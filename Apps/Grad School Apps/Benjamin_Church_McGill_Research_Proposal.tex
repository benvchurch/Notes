\documentclass[10pt]{amsart}
\usepackage{import}
\import{"../Algebraic Geometry/"}{AlgGeoCommands}

\newcommand{\Loc}[1]{\mathfrak{Loc}\left( #1 \right)}
\newcommand{\AbGrp}{\mathbf{AbGrp}}

\newtheorem*{defnn}{Definition}
\newtheorem*{conj}{Conjecture}

\usepackage{hyperref}
\usepackage{fancyhdr}

\pagestyle{fancy}
\fancyhead[LH]{\textbf{Benjamin Church}}
\fancyhead[RH]{\textbf{Research Proposal}}
\setlength{\headheight}{15pt}
\setlength{\headsep}{0.2in}

\begin{document}

\thispagestyle{fancy}

\section*{Background}
\subsection*{The L\"{u}roth Problem}

A variety $X$ is \textit{unirational} if there exists a dominant rational map $\P^n \rat X$ and \textit{rational} if the map can be birational. L\"{u}roth problem asks if unirationality implies rationality. The answer is affirmative for surfaces over a field of characteristic zero because separability ensures that unirational dominations are generically \etale implying that any variety dominated by $\P^n$ must have vanishing canonical invariants and thus must be rational by Castelnuovo's criterion. However, in positive characteristic, this argument fails due to the existence of inseparable maps and, consequently, counterexamples do exist. The best known are due to Zariski \cite{zariski1958}, for surfaces of the form $z^p = f(x, y)$, and Shioda \cite{shioda1974}, for certain Fermat surfaces. Unlike the case of rational surfaces for which Castelnuovo's criterion applies, there are no known numerical invariants for detecting unirationality.

\subsection*{Supersingular Varieties}

Let $X_0$ be a smooth proper variety over the finite field $\mathbb{F}_q$ and $X = X_0 \times \overline{\mathbb{F}}_q$. For a prime $\ell \ndivides q$ we write $H^{i}(X_{\et}, \Q_\ell)$ for the $\ell$-adic \etale cohomology of $X$ in degree $i$. For smooth proper surfaces, even cohomology of $X$ is generated by algebraic cycles iff the Picard number $\rho(X) := \mathrm{rank}(\NS{X})$ equals the second betti number $b_2(X) := \dim_{\Q_\ell} H^2(X_{\et}, \Q_\ell)$ in which case we say that $X$ is \textit{Shioda supersingular}.
Taking into account the Galois action, $[Z]$ is naturally an invariant element of $H^{2r}(X_{\et}, \Q_\ell(r))$ where coefficients are twisted by the cyclotomic character. According to the Tate conjecture, this condition on the Galois action exactly characterizes algebraic cycles. Consequently, we say a smooth proper variety over $\mathbb{F}_q$ is \textit{supersingular} if, for each $i$, the eigenvalues of Frobenius on $H^{i}(X_{\et}, \Q_\ell)$ are $q^{i/2}$ times a root of unity. 

\subsection*{The Conjecture of Shioda}

We would like to find sufficient computable invariants to determine unirationality. Since Frobenius action is preserved under rational domination, unirationality implies supersingularity. In important known cases, the converse holds as well. For example, the converse is proven for K3 surfaces\footnote{A. J. de Jong has privately informed me that an error has been identified in Liedtke's proof, however, the result is still believed to hold.} \cite{liedtke}, Kummer surfaces \cite{shioda_some_results},  and Fermat surfaces \cite{shioda_on_fermat}. Given these examples, Shioda formulated \cite{shioda_some_results} the following conjecture: let $X$ be a simply-connected ($\pi_1^{\et}(X) = 0$) surface over $\mathbb{F}_q$ then $X$ is unirational if and only if $X$ is Shioda supersingular.

\section*{Research Plan}

I plan to investigate the Shioda conjecture for diagonal weighted-projective hypersurfaces. Explicitly, these are hypersurfaces in $\P(q_0, q_1, q_2, q_3)$ cut out by an equation of the form,
\[ a_0 X_0^{n_0} + a_1 X_1^{n_1} + a_2 X_2^{n_2} + a_3 X_3^{n_3} = 0 \]
These surfaces are well-suited to studying the Shioda conjecture because, by a seminal result of Andr\'{e} Weil \cite{weil_counting}, their zeta functions may be efficiently computed in terms of Jacobi sums. I propose the following broad goals:
\begin{enumerate}
\item[(1)] employ a computer search to find new examples of supersingular hypersurfaces
\item[(2)] classify these explicit examples into infinite families of supersingular examples
\item[(3)] determine which of these examples are unirational.
\end{enumerate}
Of these goals, the third presents the biggest challenge since there are no good tools available to determine unirationality. It may be more tractable to determine the density of rational curves on a surface. We say a variety is rationally-connected if each pair of points lies on a rational curve. We expect unirational surfaces to be rationally connected and non-unirational surfaces to have finitely many rational curves. Therefore, investigating the families of rational curves on $X$ gives insight into this problem. I intend to continue my efforts on this related problem attempting to adapt the methods of \cite{lang} based on Bogomolov's finiteness results.
\par
In the Columbia 2018 REU \cite{REU}, my team and I successfully implemented Weil's method, completing goal (1), and gave a partial answer for (2). In particular, we discovered two infinite families of supersingular diagonal hypersurfaces which have the form $(n_0, n_1, n_2, n_3) = (p, q, ps, qs)$ for distinct primes $p,q$ such that $p,q \equiv 1 \mod s$ and of the form $(n_0, n_1, n_2, n_3) = (p, q, r, pqr)$ for distinct primes $p, q, r$. We proved that surfaces of these forms are supersingular when the characteristic satisfies an explicit numerical criterion. Because any surface dominated by a supersingular surface is again supersingular, classifying which diagonal surfaces whose exponents are combinations of a small number of primes are supersingular will provide information about a much wider class of diagonal surfaces.  
\par
From Shioda's work \cite{shioda_on_fermat}, we already know an infinite family of examples, namely, Fermat surfaces where $p^\nu \equiv - 1 \mod n$ as well as any diagonal surface which may be dominated by a Fermat surface of this type. However, these examples are already known to be unirational. Thus, the real success of our methods is the discovery of additional infinite families of supersingular surfaces that are not of the above form. Over the summer of 2020, I worked with Prof.\ Johan de Jong to determine if these new examples are unirational as the Shioda conjecture would imply. We have reduced the question of rational-connectedness to one about certain loci in the moduli space of cyclic $3$-covers of $\P^1$ which we hope to be more tractable. 


\bibliographystyle{unsrt}
\bibliography{bibliography}

\end{document}