\documentclass[11pt]{article}
\usepackage{import}
\import{"../Algebraic Geometry/"}{AlgGeoCommands}

\newcommand{\Loc}[1]{\mathfrak{Loc}\left( #1 \right)}
\newcommand{\AbGrp}{\mathbf{AbGrp}}

\newtheorem*{defnn}{Definition}
\newtheorem*{conj}{Conjecture}

\usepackage{hyperref}
\usepackage{fancyhdr}

\geometry{margin=0.7in}
\pagestyle{fancy}
\fancyhead[LH]{\textbf{Benjamin V. Church}}
\fancyhead[RH]{\textbf{Application Questions: University of Washington}}
\setlength{\headheight}{15pt}
\setlength{\headsep}{0.2in}



\usepackage[backend=bibtex, citestyle=apa, style=phys]{biblatex}
\addbibresource{bibliography.bib}

\begin{document}



\noindent \textbf{1. Almost all of our PhD students are supported as Teaching Assistants. Do you want to apply for such support? If yes, please explain your prior work experience. If the answer is no, please briefly explain your plans for paying tuition and living expenses.}
\bigskip\\
I do intend to apply for teaching-based financial support. Teaching is an important component of my academic life. I have been passionate about teaching since high school when I organized a student-run seminar promoting interest in pure mathematics. I have volunteered to teach over two dozen classes at Columbia and MIT Splash aimed at curious high school students, a six-week HSSP course introducing high school students to elliptic curves, and a dozen talks at Columbia's math and physics clubs aimed at undergraduates. Last year, in collaboration with the Columbia Association for Women in Mathematics, I helped create introductory talks, materials, and help sessions aimed at supporting freshmen who were new to college-level mathematics courses. I was specifically selected by Prof.\ Brian Cole to teach weekly recitations for his accelerated physics course, an unusual honor for an undergraduate. Teaching was one of the highlights of my college career as I got the pleasure of leading the next cohort of eager students and seeing the fascination I had for the subject mirrored in them. This past summer, I worked as a Ross Mathematics Program counselor, leading a group of five students through Ross’ number theory course and optional advanced courses taught at The Ohio State University. My responsibilities included: leading daily lectures on number theory topics, leading students through guided examples, providing help sessions, and grading problem sets.
\bigskip\\
\noindent\textbf{2. Why do you want to pursue a graduate degree in mathematics, specifically at the University of Washington? And, if you are applying for the Master’s program, please explain why you chose that instead of the Ph.D.}
\bigskip\\
The University of Washington has an outstanding research group in algebraic geometry whose influential and broad scholarship and would foster my intellectual growth, inspire me to explore my current interests, and drive me to expand into new areas of research. I am particularly interested in Prof.\ Max Lieblich’s work on derived equivalence, especially his investigation of the conditions under which derived equivalence implies birationality. I became especially interested in birational geometry after the 2018 Columbia REU and I continue to investigate certain unirational surfaces in positive characteristic with guidance from Prof.\ Johan de Jong. Employing methods from his study of derived equivalences, Prof.\ Lieblich has additionally made fascinating contributions to the theory of K3 surfaces including understanding Fourier-Mukai partners of K3 surfaces and proving that the Tate conjecture for K3 surfaces is equivalent to there being finitely many K3 surfaces defined over a finite field extension. Other faculty members whose’ research interests align with mine include Prof.\ Jarod Alper and Prof.\ S\'{a}ndor Kov\'{a}cs. Prof.\ Alper studies the structure of algebraic stacks and their moduli. I am especially interested in his results on the \etale-local structure of the moduli stack of curves which are equivalent to the existence of certain deformations of generic curves. Prof.\ Kov\'{a}cs has made significant contributions to the minimal model program and classifying singularities of algebraic varieties. His work on rational singularities and his construction of singular and of positive characteristic counterexamples to generic vanishing for morphisms of varieties to abelian varieties are of particular interest to me.
\bigskip\\
\noindent \textbf{3. (Optional) If you created a video presentation, where is it posted so we can access it?}
\bigskip\\
I do not have a video presentation.
\bigskip\\
\noindent \textbf{4. (Optional) Describe a recent mathematical experience that influenced your decision to apply to graduate school in math.}
\bigskip\\
My undergraduate thesis work was extremely influential in motivating me to pursue a Ph.D. in mathematics. Under Prof.\ de Jong, I studied the problem of embedding smooth curves in toric surfaces. Using a result of Harris and Mumford \footfullcite{harris1982kodaira}, I gave a proof that very general curves cannot be embedded in any toric surface and I studied obstructions to these embeddings having transverse intersection with the toric divisor. This project aimed to investigate the regularity conditions introduced in a paper of Dokchitser \footfullcite{models_of_curves} which provides an algorithm to construct the minimal r.n.c. models of certain curves over DVRs using toric embeddings. I constructed a degeneration of a genus 5 curve with a nontrivial Galois action on the components of its special fiber and showed that such an r.n.c. model cannot result from Dokchitser's method. Thus, no affine equation for this curve can satisfy required regularity conditions on its intersections with the toric divisors. For my thesis, I was awarded the John Dash van Buren Jr. Prize in Mathematics. This work was an invaluable learning experience in how research is conducted in mathematics: what techniques to try, how to manage frustration, and how problems evolve. The research process was also a valuable learning opportunity to dive more deeply into specific areas of algebraic geometry. Thanks to Prof.\ de Jong’s excellent mentorship and our working relationship, the process was also personally extremely rewarding and enjoyable; it fully convinced me that mathematical research is what I want to spend my life doing. 
\bigskip\\
\noindent \textbf{5. (Optional) What academic or scholarly experience has been most challenging for you, and how did you handle it?}
\bigskip\\
In the fall of my sophomore year, I enrolled in Prof.\ Michael Harris's course on algebraic number theory. At that point, outside of a course on groups, I had little formal experience with abstract algebra. Immediately, I felt completely over my head, becoming lost in the weeds of DVRs and cyclotomic fields, struggling against a firehose of new concepts. The weekly problem sets endlessly confronted me with my confusion about basic properties. With generous help from more experienced classmates, however, the pieces began to click. By the end of the course, I was able to approach the homework independently and felt confident solving the problems on our take-home final exam on which I scored top marks. Although the experience was difficult and taxing, it bolstered my confidence in mathematics and in learning topics that appear, at first, completely out of reach. Struggling with this course also taught me the importance of working both alone and in groups; while a group can provide the flash of insight or difference in perspective needed to overcome a roadblock, independent work is also necessary to fully digest and internalize the foundations. Encouraged by Prof.\ Harris, I enrolled in his graduate course on class field theory the following semester. At first, I was once again bombarded by unfamiliar concepts such as adelic class groups and group cohomology but, armed with experience from the previous semester, I reformed my study group and began the methodical process of unwinding this tangled theory, this time confident in my ability to pull out the threads.
\bigskip\\
\noindent \textbf{6. (Optional) Describe any experiences in your personal history that have influenced your intellectual development, interests, career plans, and goals, such as family, educational, cultural background, economic background, prior work experience.}
\bigskip\\
From a young age, my parents instilled in me a love of learning. My parents would excitedly take me to science museums, encourage me to take apart discarded electronics, and explore the wonders of arithmetic. In middle school, my father began teaching me to program in C and Lisp, tasking me with writing a program to symbolically compute derivatives. I fell in love with the quirks of Lisp, especially how, inspired by $\lambda$-calculus, it blurred the lines between data and procedures, a curiosity that would eventually lead me to an interest in mathematical logic. Early in high school, I devoured Walter Lewin’s lectures on physics and Gilbert Strang's linear algebra. Soon, the world became my laboratory; I began carrying around a polarizer film in my pocket wherever I went to investigate atmospheric optical phenomena. My parents gifted me the Feynman Lectures on Physics which again caught my imagination and led me to pursue multivariable calculus. By this point, I was convinced I would become a physicist. However, a year later, I enrolled in a summer number theory course and was immediately awestruck. The course opened my eyes to the intricate, subtle, and surprising patterns in the most familiar and mundane objects: the integers. Now, mathematics captivated me with the same sort of beauty and thrill of discovery that had drawn me to physics. From there, bit by bit, I began to glimpse the startling depth and overwhelming beauty that springs forth from even basic mathematical objects and their interrelations, solidifying this curiosity into a focused passion. 
\bigskip\\
\noindent \textbf{7. (Optional) Have you initiated or participated in activities that work to increase participation of underrepresented populations in the mathematical sciences?}
\bigskip\\
When I served on the board of Columbia Undergraduate Mathematics Society (UMS), I, in collaboration with the Columbia Association for Women in Mathematics (AWM), gave help sessions and wrote example-based materials aimed at students who were new to proof-based mathematics courses. The transition from high school math to college math is challenging for most students and often accentuates inequalities. Recognizing that disparities in higher education are widened by disparities in access to advanced preparation at the high school level, we aimed to mitigate the attrition of underrepresented groups in mathematics by helping incoming students bridge the gap between high school and college. We offered help sessions, problem-based written materials, and guided sessions which exposed students to proof techniques and examples, and we advertised available resources for students, especially for women, offered by AWM and UMS. Furthermore, as president of the Columbia Society of Physics Students, I helped organize a mentorship program in collaboration with Columbia Society Women in Physics (now Columbia Spectra). The program connected new students interested in physics with peer mentors who could speak to student-specific issues more directly than faculty advisors. In particular, we aimed to pair students from underrepresented populations (per the student's preferences) with seniors from similar backgrounds who might be better able to address, from personal experience, issues specific to these groups. We hoped that one-on-one mentorship would ameliorate the discouraging effects that underrepresentation has on incoming students. 
\bigskip\\
\noindent \textbf{8. (Optional) Is there anything else you would like us to know about you that we should consider as part of your application?}
\bigskip\\
I graduated from my undergraduate program in May 2020. My plan, made in December 2019, was to take a gap year between my undergraduate and graduate studies. I intended to spend the year abroad and get work experience before starting work on my Ph.D. I specifically had lined up work teaching math in Shanghai and plans to travel in China. Due to the Covid-19 pandemic, this became impossible. Instead, I have been doing independent study and online coursework to prepare for graduate school. Specifically, I have worked on a project with Prof.\ Johan de Jong trying to determine the unirationality of a class of supersingular surfaces over finite fields which my team and I discovered in the Columbia 2018 REU. I am attempting to complete every exercise in Hartshorne’s Algebraic Geometry and Huybrechts’ Complex Geometry: An Introduction. Furthermore, I have been attending Prof.\ Daniel Litt’s course on \etale cohomology, Prof.\ Max Lieblich’s algebraic geometry seminar, and the course on resolution of singularities taught by Prof.\ Lieblich and Prof.\ de Jong. My passion for mathematics has grown into dedication which has continued to motivate me to pursue intense coursework, research projects, and tenacious self-study. This dedication has prepared me with the technical background and perseverance necessary to thrive in the rigorous Ph.D. program at UW. I strongly believe that UW is an ideal place for my intellectual development and mathematical studies, and I sincerely hope to be given the opportunity to learn from and contribute to this vibrant community. 
\end{document}