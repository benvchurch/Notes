\documentclass[11pt]{amsart}
\usepackage{amssymb}
\usepackage[margin=1in]{geometry}

\begin{document}

\thispagestyle{empty}

\noindent
\underline{\textbf{Personal Background}}
\\
Mathematics has been a lifelong interest and an ever growing passion of mine. Although I enjoyed and excelled at math from an early age, my curiosity and attention were more focused on the natural world: using mathematics to uncover nature's mysteries through physics and astronomy. However, I was unaware of the startling depth and overwhelming beauty to be found within mathematics itself. In high school, a summer course in number theory opened my eyes to the intricate, subtle, and surprising patterns in what seem to be the most familiar and mundane objects: the integers. From then on, I was hooked. Through New York Math Circle, I was exposed to fascinating problems and the seeds of modern techniques which caused me to fall even deeper in love with mathematics for its own sake as I prepared to enter College. Now, having completed a wide array of mathematics coursework, including at the graduate level, and dipped my toes into the world mathematical research, I am more excited than ever to pursue these mysteries in a PhD program where I will be able to learn and collaborate with motivated students and professors alike. 
\\
\\
\noindent
\underline{\textbf{Intellectual Merit}}
\\
Due to my broad interests in science, I have explored various areas through coursework, research, and independent study. I completed a significant depth of coursework in my majors,  maintaining a high academic standing whilst taking advanced graduate coursework in mathematics and physics simultaneously. Due to my extensive coursework in these areas and enthusiasm for the subjects, I was awarded departmental honors in both the mathematics and physics departments. 
\par 
Outside the classroom, I have eagerly pursued scientific research. I began my first serious project under the direction of Professor Jessica Mar studying differential skewness in gene-expression data collected from cancer cells. During the project, she recommended that I read foundational texts on probability theory and statistics to ensure I fully understood the mathematical underpinnings of techniques such as correction for multiple testing and Bayesian analysis. This showed me the value of rigorous pure mathematics as a guide and a tool even in applied settings. I believe this sort of interdisciplinary investigation is extremely promising and is a direction I am interested in pursuing later in my career. For my project, I was selected as a Columbia Science Research Fellow. Our work was published in BMC Bioinformatics and I presented it at the 2019 International Conference on Intelligent Biology and Medicine \cite{skew}.
\par 
My next research project was in galaxy astrophysics and dark matter physics under Professor J. P. Ostriker. The fuzzy cold dark matter (FDM) model which hypothesizes that a significant fraction of the dark sector is comprised of ultra-light axion-like bosons. Unlike standard cold dark matter (CDM) models, FDM preducts that quantum-mechanical effects, such as interference and quantum pressure, should  be manifest on galactic scales. During dark matter collapse in galaxy formation, FDM interference can set up long-living standing waves which produce gravitational fluctuations that pump energy into the stars forming a galactic disc. We proposed this as a mechanism for causing the observed thickness of the Milky Way disc. I modeled the evolution of adiabatic invariants of the disc as it underwent gravitational interactions with the standing waves formed by the collision-less dark fluid which allowed me to derived the history and radial profile of this disc heating. Such predictions are in close agreement with observations of the Milky Way and the observed evolution profiles of glactic discs. Fitting this model to observations, I was able to set a lower bound on the mass of the ultra-light particle hypothesized to constitute such dark matter. Our work was published in the Monthly Notices of the Royal Astronomical Society \cite{FDM}.
\par 
Following these projects, in the summer of 2018, I participated in the Columbia math REU on curves in surfaces over finite fields. Our group studied the zeta functions of diagonal weighted-projective surfaces over finite fields using computation methods. We aimed to generalize Shioda’s classification of supersingular Fermat varieties to weighted-projective diagonal hypersurfaces. However, the naive generalization given by applying Shioda’s classification to the minimal covering Fermat surface provided a sufficient but not necessary condition for supersingularity. To find a stronger criteria on supersingularity, we approached the topic from an arithmetic perspective by using a result of Weil to compute the zeta functions of these diagonal hypersurfaces in terms of Gaussian sums. We then applied Stickelberger's theorem to determine the factorization of ideals generated by Gaussian sums and thus determine the roots and poles of the zeta function corresponding to eigenvalues of the Frobenius action on the variety’s $\ell$-adic cohomology. This method allowed us to reduce the problem of checking supersingularity to a purely arithmetic condition on the exponents and characteristic. 
From a computer search based on this arithmetic condition, I was able to identify a pattern in certain new examples of supersingular surfaces. From this observation, I proved the existence of an infinite family of supersingular weighted-projective surfaces over some infinite family of characteristics such that the minimal Fermat surface parametrizing them fails to be supersingular. We then identified other infinite families with this same property. I found this research topic highly compelling since it required approaches both from the algebraic geometry and from the number-theoretic perspectives. This problem has shaped and informed my current interests proposed research.
\par
The summer of 2019, I had the wonderful opportunity to study toric geometry and inequalities in convex geometry in Paris through a joint program between Columbia and Paris Diderot University. Under the direction of Professor Huayi Chen, my group studied the relationship between intersection pairing of big nef divisors and inequalities in convex geometry. and the relationships between these inequalities and constructions on toric varieties. Associated to such divisors are compact convex sets called Okunkov bodies whose volume reflects the intersection pairing and asymptotic number of sections. Variants of the Brunn-Minkowski, Alexandrov-Finchel, and isoperemetric inequalities applied to these Okunkov bodies can be strengthened by introducing probabilistic techniques \cite{probabiliste}. Specifically, an important term arising in these inequalities is the correlation between convex bodies which has a form similar to a Kantorovich optimal transport problem  \cite{isoperimetric}. We established an upper bound on the correlation between special cases of convex pairs using the Brenier map.
\par
One of my most impactful experiences in college was the ability to work closely with professors through independent study. These were an invaluable opportunity to form working relationships with professors, gain vital insight into the field, and provided excellent preparation for graduate studies. In spring 2018, I studied elliptic curves with Professor David Hansen initially from the perspective of number theory as I was concurrently taking graduate level coursework in class field theory. However, this independent study was my first exposure to algebraic geometry and I became so enamored with the subject that I pivoted the course towards algebraic curves which inspired my current interests. To broaden this perspective, I decided to learn more about the analytic approach via modular forms through an independent study with Professor Chao Li covering Diamond and Shurman. That year, I also did a reading course with Professor Brian Cole on advanced topics in electricity and magnetism using Jackson. This independent study expanded upon the basic topics covered in the undergraduate curriculum and bridged the gap between the classical theory and the results of quantum electrodynamics I was simultaneously learning in quantum field theory. 
\par
My senior year, I focused primarily on independent study and my undergraduate thesis work. I worked through Deligne's proof that Hodge cycles on abelian varieties are absolutely Hodge in my independent study with Professor Michael Harris. The methods used in this proof brought together nearly all my mathematical studies from algebraic geometry, to number theory and Galois theory, to representation theory, to algebraic topology. I found the topic deeply compelling because it linked together these desperate pieces I had been learning one-by-one to form an incredible beautiful and powerful argument. That year, I also did an independent study with Professor Evan Warner on algebraic K-theory and a reading course with Professor Michael Harris on Shimura varieties extending what I had learned through Deligne's proof about these objects as moduli spaces .
\par
The Paris Diderot program gave me a solid background in toric geometry which prepared me for my thesis topic. I studied the problem of embedding smooth curves in toric surfaces. I showed, using a result of Harris and Mumford \cite{harris1982kodaira}, that very general curves cannot be embedded in any toric surface. Furthermore, I gave various examples showing restrictions to embedding curves into toric surfaces as Cartier divisors. This project was inspired by a recent paper by Tim Dokchitser \cite{models_of_curves} which provides an algorithm which constructs for a given curve the minimal regular normal crossings model defined over a DVR. Dokchitser's method requires embedding the curve in a toric surface and constructs the model through gluing semi-toric surfaces associated to subdivisions of the Newton polygon. I was interested in finding examples of curves where this method fails to produce a minimal model. I constructed a degeneration of a genus $5$ curve with a nontrivial Galois action on the component graph of its special fiber and showed that such a regular normal crossings model cannot result from Dokchitser's method. For my thesis, I was awarded the John Dash Van Buren Jr. Prize in Mathematics. This work was an invaluable learning experience about how research is conducted in mathematics: what techniques to try, how to redirect frustration, and how problems evolve. It was also pivotal in clinching my decision to pursue graduate studies in mathematics. Due to my variety of interests, coursework, and research experience, I had previously felt conflicted about choosing this area. However, working on this problem was extremely rewarding and, largely due to the invigorating working relationship with Professor Johan de Jong and his excellent mentorship, I ended the process fully convinced that mathematical research is what I want to spend my life doing.
\par 
This past summer, I continued studying independently with Professor Johan de Jong. Primarily, I began working on determining the unirationality of various special examples of supersingular surfaces which I discovered during my 2018 REU. These surfaces may be described as cyclic covers of $\mathbb{P}^2$ and also as cyclic quotients of products of easily understood algebraic curves. I have been researching methods to find rational curves on such surfaces which can be extended to use in positive characteristic.
At the same time, Professor Johan de Jong graciously mentored me through various topics I was interested in reading about. In particular, outside my direct research aims, I read papers on N\'{e}ron models of abelian varieties, algebraic connections and Atiyah classes of vector bundles, and Cartier's theorem relating Frobenius descent to integrable algebraic connections with vanishing $p$-curvature. Currently, I am attending a seminar at UW organized by Professor Johan de Jong and Professor Max Lieblich where I will be presenting a paper by Artin and Mumford which constructs examples of unirational complex $3$-folds which are not rational. This is a seminal paper in birational geometry which is particularly relevant to by current research interests in nonrational unirational varieties.
\\
\\
\noindent
\underline{\textbf{Broader Impact}}
\\
Throughout my college career, I have not lost sight of the stepping stones which supported me. I am indebted deeply to the opportunities and people who made me see mathematics the way I do today. Therefore, I see it as my duty to be that person for a new generation of aspiring mathematicians and scientists, to open curious minds to the mysteries around them and to provide resources and support for those who choose to pursue them. 
I have volunteered to teach over 20 splash classes at Columbia and MIT aimed at high school students curious about physics and mathematics beyond the classroom.
This past summer, I taught in the Ross Mathematics Program leading students through their number theory course. I also volunteered to teach a six-week course on Elliptic curves for high school students through HSSP. I had nearly 30 highly motivated students who I led on a meandering journey through some of the most surprising and wonderful connections between algebra and geometry. I will be very proud if I succeeded, as I think I did, in instilling an appreciation for the beautify geometry hiding in high-school level algebra and possibly inspiring a career in mathematics. Teaching is not only a way to give back, I also believe it is essential for my own intellectual growth. Teaching forces one to critically reexamine what one thinks they know and to consider alternative perspectives and mental models. 
\par
I was also significantly involved with the math and physics communities at Columbia through two student run organizations: the Columbia Undergraduate Mathematics Society (UMS) and the Columbia chapter of the Society of Physics Students (SPS). Both groups offer educational outreach and exposure to a board array of ideas in the field. My senior year, I served as president of SPS during which time I organized weekly talks by professors and graduate students to help undergraduates understand the field and possibly find research topics which interest them. Furthermore, SPS provides physics outreach and demonstrations for New York City middle school groups for whom I participated multiple years by performing exciting and thought provoking demonstrations of natural phenomena. Along with our sister organization, Columbia Society for Women in Physics, SPS organizes mentorship, advising, study and help sessions, as well as a relaxed environment for students to meet and learn from their peers. I also served on the board of UMS which provided similar community, outreach, and weekly talks for the math community. I participated in and helped create an undergraduate speaker series for undergraduates to practice their presentation skills and to promote teaching and learning between peers. Over the course of my involvement with these clubs, I have given over 10 talks and presentations on a wide variety of topics from the whimsical, "is it possible to start a fire with moonlight" aimed at exciting new SPS members, to more technical topics such as ``Galois representation on elliptic curves'' and ``L\"{o}b's theorem and formalized G\"{o}del incompleteness.'' My final year, I was involved with a joint project along with the Columbia Association for Women in Mathematics to create a series of introductory talks, materials, and help sessions aimed at supporting freshman who were new to proof-based college level mathematics courses. I produced and edited a large portion of the materials which led new students through proof methods by example and I led guided help sessions to assist new students in putting these methods into practice.
\\
\\
\noindent
\underline{\textbf{Future Goals}}
\\
My interests have coalesced around algebraic geometry, specifically birational geometry in positive characteristic. This area is full with beguiling unknowns and wide open problems which I hope to investigate during my graduate studies. Currently, I am particularly interested in the Shioda conjecture and supersingular algebraic surfaces. Immediately following the completion of my senior thesis, I began working on the related problem (outlined in my research statement) under the direction of Professor Johan de Jong. I intend to pursue this question further during my graduate studies as well as incorporating and exploring the deep interplay between algebraic geometry and related branching out into related areas in number theory and complex geometry which I have only yet begun to study. I also hope to attend a graduate program with a strong mathematical physics group because I continue to be extremely interested in the ways algebraic geometry, in particular, finds applications in modern physics, in particular, string theory. It is my hope that in my graduate studies, I will be able to unify these two interests towards solving research problems on the boundary between mathematics and physics.
\par
After I complete my doctoral degree, I plan to pursue postdoctoral research with the eventual goal of becoming a professor. However, my intention to pursue a professorship goes beyond having a research position. I see teaching as a vital aspect of my career both in order to repay the excellent instruction I was given and because I believe teaching others to be an invaluable part of my own intellectual development.


\bibliographystyle{unsrt}
\bibliography{bibliography}

\end{document}