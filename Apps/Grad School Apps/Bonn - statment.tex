\documentclass[11pt]{article}
\usepackage{import}
\import{"../Algebraic Geometry/"}{AlgGeoCommands}

\newcommand{\Loc}[1]{\mathfrak{Loc}\left( #1 \right)}
\newcommand{\AbGrp}{\mathbf{AbGrp}}

\newtheorem*{defnn}{Definition}
\newtheorem*{conj}{Conjecture}

\usepackage{hyperref}
\usepackage{fancyhdr}

\pagestyle{fancy}
\fancyhead[LH]{\textbf{Benjamin V. Church}}
\fancyhead[RH]{\textbf{Statement of Purpose: BIGS}}
\setlength{\headheight}{15pt}
\setlength{\headsep}{0.2in}

\geometry{margin=0.7in}


\usepackage[backend=bibtex, citestyle=apa, style=phys]{biblatex}
\addbibresource{bibliography.bib}

\begin{document}
Mathematical research is what I want to spend my life doing. My goal is to complete a doctorate in mathematics and pursue a career in academia. Number theory first opened my eyes to the intricate, subtle, and surprising patterns in the most familiar and mundane objects: the integers. As my courses abstracted and generalized, I began to glimpse the startling depth and overwhelming beauty that springs forth from even basic objects and their interrelations, solidifying my curiosity into a focused passion. Although I cast a wide net of scientific interests and research projects, including published work in astrophysics and bioinformatics, my interests have firmly coalesced around algebraic and arithmetic geometry. However, I retain a connection to physics and a drive to engender greater dialogue between mathematics and the sciences. Having completed upper-level mathematics coursework and dipped my toes into mathematical research, I aspire to earn a doctorate at Bonn to pursue these mysteries and build my career as an academic researcher. At Bonn, I would be honored to work with innovative Professors in my areas of interest, particularly Peter Scholze, Gerd Faltings, and David Hansen. 
\par
Working closely with professors through independent study was an invaluable opportunity to explore interesting topics, gain vital insight into the field, and prepare for graduate-level work. In spring 2018, I studied elliptic curves with Prof David Hansen initially focused on number theory. This independent study was my first exposure to algebraic geometry and I became enamored with algebraic curves. I am fascinated by the interplay between number theory, complex geometry of elliptic curves, and geometry in positive characteristic. To deepen my understanding of these relationships, I learned about modular forms and Galois representations through an independent study with Prof. Chao Li. 
\par
I participated in the 2018 Columbia math REU studying the zeta functions of diagonal weighted-projective surfaces over finite fields. We aimed to generalize Shioda’s classification of supersingular Fermat varieties \footfullcite{shioda_on_fermat} to weighted-projective diagonal hypersurfaces. Using a result of Weil \footfullcite{weil_counting}, we computed the roots and poles of the zeta functions in terms of Gaussian sums and then applied Stickelberger's theorem to efficiently determine supersingularity. Using a computer search, I was able to identify patterns in certain new examples of supersingular surfaces. From this observation, I proved the existence of an infinite family of supersingular weighted-projective surfaces such that the minimal covering Fermat surface fails to be supersingular. My team then identified other infinite families with these property. This project solidified my love of algebraic geometry, especially geometry in positive characteristic and its relations to arithmetic, and introduced me to the Weil conjectures. It inspired me to study scheme theory and \etale cohomology, devoting myself to EGA, Hartshorne exercises, and Milne's \etale cohomology, in order to understand the proofs by Grothendieck and Deligne. The introduction of \etale cohomology theory to explain, geometrically, properties of the generating function for counting solutions to polynomials over finite fields remains my absolute favorite piece of math. 
\par
The summer of 2019, I had the wonderful opportunity to study toric geometry and inequalities in convex geometry in Paris through a joint REU program between Columbia and Paris Diderot University. Under Prof Huayi Chen, my group studied intersection pairings of divisors and their relationship to mixed volumes of the associated Okunkov bodies. Variants of the Brunn-Minkowski for divisors can be strengthened by introducing probabilistic techniques \footfullcite{probabiliste}.
An important term arising in these inequalities is the correlation index between two convex bodies which takes the form of a Monge-Kantorovich optimal transport problem. Based on the Monge formulation, we applied the Brenier map and Knothe map whose Jacobians, and thus the Radon-Nikodym derivative of their induced measure, give an upper bound on the correlation purely in terms of the volumes of the convex bodies. Although our results turned out to be known in the literature, the experience exposed me to new methods in geometry and improved my flexibility in approaching research.
\par
My senior year, I studied Deligne's proof that Hodge cycles on abelian varieties are absolutely Hodge with Prof Michael Harris. The methods used in Delinge's proof brought together nearly all my mathematical studies from algebraic geometry, number theory, Galois theory, representation theory, and algebraic topology, linking together disparate pieces to form an incredibly beautiful, powerful, and surprising argument. 
Concurrently, I attended Prof Johan de Jong's weekly seminar on Weil cohomology theories and algebraic de Rham cohomology which led me to read further about the comparison theorems used in Milnes's text. In doing so, I read about Hodge theory to relate algebraic and smooth de Rham cohomology and Grothendieck's ``Tohoku'' paper to understand cohomological methods like hypercohomology and universal $\delta$-functors employed in Milne's text. I also read Milne's notes on motives to understand the motivic perspective on absolute Hodge cycles. An important ingredient in Deligne's proof involves constructing a family of abelian varieties such that a distinguished fiber is of CM-type. The base turns out to be a Shimura variety encoding moduli of abelian varieties sparking my interest in arithmetic geometry so I pursued further independent study with Prof Harris on Milne's Shimura varieties.
\par
The Paris Diderot program gave me a solid background in toric geometry which prepared me for my thesis topic. Under Prof  de Jong, I studied the problem of embedding smooth curves in toric surfaces. Using a result of Harris and Mumford \footfullcite{harris1982kodaira}, I showed that very general curves cannot be embedded in any toric surface and I gave examples exhibiting obstructions to these embeddings being Cartier. This project was inspired by a paper of Dokchitser \footfullcite{models_of_curves} which provides an algorithm to construct the minimal r.n.c. models of curves over DVRs using toric embeddings. I constructed a degeneration of a genus 5 curve with a nontrivial Galois action on the components of its special fiber and showed that such a r.n.c. model cannot result from Dokchitser's method meaning that no affine equation for this curve can satisfy required regularity conditions on its intersections with the toric divisors. For my thesis, I was awarded the John Dash van Buren Jr. Prize in Mathematics. This work was an invaluable learning experience about how research is conducted in mathematics: what techniques to try, how to manage frustration, and how problems evolve. Thanks to Prof de Jong’s excellent mentorship and our working relationship, I am fully convinced that mathematical research is what I want to spend my life doing. Immediately following the completion of my thesis, I began a project with Prof de Jong determining the unirationality of various supersingular surfaces which I discovered during my 2018 REU. These surfaces may be described as cyclic covers of $\mathbb{P}^2$ and also as cyclic quotients of products of easily understood algebraic curves. I am working methods to find rational curves on such surfaces which can be extended to use in positive characteristic. Outside my direct research aims, Prof de Jong guided my reading of papers on N\'{e}ron models of abelian varieties, algebraic connections and Atiyah classes of vector bundles, and Cartier's theorem relating Frobenius descent to integrable algebraic connections with vanishing $p$-curvature.
\par
Teaching is also I major part of who I am and what I want to spend my life doing. I have been passionate about teaching since high school when I organized a student-run seminar promoting interest in pure mathematics. I have volunteered to teach over 20 extra-curricular classes at Columbia and MIT aimed at advanced high school students. Last year, in collaboration with the Columbia Association for Women in Mathematics, I helped create introductory talks, materials, and help sessions aimed at supporting freshman who were new to college-level mathematics courses. I was specifically selected by Prof Brian Cole to teach weekly recitations for  his accelerated physics course, an unusual honor for an undergraduate. Teaching was one the highlights of my college career as I got the pleasure of leading the next cohort of eager students through what had fascinated me about the subject and seeing that same fascination mirrored in them.
\par
The arithmetic algebraic geometry group at Bonn is world-class and extremely innovative. I am particularly inspired by Prof Scholze’s introduction of perfectoid spaces, prismatic cohomology, and proofs of foundational results on the $B_{\mathrm{dR}}^+$-affine Grassmannian which are some of the most significant contributions to modern arithmetic geometry. Additionally, the partnership between the University of Bonn and the Max Planck Institute for Mathematics at Bonn, which specializes in topics relating to my interests such as arithmetic geometry and moduli problems, would provide me even more avenues to explore my academic interests and situate myself in the field. In particular, Prof Gerd Faltings’ proof of the Mordell conjecture and more recent work on p-adic period domains are massive contributions which interest me significantly. The work of Prof Hansen, who seeded my interest in algebraic geometry, on completed cohomology of Shimura varieties and his proof of Artin-Grothendieck vanishing in the rigid setting is particularly compelling. My passion for mathematics has grown into dedication which motivated me to pursue intense coursework, research projects, and tenacious self-study. This dedication has prepared me with the technical background and perseverance necessary to thrive in Bonn’s rigorous PhD program. I strongly believe that Bonn an ideal place for my intellectual development and mathematical studies and I sincerely hope to be given the opportunity to learn from and contribute to this vibrant community.  

\end{document}