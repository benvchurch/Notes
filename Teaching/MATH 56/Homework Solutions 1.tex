\documentclass[12pt]{article}
\usepackage{import}
\import{../../General/}{General_Includes}


\begin{document}

\title{Homework 1 Solutions}

\maketitle

\section*{1}

Suppose that $\sqrt{3}$ were rational. Then we could write $\sqrt{3}$ as a rational number in reduced form meaning,
\[ \frac{a}{b} = \sqrt{3} \]
for some integers $a,b$ (with $b \neq 0$) sharing no common factors. Therefore,
\[ a^2 = 3 b^2 \]
Thus, $a^2$ is divisible by $3$ so $a$ is divisible by $3$ (think about why). This means we can write $a = 3 r$ for some integer $r$. Plugging in and diving by $3$ gives,
\[ b^2 = 3 r^2 \]
so by the same argument $b$ is divisible by $3$. Therefore, $a$ and $b$ share the factor $3$ contradicting our assumption that $\frac{a}{b}$ is in reduced form. This contradicts our assumption that $\sqrt{3}$ is rational proving that $\sqrt{3}$ is not rational.
\bigskip\\
Notice that this argument only relies on $3$ being prime and the following fact: if $p$ is prime and $p$ divides $a^2$ then $p$ divides $a$ (see if you can prove this in general).
\bigskip\\
Since $6$ is not prime, it might seem that the above argument does not go through. However, $6$ is square-free, meaning that it is not divisible by any square (besides $1$). Furthermore, I claim that if $r$ is square-free and $a^2$ is divisible by $r$ then $a$ is divisible by $r$ (how would you prove this?). Therefore, the exact same argument works to show that $\sqrt{6}$ is irrational.

\section*{2}

Let $A, B \subset S$. 

\subsubsection*{(a)}

If $x \in (A \cap B)^c$ then $x \notin A \cap B$. However, if $x \in A$ then $x \notin B$ (else $x \in A \cap B$ which it is not) and likewise if $x \in B$ then $x \notin B$ (if you are familiar with DeMorgan's law in logic, identify how it is used in this sentence). Therefore, $x \notin A$ or $x \notin B$ so by definition $x \in A^c$ or $x \in B^c$. By the definition of the union, this implies that $x \in A^c \cup B^c$. Therefore,
\[ (A \cap B)^c \subset A^c \cup B^c \]

\subsubsection*{(b)}

If $x \in A^c \cup B^c$ then $x \notin A$ or $x \notin B$. Suppose that $x \in (A \cap B)$ then both $x \in A$ and $x \in B$ which we just saw cannot be true. Therefore $x \notin (A \cap B)$ so,
\[ A^c \cup B^c \subset (A \cap B)^c \]


\section*{3}

Let $f : X \to Y$ be a function and $A,B \subset X$.

\subsubsection*{(a)}

If $y \in f(A \cap B)$ then by definition there is some $x \in A \cap B$ such that $f(x) = y$. However, $x \in A$ so $y = f(x) \in f(A)$. Likewise $x \in B$ so $y = f(x) \in f(B)$. Thus $y \in f(A) \cap f(B)$ so we see that,
\[ f(A \cap B) \subset f(A) \cap f(B) \]

\subsubsection*{(b)}

Let $y \in f(A \cup B)$ then by definition there is some $x \in A \cup B$ such that $f(x) = y$. Now $x \in A$ or $x \in B$ and if $x \in A$ then $y = f(x) \in f(A) \subset f(A) \cup f(B)$. and if $x \in B$ then $y = f(x) \in f(B) \subset f(A) \cup f(B)$ so in either case $y \in f(A) \cup f(B)$ and thus,
\[ f(A \cup B) \subset f(A) \cup f(B) \]
Conversely, since $A, B \subset A \cup B$ we see that $f(A), f(B) \subset f(A \cup B)$ and therefore,
\[ f(A) \cup f(B) \subset  f(A \cup B) \]
because a set that contains two subsets also contains their union. Therefore,
\[ f(A \cup B) = f(A) \cup f(B) \]

\subsubsection*{(c)}

A good example is when $f(A) = f(B)$ is some nonempty set but $A$ and $B$ are disjoint ($A \cap B = \empty$). For example, take $f : \RR \to \RR$ given by $x \mapsto x^2$. Then let $A = \{ + 1 \}$ and $B = \{ - 1 \}$. Then $f(A) = f(B) = \{ + 1 \}$ which implies that $f(A) \cap f(B) = \{ + 1 \}$. However, $A \cap B = \empty$ so $f(A \cap B) = \empty$ so $f(A \cap B)$ and $f(A) \cap f(B)$ are not equal.

\section*{4}

Let $f : X \to Y$ be a function and $A, B \subset Y$.

\subsubsection*{(a)}

Take $x \in f^{-1}(A \cap B)$ then by definition $f(x) \in A \cap B$ so $f(x) \in A$ and $f(x) \in B$. This means that $x \in f^{-1}(A)$ and $x \in f^{-1}(B)$ so $x \in f^{-1}(A) \cap f^{-1}(B)$ proving that,
\[ f^{-1}(A \cap B) \subset f^{-1}(A) \cap f^{-1}(B) \]
Conversely, if $x \in f^{-1}(A) \cap f^{-1}(B)$ then $x \in f^{-1}(A)$ and $x \in f^{-1}(B)$ so $f(x) \in A$ and $f(x) \in B$ so $f(x) \in A \cap B$ meaning that $x \in f^{-1}(A \cap B)$. Therefore,
\[ f^{-1}(A \cap B) = f^{-1}(A) \cap f^{-1}(B) \]

\subsubsection*{(b)}

Take $x \in f^{-1}(A \cup B)$ then by definition $f(x) \in A \cup B$ so $f(x) \in A$ or $f(x) \in B$. This means that $x \in f^{-1}(A)$ or $x \in f^{-1}(B)$ so $x \in f^{-1}(A) \cup f^{-1}(B)$ proving that,
\[ f^{-1}(A \cup B) \subset f^{-1}(A) \cup f^{-1}(B) \]
Conversely, if $x \in f^{-1}(A) \cup f^{-1}(B)$ then $x \in f^{-1}(A)$ or $x \in f^{-1}(B)$ so $f(x) \in A$ or $f(x) \in B$ so $f(x) \in A \cup B$ meaning that $x \in f^{-1}(A \cup B)$. Therefore,
\[ f^{-1}(A \cup B) = f^{-1}(A) \cup f^{-1}(B) \]

\end{document}