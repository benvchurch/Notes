\documentclass[12pt]{article}
\usepackage[margin=2.5cm]{geometry}
\usepackage{amsmath,amsthm,amssymb,graphicx,float}
\usepackage{mdframed}
\usepackage{pgfplots}
\usepackage{comment}
\usepgfplotslibrary{fillbetween}
\pgfplotsset{compat=1.15}
\usepackage[labelsep=space]{caption}
\usepackage{float}
\usepackage{wrapfig}
\usepackage{tikz-cd}
\usepackage{tikz}
\usepackage{enumitem}
\setlist[enumerate]{leftmargin=*}
\theoremstyle{definition}
	\newmdtheoremenv{prob}{Problem}
\theoremstyle{definition}
	\newtheorem*{soln}{Solution}
\newcommand{\N}{\mathbb{N}}
\newcommand{\Z}{\mathbb{Z}}
\newcommand{\Q}{\mathbb{Q}}
\newcommand{\R}{\mathbb{R}}
\newcommand{\C}{\mathbb{C}}
\newcommand{\F}{\mathbb{F}}
\let\i\relax
\newcommand{\i}{\mathbf{i}}
\let\j\relax
\newcommand{\j}{\mathbf{j}}
\newcommand{\T}{\mathbf{T}}
\let\r\relax
\newcommand{\r}{\mathbf{r}}
\let\k\relax
\newcommand{\k}{\mathbf{k}}
\newcommand{\Ker}{\operatorname{Ker}}
\let\Im\relax
\newcommand{\Im}{\operatorname{Im}}
\newcommand{\Coker}{\operatorname{Coker}}
\newcommand{\Ext}{\operatorname{Ext}}
\newcommand{\Hom}{\operatorname{Hom}}
\newcommand{\Span}{\operatorname{span}}
\newcommand{\Null}{\operatorname{null}}
\newcommand{\range}{\operatorname{range}}

\begin{document}

\title{Math 56: Proofs and Modern Mathematics\\ Homework 9 Solutions}
\author{Naomi Kraushar and Ben Church}
\maketitle


\begin{prob}[Abbott, Exercise 2.3.11: Ces\`{a}ro means]
\begin{enumerate}[label=(\alph*)]
\item Show that if $(x_n)$ is a convergent sequence then the sequence given by the averages
\[y_n=\frac{x_1+x_2+\dots+x_n}{n}\]
also converges to the same limit.

\item Given an example to show that it is possible for the sequence $(y_n)$ of averages to converge even if $(x_n)$ does not.
\end{enumerate}
\end{prob}

\begin{soln}
\begin{enumerate}[label=(\alph*)]
\item Let $x=\lim_{n\to\infty} x_n$. Fix arbitrary $\varepsilon>0$. By definition, there exists some $N_\varepsilon\in\N$ such that $n\geq N_\varepsilon$ implies $|x_n-x|<\varepsilon/2$. Nnow consider the sequence of averages for $n\geq N_\varepsilon$: we have
\begin{align*}
|y_n-x| &= \left|\frac{\sum_{k=1}^n x_k}{n}-x\right| \tag{definition of the average $y_n$}\\
&= \left|\frac{\sum_{k=1}^n (x_k-x)}{n}\right| \tag{moving $x$ inside the sum}\\
&\leq \frac{1}{n}\sum_{k=1}^{N_\varepsilon} |x_k-x| +\frac{1}{n} \sum_{k=N_\varepsilon+1}^n |x_k-x| \tag{using the triangle inequality and taking $1/n$ out of the sums.}
\end{align*}
The first sum is finite, and depends only on $N_\varepsilon$, and hence on $\varepsilon$, so let $S_\varepsilon=\sum_{k=1}^{N_\varepsilon} |x_k-x|$. Every term in the second sum is less than $\varepsilon/2$, so we have
\begin{align*}
|y_n-x| &\leq \frac{1}{n}\sum_{k=1}^{N_\varepsilon} |x_k-x| +\frac{1}{n} \sum_{k=N_\varepsilon+1}^n |x_k-x| \\
&< \frac{S_\varepsilon}{n}+\frac{\varepsilon(n-N_\varepsilon+1)}{2n} \\
&\leq \frac{S_{\varepsilon}}{n}+\frac{\varepsilon}{2} \tag{since $n\geq N_\varepsilon$, so $n-N_\varepsilon+1\leq n$.}
\end{align*}
Recall that $S_\varepsilon$ depends on $\varepsilon$, not $n$, so we can now choose $n$ so large that $\frac{S_\varepsilon}{n}<\frac{\varepsilon}{2}$, which then gives $|y_n-x|<\varepsilon$. Hence the sequence of averages has the same limit as the original sequence.

\item Let $x_n=(-1)^n$, so odd terms are $-1$ and even terms are $1$. The sum of the first $n$ terms is $0$ if $n$ is even, and $-1$ if $n$ is odd. This gives us the sequence of averages
\[y_n = \begin{cases} 0 &\text{if $n$ is even} \\ -1/n &\text{if $n$ is odd}\end{cases}\]
Then for any $\varepsilon>0$, if we choose $N$ sufficiently large so that $1/N<\varepsilon$, then $|y_n|\leq 1/N<\varepsilon$ for all $n\geq N$, so $y_n\to 0$. Hence it is possible for a divergent sequence to have its sequence of averages converge.
\end{enumerate}
\end{soln}

\break

\begin{prob}[Abbott, Exercise 2.6.3]
If $(x_n)$ and $(y_n)$ are Cauchy sequences then one easy way to prove that $(x_n+y_n)$ is Cauchy is to use the Cauchy criterion. By Theorem 2.6.4, $(x_n)$ and $(y_n)$ must be convergent, and the Algebraic Limit Theorem implies that $(x_n+y_n)$ is convergent, hence Cauchy.
\begin{enumerate}[label=(\alph*)]
\item Give a direct argument that $(x_n+y_n)$ is a Cauchy sequence that does not use the Cauchy criterion or the Algebraic Limit Theorem.

\item Do the same for the product $(x_ny_n)$.
\end{enumerate}
\end{prob}



\begin{soln}
\begin{enumerate}[label=(\alph*)]
\item We have that $(x_n)$ and $(y_n)$ are Cauchy sequence, so, by definition, there exists $N_1\in \N$ such that $m,n\geq N_1$ implies $|x_n-x_m|<\varepsilon/2$ and there exists $N_2\in\N$ such that $m,n\geq N_2$ implies $|y_n-y_m|<\varepsilon/2$. Let $N=\max\{N_1,N_2\}$, so that for $m,n\geq N$, we have
\begin{align*}
|(x_n+y_n)-(x_m+y_m)| &= |(x_n-x_m)+(y_n-y_m)|\\
&\leq |x_n-x_m|+|y_n-y_m| \tag{by the triangle inequality}\\
&<\frac{\varepsilon}{2}+\frac{\varepsilon}{2}=\varepsilon.
\end{align*}
Hence $(x_n+y_n)$ is a Cauchy sequence.

\item For this part of the problem, we use Lemma 2.6.3: every Cauchy sequence is bounded. Hence there exists some $X,Y\in \R$ such that $|x_n|\leq X$ and $|y_n|\leq Y$ for all $n\in\N$. By definition, there exists $N_1\in\N$ such that $m,n\geq N_1$ implies $|x_n-x_m|<\varepsilon/2Y$ and there exists $N_2\in\N$ such that $n\geq N_2$ implies $|y_n-y_m|<\varepsilon/2X$. Let $N=\max\{N_1,N_2\}$, so that for $m,n\geq N$, we have
\begin{align*}
|x_ny_n-x_my_m| &= |x_ny_n-x_my_n+x_my_n-x_my_m|\\
&\leq |x_ny_n-x_my_n|+|x_my_n-x_my_m| \tag{by the triangle inequality}\\
&= |x_n-x_m| |y_n| + |x_m| |y_n-y_m| \\
&< \left(\frac{\varepsilon}{2Y}\right)Y + X\left(\frac{\varepsilon}{2X}\right)\\
&= \frac{\varepsilon}{2}+\frac{\varepsilon}{2}=\varepsilon.
\end{align*}
Hence $(x_ny_n)$ is a Cauchy sequence.
\end{enumerate}
\end{soln}

\break

\begin{prob}[From the writing assignment]
Prove that if all the elements $b_n\geq 0$, then
\[\sum_{n=1}^\infty b_n=\sup\left\{\sum_{i\in A}b_i:A\subset\N, A \text{ finite}\right\},\]
where the supremum is taken over all finite subsets of $\N=\{1,2,3,\dots\}$.
\end{prob}

\begin{soln}
Let $(s_N)$ be the sequence of partial sums, i.e. $s_N=\sum_{n=1}^N b_n$. Let $X$ be the set of all partial sums, i.e. $X=\{s_N\}_{N\in\N}$ and let $Y=\left\{\sum_{i\in A}b_i:A\subset\N, A \text{ finite}\right\}$ be the set in the question. We will prove that $\sup X=\sup Y$. First, we observe that $\{1,2,\dots,N\}$ is a finite subset of $\N$, so that
\[s_N = \sum_{i\in\{1,\dots,N\}} b_i\]
is in $Y$. Hence every partial sum is in the set $Y$, so $X\subset Y$, which means that $\sup X\leq \sup Y$.

Next, let $A$ be an arbitrary finite subset of $\N$. Since $A$ is finite, it has a maximal element $N$. Since $b_i\geq 0$ for all $i\in\N$, we have
\[\sum_{i\in A} b_i\leq \sum_{i\in A}b_i+\sum_{\substack{i\leq N\\i\notin A}} b_i = \sum_{i=1}^N b_i = s_N.\]
Hence every element in $Y$ is bounded above by some partial sum, which is an element in $X$, so $\sup Y\leq \sup X$. Since we also have $\sup X\leq \sup Y$, this means that $\sup X=\sup Y$.

Finally, we want to show that $\sum_{n=1}^\infty b_n=\sup X$. By definition, $\sum_{n=1}^\infty b_n=\lim_{N\to\infty} s_N$. Since $b_n\geq 0$ for all $n$, the sequence of partial sums is a monotone increasing sequence. From the proof of Theorem 2.4.2 (Monotone Convergence Theorem), if this sequence is bounded above, it converges to its supremum, which is $\sup X$. Hence in the case where the series converges, we have $\sum_{n=1}^\infty b_n =\sup X=\sup Y$ as required. If the sequence $(s_N)$ is not bounded above, the supremum $\sup X=\infty$, and also $\lim s_N=\sum_{n=1}^\infty b_n=\infty$, so we again have $\sum_{n=1}^\infty =\sup X=\sup Y (=\infty)$, as required.
\end{soln}


\break

\begin{prob}
Recall that $A$ is closed if whenever $(a_n)$ is a sequence in $A$ with $x=\lim a_n$ existing in $\R$, in fact $x\in A$.  Show directly from the definition of being closed that
\begin{enumerate}[label=(\alph*)]
\item Any intersection of any collection of closed sets is closed.

\item The finite union of closed sets is closed. 

{\small (Hint: if $(a_n)$ is a sequence in $C_1\cup\dots\cup C_N$, show that there is a subsequence $(a_{n_k})$ and an index $j$ such that $a_{n_k}\in C_j$ for all $k$.)}
\end{enumerate}
\end{prob}

\begin{soln}
\begin{enumerate}[label=(\alph*)]
\item Let $\{C_\alpha\}_{\alpha\in A}$ be an arbitrary collection of closed sets in $\R$, with $A$ some indexing set, and suppose that $(a_n)$ is a sequence in $\bigcap_\alpha C_\alpha$ for all $\alpha\in A$, with $x=\lim a_n$ existing in $\R$. Since $a_n\in \bigcap_{\alpha\in A} C_\alpha$, we have $a_n\in C_\alpha$ for all $\alpha\in A$. Since every $C_\alpha$ is closed, we have $x\in C_\alpha$ for all $\alpha\in A$. Hence $x\in \bigcap_{\alpha\in A} C_\alpha$. Hence the intersection of closed sets is closed.

\item Let $C_1,\dots,C_N$ be a finite collection of closed sets in $\R$, and suppose that $a_n$ is a sequence in $\bigcup_{j=1}^N C_j$ for all $n$, with $x=\lim a_n$ existing in $\R$. We now use the hint: suppose that there is no subsequence $(a_{n_k})$ in $C_j$ for some $j$. This means that there are only finitely many terms of the original sequence in $C_j$. This means that if there is no subsequence of $(a_n)$ in any $C_j$, then each $C_j$ only has finitely many terms of the sequence, so the union $\bigcup_{j=1}^N C_j$ has only finitely many terms in the sequence, which contradicts the assumption that $(a_n)$ is a sequence in $\bigcup_{j=1}^N C_j$. Hence there must be some index $j$ such that $C_j$ contains some subsequence $(a_{n_k})$ of the sequence $(a_n)$. Since subsequences of a convergent sequence converge to the same limit, $a_{n_k}$ converges to $x$, and since $C_j$ is closed, this means that $x\in C_j\subset \bigcup_{j=1}^N C_j$. Hence the union does contain the limit, so finite unions of closed sets are closed.
\end{enumerate}
\end{soln}

\break

\begin{prob}
[Abbott, Exercise 3.3.3] Suppose that $K \subset \R$ is closed and bounded. Show that every sequence $(a_n)$ in $K$ has a subsequence converging to a point in $K$, i.e. that $K$ is compact.
\end{prob}

\begin{soln}
Because $K$ is bounded and $a_n \in K$ this means that the sequence $a_n$ is bounded. Therefore, by the Bolzano-Weierstrass theorem, there exists a subsequence $a_{n(j)}$ that coverges to some $a \in \mathbb{R}$. Therefore it suffices to show that $a \in K$. However, $K$ is closed and each $a_{n(j)} \in K$ which implies that the limit $a \in K$ by Theorem 3.2.5. 
\end{soln}

\end{document}