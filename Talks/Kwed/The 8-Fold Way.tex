\documentclass[12pt]{article}
\usepackage{import}
\import{"../../Algebraic Geometry/"}{AlgGeoCommands}

\begin{document}

\section{The Particle Zoo}

You are a particle physics theorist in 1960. For the past decade your colleges have been building accelerators and detectors to find exotic high-energy particles. They have been a little sucessful. Each year new particles are being discovered at a dizzying rate. Used to mocking the chemists 

Baryons:


\renewcommand{\arraystretch}{1.30}

\begin{center}
\begin{tabular}{||c | c c c c ||} 
 \hline
 & spin & M (MeV) & Q & S \\ [0.5ex] 
 \hline\hline
$p$ & $\tfrac{1}{2}$ & 938 & 1 & 0 \\ 
 \hline
$n$ & $\tfrac{1}{2}$ & 940 & 0 & 0 \\ 
 \hline
$\Lambda^0$ & $\tfrac{1}{2}$ & 1116 & 0 & -1 \\ 
 \hline
$\Sigma^+$ & $\tfrac{1}{2}$ & 1189 & 1 & -1 \\ 
 \hline
$\Sigma^0$ & $\tfrac{1}{2}$ & 1193 & 0 & -1 \\ 
 \hline
$\Sigma^{-}$ & $\tfrac{1}{2}$ & 1189 & -1 & -1 \\ 
 \hline
$\Xi^0$ & $\tfrac{1}{2}$ & 1315 & 0 & -2 \\ 
 \hline
$\Xi^{-}$ & $\tfrac{1}{2}$ & 1322 & -1 & -2 \\ 
 \hline
\end{tabular}
\end{center}

You notice that not all possible reactions between these particles is allowed. For example,
\[ \Lambda^0 + \Sigma^+ \to \Xi^0 + p \]
is possible but you don't see any reactions like,
\[ p + n \to \Sigma^+ + \Xi^0 \]
Following the general principle of quantum mechanics ``anything that is allowed will happen with some probability'' you conjecture that there is some additional conserved quantum number associated to these new particles which forbids certain processes if the number is not conserved. Since it is very odd you descide to call it ``strangeness'' $S$. Naturally you set the strangness of the proton and neutron to $0$ and by painstakeingly tracking which processes occur in cloud chambers and which do not you figure out how to assign strangeness numbers so that the processes do not change strangness\footnote{Not all processes preserve strangeness. The rule of thumb you would find is that slow decay processes (radioactive decay though the weak force) do not preserve strangness but ``fast'' reactions in particle collisons do preserve the total strangness}.
\\
But your colleges are hard at work and the find a whole slew of more exotic particles with higher spin. The first are spin = 3/2 excited versions of the proton and neutron called $\Delta$ baryons. Let's tabulate the spin = 3/2 baryons discovered.

\begin{center}
\begin{tabular}{||c | c c c c ||} 
 \hline
 & spin & M (MeV) & Q & S \\ [0.5ex] 
 \hline\hline
$\Delta^{++}$ & $\tfrac{3}{2}$ & 1232 & 2 & 0 \\ 
 \hline
$\Delta^{+}$ & $\tfrac{3}{2}$ & 1232 & 1 & 0 \\ 
 \hline
$\Delta^{0}$ & $\tfrac{3}{2}$ & 1232 & 0 & 0 \\ 
 \hline
$\Delta^{-}$ & $\tfrac{3}{2}$ & 1232 & -1 & 0 \\ 
 \hline
$\Sigma^{* +}$ & $\tfrac{3}{2}$ & 1383 & 1 & -1 \\ 
 \hline
$\Sigma^{* 0}$ & $\tfrac{3}{2}$ & 1383 & 0 & -1 \\ 
 \hline
$\Sigma^{* -}$ & $\tfrac{3}{2}$ & 1387 & -1 & -1 \\ 
 \hline
$\Xi^{*0}$ & $\tfrac{3}{2}$ & 1532 & 0 & -2 \\ 
 \hline
$\Xi^{*-}$ & $\tfrac{3}{2}$ & 1535 & -1 & -2 \\ 
 \hline
(?) $\Omega^{-}$ & $\tfrac{3}{2}$ & 1672 & -1 & -3 \\ 
 \hline
\end{tabular}
\end{center}


PLOT THESE PARTICLES

\section{Weights}

\newcommand{\g}{\mathfrak{g}}
\newcommand{\h}{\mathfrak{h}}
\newcommand{\ad}{\mathrm{ad}}

Let $(G, T)$ be a pair of a Lie group (or algberaic group) and a maximal torus. Then the character lattice $X(T)$ is the abelian group of characters. Often we work inside $X(T)_{\RR}$. 

\begin{defn}
The action $T \acts G$ by conjugation induces an action $T \acts \g$ and hence a weight decomposition,
\[ \g = \g_0 \oplus \bigoplus_{\alpha \in \Phi} \g_\alpha \]
where $\Phi \subset X(T)$ is the set of \textit{roots} i.e. nonzero characters of $T$ such that,
\[ \g_\alpha = \{ X \in \g \mid \forall t \in T : t \cdot X = \alpha(t) X \} \]
is nonempty.
\end{defn}

This is a special case of the general theory of weights. 

\begin{defn}
A \textit{weight} of a Lie algebra over $k$ is a $k$-linear map $\lambda \in \g^*$ such that $\lambda([x,y]) = 0$. 
\end{defn}

\begin{rmk}
A weight is just a morphism of Lie algebras to the 1-dimensional abelian Lie algebra. Indeed, for any algebra a character should be an algebra map to the corresponding ``trivial'' 1-dimensional such algebra. For example, for associative algebras this recovers a multiplicative character.
\end{rmk}

\begin{rmk}
Clearly any weight of $\g$ factors through $\g \to \g / [\g, \g]$ which is abelian. Therefore, it is most natural to consider weights only for abelian Lie algebras. This is why we pass to a Cartan subalgebra (corresponding to a maximal torus). 
\end{rmk}

For now fix a pair $(\g, \h)$ of a Lie algebra $\g$ and a Cartan subalgebra $\h \subset \g$.

\begin{defn}
Let $\rho : \g \to \End{V}$ be a representation of a Lie algebra $\g$ on a vectorspace $V$. Let $\lambda$ be a weight of $\h$. Then the \textit{weight space} of $V$ with weight $\lambda$ is the subspace,
\[ V_\lambda := \{ v \in V \mid \forall H \in \h : \rho(H) \cdot v = \lambda(H) \cdot v \} \]
A \textit{weight} of $\rho$ is a nonzero weight $\lambda$ of $\h$ such that $V_{\lambda}$ is nonempty.
\end{defn}

For example, if we consdier the adjoint representation $\ad : \g \to \End{\g}$ then the weights of $\ad$ are exactly the roots under the identification of $X(T)_k = \h^*$ if $(\g, \h)$ is the Lie algebra of $(G, T)$. Indeed, it means there is a root vector $X \in \g$ such that,
\[ \forall H \in \h : [H, X] = \ad_{H} \cdot X = \alpha(H) X \]
For other representations of $\g$ we can think of weights as living in the same space as the root lattice. 

\begin{rmk}
Moreover the roots act on the weights of any representation as follows. For $X \in \g_\alpha$ meaning a root vector corresponding to the root $\alpha \in \Phi$ and $v \in V_\lambda$ meaning a weight vector corresponding to the weight $\lambda$ then consider, for any $H \in \h$,
\[ \rho(H)(\rho(X) \cdot v) = \rho(X) (\rho(H) \cdot v) + [\rho(H), \rho(X)] \cdot v = \lambda(H) (\rho(X) \cdot v) + \alpha(H) \rho(X) \cdot v = (\lambda + \alpha)(H) \, \rho(X) \cdot v \]
and therefore $\rho(X) \cdot v$ is either zero or a weight vector of weight $\lambda + \alpha$.
\end{rmk}


\begin{lemma}
$[\g_\alpha, \g_\beta] \subset \g_{\alpha + \beta}$
\end{lemma}

\begin{proof}
Choose $X \in \g_\alpha$ and $Y \in \g_{\beta}$. For any $H \in \h$,
\[ [H, [X,Y]] = [[H,X],Y] + [X, [H,Y]] = \alpha(H) [X,Y] + \beta(H) [X, Y] \]
and therefore we see that $[X,Y]$ is a root vector for $\alpha + \beta$.
\end{proof}

\subsection{The Weyl Group Action}

Roots and weights are computed with respect to a fixed Cartan $\h$ or maximal torus $T$. However, the answer should not really depend on the choice. Indeed it does not because any two Cartans or maximal tori are conjugate (at least over $\CC$) so we can define an abstract isomorphism of roots systems between the two induced by this conjugation. However this suggests extra symmetry since there are conjugation actions that fix the maximal torus. Let,
\[ W = N_G(T)/T \]
be the Weyl group. Then we claim that $W$ acts on the roots and weights. Indeed for $w \in W$ and $\alpha \in \Phi$ consider $w \cdot \alpha$ meaning the character $(w \cdot \alpha)(t) = \alpha(w^{-1} t w)$ which is well-defined since $T$ is abelian. However, if $X \in \g_{\alpha}$ is a root vector then for any $t \in T$ we have,
\[ t \cdot (w \cdot X) = w (w^{-1} t w) \cdot X = w \cdot \alpha(w^{-1} t w) X = (w \cdot \alpha)(t) w \cdot X \]
Thus $w : \g_{\alpha} \iso \g_{w \cdot \alpha}$ gives an isomorphism permuting the roots and the root spaces. Similarly, if $\lambda$ is a weight of $\rho$ and $v \in V_\lambda$ is a weight vector then for any $H \in \h$,
\[ \rho(H) \cdot (\rho(w) \cdot v) = \rho(w) \rho(w^{-1} H w) \cdot v = \rho(w) \lambda(w^{-1} H w) v = (w \cdot \lambda)(H) \, \rho(w) \cdot v \]
and thus $w : V_\lambda \iso V_{w \cdot \lambda}$. Therefore when drawing the root system we should respect the symmetry under the Weyl group. The best way to do this is to choose a Weyl-invariant inner product on $X(T)_{\RR}$.  

\subsection{Killing Form}

\begin{defn}
The \textit{Killing form} of $\g$ is the map,
\[ K : \g \times \g \to \CC \]
defined by $K(X,Y) = \tr{(\ad_X \circ \ad_Y)}$. This is a bilinear form. 
\end{defn}

\begin{defn}
An involution $\theta : \g \to \g$ of a real Lie algebra is called a \textit{Cartan} involution if $B_\theta(X,Y) := -K(X, \theta Y)$ is positive-definite.
\end{defn}

\begin{prop}
If $\g$ is a semisimple real Lie algebra then the following are equivalent,
\begin{enumerate}
\item $\id : \g \to \g$ is a Cartan involution
\item $K$ is negative definite
\item $\g$ is the Lie algebra of a compact semisimple Lie group.
\end{enumerate}
\end{prop}

\begin{cor}
If $\g = (\g_0)_{\RR}$ then complex conjugation on $\g$ is a Cartan involution of $\g$ if and only if $\g_0$ is the Lie algebra of a compact Lie group.
\end{cor}

\begin{theorem}[Cartan]
$\g$ is semisimple if and only if $K$ is nondegenerate.
\end{theorem}

\begin{cor}
If $G$ is compact and $Z(\g) = 0$ then $\g$ is semisimple.
\end{cor}
\renewcommand{\t}{\mathfrak{t}}
\renewcommand{\k}{\mathfrak{k}}

Given a Cartan involution $\theta$ we get a Cartan pair $\g = \k \oplus \p$ which are the $+1$ and $-1$ eigenspaces. Since $\theta$ is a Lie algebra homomorphism we see that,
\[ [\k, \k] \subset \k \quad [\k, \p] \subset \p \quad [\p, \p] \subset \k \]
so $\k$ is a Lie subalgebra while any subalgebra of $\p$ is abelian. The data of $(\k, \p)$ determines $\theta$. 
\bigskip\\
In particular $K$ is negative definite on $\k$ and positive definitie on $\p$ and $\g = \k \oplus \p$ is an othogonal decomposition with respect to $K$. 
\bigskip\\
In the case that $\g = (\g_0)_{\CC}$ then a Cartan subalgebra is given by $\h = \t \oplus i \t$ where $\t$ is the Lie algbra of a maximal torus $T$ of the compact group corresponding to $\g_0$. Then there is a natural identification $X(T)_{\RR} = \t^*$. We choose to instead make the identification $X(T)_{\RR} = i \t^*$ and the Killing form is positive-definite on $i \t$ hence defines an isomorphism $i \t \cong i \t^*$ and therefore gives an inner produce on $X(T)_{\RR}$ which is Weyl invariant. This is where the metric on the root lattice arises from. 
\bigskip\\
In the theory of algebraic groups this metric is not defined so simply since there is not a good notion of compact real form. Instead we consider for each root $a \in \Phi(G, T)$ a dual coroot $a^{\vee} \in \Phi^{\vee}(G, T)$ which is the unique cocharacter $a^{\vee} : \Gm \to S_a$ where $S_a := T \cap D(Z_G(T_a))$ such that $a \circ a^\vee$ is $z \mapsto z^2$ as a map $\Gm \to \Gm$. This associatation between roots and coroots replaces the metric provided by the Killing form as a way to take inner products between roots. 


\subsection{$\SU(2)$}

\newcommand{\su}{\mathfrak{su}}

If we choose $T = U(1) \subset \SU(2)$ as the diagonal torus,
\[ T = \left\{ 
\begin{pmatrix}
e^{i \theta} & 0
\\
0 & e^{-i\theta} 
\end{pmatrix} \right\} \]
Then the weight space decomposition is familar from physics. Indeed, $\h$ is $1$-dimensional generated by the matrix,
\[ H = \begin{pmatrix}
1 & 0 
\\
0 & -1 
\end{pmatrix} \]
which is the Pauli matrix $\sigma_z$. In terms of angular momentum $J_z = \tfrac{1}{2} \sigma_z$ which is why there will be some factors of $2$ floating around compared to the physics literature. Indeed, usually physcists parametrize this digonal maximal torus in terms of rotations around the $z$-axis under the standard action on $\RR^3$ via conjugation on traceless Hermitian of determinant 1 (equivalently on unit imaginary quaterions) as,
\[ T = \left\{ 
\begin{pmatrix}
e^{\frac{i \theta_z}{2}} & 0
\\
0 & e^{-\frac{i \theta_z}{2}} 
\end{pmatrix} \right\} \]
Anyway the weight spaces are exactly the eigenspaces of $\rho(H)$ and since this is an action of $U(1)$ the eigenvalues must be integers (hence the eigenvalues of $J_z$ are half-integers). This is exactly what we do when we describe the states of $J$ multiplets in terms of their $J_z$ eigenvalue $m$. The simulatenous diagonalization of $J$ and $J_z$ is nothing other than the weight decomposition. 
\bigskip\\
Now consider the adjoint action $\SU(2) \to \GL(\su(2)_{\CC})$. We get,
\begin{align*}
\begin{pmatrix}
e^{i \theta} & 0
\\
0 & e^{-i\theta} 
\end{pmatrix}
\begin{pmatrix}
a & b
\\
c & d
\end{pmatrix}
\begin{pmatrix}
e^{-i \theta} & 0
\\
0 & e^{i\theta} 
\end{pmatrix}
= 
\begin{pmatrix}
a & b e^{2i \theta}
\\
c e^{-2i \theta} & d
\end{pmatrix}
\end{align*} 
therefore the roots are $\pm 2$ with root vectors,
\[ X = \begin{pmatrix}
0 & 1
\\
0 & 0 
\end{pmatrix} 
\quad \quad 
Y = \begin{pmatrix}
0 & 0 
\\
1 & 0
\end{pmatrix} \]
these are exactly the rasing and lowering operators. Raising and lowering $J_z$ corresponds exactly to shifting the weight by $+2$ or $-2$ which is how we saw the root vectors act on the weight spaces. Note that $X,Y$ live in $\su(2)_{\CC}$ not in $\su(2)$ corresponding to the fact that in physics we define them as $L_x + i L_y$ and $L_x - i L_y$ which require a factor of $i$. The root vectors act additivley on the weights of any representation. This shows that the raising and lower operators shift the weight up and down by $2$ (in the physicists notation by $1$ i.e. a whole integer rather than by a half-integer while the weights may be half-integers).
\subsection{$\SU(3)$}

The maximal torus $T \subset G = \SU(3)$ is given by matrices,
\[ \begin{pmatrix}
e^{i \theta_1} 
\\
& e^{i \theta_2}
\\
& & e^{i \theta_3}
\end{pmatrix} \]
with determinant $1$. Therefore we take the Lie algebras $(\g, \h)$ where $\g$ is the Lie algebra of 3 x 3 anti-Hermtian matrices and $\h$ is the imaginary diagonal matrices. 
\bigskip\\
There are three standard characters that produce $e^{i \theta_i}$ called $L_i$. It is easier to work with the complexified Lie group $G_{\CC} = \SL_3$ in which case we get the (complex) maximal torus,
\[ \begin{pmatrix}
t_1
\\
& t_2
\\
& & t_3
\end{pmatrix} \]
where $t_1 t_2 t_3 = 1$. Then $L_i$ sends this matrix to $t_i$ so we see that $L_1 + L_2 + L_3 = 0$. Therefore, these define three points in the two-dimensional space $X(T)_{\RR}$ with barycenter at the origin. Thus it is conventional to represent their span as a triangular lattice. These are the fundamental weights, the weights of the fundamental representation. Now we compute the roots. 
Now,
\[ 
\begin{pmatrix}
t_1 & 0 & 0
\\
0 & t_2 & 0
\\
0 & 0 & t_3
\end{pmatrix}
\begin{pmatrix}
a & b & c
\\
d & e & f
\\
g & h & i
\end{pmatrix}
\begin{pmatrix}
t_1 & 0 & 0
\\
0 & t_2 & 0
\\
0 & 0 & t_3
\end{pmatrix} ^{-1}
= 
\begin{pmatrix}
a & t_1 t_2^{-1} b & t_1 t_3^{-1} c
\\
t_2 t_1^{-1} d & e & t_2 t_3^{-1} f
\\
t_3 t_1^{-1} g & t_3 t_2^{-1} h & i
\end{pmatrix} \]
Therefore, the roots and corresponding root vectors are,
\begin{align*}
L_1 - L_2  & \quad \quad E_{12} = 
\begin{pmatrix}
0 & 1 & 0
\\
0 & 0 & 0
\\
0 & 0 & 0
\end{pmatrix}
\\
L_2 - L_3  & \quad \quad E_{23} = 
\begin{pmatrix}
0 & 0 & 0
\\
0 & 0 & 1
\\
0 & 0 & 0
\end{pmatrix}
\\
L_1 - L_3  & \quad \quad E_{13} = 
\begin{pmatrix}
0 & 0 & 1
\\
0 & 0 & 0
\\
0 & 0 & 0
\end{pmatrix}
\\
L_2 - L_1  & \quad \quad E_{21} = 
\begin{pmatrix}
0 & 0 & 0
\\
1 & 0 & 0
\\
0 & 0 & 0
\end{pmatrix}
\\
L_3 - L_2 & \quad \quad E_{32} = 
\begin{pmatrix}
0 & 0 & 0
\\
0 & 0 & 0
\\
0 & 1 & 0
\end{pmatrix}
\\
L_3 - L_1  & \quad \quad E_{31} = 
\begin{pmatrix}
0 & 0 & 0
\\
0 & 0 & 0
\\
1 & 0 & 0
\end{pmatrix}
\end{align*}
where $E_{ij}$ has a single $1$ in the $(i,j)$ entry and zeros elsewhere.

\section{Appearence of the adjoint representation for mesons}

\newcommand{\gl}{\mathfrak{gl}}
\newcommand{\inner}[2]{\left< #1 , #2 \right>}

Let $\g$ be a Lie algebra and $V$ an faithful $\g$-representation meaning that the map,
\[ \g \to \End{V} \]
is injective. This gives a map of Lie algebras $\g \to \gl(V)$. Then the adjoint representation lives as a subrepresentation of $V^* \ot V$. Indeed, we simply take,
\[ \g \embed \End{V} \cong V^* \ot V \]
Indeed, the first is actually a map of $\g$-representations (or $U \g$-modules) where $\g$ is given the adjoint representation. Indeed, we need to check that,
\[ \rho(\ad_X(Y)) = X \cdot \rho(Y) \]
but by definition of the action of $\g$ on $\End{V}$ we have,
\[ X \cdot \varphi = [\rho(X), \varphi] \]
and thus we see that,
\[ \rho(\ad_X(Y)) = \rho([X,Y]) = [\rho(X), \rho(Y)] = X \cdot \rho(Y) \]
as expected. 
\bigskip\\
Furthermore, if $V$ is equipped with a $\g$-compatible Hermitian inner product (meaning $\inner{X \cdot v}{w} + \inner{v}{X \cdot w} = 0$ so -- if it exists -- the associated $G$-representation is unitary) then it induces an isomorphism of $\g$-representations $\overline{V} \to V^*$. Therefore, we also get a copy of $\mathbf{adj} \subset \overline{V} \ot V$. This is what physicists mean when they say the following. Let $\lambda_i$ be a collection of matrices with the specified commutation relations and $e_i$ the basis vectors of the space on which $\lambda_i$ acts. Let $\overline{e}_i$ be the complex conjugate basis. Then we get a singlet representation by choosing $\overline{e}_\alpha^* e_\alpha$ and we get an adjoint multiplet by choosing $s_j = \overline{e}_\alpha \lambda_j e_\alpha$. The first is looking at $\id \in \End{V}$ which is invariant under $\g$. The second is looking at the image of $\g \embed \End{V}$ where each $X \in \g$ is represented in terms of $\ol{V} \ot V$ via $\ol{e}_j \ot X_{ij} e_i$ where $X_{ij}$ is the matrix representation in the basis $e_i$. 

\section{The quark model}


Notice that we see the two shapes in the decomposition of $V_3^{\ot 3}$: triangle with 10 weight spaces and hexagon with 2-dimensional weight space at the origin in the spectrum of baryons. Actually, we are missing one weight space in the decouplet. Murray-Gel Man hypothesized the existence of $\Omega^{-}$ with charge $-1$ and strangness $-3$ to fill out the diagram. By looking at the linear relationship of masses he also predicted it should have a mass about $1600 MeV$. This was discovered three years later in 1964 at Brookhaven national laboratory. 
\bigskip\\
How could this correspondence to weight diagrams of $\SU(3)$ make sense? Suppose there was a particle $q$ called the quark and there are three types which we call up $(u)$, down $(d)$, and strange $(s)$ and these particles have nearly identical properties from the perspective of whatever interaction holds them together. Then superpositions of these three particles will also have similar properties. Therefore the group $\SU(3)$ acts on this particle in the fundamental representation where we choose to represent,
\[ u = \begin{pmatrix}
1 
\\
0
\\
0
\end{pmatrix}
\quad \quad
d = \begin{pmatrix}
0 
\\
1
\\
0
\end{pmatrix}
\quad \quad
s = \begin{pmatrix}
0 
\\
0
\\
1
\end{pmatrix} \]
Then suppose we had three of these particles glued together. This composite system transforms in the representation,
\[ V_3^{\ot 3} = D(3,0) \oplus D(1,1) \oplus D(1,1) \oplus D(0,0) \]
so inside here there are representations $D(3,0)$ and $D(1,1)$ whose weight spaces correspond to the different particles built out of these three quarks. But why should weight spaces correspond to different particles that we observe? Well to explain this we need that the three flavors of quark $u, d, s$ are not actually completely identical. If flavor $\SU(3)$ were a perfect symmetry, meaning that all processes act identically for the three types of quarks, then there would be no way to observe the differences between the particles corresponding to different weight vectors, we would have described it as different internal states of the same particle. For example, an electron can have spin up and spin down. From the mathematical perspective, the spin up electron and the spin down electron are two different particles and correspond to two different fields. However, they are interchanged by an exact $\SU(2)$ symmetry arising from the rotations of 3d space. Therefore, since these two particles behave identically up to a rotation, we think of spin up and spin down as being two different internal states of the electron instead of two different similar particles.
\bigskip\\
Here, the situation is similar with respect to how the particles are built, the $\SU(3)$ is an exact symmetry of the strong force, the force that binds the quarks together but it is not an exact symmetry when we include other possible interacitons. In particular, we see that the particles that should be in the same irrep are \textit{not} identical. They differ in mass and change as well as strangness. We can explain this by the symmetry exchanging the flavors not being an exact symmetry. Lets consider how the torus acts on this fundamental representation,
\[ \begin{pmatrix}
t_1 & 0 & 0
\\
0 & t_2 & 0
\\
0 & 0 & t_3 
\end{pmatrix}
\cdot e_i = t_i e_i 
\]
so the pure flavor states $u,d,s,$ are those for which $T$ acts by the characters $L_1, L_2, L_3$ respectively. Therefore, the three flavor states are the weight vectors corresponding to the fundamental weights $L_1, L_2, L_3$. Then for any representation $\SU(3) \acts V$ we can think of the weight vectors as states of definite number of quarks of each flavor where the weight $\lambda = a L_1 + b L_2 + c L_3$ is composed of $a$ up quarks, $b$ down quarks, and $c$ strange quarks. This isn't quite right because we have $L_1 + L_2 + L_3 = 0$ so really the characters $L_1, L_2, L_3$ are measuing the \textit{excess} of quarks of a certain type so we need to set $a + b + c = n$ where $n$ is the total number of quarks (not an invariant of the isomorphism type of the representation, we will see this for mesons) and then there is a unique way to write $\lambda = a L_1 + b L_2 + c L_3$ and $a,b,c$ then can be interpreted as the number of up, down, and strange quarks.
\bigskip\\
Then we can understand the properties of the particles in our tables. To reiterate, the fact that we observe these particles as not identical means that the flavor $\SU(3)$ is not an exact symmetry meaning that the $u,d,s$ states of the quark should have different properties. We can explain the table if the quarks have the following properties, 

\begin{center}
\begin{tabular}{||c | c c c c ||} 
 \hline
 & spin & M (MeV) & Q & S \\ [0.5ex] 
 \hline\hline
$u$ & $\tfrac{1}{2}$ & ~336\footnote{This is a big lie. Most of the mass of the particle comes from the strong interaction not the rest mass of the quarks (this is anyway tricky concept since quarks are confined so you need to decide on a reasonable renormalization point for the quark masses anyway) so this is what is called the ``constituent mass'' of each flavor of quark which approximately captures its contribution to the binding energy.} & $\tfrac{2}{3}$ & 0 \\ 
 \hline
$d$ & $\tfrac{1}{2}$ & ~340 & $-\tfrac{1}{3}$ & 0 \\ 
 \hline
$s$ & $\tfrac{1}{2}$ & ~486 & $\tfrac{2}{3}$ &  -1 \\ 
 \hline
\end{tabular}
\end{center}
Then the $U(1)$ EM charged operation is given by,
\[ \begin{pmatrix}
e^{2i \theta/3} & 0 & 0
\\
0 & e^{-i\theta/3} & 0
\\
0 & 0 & e^{2 i \theta/3}
\end{pmatrix} = e^{i \theta/3} \begin{pmatrix}
e^{i \theta/3} & 0 & 0
\\
0 & e^{-2i\theta/3} & 0
\\
0 & 0 & e^{i \theta/3}
\end{pmatrix} \]
but if a state has total charge $Q$ then this means the above matrix acts by $e^{i Q \theta}$. Hence this occurs if the state is a weight vector of weight $\lambda$ and,
\[ Q = \lambda   \begin{pmatrix}
e^{i \theta/3} & 0 & 0
\\
0 & e^{-2i\theta/3} & 0
\\
0 & 0 & e^{i \theta/3}
\end{pmatrix} + \frac{n}{3} = \frac{a - 2 b + c + n}{3} \quad \text{ where } \quad \lambda = a L_1 + b L_2 + c L_3  \]
This gives the weight spaces the correct charges on our diagram. Likewise, $S$ is just minus the nubmer of strange quarks so it just corresponds to $-c$ on our weight space diagram chosing as before $a + b + c = n$.
\bigskip\\
We can think of the extra constant shifts floating around in these formulas as showing that really we should look at representations of $\U(n)$ not $\SU(n)$ then the maximal torus is $3$-dimensional from the extra $\det : \U(n) \to S^1$ and but the roots are the same and lie in the plane since the new torus element is in the center. However, the weights can involve this extra direction. The weights for irreps of $\U(n)$ are those of $\SU(n)$ shifted up or down by a constant. In this weight space the number of each flavor, $u,d,s$ and $Q, S$ are linear functions of the weight coordinates.
\bigskip\\
But why don't we see two copies of the adjoint representation nor do we see a trivial representation. Also why is the $D(3,0)$ correspond to spin = 3/2 while the $D(1,1)$ correspons to spin = 1/2. We need to bring spin into the discussion and also the fact we have been sweeping under the rug that the quarks are identical particles.

\subsection{Schur-Weyl Duality}

Since the quarks are identical particles, it makes no sense to take about the composite state $v \ot v'$ vs $v' \ot v$ in $V^{\ot 2}$. This would be if we could label the two quarks and say the first is in state $v$ and the second is in state $v'$. But these particles are completely identical so we can't tell which one is in state $v$ and which one is in state $v'$. The only thing that makes sense to say is that there is one quark in state $v$ and one quark in state $v'$. What we should do is only consider states that respect this symmetry meaning the real space of states of the combined system of two quarks is not $V^{\ot 2}$ but $(V^{\ot 2})^{S_2} = \nSym{2}{V}$ and likewise the space of states of three quarks is $(V^{\ot 3})^{S_3} = \nSym{3}{V}$. But $\nSym{3}{V} = D(3,0)$ so now we're missing the adjoint!
\bigskip\\
To understand what's really going on we need to introduce spin. As a representation of $\SU(2) \times \SU(3)$ the quark is $S_{1/2} \ot V_3$ so we need to consider,
\[ W = ((S_{1/2} \ot V_3)^{\ot 3})^{S_3} \]
which is the representation of $\SU(2) \times \SU(3)$ of the composite system of three quarks. This looks nasty but luckily there is an intimate relationship between representation theory of $S_k$ and $\SU(n)$ given by Schur-Weyl duality:

\renewcommand{\SS}{\mathbb{S}}
\newcommand{\alt}{\mathrm{alt}}

\begin{defn}
Let $V$ be the fundamental representation of $\SU(k)$. Then as $S_n \times \SU(k)$ representations we have a decomposition into irreducibles,
\[ V^{\ot n} = \bigoplus_{\lambda \vdash n} V_\lambda \ot \SS_\lambda V \]
where $\lambda$ runs over the permutations of $\lambda$ (or equivalently Young diagrams) which index the irreps $V_\lambda$ of $S_k$ and $\SS_\lambda$ is the Schur functor,
\[ \SS_\lambda W = \Hom{S_n}{V_\lambda}{W^{\ot n}} \]
\end{defn}

\begin{example}
Let $\lambda \vdash n$ be $\lambda = (n)$ (the horizontal Young diagram) then $V_\lambda$ is trivial then $S_\lambda W = \Hom{S_n}{\CC}{W^{\ot n}} = \nSym{n}{W}$. Let $\lambda = (1, \dots, 1)$ the (vertical Young diagram) then $V_\lambda = \alt$ is the totally alternating sign character. Hence $\S_\lambda W = \Hom{S_n}{\alt}{W^{\ot n}} = \bigwedge^n W$ since swapping any two entries picks up a minus sign from the alternating character.
\end{example}

Now we can use this to understand the three-quark representation. First,
\[ S_{1/2}^{\ot 3} = V_{\square \square \square} \ot \nSym{3}{S_{1/2}} \oplus V_{\substack{\square \square \\ \square \,\,\,\,}} \ot \SS_{\substack{\square \square \\ \square \,\,\,\,}}(S_{1/2}) \oplus V_{\substack{\substack{\square \\ \square} \\ \square}} \oplus \wedge^3 S_{1/2} = S_{3/2} \oplus V_{\substack{\square \square \\ \square \,\,\,\,}} \ot S_{1/2}   \]
we can see that $\SS_{\substack{\square \square \\ \square \,\,\,\,}}(S_{1/2}) = S_{1/2}$ just by dimension counts. We know as $\SU(2)$-reps $S_{1/2}^{\ot 3} = S_{3/2} \oplus S_{1/2} \oplus S_{1/2}$ so the second two must fit together into the 2-dimensional $S_3$-rep. Likewise,
\[ V_3^{\ot 3} = V_{\square \square \square} \ot \nSym{3}{V_3} \oplus V_{\substack{\square \square \\ \square \,\,\,\,}} \ot \SS_{\substack{\square \square \\ \square \,\,\,\,}}(V_3) \oplus V_{\substack{\substack{\square \\ \square} \\ \square}} \oplus \wedge^3 V_3 = D(3,0) \oplus V_{\substack{\square \square \\ \square \,\,\,\,}} \ot D(1,1) \oplus \alt \ot D(0,0) \]
again we can see that $\SS_{\substack{\square \square \\ \square \,\,\,\,}}(V_3) = D(1,1)$ by looking at our decomposition of $V_3^{\ot 3} = D(3,0) \oplus D(1,1)^{\oplus 2} \oplus D(0,0)$ and seeing that only $D(1,1)$ appears twice so it is the only factor that could be paired with the 2-dimensional $S_3$-rep. Now remember we want the $S_3$-invariants of the tensor products of these so we just need to see which products of $S_3$-reps contain the trivial rep. Indeed, using character theory one checks,
\begin{align*}
V_{\substack{\square \square \\ \square \,\,\,\,}} \ot V_{\substack{\square \square \\ \square \,\,\,\,}} & \cong V_{\substack{\square \square \\ \square \,\,\,\,}} \oplus \alt \oplus \CC
\\
V_{\substack{\square \square \\ \square \,\,\,\,}} \ot \alt & \cong V_{\substack{\square \square \\ \square \,\,\,\,}}
\end{align*}
Therefore we can pair the two trivial reps and the two $V_{\substack{\square \square \\ \square \,\,\,\,}}$ to get,
\[ ((S_{1/2} \ot V_3)^{\ot 3})^{S_3} = S_{3/2} \ot D(3,0) \oplus S_{1/2} \ot D(1,1) \]
forming one $D(3,0)$ representation with spin = 3/2 and one $D(1,1)$ representation with spin = 1/2. Note that $V_{\substack{\square \square \\ \square \,\,\,\,}}^{\ot 2}$ contains only one copy of $\CC$ so does not give two copies of $D(1,1)$ explaining why we only saw one adjoint octet of baryons and not two.

\subsection{Isospin}

This corresponds to the inclusion $\SU(2) \embed \SU(3)$ which fixes the strange quark $s$. Notice that this action mixes the proton and neutron but preserves the fixed strangeness rows of the octet and decouplet. Therefore, the rows must be decompose into irreps of $\SU(2)$ which are called \textit{isospin multiplets}. It is called ``isospin'' because it is a representation of $\SU(2)$ so behaves mathematically like spin but this is just an analogy there is no physical rotations in the picture. Therefore, we can add an extra label to the weight spaces which is $I_3$ the $\SU(2)$ weight corresponding to isospin (the three is because usually in physics they write the Cartan generator $\sigma = \sigma_3$ as the third pauli matrix since it in the spin operator along $+z$ axis (the third axis) with the standard coordinate orientation and the standard maximal torus of $\SU(2)$).

\section{Mesons}

There is another set of particles called the mesons. These are bosons hence have integer spin.

\begin{center}
\begin{tabular}{||c | c c c c ||} 
 \hline
 & spin & M (MeV) & Q & S \\ [0.5ex] 
 \hline\hline
$\pi^{+}$ & 0 & 140 & 1 & 0 \\ 
 \hline
$\pi^{0}$ & 0 & 135 & 0 & 0 \\ 
 \hline
$\pi^{-}$ & 0 & 140 & -1 & 0 \\ 
 \hline
$K^+$ & 0 & 494 & 1 & 1 \\ 
 \hline
$K^0$ & 0 & 498 & 0 & 1 \\ 
 \hline
$K^-$ & 0 & 494 & -1 & -1 \\ 
 \hline
$\ol{K}^0$ & 0 & 498 & 0 & -1 \\ 
 \hline
$\eta$ & 0 & 548 & 0 & 0 \\ 
 \hline
$\eta'$ & 0 & 948\footnote{since this is much larger than the others we should isolate it on its own} & 0 & 0 \\ 
 \hline
\end{tabular}
\end{center}

These correspond to $V_3 \ot \ol{V}_3 = D(1,1) \oplus D(0,0)$. Indeed the larger mass of the $\eta'$ is a clue that it lives in the trivial $D(0,0)$ and not the $D(1,1)$ adjoint representations like the other mesons.

\subsection{Confinement}

But why do we see $V \ot \ol{V}$ and $V^{\ot 3}$ but not any other states. Also I said above that the quarks should be symmetric under exchange but they are spin = 1/2 and hence fermions. Doesn't this contradict the spin-statistic theorem? This problem seemed so egregious to Murray Gell-Mann that he first proposed quarks only as a mathematical abstraction not as real particles since they would have to violate the spin-statistic theorem.
\bigskip\\
Indeed, this would be fixed if the quarks had another property for which they could be totally antisymmetric in under exchange. Then if they were symmetric with respect to spin and flavor $\SU(3)$ they would overall satisfy the fermionic antisymmetry. Then some symmetry group acts on this new type of charge to mix its values. We propose that only systems in the trivial representation for this new symmetry can exist. Lets suppose this new property had two types and therefore transformed under $\SU(2)$ but then two quarks would correspond to $S_{1/2}^{\ot 2}$ which contains the trivial representation. Since we don't see any two-quark baryons this can't be right. What if this new charge has three types and hence transforms under $\SU(3)$. Then if each quark transforms under $V_3$ then a two-quark system has representation,
\[ V_3^{\ot 2} = \nSym{2}{V_3} \oplus \wedge^2 V_3 \]
both are irreps (in fact they are $D(2,0)$ and $D(0,1)$ respectively) so it does not contain a trivial representation. However, we have seen that,
\[ V_3^{\ot 3} = D(3,0) \oplus D(1,1) \oplus D(1,1) \oplus D(0,0) \]
does contain one copy of the irreducible representation. Morever, in the Schur-Weyl decomposition it arises as $\alt \ot D(0,0)$ so it is a totally antisymmetric $\SU(3)$-trivial representation. This explains perfectly why three but not fewer quarks are allowed as well as explaining how we can get total $S_3$ antisymmetry as expected for fermions.
\bigskip\\
Likewise, this explains how the mesons can exist with one quark and one anti-quark since $V_3 \ot \ol{V}_3 \cong D(1,1) \oplus D(0,0)$ contains a copy of the tivial representation. 
\bigskip\\
The only thing left to explain is why only trivial reps of this new $\SU(3)$ appear. We call this color $\SU(3)$ and the three fundamental weights red $(r)$, green $(g)$, and blue $(b)$. The color metaphor is the only uncharged states are equal mixtures of all three colors $r + g + b$ making white. Basically the representation of color $\SU(3)$ corresponds to the interaction of the composite system with the strong force. The strong force is so strong that it collects together all color charged particles so powerfully that you never see color charged particles on their own. This is like how atoms are almost always neutral because if they had electric charge they would find like charges and cancel except that this effect is much much stronger and hence is given a new name: confinement\footnote{It is not just stronger, it is qualatatively very different from recombination. Basically, there is enough energy in the strong field between two color-charged particles that if you pull two appart it will pair produce a quark-antiquark pair to cancell their charges hence the confinement is effectively total.}

\end{document}
