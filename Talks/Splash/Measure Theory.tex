\documentclass{article}
\usepackage[utf8]{inputenc}

\usepackage{amsthm, amssymb, amsmath, centernot}
\usepackage{dsfont}

\newcommand{\notimplies}{%
  \mathrel{{\ooalign{\hidewidth$\not\phantom{=}$\hidewidth\cr$\implies$}}}}

\renewcommand\qedsymbol{$\square$}
\newcommand{\cont}{$\boxtimes$}
\newcommand{\divides}{\mid}
\newcommand{\ndivides}{\centernot \mid}
\newcommand{\Z}{\mathbb{Z}}
\newcommand{\N}{\mathbb{N}}
\newcommand{\C}{\mathbb{C}}
\newcommand{\Zplus}{\mathbb{Z}^{+}}
\newcommand{\Primes}{\mathbb{P}}
\newcommand{\ball}[2]{B_{#1} \! \left(#2 \right)}
\newcommand{\Q}{\mathbb{Q}}
\newcommand{\R}{\mathbb{R}}
\newcommand{\Rplus}{\mathbb{R}^+}
\newcommand{\invI}[2]{#1^{-1} \left( #2 \right)}
\newcommand{\End}[1]{\text{End}\left( A \right)}
\newcommand{\legsym}[2]{\left(\frac{#1}{#2} \right)}
\renewcommand{\mod}[3]{\: #1 \equiv #2 \: \mathrm{mod} \: #3 \:}
\newcommand{\nmod}[3]{\: #1 \centernot \equiv #2 \: mod \: #3 \:}
\newcommand{\ndiv}{\hspace{-4pt}\not \divides \hspace{2pt}}
\newcommand{\finfield}[1]{\mathbb{F}_{#1}}
\newcommand{\finunits}[1]{\mathbb{F}_{#1}^{\times}}
\newcommand{\ord}[1]{\mathrm{ord}\! \left(#1 \right)}
\newcommand{\quadfield}[1]{\Q \small(\sqrt{#1} \small)}
\newcommand{\vspan}[1]{\mathrm{span}\! \left\{#1 \right\}}
\newcommand{\galgroup}[1]{Gal \small(#1 \small)}
\newcommand{\sm}{\! \setminus \!}
\newcommand{\topo}{\mathcal{T}}
\newcommand{\base}{\mathcal{B}}
\renewcommand{\bf}[1]{\mathbf{#1}}
\renewcommand{\Im}[1]{\mathrm{Im} \: #1}
\renewcommand{\empty}{\varnothing}
\newcommand{\indic}[1]{\mathds{1}_{#1}}


\newenvironment{definition}[1][Definition:]{\begin{trivlist}
\item[\hskip \labelsep {\bfseries #1}]}{\end{trivlist}}

\theoremstyle{theorem}
\newtheorem{theorem}{Theorem}[section]
\newtheorem{lemma}[theorem]{Lemma}
\newtheorem{corollary}[theorem]{corollary}

\theoremstyle{definition}
\newtheorem*{problem}{Problem}

\theoremstyle{definition}
\newtheorem*{proposition}{Proposition}

\theoremstyle{remark}
\newtheorem*{fact}{Fact}

\theoremstyle{definition}
\newtheorem{example}{Example}[section]

\theoremstyle{remark}
\newtheorem{remark}{Remark}[subsection]

\begin{document}
\author{Benjamin Church}
\title{\Huge Measure Theory}

\maketitle
\tableofcontents
\newpage

\newcommand{\Rcomp}{\hat{\R}^+}
\newcommand{\Power}[1]{\mathcal{P}\left( #1 \right)}
\newcommand{\measure}[1]{\mu\left( #1 \right)}
\newcommand{\outmeasure}[1]{\mu^*\left( #1 \right)}

\section{Introduction}

\section{A First Attempt at Measure Theory}

We want to define a function which measures the size of a set. First let us work over $\R$. Then our measure is a map from subsets of the real line to nonegative reals or infinity if our set is infinite in length. 

\begin{definition}
The domain of a mesure will be in the set,
\[ \Rcomp = \{ x \in \R \mid x \ge 0 \} \cup \{ \infty \} \]
which has the topology of a closed interval. 
\end{definition}

\begin{definition}
A \textit{measure} is a function $\mu : \Power{\R} \to \Rcomp$ satisfying,
\begin{enumerate}
\item $\measure{\varnothing} = 0$

\item For any countible collection of pairwise disjoint sets $\{ E_i \}_{i = 1}^{\infty}$ for $E_i \subset \R$ we have additivity,
\[ \measure{\bigcup_{i = 1}^\infty E_i} = \sum_{i = 1}^\infty \measure{E_i} \]
\end{enumerate}
\end{definition}

\begin{lemma}
Let $\mu$ be a measure. If $A \subset B$ then $\measure{A} \le \measure{B}$. 
\end{lemma}

\begin{proof}
We can write $B = A \cup (B \setminus A)$ and $A \cap (B \setminus A) = \varnothing$. Then, applying the second property of a measure,
\[ \measure{B} = \measure{A} + \measure{B \setminus A} \ge \measure{A} \]
because $\measure{B \setminus A} \ge 0$ for any set. 
\end{proof}

\begin{example}
The following are well-defined measures on all subsets of $\R$:
\begin{enumerate}
\item The counting measure is defined by $\measure(S) = \#(S)$ when $S$ is finite and $\measure{S} = \infty$ when $S$ is infinite.

\item The dirac measure $\delta_a$ for $a \in \R$ is given by,
\[ \delta_a(S) = \indic{S}(a) = 
\begin{cases}
1 & a \in S
\\
0 & a \notin S
\end{cases} \]
where $\indic{S}$ is the indicator function given by,
\[ \indic{S}(x) = 
\begin{cases}
1 & x \in S
\\
0 & x \notin S
\end{cases} \]

\item Let $\{q_i\}$ be a fixed enummeration of the rational numbers $\Q$. Define $\mu_\Q : \Power{\R} \to \Rcomp$ by,
\[ \mu_{\Q}(S) = \sum_{i = 1}^\infty \frac{\indic{S}(q_i)}{2^i} \]
Since the sum,
\[ \sum_{i = 1}^\infty \frac{1}{2^i} = 1 \]
converges, the measure $\mu_{\Q}(S) \le 1$ so it is never infinite. This function is indeed a measure because the measure of a disjoint union gives the sum over all rationals in each piece with is exactly the sum of the measures. 
\end{enumerate}
\end{example}

\begin{definition}
We say a measure $\mu : \Power{\R} \to \Rcomp$ is \textit{translation-invariant} if $\measure{S + x} = \measure{S}$ for any $S \subset \R$ and $x \in R$ where,
\[ S + x = \{ s + x \mid s \in S \} \]
\end{definition}

\begin{example}
\item The counting measure is translation-invariant since $S + x$ has the same number of elements as $S$.

\item The dirac measure is not translation-invariant since $\delta_a(\{a\}) = 1$ but if $x \neq 0$ then $\delta_{a}(\{a\} + a) = \delta_a(\{a + x\}) = 0$.

\item  $\mu_{\Q}$ is not translation-invariant because different rational numbers will appear in a shifted interval.
\end{example}

\begin{definition}
We say a measure $\mu : \Power{\R} \to \Rcomp$ is \textit{interval-length-compatible} if for any real numbers $a < b$ we have $\measure{[a,b]} = b - a$. The weaker notion of being \textit{nontrivial on intervals} holds if $\mu([a,b]) \neq 0, \infty$ for all such intervals.
\end{definition}

\begin{example}
\item The counting measure is trivial on intervals because $\mu([a,b]) = \infty$. 

\item The dirac measure $\delta_a$ is trivial on all intervals which do not conatain $a$. 

\item $\mu_{\Q}$ is nontrivial on intervals since every interval contains a rational number $q_i \in [a,b]$ so $2^{-1} \le \mu_{\Q}([a,b]) < \infty$.  
\end{example}

\begin{remark}
None of the examples discussed are both translation-invariant and nontrivial on all intervals. This is not an accident as we will now demonstrate. 
\end{remark}

\begin{theorem}[Vitali]
There does not exist a translation-invariant measure on $\R$ which is nontrivial on intervals.
\end{theorem}

\begin{proof}
We will define an equivalence relation $\sim$ on $\R$ by,
\[ x \sim y \iff \exists q \in \Q : x + q = y \]
This equivalence relation measures the ``irrational part'' of a number. Consider the set of equivalence classes,
\[ \R / \Q = \{ [x] \mid x \in \R \} \quad \text{where} \quad [x] = \{t \in \R \mid x \sim y \} \]
This is actually a quotient of groups since $[x] = x + \Q$ so we can also write,
\[ \R / \Q = \{ x + \Q \mid x \in \R \} \]
Now we create a set $V$ by choosing a single element of each equivalence class such that this element lies in $[0,1]$. That is, if $x \in V$ then $V \cap [x] = \{x\}$ so no element equivalent to $x$ (i.e. differing by a rational from $x$) can lie in $V$. Given any choice of a representitive for $[x]$ we can shif by rationals until we land in $[0, 1]$. Constructing $V$ formally requires the axiom of choice but more on this latter. 
\bigskip\\
Now, for $q \in \Q \cap [-1, 1] = \Q_{1}$ consider the sets $V + q$. Given any $x \in [-1, 1]$ we know that there exists some $y \in [x] \cap V$ with $y \in [0, 1]$. Thus, $x - y \in \Q$ since $x \sim y$ and $x - y \in [-1, 1]$ since $x,y \in [0, 1]$. Thus, $x = y + q$ for some $q \in \Q \cap [-1,1]$. However, $y \in V$ so $x \in V + q$. But furthermore, if $x \in V$ then $x \in [0, 1]$ so $x + q \in [-1, 2]$ for $q \in \Q \cap [-1,1]$. Therefore,
\[ [0, 1] \subset \bigcup_{q \in \Q_{1}} V + q \subset [-1, 2] \]
Finally, let $\mu : \Power{\R} \to \Rcomp$ be a translation-invation measure on $\R$ which is nontrivial on intervals. Applying this measure,
\[ \measure{[0, 1]} \le \measure{\bigcup_{q \in \Q_{1}} V + q} \le \measure{[-1,2]} \]
However, if $q \neq q'$ then $V + q$ and $V + q'$ are disjoint because if $x \in V + q$ and $x \in V + q'$ then we would have $x - q, x - q' \in V$ but $(x - q) + (q - q') = x - q'$ so these must lie in the same equivalence class and thus $x - q = x - q'$ so $q = q'$ since there is exactly one element from each equivalence class in $V$. Furthermore, since $\Q$ is countible $\Q_1 = \Q \cap [-1,1]$ is also a countible index set. Therefore, since $\mu$ is a measure, it is additive over countible collections of disjoint set so we have,
\[ \measure{\bigcup_{q \in \Q_{1}} V + q} = \sum_{q \in \Q_{1}} \measure{V + q} \]
Furthermore, $\mu$ is translation invariant so,
\[ \measure{V + q} = \measure{V} \]
Therefore,
\[ \measure{\bigcup_{q \in \Q_{1}} V + q} = \sum_{q \in \Q_{1}} \measure{V} \]
Plugging into the innequality,
\[ \measure{[0, 1]} \le \sum_{q \in \Q_{1}} \measure{V} \le \measure{[-1,2]} \]
Finally, because $\mu$ is nontrivial on intervals we know that $\measure{[0,1]}$ and $\measure{[-1,2]}$ are positive real numbers (not $\infty$). This is the desired contradiction because,
\[ \sum_{q \in \Q_{1}} \measure{V} = \measure{V} \sum_{q \in \Q_1} 1 
= 
\begin{cases}
\infty & \measure{V} \neq 0
\\
0 & \measure{V} = 0
\end{cases}\] 
so this value cannot possibly fit in the innequality between two positive real numbers. 
\end{proof}

\begin{remark}
The axiom of choice is a somewhat controversial axiom of set theory which states that given any collection of nonempty sets there exists a set which contains exactly one element from each set in the collection. Applying this axiom to $\R / \Q$ gives us a Vitali set $V$. We can write this axiom in formal logic as,
\[ \forall X [ \varnothing \notin X \Longrightarrow \exists f : X \rightarrow \bigcup X \quad \forall A \in X :  f ( A ) \in A  ] \]
which states that there exists a choice function taking a set $A$ and choosing some element $f(A) \in A$. 
\end{remark}

\begin{remark}
This is a devestating result. We certainally wanted any candidate length function to be a translation-invariant measure which respects the lengths of intervals. Vitali showed that this is impossible. We will discuss how the modern theory circumvents this difficulty in the following section. 
\end{remark}

\section{Sigma Algebras and Measure Spaces}

\begin{definition}
An \textit{outer-measure} is a function $\mu^* : \Power{X} \to \Rcomp$ satisfying,
\begin{enumerate}
\item $\outmeasure{\varnothing} = 0$

\item For any subsets $A,B \subset X$ we have,
 \[ A \subset B \implies \outmeasure{A} \le \outmeasure{B} \]

\item For any countible collection of pairwise disjoint sets $\{ E_i \}_{i = 1}^{\infty}$ for $E_i \subset X$ we have subadditivity,
\[ \outmeasure{\bigcup_{i = 1}^\infty E_i} \le \sum_{i = 1}^\infty \outmeasure{E_i} \]
\end{enumerate}
\end{definition}

\begin{definition}
The \textit{Lebesgue outer-measure} $\mu^* : \Power{\R} \to \Rcomp$ is defined as follows. Let $I$ denote an open interval of the form $I = (a,b)$ and $\ell(I) = b - a$ its canonical length. Then for $S \subset \R$ we set,
\[ \outmeasure{E} = \inf{\left\{ \sum_{k = 1}^{\infty} \ell(I_k) \: \middle| \: \{I_k\}_{k\in\N} \text{ is a cover of } E \text{ by open intervals i.e. } E \subset \bigcup_{k = 1}^\infty I_k \right\}} \]
\end{definition}

\begin{proposition}
The Lebesgue outer-measure defined above satisfies the outer-measure axioms.
\end{proposition}

\begin{proof}

\end{proof}

\begin{remark}
The concept of an outer-measure will allow us to define the space of measureable sets. We first need to know what kind of space this will be.
\end{remark}

\begin{definition}
A $\sigma$-\textit{algebra} on $X$ is a collection $\Sigma \subset X$ of subsets of $X$ satisfying,
\begin{enumerate}
\item $X \in \Sigma$ and $\varnothing \in \Sigma$
\item If $E \in \Sigma$ then $E^c = X \setminus E \in \Sigma$.
\item or any countible collection of pairwise disjoint sets $\{ E_i \}_{i = 1}^{\infty}$ for $E_i \in \Sigma$ then,
\[ \bigcup_{i = 1}^\infty E_i \in \Sigma \] 
By taking the compliment of the union of the compliments we also get coutible intersections i.e.
\[ \bigcap_{i = 1}^\infty E_i \in \Sigma \] 
\end{enumerate}
We call the pair $(X, \Sigma)$ a measureable space. 
\end{definition}

\begin{definition}
Let $(X, \Sigma_X)$ and $(Y, \Sigma_Y)$ be measureable spaces. A function $f : X \to Y$ is called \textit{measureable} if for any $Y$-measurable set $E \in \Sigma_Y$ its pre-image is $X$-measureable i.e. $f^{-1}(E) \in \Sigma_X$. 
\end{definition}

\begin{remark}
We now have the tools to give a correct modern definition of a measure.
\end{remark}

\begin{definition}
Let $(X, \Sigma)$ be a measureable space i.e. $\Sigma$ is a $\sigma$-algebra on $X$. Then a \textit{measure} on $(X, \Sigma)$ is a function $\mu : \Sigma \to \Rcomp$ satisfying,
\begin{enumerate}
\item $\measure{\varnothing} = 0$

\item For any countible collection of pairwise disjoint sets $\{ E_i \}_{i = 1}^{\infty}$ for $E_i \in \Sigma$ we have additivity,
\[ \measure{\bigcup_{i = 1}^\infty E_i} = \sum_{i = 1}^\infty \measure{E_i} \]
\end{enumerate}
We call the triple $(X, \Sigma, \mu)$ a \textit{measure} space. 
\end{definition}

\begin{definition}
A measure space $(X, \Sigma, \mu)$ is \textit{complete} if for any $E \in \Sigma$ such that $\mu(E) = 0$ and any $S \subset E$ we have $S \in \Sigma$.  
\end{definition}

\begin{definition}
Let $\mu^* : \Power{X} \to \Rcomp$ be an outer-measure. We say that $E \subset X$ is \textit{measureable} if for any $A \subset X$ we have,
\[ \outmeasure{A} = \outmeasure{A \cap E} + \outmeasure{A \cap E^c} \]
\end{definition}

\begin{lemma}
If $E_1, E_2 \subset X$ are $\mu^*$-measurable then $E_1 \cup E_2$ is also $\mu^*$-measurable.
\end{lemma}

\begin{proof}
If $E_1, E_2 \in \Sigma$ then,
\[ \outmeasure{A} = \outmeasure{A \cap E_1} + \outmeasure{A \cap E_1^c}  \]
for any $A$. Furthermore, taking $A \cap E_1^c$ as the arbitrary subset and applying the measurability of of $E_2$,
\[ \outmeasure{A \cap E_1^c} = \outmeasure{A \cap E_1^c \cap E_2} + \outmeasure{A \cap E_1^c \cap E_2^c} \]
Furthermore, we can split the set $A \cap (E_1 \cup E_2)$ as the union of $A \cap E_1$ and $A \cap E_1^c \cap E_2$. By subadditivity,
\[ \outmeasure{A \cap (E_1 \cup E_2)} \le \outmeasure{A \cap E_1} + \outmeasure{A \cap E_1^c \cap E_2} \]
Combining these results,
\begin{align*}
\outmeasure{A \cap (E_1 \cup E_2)} + \outmeasure{A \cap (E_1^c \cap E_2^c)} & \le \outmeasure{A \cap E_1} + \outmeasure{A \cap E_1^c \cap E_2} + \outmeasure{A \cap (E_1^c \cap E_2^c)} 
\\
& = \outmeasure{A \cap E_1} + \outmeasure{A \cap E_1^c} = \outmeasure{A}
\end{align*}
However, $A$ can be decomposed as the disjoint union of $A \cap (E_1 \cup E_2)$ and $A \cap (E_1^c \cap E_2^c)$ so by subadditivity,
\[ \outmeasure{A} \le \outmeasure{A \cap (E_1 \cup E_2)} + \outmeasure{A \cap (E_1^c \cap E_1^c)} \]
Therefore,
\[ \outmeasure{A} = \outmeasure{A \cap (E_1 \cup E_2)} + \outmeasure{A \cap (E_1^c \cap E_1^c)} \]
for any set $A$. Thus, $E_1 \cup E_2 \in \Sigma$ is measureable. 
\end{proof}

\begin{lemma} \label{sigma_alg_limits}
If $\{ E_i \}_{i = 1}^{\infty}$ is a countible increasing collection of $\mu^*$-measureable sets then, for any set $A \subset X$,
\[ \outmeasure{A \cap \bigcup_{i = 1}^\infty E_i} = \lim_{n \to \infty} \outmeasure{A \cap E_n} \]
\end{lemma}

\begin{proof}
Define,
\[ E = \bigcup_{i = 1}^\infty E_i \]
By monotonicity, 
\[ \outmeasure{A \cap E_n} \le \outmeasure{A \cap E} \implies \lim_{n \to \infty} \outmeasure{A \cap E_n} \le \outmeasure{A \cap E} \]
We can write,
\[ A \cap E = \bigcup_{i = 1}^\infty A \cap E_i = \bigcup_{i = 0}^\infty A \cap E_{i+1} \cap E_i^c \]
since $E_{i+1} \supset E_i$ this is a disjoint union since if $i < j$ then $E_{j+1} \cap E_{j}^c$ is disjoint from $E_j \supset E_i$. Applying subadditivity,
\begin{align*}
\outmeasure{A \cap E}  \le \sum_{i = 0}^\infty \outmeasure{A \cap E_{i + 1} \cap E_i^c}
\end{align*}
Since $E_i$ is $\mu^*$-measureable then taking $A \cap E_{i+1}$, 
\[ \outmeasure{A \cap E_{i+1}} = \outmeasure{A \cap E_{i+1} \cap E_i} + \outmeasure{A \cap E_{i+1} \cap E_i^c} \]
with $E_0 = \varnothing$. Thus,
\begin{align*}
\outmeasure{A \cap E} & \le \sum_{i = 0}^\infty \outmeasure{A \cap E_{i + 1} \cap E_i^c} = \sum_{i = 0}^\infty \left[  \outmeasure{A \cap E_{i+1}} - \outmeasure{A \cap E_{i+1} \cap E_i} \right] 
\\
& = \sum_{i = 0}^\infty \left[  \outmeasure{A \cap E_{i+1}} - \outmeasure{A \cap E_i} \right] = \lim_{n \to \infty} \outmeasure{A \cap E_{n}} - \outmeasure{A \cap E_0} = \lim_{n \to \infty} \outmeasure{A \cap E_{n}}
\end{align*}
because,
\[ \outmeasure{A \cap E_0} = \outmeasure{A \cap \varnothing} = 0\]
Therefore,
\[ \outmeasure{A \cap E} = \lim_{n \to \infty} \outmeasure{A \cap E_{n}} \]
\end{proof}

\begin{theorem}
The collection of $\mu^*$-measureable sets $\Sigma_{\mu}$ is a $\sigma$-algebra on $X$ and $\mu$, the restiction of $\mu^*$ to $\Sigma_\mu$, makes $(X, \Sigma_\mu, \mu)$ a complete measure space. 
\end{theorem}

\begin{proof}
If $E = X$ or $E = \varnothing$ then clearly,
\[ \outmeasure{A \cap E} + \outmeasure{A \cap E^c} = \outmeasure{A} + \outmeasure{\varnothing} = \outmeasure{A} \]
so $X, \varnothing \in \Sigma_{\mu}$. Furthermore $E \in \Sigma_{\mu}$ if and only if 
\[ \outmeasure{A} = \outmeasure{A \cap E} + \outmeasure{A \cap E^c} \]
for each $A \subset X$. So clearly $E \in \Sigma_\mu \iff E^c \in \Sigma$.
We have shown that $\Sigma_\mu$ contains finite unions. Taking $A = E_1$ with disjoint $E_1, E_2 \in \Sigma_{\mu}$ gives,
\[ \outmeasure{E_1 \cup E_2} = \outmeasure{(E_1 \cup E_2) \cap E_1} + \outmeasure{(E_1 \cup E_2) \cap E_1^c} = \outmeasure{E_1} + \outmeasure{E_2} \]
so we have finite additivity on $\Sigma_{\mu}$. 
If we have a countible collection of pairwise disjoint sets $\{ E_i \}_{i = 1}^{\infty}$ for $E_i \in \Sigma_\mu$. We have shown that the unions,
\[ T_n = \bigcup_{i = 1}^n E_n \in \Sigma_\mu \]
are measureable.  Then,
\[ \outmeasure{A} = \outmeasure{A \cap T_n} + \outmeasure{A \cap T_n^c} \]
Furthermore, define,
\[ E = \bigcup_{i = 1}^\infty E_i \]
and then,
\[ A \cap E^c \subset A \cap T_n^c \]
so we have,
\[ \outmeasure{A \cap E^c} \le \outmeasure{A \cap T_n^c} \]
Thus,
\[  \outmeasure{A} \ge \outmeasure{A \cap T_n} + \outmeasure{A \cap E^c} \]
which implies, via Lemma \ref{sigma_alg_limits}, that
\[ \outmeasure{A} \ge \lim_{n \to \infty} \outmeasure{A \cap T_n} + \outmeasure{A \cap E^c} = \outmeasure{A \cap E} + \outmeasure{A \cap E^c} \]
Finally, by subadditivty,
\[ \outmeasure{A} \le \outmeasure{A \cap E} + \outmeasure{A \cap E^c} \]
and therefore, 
\[ \outmeasure{A} = \outmeasure{A \cap E} + \outmeasure{A \cap E^c} \]
So $E \in \Sigma_\mu$. Therefore $\Sigma_{\mu}$ is a $\sigma$-algebra.
Furthermore, if $E \in \Sigma_{\mu}$ with $\outmeasure{E} = 0$ and take $S \subset E$ then for any $A \subset X$ using monotonicity we have,
\[ \outmeasure{A \cap S^c} \le \outmeasure{A} \]
and also, 
\[ \outmeasure{A \cap S} \le \outmeasure{A \cap E} \le \outmeasure{E} = 0 \]
Thus,
\[ \outmeasure{A \cap S^c} + \outmeasure{A \cap S} \le \outmeasure{A} \]
and also, by subadditivity,
\[ \outmeasure{A} \le \outmeasure{A \cap S} + \outmeasure{A \cap S^c} \]
Thus,
\[ \outmeasure{A} = \outmeasure{A \cap S} + \outmeasure{A \cap S^c} \]
so $S \in \Sigma_{\mu}$.
Finally, we have,
\[ \outmeasure{T_n} = \sum_{i = 1}^n \outmeasure{E_i} \]
but finite additivity.
Thus,
\[ \outmeasure{E} = \lim_{n \to \infty} \sum_{i = 1}^n \outmeasure{E_i} = \sum_{i = 1}^{\infty} E_i \]
Therefore, $(X, \Sigma_{\mu}, \mu^*)$ is a complete measure space.   
\end{proof}

\begin{definition}
A $\sigma$-algeba $\Sigma$ on a topological space $X$ is called \textit{Borel} if $\Sigma$ contains every open set of $X$. If $\Sigma$ is Borel then we say that the measureable space $(X, \Sigma)$ is a Borel space and any measure on $(X, \Sigma)$ is a Borel measure. Furthermore, \textit{the Borel algebra} $\mathfrak{B}(X)$ is the intersection of all Borel $\sigma$-algebras on $X$ so $\mathfrak{B}(X)$ is the minimal $\sigma$-algebra containing all open and thus all closed sets of $X$.  
\end{definition}

\begin{theorem}
The $\sigma$-algebra of Lebesgue-measurable sets $\Sigma_{\mathcal{L}}$ is Borel over $\R$. 
\end{theorem}

\begin{proof}

\end{proof}

\begin{theorem}
The Lebesgue measure on $(X, \Sigma_{\mathcal{L}})$ is a translation-invariant measure which is nontrivial on intervals. 
\end{theorem}

\begin{proof}

\end{proof}

\begin{remark}
We can generalize the Lebesgue measure to $\R^n$ for arbitrary dimensions by,
\[ \outmeasure{E} = \inf{\left\{ \sum_{k = 1}^{\infty} \ell(I_k) \: \middle| \: \{I_k\}_{k\in\N} \text{ is a cover of } E \text{ by open intervals i.e. } E \subset \bigcup_{k = 1}^\infty I_k \right\}} \]
where $I_k$ is a primitive open set $[a_1, b_1] \times \cdots \times [a_n, b_n]$ and 
\[ \ell(I_k) = (b_1 - a_1) \cdots (b_n - a_n)\]
is the canonical volume. 
\end{remark}

\section{Haar Measures}

\section{Probability Theory}

\section{Lebesge Integration}

\section{Hausdorff Measures}

\section{Banach-Tarski}

\end{document}