\documentclass[11pt,a4paper]{article}

\usepackage[english]{babel}
\usepackage[utf8]{inputenc}

\usepackage{amsthm, amssymb, amsmath, centernot}

\newcommand{\notimplies}{%
  \mathrel{{\ooalign{\hidewidth$\not\phantom{=}$\hidewidth\cr$\implies$}}}}

\renewcommand\qedsymbol{$\square$}
\newcommand{\cont}{$\boxtimes$}
\newcommand{\divides}{\mid}
\newcommand{\ndivides}{\centernot \mid}
\newcommand{\Z}{\mathbb{Z}}
\newcommand{\N}{\mathbb{N}}
\newcommand{\C}{\mathbb{C}}
\newcommand{\Zplus}{\mathbb{Z}^{+}}
\newcommand{\Primes}{\mathbb{P}}
\newcommand{\ball}[2]{B_{#1} \! \left(#2 \right)}
\newcommand{\Q}{\mathbb{Q}}
\newcommand{\R}{\mathbb{R}}
\newcommand{\Rplus}{\mathbb{R}^+}
\newcommand{\invI}[2]{#1^{-1} \left( #2 \right)}
\newcommand{\End}[1]{\text{End}\left( A \right)}
\newcommand{\legsym}[2]{\left(\frac{#1}{#2} \right)}
\renewcommand{\mod}[3]{\: #1 \equiv #2 \: \mathrm{mod} \: #3 \:}
\newcommand{\nmod}[3]{\: #1 \centernot \equiv #2 \: mod \: #3 \:}
\newcommand{\ndiv}{\hspace{-4pt}\not \divides \hspace{2pt}}
\newcommand{\finfield}[1]{\mathbb{F}_{#1}}
\newcommand{\finunits}[1]{\mathbb{F}_{#1}^{\times}}
\newcommand{\ord}[1]{\mathrm{ord}\! \left(#1 \right)}
\newcommand{\quadfield}[1]{\Q \small(\sqrt{#1} \small)}
\newcommand{\vspan}[1]{\mathrm{span}\! \left\{#1 \right\}}
\newcommand{\galgroup}[1]{Gal \small(#1 \small)}
\newcommand{\sm}{\! \setminus \!}
\newcommand{\topo}{\mathcal{T}}
\newcommand{\base}{\mathcal{B}}
\renewcommand{\bf}[1]{\mathbf{#1}}
\renewcommand{\Im}[1]{\mathrm{Im} \: #1}
\renewcommand{\empty}{\varnothing}


\newenvironment{definition}[1][Definition:]{\begin{trivlist}
\item[\hskip \labelsep {\bfseries #1}]}{\end{trivlist}}

\theoremstyle{theorem}
\newtheorem{theorem}{Theorem}[section]
\newtheorem{lemma}[theorem]{Lemma}

\theoremstyle{definition}
\newtheorem*{problem}{Problem}

\theoremstyle{definition}
\newtheorem*{proposition}{Proposition}

\theoremstyle{remark}
\newtheorem*{fact}{Fact}

\theoremstyle{definition}
\newtheorem{example}{Example}[section]

\theoremstyle{remark}
\newtheorem{remark}{Remark}[subsection]

\begin{document}
\author{Benjamin Church}
\title{\Huge Constructible Polygons}

\maketitle
\tableofcontents
\newpage


\section{Introduction}

\section{Compass and Straight Edge Constructons}

\subsection{Perpendicular Bisectors}

\subsection{Finding the Center of A Circle}

\subsection{Construcing Regular Polygons}

\subsubsection{The Equilateral Triangle}

\subsubsection{The Square}

\subsubsection{The Pentagon}

\subsubsection{The Hexagon}

\subsubsection{The Heptagon: Where to Go From Here?}

\section{Field Theory}

\subsection{Definition of a Field}

\subsection{Field Extensions}

\subsection{Constructable Fields}

\subsubsection{Constructing Sums, Products, and Quotients}

\subsubsection{The Field of Constructable Numbers}

\subsubsection{Constructable Extensions}

\section{Some Classic Impossibility Results}

\subsection{Doubling The Cube}

\subsection{Trisecting an Angle}

\subsection{Squaring the Circle}

\section{Some Not So Classic Possibility Results}

\subsection{Generalizations of Compass and Streight Edge Constructions}

\subsubsection{$\nu \varepsilon \upsilon \sigma \iota \varsigma$ Constructions}

\subsubsection{Using a Tomahawk to Trisect an Angle}

\subsubsection{Constructions with Conic Sections}

\subsection{The Mohr–Mascheroni theorem}

\subsection{The Poncelet–Steiner theorem}

\subsection{Archemedes' Axiom and First Order Logic}

\section{Galois Theory}

\subsection{Motivation}

\subsection{Some Basic Group Theory}

\subsection{The Fundamental Theorem of Galois Theory}

\subsection{An Example of Galois Theory}

\subsection{$p$-Groups}

\begin{theorem}
Let $P$ be a $p$-group then there exists a chain of normal subgroups,
\[P_0 \triangleleft P_1 \triangleleft \cdots \triangleleft P_k = P\]
such that $|P_i| = p^i$. 
\end{theorem}

\begin{remark}
Such a chain is known as a composition series because $P_{i + 1}/P_i \cong C_p$ which is a simple group (since the order is $p^{i+1}/p^i = p$ and all prime groups are cyclic and simple). By the Jordan-Holder theorem such a composition series always exists and is unique upto rearaning the factor groups. Furthermore, since the factor groups $P_{i + 1}/P_i \cong C_p$ are not just simple they are also abelian, this is known as a solvable series which tells us that any $p$-group is a solvable group.  
\end{remark}

\subsection{Characterizing Constructible Field Extensions}

\begin{theorem}
If $x$ is a constructible number than $\sqrt{x}$ is a constructible number.
\end{theorem}

\begin{theorem}
Let $K \subset \C$ be a subfield of the complex numbers. For any $x \in K$ there is a constructible extension $L/K$ such that $\sqrt{x} \in K$.  
\end{theorem}

\begin{theorem}
$L/K$ is a constructible extension if and only if $[L : K] = 2^k$ for some $k$. 
\end{theorem}

\section{The Main Theorem}

\subsection{The Euler $\varphi$ Function}

\subsection{The Galois Group of A Cyclotomic Field}

\subsection{Characterizing Constructable Polygons}

\section{Further Developments}

Gauss' marvelous theorem reduced the question of constructability of regular polygons to a number theoretic question. In this section, we will discuss a bit of the relevant number theory and modern developments in the theory.

\subsection{Fermat Numbers}

\subsection{On the Infinitude of Fermat Primes}

\subsection{The (Odd Sized) Construcitble Polygons}

\subsection{A Vision For The Future}

\end{document}
