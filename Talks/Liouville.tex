\documentclass{article}
\usepackage[utf8]{inputenc}

\usepackage{amsthm, amssymb, amsmath, centernot}

\newcommand{\notimplies}{%
  \mathrel{{\ooalign{\hidewidth$\not\phantom{=}$\hidewidth\cr$\implies$}}}}

\renewcommand\qedsymbol{$\square$}
\newcommand{\cont}{$\boxtimes$}
\newcommand{\divides}{\mid}
\newcommand{\ndivides}{\centernot \mid}
\newcommand{\Z}{\mathbb{Z}}
\newcommand{\N}{\mathbb{N}}
\newcommand{\C}{\mathbb{C}}
\newcommand{\Zplus}{\mathbb{Z}^{+}}
\newcommand{\Primes}{\mathbb{P}}
\newcommand{\ball}[2]{B_{#1} \! \left(#2 \right)}
\newcommand{\Q}{\mathbb{Q}}
\newcommand{\R}{\mathbb{R}}
\newcommand{\Rplus}{\mathbb{R}^+}
\newcommand{\invI}[2]{#1^{-1} \left( #2 \right)}
\newcommand{\End}[1]{\text{End}\left( A \right)}
\newcommand{\legsym}[2]{\left(\frac{#1}{#2} \right)}
\renewcommand{\mod}[3]{\: #1 \equiv #2 \: \mathrm{mod} \: #3 \:}
\newcommand{\nmod}[3]{\: #1 \centernot \equiv #2 \: mod \: #3 \:}
\newcommand{\ndiv}{\hspace{-4pt}\not \divides \hspace{2pt}}
\newcommand{\finfield}[1]{\mathbb{F}_{#1}}
\newcommand{\finunits}[1]{\mathbb{F}_{#1}^{\times}}
\newcommand{\ord}[1]{\mathrm{ord}\! \left(#1 \right)}
\newcommand{\quadfield}[1]{\Q \small(\sqrt{#1} \small)}
\newcommand{\vspan}[1]{\mathrm{span}\! \left\{#1 \right\}}
\newcommand{\galgroup}[1]{Gal \small(#1 \small)}
\newcommand{\sm}{\! \setminus \!}
\newcommand{\topo}{\mathcal{T}}
\newcommand{\base}{\mathcal{B}}
\renewcommand{\bf}[1]{\mathbf{#1}}
\renewcommand{\Im}[1]{\mathrm{Im} \: #1}
\renewcommand{\empty}{\varnothing}


\newenvironment{definition}[1][Definition:]{\begin{trivlist}
\item[\hskip \labelsep {\bfseries #1}]}{\end{trivlist}}

\theoremstyle{theorem}
\newtheorem{theorem}{Theorem}[section]
\newtheorem{lemma}[theorem]{Lemma}
\newtheorem{corollary}[theorem]{corollary}

\theoremstyle{definition}
\newtheorem*{problem}{Problem}

\theoremstyle{definition}
\newtheorem*{proposition}{Proposition}

\theoremstyle{remark}
\newtheorem*{fact}{Fact}

\theoremstyle{definition}
\newtheorem{example}{Example}[section]

\theoremstyle{remark}
\newtheorem{remark}{Remark}[subsection]

\begin{document}
\author{Benjamin Church}
\title{\Huge Transcendental Numbers}

\maketitle
\tableofcontents
\newpage


\section{Introduction}

The rational numbers ($\Q$) are incomplete in two different ways. Firstly, $\Q$ is not algebraically closed because there exist polynomials with rational coeficients which have no roots in $\Q$. For example, $x^2 - 2 = 0$. Furthermore, $\Q$ is not complete because there are sequences of rational numbers which converge in the real numbers but not in the rational numbers. For example, let $F_n$ be the $n$-th Fibonacci number then $\lim\limits_{n \to \infty} \frac{F_{n+1}}{F_n} = \varphi$ where $\varphi = \frac{1 + \sqrt{5}}{2} \notin \Q$. If we complete $\Q$ by adding in the limit of every sequence, we get the real numbers $\R$. If take the algebraic closure of $\Q$ by adding in the roots of every polynomial with coeficients in $\Q$ we get the algebraic numbers $\bar{\Q}$. The relationship between these two sets was of great historically importance. In particular, $\bar{\Q}$ contains complex numbers (for example $i$ solves $x^2 + 1 = 0$) and $\R$ does not. The question arises, is $\R$ contained in $\bar{\Q}$. Equivalently, does there exist a non-algebraic real number. Such a number is called \textit{transcendental} because the number ``transcends'' algebraic definition.   

\section{Algebraic Numbers and Cantor's Theorem}

\begin{definition}
$\Q[X]$ is the set of polynomials with coeficients in $\Q$ and $\Z[X]$ is the set of polynomials with coeficients in $Z$.  
\end{definition}

\begin{definition}
A complex number $\alpha \in \C$ is \textit{algebraic} if there exists a polynomial $f \in \Q[X]$ such that $f(\alpha) = 0$. Otherwise, $\alpha$ is \textit{transcendental}. 
\end{definition}

\begin{proposition}
$\alpha \in \C$ is algebraic iff there exists a polynomial $f \in \Z[X]$ with \textit{integer} coeficients such that $f(\alpha) = 0$. 
\end{proposition}

\begin{proof}
Let $\alpha \in \C$ be a root of a polynomial $f \in \Z[X]$. Because $\Z \subset \Q$ then $\Z[X] \subset \Q[X]$ so $f \in \Q[X]$ so $\alpha$ is algebraic. Conversely, let $\alpha \in \C$ be algebraic. Then $\alpha$ is the root of some polynomial with rational coeficients, 
\begin{align*}
f(\alpha) &= \frac{p_n}{q_n} \alpha^n + \cdots + \frac{p_1}{q_1} \alpha + \frac{p_0}{q_0} \\ & = p_n (q_{n-1} q_{n-2} \cdots q_0) \alpha^n + \cdots p_1 (q_n q_{n-1} \cdots q_{2} q_0) \alpha + p_0 (q_{n} q_{n-1} \cdots q_2 q_1) \\ & = 0 
\end{align*}
Therefore, $\alpha$ is the root of a polynomial with coeficients in $\Z$. 
\end{proof}

\begin{definition}
Let $\alpha \in \C$ be algebraic. The degree of $\alpha$, denoted as $\deg{\alpha}$, is the minimum degree of a polynomial $f \in \Z[X]$ such that $f(\alpha) = 0$.  
\end{definition}

\begin{definition}
A function $f : X \to Y$ is a \textit{surjection} if for every $y \in Y$ there exists $x \in X$ such that $f(x) = y$. 
\end{definition}

\begin{definition}
A set $X$ is \textit{countable} if there is a surjection $f : \N \to X$. Otherwise, $X$ is \textit{uncountable}. Such a function is called a list of $X$. 
\end{definition}

\begin{definition}
The set $\bar{\Q} \subset \C$ is the set of algebraic numbers. That is, given a complex number $\alpha \in \C$ then $\alpha \in \bar{\Q}$ if any only if $\alpha$ is the root of some rational polynomial $p \in \Q[X]$. 
\end{definition}

\begin{theorem}
$\bar{\Q}$ is countable.
\end{theorem}

\begin{proof}
Every polynomial has a finite number of roots so it suffices to show that we can list all polynomials with integer coeficients. The list goes as follows, list all polynomials with degree less than $n$ and coeficients with absolute value less than $n$. There are a finite number of such polynomials so we will list every polynomial by incimentally increasing $n$. This is surjective becuase given a polynomial of degree $n$ there are a finite number of polynomials with smaller degree and smaller coeficients so the function will reach this given polynomial in a finite number of steps.   
\end{proof}

\begin{theorem}[Cantor]
$\R$ is uncountable.
\end{theorem}

\begin{proof}
This is a classic proof by contradiction. Suppose I had such a function $f : \N \to \R$. Now, I will construct some $r \in \R$ not in the image of $f$. For simplicty let us only take real numbers on the interval $[0, 1]$ this will suffice. Let $r_i$ be the $i$-th digit of $r$ in some base $b$. Define the number $s \in \R$ by its expansion base $b$ as $s_i = f(i)_i + 1 \: \mathrm{mod} \: b$. This is the $i$-th digit of the $i$-th number plus one reduced by $b$. I claim that there does not exist any $n \in \N$ such that $f(n) = s$. Suppose such and $n$ exists. Then $s_n = f(n)_n + 1 \: \mathrm{mod} \: b$ but $s = f(n)$ so $s_n = f_n$ which is a contradiction. Therefore, $f$ cannot be a bijection.   
\end{proof}

\begin{corollary}
Transcendental numbers exist.
\end{corollary}

\begin{proof}
If $\R \subset \bar{\Q}$ then because $\bar{\Q}$ is countable then $\R$ would be countable because any sujection onto $\bar{\Q}$ can be reduced to a surjection onto $\R$ by mapping every $x \in \bar{\Q}$ such that $x \notin \R$ to some fixed point of $\R$. However, $\R$ is uncountable so there must be some $r \in \R$ such that $r \notin \bar{\Q}$. 
\end{proof}

\section{Diophantine Approximation}

\begin{definition}
For $\alpha \in \R$, an $n$-\textit{good} Diophantine approximation is $\frac{p}{q} \in \Q$ so that, 
\[ 0 < \Big| \alpha - \frac{p}{q} \Big| < \frac{1}{q^n}\]
\end{definition}

\begin{definition}
A number $\alpha \in \R$ is $n$-\textit{approximable} if there exist infinitely many $n$-good Diophantine approximations.  
\end{definition}

\begin{definition}
\[G_n(\alpha) = \left\{ \frac{p}{q} \in \R \mid 0 < \Big|\alpha - \frac{p}{q} \Big| < \frac{1}{q^n} \right\}\]
the set of $n$-good approximations of $\alpha$. $\alpha$ is $n$-approximable when $|G_n(\alpha)| = \infty$.
\end{definition}

\begin{lemma}
Let $\alpha$ be a root of $f \in \Z[X]$ with $\deg{f} = n$ then there exists $C \in \Rplus$ such that for every $\frac{p}{q} \in \Q$ such that $\frac{p}{q} \neq \alpha$ we have,
\[ \Big| \alpha - \frac{p}{q} \Big| > \frac{C}{q^n}\]
\end{lemma}

\begin{proof}
Let $f(x) = a_n x^n + \cdots + a_0$ with coeficients $a_i \in \Z$. Take $f(\alpha) = 0$. There are at most $n$ roots of $f$ labled $\alpha_1, \alpha_n, \cdots, \alpha_k$ and $\alpha$. Define, 
\[r = \min\{|\alpha - \alpha_1|, |\alpha - \alpha_2|, \dots, |\alpha - \alpha_k|\}\]
Therefore, $f$ has no roots except $\alpha$ on the interval $(\alpha - r, \alpha + r)$. Define, 
\[ M = \max \{ |f'(x)| \mid x \in (\alpha - r, \alpha + r) \}\]
and take any positive real number $C < \min\{r, \frac{1}{M}\}$. 
Now, take any $\frac{p}{q} \in \Q$ with $\frac{p}{q} \neq \alpha$. If $\Big| \alpha - \frac{p}{q} \Big| > C > \frac{C}{q^n}$ then we are done. Otherwise, $\Big| \alpha - \frac{p}{q} \Big| \le C \le r$ so $\frac{p}{q} \in (\alpha - r, \alpha + r)$ but $\alpha \neq \frac{p}{q}$ so $f(\frac{p}{q}) \neq 0$ because there are no other roots on this interval. Consider,
\[q^n f\left(\frac{p}{q}\right) = a_n p^n + a_{n-1} p^{n-1}q + \cdots + a_0 q^n \in \Z\]
However, $q^n f(\frac{p}{q}) \neq 0$ so $|q^n f(\frac{p}{q})| \ge 1$ because it is a nonzero positive integer. Thus, $|f(\frac{p}{q})| \ge \frac{1}{q^n}$. By the mean value theorem, there exists $\xi \in (\alpha, \frac{p}{q}) \subset (\alpha - r, \alpha + r)$ such that, \[f'(\xi) = \frac{f(\frac{p}{q}) - f(\alpha)}{\frac{p}{q} - \alpha}\]
Therefore, 
\[\Big| \alpha - \frac{p}{q} \Big| = \Bigg| \frac{f(\frac{p}{q})}{f'(\xi)} \Bigg| \] 
However, $|f'(\xi)| \le M$ and $|f(\frac{p}{q})| \ge \frac{1}{q^n}$ so,
\[\Big| \alpha - \frac{p}{q} \Big| = \Bigg| \frac{f(\frac{p}{q})}{f'(\xi)} \Bigg| \ge \frac{1}{M q^n} > \frac{C}{q^n} \] 
\end{proof}

\begin{corollary}
Let $\alpha$ be algebraic with degree $n$, then for any $k > n$, $\alpha$ is not $k$-approximable. 
\end{corollary}

\begin{proof}
Suppose that $k > n$ and $\alpha$ is $k$-approximable then $G_k(\alpha)$ is infinite and thus must contain $\frac{p}{q}$ with arbitrarily large $q$. Therefore, given any $C \in \Rplus$ we can choose $\frac{p}{q} \in G_k(\alpha)$ such that $q^{k-n} > C$ which is possible because $k - n > 0$. Then, because $\frac{p}{q} \in G_k(\alpha)$,
\[ 0 < \Big| \alpha - \frac{p}{q} \Big| < \frac{1}{q^k} = \frac{C}{q^n} \cdot \frac{C}{q^{n-k}} < \frac{C}{q^n}\]
Since $\frac{p}{q} \neq \alpha$, this contradicts the previous lemma because $\alpha$ is a root to some $f \in Z[X]$ with $\deg{f} = n$. However, there could not exist any $C \in \Rplus$ such that, 
\[ \Big| \alpha - \frac{p}{q} \Big| > \frac{C}{q^n}\]
for every $\frac{p}{q} \in \Q$ with $\frac{p}{q} \neq \alpha$. Thus, $\alpha$ is not $n$-approximable. 
\end{proof}

\section{Irrationality Measure} 

We can use the previous definitions and results to define a measure of how irrational a number is. Essentially, the irrationality measure tells us how well a number can be approximated by rational numbers. Perhaps unintuitively, the more irrational the number, the \textit{better} it can be approximated by rationals.

\begin{definition}
The \textit{irrationality measure} is $\mu(\alpha) = \sup\{n \in \Rplus \mid |G_n(\alpha)| = \infty\}$   
\end{definition}

\begin{proposition}
Let $\alpha$ be algebraic of degree $n$, then $\mu(\alpha) \le n$. 
\end{proposition}

\begin{proof}
Suppose that $\mu(\alpha) > n$. Then there would exist some $k > n$ such that $|G_k(\alpha)| = \infty$ else the supremum would be $n$. However, $\mu(\alpha)$ is algebraic of order $n$ and $k > n$ so $\alpha$ is not $k$-approximable. Therefore, $\mu(\alpha) \le n$. 
\end{proof}

\begin{proposition}
If $\alpha \in \Q$ then $\mu(\alpha) = 1$
\end{proposition}

\begin{proof}
Take $\epsilon < 1$ and $\alpha = \frac{p}{q}$. Then, for any $n \in \N$ consider $p_n = np + 1$ and $q_n = nq$. Now, $\alpha - \frac{p_n}{q_n} = \frac{np - p_n}{nq} = \frac{1}{nq} = \frac{1}{q_n}$. Also, $q_n^\epsilon < q_n$. Therefore,
\[  0 < \Big| \alpha - \frac{p_n}{q_n} \Big| = \frac{1}{q_n} < \frac{1}{q_n^\epsilon}\]
Also, $\frac{p_n}{q_n} = \frac{p}{q} + \frac{1}{nq}$ so these solutions are all distinct. Thus, there are infinitely many $\epsilon$-good approximations of $\alpha$ so $\mu(\alpha) \ge \epsilon$ for every $\epsilon < 1$ so $\mu(\alpha) \ge 1$. Futhermore, $\alpha = \frac{p}{q}$ solves $f(x) = qx - p$ which has degree $1$ so $\mu(\alpha) \le 1$. Therefore, $\mu(\alpha) = 1$. 
\end{proof}

\begin{theorem}[Roth, 1955]
If $\alpha$ is algebraic then $\mu(\alpha) = 2$.
\end{theorem}

Klaus Roth was awarded the Fields Medal for the proof of this theorem. Needless to say, we will not prove it here.  

\begin{example}
The best known upper bound on the irrationality measure of $\pi$ was given in 2008 by Salikhov as $\mu(\pi) \le 7.6063$ 
\end{example}

\begin{example}
Borwein and Borwein proved in 1987 that $\mu(e) = 2$. 
\end{example}

\section{Liouville Numbers} 

\begin{definition}
$L$ is a \textit{Liouville number} if for every $n \in \Zplus$ there exists $\frac{p}{q}  \in \Q$ such that,
\[ 0 < \Big| \alpha - \frac{p}{q} \Big| < \frac{1}{q^n} \]
\end{definition}

\begin{proposition}
$L$ is Liouville if and only if $\mu(L) = \infty$
\end{proposition}

\begin{proof}
Let $\mu(L) = \infty$. then for any $n \in \Zplus$ there must be a $k > n$ such that $L$ is $k$-approximable because $\mu(L) > n$. Therefore, there is a solution $\frac{p}{q} \in \Q$ (in fact infinitely many) to the inequality,
\[0 < \Big| \alpha - \frac{p}{q} \Big| < \frac{1}{q^k} < \frac{1}{q^n}\] 
so $L$ is Liouville. Conversely, suppose that $L$ is Liouville. Then take any $k$ and choose $n > k$ with $n \in \Zplus$. Because $L$ is Liouville, for each $n$ there must be a solution $\frac{p_n}{q_n} \in \Q$ to,
\[0 < \Big| \alpha - \frac{p_n}{q_n} \Big| < \frac{1}{p_n^n} < \frac{1}{p_n^k}\]
Therefore, each $\frac{p_n}{q_n} \in G_k(L)$. I claim that this is an infinite number of distinct solutions. Otherwise, there would be a single value $\frac{p'}{q'}$ which appears infinitely many times. Thus,  
\[0 < \Big| \alpha - \frac{p'}{q'} \Big| < \frac{1}{(q')^n}\]
for infinitely many values of $n \in \Zplus$ which is impossible because,
\[\Big| \alpha - \frac{p'}{q'} \Big| \neq 0\]
but $\frac{1}{(q')^n} \to 0$. Therefore, $|G_k(L)|$ is infinite for every $k \in \Rplus$. Thus, $\mu(L) \ge k$ for all $k \in \Rplus$ so $\mu(L) = \infty$.
\end{proof}

\begin{theorem}
Liouville numbers are trancendental. 
\end{theorem}

\begin{proof}
Let $L$ be algebraic then there exists some $f \in Z[X]$ such that $f(L) = 0$. However, then $\mu(L) \le \deg{f}$ which is finite so $\mu(L) < \infty$ and thus $L$ is not Liouville. Thus, if $L$ is Liouville, then $L$ is not algebraic so $L$ is trancendental.
\end{proof}

\begin{theorem}
Take $b \in \Z$ with $b \ge 2$ and $a_k \in \{0, 1, 2, \cdots, b - 1\}$ for every $k \in \N$, then, the number,
\[L = \sum_{k = 1}^\infty \frac{a_k}{b^{k!}}\] is Liouville number and thus trancendental. In particular, we have uncountably many explict examples of trancendental numbers. 
\end{theorem}

\begin{proof}
Let $q_n = b^{n!}$ and $p_n = q_n \sum\limits_{k = 1}^n \frac{a_k}{b^{k!}}$ then,
\begin{align*}
0 < \Big| \alpha - \frac{p}{q} \Big| & = \sum_{k = n + 1}^\infty \frac{a_k}{b^{k!}} <   \sum_{k' = (n + 1)!}^\infty \frac{a_k}{b^{k'}} \le \sum_{k' = (n + 1)!}^\infty \frac{b-1}{b^{k'}} = \frac{b - 1}{b^{(n+1)!}} \sum_{k = 0}^\infty \frac{1}{b^{k}} \\ & = \frac{b - 1}{b^{(n+1)!}} \frac{b}{b-1} = \frac{b}{b^{(n+1)!}} \le \frac{b^{n!}}{b^{(n+1)!}}
\end{align*}
Now, $(n+1)! - n! = (n+1) \cdot n! - n! = n \cdot n!$ and thus,
\[ 0 < \Big| \alpha - \frac{p_n}{q_n} \Big| < \frac{b^{n!}}{b^{(n+1)!}} = \frac{1}{b^{n \cdot n!}} = \frac{1}{(b^{n!})^n} = \frac{1}{q_n^n}\]
Therefore, the inequality,
\[0 < \Big| \alpha - \frac{p_n}{q_n} \Big| < \frac{1}{q_n^n}\]
has a solution for every integer $n$. For any $k$ we can take an integer $n > k$ such that,
\[0 < \Big| \alpha - \frac{p_n}{q_n} \Big| < \frac{1}{q_n^n} < \frac{1}{q_n^k}\]
has a solution.  
\end{proof}
\end{document}
