\documentclass{article}
\usepackage[utf8]{inputenc}

\usepackage{amsthm, amssymb, amsmath, centernot}
\usepackage{dsfont}

\newcommand{\notimplies}{%
  \mathrel{{\ooalign{\hidewidth$\not\phantom{=}$\hidewidth\cr$\implies$}}}}

\renewcommand\qedsymbol{$\square$}
\newcommand{\cont}{$\boxtimes$}
\newcommand{\divides}{\mid}
\newcommand{\ndivides}{\centernot \mid}
\newcommand{\Z}{\mathbb{Z}}
\newcommand{\N}{\mathbb{N}}
\newcommand{\C}{\mathbb{C}}
\newcommand{\Zplus}{\mathbb{Z}^{+}}
\newcommand{\Primes}{\mathbb{P}}
\newcommand{\ball}[2]{B_{#1} \! \left(#2 \right)}
\newcommand{\cball}[2]{\overline{B}_{#1} \! \left(#2 \right)}
\newcommand{\Q}{\mathbb{Q}}
\newcommand{\R}{\mathbb{R}}
\newcommand{\Rplus}{\mathbb{R}^+}
\newcommand{\invI}[2]{#1^{-1} \left( #2 \right)}
\newcommand{\End}[1]{\text{End}\left( A \right)}
\newcommand{\legsym}[2]{\left(\frac{#1}{#2} \right)}
\renewcommand{\mod}[3]{\: #1 \equiv #2 \: \mathrm{mod} \: #3 \:}
\newcommand{\nmod}[3]{\: #1 \centernot \equiv #2 \: mod \: #3 \:}
\newcommand{\ndiv}{\hspace{-4pt}\not \divides \hspace{2pt}}
\newcommand{\finfield}[1]{\mathbb{F}_{#1}}
\newcommand{\finunits}[1]{\mathbb{F}_{#1}^{\times}}
\newcommand{\ord}[1]{\mathrm{ord}\! \left(#1 \right)}
\newcommand{\quadfield}[1]{\Q \small(\sqrt{#1} \small)}
\newcommand{\vspan}[1]{\mathrm{span}\! \left\{#1 \right\}}
\newcommand{\galgroup}[1]{Gal \small(#1 \small)}
\newcommand{\sm}{\! \setminus \!}
\newcommand{\topo}{\mathcal{T}}
\newcommand{\base}{\mathcal{B}}
\renewcommand{\bf}[1]{\mathbf{#1}}
\renewcommand{\Im}[1]{\mathrm{Im} \: #1}
\renewcommand{\empty}{\varnothing}
\newcommand{\indic}[1]{\mathds{1}_{#1}}


\newenvironment{definition}[1][Definition:]{\begin{trivlist}
\item[\hskip \labelsep {\bfseries #1}]}{\end{trivlist}}

\theoremstyle{theorem}
\newtheorem{theorem}{Theorem}[section]
\newtheorem{lemma}[theorem]{Lemma}
\newtheorem{corollary}[theorem]{Corollary}

\theoremstyle{definition}
\newtheorem*{problem}{Problem}

\theoremstyle{definition}
\newtheorem*{proposition}{Proposition}

\theoremstyle{remark}
\newtheorem*{fact}{Fact}

\theoremstyle{definition}
\newtheorem{example}{Example}[section]

\theoremstyle{remark}
\newtheorem{remark}{Remark}[subsection]

\begin{document}
\author{Benjamin Church}
\title{\Huge The Baire Category Theorem}

\maketitle
\tableofcontents
\newpage

\newcommand{\Rcomp}{\hat{\R}^+}
\newcommand{\Power}[1]{\mathcal{P}\left( #1 \right)}
\newcommand{\measure}[1]{\mu\left( #1 \right)}
\newcommand{\outmeasure}[1]{\mu^*\left( #1 \right)}

\section{Thomae's function: an Invitation}

Consider the curious real function,
\[ f(x) = 
\begin{cases}
\frac{1}{b} & x = \frac{a}{b} \in \Q \text{ writen in reduced form} 
\\
0 & x \notin \Q
\end{cases}
\]
It turns out that $f$ is continuous exactly on the irrational numbers $\R \setminus \Q$. 
(PROVE IT)
The natural question arises: can one construct a function which is exactly continuous on the rationals and discontinuous on the irrationals. This is quite a difficult question. I will spare you from many attempts and failures only remarking that we could make a function continuous on exactly the irrationals because a function measuring the ``complexity'' of a rational which went to zero as you approached any irrational exists, namely $\frac{a}{b} \mapsto \frac{1}{b}$. The difficulty begins when we realize that there is not an abvious candidate for a function measuring how ``rational'' an irrational number is which goes to zero as we approach any rational. The fact that no such function exists will be a consequence of the more systematic treatment to come.

\section{The Topology of Continuity in Euclidean Space}

Since there are not obvious candidate functions which satisfy the property we are looking for, we are going to need to be more systematic and rigorous in our quest. In particular, we need to nail down what the possible set of continuity points ``looks like.'' The notion of how a set ``looks'' is nicely captured by what we call its topology. In this section, we will introduce the basic notions of the topology of Euclidean space and consider its relationship to continuity.  

\begin{definition}
In $\R^n$ let $|\bullet|$ denote the usual Euclidean norm. The \textit{open ball} of radius $\delta > 0$ centered at $\bf{x}$ is the set,
\[ \ball{\delta}{\bf{x}} = \{ \bf{y} \in \R^n \mid |\bf{x} -\bf{y}|  < \delta \}\]
\end{definition}

\begin{definition}
We call a set $U \subset \R^n$ \textit{open} if for each point $\bf{x} \in U$ there exists $\delta > 0$ such that $\ball{\delta}{\bf{x}} \subset U$. A set is \textit{closed} if its complement is open.  
\end{definition} 

\begin{proposition}
The set of points at which $f : \R \to \R$ is continuous can be written as a countable intersection of open sets. 
\end{proposition}

\begin{proof}
For $n \in \Zplus$ define the set,
\[ C_n = \{ x \in \R \mid \exists \delta > 0 : y,z \in \ball{\delta}{x} \implies |f(y) - f(z)| < \tfrac{1}{n} \} \]
I claim that these sets are open. If $x \in C_n$ then there exists some $\delta$ associated to $x$ such that $|f(y) - f(z)| < \frac{1}{n}$ whenever $y,z \in \ball{\delta}{x}$. Take $\delta' = \frac{1}{2} \delta$. Then, I claim that $\ball{\delta'}{x} \subset C_n$. If $x' \in \ball{\delta'}{x}$ then for any $y,z \in \ball{\delta'}{x'}$ we know that $\ball{\delta'}{x'} \subset \ball{\delta}{x}$ since $|y - x'| < \delta' \implies |y - x| < |y - x'| + |x' - x| < \delta$. Thus, $|f(y) - f(z)| < \frac{1}{n}$ so $x' \in C_n$. Furthermore, I claim that the continuity points of $f$ can be written as,
\[ C(f) = \bigcap_{n = 1}^{\infty} C_n \]
This is because if $f$ is continuous at $x$ then for any $n$ choose $\varepsilon < \frac{1}{2n}$ then we get a $\delta_n > 0$ such that $f(\ball{\delta_n}{x}) \subset \ball{\varepsilon}{f(x)}$. Therefore, if $y,z \in \ball{\delta_n}{x}$ then $|f(y) - f(x)| < \varepsilon$ and $|f(z) - f(x)| < \varepsilon$. This implies that,
\[ |f(y) - f(z)| < |f(y) - f(x)| + |f(x) - f(z)| < 2 \varepsilon < \tfrac{1}{n} \]
so $x \in C_n$ for each $n$. Conversely, if $x \in C_n$ for each $n$ then, given any $\varepsilon > 0$ take $n$ such that $\frac{1}{n} < \varepsilon$ and get a $\delta$ since $x \in C_n$ such that 
\[ y,z \in \ball{\delta}{x} \implies |f(y) - f(z)| < \tfrac{1}{n} < \varepsilon\]
Therefore, $|x - y| < \delta \implies |f(x) - f(y)| < \varepsilon$ for any positive $\varepsilon$. Therefore, $f$ is continuous at $x$ i.e. $x \in C(f)$ so we have equality. 
\end{proof}

\begin{remark}
This may not seem like a much of a restriction on the possible set of continuity points but it turns out to determine this set's membership in the Borel Hierarchy. 
\end{remark}

\begin{definition}
We call a set $G_{\delta}$ if it is the countable intersection of open sets and we call it $F_{\sigma}$ if it is the countable union of closed sets.  
\end{definition}

\begin{proposition}
\[ A \in G_{\delta} \iff A^C \in F_{\sigma} \]
\end{proposition}

\begin{proof}
If $A \in G_{\delta}$ then for some open $U_i$,
\[ A = \bigcap_{i = 1}^{\infty} U_i \implies A^C = \bigcup_{i = 1}^{\infty} U_i^C \implies A^C \in F_{\sigma} \]
since $U_i^C$ is closed. Conversely, if $A^C \in F_{\sigma}$ then for some closed $D_i$,
\[ A^C = \bigcup_{i = 1}^\infty D_i \implies A = \bigcap_{i = 1}^\infty D_i^C \implies A \in G_{\delta} \]
since $D_i^C$ is open. 
\end{proof}

\begin{theorem}
Let $A \subset \R$ then there exits a function $f : \R \to \R$ such that $C(f) = A$ if and only if $A$ is $G_{\delta}$. 
\end{theorem}

\begin{proof}
We have shown that if $C(f) = A$ then $A$ is $G_{\delta}$. Conversely, suppose we can write,
\[ A = \bigcap_{i = 1}^{\infty} A_i \]
with open sets $A_i$. Define the sequence of open sets,
\[ U_n = \bigcap_{i = 1}^n A_i \]
then define the function,
\[ f(x) 
= 
\begin{cases}
0 & x \in A
\\
\frac{1}{n} & x \in U_{n-1} \quad \text{and} \quad x \notin U_{n} \quad \text{and} \quad x \in \Q
\\
- \frac{1}{n} & x \in U_{n-1} \quad \text{and} \quad x \notin U_{n} \quad \text{and} \quad x \notin \Q
\end{cases} \]
We need to prove that $C(f) = A$. If $x \in A$ then $x \in U_n$ so for each $n$ there exists $\delta_n$ such that $\ball{\delta_n}{x} \subset U_n$. For any $\epsilon > 0$ take $n \in \Zplus$ such that $\frac{1}{n} < \epsilon$ and consider $\delta_n$. If $y \in \ball{\delta_n}{y} \subset U_n$ then the minimal $m$ such that $y \notin U_m$ is greater than $n + 1$ so $|f(y) - f(x)| = |f(y)| \le \frac{1}{n + 1} < \epsilon$ so $A \subset C(f)$. Furthermore, if $x \notin A$ then $f(x) = \pm \frac{1}{n}$ for some $n$. In every neighborhood of $x$ there will be a $y$ with the opposite rationality as $x$ such that $f(y) \le 0$. Thus, $|f(x) - f(y)| \ge \frac{1}{n}$ for some point in any neighborhood contradicting continuity for any choice of $\epsilon < \frac{1}{n}$. Thus, $C(f) \subset A$. 
\end{proof}

\begin{remark}
For Thomae's function, we can take the following decomposition of $C(f) = \R \setminus \Q$ as,
\[ \R \setminus \Q = \bigcap_{n = 1}^{\infty} \R \setminus \left\{ \frac{a}{n} \: \middle| \: a \in \Z \right\} \]
The sets,
\[ A_n = \R \setminus \left\{ \frac{a}{n} \: \middle| \: a \in \Z \right\} \]
are unions of open intervals and thus open. Furthermore, if $x \in U_{n-1}$ but $x \notin U_{n}$ then $x = \frac{a}{n}$ in least terms. Therefore, $f_A$ constructed above is exactly Thomae's function since if $x \notin \R \setminus \Q$ then $x \in \Q$ so the output is always positive.
\end{remark}

\begin{remark}
Therefore, we have reduced our problem to determining which subsets of $\R$ are $G_{\delta}$. This turns out to be much more sublte than it may appear and will lead to the Baire category theorem, one of the most powerful theorems in modern analysis. 
\end{remark}

\section{General Topology and Metric Spaces}

\newcommand{\T}{\mathcal{T}}

\subsection{Point-Set Topology}

\begin{definition}
A \textit{topology} on $X$ is a collection of subsets $\T \subset \Power{X}$ such that,
\begin{enumerate}
\item $\varnothing, X \in \T$ 

\item If $\mathcal{C} \subset \T$ is any collection of sets in $\T$ then their union is an element of $\T$,
\[ \bigcup_{U \in \mathcal{C}} U \in \T \]

\item If $A, B \in \T$ then $A \cap B \in \T$ so all finite intersections are in $\T$
\end{enumerate}
We call the sets in $\T$ the \textit{open} sets of $X$ and the closed sets of $X$ are those whose complement is in $\T$ i.e. $D \subset X$ is closed exactly when $X \setminus D$ is open. Therefore, a toplogy on $X$ is simply a choice of which sets to call open making sure that unions and finite intersections of open sets are open and that the empty set and full set are open. Furthermore, if $\T$ is a topology on $X$ we call the pair $(X, \T)$ a topological space. 
\end{definition}

\begin{remark}
There is no reason that a set need be either closed or open. In fact, many sets are neither open nor closed, for example the interval $[a, b)$. Also, some sets are \textit{both} open and closed, for example $\varnothing$ and $X$. Such sets are hilariously refered to as ``clopen.''  
\end{remark}

\begin{definition}
A function $f : (X, \T_X) \to (Y, \T_Y)$ between topological spaces is \textit{continuous} if for any $U \in \T_Y$ we have $f^{-1}(U) \in \T_X$. That is, the pre-image of open sets is open. Similarly, $f$ is continuous at some point $x \in X$ if for any neighborhood $U$ of $f(x)$, i.e. an open set containing $f(x)$, its preimage $f^{-1}(U)$ contains a neighborhood of $x$.  
\end{definition}

\begin{definition}
The \textit{closure} of a set $A \subset X$, denoted $\overline{A}$, is the intersection of all closed sets containing $A$. 
\end{definition}

\begin{proposition}
For $A \subset X$, its \textit{closure} $\overline{A}$ is the smallest closed set containing $A$. 
\end{proposition}

\begin{definition}
A set $D \subset X$ is \textit{dense} in $X$ if $\overline{D} = X$.
\end{definition}

\begin{proposition}
A set $D \subset X$ is dense if and only if it intersects with every nonempty open set of $X$. 
\end{proposition}

\begin{proof}
Let $U$ be open then $D \cap U = \varnothing \iff D \subset U^C \iff \overline{D} \subset U^C$ since $U^C$ is closed. Therefore, $D$ intersects with every nonempty open set iff $\overline{D} = X$.  
\end{proof}

\begin{remark}
For example, the rationals $\Q \subset \R$ are dense in $\R$ since if $\Q \subset D$ with $D$ closed then $D^C$ must be an open set which intersects no rationals. However, every open set contains an interval about each point and thus contains some rational unless it is empty. Thus, $D = \R$. 
\end{remark}

\begin{definition}
The \textit{interior} of a set $A \subset X$, denoted $A^\circ$, is the union of all open sets contained in $A$. 
\end{definition}

\begin{definition}
A set $A \subset X$ is \textit{nowhere dense} if its closure has empty interior.    
\end{definition}


\begin{definition}
The \textit{boundary} of a set $A \subset X$, denoted $\partial A$, is $\partial A = \overline{A} \setminus A^{\circ}$. 
\end{definition}

\begin{definition}
$x \in X$ is a \textit{limit point} of $A$ if for every open $U \subset X$ containing $x$ we have $U \cap (A \setminus \{x\}) \neq \varnothing$. 
\end{definition}

\begin{definition}
$x \in A$ is an \textit{isolated point} of $A$ if $x$ is not a limit point of $A$.
\end{definition}

\begin{definition}
A set $A \subset X$ is \textit{dense in itself} if it contains no isolated points.    
\end{definition}

(DO THIS PROPOSITION)

\begin{lemma}[Categorization of Toplogical Sets]
\leavevmode
\begin{enumerate}
\item 
\end{enumerate}
\end{lemma}

\subsection{Metric Spaces}

\begin{definition}
A \textit{metric space} is a space $X$ and a function $d : X \times X \to \R$ such that for all $x,y,z \in X$,
\begin{enumerate}
\item $d(x,y) \ge 0$
\item $d(x,y) = 0 \iff x = y$
\item $d(x,y) = d(y,x)$
\item $d(x,z) + d(z,y) \ge d(x,y)$
\end{enumerate} 
\end{definition}

\begin{definition}
Let $X$ be a metric space and $x \in X$. The \textit{open ball} of radius $\delta > 0$ centered at $x$ is the set,
\[ \ball{\delta}{x} = \{ y \in X \mid d(x,y) < \delta \}\]
\end{definition}

\begin{proposition}
Any metric space $(X, d)$ is a topological space with the induced topology,
\[ \T = \{ U \subset X \mid \forall p \in U : \exists \delta > 0 : \ball{\delta}{p} \subset U \} \]
\end{proposition}

\begin{proof}
(DO PROOF)
\end{proof}

\begin{definition}
A sequence $(a_i)$ in a metric space is \textit{Cauchy} if for any $\varepsilon > 0$ there exists a natural number $N_{\varepsilon}$ such that for any $n,m > N_{\varepsilon}$ we have $d(a_n, a_m) < \varepsilon$. 
\end{definition}

\begin{definition}
A sequence $(a_i)$ in a metric space has limit $L$, denoted $\lim\limits_{n \to \infty} a_n = L$ if for any $\varepsilon > 0$ there exists a natural number $N_{\varepsilon}$ such that for all $n > N_{\varepsilon}$ we have $d(a_n, L) < \varepsilon$. If a sequence has some limit we say that it \textit{converges}.
\end{definition}

\begin{proposition}
If a sequence converges then it is Cauchy. 
\end{proposition}

\begin{proof}
Suppose that $a_n \to L$. Then for any $\varepsilon > 0$ there exists $N_{\varepsilon/2} \in \N$ such that, $d(a_n, L) < \varepsilon/2$ for all $n > N_{\varepsilon}$. Therefore, if $n,m > N_{\varepsilon/2}$ then,
\[ d(a_n, a_m) < d(a_n, L) + d(L, a_m) < \frac{\varepsilon}{2} + \frac{\varepsilon}{2} = \varepsilon \]
so $(a_n)$ is Cauchy. 
\end{proof}

\begin{remark}
In general, the converse is false. For example, in the metric space $\Q$, decimal approximations of $\pi$ of increasing length form a Cauchy sequence that does not converge to any element of $\Q$ since $\pi$ is irrational.  
\end{remark}

\begin{definition}
A metric space $(X, d)$ is \textit{complete} if every Cauchy sequence with respect to the metric $d$ converges in $X$ i.e. converges to some limit in $X$.  
\end{definition}

\begin{theorem}
Euclidean space $\R^n$ with the Euclidean distance $d(\bf{x}, \bf{y}) = |\bf{x} - \bf{y}|$ is a complete metric space. 
\end{theorem}

\begin{remark}
In fact, one possible definition of $\R^n$ is as the completion of $\Q^n$ with the standard metric i.e. the smallest complete metric space containing $\Q^n$ whose metric extends the standard metric on $\Q$ i.e. the metric of the larger complete space resticted to $\Q$ is just the standard metric. 
\end{remark}

\newcommand{\diam}[1]{\mathrm{diam}\left(#1\right)}

\begin{definition}
Let $X$ be a metric space and $A \subset X$. Then,
\[ \diam{A} = \sup\{ d(x,y) \mid x,y \in A \} \]
\end{definition}

\begin{theorem}
A metric space $(X, d)$ is complete if and only if every descending chain of closed sets,
\[ F_0 \supset F_1 \supset F_2 \supset \cdots \]
such that $\diam{F_n} \to 0$ has a unique point in its intersection. 
\end{theorem}

\begin{proof}
(DO PROOF)
\end{proof}

\subsection{Continuity of Maps Between Metric Spaces}

\begin{proposition}
A map $f : (X, d_X) \to (Y, d_Y)$ of metric spaces is continuous at $x \in X$ if and only if for every $\varepsilon > 0$ there exists $\delta > 0$ such that,
\[ f(\ball{\delta}{x}) \subset \ball{\varepsilon}{f(x)} \] 
Which is equivalent to,
\[ d(x, y) < \delta \implies d(f(x), f(y)) < \varepsilon \]
\end{proposition}

\begin{proof}
(DO PROOF)
\end{proof}

\begin{definition}
Let $f : (X, d_X) \to (Y, d_Y)$ be a map of metric spaces. Then the \textit{oscillation} of $f$ over a set $S \subset X$ is,
\[ \omega_f(S) = \diam{f(S)} = \sup\{d(f(x), f(y)) \mid x,y \in S \} \]
Furthermore, we may define the oscillation at a point $x \in X$ via,
\[ \omega_f(x) = \inf_{\delta > 0}\omega_f(\ball{\delta}{x}) \]
\end{definition}

\begin{lemma}
Let $f : (X, d_X) \to (Y, d_Y)$ be a map of metric spaces then the \textit{continuity points} of $f$ are equivalent to,
\[ C(f) = \{ x \in X \mid \omega_f(x) = 0 \} \]
and likewise the \textit{discontinuity points} are $X \setminus C(f)$. 
\end{lemma}

\begin{proof}
Suppose that $\omega_f(x) = 0$. Then for any $\varepsilon > 0$ there must exist a $\delta > 0$ such that $\omega_f(\ball{\delta}{x}) < \varepsilon$ and thus $\diam{f(\ball{\delta}{x})} < \varepsilon$ so $f(\ball{\delta}{x}) \subset \ball{\varepsilon}{f(x)}$ since no points of $f(\ball{\delta}{x})$ are seperated by more than $\varepsilon$. Thus, $f$ is continuous at $x$. Conversely, assuming that $f$ is continuous at $x$ we know that given any $\varepsilon > 0$ we can find $\delta_{\varepsilon/2} > 0$ such that $f(\ball{\delta}{x}) \subset \ball{\varepsilon/2}{f(x)}$. Therefore, for all $y,z \in \ball{\varepsilon/2}{f(x)}$ we know that,
\[ d_Y(y,z) \le d_Y(y, f(x)) + d_Y(f(x), z) < \varepsilon \]
so $\omega_f(\ball{\delta}{x}) = \diam{f(\ball{\delta}{x})} < \varepsilon$. Therefore, 
\[ \omega_f(x) = \inf_{\delta > 0} \omega_f(\ball{\delta}{x}) < \varepsilon\]
for all $\varepsilon > 0$ which implies that $\omega_f(x) = 0$. Thus, 
\[ x \in C(f) \iff f \text{ is continuous at } x \]  
\end{proof}

\begin{theorem}
Let $f : X \to Y$ be a map of metric spaces. Then the set of continuity points $C(f)$ is $G_{\delta}$ and the set of discontinuity points $C(f)^C$ is $F_{\sigma}$. 
\end{theorem}

\begin{proof}
Clearly,
\[ C(f) = \{ x \in X \mid \omega_f(x) = 0\} = \bigcap_{n = 1}^{\infty} \{ x \in X \mid \omega_f(x) < \tfrac{1}{n} \} \]
so it suffices to show that each,
\[ C_n(f) = \{ x \in X \mid \omega_f(x) < \tfrac{1}{n} \} \]
is open to prove that $C(f)$ is the countable intersection of open sets i.e. $G_{\delta}$. Suppose that $\omega_f(x) < \frac{1}{n}$ or equivalently that there exists some $\delta > 0$ such that $\omega_f(\ball{\delta}{x}) < \frac{1}{n}$. For $x' \in \ball{\delta/2}{x}$ consider,
\[ \omega_f(\ball{\delta/2}{x'}) = \diam{f(\ball{\delta/2}{x'})} \]
If $y,z \in \ball{\delta/2}{x'} \subset \ball{\delta}{x}$ then $d(f(x), f(z)) \le \diam{f(\ball{\delta}{x})} < \frac{1}{n}$. Then, 
\[ \omega_f(x') \le \diam{f(\ball{\delta/2}{x'})}  = \diam{f(\ball{\delta/2}{x'})} < \tfrac{1}{n} \]
which implies that $x' \in C_n(f)$. Thus, $\ball{\delta/2}{x} \subset C_n(f)$ so $C_n(f)$ is open. Therefore, $C(f)$ is $G_{\delta}$ and thus $X \setminus C(f)$ is $F_{\sigma}$ since it is a countable union of closed sets. 
\end{proof}

\begin{remark}
It now remains to understand the properties of $G_{\delta}$ and $F_{\sigma}$ sets. Specifically, we will be interested in \textit{dense} $G_{\delta}$ and $F_{\sigma}$ sets. 
\end{remark}

\section{The Baire Category Theorem}

\begin{definition}
A set $A \subset X$ is \textit{meager} if it is the countable union of nowhere dense sets. We say $A \subset X$ is \textit{comeager} if its complement is meager or equivalently if it is the countible intersection of sets with dense interiors.  
\end{definition}

\begin{proposition}
Any subset of a meager set is meager.
\end{proposition}

\begin{proof}
Let $S \subset M$ and $M$ be meager then there exit nowhere dense sets $N_i$ such that,
\[ S \subset M = \bigcup_{i = 1}^{\infty} N_i \implies S = \bigcup_{i = 1}^{\infty} N_i \cap S \]
then $N_i \cap S$ is nowhere dense since $(\overline{N_i \cap S})^\circ \subset (\overline{N_i})^\circ = \varnothing$.
\end{proof}

\begin{remark}
A set $A \subset X$ being nowhere dense is equivalent to its intersection with any nonempty open set $U \subset X$ not being dense in $U$. If $A$ is dense in some $U$ then for any closed $D \subset X$ if $A \cap U \subset D \cap U$ then $U \subset D$. Therefore, if $A \subset D$ then $U \subset D$ so $D$ has nonempty interior. Converely if $\overline{A}$ has nonempty interior then there must exist a nonempty open set $U$ contained in every closed set containing $A$. However, if $A \cap U \subset D \cap U$ then $A \subset D \cup U^C$ which is closed (since $U$ is open) so $U \subset D \cup U^C$ which implies that $U \subset D$ because $U$ is contained in $\overline{A}$. Thus, $D \cap U = U$ and thus $A \cap U$ is dense in $U$.  
\end{remark}

\begin{remark}
It is clear that if $U$ is open then $U$ is dense iff it has dense interior and likewise if $D$ is closed then $D$ has empty interior iff $D$ is nowhere dense. 
\end{remark}

\begin{proposition}
A topological space $X$ is a \textit{Baire space} if one of the following equivalent properties holds,
\begin{enumerate}
\item every countable intersection of open dense sets is dense
\item every countable union of closed nowhere dense sets has empty interior
\item every meager set has empty interior
\item every comeager set is dense.
\item every nonempty open set is nonmeager
\end{enumerate}
\end{proposition}

\begin{proof}
I will show that these properties are equivalent.
A set $U$ is a dense open set if and only if $U^C = X \setminus U$ is a closed nowhere dense set because $\overline{U^C}^\circ = (U^C)^\circ = (\overline{U})^C$. Therefore, 
\[ U \text{ is dense } \iff \overline{U} = X \iff \overline{U^C}^\circ = (\overline{U})^C = \varnothing \iff U^C \text{ is nowhere dense} \]
Therefore the first two properties are equivalent because,
\[ \overline{\left( \bigcap_{i = 1}^\infty U_i \right)} = \overline{\left( \bigcup_{i = 1}^{\infty} U_i^C \right)^C} = \left[ \left( \bigcup_{i = 1}^{\infty} U_i^C \right)^\circ \right]^C \]
Therefore,
\[ \overline{\left( \bigcap_{i = 1}^\infty U_i \right)} = X \iff \left( \bigcup_{i = 1}^{\infty} U_i^C \right)^\circ = \varnothing \]
Thus, every intersection of dense open sets is dense if and only if any union of closed nowhere dense sets has empty interior. 
\bigskip\\
Now, I will show that properties 2 and 3 are equivalent. Assuming 2, if $M$ is a meager set then,
\[ M = \bigcup_{i = 1}^\infty N_i \]
where $N_i$ is nowhere dense. Then,
\[ M \subset \bigcup_{i = 1}^\infty \overline{N_i}\] 
By 2, the union has no interior points and thus neither does $M$ proving 3. Conversely, assuming 3, a countible union of closed nowhere dense sets is a meager set and thus, by 3, has empty interior proving 2.
\bigskip\\
Now, I will show that properties 1 and 5 are equivalent. Assuming 1, if $A$ is comeager then it is a countable intersection of sets with dense interior which are thus dense. Therefore, by 1, $A$ is dense proving 4. Conversely, assuming 4, if $A$ is the countable intersection of open dense sets then it is comeager (since open dense sets trivially have dense interiors) and thus dense by 4, proving 1. Furthermore, 3 and 4 are clearly equivalent because the complement of a meager set $M$ is dense exactly when $M$ has empty interior.
\bigskip\\
Finally, I will show that properties 3 and 5 are equivalent. Assuming 3, if $U$ is open and meager then its interior, by 3, is empty so $U = U^\circ = \varnothing$ since it is open proving 5. Conversely, assuming 5, if $M$ is meager then $M^\circ \subset M$ and thus is meager. However, $M^\circ$ is also open so, by 5, $M^\circ$ is empty proving 3. 
\end{proof}

\begin{theorem}[Baire]
$X$ is a Baire space if either of the folowing sufficient conditions holds,
\begin{enumerate}
\item $X$ is a complete metric space
\item $X$ is a locally compact Hausdorff space.
\end{enumerate}
\end{theorem}

\begin{proof}
Let $X$ be a complete metric space. Suppose that $\{ U_i \}$ is a countable collection of dense open sets. We want to show that,
\[ U = \bigcap_{i = 1}^\infty U_i \]
is dense. Let $W$ be a nonempty open set of $X$. Since $U_1$ is dense, $\exists x_1 \in U_1 \cap W$  but $U_1 \cap W$ is open so $\exists r_1 \in (0, 1) : \cball{r_1}{x_1} \subset U_1 \cap W$. Now we define a sequence recursively. Given $x_n$ and $r_n$, we know that $U_{n+1} \cap \ball{r_n}{x_n}$ is open and nonempty since $U_{n+1}$ is dense. Thus, there exists $x_{n+1} \in U_{n+1} \cap \ball{r_n}{x_n}$ and since the set is open, 
\[ \exists r_{n+1} \in (0, \tfrac{1}{n})  : \cball{r_{n+1}}{x_{n+1}} \subset U_{n+1} \cap \ball{r_n}{x_n} \]
Therefore, there exist sequences\footnote{We can define any finite number of terms in such a sequence recursively. However, to get the entire infinite sequence we need to invoke the axiom of choice. In fact, the Baire category theorem is equivalent to a weak form of the axiom of choice.} $(x_n)$ in $X$ and $(r_n)$ in $\R$ such that,
\[ \cball{r_{n+1}}{x_{n+1}} \subset U_{n+1} \cap \cball{r_{n}}{x_{n}} \]
and $r_n \in (0, \tfrac{1}{n})$. Since $\cball{r_n}{x_n} \subset \ball{r_m}{x_m} \subset U_m \cap W$ when $n > m$. Therefore, $x_n \in U_1 \cap \cdots \cap U_n$ and $x_n \in W$. Furthermore, $d(x_n, x_m) < r_m < \tfrac{1}{m}$ so for any $\varepsilon > 0$ if we choose $n,m > N$ such that $\frac{1}{N} < \varepsilon$ then $d(x_n, x_m) < \frac{1}{N} < \varepsilon$ so $(x_n)$ is Cauchy. Therefore, since $X$ is complete $x_n$ converges to a limit $x \in X$. Since $x_n \in \cball{r_m}{x_m}$ for all $n > m$ and $\cball{r_m}{x_m}$ is closed, we must have $x \in \cball{r_m}{x_m} \subset U_m \cap W$ so $x \in U_m$ for all $m$ and $x \in W$. Therefore, $x \in U \cap W$ so $U$ intersects any nonempty open set and is therefore dense.  
(PROOF FOR LCH)
\end{proof}

\begin{corollary}
In a Baire space $X$, $Q$ and $Q^C$ cannot both be dense $G_{\delta}$ sets.
\end{corollary}

\begin{proof}
Suppose that $Q$ and $Q^C$ were dense $G_{\delta}$ sets then both $Q$ and $Q^C$ would be the intersection of countably many dense\footnote{Since $Q$ is dense, if we can write $Q$ as an intersection of $\{ U_i \}$ then $Q \subset U_i$ so $\overline{U_i} \supset \overline{Q} = X$ so each $U_i$ is dense.} open sets. Therefore, $Q \cap Q^C$ is the countable intersection of dense open sets. However, $Q \cap Q^C = \varnothing$ is not dense contradicting $X$ being a Baire space. 
\end{proof}

\section{An Answer At Last}

\begin{theorem}
There does not exist a real function continuous exactly on $\Q$.
\end{theorem}

\begin{proof}
We know that the set of continuity points of any function must be $G_{\delta}$ so it suffices to prove that $\Q$ is not $G_{\delta}$. Since $\R$ is a complete metric space it is a Baire space. Therefore, no set and its complement can be dense $G_{\delta}$ sets. However, both $\Q$ and $\R \setminus \Q$ are dense and,
\[ \R \setminus \Q = \bigcap_{q \in \Q} \R \setminus \{q\} \]
is $G_{\delta}$ since $\Q$ is countible and $\R \setminus \{q\}$ is open. Therefore, $\Q$ cannot be $G_{\delta}$. 
\end{proof}

\begin{remark}
This proof easily generalizes to show that if $Q$ is a dense $F_{\sigma}$ set with dense complement then $Q$ cannot be the set of continuity points of any function. 
\end{remark}

\section{The Borel Hierarchy} 

We have demonstrated the importance of $G_{\delta}$ and $F_{\sigma}$ sets in proving facts about continuity points. However, we can generalize the definitions of these classes to form an infinite hierarchy of increasingly complex sets. 

\newcommand{\BHS}[1]{\mathbf{\Sigma}^0_{#1}}
\newcommand{\BHP}[1]{\mathbf{\Pi}^0_{#1}}
\newcommand{\BHD}[1]{\mathbf{\Delta}^0_{#1}}

\begin{definition}
The Borel Hierarchy of a topological space $X$ consists of classes of sets $\BHS{\alpha}$, $\BHP{\alpha}$, and, $\BHD{\alpha}$ for any countable ordinal\footnote{An ordinal is a generalization of the counting numbers $\N$ to counting infinite things. In particular, $\omega$ is the set of natural numbers and $\omega + 1$ is the union of that set and $\{ \omega \}$. Furthermore, $\omega_1$ is the set of all countable ordinals and which is, itself, an uncountable ordinal} greater than zero $0 < \alpha < \omega_1$ which are defined by the following inductive rules,
\begin{enumerate}
\item A set is $\BHS{1}$ exactly when it is open.
\item A set is $\BHP{\alpha}$ if and only if complement is $\BHS{\alpha}$.
\item A set is $\BHD{\alpha}$ exactly when it is both $\BHS{\alpha}$ and $\BHP{\alpha}$.
\item A set $A$ is $\BHS{\alpha}$ if and only if there is a countable sequence $\{ A_i \}$ such that $A$ is $\BHP{\alpha_i}$ for $\alpha_i < \alpha$ and,
\[ A = \bigcup_{i = 1}^{\infty} A_i \]
\end{enumerate}
\end{definition}

\begin{remark}
It follows that a set $A$ is $\BHP{\alpha}$ if and only if there is a countable sequence $\{ A_i \}$ such that $A$ is $\BHS{\alpha_i}$ for $\alpha_i < \alpha$ and,
\[ A = \bigcap_{i = 1}^{\infty} A_i \]
\end{remark}

\begin{remark}
We see that $F_{\sigma} = \BHS{2}$ since these are exactly the countable unions of closed sets and $G_{\delta} = \BHP{2}$ is their complements which is exactly the countable intersection of open sets. 
\end{remark}

\begin{remark}
Since $\BHD{\alpha} = \BHS{\alpha} \cap \BHP{\alpha}$ we see that no $\BHD{2}$ subset of a Baire space can be dense and have dense complement. For example, $\R \setminus \{0\}$ is open and thus $G_{\delta} = \BHP{2}$ but,
\[ \R \setminus \{0\} = \bigcup_{i = 1}^{\infty} (-\infty, \tfrac{1}{i}] \cup [\tfrac{1}{i}, \infty) \]
which is $F_{\sigma} = \BHS{2}$ so $\R \setminus \{0\}$ is $\BHD{2}$. However, notice that its complement is just the point zero which is, of course, not dense. 
\end{remark}

\begin{lemma}
For and ordinal $\alpha$, we have $\BHS{\alpha} \cup \BHP{\alpha} \subset \BHD{\alpha + 1}$. Therefore, higher ordinal classes of the Hierarchy  contain all lower classes.
\end{lemma}

\begin{proof}
If $A \in \BHS{\alpha}$ then clearly $A \in \BHP{\alpha + 1}$ since it is trivially an intersection of $\BHS{\alpha}$ sets. Furthermore, we can write,
\[ A = \bigcup_{i = 1}^{\infty} A_i \]
with $A_i \in \BHP{\alpha_i}$ for $\alpha_i < \alpha < \alpha + 1$ which means that $A \in \BHS{\alpha+1}$. Therefore, $A \in \BHD{\alpha + 1}$. Furthermore, if $A \in \BHP{\alpha}$ then clearly $A \in \BHS{\alpha + 1}$ since it is trivially a countable union of $\BHP{\alpha}$ sets. Furthermore, we can write,
\[ A = \bigcap_{i = 1}^{\infty} A_i \]
with $A_i \in \BHS{\alpha_i}$ for $\alpha_i < \alpha < \alpha + 1$ which means that $A \in \BHP{\alpha+1}$ so $A \in \BHD{\alpha + 1}$.
\end{proof}

\begin{definition}
A topological space $X$ is \textit{seperable} if there exists a countable dense subset. We call a seperable complete metric space a \textit{Polish space}.
\end{definition}

\begin{proposition}
Let $X$ be an uncountable Polish space. Then $\BHS{\alpha} \not\subset \BHP{\alpha}$ and $\BHP{\alpha} \not \subset \BHP{\alpha}$ for any $\alpha < \omega_1$. Therefore, the Borel Hierarchy does not collapse. 
\end{proposition}

Now let us go further!
\newcommand{\Borel}[1]{\mathcal{B}(X)}

\begin{proposition}
Define,
\begin{align*}
\BHS{\omega_1} & = \bigcup_{\alpha < \omega_1} \BHS{\alpha} 
\\
\BHP{\omega_1} & = \bigcup_{\alpha < \omega_1} \BHP{\alpha} 
\\
\BHD{\omega_1} & = \bigcup_{\alpha < \omega_1} \BHD{\alpha} 
\end{align*}
Then,
\[ \Borel{X} = \BHS{\omega_1} = \BHP{\omega_1} = \BHD{\omega_1} \]
is called the $\sigma$-algebra of Borel sets and $\mathcal{B}(X)$ is closed under countable unions, countable intersections, and complements.
\end{proposition}

\begin{proof}
If $A \in \BHS{\omega_1}$ or $A \in \BHP{\omega_1}$ then $A \in \BHS{\alpha}$ or $A \in \BHP{\alpha}$ for some $\alpha < \omega_1$. But we know that $\BHS{\alpha} \cup \BHP{\alpha} \subset \BHD{\alpha+1} \subset \BHD{\omega_1}$ so $A \in \BHD{\omega_1}$. However, we clearly have $\BHD{\omega_1} \subset \BHS{\omega_1}$ and $\BHD{\omega_1} \subset \BHP{\omega_1}$ since $\BHD{\alpha} = \BHS{\alpha} \cap \BHP{\alpha}$.  
\bigskip\\
Since $\Borel{X} = \BHS{\omega_1} = \BHP{\omega_1} = \BHD{\omega_1}$ it is closed under complements since $A \in \BHD{\omega_1}$ implies that $A \in \BHS{\omega_1} \cap \BHP{\omega_1}$ so $A^C \in \BHP{\omega_1} \cap \BHS{\omega_1} = \Borel{X}$. It remains to check that $\Borel{X}$ is closed under countable unions (since we get countable intersections from this and complements). Suppose that $\{ A_i \}$ is a countable sequence of Borel sets then $A_i \in \BHD{\omega_1}$ so $A_i \in \BHD{\alpha_i}$ for some $\alpha_i < \omega_1$. I claim that the supremum $\alpha = \sup{\alpha_i}$ exists and is a countable ordinal $\alpha < \omega_1$. Assuming this, the union,
\[ A = \bigcup_{i = 1}^{\infty} A_i \]
is $\BHS{\alpha + 1}$ because $A_i \in \BHD{\alpha_i} \subset \BHP{\alpha_i}$ for $\alpha_i < \alpha + 1$. Thus $A \in \BHS{\alpha+1} \subset \BHS{\omega_1} = \Borel{X}$. Now I need to prove my claim. I define,
\[ \alpha = \sup{\alpha_i} = \bigcup_{i = 1}^{\infty} \alpha_i \]
which is countable since it is a countable union of countable sets and $\alpha \supset \alpha_i$ so $\alpha \ge \alpha_i$.
\end{proof}


\end{document}