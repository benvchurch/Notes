\documentclass[12pt]{extarticle}
\usepackage[utf8]{inputenc}
\usepackage[english]{babel}
\usepackage[a4paper, total={7.25in, 9.5in}]{geometry}
\usepackage{tikz-feynman}
\tikzfeynmanset{compat=1.0.0} 
\usepackage{subcaption}
\usepackage{float}
\floatplacement{figure}{H}
\usepackage{simpler-wick}
 
\newcommand{\field}{\hat{\Phi}}
\newcommand{\dfield}{\hat{\Phi}^\dagger}
 
\usepackage{amsthm, amssymb, amsmath, centernot}
\usepackage{mathrsfs}
\usepackage{slashed}
\newcommand{\notimplies}{%
  \mathrel{{\ooalign{\hidewidth$\not\phantom{=}$\hidewidth\cr$\implies$}}}}
 
\renewcommand\qedsymbol{$\square$}
\newcommand{\cont}{$\boxtimes$}
\newcommand{\divides}{\mid}
\newcommand{\ndivides}{\centernot \mid}
\newcommand{\Z}{\mathbb{Z}}
\newcommand{\N}{\mathbb{N}}
\newcommand{\C}{\mathbb{C}}
\newcommand{\Zplus}{\mathbb{Z}^{+}}
\newcommand{\Primes}{\mathbb{P}}
\newcommand{\ball}[2]{B_{#1} \! \left(#2 \right)}
\newcommand{\Q}{\mathbb{Q}}
\newcommand{\R}{\mathbb{R}}
\newcommand{\Rplus}{\mathbb{R}^+}
\newcommand{\invI}[2]{#1^{-1} \left( #2 \right)}
\newcommand{\End}[1]{\text{End}\left( A \right)}
\newcommand{\legsym}[2]{\left(\frac{#1}{#2} \right)}
\renewcommand{\mod}[3]{\: #1 \equiv #2 \: \mathrm{mod} \: #3 \:}
\newcommand{\nmod}[3]{\: #1 \centernot \equiv #2 \: mod \: #3 \:}
\newcommand{\ndiv}{\hspace{-4pt}\not \divides \hspace{2pt}}
\newcommand{\finfield}[1]{\mathbb{F}_{#1}}
\newcommand{\finunits}[1]{\mathbb{F}_{#1}^{\times}}
\newcommand{\ord}[1]{\mathrm{ord}\! \left(#1 \right)}
\newcommand{\quadfield}[1]{\Q \small(\sqrt{#1} \small)}
\newcommand{\vspan}[1]{\mathrm{span}\! \left\{#1 \right\}}
\newcommand{\galgroup}[1]{Gal \small(#1 \small)}
\newcommand{\bra}[1]{\left| #1 \right>}
\newcommand{\Oa}{O_\alpha}
\newcommand{\Od}{O_\alpha^{\dagger}}
\newcommand{\Oap}{O_{\alpha '}}
\newcommand{\Odp}{O_{\alpha '}^{\dagger}}
\renewcommand{\Im}[1]{\mathrm{Im} \: #1}
\newcommand{\ket}[1]{\left| #1 \right>}
\renewcommand{\bra}[1]{\left< #1 \right|}
\newcommand{\inner}[2]{\left< #1 | #2 \right>}
\newcommand{\expect}[2]{\left< #1 \right| #2 \left| #1 \right>}
\renewcommand{\d}[1]{ \mathrm{d}#1 \:}
\newcommand{\dn}[2]{ \mathrm{d}^{#1} #2 \:}
\newcommand{\deriv}[2]{\frac{\d{#1}}{\d{#2}}}
\newcommand{\nderiv}[3]{\frac{\dn{#1}{#2}}{\d{#3}^{#1}}}
\newcommand{\pderiv}[2]{\frac{\partial{#1}}{\partial{#2}}}
\newcommand{\parsq}[2]{\frac{\partial^2{#1}}{\partial{#2}^2}}
\newcommand{\topo}{\mathcal{T}}
\newcommand{\base}{\mathcal{B}}
\renewcommand{\bf}[1]{\mathbf{#1}}
\renewcommand{\a}{\hat{a}}
\newcommand{\adag}{\hat{a}^\dagger}
\renewcommand{\b}{\hat{b}}
\newcommand{\bdag}{\hat{b}^\dagger}
\renewcommand{\c}{\hat{c}}
\newcommand{\cdag}{\hat{c}^\dagger}
\newcommand{\hamilt}{\hat{H}}
\renewcommand{\L}{\hat{L}}
\newcommand{\Lz}{\hat{L}_z}
\newcommand{\Lsquared}{\hat{L}^2}
\renewcommand{\S}{\hat{S}}
\renewcommand{\empty}{\varnothing}
\newcommand{\J}{\hat{J}}
\newcommand{\lagrange}{\mathcal{L}}
\newcommand{\dfourx}{\mathrm{d}^4x}
\newcommand{\meson}{\phi}
\newcommand{\dpsi}{\psi^\dagger}
\newcommand{\ipic}{\mathrm{int}}
\newcommand{\parity}{\mathbf{P}}
\newcommand{\conj}{\mathbf{C}}
\newcommand{\tr}[1]{\mathrm{Tr} \left( #1 \right)}
\newcommand{\Tr}[1]{\mathrm{Tr} \left( #1 \right)}
\newcommand{\EV}[1]{\left< #1 \right>}

\renewcommand{\theenumi}{(\alph{enumi})}

\newcommand{\atitle}[1]{\title{% 
	\large \textbf{Physics GR6047 Quantum Field Theory I
	\\ Assignment \# #1} \vspace{-2ex}}
\author{Benjamin Church }
\maketitle}

 
\newtheorem{theorem}{Theorem}[section]
\newtheorem{lemma}[theorem]{Lemma}
\newtheorem{proposition}[theorem]{Proposition}
\newtheorem{corollary}[theorem]{Corollary}
\newtheorem{remark}[theorem]{Remark}
\newtheorem{definition}[theorem]{Definition}
 



\usepackage{cancel}


\begin{document}
\atitle{4}

\newcommand{\nderiv}[3]{\frac{\mathrm{d}^{#1}{#2}}{\mathrm{d}{#3}^{#1}}}

\section{Problem 1}

Consider a 1D lattice with a two-state spin at each lattice site. The Hamiltonian is,
\[ H = - J \sum_{\left< i, j \right>} \sigma_i \cdot \sigma_j \]
We have $N$ spins and choose periodic boundary conditions $\sigma_{N+1} = \sigma_1$. The partiton function is,
\[ Z = \Tr{e^{-\beta H}} = \sum_{\sigma} \prod_{\left<i, j \right>} e^{-\beta J \sigma_i \sigma_j } \]
Then we use the following trick,
\[ e^{-\beta J \sigma_i \cdot \sigma_j} =  \cosh{\beta J} + \sigma_i \sigma_j \sinh{\beta J} \]
If we consider the function,  
\[ Z^{ij} =  \cosh{\beta J} + \sigma_i \sigma_j \sinh{\beta J} \]
as a matrix in $\sigma_{i}$ and $\sigma_j$,
\[ M_{\sigma_1 \sigma_2} =  \cosh{\beta J} + \sigma_1 \sigma_2 \sinh{\beta J} \]
then we find,
\[ Z = \sum_{\sigma_1, \dots, \sigma_N} Z_{12} Z_{23} \cdots Z_{N1} = \sum_{\sigma_1, \dots, \sigma_N} M_{\sigma_1 \sigma_2} M_{\sigma_2 \sigma_3} \cdots M_{\sigma_N \sigma_1} = \Tr{M^N} \]
Therefore, it suffices to compute the eigenvalues of the matrix, 
\[ M = \begin{pmatrix}
\cosh{\beta J} + \sinh{\beta J} & \cosh{\beta J} - \sinh{\beta J}
\\
\cosh{\beta J} - \sinh{\beta J} & \cosh{\beta J} + \sinh{\beta J} 
\end{pmatrix} 
= 
\begin{pmatrix}
e^{\beta J} & e^{-\beta J}
\\
e^{-\beta J} & e^{\beta J}
\end{pmatrix} \]
which are $2 \cosh{\beta J}$ and $2 \sinh{\beta J}$. Therefore,
\[ Z = \Tr{M^N} = (2 \cosh{\beta J})^N + (2 \sinh{\beta J})^N \]
(COMPUTE CORRELATION LENGTH AND HEAT CAPACITY)

\section{Problem 2}

Consider a 1D lattice with a unit planar vector at each lattice site. That is there is a vector $\bf{S}_i$ with length $\bf{S}_i^2 = 1$ which can thus be parametrized by an angle $\phi_i$. Then the Hamiltonian is,
\[ H(\bf{S}) = - J \sum_{\left< i, j \right>} \bf{S}_i \cdot \bf{S}_j = - J \sum_{\left< i, j \right>} \cos{(\phi_i - \phi_j)} \]
Therefore, the partition function is,
\[ Z = \int_{\bf{S}} e^{-\beta H(\bf{S})} = \int_{\theta \in [0 , 2 \pi]^N} e^{-\beta \sum\limits_{\left< i, j \right>} \cos{(\phi_i - \phi_j)}} \d{\bf{S}} \]
with $N$ spins and thus $N - 1$ interaction energies.
Now we perform a change of variables, $\theta_i = \phi_{i + 1} - \phi_i$ for $1 \le i \le N-1$. To preserve degrees of freedom, we also need to consider as a coordinate,
\[ \theta = \sum_{i = 1}^N \phi_i \]
so that the $\theta_i$ and $\theta$ can reproduce the $\phi_i$. 
Therefore,
\begin{align*}
Z = \int_{\theta \in [0, 2 \pi]^N} \prod_{i = 1}^{N-1} e^{-\beta \cos{\theta_i}} \: \dn{N}{\theta_i} = 2 \pi \prod_{i = 1}^{N-1} \int_{0}^{2 \pi} e^{-\beta \cos{\theta_i}} \d{\theta} = 2 \pi \left( \int_0^{2 \pi} e^{-\beta \cos{\theta}} \: \d{\theta} \right)^{N-1}
\end{align*}
Furthermore, 
\[ \int_0^{2 \pi} e^{-\beta \cos{\theta}} \: \d{\theta} =  \]
(PERIODIC BOUNDARY CONDITIONS?)
(HOW TO DO THIS INTEGRAL?)

\section{Problem 3}

Consider a free energy functional,
\[ F [ \theta(\bf{r}) ] = \tfrac{1}{2} \epsilon_0 \int \left[ \ell^2 (\nabla \theta(\bf{r}))^2 - \cos^2{\theta(\bf{r})} \right] \dn{3}{r} \]
The fixed points are found by minimizing the free energy,
\[ \frac{\delta F}{\delta \bf{r}} = - \epsilon_0 \left[ \ell^2  \nabla^2 \theta(\bf{r}) - \sin{\theta(\bf{r})} \cos{\theta(\bf{r})} \right] = 0 \]
This gives the PDE,
\[ \ell^2 \nabla^2 \theta - \sin{\theta} \cos{\theta} = 0 \]
First, lets consider a domain-wall transition along the $z$-axis homogenous in the $x$-$y$ plane. Then we need to solve,
\[ \ell^2 \nderiv{2}{\theta}{z} = \sin{\theta} \cos{\theta} \]
This does not have a solution in terms of elementary functions in general. However, we can solve it for the specific boundary conditions $\theta(z \to -\infty) = 0$ and $\theta(z \to + \infty) = \pi$ as follows. First, consider the first integral,
\[ \mathcal{E} = \ell^2 \left(\deriv{\theta}{z}\right)^2 + \cos^2{\theta} \]
which is conserved. Therefore,
\[ \deriv{\theta}{z} = \frac{1}{\ell} \sqrt{\mathcal{E} - \cos^2{\theta}} \]
For our boundary conditions, $\mathcal{E} = 1$ and thus,
\[ \deriv{\theta}{z} = \frac{1}{\ell} \sqrt{1 - \cos^2{\theta}} = \frac{\sin{\theta}}{\ell} \]
and thus we find,
\[ z = \ell \int_{\frac{\pi}{2}}^{\theta} \frac{\d{\theta}}{\sin{\theta}} =  \log{\tan{\frac{\theta}{2}}}   \] 
Inverting gives,
\[ \theta(z) = 2 \arctan{(e^{z / \ell})} \]
From this we may compute the free energy stored in the surface per surface area,
\begin{align*}
\frac{\Delta F}{A} & = \int_{-\infty}^{\infty} \left[ \ell^2 \left( \deriv{\theta}{z} \right)^2 - \cos^2{\theta} + 1 \right] \d{z}
\end{align*}
We know that,
\[ \mathcal{E} = \ell^2 \left( \deriv{\theta}{z} \right)^2 + \cos^2{\theta} = 1 \]
and therefore,
\[ \frac{\Delta F}{A} = 2 \int_{-\infty}^{\infty} \sin^2{\theta(t)} \: \d{z} = 4 \int_{0}^{\infty} \sin^2{\theta(t)} \: \d{z} \]
Now we perform a change of variables,
\[ u = \sin{\theta(t)} \]
Then recall that,
\[ \deriv{u}{z} = - \sin{\theta} \deriv{\theta}{z} = - \frac{\sin^2{\theta}}{\ell} \]
therefore,
\[ \frac{\Delta F}{A} = 4 \ell \int_0^{1} \d{u} = 4 \ell \]

\section{Problem 4}


Consider a gas with equation of state,
\[ P = \frac{k_B T}{v - b} \exp{\left[ - \frac{a}{k_B T v} \right]} \]
where $v = V / N$. The critial point occurs for a critical isotherm,
\begin{align*}
\left( \pderiv{P}{V} \right)_{T} & = 0 
\\ 
\left( \frac{\partial^2 P}{\partial V^2} \right)_{T} & = 0
\end{align*}
Computing these derivatives, we find,
\begin{align*}
N \left( \pderiv{P}{V} \right)_{T} & = - \frac{k_B T}{(v - b)^2} \exp{\left[ - \frac{a}{k_B T v} \right]} + \frac{k_B T}{v - b} \left( \frac{a}{k_B T v^2} \right) \exp{\left[ - \frac{a}{k_B T v} \right]} 
\\
& = \left[ -\frac{1}{v - b} + \frac{a}{k_B T v^2} \right] \frac{k_B T}{v - b}  \exp{\left[ - \frac{a}{k_B T v} \right]}
\\
& = \left[ \frac{a}{v^2} - \frac{k_B T}{v - b} \right] \frac{1}{v - b}  \exp{\left[ - \frac{a}{k_B T v} \right]} 
\end{align*}
Taking a second derivative gives,
\begin{align*}
N^2 \left( \frac{\partial^2 P}{\partial V^2} \right)_{T} & = \left[ a^2 (v - b)^2 + 2 (k_B T)^2 v^4 - 2 a (k_B T) v (b^2 - 3 b v + 2 v^2) \right] \frac{1}{(k_B T) v^4 (v - b)^3} \exp{\left[ - \frac{a}{k_B T v} \right]} 
\end{align*}
Therefore, we need to set,
\[ \frac{a}{v^2} = \frac{k_B T}{v - b} \]
and,
\[  a^2 (v - b)^2 + 2 (k_B T)^2 v^4 - 2 a (k_B T) v (b^2 - 3 b v + 2 v^2) = 0 \]
Using the first relation, we find,
\[ a^2 (v - b)^2 + 2 a^2 (v - b)^2 - 2 a^2 (v - b) v^{-1} (b^2 - 3 b v + 2 v^2) = 0 \]
Therefore,
\[ 3 v (v - b) = 2 (b^2 - 3 b v + 2 v^2) \]
Rearranging, 
\[ v^2 - 3 b v + 2 b^2 = 0 \]
which has a solutions,
\[ v = \frac{3 b \pm \sqrt{(3 b)^2 - 8 b^2}}{2} = \frac{3 b \pm b}{2} = 
\begin{cases}
b
\\
2 b
\end{cases} \]
However, we require that $v > b$ else there is a divergence in the equation of state. Thus, $v_v = \tfrac{3}{2} v$. 
Then, at the critical point,
\[ \frac{a}{4 b^2} = \frac{k_B T_c}{b} \]
and thus,
\[ k_B T_c = \frac{a}{4 b} \]
Therefore, plugging in,
\[ \frac{P_c v_c}{k_B T_c} = \frac{a}{4 b^2} \: e^{-4} \]

\end{document}

