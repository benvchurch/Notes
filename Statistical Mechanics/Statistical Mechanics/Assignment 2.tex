\documentclass[12pt]{extarticle}
\usepackage[utf8]{inputenc}
\usepackage[english]{babel}
\usepackage[a4paper, total={7.25in, 9.5in}]{geometry}
\usepackage{tikz-feynman}
\tikzfeynmanset{compat=1.0.0} 
\usepackage{subcaption}
\usepackage{float}
\floatplacement{figure}{H}
\usepackage{simpler-wick}
 
\newcommand{\field}{\hat{\Phi}}
\newcommand{\dfield}{\hat{\Phi}^\dagger}
 
\usepackage{amsthm, amssymb, amsmath, centernot}
\usepackage{mathrsfs}
\usepackage{slashed}
\newcommand{\notimplies}{%
  \mathrel{{\ooalign{\hidewidth$\not\phantom{=}$\hidewidth\cr$\implies$}}}}
 
\renewcommand\qedsymbol{$\square$}
\newcommand{\cont}{$\boxtimes$}
\newcommand{\divides}{\mid}
\newcommand{\ndivides}{\centernot \mid}
\newcommand{\Z}{\mathbb{Z}}
\newcommand{\N}{\mathbb{N}}
\newcommand{\C}{\mathbb{C}}
\newcommand{\Zplus}{\mathbb{Z}^{+}}
\newcommand{\Primes}{\mathbb{P}}
\newcommand{\ball}[2]{B_{#1} \! \left(#2 \right)}
\newcommand{\Q}{\mathbb{Q}}
\newcommand{\R}{\mathbb{R}}
\newcommand{\Rplus}{\mathbb{R}^+}
\newcommand{\invI}[2]{#1^{-1} \left( #2 \right)}
\newcommand{\End}[1]{\text{End}\left( A \right)}
\newcommand{\legsym}[2]{\left(\frac{#1}{#2} \right)}
\renewcommand{\mod}[3]{\: #1 \equiv #2 \: \mathrm{mod} \: #3 \:}
\newcommand{\nmod}[3]{\: #1 \centernot \equiv #2 \: mod \: #3 \:}
\newcommand{\ndiv}{\hspace{-4pt}\not \divides \hspace{2pt}}
\newcommand{\finfield}[1]{\mathbb{F}_{#1}}
\newcommand{\finunits}[1]{\mathbb{F}_{#1}^{\times}}
\newcommand{\ord}[1]{\mathrm{ord}\! \left(#1 \right)}
\newcommand{\quadfield}[1]{\Q \small(\sqrt{#1} \small)}
\newcommand{\vspan}[1]{\mathrm{span}\! \left\{#1 \right\}}
\newcommand{\galgroup}[1]{Gal \small(#1 \small)}
\newcommand{\bra}[1]{\left| #1 \right>}
\newcommand{\Oa}{O_\alpha}
\newcommand{\Od}{O_\alpha^{\dagger}}
\newcommand{\Oap}{O_{\alpha '}}
\newcommand{\Odp}{O_{\alpha '}^{\dagger}}
\renewcommand{\Im}[1]{\mathrm{Im} \: #1}
\newcommand{\ket}[1]{\left| #1 \right>}
\renewcommand{\bra}[1]{\left< #1 \right|}
\newcommand{\inner}[2]{\left< #1 | #2 \right>}
\newcommand{\expect}[2]{\left< #1 \right| #2 \left| #1 \right>}
\renewcommand{\d}[1]{ \mathrm{d}#1 \:}
\newcommand{\dn}[2]{ \mathrm{d}^{#1} #2 \:}
\newcommand{\deriv}[2]{\frac{\d{#1}}{\d{#2}}}
\newcommand{\nderiv}[3]{\frac{\dn{#1}{#2}}{\d{#3}^{#1}}}
\newcommand{\pderiv}[2]{\frac{\partial{#1}}{\partial{#2}}}
\newcommand{\parsq}[2]{\frac{\partial^2{#1}}{\partial{#2}^2}}
\newcommand{\topo}{\mathcal{T}}
\newcommand{\base}{\mathcal{B}}
\renewcommand{\bf}[1]{\mathbf{#1}}
\renewcommand{\a}{\hat{a}}
\newcommand{\adag}{\hat{a}^\dagger}
\renewcommand{\b}{\hat{b}}
\newcommand{\bdag}{\hat{b}^\dagger}
\renewcommand{\c}{\hat{c}}
\newcommand{\cdag}{\hat{c}^\dagger}
\newcommand{\hamilt}{\hat{H}}
\renewcommand{\L}{\hat{L}}
\newcommand{\Lz}{\hat{L}_z}
\newcommand{\Lsquared}{\hat{L}^2}
\renewcommand{\S}{\hat{S}}
\renewcommand{\empty}{\varnothing}
\newcommand{\J}{\hat{J}}
\newcommand{\lagrange}{\mathcal{L}}
\newcommand{\dfourx}{\mathrm{d}^4x}
\newcommand{\meson}{\phi}
\newcommand{\dpsi}{\psi^\dagger}
\newcommand{\ipic}{\mathrm{int}}
\newcommand{\parity}{\mathbf{P}}
\newcommand{\conj}{\mathbf{C}}
\newcommand{\tr}[1]{\mathrm{Tr} \left( #1 \right)}
\newcommand{\Tr}[1]{\mathrm{Tr} \left( #1 \right)}
\newcommand{\EV}[1]{\left< #1 \right>}

\renewcommand{\theenumi}{(\alph{enumi})}

\newcommand{\atitle}[1]{\title{% 
	\large \textbf{Physics GR6047 Quantum Field Theory I
	\\ Assignment \# #1} \vspace{-2ex}}
\author{Benjamin Church }
\maketitle}

 
\newtheorem{theorem}{Theorem}[section]
\newtheorem{lemma}[theorem]{Lemma}
\newtheorem{proposition}[theorem]{Proposition}
\newtheorem{corollary}[theorem]{Corollary}
\newtheorem{remark}[theorem]{Remark}
\newtheorem{definition}[theorem]{Definition}
 



\usepackage{cancel}


\begin{document}
\atitle{2}
 
\section{Problem 1}

Consider the Boltzmann statistics of a gas of $N$ noninteracting highly-relativistic identical particles in a box of volume $V$. The single particle dispersion relation for these particles is,
\[ E = c |p| \]
and therefore the Hamiltonian is,
\[ H(q_i, p_i) = \sum_{i = 1}^N c |p_i| \]
Now the partition function is,
\begin{align*}
Z & = \frac{1}{N!} \frac{1}{h^{3 N}} \int \dn{3}{q_1} \cdots \dn{3}{q_N} \dn{3}{p_1} \cdots \dn{3}{p_N} e^{- \beta H}
\\
& = \frac{V^N}{N!} Z_1^N
\end{align*}
where $Z_1$ is the single particle momentum partiton function,
\begin{align*}
Z_1 & = \frac{1}{h^3} \int \dn{3}{p} e^{-\beta c |p|} = \int_{0}^{\infty} 4 \pi p^2 \d{p} e^{-\beta c p} 
\\
& = \frac{8 \pi}{\beta^3 h^3 c^3} 
\end{align*}
Therefore,
\[ Z = \frac{V^N}{N!} \left( \frac{8 \pi}{\beta^3 h^3 c^3} \right)^N \]
First, we compute the free energy,
\[ F = - k_B T \log{Z} = - N k_B T \left[ \log{\left( \frac{k_B^3 T^3 V}{h^3 c^3} \right)} - (\log{N} - \log{(8 \pi)} - 1)  \right] \]
Then we can compute the entropy and pressue,
\begin{align*}
P & = - \left( \pderiv{F}{V} \right)_{T} = \frac{N k_B T}{V} 
\\
S & = - \left( \pderiv{F}{T} \right)_{V} = N k_B \left[ \log{\left( \frac{k_B^3 T^3 V}{h^3 c^3} \right)} - (\log{N} - \log{(8 \pi)} - 4)  \right]
\end{align*}
Next, we compute the energy via,
\begin{align*}
E - \pderiv{\log{Z}}{\beta} = 3 N k_B T 
\end{align*}
and therfore the heat capacity is,
\[ c_V = \pderiv{E}{T} = 3 N k_B \]

\section{Problem 2}

Consider an isotropic quantum rigid rotor with moment of inertia $I$ which has a Hamiltonian,
\[ \hat{H} = \frac{L^2}{2 I} \]
which has energy levels,
\[ E_J = \frac{\hbar^2}{2 I} J(J + 1) \]
and for each $J$ there is a $J$-multiplet with $g_J = 2 J + 1$ states. 

\subsection*{(a)}

Then we can write down the partition function,
\[ Z = \Tr{e^{-\beta \hat{H}}} = \sum_{J = 0}^\infty g_J e^{- \beta E_J} = \sum_{J = 0}^\infty (2 J + 1) e^{-\frac{\beta \hbar^2}{2 I} J(J + 1)} \]
For large temperatures we have $u = \frac{\beta \hbar^2}{2 I} \ll 1$ and thus the Boltzmann factor is approximatly flat for low $J$ values and thus the degeneracy $2 J + 1$ is dominant. Thus we can approximate this by an integral since,
\[ \sum_{J = 0}^{J'} (2 J + 1) = J'(J' + 1) + J' \approx \int_{0}^{J'} (2 J + 1) \d{J} \]
for large $J$.
Therefore,
\begin{align*}
Z & \approx \int_0^{\infty} 2 (2 J + 1) e^{-u J(J + 1)} \: \d{J}
\end{align*}
Notice that,
\[ 2 J + 1 = \pderiv{}{J} [J ( J + 1) ] \]
and therefore,
\[ Z = \int_0^{\infty} \deriv{}{J} e^{-u J ( J + 1)} \d{J} = \frac{1}{u} = \frac{2 I}{\beta \hbar^2} \]
which is unreasonably simple. 

\subsection*{(b)}

The energy is then,
\[ E = - \pderiv{\log{Z}}{\beta} = \frac{1}{\beta} = k_B T \]
Therefore, at high temperature,
\[ c_V = \pderiv{E}{T} = k_B \]

\subsection*{(c)}

At low temperature, we may apprimate the partition function by taking only the $J = 0$ and $J = 1$ states to be accessable. Then,
\[ Z = 1 + 3 e^{-2 u } \]
Then we have,
\begin{align*}
E & = - \pderiv{\log{Z}}{\beta} = \frac{\hbar^2}{I} \cdot \frac{3}{e^{\frac{\beta \hbar^2}{I}} + 3} = \frac{\hbar^2}{I} \cdot \frac{3}{e^{\frac{\hbar^2}{I k_B T}} + 3} 
\end{align*}
Now define,
\[ T_Q = \frac{\hbar^2}{I k_B} \]
and thus,
\[ E = k_B T_Q \cdot \frac{3}{e^{\frac{T_Q}{T}} + 3} \]
Thus,
\begin{align*}
c_V & = k_B \left( \frac{T_Q}{T} \right)^2 \cdot \frac{3 e^{\frac{\hbar^2}{I k_B T}}}{(e^{\frac{T_Q}{T}} + 3)^2}
\end{align*}
Then, at low temperatures, we have,
\[ c_V \approx 3 k_B \left( \frac{T_Q}{T} \right)^2 \: e^{-\frac{T_Q}{T}}  \]
which dies exponentially because the system is gapped. This approximaton is valid for,
\[ T \ll T_Q \]

\section{Problem 3}

Consider an ideal gas of classical identical particles with mass $m$ and no internal degrees of freedom enclosed in a cylinder of radius $b$ and length $L$. The cylinder is rotating with angular velocity $\omega$ about is symmetry axis. The ideal gas is in thermal equilibrium at temperature $T$ in the rotating coordinate system.

\subsection*{(a)}

We first need to compute the Hamiltonian in a rotating coordinate system. First, we write down the Lagrangian for a single particle,
\[ \lagrange = \tfrac{1}{2} m \dot{\vec{r}}^{\, 2} \]
in the lab frame coordinates. Then, in terms of the rotating coordinate system,
\[ \dot{\vec{r}} = R(t) (\dot{\vec{q}} + \vec{\omega} \times \vec{q} ) \] 
where $R(t)$ is the rotation matrix sending the rotating frame coordinates to the lab frame coodinates. Therefore,
\[ \lagrange = \tfrac{1}{2} m (\dot{\vec{q}} + \vec{\omega} \times \vec{q})^2 \] 
Then the canonical momentum associated to $q$ is,
\[ p_j = \pderiv{\lagrange}{\dot{q}_j} = m (\dot{\vec{q}} + \vec{\omega} \times \vec{q})_j \]
Incidentally, we could write down the Euler Lagrange equations,
\[ \deriv{}{t} p_j = \pderiv{\lagrange}{q_j} \]
which gives,
\[ m \ddot{\vec{q}} + m (\vec{\omega} \times \dot{\vec{q}}) + m (\dot{\vec{\omega}} \times \vec{q}) = m (\dot{\vec{q}} + \vec{\omega} \times \vec{q}) \times \vec{\omega} \]
which allows us to identify the pseudoforces,
\[ F_{\text{pseudo}} = - m (\dot{\vec{\omega}} \times \vec{q}) - 2 m (\vec{\omega} \times \dot{\vec{q}}) - m \vec{\omega} \times (\vec{\omega} \times \vec{q}) \]
Back to our problem of computing the Hamiltonian. We have,
\[ H = \vec{p} \cdot \dot{\vec{q}} - \lagrange = \tfrac{1}{m} \vec{p} \cdot (\vec{p} -  \vec{\omega} \times \vec{q}) - \tfrac{1}{2m} \vec{p}^{\, 2} = \frac{(\vec{p} - \vec{\omega} \times \vec{q})^2}{2 m} - \frac{(\vec{\omega} \times \vec{q})^2}{2m} \] 
Thus, the Hamiltonian for our gas is,
\[ H(\vec{q}_i, \vec{p}_i) = \sum_{i = 1}^N \frac{1}{2 m} \left[ (\vec{p}_i - \vec{\omega} \times \vec{q}_i)^2 - (\vec{\omega} \times \vec{q}_i)^2 \right] \]

\subsection*{(b)}

Now we can compute the partition function,
\[ Z = \frac{1}{N! h^{3N}} \int \dn{3}{q_1} \cdots \dn{3}{q_N} \dn{3}{p_1} \cdots \dn{3}{p_N} \exp{\left[ - \frac{\beta}{2 m} \sum_{i = 1}^{\infty} \left( (\vec{p}_i - \vec{\omega} \times \vec{q}_i)^2 - (\vec{\omega} \times \vec{q}_i)^2 \right) \right]} \]
Notice that these integrals factor since the Hamiltonian for each particle decouples (no interactions). Furthermore, I can seperate the integrals into position and momentum parts as follows,
\[ Z = \frac{1}{N! h^{3 N}} \left( \int \dn{3}{q} \exp{\left[ \frac{\beta}{2 m} (\vec{\omega} \times \vec{q})^2 \right]} \int \dn{3}{p} \exp{\left[ - \frac{\beta}{2 m}  (\vec{p} - \vec{\omega} \times \vec{q})^2 \right]}   \right)^N \]
The integral over $p$ appears to involve $q$ as well but since we integrate over $p$ first we may consider $q$ fixed and perform a change of variables, 
\[ \vec{p}' = \vec{p} - \vec{\omega} \times \vec{q} \]
and the $q$-dependence vanishes,
\[ Z = \frac{1}{N! h^{3 N}} \left( \int \dn{3}{q} \exp{\left[ \frac{\beta}{2 m} (\vec{\omega} \times \vec{q})^2 \right]} \int \dn{3}{p} \exp{\left[ - \frac{\beta}{2 m} \vec{p}^{\, 2} \right]}   \right)^N \]
We now consider these two integrals in detail. First,
\begin{align*}
Z_p \int \dn{3}{p} \exp{\left[ - \frac{\beta}{2 m} \vec{p}^{\, 2} \right]}  & = \left( \int_{-\infty}^{\infty} \d{p} e^{-\frac{\beta}{2m} p^2} \right)^3 = \left( \frac{2 \pi m}{\beta} \right)^{\frac{3}{2}}
\end{align*} 
Next we need to integrate over the cylinder,
\begin{align*}
Z_q = \int \dn{3}{q} \exp{\left[ \frac{\beta}{2 m} (\vec{\omega} \times \vec{q})^2 \right]} & = \int_0^{b} \d{r} \int_{0}^L \d{z} \int_0^{2 \pi} r \d{\phi} e^{-\frac{\beta}{2 m} r^2 \omega^2} 
\\
& = 2 \pi L \int_0^b r \d{r} e^{-\frac{\beta}{2 m} r^2 \omega^2} = 2 \pi L \int_0^b  \frac{2 m}{\beta \omega^2} \deriv{}{r} e^{-\frac{\beta}{2 m} r^2 \omega^2} \d{r}
\\
& = \frac{4 m \pi L}{\beta \omega^2} \left(1 - e^{-\frac{\beta b^2 \omega^2}{2m}} \right)
\end{align*}
Putting everything together, we find,
\[ Z = \frac{1}{N!} \left( \frac{4 m \pi L}{\beta h^3 \omega^2} \right)^N \left(1 - e^{-\frac{\beta b^2 \omega^2}{2m}} \right)^N \left( \frac{2 \pi m}{\beta} \right)^{\frac{3N}{2}} \]

\subsection*{(c)}

The density of particles at a given radius is given by the operator,
\[ \rho_r(\vec{q}_1, \dots, \vec{q}_N) = \sum_{i = 1}^N \delta(r_i - r) \]
The average particle density is thus given by the expectation of this operator,
\begin{align*}
\rho(r) = \EV{\rho_r} = \frac{1}{N! h^{3 N}} \int \dn{3}{q_1} \cdots \dn{3}{q_N} \dn{3}{p_1} \cdots \dn{3}{p_N} \rho_r(\vec{q}_1, \dots, \vec{q}_N) P(\vec{q}_1, \dots, \vec{q}_N, \vec{p}_1, \dots, \vec{p}_N)
\end{align*}
where,
\[ P(\vec{q}_1, \dots, \vec{q}_N, \vec{p}_1, \dots, \vec{p}_N)) = \frac{1}{Z} e^{-\beta H(\vec{q}_1, \dots, \vec{q}_N, \vec{p}_1, \dots, \vec{p}_N)} \]
The above integral reduces to,
\begin{align*}
\rho(r) & = \frac{1}{Z_q} \sum_{i = 1}^N \int \dn{3}{q_i} \left[ \delta(r_i - r) e^{\frac{\beta}{2 m} (\vec{\omega} \times \vec{q}_i)^2} \right]
\\
& = \frac{N}{Z_q} \int_0^{b} \d{r'} \int_{0}^L \d{z} \int_0^{2 \pi} r' \d{\phi} \delta(r' - r) e^{-\frac{\beta}{2 m} r'^2 \omega^2} 
\\
& = \frac{N}{Z_q} (2 \pi L) \int_0^b r' \d{r'} \delta(r' - r) e^{-\frac{\beta}{2 m} r'^2 \omega^2} 
\\
& = \frac{N}{Z_p} (2 \pi L) \: r e^{-\frac{\beta}{2 m} r^2 \omega^2} 
\end{align*}
Therefore, we have shown that,
\[ \rho(r) = \frac{N \omega^2}{2 m k_B^2 T^2} \left(1 - e^{-\frac{b^2 \omega^2}{2m k T}} \right)^{-1} r e^{-\frac{r^2 \omega^2 }{2 m k T} } \]

\section{Problem 4}

Consider a material of $n$ independent particles inside a weak magnetic field $H$. Each particle has spin $J$ and a magnetic moment $\mu m_J$ for $m_J \in \{ - J , -J + 1, \dots, J \}$. The system is in thermal equilibrium at constant temperature $T$.

\subsection{(a)}

First we need to compute the partition function. The Hamiltnian is,
\[ \hat{H} = \sum_{i = 1}^n \mu \hat{S}_i \cdot H \]
and thus, 
\[ Z = \sum_{m \in [-J, J]^n} e^{-\beta \mu H \sum_{i = 1}^n m_i } = \left( \sum\limits_{m \in [-J, J]} e^{-\beta \mu m} \right)^n \]
However, we can sum the geometric,
\[ \sum_{m \in [-J, J]} e^{-\beta \mu H m} = \frac{e^{\beta \mu H J} - e^{- \beta  \mu H  (J+1)}}{1 - e^{-\beta \mu H}} = \frac{e^{\beta \mu H (J + \frac{1}{2})} - e^{- \beta \mu H (J + \frac{1}{2}}}{e^{\tfrac{1}{2} \beta \mu H} - e^{-\tfrac{1}{2} \beta \mu H }} = \frac{\sinh{(\beta \mu H (J + \frac{1}{2}))}}{\sinh{(\frac{1}{2} \beta \mu H )}} \]
We can compute the average of the particle magnetic moment over the statistical ensemble,
\[ \EV{M} = \EV{\sum_{i = 1}^n \mu m_i} = \frac{1}{Z} \sum_{m \in [-J, J]^n} \sum_{i = 1}^n \mu m_i \: e^{- \beta \mu H \sum\limits_{i = 1}^n m_i} = - \frac{1}{Z H} \deriv{Z}{\beta} = - \frac{1}{H} \deriv{\log{Z}}{\beta} = \frac{E}{H}  \]
So in this case we see that the magnetization is also equal to the proportionality between energy and applied field. Computing,
\[ -E = \deriv{\log{Z}}{\beta} = \tfrac{1}{2} \mu H \left[ \coth{(\tfrac{1}{2} \beta \mu H)} -  (2J + 1) \coth{(\beta \mu H (J + \tfrac{1}{2}))} \right] \]
Therefore,
\[ M = \tfrac{1}{2} \mu \left[  (2J + 1) \coth{(\beta \mu H (J + \tfrac{1}{2}))} - \coth{(\tfrac{1}{2} \beta \mu H)} \right] \]
We can also compute the magnetic susceptibility which is the thermodynamic conjugate variable to the magnetizing field $H$,
\[ \EV{M} = \pderiv{E}{H} \bigg|_{S} = \pderiv{F}{H} \bigg|_T \]  

\subsection{(b)}

Now we apply the series,
\[ \coth{x} = \frac{1}{x} + \frac{x}{3} - \frac{x^3}{45} + O(x^5) \]
Then we find, at high $T$,
\begin{align*}
M & = \tfrac{1}{2} \mu \left[ \frac{2}{\beta \mu H} + \frac{\beta \mu H}{6} (2 J + 1)^2 - \frac{(\beta \mu H)^3}{360} (2 J + 1)^4 + O(\beta^5) - \frac{2}{\beta \mu H} - \frac{\beta \mu H}{6} + \frac{(\beta \mu H)^3}{360} + O(\beta^5)  \right]
\\
& = \mu \left[ \frac{(2 J + 1)^2 - 1}{12} \cdot \frac{\mu H}{k_B T} - \frac{(2 J + 1)^4 - 1}{720} \cdot \left( \frac{\mu H}{k_B T} \right)^3 + O(T^{-5}) \right] 
\end{align*}

\section{Problem 7}

Consider a classical system in phase space with $3 N$ dimensions and a Hamiltonian,
\[ H(\vec{r}_r, \vec{p}_i) \]
Conisder the application of an external magnetic field which has the effect of transforming the momenta,
\[ \vec{p}_i \mapsto \vec{p}_i - \tfrac{e}{c} \vec{A}(\vec{r}_i) \]
Then we consider the partition function,
\[ Z = \frac{1}{N! h^{3N}} \int \dn{3}{r_1} \cdots \dn{3}{r_N} \dn{3}{p_1} \cdots \dn{3}{p_N} \exp{\left[- \beta H(r_i, p_i - \tfrac{e}{c} \vec{A}_i(r_i)) \right]} \]
We perform the integral over momenta first,
\[ Z = \frac{1}{N! h^{3N}} \int \dn{3}{r_1} \cdots \dn{3}{r_N} \int \dn{3}{p_1} \cdots \dn{3}{p_N} \exp{\left[- \beta H(\vec{r}_i, \vec{p}_i - \tfrac{e}{c} \vec{A}(\vec{r}_i)) \right]} \]
Notice that we may treat the $r_i$ as fixed in the momentum integral and thus we can view $\vec{A}(\vec{r}_i)$ as a constant in this integral. Therefore, we are free to perform a substitution setting $p_i' = p_i - \tfrac{e}{c} \vec{A}$ and then the partition function loses its dependence on $A$, 
\[ Z = \frac{1}{N! h^{3N}} \int \dn{3}{r_1} \cdots \dn{3}{r_N} \int \dn{3}{p'_1} \cdots \dn{3}{p'_N} \exp{\left[- \beta H(\vec{r}_i, \vec{p'}_i) \right]} \]
therefore we have proven that all thermodynamic potentials derivable from $Z$ must be independent of the applied magnetic field. 




\end{document}

