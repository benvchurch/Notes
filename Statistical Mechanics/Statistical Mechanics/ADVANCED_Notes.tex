\documentclass[12pt]{article}
\usepackage[utf8]{inputenc}
\usepackage[english]{babel}
\usepackage[a4paper, total={7.25in, 9.5in}]{geometry}
\usepackage{tikz-feynman}
\tikzfeynmanset{compat=1.0.0} 
\usepackage{subcaption}
\usepackage{float}
\floatplacement{figure}{H}
\usepackage{simpler-wick}
 
\newcommand{\field}{\hat{\Phi}}
\newcommand{\dfield}{\hat{\Phi}^\dagger}
 
\usepackage{amsthm, amssymb, amsmath, centernot}
\usepackage{mathrsfs}
\usepackage{slashed}
\newcommand{\notimplies}{%
  \mathrel{{\ooalign{\hidewidth$\not\phantom{=}$\hidewidth\cr$\implies$}}}}
 
\renewcommand\qedsymbol{$\square$}
\newcommand{\cont}{$\boxtimes$}
\newcommand{\divides}{\mid}
\newcommand{\ndivides}{\centernot \mid}
\newcommand{\Z}{\mathbb{Z}}
\newcommand{\N}{\mathbb{N}}
\newcommand{\C}{\mathbb{C}}
\newcommand{\Zplus}{\mathbb{Z}^{+}}
\newcommand{\Primes}{\mathbb{P}}
\newcommand{\ball}[2]{B_{#1} \! \left(#2 \right)}
\newcommand{\Q}{\mathbb{Q}}
\newcommand{\R}{\mathbb{R}}
\newcommand{\Rplus}{\mathbb{R}^+}
\newcommand{\invI}[2]{#1^{-1} \left( #2 \right)}
\newcommand{\End}[1]{\text{End}\left( A \right)}
\newcommand{\legsym}[2]{\left(\frac{#1}{#2} \right)}
\renewcommand{\mod}[3]{\: #1 \equiv #2 \: \mathrm{mod} \: #3 \:}
\newcommand{\nmod}[3]{\: #1 \centernot \equiv #2 \: mod \: #3 \:}
\newcommand{\ndiv}{\hspace{-4pt}\not \divides \hspace{2pt}}
\newcommand{\finfield}[1]{\mathbb{F}_{#1}}
\newcommand{\finunits}[1]{\mathbb{F}_{#1}^{\times}}
\newcommand{\ord}[1]{\mathrm{ord}\! \left(#1 \right)}
\newcommand{\quadfield}[1]{\Q \small(\sqrt{#1} \small)}
\newcommand{\vspan}[1]{\mathrm{span}\! \left\{#1 \right\}}
\newcommand{\galgroup}[1]{Gal \small(#1 \small)}
\newcommand{\bra}[1]{\left| #1 \right>}
\newcommand{\Oa}{O_\alpha}
\newcommand{\Od}{O_\alpha^{\dagger}}
\newcommand{\Oap}{O_{\alpha '}}
\newcommand{\Odp}{O_{\alpha '}^{\dagger}}
\renewcommand{\Im}[1]{\mathrm{Im} \: #1}
\newcommand{\ket}[1]{\left| #1 \right>}
\renewcommand{\bra}[1]{\left< #1 \right|}
\newcommand{\inner}[2]{\left< #1 | #2 \right>}
\newcommand{\expect}[2]{\left< #1 \right| #2 \left| #1 \right>}
\renewcommand{\d}[1]{ \mathrm{d}#1 \:}
\newcommand{\dn}[2]{ \mathrm{d}^{#1} #2 \:}
\newcommand{\deriv}[2]{\frac{\d{#1}}{\d{#2}}}
\newcommand{\nderiv}[3]{\frac{\dn{#1}{#2}}{\d{#3}^{#1}}}
\newcommand{\pderiv}[2]{\frac{\partial{#1}}{\partial{#2}}}
\newcommand{\parsq}[2]{\frac{\partial^2{#1}}{\partial{#2}^2}}
\newcommand{\topo}{\mathcal{T}}
\newcommand{\base}{\mathcal{B}}
\renewcommand{\bf}[1]{\mathbf{#1}}
\renewcommand{\a}{\hat{a}}
\newcommand{\adag}{\hat{a}^\dagger}
\renewcommand{\b}{\hat{b}}
\newcommand{\bdag}{\hat{b}^\dagger}
\renewcommand{\c}{\hat{c}}
\newcommand{\cdag}{\hat{c}^\dagger}
\newcommand{\hamilt}{\hat{H}}
\renewcommand{\L}{\hat{L}}
\newcommand{\Lz}{\hat{L}_z}
\newcommand{\Lsquared}{\hat{L}^2}
\renewcommand{\S}{\hat{S}}
\renewcommand{\empty}{\varnothing}
\newcommand{\J}{\hat{J}}
\newcommand{\lagrange}{\mathcal{L}}
\newcommand{\dfourx}{\mathrm{d}^4x}
\newcommand{\meson}{\phi}
\newcommand{\dpsi}{\psi^\dagger}
\newcommand{\ipic}{\mathrm{int}}
\newcommand{\parity}{\mathbf{P}}
\newcommand{\conj}{\mathbf{C}}
\newcommand{\tr}[1]{\mathrm{Tr} \left( #1 \right)}
\newcommand{\Tr}[1]{\mathrm{Tr} \left( #1 \right)}
\newcommand{\EV}[1]{\left< #1 \right>}

\renewcommand{\theenumi}{(\alph{enumi})}

\newcommand{\atitle}[1]{\title{% 
	\large \textbf{Physics GR6047 Quantum Field Theory I
	\\ Assignment \# #1} \vspace{-2ex}}
\author{Benjamin Church }
\maketitle}

 
\newtheorem{theorem}{Theorem}[section]
\newtheorem{lemma}[theorem]{Lemma}
\newtheorem{proposition}[theorem]{Proposition}
\newtheorem{corollary}[theorem]{Corollary}
\newtheorem{remark}[theorem]{Remark}
\newtheorem{definition}[theorem]{Definition}
 




\renewcommand{\H}{\mathcal{H}}

\begin{document}

\section{Phonons}

We have a prescription,
\[ \mathcal{H}(\mu) \to Z(P,V,T) \]
Given some symmetry we can produce an effective Hamiltonian which gives the same partition function. We give the example of phonons in a 1D lattice with spacing $a$. Let $q_n^{(0)} = na$ and $q_n = q_{n}^{(0)} + u_n$ small dispacement.
Then,
\[ U = U_0 +  \sum_{j} K_j \sum_{n} (u_n - u_{n + k})^2 \]
And,
\[ T = \sum_n \tfrac{1}{2} m \dot{u}_n^2 \]
We replace $u_n$ with its fourier transform $u(k)$ to give,
\[ \H = U_0 + \tfrac{1}{2} \sum_k \left[ \frac{p(k)^2}{2m} + \kappa(k) u(k) u^*(k) \right] \]
where
\[ \kappa(k) = \sum_j K_j (1 - \cos{k \alpha_j}) \]
Then,
\[ \omega(k) = \sqrt{\frac{\kappa(k)}{m}} \]
Thus in the limit $k \to 0$ we find $\kappa \propto k^2$ and $\omega(k) \propto k$ so we get a linear dispersion law with leads to $C \propto T$. 
\bigskip\\
The question is if we move away from an ideal monotomic 1D lattice, do these properties change?

\subsection{Phenomenological Approach}

Concisder $a \ll d \ll \lambda_T$ where,
\[ \lambda_T = \frac{2 k}{k} = \frac{2 \pi V}{T} \]
is the reduced thermal wavelength. Then w let $u(x)$ be the averaged $u_n$ given by,
\[ u(x) = \sum_{n} \phi_d(x - an) u_n \]
where $\phi_d$ is some weighting function of width $d$. (In class he writes $\phi_d$ is a step function $\frac{1}{d} \cdot 1_{[-d/2, d/2]}$ but I think it must be continuous for $u(x)$ to be continuous). Now define the density $\rho = \frac{m}{a}$ and, ignoring the internal clump energy (which is not important in the long wavelength approximation),
\[ T = \rho \int \tfrac{1}{2} \dot{u}(x)^2 \d{x} \]
Now we need a form of the potential energy which we will express as,
\[ U[u(x)] = \int \d{x} \Phi \left( u, \deriv{x}{x}, \nderiv{2}{u}{x}, \cdots \right) \]
We make the following assumption,
\begin{enumerate}
\item Locality: higher derivatives are vanishing in importance.
\item translation invarianc:
\[ \Phi[u(x) + b] = \Phi[u(x)] \]
\item stability: no dependence on $\deriv{u}{x}$ so we may write,
\[ U[u(x)] = \int \d{x} \left \{ \frac{1}{2} K \left( \deriv{u}{x} \right)^2 + \frac{1}{2} L \left( \nderiv{2}{u}{x} \right)^2 + \cdots + M \left( \deriv{u}{x} \right)^3 + \cdots \right\} \]
\end{enumerate}
Now replacing $u(k)$ with its fourier transform gives,
\begin{align*}
U[u(x)] & = \int \frac{\d{k}}{2 \pi} \left[ \frac{1}{2} k^2 + \frac{1}{2} L k^4 + \cdots \right] |u(k)|^2
\\
& \quad - i M \int \frac{\d{k_1} \d{k_2}}{(2 \pi)^2} k_1 k_2 (k_1 + k_2) u(k_1) u(k_2) u(-k_1 - k_2) 
\end{align*}
Which, again, in the low $k \to 0$ is quadratic and thus gives linear dispersion. 
\bigskip\\
Now consider dimension greater than $1$. We have $u(\vec{k})$. If we restrict ourselves to Isotropic media then symmetry restricts of the form of the lowest-order terms to be,
\[ \vec{u}(\vec{k})^2 \quad \text{ or } \quad (\vec{k} \cdot \vec{u}(\vec{k}) )^2 \]
Then,
\[ \H_{\text{eff}} = \int \frac{\dn{d}{k}}{(2 \pi)^d} \left[ \frac{1}{2} \rho \pderiv{\vec{u}}{t} \cdot \pderiv{\vec{u}}{t} + \frac{1}{2} \mu k^2 \vec{u}^2 + \frac{1}{2} (\mu + \lambda) [\vec{k} \cdot \vec{u} ]^2 \right] \]
This gives linear dispersion but for two types of waves:
\begin{enumerate}
\item Longitudinal waves have $\vec{k} \parallel \vec{u}$ and speed $c_\ell = \sqrt{\frac{2 \mu + \lambda}{\rho}}$.
\item Transverse waves have $\vec{k} \perp \vec{u}$ and speed $c_t = \sqrt{\frac{\mu}{\rho}}$. 
\end{enumerate}
Then the bulk energy due to photons can be expressed as,
\[ E(T) = L^d \int \frac{\dn{d}{k}}{(2 \pi)^d} \left[ \frac{\hbar v_\ell k}{1 - e^{\beta \hbar v_\ell}} + (d - 1) \frac{\hbar v_t k}{1 - e^{\beta \hbar v_t \beta}} \right] \approx A(v_\ell, v_t) T^{d + 1} \]
\end{document}
