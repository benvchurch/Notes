\documentclass[12pt]{article}
\usepackage{hyperref}
\hypersetup{
    colorlinks=true,
    linkcolor=blue,
    filecolor=magenta,      
    urlcolor=blue,
}

\usepackage{import}
\import{./}{LogicCommands}

\newcommand{\RRed}{\triangleright}
\newcommand{\aconv}{\equiv_{\alpha}}
\newcommand{\bredo}{\RRed_{\beta,1}}
\newcommand{\bred}{\RRed_{\beta}}
\newcommand{\etared}{\RRed_{\eta}}
\newcommand{\etaredo}{\RRed_{\eta,1}}
\newcommand{\bered}{\RRed_{\beta\eta}}
\newcommand{\beredo}{\RRed_{\beta\eta,1}}

\renewcommand{\P}{\mathbb{P}}
\newcommand{\B}{\mathcal{B}}
\newcommand{\cP}{\mathcal{P}}
\usepackage{stmaryrd}

\newcommand{\dbrack}[1]{\llbracket #1 \RRbracket}
\renewcommand{\L}{\mathcal{L}}
\newcommand{\as}{\mathrm{a.s.}}
\let\oldimplies\implies
\renewcommand{\implies}{\hspace{-1ex} \oldimplies \hspace{-1ex}}


% Bib Formatting

\usepackage{blindtext}
\usepackage[style=nature, backend=bibtex8,
            autocite=footnote, notetype=endonly, labeldateparts]{biblatex}
            
\usepackage{hyperref}
\usepackage{fancyhdr}

\addbibresource{bibliography.bib}

\begin{document}

\pagestyle{fancy}
\fancyhead[LH]{Geometric models of comparatives.}
\fancyhead[RH]{Ben Church}
\setlength{\headheight}{11pt}
\setlength{\headsep}{0.2in}


\section{Abstract}

Motivated by the ``arguments from small improvements'' we propose a geometric-topological model of comparatvies where the geometry of Morse functions and vector fields reproduce features of positive and comparative adjectives respectively. Moreover, this gives a framework for understanding ``improper'' comparatives which do not appear to be derived from a corresponding positive form as well as ``multi-dimensional'' properties which do not clearly derive a comparative form from their constituent parts and hence from their positive form. 

\section{Introduction}

A central problem in the formal semantics of adjectives is to derive semantics of the comparative form of an adjective from its corresponding positive form. For example to derive semantics for `taller' from semantics for 'tall'. This thesis that,
\begin{center}
if $A$ is a positive adjective, then the meaning of `$A$-er than' is a function of the meaning of $A$
\end{center}
(WHAT DO THE BRACETS IN KLEIN MEAN)
traces to [EWAN KLEIN] who identified that such a derivation must exist if we hold Frege’s principle of compositionality. Furthermore, the thesis is supported from natural language morphology. It is common (as is the case in english and romance languages) for comparatives to be derived from a formally unmarked root positive adjective via a regular morphological process, for example `tall' becomes `taller' where `tall' is an unmarked morpheme. 


(COMPARATIVE -> POSiTIVE STRATEGY)

Klein dismisses any such strategy as ``inadequate'' for violating the minimal form of Frege's principle, requiring unmotivated appeals to complex underlying structure, and disregarding natural language structure.
Another difficulty with this strategy is nicely illustrated by an example of [DORR] (WESTERLY) which shows that a putative comparative can have nice `local' (for cities close by) structure but improper `global' (for cities far apart) properties (for example, failures of transitivity). Like Dorr, we do not expect that `more westerly' should be derivable from a semantically-coherent positive property `westerly'. However, where as Dorr dismisses this example and similar edge cases saying ``these expressions [are not] comparatives in the sense we are concerned with'' (Dorr, p.5) we take them to be central to our analysis.
\par 
The strategy laid out here does not attempt to derive, in specific or general cases, the semantics of a comparative from a positive or vice versa but rather to investigate structures where we can ask if positive or comparative forms can exist compatibly. As such, this work progresses in a dialectical fashion between the positive and the comparative using formal desiderata of each form to constrain our model of the other, hence arriving at a set of shared mutually compatible postulates. We will find, in agreement with Klein's thesis, that positive forms have derived comparative forms but there exist (local) comparative adjectives which do not derive from an associated positive form. 

(TAKE AS BASIS CRESSWELL DEGREE THEORY) 

(CENTRAL INSIGHT OF SMALL IMPROVMENTS ARGUMENT)

\section{Degree of Comparison}


(TAKE AS BASIS CRESSWELL DEGREE THEORY) 

NEED TO DISCUSS DEGREE OF COMPARISON CAN USE VON-NEUMAN THEOREM FROM ECON. 

\section{Local Comparatives and Small Improvements}

Our distinction between ``local'' and ``global'' regimes is motivated by arguments from small improvements [DORR, CHANG, RAZ]. These small improvment arguments are designed to support or challenge strict comparativity (i.e. whether comparative adjectives represent total or partial orders). Chang designs his small improvement argument as a chanllenge to  

Considering the monetary example, because we defined our small improvements based on fixed cash amounts and it is reasonable to assume that in the generic case $\$ 2$ ought to provide a greater increase in utility, or whichever measure of value is under discussion, than $\$ 1$ case, any base unit by whose addition we define a small improvement will necessarily be arbitrary. For any base amount, providing half the cash sum should produce a yet smaller small improvement. It it therefore most natural to take a continuum limit of small improvements shifting our focus from ``smallness'' in the sense of measure (since on a continuum any nonzero quantity can be made arbitrarily small or large via rescalling) to ``locality'' in the sense of geometry. If we take the leap to allowing for arbitrary small improvements along a continuum we are led to a notion of space on which our comparatives act which locally looks like the standard continuum $\RR$ (or as is argued in the sequel, we may want our space to be locally modeled on $\RR^n$). However, this is not to necessarily make any claim for metaphysical continuity as opposed to discretization in the actual possibility of small improvements. Indeed, we may view continuity as merely an useful analytical tool for capturing the behavior of very small quantities without a prescribed indecomposable quantum (or even when such a quantum is readily available) as is often done in the natural sciences\footnote{For example, in the analysis of radioactive substances one writes -- without comment -- that the number of nuclei of a radioactive species after time $t$ is $N(t) = N_0 e^{-t / \tau}$. Strictly, if we interpret ``number of atoms'' in the usual sense as a cardinality (in particular an integer) this is nonsense. We approximate these integers by a continuous value even those there is a clear base unit, $1$ atom, without issue for analytical convenience. One might object that the exponential law is justified even in the case of a small number of nuclei by quantum mechanics which predicts exponentially ditributed decay probabilities and hence the exact exponential law for the expected number; thus that the ``actual'' approximation underlying the exponential law is replacing a random integer variable (the number of nuclei of the radioactive species) by its expected value which has no reason to be an integer. This objection misses that continuum approximations are useful in macroscopic physics where the underlying behavor does not depend on quantum mechanics such as molecules constituting a liquid (although one might still say that the continuous density field of the liquid represents a quantum mechanical density function) or even outside of atomic physics such as modeling traffic flow as a fluid approximating the discrete number of cars on a road by a continuous density.}

\section{Multidimensionality}

\section{Topological Manifolds and Vector Fields}

We have justified why our models should be manifolds by appeal to their local structure. However, if manifolds did not have interesting global structure they would be of no interest to study and mathematicians and practioners would be content to keep their signts fixed on $\RR^n$. Likewise, for this work to add anything beyond mathematical jargon to the study of comparatiives we should justify why we expect manifolds with interesting global structure to arise. 

THE DIRECTION CAN CHANGE AS YOU MOVE AROUND THE MANIFOLD

WESTERLY EXAMPLE


\section{Comparability}

\section{Conclusion}

\section{Addendum: Long Manifolds}

\section{Appendix A: Topological Manifolds}

\section{Appendix B: Calculus on Manifolds}

\section{Appendix C: Morse Theory}



\end{document}
