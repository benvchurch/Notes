\documentclass[12pt]{article}
\usepackage{hyperref}
\hypersetup{
    colorlinks=true,
    linkcolor=blue,
    filecolor=magenta,      
    urlcolor=blue,
}

\usepackage{import}
\import{./}{LogicCommands}

\newcommand{\red}{\triangleright}
\newcommand{\aconv}{\equiv_{\alpha}}
\newcommand{\bredo}{\red_{\beta,1}}
\newcommand{\bred}{\red_{\beta}}
\newcommand{\etared}{\red_{\eta}}
\newcommand{\etaredo}{\red_{\eta,1}}
\newcommand{\bered}{\red_{\beta\eta}}
\newcommand{\beredo}{\red_{\beta\eta,1}}

\renewcommand{\P}{\mathbb{P}}
\newcommand{\B}{\mathcal{B}}
\newcommand{\cP}{\mathcal{P}}
\usepackage{stmaryrd}

\newcommand{\dbrack}[1]{\llbracket #1 \rrbracket}
\renewcommand{\L}{\mathcal{L}}
\newcommand{\as}{\mathrm{a.s.}}
\let\oldimplies\implies
\renewcommand{\implies}{\hspace{-1ex} \oldimplies \hspace{-1ex}}


% Bib Formatting

\usepackage{blindtext}
\usepackage[style=nature, backend=bibtex8,
            autocite=footnote, notetype=endonly, labeldateparts]{biblatex}
            
\usepackage{hyperref}
\usepackage{fancyhdr}

\addbibresource{bibliography.bib}

\begin{document}

\pagestyle{fancy}
\fancyhead[LH]{Notes on Bacon, ``Stalnaker's Thesis in Context'' (2015)}
\fancyhead[RH]{Ben Church}
\setlength{\headheight}{11pt}
\setlength{\headsep}{0.2in}

\section{The Construction}

Consider a probability space $X = (W, \Sigma, \P)$ where $W$ is a set of worlds, $\Sigma$ is a $\sigma$-algebra, $\P$ a probability function. We are going to define a lift to uncountably infinite sequences of worlds. 

\begin{defn}
The probability space is the product space,
\[ X_{\infty} = \prod_{\alpha < \omega_1} X \]
\end{defn}

\subsection{Products of Measure Spaces}

\begin{defn}
The product of a (possibility infinite) collection of probability spaces $\{ X_i = (W_i, \Sigma_i, \P_i) \}_{i \in I}$ has the following components. The underlying set is,
\[ W = \prod_{i \in I} W_i \]
with a $\sigma$-algebra constructed from the following the basis sets,
\[ \B = \left\{ \prod_{i \in I} S_i \, \middle| \, S_i \in \Sigma_i \text{ and } S_i = W_i \text{ for all but finitely many } i \right\} \]
where we let $\Sigma = \overline{\B}$ be the smallest $\sigma$-algebra containing $\B$ meaning the closure of $\B$ under complements and countable unions. Finally, we define the probability measure $\P$ on the basis sets,
\[ \P \left( \prod_{i \in I} S_i \right) = \prod_{i \in I} \P_i(S_i) \]
which makes sense because $\P_i(S_i) = 1$ for all but finitely many $i \in I$. Then Caratheodory's theorem proves that $\P$ extends uniquely to $\Sigma$.
\end{defn}

\begin{rmk}
It is worth explaining why the product $\sigma$-algebra has this strange finiteness condition. One reason is that otherwise we would not know that,
\[ \prod_{i \in I} \P_i(S_i) \]
is meaningful without worrying deeply about convergence issues. A more fundamental issue is the following. In general, product objects are distinguished by the following \textit{universal property}. For any test space $T$ the two sets of data are equivalent,
\[ f : T > \prod_{i \in I} X_i \iff \{ f_i : T > X_i \}_{i \in I} \]
Let's do an example to see why this requires some finiteness (really its a countability) restriction on the basis sets. Let $I$ be an uncountable set. The identity function $\id : \RR > \RR$ is of course measurable. So this condition should tell us that these glue together to give a measurable function,
\[ f : \RR > \prod_{i \in I} \RR \quad \quad f(x) = (x,x,x,\dots) \]
However, consider,
\[ f^{-1} \left( \prod_{i \in I} S_i \right) = \{ x \in \RR \mid f(x) = (x,x, \dots) \in \prod_{i \in I} A_i \} = \bigcap_{i \in I} A_i \]
However, the sets $A_i$ are arbitrary measurable sets so this can only be possibly if the intersection of arbitrarily many measurable sets is measurable. This is false (assuming the axiom of choice). Let $S \subset \RR$ be any non-measurable set and consider,
\[ S = \bigcap_{x \in (\RR \sm S)} (\RR \sm \{ x \}) \]
but clearly the sets $\RR \sm \{ x \}$ are measurable. 
\end{rmk}

\subsection{Bacon's Selection Function}

Now that we have a probability space $X_{\infty} = (W_{\infty}, \Sigma_{\infty}, \P_{\infty})$ we can use the structure of $\omega_1$ to define a selection function.

\begin{rmk}
Here we equivocate between an ordinal $\alpha$ and its set representation as $\{ \beta \mid \beta < \alpha \}$. For example,
\begin{align*}
0 \iff & \{ \}
\\
1 \iff & \{ 0 \}
\\
2 \iff & \{ 0, 1 \}
\\
\vdots &
\\
\omega \iff & \{0, 1, 2, \dots \} = \N 
\end{align*}
\end{rmk}

\begin{defn}
For $\alpha < \omega_1$ and $x, y \in W_{\infty}$ we view $\alpha \subset \omega_1$ and say $x \sim_{\alpha} y$ iff $x|_{\alpha} = y|_{\alpha}$. Say that $A$ is $\alpha$-\textit{insensitive} if,
\[ \forall x, y \in W_{\infty} : x \sim_{\alpha} y \oldimplies (x \in A \iff y \in A) \]
\end{defn}

\begin{rmk}
Explicitly,
\begin{enumerate}
\item $x \sim_0 y$ is vacuously true
\item $x \sim_1 y \iff x(0) = y(0)$
\item $x \sim_n y \iff x(m) = y(m)$ for all $m < n$
\item $x \sim_{\omega} y \iff x(m) = y(m)$ for all $m \in \N$
\item $x \sim_{\omega + 1} y \iff x(m) = y(m)$ for all $m \in \N$ and $x(\omega) = y(\omega)$.
\end{enumerate}
\end{rmk}

\begin{defn}
For $A \in \Sigma_{\infty}$ define the rank of $A$,
\[ \rank(A) = \min \{ \alpha \mid A \text{ is } \omega^\alpha\text{-insensitive} \} \]
\end{defn}

\begin{prop}
For any $A \in \Sigma_{\infty}$ the rank exists and $\rank{(A)} < \omega_1$.
\end{prop}

\begin{proof}
Because any nonempty set of ordinals has a least element $\rank{(A)} < \omega_1$ is equivalent to: there is some $\alpha < \omega_1$ such that $A$ is $\alpha$-insensitive.
\bigskip\\
We will prove that the set $\Sigma_{\rank} \subset \cP(W_{\infty})$ of sets $A \subset W_{\infty}$ with $\rank(A) < \omega_1$ is a $\sigma$-algebra containing $\B$ This suffices because then $\Sigma_{\infty} \subset \Sigma_{\rank}$ as $\Sigma_{\infty}$ is the \textit{smallest} $\sigma$-algebra containing $\B$.
\bigskip\\
First, for any $A \in \B$ using the finiteness condition, take $\alpha < \omega_1$ to be the maximum of the nontrivial indices. If $A$ is $\alpha$-insensitive then $A^C$ is $\alpha$-insensitive so $A \in \Sigma_{\rank} \iff A^C \in \Sigma_{\rank}$. Finally, we need to show closure under countable unions. 
\bigskip\\
For any sequence $\{ A_i \}_{i \in I}$ if $A_i$ is $\alpha$-insensitive for each $i$ then $A = \bigcup_{i \in I} A_i$ is $\alpha$-insensitive therefore if $A_i$ is $\alpha_i$-insensitive then letting,
\[ \alpha = \sup \{ \alpha_i \mid i \in I \} \]
we see that each $A_i$ is $\alpha$-insensitive and hence $A$ is $\alpha$-insensitive the only issue is that $\alpha = \omega_1$ is possible. However, if $I$ is countable then $\alpha$ is a supremum of a countable set of countable ordinals and hence $\alpha < \omega_1$. Thus if $A_i \in \Sigma_{\rank}$ then $A \in \Sigma_{\rank}$ proving that $\Sigma_{\rank}$ is a $\sigma$-algebra.
\end{proof}

\begin{defn}
For $\pi \in W_{\infty}$ define $\pi[\alpha]$ to be the sequence $\pi[\alpha](\beta) = \pi(\alpha + \beta)$. This is the same sequence ``forgetting the first $\alpha$ terms.'' 
\end{defn}

\begin{defn}
We say that $\pi$ \textit{occurs} in $A$ if there is some integer $i \in \N$ such that $\pi[\omega^\alpha \cdot i] \in A$. If $\pi$ occurs in $A$ we call the minimal such value of $i$ the \textit{index} of $\pi$ with respect to $A$.
\end{defn}

\begin{defn}
The selection function $f : \Sigma_{\infty} \times W_{\infty} > W_{\infty}$ is defined by,
\[ f(A, \pi) = 
\begin{cases}
\perp & A = \empty 
\\
\pi[\omega^\alpha \cdot i] & \pi \text{ occurs in } A \text{ and } i \text{ is the index}
\\
\tau_A & \pi \text{ does no occur in } A
\end{cases} \]
where $\perp$ is the list of contradictory worlds 
\end{defn}

\begin{defn}
Given $A, B \in \Sigma_{\infty}$ define the implicative,
\[ (A \implies B) := \{ \pi \in W_{\infty} \mid f(A, \pi) \in B \} \] 
\end{defn}

\begin{thm}
For any $A, B \in \Sigma_{\infty}$ we have $(A \implies B) \in \Sigma_{\infty}$ and if $\P(A) > 0$ then,
\[ \P(A \implies B) = \P(B \mid A) = \frac{\P(A \cap B)}{\P(A)} \]
\end{thm}

\begin{proof}
Let $\alpha = \rank(A)$. Then,
\[ (A \implies B) = \left[ \bigcup_{i \in \N} \{ \pi \mid \pi[\omega^\alpha \cdot i] \in A \cap B \} \right] \cup \{ \pi \mid \text{not occurring in } A \text { and } \tau_A \in B \} \]
Each $\{ \pi \mid \pi[\omega^\alpha \cdot i] \in A \cap B \} \in \Sigma_{\infty}$ because $\omega^\alpha \cdot i$ is countable. Thus the first term and the second, 
\[ \{ \pi \mid \text{not occurring in } A \} = \bigcap_{i \in \N} \{ \pi \mid \pi[\omega^\alpha \cdot i] \notin A \} \]
are both measurable. Now since $\rank(A) = \alpha$ the sets $S_i = \{ \pi \mid \pi[\omega^\alpha \cdot i] \notin A \}$ are independent with equal probability (they are each the set of $\pi$ when restricted to the range $\omega^\alpha \cdot i$ to $\omega^{\alpha} \cdot (i+1)$ lands in the interesting part of $A$) so,
\[ \P(\{ \pi \mid \text{not occurring in } A \}) = \lim_{n > \infty} \P(A^C)^n = 0 \]
because $P(A) > 0$. Likewise, we can expand the first term as a disjoint union,
\[ \bigcup_{i \in \N} \{ \pi \mid \pi[\omega^\alpha \cdot i] \in A \cap B \} = \bigcup_{i \in \N} \bigcap_{j < i} S_j \cap \{ \pi \mid \pi[\omega^\alpha \cdot i] \in A \cap B \} \]
The union is disjoint and the intersection is over independent sets so,
\[ \P(A \implies B) = \sum_{i = 1}^\infty \P(S_0) \cdots \P(S_{i-1}) \P(A \cap B) = \sum_{i = 1}^{\infty} \P(A^C)^i \P(A \cap B) = \frac{\P(A \cap B)}{1 - \P(A^C)} = \frac{\P(A \cap B)}{\P(A)} \]
\end{proof}

\section{Triviality Results}

\subsection{Probabilistic Triviality}

\begin{defn}
We say a probability space $X = (W, \Sigma, \P)$ is \textit{non-trivial} if there exist events $A, B \in \Sigma$ such that,
\begin{enumerate}
\item $\P(A) > 0$
\item $\P(B) > 0$
\item $\P(A \wedge B) = 0$
\item $\P(A \vee B) < 1$
\end{enumerate}
otherwise we say that $X$ is \textit{trivial}.
\end{defn}

\begin{lemma}
The following are equivalent,
\begin{enumerate}
\item $X$ is trivial
\item $\P(B | A) = \P(B)$ for all $A, B \in \Sigma$ with $\P(A \wedge B) > 0$ and $\P(A \wedge \neg B) > 0$
\item all $A, B \in \Sigma$ are independent or $\P(A \wedge B) = 0$ or $\P(A \wedge \neg B) = 0$
\end{enumerate}
\end{lemma}

\begin{proof}
Suppose $X$ is trivial. If $\P(A \wedge B) > 0$ and $\P(A \wedge \neg B)$ then applying triviality $\P(A) = \P((A \wedge B) \vee (A \wedge \neg B)) = 1$. Hence, $\P(B|A) = \P(A \wedge B) = \P(B)$. Also (b) and (c) are equivalent since $\P(B|A) = \P(B)$ iff $\P(A \wedge B) = \P(A) \P(B)$ iff $A$ and $B$ are independent. Finally, assume (c) then we claim that $X$ is trivial. Indeed, if $A, B \in \Sigma$ with $\P(A) > 0$ and $\P(B) > 0$ and $\P(A \wedge B) = 0$ then let $C = A \vee B$. Then $\P(C \wedge A) = \P(A) > 0$ and $\P(C \wedge \neg A) = \P(B) > 0$ so by (c) $\P(C \wedge A) = \P(C) \P(A)$ but $\P(C \wedge A) = \P(A)$ and thus $\P(C) = 1$ proving triviality. 
\end{proof}

\begin{lemma}[Lewis]
Let $0 < \P(B) < 1$. The following cannot all be true,
\begin{enumerate}
\item $\P(A > B | B) = 1$
\item $\P(A > B | \neg B) = 0$
\item $\P(A > B) \neq \P(B)$
\end{enumerate}
\end{lemma}

\begin{proof}
Suppose (a) and (b) then the law of total probability gives,
\[ \P(A > B) = \P(A > B | B) \P(B) + \P(A > B | \neg B) \P(\neg B) = \P(B) \]
\end{proof}

\begin{cor}
If the following hold,
\begin{enumerate}
\item[(L1)] for all $A, B \in \Sigma$ with $\P(A \wedge B) > 0$ and $\P(A \wedge \neg B) > 0$ we have $\P(A > B | B) = 1$
\item[(L2)] for all $A, B \in \Sigma$ with $\P(A \wedge B) > 0$ and $\P(A \wedge \neg B) > 0$ we have $\P(A > B | \neg B) = 0$
\end{enumerate}
then $X$ is trivial.
\end{cor}

\begin{cor}
If for all $A, B, C \in \Sigma$ with $\P(A \wedge C) > 0$,
\[  \P(A > B | C) = \P(B | A \wedge C) \]
then $X$ is trivial.
\end{cor}

\begin{prop}[Lewis]
The set of events for which the thesis holds for all conditionalized probability measures is trivial. 
\end{prop}

\begin{proof}
We verify the property,
\[ \P(A > B | C) = \P(B | A \wedge C) \]
when $\P(A \wedge C) > 0$. Indeed,
\[ \P_C(A > B) := \P(A > B | C) \]
and since the thesis holds for $\P_C$ we have,
\[ \P_C(A > B) = \P_C(B|A) = \frac{\P_C(B \wedge A)}{\P_C(A)} = \frac{\P(B \wedge A | C)}{\P(A|C)} = \frac{\P(B \wedge A \wedge C) \P(C)}{\P(A \wedge C) \P(C)} = \P(B | A \wedge C) \]
\end{proof}

\begin{rmk}
Stalnaker claims that the above argument relies on the hidden assumption: ``metaphysical realism'' that the proposition expressed by a conditional sentence is independent of the probability function. This he claims is implicit in that the content of $A > B$ is the same in the contexts $\P(A > B)$ and $\P_C(A > B)$ which is needed in claiming that,
\[ \P_C(A > B) = \P(A > B | C) \]
The following argument of Fitelson avoids conditionalization and therefore removes this assumption as well as the assumption that the thesis need be preserved under conditionalization.
\end{rmk}

\begin{prop}[Fitelson] 
The set of events $A, B \in \Sigma$ such that,
\begin{enumerate}
\item[(PIE)] $\P(A > (B > C)) = \P((A \wedge B) > C)$
\item[(T)] $\P(A > B) = \P(B|A)$ for $\P(A) > 0$
\end{enumerate}
is trivial.
\end{prop}

\begin{proof}
We verify the property,
\[ \P(A > B | C) = \P(B | A \wedge C) \]
as follows: using $\P(C) > 0$ the thesis says,
\[ \P(A > B | C) = \P(C > (A > B)) \]
using PIE,
\[ \P(C > (A > B)) = \P((A \wedge C) > B) \]
then because $\P(A \wedge C) > 0$ we can apply the thesis to conclude,
\[ \P((A \wedge C) > B) = \P(B | A \wedge C) \]
\end{proof}

\subsection{Stalnaker's Triviality Result}

\begin{rmk}
Stalnaker isolates the fragment of the propositional calculus of the conditional which is problematic when combined with the thesis. (CITE BACON's OTHER PAPER FOR MORE DETAIL)
\end{rmk}


\begin{defn}
Stalnaker's logic \textbf{C2} for the conditional is generated on the language $\L$ with signature $\sigma = \{ \wedge, \vee, \neg, \to,  > \}$ via the standard Hilbert-style calculus for the fragment $\{ \wedge, \vee, \neg, \to \}$ and additional axioms,
\begin{enumerate}
\item[LT] $A > \top$

\item[ID] $A > A$

\item[AND] $((A > B) \wedge (A > C)) \to (A > (B \wedge C))$

\item[OR] $((A > C) \wedge (B > C)) \to ((A \vee B) > C)$

\item[CCut] $((A > V) \wedge ((A \wedge B) > C)) \to (A > C)$

\item[CMon] $((A > B) \wedge (A > C)) \to ((A \wedge B) > C)$

\item[CSO] $((A > B) \wedge (B > A)) \to ((A > C) \equiv (B > C))$

\item[SM] $(A > B) \to (A \to B)$

\item[CS] $(A \wedge B) \to (A \to B)$

\item[RMon] $((A \to B) \wedge \neg(A > \neg C)) \to ((A \wedge C) > B)$

\item[CEM] $(A > B) \vee (A > \neg B)$
\end{enumerate}
along with additional rules of inference,
\begin{enumerate}
\item[LLE] $B \equiv C \proves (B > A) \equiv (C > A)$

\item[RW] $(B \to C) \proves (A > B) \to (A > C)$

\item[RCK] $(B_1 \wedge \dots \wedge B_n) \to C \proves (A > B_1) \wedge \dots \wedge (A > B_n) \to (A > C)$.
\end{enumerate}
\end{defn}

\begin{lemma}
The schema \textit{Strict Centering} follows from \textbf{C2},
\begin{enumerate}
\item[C2] $A \wedge (A \to B) \equiv A \wedge B$
\end{enumerate}
\end{lemma}

\begin{proof}
DO THIS IT'S CLEAR
\end{proof}


\begin{rmk}
Here, $\to$ is the material conditional so $A \to B :\equiv \neg A \vee B$.
\end{rmk}

\begin{defn}
A \textit{credence-conditional space} $X = (W, \Sigma, \P, \implies)$ is a probability space $(W, \Sigma, \P)$ along with a binary operation $\implies$ on $\Sigma$. An \textit{interpretation function} is an assignment of propositional variables, $p \mapsto \dbrack{p} \in \Sigma$ extended to $\L$ via,
\begin{enumerate}
\item $\dbrack{A \wedge B} = \dbrack{A} \cap \dbrack{B}$
\item $\dbrack{A \vee B} = \dbrack{A} \cup \dbrack{B}$
\item $\dbrack{\neg A} = W \sm \dbrack{A}$
\item $\dbrack{A > B} = \dbrack{A} \implies \dbrack{B}$.
\end{enumerate}
We say $X$ \textit{verifies almost surely} a sentence $A \in \L$ if $\P(\dbrack{A}) = 1$ for any interpretation function $\dbrack{\bullet}$ which we write as $X \entails_{\as} A$.
\end{defn}


\begin{defn}
A credence-conditional space $X$ is a \textit{model} of a theory $\Gamma$ over $\sigma$ if $X \entails_{\as} \Gamma$.
\end{defn}

(WHAT IS THE CORRECT THING TO SAY ABOUT THE RULES OF INFERENCE OR CAN I MAKE THEM JUST AXIOMS?)

\begin{theorem}[Stalnaker]
Suppose that $X$ is a credence-conditional space such that,
\begin{enumerate}
\item $X$ is a model of \textbf{C2} i.e. $X \entails_{\as} \textbf{C2}$
\item $X$ verifies the thesis i.e. $\P(A \implies B) = \P(B | A)$ for all $A, B \in \Sigma$ with $\P(A) > 0$.
\end{enumerate}
then $X$ is a trivial probability space. 
\end{theorem}

\begin{proof}
Step 1, we show that if $\P(\neg A) > 0$ then,
\[ \P(A > B | \neg A) = \P(A > B) \]
Indeed,
\[ \P(A > B | \neg A) = \frac{\P(A > B) - \P(A \wedge (A > B))}{\P(\neg A)} = \frac{\P(A > B) - \P(A \wedge B)}{\P(\neg A)} \] 
If $\P(A) = 0$ then we conclude $\P(A > B | \neg A) = \P(A > B)$. Otherwise, we may apply the thesis,
\[ \P(A > B | \neg A) = \frac{\P(A \wedge B) - \P(A \wedge B) \P(A)}{\P(A) \P(\neg A)} = \frac{\P(A \wedge B)}{\P(A)} = \P(B | A) = \P(A > B) \]
Step 2, assume that $X$ is nontrivial with representatives $A, B \in \Sigma$
\end{proof}


\begin{theorem}
Bacon's Model $X_{\infty}$ satisfies,
\begin{enumerate}
\item $X_{\infty} \entails_{\as} \textbf{C2} \sm \text{CSO}$
\item $X_{\infty}$ verifies the thesis: $\P_\infty(A \implies B) = \P(B | A)$ for $A, B \in \Sigma_{\infty}$ with $\P_\infty(A) > 0$.
\end{enumerate}
\end{theorem}

\begin{proof}
WORK IN PROGRESS
\end{proof}

\begin{cor}
Bacon's model $X_{\infty}$ must violate \text{CSO}. 
\end{cor}

\begin{rmk}
Bacon proposes a number of potential counter-examples to CSO in natural language \cite{Bacon}.
\end{rmk}


\section{Appendix: relation of my notation to Bacon's}

In Bacon's paper, Stalnaker's Thesis in Context, he formulates the construction of his semantics in a somewhat different form. He writes $W_\alpha = W^{\omega^\alpha}$ which has the inductive form,
\begin{enumerate}
\item $W_0 = W$
\item $W_{\alpha+1} \cong W_\alpha^\omega$
\end{enumerate} 
Then abusing notation to write $A \times W_{\infty}$ for $A \subset W_{\alpha}$ to mean the set of sequences starting $\pi$ such that $\pi|_{\omega^\alpha} \in A$ (with the remainder arbitrary), he defines inductively a filtered $\sigma$-algebra,
\begin{enumerate}
\item $\Sigma_0 = \{ A \times W_{\infty} \mid A \in \Sigma \}$
\item $\Sigma_{\alpha + 1} = \overline{\{ A_0 \times \cdots \times A_n \times W_{\infty} \mid n \in \N \text{ and } A_i \times W_{\infty} \in \Sigma_{\alpha} \text{ for } 0 \le i \le n \}}$
\item $\Sigma_{\gamma} = \overline{ \bigcup_{\alpha < \gamma} \Sigma_\alpha}$
\end{enumerate}
Then Bacon's definition of $\Sigma_{\infty}^{(B)}$ is the union of this filtered sequence of $\sigma$-algebras.

\begin{prop}
These definitions agree, meaning $\Sigma^{(B)}_{\infty} = \Sigma_{\infty}$ and moreover,
\[ A \in \Sigma_{\alpha} \iff \rank(A) \le \alpha \]
\end{prop}

\begin{proof}
We prove by transfinite induction that $\Sigma_{\alpha} \subset \Sigma_{\infty}$. It is clear that $\Sigma_0 \subset \Sigma_{\infty}$. If $\Sigma_{\alpha} \subset \Sigma_{\infty}$ then for $A_i \in \Sigma_\alpha$ the set $A_0 \times \cdots \times A_n \times W_{\infty}$ is a finite intersection of shifts of elements of $\Sigma_{\infty}$ and thus an element of $\Sigma_{\infty}$ so $\Sigma_{\alpha+1} \subset \Sigma_{\infty}$. Finally, the limit ordinal case is clear because if each $\Sigma_\alpha \subset \Sigma_{\infty}$ then the union and closure (since $\Sigma_{\infty}$ is a $\sigma$-algebra) also lies inside $\Sigma_{\infty}$ proving by transfinite induction that all $\Sigma_{\alpha} \subset \Sigma_{\infty}$ and hence $\Sigma_{\infty}^{(B)} \subset \Sigma_{\infty}$.
\bigskip\\
To prove that $\Sigma_{\infty}$ is exhausted by the $\Sigma_\alpha$ it suffices to prove the second claim since every $A \in \Sigma_{\infty}$ has some rank and hence would lie in some $\Sigma_{\alpha} \subset \Sigma_{\infty}^{(B)}$.
\bigskip\\
Now we prove,
\[ A \in \Sigma_\alpha \iff \rank(A) \le \alpha \]
by transfinite induction. Let $\Sigma_{\rank \le \alpha} = \{ A \in \Sigma_{\infty} \mid \rank(A) \le \alpha \}$. I claim that $\Sigma_{\rank \le \alpha}$ is the $\sigma$-algebra generated by $\B \cap \Sigma_{\rank \le \alpha}$. It is clear that $\B \cap \Sigma_{\rank \le \alpha} \subset \Sigma_{\alpha}$ so $\Sigma_{\rank \le \alpha} \subset \Sigma_{\alpha}$. Now, $A \in \Sigma_0 \iff \rank(A) = 0$ because both describe $1$-insensitive sets. Now we show the inductive steps.  Suppose $\Sigma_{\alpha} = \Sigma_{\rank \le \alpha}$. For any set of the form,
\[ A = A_0 \times \cdots \times A_n \times W_{\infty} \]
with $A_i \times W_{\infty} \in \Sigma_{\alpha}$ then by definition $\rank{(A)} \le \alpha + 1$. Since $\Sigma_{\alpha+1}$ is generated by sets of this form, $\Sigma_{\alpha+1} \subset \Sigma_{\rank \le \alpha + 1}$. Finally, we do the limit step. For $\alpha < \gamma$ we have $\Sigma_{\alpha} = \Sigma_{\rank \le \alpha} \subset \Sigma_{\rank \le \gamma}$ so $\Sigma_{\gamma} \subset \Sigma_{\rank \le \gamma}$.
\end{proof}


\end{document}
