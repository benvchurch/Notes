\documentclass[12pt]{article}
\usepackage[utf8]{inputenc}
\usepackage[english]{babel}
\usepackage[a4paper, total={6in, 9in}]{geometry}

\usepackage{amsthm, amssymb, amsmath, centernot}

\newcommand{\notimplies}{%
  \mathrel{{\ooalign{\hidewidth$\not\phantom{=}$\hidewidth\cr$\implies$}}}}

\renewcommand\qedsymbol{$\square$}
\newcommand{\cont}{$\boxtimes$}
\newcommand{\divides}{\mid}
\newcommand{\ndivides}{\centernot \mid}
\newcommand{\Z}{\mathbb{Z}}
\newcommand{\N}{\mathbb{N}}
\newcommand{\C}{\mathbb{C}}
\newcommand{\Zplus}{\mathbb{Z}^{+}}
\newcommand{\Primes}{\mathbb{P}}
\newcommand{\ball}[2]{B_{#1} \! \left(#2 \right)}
\newcommand{\Q}{\mathbb{Q}}
\newcommand{\R}{\mathbb{R}}
\newcommand{\Rplus}{\mathbb{R}^+}
\newcommand{\invI}[2]{#1^{-1} \left( #2 \right)}
\newcommand{\End}[1]{\text{End}\left( A \right)}
\newcommand{\legsym}[2]{\left(\frac{#1}{#2} \right)}
\renewcommand{\mod}[3]{\: #1 \equiv #2 \: \mathrm{mod} \: #3 \:}
\newcommand{\nmod}[3]{\: #1 \centernot \equiv #2 \: mod \: #3 \:}
\newcommand{\ndiv}{\hspace{-4pt}\not \divides \hspace{2pt}}
\newcommand{\finfield}[1]{\mathbb{F}_{#1}}
\newcommand{\finunits}[1]{\mathbb{F}_{#1}^{\times}}
\newcommand{\ord}[1]{\mathrm{ord}\! \left(#1 \right)}
\newcommand{\quadfield}[1]{\Q \small(\sqrt{#1} \small)}
\newcommand{\vspan}[1]{\mathrm{span}\! \left\{#1 \right\}}
\newcommand{\galgroup}[1]{Gal \small(#1 \small)}
\newcommand{\sm}{\! \setminus \!}
\newcommand{\topo}{\mathcal{T}}
\newcommand{\base}{\mathcal{B}}
\renewcommand{\bf}[1]{\mathbf{#1}}
\renewcommand{\Im}[1]{\mathrm{Im} \: #1}
\renewcommand{\empty}{\varnothing}



\newcommand{\proves}{\vdash}
\newcommand{\entails}{\models}
\newcommand{\Mmod}{\mathcal{M}}
\newcommand{\Nmod}{\mathcal{N}}
\newcommand{\ul}[1]{\underline{#1}}
\newcommand{\maxideal}{\mathfrak{m}}
\newcommand{\Th}[2]{\mathrm{Th}_{#1} \left( #2 \right)}

\newcommand{\VAR}[1]{\mathrm{VAR}_{#1}}
\newcommand{\TRM}[1]{\mathrm{TRM}_{#1}}
\newcommand{\FOR}[1]{\mathrm{FOR}_{#1}}
\newcommand{\SEN}[1]{\mathrm{FOR}_{#1}}


\newenvironment{definition}[1][Definition:]{\begin{trivlist}
\item[\hskip \labelsep {\bfseries #1}]}{\end{trivlist}}

\theoremstyle{theorem}
\newtheorem{theorem}{Theorem}[section]
\newtheorem{lemma}[theorem]{Lemma}
\newtheorem{corollary}[theorem]{Corollary}

\theoremstyle{definition}
\newtheorem*{problem}{Problem}

\theoremstyle{definition}
\newtheorem*{proposition}{Proposition}

\theoremstyle{remark}
\newtheorem*{fact}{Fact}

\theoremstyle{definition}
\newtheorem{example}{Example}[section]

\theoremstyle{remark}
\newtheorem{remark}{Remark}[subsection]

\begin{document}
\author{Benjamin Church}
\title{\Huge Mathematical Logic}

\maketitle
\tableofcontents
\newpage


\section{Introduction}

Mathematical logic is divided into two parts: semantics and syntactics. Semantics is the study to interpretations and models of a theory, think examples. Syntactics is the study of formal deduction systems and provability. We will study first-order or predicate quantifier logic which has an extremely powerful metatheory. Higher-order logics also exist but do not admit complete proof theory and many of the desirable metalogical properties of first-order logic fail to hold. A first-order logic has two pieces, a class of first-order formal languages which we will define inductivly and a deduction system defined by rules of inference.

\subsection{First-Order Languages}.

\begin{definition}
A \textit{vocabulary} or \textit{signature} $\sigma$ is a set of ``nonlogical'' symbols which may be of three types:
\begin{enumerate}
\item Constant symbols (e.g. $0$)
\item n-ary function symbols (e.g. $+$)
\item n-ary relation symbols (e.g. $\in$)
\end{enumerate}
Along with the signature, a first-order language has a set of ``logical'' symbols:
\begin{enumerate}
\item A countable list of variable symbols: $x_1, x_2, x_3, \cdots$
\item Logical connectives: $\neg, \vee, \wedge, \to$
\item Quantifiers: $\forall$ (we get $\exists \iff \neg \forall \neg$ for free)
\item An equality relation: $=$
\item Punctuation: $( \: ) \: ,$ etc.
\end{enumerate}
\end{definition}

\begin{definition}
The set of \textit{terms} of a first-order language $L$ with vocabulary $\sigma$ is defined inductively as follows:
\begin{enumerate}
\item Any variable or constant symbol is a term.
\item If $f$ is an n-ary function symbol and $t_1, \dots, t_n$ are terms then $f(t_1, \dots, t_n)$ is a term. For a binary operator (2-ary function), say $\circ$, we will often write $(t_1 \circ t_2)$ to mean $\circ(t_1, t_2)$. 
\end{enumerate} 
\end{definition}

\begin{definition}
The set of \textit{formulas} of a first-order language $L$ with vocabulary $\sigma$ is defined inductivly as follows:
\begin{enumerate}
\item If $s,t$ are terms then $(s = t)$ is a fomula. Furthermore if $R \in \sigma$ is an n-ary relation symbol and $t_1, \dots, t_n$ are terms then $R(t_1, \dots, t_n)$ is a formula. For a 2-ary relation we will often write $s R t$ to mean $R(s, t)$. 
\item If $A$ and $B$ are formulas then $\neg A$, $(A \vee B)$, $(A \wedge B)$, and $(A \to B)$ are all formulas.
\item If $x$ is a variable symbol and $\varphi$ a formula in which $x$ is free ($\varphi$ contains $x$ but no quantifiers over $x$) then $\forall x \: \varphi$ and $\exists x \: \varphi$ are formulas. 
\end{enumerate}
\end{definition}

\begin{definition}
A \textit{sentence} of a first-order language is a formula with no free variables.  
\end{definition}

\begin{definition}
A first-order theory is a first-order language $L$ along with a set $\Gamma$ of first-order $L$-sentances which are refered to as axioms. 
\end{definition}

\subsection{Proof Theory}

There are many possible first-order deduction systems each with its own unique flavor. A deduction system has logical axioms and rules of inference on formulas of $L$. A formal proof begining with some assumptions is a sequence of $L$-formulas each of which is either a logical axiom, an assumption, or the result of a rule of inference applied to previous formulas.  

\begin{definition}
We say that a first-order theory $\Gamma$ \textit{syntactically entails} or, more simply, \textit{proves} $A$ if there exists a formal proof using axioms of $\Gamma$ and first-order rules of inference. We write this as $\Gamma \vdash A$.
\end{definition}

\begin{definition}
A first-order theory $\Gamma$ is \textit{consistent} if there does not exist a statement $A$ such that $\Gamma \proves A$ and $\Gamma \proves \neg A$. 
\end{definition}

\begin{definition}
A first-order theory $\Gamma$ is \textit{complete} if for every $L$-sentence $A$ we have either $\Gamma \proves A$ or $\Gamma \proves \neg A$. 
\end{definition}

\begin{lemma}[Deduction]
Let $A$ be an $L$-sentence. If $\Gamma \cup \{ A \} \proves B$ then $\Gamma \proves A \to B$. 
\end{lemma}

\begin{proof}
This proof is a simple induction on theorems. We suppose it is true for all proofs of length $n$ or less and show that any proof of $B$ with length $n+1$ must contain a subproof of $B'$ with length at most $n$ such that a single rule of inference can take $B'$ to $B$. Using the induction hypothesis we get $\Gamma \proves A \to B'$. It is then a simple yet tedious excersize in first-order logic to arrive at $A \to B$ from $A \to B'$ and the rule of inference which takes $B'$ to $B$. 
\end{proof}

\begin{lemma}[Categorization of Consistency]
$\Gamma$ is proof-theoretically consistent if and only if there exists a first-order sentence $A$ such that $\Gamma \not\proves A$. 
\end{lemma}

\begin{proof}
If $\Gamma$ is consistent and $\Gamma \proves A$ then $\Gamma \not\proves \neg A$. If $\Gamma$ is not consistent then $\Gamma \proves A$ and $\Gamma \proves \neg A$ for some $A$. However, for any $B$, we have $\Gamma \proves A \vee B$ since $\Gamma \proves A$ and also $\Gamma \proves \neg A$ so $\Gamma \proves B$. Thus, if $\Gamma$ is inconsistent then it proves everything.  
\end{proof}

\begin{lemma}
$\Gamma \cup \{ \neg A \}$ is consistent $\iff \Gamma \not\proves A$.
\end{lemma}

\begin{proof}
Suppose $\Gamma \cup \{ \neg A \}$ is inconsistent. Then it proves anything including $A$. By the deduction lemma, $\Gamma \proves \neg A \to A$ and thus $\Gamma \proves A$.
\bigskip\\
Conversely,
if $\Gamma \proves A$ then $\Gamma \cup \{ \neg A \} \proves A$ and clearly $\Gamma \cup \{ \neg A \} \proves \neg A$ so it is inconsistent. 
\end{proof}

\subsection{Interpretations, Models, and Truth}

\begin{definition}
Let $L$ be a first-order language with signature $\sigma$. An $L$-interpretation or $\sigma$-structure $\Mmod = (M, I)$ is a nonempty set $M$ called the domain and an interpretation function $I$ satisfying,
\begin{enumerate}
\item For each constant symbol $c \in \sigma$, an element of the domain, $c^{\Mmod} = I(c) \in M$.
\item For each n-ary function symbol $f \in \sigma$, n-ary function $f^{\Mmod} = I(f) : M^n \to M$.
\item For each n-ary relation symbol $R \in \sigma$, an n-ary relation $R^{\Mmod} = I(R) \subset M^n$. Furthermore $=^{\Mmod}$ is the diagonal relation $\Delta_M$.
\end{enumerate}
\end{definition}

\begin{definition}
Let $\Mmod$ be an $L$-interpretation and given a map $\alpha : \VAR{L} \to M$ we extend $\alpha$ to an \textit{assignment} on terms inductivly,
\begin{enumerate}
\item If $t$ is a constant symbol $c$ then $\alpha(t) = c^{\Mmod}$
\item If $t = f(t_1, \dots, f_n)$ then $\alpha(t) = f^{\Mmod}(\alpha(t_1), \dots, \alpha(t_n))$.  
\end{enumerate} 
For a tuple $\ul{a} \in M$ we write $[\ul{x} := \ul{a}]$ for the assigment sending $x_i \mapsto a_i$.
\end{definition}

\begin{definition}
Let $\Mmod$ be an $L$-interpretation and $\alpha$ an assigment. Let $\varphi$ be an $L$-formula then we define \textit{satisfaction} of a formula by $\alpha$, denoted $\Mmod[\alpha] \entails \varphi$ inductively:
\begin{enumerate}
\item If $\varphi$ is $R(t_1, \dots, t_n)$ then $\Mmod[\alpha] \entails \varphi \iff (\alpha(t_1), \dots, \alpha(t_n)) \in R^{\Mmod}$.
\item If $\varphi$ is $(s = t)$ then $\Mmod[\alpha] \entails \varphi \iff \alpha(s) = \alpha(t) \iff (\alpha(s), \alpha(t)) \in \Delta_M$. 
\item If $\varphi$ is $\neg A$ then $\Mmod[\alpha] \entails \varphi \iff \Mmod[\alpha] \not\entails A$.
\item If $\varphi$ is $A \to B$ then $\Mmod[\alpha] \entails \varphi \iff \Mmod[\alpha] \entails A \text{ implies } \Mmod[\alpha] \entails B$.
\item If $\varphi$ is $A \vee B$ then $\Mmod[\alpha] \entails \varphi \iff \Mmod[\alpha] \entails A \text{ or } \Mmod[\alpha] \entails B$.
\item If $\varphi$ is $A \wedge B$ then $\Mmod[\alpha] \entails \varphi \iff \Mmod[\alpha] \entails A \text{ and } \Mmod[\alpha] \entails B$.
\item If $\varphi$ is $\forall x \: A$ then $\Mmod[\alpha] \entails \varphi \iff \Mmod[\alpha'] \entails A$ for every assigment $\alpha'$ such that $\alpha(z) = \alpha'(z)$ for all variables besides $x$.
\item If $\varphi$ is $\exists x \: A$ then $\Mmod[\alpha] \entails \varphi \iff \Mmod[\alpha'] \entails A$ for some assigment $\alpha'$ such that $\alpha(z) = \alpha'(z)$ for all variables besides $x$.
\end{enumerate}
An $L$-formula $\varphi$ is true in the interpretation $\Mmod$, written as $\Mmod \entails \varphi$, if $\Mmod[\alpha] \entails \varphi$ for each assignment $\alpha$. The satisfaction of a sentence $A$ does not depend on the choice of assignment and thus either $\Mmod[\alpha] \entails A$ for all $\alpha$ or for no $\alpha$. Therefore, either $\Mmod \entails A$ or $\Mmod \entails \neg A$. A formula can be neither true nor false in $\Mmod$ when it has free variables since different assigments may disagree on its satisfaction. However, sentences are always either true or false. If $\varphi$ is a formula with free variables amoung $\ul{x}$ then for a tuple $\ul{a} \in M$ we write $\Mmod \entails \varphi(\ul{a})$ to mean $\Mmod[\ul{x} := \ul{a}] \entails \varphi$. 
\end{definition}

\begin{definition}
Let $\Gamma$ be a first-order theory and $\Mmod$ an $L$ interpretation. If $\Mmod \entails \Gamma$ i.e. every axiom of $\Gamma$ is satisfied in $\Mmod$, then we say that $\Mmod$ is a \textit{model} of $\Gamma$.
\end{definition}

\begin{definition}
Let $\Mmod$ be an $L$-interpretation. Then $\Th{L}{\Mmod}$, the theory of $\Mmod$, is the set of true $L$-sentances of $\Mmod$.
\end{definition}

\begin{definition}
Two $L$-interpretations $\Mmod$ and $\Nmod$ are said to be elementary equivalent, written $\Mmod \equiv \Nmod$, if they have the same set of true $L$-sentences.  
\end{definition}

\begin{lemma}
If $\Th{L}{\Mmod} \subset \Th{L}{\Nmod}$ then $\Th{L}{\Mmod} = \Th{L}{\Nmod}$ and thus $\Mmod \equiv \Nmod$.
\end{lemma}

\begin{proof}
Let $A$ be an $L$-sentence. If $\Mmod \entails A$  then by hypothesis $\Nmod \entails A$. However, if $\Mmod \not\entails A$ then, because $A$ is a sentence, $\Mmod \entails \neg A$ so $\Nmod \entails \neg A$ and thus $\Nmod \not\entails A$.  
\end{proof}

\begin{definition}
Two $L$-interpretations $\Mmod$ and $\Nmod$ are said to be isomorphic, written $\Mmod \cong \Nmod$, if there is a bijection $\sigma : M \to N$ such that,
\begin{enumerate}
\item For each constant symbol $c$, we have $\sigma(c^{\Mmod}) = c^{\Nmod}$.
\item For each relation $R$ and tuple $\ul{a} \in M$ we have $\ul{a} \in R^{\Mmod} \iff \sigma(\ul{a}) \in R^{\Nmod}$.
\item For each function $f$ and tuple $\ul{a} \in M$ we have $\sigma(f^{\Mmod}(\ul{a})) = f^{\Nmod}(\sigma(\ul{a}))$. 
\end{enumerate} 
\end{definition}

\begin{proposition}
Let $\sigma : \Mmod \to \Nmod$ be an isomorphism of $L$-interpretations. Then we have
\[\Mmod \entails \varphi(\ul{a}) \iff \Nmod \entails \varphi(\sigma(\ul{a}))\]
for any $\ul{a} \in M$ and $L$-fomula $\varphi$ with parameters. 
\end{proposition}

\begin{proof}
Induction on the complexity of a formula. 
\end{proof}

\begin{corollary}
If $\Mmod \cong \Nmod$ then $\Mmod \equiv \Nmod$.
\end{corollary}

\begin{proof}
The above proposition applied to parameter-free formulas i.e. sentences says,
\[ \Mmod \entails A \iff \Nmod \entails A \]
which is exactly the content of $\Mmod \equiv \Nmod$. 
\end{proof}

\begin{definition}
A first-order theory $\Gamma$ \textit{semantically entails} or simply \textit{entails} an $L$-formula $A$, written $\Gamma \entails A$, if $A$ is true in every model of $\Gamma$ i.e. whenever $\Mmod \entails \Gamma$ then $\Mmod \entails A$.  
\end{definition}

\begin{definition}
A first-order theory is \textit{satisfiable} (or consistent) if it admits a model.  
\end{definition}

\begin{definition}
A first-order theory is \textit{model complete} if any two models are elementary equivalent.
\end{definition}


\section{The Completeness of First-Order Logic}

The crowning achievment of mathematical logic is to join syntactics and semantics into a unified theory. This was accomplished in one fell swoop by the greatest logician to ever live, Kurt G\"{o}del, in his celebrated ``completeness theorems'' for first-order logic. Care must be taken to not mistake the ``completeness theorems'' with G\"{o}del's most famous work, his ``incompletness theorems.'' The situation seems designed for confusion. Hopfully this distinction will clear things up. The \textit{incompletness} theorems deal with the technical notion of proof-theoretic and model-theoretic completeness we discussed earlier and show that various theories cannot be complete in this sense. On the other hand, the \textit{completness} theorems consider the informal notion of the completness of first-order logic as a whole in the sense that proof theory and model theory complete eachother. 

\begin{theorem}[Model Existence]
A first-order theory is satisfiable if and only if it is consistent in the proof-theoretic sense. Furthermore, any consistent $L$-theory has a model of cardinality at most $|L|$. 
\end{theorem}

\begin{proof}
The proof of this theorem is long and highly technical so we cannot cover it here. The proof first constructs a maximally consistent set (consistent and every sentence or its negation is included) of sentances containing the theory and then constructs a model in which these are exactly the true sentances.   
\end{proof}

\begin{theorem}[Adequacy]
If $\Gamma \entails A$ then $\Gamma \proves A$. That is, if $A$ is true in every model then there exists a formal proof of $A$.
\end{theorem}

\begin{proof}
Suppose that $\Gamma \not\proves A$ then we know that $\Gamma \cup \{\neg A\}$ is consistent. By the model existence theorem there exists a model of $\Gamma \cup \{ \neg A \}$. However, this is a model of $\Gamma$ in which $A$ is false. Thus $\Gamma \not\entails A$.  
\end{proof}

\begin{theorem}[Soundness]
If $\Gamma \proves A$ then $\Gamma \entails A$ i.e. $A$ is true in every model.
\end{theorem}

\begin{proof}
The proof is quite simple and uses induction on proofs. The only piece of input is to check that one application of a rule of inference preserves truth value for any truth assignment. This follows easily from the inductive definition of truth assigments. 
\end{proof}

\begin{theorem}
All models of $\Gamma$ are elementary equivalent if and only if $\Gamma$ is complete in the proof-theoretic sense. 
\end{theorem}

\begin{proof}
Suppose that $\Gamma$ is complete and let $\Mmod$ be a model of $\Gamma$.
For any first order sentence $A$, if $\Gamma \proves A$ then $\Mmod \entails A$. Furthermore if $\Gamma \not\proves A$ then by completeness $\Gamma \proves \neg A$ and thus $\Mmod \entails \neg A \iff \Mmod \not\entails A$. Thus, 
\[ \Mmod \entails A \iff \Gamma \proves A \]
Therefore, every model of $\Gamma$ has the same set of true first-order sentences and are thus all elementary equivalent. 
\bigskip\\
Conversely, suppose that all models of $\Gamma$ are elementary equivalent. For any model $\Mmod$ of $\Gamma$ and any sentence $A$ either $\Mmod \entails A$ or $\Mmod \entails \neg A$. Furthermore, for any other model $\Nmod$ we have $\Mmod \equiv \Nmod$ so either $A$ or $\neg A$ is true in every model of $\Gamma$. Therefore, by the adequacy theorem, either $\Gamma \proves A$ or $\Gamma \proves \neg A$ so $\Gamma$ is proof-theoretically complete. 
\end{proof}
A finaly elegant summary of these results is given by G\"{o}del's monumentous theorem:

\begin{theorem}[G\"{o}del]
For any first-order theory, $\Gamma \proves A \iff \Gamma \entails A$. 
\end{theorem}

\section{The Compactness of First-Order Logic}

Here we will dive into model-theory proper in which we want to study properties of the set of all possible models of a given theory. However, it is often the case that proof-theoretic methods will provide insight and clever proofs even for purely model-theoretic statements. 


\begin{theorem}[Compactness]
A first-order theory is satisfiable if and only if it is finitely satifiable. 
\end{theorem}

\begin{proof}
Suppose that $\Gamma$ is not satisfiable. By the model existence theorm, $\Gamma$ must be incomplete so there exist proofs $\Gamma \proves A$ and $\Gamma \proves \neg A$. However, every proof is finite so each can only use a finite set of axioms in $\Gamma$. Call the set of all axioms in $\Gamma$ used in either proof $\Delta \subset \Gamma$. Thus, $\Delta \proves A$ and $\Delta \proves \neg A$ so $\Delta$ cannot admit a model and is finite. Therefore, if every finite subtheory $\Delta \subset \Gamma$ has a model then $\Gamma$ must have a model.
\end{proof}

\begin{remark}
There exist purely model-theoretic proofs of the Compactness theorem but they require more sophisticated ideas such as ultra-products.
\end{remark}


\subsection{Ultra-Products}

\section{The L\"{o}wenheim–Skolem Theorem}

\begin{definition}
Consider the first-order statments of the form,
\[ \exists x \exists y \exists z : x \neq y \wedge y \neq z \wedge z \neq x \]
which express that there exist at least $n$ elements. Call $\Sigma$ the set of all such statments. Furthermore, for any set of constant symbols $C$, define,
\[ \Sigma_C = \{ \neg (a = b) \mid a, b \in C \text{ such that } a \neq b \} \]
the set of all sentances of the form $a \neq b$.
\end{definition}

\begin{theorem}
Let $\Gamma$ be a first order theory. If $\Gamma$ has arbitrarily large finite modeles then $\Gamma$ admits an infinite model.
\end{theorem}

\begin{proof}
Consider the first-order theory $\Gamma \cup \Sigma$. Since $\Gamma$ has arbitrarily large finite models we know that any finite subset $\Delta \subset \Gamma \cup \Sigma$ is satisfiable by a large enough finite model of $\Gamma$. By the Compactness theorem, there exists a model of $\Gamma \cup \Sigma$ which is a model of $\Gamma$ which cannot have any finite number of elements. 
\end{proof}

\begin{remark}
This theorem about deducing the sizes of possible models gives a small taste of the powerful L\"{o}wenheim–Skolem theorem yet to come. However, the finite version is very important for proving the consistency of a new theory. Furthermore, the technique of adding an infinite set of first-order theorem which together impose a strict condition but in isolation are easily satisfied and then applying the compactness theorem comes up over and over in mathematical logic. 
\end{remark}


\begin{theorem}[L\"{o}wenheim–Skolem]
Let $L$ be a first-order language and $\Mmod$ an infinite $L$-interpretation. Then for any cardnal $\kappa \ge |L|$ there exists an $L$-interpretation $\Nmod$ such that $\Mmod \equiv \Nmod$ and $|\Nmod| = \kappa$. We call such an $\Nmod$ an elementary substructure or elementary extension. 
\end{theorem} 

\begin{proof}
First suppose that $\kappa > \max{(|\Mmod|, |L|)} \ge \aleph_0$. Let $C$ be a set of constant symbols with $|C| = \kappa$ and construct the language $L^+$ generated by $L \cup C$. Now, consider the $L^+$-theory $\Gamma_{\Mmod} = \Th{L}{\Mmod} \cup \Sigma_C$. For any finite subtheory $\Delta \subset \Gamma_{\Mmod}$, I claim that $\Mmod \entails \Delta$. This is because $\Delta$ contains only true $L$-sentences of $\Mmod$ and a finite number of $\Sigma_C$ statements which can be interpreted by sending any distinct elements of $\Mmod$ to the finite number of $C$-constants which appear in $\Delta$. Thus, $\Gamma_{\Mmod}$ is finitely satisfiable so, by compactness, $\Gamma_{\Mmod}$ is satisfiable and thus consistent. By the model existence theorem there exists a model $\Mmod' \entails \Gamma_{\Mmod}$ such that $|\Mmod'| \le |L^+| = |C| = \kappa$ (since $\kappa > |L| \ge \aleph_0$). However, since $\Sigma_C \subset \Gamma_{\Mmod}$ we must have an injection $C \to M'$ because not two $C$-constants can be interpreted as equal (since $a \neq b$ is true in $\Gamma_{\Mmod}$ for all $a,b \in C$). Thus, $|\Mmod'| = |C| = \kappa$. We may view $\Mmod'$ as an $L$-interpretation since $L \subset L^+$. Furthermore, $\Th{L}{\Mmod} \subset \Gamma_{\Mmod}$ and $\Mmod' \entails \Gamma_{\Mmod}$ so $\Th{L}{\Mmod} \subset \Th{L}{\Mmod'}$ and thus $\Mmod \equiv \Mmod'$ as $L$-interpretations. 
\end{proof}

\begin{remark}
An immediate consequence of the preceeding theorems is that having arbitrarily large finite models implies having models of every infinite cardinality. This is very powerful and surprising. The L\"{o}wenheim–Skolem theorem and they compactness theorem show that first-order logic is insufficient to constrain the size of its models. We will make this notion percise within the framework of first-order properties.
\end{remark}

\section{Skolem's Paradox and Higher-Order Logic}

\subsection{First-Order Properties}

(FINITENESS)
(SIZE)
(REAL NUMBERS)

\subsection{Higher-Order Logic}

\subsection{Nonstandard Models of Set Theory}


\section{Categoricity}

\begin{definition}
A first-order theory is $\kappa$-categorical if all models of size $\kappa$ are isomorphic.
\end{definition}

\begin{theorem}[Vaught]
Supppose that $\Gamma$ is $\kappa$-categorical for some $\kappa \ge |L|$ and has only infinite models then $\Gamma$ is model complete.
\end{theorem}

\begin{proof}
Let $\Mmod \entails \Gamma$ and $\Nmod \entails \Gamma$ be models. By the L\"{o}wenhiem-Skolem theorem, there exists models $\Mmod'$ and $\Nmod'$ of cardinaility $\kappa$ such that $\Mmod \equiv \Mmod'$ and $\Nmod \equiv \Nmod'$. However, by $\kappa$-categoricity, we know that $\Mmod' \cong \Nmod'$ since $|\Mmod'| = |\Nmod'| = \kappa$. Therefore, $\Mmod \equiv \Mmod' \equiv \Nmod' \equiv \Nmod$ so $\Gamma$ is model complete.  
\end{proof}


\section{Applications To Algebraic Geometry}

\newcommand{\ACF}[1]{\mathrm{ACF}_{#1}}

\begin{definition}
The first-order language of fields, denoted $F$, is generated by the signature $\sigma_F = \{0,1,+,\cdot\}$ where $0$ and $1$ are constant symbols and $+$ and $\cdot$ are 2-ary function symbols. 
\end{definition}

\begin{definition}
The theory $\ACF{}$ is the first-order theory of algebraically closed fields is defined over $F$ and has axioms:
\begin{enumerate}
\item Field Axioms.
\item Algebraic Closure: for each positive integer $n$, the sentence,
\[ \forall a_0 \forall a_2 \cdots \forall a_n \exists x \: \left[ a_n \cdot x^n + \cdots + a_1 \cdot x + a_0 = 0 \right] \]
\end{enumerate}
where exponentiation by a fixed integer is shortand for repeated multiplication. 
\end{definition}

\begin{definition}
The theory $\ACF{p}$ is the first-order theory of algebraically closed fields of characteristic $p$ is defined over $F$ and has axioms $\ACF{p} = \ACF{} \cup C_p$ where $C_p$ limits the characteristic. Let $\sigma_k$ be the sentence,
\[ 1 + 1 + \cdots + 1 = 0 \]
with exactly $k$ ones. Then $C_p = \{ \neg \sigma_k \mid k < p \} \cup \{ \sigma_p \}$ and $C_0 = \{ \neg \sigma_p \mid k \in \Zplus \}$.  
\end{definition}

\begin{theorem}[Steinitz]
All algebraically closed fields of the same uncountable cardinality and characteristic are isomorphic. That is, $\ACF{p}$ is uncountably-categorical for any characteristic.
\end{theorem}

\begin{proof}
This theorem uses the axiom of choice and equivalences of transendence bases. It is not wildly difficult but is outisde the scope of this discussion.
\end{proof}


\begin{corollary}
$\ACF{p}$ is complete.
\end{corollary}

\begin{proof}
There are infinitely many irreduicble polynomials and each have distinct roots. Thus any algebraially closed field is infinite. Let $ \Mmod \entails \ACF{p}$ and $\Nmod \entails \ACF{p}$ be models. By the L\"{o}wenhiem-Skolem theorem, there exists a model $\Mmod'$ of cardinaility $|\Nmod|$ (since it is infinite) such that $\Mmod \equiv \Mmod'$. However, by categoricity, we know that $\Mmod' \cong \Nmod$ since they are models with the same cardinality. Therefore, $\Mmod \equiv \Mmod' \equiv \Nmod$ so $\ACF{p}$ is model complete.  
\end{proof}

\begin{theorem}[Lefschetz Principle]
The true first-order sentences about $\C$ and about $\overline{\Q}$ are the same. This is often stated as, the first-order theory of algebraic geometry over $\C$ is the same as over $\overline{\Q}$.
\end{theorem}

\begin{proof}
By the completeness of $\ACF{0}$ we have $\overline{\Q} \equiv \C$. 
\end{proof}

\begin{theorem}[Cross-Characteristic Transfer]
Let $A$ be a first order $F$-sentence. Then $\ACF{0} \entails A$ if and only if $\overline{\finfield{p}} \entails A$ for all but finitely many $p$.   
\end{theorem}

\begin{proof}
Consider any finite subtheory $\Delta \subset \ACF{0} \cup \{A\}$. Since $\Delta$ may contain only finitely many sentences constraining the characteristic and for large enough $p$ we know that $\overline{\finfield{p}} \entails A$, we can choose $p$ large enough such that $\overline{\finfield{p}}  \entails \Delta$. Therefore, $\Delta$ is satisfiable so $\ACF{0} \cup \{A\}$ is finitely-satisfiable. Thus, by the compactness of first-order logic, $\ACF{0} \cup \{A\}$ is satisfiable so there exists a model of $\ACF{0}$ in which $A$ is true. Thus, by the completeness of $\ACF{0}$ we know that $\ACF{0} \entails A$.   
\bigskip\\
Conversely suppose we can find arbitrarily large $p$ such that $\overline{\finfield{p}} \entails \neg A$. Again, take a finite subtheory $\Delta \subset \ACF{0} \cup \{\neg A\}$ and choose $p$ large enought that it satisfies the finite number of sentences in $\Delta$ constraining the characteristic in which $A$ is false. Thus, $\overline{\finfield{p}} \entails \neg A$ so $\ACF{0} \cup \{\neg A\}$ is finitely satisfiable and thus, by compactness, satisfiable. As before, there exists a model of $\ACF{0}$ in which $A$ is false so, by completeness, $\ACF{0} \entails \neg A$. 
\end{proof}

\begin{theorem}[Ax-Grothendiek]
If a polynomial map $f : \C^n \to \C^n$ is injective then it is surjective.
\end{theorem}

\begin{proof}
For a fixed natural number $d$ consider the first order sentence,
\begin{align*}
AG = \forall \ul{a}_0 \forall \ul{a}_1 \cdots \forall \ul{a}_d \bigg[ &  \forall \ul{x} \forall \ul{y} \Big[ \left( \ul{a}_d \cdot \ul{x}^d + \cdots + \ul{a}_1 \cdot \ul{x} + \ul{a}_0 = \ul{a}_d \cdot \ul{y}^d + \ul{a}_1 \cdot \ul{y} + \cdots + \ul{a}_0 \right) \to \ul{x} = \ul{y} \Big]
\\
&\to \Big[ \forall \ul{y} \exists \ul{x} : \ul{a}_d \cdot \ul{x}^d + \cdots + \ul{a}_1 \cdot \ul{x} + \ul{a}_0 = \ul{y}  \Big] \bigg]
\end{align*}
which expresses the Ax-Grothendiek theorem for degree $n$. Here exponentiation is short for multiplication written a fixed number of times and underlined variables represent $n$-tuples with operations and comparisions defined componentwise i.e. each comparision is a conjunction of $n$ comparisions on the components. Furthermore, suppose that a map $f : \overline{\finfield{p}}^{\,n} \to \overline{\finfield{p}}^{\,n}$ given by polynomials is injective. For each element $\ul{y} \in \overline{\finfield{p}}^{\,n}$ consider the field extension $k = \finfield{p}[\ul{a}_0, \dots, \ul{a}_d, \ul{y}]$ given by adjoing the coefficients of $f$ and coordinates of $y$. Since these are elements of $\overline{\finfield{p}}$ each is algebraic and thus $k$ is a finite extension of $\finfield{p}$ and thus a finite field. Furthermore $f$ restricts to a map $f : k^n \to k^n$ since it is given by polynomials with coefficients in $k$. However, this restriction is still injective and therefore surjective because $k^n$ is a finite set. Thus, $\exists \ul{x} \in k^n$ such that $f(\ul{x}) = \ul{y} \in k^n$ so $f$ is surjective. 
\bigskip\\
Since $\overline{\finfield{p}} \entails AG$ for every $p$ we know that $\ACF{0} \entails AG$ and thus the Ax-Grothendiek theorem is true for every algebracially closed field of characteristic zero inculuding $\C \entails AG$. 
\end{proof}

\begin{theorem}[Strong Ax-Grothendiek]
Let $V$ be an affine variety over an algebraically closed field and $f : V \to V$ a morphism. If $f$ is injective then it is surjective. 
\end{theorem}

\begin{theorem}
Every extension of algebracially closed fields is elementary.
\end{theorem}

\begin{proof}
This follows from quantifier elimination.
\end{proof}

\begin{theorem}[Hilbert's Nullstellensatz]
Let $I \subset \overline{K}[\ul{X}]$ be a proper ideal then the common vanishing set $V(I) = \{p \in \bar{K}^n \mid \forall f \in I : f(p) = 0 \}$ is nonempty. 
\end{theorem}

\begin{proof}
The ideal $I$ is contained in a maximal ideal. Consider the field $L = \overline{\overline{K}[\ul{X}] / \maxideal}$ and projecton map $\overline{K}[\ul{X}] \to L$. Since $I \subset \maxideal$ the image of $I$ is zero in $L$ and thus the image of $\ul{X}$ is a common solution for $I$. Since $K \subset L$ it is an elementary extension.
Since $\overline{K}[\ul{X}]$ is Noetherian, we can find generators $I = (f_1, \dots, f_n)$. Let $\ul{a} \in \overline{K}$ a tuple encoding the coefficients of the polynomials $f_1, \dots, f_n$ and $\psi(\ul{a})$ the first order formula encoding the idea that the polynomials with coefficients $\ul{a}$ have a common zero. Since the extension is elementary and $\ul{a} \in K$ we have,
\[ K \entails \psi(\ul{a}) \iff L \entails \psi(\ul{a}) \]
However, $\ul{X}$ is a solution in $L$ so $K \entails \psi(\ul{a})$. Thus any proper ideal has nonempty vanishing set. 
\end{proof}

\section{Incompletness and Decidability}

\section{Provability Logic}

\end{document}
