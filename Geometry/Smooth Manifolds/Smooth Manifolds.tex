\documentclass[12pt]{extarticle}
\usepackage[utf8]{inputenc}
\usepackage[english]{babel}
\usepackage[a4paper, total={6in, 9in}]{geometry}
\usepackage{tikz-cd}
 
\usepackage{amsthm, amssymb, amsmath, centernot}
\usepackage{mathrsfs} 

\newcommand{\notimplies}{%
  \mathrel{{\ooalign{\hidewidth$\not\phantom{=}$\hidewidth\cr$\implies$}}}}
 
\renewcommand\qedsymbol{$\square$}
\newcommand{\cont}{$\boxtimes$}
\newcommand{\divides}{\mid}
\newcommand{\ndivides}{\centernot \mid}
\newcommand{\Z}{\mathbb{Z}}
\newcommand{\R}{\mathbb{R}}
\newcommand{\N}{\mathbb{N}}
\newcommand{\Zplus}{\mathbb{Z}^{+}}
\newcommand{\Primes}{\mathbb{P}}
\newcommand{\colim}[1]{\mathrm{colim}(#1)}
\newcommand{\Ob}[1]{\mathrm{Ob}(#1)}
\newcommand{\cat}[1]{\mathcal{#1}}
\newcommand{\id}{\mathrm{id}}
\newcommand{\Hom}[2]{\mathrm{Hom}\left( #1, #2 \right)}
\newcommand{\catHom}[3]{\mathrm{Hom}_{#1}\left( #2, #3 \right)}
\newcommand{\End}[1]{\mathrm{End}\left(#1\right)}
\newcommand{\Top}{\mathbf{Top}}
\newcommand{\pTop}{\mathbf{Top}_{\bullet}}
\newcommand{\Set}{\mathbf{Set}}
\newcommand{\pSet}{\mathbf{Set}_\bullet}
\newcommand{\hTop}{\mathbf{hTop}}
\newcommand{\phTop}{\mathbf{hTop}_{\bullet}}
\renewcommand{\Im}[1]{\mathrm{Im}(#1)}
\newcommand{\homspace}[2]{\left< #1, #2 \right>}
\newcommand{\rp}{\mathbb{RP}}
\newcommand{\coker}[1]{\mathrm{coker}\: #1}

\renewcommand{\d}[1]{ \mathrm{d}#1 \:}
\newcommand{\dn}[2]{ \mathrm{d}^{#1} #2 \:}
\newcommand{\deriv}[2]{\frac{\d{#1}}{\d{#2}}}
\newcommand{\nderiv}[3]{\frac{\dn{#1}{#2}}{\d{#3^{#1}}}}
\newcommand{\pderiv}[2]{\frac{\partial{#1}}{\partial{#2}}}
\newcommand{\fderiv}[2]{\frac{\delta #1}{\delta #2}}

\theoremstyle{definition}
\newtheorem{theorem}{Theorem}[section]
\newtheorem{lemma}[theorem]{Lemma}
\newtheorem{proposition}[theorem]{Proposition}
\newtheorem{example}[theorem]{Example}
\newtheorem{corollary}[theorem]{Corollary}
\newtheorem{remark}{Remark}

\newenvironment{definition}[1][Definition:]{\begin{trivlist}
\item[\hskip \labelsep {\bfseries #1}]}{\end{trivlist}}


\newenvironment{lproof}{\begin{proof} \renewcommand{\qedsymbol}{}}{\end{proof}}
\renewcommand{\mod}[3]{\: #1 \equiv #2 \: mod \: #3 \:}
\newcommand{\nmod}[3]{\: #1 \centernot \equiv #2 \: mod \: #3 \:}
\newcommand{\ndiv}{\hspace{-4pt}\not \divides \hspace{2pt}}
\newcommand{\gen}[1]{\langle #1 \rangle}
\newcommand{\hook}{\hookrightarrow}
\newcommand{\Tor}[4]{\mathrm{Tor}^{#1}_{#2} \left( #3, #4 \right)}
\newcommand{\Ext}[4]{\mathrm{Ext}^{#1}_{#2} \left( #3, #4 \right)}

\tikzset{
    labl/.style={anchor=south, rotate=90, inner sep=.5mm}
}

\renewcommand{\bf}[1]{\mathbf{#1}}
\newcommand{\C}[1]{\mathcal{C}^{#1}}
\newcommand{\res}{\mathrm{res}}
\newcommand{\F}{\mathcal{F}}
\newcommand{\G}{\mathcal{G}}
\renewcommand{\O}{\mathcal{O}}
\newcommand{\m}{\mathfrak{m}}

\newcommand{\GL}[1]{\mathrm{GL}\left(#1\right)}
\newcommand{\SL}[1]{\mathrm{SL}\left(#1\right)}
\newcommand{\PGL}[1]{\mathrm{PGL}\left(#1\right)}
\newcommand{\PSL}[1]{\mathrm{PSL}\left(#1\right)}

\newcommand{\Orth}[1]{\mathrm{O}\left(#1\right)}
\newcommand{\U}[1]{\mathrm{U}\left(#1\right)}
\newcommand{\SO}[1]{\mathrm{SO}\left(#1\right)}
\newcommand{\SU}[1]{\mathrm{SU}\left(#1\right)}


\begin{document}


\section{Smooth Functions}

First we recall the definition of smoothness for domains in $\R^n$. Let $U \subset \R^n$ be open with $\bf{p} \in U$.

\begin{definition}
We say that $f : U \to \R^m$ is differentiable at $\bf{p} \in U$ if there is a linear map $f'_{\bf{p}} : \R^n \to \R^n$ such that,
\[ \lim_{\bf{h} \to 0} \frac{1}{|\bf{h}|} \left| f(\bf{p} + \bf{h}) - f(\bf{p}) - f'_{\bf{p}} (\bf{h})  \right| = 0 \] 
$f : U \to \R^m$ is differentiable if $f$ is differentiable at each $\bf{p} \in U$.
\end{definition}

\begin{definition}
Denote the vectorspace of continuous functions $U \to \R^m$ by $\C{0}$ and for $n > 0$ define, 
\[ \C{n} = \{ f : U \to \R^m \mid \forall p \in U : f'_p \text{ exists and } \forall \bf{v} \in \R^n : f'_\bf{v} \in \C{n-1} \} \] 
where $f'_\bf{v}$ is the map $\bf{p} \mapsto f'_{\bf{p}}(\bf{v})$. Furthermore, the space of smooth functions is,
\[ \C{\infty} = \bigcap_{k} \C{k} \] 
\end{definition}

\begin{proposition}
A function $f : U \to \R^m$ is $\C{k}$ if $f$ is differentiable $k$ times and the $k^{\mathrm{th}}$-derivative $f^{(k)} : U \to \R^m$ is continuous. Furthermore $f$ is called $\C{\infty}$ or smooth if it is $\C{k}$ for all $k \le 0$.
\end{proposition}

\begin{proposition}
If $f : U \to \R^m$ is differentiable then $f$ is continuous. 
\end{proposition}

\begin{proof}
Becuase $f'_p$ is linear it has limit zero as $\bf{h} \to 0$,
\[ \lim_{\bf{h} \to 0} | f(\bf{p} + \bf{h}) - f(\bf{p})| = \lim_{\bf{h} \to 0} \left| f(\bf{p} + \bf{h}) - f(\bf{p}) - f'_p(\bf{h})  \right| = 0 \]
and the second term has zero limit by differentiability.
\end{proof}

\begin{proposition}
Let $f : U \to V$ and $g : V \to \R^m$ be differentiable. Then the composition has derivative, 
\[ (g \circ f)'_{\bf{p}} = g'_{f(\bf{p})} \circ f'_{\bf{p}} \]
\end{proposition}

\begin{proof}

\end{proof}

\begin{proposition}
Let $f,g : U \to \R$ be differentiable functions on $U \subset \R^n$ then,
\[ (fg)'_{\bf{p}} = f(\bf{p}) g'_{\bf{p}} + g(\bf{p}) f'_{\bf{p}} \]
\end{proposition}

\begin{proof}
Consider,
\begin{align*}
(fg)(\bf{p} + \bf{h}) - (fg)(\bf{p}) & = f(\bf{p} + \bf{h}) g(\bf{p} + \bf{h}) - f(\bf{p}) g(\bf{p})
\\
& = f(\bf{p} + \bf{h}) [g(\bf{p} + \bf{h}) - g(\bf{p})] + [f(\bf{p} + \bf{h}) - f(\bf{p})] g(\bf{p}) 
\end{align*}
Therefore,
\begin{align*}
Q(\bf{h}) & =
(fg)(\bf{p} + \bf{h}) - (fg)(\bf{p}) - [f(\bf{p} + \bf{h}) g'_{\bf{p}}(\bf{h}) + g(\bf{p}) f'_{\bf{p}}(\bf{h}) ]
\\
& = f(\bf{p} + \bf{h}) [g(\bf{p} + \bf{h}) - g(\bf{p}) - g'_{\bf{p}}(\bf{h})] + [f(\bf{p} + \bf{h}) - f(\bf{p}) - f'_{\bf{p}}(\bf{h})] g(\bf{p}) 
\end{align*}
which implies that,
\[ \lim_{\bf{h} \to 0} \frac{1}{|\bf{h}|} |Q(\bf{h})| = 0 \]
since both,
\[ \lim_{\bf{h} \to 0} \frac{1}{|\bf{h}|} \left| f(\bf{p} + \bf{h}) - f(\bf{p}) - f'_{\bf{p}} (\bf{h})  \right| = 0 \] 
and 
\[ \lim_{\bf{h} \to 0} \frac{1}{|\bf{h}|} \left| g(\bf{p} + \bf{h}) - g(\bf{p}) - g'_{\bf{p}} (\bf{h})  \right| = 0 \] 
Finally,
\[ D(\bf{h}) = (fg)(\bf{p} + \bf{h}) - (fg)(\bf{p}) - [f(\bf{p}) g'_{\bf{p}}(\bf{h}) + g(\bf{p}) f'_{\bf{p}}(\bf{h}) ] = Q(\bf{h}) + [f(\bf{p} + \bf{h}) - f(\bf{p})] g'_{\bf{p}}(\bf{h}) \]
And thus,
\[ \frac{1}{|\bf{h}|} |D(\bf{h})| =  \frac{1}{|\bf{h}|} \left| Q(\bf{h}) + [f(\bf{p} + \bf{h}) - f(\bf{p})] g'_{\bf{p}}(\bf{h}) \right| \le \frac{1}{|\bf{h}|} |Q(\bf{h})| + \frac{|g'_{\bf{p}}(\bf{h})|}{|\bf{h}|} \left| f(\bf{p} + \bf{h}) - f(\bf{p}) \right| \]
Since $g'_{\bf{p}}$ is linear, by Lemma \ref{bounded_lin_op}, we can find some bound $M$ such that,
\[  \frac{|g'_{\bf{p}}(\bf{h})|}{|\bf{h}|} \le M \]
for all $\bf{h}$. Therefore,
\[ \frac{1}{|\bf{h}|} |D(\bf{h})|  \le \frac{1}{|\bf{h}|} |Q(\bf{h})| + M  \left| f(\bf{p} + \bf{h}) - f(\bf{p}) \right| \]
and both terms have limit zero. Thus,
\[ \lim_{bf{h} \to 0} \frac{1}{|\bf{h}|} |D(\bf{h})| = 0 \]
\end{proof}

\section{Manifolds}

\begin{definition}
A topolgoical space $M$ is an $n$-manifold if it is Hausdorff, second countable, and locally Euclidean. That is, there exists an open cover by charts $(U, \varphi)$ with homeomorphisms $\varphi : U \to V$ where $V \subset \R^n$ is open. Such an open cover by charts $(U, \varphi)$ is called an atlas.    
\end{definition}

\begin{definition}
An atlas is smooth if for each pairs of charts $(U_{\alpha}, \varphi_{\alpha})$ and $(U_{\beta}, \varphi_{\beta})$ the transition maps 
\[\varphi_{\beta} \circ \varphi_{\alpha}^{-1} : \varphi_{\alpha}(U_{\alpha} \cap U_{\beta}) \to \varphi_{\beta}(U_{\alpha} \cap U_{\beta}) \]
are $\C{\infty}$. Likewise, an atlas is analytic, rational, or holomorphic if the transition maps are.  
\end{definition}

\begin{definition}
A smooth atlas is maximal if whenever $(U, \varphi)$ is a chart compatible with the atlas then $(U, \varphi)$ is a member of the atlas. 
\end{definition}

\begin{proposition}
Every smooth atlas is contained in a unique maximal atlas.
\end{proposition}

\begin{definition}
A topological space $M$ is a smooth $n$-manifold if it is a manifold with a smooth structure. That is $M$ equiped with a maximal smooth atlas. 
\end{definition}

\begin{remark}
Any smooth atlas is contained in a unique maximal atlas and thus any smooth atlas on $M$ defines a unique smooth structure. 
\end{remark}

\begin{definition}
A map $F : M \to N$ between smooth manifolds is smooth if for each point $\bf{p} \in M$ there exists a chart $(U, \varphi)$ of $M$ and $(V, \psi)$ of $N$ such that $\bf{p} \in U$ and $F(U) \subset V$ and $\psi \circ F \circ \varphi^{-1} : \varphi(U) \to \psi(V)$ is smooth. 
\end{definition}

\section{Sheaves}

\begin{definition}
Let $X$ be a topological space and $\mathcal{C}$ a category. A pre-sheaf on $X$ is a contravariant functor $\F : X^{\mathrm{op}} \to \mathcal{C}$ where $X$ is a directed category on the open sets with inclusion maps. We call $\F(U)$ the sections restricted to $U$ and for $U \subset V$ the maps $\res : \F(V) \to \F(U)$ restriction maps denoted $s|_U = \res_{U, V}(s)$.    
\end{definition}

\begin{definition}
A sheaf on $X$ is a pre-sheaf such that for each open $U \subset X$ and open cover $\{ U_{\alpha} \}$ of $U$ the diagram,
\begin{center}
\begin{tikzcd}
\F(U) \arrow[r, "eq"] & \prod \F(U_{\alpha}) \arrow[r, shift left = 0.75ex] \arrow[r, shift right = 0.75ex] & \prod \F(U_{\alpha} \cap U_{\beta}) 
\end{tikzcd}
\end{center}
is an equalizer of the maps defined by the products $\res : U_{\alpha} \to U_{\alpha} \cap U_{\beta} $ and the products of the maps $\res : U_{\beta} \to U_{\alpha} \cap U_{\beta}$ respectively. This is equivalent to the following conditions: let $\{ U_{\alpha} \}$ be an open cover of $U \subset X$ then,
\begin{itemize}
\item If $s, t \in F(U)$ such that $s|_{U_{\alpha}} = t|_{U_{\alpha}}$ for each $U$ then $s = t$.

\item Suppose we have $s_{\alpha} \in \F(U_{\alpha})$ such that $s_{\alpha}|_{U_{\alpha} \cap U_{\beta}} = s_{\beta}|_{U_{\alpha} \cap U_{\beta}}$ then there exists a section $s \in F(U)$ such that $s|_{\alpha} = s_{\alpha}$ for each $\alpha$. 
\end{itemize} 
\end{definition}

\begin{definition}
Let $\F$ and $G$ be sheaves on $X$. Then a morphisms of sheaves $\varphi : \F \to \G$ is a natural transformation between the functors $\F$ and $\G$. That is a collection of maps $\varphi_U : \F(U) \to \G(U)$ such that,
\begin{center}
\begin{tikzcd}
\F(U) \arrow[d, "\varphi_U"] \arrow[r, "\res_{V,U}"] & \F(V) \arrow[d, "\varphi_V"]
\\
\G(U) \arrow[r, "\res_{V,U}"] & \G(V) 
\end{tikzcd}
\end{center} 
whenever $V \subset U$. 
\end{definition}

\begin{definition}
Let $f : X \to Y$ be a continuous map and $\F$ a sheaf on $X$. Then $f_* \F$ is a sheaf on $Y$ defined by $f_* \F(V) = \F(f^{-1}(V))$ for $V \subset Y$ with the restriction maps on the preimages. Furthermore, $f_*$ is a functor from the category of sheaves over $X$ to sheaves over $Y$ by sending a sheaf map $g^{\#} : \F \to \G$ to the sheaf map $f_* g^{\#} : f_* \F \to f_* \G$ given by $(f_* g^{\#})_U = g^{\#}_{f^{-1}(U)}$ such that the diagram commutes due to naturality of $g^{\#}$,
\begin{center}
\begin{tikzcd}[column sep = huge, row sep = huge]
\F(f^{-1}(U)) \arrow[d, "g^{\#}_{f^{-1}(U)}"] \arrow[r, "f_* \res_{V, U}"] & \F(f^{-1}(V)) \arrow[d, "g^{\#}_{f^{-1}(V)}"]
\\
\G(f^{-1}(U))  \arrow[r, "f_* \res_{V,U}"] & \G(f^{-1}(V))
\end{tikzcd}
\end{center}   
\end{definition}

\begin{definition}
Let $\F$ be a sheaf on $X$. For $p \in X$ the stalk at $p$ is given by,
\[ \F_p = \varinjlim_{p \in U} \F(U) \]
under the directed system given by restricting two neighborhoods to their intersection. 
\end{definition}

\begin{definition}
Let $(X, \O_X)$ and $(Y, \O_Y)$ be ringed spaces. Then a morphism of ringed spaces $f : (X, \O_X) \to (Y, \O_Y)$ is a continuous map $f : X \to Y$ and a morphism of sheaves $f^{\#} : \O_Y \to f_* \O_X$. 
\end{definition}

\begin{lemma}
If $f^{\#} : \F \to \G$ is a map of sheaves on $X$ then $f^{\#}$ induces a map on stalks $f^{\#} : \F_p \to \G_p$ for any $p \in X$. 
\end{lemma}

\begin{proof}
Since $f^{\#}$ is a map of sheaves, each inclusion of $f^{\#}_U$ into $\G_p$ is compatible with the restriction maps and thus lists to a map $f^{\#} : \F_p \to \G_p$. Furthermore, 
\end{proof}

\begin{definition}
Given a continous map $f : X \to Y$ and a sheaf $\F$ on $X$ there is a natural inclusion $(f_* \F)_{f(p)} \to \F_{p}$. 
\end{definition}

\begin{proof}
By definition, the inclusion maps $\iota : \F(f^{-1}(V)) \to \F_p$ for $f(p) \subset V$ are compatible with restrictions. Therefore, we get a map $(f_* \F)_{f(p)} \to \F_p$.  
\end{proof}

\begin{definition}
A space $(X, \O_X)$ is locally ringed if $\O_{X,p}$ is a local ring for each $p \in X$. A morphism of locally ringed spaces is a morphism of ringed spaces such that the induced map $f^{\#} : \O_{Y,f(p)} \to (f_* \O_X)_{f(p)} \to \O_{X, p}$ is a local map. That is, considering the unique maximal ideals $\mathfrak{m}_{Y, f(p)} \subset \O_{Y,f(p)}$ and  $\mathfrak{m}_{X, p} \subset \O_{X, p}$ then $f^{\#}(\mathfrak{m}_{Y, f(p)}) \subset \mathfrak{m}_{X, p}$.   
\end{definition}

\subsection{The Sheaf of Smooth Functions}

\begin{definition}
Let $M$ be a smooth manifold and let $\C{k}_M$ be the sheaf of $\C{k}$ functions on $M$. Define $\O_M = \C{\infty}_M$ to be the sheaf of smooth functions on $M$.
\end{definition}


Now we can redefine the basics of smooth manifolds in terms of structure sheaves.

\begin{definition}

\end{definition}

\begin{definition}
A smooth manifold is a locally ringed second countable Hausdroff space $(M, \O_M)$ with a covering by open sets $(U, \O_U)$ which are isomorphic as ringed spaces to $(V, \O_M)$ for some open $V \subset \R^n$ where $\O_V$ is the sheaf of smooth functions on $V$.  
\end{definition}

\begin{definition}
Let $(M, \O_M)$ and $(N, \O_N)$ be smooth manifolds. A smooth map from $M$ to $N$ is a morphism of locally ringed spaces $(F, F^{\#}) : (M, \O_M) \to (N, \O_N)$. 
\end{definition}

\begin{theorem}
These definitions coincide with the classical definitions. 
\end{theorem}

\section{The Tangent Space}

\begin{definition}
Let $(M, \O)$ be a smooth-manifold. The cotangent space at $p\in M$ is the quotient $T_p^* M = \m_p / \m_p^2$ where $\m_p$ is the maximal ideal of the stalk $\O_p$ given by germs of functions vanishing at $p$. The tangent space is the dual $T_p M =(T_p^* M)^*$.   
\end{definition}

\begin{remark}
Since $T_p M$ is defined from the stalk of $\O$ then $T_p U$ is identical to $T_p M$ for any open $U$ containing $p$. 
\end{remark}

\begin{proposition}
Given the manifold $(\R^n, \C{\infty}_{\R^n})$ we have $T_{\bf{p}}^* \R^n \cong (\R^n)^*$. 
\end{proposition}

\begin{proof}
Let $f \in \C{\infty}_{\R^n}$ be smooth. Then we have,
\[ f(\bf{x}) = f(\bf{p}) + f'_{\bf{p}}(\bf{x} - \bf{p}) + E(\bf{x}) \]
where $E(\bf{p}) = E'_{\bf{p}} = 0$. Then we have,
\[ \m_{\bf{p}} = \{ [f] \mid f \in \C{\infty}_U \text{ such that } f(\bf{p}) = 0 \} \]
If $[g] = [h_1 h_2]$ so on some neighborhood $U$ we have $g = a b$ with $a(\bf{p}) = b(\bf{p}) = 0$. Thus,
\[ g'_{\bf{p}}(\bf{x}) = a(\bf{p}) \cdot b'_{\bf{p}}(\bf{x}) + a'_{\bf{p}}(\bf{x}) \cdot b(\bf{p}) = 0 \]
so if $[g] \in \m_{\bf{p}}^2$ then $g'_{\bf{p}} = 0$.
Furthermore, if $E(\bf{p}) = 0$ and $E'_{\bf{p}} = 0$ then I claim that $[E] \in \m_{\bf{p}}^2$. Consider the smooth functions,
\[ 
f_i = \frac{x_i - p_i}{(\bf{x} - \bf{p})^2} E \]
These are smooth because 
\[ \lim_{|\bf{h}| \to 0} |f_i(\bf{p} + \bf{h})| = \lim_{|\bf{h}| \to 0} \left| \frac{E(\bf{p} + \bf{h})}{|\bf{h}|} \right| \cdot \frac{|h_i|}{|\bf{h}|} = 0 \]
which is zero because $E$ has zero derivative and value at $\bf{p}$. Let $g_i(\bf{x}) = (x_i - p_i)$ then 
\[ E = \sum_{i = 1}^n f_i(\bf{x}) g_i(\bf{x}) \]
where $f_i(\bf{p}) = g_i(\bf{p}) = 0$ and thus $E \in \m_{\bf{p}}^2$. Define the map, $\Phi : \m_{\bf{p}} \to (\R^n)^*$ by $\Phi([f]) = f_{\bf{p}}'$ which is a linear functional. We have shown that $\ker{\Phi} = \m_{\bf{p}}^2$. Furthermore, $\Phi$ is surjective because $[g_i] \mapsto \hat{e}_i$. By the first isomorphism theorem,
\[ T_{\bf{p}}^* \R^n  = \m_{\bf{p}} / \m_{\bf{p}}^2 \cong (\R^n)^* \]
\end{proof}

\begin{definition}
A smooth $f : M \to N$ defines a linear map $f^*_p : T_{f(p)}^* N \to T_{p}^* M$ given by sending $[g] \in \m^N_{f(p)}$ to $[g \circ f] \in \m^M_p$. The dual map defines the differential, 
\[ \d{f}_p = (f_*)_p = (f^*_p)^* : T_p M \to T_{f(p)} M \]  
\end{definition}

\begin{remark}
The map $f^*_p : T_{f(p)}^* N \to T_{p}^* M$ is well-defined by Lemma \ref{ring_map_quotient} because the induced map $f^*_p : \m_{f(p)}^N \to \m_p^M$ is an algebra homomorphism. 
\end{remark}

\begin{proposition}
The tangent $T_p$ is a covariant and the cotangent $T_p^*$ is a contravariant functor from the category of smooth manifolds to the category of $\R$-vectorspaces.  
\end{proposition}

\begin{proof}
For $f : M \to N$ and $g : N \to R$ smooth maps then, 
\[ (g \circ f)^*_p([h]) = [h \circ g \circ f] = f_p^* [h \circ g] = f_p^* (g_{f(p)}^*([h])) \]
and $\id_p^*([h]) = [h \circ \id] = [h]$. Thus, taking the dual map of vectorspaces,
\[ \d{(g \circ f)}_p = ((g \circ f)^*_p)^* = (f^*_p \circ g^*_{f(p)})^* = \d{g}_{f(p)} \circ \d{f}_p \] 
\end{proof}

\begin{corollary}
Let $M$ be a smooth $n$-manifold. Then $T_p M \cong \R^n$ at each $p \in M$.
\end{corollary}

\begin{proof}
For $p \in M$ there is some chart $\varphi : U \xrightarrow{\sim} V$ for open $V \subset \R^n$. with $p \in U$. Thus, since $\varphi$ has a smooth inverse, the map $\d{\varphi}_p : T_p M \xrightarrow{\sim} T_{\varphi(p)} \R^n \cong \R^n$ is an isomorphism.    
\end{proof}

\begin{definition}
A linear functional $X : \O_p \to \R$ is called a derivation if is satisfies the Leibniz rule, \[ X(fg) = X(f) g(p) + f(p) X(g) \] 
\end{definition}

\begin{proposition}
$T_p M$ is canonically isomorphic to the space of derivations at $p$. 
\end{proposition}

\begin{proof}
Take $X \in T_p M = (T_p^* M)^*$ then $X : \m_p / \m_p^2 \to \R$. Given any germ $f \in \O_p$ we can extend $X$ to act on $f$ by $X(f) = X(\tilde{f})$ where $\tilde{f} = f - f(p) \in \m_p / \m_p^2$. Then take $f, g \in \O_p$ and consider,
\begin{align*}
X(fg) & = X(fg - f(p) g(p)) = X(\tilde{f} \tilde{g} + \tilde{f} g(p) + f(p) \tilde{g})
\\
& = X(\tilde{f} \tilde{g}) + X(\tilde{f}) g(p) + f(p) X(\tilde{g}) = X(f) g(p) + f(p) X(g) 
\end{align*}
because $\tilde{f} \tilde{g} \in \m_p^2$ so $X(\tilde{f} \tilde{g}) = 0$. Thus, $X$ is a derivation at $p$. Furthermore, any derivation is automatically a linear map $\m_p \to \R$ so we only need to show that it descends to the quotient. Take $f, g \in \m_p$ then we have,
\[ X(fg) = X(f) g(p) + f(p) X(g) = 0 \] 
because $f(p) = g(p) = 0$. Thus, $X$ is zero on $\m_p^2$ so it factors through the quotient as a linear map $\tilde{X} : \m_p / \m_p^2 \to \R$ which an element of the dual space $(\m_p / \m_p^2)^* = T_p M$. Therefore, $T_p M$ is canonically identified with the space of derivations. 
\end{proof}

\subsection{The Tangent Bundle}

\subsection{Product Manifolds}

\begin{proposition}
Let $M$ and $N$ be smooth manifolds of dimensions $m$ and $n$ then $M \times N$ natually has a smooth stucture making it a smooth $m + n$-manifold such that the projection maps are smooth. 
\end{proposition}

\begin{proof}

\end{proof}

\begin{proposition}
For $(p,q) \in M \times N$ we have, $T_{p,q} (M \times N) \cong T_p M \oplus T_q N$.
\end{proposition}

\begin{proof}
Consider the map $\Phi : T_p^* M \oplus T_q^* N \to T_{p,q}^* (M \times N)$ defined by $\Phi = \pi_1^* + \pi_2^*$ acting on $f \in \m_p$ and $g \in \m_q$ via $[f] \oplus [g] \mapsto [f \circ \pi_1 + g \circ \pi_2 ]$.
Suppose that $\Phi([f] \oplus [g]) = 0$ then let $F = f \circ \pi_1 + g \circ \pi_2 \in \m_{p,q}$ on some neighborhood of $p,q$. Thus we can write,
\[ F = \sum_{i = 1}^n a_i b_i \]
with $a_i, b_i \in \m_{p,q}$. Then, 
\[ F(x, q) = \sum_{i = 1}^n a_i(x, q) b_i(x, q) \]
and $a_i(-,q), b_i(-,q) \in \m_p$ thus $F(-,q) \in \m_p^2$.
However, $F(x,p) = f(x) + g(q) = f(x)$ so $f \in \m_p^2$. Similarly, $g \in \m_q$. Thus $\Phi$ is injective. Clearly, $\Phi$ is linear. Furthermore, 
\[ \dim{(T_p^* M \oplus T_q^* N)} = m + n = \dim{T_{p,q}^* (M \times N)} \]
Thus, $\Phi$ must be surjective by rank-nullty. Thus, $\Phi$ is an isomorphism. Taking the dual map we get an isomorphism,
\[ \Phi^* : T_{p,q} (M \times N) \xrightarrow{\sim} T_p M \oplus T_q N \]
given explictly by $\Phi^* = (\pi_1)_* \oplus (\pi_2)_*$ since, 
\begin{align*}
\Phi^*(X)([f]\oplus[g]) & = X(\Phi([f] \oplus [g])) = X([f \circ \pi_1 + g \circ \pi_2]) 
\\
& = X([f \circ \pi_1]) + X([g \circ \pi_2]) = (\pi_1)_* X([f]) + (\pi_2)_* X([g])
\\
& = \Big( (\pi_1)_* X \oplus (\pi_2)_* X \Big) ([f] \oplus [g])
\end{align*}
\end{proof}

\begin{proposition}
Let $f : P \to M \times N$ a smooth map of smooth manifolds then 
\[ \d{f}_p = \d(\pi_M \circ f)_p \oplus \d{(\pi_N \circ f)}_p : T_p P \to T_{\pi_1(f(p))} M \oplus T_{\pi_2(f(p))} N \]
\end{proposition}

\begin{proof}
Let $X \in T_p P$ be a derivation (or linear functional on $T_p^* P = \m_p / \m_p^2$). Then consider,
\begin{align*}
\Phi^* \circ \d{f}_p(X) = (\pi_1)_* \circ \d{f}_p X \oplus (\pi_2)_* \circ \d{f}_p X = \Big( \d(\pi_M \circ f)_p \oplus \d{(\pi_N \circ f)}_p \Big)(X)
\end{align*}
\end{proof}

\begin{proposition}
Let $f : M \times N \to P$ a smooth map of smooth manifolds then 
\[ \d{f}_{p,q} = \d(f \circ \iota_M^q)_p + \d{(f \circ \iota_N^p)}_q \]
where $\iota_M^q : M \to M \times N$ is the inclusion $x \mapsto (x, q)$. 
\end{proposition}

\begin{proof}
Let $X_1 \oplus X_2 \in T_{p} M \oplus T_{q} N$ be derivations corresponding to $X \in T_{p,q}(M \times N)$ 

I claim that $(\Phi^*)^{-1}(X_1 \oplus X_1) = (\iota_M^q)_* X + (\iota_N^p)_* X$. It is easy to check that,
\begin{align*}
\Phi^* \circ (\Phi^*)^{-1}(X_1 \oplus X_1) & = \Phi^*((\iota_M^q)_* X_1 + (\iota_N^p)_* X_2) = (\pi_1)_* (\iota_M^q)_* X_1 \oplus (\pi_2)_* (\iota_N^p)_* X_2
\\
& = (\pi_1 \circ \iota_M^q)_* X \oplus (\pi_2 \circ \iota_N^p)_* X
\end{align*}
where the cross terms vanish because $\pi_1 \circ \iota_N^p$ is a constant map which has zero differential. Furthermore, $\pi_1 \circ \iota_M^q = \id_M$ so $(\pi_1 \circ \iota_M^q)_* = \id_{T_p M}$. Thus,
\[ \Phi^* \circ (\Phi^*)^{-1}(X_1 \oplus X_1) = (\pi_1)_* (\iota_M^q)_* X_1 \oplus (\pi_2)_* (\iota_N^p)_* X_2 = X_1 \oplus X_2 \]
Since $\Phi^*$ is invertible, this must be its two-sided inverse. Now consider,
\begin{align*}
\d{f}_{p,q} \circ (\Phi^*)^{-1}(X_1 \oplus X_2) & = \d{f}_{p,q}((\iota_M^q)_* X_1 + (\iota_N^p)_* X_2) 
\\
& = \big( \d{f}_{p,q} \circ (\iota_M^q)_* \big) X_1 + \big( \d{f}_{p,q} \circ (\iota_N^p)_* \big) X_2 
\\
& = \d{(f \circ \iota_M^q)} X_1 + \d{(f \circ \iota_N^p)} X_2
\end{align*}
\end{proof}


\subsection{Vector Fields}

\subsection{Tensors and Tensor Fields}

\section{Oct. 2}

\newcommand{\Supp}[1]{\mathrm{Supp}\left( #1 \right)}
\renewcommand{\L}{\mathcal{L}}

\begin{lemma}
Let $X$ be a smooth vector field on a smooth manifold $M$ such that $\Supp{X} = \overline{\{ p \in M \mid X_p \neq 0 \}}$ is compact. Then there exists a smooth $\phi : \R \times M \to M$ such that $\d{\phi_p}(\deriv{}{t}) = X(\phi(t, p))$ and $\phi(0, p) = p$. 
\end{lemma}

\begin{proof}
Let $K = \Supp{X}$ which is compact in $M$. For $p \notin K$ then $X(p) = 0$ so $\phi(t, p) = \phi(p)$. For $p \in K$ let $U_p$ be an open neighborhood of $p$ then a function satisfying the properties $\phi : (-\epsilon_p, \epsilon_p) \times U_p \to M$ is defined. By the compactess of $K$ we need only a finite number of $U_p$ to cover $K$ so we can take the minimum of the $\epsilon_p$. Then $\phi$ is defined on $K$. 
\end{proof}

\begin{definition}
Take $X \in \C{\infty}(M, TM)$. The Lie derivative $\L_X : \C{\infty}{M} \to \C{\infty}{M}$ given by $\L_X(f)(p) = X_p(f)$ is $\R$-linear and satisfies the product rule. 
\end{definition}

\begin{definition}
For $X,Y \in \C{\infty}(M, TM)$ then the Lie derivative on vector fields is a map $\L_X : \C{\infty}(M, TM) \to \C{\infty}(M, TM)$ given by $\L_X Y = [X, Y]$ and is $\R$-linear and satisfies the product rule,
\[ \L_X(fY) = (\L_X f) Y + f \L_X Y \]
\end{definition}

\begin{definition}
Let $F ; M \to N$ be a diffeomorphism and $X \in \C{\infty}(M, TM)$ define $F_* X$ by $(F_* X)_q =  \d{F}_{F^{-1}(q)}  \left( X_{F^{-1}(q)} \right)$ and for $Y \in \C{\infty}(N, TN)$ then $F^* Y = (F_{-1})_* Y$. 
\end{definition}

\begin{proposition}
Let $X \in \C{\infty}(M, TM)$ and $\phi$ the local flow of $X$ then,
\begin{enumerate}
\item For $f \in \C{\infty}(M)$ we have,
\[ X_p(f) = \deriv{}{t} \bigg|_{t = 0} \left( \phi_t^* f \right)(p) = \lim_{t \to 0} \frac{f \circ \phi_t(p) - f(p)}{t} \]
\item For $Y \in \C{\infty}(M, TM)$ we have,
\[ [X, Y]_p = - \deriv{}{t} \bigg|_{t = 0} \left( [\phi_t]_* Y \right)_p = \lim_{t \to 0} \frac{Y_p - ([\phi_t]_* Y)_p}{t} \]
\end{enumerate}
\end{proposition}

\begin{proof}
Let,
\[ T = \deriv{}{t} \bigg|_{t = 0}\]
be the standard derivation at zero on $\R$. Then $T(f \circ \phi_t(p)) = \d{\phi_p}(T)(f) = X_p(f)$. 
\end{proof}

\section{Vector Bundles}

\begin{definition}
$\pi_E : E \to M$ is a $\C{\infty}$-vector bundle over $M$ of rank $r$ if there exists an open cover $\{ U_\alpha \}$ of $M$ such that $\pi_E$ is locally trivialized via,
\begin{center}
\begin{tikzcd}[column sep = huge, row sep = huge]
\pi_E^{-1}(U_\alpha) \arrow[d, "\pi_E"'] \arrow[r, "\sim"] & U_{\alpha} \times \R^r \arrow[ld, "\pi_1"] 
\\
U_{\alpha}
\end{tikzcd}
\end{center}
for each $U_\alpha$.
\end{definition}


\begin{definition}
The $(r,s)$-tensor bundle on $M$ is the bundle,
\[ T^r_s M = (T M)^{\otimes r} \otimes (T^* M)^{\otimes s} \]
Furthermore, the bundle of $k$-forms is given by the $k^{\mathrm{th}}$ exterior power,
\[  \bigwedge^k T^* M \subset (T^*M)^{\otimes k} \]
Thus every $k$-form is a $(0,k)$-tensor.  
\end{definition}

\section{Differential Forms}

\begin{definition}
The space of differential $k$-forms on $M$ is,
\[ \Omega^k(M) = \mathcal{C}^{\infty}\left(M, \bigwedge^k T^* M\right) \]
which are smooth sections of the bundle of $k$-forms. 
\end{definition}

\begin{definition}
The exterior derivative is an $\R$-linear map,
\[ \d{} : \Omega^s(M) \to \Omega^{s+1}(M) \]
satisfying,
\begin{enumerate}
\item For $f \in \Omega^0(M) = \C{\infty}(M)$ the $1$-form $\d{f}$ is the differential.
\item For $f \in \Omega^0(M)$ we have $\d{\d{f}} = 0 $.
\item For $\alpha \in \Omega^r(M)$ and $\beta \in \Omega^s(M)$ then $\d{(\alpha \wedge \beta)} = \d{\alpha} \wedge \beta + \alpha \wedge \d{\beta}$. 
\item We have $\d{} \circ \d{} = 0$ i.e. $\forall \omega \in \Omega^{s}(M)$ we have $\d{\d{\omega}} = 0$.
\item Let $\phi : M \to N$ be smooth and $\omega \in \Omega^s(N)$ then $\phi^* \d{\omega} = \d{(\phi^* \omega)}$ i.e. $\d{} \circ \phi^* = \phi^* \circ \d{}$.
\item Let $X$ be a smooth vector field then $\d{} \circ \L_X = \L_X \circ \d{}$.
\item For $\alpha \in \Omega^s(M)$ and $X_0, \dots, X_s$ are smooth vector fields then,
\[ \d{\alpha}(X) = \sum_{i = 0}^s (-1)^i X_i \left( \alpha(X|_{\text{not } i}) \right) + \sum_{1 \le i < j \le s} (-1)^{i + j} \alpha([X_i, X_j], X_0, \dots, \hat{X}_i, \dots, \hat{X}_j, \dots, X_s) \] 
\end{enumerate}
\end{definition}

\begin{definition}
The space of all differential forms is,
\[ \Omega^{\ast}(M) = \bigoplus_{k = 1}^\infty \Omega^k(M) \]
\end{definition}

\begin{definition}
Consider the cochain complex,
\begin{center}
\begin{tikzcd}
0 \arrow[r] & \Omega^0(M) \arrow[r, "\d{}^0"] & \Omega^1(M) \arrow[r, "\d{}^1"] & \Omega^2(M) \arrow[r, "\d{}^2"] & \Omega^3(M) \arrow[r, "\d{}^3"] & \Omega^4(M) \arrow[r] & \cdots 
\end{tikzcd}
\end{center}
The de Rham cohomology is the cohomology of this complex,
\[ H_{\text{dR}}^k(M, \R) = \ker{d^k}/\Im{d^{k-1}} \]
\end{definition}

\begin{definition}
The interior derivative, for a smooth vector field $X \in \C{\infty}(M, TM) = \mathscr{X}(M)$, is an $\R$-linear map on forms,
\[ \iota_X : \Omega^s(M) \to \Omega^{s-1}(M) \]
defined by,
\[ \iota_X(\alpha)(Y_1, \dots, Y_{s-1}) = \alpha(X, Y_1, \dots, Y_{s-1}) \]
for any smooth vector fields $Y_1, \dots, Y_{s-1} \in \mathscr{X}(M)$. 
\end{definition}

\section{Remannian Manifolds}

\begin{definition}
A Remannian metric $g$ on a smooth manifold $M$ is a smooth $(0,2)$-tensor such that $g_p : T_p M \times T_p M \to \R$ is an inner product on $T_p M$. 
\end{definition}

\begin{definition}
A Remannian manifold $(M, g)$ is a smooth manifold $M$ with a Remannian metric $g$ on $M$. 
\end{definition}

\subsection{Isometric Immersions}

Let $f : M \to N$ be a smooth map and $(N, g)$ a Riemannian manifold. Then $f^*g$ is a symmetric $(0,2)$-tensor on $M$. When is $f^* g$ a Riemannian manifold on $M$? We have that,
\[ (f^* g)_p(v, v) = g(\d{f}_p(v), \d{f}_p(v)) \]
Thus we must have that $\d{f}_p(v) = 0$ implies $v = 0$ in order that $f^* g$ be nondegenerate.

\begin{proposition}
If $f : M \to N$ is a smooth immersion and $g$ is a Riemannian metric on $N$ then $f^* g$ is a Riemannian metric on $M$.
\end{proposition}

\begin{proof}
We know that $f^* g$ is a symmetric $(0,2)$-tensor. Furthermore, we know that,
\[ (f^* g)_p(v,v) = g(\d{f}_p(v), \d{f}_p(v)) = 0 \implies \d{f}_p(v) = 0 \]
but $f$ is an immersion so $\d{f}_p(v) = 0$ implies that $v = 0$. Thus, $f^* g$ is nondegenerate. Then, since $g$ is positive-defininite, so is $f^* g$. 
\end{proof}

\begin{definition}
A map $f : (M, g_M) \to (N, g_N)$ is,
\begin{enumerate}
\item an isometry if $f$ is a diffeomorphism and $f^* g_N = g_M$. 
\item an isometric immersion if $f$ is a smooth immersion and $f^* g_N = g_M$.
\item an isometric embedding if $f$ is a smooth embedding and $f^* g_N = g_M$.
\item  a local isometry if $f$ is a local diffeomorphism and $f^* g_N = g_M$. 
\end{enumerate}
\end{definition}

\begin{proposition}
Let $(M_1, g_1)$ and $(M_2, g_2)$ be Riemannian manifolds, then their product $(M_1 \times M_2, g)$ is canonically a Riemannian manifold with $g = \pi_1^* g_1 + \pi_2^* g_2$. 
\end{proposition}

\begin{proof}
We must check that $g$ is non-degenerate. Suppose,
\[ g((v,u),(v,u)) = g_1(v) + g_2(u) = 0\]
then since $g_1$ and $g_2$ are positive-definite we must have $v = u = 0$. 
\end{proof}

\subsection{Distance}

\newcommand{\len}[1]{\mathrm{len}\left( #1 \right)}

\begin{definition}
Let $(M, g)$ be a connected Remannian manifold. For $x,y \in M$ we have $d(x,y) = \inf{\len{\gamma}}$ for all picewise smooth paths $\gamma : I \to M$ from $x$ to $y$. 
\end{definition}

\begin{proposition}
For all $x,y,z \in M$ we have,
\[ d(x,z) + d(z, y) \ge d(x, y) \]
\end{proposition}

\begin{proof}
Let $\gamma_1, \gamma_2 : I \to M$ be picewise smooth paths from $x$ to $z$ and $z$ to $y$ respectively. Then $\gamma_2 * \gamma_1$ is a piecewise smooth path from $x$ to $y$ and the lengths add.
\end{proof}

\subsection{Volume Forms}

\begin{definition}
Let $M$ be a smooth $n$-manifold then a \textit{volume form} on $M$ is a nonvanishing smooth $n$-form $\omega \in \Omega^n(M)$.  
\end{definition}

\begin{definition}
An orientation on a smooth manifold $M$ is an atlas on $M$ such that each transition map has positive jacobian i.e. its differential has positive determinant. 
\end{definition}

\begin{lemma}
Let $(M, g)$ be an oriented Riemannian $n$-manifold then there exists a unique volume form $\omega \in \Omega^n(M)$ such that at each point $p \in M$ there exists an ordered basis of $(T_p M, g_p)$ compatible with the orientation.
\end{lemma}


\begin{proof}
For any $p \in M$ define $\omega(p) = e_1^* \wedge \cdots \wedge e_n^* $ where $(e_1^*, \dots, e_n^*)$ is an ordered dual basis of $T_p^*M$ orthonormal with respect to $g_p$ and compatible with the orientation. If we choose a different set $(\tilde{e}_1, \dots, \tilde{e}_n)$ then we can write,
\[ \tilde{e}_i^* = \sum_{j = 1}^n A_{ji} e_j^* \]
with $A \in \Orth{n}$ because it must preserve the metric $g$. Furthermore, since $e_i'$ is also compatible with the orientation we must have $\det{A} > 0$ so $A \in \SO{n}$ and thus $\det{A} = 1$. Furthermore,
\[ \omega'(p) = \tilde{e}_1^* \wedge \cdots \wedge \tilde{e}_n^* = e_1^* \wedge \cdots \wedge e_n^* \det{A} = e_1^* \wedge \cdots \wedge e_n^* = \omega(p) \]
\end{proof}

\begin{lemma}
Let $M$ be a smooth $n$-manifold. There exists a volume form on $M$ if and only if $\bigwedge^n T^* M$ is trivial.
\end{lemma}

\begin{proof}
There exists a volume form on $M$ if and only if there exists a nonvanishing smooth section of $\bigwedge^n T^* M$ if and only if there there exists a smooth global frame of $\bigwedge^n T^* M$ if and only if $\bigwedge^n T^* M$ is a trivial vector bundle of rank $1$ over $M$. 
\end{proof}

\begin{lemma}
Let $M$ be a smooth $n$-manifold then every volume form is in corresponence to a choice of orientation. Thus, $M$ admits a volume form if and only if $M$ is orientible. 
\end{lemma}

\section{Connections}

\begin{definition}
An affine connection $\nabla$ on $M$ is a bilinear map,
\[ \mathscr{X}(M) \times \mathscr{X}(M) \to \mathscr{X}(M) \]
\end{definition}

\begin{definition}
Let $M$ be a $\C{\infty}$ manifold. An affine connection $\nabla$ on $M$ is symmetric if,
\[ \nabla_X Y - \nabla_Y X = [X, Y] \]
for any $X, Y \in \mathscr{X}(M)$. 
\end{definition}

\begin{definition}
The torsion of $\nabla$ is defined as,
\[ T_{\nabla} : \mathscr{X}(M) \times \mathscr{X}(M) \to \mathscr{X}(M) \]
where,
\[ T_{\nabla}(X, Y) = \nabla_X Y - \nabla_Y X - [X, Y] \]
Then $T_{\nabla}$ is bilinear and antisymmetric so $T_{\nabla} \in \C{\infty}(M, \Omega^2(M, TM) \otimes TM)$. 
\end{definition}

\begin{remark}
An affine connection $\nabla$ is symmetric or torsion-free iff $T_{\nabla} = 0$.
\end{remark}

\begin{proposition}
The space of all affine connections on $M$ is an affine space with associated vectorspace $\C{\infty}(M, T^1_2 M)$. The space of all symmetric affine connections is also an affine space with associate vector space $\C{\infty}(M, \mathrm{Sym}^2(T^*M) \otimes TM)$. 
\end{proposition}

\begin{definition}
Let $(M, g)$ be a Riemannian manifold. An affine connection $\nabla$ on $M$ is compatible with the Riemannian structure $g$ if,
\[ X(g(Y,Z)) = g(\nabla_X Y, Z) + g(Y, \nabla_X Z) \]
for any $X,Y,Z \in \mathscr{X}(M)$. 
\end{definition}

\begin{remark}
If $g$ is a symmetric $T_2$ field then $\nabla_X g \in \C{\infty}(M, (T^* M)^{\otimes 2})$ defined by,
\[ (\nabla_X g)(Y, Z) = X(g(Y, Z)) - g(\nabla_X Y, Z) - g(Y, \nabla_X Z) \]
Therefore, $\nabla$ is compatible with $g \iff \nabla_X g = 0 \quad \forall X \in \mathscr{X}(M)$. 
\end{remark}

\begin{theorem}[Levi-Civita]
If $(M, g)$ is a Riemannian manifold then there exists a unique symmetric affine connection $\nabla$ on $M$ which is compatible by the Riemannian structure. 
\end{theorem}

\begin{proof}
Suppose that $\nabla$ is an affine connection on $M$ which is symmetric and compatible with $g$. Then,  by compatibility,
\begin{align*}
X(g(Y,Z)) & - g(\nabla_X Y, Z) - g(Y, \nabla_X Z) = 0
\\
Y(g(X,Z)) & - g(\nabla_Y X, Z) - g(X, \nabla_Y Z) = 0
\\
Z(g(X,Y)) & - g(\nabla_Z X, Y) - g(X, \nabla_Z Y) = 0
\end{align*}
Then take (1) + (2) - (3), and use symmetry,
\begin{align*}
X(g(X,Z)) + Y(g(Z,X)) - Z(g(X,Y)) = g(X, [Y, Z]) + g(Y, [X,Z]) + g(Z, [X, Y]) + 2 g(Z, \nabla_Y X) 
\end{align*}
Therefore,
\[ g(Z, \nabla_Y X) = \tfrac{1}{2} \left[ X(g(Y, Z)) + Y(g(Z, X)) - Z(g(X, Y)) - g(X, [Y,Z]) - g(Y, [X,Z]) - g(Z, [X, Y]) \right] \]
Which implies that $\nabla$ is uniquely determined by the metric $g$. To show existence, define $\nabla_X Y$ by the above equation for any $Z \in \mathscr{X}(M)$. 
\end{proof}

\begin{definition}
Let $(M, g)$ be a Riemannian manifold and $p \in M$ then there exists an open neighborhood $V$ of $p$ in $M$ and $\epsilon > 0$ such that $\phi(t, q, w)$ and $\gamma(t, q, w)$ are defined for $|t| < 2$ and $q \in V$ and $|w| \le \epsilon$ then $\exp : U_{(v,\epsilon)} \to M$ is defined by $\exp(q, w) = \gamma(1, q, w)$. 
\end{definition}

\section{Curvature}

\begin{definition}
Let $(M,g)$ be a Riemannian manifold and $\nabla$ the Levi-Civita connection determined by $g$. Then given $X, Y \in \mathscr{X}(M)$ the Riemann map,
\[ R(X, Y) : \mathscr{X}(M) \to \mathscr{X}(M) \]
is defined by,
\[ R(X, Y)(Z) = \nabla_Y \nabla_X Z - \nabla_X \nabla_Y Z + \nabla_{[X, Y]} Z \]
\end{definition}

\begin{proposition}
Viewing the Riemann map as,
\[ R : \mathscr{X}(M) \times \mathscr{X}(M) \times \mathscr{X}(M) \to \mathscr{X}(M) \]
we have,
\begin{enumerate}
\item $R$ is antisymmetric in the first two arguments, $R(X, Y, Z) = - R(Y, X, Z)$.
\item $R$ is $\C{\infty}(M)$-linear viewing $\mathscr{X}(M)$ as a $\C{\infty}(M)$-module.
\item $R$ is equivalent to an element of $\C{\infty}(M, (\Lambda^2 T^* M) \otimes T^* M \otimes T M) = \Omega^2(M, \End{TM})$ so $R$ is an $\End{TM}$-valued $2$-form. 
\end{enumerate}
\end{proposition}

\begin{remark}
Let $\pi : E \to M$ be a $\C{\infty}$ vector bundle over $M$ with connection $\nabla : \Omega^0(M, E) \to \Omega^1(M, E)$ then we may define,
\[ R_{\nabla} : \mathscr{X}(M) \times \mathscr{X}(M) \times \Omega^0(M, E) \to \Omega^0(M, E) \]
by,
\[ R_{\nabla}(X,Y)(S) = \nabla_X \nabla_Y S - \nabla_Y \nabla_X S - \nabla_{[X,Y]} S \]
\end{remark}

\begin{theorem}[Bianchi]
\[ R(X,Y)(Z) + R(Y,Z)(X) + R(Z, X)(Y) = 0 \]
\end{theorem}

\begin{proof}
This property follows from the symmetry of the Levi-Civita connection and the Jacobi identity. We have,
\begin{align*}
Q = R(X,Y)(Z) + R(Y,Z)(X) + R(Z, X)(Y) & = \nabla_Y \nabla_X Z - \nabla_X \nabla_Y Z - \nabla_{[Y,X]} Z
\\
& + \nabla_Z \nabla_Y X - \nabla_Y \nabla_Z X - \nabla_{[Z,Y]} X
\\
& + \nabla_X \nabla_Z Y - \nabla_Z \nabla_X Y - \nabla_{[X,Z]} Y
\end{align*}
Using symmetry,
\begin{align*}
Q & = \nabla_Y [X, Z] - \nabla_{[Y,X]} Z + \nabla_X [Z, Y] - \nabla_{[Z,Y]} X + \nabla_Z [Y, X] - \nabla_{[X,Z]} Y
\\
& = [Y, [X,Z]] + [X, [Z, Y]] + [Z [Y, X]] = 0 
\end{align*}
Which is zero by the Jacobi identity. 
\end{proof}

\newcommand{\Rie}{\mathcal{R}}

\begin{definition}
For $X, Y, Z, W \in \mathscr{X}(M)$, the Riemman tensor is defined by,
\[ \Rie(X, Y, Z, W) = g(R(X, Y)(Z), W) \in \C{\infty}(M) \]
Thus, $\Rie$ is a smooth $(0,4)$-tensor field on $M$.
\end{definition}

\begin{proposition}
The Riemann tensor $\Rie$ satisfies,
\begin{enumerate}
\item $\Rie(X, Y, Z, W) + \Rie(Y, Z, X, W) + \Rie(Z, X, Y, W) = 0$.
\item $\Rie \in \C{\infty}(M, \mathrm{Sym}^2(\Lambda^2 T^* M))$ or equivalently,
\begin{enumerate}
\item $\Rie(X, Y, Z, W) = - \Rie(Y, X, Z, W)$
\item $\Rie(X, Y, Z, W) = - \Rie(X, Y, W, Z)$
\item $\Rie(X, Y, Z, W) = \Rie(Z, T, X, Y)$
\end{enumerate}
\end{enumerate}
\end{proposition}

\section{Covariant Derivatives of Tensors}

Let $M$ be a smooth $n$-manifold with an affine connection $\nabla$ such that for $X \in \mathscr{X}(M)$ we have an $\R$-linear map, $\nabla_X : \mathscr{X}(M) \to \mathscr{X}(M)$. 
We will extend this to a derivative on all tensors inductively by imposing the Leibniz rule.
\begin{align*}
(0,0) \quad & f \in \C{\infty}(M) \quad \nabla_X f = X(f) 
\\
(1, 0) \quad & Y \in \mathscr{X}(M) \quad \nabla_X Y
\\
(0, 1) \quad & \omega \in \Omega^1(M) \quad (\nabla_X \omega)(Y) = X(\alpha(Y)) - \alpha(\nabla_X Y)
\end{align*} 
Suppose that $T_1$ is an $(r_1, s_1)$-tensor and $T_2$ is an $(r_2, s_2)$-tensor. Then we want,
\[ \nabla_X (T_1 \otimes T_2) = \nabla_X T_1 \otimes T_2 + T_1 \otimes \nabla_X T_2 \]
This extends $\nabla_X$ to a map $\nabla_X : \C{\infty}(M, T^{r}_s M) \to \C{\infty}(M, T^r_s M)$ 
Given $T$ a $(r, s)$-tensor we can write $T$ as a map,
\[ T : \mathscr{X}(M)^{\otimes s} \to \mathscr{X}(M)^{\otimes r} \]
which is $C^{\infty}(M)$-linear. Now, define,
\[ \nabla T : \mathscr{X}(M)^{\otimes (s + 1)} \to \mathscr{X}(M)^{r} \]
given by,
\[ (\nabla T)(X_1, \dots, X_{s+1}) = (\nabla_{X_{s+1}} T)(X_1, \dots, X_s) \]
Thus, $\nabla T$ is a $(r, s + 1)$-tensor. 

\begin{proposition}
Let $\nabla$ be an affine connection on a Riemannian manifold $(M, g)$.
\begin{enumerate}
\item $\nabla$ is symmetric $\iff \d{\alpha}(X, Y) = \nabla \alpha(Y, X) - \nabla \alpha(X, Y)$ for any $\alpha \in \Omega^1(M)$ and $X,Y \in \mathscr{X}(M)$
\item $\nabla$ is compatible with $g \iff \nabla g = 0$.
\end{enumerate}
\end{proposition}

\begin{proof}

\end{proof}

\subsection{Covariant Derivatives In Local Coordinates}

Take the local coordinates $(x_1, \dots, x_n)$ on $U$. We can write,
\[ \nabla_{\pderiv{}{x_i}} \pderiv{}{x_j} = \sum_{k = 1}^n \Gamma_{ij}^k \pderiv{}{x_k} \]
for some $\Gamma_{ij}^k \in \C{\infty}(M)$ since this is a $\C{\infty}(M)$-basis of vector fields. Now,
\begin{align*}
\nabla_{\pderiv{}{x_i}} \d{x_j} & = \sum_{k = 1}^n \left( \nabla_{\pderiv{}{x_i}} \d{x_j} \right) \left( \pderiv{}{x_k} \right) \d{x_k} = \sum_{k = 1}^n \left( \pderiv{}{x_i} \left( \d{x_j} \left( \pderiv{}{x_k} \right) \right) - \d{x_j} \left( \nabla_{\pderiv{}{x_i}} \pderiv{}{x_k} \right) \right) \d{x_k}
\\
& =  \sum_{k = 1}^n \left( \pderiv{}{x_i} \delta_{jk} - \d{x_j} \left( \Gamma_{ik}^\ell \pderiv{}{x_\ell} \right) \right) \d{x_k} = - \sum_{k = 1}^n \Gamma_{ik}^j \d{x_k} 
\end{align*}
Finally, we can compute the covariant derivative of an arbitrary tensor in local coordinates.


\section{Jacobi Fields}

Let $(M, g)$ be a Riemannian manifold and $\gamma : [0, a] \to M$ a geodesic. Then a Jacobi field may arise as follows. Let $f_s : [0, a] \to M$ with $s \in (-\epsilon, \epsilon)$ be a smooth family of geodesics. That is,
\[ f : (-\epsilon, \epsilon) \times [0, a] \to M \]
such that $f_s$ is a geodesic and $f_0 = \gamma$. Then set $J(t) = \pderiv{f}{s}(0, t)$. 


\section{Feb 3}

Let $f : (M, g) \to (\bar{M}, \bar{g})$ be an isometric immersion. Then let $\nabla$ and $\bar{\nabla}$ be the Levi-Civita connections and $\nabla = f^* \bar{\nabla}$ on $f^* T\bar{M}$. 


\section{Feb 13}

\begin{theorem}
Let $(M, g)$ be a connected Riemannian manifold then $(M, d)$ is a metric space with $d$ induced by $g$.
\end{theorem}

\begin{proof}

\end{proof}


\begin{theorem}[Hopf-Rinow]
Let $(M, g)$ be a connected Riemannian manifold so $(M, d)$ is a metric space with $d$ induced by $g$. Then for any $p \in M$ TFAE,
\begin{enumerate}
\item $\exp_p$  is defined on $T_p M$ 
\item closed bounded subsets of $M$ are compact
\item $(M, d)$ is complete
\item $(M, g)$ is geodesically  complete
\item there exists compact sets $K_n \subset M$ covering $M$ such that $q_n \notin K_n \implies d(p, q_n) \to \infty$ as $n \to \infty$.
\item $\forall q \in M$ there exists minimizing geodesic from $p$ to $q$.
\end{enumerate}
\end{theorem}

\begin{proof}

\end{proof}

\begin{corollary}
If $M$ is a compact smooth manifold ten $(M, g)$ is a geodesically complete Riemannian manifold for any Riemannian metric $g$. 
\end{corollary}

\begin{definition}
A connected Riemannian manifold $(M, g)$ is \textit{expandible} if there exists a connected Riemannian manifold $(M', g')$ and an isometric proper open embedding $\iota : (M, g) \to (M', g')$. Otherwise $(M, g)$ is nonextendible.  
\end{definition}

\begin{proposition}
If $(M, g)$ is complete then $(M, g)$ is extendible.
\end{proposition}

\begin{proof}
If there exists $\iota : (M, g) \to (M', g')$ an isometric proper open embedding then $\iota(M)$ must be geodesically incomplete because it is a proper open subset of $M'$. Thus $M$ is geodesically incomplete because $\iota$ is an isometric embedding and thus an isometry onto its image.
\end{proof}

\begin{corollary}
Any (induced) connected Riemannian submanifold of a complete connected Riemannian manifold is complete.
\end{corollary}

\begin{proof}
Let $(N, \iota^* g) \xrightarrow{\iota} (M, g)$ be a Riemannian submanifold. Then $\forall p, q \in N$ we have $d_N(p, q) \ge d_M(p, q)$ so any Cauchy sequence in $N$ is also Cauchy on $M$ and thus converges in $M$. However, $N \subset M$ is closed and thus contains all its limit points. Thus all Cauchy sequences converge in $N$. 
\end{proof}

\begin{definition}
Let $(M, g)$ be a connected complete Riemannian manifold then $\forall p \in M$ we have $\exp_p : T_p : M \to M$. We say that $p$ is a pole if $\exp_p : T_p M \to M$ is a local diffeomorphism i.e. $\forall p \in T_p M$ the map,
\[ \d{(\exp_p)_v} : T_v (T_p M) \to T_{\exp_p(v)} M \]
is a linear isomorphism. 
\end{definition}

\begin{lemma}
Let $(M g)$ be a connected complete Riemannian manifold such that $\forall p \in M$ and any $2$-plane $\sigma \subset T_p M$ then $K(p, \sigma) \le 0$ then $\forall p \in M$ the point $p$ is a pole.
\end{lemma}

\begin{proof}

\end{proof}

\begin{lemma}
Let $(M, g)$ and $(N, h)$ be complete Riemannian manifolds with smooth surjective map $f : (M, g) \to (N, h)$ which is a local diffeomorphism satisfying $\forall p \in M, v \in T_p M$ then $|\d{f_p(v)} |_{f(p)} \ge |v|_p$ then $f : (M, g) \to (N, h)$ is a covering map. 
\end{lemma}

\begin{lemma}
Let $(M, g)$ be a complete connected Riemannian manifold and $p \in M$ is a pole then $\exp_p : T_p M \to M$ is a covering map. 
\end{lemma}

\begin{corollary}
If $(M, g)$ is a complete, connected, simply-connected Riemannian manifold with a pole $p$ then $\exp_p : T_p M \to M$ is a diffeomorphism. 
\end{corollary}

\begin{theorem}[Cartan-Hadamard]
Let $(M, g)$ be a connected complete Riemannian manifold with $K(p, \sigma) \le 0$ forall $p \in M$ and $\sigma \in \mathrm{Gr}(2, T_p M)$ then $\forall p \in M$ the exponential map $\exp_p : T_p M \to M$ is a covering map. In particular, if $M$ is simply-connected then $M \cong \R^{\dim{M}}$.  
\end{theorem}

\section{General Lemmata}

\begin{lemma} \label{bounded_lin_op}
Let $T : V \to W$ be a linear map of finite-dimensional real normed spaces. Then there exists $M \in \R$ such that for all $v \in V$,
\[ ||T(v)|| \le M ||v|| \]
\end{lemma}

\begin{proof}
For $v = 0$, the inequality is trivial. Suppose $v \neq 0$ then we can always scale,
\[ T(v) = ||v|| T \left( \frac{v}{||v||} \right) \]
Let $\dim{V} = n$ and thus $V \cong \R^n$ as normed spaces. Thus, $\{ v \in V \mid ||v|| = 1\} \cong S^{n-1}$. Furthermore, $T : V \to W$ is linear and thus continuous. Therefore, the restriction $T : S^{n-1} \to \R^n$ is also continuous. Since $S^{n-1}$ is compact its image $T(S^{n-1}) \subset W$ is compact and is thus bounded by Heine-Borel. Therefore, there exists $M \in \R$ such that whever $||v|| = 1$ then $||T(v)|| \le M$. Finally, for any $v \in V$,
\[ T(v) = ||v|| T \left( \frac{v}{||v||} \right) \le M ||v|| \]
becaues $v / ||v||$ has unit norm. 
\end{proof}

\begin{lemma} \label{ring_map_quotient}
Let $f : A \to B$ be an $K$-algebra homomorphism. The quotient $A / A^k$ is a $K$-vector space and $f : A / A^k \to B / B^k$ is a well-defined map of $K$-vectorspaces. 
\end{lemma}

\begin{proof}
Clearly, $A^k \subset A$ is a subvector space so the quotuent is a $K$-vectorspace. Furthermore, consider the map,
\begin{center}
\begin{tikzcd}
A \arrow[r, "f"] & B \arrow[r, "\pi"] & B / B^k
\end{tikzcd}
\end{center}
However, $f(A^k) \subset B^k$ since $f$ is an algebra homomorphism and thus $\pi \circ f(A^k) = (0)$ so $A^k \subset \ker{\pi \circ f}$ and thus $\pi \circ f$ factors through the quotient $A / A^k$.  
\end{proof}


\section{Feb. 24}

\begin{theorem}[Hadamard Theorem]

\end{theorem}

\begin{theorem}[Cartan]

\end{theorem}


\begin{lemma}
Let $f : (M_1, g_1) \to (M_2, g_2)$ be a surjective local diffeomorphism and $\forall p \in M : |\d{f_p}(v)| \ge |v|$ and $(M_1, g_1)$ is complete then $f$ is a covering map. 
\end{lemma}

\begin{proof}

\end{proof}

\begin{theorem}
Let $(\tilde{M}, \tilde{g})$ be a simply connectd, complete manifold with constant sectional curvature $K$ then $(tilde{M}, \tilde{g})$ is isometric to
\begin{enumerate}
\item $(H^n, g_{\text{can}})$ if $K = -1$
\item $(R^n, g_{\text{can}})$ if $K = 0$
\item $(S^n, g_{\text{can}})$ if $K = 1$
\end{enumerate} 
\end{theorem}

\begin{proof}
Let $\Delta$ be the required space in the cases $K = -1$ and $K = 0$. Take $\tilde{p} \in \tilde{M}$ and $p \in \Delta$ be any point. Let $\iota : T_{\tilde{p}} \tilde{M} \to T_p \Delta$ be any linear isometry. By Hadamard's theorem, the exponetial maps are diffeomorphism. Therefore, $f = \exp_p \circ \circ \iota \circ \exp_{\tilde{p}}^{-1}$ is an isometry $\tilde{M} \to \Delta$. 
\bigskip\\
Now conisder the case $K = 1$. Let $p \in S^n$ and $\tilde{p} \in \tilde{M}$ be some maps and $\iota : T_p S^n \to T_{\tilde{p}} \tilde{M}$ any linear isometry. Again, define the map $f = \exp_{\tilde{p}} \circ \iota \circ \exp_p^{-1}$ taking the open neighborhood $S^{n} \setminus \{-p\}$ to $\tilde{M}$. By Cartan, $f$ is a local isometry. Now, choose any other $p' \in S^n$ besides $p$ and $-p$ and $\tilde{p}' = f(p')$. Then we may construct the map $f' : S^n \setminus \{ - p' \} \to \tilde{M}$ via $f = \exp_{p'} \circ \iota' \circ \exp_{\tilde{p'}}$ via the linear isometry defined as $\iota' = \d{f_{p'}} : T_{p'} S^n \to T_{\tilde{p}'} \tilde{M}$. Therefore, $f(p') = f'(p')$ and $\d{f_{p'}} = \iota = \d{f'_{p'}}$ and thus, on the overlap $f = f'$. Therefore the two functions glue to form a local isometry $h : S^n \to \tilde{M}$. By the lemma, $h$ is a covering map but $\tilde{M}$ is simply connected so $h$ is a diffeomorphism and thus a global isometry.     
\end{proof}

\begin{corollary}
Let $(M^n, g)$ be a space form (a complete Riemannian manifold with constant sectional curvature $K$) then $(M^n, g)$ is isometric to $(\tilde{M} / \Gamma, \hat{g})$ where,
\[ (\tilde{M}, \hat{g}) 
= \begin{cases}
(S^n, \lambda^{-1} g_{\text{can}}) & K = \lambda
\\
(\R^n, g_{\text{can}}) & K = 0
\\
(H^n, \lambda^{-1} g_{\text{can}}) & K = - \lambda
\end{cases} \]
where $\Gamma$ is a discrete subgroup of isometries of $\tilde{M}$ which acts freely and properly discontinuously. Furthermore the map $(\tilde{M}, \tilde{g}) \to (\tilde{M} / \Gamma, \hat{g})$ is a covering map and local isometry. 
\end{corollary}

\begin{proof}
Let $\tilde{M}$ be the universal cover of $M$. Then equip $\tilde{M}$ with the unique smooth structure such that $\pi : \tilde{M} \to M$ is a local diffeomorphism. Let $\tilde{g} = \pi^*(g)$ then $\tilde{g}$ is a Riemannian metric on $\tilde{M}$. Let $\Gamma = D(\pi)$ be the group of deck transformations of the covering map $\pi : \tilde{M} \to M$. Since $\tilde{M}$ is simply-connected $D(\pi) \cong \pi_1(M)$. We have that $\Gamma$ facts isometrically on $(\tilde{M}, \tilde{g})$ since it commutes with $\pi$ and $\tilde{g} = \pi^*(g)$. Furthermore, $\Gamma$ acts freely and properly discontinuously since it is the deck transformations $\tilde{M} \to M$ is a covering map.  
\end{proof}

\section{Feb. 28}

\newcommand{\RP}{\mathbb{RP}}

\begin{proposition}
Let $(M^n, g)$ ne a complete Riemannian manifold with constant sectional curvature $K = +1$ and $n = 2m$ even then $M^n = S^n / \Gamma$ for $\Gamma \subset O(n+1)$ and thus $(M^n, g)$ is isometric to either $(S^n, g_{\text{can}})$ or $(\RP^n, \hat{g})$. In particular, if $M^n$ is orientable then $M^n \cong S^n$. 
\end{proposition}

\begin{proof}
We have $M^n \cong S^{2m} / \Gamma$ with $\Gamma \subset O(n+1)$. Then $\Gamma$ acts freely and properly discontinuously on $S^{2m}$. All $O(n+1)$ maps are normal and thus diagonalizable with an odd number of eigenvalues each of the from $e^{i \theta}$. If $\gamma \in \Gamma$ has a $+1$ eigenvalue then $\gamma$ has a fixed point on $S^{2m}$ but the action is free so $\gamma = \id$. For any $\gamma \in \Gamma$, if $\det{\gamma} = +1$ then $\gamma$ has $+1$ as an eigenvalue so $\gamma = \id$. Otherwise for $\gamma \in O(n+1)$ we must have $\det{\gamma} = - 1$ so $\det{\gamma^2} = 1$ and therefore $\gamma^2 = \id$. This implies that all the eigenvalues of $\gamma$ are $\pm 1$. If $\gamma \neq \id$ then all its eigenvalues must be $-1$ and thus $\gamma = - \id$. Therefore either $\Gamma = \{\id\}$ or $\Gamma = \{\id, -\id\}$. 
\end{proof}

\subsection{Conformal Maps}

\newcommand{\inner}[2]{\left< #1 , #2 \right>}

\begin{definition}
Let $V$ and $W$ be $n$-dimensional inner product spaces. Then $T : V \to W$ is a linear conformal map if $T$ is a linear isomorphism and,
\[ \cos{\theta(T(v), T(u))} = \frac{\inner{T(v)}{T(u)}}{|T(u)|\cdot |T(v)|} = \frac{\inner{v}{u}}{|v| \cdot |u|} = \cos{\theta(v, u)} \]
\end{definition}

\begin{lemma}
Let $V$ and $W$ be inner product spaces of dimension $n$ and $T : V \to W$ a linear map. Then $T$ is a linear conformal map iff there exists $\lambda > 0$ such that $|L(v)|_W = \lambda |v|_V$ for all $v \in V$ iff $\inner{T(v)}{T(u)}_W = \lambda^2 \inner{v}{u}_V$ for all $v,u \in V$. 
\end{lemma}

\begin{definition}
Let $(M, g)$ and $(N, h)$ be Riemannian manifolds then a smooth map $f : M \to N$ is \textit{conformal} if $\forall p \in M$ the differential $\d{f_p} : T_p M \to T_{f(p)} N$ is a linear conformal map.
\end{definition}

\begin{remark} 
linear conformal map $\implies$ linear isomorphism $\implies \dim{M} = \dim{N}$ and $f$ is a local diffeomorphism. By the lemma, $f$ is  confroal map if $f$ is a clocal diffomorphism and $f^*h = \lambda^2 g$ for some smooth function $\lambda : M \to \R^+$. In particular, if $f$ is a local isometry then it is conformal wit $\lambda = 1$. Local isometry $\implies$ conformal $\implies$ local diffeomorphism but neither arrow is reversable. 
\end{remark}

\begin{example}
Take $f_\lambda : \R^n \to \R^n$ given by $f(\vec{x}) = \lambda \vec{x}$ for $\lambda \in \R^+$. Then,
\[ f^*_\lambda g = f^*_\lambda (\d{x_1^2} + \cdots + \d{x_n^2}) = \lambda^2 (\d{x_1^2} + \cdots + \d{x_n^2}) = \lambda^2 g \]
Then $\forall \vec{x} : \d{f_\lambda}_{\vec{x}} : T_{\vec{x}} \R^n \to T_{\vec{x}} \R^n$. Also $\det{\d{f_\lambda}} = \lambda^n > 0$ so $f_\lambda$ is a conformal orientation preserving map $(\R^n, g_0) \to (\R^n, g_0)$.
\end{example}

\begin{example}
For $\vec{x}_0 \in \R^n$ take $\iota_{\vec{x}_0} : \R^n \setminus \{\vec{x}_0\} \to \R^n \setminus \{ \vec{x}_0 \}$ which inverts the point across the unit sphere arround $\vec{x}_0$ such that,
\[ |\iota_{\vec{x}_0}(\vec{x}) - \vec{x}_0| = \frac{1}{|\vec{x} - \vec{x}_0|} \]
We define,
\[ \iota_{\vec{x}_0}(\vec{x}) = \frac{\vec{x} - \vec{x}_0}{|\vec{x} - \vec{x}_0|^2} + \vec{x}_0 \]
Then the differential is,
\[ \d{\iota_{\vec{x}_0}}_{\vec{x}}(\vec{v}) = \frac{1}{|\vec{x} - \vec{x}_0|^2} \left( \vec{v} - 2 \frac{\inner{\vec{x} - \vec{x}_0}{\vec{v}}}{|\vec{x} - \vec{x}_0|^2} (\vec{x} - \vec{x}_0) \right) \]
Therefore, the differential simply scales by $|\vec{x} - \vec{x}_0|^{-2}$ and reverses the component of $\vec{v}$ perpendicular to the unit sphere which leaves the length of vectors invariant up to the overall scaling. Therefore $\iota_{\vec{x}_0}$ is conformal. 
\end{example}

\begin{theorem}[Liouville]
Let $U \subset \R^n$ is connected open and $f : U \to \R^n$ is conformal with respect to $g_0$. If $n \ge 3$ then $f$ is the restriction to $U$ of a composition of isometries, dilations $(f_\lambda)$, and inversions $\iota_{\vec{x}_0}$ at most one of each.
\end{theorem}

\subsection{M\"{o}bius Transformation}

Consider the map,
\[ f(z) = \frac{az + b}{cz + d} \quad \quad 
\begin{pmatrix}
a & b
\\
c & d
\end{pmatrix}
\in \PSL{2, \mathbb{C}} \]


\begin{theorem}
For $n \ge 2$, the isometries of $H^n$ are restrictions to $H^n \subset \R^n$ of the conformal transformations of $\R^{n}$ that map $H^n \to H^n$. 
\end{theorem}

\begin{proof}

\end{proof}

\section{March 4}

\begin{definition}
Let $c : [0, a] \to M$ be a piecewise smooth curve in a smooth manifold $M$. A \textit{variation} of $c$ is a continuous map $f : (-\epsilon, \epsilon) \times [0, a] \to M$ denoted as $(s,t) \mapsto f_s(t)$ such that,
\begin{enumerate}
\item $f_0(t) = c(t)$
\item $\exists 0 = t_0 < t_1 < \cdots < t_k < k_{k+1} = a$ such that $f|_{(-\epsilon, \epsilon) \times [t_k, t_{k+1}]}$ is smooth.
\end{enumerate}
Given a variation we have the following situations,
\begin{enumerate}
\item For fixed $s \in (-\epsilon, \epsilon)$ the map $f_s : [0, a] \to M$ is called the \textit{curve of variation} of $f$. 
\item For fixed $t \in (0, a]$ the function $g_t : (-\epsilon, \epsilon) \to M$ given by $g_t(s) = f_s(t)$ is called a \textit{transverse curve in the variation} $f$.
\item The vector $V(t) = \pderiv{f}{s}(s,t)$ for $t \in [0, a]$ is called the \textit{variational field} of $f$. 
\item We say that $f$ is \textit{proper} if $\forall s \in (-\epsilon, \epsilon)$ we have $f_s(0) = c(0)$ and $f_s(a) = c(a)$ i.e. $V(0) = V(a) = 0$. 
\end{enumerate}
\end{definition}

\begin{proposition}
Let $c : [0, a] \to M$ be a piecewise smooth curve. For any piecewise smooth vector-field $V(t)$ along $c(t)$, there exists a variation $f : (-\epsilon, \epsilon) \times [0, a]$ of $c$ for some $\epsilon > 0$ such that, $V$ is the variational field of $f$. Moreover, if $V(0) = V(a) = 0$ then $f$ can be chosen to be proper. 
\end{proposition}

\begin{definition}[Energy Function]
Given a variation $f : (-\epsilon, \epsilon) \times [0, a] \to M$ of some piecewise smooth curve $c : [0, a] \to M$ on a Riemannian manifold $M$ we define the energy function,
\[ E_f(s) = \int_0^a \left| \pderiv{f}{t}(s,t) \right|^2 \d{t} \]
which is a (piecewise) smooth map $E_f : (-\epsilon, \epsilon) \to \R$ i.e. a map $E(f_s)$. Let $V$ be the variational field then,
\[ \d{E_c}(V) = E'(0) \]
\end{definition}

\begin{theorem}[Formula For the First Variation of Energy]
Let $(M, g)$ be a Riemannian manifold and $c : [0, a] \to M$ a piecewise smooth curve and $f : (-\epsilon, \epsilon) \times [0, a] \to M$ the variation of $c$ with variational field $V$. Let $E : (-\epsilon, \epsilon) \to \R$ be the energy function of $f$. Then we may compute,
\begin{align*}
\tfrac{1}{2} E'(0) & = - \int_0^a \inner{V(t)}{\frac{D}{\d{t}} \deriv{c}{t}}\d{t} + \sum_{i = 1}^k \inner{V(t_i)}{\deriv{c}{t}(t_i^+) - \deriv{c}{t}(t_i^{-})} 
\\
& \quad - \inner{V(0)}{\deriv{c}{t}(0)} + \inner{V(a)}{\deriv{c}{t}(0)} 
\end{align*} 
\end{theorem}

\begin{proof}
Consider the Energy function,
\[ E(s) = \int_0^a \inner{\pderiv{f}{t}}{\pderiv{f}{t}} \d{t} = \sum_{i = 1}^k \int_{t_i}^{t_{i+1}} \inner{\pderiv{f}{t}}{\pderiv{f}{t}} \d{t} \]
Consider, a single term,
\begin{align*}
\frac{1}{2} \deriv{}{s} \int_{t_i}^{t_{i+1}} \inner{\pderiv{f}{t}}{\pderiv{f}{t}} \d{t} & = \int_{t_i}^{t_{i+1}} \inner{\frac{D}{\d{s}} \pderiv{f}{t}}{\pderiv{f}{t}} \d{t} 
\\
& = \int_{t_i}^{t_{i+1}} \inner{\frac{D}{\d{t}} \pderiv{f}{s}}{\pderiv{f}{t}} \d{t} 
\\
& = \int_{t_i}^{t_{i+1}} \left( \pderiv{}{t} \inner{\pderiv{f}{s}}{\pderiv{f}{t}} - \inner{\pderiv{f}{s}}{\frac{D}{\d{t}} \pderiv{f}{t}} \right) \d{t}
\\
& = \inner{\pderiv{f}{s}}{\pderiv{f}{s}} \bigg|_{t = t_i^+}^{t = t_{i+1}^{-}} - \int_{t_i}^{t_{i+1}} \inner{\pderiv{f}{s}}{\frac{D}{\d{t}} \pderiv{f}{t}} \d{t}
\end{align*}
Therefore,
\begin{align*}
\tfrac{1}{2} E'(s) = - \int_0^a \inner{\pderiv{f}{s}}{\frac{D}{\d{t}} \pderiv{f}{t}} \d{t}  + \sum_{i = 1}^k \inner{\pderiv{f}{s}}{\deriv{f}{t}} \bigg|_{t = t_i^+}^{t = t_{i+1}^-}
\end{align*}
However, at $s = 0$ we have $\pderiv{f}{s}(0, t) = V(t)$ and $\pderiv{f}{t}(0, t) = c'(t)$. Therefore,
\begin{align*}
\tfrac{1}{2} E'(0) & = - \int_0^a \inner{V(t)}{\frac{D}{\d{t}} \deriv{c}{t}}\d{t} + \sum_{i = 1}^k \inner{V(t_i)}{\deriv{c}{t}(t_i^+) - \deriv{c}{t}(t_i^{-})} 
\\
& \quad - \inner{V(0)}{\deriv{c}{t}(0)} + \inner{V(a)}{\deriv{c}{t}(0)} 
\end{align*} 
\end{proof}

\begin{proposition}
Consider the critical points of $E : \Omega_{pq} \to \R$ where $c : [0, a] \to M$ is a piecewise smooth cure. Then for any proper variation $f$ of $c$, the energy function satisfies $E'(0) = 0$ if and only if $c$ is a geodesic. 
\end{proposition}

\begin{proof}
If $c$ is a geodesic then,
\[ \frac{D}{\d{t}} \deriv{c}{t} = 0 \]
and because $c$ is smooth we have $c(t_i^+) = c(t_i^-)$. Therefore, by the above formula $E'(0) = 0$.
\bigskip\\
Conversely, consider two particular variations of $c$. 
\end{proof}

\section{Principal Bundles}

\begin{definition}
We say that $\pi : P \to M$ is a principal $G$-bundle for a Lie group $G$ acting freely on the right on $P$ such that $\pi : P \to M$ is the quotient map $P \to P / G$ if there are local trivializations $(U_\alpha, \psi_\alpha)$ such that,
\begin{center}
\begin{tikzcd}
\pi^{-1}(U_\alpha) \arrow[rd] \arrow[rr, "\psi_\alpha"] & & U_\alpha \times G \arrow[ld]
\\
& U_\alpha  
\end{tikzcd}
\end{center}
commutes and $\psi_\alpha$ is $G$-equivariant i.e. $\psi_\alpha(p \cdot g) = \psi_\alpha(p) \cdot g$ where the action of $G$ on $U_\alpha \times G$ is $(x, h) \cdot g = (x, hg)$. 
\end{definition}

\begin{definition}
Let $\pi : P \to M$ be a pricipal $G$-bundle. Let $F$ be a smooth manifold equiped with a left $G$-action. Then $G$ actions on $P \times F$ freely on the right via $(p, \xi) \cdot g = (p \cdot g, g^{-1} \cdot \xi)$. 
\end{definition}

\begin{definition}
Let $P \times_G F$ denote the fibre product given by $(P \times F)/G$. Then the map $\tilde{\pi} : P \times_G F \to M$ given by $[p, \xi] \mapsto \pi(p)$ is a fibre bundle with base $M$ and fiber $F$. 
\end{definition}

\newcommand{\Aut}[1]{\mathrm{Aut}\left( #1 \right)}

\begin{definition}
Let $\pi : P \to M$ be a principal $G$-bundle and $\rho : G \to \Aut{V}$ a representation. Then we may take $P \times_\rho V = P \times_G V$, the associate vector bundle. 
\end{definition}

\begin{example}
Let $\pi_E : E \to M$ be a vector bundle of rank $r$ over $M$ and take the frame bundle $\Aut{E} \to M$ which is a principle $\Aut{\R^r}$-bundle. Then we may consider the associated vector bundle $\Aut{E} \times_\rho \R^r = E$. However, we may also consider the vector bundle associated to the dual representation, $\Aut{E} \times_{\rho^*} \R^r = E^*$. More generally, we may take the representation,
\[ \rho^{\otimes s} \otimes (\rho^*)^{\otimes t} : \Aut{\R^r} \to \Aut{\R^{r^{s+t}}} \]
then we find the associated vector bundle,
\[ \Aut{E} \times_{\rho^{\otimes s} \otimes (\rho^*)^{\otimes t}} \R^{r^{s + t}} = E^{\otimes s} \otimes (E^*)^{\otimes t} \]
In particular, 
\[ \Aut{TM} \times_{\rho^{\otimes s} \otimes (\rho^*)^{\otimes t}} \R^{n^{s + t}} = T^s_t M \]
\end{example}

\renewcommand{\U}{\mathrm{U}}
\renewcommand{\C}{\mathrm{C}}

\begin{example}
Let $h$ be an inner product on a real vector bundle $E$. Then consider the orthonormal frame bundle $\O(E, h) \to M$ which is a principal $\O(r)$-bundle where $r$ is the rank of $E$. There is a representation $\rho : \O(n) \to \Aut{\R^r}$ and then $\O(E, h) \times_\rho \R^r = E$ and $\O(E, h) \times_{\rho^*} \R^r = E^*$ but these are isomorphic because $\rho = \rho^*$ since it is the orthonormal representation.
\bigskip\\
Now let $h$ be a hermitian metric on a complex vector bundle $E \to M$ of rank $r$ and consider $\U(E, h) \to M$ the unitary frame bundle. Then there are representations $\rho : \U(r) \to \Aut{\C^r}$ and $\rho^* : \U(r) \to \Aut{\C^r}$ which give $\U(E, h) \times_\rho \C^r = E$ and $\U(E, h) \times_{\rho^*} \C^r = E^*$. However, $\rho^* = \overline{\rho}$ since it is the representation of the unitary group. Thus, $\U(E, h) \times_{\rho^*} \C^r = \overline{E}$. 
\end{example}

\section{March 27}

\subsection{Cross Section}

\begin{definition}
A \textit{cross section} of a fiber bunle $\pi : E \to $ with fiber $F$ is a smooth map $\sigma : M \to E$ such that $\pi \circ \sigma = \id_M$.
\end{definition}


\begin{lemma}
Let $\pi : E \to M$ be a trivial fiber bundle with fiber $F$ then cross sections correspond exactly to smooth maps $M \to F$.
\end{lemma}

\begin{proof}
Consider,
\begin{center}
\begin{tikzcd}
E \arrow[rr, "\psi"] \arrow[dr, "\pi"] & & M \times F \arrow[dl]
\\
& M \arrow[ru, bend right, "s"']
\end{tikzcd}
\end{center}
$s : M \to M \times F$ is a section of $M \times F \to M$ and thus is a map $M \to F$.
\end{proof}

\begin{lemma}
Let $\pi : P \to M$ be a principal $G$-bundle then $\pi : P \to M$ is trivial iff it admits a cross section.
\end{lemma}

\begin{proof}
If $\pi : P \to M$ is trivial then the above lemma gives a section. Conversely, if $\sigma : M \to P$ is a corss section then define $\phi : M \times G \to P$ via $\phi(x,g) = \sigma(x) \cdot g$. Then $\phi$ is a $G$-equivariant diffeomorphism since 
\[ \phi((x, a) \cdot g)) = \phi(x,ag) = \sigma(x) \cdot (ag) = (\sigma(x) \cdot a) \cdot g = \phi(x,a) \cdot g \]
and furthermore the following diagram commutes,
\begin{center}
\begin{tikzcd}
M \times G \arrow[rr, "\phi"] \arrow[rd] & & P \arrow[dl, "\pi"] 
\\
& M
\end{tikzcd}
\end{center}
because $\pi \circ \phi(x,g) = \pi(\sigma(x) \cdot g) = \pi(\sigma(x)) = x$. 
\end{proof}

\subsection{Vertical Spaces}

\begin{definition}
Let $\pi : E \to M$ be a fibre bundle with fibre $F$. For any $u \in E$ let $x = \pi(u) \in M$ then $\iota_x : \pi^{-1}(x) \hookrightarrow E$ be the inclusion of the firbre $\pi^{-1}(x) = E_x \cong F$. The vertical space $V_u = \Im{(\d{\iota_x})_u} \subset T_u E$. Then $\dim{V_u} = \dim{F} = N$. Then $\{ V_u \subset T_u E \}$ is a smooth distribution. Equivalently $V \to E$ is a smooth subbundle of $TE \to E$ of rank $N$. In particular if $E$ is a vector bundle then $V \cong \pi^* E$.  
\end{definition}

\newcommand{\g}{\mathfrak{g}}
\newcommand{\Lie}{\mathrm{Lie}}
\renewcommand{\C}[1]{\mathcal{C}^{#1}}

\begin{definition}
Let $G$ be a Lie group, $Y$ a smooth manifold and $G$ acts on $Y$ on the right. Given any $\xi \in \g$ we define the fundamental vector field $X^Y_\xi \in \mathscr{X}(Y)$ via,
\[ X^Y_{\xi}(y) = \deriv{}{t} \bigg|_{t = 0} y \cdot \exp(t \xi) \]
In particular if $Y = G$ and $G$ acts on $G$ by right multiplication then,
\[ X^G_{\xi} = X^L_{\xi} \]
is the unique left invariant vector field on $G$ with $X^L_\xi(e) = \xi$. If $Y = P$ is a princiapl $G$-bundle over $M$ then $X^P_\xi(u) \in V_u$ because the curve $t \mapsto y \cdot \exp(t\xi)$ is contained in the fiber $E_y$. Thus, $X^P_\xi \in \C{\infty}(P, V)$. 
\end{definition}

\begin{lemma}
Let $\pi : P \to M$ be a principal $G$-buundle. Then the vertical bundle is given by $V \cong P \times \g$ where $\g = \Lie(G)$. 
\end{lemma}

\begin{proof}
Define $\phi : P \times \g \to V$ via $\phi(u, \xi) = X^P_\xi(u) \in V_u$. This is an isomorphism of vector bundles over $P$.
\end{proof}

\begin{remark}
In particular, if $P = G$ is a principal $G$-bundle over a point then $V = TG \cong G \times \g$. 
\end{remark}

\newcommand{\Ad}{\mathrm{Ad}}

\subsection{Connections on Principal Bundles}

\begin{remark}
A connection on a principal bundle $\pi : P \to M$ is of one,
\begin{enumerate}
\item horizontal spaces

\item connection $1$-form

\item parallel transport
\end{enumerate}
\end{remark}


\begin{definition}
A \textit{connection} on a principal $G$-bundle $\pi : P \to M$ s an assignment of hoizontal spaces $\{ H_u \subset T_u P \mid u \in P \}$ which is a smooth distribution of $n$-planes on $P$ where $n = \dim{M}$ such that,
\begin{enumerate}
\item $\forall u \in P : T_u P = Vu \oplus H_u$ and thus $TP = V \oplus H$
\item $\forall u \in P : \forall a \in G : H_{u \cdot a} = (\d{R_a})_u (H_u)$ with $R_a : P \to P$ via $u \mapsto u \cdot a$. 
\end{enumerate}
\end{definition}

\begin{definition}
A \textit{connection $1$-form} on a principal $G$-bundle $\pi : P \to M$ is a smooth $\g$-valued $1$-form $\omega \in \Omega^1(P, \g)$ (i.e. for each $X \in \mathscr{X}(P)$ we have a smooth map $\omega(X) : P \to \g$) such that
\begin{enumerate}
\item $\forall \xi \in \g : \omega(X^P_\xi) = (u \mapsto \xi)$
\item $\forall a \in G : R_a^* \omega = \Ad(a^{-1}) \omega$ i.e. pointwise, 
\[ \forall u \in P : \forall a \in G : \forall y \in T_u P : \omega(u \cdot a)((\d{R_a})_u(y)) = \Ad(a^{-1})(\omega(u)(y)) \]
\end{enumerate}
\end{definition}

\begin{lemma}
$(R_a)_* X^P_\xi = X^P_{\Ad(a^{-1}) \xi}$ 
\end{lemma}

\begin{proof}
At $u \in P$ we have,
\begin{align*}
[(R_a)_* X^P_\xi](u) & = (\d{R_a})_{u \cdot a^{-1}}(X^P_\xi(u \cdot a^{-1}) = (\d{R}_a)_{u \cdot a^{-1}} \deriv{}{t} \bigg|_{t = 0} u \cdot a^{-1} \exp(t \xi)
\\
& = \deriv{}{t} \bigg|_{t = 0} u \cdot a^{-1} \cdot \exp(t \xi) \cdot a = \deriv{}{t} \bigg|_{t = 0} u \cdot \exp(t \Ad(a^{-1}) \xi) = X^P_{\Ad(a^{-1}) \xi}(u) 
\end{align*}
\end{proof}

\begin{lemma}
Horizontal spaces and connection $1$-forms are in correspondence.
\end{lemma}

\begin{proof}
Given $\{ H_u \subset T_u P \}$ satisfying the conditions, define, $\omega \in \Omega^1(P, V)$ as follows. $\forall u \in P : \forall u \in T_u P = H_u \oplus V_u$ then write $y = y^H + y^V$. Since $V \cong P \times \g$ via the fundamental vector fields then $y^V = X_\xi^P(u)$ for a unique $\xi$. Define $\omega(u)(y) = \xi$. Clearly, $\omega(X^P_\xi) = \xi$.
\bigskip\\
Now, for $u \in P, a \in G, y \in T_u P$ then in the case $y \in H_u$ we have $(\d{R_a})_u(y) \in H_{u \cdot a}$ by assumption. Therefore, by construction,
\[ \omega(u \cdot a)((\d{R_a})_u(y)) = 0 = \Ad(a^{-1})(\omega(u)(y)) \]
For the case $y \in V_u$ we have $y = X^P_\xi(u)$ for some $\xi \in \g$ and thus,
\[ \omega(u \cdot a)((\d{R_a})_u (y)) = \omega(u \cdot a)((R_a)_* X^P_\xi(u \cdot a)) = \omega(u \cdot a)(X^P_{\Ad{a^{-1}} \xi}) = \omega(X^P_{\Ad(a^{-1}) \xi})(u \cdot a) = \Ad(a^{-1}) \xi \]
Furthermore,
\[ \Ad(a^{-1}) \omega(u)(y) = \Ad(a^{-1}) \omega(u)(X^P_{\xi}(u)) = \Ad(a^{-1}) \xi \]
and the general case follows by linearity. 
\end{proof}

\section{April 1}

Consider $\omega \in \Omega^1(P, \g)$ satisfying the connection $1$-form conditions. Then let $H_u = \ker{(\omega(u) : T_u P \to \g)}$. The first property gives $\omega(u)|_{V_u} : V_u \to \g$ is a linear isomorphism so $T_u P = V_u \oplus H_u$. The second property is equivalent to,
\begin{center}
\begin{tikzcd}
T_u P \arrow[r, "\omega(u)"] \arrow[d, "(\d{R_a})_u"] & \g \arrow[d, "\Ad(a^{-1})"]
\\
T_{(ua)} P \arrow[r, "\omega(u \cdot a)"] & \g
\end{tikzcd}
\end{center}
\[ v \in H_u = \ker{\omega(u)} \iff (\d{R_a})_u(v) \in \ker{\omega(u \cdot a)} = H_{ua} \]

\subsection{?}

We now fix a principal $G$-bundle $\pi : P \to M$ of the base $M$ and local trivializations $\phi_\alpha : \pi^{-1}(U_\alpha) \to U_\alpha \times G$ which are $G$-equivariant diffeomorphisms and cross sections $\sigma_\alpha  : U_\alpha \to \pi^{-1}(U_\alpha)$ of $\pi^{-1}(U_\alpha) \to U_\alpha$ given by $\sigma_\alpha(x) = \psi_\alpha^{-1}(x, e)$ for $e \in G$. We want to consider all possible connections on this bundle. 

\begin{definition}
Let $G$ be a Lie group. Then Mauser-Cartan form is the unqiue $\g$-valued $1$-form $\theta \in \Omega^1(G, \g)$ on $G$ which is left invariant and $\theta(e) : T_e G \to \g$ is the identity i.e. $\theta(X^L_\xi = \xi$ for any $\xi \in \g$. Equivalently,
\[ \forall g \in G : \theta(g) = (\d{L_g})_{g^{-1}} : T_g P \to T_e G  \]
We use the notation $\theta = g^{-1} \d{g}$ for a general group $G$. Then,
\[ R_a^* \theta = R_a^* L_{a^{-1}}^* \theta = \Ad{(a^{-1}} \theta \]
\end{definition}

\begin{example}
Let $G = \Aut{\R^r}$ then $TG = G \times \g$ 
\end{example}

\begin{definition}
Given any connection $1$-form $\omega \in \Omega^1(P, \g)$ on a principal bundle $\pi : P \to G$ with given trivializations and sections. Define $\omega_\alpha = \sigma_\alpha^* \omega \in \Omega^1(U_\alpha, \g)$ then $\phi_\alpha^{-1}(\omega) \in \Omega^1(U_\alpha \times G, \g)$
\end{definition}

\begin{lemma}
Then,
\[ [(\psi^{-1}_\alpha)^* \omega](x,g) = \Ad{(g^{-1})} \omega_\alpha(x) + \theta(g) \]
via,
\begin{center}
\begin{tikzcd}
((\psi_\alpha^{-1})^* \omega)(x, g) : &  T_{(x,g)}(U_\alpha \times G) \arrow[r] \arrow[d, equals] & \g \arrow[d, equals]
\\
\Ad{(g^{-1})} \omega_\alpha(x) + \theta(g) : & T_x U_\alpha \oplus T_g G \arrow[r] & \g
\end{tikzcd}
\end{center}
\end{lemma}

\begin{proof}
First,
\[ ((\psi_\alpha^{-1})^* \omega)(x,g) \bigg|_{T_g G} = \theta(g) \]
and
\[ (\psi^{-1}_\alpha)^* \omega(x,e) \bigg|_{T_x U_\alpha} = (\sigma_\alpha^* \omega)(x) = \omega_\alpha(x) \]
and,
\[ (\psi_\alpha^{-1})^* \omega(x, g) \bigg|_{T_x U_\alpha} = ((R_g \circ \sigma_\alpha)^* \omega)(x) = [\sigma_\alpha^* R_g^* \omega](x) = [\sigma_\alpha^*(\Ad{(g^{-1})}) \omega](x) = \Ad{(g^{-1})} \omega_\alpha(x) \]
\end{proof}

\begin{lemma}
On $U_\alpha \cap U_\beta$ with $\omega_\alpha = \sigma_\alpha^* \omega$ and $\omega_\beta = \sigma_\beta^* \omega$. Then,
\[ \omega_\alpha = \Ad{(\psi^{-1}_{\alpha \beta})} \omega_\beta + (\psi^{-1}_{\alpha \beta}) \theta \]
\end{lemma}

\begin{proof}

\end{proof}

\begin{example}
Let $\pi : P \to M$ be a pricipal $\GL(r, F)$-bundle. Consider the fundamental representation $\rho$ given by acting on $F^r$. Then $E = P \times_\rho F^r$ is a vector bundle of rank $r$ over $M$. Then $P = \Aut{E}$. Given a connection on $P = \Aut{E}$ we define a conection $\nabla : \Omega^0(E) \to \Omega^1(E)$ on $E$ as follows. LEt $\{ U_\alpha : \alpha \in I \}$ be an open cover of $M$. Let $\sigma_\alpha(x) = (e_{\alpha, d}(x), \dots, e_{\alpha, 1}(x))$ frame of $E|_{U_\alpha} \to U_\alpha$ with $\pi : P \to M$. On $E|_{U_\alpha}$ we define,
\[ \nabla e_{\alpha,i} = \sum_{j = 1}^n e_{\alpha, j} \otimes \theta_{ji} \quad \quad \quad \omega_\alpha = \sigma_\alpha^* \omega = (\theta_{ij}) \in \Omega^1(U_\alpha, \g) \] 
\end{example}

\begin{theorem}
Given a connection of a principal bundle $\pi : P \to M$ then there is a connection on the associate vector bundle. 
\end{theorem}

\end{document}


