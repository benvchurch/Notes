\documentclass[12pt]{extarticle}
\usepackage[utf8]{inputenc}
\usepackage[english]{babel}
\usepackage[a4paper, total={7in, 10in}]{geometry}
\usepackage{amsthm, amssymb, amsmath, centernot, enumitem, soul}
\usepackage{marvosym}

\newcommand{\notimplies}{%
  \mathrel{{\ooalign{\hidewidth$\not\phantom{=}$\hidewidth\cr$\implies$}}}}
 
\renewcommand\qedsymbol{$\square$}
\newcommand{\cont}{$\boxtimes$}
\newcommand{\divides}{\mid}
\newcommand{\ndivides}{\centernot \mid}
\newcommand{\Z}{\mathbb{Z}}
\newcommand{\N}{\mathbb{N}}
\newcommand{\C}{\mathbb{C}}
\newcommand{\Zplus}{\mathbb{Z}^{+}}
\newcommand{\Primes}{\mathbb{P}}
\newcommand{\ball}[2]{B_{#1} \! \left(#2 \right)}
\newcommand{\Q}{\mathbb{Q}}
\newcommand{\R}{\mathbb{R}}
\newcommand{\Rplus}{\mathbb{R}^+}
\newcommand{\invI}[2]{#1^{-1} \left( #2 \right)}
\newcommand{\End}[1]{\text{End}\left( A \right)}
\newcommand{\legsym}[2]{\left(\frac{#1}{#2} \right)}
\renewcommand{\mod}[3]{\: #1 \equiv #2 \: \mathrm{mod} \: #3 \:}
\newcommand{\nmod}[3]{\: #1 \centernot \equiv #2 \: mod \: #3 \:}
\newcommand{\ndiv}{\hspace{-4pt}\not \divides \hspace{2pt}}
\newcommand{\finfield}[1]{\mathbb{F}_{#1}}
\newcommand{\finunits}[1]{\mathbb{F}_{#1}^{\times}}
\newcommand{\ord}[1]{\mathrm{ord}\! \left(#1 \right)}
\newcommand{\quadfield}[1]{\Q \small(\sqrt{#1} \small)}
\newcommand{\vspan}[1]{\mathrm{span}\! \left\{#1 \right\}}
\newcommand{\galgroup}[1]{Gal \small(#1 \small)}
\newcommand{\sm}{\! \setminus \!}
\newcommand{\topo}{\mathcal{T}}
\newcommand{\base}{\mathcal{B}}
\newcommand{\id}{\mathrm{id}}
\newcommand{\ket}[1]{\left| #1 \right>}
\newcommand{\inner}[2]{\left< #1 | #2 \right>}
\newcommand{\expect}[2]{\left< #1 \right| #2 \left| #1 \right>}
\renewcommand{\d}[1]{\mathrm{d}#1}
\newcommand{\deriv}[2]{\frac{\d{#1}}{\d{#2}}}
\newcommand{\pderiv}[2]{\frac{\partial{#1}}{\partial{#2}}}
\newcommand{\parsq}[2]{\frac{\partial^2{#1}}{\partial{#2}^2}}
\newcommand{\CP}{\mathbb{CP}}
\newcommand{\bat}{\text{\Bat}}
\newtheorem{theorem}{Theorem}
\newtheorem{lemma}[theorem]{Lemma}

\newcommand\norm[1]{\left\lVert#1\right\rVert}
\newenvironment{myproof}[1][\proofname]{%
  \begin{proof}[#1]$ $\par\nobreak\ignorespaces
}{%
  \end{proof}
}
\begin{document}
\pagestyle{empty}

{\bf Math 4081 HW\#1, due WEDNESDAY 1/31/18} \hspace{.5in} \textbf{NAME: Benjamin Church} \hspace{4in}
%outlier malcom
\begin{enumerate}
\item Lee 1-5 [SECOND edition] = Lee 1-7 [FIRST edition 1-7].  So either way you will have done it (since hw 1 originally read 1. Lee 1-5, 2. Lee 1-7). \\


 \textbf{ONLY DO THE CASE WHEN $n=2$.}\\

Let $N$ denote the north pole $(0,...,0,1) \in S^n \subset \R^{n+1}$   and let $S$ denote the south pole $(0,...,0,-1)$. Define the \textbf{stereographic projection} $\sigma: S^n\setminus \{N\} \to \R^n$ by  
\[
\sigma(x^1,...,x^{n+1}) = \frac{(x^1,...,x^{n})}{1-x^{n+1}}
\]
Let $\widetilde{\sigma}(x) = -\sigma(-x)$ for $x\in S^n \setminus \{ S \}$. \\
(a) For any $x\in S^n \setminus \{N\}$, show that $\sigma(x)=u$, where $(u,0)$ is the point where the line through $N$ and $x$ intersects the linear subspace where $x^{n+1}=0$ (Fig. 1.13 in LEE SECOND and FIRST). Similarly, show that  $\widetilde{\sigma}(x) $is the point where the line through $S$ and $x$ intersects the same subspace. (For this reason,  $z$ is called \textbf{stereographic projection from the south pole.}) \\

\textbf{Solution:}
The following is restricted to the case $n = 2$.
Consider the line through the points $(0, 0, 1), (x, y, z) \in S^2$ which can be parametrized by $\{ (tx, ty, (1 - t) + tz) \mid t \in \R \}$. When this intersects the equitorial plane, $(1 - t) + tz = 0$ so $t = \frac{1}{1 - z}$. Therefore, the intersection point is at, $\left(\frac{x}{1 - z}, \frac{y}{1 - z}, 0 \right)$ which equals $\sigma(x, y, z)$ with $\R^2$ identified as the equitorial plane. The case for the south pole is identical. \bigskip \\ 
Consider the line through the points $(0, 0, -1), (x, y, z) \in S^2$ which can be parametrized by $\{ (tx, ty, -(1 - t) + tz) \mid t \in \R \}$. When this intersects the equitorial plane, $-(1 - t) + tz = 0$ so $t = \frac{1}{1 + z}$. Therefore, the intersection point is at, $\left(\frac{x}{1 + z}, \frac{y}{1 + z}, 0 \right)$ which equals $\tilde{\sigma}(x, y, z)$ with $\R^2$ identified as the equitorial plane.


(b) Show that  $\sigma$ is bijective, and
 \[
 \sigma^{-1}(u^1, ... ,u^n) =\frac{(2u^1,..2u^n, |u|^2 -1)}{|u|^2+1}
 \]

\textbf{Solution:}

It suffices to show that the compositions $\sigma \circ \sigma^{-1} = \id_{\R^2}$ and $\sigma^{-1} \circ \sigma = \id_{S^2}$. This follows directly by computation, take $(x,y,z) \in S^2$ and $(u, v) \in \R^2$,
\begin{align*}
\sigma^{-1}\circ \sigma(x,y,z) &= \sigma^{-1}  \left(\frac{(x,y)}{1-z}\right) = \frac{\left(\frac{2x}{1-z},\frac{2y}{1-z}, \left(\frac{x}{1-z}\right)^2+\left(\frac{y}{1-z}\right)^2 -1\right)}{\left(\frac{x}{1-z}\right)^2+\left(\frac{y}{1-z}\right)^2 +1}
\\
&=\frac{\left(2x(1-z),2y(1-z),x^2+y^2-(1-z)^2\right)}{x^2+y^2+(1-z)^2}
\\
&=\frac{\left(2x(1-z),2y(1-z),-2z^2+2z\right)}{2(1-z) + (x^2 + y^2 + z^2 - 1)} = \left(x, y, z\right) = \id_{S^2}(x, y, z)
\\
\sigma\circ \invI{\sigma}{u, v} &= \sigma\left(\frac{(2u,2v, u^2+v^2 -1)}{u^2 + v^2 + 1}\right) = \frac{(2u,2v)}{(u^2 + v^2 + 1)\left(1-\frac{u^2 + v^2 - 1}{u^2 + v^2 + 1}\right)}\\
&=\frac{(2u, 2v)(u^2 + v^2 + 1)}{2(u^2 + v^2 + 1)} = (u, v) = \id_{\R^2}(u, v)
\end{align*}

(c) Compute the transition map  $\widetilde{\sigma} \circ \sigma^{-1}$ and verify that the atlas consisting of the two charts $(S^n\setminus \{N\}, \sigma)$ and $(S^n\setminus \{S\}, \widetilde{\sigma})$ defines a smooth structure on $S^n$. (The coordinates defined by  $\sigma$ or $\tilde{\sigma}$ are called stereographic coordinates.) \\

\textbf{Solution:}
Take $(x, y) \in \R^2$,
\begin{align*}
\tilde{\sigma} \circ \sigma^{-1}(x, y) & = -\sigma(-\sigma^{-1}(x, y)) = -\frac{\left(-2x, -2y \right)}{(x^2+y^2 + 1) \left(1-\frac{x^2+y^2-1}{x^2+y^2+1} \right)} \\
&  = \frac{2(x,y)}{x^2+y^2+1-(x^2+y^2-1)} = (x,y)
\end{align*}
is a diffeomorphism. Therefore, $(S^n\setminus{\{N\}},\sigma)$ and $(S^n\setminus{\{S\}},\widetilde{\sigma})$ are smoothly compatible charts and thus there exists a unique maximal atlas containing these charts which defines a smooth structure on $S^2$. 

(d) Show that this smooth structure is the same as the one defined in Example 1.31 from LEE SECOND or Example 1.20 in LEE FIRST.

\textbf{Solution:}

It suffices to show that $\sigma$ and $\widetilde{\sigma}$ are smoothly compatible with $\phi^{\pm}_i$, because then the union of the smooth atlases is a smooth atlas which implies that the maximal atlases defined by the two sets of charts are in fact equal.\\
Consider $\phi_i^\pm \circ\sigma^{-1}:\sigma(U_i^\pm\cap \left(S^2\setminus {\{N\}}\right))\rightarrow \phi_i^\pm(U_i^\pm\cap \left(S^2\setminus{\{N\}}\right))$. Take $(x, y) \in \R^2$ then, 
\[\phi_1^\pm \circ\sigma^{-1}(x, y) = \phi^{\pm}_1 \left(\frac{\left(2x, 2y, x^2+y^2-1\right)}{x^2+y^2+1} \right) = \frac{(2x,-z^2)}{x^2+y^2+1}\]

\[\phi_2^\pm \circ\sigma^{-1}(x, y) = \phi^{\pm}_2 \left(\frac{\left(2x, 2y, x^2+y^2-1\right)}{x^2+y^2+1} \right) = \frac{(2y,-z^2)}{x^2+y^2+1}\]

\[\phi_2^\pm \circ\sigma^{-1}(x, y) = \phi^{\pm}_2 \left(\frac{\left(2x, 2y, x^2+y^2-1\right)}{x^2+y^2+1} \right) = \frac{(2x, 2y)}{x^2+y^2+1}\]
which are all diffeomorphic by basic analysis. \bigskip \\
Similarly, for  $\phi_i^\pm \circ\widetilde{\sigma}^{-1}:\widetilde{\sigma}(U_i^\pm\cap \left(S^n\setminus {S}\right))\rightarrow \phi_i^\pm(U_i^\pm\cap \left(S^n\setminus{S}\right))$. 
Take $(x, y) \in \R^2$ then, 
\[\phi_1^\pm \circ \tilde{\sigma}^{-1}(x,y) = \phi^{\pm}_1 \left(\frac{\left(2x, 2y, 1 - x^2 - y^2\right)}{x^2+y^2+1} \right) = \frac{(2x, z^2)}{x^2+y^2+1}\]

\[\phi_2^\pm \circ \tilde{\sigma}^{-1}(x,y) = \phi^{\pm}_2 \left(\frac{\left(2x, 2y, 1 - x^2 - y^2\right)}{x^2+y^2+1} \right) = \frac{(2y, z^2)}{x^2+y^2+1}\]

\[\phi_2^\pm \circ \tilde{\sigma}^{-1}(x,y) = \phi^{\pm}_2 \left(\frac{\left(2x, 2y, 1 - x^2 - y^2\right)}{x^2+y^2+1} \right) = \frac{(2x, 2y)}{x^2+y^2+1}\]
which are all diffeomorphic by basic analysis.

\vfill
\eject




\item Lee 1.6 [SECOND EDITION] (probably more interesting than the first option) \\ Let $M$ be a nonempty topological manifold of dimension $n \geq 1$. If $M$ has a smooth structure, show that it has uncountably many distinct ones. [Hint: first show that for any $s > 0, \ F_s(x) = |x|^{s-1}x $ defines a homeomorphism from $B^n$ to itself, which is a diffeomorphism if and only if $s =1$.]

\textbf{Solution:}
Let $M$ be a nonempty smooth $n$-manifold for $n \ge 1$ with some smooth atlas $\mathcal{A}$ and pick a point $p \in M$. First, we will need a lemma:

\begin{lemma}
For any $p \in M$ there exists a smooth atlas $\mathcal{A'}$ such that $p$ is contained in the domain of exactly one chart in $\mathcal{A}'$ and both $\mathcal{A}$ and $\mathcal{A}'$ define the same smooth structure on $M$. 
\end{lemma}

\begin{proof}
Let $\mathcal{C}_p \subset \mathcal{A}$ be the set $\mathcal{C}_p = \{ (U, \phi) \in \mathcal{A} \mid p \in U\}$. Because $M$ is Hausdorff, the sigleton $\{p\}$ is closed so the sets $U \sm \{p\}$ are open. Define $\mathcal{A'}$ by replaceing each $(U, \phi) \in \mathcal{C}_p$ with $(U \sm \{p\}, \phi|_{U \sm \{p\}} )$ which is still a chart because $U \sm \{p\}$ is open and the restiction of any homeomorphism is still a homeomorphism onto its image. Clearly, $\mathcal{A'}$ is a smooth atlas because the transition maps are simply restrictions of the transition maps of $\mathcal{A}$ which are smooth. To prove that $\mathcal{A}$ and $\mathcal{A'}$ define the same smooth structure, it suffices to show that their union is a smooth atlas. However, the transition maps between $\mathcal{A}$ and $\mathcal{A}'$ are also restrictions of smooth maps and are therefore smooth.   
\end{proof}

By the previous lemma, we can assume that $\mathcal{A}$ has exactly one chart containing $p$, namely, $(U_p, \phi_p)$. We will construct a new atlas $\mathcal{B}_s$ by replacing $(U_p, \phi_p)$ with $(U_p, \phi_s)$ defined as,
\[ \phi_s(x) = F_s (\phi_p(x) - \phi_p(p)) \]
For any $s > 0$, I claim that $\mathcal{B}_s$ is a smooth atlas. This is because for any other chart $(V, \psi) \in \mathcal{B}_s$ we know $ p \notin U_p \cap V$ so the map,
\[ \psi \circ \phi_s^{-1} : \phi_s(U \cap V_p) \to \psi(U \cap V_p) \]
cannot contain $0$ in its domain. This is because $\phi_s(p) = 0$ but $\phi_s$ is an injection and $p \notin U \cap V_p$ so $0$ is not in the image unde $\phi_s'$. I claim that $F_s$ is a diffeomorphism on any set not containing $0$. I will show this by exhibting an inverse,
\[ F_s^{-1}(x) = \frac{x}{|x|^{1 - \frac{1}{s}}}\] 
which is well defined ($s > 0$) and smooth as long as $x \neq 0$. Furthermore, 
\[ F_s \circ F_s^{-1}(x) = \frac{|x|^{s - 1}}{|x|^{(1 - \frac{1}{s})(s - 1)}} \frac{x}{|x|^{1 - \frac{1}{s}}} = x \quad \quad F_s^{-1} \circ F_s(x) = \frac{|x|^{s - 1} x}{|x|^{s(1 - \frac{1}{s})}} = x\] 
Thus, $F_s$ is a diffeomorphism away from $x = 0$. Therefore, 
\[\psi \circ \phi_s^{-1}(x) = \psi \circ \phi_p^{-1} \circ (F_s^{-1}(x) + \phi_p(p))\] 
is a diffeomorphism because $x \neq 0$. Furthermore, the transition map of any two charts in $\mathcal{B}_s$ neither of which are $(U_p, \phi_s)$ are smooth because they are also charts of $\mathcal{A}$ which is a smooth atlas. \bigskip \\
Finally, we need to show that $\mathcal{B}_s$ and $\mathcal{B}_{s'}$ define different smooth structures on $M$ if $s \neq s'$. If $\mathcal{B}_s$ and $\mathcal{B}_{s'}$ defined the same smooth structure then their union would be a smooth atlas. However,
\[ \phi_s' \circ \phi_s^{-1}(x) = F_{s'}(\phi_p(\phi_p^{-1}(F_s^{-1}(x) + \phi_p(p)) - \phi_p(p)) = F_{s'}(F_s^{-1}(x))\] 
which is not smooth unless $s = s'$. This holds because,
\[F_{s'} \circ F_s^{-1}(x) = \frac{|x|^{s' - 1}}{|x|^{(1 - \frac{1}{s})(s' - 1)}} \frac{|x|^{s'-1}}{|x|^{s' - \frac{s'}{s}}} x = |x|^{\frac{s'}{s} - 1} x\]
which is not differentiable at $x = 0$ unless $s = s'$. Thus, the charts of $\mathcal{B}_s$ and $\mathcal{B}_{s'}$ are not smoothly compatable. Therefore, there is a distinct smooth structure on $M$ for each $s \in (0, \infty)$ which is an uncountable set.    

\bigskip


\vfill
\eject

\item DO ONE OF the following:

Lee 1.7 [First Edition] = Lee 1.9 [Second Edition] \\

\textbf{  (Just do the case $n=1$.) Also check that the projection $\C^{2}\setminus\{0\}\to\C P^1$ is smooth.} \\ 

\textbf{ Complex projective $n$-space}, denoted by $CP^n$, is the set of all 1-dimensional complex-linear subspaces of $C^{n+1}$, with the quotient topology inherited from the natural projection  $\pi: \C^{n+1} \setminus \{ 0 \} \to CP^n$.  Show that $CP^n$ is a compact 2n-dimensional topological manifold, and show how to give it a smooth structure analogous to the one we constructed for $RP^n$ (done in an earlier example in Chapter 1 in both versions). We use the correspondence
 \[
(x^i + iy^1,...,x^{n+1}+iy^{n+1}) \leftrightarrow (x^1, y^1,... x^{n+1}, y^{n+1})
  \]
  to identify $\C^{n+1}$ with $\R^{2n+2}$
  
\textbf{Solution:}

We first show that $\CP^1$ is locally euclidean.
Notation: $\mathcal{C} : \R^2 \to \C$ defined by $x + iy \mapsto (x, y)$.

I will use the charts on $\CP^1$ defined by the charts,
\begin{align*}
(\CP^1 \sm \{[1, 0]\}, \phi_1) \quad \text{with} \quad & \phi_1 : [x + i y, u + i v] \mapsto \mathcal{C}\left(\frac{x + iy}{u + iv}\right)
\\
& \phi_1^{-1} : (x, y) \mapsto [x + i y, 1]
\\
(\CP^1 \sm \{[0, 1]\}, \phi_2) \quad \text{with} \quad & \phi_2 : [x + i y, u + i v] \mapsto \mathcal{C}\left(\frac{u + iv}{x + iy}\right)
\\
& \phi_2^{-1} : (x, y) \mapsto [1, x + i y]
\end{align*}
We note that these are, in fact, inverses because. We have,
\[\phi_1^{-1}\circ \phi_1([x + iy, u + i v]) = \phi_1^{-1}\left(\frac{x + i y}{u + i v}\right) = \left[\frac{x + iy}{u + iv},1\right] \cong [x + iy, u + iv]\]
which is well defined because $u + i v \neq 0$ on the domain of $\phi_1$. Similarly,
\[\phi_1\circ\phi_1^{-1}\left(x, y\right) = \phi_1[x + iy,1] = \mathcal{C} \left(\frac{x + iy}{1}\right) = (x, y)\]
Likewise,
\[\phi_2^{-1}\circ \phi_2([u + iv, x + i y]) = \phi_2^{-1}\left(\frac{u + i v}{x + i y}\right) = \left[1, \frac{u + iv}{x + iy}\right] \cong [x + iy, u + iv]\]
which is well defined because $x + i y \neq 0$ on the domain of $\phi_2$. Similarly,
\[\phi_2 \circ \phi_2^{-1} \left(x, y\right) = \phi_2[1, x + iy] = \mathcal{C} \left(\frac{x + iy}{1}\right) = (x, y)\]
Consider the maps from $\C^2 \to \R^2$,
\[ \phi_1 \circ \pi(x + i y, u + i v) = \phi_1([x + i y, u + i v]) = \mathcal{C}\left(\frac{x + iy}{u + iv}\right) \]
and 
\[ \phi_2 \circ \pi(x + i y, u + i v) = \phi_2([x + i y, u + i v]) = \mathcal{C}\left(\frac{u + iv}{x + iy}\right) \]
which are continuous because the domains are restricted to the sets where these denominators are nonzero. By the properties of quotient maps, $\phi_1$ and $\phi_2$ must be continuous. Similarly, the inverses are easily seen to be continuous. Therefore $\phi_1$ and $\phi_2$ are homeomorphisms onto their image. The Hausdorff and second countable properties are easily checked via the identification with $\R^4$. Therefore $\CP^1$ is a 2-manifold. \bigskip \\
It remains to show that the atlas given by the charts $\phi_1$ and $\phi_2$ defines a smooth structure on $\CP^1$. Consider the transition map, 
\[ \phi_1 \circ \phi_2^{-1} : \phi_2(U_1 \cap U_2) \to \phi_1(U_1 \cap U_2) \]
where $U_1 \cap U_2 = \CP^1 \sm \{[1,0], [0,1]\}$. Now, consider, $(x,y) \in \phi_2(U_1 \cap U_2)$ then $(x, y) \neq 0$ since $\phi_2([1, 0]) = (0, 0) \notin \phi_2(U_1)$ and,
\[ \phi_1 \circ \phi_2^{-1}(x, y) = \phi_1([1, x + i y]) = \mathcal{C}\left(\frac{1}{x + i y} \right) = \left(\frac{x}{x^2 + y^2}, \frac{-y}{x^2 + y^2} \right)\]
which is smooth in any region of the plane minus the origin. Similarly, the opposite transition map,
\[ \phi_2 \circ \phi_1^{-1} : \phi_1(U_1 \cap U_2) \to \phi_2(U_1 \cap U_2) \]
and take $(x,y) \in \phi_1(U_1 \cap U_2)$ so $(x, y) \neq 0$ since $\phi_1([0, 1]) = (0, 0) \notin \phi_1(U_2)$. Then,
\[ \phi_2 \circ \phi_1^{-1}(x, y) = \phi_1([x + i y, 1]) = \mathcal{C}\left(\frac{1}{x + i y} \right) = \left(\frac{x}{x^2 + y^2}, \frac{-y}{x^2 + y^2} \right)\]
which is likewise smooth in any region of the plane minus the origin. Therefore, the maps $\phi_1$ and $\phi_2$ are smoothly compatable so they define a smooth structure on $\CP^1$. Under this smooth stucture, we will show that $\CP^1$ is diffeomorphic to $S^2$ and thus homeomorphic because any smooth map is continuous. Therefore, $\CP^1$ is compact because $S^2$ is compact. \bigskip \\
Finally, consider the projection map $\pi : \C^2 \sm \{0\} \to \CP^1$. The chart, $\psi : \C^2 \sm \{0\} \to \R^4$ given by,
\[ \psi : (x + iy, u + iv) \to (x, y, u, v)\] 
is obviously smoothly compatable with itself and therefore defines a smooth structure on $\C^2 \sm \{0\}$. Therefore, the coordinate representation of the projection, 
\begin{align*}
\phi_1 \circ \pi \circ \psi^{-1}(x, y, u, v) = \phi_1([x + iy, u + iv]) = \mathcal{C}\left(\frac{x + iy}{u + iv}\right)
\end{align*} 
which is smooth on a domain in which $(u, v) \neq 0$. For a point such that $u = v = 0$, the image of the projection is not within the domain of $\phi_1$ so we must use the other chart,
\begin{align*}
\phi_2 \circ \pi \circ \psi^{-1}(x, y, u, v) = \phi_1([x + iy, u + iv]) = \mathcal{C}\left(\frac{u + iv}{x + iy}\right)
\end{align*}
which is smooth on a domain in which $(x, y) \neq 0$. Therefore, one of these coordinate representations is well defined and smooth whenever $(x, y, u, v) \neq 0$ which always holds on $\C^2 \sm \{0\}$. Thus, $\pi$ is a smooth map.

\vfill
\eject

\item Show that $\C P^1$ is diffeomorphic to $S^2$.

\textbf{Solution:}

Notation: $\mathcal{C} : \R^2 \to \C$ defined by $x + iy \mapsto (x, y)$.

I will use the atlas on $\CP^1$ defined by the charts,
\begin{align*}
(\CP^1 \sm \{[1, 0]\}, \phi_1) \quad \text{with} \quad & \phi_1 : [x + i y, u + i v] \mapsto \mathcal{C}\left(\frac{x + iy}{u + iv}\right)
\\
& \phi_1^{-1} : (x, y) \mapsto [x + i y, 1]
\\
(\CP^1 \sm \{[0, 1]\}, \phi_2) \quad \text{with} \quad & \phi_2 : [x + i y, u + i v] \mapsto \mathcal{C}\left(\frac{u + iv}{x + iy}\right)
\\
& \phi_2^{-1} : (x, y) \mapsto [1, x + i y]
\end{align*}
and the atlas on $S^1$ given by the steriographic projections defined above,  
\begin{align*}
(S^2 \sm \{(0,0,1)\}, \sigma) \quad \text{with} \quad & \sigma : (x, y, z) \mapsto \left(\frac{x}{1 - z}, \frac{y}{1 - z}\right)
\\
& \sigma^{-1} : (x, y) \mapsto \left(\frac{2x}{x^2 + y^2 + 1}, \frac{2y}{x^2 + y^2 + 1}, \frac{x^2 + y^2 - 1}{x^2 + y^2 + 1}\right)
\\
(S^2 \sm \{(0,0,-1)\}, \tilde{\sigma}) \quad \text{with} \quad & \tilde{\sigma} : (x, y, z) \mapsto \left(\frac{x}{1 + z}, \frac{y}{1 + z}\right)
\\
& \sigma^{-1} : (x, y) \mapsto \left(\frac{2x}{x^2 + y^2 + 1}, \frac{2y}{x^2 + y^2 + 1}, \frac{1 - x^2 - y^2}{x^2 + y^2 + 1}\right)
\end{align*}

Now, I define the map $F : S^2 \to \CP^1$ by,
\[ F : (x, y, z) \mapsto
\begin{cases}
\left[\frac{x + iy}{1 - z}, 1 \right] & z \neq 1 \\
[1, 0] & z = 1
\end{cases}
\]

It remains to check that $F$ is bijective, smooth, and has smooth inverse. First, we show that $F$ is a bijection by exhibiting an inverse function,
\[ F^{-1} : [x + iy, u + iv] \mapsto
\begin{cases}
\left(\frac{2 \alpha}{\alpha^2 + \beta^2 + 1}, \frac{2 \beta}{\alpha^2 + \beta^2 + 1}, \frac{\alpha^2 + \beta^2 - 1}{\alpha^2 + \beta^2 + 1} \right) & u + iv \neq 0\\
(0, 0, 1) & u + iv = 0
\end{cases}
\]
where $\alpha + i \beta = \frac{x + iy}{u + i v}$ which is well defined in $\CP^1$ because if $x + iy$ and $u + iv$ are scalled by the same nonzero complex number then their ratio $\alpha + i \beta$ remains constant. The following calculation shows that $F$ is a bijection,
\begin{align*}
F \circ F^{-1}([x + iy, u + iv]) & = 
\begin{cases}
F\left(\frac{2 \alpha}{\alpha^2 + \beta^2 + 1}, \frac{2 \beta}{\alpha^2 + \beta^2 + 1}, \frac{\alpha^2 + \beta^2 - 1}{\alpha^2 + \beta^2 + 1} \right) & u + iv \neq 0
\\
F(0, 0, 1) & u + iv = 0 
\end{cases}
\\
& = 
\begin{cases}
\left[ \frac{2 \alpha + 2 i \beta}{\alpha^2 + \beta^1 + 1 - (\alpha^2 + \beta^2 + 1)}, 1 \right]  & u + iv \neq 0
\\
[1, 0] & u + iv = 0 
\end{cases}
\\
& =  
\begin{cases}
\left[ \alpha + i \beta, 1 \right]  & u + iv \neq 0
\\
[1, 0] & u + iv = 0 
\end{cases}
\end{align*}
However, if $u + i v \neq 0$ then $[x + iy, u + iv] \sim [\alpha + i \beta, 1]$ and otherwise $[x + i y, 1] \sim [1, 0]$. Therefore, $F \circ F^{-1} = \id_{\CP^1}$. Likewise,
\begin{align*}
F^{-1} \circ F(x, y, z) & =
\begin{cases}
F^{-1}\left(\left[ \frac{x + i y}{1 - z}, 1 \right] \right) & z \neq 1 \\
F^{-1}([1, 0]) & z = 1
\end{cases}
\\ 
& = 
\begin{cases}
\left( \frac{2x ( 1 - z)}{x^2 + y^2 + (1 - z)^2}, \frac{2y ( 1- z)}{x^2 + y^2 + (1 - z)^2}, \frac{x^2 + y^2 - (1 - z)^2}{x^2 + y^2 + (1 - z)^2} \right) & z \neq 1 \\
(0, 0, 1) & z = 1
\end{cases}
\\
& = 
\begin{cases}
\left( x, y, z \right) & z \neq 1 \\
(0, 0, 1) & z = 1
\end{cases}
\end{align*}
where the last line follows because $x^2 + y^2 + z^2 = 1$. Therefore $F^{-1} \circ F = \id_{S^2}$. Now, we must check that $F$ and $F^{-1}$ are smooth. The charts $\sigma, \tilde{\sigma}$ cover $S^2$ so we need to show that some choice of chart on $\CP^1$ makes the coordinate representation smooth. Consider,
\begin{align*}
\phi_1 \circ F \circ \sigma^{-1}(x, y) = \phi_1 \circ F \left(\frac{2x}{x^2 + y^2 + 1}, \frac{2y}{x^2 + y^2 + 1}, \frac{x^2 + y^2 - 1}{x^2 + y^2 + 1}\right) = \phi_1([x + i y, 1]) = (x, y)
\end{align*} 
This map is a diffeomorphism $\R^2 \to \R^2$. Because $z = \frac{x^2 + y^2 - 1}{x^2 + y^2 + 1} \neq 1$ we have that the domain of $\sigma$ is mapped to within the domain of $\phi_1$. Similarly,
\begin{align*}
\phi_2 \circ F \circ \tilde{\sigma}^{-1}(x, y) & = \phi_2 \circ F \left(\frac{2x}{x^2 + y^2 + 1}, \frac{2y}{x^2 + y^2 + 1}, \frac{1 - x^2 - y^2}{x^2 + y^2 + 1}\right) 
\\
& = 
\begin{cases}
\phi_2 \left(\left[ \frac{2 x + 2 i y}{x^2 + y^2 + 1 - (1 - x^2 - y^2)}, 1 \right] \right) & (x, y) \neq 0 \\
\phi_2([1, 0]) & x = y = 0
\end{cases}
\\
& =
\begin{cases}
\phi_2 \left(\left[ \frac{x + i y}{x^2 + y^2}, 1 \right] \right) & (x, y) \neq 0 \\
\phi_2([1, 0]) & x = y = 0
\end{cases}
\\
& =
\begin{cases}
(x, -y) & (x, y) \neq 0 \\
(0, 0) & x = y = 0
\end{cases}
\\
& = (x, -y)
\end{align*} 
This map is a diffeomorphism $\R^2 \to \R^2$. Because $F(x,y,z) = [0, 1]$ only when $(x,y,z) = (0 ,0, -1)$ which is not in the domain of $\tilde{\sigma}$. Therefore, $F$ maps the domain of $\tilde{\sigma}$ to inside the domain of $\phi_2$. Because every point is in one of $S^2$ is contained in one of these domains, $F$ is a smooth map. In fact, the coordinate representations of $F$ are diffeomorphisms and $F$ is a bijection so immediately, the coordinate maps, $(\phi_1 \circ F \circ \sigma^{-1})^{-1} = \sigma \circ F^{-1} \circ \phi_1^{-1}$ and $(\phi_2 \circ F \circ \tilde{\sigma}^{-1})^{-1} = \tilde{\sigma} \circ F^{-1} \circ \phi_2^{-1}$ are also smooth (since they are the inverses of diffeomorphisms). Thus, $F$ is a diffeomorphism so $\CP^1 \cong S^2$. 
 

\vfill
\eject

\item Consider spherical coordinates on $\R^3$ (not including the line $x=y=0$) $\rho,\phi,\theta$ defined in terms of the Euclidean coordinates $x,y,z$ by
\[
x = \rho\sin\phi\cos\theta, \quad y = \rho\sin\phi\sin\theta, \quad z=\rho\cos\phi.
\]
\begin{enumerate}
\item
Express $\partial/\partial \rho$, $\partial/\partial\phi$, and $\partial/\partial\theta$ as linear combinations of $\partial/\partial x$, $\partial/\partial y$, and $\partial/\partial z$. \\
(The coefficients in these linear combinations will be functions on $\R^3\setminus(x=y=0)$.)

\textbf{Solution:}

For any function $f(x,y,z)$ on the set $\R^3 \setminus \{0\}$ we find that,
\begin{align*}
\pderiv{f(x,y,z)}{\rho} &= \pderiv{x}{\rho} \pderiv{f}{x} + \pderiv{y}{\rho} \pderiv{f}{y} + \pderiv{z}{\rho} \pderiv{f}{z} 
\\
&=  \sin{\phi} \cos{\theta} \pderiv{f}{x} + \sin{\phi} \sin{\theta} \pderiv{f}{y} + \cos{\phi} \pderiv{f}{z}
\\
\pderiv{f(x,y,z)}{\phi} &= \pderiv{x}{\phi} \pderiv{f}{x} + \pderiv{y}{\phi} \pderiv{f}{y} + \pderiv{z}{\phi} \pderiv{f}{z} 
\\
&= \rho \cos{\phi} \cos{\theta} \pderiv{f}{x} + \rho \cos{\phi} \sin{\theta} \pderiv{f}{y} + 0 \cdot \pderiv{f}{z}
\\
\pderiv{f(x,y,z)}{\theta} &= \pderiv{x}{\theta} \pderiv{f}{x} + \pderiv{y}{\theta} \pderiv{f}{y} + \pderiv{z}{\theta} \pderiv{f}{z} 
\\
&= (-\rho \sin{\phi} \sin{\theta}) \pderiv{f}{x} + \rho \sin{\phi} \cos {\theta}  \pderiv{f}{y} + \rho \sin{\phi} \cos{\theta} (-\sin{\theta}) \pderiv{f}{z}
\end{align*}
Therefore, 
\begin{align*}
\pderiv{}{\rho} &=  \sin{\phi} \cos{\theta} \pderiv{f}{x} + \sin{\phi} \sin{\theta} \pderiv{f}{y} + \cos{\phi} \pderiv{f}{z}
\\
\pderiv{}{\phi} &= \rho \cos{\phi} \cos{\theta} \pderiv{}{x} + \rho \cos{\phi} \sin{\theta} \pderiv{}{y}
\\
\pderiv{}{\theta} &= (-\rho \sin{\phi} \sin{\theta}) \pderiv{}{x} + \rho \sin{\phi} \cos {\theta}  \pderiv{}{y} + \rho \sin{\phi} \cos{\theta} (-\sin{\theta}) \pderiv{}{z}
\end{align*}

\item
Express $d\rho$, $d\phi$, and $d\theta$ as linear combinations of $dx$, $dy$, and $dz$.

\textbf{Solution:}

Note that $\rho^2 = x^2+y^2 + z^2$ so, 
\[2\rho \d{\rho} = 2 x \d{x} + 2 y \d{y} + 2 z \d{z} \implies \d{\rho} = \frac{x \d{x} +  y \d{y} + z \d{z}}{\d{\rho}} = \frac{x \d{x} + y \d{y} + x \d{z}}{\sqrt{x^2 + y^2 + z^2}}\]
Similarly, $z = \rho \cos\phi$ and $\sin{\phi} = \sqrt{\frac{x^2 + y^2}{x^2 + y^2 + z^2}}$. Therefore,
\begin{align*}
\d{z} &= - \d{\phi} \rho \sin{\phi} + \d{\rho} \cos{\phi}
\\
\d{\phi} & = \frac{\frac{z}{\rho} - \d{\rho}}{\sqrt{x^2 + y^2}} = \frac{z \frac{x \d{x} + y \d{y} + z \d{z}}{x^2 + y^2 + z^2} - \d{z}}{\sqrt{x^2 + y^2}}
\\
& = \frac{zx \d{x} + zy \d{y} - (x^2 + y^2) \d{z}}{(x^2 + y^2 + z^2)\sqrt{x^2 + y^2}} 
\end{align*}
Finally, using $\tan{\theta} = \frac{y}{x}$ and differentiating both sides, 
\begin{align*}
\d{y} & = x \sec^2{\theta} d{\theta} + \d{x} \tan{\theta} 
\\
d{\theta} & = \frac{\d{y} - \d{x} \tan{\theta}}{x\sec^2{\theta}} = \frac{\sin^2{\theta}}{x} \left(\d{y} - \d{x} \tan{\theta} \right) = \frac{x}{x^2 + y^2} \left(\d{y} - \d{x} \frac{y}{x} \right)
\\
& = \frac{x \d{y} - y \d{x}}{x^2 + y^2}
\end{align*}

\end{enumerate}

\vfill
\eject

\item Let $V$ and $W$ be finite dimensional vector spaces and let $A:V\to W$ be a linear map. Show that the dual map $A^*:W^*\to V^*$ is given in coordinates as follows. Let $\{e_i\}$ and $\{f_j\}$ be bases for $V$ and $W$, and let $\{e^i\}$ and $\{f^j\}$ be the corresponding dual bases for $V^*$ and $W^*$. 
If $Ae_i=A^j_if_j$ then $A^*f^j=A^j_ie^i$.

\textbf{Solution:}

Suppose that $Ae_i = A^j_i f_j$ then, $A^* f^j$ is a linear functional on $V$ such that, 
\[(A^* f^j) (e_k) = f^j(Ae_k) = f^j(A^r_i f_r) = A^r_i f^j(f_r) = A^r_i \delta^j_r = A^j_i\]
However, $A^* f^j$ can be expressed in the dual basis, $A^* f^j = C^j_i e^i$ and $C^j_i e^i(e_k) = C^j_i \delta^i_k = C^j_k$. Thus, $A^*f^j = A^j_i e^i$.  

\vfill
\eject

\item Let $V$ be a finite dimensional vector space and let $\langle\cdot,\cdot\rangle$ be an inner product on $V$. The inner product determines an isomorphism $\phi: V\to V^*$. 
\begin{enumerate}
\item
Show that the isomorphism $\phi$ is given in coordinates as follows. Let $\{e_i\}$ be a basis for $V$, let $\{e^i\}$ be the dual basis, and write $g_{ij}=\langle e_i,e_j\rangle$. Then $\phi(e_i)=g_{ij}e^j$. 

\textbf{Solution:}

Let the isomorphism $\phi : V \to V^*$ be given by $\phi(v) \mapsto \left<v, \cdot \right>$. Then, 
\[\phi(e_i) (e_k) = \left<e_i, e_k \right> = g_{ik}\]
however, $\phi(e_i) \in V^*$ so $\phi(e_i)$ can be expressed in terms of the dual basis $\phi(e_i) = C_{ij} e^j$ and $C_{ij} e^j(e_k) = C_{ij} \delta^j_k = C_{ik}$ so $C_{ik} = g_{ik}$. Therefore, $\phi(e_i) = g_{ij} e^j$.  

\item
The inner product, together with the isomorphism $\phi$, define an inner product on $V^*$. Write this in coordinates as $g^{ij}=\langle e^i,e^j\rangle$. Show that the matrix $(g^{ij})$ is the inverse of the matrix $(g_{ij})$.

\textbf{Solution:}

Given the inner product $\left< \cdot, \cdot \right> : V \times V \to \R$ and the isomorphism $\phi : V \to V^*$, we can define an inner product on the dual space by, $\left< u, w \right> = \left< \phi^{-1}(u), \phi^{-1}(w) \right>$ for $u, w \in V_*$. Now, define the upper components by $g^{ij} = \left<e^i, e^j\right>$. Consider, 
\[\phi^{-1}(e^i) = C^{ij} e_j \implies \phi(C^{ij} e_j) = C^{ij} \phi(e_j) = C^{ij} g_{jk} e^k \implies C^{ij} g_{jk} = \delta^i_k\]
Thus $C = g^{-1}$ but $g$ is symmetric because,
\[ g_{ij} = \left<e_i, e_j \right> = \left<e_j, e_i \right> = g_ji\]
so $C$ is also symmetric and $g_{ij} C^{jk} = \delta_i^j$. 
Now, define $g^{ij} = \left< e^i, e^j \right>$ then,
\begin{align*}
g_{ij} g^{jk} & = g_{ij} \left< e^j, e^k \right> = \left< g_{ij} e^j, e^k \right> = \left< \phi(e_i), e^k \right> = \left< e_i, \phi^{-1}(e^k) \right> = \left< e_i, C^{k l}e_l \right> 
\\
& = \left< e_i, e_l \right> C^{k l} = g_{il} C^{kl} = \delta^i_k
\\
g^{ij} g_{jk} & = \left< e^i, e^j \right> g_{jk} = \left< e^i, g_{jk} e^j \right> = \left< e^i, \phi(e_k) \right> = \left< \phi^{-1}(e_i), e_k \right> = \left< C^{i l} e_l, e_k \right> = 
\\
& = C^{i l} \left< e_l, e_k \right>= C^{il} g_{lk} = \delta^i_k
\end{align*}
Therefore, $g^{ij}$ is the inverse matrix of $g_{ij}$. In particular, $C^{ij} = g^{ij}$.

\end{enumerate}




\vfill
\eject


\item How difficult was this assignment?  How many hours did you spend on it?
\bigskip \\
I would not say this assignment was exactly difficult. It was time consuming and at times tedious but the ideas were not too difficult. Rather, I got bogged down in computations and notation. I spent about 7 - 8 hours on it.   

\vfill
\eject

\end{enumerate}

\end{document}



