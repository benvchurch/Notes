\documentclass[12pt]{article}
\usepackage{import}
\import{../}{GeometryCommands}


\begin{document}


\section{Tensor Densities}

\begin{defn}
For any vector bundle $\pi : E \to M$ there is an associated frame bundle $F(E)$. For $s \in \R$, consider the homomorphism $\rho_s : \GL{n,\R} \to \R^\times$ via $A \mapsto |\det{A}|^{-s}$ which gives a $1$-dimensional representation. Then the \textit{bundle of $E$ densities} is the line bundle,
\[ D^s(E) = F(E) \times_{\rho_s} \R \]
Because $\rho$ has image inside $\R^+$ and therefore $\rho$ gives an action on $\R^+$ by multiplication. Therefore, there is a canonical principal $\R^+$-bundle inside $D^s(E)$,
\[ D_+^s(E) = F(E) \times_{\rho_s} \R^+ \embed F(E) \times_{\rho_s} \R = D^s(E) \]
corresponding to the positive densities. This is a restriction of the structue group of $D^s(E)$ from $\R^\times$ to $\R^+$ which corresponds to putting an orientation on $D^s(E)$ and thus a notion of positivity.
\end{defn}

\begin{prop}
The line bundle $D^s(E)$ is always trivial.
\end{prop}

\begin{proof}
The subbundle $D^s_+(E) \embed D^s(E)$ gives an orientation of $D^s(E)$ and thus $D^s(E)$ is trivial. Furthermore, all principal $\R^+$-bundles are trivial and thus $D^s_+(E)$ has a section which gives a nonvanishing section of $D^s(E)$ that trivializes it.
\end{proof}

\begin{rmk}
Notice: there does not exist a canonical trivialization of $D^s(E)$. Here we recall some cohomological arguments for the structure of the torsors in question. There is a morphism of exact sequences of abelian sheaves on $M$ where the second row is $\exp : \struct{M} \to \struct{M}^\times$ replaced by its image,
\begin{center}
\begin{tikzcd}
0 \arrow[r] & \struct{M} \arrow[d, "\exp"']  \arrow[r, "\exp"] & \struct{M}^\times \arrow[r] & \underline{\{ \pm 1 \}} \arrow[r] & 0
\\
0 \arrow[r] & \struct{M}^+ \arrow[r] & \struct{M} \arrow[u, equals] \arrow[r] & \underline{\{ \pm 1 \}} \arrow[u, equals] \arrow[r] & 0
\end{tikzcd}
\end{center}
Then we get a morphism of long exact sequence of cohomology,
\begin{center}
\begin{tikzcd}
H^1(M, \struct{M}) \arrow[d, "\exp"] \arrow[r] & H^1(M, \struct{M}^\times) \arrow[r] \arrow[d, equals] & H^1(M, \{ \pm 1 \}) \arrow[d, equals] \arrow[r] & H^2(M, \struct{M}) \arrow[d, "\exp"]
\\
H^1(M, \struct{M}^+) \arrow[r] & H^1(M, \struct{M}^\times) \arrow[r] & H^1(M, \{ \pm 1 \}) \arrow[r] & H^2(M, \struct{M})
\end{tikzcd}
\end{center}
but $\struct{M}$ is soft so $H^i(M, \struct{M}) = 0$ for all $i > 0$ and thus $w_1 : H^1(M, \struct{M}^\times) \to H^1(M, \Z / 2 \Z)$ is an isomorphism classifying line bundles by their first Steifel-Whitney class. Furthemore, since $\exp$ is an isomorphism we see that $H^i(M, \struct{M}^+) = 0$ so in particular, every prinicpal $\R^+$-bundle is trivial. A more prosaic reason is that principal $\R^+$-bundles are reduction of structure groups of a line bundle $\L$ thus giving the strucutre of an orientation on $\L$ but an oriented line bundle is trivial. 
\end{rmk}

\begin{rmk}
Warning! In topology, the singular cohomology group $H^1(M, G)$ only classifies $G$-bundles \underline{for a discrete groups} $G$. To classify $G$-bundles for a topoolgical group we need to use the sheaf $\F_G$ of continuous maps to $G$. Then sheaf cohomology $H^1(M, \F_G)$ classifies $G$-torsors on $M$. Warning $\F_G$ is NOT the constant sheaf $\underline{G}$ unless $G$ is discrete and thus,
\[ H^1_{\text{sing}}(M, G) = H^1(M, \underline{G}) \neq H^1(M, \F_G) = \{ \text{principal } G \text{-bundles on } M \} \]
For example, principal $\R$-bundles are all trivial but $H^1(M, \R)$ need not be zero.
\end{rmk}


\section{Conformal Geometry}

\renewcommand{\C}{\mathcal{C}}

\begin{defn}
Let $M$ be a smooth manifold. Two Riemannian metrics $g_1, g_2$ are \textit{conformally equivalent} if there exists a positive smooth function $\lambda : M \to \R$ such that $g_1 = \lambda^2 g_2$.
\end{defn}

\begin{defn}
A smooth map $f : M \to N$ of Riemannian manifolds is \textit{conformal} if $g_M$ and $f^* g_N$ are conformally equivalent. 
\end{defn}

\begin{defn}
A \textit{conformal manifold} is a pair $(M, [g])$ where $M$ is a smooth manifold and $[g]$ is a conformal class of Riemannian metrics on $M$. A conformal map $f : (M, [g]) \to (N, [h])$ is a smooth map $f : M \to N$ such that $f^* [h] = [g]$.
\end{defn}

\begin{defn}
We say that an almost complex structure $I$ is \textit{compatible} with a metric $g$ if
\[ g(I(v), I(u)) = g(v, u) \]
that is $I \in \mathrm{O}(TM)$
\end{defn}

\begin{prop}
Let $(X, g, I)$ be a Riemannian manifold with a compatible almost complex structue. Then $\omega(-,-) = g(I(-), -)$ is a 2-form called the fundamental form.
\end{prop}

\begin{proof}
We need to show that $\omega$ is antisymmetric. Note,
\[ \omega(v, u) = g(I(v), u) = g(I^2(v), I(u)) = - g(v, I(u)) = - g(I(u), v) = - \omega(u, v) \]
\end{proof}

\begin{prop}
If $I$ is compatible with $g$ then 
\end{prop}

\begin{prop}
Let $(X, I)$ be a (paracompact) smooth manifold with an almost complex structure. Then there exists a Riemannian metric $g$ on $X$ compatible with $I$.
\end{prop}

\begin{proof}
First we show that there exsits a metric $g'$ on $X$. Choose charts $\{(U_i, \varphi_i)\}$ which we refine such that it is locally finite and choose a subordinate partition of unity $\chi_i$. The standard metric gives a metric $g_i$ on $U_i$. Now consider,
\[ g = \sum \chi_i g_i \]
then $g$ is symmetric and positive definite since $\chi_i \ge 0$ with at least one positive at each point and $g_i(v, v) > 0$ for $v \neq 0$ so $g(v, v) > 0$. 
\bigskip\\
Now consider,
\[ g(v, w) = g'(v, w) + g'(I(v), I(w)) \]
It is clear that $g$ is symmetric and positive definite because $g(v, v) = g'(v, v) + g'(I(v), I(v))$ is positive unless $v = 0$ so $g$ is a metric. Furthermore,
\[ g(I(v), I(u)) = g'(I(v), I(u)) + g'(I^2(v), I^2(u)) = g'(v, u) + g'(I(v), I(u)) = g(v, u) \]
\end{proof}

\begin{defn}
A \textit{pseudo-holomorphic} map $f : (X, I) \to (X', I')$ is a smooth map $f : X \to X'$ such that $\d{f} \circ I = I' \circ \d{f}$. 
\end{defn}

\begin{rmk}
When $(X, I)$ and $(X', I')$ are integrable almost complex structues (i.e. are induced by complex structues on $X$ and $X'$) then pseudo-holomorphic maps are exactly holomorphic maps $f : X \to X'$.
\end{rmk}

\subsection{The Two Dimensional Case}


\begin{lemma}
Let $V$ be a oriented 2-dimensional $\R$-vectorspace. Let $J : V \to V$ be an endomorphism such that $v, J(v)$ is a positively ordered basis for each $v \neq 0$. Then $J \in \mathrm{GL}^{+}(V)$.
\end{lemma}


\begin{proof}
The orientation induces a notion of positivity on $\bigwedge^2 V$. The form $q(v) = v \wedge J(v)$ is positive definite. Choosing a basis we write $J$ in a matrix form,
\[ A = 
\begin{pmatrix}
a & b 
\\
c & d 
\end{pmatrix} \]
then $q(x) = x^\top B x$ where,
\[ B = \tfrac{1}{2} (SA - A^\top S) \]
where,
\[ S = 
\begin{pmatrix}
0 & -1
\\
1 & 0
\end{pmatrix} \]
then,
\[ B = 
\begin{pmatrix}
-c & \frac{a-d}{2}
\\
\frac{a-d}{2} & b 
\end{pmatrix} \]
Since $q$ is positive definite,
\[ \det{B} = -cb - \left( \frac{a-d}{2} \right)^2 > 0 \]
Therefore,
\[ ad - bc > ad + \left( \frac{a-d}{2} \right)^2 = \left( \frac{a + d}{2} \right)^2 > 0 \]
and thus $J \in \mathrm{GL}^{+}(V)$.
\end{proof}

\begin{lemma}
Let $V$ be a 2-dimensional $\R$-vectorspace with an inner product $\inner{-}{-}$. Suppose that $J : V \to V$ is an endomorphism such that $\inner{v}{J(v)} = 0$ for all $v \in V$. Then $J^2 = - \lambda^2 \id$.
\end{lemma}

\begin{proof}
Choose an orthonormal basis $\{ e_i \}$ then we write $I$ in a matrix form,
\[ A = 
\begin{pmatrix}
a & b 
\\
c & d 
\end{pmatrix} \]
and $\inner{v}{J(v)} = x^\top A x$ where $v = x^i e_i$. Thus if the form is zero we must have $A^\top = -A$ and thus,
\[ A = 
\begin{pmatrix}
0 & -\lambda 
\\
\lambda & 0
\end{pmatrix} \]
meaning that $A^2 = - \lambda^2 I$ and thus $J^2 = - \lambda^2 \id$.
\end{proof}

\begin{lemma}
Let $V$ be a 2-dimension $\R$-vectorspace and $J : V \to V$ an endomorphism such that $J^2 = - \id$. Then $J \in \mathrm{SL}(V)$.
\end{lemma}

\begin{proof}
We can choose a basis $v, J(v)$ for any $v \neq 0$ since if $J(v) = \lambda v$ then $\lambda^2 = -1$ which is not possible over $\R$. Thus $J(v \wedge J(v)) = J(v) \wedge J^2(v) = - J(v) \wedge v = v \wedge J(v)$ and thus $J$ preserves $\det{V}$ so $\det{J} = 1$ and thus $J \in \mathrm{SL}(V)$. 
\end{proof}

\begin{prop}
Let $X$ be an oriented 2-manifold. Then the following data are equivalent,
\begin{enumerate}
\item an almost complex structure $(X, I)$ compatible with the orientation
\item a conformal structure $(X, [g])$
\end{enumerate}
The graph of this correspondence is the set of compatible pairs $(I, [g])$.
\end{prop}

\begin{proof}
Given a conformal class $[g]$ choose a representative $g$. For any vector field $\sigma$ consider the line bundle $L = \ker{g(\sigma, -)} \subset T X$ giving an exact sequence,
\begin{center}
\begin{tikzcd}
0 \arrow[r] & L \arrow[r] & T X \arrow[r, "\iota_\sigma g"] & \underline{\R} \arrow[r] & 0
\end{tikzcd}
\end{center} 
Therefore, we find,
\[ \det{T X} \cong L \]
An orientation of $X$ induces an orientation on $\det{T X}$ and thus on $L$ so $L$ has nonvanishing global sections. Then the fixed norm bundle $U_\sigma(L) = \{ (x, v) \in L \mid g(v, v) = g(\sigma, \sigma) \}$ has a unique positive section $\sigma^\perp \in \Gamma(X, U(L)) \subset \Gamma(X, TX)$ under the induced orientation. Then $\sigma \mapsto \sigma^\perp$ is $\C^{\infty}$-linear giving an endomorphism $I : TX \to TX$. Furthermore, $g(\sigma, I(\sigma)) = 0$ so, applying the previous lemmas, $I^2 = - \lambda^2 \id$ but $g(I(\sigma), I(\sigma)) = g(\sigma, \sigma)$ and thus $\lambda^2 = 1$ so $I^2 = -\id$. Finally, since $v \wedge I(v)$ is positive, we see $I$ induces a compatible orientation on $X$. Furthermore, clearly $I$ is the unique almost complex structure compatible with $g$ and orientation since $U_\sigma(L)$ has only two sections the other of which is oppositely oriented. 
\bigskip\\
Conversely, given a complex structure $(X, I)$ compatible with the orientation consider the set $c(I)$ of metrics on $X$ compatible with $I$. Suppose $g$ and $g'$ are two such metrics. At a point $x \in X$ choose some $v \neq 0$. We know $v, I(v)$ forms a basis $T_x X$ so $g_x$ and $g_x'$ are determined by their values on $v, I(v)$. Choose $\lambda_x > 0$ such that $g_x'(v, v) = \lambda_x^2 g_x(v, v)$. Furthermore,
\[ g_x'(I(v), I(v)) = g_x'(v, v) = \lambda_x^2 g(v, v) = \lambda_x^2 g_x(I(v), I(v)) \]
and $g_x'(v, I(v)) = 0$ and $g_x(v, I(v))$ so $g_x' = \lambda_x^2 g_x$. Then because $g$ and $g'$ are smooth and nonzero tensors we see that $\lambda$ is a smooth function so $g$ and $g'$ are conformally equivalent. 
\end{proof}

\begin{prop}
Let $(X, I, [g])$ and $(X', I', [g'])$ be 2-manifolds with compatible conformal and almost complex structures. Let $f : X \to X'$ be a smooth map. Then the following are equivalent,
\begin{enumerate}
\item $f$ is conformal and orientation preserving
\item $f$ is a pseudo-holomorphic local diffeomorphism.
\end{enumerate}
\end{prop}

\begin{proof}
If $f : X \to X'$ is conformal then choosing a representative, $g'$ for the conformal class on $X'$ we see that $f^* g'$ is a metric i.e. positive definite. In particular, $(f^* g')(v, v) = g'(\d{f}(v), \d{f}(v))$ but if $\d{f}(v) = 0$ then $v = 0$ since $(f^* g')(v, v) = 0$. Thus $f$ is a local diffeomorphism and since $f^* g' \in [g]$, we see that $I$ is compatible with with $f^* g'$. Furthermore, $\tilde{I} = \d{f}^{-1} \circ I' \circ \d{f}$ is an almost complex structure on $X$ compatible with $f^* g'$ because,
\begin{align*}
(f^* g')(\tilde{I}(v), \tilde{I}(u)) & = g'(\d{f}(I(v)), \d{f}(I(u))) = g'(I'(\d{f}(v)), I'(\d{f}(u))) 
\\
& = g'(\d{f}(v), \d{f}(u)) = (f^* g')(v, u)
\end{align*}
Furthermore, if $f$ is orientation preserving then $\tilde{I}$ is compatible with the orientation. To see this, note that $\d{f}(v) \wedge I' \circ \d{f}(v)$ is positive (because $I'$ is compatible with the orientation on $X$) so applying $\d{f}^{-1}$ we see that $v \wedge \tilde{I}(v)$ is positive meaning that $\tilde{I}$ is compatible with the orientation on $X$. Thus $\tilde{I} = I$ because there is a unique almost complex structure compatible with the orientation and conformal structure so $f$ is pseudo-holomorphic.
\bigskip\\
Conversely, suppose that $f$ is pseudo-holomorphic and a local diffeomorphism. Choose a representative $g'$ for the conformal class. Then consider,
\[ (f^* g')(I(v), I(u)) = g'(\d{f} \circ I(v), \d{f} \circ I(u)) = g'(I' \circ \d{f}(v), I' \circ \d{f}(u)) = g'(\d{f}(v), \d{f}(u)) = (f^* g')(v, u) \]
because $I'$ is compatible with $g'$. Thus, $f^* g'$ is compatible with $I$. Because $f$ is a local diffeomorphism $f^* g'$ is a metric on $X$ and thus $f^* g' \in [g]$ because $f^* g'$ is compatible with $I$ and thus defines the same conformal class. Therefore, $f : (X, [g]) \to (X', [g'])$ is conformal. Furthermore, $v \wedge I(v)$ positive but $\d{f}(I(v)) = I'(\d{f}(v))$ so $\d{f}(v \wedge I(v)) = \d{f}(v) \wedge I'(\d{f}(v))$ which is positive because $I'$ is compatible with the orientation on $X'$.  
\end{proof}


\end{document}


