\subimport{../General/}{General_Includes}

% what do these do?

\usepackage{stmaryrd}
\usepackage{relsize}
\usepackage{mathtools}

% needed for correctly formatting words

\usepackage{xspace}
\newcommand{\et}{\text{\'{e}t}}
\newcommand{\etale}{\'{e}tale\xspace}
\newcommand{\kahler}{\text{K\"{a}hler}\xspace}

 

% Caligraphic Sheaf Operators

\DeclareMathOperator{\calHom}{\mathscr{H}\text{\kern -5pt {\calligra\large om}}}
\DeclareMathOperator{\calExt}{\mathscr{E}\text{\kern -3pt {\calligra\large xt}}}
\DeclareMathOperator{\calTor}{\mathscr{T}\text{\kern -5pt {\calligra\large or}}}
\DeclareMathOperator{\calEnd}{\mathscr{E}\text{\kern -3pt {\calligra\large nd}}}
\DeclareMathOperator{\calDer}{\mathscr{D}\text{\kern -3pt {\calligra\large er}}}
\DeclareMathOperator{\calDiff}{\mathscr{D}\text{\kern -3pt {\calligra\large iff}}}
\DeclareMathOperator{\calAnn}{\mathscr{A}\text{\kern -3pt {\calligra\large nn}}}

\newcommand{\shHom}[3]{\calHom_{#1} \! \! \left( #2, #3 \right)}
\newcommand{\shHomover}[3]{\calHom_{#1} \! \! \left( #2, #3 \right)}
\newcommand{\shExt}[4]{\calExt^{\: \: \: #1}_{#2} \! \! \left( #3, #4 \right)}
\newcommand{\shTor}[4]{\calTor^{\, #1}_{\, #2} \! \! \left( #3, #4 \right)}
\newcommand{\shEnd}[2]{\calEnd_{#1} \! \! \left( #2 \right)}
\newcommand{\shDer}[2]{\calDer\left( #1, #2 \right)}
\newcommand{\shDero}[3]{\calDer_{#1}\! \left( #2, #3 \right)}
\newcommand{\shDiff}[4]{\calDiff^{#1}_{#2}\! \left( #3, #4 \right)}
\newcommand{\Sym}[2]{\mathrm{Sym}_{#1}\left( #2 \right)}
\newcommand{\shAnn}[2]{\calAnn_{\, #1} \!\left(#2 \right)}

% Dimensionality

\newcommand{\codim}[1]{\mathrm{codim}\left( #1 \right)}

% Chow Groups

\newcommand{\CH}{\mathrm{CH}}

% Names for Prime Ideals

\renewcommand{\a}{\mathfrak{a}}
\newcommand{\p}{\mathfrak{p}}
\newcommand{\q}{\mathfrak{q}}
\newcommand{\m}{\mathfrak{m}}


% Classical Lie Groups

\newcommand{\U}[1]{\mathrm{U}(#1)}
\renewcommand{\O}[1]{\mathrm{O}(#1)}
\newcommand{\SU}[1]{\mathrm{SU}(#1)}
\newcommand{\SO}[1]{\mathrm{SO}(#1)}
\newcommand{\GL}[2]{\mathrm{GL}_{#1}(#2)}
\newcommand{\SL}[2]{\mathrm{SL}_{#1}(#2)}
\newcommand{\PGL}[2]{\mathrm{PGL}_{#1}(#2)}
\newcommand{\PSL}[2]{\mathrm{PSL}_{#1}(#2)}

% Important Categories

\newcommand{\Mod}[1]{\mathbf{Mod}_{#1}}
\newcommand{\Grp}{\mathbf{Grp}}
\newcommand{\Ab}{\mathbf{Ab}}
\newcommand{\Ring}{\mathbf{Ring}}
\newcommand{\Top}{\mathbf{Top}}
\newcommand{\Ch}{\mathbf{Ch}}

\newcommand{\Var}{\mathbf{Var}}
\newcommand{\Sch}{\mathbf{Sch}}
\newcommand{\Sh}{\mathfrak{Sh}}
\newcommand{\Coh}[1]{\mathfrak{Coh}\left(#1\right)}
\newcommand{\QCoh}[1]{\mathfrak{QCoh}\left(#1\right)}
\newcommand{\Qcoh}[1]{\mathfrak{QCoh}\left(#1\right)}
\newcommand{\shMod}[1]{\mathscr{M}\text{\kern -3pt \calligra\large od} \: \left(#1\right)}

% Names for Fields 

\newcommand{\R}{\mathbb{R}}
\newcommand{\C}{\mathbb{C}}

% Names for Sheaves

\newcommand{\F}{\mathscr{F}}
\newcommand{\G}{\mathscr{G}}
\renewcommand{\H}{\mathscr{H}}
\newcommand{\I}{\mathscr{I}}
\newcommand{\J}{\mathcal{J}}
\newcommand{\K}{\mathscr{K}}
\renewcommand{\L}{\mathcal{L}}
\newcommand{\E}{\mathcal{E}}
\newcommand{\T}{\mathcal{T}}
\newcommand{\Csh}{\mathscr{C}}
\renewcommand{\S}{\mathcal{S}}


\newcommand{\sA}{\mathscr{A}}
\newcommand{\sB}{\mathscr{B}}
\newcommand{\sN}{\mathcal{N}}

% Ring-Theoretic Commands

\newcommand{\Frac}[1]{\mathrm{Frac}\left(#1\right)}
\newcommand{\Jac}[1]{\mathrm{Jac}\left( #1 \right)}
\newcommand{\Ann}[2]{\mathrm{Ann}_{#1}\left(#2\right)}
\newcommand{\Ass}[2]{\mathrm{Ass}_{#1}\left( #2 \right)}
\newcommand{\supp}[2]{\mathrm{Supp}_{#1} \left( #2 \right) }
\newcommand{\Supp}[2]{\mathrm{Supp}_{#1}\left(#2 \right)}
\newcommand{\spec}[1]{\mathrm{Spec}\left( #1 \right)}
\newcommand{\rSpec}[2]{\mathbf{Spec}_{#1}\left( #2 \right)}
\newcommand{\Spec}[1]{\mathrm{Spec}\left( #1 \right)}
\newcommand{\mspec}[1]{\mathrm{mSpec}\left( #1 \right)}
\newcommand{\mSpec}[1]{\mathrm{mSpec}\left( #1 \right)}
\newcommand{\Proj}[1]{\mathrm{Proj}\left( #1 \right)}
\newcommand{\rProj}[2]{\mathbf{Proj}_{#1}\left( #2 \right)}
\newcommand{\proj}[1]{\mathrm{Proj}\left( #1 \right)}
\newcommand{\rad}[1]{\mathrm{rad}\left( #1 \right)}
\newcommand{\nilrad}[1]{\mathrm{nilrad}\left( #1 \right)}
\newcommand{\gr}[2]{\mathbf{gr}_{#1}\left(#2\right)}
\newcommand{\height}[1]{\mathbf{ht}\left( #1 \right)}
\newcommand{\length}[2]{\mathrm{length}_{#1}\left( #2 \right)}
\newcommand{\modSym}[2]{\mathrm{Sym}^{#1}\left( #2 \right)}
\newcommand{\shSym}[2]{\mathrm{Sym}_{#1}\left( #2 \right)}
\newcommand{\trdeg}[2]{\mathrm{trdeg}_{#1}(#2)}
\newcommand{\red}{\mathrm{red}}
\newcommand{\inner}[2]{\left< #1, #2 \right>}


% Sheaf-Theoretic Commands

\usepackage{accents}
\DeclareMathAccent{\wt}{\mathord}{largesymbols}{"65}

\newcommand{\res}{\mathrm{res}}

\newcommand{\struct}[1]{\mathcal{O}_{#1}}
\newcommand{\stalk}[2]{\mathcal{O}_{#1, #2}}

% Divisors and Picard Group

\newcommand{\Pic}[1]{\mathrm{Pic} \left( #1 \right)}
\newcommand{\Cl}[1]{\mathrm{Cl} \left( #1 \right)}
\newcommand{\CaCl}[1]{\mathrm{CaCl}\left( #1 \right)}
\newcommand{\NS}[1]{\mathrm{NS}\left( #1 \right)}

\newcommand{\shDiv}{\mathfrak{Div}}

\newcommand{\Div}[1]{\mathrm{Div}\left( #1 \right)}
\renewcommand{\div}{\mathrm{div}}

% GAGA

\newcommand{\an}{\mathrm{an}}

% Cohomology Commands

\newcommand{\dR}{\mathrm{dR}}
\newcommand{\Cech}{\check{\mathscr{C}}}
\newcommand{\ch}{\mathrm{ch}}

% Important Spaces 

\newcommand{\A}{\mathbb{A}}
\renewcommand{\P}{\mathbb{P}}
\newcommand{\Ga}{\mathbb{G}_a}
\newcommand{\Gm}{\mathbb{G}_m}

\newcommand{\Xcut}{X_{\text{cut}}}


% Operations on Schemes

\newcommand{\Sing}[1]{\mathrm{Sing}\left( #1 \right)}


% Rational Maps

\newcommand*{\DashedArrow}[1][]{\mathbin{\tikz [baseline=-0.25ex,-latex, dashed,#1] \draw [#1] (0pt,0.5ex) -- (1.3em,0.5ex);}}%
\newcommand*{\DashedBiArrow}[1][]{\mathbin{\tikz [baseline=-0.25ex,-latex, dashed, #1] \draw [#1] (0pt,0.5ex) -- (1.3em,0.5ex) node[midway, above] {$\sim$} ;}}
\newcommand{\rat}{\DashedArrow[densely dashdotted]}
\newcommand{\birat}{\DashedBiArrow[densely dashdotted]}
\newcommand{\Dom}[1]{\mathrm{Dom}\left(#1 \right)}

% Tikzcd

\newsavebox{\pullback}
\sbox\pullback{%
\begin{tikzpicture}%
\draw (0,0) -- (1ex,0ex);%
\draw (1ex,0ex) -- (1ex,1ex);%
\end{tikzpicture}}
