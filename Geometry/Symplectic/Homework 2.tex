\documentclass[12pt]{article}
\usepackage{import}
\import{./}{Includes}


\begin{document}

\atitle{2}

\newcommand{\can}{\mathrm{can}}

\section{Problem 1}

\subsection{Chern Classes are Determined by Connected Component of the Almost Complex Structure}

Let $M$ be a smooth manifold of even dimension which admits an almost complex structure (for example if $M$ is symplectic). 
I claim that for any smooth path $J_t$ of almost complex structures, the Chern classes,
\[ c_k(TM) \in H^{2k}(M, \Z) \]
are constant. Indeed, the complex vector bundles $(TM, J_0)$ and $(TM, J_1)$ are always isomorphic. To see this, notice that such a path $J_t$ defines a complex structure on the vector bundle $\pi_1^* TM$ on $M \times \R$ whose action on $T_p M$ at $(p,t)$ is,
\[ (a + i b) \cdot v = a v + b J_t v \]
By Proposition 1.7 of Hatcher's Vector Bundles this implies that on the sections $M \times \{ 0 \}$ and $M \times \{ 1 \}$ the complex vector bundles are isomorphic proving the claim. 
\bigskip\\
We can rephrase this in terms of classifying spaces. The path of almost complex structures $(TM, J_t)$ defines a homotopy of classifying maps,
\[ f_t : M \to \mathrm{BGL}_n(\CC) \]
between the classifying maps $f_0$ and $f_1$ of the complex vector bundles $(TM, J_0)$ and $(TM, J_1)$ and hence these define isomorphic vector bundles. This proof is equivalent because $f_t$ is just the classifying map of $(\pi_1^* TM, J)$,
\[ f : M \times \R \to \mathrm{BGL}_n(\CC) \]

\subsection{Invariance Under Choice of Tamed Structure}

Now we show that if $(M, \omega)$ is symplectic then $c_k(TM)$ is independent of the choice of tamed almost complex structure $J$. Indeed, the space of tamed structures is contractible and hence path connected so this follows immediately from our previous result.

\subsection{Invariance Under Symplectic Deformation} 

Let $\omega_t$ be a symplectic deformation on $M$. By the previous discussion, to conclude that $c_k(TM)$ are independent of $t$ it suffices to show there exists a path $J_t$ of almost complex structures which are tamed for $\omega_t$. This follows from continuity in the polar decomposition. 

\section{Problem 2}

\begin{enumerate}
\item Let $f_n(z) = [z^2, z, \tfrac{1}{n}]$. The limit $n \to \infty$ is not-well-defined at $z = 0$ and thus we need to catch a bubble. Rescale to let $w = n z$ then, 
\[ f_n(w) = [ \tfrac{1}{n^2} w^2, \tfrac{1}{n} w, \tfrac{1}{n}] = [ \tfrac{1}{n} w^2, w, 1] \]
which is not well-defined at $w = \infty$ in the limit. Therefore, we get a limit consisting of two degree one maps $f_{\infty}(z) = [z, 1, 0]$ and $f_{\infty}(w) = [0, w, 1]$ which glue at $z = 0$ and $w = \infty$.

\item Let,
\[ f_n(z) = [z(z-\tfrac{1}{n}), z, \tfrac{1}{n}] \]
The limit $n \to \infty$ is not-well-defined at $z = 0$ and thus we need to catch a bubble. Rescale to let $w = n z$ then, 
\[ f_n(w) = [ \tfrac{1}{n^2} w(w-1), \tfrac{1}{n} w, \tfrac{1}{n}] = [ \tfrac{1}{n} z(z-1), z, 1] \]
which is not well-defined at $z = \infty$ in the limit. Therefore, we get a limit consisting of two degree one maps $f_{\infty}(z) = [z, 1, 0]$ and $f_{\infty}(w) = [0, w, 1]$ which glue at $z = 0$ and $w = \infty$.

\item Let,
\[ f_n(z) = [z^2 - \tfrac{1}{n^2}, z - \tfrac{1}{n^2}, \tfrac{1}{n}] \]
The limit $n \to \infty$ is not-well-defined at $z = 0$ and thus we need to catch a bubble. Rescale to let $w = n z$ then, 
\[ f_n(w) = [ \tfrac{1}{n^2} (z^2-1), \tfrac{1}{n} z - \tfrac{1}{n^2}, \tfrac{1}{n} ] = [ \tfrac{1}{n} (z^2 - 1), z - \tfrac{1}{n}, 1] \]
which is not well-defined at $z = \infty$ in the limit. Therefore, we get a limit consisting of two degree one maps $f_{\infty}(z) = [z, 1, 0]$ and $f_{\infty}(w) = [0, w, 1]$ which glue at $z = 0$ and $w = \infty$.
\end{enumerate}

\section{Problem 3}

Let $\M_{g,n}$ be the Deligne-Mumford moduli space of stable genus $g$ curves with $n$ marked points.

\newcommand{\h}{\mathbb{h}}

\begin{enumerate}
\item Let $x_0, x_1 \in \Sigma$ be two points. I claim there exists a disk $D \subset \Sigma$ containing $x_0, x_1 \in D^\circ$ in the interior. Given this it is always possible to find a homeomorphism (even a diffeomorphism!) $D \to D$ which fixes the boundary sending $x_0 \mapsto x_1$ by using bump functions. This gives a homeomorphism $\Sigma \to \Sigma$ sending $x_0 \mapsto x_1$. If $g = 0$ then $\Sigma = S^2$ so removing a point not equal to $x_0$ or $x_1$ gives the required disk. Otherwise, choose a basis of homology cycles on $\Sigma$ not intersecting $x_0$ and $x_1$ and cutting along these $\Sigma$ is homoeomorphic to a $4g$-sided polygon which is convex and hence $x_0$ and $x_1$ are contained in some common disk.
\bigskip\\
However, if $\Sigma$ has a node then no homeomorphism can take a node to a non-node since these have topologically distinct neighborhoods (a node is not locally euclidean). 

\item We need to show that any pair of genus $g$ surfaces $\Sigma$ with $n$ marked points $(\Sigma, x_1, \dots, x_n)$ are homeomorphic. The same argument as previously reduces to the case of $n$ distinct points $x_1, \dots, x_n \in D^\circ$ in the interior of a disk. These points may be moved arbitrarily while fixing the boundary. I will draw the types on another page.

\item Consider the graph $G$ whose vertices are the irreducible components and whose edges correspond to nodes. This graph has nodes labeled by their genus $g$. The number of cycles is,
\[ \# \text{cycles} = \# E - \# V + 1 \]
and $E = N$ is the set of nodes and $V = C$ is the set of components so the genus becomes,
\[ g(G) = \sum_{c \in C} g_c + \# N - \# C + 1 \]
Now let $C = C_0 \sqcup C_1 \sqcup C_{\ge 2}$ be the components of genus $g = 0$ and $g = 1$ and $g \ge 2$ respectively. Thus,
\[ g(G) = \sum_{c \in C_{\ge 2}} (g_c - 1) - \# C_0 + \# N + 1 \]
Furthermore, the stability condition says that each genus $0$ component has at least three marked points or nodes and each genus $1$ component at least $1$ meaning,
\[ 3 \# C_0 + \# C_1 \le 2 \# N + n \]
because each node may count on two components or twice if it is a self-intersection but each marked point lies on exactly one irreducible component (since it is required to be a nonsingular point). Therefore,
\[ \sum_{c \in C_{\ge 2}} 3 (g_c - 1) - 3 (g - 1) + 3 \# N = 3 \# C_0 \le 2 \# N + n - \# C_1 \]
which implies that,
\[ \# N + \# C_1 + \sum_{c \in C_{\ge 2}} 3(g_c - 1) \le 3 g - 3 + n = \dim{\M_{g,n}} \]
In particular, since all numbers on the right hand side are non-negative,
\[ \# N \le \dim{\M_{g,n}} \]
and I claim that equality is possible. For the cases in question, I gave explicit topological types with $\dim{\M_{g,n}}$ nodes. Furthermore, 
\[ 3 \# C_0 + \# C_1 \le 2 \# N + n = 2 g - 2 + n - \sum_{c \in C_{\ge 2}} 2(g_c - 2) + 2 \# C_0 \]
and therefore,
\[ \# C_0 + \# C_1 + \sum_{c \in C_{\ge 2}} 2 (g_c - 1) \le 2g - 2 + n \]
\end{enumerate}

\end{document}