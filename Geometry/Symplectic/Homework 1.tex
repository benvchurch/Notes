\documentclass[12pt]{article}
\usepackage{import}
\import{./}{Includes}


\begin{document}

\atitle{1}

\newcommand{\can}{\mathrm{can}}

\section{Problem 1}

\begin{enumerate}
\item Given a $1$-from $\alpha$ define its graph,
\[ \Gamma_\alpha = \{ (x, \alpha_x) \mid x \in X \} \]
We claim that $\Gamma_\alpha$ is Lagrangian if and only if $\d{\alpha} = 0$.
Because $\Gamma_\alpha$ is half-dimensional, it suffices to show that $\omega_{\can}|_{\Gamma_\alpha} = 0$ if and only if $\d{\alpha} = 0$. Denote the section determined by $\alpha$ as $s_\alpha : X \to T^*X$. Indeed, the tautological form satisfies the tautological property ($\lambda_{\can}$ is the universal $1$-form) that $\alpha = s_\alpha^* \lambda_{\can}$ and therefore $\d{\alpha} = s_\alpha^* \d{\lambda_{\can}} = - s_{\alpha}^* \omega_{\can}$. Since $s_\alpha$ is an isomorphism onto $\Gamma_\alpha$ we see that 
\[ \omega_{\can} |_{\Gamma_\alpha} = 0 \iff s_\alpha^* \omega_{\can} = 0 \iff \d{\alpha} = 0 \]
Now we check that $\lambda_{\can}$ is universal. Indeed, 
\[ (s_\alpha^* \lambda_{\can})_x = (\lambda_{\can} \circ \d{s_\alpha})_x = (\lambda_{\can})_{(x, \alpha_x)} \circ (\d{s_\alpha})_x = \alpha_x \circ (\d{\pi})_{(x, \alpha_x)} \circ (\d{s_{\alpha}})_x = \alpha_x \circ \id = \alpha_x \]
so we win. 

\item Let $Y \subset X$ be a submanifold and consider the conormal bundle,
\[ L_Y = \{ \alpha \in T^* X |_Y \mid \alpha|_{TY} = 0 \} \]
We need to consider $T L_Y \subset T(T^* X)$ and show that $\omega_{\can}|_{L_Y} = 0$ since,
\[ \dim{L_Y} = \dim{Y} + \rank{T^*M} - \rank{TY} = \dim{M} = \tfrac{1}{2} \dim{T^* M} \] 
Even better, I claim that $\lambda_{\can} |_{L_Y} = 0$. Indeed, at $(x, \alpha) \in L_Y$ we know,
\[ (\lambda_{\can})_{(x, \alpha)} = \d{\pi}^* \alpha = \alpha \circ \d{\pi}_{(x, \alpha)} \]
and therefore because $\alpha_{TY} = 0$ using the inclusion $\iota_{L_Y} : L_Y \embed T^*X$,
\[ (\lambda_{\can} |_{L_Y})_{(x, \alpha)} = \alpha \circ \d{\pi} \circ \d{\iota_{L_Y}} = 0 \]
since $\pi \circ \iota_{L_Y}$ is the projection $L_Y \to Y \subset X$ and hence $\d{(\pi \circ \iota_{L_Y})} : T L_Y \to TY \subset TX$.

\item Let $\varphi : (M_1, \omega_1) \to (M_2, \omega_2)$ be a diffeomorphism and consider the graph morphism,
\[ g_\varphi : M_1 \to M_1 \times M_2 \]
which is a closed embedding. Its image be the embedded submanifold $\Gamma_\varphi$. Since $\varphi$ is a diffeomorphism $\dim{M_1} = \dim{M_2}$ so $\Gamma_\varphi$ is half-dimensional. Thus, $\Gamma_\varphi \subset M_1 \times M_2$ is Lagrangian for $(M_1, \times M_2, \omega)$ with $\omega = \omega_1 \oplus (-\omega_2)$ if and only if $\omega|_{\Gamma_\varphi} = 0$. Since $g_{\varphi} : M_1 \to \Gamma_{\varphi}$ is an isomorphism we see that,
\begin{align*}
\Gamma_{\varphi} \text{ is Lagrangian} & \iff \omega|_{\Gamma_{\varphi}} = 0 \iff g_{\varphi}^* \omega = 0 \iff g_\varphi^* (\omega_1 \oplus (-\omega_2)) = \omega_1 - \varphi^* \omega_2 = 0
\\
&  \iff \omega_1 = \varphi^* \omega_2 \iff \varphi \text{ is a symplectomorphism} 
\end{align*}
\end{enumerate}

\section{Problem 2}

\begin{enumerate}
\item Let $W \subset V$ be a linear subspace of a symplectic space $(V, \omega)$. Let,
\[ K = \ker{\omega|_W} = W \cap W^\omega \]
By definition $\omega$ is a well-defined $2$-form on $W/K$. It suffices to show that $\omega$ on $W/K$ is nondegenerate. Suppose that $\omega([w], -) = 0$ then $w \in K$ so $[w] = 0$ by definition proving the claim. 






Choose a compatible complex structure $J$ on $V$ and thus we get a metric,
\[ g(v,w) = \omega(v, J w) \]
It is clear that,
\[ J(W^\omega) = W^\perp \]
Furthermore, $g$ restricts to a metric on $K$ and therefore defines an isomorphism $q^{-1} : K \to K^*$ via $v \mapsto g(v,-)$. Notice that, $J K \subset (W + W^\omega)^\perp \subset K^\perp$ because if $v \in W$ and $u \in W^\omega$ and $k \in K$ then,
\[ g(Jk, v + u) = \omega(k,v) + \omega(k,u) = 0 \]
because $k \in W \cap W^\omega$. Now consider the map,
\[  \Phi : (W / K) \oplus (W^\omega / K) \oplus (K \oplus K^*) \to V \]
defined by,
\[ ([w], [u], v, \varphi) \mapsto w + u + v + J q(\varphi) \]
where $w$ and $u$ are the unique representative in $W \cap K^\perp$ and $W^\omega \cap K^\perp$ so the map is well-defined. I claim this map is injective. Suppose that,
\[ w + u + v + J q(\varphi) = 0 \]
Since $w + u + J q(\varphi) \in K^\perp$ and $v \in K$ we see that $v = 0$ so,
\[ w + u + J q(\varphi) = 0 \]
Since $J q(\varphi) \in (W + W^\omega)^\perp$ and $w + u \in W + W^\omega$ we see that $\varphi = 0$ and $w + u = 0$ so $w, u \in W \cap W^\perp = K$ but also both lie in $K^\perp$ so $u = w = 0$ so the map is indeed injective. Since the two sides have the same dimension, $\Phi$ is an isomorphism. Now we need to check that $\Phi$ is a symplectomorphism,
\[ \Phi : (W/K, \omega) \oplus (W^\omega / K, \omega) \oplus (K \oplus K^*, \omega_{\can}) \to (V, \omega) \]
Indeed, consider,
\begin{align*}
\omega(w + u + v + J q(\varphi), w' + u' + v' + J q(\varphi')) & = \omega(w, w') + \omega(u, w') + \omega(v, w') - g(q(\varphi), w') 
\\
&+ \omega(w, u') + \omega(u, u') + \omega(v, u') - g(q(\varphi), u') 
\\
&+ \omega(w, v') + \omega(u, v') + \omega(v, v') - g(q(\varphi), v') 
\\
&+ g(w, q(\varphi')) + g(u, q(\varphi')) + g(v, q(\varphi')) + \omega(q(\varphi), q(\varphi'))
\\
& = \omega(w, w') + \omega(u, u') - \varphi(v') + \varphi'(v)
\end{align*}
Because,
\[ \omega(u, w') = \omega(v, w') = \omega(w.u') = \omega(v, u') = \omega(w, v') = \omega(u. v') = \omega(v, v') = 0 \]
by paring $W$ with $W^\omega$. Furthermore, $q(\varphi), q(\varphi') \in K$ so because $w,u,w',u' \in K^\perp$ we have,
\[ g(q(\varphi), w') = g(q(\varphi), u') = g(q(\varphi), v') = g(w, q(\varphi')) = g(u, q(\varphi')) = 0 \]
Likewise $\omega(q(\varphi), q(\varphi')) = 0$ since $K$ is isotropic. Therefore, 
\[ \Phi^* \omega = \omega \oplus \omega \oplus \omega_{\can} \]
proving the claim.


\item Let $W$ be a submanifold of a symplectic manifold $(M, \omega)$ such that
\[ K = (TW) \cap (TW)^\omega = \ker{\omega|_{TW}} \]
is constant rank. Choose an almost complex structure $J$ on $TM$ compatible with $\omega$. Since the previous map $\Phi_J$ is canonical, the same formula defines a map of symplectic bundles,
\[ \Phi_J : (TW/K, \omega) \oplus (TW^\omega/K, \omega) \oplus (K \oplus K^*, \omega_{\can}) \iso TM |_W \]
which is fiberwise an isomorphism and therefore is an isomorphism of vector bundles.
\bigskip\\
If $W$ is Lagrangian then $K = TW$ so we recover an isomorphism,
\[ (TM|_W, \omega) \cong (TW \oplus T^*W, \omega_{\can}) \]
If $W$ is symplectic then $K = 0$ so we recover,
\[ (TM|_W, \omega) \cong (TW, \omega) \oplus ((TW)^\omega, \omega) \]

\item Let $W$ be a symplectic submanifold of $(M, \omega)$. Since $W$ is symplectic,
\[ (TM|_W, \omega) \cong (TW, \omega) \oplus ((TW)^\omega, \omega) \]
Consider two symplectic forms $\omega_1, \omega_2$ on $M$ which agree as symplectic structures on $TW$ and $(TW)^\omega$ meaning, from the above decomposition (which holds for any symplectic form) they must agree as symplectic structures on $TM|_W$. Therefore, we may apply the relative Moser theorem (since $M$ is compact) to conclude that there exist tubular neighborhoods $U_0$ and $U_1$ of $W \subset M$ and a diffeomorphism $\varphi : U_0 \to U_1$ such that $\varphi|_W = \id$ and $\varphi^* \omega_1 = \omega_2$. This proves that, up to diffeomorphism, the symplectic structure on a sub tubular neighborhood of $U$ is determined by the data of the symplectic structure on $TW$ and $(TW)^\omega$. 
\end{enumerate}

\section{Problem 3}

\begin{enumerate}
\item Let $X = \CP^1 \times \CP^1$ with the family of symplectic forms,
\[ \omega_\lambda = \lambda \omega_0 \oplus \lambda^{-1} \omega_0 \]
for $\lambda > 0$. These all have the same volume form because,
\[ \omega_{\lambda}^{\wedge 2} = (\lambda \omega_0 \oplus \lambda^{-1} \omega_0) \wedge (\lambda \omega_0 \oplus \lambda^{-1} \omega_0) = \omega_0 \ot \omega_0 \in \Omega^4(\CP^1 \times \CP^1) = \Omega^2(\CP^1) \ot \Omega^2(\CP^1) \]
This is much clearer if we write,
\[ \omega_{\lambda} = \lambda  \pi_1^* \omega_0 + \lambda^{-1} \pi_2^* \omega_0 \]
for $\pi_i : \CP^1 \times \CP^1 \to \CP^1$
and notice that,
\[ (\pi_i^* \omega_0) ^{\wedge 2} = \pi_i^* \omega_0^{\wedge 2} = 0 \]
because $\dim{\CP^1} = 2$. Therefore, 
\[ \omega_{\lambda}^{\wedge 2} = \pi_1^* \omega_0 \wedge \pi_2^* \omega_0 \]
is constant. We need to show that $(X, \omega_\lambda)$ are not symplectomorphic for all $\lambda$. Consider a diffeomorphism $f : X \to X$ which induces an automorphism,
\[ f^* : H^2(X, \Z) \to H^2(X, \Z) \]
and since $H^2(X, \Z) = \Z^{\oplus 2}$ by Kunneth we see that $f$ is given by a $\GL{2,\Z}$ matrix. Then,
\[ f^* : H^2(X, \R) \to H^2(X, \R) \]
is induced by the same integer matrix (the map tensored with $\R$). Now I claim that $[\omega_\lambda] \in H^2_{\dR}(X)$ is not in the $\GL{2,\Z}$-orbit of $[\omega_1] \in H^2_{\dR}(X)$. Because $e = [\omega_0] \in H^2_{\dR}(\CP^1)$ is a generator we see that $[\omega_\lambda ] = \lambda e_1 + \lambda^{-1} e_2$ which is not in the $\Z$-lattice for $\lambda > 1$ proving that these cannot be symplectomorphic.

\item Let $X = \overline{\CP^2}$ with the opposite orientation meaning we choose the distinguished element $-[\CP^2] \in H^4(X, \Z)$ as an orientation. Suppose that $\omega$ is a symplectic form inducing the correct orientation. Then $[\omega] \in H^2(X, \R)$ and $\omega^{\wedge 2}$ is a volume form inducing the correct orientation meaning that $[\omega]^2 < 0$ but this is not possible because under $H^2(X, \R) \cong \R$ the cup product is multiplication and squares of real numbers are always positive.
\end{enumerate}

\end{document}