\documentclass[12pt]{extarticle}
\usepackage[utf8]{inputenc}
\usepackage[english]{babel}
\usepackage[a4paper, total={6in, 9in}]{geometry}
\usepackage{tikz-cd}
 
\usepackage{amsthm, amssymb, amsmath, centernot}

\newcommand{\notimplies}{%
  \mathrel{{\ooalign{\hidewidth$\not\phantom{=}$\hidewidth\cr$\implies$}}}}
 
\renewcommand\qedsymbol{$\square$}
\newcommand{\cont}{$\boxtimes$}
\newcommand{\divides}{\mid}
\newcommand{\ndivides}{\centernot \mid}
\newcommand{\Z}{\mathbb{Z}}
\newcommand{\R}{\mathbb{R}}
\newcommand{\N}{\mathbb{N}}
\newcommand{\C}{\mathbb{C}}
\newcommand{\Zplus}{\mathbb{Z}^{+}}
\newcommand{\Primes}{\mathbb{P}}
\newcommand{\colim}[1]{\mathrm{colim}(#1)}
\newcommand{\Ob}[1]{\mathrm{Ob}(#1)}
\newcommand{\cat}[1]{\mathcal{#1}}
\newcommand{\id}{\mathrm{id}}
\newcommand{\Hom}[2]{\mathrm{Hom}\left( #1, #2 \right)}
\newcommand{\catHom}[3]{\mathrm{Hom}_{#1}\left( #2, #3 \right)}
\newcommand{\Top}{\mathbf{Top}}
\newcommand{\pTop}{\mathbf{Top}_{\bullet}}
\newcommand{\Set}{\mathbf{Set}}
\newcommand{\pSet}{\mathbf{Set}_\bullet}
\newcommand{\hTop}{\mathbf{hTop}}
\newcommand{\phTop}{\mathbf{hTop}_{\bullet}}
\renewcommand{\Im}[1]{\mathrm{Im}(#1)}
\newcommand{\homspace}[2]{\left< #1, #2 \right>}
\newcommand{\rp}{\mathbb{RP}}
\newcommand{\coker}[1]{\mathrm{coker}\: #1}
\newcommand{\Tr}[1]{\mathrm{Tr}\left( #1 \right)}

\renewcommand{\d}[1]{\: \mathrm{d}#1 \:}
\newcommand{\dn}[2]{\: \mathrm{d}^{#1} #2 \:}
\newcommand{\deriv}[2]{\frac{\d{#1}}{\d{#2}}}
\newcommand{\nderiv}[3]{\frac{\dn{#1}{#2}}{\d{#3^2}}}
\newcommand{\pderiv}[2]{\frac{\partial{#1}}{\partial{#2}}}
\newcommand{\parsq}[2]{\frac{\partial^2{#1}}{\partial{#2}^2}}

\theoremstyle{definition}
\newtheorem{theorem}{Theorem}[section]
\newtheorem{lemma}[theorem]{Lemma}
\newtheorem{proposition}[theorem]{Proposition}
\newtheorem{example}[theorem]{Example}
\newtheorem{corollary}[theorem]{Corollary}
\newtheorem{remark}{Remark}

\newenvironment{definition}[1][Definition:]{\begin{trivlist}
\item[\hskip \labelsep {\bfseries #1}]}{\end{trivlist}}


\newenvironment{lproof}{\begin{proof} \renewcommand{\qedsymbol}{}}{\end{proof}}
\renewcommand{\mod}[3]{\: #1 \equiv #2 \: mod \: #3 \:}
\newcommand{\nmod}[3]{\: #1 \centernot \equiv #2 \: mod \: #3 \:}
\newcommand{\ndiv}{\hspace{-4pt}\not \divides \hspace{2pt}}
\newcommand{\gen}[1]{\langle #1 \rangle}
\newcommand{\hook}{\hookrightarrow}
\newcommand{\Tor}[4]{\mathrm{Tor}^{#1}_{#2} \left( #3, #4 \right)}
\newcommand{\Ext}[4]{\mathrm{Ext}^{#1}_{#2} \left( #3, #4 \right)}

\tikzset{
    labl/.style={anchor=south, rotate=90, inner sep=.5mm}
}

\renewcommand{\bf}[1]{\mathbf{#1}}
\newcommand{\Class}[2]{\mathcal{C}^{#1} \left( #2 \right)}
\newcommand{\Res}[2]{\mathrm{Res}_{#1} \: #2}
\newcommand{\F}{\mathcal{F}}
\newcommand{\G}{\mathcal{G}}
\renewcommand{\O}{\mathcal{O}}
\newcommand{\Aut}[1]{\mathrm{Aut}\left( #1 \right)}
\newcommand{\GL}[2]{\mathrm{GL}_{#1}\left( #2 \right)}

\newcommand{\Xcut}{X_{\text{cut}}}

\begin{document}

\title{Differential Galois Theory}
\author{Ben Church}
\maketitle
\tableofcontents
\newpage

\section{Holomorphic Maps}

\begin{definition}
A subset $\Omega \subset \C$ is a domain if $\Omega$ is open and connected.
\end{definition}

\begin{definition}
A map $f : \Omega \to \C$ is \textit{holomorphic} at $z \in \Omega$ if the limit,
\[ f'(z) = \lim_{h \to 0} \frac{ f(z + h) - f(z) }{h} \]
exists. The map $f$ is holomorphic on $\Omega$ if it is holomorphic at each $z \in \Omega$. 
\end{definition}

\begin{definition}
We say a map $f : \C \to \C$ is \textit{entire} if it is holomorphic on all of $\C$.
\end{definition}

\begin{proposition}
Let $f : \Omega \to \C$ be holomorphic at $z \in \Omega$. Then we may write $f$ as a function of two real variables as, $f(x, y) = f(x + i y)$. This done,
\[ f'(z) = \pderiv{f}{x} = \frac{1}{i} \pderiv{f}{y} \]
and thus,
\[ \pderiv{f}{x} + i \pderiv{f}{y} = 0 \] 
\end{proposition}


\begin{proposition}
\[ \pderiv{f}{z} = \frac{1}{2} \left[ \pderiv{f}{x} - i \pderiv{f}{y} \right] \quad \text{and} \quad \pderiv{f}{\bar{z}} = \frac{1}{2} \left[ \pderiv{f}{x} + i \pderiv{f}{y} \right] \]
Therefore, if $f$ is holomorphic then 
\[ \pderiv{f}{z} = f'(z) \quad \text{and} \quad \pderiv{f}{\bar{z}} = 0 \]
\end{proposition}


\begin{remark}
If we write $f : \Omega \to \C$ in real form i.e. as a function $F : \R^2 \to \R^2$ with $F(x,y) = (A(x,y), B(x,y))$ and $f(x + iy) = A(x,y) + i B(x,y)$ then,
\[ \pderiv{f}{\bar{z}} = \frac{1}{2} \left[ \pderiv{f}{x} + i \pderiv{f}{y} \right] = \frac{1}{2} \left[ \pderiv{A}{x} + i \pderiv{B}{x} + i \pderiv{A}{y} - \pderiv{B}{y} \right] \]
Therefore,
\[ \pderiv{f}{\bar{z}} = 0 \iff \pderiv{A}{x} = \pderiv{B}{y} \quad \text{and} \quad \pderiv{B}{x} = - \pderiv{A}{y} \]
These are known as the Cauchy-Riemann equations. We will see that satisfying these equations along with some weak regularity is necessary and sufficient for a function to be holomorphic. 
\end{remark}

\begin{definition}
Let $U \subset \R^m$ then denote the vectorspace of continuous functions $U \to \C$ by $\Class{0}{U}$ and for $n > 0$ define, 
\[ \Class{n}{U} = \{ f : U \to \R^m \mid \forall p \in U : f'_p \text{ exists and } \forall \bf{v} \in \R^n : f'(\bf{v}) \in \Class{n-1}{U} \} \] 
where $f' \cdot \bf{v}$ is the map $p \mapsto f'_p(\bf{v})$. Furthermore, the space of smooth functions is,
\[ \Class{\infty}{U} = \bigcap_{k} \Class{k}{U} \] 
\end{definition}

\begin{theorem}
Let $\Omega$ be a domain and $f : \Omega \to \C$. Then the following are equivalent,
\begin{enumerate}
\item $f : \Omega \to \C$ is holomorphic.
\item $f \in \Class{1}{\Omega}$ and 
\[ \pderiv{f}{\bar{z}} = 0 \]
\end{enumerate}
\end{theorem}

\begin{theorem}
Let $\Omega$ be a domain and $f : \Omega \to \C$. Then the following are equivalent,
\begin{enumerate}
\item $f : \Omega \to \C$ is holomorphic.
\item $f \in \Class{1}{\Omega}$ and 
\[ \pderiv{f}{\bar{z}} = 0 \]
\item $f \in \Class{1}{\Omega}$ and for $D \subseteq \Omega$ with piecewise $\Class{1}{\Omega}$ boundary we have \[ \oint_{\partial D} f(z) \d{z} = 0 \]
\item $\forall B_{r}(w) \subsetneq \Omega$ we have,
\[ f(z) = \frac{1}{2 \pi i} \oint_{\partial B_{r}(w)} \frac{f(\zeta)}{\zeta - z} \d{\zeta} \] 
for all $z \in B_r(w)$. 
\item $f$ is complex analytic: $\forall w  \in \Omega : \exists r > 0$ such that whenever $|z - w| < r$ we have,
\[ f(z) = \sum_{n = 0}^\infty a_n(x - w)^n \]
\end{enumerate}
\end{theorem}

\begin{theorem}[Cauchy]
Let $f : \Omega \to \C$ be holomorphic, for any subset $D \subset \Omega$ homeomorphic to a disc with $\Class{1}{I}$ boundary and $w \in D^\circ$ we have,
\[ f^{(n)}(w) = \frac{n !}{2 \pi i} \oint_{\partial D} \frac{f(z)}{(z - w)^{n+1}} \d{z} \]
In particular, the coefficients of the series expansion about $w$ are,
\[ a_n = \frac{1}{2 \pi i} \oint_{\partial D} \frac{f(z)}{(z - w)^{n+1}} \d{z} \]
\end{theorem}


\begin{lemma}
For any $z_0 \in \Omega$, either $f \equiv 0$ in a neighborhood of $z_0$ or we can express $f = (z - z_0)^n u(z)$ for $u(z)$ holomorphic and $u(z) \neq 0$.
\end{lemma}



\begin{proof}
In a neighborhood of $z_0$, we can write,
\[ f(z) = \sum_{n = 0}^\infty n_n(z - z_0)^n\]
Either $c_n = 0$ for each $n$ so $f = 0$ or $c_N \neq 0$ for some $n$ and $c_n - 0$ for $n < N$. Therefore,
\[ f(z) = \sum_{n \ge N}^\infty c_n(z - z_0)^n = (z - z_0)^N \left( \sum_{m = 0}^\infty c_{N + m} (z - z_0)^m \right) = (z - z_0)^N u(z) \]
Furthermore, $u(z_0) = c_N \neq 0$ so, by continuity, there exists a neighborhood of $z_0$ on which $n(z) \neq 0$.  
\end{proof}


\begin{theorem}
Let $f$ be holomorphic on a domain $\Omega$. If $f \equiv 0$ on some open set inside $\Omega$ then $f \equiv 0$ on all of $\Omega$. 
\end{theorem}

\begin{proof}
Define,
\[ \Omega' = \{ z \in \Omega \mid f \equiv 0 \text{ on an open neighborhood of } z \} \]
Clearly $\Omega'$ is open in $\Omega$ because each $z \in \Omega'$ in inside an open neighborhood of $\Omega$ on which $f$ vanishes so is contained in an open neighborhood of $\Omega'$.
\\
Take $z_1 \notin \Omega'$. Thus, $f$ does not vanish identically on every neighborhood of $z$ so there exists a neighborhood $U$ such that $f(z) = (z - z_1)^N u(z)$ for $u(z) \neq 0$. Then $f(z) \neq 0$ on $U \setminus \{z_1\}$. Therefore, $U \subset (\Omega')^C$ because $f$ is nonzero on $U \setminus \{z\}$ and thus cannot be identically zero on any neighborhood of any point of $U$. Thus, $(\Omega')^C$ is open so $\Omega'$ is clopen. However, $\Omega$ is connected and thus $\Omega' = \Omega$.  
\end{proof}


\begin{remark}
The previous theorem opens up the idea of analytic continuation. If two holomorphic functions agree on some open overlap then they are identically equal on their domain of definition. Therefore, if it exists, there is a unique holomorphic (analytic) extension of a holomorphic function to a any larger domain. 
\end{remark}

\begin{proposition}
Let $f : \Omega \to \C$ be holomorphic (and not identically zero) then the set of zeros, $f^{-1}(0)$ is discrete.
\end{proposition}

\begin{proof}
Let $f$ vanish at $z_0$. If $f$ were identically zero on some open neighborhood of $z_0$ then $f$ would be identically zero on $\Omega$. Thus, by the lemma, we can write $f = (z - z_0)^n u(z)$ on some open neighborhood $U$ of $z_0$ where $u(z)$ is nonvanishing on $U$. Furthermore, $(z - z_0)^n$ vanishes exactly at $z_0$ so we have $f^{-1}(0) \cap U = \{ z_0 \}$ so the singleton $\{ z_0 \} \subset f^{-1}(0)$ is open implying that $f^{-1}(0)$ is discrete. 
\end{proof}

\begin{lemma}
Let $X$ be second-countable then every discrete subspace $D \subset X$ is countable.
\end{lemma}

\begin{proof}

\end{proof}

\begin{lemma}
A discrete space $K$ is compact iff it is finite.
\end{lemma}

\begin{proof}

\end{proof}

\begin{lemma}
Let $K$ be compact and $E \subset K$ be closed and discrete then $E$ is finite. 
\end{lemma}

\begin{proof}
Since $E \subset K$ is closed and $K$ is compact then $E$ is compact and $E$ is discrete so $E$ is finite.
\end{proof}

\begin{proposition}
The set $f^{-1}(0)$ is countable and if $K \subset \Omega$ is compact then $K \cap f^{-1}(0)$ is finite. 
\end{proposition}

\begin{proof}
$\Omega \subset \C$ is a secound-countable space (consider rational radius balls with rational centers) and $f^{-1}(0)$ is discrete so it is countable. Furthermore, if $K \subset \Omega$ is compact $K \cap f^{-1}(0) \subset K$ is discrete. Furthermore, $f : \Omega \to \C$ is continuous, $\C$ is Hausdorff so $\{ 0 \}$ is closed and thus $f^{-1}(0)$ is closed which implies that $K \cap f^{-1}(0) \subset K$ is closed and discrete so by the lemma it is finite. 
\end{proof}

\section{Meromorphic Functions}

\begin{definition}
A function $f : \Omega \to \C$ is meromorphic if, near any $z_0 \in \Omega$, it can be written as,
\[ f(z) = \sum_{n \ge - N} c_n (z - z_0)^n \] 
We call $N$ the order of the pole (assuming that $c_n \neq 0$) and $c_{-1}$ the residue at $z_0$. 
\end{definition}


\begin{theorem}[Residue]
Let $f : \Omega \to \C$ be meromorphic and $D \subset \overline{D} \subset \Omega$ be a domain in $\Omega$ with piecewise smooth boundary $\partial D$ such that no poles of $g$ lie on $\partial D$. Then,
\[ \oint_{\partial D} f(z) \d{z} = 2 \pi i  \sum_{p \in D} \Res{f(p)} \]
\end{theorem}

\begin{proof}
We can deform the path $\partial D$ to a sum of small circles of radius $r$ surrounding each pole. Since $f$ is holomorphic on the region $D$ minus these circles the two integrals along these paths (whose difference is the integral over the boundary) are equal. Thus,
\begin{align*}
\oint_{\partial D} f(z) \d{z} - 2 \pi i \sum_{p \in D} \Res{p}{f} & = \sum_{p \in D} \left[ \oint_{\partial B_r(p)}  f(p + z) \d{z}  - 2 \pi i \Res{p}{g}   \right]
\\
& = \sum_{p \in D} \left[ \int_0^{2\pi} i \bigg( f(p + r e^{i\theta}) r e^{i \theta}  - \Res{p}{g} \bigg) \d{\theta}   \right]
\end{align*}
However,
\[ \Res{p}{f} = \lim_{z \to p} (z - p) f(z) = \lim_{h \to 0} f(p + h) h \]
and thus, for each $\epsilon > 0$ we can choose some $\delta$ such that $r < \delta$ implies that,
\[ \left| f(z + r r^{i \theta}) r e^{i \theta} - \Res{p}{f} \right| < \epsilon \]
Therefore,
\begin{align*}
\left| \oint_{\partial D} f(z) \d{z} - 2 \pi i \sum_{p \in D} \Res{p}{f} \right| & \le \sum_{p \in D} \left[ \int_0^{2\pi} \Big| f(p + r e^{i\theta}) r e^{i \theta}  - \Res{p}{g} \Big| \d{\theta}   \right]
\\
\le \sum_{p \in D} \int_0^{2 \pi} \epsilon = 2 \pi N \epsilon 
\end{align*}
where $N$ is the number of poles. Since $\epsilon$ is arbitrary,
\[ \oint_{\partial D} f(z) \d{z} = 2 \pi i \sum_{p \in D} \Res{p}{f} \]
\end{proof}

\begin{theorem}
Let $f : \Omega \to \C$ be meromorphic and $D \subset \overline{D} \subset \Omega$ be a domain in $\Omega$ with piecewise $\mathcal{C}^1$ boundary $\partial D$ such that no poles of $g$ lie on $\partial D$. Then,
\[ 
\frac{1}{2 \pi i} \oint_{\partial D} \frac{f'(z)}{f(z)} \d{z} = \text{(\# of zeros)} - \text{(\# of poles)}
\]
\end{theorem}

\begin{proof}
At each point $p \in D$ we can expand,
\[ f(z) = (z - p)^N u(z) \]
where $u$ is holomorphic and nonvanishing. Therefore,
\[ \frac{f'(z)}{f(z)} = \deriv{}{z} \log{f(z)} = \deriv{}{z} \left[ (z - p)^N u(z) \right] = \frac{N}{x - p} + \frac{u'(z)}{u(z)} \]
Thus when $f$ has either a zero ($N > 0$) or a pole ($N < 0$) the logarithmic derivative has residue,
\[ \Res{p}{\left(\frac{f'}{f}\right)} = N \]
Therefore the result holds by the residue theorem. 
\end{proof}

\begin{corollary}
Let $f : \Omega \to \C$ be holomorphic take $w \in \C$, then the number of solutions in $D$ to the equation $f(z) - w = 0$ is equal to,
\[ \#\{ z \in D \mid f(z) = w \} = \oint_{\partial D} \frac{f'(z)}{f(z) - w} \d{z}  \]
\end{corollary}

\begin{proof}
Since $f - w$ is holomorphic on $\Omega$ is has no poles. Therefore, the only residues are from roots of $f - w$ i.e. solutions to $f(z) - w = 0$. As above, the integral of the logarithmic derivative counts the number of such poles.  
\end{proof}


\section{Analytic Continuation}

\begin{definition}
Let $\Omega \subset \Omega' \subset \C$ be domians and $f : \Omega \to \C$ holomorphic. We say that a holomorphic function $g : \Omega' \to \C$ is a \textit{(direct) analytic continuation} of $f$ if $g|_\Omega = f$. 
\end{definition}

\begin{lemma}
Any analytic continuation of $f : \Omega \to \C$ to $g : \Omega' \to \C$ is unique.
\end{lemma}

\begin{proof}
Suppose that $g, g' : \Omega' \to \C$ both analyticially continue $f$. Then $g|_\Omega = g'|_\Omega = f$ so $(g - g')|_\Omega = 0$. Since $\Omega \subset \Omega'$ is open then $g - g' \equiv 0$ on all of $\Omega'$ so $g = g'$.
\end{proof}

\begin{remark}
We have shown that, if it exits, direct analytic continuation is unique. However, we must now adress the question of existence.
\end{remark}

\begin{definition}
Given a curve $\gamma : I \to C$ and a holomorphic function $f : U \to \C$ with $U$ an open disc about $\gamma(0)$, an \textit{analytic continuation of} $(f, U)$ \textit{along the curve} $g$ is a collection of pairs $(f_t, U_t)$ satisfying
\begin{itemize}
\item $f_0 = f$ and $U_0 = U$
\item for each $t \in I$ the set $U_t$ is a disc centered at $\gamma(t)$ and $f_t : U_t \to \C$ is holomorphic
\item for each $t \in I$ there is some $\epsilon > 0$ such that if $|t' - t| < \epsilon$ then $\gamma(t') \in U_t$ (implying that $U_t \cap U_{t'} \neq \varnothing$) and $(f_t - f_{t'})|_{U_t \cap U_{t'}} = 0$
\end{itemize}
\end{definition}

\begin{example}
Draw picture for $z^2$, $e^z$, $z^{1/n}$, $\log{z}$ and show failure.
\end{example}

\begin{theorem}[Monodromy]
Let $f : \Omega \to \C$ be holomorphic and let $\gamma_0, \gamma_1 : I \to \C$ be paths with equal endpoints. Suppose there exists a path homotopy $\gamma : I^2 \to C$ from $\gamma_0$ to $\gamma_1$ such that $f$ admits analytic continuation along each curve $\gamma_s$. Then the analytic continuations of $f$ along $\gamma_0$ and $\gamma_1$ yield the same function in a neighborhood of $\gamma_0(1) = \gamma_1(1)$. 
\end{theorem}


\begin{corollary}
Let $\Omega$ be a simply-conncted domain and $f : U \to \C$ be holomorphic on some disc $U \subset \C$. Suppose further that $f$ has an analytic continuation along any curve in $\Omega$ then $f$ admits a direct analytic continuation to $\Omega$.
\end{corollary}

\begin{proof}
Since all paths with the same endpoints are homotopic in the simply-connected domain $\Omega$, by the principle of monodromy we get a uniquely determined analytic continuation of $f$ by continuing $f$ along any curve from $U$ to a desired point $z \in \Omega$. 
\end{proof}

\begin{example}
Works for $z^2$ and $e^z$ not for $z^{1/n}$ or $\log{z}$.
\end{example}

\section{Monodromy}

\subsection{Covering Maps}

\begin{definition}
A \text{fiber bundle} $p : X \to B$ with fiber $F$ if there exists an open cover $\mathcal{U}$ of $B$ such that for each $U \in \mathcal{U}$ there exists a local trivialization homeomorphisms $e_U : p^{-1}(U) \to U \times F$ s.t. the diagram,
\begin{center}
\begin{tikzcd}[row sep = huge]
p^{-1}(U) \arrow[rr, "e_U"] \arrow[rd, "p"'] & & U \times F \arrow[dl, "\text{proj}_1"]
\\
& U 
\end{tikzcd}
\end{center} 
commutes for each $U$. 
\end{definition}

\begin{example}
Trivial Bundles, the Mobius bundle, Hopf fibration
\end{example}

\begin{definition}
A \textit{covering map} $p : \tilde{X} \to X$ with fiber $\Lambda$ is a fiber bundle  such that the fiber $\Lambda$ is a discrete space.  
\end{definition}

\begin{proposition}[Lifting]
Let $p : (\tilde{X}, \tilde{x}_0) \to (X, x_0)$ is a covering map and a continous map $f : (Y, y_0) \to (X, x_0)$ with $Y$ path-connected and locally path-connected. Then there exists a lift $\tilde{f} : (Y, y_0) \to (\tilde{X}, \tilde{x}_0)$ of $f$ iff $f_*(\pi_1(Y,y_0)) \subset p_*(\pi_1(\tilde{X}, \tilde{x}_0))$. Furthermore, such a lift is unique. 
\end{proposition}

\begin{proposition}[Homotopy Lifting]
Given a covering map $p : \tilde{X} \to X$, a homotopy $f : Y \times I \to X$, and a map $\tilde{f}_0 : Y \to \tilde{X}$ lifting $f_0$ there exists a lift $\tilde{f}_t : Y \to \tilde{X}$ such that $p \circ \tilde{f}_t = f_t$.  
\end{proposition}

\begin{proposition}[Path Lifting]
Let $p : \tilde{X} \to X$ be a covering map. Fix a base point $x_0 \in X$ and a lift $\tilde{x}_0$. Then any path $\gamma : I \to X$ at $x_0$ can be lifted uniquely to $\tilde{\gamma} : I \to \tilde{X}$ at $\tilde{x}_0$. 
\end{proposition}

\begin{proof}
Take $Y$ to be a point in the above. Then, given a map $Y \to \tilde{X}$ i.e. a point $\tilde{x}_0 \in \tilde{X}$ and a homotopy $\gamma : I \to X$ then there is a unique lift $\tilde{\gamma} : I \to \tilde{X}$ such that $\gamma(0) = \tilde{x}_0$. 
\end{proof}

\begin{proposition}
Let $p : (\tilde{X}, \tilde{x}_0) \to (X, x_0)$ be a covering map. Suppose that two paths $\gamma_0, \gamma_1 : I \to X$ at $x_0$ are homotopic via a path-homotopy $\gamma : I^2 \to X$. These lift uniquely to paths $\tilde{\gamma}_0, \tilde{\gamma}_1 : I \to \tilde{X}$ at $\tilde{x}_0$ and a path-homotopy $\tilde{\gamma} : I^2 \to \tilde{X}$.
\end{proposition}

\begin{proof}
We know there is an ordinary homotopy $\tilde{\gamma} : I^2 \to \tilde{X}$. Now, $\tilde{\gamma}_t(\text{endpoints})$ as a function of $t$ is a lift of the constant path since $p \circ \tilde{\gamma}_t = \gamma_t$ which fixes the endpoints. By the uniqueness of path lifts (also the fiber $p^{-1}(x_0)$ is discrete) this lift must be constant so $\tilde{\gamma}$ is a path-homotopy. 
\end{proof}

\begin{corollary}
Let $p : (\tilde{X}, x_0) \to (X, x_0)$ be a covering map. Then the induced map $p_* : \pi_1(\tilde{X}, x_0) \to \pi_1(X, x_0)$ is injective.
\end{corollary}

\begin{proof}
Suppose that $p_*([\tilde{\gamma}_0]) = p_*([\tilde{\gamma}_1]$ then we have $p \circ \tilde{\gamma}_0$ is homotopic to $p \circ \tilde{\gamma}_1$. Since $\tilde{\gamma}_0$ and $\tilde{\gamma}_1$ are lifts of these paths, by homotopy lifting, $[\tilde{\gamma}_0] = [\tilde{\gamma}_1]$. 
\end{proof}

\begin{remark}
In particular, any two homotopic paths in $X$ lifted at the same basepoint will have the same endpoint since they are related by a path-homotopy in $\tilde{X}$. This allows the definition of the monodromy action of $\pi_1(X, x_0)$.  
\end{remark}

\begin{proposition}
Let $p : \tilde{X} \to X$ be a covering map and $x_0 \in X$. There is a map $\Phi : \pi_1(X, x_0) \to \Aut{p^{-1}(x_0)}$ called the monodromy action.  
\end{proposition}

\begin{proof}
Given a loop $\gamma$ at $x_0$ we get a permutation $\Phi_\gamma : p^{-1}(x_0) \to p^{-1}(x_0)$ as follows. $L_\gamma(y) = z$ if the unique lift $\tilde{\gamma}$ of $\gamma$ at $y$ has $\tilde{\gamma}(1) = z$. We have shown that endpoints are homotopy invariant so $\Phi_\gamma$ depends only on the class $[\gamma]$. Furthermore, consider two classes, $[\gamma], [\delta] \in \pi_1(X, x_0)$. Then $\Phi_{[\delta] * [\gamma]}(y) = \Phi_{[\delta * \gamma]}(y)$ is defined by a lift of $\delta * \gamma$ at $y$ which is given by lifting $\gamma$ at $y$ then $\delta$ at $\tilde{\gamma}(1)$ to give $\tilde{\delta}(1)$ which is exactly $\Phi_{[\delta]} \circ \Phi_{[\gamma]}$. Thus $\Phi$ is an homomorphism. 
\end{proof}

\begin{remark}
A related notion to that of monodromy is the action of the group of deck transformations which we will now define.
\end{remark}

\begin{definition}
Let $p : \tilde{X} \to X$ be a covering map. A deck transformation is a homeomorphism $f : \tilde{X} \to \tilde{X}$ such that $p \circ f = p$ i.e. it is an automorphism of covering spaces of $X$. We write $\Aut{p}$ for the group of deck transformations of $p$. 
\end{definition}

\begin{proposition}
The group $\Aut{p}$ acts freely on the fiber $p^{-1}(x_0)$.
\end{proposition}

\begin{proof}
Any deck transformation is a lift of $p$ and thus is unique up to a choise of basepoint. Therefore, two deck transformations which argree on the fiber are equal.
\end{proof}

\begin{definition}
We say that a cover $p : \tilde{X} \to X$ is \textit{normal} if $\Aut{p}$ acts transitivly on the fiber $p^{-1}(x_0)$. 
\end{definition}

\begin{theorem}[Galois Correspondence of Covering Maps]

\end{theorem}

\subsection{Germs}

\begin{definition}
For any open set $U \subset \C$ we let $\O(U)$ denote the $\C$-algebra of holomorphic functions $f : U \to \C$ on $U$. These data together make the sheaf $\O$ of holomorphic functions.
\end{definition}

\begin{definition}
Take $z \in \C$. A \textit{germ} at $z$ is an equivalence class of pairs $(f, U)$ with $f \in \O(U)$ and $z \in U$ where $(f_1, U_1) \sim (f_2, U_2)$ if there exists an open $V \subset U_1 \cap U_2$ such that $f_1 |_V =f_2 |_V$. We denote the $\C$-algebra of germs by $\O_z$. 
\end{definition}

\begin{remark}
We think of germs as functions with arbitrarily small domains of definition. Because of the uniqueness of analytic continuations if $(f_1, U_1) \sim (f_2, U_2)$ then $f_1 |_{U_1 \cap U_2} = f_2 |_{U_1 \cap U_2}$ so $f_1$ and $f_2$ glue to give a holomorphic function on $U_1 \cup U_2$. Thus the space of holomorphic germs can be represented by maximally extended holomorphic functions whose domains include $z$. 
\end{remark}

\begin{remark}
The germs $\O_z$ are a special case of the stalk of a sheaf which is defined as,
\[ \F_x = \varinjlim_{x \in U} \F(U) \]
\end{remark}

\begin{definition}
Let $A$ be an $R$-algebra. Then an $R$-derivation is a $R$-linear map $\d : A \to A$ satisfying the Leibniz rule for all $a, b \in A$,
\[ \d{(ab)} = \d{(a)}b + a \d{(b)} \]
We call an $R$-algebra with an $R$-derivation a differential $R$-algebra.
\end{definition}

\begin{definition}
we define the following differential $\C$-algebras,
\begin{enumerate}
\item $\O(\Omega)$ with complex differentiation $\d{f} = f'$
\item $\C[z]$ the ring of polynomials
\item $\C[[z]]$ the (local)ring of formal power series
\item $\C(z)$ the field of rational functions
\item $\C((z))$ the field of Laurent series (bounded below in degree) 
\item $\C\{ z \}$ the subalgebra of $\C[[z]]$ of convergent power series
\item $\C(\{ z \})$ the subalgebra of convergent Laurent series.
\end{enumerate}
Note that $\C((z)) = \C[[z]][z^{-1}]$ is the field of fractions of $\C[[z]]$ and $\C( \{ z \}) = \C\{ z \}[z^{-1}]$ is the field of fractions of $\C\{z \}$. 
\end{definition}

\begin{definition}
A morphism of differential $R$-algebras $f : A \to B$ is an $R$-algebra map preserving the differential, $\d{f(a)} = f(\d{a})$. 
\end{definition}

\begin{proposition}
The map $\Omega(U) \to \O_z$ is an inclusion of differential $\C$-algebras making $\Omega(U)$ a differential subalgebra. 
\end{proposition}

\begin{proposition}
There is an isomorphism of differential $\C$-algebras,
\[ \O_w \to \C\{ z - w \} \]
sending a germ to its convergent power series representation at $w$. 
\end{proposition}

\begin{proof}
Two power series define the same function on some open neighborhood exactly when they agree term by term and furthermore every holomorphic function has a power series representation on some open neighborhood. These facts prove that the given map is an isomorphism. 
\end{proof}

\begin{remark}
Notice in the definition of analytic continuation along a curve we only need the initial function to be defined on some small disc. Therefore, we can apply analytic continuation along a curve to germs and consider only the germ class along the curve. This will be our approach in the next section.
\end{remark}

\subsection{Monodromy Actions}

\begin{definition}
Let $\gamma : I \to \Omega$ be a $\Class{1}{I}$ curve from $a$ to $b$ and $f \in \O_a$ a germ. Then, if it exists, we write $\gamma \cdot f \in \O_b$ for the result of analytic continuation along $\gamma$. Note that this continuation is uniquely determined as a germ. We write $\tilde{\O}_a^\Omega$ for the differential $\C$-algebra of germs which can be analytically continued along any curve from $a$. 
\end{definition}

\begin{remark}
Differentiating every function in an analytic continuation along a curve gives an analytic continuation of the derivative justifying why $\tilde{\O}_a^\Omega$ is a differential sub-algebra of $\O_a$.
\end{remark}

\begin{proposition}
The map $f \mapsto \gamma \cdot f$ satisfies,
\begin{enumerate}
\item $f \mapsto [\gamma] \cdot f$ is a morphism of differential $\C$-algebras,
\begin{enumerate}
\item $\gamma \cdot (\lambda f + \mu g) = \lambda \gamma \cdot f + \mu \gamma \cdot g$
\item $\gamma \cdot (fg) = (\gamma \cdot f) (\gamma \cdot g)$
\item $\gamma \cdot f' = (\gamma \cdot f)'$
\end{enumerate}
\item the map $f \mapsto \gamma \cdot f$ depends only on the homotopy class $[\gamma]$ this follows from the principle of monodromy which states that,
\[ \gamma_1 \sim \gamma_2 \implies \gamma_1 \cdot f = \gamma_2 \cdot f \]
so we may define a map,
\[ \Pi_1(\Omega, a, b) \to \text{Hom}(\tilde{\O}_a^\Omega, \tilde{\O}_b^\Omega) \]
via $f \mapsto [\gamma] \cdot f = \gamma \cdot f$
\item these maps are natural with respect to composition, if $\gamma_1$ is a path $a$ to $b$ and $\gamma_2$ is a path $b$ to $c$ then,
\[ ([\gamma_2] * [\gamma_1]) \cdot f = (\gamma_2 * \gamma_1) \cdot f = \gamma_2 \cdot (\gamma_1 \cdot f) = [\gamma_2] \cdot ([\gamma_1] \cdot f) \]
because analytically continuing from $a$ to $c$ along $\gamma_2 * \gamma_1$ is the same as analytically continuing from $a$ to $b$ along $\gamma_1$ and then from $b$ to $c$ along $\gamma_2$
\item analytic continuation acts via isomorphism so the image of the defined maps lie in the set of isomorphism of differential $\C$-algebra
\[ \Pi_1(\Omega, a, b) \to \text{Hom}(\tilde{\O}_a^\Omega, \tilde{\O}_b^\Omega) \] 
since we know that inverse paths compose to the identity class (which clearly maps to the identity since it corresponds to analytic continuation along the trivial path) so we conclude that the composition of analytic continuation in opposite directions are inverse so this continuation gives an isomophism
\item taking $a = b$ we have a homomorphism of groups,
\[ \pi_1(\Omega, a) \to \mathrm{Iso}(\tilde{\O}_a^\Omega) \]
and therefore a left action of $\pi_1(\Omega, a)$ on $\mathrm{Iso}(\tilde{\O}_a^\Omega)$. 
\end{enumerate}
\end{proposition}

\begin{remark}
We can fully capture the previous proposition by saying that analytic continuation about a point defined a functor $A : \Pi_1(\Omega) \to \mathrm{DiffAlg}_\C$ from the fundamental groupoid of $\Omega$ to the category of differential $\C$-algebras via $a \mapsto \tilde{\O}_a^\Omega$ on objects and $\Pi_1(\Omega, a, b) \to \text{Hom}(\tilde{\O}_a^\Omega, \tilde{\O}_b^\Omega)$ on hom sets via analytic continuation. 
\end{remark}

\begin{example}
Consider $\Omega = \C^\times$ and the germ of the logarithm at $1 \in \Omega$. Then $\pi_1(\Omega, 1) = \Z$ acts on this germ by $[n] \cdot f = f + 2 \pi i n$ which is the ambiguity in the logarithm which builds up as we perform monodromy about the singularity.
\end{example}

\begin{example}
Consider $\Omega = \C^\times$ and the germ of $z^{1/m}$ at $1 \in \Omega$. Then $\pi_1(\Omega, 1) = \Z$ acts on this germ by $[n] \cdot f = z^{2 \pi i/m} f$ which is the ambiguity in taking roots which builds up as we perform monodromy about the singularity.
\end{example}

\begin{theorem}
The fixed points of $\tilde{\O}_a^\Omega$ under the monodromy action of $\pi_1(\Omega, a)$ are exactly the restrictions of global holomorphic functions $\O(\Omega)$. 
\end{theorem}

\begin{proof}
Every germ in $\tilde{\O}_a^\Omega$ may be analytically extended arbitrarily along any curve. If $f \in \tilde{\O}_a^\Omega$ is fixed by monodromy then its extension along any two curves are equal so the germ admits a well-defined analytic continuation to $\O(\Omega)$. 
\end{proof}

\begin{remark}
This fact is exaclty the vital input in Galois theory that the fixed points of a Galois extension $E / F$ under the action of Galois automorphisms $\mathrm{Gal}(E/F)$ is the base field $F$. 
\end{remark}

\subsection{Complex Differential Equations and Monodromy Representations} 

\begin{proposition}
Let $Q[f] = 0$ be a differential equation in terms of $f$ with holomorphic coefficient functions on $\Omega$. If $f : U \to \C$ is a holomorphic solution and $g : \Omega \to \C$ is an analytic continuation of $f$ then $Q[g] = 0$ i.e. analytic continuations o solution to the equation satisfy the same equation.
\end{proposition}

\begin{proof}
The function $Q[g]$ is holomorphic on $\Omega$ and $Q[g]|_U \equiv 0$ on $U$ since $E$ is local so $Q[g] = Q[f] = 0$ on $U$. Thus $Q[g] \equiv 0$ on all of $\Omega$.
\end{proof}

\begin{corollary}
Analytic continuation along a curve $\gamma$ from $a$ to $b$ of a germ $f \in \tilde{\O}_a^\Omega$ solving $Q[f] = 0$ locally gives a local solution $Q[\gamma \cdot f] = 0$ to the same equation at $b$. 
\end{corollary}

\begin{definition}
Let $Q$ be a differential operator holomorphic on $\Omega$ then define the space of holomorphic solutions $\F^Q(U)$ on $U$ i.e. $f \in \O(U)$ s.t. $Q[f] = 0$. This forms a subsheaf $\F^Q$ of $\O$ (as a sheaf of sets not differential algebras). In particular, if $Q$ is a linear differential operator then $\F^Q(U)$ is a $\C$-vectorspace so $\F^Q$ is a subsheaf of $\C$-modules. The stalks $\F^Q_w$ are the germs of holomorphic solutions at $w$.  
\end{definition}

\begin{proposition}
Let $\Omega$ be such that all germs $\F^Q_a$ can be extended analytically continued along any path in $\Omega$. Then the same argument as before defines the monodromy functor $\Pi_1(\Omega, a, b) \to \text{Iso}(\F^Q_a, \F^Q_b)$ and thus a monodromy action
action of $\pi_1(\Omega, w)$ on $\F^Q_w$. In the case that $Q$ is linear then we have seen that this monodromy functor takes $\Pi_1(\Omega) \to \text{Vect}_\C$ so the action is $\C$-linear i.e. defining a group homomorphism,
\[ \pi_1(\Omega, w) \to \GL{}{\F_w} \]
we call this map the \textit{monodromy representation} of $Q$ at $a$. 
\end{proposition}

\begin{example}
We wish to compute $z^\alpha$ for $\alpha \in \C$. We may note that such a function satisfies the differential equation $z f' = \alpha f$ which is linear. Furthermore, it is first-order so we expect $\F^Q(U)$ to be one-dimensional when there exist solutions. There always exists a local solution so $\F^Q_w = \{ c z^\alpha \mid c \in \C \}$ is one-dimensional. Then the monodromy action is,
\begin{center}
\begin{tikzcd}
\pi_1(\C^\times, w) \arrow[r] \arrow[d, equals] & \GL{}{\F^Q_w} \arrow[d, equals]
\\
\Z \arrow[r] \arrow[r, "n \mapsto e^{2 \pi i \alpha n}"] & \GL{1}{\C}
\end{tikzcd}
\end{center}
\end{example}

\begin{example}
We wish to compute $\log{z}$ on $\C^\times$. Note that $\log{z}$ satisfies the differential equation,
\[ f' = \frac{1}{z} \]
however this is not homogeneous (not a linear operator). Alternativly, consider,
\[ z f'' = - \frac{1}{z} = -  f' \] 
Therefore, $\log{z}$ also satisfies the linear differential equation,
\[ z f'' + f' = 0 \]  
This is second-order so we expect $\F^Q(U)$ to be two-dimensional when there exist solutions. Formally the solutions are,
\[ f(z) = a \log{z} + b \]
There always exists a local solution so $\F^Q_w = \{ a \log{z} + b \mid a,b \in \C \}$ is two-dimensional. Then the monodromy action is,
\begin{center}
\begin{tikzcd}
\pi_1(\C^\times, w) \arrow[r] \arrow[d, equals] & \GL{}{\F^Q_w} \arrow[d, equals]
\\
\Z \arrow[r] \arrow[r] & \GL{2}{\C}
\end{tikzcd}
\end{center}
where, choosing the basis $1, \log{z}$ the bottom map is,
\[ n \mapsto 
\begin{pmatrix}
1 & 0
\\
2 \pi i n & 1 
\end{pmatrix} \]
which is a representation.
\end{example}

\begin{example}
Consider the function $f = \sqrt{z} \log{(1 - z)}$
which has arbitrary analytic extensions along curves in $\C \setminus \{ 0, 1 \}$ and satisfies the differential equation,
\[ z^2(1 - z) f'' - z f' + \tfrac{1}{4}(3-z) f = 0  \]
This is second-order so we expect $\F^Q(U)$ to be two-dimensional when there exist solutions. Formally the solutions are,
\[ f(z) = a \sqrt{z} \log{(1 - z)} + b \sqrt{z} \]
There always exists a local solution so $\F^Q_w = \{ a \sqrt{z} \log{(1 - z)} + b \sqrt{z} \mid a,b \in \C \}$ is two-dimensional. Then the monodromy action is,
\begin{center}
\begin{tikzcd}
\pi_1(\C \setminus \{ 0, 1 \}, w) \arrow[r] \arrow[d, equals] & \GL{}{\F^Q_w} \arrow[d, equals]
\\
\Z * \Z \arrow[r] \arrow[r] & \GL{2}{\C}
\end{tikzcd}
\end{center}
where, choosing the basis $\sqrt{z} \log{(1 - z)}$, $\sqrt{z}$, the bottom map sends the generators $\Z * \Z = \left< \gamma_0, \gamma_1 \right>$ where $\gamma_0$ is a small loop about $0$ and $\gamma_1$ is a small loop about $1$ to,
\[ \gamma_0 \mapsto 
\begin{pmatrix}
-1 & 0
\\
0 & -1 
\end{pmatrix}  \quad \quad \gamma_1 \mapsto 
\begin{pmatrix}
1 & 0
\\
- 2 \pi i & 1 
\end{pmatrix} \]
which is a representation.
\end{example}

\section{Singularities}

\begin{remark}
Consider the differential equations with basis of solutions,
\begin{align*}
z f' - f & = 0 \quad \quad \{ z \} 
\\
z f' + f & = 0 \quad \quad \{ \frac{1}{z} \}
\\
z^2 f' + f & = 0 \quad \quad \{ e^{\frac{1}{z}} \}  
\end{align*}
All three have trivial monodromy so monodromy alone in insufficient to characterize solutions near a singularity.
\end{remark}

\begin{definition}
For $A, B \in \mathrm{Mat}_n(\C(\{ z\}))$ then $A \sim B$ if $F \in \GL{n}{\C(\{z \})}$ such that $F[A] = B$ i.e. $F' = BF - FA$.
\end{definition}

\begin{definition}
$\dot{D} = \{ z \in \C \mid |z| \in (0,1) \}$ 
\end{definition}

\begin{definition}
A multivalued function on $\dot{D}$ is a collection $F$ of all the germs obtained by some analytic continuouation of $f$ arround all loops.
\end{definition}

\begin{definition}
For any open $U \subset \dot{D}$ an analytic function on $U$ is a determination of $F$ on $U$ if all its germs belong in $F$.
\end{definition}

\begin{definition}
A multivalued function $F$ has moderate growth in sectors $D \cap S_{a,b}$ wher,
\[ S_{a,b} = \{ re^{i \theta} \mid r > 0 \text{ and } a < \theta < b \} \]
if for any determination $f$ of $F$ on $U = \dot{D} \cap S_{a,b}$ there exists $N$ such that $f(z) = O(z^{-N})$ as $z \to 0$ in $U$. 
\end{definition}

\begin{example}
Any determination of the logarithm on $\dot{D} \cap S_{a,b}$ is of the form $L(r e^{i \theta}) = \log{r} + i \theta + 2 \pi i n$ for some $n \in \Z$. Then, $L(z)  = O(z^{-1})$ as $z \to 0$. 
\end{example}

\begin{example}
Any determination of $z^\alpha$ on $\dot{D} \cap S_{a,b}$ is of the form, 
\[ f(r e^{i \theta}) = r^\alpha \left[ \cos{(\theta + 2 \pi i n) \alpha} + i \sin{(\theta + 2 \pi i n) \alpha} \right] \]
\end{example}

\begin{remark}
$z^A = \exp(A \log{z})$ for a matrix $A$ has moderate growth. If $f$ is uniform on $\dot{D}$ then $f$ having moderate growth is equivalent to $f$ being meromorphic on $D$. 
\end{remark}

\begin{theorem}
If $f$ has moderate growth in sectior in $\dot{D}$ then so does $f'$. 
\end{theorem}

\end{document}

