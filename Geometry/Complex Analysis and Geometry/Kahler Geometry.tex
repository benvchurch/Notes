\documentclass[12pt]{extarticle}
\usepackage[utf8]{inputenc}
\usepackage[english]{babel}
\usepackage[a4paper, total={6in, 9in}]{geometry}
\usepackage{tikz-cd}
\usepackage{mathrsfs}  
 
\usepackage{amsthm, amssymb, amsmath, centernot}

\newcommand{\notimplies}{%
  \mathrel{{\ooalign{\hidewidth$\not\phantom{=}$\hidewidth\cr$\implies$}}}}
 
\renewcommand\qedsymbol{$\square$}
\newcommand{\cont}{$\boxtimes$}
\newcommand{\divides}{\mid}
\newcommand{\ndivides}{\centernot \mid}
\newcommand{\Z}{\mathbb{Z}}
\newcommand{\R}{\mathbb{R}}
\newcommand{\N}{\mathbb{N}}
\newcommand{\C}{\mathbb{C}}
\newcommand{\Zplus}{\mathbb{Z}^{+}}
\newcommand{\Primes}{\mathbb{P}}
\newcommand{\colim}[1]{\mathrm{colim}(#1)}
\newcommand{\Ob}[1]{\mathrm{Ob}(#1)}
\newcommand{\cat}[1]{\mathcal{#1}}
\newcommand{\id}{\mathrm{id}}
\newcommand{\Hom}[2]{\mathrm{Hom}\left( #1, #2 \right)}
\newcommand{\catHom}[3]{\mathrm{Hom}_{#1}\left( #2, #3 \right)}
\newcommand{\Top}{\mathbf{Top}}
\newcommand{\pTop}{\mathbf{Top}_{\bullet}}
\newcommand{\Set}{\mathbf{Set}}
\newcommand{\pSet}{\mathbf{Set}_\bullet}
\newcommand{\hTop}{\mathbf{hTop}}
\newcommand{\phTop}{\mathbf{hTop}_{\bullet}}
\renewcommand{\Im}[1]{\mathrm{Im}(#1)}
\newcommand{\homspace}[2]{\left< #1, #2 \right>}
\newcommand{\rp}{\mathbb{RP}}
\newcommand{\coker}[1]{\mathrm{coker}\: #1}
\newcommand{\Tr}[1]{\mathrm{Tr}\left( #1 \right)}

\renewcommand{\d}[1]{\: \mathrm{d}#1 \:}
\newcommand{\dn}[2]{\: \mathrm{d}^{#1} #2 \:}
\newcommand{\deriv}[2]{\frac{\d{#1}}{\d{#2}}}
\newcommand{\nderiv}[3]{\frac{\dn{#1}{#2}}{\d{#3^2}}}
\newcommand{\pderiv}[2]{\frac{\partial{#1}}{\partial{#2}}}
\newcommand{\parsq}[2]{\frac{\partial^2{#1}}{\partial{#2}^2}}

\theoremstyle{definition}
\newtheorem{theorem}{Theorem}[section]
\newtheorem{lemma}[theorem]{Lemma}
\newtheorem{proposition}[theorem]{Proposition}
\newtheorem{example}[theorem]{Example}
\newtheorem{corollary}[theorem]{Corollary}
\newtheorem{remark}{Remark}

\newenvironment{definition}[1][Definition:]{\begin{trivlist}
\item[\hskip \labelsep {\bfseries #1}]}{\end{trivlist}}


\newenvironment{lproof}{\begin{proof} \renewcommand{\qedsymbol}{}}{\end{proof}}
\renewcommand{\mod}[3]{\: #1 \equiv #2 \: mod \: #3 \:}
\newcommand{\nmod}[3]{\: #1 \centernot \equiv #2 \: mod \: #3 \:}
\newcommand{\ndiv}{\hspace{-4pt}\not \divides \hspace{2pt}}
\newcommand{\gen}[1]{\langle #1 \rangle}
\newcommand{\hook}{\hookrightarrow}
\newcommand{\Tor}[4]{\mathrm{Tor}^{#1}_{#2} \left( #3, #4 \right)}
\newcommand{\Ext}[4]{\mathrm{Ext}^{#1}_{#2} \left( #3, #4 \right)}

\tikzset{
    labl/.style={anchor=south, rotate=90, inner sep=.5mm}
}

\renewcommand{\bf}[1]{\mathbf{#1}}
\newcommand{\Class}[2]{\mathcal{C}^{#1} \left( #2 \right)}
\newcommand{\Res}[2]{\mathrm{Res}_{#1} \: #2}
\newcommand{\F}{\mathcal{F}}
\newcommand{\G}{\mathcal{G}}
\renewcommand{\O}{\mathcal{O}}

\newcommand{\Xcut}{X_{\text{cut}}}
\newcommand{\CP}{\mathbb{CP}}

\begin{document}

\title{Complex Geometry}
\author{Ben Church}
\maketitle
\tableofcontents
\newpage

\section{K\"{a}hler Forms}

\begin{definition}
A pair $(V, I)$ is vector space with a complex structure if $I$ is an $\mathbb{R}$-linear endomorphism s.t. $I^2 = - \id$. We say an $\R$-bilinear form $h : V \times V \to \C$ is a \textit{hermitian} form on $(V, I)$ if,
\begin{enumerate}
\item $\overline{h(v,u)} = h(u,v)$
\item $h(I v, u) = i h(v, u)$
\end{enumerate} 
and $h$ is a \textit{hermitian metric} if furthermore,
\begin{enumerate}
\item $h(v, v) \ge 0$ 
\item $h(v, v) = 0 \iff v = 0$.
\end{enumerate}
\end{definition}

\begin{proposition}
A complex structure $I$ induces a $\C$-vectorspace decomposition $V_\C = V^{1,0} \oplus V^{0,1}$ such that $(V, I) \to (V^{1,0}, i)$ is an isomorphism of complex structures. Furthermore, $V_\C = V^{1,0} \oplus V^{0,1}$ is a Hodge structure and such a weight $1$ Hodge structure is equivalent to the complex structure $(V, I)$. 
\end{proposition}

\begin{proof}
Let $(V, I)$ be a complex structure. Then we $\C$-linearlly extend $I$ to give an endomorphism $I : V_\C \to V_\C$ such that $I^2 + \id = 0$. Therefore, $(I + i \cdot \id)(I - i \cdot \id) = 0$. Given any $v \in V_\C$ consider,
\[ v = \tfrac{1}{2} (\id - i I) v + \tfrac{1}{2} (\id + i I) v \]
Then $(I - i \cdot \id)$ kills the first factor and $(I + i \cdot \id)$ kills the second factor. Furthermore, the map $v \mapsto \tfrac{1}{2} (\id - i I)v$ is an isomorphism $(V, I) \to (V^{1,0}, i)$ since $v \mapsto 0$ iff $I(v) = - i v$ which cannot occur for $v \in V$ and the map is surjective because if $w \in V^{1,0}$ then $I(w) = i w$ and write $w = u_1 + i u_2$ for $u_1, u_2 \in V$ then $I(u_1) = - u_2$ and $I(u_2) = u_1$ meaning that $w = \tfrac{1}{2} (\id - i I) (2u_1)$. Consider the $\C$-antilinear isomorphism $V^{1,0} \to V^{0,1}$ via complex conjugation. This is well-defined because,
\[ I(w) = i w \iff I(\overline{w}) = \overline{I(w)} = - i \overline{w} \]
Conversely, given a Hodge structure $V_\C = V^{1,0} \oplus V^{0,1}$ consider the map $V \to V^{1,0}$ which is an isomorphism because $v^{1,0} = 0 \iff v \in V^{0,1}$ but $V \cap V^{0,1} = (0)$.  
Furthermore, for any $w \in V^{1,0}$ then $w + \bar{w} \in V$ so $V \to V^{1,0}$ is surjective. Thus we get a complex structure on $V$ via pulling back $i$. Explicitly, 
\[ I(v) = i v^{1,0} - i v^{0,1} \quad \text{where} \quad v = v^{1,0} + v^{0,1} \quad \text{and} \quad v^{0,1} = \overline{v^{1,0}} \]
\end{proof}

\begin{remark}
In general, if the minimal polynomial of an operator $T$ splits into linear factors then then $T$ is diagonalizable and its eigenvalues are the roots of the minimal polynomial.
\end{remark}

\newcommand{\inner}[2]{\left< #1, #2 \right>}
\renewcommand{\Re}[1]{\mathrm{Re}\left( #1 \right)}

\begin{definition}
An inner product $\inner{-}{-}$ is compatible with $I$ if $\inner{I(v)}{I(u)} = \inner{v}{u}$.
\end{definition}

\begin{proposition}
Let $\inner{-}{-}$ be a compatible inner product on $(V, I)$ then the associated fundamental form $\omega(v,u) = g(I(v), u)$ is real, antisymmetric and its $\C$-linear extension is of type $(1,1)$ and $I$-invariant.
\end{proposition}

\begin{proof}
Notice $\omega(u,v) = g(I(u), v) = g(v, I(u)) = g(I(v), I^2(u)) = - g(I(v), u) = - \omega(v, u)$ therefore $\omega$ and its $\C$-linear extension are antisymmetric. Furthermore, $\omega(I(u), I(v)) = \omega(u,v)$ because $g$ is $I$-invariant. We need to show that $\omega$ is a $(1,1)$-form. This is equivalent to $I$-invariance since if $v,u \in V^{1,0}$ then,
\[ \omega(u,v) = \omega(I(u), I(v)) = \omega(iu, iv) = i^2 \omega(u,v) = - \omega(u, v) \]
so $\omega(u,v) = 0$. The same holds if $u,v \in V^{0,1}$. 
\end{proof}

\begin{proposition}
The map $h \mapsto - \Im{h}$ gives a one-to-one correspondence between hermitian forms $h$ on $(V, I)$ and real $(1,1)$-forms.  
\end{proposition}

\begin{proof}
Let $h$ be hermitian and consider $\omega = - \Im{h}$ which is antisymmetric because $h$ is conjugate symmetric. Clearly $\omega$ is real. Since $h(I(u), I(v)) = h(u, v)$ we see $\omega$ is $I$-invariant and thus of type $(1,1)$. Conversely, if $\omega$ is a real $(1,1)$-form then consider,
\[ h(u,v) = \omega(u, I(v)) - i \omega(u, v) \]
Clearly $h$ is conjugate symmetric. Furthermore, \begin{align*}
h(I(u), v) & = \omega(I(u), I(v)) - i \omega(I(u), v) = \omega(u, v) + i \omega(u, I(v)) 
\\
& = i \cdot (\omega(u, I(v)) - i \omega(u, v)) = i h(u, v)
\end{align*}
These are inverse operations. First $\omega \mapsto h \mapsto \omega$ is clear. Next $h \mapsto - \Im{h} \mapsto h'$ where,
\[ h'(u,v) = -\Im{h}(u, I(v)) + i \Im{h}(u, v) \]
so we need to show $\Re{h}(u, v) = - \Im{h}(u, I(v))$ but,
\[ h(u, I(v)) = -i h(u, v) \]
so this follows. 
\end{proof}

\begin{proposition}
Let $\inner{-}{-}$ be a compatible inner product on $(V, I)$ then its hermitian extension to $V_\C$ via $\inner{\alpha \otimes u}{\beta \otimes v}_\C = \alpha \bar{\beta}\inner{u}{v}$ satisfies the following properties:
\begin{enumerate}
\item $V_\C = V^{1,0} \oplus V^{0, 1}$ is orthogonal
\item under the isomorphism $(V,I) \to (V^{1,0}, i)$ the metric $2 \cdot \inner{-}{-}_\C$ is the hermitian form associated to the fundamental form $\omega$.
\end{enumerate}
\end{proposition}

\begin{proof}
We can write $u^{1,0} = (\id  - i I)u$ and $v^{0,1} = (\id + i I) v$ then,
\begin{align*}
\inner{u^{1,0}}{v^{0,1}}_\C & = \inner{u - i I(u)}{v + i I(v)}_\C = \inner{u}{v}_\C - i \inner{I(u)}{v}_\C - i \inner{u}{I(v)} - \inner{I(u)}{I(v)}_\C
\\
& = \inner{u}{v}_\C - \inner{u}{v}_\C + i \inner{u}{I(v)}_\C - i \inner{u}{I(v)}_\C = 0
\end{align*}
Furthermore, for $u,v \in V$ consider,
\begin{align*}
2 \inner{\tfrac{1}{2}(\id - i I) u}{\tfrac{1}{2}(\id - i I) v} & = \tfrac{1}{2} \left( \inner{u}{v}_\C + i \inner{u}{I(v)}_\C - i \inner{I(u)}{v}_\C + \inner{I(u)}{I(v)}_\C  \right)
\\
& = \left( \inner{u}{v} + i \inner{u}{I(v)} \right) 
\end{align*}
Furthermore,
\[ h(u, v) = \omega(u, I(v)) - i \omega(u, v) = \inner{I(u)}{I(v)} - i \inner{I(u)}{v} = \inner{u}{v} + i \inner{u}{I(v)} \]
\end{proof}

\begin{remark}
Note that $\omega(u,v) = \inner{I(u)}{v}$ on $V$ however warning $\omega(u, v) \neq \inner{I(u)}{v}_\C$ on $V_\C$ because $\omega$ is extended $\C$-linearly while $\inner{-}{-}$ is extended $\C$-sesquilinearly.
\end{remark}


\begin{remark}
An example of a real $(1,1)$-form.
\[ \tfrac{i}{2} z \wedge \bar{z} = \tfrac{i}{2} (x + i y) \wedge (x - i y) = x \wedge y = \tfrac{1}{2}(z + \bar{z}) \wedge \tfrac{1}{2 i}(z - \bar{z})  \] 
\end{remark}

\begin{remark}
Choose a basis $x_i, y_i \in V$ s.t. $I(x_i) = y_i$. Then we define,
\begin{align*}
z_i & = \tfrac{1}{2}(\id - i I) x_i = \tfrac{1}{2}(x_i - i y_i)
\\
\bar{z}_i & = \tfrac{1}{2}(\id + i I) x_i = \tfrac{1}{2} (x_i + i y_i)
\end{align*}
which are $\C$-bases of $V^{1,0}$ and $V^{0,1}$. Let $\inner{-}{-}$ be $I$-compatible. Then let $\tfrac{1}{2} h_{ij}$ be the matrix coefficients of $\inner{-}{-}_\C$ on $V^{1,0}$,
\[ \inner{z_i}{z_j}_\C = \tfrac{1}{2} h_{ij} \]
which is a positive-definite hermitian matrix. Then,
\begin{align*}
\inner{z_i}{z_j}_\C & = \tfrac{1}{2} h_{ij}
\\
\inner{z_i}{\bar{z}_j}_\C & = 0
\\
\inner{\bar{z}_i}{z_j}_\C & = 0
\\
\inner{\bar{z}_i}{\bar{z}_j}_\C & = \tfrac{1}{2} \bar{h}_{ij}
\end{align*}  
Therefore, using the lemma and the fact that $h$ is Hermitian,
\begin{align*}
h(x_i, x_j) & = h_{ij} 
\\
h(y_i, x_j) & = i h_{ij}
\\
h(x_i, y_j) & = -i h_{ij}
\\
h(y_i, y_j) & = h_{ij}
\end{align*}
Therefore,
\begin{align*}
\inner{x_i}{x_j} & = \Re{h_{ij}}
\\
\inner{y_i}{x_j} & = - \Im{h_{ij}}
\\
\inner{x_i}{y_j} & = \Im{h_{ij}}
\\
\inner{y_i}{y_j} & = \Re{h_{ij}}
\end{align*}
then indeed,
\begin{align*}
\inner{\bar{z}_i}{\bar{z}_j}_\C & = \tfrac{1}{4} \left( \inner{x_i}{x_j} + i \inner{y_i}{x_j} - i \inner{x_i}{y_j} + \inner{y_i}{y_j} \right)
\\
& = \tfrac{1}{2} \left( \Re{h_{ij}} - i \Im{h_{ij}} \right) = \tfrac{1}{2} \bar{h}_{ij}
\end{align*}
and likewise,
\begin{align*}
\omega(x_i, x_j) & = -\Im{h_{ij}}
\\
\omega(y_i, x_j) & = - \Re{h_{ij}}
\\
\omega(x_i, y_j) & = \Re{h_{ij}}
\\
\omega(y_i, y_j) & = - \Im{h_{ij}}
\end{align*}
Finally, extending $\C$-linearly we can compute,
\begin{align*}
\omega(z_i, z_j) & = \tfrac{1}{4} (\omega(x_i, x_j) - i \omega(x_i, y_j) - i \omega(y_i, x_j) - \omega(y_i, y_j) 
\\
& = - \tfrac{1}{4} (\Im{h_{ij}} + i \Re{h_{ij}} - i \Re{h_{ij}} - \Im{h_{ij}}) = 0
\end{align*}
which we already knew because $\omega$ is a $(1,1)$-form. Furthermore, 
\begin{align*}
\omega(z_i, \bar{z}_j) & = \tfrac{1}{4} (\omega(x_i, x_j) + i \omega(x_i, y_j) - i \omega(y_i, x_j) + \omega(y_i, y_j) 
\\
& = \tfrac{1}{2} (-\Im{h_{ij}} + i \Re{h_{ij}} + i \Re{h_{ij}} - \Im{h_{ij}}) = \tfrac{i}{2} h_{ij}
\end{align*}
Thus,
\begin{align*}
\omega(z_i, z_j) & = 0
\\
\omega(z_i, \bar{z}_j) & = \tfrac{i}{2} h_{ij}
\\
\omega(\bar{z}_i, z_j) & = \overline{\tfrac{i}{2} h_{ij}} = - \tfrac{i}{2} \bar{h}_{ij} = - \tfrac{i}{2} h_{ji}
\\
\omega(\bar{z}_i, \bar{z}_j) & = 0
\end{align*}
Therefore, we see that,
\[ \omega = \frac{i}{2} \sum_{i,j = 1}^n h_{ij} \, \d{z^i} \wedge \d{\bar{z}^j} \]
\end{remark}


\section{Complex Manifolds}

\begin{definition}
Let $M$ be a topological space. If there exists an open cover of $M$ by open charts $(U_\alpha, \varphi_\alpha)$ where $\varphi_\alpha : U_\alpha \to \C^n$ is a homeomorphism onto its image such that,
\[ \varphi_\alpha \circ \varphi^{-1}_\beta : \varphi_\beta(U_\alpha \cap U_\beta) \to \varphi_\alpha(U_\alpha \cap U_\beta) \]
is a holomorphic map on subsets of $\C^n$. Then $M$ is a complex manifold of dimension $\dim_{\C}(M) = n$ and real dimension $\dim_{\R}(M) = 2n$ as a real smooth manifold. 
\end{definition}

\begin{definition}
A continuous map $f : M \to \C$ on a complex manifold is \textit{holomorphic} if $f \circ \varphi_{\alpha}^{-1} : \varphi_\alpha(U_\alpha) \to \C$ is holomorphic as a map of complex space. 
\end{definition}

\begin{proposition}
$\CP^{n} = (\C^{n+1} \setminus \{0\}) / \C^\times$ is a complex manifold of dimension $n$. 
\end{proposition}

\begin{definition}
Let $M$ be a real smooth manifold. An endomorphism $J : TM \to TM$ is called an \textit{almost complex structure} on $M$ if $J^2 = - \id_{TM}$. 
\end{definition}

\begin{proposition}
If $M$ admits an almost complex structure then $\dim_{\R}(M)$ is even.
\end{proposition}

\begin{proof}
On the tangent space at a point, $J_p : T_p M \to T_p M$ is an endomorphism such that $J_p^2 = - \id$. Therefore, $\det{J_p^2} = \det{(-\id)} = (-1)^{\dim_{\R}(M)}$. However, $(\det{J_p^2})^2$ is positive so $\dim_{\R}(M)$ is even.   
\end{proof}

\begin{proposition}
If $M$ is a complex $n$-manifold then for each $p \in M$ then $T_p M \cong \C^{n}$ then $J : T_p M \to T_p M$ via $u \mapsto i u$ is an almost complex structure which is independent of the choice of complex coordinates.
\end{proposition}

\begin{proof}

\end{proof}

\begin{definition}
An almost complex structure is \textit{integrable} if it is induced from a complex structure. Thus a complex manifold is a manifold with an integrable almost complex structure. 
\end{definition}

\begin{definition} 
The Nijenhuis tensor is $N_J : TM \times \to TM$ given by,
\[ N_J(X, ) = [X, Y] + J[X, JY] + J[JX, U] - [JX, JY] \]
\end{definition}

\begin{theorem}
If $N_J$ vanishes then $J$ is integrable. In this case, at each point there exist local holomorphic coordinates s.t. $J$ acts as locally as multiplication by $i$. 
\end{theorem}

\begin{definition}
The complex tangent space is defined by,
\[ T_\C M = TM \otimes_{\R} \C \]
Then $J : T_\C M \to T_\C M$ has eigenvalues $\pm i$. Therefore, we can decompose $T_\C M$ into its $J$-eigenspaces,
\[ T_\C M = T^{1,0} M \oplus T^{0, 1} M \]
\end{definition}

\section{Hermitian K\"{a}hler Metrics}

\begin{definition}
Let $(M, g)$ be a Riemannian manifold and $J$ an almost complex structure. Then the Riemannian metric $g$ is \textit{Hermitian} i.e. $J$-invariant if for all $X, Y \in \mathscr{X}(M)$,
\[ g(J X, J Y) = g(X, Y) \] 
\end{definition}

\begin{remark}
Every complex manifold admits a Hermitian metric.
\end{remark}

\begin{proposition}
If $g$ is Hermitian then $g(X, JX) = g(JX, J^2 X) = - g(JX, X) = - g(X, J X) \implies g(X, JX) = 0$. Therefore, $X$ and $JX$ are always orthogonal. 
\end{proposition}

\begin{lemma}
Define the tensor $\omega(X, Y) = g(JX, Y)$ then $\omega(X, Y) = - \omega(Y, X)$. Therefore, $\omega$ is a real $2$-form (actually a $(1,1)$-form). 
\end{lemma}

\begin{proof}
\[ \omega(X, Y) = g(J X, Y) = g(J^2 X, J Y) = - g(X, J Y) = - g(JY, X) = - \omega(Y, X) \]
\end{proof}

\begin{remark}
For the above discussion if suffices for $J$ to define an almost complex structure (integrability is not required). 
\end{remark}

\begin{definition}
A Riemannian manifold $(R, g)$ is K\"{a}hler if $g$ is Hermitian and $\omega$ is a $\d{}$-closed $(1,1)$-form. We call $(M, \omega, J)$ a K\"{a}hler manifold. 
\end{definition}

\section{Jan 29}

\begin{lemma}[$\partial \bar{\partial}$]
Whenever $[\omega_1] = [\omega_2]$ are real $(1,0)$ forms representing the same cohomology class then there exists a real function $\varphi : \Class{\infty}{M, \R}$ such that \[ \omega_1 = \omega_2 + i \partial \bar{\partial} \varphi \]
\end{lemma}

\begin{lemma}[normal coordinates for Riemmanian geometry]
Let $(M, g)$ be a Riemannian manifold then for any point $p \in M$ there exists a coordinate frame such that $g_{ij}(p) = \delta_{ij}$ and $\d{g_{ij}}(p) = 0$. 
\end{lemma}

\begin{lemma}[normal coordinates for K\"{a}hler geometry]
Let $(M, \omega, J)$ be a K\"{a}hler manifold. Then for any point $p$ there exists a local chart $U$ with holomorphic coordinates $z_1, \dots, z_n$ such that $g_{i, \bar{j}}(p) = 0$ and 
\[ \deriv{g_{i \bar{j}}(p)}{z_k} = \deriv{g_{i\bar{j}}(p)}{\bar{z}_k} = 0 \]
\end{lemma}

\begin{remark}
The K\"{a}hler property is necessary for the existence of such normal coordinates.
\end{remark}

\begin{proof}
We can choose holomorphic coordinates $w_1, \dots, w_n$ such that $\delta_{i \bar{j}}(p) = \delta_{ij}$ simply by diagonalization and scaling since $g_{i \bar{j}}$ is a Hermitian matrix. In these coordinates, we expand the K\"{a}hler form $\omega$ as,
\[ \omega = \sum \left( \delta_{ij} + \deriv{g_{i \bar{j}}}{w_k} w_k + \deriv{g_{i \bar{j}}}{\bar{w}_k} \bar{w}_k + O(|w|^2) \right) \d{w_i} \wedge \d{\bar{w}_j}  \] 
which we write as,
\[ \omega = \sum \left( \delta_{ij} + a_{i \bar{j} k} w_k + a_{i \bar{j} \bar{k}} \bar{w}_k + O(|w|^2) \right) \]
The K\"{a}hler condition implies that $a_{i \bar{j} k} = a_{k \bar{j} i}$ and $a_{i \bar{j} \bar{k}} = a_{i\bar{k} \bar{j}}$. We can find (locally) coordinates $(z_1, \dots, z_n)$ such that,
\[ w_i = z_i - \tfrac{1}{2} b_{ijk} z_j z_k \]
where $b_{ijk} = b_{ikj}$. Such coordinates exist by the implicit function theorem. Thus we have,
\[ \d{w_i} = \d{z_i} - b_{ijk} z_j \d{z_k} \quad \quad \d{\bar{w}_j} = \d{\bar{z}_j} - \overline{b_{jml}} \bar{z_m} \d{\bar{z_l}} \]
Therefore, in these coordinates, we find,
\begin{align*}
\omega & = \sum \left( \delta_{ij} + a_{i \bar{j} k} z_k + a_{i \bar{j} \bar{k}} \bar{z}_k + O(|z|^2) \right) \left( \d{z_i} - b_{ijk} z_j \d{z_k} \right) \wedge \left( \d{\bar{z}_j} - \overline{b_{jml}} \bar{z_m} \d{\bar{z_l}} \right)
\\
& = \sum \left( \delta_{ij} + [ a_{i \bar{j} k} - b_{jki} ] z_k + [a_{i \bar{j} \bar{k}} - \overline{b_{ikj}}] \bar{z}_k \right) \d{z_i} \wedge \d{z_j} 
\end{align*}
where the quadratic terms in $w_i = z_i - \tfrac{1}{2} b_{ijk} z_j z_k$ are absorbed into the $O(|z|^2)$. Therefore, choose $b_{jki} = a_{i \bar{j} k}$ then, by the Hermitian condition, $\overline{b_{ikj}} = a_{i \bar{j} \bar{k}}$ so the linear terms vanish.  
\end{proof}

\subsection{Covariant Derivative}

\begin{remark}
Given a K\"{a}hler manifold $(M, \omega, J)$ and let $(M, g)$ be the underlying Riemannian manifold. $g$ defines a Levi-Civita connection $\nabla$ so $\nabla J$ is a tensor. In local coordinates,
\[ J = \sum \delta_i^j \sqrt{-1} \pderiv{}{z^j} \otimes \d{z^i} \] 
This tensor has constant coefficients and thus $\nabla J = 0$.	Then $\omega(X, Y) = g(J X, Y)$ which implies that $\nabla \omega = 0$ since $\nabla g = 0$. In local coordinates $(z^1, \dots, z^n)$,
\[ \nabla_{\pderiv{}{z^i}} \left( J \left( \pderiv{}{z_j} \right) \right) = \sqrt{-1} \nabla_{\pderiv{}{z^i}} \pderiv{}{z^j} \]
But,
\[ \nabla_{\pderiv{}{z^i}} \left( J \left( \pderiv{}{z_j} \right) \right) = \left( \nabla_{\pderiv{}{z^i}} J \right) \left( \pderiv{}{z^j} \right) + J \left( \nabla_{\pderiv{}{^i}} \pderiv{}{z^j} \right) \]
But $\nabla J = 0$ so this implies,
\[ J \left( \nabla_{\pderiv{}{^i}} \pderiv{}{z^j} \right) = \sqrt{-1} \nabla_{\pderiv{}{^i}} \pderiv{}{z^j} \]
and therefore $\nabla$ preserves the type $(1, 0)$ of these  vector fields. Thus,
\[ \nabla_{\pderiv{}{z^i}} X \quad \quad \nabla_{\pderiv{}{\bar{z}^i}} X \]
preserve the type of $X$. Therefore, we may define,
\[ \nabla_{\pderiv{}{z^i}} \pderiv{}{z^j} = \Gamma^k_{ij} \pderiv{}{z^k} \]
The fact that $\nabla$ is torsion-free implies that $\Gamma^k_{ij} = \Gamma^k_{ji}$. The torsion-free property also gives,
\[ \nabla_{\pderiv{}{z^i}} \pderiv{}{\bar{z}^j} - \nabla_{\pderiv{}{z^j}} \pderiv{}{\bar{z}^i} = 0 \] 
However, these vector fields have different types and thus each are individually zero. Therefore,
\[ \Gamma_{i \bar{j}}^k = \Gamma^{\bar{k}}_{i \bar{j}} = 0 \]
\end{remark}

\begin{lemma}
Define $g(u, v) = \omega(u, J v)$ which is a metric and let $\nabla_g$ be the associated Levi-Civita connection. Then $\omega$ is K\"{a}hler $\iff \nabla J = 0$.
\end{lemma}	

\begin{lemma}
\[ \Gamma_{ij}^k = g^{k \bar{l}} \pderiv{g_{i \bar{l}}}{z^j} \]
\end{lemma}

\begin{corollary}
In normal coordinates the Christoffel symbols vanish.
\end{corollary}       

\begin{definition}
The Riemman tensor in complex coordinates is defined by,
\[ \left( \nabla_{\pderiv{}{z^k}} \nabla_{\pderiv{}{\bar{z}^l}}  - \nabla_{\pderiv{}{\bar{z}^l}} \nabla_{\pderiv{}{z^k}} \right) \left( \pderiv{}{z^j} \right) = R_{i \: k \bar{l}}^{\: j} \pderiv{}{z^j} \]
The first term of the LHS is zero and the second simplifies to,
\[ - \nabla_{\pderiv{}{z^l}} \left( \Gamma^j_{ki} \pderiv{}{z^j} \right) = - \pderiv{}{\bar{z}^l} \Gamma^j_{ki} \pderiv{}{z^j} \pderiv{}{z^j} \]
Therfore,
\[ R_{i \: k \bar{l}}^{\: j} = - \pderiv{}{\bar{z}^l} \Gamma^j_{ki} \]
which implies that,
\[ R_{i \bar{j} k \bar{l}} = - \frac{\partial^2}{\partial z_k \partial \bar{z}_l} + g^{p \bar{q}} \pderiv{g_{i \bar{q}}}{x^k} \pderiv{g_{p \bar{j}}}{\bar{z}^l} \]
\end{definition}

\section{Jan 31}

\begin{proposition}
\[ R_{i \bar{j} k \bar{l}} = R_{k \bar{j} i \bar{l}} = R_{i \bar{l} k \bar{j}} = R_{k \bar{l} i \bar{j}} \]
\end{proposition}

\begin{proposition}[Bianchi]
\[ \nabla_m R_{i \bar{j} k \bar{l}} = \nabla_i R_{m \bar{j} k \bar{l}} \]
\end{proposition}

\begin{definition}
A K\"{a}hler manifold has constant bisectional curvature if 
\[ R_{i \bar{j} k bar{l}} = \mu \left( g_{i\bar{j}} g_{k \bar{l}} + g_{i \bar{l}} g_{k \bar{j}} \right) \]
\end{definition}

\begin{theorem}[Uniformization]
Let $(M, \omega, J)$ be a K\"{a}hler manifold with constant bisectional curvature and $\pi_1(M) = 0$ then $(M, \omega)$ must be biholomorphic to one of,
\[ (\CP^, \omega_{FS}) \quad \quad (\C^n, \omega_{\C^n}), (B, \omega_P) \]
where $B$ is the hyperbolic ball.
\end{theorem}

\begin{definition}
The Ricci curvature is $R_{k \bar{l}} = g^{ij} R_{i \bar{j} k \bar{l}}$ and Ricci scalar $R = g^{k \bar{l}} R_{k \bar{l}}$.
\end{definition}

\begin{lemma}
\[ R_{k \bar{l}} = - \frac{\partial^2}{\partial z^k \partial \bar{z}^l} \log{\det{g}} \]
\end{lemma}

\newcommand{\Ric}{\mathrm{Ric}}

\begin{definition}
The Ricci form is,
\[ \Ric(\omega) = \sum_{k,l} = R_{k \bar{l}} \sqrt{-1} \d{z^k} \wedge \d{\bar{z}^l} = - \sqrt{-1} \partial \bar{\partial} \log{\det{g}} \] 
Then $\d{\Ric(\omega)} = 0$ and $\Ric$ is a real $(1, 1)$-form.
\end{definition}

\section{The Calabi-Yau Theorem}

\begin{theorem}
The Ricci form represents the first Chern class 
\[ \Ric(\omega) \in 2 \pi c_1(M) \in H^{1,1}(M, \C) \cap H^2(M, \R)\]
Then the map $\omega \mapsto [\Ric(\omega)]$ is surjective onto cohomology classes. 
\end{theorem}

\subsection{Preliminaries}

Let $(M, \omega)$ be a given K\"{a}hler manifold and $\alpha \in 2 \pi c_1(M)$. Then, 
\[ [\Ric(\omega)] = [\alpha] = 2\pi c_1(M)\]
By the $\partial \bar{\partial}$ Lemma,
\[ \Ric(\omega) - \alpha = i \partial \bar{\partial} F \]
Then if $\omega_\varphi = \omega + i \partial \bar{\partial} \varphi > 0$ satisfies,
\[ \Ric(\omega_\varphi) - \Ric(\omega) = \alpha - \Ric(\omega) \]
then we find,
\[ \partial \bar{\partial} \log{ \frac{\omega^n}{\omega_\varphi^n}} = \partial \bar{\partial} (- F) \implies \partial \bar{\partial} \left( \log{\left( \frac{\omega^n}{\omega_\varphi^n} e^F \right)} \right) \] 


\section{Feb. 7}

\[ (*) \quad \quad \quad (\omega + i \partial \bar{\partial} + \varphi_t)^n = e^{t F + C_t} \omega^n \]
Where $C_t$ is chosen such that,
\[ \int e^{t G + C_t} \omega^n = \int \omega^n \]
We want to show that he set,
\[ I = \{ t \in [0, 1] \mid (*) \text{admits a $C^{3, \alpha}$ solution} \} \]
is open, closed, and nonempty. Clearly, $0 \in I$ so we need to show that $I$ is clopen. 

\newcommand{\norm}[3]{||#3||_{C^{#1 #2}}}

\begin{definition}
Let $B \subset \R^n$ be the unit ball. Then $u \to \R$ is said to be $C^{k \alpha}$ for an integer $k$ if the norm,
\[ \norm{u}{k}{\alpha} = \sup \sum_{i = 0}^k |\nabla^i u|(x) + \sup_{x \neq y} \frac{|\nabla^k u(x) - \nabla^k u(y)|}{|x - y|^\alpha} \]
is finite. The set of such functions is called $C^{k, \alpha}(B)$. If $(M, g)$ is a compact Riemannian manifold there exists a finite trivializing cover $(U_a, \varphi_a)$ such that each $U_a$ is homeomorphic to $B$. For a function $f : M \to \R$ we define,
\[ \norm{k}{\alpha}{f} = \sum_a || f \circ \varphi^{-1}_a||_{C^{k, \alpha}(U_\alpha)} \] 
This is not independent of the charts but the set of $f$ with finite norm i.e. $C^{k, \alpha}(M)$ is independent of such choices.   
\end{definition}

\begin{proposition}
$I$ is open.
\end{proposition}

\begin{proof}
Define the map $\Psi : C^{3, \alpha} \times [0, 1] \to C^{1, \alpha}$ via,
\[ (\phi, s) \mapsto \log{\left( \frac{(\omega + i \partial \bar{\partial} \varphi)^n}{\omega^n} \right)} - s F - C_s \]
Therefore, $t \in I \implies \exists \varphi_t : \Psi(\varphi_t, t) = 0$. By the implicit function theorem, it suffices to show that the linearized operator $D \Phi_(\varphi_s, t) : C^{3, \alpha} \times [0, 1] \to C^{1, \alpha}$ is invertible where we define,
\[ D \Psi_{\varphi_t, t}(\phi, 0) = \deriv{}{\theta} \bigg|_{\theta = 0} \Psi(\varphi^\theta, 0) = \Delta_{\omega_{\phi_t}} \phi \] 
where we set,
\[ \deriv{}{t} \bigg|_{\theta = 0} \varphi^\theta = \phi \]
Unfortunately, this operator is not invertable as defined.
However, we see that,
\[ e^{- \Psi(\varphi, s)} \omega_{\phi}^n = e^{s F + C_s} \frac{\omega^n}{\omega_\phi^n} \omega_\phi^n = e^{s F + C_s} \omega^n \implies \int e^{-\Psi(\varphi, s)} \omega_{\varphi}^n = \int \omega^n = \mathrm{Vol}(M, \omega) \]
Thus, we must restrict ourselves to $\varphi^\theta$ such that,
\[ \int e{- \Psi(\varphi^\theta, s)} \omega_{\varphi^\theta}^n = \int \omega^n = \mathrm{Vol}(M, \omega) \]
So let $\delta \Psi = G$ then,
\[ \int G \omega_{\varphi_t}^n = 0 \]
Therefore, consider $C_0^{1, \alpha}$ to be the set of functions satisfying this property i.e.,
\[ \left\{ G \in C_0^{1, \alpha} \: \middle| \: \int G \omega_{\varphi_t}^n = 0 \right\} \]
We need to show that if $G \in C_0^{1, \alpha}$ then $\Delta_{\omega_{\varphi_t}} \phi = G$ alwsays admits a solution. If we furthermore impose the normalization condition,
\[ \int \phi \omega^n = 0 \]
then we also have uniqueness $\Delta_{\omega_{\varphi_t}} \phi_1 = \Delta_{\omega_{\varphi_t}} \phi_2 \implies \phi_1 =  \phi_2$. Thus, invertibility gives us a solution $\Phi(\varphi_s, s) = 0$ in some open about $t$. Furthermore,
\[ || \varphi_s - \varphi_t||_{C^{3, \alpha}} \le C(|s- t|) \ll 1 \]
Therefore, if $\omega + i \partial \bar{\partial} \varphi_t > 0$ then $\omega + i \partial \bar{\partial} \varphi_s > 0$. 
\end{proof}

\subsection{$I$ is closed}

We need to prove $C^0$ and $C^2$ and $C^3$ and $C^{3, \alpha}$ estimates. Suppose these estimates hold for any $\varphi_t$ for $t \in I$. Then if $t_i \in I$ with $t_i \to t_{\infty}$ we need to show that $t_{\infty} \in I$ to prove that $I$ is closed. These estimates show $\varphi_{t_i} \to \varphi_{\infty} \in C^{3, \alpha}$ with $C^{3, \alpha}$ convergence. Furthermore, $C^{-1} \omega \le \omega_{\varphi_{\infty}} \le C \omega$ shows that $\omega_{\varphi_{\infty}}$ is a K\"{a}hler form. 


\subsubsection{$C^2$ estimates}

Let $\tilde{g} = \omega + i \partial \bar{\partial} \varphi$ and $g = \omega$. At $p \in M$ choose normal coordinates for $g$ i.e.
\[ g_{i \bar{j}}(p) = \delta_{ij} \quad \quad \d{g_{i \bar{j}}}(p) = 0 \]
such that $\tilde{g}_{i \bar{j}}(p) = \tilde{g}(p)_{i \bar{i}} \delta_{ij}$ which is accomplished by unitary transformation. 


\section{Feb 14}
\newcommand{\tr}[1]{\mathrm{tr}_{#1} \:}
We are solivng the equation,
\begin{align*}
(\omega + i \partial \bar{\partial} \varphi)^n = e^f \omega^n 
\end{align*}
and we require normalization,
\[ \int \varphi \omega^n = 0 \]
Consider $g = \omega$ and $\tilde{g} = \omega + i \partial \bar{\partial} \varphi > 0$. We have shown the innequality,
\[ \Delta_{\tilde{\omega}} \log{\tr{\omega} \tilde{\omega}} \ge \frac{1}{\tr{\omega}\tilde{\omega}} \left( \tilde{g}^{i \bar{i}} R_{i\bar{i}k\bar{k}} - R - \Delta_\omega f \right) \]
(at a point $p \in M$ with normal coordinates of $g$ and $\tilde{g}$ diagonal at $p$.)
The first term on the left can be simplified to the form,
\begin{align*}
\frac{\tilde{g}_{k \bar{k}}}{\tilde{g}_{i \bar{i}}} R_{i \bar{i} k \bar{k}} - R_{i \bar{i} k \bar{k}} 
& = \frac{1}{2} \left( \frac{\tilde{g}_{k \bar{k}}}{\tilde{g}_{i\bar{i}}} + \frac{\tilde{g}_{i \bar{i}}}{\tilde{g}_{k \bar{k}}}  \right) R_{i \bar{i} k \bar{k}} - R_{i \bar{i} k \bar{k}} 
\\
& = \tfrac{1}{2} R_{i \bar{i} k \bar{k}} \left( \frac{\tilde{g}_{i \bar{i}}}{\tilde{g}_{k \bar{k}}} + \frac{\tilde{g}_{b \bar{k}}}{\tilde{g}_{i \bar{i}}} - 2 \right) 
\\
& \ge \tfrac{1}{2} \inf R_{i \bar{i} k \bar{k}} \sum_{i,k = 1}^n \left( \frac{\tilde{g}_{i \bar{i}}}{\tilde{g}_{k \bar{k}}} + \cdots \right)
\\
& \ge - C_0 \left( \tr{\omega} \tilde{\omega} \tr{\tilde{\omega}} \omega + 1 \right) 
\end{align*}
Therfore,
\[ \Delta_{\tilde{\omega}} \log{\tr{\omega} \tilde{\omega}} \ge - C_0 \tr{\tilde{\omega}} \omega - \frac{C_1}{\tr{\omega} \tilde{\omega}} \]
However, this is not enough to prove the $C_0$ estimate. Furthermore, consider,
\[ \Delta_{\tilde{\omega}} (\varphi) = \tr{\tilde{\omega}}( i \partial \bar{\partial} \varphi) = \tr{\tilde{\omega}} \left( \tilde{\omega} - \omega \right) = n - \tr{\tilde{\omega}} \omega \] 
Thus consider,
\begin{align*}
\Delta_{\tilde{\omega}} \left( \log{\tr{\omega} \tilde{\omega}} - A \varphi \right) \ge - C_0 \tr{\tilde{\omega}} \omega - \frac{C_1}{\tr{\omega} \tilde{\omega}} - A n + A \tr{\tilde{\omega}} \omega
\end{align*}
Then we may take $A = C_0 + 1$ to find,
\begin{align*}
\Delta_{\tilde{\omega}} \left( \log{\tr{\omega} \tilde{\omega}} - A \varphi \right) \ge \tr{\tilde{\omega}} \omega - \frac{C_1}{\tr{\omega} \tilde{\omega}} - C_2
\end{align*}
However we neet to compute innequalities in terms of $\tr{\omega}{\tilde{\omega}}$ rather than $\tr{\tilde{\omega}}{\omega}$. We need the innequality,
\[ \left( \tr{\tilde{\omega}} \omega \right)^{n-1} \frac{\tilde{\omega}^n}{\omega^n} \ge \tr{\omega} \tilde{\omega} \]
Using this innequality we find,
\begin{align*}
\tr{\tilde{\omega}} \omega - \frac{C_1}{\tr{\omega} \tilde{\omega}} - C_2 
& \ge e^{- \frac{f}{n-1}} \left( \tr{\omega}{\tilde{\omega}} \right)^{\frac{1}{n-1}} - \frac{C_1}{\tr{\omega}{\tilde{\omega}}} - C_2
\\
& \ge C_3 (\tr{\omega}{\tilde{\omega}} )^{\frac{1}{n-1}} - \frac{C_1}{\tr{\omega} \tilde{\omega}} - C_2 
\end{align*}
Let $H = \log{\tr{\omega}{\tilde{\omega}}} - A \varphi$ which is a smooth function on a compact manifold and thus must have maximum at $p$ at which the Hessian must be negative semi-definite. Thus,
\[ \Delta_{\tilde{\omega}} H|_p \ge 0 \implies \frac{1}{\tr{\omega}{\tilde{\omega}}} \left( C_3 (\tr{\omega}{\tilde{\omega}})^{\frac{n}{n1}} - C_1 - C_2 \tr{\omega}{\tilde{\omega}} \right) \le 0 \]
Therefore applying Young's innequality,
\[ \frac{1}{p} + \frac{1}{q} = 1 \implies ab \le \frac{a^p}{p} + \frac{a^q}{q} \]
We find that,
\[ C_3 \left( \tr{\omega}{\tilde{\omega}} \right)^{\frac{n}{n-1}} \le C_1 + C_2 \tr{\omega} {\tilde{\omega}} \le C_1 + \epsilon \left( \tr{\omega}{\tilde{\omega}} \right)^{\frac{n}{n-1}} + C(\epsilon) \]
And thus,
\[ \tr{\omega}{\tilde{\omega}} |_p \le C \]
Therefore, because $p$ is the maximum of $H$, we have found that,
\[ H(x) \le H(p) = \log{\tr{\omega}\tilde{\omega}|_p} - A \varphi(p) \le C - A \inf_M \varphi \]
However, by definition,
\[ H(x) = \log{\tr{\omega}{\tilde{\omega}}}(x) - A \varphi(x) \]
Taking the exponential of the given innequality,
\[ \tr{\omega}{\tilde{\omega}}(x) \le C \exp{\left(A \left( \varphi(x) - \inf\limits_M \varphi \right)\right)} \]
This is what we call Yau's $C^2$-estimate. 

\subsection{The $C^0$-estimate}

\begin{proposition}
There exists a uniform constant $C$ such that $||\varphi||_{C^0} \le C \left( ||e^f ||_{L^\infty}, \omega \right)$ where $C$ depends on the parameters $||e^f||_{L^\infty}$ and $\omega$.  
\end{proposition}

\begin{proof}
To come later. 
\end{proof}

Appling the $C^0$-estimate and the above $C^2$-estimate we find,
\[ \tr{\omega}{\tilde{\omega}} \le C' \]
for some uniform constant. This implies that $\tilde{\omega} \le C' \omega$ which implies that every eigenvalue $\lambda_i$ of $\tilde{\omega}$ is bounded by $C'$ (because at the origin of the normal coordinates that $\omega = \id$). Using the fact that $\tilde{\omega}^n = e^f \omega^n$ we find,
\[ \prod_{i = 1}^n \lambda_i = e^f \]
and thus for any $i$ we get,
\[ C'^{n-1} \lambda_i \ge^f  \ge e^f \ge e^{\inf\limits_M f} \]
This implies that,
\[ \lambda_i \ge C'^{-(n-1)} e^{\inf\limits_M f} \]
Since each eigenvalue is bounded we find that,
\[ \tilde{\omega} \ge C'^{-(n-1)} e^{\inf\limits_M f} \omega \]
since this constant is uniform we may combine this with an earlier innequality we find,
\[ C^{-1} \omega \le \tilde{\omega} \le C \omega \]
for some $C > 1$ which is the needed $C^2$-estimate. 

\subsection{Higher-Order Estimates}

Consider the Christoffel symbols associated to the metrics $g$ and $\tilde{g}$ which we write as $\Gamma$ and $\tilde{\Gamma}$. Then define,
\[ S^k_{ij} = \tilde{\Gamma}^k_{ij} - \Gamma^k_{ij} \]
It turns out that this is a tensor even though each connection is not. Furthermore, we can compute,
\[ |S|_{\tilde{g}}^2 = S^k_{ij} \overline{S^r_{pq}} \tilde{g}^{i \bar{p}} \tilde{g}^{j \bar{q}} \tilde{g}_{k \bar{r}} \]
\begin{lemma}
$|S|^2_{\tilde{g}} \le C$
\end{lemma}

\begin{proof}

\end{proof}

\subsection{The $C^0$ Estimate}

\newcommand{\Vol}[1]{\mathrm{Vol}\left(#1\right)}

\begin{proposition}
$|| \varphi ||_{C^0(M)} \le C$ for some constant $C$ only depending on $||e^f||_{L^\infty}$ and $M$ and $\omega$.
\end{proposition}

\begin{proof}
By Green's formula $\forall x \in M$ we have,
\[ \varphi(x) = \frac{1}{\Vol{M}} \int_M \varphi \omega^n - \frac{1}{\Vol{M}} \int_M G(x, y) \Delta \varphi(y) \omega^n(y) \]
Where,
\[ \Vol{M} = \int_M \omega^n \]
and $G(x,y)$ is the Green's function of $\Delta_\omega$ and $\Delta_\omega G(x, y) = \delta_x(y)$. We know that $G$ is bounded below by $- C_\omega$. Taking the trace of $\omega_\varphi = \omega + i \partial \bar{\partial} \varphi > 0$ gives, 
\[ \tr{\omega}{\omega_\varphi} = n + \Delta_\omega \varphi > 0 \]
and therefore,
\[ \varphi(x) = \frac{1}{\Vol{M}} \int_M \left( G(x, y) + C_\omega \right) \left( - \Delta_\omega \varphi(y) \right) \omega^n \le \frac{n}{\Vol{M}} \int_M \left( G(x, y) + C_\omega \right) \omega^n \]
Which implies that,
\[ \varphi(x) \le C_0(M, \omega) \]
Now we need to prove the lower bound. Set $\phi = - \left( \varphi - C_0 - 1 \right) \ge 1$. We need to show that $\phi$ is bounded above. From the defining equation, for any $p \ge 2$,
\[ \int_M \phi^{p - 1} \omega_\phi^n - \omega^n = \int \phi^{p-1} \left( e^f - 1 \right) \omega^n \le C \int \phi^{p - 1} \omega^n \]
where the bound comes from the $L^1$ boundedness of $f$. However,
\begin{align*}
\int_M \phi^{p - 1} \left( \omega^n_\varphi - \omega^n \right) = \int_M \phi^{p-1} i \partial \bar{\partial} \varphi \wedge \left( \omega_\varphi^{n-1} + \cdots + \omega^{n - 1} \right)
\end{align*}
However, $i \partial \bar{\partial} \varphi = - i \partial \bar{\partial} \phi$. Then applying Stoke's theorem,
\begin{align*}
\int_M \phi^{p-1} i \partial \bar{\partial} \varphi \wedge \left( \omega_\varphi^{n-1} + \cdots + \omega^{n - 1} \right) 
& = \int_M (p - 1) \phi^{p-2} i \partial \phi \wedge \bar{\partial} \phi \wedge \left( \omega_\varphi^{n- 1} + \cdots + \omega^{n-1} \right) 
\\
& \ge (p-1) \int_M \phi^{p-2} i \partial \phi \wedge \bar{\partial} \phi \wedge \omega^{n-1} = \frac{p-1}{n} \int_M \phi^{p-2} |\nabla \phi |_{\omega}^@ \omega^n
\\
& = \frac{4 (p - 1)}{n p^2} \int_M |\nabla \phi^{\frac{p}{2}} |^2_{\omega} \omega^n 
\end{align*}
Furthermore,
\begin{align*}
\int_M |\nabla \phi^{\frac{p}{2}} |^2_\omega \omega^n \le \frac{C n p^2}{4 (p - 1)} \int_M \phi^{p-1} \omega^n \le C p \int_M \phi^{p-1} \omega^{n} 
\end{align*}
Now recall the Sobolev innequality for $(M, \omega)$ which states that for any positive $\eta \in C^1(M)$ there exists $C$ depending only on $M$ and $\omega$ such that,
\[ \left( \int_M \eta^{\frac{2n}{n-1}} \omega^n \right)^{\frac{n-1}{n}} \le C \left( \int_M \left( |\nabla \eta|^2 + \eta^2 \right) \omega^n \right) \]
We apply this innequality in the case $\eta = \phi^{\frac{p}{2}}$. Then we find that,
\begin{align*}
\left( \int_M \phi^{p \cdot \frac{2n}{n-1}} \omega^n \right)^{\frac{n-1}{n}} \le C \left( \int_M \left( | \nabla \phi^{\frac{p}{2}} |^2 + \phi^p \right) \omega^n \right) \le \left( \int_M \left( C p \phi^{p - 1} + \phi^p \right) \omega^n \right)
\end{align*}
Furthermore, since $\phi \ge 1$ we can assume that $\phi^p \ge \phi^{p-1}$ and therefore we find,
\[ \left( \int_M \phi^{p \cdot \frac{2n}{n-1}} \omega^n \right)^{\frac{n-1}{n}} \le C p \int \phi^p \omega^n \]
for some constant $p$. Then, taking the $p$-th root we find,
\[ \left( \int_M \phi^{p \cdot \frac{2n}{n-1}} \omega^n \right)^{\frac{n-1}{np}} \le C^{\frac{1}{p}} p^{\frac{1}{p}} \left( \int \phi^p \omega^n \right)^{\frac{1}{p}} \]
Then, letting $\chi = \frac{n}{n - 1} > 1$, we find that,
\[ || \phi ||_{L^{p \chi}} \le C^{\frac{1}{p}} p^{\frac{1}{p}} || \phi ||_{L^p} \]
for any $p \ge 2$ where the constant $C$ does not depend on $p$. We may apply this innequality inductively, on a sequence $p_k = 2 \chi^k$ using that,
\[ || \phi ||_{L^{p_{k + 1}}} \le C^{\frac{1}{p_k}} p_k^{\frac{1}{p_k}} || \phi ||_{L^{p_k}} \]
to find that,
\[ || \phi ||_{L^{p_{k + 1}}} \le C^{\sum\limits_{j = 0}^k \frac{1}{p_j}} \prod_{j = 0}^k p_j^{\frac{1}{p_j}} || \phi ||_{L^2} \] 
However, the series is geometic so,
\[ \sum_{j = 0}^\infty \frac{1}{p_j} = \frac{1}{2} \frac{1}{1 - \frac{1}{\chi}} \]
and $\chi > 1$ so this series converges. Furthermore,
\[ \prod_{j = 0}^\infty p_j^{\frac{1}{p_j}} = \prod 2^{\frac{1}{p_j}} \chi^{\frac{j}{p_j}} = 2^{\sum\limits_{j = 0}^\infty \frac{1}{p_j}} \chi^{\sum\limits_{j = 0}^{\infty} \frac{j}{p_j}} \]
is bounded. Because,
\[ \sum\limits_{j = 0}^{\infty} \frac{j}{p_j} = \sum\limits_{j = 0}^{\infty} \frac{j}{2 \chi^j} < \infty \] 
Thus we have shown that,
\[ || \phi ||_{L^p_{k + 1}} \le C || \phi ||_{L^2} \]
but $p_{k+1} \to 0$. Furthermore,
\[ \lim_{p \to \infty} \left( \int_M \phi^p \right)^{\frac{1}{p}} = || \phi ||_{L^\infty} \]
Therefore, we have shown that,
\[ || \phi ||_{L^\infty} \le C || \phi ||_{L^2} \]
Using this it suffices to prove that the $L^2$-norm is bounded. Consider,
\end{proof}

\section{ABP-Maximum Principle}

\begin{definition}
Let $\Omega \subset \R^n$ be a bounded domain and $u : \Omega \to \R$ a function. Define the \textit{upper contact set} of $u$ to be,
\[ \Gamma_u = \{ y \in \Omega \mid \exists p \in \R^n : \forall x  \in \Omega : u(x) - u(y) \le p \cdot (x - y) \} \] 
\end{definition}

\begin{remark}
If $u$ is concave then $\Gamma = \Omega$ and $p = \nabla u(y)$ at the point $y \in \Gamma 
$. In general, $D^2 u(y) \le 0$ for any $y \in \Gamma$. 
\end{remark}

\begin{definition}
The normal mapping $\chi_u : \Omega \to \R^n$ is the mapping,
\[ \chi_u(y) = 
\begin{cases}
\{ p \in \R^n \mid \forall x  \in \Omega : u(x) - u(y) \le p \cdot (x - y) \} & y \in \Gamma
\\
\varnothing & y \notin \Gamma 
\end{cases} \]
\end{definition}

\newcommand{\diam}[1]{\mathrm{diam}\left(#1\right)}

\begin{lemma}
Suppose that $u \in C^2(\Omega) \cap C^0(\overline{\Omega})$ then $\exists C(n)$ such that 
\[ \sup_{\Omega} u \le \sup_{\partial \Omega} u + C(m) \cdot \diam{\Omega} \cdot \left( \int_{\Gamma_u} |\det{D^2 u}| \right)^{\frac{1}{n}} \]
\end{lemma}

\begin{proof}
By replacing $u$ by $u - \sup_{\partial \Omega} u$ we can assume that $\sup_{\partial \Omega} u = 0$. We have that,
\[ \Vol{\chi_u(\Omega)} = \Vol{\chi_u(\Gamma_u)} = \Vol{\nabla u(\Gamma_u)} \le \int_{\Gamma_u} | \det{D^2 u}| \]
by pulling back the volume of the map $\nabla u : \Gamma_u \to \R^n$. Now assume that $u$ achieves its supremum at $y \in \Omega$ (other we are done since $\overline{\Omega}$ is compact so $u$ must achieve its supremum then on the boundary). 
\end{proof}

\subsection{Complex Version of ABP-Maximum Principle}

Let $\Omega \subset \C^n \cong \R^{2n}$ and take $u \in PSH(\Omega) \cap C^2(\Omega)$ meaning that $i \partial \bar{\partial} u \ge 0$. Then we have, $\det{u_{i \bar{j}}} = F \ge 0$ in $\Omega$ and take $u = 0$ on $\partial \Omega$. Therefore, $u \le 0$. Then we find,
\begin{theorem}
\[ \sup\limits_{\Omega}(-u) \le C(n) \cdot \diam{\Omega} \left( \int_{\Gamma_u} F^2 \right)^{\frac{1}{2n}} \]
\end{theorem}

\begin{proof}
First, $\forall p \in \Gamma_{-u}$ we have $D^2(-u)|_p \le 0$ which is the real Hessian. Choose special holomorphic coordinates $(z_1, \dots, z_n)$ near $p$ such that,
\[ (u_{i \bar{j}})|_p = 
\begin{pmatrix}
\lambda_1 & &  
\\
& \ddots & 
\\
& & \lambda_n
\end{pmatrix} \]
where the eigenvalues are,
\[ \lambda_i = \frac{\partial^2 u}{\partial z_i \partial \bar{z}_i} \bigg|_p = 4 \left( \frac{\partial^2 u}{\partial x_i^2} + \frac{\partial^2 u}{\partial y_i^2} \right) \]
were we decompose, $z_i = x_i + \sqrt{-1} y_i$. Now,
\begin{align*}
\det{u_{i \bar{j}}} = \prod_{i = 1}^n \lambda_i = 4^n \prod_{i = 1}^n  \left( \frac{\partial^2 u}{\partial x_i^2} + \frac{\partial^2 u}{\partial y_i^2} \right)
\end{align*}
Since both quantities are positive, by Cauchy Schwartz,
\[ \det{u_{i \bar{j}}} \ge 2^{3n} \prod_{i = 1}^n \sqrt{\frac{\partial^2 u}{\partial x_i^2} \frac{\partial^2 u}{\partial y_i^2}} \ge \sqrt{\det{D^2 u}} \]
Thus we have found,
\[ \det{D^2 u} \le ( \det{u_{i \bar{j}}} )^2 \]
and then applying the ABP estimate proves the desired result.   
\end{proof}

\begin{corollary}
Appling the H\"{o}lder inequality we find,
\[ \sup\limits_{\Omega}(-u) \le C(n) \cdot  \diam{\Omega} \cdot \left( \int_{\Omega} F^{2 p} \right)^{\frac{1}{2 n p}} \Vol{\Omega}^\frac{1}{2 n p'} \]
where $\frac{1}{p} + \frac{1}{p'} = 1$. 
\end{corollary}

\begin{lemma}
Take $\Omega \subset \C^n$ and $u \le 0$ take $\det{u_{i\bar{j}}} = F \ge 0$ on $\Omega$. Assume that, for $\lambda > 0$, 
\[ \overline{\{ z \in \Omega \mid u(z) < \inf\limits_{\Omega} u + \lambda \}} \subset \Omega^\circ \]
then,
\[ \sup\limits_{\Omega}(-u) \le \lambda + \left( \frac{c(n) d}{\lambda} \right)^{2n p'} \left( \int_\Omega  F^{2 p} \right)^{\frac{p'}{p}} || u ||_{L^1(\Omega)} \]
\end{lemma}

\begin{proof}
Consider the set,
\[ \Omega' = \{ z \in \Omega \mid u(Z) < \inf_{\Omega} u + \lambda \} = \{ z \in \Omega \mid v < 0 \text{ where } v = u  \inf\limits{\Omega} u - \lambda \} \]
On $\partial \Omega'$ we have $v = 0$. Then applying the previous innequality,
\[ \sup\limits_{\Omega'}(-v) \le c(n) d \left( \int_{\Omega'} F^{2 p} \right)^{\frac{1}{2 n p}} \Vol{\Omega'}^{\frac{1}{2np'}} \]
Furthermore, $\sup_{\Omega'}(-v) = \lambda$. Therfore, on $\Omega'$ we have,
\[ \int_{\Omega'} \frac{|u|}{|\inf\limits_{\Omega}(u) + \lambda|} \ge \Vol{\Omega'} \]
Which, noting that $\inf_{\Omega} u < 0$ and $\lambda > 0$ we get,
\[ \Vol{\Omega'} \le \frac{1}{|\inf\limits_{\Omega} u | - \lambda} \int_{\Omega'} |u| \]
\end{proof}

\subsection{APB-$C^0$-Estimate for $\varphi$}

First, $\exists z_0 \in M$ such that $\varphi(z_0) = \inf\limits_{M} \varphi$. By Green's formula this implies $\varphi \le C_0$. Then consider,
\begin{align*}
\int_M | \varphi | \omega^n \le \int_M \left( |\varphi - C_0| + C_0 \right) \omega^n = int \left( C_0 - \varphi + C_0 \right) \omega^n = 2 C_0 V 
\end{align*}
Therefore $||\varphi||_{L^1} \le C$. 
Without loss of generality, replace $\varphi$ by $\varphi - C_0$ by which we may assume that $\varphi \le 0$. We can choose local holomorphic coordinates $(U, z_j)$ near $z_0$. Then, locally, we can represent,
\[ \omega = i \partial \bar{\partial} \psi > 0 \]
for some strictly PSH$(U)$ function $\phi$. We then take the Taylor expansion of $\psi$ near $z_0$ to find,
\[ \psi(z_0 + h) = \psi(z_0) + 2 \Re{\left( \frac{\partial \psi}{\partial z_j}(z_0) h_j \right)} + 2 \Re{\left( \frac{\partial^2 \psi(z_0)}{\partial z_i \partial z_j} h_i h_j \right)} + 2 \frac{\partial^2 \psi(z_0)}{\partial z_i \partial \bar{z}_{\bar{j}}} h_i \bar{h_j} + O(|h|^3) \] 
Then consder,
\[ P(h) = \psi(z_0) + 2 \frac{\partial \psi}{ \partial z_j}(z_0) h_j + 2 \frac{\partial^2 \psi(z_0)}{\partial z_i \partial z_j} h_i h_j \]
Then $\Re{P(h)}$ is pluriharmonic meaning that $i \partial \bar{\partial} \Re{P(h)} = 0$. Replace $\psi$ by $\psi - \Re{P(h)} - C$ if necessary such that we may assume that,
\[ \psi(z_0 + h) \ge c_0 |h|^2 - c_1 |h|^3 \]
for sufficiently small $h$. Moreover, this innequality implies that $\psi$ acieves a strict mnimum at $z_0$. Thus for some $\lambda$,
\[ \forall z \in B(z_0, 2 r) \setminus B(z_0, r) : \psi(z) > \psi(z_0) + \lambda \]
Now we set $u = \psi + \varphi \le 0$ in $\Omega = B(z_0, 2 r)$. Applying the Global Modge-Ampere equaton,
\[ \det{u_{i \bar{j}}} = e^{f} \det{\Omega} = F \]
so $F$ is fixed. 
Furthermore, using the origional normalization and the shifting by $C$ we also have,
\[ \int_\Omega |u| \le C \]
Now we apply the previous lemma noting that $|u|$ is bounded above and $F$ is fixed. Therefore we find that $u$ is bounded above implying that $\varphi$ is bounded above. In summary,
\[ || \varphi ||_{C^0} \le C(M, \omega, p, || e^f ||_{L^p}) \]
for any $p > 2$. So we can make our bound depend only on the $L^p$ norm of $e^f$ for any $p > 2$. 

\section{Proof}

On a compact K\"{a}hler manifold $M$ if $c_1(M) =0$ then for any K\"{a}hler class $[\omega]$ there exists a unique $\omega_0 \in [\Omega]$ such that $\Ric{\omega_0} = 0$. Such manifolds are Calabi-Yau and $\omega_0$ is a Calabi-Yau metric. 
\bigskip\\
When $c_1(M) < 0$ then the K\"{a}hler-Einstein metric exists such that $\Ric{\omega_{\text{KE}]}} = - \omega_{\text{KE}}$. 
\begin{proof}
Fix some $\omega \in - c_1(M)$. We know that $[\Ric{\eta}] = c_1(M)$ and therefore $\omega_{\text{KE}} \in - c_1(M)$ so $\omega_{\text{KE}} = \omega + i \partial \bar{\partial} \varphi$. This is equivalent to,
\[ (\omega + i \partial \bar{\partial} \varphi)^n = e^{f + \varphi} \omega^n \]
and $\Ric{(\omega)} + \omega = i \partial \bar{\partial} f$ with $\int e^f \omega^n = v$. The $C^2$ and higher estimates are almost the same. We will now prove the $C^0$ estimate for $\varphi$.
\end{proof}

\begin{lemma}
$||\varphi||_{C^0} \le || f ||_{L^\infty}$
\end{lemma}

\begin{proof}
Suppose that $z_0 \in M$ such that $\varphi(z_0) = \max\limits_M \varphi$. Then at $z_0$ we have $D^2 \varphi \bigg|_{z_0} \le 0$ which implies that $i \partial \bar{\partial} \varphi \bigg|_{z_0} \le 0$. Thus,
\[ 0 < \omega + i \partial \bar{\partial} \varphi \le \omega \implies \det{(\omega + i \partial \bar{\partial} \varphi)} \le \det{\omega} \implies e^{f + \varphi} \det{\omega} \le \det{\omega} \]
This implies that $f + \varphi \le 0$ and thus $\varphi(z_0) \le - f(z_0)$. The opposite innequality is similar.
\end{proof}

\subsection{The Case of $c_1(M) > 0$}

We would want $\Ric{(\omega_{\text{KE}})} = \omega_{\text{KE}}$ on a manifold with $\omega \in c_1(M)$. We would need to solve,
\[ (\omega + i \partial \bar{\partial} \varphi)^n = e^{f - \varphi} \omega^n \]
However, in general there does not exist a solution to this equation. That said, all $C^2$ and $C^{2, \alpha}$ higher-order estimates hold but $C^0$-estimate of $\varphi$ fails for most problems.  

\section{Feb. 26}

\begin{remark}
Let $\Omega \subset \C$ a bounded domain.
\end{remark}

\begin{definition}
A function $u : \Omega \to \R$ is \textit{upper semi-continuousi} iff for all $x \in \Omega$ we have $\limsup_{z \to x} u(z) \le u(x)$. 
\end{definition}

\begin{proposition}
A function $u : \Omega \to \R$ is upper semi-continuous iff $\forall c \in R : u^{-1}(\{ x \in \R \mid x < c \})$ is open. 
\end{proposition}


\begin{definition}
A function $u : \Omega \to [-\infty, \infty)$ is subharmonic (SH) if $u$ is upper semi-continuous and satisfies the sub-mean value inequality,
\[ \forall x \in \Omega : \exists R_x > 0 : \forall r \in (0, R_x) : u(x) \le \frac{1}{2 \pi} \int_0^{2 \pi} u(x + r e^{i \theta}) \d{\theta} = \frac{1}{2 \pi r} \int_{\partial B_r(x)} u(z) \d{z} \]
\end{definition}

\begin{proposition}
Let $\Omega \subset \C$ a bounded domain. Then we have the following:
\begin{enumerate}
\item If $u, v$ are both SH on $\Omega$ then so is $\max{(u,v)}$. 
\item If $u$ is SH and $\chi : \R \to \R$ is convec and increasing then $\chi \circ u$ iis also SH.
\item If $\{ u_j \}$ is a decreasing sequence of SH functions, then $\lim_{j \to \infty} u_j = u$ is also SH.
\item Suppose that $\{ u_j \}$ is a sequence of SH functions which is locally uniformly bounded above. Then if $\{ \epsilon_j \}$ is a positive sequence with finite sum then,
\[ u = \sum_{j = 1}^\infty \epsilon_j u_j \]
is SH.
\end{enumerate}
\end{proposition}

\begin{proof}
Apply Jensen's inequality applied to the convex function $\chi$,
\[ \chi \circ u(x) = \chi(u(x)) \le \chi \left( \frac{1}{2 \pi} \int_0^{2 \pi} u(x + r e^{i \theta}) \d{\theta} \right) = \frac{1}{2\pi} \int_0^{2 \pi} \chi \circ u(x + r e^{i \theta}) \d{\theta} \] 
where I have used that $\chi$ is increasing.
\bigskip\\
Take $u = \inf \{ u_j \}$. Then we find,
\[ \{ x \in \Omega \mid u(x) < c \} = \bigcup_j \{ x \in \Omega \mid u_j(x) < c \} \]
which is a union of open sets and thus open. Therefore $u$ is upper semi-continuous. Furthermore,
\begin{align*}
u(x) = \lim_{j \to \infty} u_j(x) & \le \lim_{j \to \infty} \frac{1}{2 \pi} \int_0^{2 \pi} u_j(x + r e^{i \theta}) \d{\theta} = 
\\
& = \frac{1}{2 \pi} \int_0^{2 \pi} u(x + r e^{i \theta}) \d{\theta} 
\end{align*}
by monotone convergence theorem. 
\bigskip\\
Take a relatively compact domain $\Omega' \subset \Omega$. We may assume that $u_j \le C_0$ on $\Omega'$. Then,
\[ u = \sum_{j = 1}^\infty \epsilon_j (u_j - C_0) + C_0 \sum_{j = 1}^\infty \epsilon_j \]
Then the partial sums of the first term are a decreasing sequence of SH functions. Thus, the first term is SH by the previoud property and the second term is constant. Thus $u$ is SH. 
\end{proof}

\begin{example}
On $\C$ we have the following examples of subharmonic functions,
\begin{enumerate}
\item Harmonic functions ($\Delta u = 0$) which satisfy the mean value innequality.
\item Convex functions.
\item For $z_0 \in \C$ the function $u(z) = \log{|z - z_0|}$.
\item Given a sequence of points $\{ a_j \} \in B_1(0)$ the function $u(z) = \sum_{j = 1}^\infty \epsilon_j \log{|z - a_j|}$ for any positive sequence $\{ \epsilon_j \}$ with positive sum. If $\{ a_j \}$ is dense in $B_1(0)$ then $u$ is not locally bounded since it accieves the value $-\infty$ on every open set. 
\item If $u \in C^2(\Omega)$ then $u$ is SH iff $\Delta u \ge 0$.
\item For not necessarily $C^2(\Omega)$ functions, we may check the earlier condition in the sense of distributions,
\[ \int_\Omega u \Delta \varphi \ge 0 \]
for every $\varphi \in C^2(\Omega)$ and $\varphi \ge 0$ with compact support. This condition is equivalent to $u$ being SH.
\end{enumerate}
\end{example}

\begin{proposition}
If $u$ is SH, $x \in \Omega$ and $\delta x = d(x, \partial \Omega)$ then the function,
\[ r \mapsto M(x, r) = \frac{1}{2 \pi} \int_0^{2 \pi} u(x + r e^{i \theta}) \d{\theta} \]
is increasing for $r \in [0, \delta x)$ and $\lim\limits_{r \to 0^{+}} M(x, r) = u(x)$. 
\end{proposition}

\begin{proof}
Take any continuous function $h$ on $\partial B_r(x)$ such that $h \ge u$ on $\partial B_r(x)$. Then there exists a solution to the Dirichlet problem, $\Delta H = 0$ inside $B_r(x)$ and $H = h$ on $\partial B_r(x)$. The maximum principle then implies that $u \le H$ in $B_r(x)$ For any $0 < s < r$ we know that,
\[ u(x + s e^{i \theta}) \le H(x + re^{i \theta}) \]
Therefore, taking the integral,
\[ \frac{1}{2 \pi} \int_0^{2 \pi} u(x + se^{i \theta}) \d{\theta} \le \frac{1}{2 \pi} \int_0^{2 \pi} H(x + s e^{i \theta}) \d{\theta} \]
However, since $H$ is harmonic we have,
\[ H(x) = \frac{1}{2 \pi} \int_0^{2 \pi} H(x + s e^{i \theta}) \d{\theta} = \frac{1}{2 \pi} \int_0^{2 \pi} H(x + r e^{i \theta}) \d{\theta} = \frac{1}{2 \pi} \int_0^{2 \pi} h(x + r e^{i \theta}) \d{\theta} \]
since $H = h$ on the boundary. Therefore,
\[ \frac{1}{2 \pi} \int_0^{2 \pi} u(x + se^{i \theta}) \d{\theta} \le \frac{1}{2 \pi} \int_0^{2 \pi} h(x + se^{i \theta}) \d{\theta} \] 
We may then take a decreasing sequence of continuous functions on $\partial B_r(x)$ converging to $u$. Then we find that,
\[ \frac{1}{2 \pi} \int_0^{2 \pi} u(x + se^{i \theta}) \d{\theta} \le lim_{j \to \infty} \frac{1}{2 \pi} \int_0^{2 \pi} h_j(x + se^{i \theta}) \d{\theta} = \frac{1}{2 \pi} \int_0^{2 \pi} u(x + r e^{i \theta}) \d{\theta} \] 
by monotone convergence. Therefore $M(x, r)$ is increasing in $r$. By monotonicity, $\lim\limits_{r \to 0^{+}} M(x, r)$ exists and is greater than $u(x)$. However, $\limsup_{z \to x} u(z) \le u(x)$ if we choose a local maximum $x \in \Omega$. These imply that $\lim\limits_{r \to 0^{+}} M(x, r) = u(x)$. 
\end{proof}

\begin{corollary}
If $u$ is SH then,
\[ u(x) \le \frac{1}{\pi r^2} \int_{B_r(x)} u(z) \d{\lambda(z)} \]
where $\lambda$ is the standard Lebesgue measure of $\C$. Furthermore this becomes an equality in the limit $r \to 0^{+}$. 
\end{corollary}

\begin{corollary}
If $u, v$ are SH and $u = v$ almost everywhere in $\Omega$.
\end{corollary}

\begin{proof}
\[ u(x) = \lim\limits_{r \to 0^{+}} \frac{1}{\pi r^2} \int_{B_r(x)} u = \lim_{r \to 0^{+}} \frac{1}{\pi r^2} \int_{B_r(x)} v = v(x) \]
The integrals argree because the functions $u$ and $v$ agree almost everywhere.
\end{proof}

\subsection{Plurisubharmonic Functions}

\begin{remark}
Let $\Omega \subset \C^n$ be a bounded domain with $n \ge 1$. 
\end{remark}

\begin{definition}
A function $u : \Omega \to [-\infty, \infty)$ is plurisubharmonic (PSH) if for any complex line $\Lambda \subset \C^n$ the function $u|_{\Lambda \cap \Omega}$ is SH. This property is equivalent to the condition that $\forall \xi \in \C^n$ such that $|\xi| = 1$ then $\forall x\ in \Omega$,
\[ u(x) \le \frac{1}{2 \pi} \int_0^{2 \pi} u(x + r e^{i \theta} \xi) \d{\theta} \]
for all $0 \le r \le d(x, \partial \Omega)$. Let $PSH(\Omega)$ denote the space of PSH functions (except the constant function at $-\infty$.
\end{definition}

\begin{proposition}
The following properties of SH functions also apply to PSH functions,
\begin{enumerate}
\item If $u,v \in PSH(\Omega)$ then $\max\{(u,v)\} \in PSH(\Omega)$.
\item If $\chi : \R \to \R$ is convex and increasing then $\chi \circ u \in PSH(\Omega)$.
\item If $\{ u_j \}$ is a decreasing sequence of PSH functions with limit $u$ then $u$ is PSH.
\end{enumerate}
\end{proposition}

\begin{example}
The following functions on $\C^n$ are PSH, 
\begin{enumerate}
\item all convex functions
\item for $a \in \C^n$ the function $u(z) = \log{|z - a|}$ 
\item if $f$ is holomorphic on $\Omega$ then $u(z) = \log{|f(z)|}$ is PSH
\item if $u \in C^2(\Omega)$ then $u \in PSH(\Omega)$ iff $D^2 u$ is positive definite. 
\end{enumerate}
\end{example}

\begin{proposition}
If $\Omega'$ is relatively compact in $\Omega$ and $u \in PSH(\Omega)$ and $v \in PSH(\Omega')$ with $u \ge v$ on $\partial \Omega$ then the function,
\[ w(z) = 
\begin{cases}
\max\{(u(z), v(z)\} & z \in \Omega'
\\
u(z) & z \in \Omega \setminus \Omega'
\end{cases} \]
\end{proposition}

\begin{proof}
Clearly, $w$ is upper semi-continuous. Replace $v$ by $v - \epsilon$, define $w_\epsilon$ correspondingly. Then $u > v - \epsilon$ on $\partial \Omega'$. Therefore $w_\epsilon \in PSH(\Omega')$ but we can form an incresing sequence of PSH functions $w_\epsilon$ as $\epsilon \to 0$ to the limit $w$. Thus $w \in PSH(\Omega)$. 
\end{proof}

\section{Feb. 28}

\begin{remark}
$\omega_n$ is the volume of the unit ball in $\R^n$ then the area of $S^{n-1}$ is $n \omega_n$. 
\end{remark}

\begin{theorem}[Sub-Mean Value Inequalities]
Take $u \in PSH(\Omega)$ then $\forall x \in \Omega$
\[ n(x) \le \frac{1}{2n \omega^{2n}} \int_{|\xi| = 1} u(x + r \xi) \d{\sigma(\xi)} = \frac{1}{A(\partial B_r(x))} \int_{\partial B_r(x)} u(x) \d{\sigma} \]
and also,
\[ u(x) \le \frac{1}{r^{2n}} \int_0^r t^{2n - 1} \d{t} \int_{|\xi| = 1} \frac{1}{2 n \omega_{2n}} u(x + t \xi) \d{\sigma(\xi)} = \frac{1}{\text{vol}(B_r(x))} \int_{B_r(x)} u(z) \d{v(z)} \] 
\end{theorem}

\begin{proof}
Clealry the first statement implies the second by integrating the first expression times $t^{2n - 1}$. 
\bigskip\\
First,
\begin{align*}
\int_{|\xi| = 1} u(x + r e^{i \theta} \xi) \d{\sigma(\xi)} = \int_{|\xi| = 1} u(x + t \xi) \d{\sigma(\xi)} 
\end{align*}
\end{proof}

\begin{proposition}
$PSH(\Omega) \subset L^1_{\text{loc}}(\Omega)$ where $L^1_{\text{loc}}(\Omega)$ is the space of functions which are integrable on all compact subdomains of $\Omega$.
\end{proposition}

\begin{proof}
Given $u \in PSH(\Omega)$ with $u \neq - \infty$ indentically. Define,
\[ G = \{ x \in \Omega \mid u \text{ is integrable in an open neighborhood of } x \} \] 
Take some point $x_0 \in \Omega$ such that $u(x_0) > - \infty$. Since $\Omega$ is open there exists some $r < d(x_0, \partial \Omega)$ such that by the previous innequality,
\[ u(x_0)\omega_{2n} r^{2n} \le \int_{B_r(x)} u \implies u |_{B_r(x)} \le C \]
by upper-semi-continuity. Since $u$ is bounded above and the integral is bounded below then,
\[ \int_{B_r(x_0)} |u| < \infty \]
Thus $x_0 \in G$. Furthermore, $G$ is open. Claim that $G$ is also closed implying that $G = \Omega$. Suppose $x' \in \overline{G} \cap \Omega$ then there must be an open ball $B_r(x') \subset \Omega$ since $\Omega$ is open. Take $r < \tfrac{1}{4} d(x', \partial \Omega)$. However, since $x'$ is a limit point of $G$ then $\exists x \in G \cap B_r(x')$. Therefore, $u$ is locally integrable near $x$ so we can find $x'' \in B_r(x')$ and close to $x$ such that $u(x'') > - \infty$. Then, by the triangle innequality,
\[ B_{2r}(x'') \subset \Omega \]
and then $u$ in integrable on $B(x'', 2r)$ so $u$ is integrable on $B_r(x') \subset B_{2r}(x'')$. Therefore $x' \in G$. 
\end{proof}

\begin{theorem}
Therefore $(PSH(\Omega), L^1_{\text{loc}}(\Omega))$ is a topological space i.e. if $u_i \to u$ in $L^1_{\text{loc}}(\Omega)$ i.e. for all $\Omega' \subset \Omega$,
\[ \int_{\Omega'} |u_j - u | \to 0 \]
\end{theorem}

\begin{lemma}
The function,
\[ M(x, r) = \frac{1}{r^{2n}} \int_0^r t^{2n - 1} \d{t} \int_{|\xi| = 1} u(x + t \xi) \d{\sigma(\xi)} \]
is increasing in $r \in [0, d(x, \partial \Omega))$. Furthermore,
\[ \lim_{r \to 0^{+}} M(x, r) = u(x) \] 
\end{lemma}

\begin{proof}
Immediate consequence of the correspondsing lemma for SH functions. 
\end{proof}

\begin{proposition}
The evaluation functional,
\[ PSH(\Omega) \otimes \Omega \to \R \cup \{ -\infty \} \]
is upper-semi-continous. As a consequence, if $U \subset PSH(\Omega)$ is compact then its upper envelope,
\[ u_e(z) = \sup \{ u(z) \mid u \in \mathcal{U} \} \]
is upper-semi-continuous and hence $u_e \in PSH(\Omega)$.  
\end{proposition}

\newcommand{\PSH}[1]{\mathrm{PSH}\left( #1 \right)}

\begin{proof}
Fix $(u, z_0) \in \PSH{\Omega} \times \Omega$. Then $u_j \in \PSH{\Omega}$ such that $u_j \to u$ in $L^1_{\text{loc}}$. I claim that $\{ u_j \}$ is locally uniformly bounded above. On $K \subset \subset K' \subset \subset \Omega$ such that $K$ and $K'$ are compact then $u_j \to u$ in the $L^1(K')$ sense such that,
\[ \int_{K'} |u_j| \le C \]
is independent of $j$. Then $\forall x \in K$ there exists $r < d(K, \partial K')$ such that,
\[ u_k|_K \le C \quad \text{and thus} \quad u_j(x) \le \frac{1}{\omega_{2n} r^{2n}} \int_{B_r(x)} u_j \le C \]
By subtracting the constant $C$ we may assume that $u_j \le 0$ near $z_0$. Next. choose $r < \tfrac{1}{2} d(z_0, \partial \Omega)$ and $\delta > 0$ to be sufficiently small such that by the sub-mean-value innequality for any $x \in B_\delta(z_0)$ we have,
\[ u_j(x) \le \frac{1}{\omega_{2n} (r + \delta)^{2n}} \int_{B_{r + \delta}(x)} u_j \] 
Then $B_r(z_0) \subset B_{r + \delta}(x)$. Furthermore, $u_j \le 0$ and thus the intergral is decreasing with respect to inclusions of domains. In particular,
\[ u_j(x) \le \frac{1}{\omega_{2n} (r + \delta)^{2n}} \int_{B_{r + \delta}(x)} u_j \le \frac{1}{\omega_{2n} (r + \delta)^{2n}} \int_{B_r(z_0)} u_j \] 
Now taking the limsup of both sides,
\[ \limsup_{\substack{j \to \infty \\ x \to z}} u_j(x) \le \limsup_{\substack{j \to \infty \\ x \to z}} \frac{1}{\omega_{2n} (r + \delta)^{2n}} \int_{B_r(x)} u_j = \frac{1}{\omega_{2n} (r + \delta)^{2n}} \int_{B_r(z_0)} u \]
Now letting $\delta \to 0$ we find,
\[ \limsup_{\substack{j \to \infty \\ x \to z}} u_j(x) \le \frac{1}{\omega_{2n} r^{2n}} \int_{B_r(z_0)} u \xrightarrow{r \to 0} u(z_0) \]
\end{proof}

\begin{remark}
Take $\Omega \subset \C^n$ a pseudoconvex domain. Then for all holomorphic functions on $\Omega$ we have $c \log{|f|} \in \PSH{\Omega}$ for all $c \in R^{+}$. Conversely, the set,
\[ U = \{ c \log{|f|} \mid c > 0 \quad f \text{ holomorphic} \} \]
This may be proven by the Ohsawa-Takagoshi extension theorem or $L^2$-estimates technique of H\"{o}rmader. 
\end{remark}

\subsection{Differential Characterization of PSH}

\newcommand{\supp}[1]{\mathrm{supp}\left( #1 \right)}

\begin{remark}
Our goal is $\forall u \in \PSH{\Omega}$ there exists a sequence $u_j \in \PSH{\Omega} \cap C^\infty(\Omega)$ such that $u_j \to u$ is a decreasing limit in $L^1_{\text{loc}}(\Omega)$. This is known as PSH smoothing. 
\end{remark}

\begin{definition}
Fix a cutoff function $\rho$ on $\R$ such that $\supp{\rho} \subset [0, 1]$ and,
\[ \int_{\C^n} \rho(|z|) \d{z} = 1 \]
Therefore, $\supp{\rho(|z|)} \subset B_1(0) \subset \C^n$. For any $\epsilon > 0$ define,
\[ \rho_{\epsilon}(z) = \frac{1}{\epsilon^{2 n}} \rho \left( \frac{z}{\epsilon} \right) \]
which preserves the mass requirement,
\[ \int_{\C^n} \rho_{\epsilon}(z) \d{z} = 1 \]
Then $\supp{\rho_{\epsilon}} \subset B_\epsilon(0) \subset \C^n$. Define,
\[ \Omega_{\epsilon} = \{ x \in \Omega \mid d(x, \partial \Omega) > \epsilon \} \]
Then $\forall z \in \Omega_\epsilon$ we have,
\[ u_\epsilon(z) = (u * \rho_\epsilon)(z) = \int_{\zeta \in \C^n} u(\zeta) \rho_\epsilon(z - \zeta) \d{\zeta} \]
\end{definition}

\begin{lemma}
$u_\epsilon \in C^{\infty}(\Omega_\epsilon) \cap \PSH{\Omega_\epsilon}$ furthermore $u_\epsilon \downarrow u$ as $\epsilon \to 0$.  
\end{lemma}

\begin{proof}
We may rewrite,
\[ u_\epsilon(z) = \int_{|\zeta| < 1} u(z + \epsilon \zeta) \rho(\zeta) \d{\zeta} \]
Then $u_{\epsilon}$ satisfies the sub-mean-value innquality on any convex line. Now,
\begin{align*}
u_\epsilon(z) = \int_0^1 t^{2n - 1} \rho(t) \d{t} \int_{|\xi| = 1} u(z + \epsilon t \xi) \d{\sigma(\xi)} 
\end{align*}
The monotonicity shows that $u_\epsilon(z) \downarrow u(z)$. 
\end{proof}

\section{March 5}

We have shown that for any $u \in \PSH{\Omega}$ there exits a sequence $u\epsilon = u * \rho_\epsilon$ which a smooth and PSH such that $u_\epsilon$ converges monotonically to $u$. We also have,
\[ u \in C^2(\Omega) \cap \PSH{\Omega} \iff H(u) \ge 0 \iff \forall \xi \in \C^n  : \xi^* (i \partial \bar{\partial} u) \xi = \xi^i \frac{\partial^2 u}{\partial z_i \partial \bar{z}_j} \bar{\xi}^j \ge 0 \]


\begin{definition}
We say that the above innequality holds in the sense of distributions if $\forall \varphi \in C^\infty_0(\Omega)$ with $\varphi \ge 0$,
\[ \int_\Omega u \xi^i \overline{\xi}^j \frac{\partial^2 \varphi}{\partial z_i \bar{z}_j} \d{V} \ge 0 \]
This is well-defined since $\varphi$ is smooth and $u$ is locally integrable and $\varphi$ has compact support. Furthermore, let $D(\Omega) = C^\infty_0(\Omega)$ and $D'(\Omega) = D^*(\Omega)$ its dual space. For any $U \in D'(\Omega)$ we define its derivative $D_i U \in D'(\Omega)$ is defined by,
\[ D_i U(\varphi) = - U(D_i \varphi) \]
with sign chosen to be consistent with integration by parts. We say a distribution is non-negative if it sends non-negative functions to non-negative reals. Finally, if $u \in L^1_{\text{loc}}(\Omega)$ then,
\[ T_u(\varphi) = \int_{\Omega} u \varphi \d{V} \]
Then $T_u \in D'(\Omega)$. 
\end{definition}

\begin{proposition}
If $u \in \PSH{\Omega}$ then the above innequality holds in the sense of distributions. If $U \in D'(\Omega)$ is a distribution with positive complex hessian then there exists a unique $u \in \PSH{\Omega}$ such that $U = T_u$. 
\end{proposition}

\begin{proof}
Define,
\[ \Delta_\xi = \xi^i \overline{\xi}^j \frac{\partial^2}{\partial z_i \partial \bar{z}_j} \]
Assume that $u \in C^2(\Omega) \cap \PSH{\Omega}$ and $\Delta_\xi u \ge 0$. We need the following fact. If $u_\epsilon = u * \rho_\epsilon$ then,
\[ D_i u_\epsilon = (D_i u) * \rho_\epsilon = (D_i u)_\epsilon \]
in the sense of weak derivatives. Then $u_\epsilon \in \PSH{\Omega}$ so,
\[ (\Delta_\xi u)_\epsilon = \Delta_\xi u_\epsilon \ge 0 \]
In the limit $\epsilon \to 0$ we have $(\Delta_\xi u)_\epsilon \to \Delta_\xi u$. 
\bigskip\\
Given $U \in D'(\Omega)$ define $v_\epsilon = U * \rho_\epsilon$ which defines a function $v_\epsilon$ via $v_\epsilon(x) = U(\rho_\epsilon(x - \bullet))$.  Suppose that $U$ has positive hessian in the sense of distributions then $v_\epsilon \in \PSH{\Omega_\epsilon}$. Furthermore, $v_\epsilon$ defines a decreasing sequence and thus converges so some $u \in \PSH{\Omega}$. Then $U = T_u$. 
\end{proof}

\begin{lemma}
If $f : \Omega' \to \Omega$ is holomorphic and $\Omega' \subset \C^m$ and $\Omega \subset \C^n$ are domains and $u \in \PSH{\Omega}$ then $f^* u = u \circ f \in \PSH{\Omega'}$.
\end{lemma}

\begin{proof}
We may assume that $u \in C^2(\Omega)$. Let $\C^n$ have coordinates $z^i$ and $\C^m$ have coordinates $w$. Then,
\begin{align*}
\frac{\partial^2 u \circ f}{\partial w^i \partial \bar{w}^j} & = \pderiv{}{w^i} \pderiv{}{\bar{w}^j} (u \circ f) = \pderiv{}{w^i} \left( \pderiv{u}{z^a} \pderiv{f^a}{\bar{w}^j} + \pderiv{u}{\bar{z}^a} \pderiv{\bar{f}^a}{\bar{w}^j} \right)
\end{align*}
The first term vanishes because $f$ is holomorphic in each component. Then we get,
\begin{align*}
\frac{\partial^2 u \circ f}{\partial w^i \partial \bar{w}^j} & = \frac{\partial^2 u}{\partial z^b \partial \bar{z}^a} \pderiv{f^b}{w^i} \pderiv{\bar{f}^a}{\bar{w}^j}
\end{align*}
again using the fact that $\bar{f}$ is antiholomorphic. This matrix is positive definition when $u \in \PSH{\Omega}$. 
\end{proof}

\begin{lemma}[Hartog]
Let $\{ u_j \}$ locally uniformly bounded above sequence of PSH functions (i.e. for any compact $K \subset \subset \Omega$ then $u_j |_K \le c$). Then,
\begin{enumerate}
\item if $\{ u_j \}$ does not converge uniformly to $-\infty$ locally uniformly on $\Omega$ then there exists a subsequence $\{ v_j \} \subset \{u_j \}$ such that $v_j$ converge to $u \in \PSH{\Omega}$ in the $L^1_{\text{loc}}(\Omega)$ sense.
\item If $u_j \to U \in D'(\Omega)$ then $U = T_u$ for some $u \in \PSH{\Omega}$ and $u_j \to u$ in the $L^1_{\text{loc}}(\Omega)$ sense and $\limsup\limits_{\j \to \infty} u_j \le u$ and equality holds almost everywhere in $\Omega$. 
\end{enumerate} 
\end{lemma}

\begin{proof}
WLOG assume that $u_j \le 0$. There exists $K \subset \subset \Omega$ such that,
\[ \limsup_{\j \to \infty} \max_K u_j \ge 0 C > - \infty \]
Therefore, there exists $x_j \in K$ such that $u_j(x_j) \ge -2 C$. Then $x_{jk}$ coverges to $x_\infty \in K$. Take $v_k = u_{jk}$ then $v_k(x_{jk}) \ge - 2C$. For some small ball $B_\delta(x_\infty)$ we have,
\[ \left| \int_B v_k \right| \le C \]
Consider $B_\delta(x_\infty) \subset B_{2\delta}(x_{jk}) \subset \subset \Omega$. But $v_k$ is negative so,
\[ \int_{B_\delta(x_\infty)} v_k \ge \int_{B_{2r}(x_{jk})} v_k \ge v_k \Vol{B_{2r}(x_{jk})} \ge -2 C \Vol{B_{2r}(x_{jk})} \]
If we define the set,
\[ X = \left\{ x \in \Omega \mid \exists W : x \in W \text{ and } \left| \int_W v_k \right| \text{ is uniformly bounded} \right\} \]
Then $X$ is open and closed and $x_\infty \in X$ so $X = \Omega$. 
\end{proof}

\section{March 7}

\begin{theorem}[Tian]
Suppose that $(M, \omega)$ is a compact K\"{a}hler manifold. Then $\forall \varphi \in C^{\infty}(M)$ with $\omega + i \partial \bar{\partial} \varphi > 0 $ and $\sup_M{\varphi} = 0$. There exists $\alpha$ and $C$ depending on $\omega$ and $M$ such that,
\[ \int_M e^{-\alpha \varphi} \omega^n \le C \]
\end{theorem}

\begin{remark}
The $\alpha$-invariant of $M$ i defied as,
\[ \alpha(M) = \sup \left\{ \alpha > 0 \mid \int_M e^{-\alpha \varphi} \omega^n \le C_\alpha < \infty \right\} \]
\end{remark}

\begin{theorem}[Rieze Representation]
For any subharmonic function $\varphi$ on $B_1$,
\[ \varphi(z) = \frac{1}{2 \pi} \int_{B_1} \log{\left( \frac{|z - \zeta|}{|1 - z \bar{\zeta}|}\right)} \Delta \varphi \d{V} + \frac{1}{2 \pi} \int_{\partial B_1} \frac{1 - |z|^2}{|z - \zeta|^2} \varphi(\zeta) \d{\sigma(\zeta)} \]
\end{theorem}

\begin{lemma}[H\"{o}rmander]
Let $B = B_1(0) \subset \C^n$ there exists $C > 0$ depending only on $n$ such that $\forall \varphi \in \PSH{B}$ with $\varphi < 1$ in $B$ and $\varphi(0) = 0$ then,
\[ \int_{B_{1/2}} e^{- \varphi} \d{V} \le C \]
\end{lemma}

\begin{proof}
First assume $n = 1$. Applying the Rieze Representation theorem at $z = 0$ we get,
\begin{align*}
2 \pi \varphi(0) = 0 = \int_{B_1} \log{|\zeta|} \Delta \varphi \d{V} + \int_{\partial B_1} \varphi(\zeta) \d{\sigma(\zeta)} 
\end{align*}
Therefore, we find that,
\begin{align*}
2 \pi  = \int_{B_1} \log{\frac{1}{|\zeta|}} \Delta \varphi \d{V} + \int_{\partial B_1} \left(1 - \varphi(\zeta) \right) \d{\sigma(\zeta)} 
\end{align*}
However, both factors are positive and thus each must be less than $2 \pi$. Therefore,
\[ \int_{\partial B_1} \left( 1 - \varphi(\zeta) \right) \d{\sigma(\zeta)} \le 2 \pi \implies \int_{\partial B_1} |\varphi(\zeta)| \d{\sigma(\zeta)} \le 4 \pi \]
since $|varphi| \le |1 - \varphi| + 1$. Becuase for $|z| < 1/2$ we have,
\[ \frac{1 - |z|^2}{|z - \zeta|^2} \le \frac{1}{2} \]
then the boundary term is bounded. Fix $R \in (\tfrac{1}{2}, e^{-\tfrac{1}{2}})$ and set,
\begin{align*}
a & = \frac{1}{2 \pi} \int_{|\zeta| < R} \Delta \varphi \d{V} = \frac{1}{2 \pi} 
\end{align*}
(WHAT THE FUCK)
\end{proof}

\begin{proof}[Proof of Tian's Theorem]
Suppose that $\omega + i \partial \bar{\partial} \varphi > 0$ and $\sup_M \varphi = 0$. By Green's formula,
\[ 0 = \varphi(z_{\text{max}}) = \int \varphi \omega^n V^{-1} - \int G(z_{\text{max}}, y) \Delta \varphi(y) \omega^n \]
The second term is uniformly bounded above. Therefore,
\[ -C(n, \omega) \le \int_M \varphi \omega^n \le 0 \]
Because $M$ is compact, we can cover $M$ by finitely many euclidean balls $B_{r/2}(x_i)$ and WLOG assume that $B_{2r}(x_i)$ are also Euclidean. Over each ball we know,
\[ - C \le \int_{B_{r/2}(x_i)} \varphi \le \sup_{B_{r/2}(x_i)} \varphi \cdot \Vol{B_{r/2}(x_i)} \]
Since $\varphi$ is upper semicontinuous it must achieve its maximum at some $y_i \in B_{r/2}(x_i)$. Therefore,
\[ \varphi(y_i) \ge - \frac{C}{\Vol{(B_{r/2}(x_i)}} = C_0 \]
On this ball, which is simply connected, we may write $\omega = i \partial \bar{\partial} u_i$ and 
\[ -C_1 \le u_i \le 0 \]
Look at $\psi_i = u_i + \varphi \in \PSH{B_{2r}(x_i)}$. Then $B_{r}(y_i) \subset B_{2r}(x_i)$ also cover. Consider,
\[ \tilde{\psi}_i = \frac{\psi_i - \psi_i(y_i)}{C_0 + C_1} \] 
Then,
\[ \tilde{\psi}_i(y_i) = 0 \quad \text{and} \quad \tilde{\psi}_i < 1 \]
so by H\"{o}rmander's estimate, we get,
\[ \int_{B{r/2}(y_i)} e^{-\tilde{\psi}_i} \omega^n \le C \]
proving the result since this is valid on a cover.
\end{proof}

\subsection{Positive $(p,p)$-forms}

Let $\Omega \subset \C^n$ be a bounded domain. Denote $C^\infty_{(p,p)}(\Omega)$ to be the space of smooth $(p,p)$-forms,
\[ \alpha = \frac{1}{(p!)^2} \sum_{|J| = |K| = p} \alpha_{J \bar{K}} \d{z_{J}} \wedge \overline{z_K} \]
with $\alpha_{J \bar{K}} \in C^\infty(\Omega)$. We say such a $(p,p)$-form is simple positive if,
\[ \alpha = i^p \alpha_1 \wedge \overline{\alpha_1} \wedge \cdots \wedge \alpha_p \wedge \overline{\alpha_p} \]
For some $(1,0)$-forms $\alpha_j$. 

\begin{lemma}
The space of $(p, p)$-forms with constant coefficients is spanned by simple positive $(p, p)$-forms.  
\end{lemma}

\begin{proof}
We can write,
\[ \d{z}_j \wedge \d{z}_k = \frac{1}{4} \sum_{r = 1}^4 \left(\d{z}_j + i^r \d{z_k} \right) \wedge \overline{\left( \d{z}_j + i^r \d{z}_k \right)}  \]
\end{proof}

\begin{lemma}
If $f : \tilde{\Omega} \to \Omega$ is holomorphic and $\alpha \in C^\infty_{(p,p)}(\Omega)$ is simple positive then so is $f^* \alpha \in C^\infty_{(p,q)}(\Omega)$.
\end{lemma}

\begin{remark}
Let $\beta = \omega_{\C^n} = \sum i \d{z}_k \wedge \d{\overline{z}_k}$ be the Euclidean metric. We say that $\omega \in C^\infty_{(p,p)}(\Omega)$ is  positive $(p,p)$-form if for any simply positive $(n-p, n-p)$ form,
\[ \omega \wedge \alpha = \phi \beta^n \]
for some $\phi > 0$. 
\end{remark}

\section{March 14}

\newcommand{\dc}{\mathrm{d}^c}

\begin{example}
For any $u \in \PSH{\Omega}$ let $T = \mathrm{d}\mathrm{d}^c u = 2 i \partial \bar{\partial} u \in D'_{n-1, n-1}(\Omega)$ where,
\[ \dc = i \left( \bar{\partial} - \partial \right) \]
In fat, conversely, if $\pi(\Omega) = 0$ any closed positive current $T$ then $T = \d{\dc u}$ for some $u \in \PSH{\Omega}$. Suppose that $Z^m \subset \Omega^n$ is a complex submanifold then define $[Z] \in D_{(m,m)}(\Omega)$ as follows. $\forall \varphi \in D_{(m,m)}(\Omega)$ then we have,
\[ \inner{[Z]}{\varphi} = \int_{Z} \iota^* \varphi \]
where $\iota : Z \to \Omega$ is the natural embedding. Thus $[Z]$ is a positive form. If $\partial Z = \varnothing$ then $[Z]$ is closed. To see this,
\[ \inner{\d{[Z]}}{\varphi} = - \inner{[Z]}{\d{\varphi}} = - \int_Z \d{\varphi} = - \int_{\partial Z} \varphi = 0 \]
If $Z$ is a singular analytic set or singular comlplex variety, we may still define a current $[Z]$ via,
\[ \inner{[Z]}{\varphi} = \int_{Z^\text{reg}} \iota^* \varphi \]
where $Z^{\text{reg}}$ is the maximal smooth submanifold of $Z$. The above closedness and positivity also hold. 
\end{example}

\begin{theorem}[Poincare-Lelong]
Let $f \in \mathcal{O}(\Omega)$ be holomorphic and $Z = V(f)$ the vanishing of $f$. Then,
\[ [Z] = \frac{1}{2 \pi} \d{\dc \log{|f|^2}} \]
\end{theorem}

\begin{proof}
In the disc $D \subset X$ we have,
\[ \frac{1}{2 \pi} \d{\dc \log{|z|^2}} = \delta_{\{ 0 \}} \]
Therefore, this current is exactly supported on the vanishing $V(f)$ and on $V(f)$ has density $1$. 
\end{proof}

\begin{definition}
Let $u \in \PSH{\Omega} \cap L^\infty(\Omega)$ and $T$ a positive closed $(p,p)$-current. Then we define, $u T$ via,
\[ \inner{uT}{\varphi} = \inner{T}{u \varphi} \]
where we may need to regulate and take limits if $u \varphi$ is not smooth. If additionally, $u \in \PSH{\Omega} \cap C^2(\Omega)$ then $\d{\dc u}$ is a smooth $(1,1)$-form. Then we may define the wedge prouct $\d{\dc{u}} \wedge T$ via,
\[ \inner{\d{\dc{u}} \wedge T}{\varphi} = \inner{T}{\d{\dc{u}}}{\wedge \varphi} \]
If $T$ is closed and $u \in C^2(\Omega)$ then we have,
\[ \inner{\d{\dc{u}} \wedge T}{\varphi} = \inner{T}{u \d{\dc{\varphi}}} + \inner{T}{\d{\Psi}} \]
where $\Psi = \dc{u} \wedge \varphi - u \dc{\varphi}$. 
Therefore, using the fact that $T$ is closed,
\begin{align*}
\inner{\d{\dc{u}} \wedge T}{\varphi} & = \inner{T}{u \d{\dc{\varphi}}} + \inner{T}{\d{\Psi}} = \inner{T}{u \d{\dc{\varphi}}} - \inner{\d{T}}{\Psi} = inner{T}{u \d{\dc{\varphi}}} 
\\
& = \inner{\d{\dc{(u T)}}}{\varphi}
\end{align*}
Therefore, we find that,
\[ \d{\dc{u}} \wedge T = \d{\dc(u T)} \]
for smooth $u \in C^2(\Omega)$ and closed $T$. However, this definition makes sense for non-smooth $u$ so we will make this the definition for all $u \in \PSH{\Omega}$. 
\end{definition}


\begin{definition}
$\forall u \in \PSH{\Omega} \cap L^\infty(\Omega)$ and closed $(p,p)$-current $T$ define,
\[ \d{\dc{u}} \wedge T = \d{\dc{(uT)}} \] 
There exit $u_j \in C^\infty(\Omega) \cap \PSH{\Omega}$ with $u_j \downarrow u$ such that $u_j T \to u T$ as currents by the dominanted convergence theorem. Furthermore, as currents, $\d{\dc{(u_j T)}} \to \d{\dc{(uT)}}$. 
\end{definition}

\begin{definition}
We say that a sequence $T_j \to T$ converges as currents if $\forall \varphi$ the sequence, $\inner{T_j}{\varphi} \to \inner{T}{\varphi}$ converges.  
\end{definition}

\begin{remark}
If $T > 0$ then $\d{\dc{u}} \wedge T \ge 0$ for any $u \in \PSH{\Omega}$. 
\end{remark}

\begin{remark}
If $u_j \in \PSH{\Omega} \cap L^\infty(\Omega)$ is a sequence of bounded PSH functions then we may inductively define,
\[  \d{\dc{u_1}} \wedge \d{\dc{u_2}} \wedge \d{\dc{u_3}} \wedge \cdots \wedge \d{\dc{u_k}} \wedge T = \d{\dc{ \left( \d{\dc{u_1}} \wedge \d{\dc{u_2}} \wedge \d{\dc{u_3}} \wedge \cdots \wedge \d{\dc{u_{k-1}}} \wedge T \right) }} \] 
In particular, for $u \in \PSH{\Omega} \cap L^\infty(\Omega)$ the we may define the Monge-Ampere operator,
\[ MA(u) = (\d{dc{u}})^{\wedge n} = \d{\dc{u}} \wedge \cdots \wedge \d{\dc{n}}  \]
\end{remark}

\begin{remark}
For any $u \in \PSH{\Omega} \cap L^\infty(\Omega)$ then $\d{\dc{u}} \wedge \d{\dc{u}} \wedge T$ can be defied as follows. If $u \in C^2(\Omega)$ and $u \ge 0$ then we have,
\[ \d{\dc{u^2}} = 2 u \d{\dc{u}} + 2 \d{u} \wedge \dc{u} \]
Thus we have,
\[ \d{u} \wedge \dc{u} = \tfrac{1}{2} \d{\dc{u}} - \tfrac{1}{2} u \d{\dc{u}} \]
For a general $u \in \PSH{\Omega}$ define,
\[ \d{u} \wedge \dc{u} \wedge T = \tfrac{1}{2} \d{\dc{u^2}} \wedge T - \tfrac{1}{2} u \d{\dc{u}} \wedge T \]
which is a well-defined current. Furthermore, given a sequence $u_j \downarrow u$ then the sequence $\d{u_j} \wedge \dc{u_j} \wedge T \to \d{u} \wedge \dc{u} \wedge T$ converges as currents. 
\end{remark}

\begin{proposition}
For $u, v \in \PSH{\Omega} \cap l^\infty(\Omega)$ and $T$ a positive closed $(1,1)$-current,
\[ \d{u} \wedge \dc{b} \wedge T = \d{v} \wedge \dc{u} \wedge T \]
Also,
\[ \d{u} \wedge \dc{v} \wedge T = \left( i \partial u \wedge \bar{\partial} v + i \partial v \wedge \bar{\partial} u \right) \wedge T \]
\end{proposition}

\begin{theorem}[Cauchy-Schwartz]
\[ \left| \int_\Omega \d{u} \wedge \dc{v} \wedge T \right| \le \left| \int \d{u} \wedge \dc{u} \wedge T \right|^{\frac{1}{2}} \cdot \left| \int \d{v} \wedge \dc{v} \wedge T \right|^{\frac{1}{2}} \]
Define,
\[ (u, u) = \int \d{u} \wedge \dc{u} \wedge T \]
which then defines a positive definite inner product. 
\end{theorem}

\begin{theorem}[Stokes]
Suppose that $T$ is a smooth $(2n-1)$-form on $\overline{\Omega}$. Then,
\[ \int_{\Omega} \d{T} = \int_{\partial \Omega} T \]
Also we have a stronger statement:
\bigskip\\
If $T$ is a degree $(2n  1)$-form and $T$ is $C^1$ near $\partial \Omega$, then,
\[ \int_{\Omega} \d{T} = \int_{\partial \Omega} T \]
\end{theorem}

\begin{proof}
There exists a sequence $T_j$ of smooth $(2n - 1)$-forms such that $T_j \to T$ coverges. Let $\chi$ be a cuttoff function which is zero on an open region containing the boundary. Then,
\[ S_j = T(1 - \chi) + \chi T_j \]
is a $C^1$ form. Therefore, by the standard form of Stokes theorem,
\[ \int_{\partial \Omega} S_j = \int_{\Omega} \d{S_j} \to \int_\Omega \d{(T(1 - \chi) + \chi T)} = \int_\Omega \d{T} \]
However,
\[ \int_{\partial \Omega} S_j = \int_{\partial \Omega} T (1 - \chi) + \chi T_j = \int_{\partial \Omega} T \]
because $\chi = 0$ on the boundary. Therefore,
\[ \int_{\Omega} \d{T} = \int_{\partial \Omega} T \]
\end{proof}

\begin{corollary}
In particular, if $\supp{\chi} \subset \Omega$ then,
\[ \int_\Omega \chi \d{T} = - \int_\Omega \d{\chi} \wedge T \]
\end{corollary}

\begin{lemma}[Localization Principle]
Let $\Omega = \{ \rho < 0 \}$ for some $\rho \in \PSH{\Omega}$ which is a pseudoconvex domain. Fix a compact $K \subset \subset E \subset \subset \Omega$. Then for any $u \in \PSH{\Omega} \cap L^\infty(\Omega)$ with $u < 0$ there exists $\tilde{u} \in \PSH{\Omega} \cap L^\infty(\Omega)$ and $A > 0$ such that,
\begin{enumerate}
\item $u = \tilde{u}$ on a small neighborhood of $K$
\item $u = A \rho$ on $\Omega \setminus E$
\item $u \le \tilde{u} \le A \rho$ in $\Omega$
\end{enumerate}
\end{lemma}

\begin{proof}
For any $c > 0$, define, $\Omega_c = \{ \rho \le - c \} \subset \subset \Omega$. Since $K$ is compact we may take $K \subset \Omega_a$ for suficiently small $a > 0$. Take the constant $A$ such that,
\[ A \ge \frac{||u||_{L^\infty}}{a} \]
Then by definition, $u \ge A \rho$ on $\partial \Omega_a$ since there $\rho = - a$. Choose $b > 0$ sufficiently small such that on $\partial \Omega_b$ we have $u < A \rho$. Now define,
\[ \tilde{u}(z) = \begin{cases}
u(z) & z \in \Omega_a
\\
\max{(u(z), A \rho(z))} & z \in \Omega_b \setminus \Omega_a 
\\
A \rho(z) & z \in \Omega \setminus \Omega_b 
\end{cases} \]
\end{proof}

\begin{remark}
Now if $\{ u_1, \dots, u_k \} \subset L^\infty(\Omega) \cap \PSH{\Omega}$ by the above Localization Principle, we may assume that $u_j$ are all equal near $\partial \Omega$ and $u_j = 0$ on $\partial \Omega$. 
\end{remark}

\section{March 26}

\begin{theorem}[Chern-Levine-Ninenberg]
If $K \subset \subset U \subset \subset \Omega \subset \C^n$ then there exists $C(K, U, \Omega) > 0$ such that for all $u_1, \dots, u_p \in \PSH{\Omega} \cap L^\infty_{\text{loc}}(\Omega)$ an positive closed $(p,p)$ current $T$ then we have,
\begin{enumerate}
\item $ || \d{\dc{u_1}} \wedge \cdots \wedge \d{\dc{u_p}} \wedge T ||_K \le C || u_1 ||_{L^\infty(U)} \cdots || u_p ||_{L^\infty(U)} ||T||_U$ 
\item $ || \d{\dc{u_1}} \wedge \cdots \wedge \d{\dc{u_p}} ||_K \le C || u_1 ||_{L^1(U)} || u_1 ||_{L^\infty(U)} \cdots || u_p ||_{L^\infty(U)}$
\item $ || (-u_0) \d{\dc{u_1}} \wedge \cdots \wedge \d{\dc{u_p}} ||_K \le C || u_0 ||_{L^1(U)} \prod\limits_{j = 1}^n || u_j ||_{L^\infty(U)}$
\end{enumerate}
\end{theorem}

\begin{proof}
For $p = 1$ fix a cutoff function,
\[ \chi(x) = \begin{cases}
1 & x \in K 
\\
0 & x \in \Omega \setminus U 
\end{cases} \]
Then we have,
\[ int_K \d{\dc{u_1}} \wedge T \le \int_U \chi \d{\dc{u_1}} \wedge T = \int_U \d{\dc{(u_1 T)}} = \int_U u_1 \d{\dc{\chi}} \wedge T \le C || u_1 ||_{L^\infty(U_1)} \int_U \beta \wedge T \]
Now let $\beta = \omega_{\C^n}$ 
\end{proof}


\subsection{Capacity}

\renewcommand{\Cap}{\mathrm{Cap}}

\begin{definition}
Given $E$ a Borel set, its relative capacity is defined by,
\[ \Cap(E, \Omega) = \sup \left\{ \int_E (\d{\dc{u}})^n \: \middle| \: u \in \PSH{\Omega} \quad -1 \le u \le 0 \right\} \]
If $T$ is a positive closed $(p,p)$-current then,
\[ \Cap_T(E, \Omega) = \sup \left\{ \int_E (\d{\dc{u}})^p \wedge T \: \middle| \: u \in \PSH{\Omega} \quad -1 \le u \le 0 \right\} \]
\end{definition}

\begin{remark}
by CLN inneqality, $\Cap(E, \Omega)$ and $\Cap_T(E, \Omega)$ are both well-defined. Take,
\[ u = \frac{|z|^2}{\diam{\Omega}^2} \implies s\Cap(E, \Omega) \ge \int_E (\d{\dc{u}})^n = C \Vol{E} \]
\end{remark}

\begin{lemma}
Elementary properties of Capacity,
\begin{enumerate}
\item if $E_1 \subset E_2 \subset \Omega$ then $\Cap(E_1, \Omega) \subset \Cap(E_2, \Omega)$

\item if $E_j \uparrow E \subset \Omega$ then $\lim \Cap(E_j, \Omega) \le \Cap(E, \Omega)$

\item if $E = \bigcup E_j$ then $\Cap(E, \Omega) \le \sum \Cap(E_j, \Omega)$
\end{enumerate}
\end{lemma}

\begin{lemma}
If $K \subset \subset U \subset \subset \Omega$ then there exists $C > 0$ such that $\forall u < 0$ such that $u \in \PSH{\Omega}$ then,
\[ \Cap\left( \{ z \in K \mid u(z) < -j \}, \Omega \right) \le \frac{C ||u||_{L^1(U)}}{j} \]
\end{lemma}

\begin{proof}
Take any $v \in \PSH{\Omega}$ such that $-1 \le v \le 0$ then we have,
\[ \int_{K_j} (\d{\dc{v}})^n \le \int_{K_j} \frac{(-u)}{j} (\d{\dc{v}})^n  \le \frac{C}{j} \int_U (-u) \]
\end{proof}

\begin{definition}
Let $\mu$ be a measure then $f_i \to f$ in the measure sense if,
\[ \lim \mu(\{ x \mid |f_i(x) - f(x)| > \delta \}) = 0 \]
\end{definition}

\begin{definition}
A sequence of PSH functions $\{u_j\} : \Omega \to \R$ converges to $u$ with respect to capacity if $\forall \delta > 0$ and all compact $K \subset \subset \Omega$ we have,
\[ \lim_{j \to \infty} \Cap( \{ x \in K \mid |u_j - u| > \delta \}, \Omega) = 0 \]
\end{definition}

\begin{remark}
If $u_j \to u$ in the sense of capacity then their corresponding Modge-Ampere operators coverge in the sense of currents. 
\end{remark}

\begin{theorem}[Convergence]
Suppose that $\{ u^j_k \} \subset \PSH{\Omega}$ and locally uniformly bounded with $k = 1, \dots, n$ such that $\forall k : u^j_k \xrightarrow{j \to \infty} u_k$ with respect to capacity. Then,
\[ \d{\dc{u_1^j}} \wedge \cdots \d{\dc{u_n^j}} \to \d{\dc{u_1}} \wedge \cdots \wedge \d{\dc{u_n}} \]
in the sense of currents. 
\end{theorem}

\begin{proof}
This is a local statment since convergence of currents means that the integral agree on compact sets so if we prove the theorem for all open sets then there must be an open cover of each compact on which these integrals agree and thus they agree on a finite cover and thus they agree over the entire compact set. 
\bigskip\\
Assume all functions have range $[-1,0]$. Observe that,
\begin{align*}
\d{\dc{v_1}} & \wedge \d{\dc{v_2}} \wedge \cdots \d{\dc{v_n}} - \d{\dc{u_1}} \wedge \cdots \wedge \d{\dc{u_n}}
\\
& = \sum_{j = 1}^n \d{\dc{u_1}} \wedge \cdots \d{\dc{u_{j-1}}} \wedge \d{\dc{(v_j - u_J)}} \wedge \d{\dc{v_{j+1}}} \wedge \cdots \wedge \d{\dc{v_n}} 
\end{align*}
Applying this formula to $u^j_k$, the theorem is reduced to showing that if $u_j \to u$ in the capacity sense then $\d{\dc{(u_j - u)}} \wedge T_j \to 0$ in the current sense where $T_j$ is a $(1,1)$-current of the form $\d{\dc{v_1}} \wedge \cdots \wedge \d{\dc{v_{r-1}}}$. 
\end{proof}

\section{March 27}

\begin{remark}
Recall that if $\{ u_j \} \subset \PSH{\Omega} \cap L^\infty_{loc}(\Omega)$ converges to $u$ in the sense of capacity then,
\[ MA(u_j) \to MA(u) = (\d{\dc{u})})^n \]
However, this is difficult to check in general.
\end{remark}

\begin{proposition}
Take $\{ u_j \} \subset \PSH{\Omega} \cap L^\infty_{loc}(\Omega)$ locally uniformly bounded and $u_j \downarrow u \in \PSH{\Omega} \cap L^\infty_{loc}(\Omega)$ then $u_J \to u$ in the sense of capacity.
\end{proposition}

\newcommand{\ddc}[1]{\d{\dc{#1}}}

\begin{proof}
By localization princile we may assume that $u_j = u = A \rho$ on $\Omega \setminus E$. Furthermore, assume that all functions take values in $[-1,0]$. For any fixed PSH function $-1 \le v \le 0$ we define the functional,
\[ I_0(v) = \int_\Omega (u_j - u) (\ddc{v})^n \ge 0 \]
On the set $E_\delta = E \cap \{ u_j - u > \delta \}$ then,
\[ \int_{E_\delta} (u_j - u) (\ddc{v})^n \ge \int_{E_\delta} \delta (\ddc{v})^n \]
Therefore,
\[ \Cap(E_\delta, \Omega) \le \frac{1}{\delta} \sup_v I_0(v) \]
Now define,
\begin{align*}
I_k(v)  & = \int_{E} (u_j - u) (\ddc{v})^{n - k} \wedge (\ddc{u})^k
\\
& = - \int_E \d{(u_j - u)} \wedge \dc{v} \wedge (\ddc{v})^{n - k - 1} \wedge (\ddc{u})^{k}
\\
& \le \left( \int_E \d{(u_j - u)} \wedge \dc{(u_j - u)} \wedge (\ddc{v})^{n - k - 1} \wedge (\ddc{u})^{k} \right)^{\frac{1}{2}}
\\
& \cdot \left( \int_E \d{v} \wedge \dc{v} \wedge (\ddc{v})^{n - k - 1} \wedge (\ddc{u})^{k} \right)^{\frac{1}{2}}
\end{align*}
By the Cauchy-Schwartz innequality. Furthermore,
\[ \ddc{(v + 1)^2} = 2 (v + 1) \ddc{v} + 2 \d{v} \wedge \dc{v} \]
and thus,
\[ \int_E \d{v} \wedge \dc{v} \wedge (\ddc{v})^{n - k - 1} \wedge (\ddc{u})^{k} \le \int_E \ddc{(v + 1)^2} \wedge (\ddc{v})^{n - k - 1} \wedge (\ddc{u}))^k \]
However,
\[ \ddc{(n+1)^2} \le \ddc{( ( v + 1)^2 + v + u )} \]
and thus,
\[ \int_E \ddc{(v + 1)^2} \wedge (\ddc{v})^{n - k - 1} \wedge (\ddc{u}))^k \le \int_E \left( \ddc{((v + 1)^2 + v + u)} \right)^n \le 3^n \Cap(E, \Omega) \le \infty \]
Furthermore, 
\begin{align*}
\int_E \d{(u_j - u)} \wedge \dc{(u_j - u)} \wedge (\ddc{v})^{n - k - 1} \wedge (\ddc{u})^{k} & = - \int_\Omega (u_j - u) \ddc{(u_j - u)} \wedge (\ddc{v})^{n - k - 1} \wedge (\ddc{u})^{k}
\\
& \le \int_E (u_j - u) \ddc{u} \wedge (\ddc{v})^{n - k - 1} \wedge (\ddc{u})^k = I_{k+1}(v) 
\end{align*}
Therefore, $I_k \le C I_{k+1}^{\frac{1}{2}}$ for $k < n$. Iterating, we find,
\[ I_0 \le C I_n^{\frac{1}{2^n}} = C \left( \int_\Omega (u_j - u) (\ddc{u})^n \right)^{\frac{1}{2^n}} \]
Then by dominated convergence theorem, this goes to zero. 
\end{proof}

\begin{corollary}
If $u_1, \dots, u_n \in \PSH{\Omega} \cap L^\infty_{loc}(\Omega)$ the map,
\[ (u_1, \dots, u_n) \mapsto \ddc{u_1} \wedge \cdots \wedge \ddc{u_n} \]
is symmetric.
\end{corollary}

\begin{proof}
This is true if all are smooth and in general follows by regularization. 
\end{proof}

\begin{theorem}
$\forall u \in \PSH{\Omega} : \forall \epsilon > 0 : \exists$ an open set $U$ such that $\Cap(U, \Omega) < \epsilon$ and $u|_{\Omega \setminus U}$ is continuous. 
\end{theorem}

\begin{proof}
Fix a compact $K \subset \subset \Omega$. Recall that,
\[ \Cap(\{ u < - N \} \cap K, \Omega) \le \frac{C}{N} || u ||_{L^1} \]
Fix $N$ sufficiently large such that $C || u ||_{L^1} / N < \epsilon / 10$. Consider $\max(u, -N)$ and take $\{ u_j \} \downarrow \max(u, - N)$. The previous proposition implies that $u_j \to \max(u, - N)$ in the sense of capacity. This implies that $\forall k \in \N$ we may take some $j_k$ such that,
\[ U_k = \{ u_{j_k} - \max(u, - N) \ge \tfrac{1}{k} \} \cap K \]
has arbitrarily large capaciy, in particular, we make take,
\[ \Cap(U_k, \Omega) < \frac{\epsilon}{2^{k+1}} \]
Define,
\[ U = \bigcup_{k = 1}^\infty U_k \] 
Then by subadditivity of capacity,
\[ \Cap(U, \Omega) < \epsilon \]
Furthermore, on $K \setminus U$ we know that $u_{j_k} - \max(u, -N) \le \frac{1}{k}$ which implies that the subsequence $u_{j_k} \to \max(u, -N)$ converges uniformly. Since each $u_{j_k}$ is continuous, uniform convergence implies that $\max(u, -N)$ is continuous on $K \setminus U$. Therefore, $u$ is continuous on the set $(K \setminus U) \setminus \{ x \in K \mid u(x) < - N \}$. Adding this sublevel set does not ruin the bound on capacity because we have forced this set to have arbitrarily small capcity via a judicious choice of $N$.
\bigskip\\
Now take a compact exhaustion $K_j \uparrow \Omega$. Then there exists $U_j \subset K_j$ such that $\Cap(U_J, \Omega) < \frac{\epsilon}{2^n}$ and $u$ is continuous on $K_j \setminus U_j$. Finally, take,
\[ U = \bigcup_{j} U_j \]
\end{proof}

\newcommand{\U}{\mathcal{U}}

\begin{corollary}
Denote $\U = \{ u \in \PSH{\Omega} \mid -1 \le u \le 0 \}$. The currents $T_j$ and $T$ are the wedge products of $\ddc{u}$ for $u \in \U$. If $T_j \to T$ in the sense of currents then for any $u \in \PSH{\Omega}$ the current $u T_j \to u T$ in the sense of currents. 
\end{corollary}

\begin{proof}
For any $\epsilon > 0$ we can find a continuous function $v \in C^0(\Omega)$ such that,
\[ \Cap(\{x \in \Omega \mid u \neq v \}, \Omega) < \epsilon \]
by taking a continuous extension of $u$ and using the previous proposition to find the bound on capacity. Now we use,
\[ \Cap(U, \Omega) \ge C \Vol{U} \]
to find that,
\[ \Vol{ \{x \in \Omega \mid u \neq v \} } < C^{-1} \epsilon \]
Therefore,
\[ ||(u - v) T_j ||_k \le C || u -v ||_{L^1} \quad \quad \quad || (u - v) T ||_k \le C || u - v||_{L^1} \]
Now recall:

\begin{proposition}
For any measureable $f \in L^1(\Omega)$ then $\forall \epsilon > 0 : \exists \delta > 0$ such that,
\[ \Vol{U} < \delta \quad \quad \quad || f ||_{L^1(U)} < \epsilon \]
\end{proposition}
Therefore, we can make $|| u - v ||_{L^1}$ arbitrarily small. 
\end{proof}

\begin{theorem}
Let $\{ u^j_k \} \subset \PSH{\Omega} \cap L^\infty_{loc}(\Omega)$ for $k = 1, \dots, N$ and suppose that $u^j_k \uparrow u_k \in \PSH{\Omega} \cap L^\infty_{loc}(\Omega)$. Then,
\[ \ddc{u_1^j} \wedge \cdots \wedge \ddc{u_N^j} \xrightarrow{j \to \infty} \ddc{u_1} \wedge \cdots \wedge \ddc{u_N} \]
\end{theorem}

\begin{proof}
For $N = 1$ this is clear via integration of parts. Assume that is holds for any $1, \dots, N$. Then,
\[ T_j = \ddc{u_1^j} \wedge \cdots \wedge \ddc{u_N^j} \to T = \ddc{u_1} \wedge \ddc{u_N} \]
Goal: if $u_j \uparrow u$ then $\ddc{v_j} \wedge T_j \to \ddc{v} \wedge T = \ddc{(v T)}$. Claim: $v_j T_j \to v T$. To prove this claim, use the localization principle. Assume that all functions involved can be written as $A \rho$ near $\partial \Omega$. Then $v_j T_j \le v Tj$ and by our previous corollary $v T_j \to v T$ and thus,
\[ \limsup_{j \to \infty} v_j T_j \le v T_j = v T \]
Now take an arbitrary (simple) closed positive $(n - N, n - N)$-form $\omega$. Consider,
\[ \liminf_{j \to \infty} \int_\Omega v_j T_j \wedge \omega \ge \liminf_{j \to \infty} \in v_k T_j \wedge \omega \]
For any fixed $k \in \N$, the corollary implies that,
\[ \liminf_{j \to \infty} \int_\Omega v_k T_j \wedge \omega = \int_\Omega v_k T \wedge \omega \] 
By integration by parts,
\[ \int_\Omega v_k T \wedge \omega = \int_\Omega u_1 \ddc{v_k} \wedge \ddc{u_2} \wedge \cdots \wedge \ddc{u_N} \wedge \omega \]
By the induction hypothesis and the corrolary,
\[ u_1 \ddc{v_k} \wedge \ddc{u_2} \wedge \cdots \wedge \ddc{u_N} \wedge \omega \to u_1 \ddc{v} \wedge \ddc{u_2} \wedge \cdots \wedge \ddc{u_N} \wedge \omega \] 
Therefore,
\[ \int_\omega u_1 \ddc{v_k} \wedge \ddc{u_2} \wedge \cdots \wedge \ddc{u_N} \wedge \omega \to \int_\Omega u_1 \ddc{v} \wedge \ddc{u_2} \wedge \cdots \wedge \ddc{u_N} \wedge \omega \]
Finally, by integration by parts,
\[  \int_\Omega u_1 \ddc{v} \wedge \ddc{u_2} \wedge \cdots \wedge \ddc{u_N} \wedge \omega = \int_\Omega v \ddc{u_1} \wedge \ddc{u_2} \wedge \cdots \wedge \ddc{u_N} \wedge \omega = \int_\Omega v T \wedge \omega \]
Therefore,
\[ \liminf_{j \to \infty} \int_\Omega v_k T_j \wedge \omega \ge \int_\Omega v T \wedge \omega \]
\end{proof}

\section{April 2}

\subsection{Comparison Principle}

\begin{theorem}
Let $\Omega \subset \C^n$ be bounded open and $u,v \in \PSH{\Omega} \cap L^\infty(\Omega)$ and $u \ge v$ on $\partial \Omega$ in the sense that,
\[ \liminf_{z \to \partial \Omega} (u - v) \ge 0 \]
Then,
\[ \int_{\{z \in \Omega \mid u(z) < v(z) \}} (\ddc{v})^n \le \int_{\{z \in \Omega \mid u(z) < v(z) \}} (\ddc{u})^n \] 
\end{theorem}

\begin{proof}
Assume $u,v \in C^\infty(\Omega)$ and $E = \{ z \in \Omega \mid u(z)  _ \epsilon < v(z) \} \subset \subset \Omega$. Ket $v_k = \max(v, u + \tfrac{1}{k}) \in \PSH{\Omega} \cap L^\infty(\Omega)$. Then, near $\partial \Omega$ and $\partial E$ we have $v_k = u + \frac{1}{k}$. On open set $E$ we have $v_k \downarrow v$ which implies that $(\ddc{v_k})^n \to (\ddc{v})^n$. For any compact $K \subset \subset E$ take a bump function $\phi$ which is $1$ on $K$ and $0$ near $\partial E$ then,
\[ \int_{K_j} (\ddc{V})^n \le \int_E \phi (\ddc{v})^n = \lim_{k \to \infty} \int_E \phi (\ddc{v_k})^n \le \lim_{k \to \infty} \int_E (\ddc{v_k})^n \]
Applying Stokes' theorem,
\[ \lim_{k \to \infty} \int_E (\ddc{v_k})^n = \lim_{k \to \infty} \int_{\partial E} \dc{v_k} \wedge (\ddc{v_k})^{n-1} = \lim_{k \to \infty} \int_{\partial E} \dc{u} \wedge (\ddc{u})^{n-1} = \in_E (\ddc{u})^n \]
Choose $K_i \uparrow E$ then,
\[ \int_E (\ddc{u})^n \le \int_E (\ddc{u})^n \]
For general $u,v$ by almost continuity in the Capacity sense, for fixed $\epsilon > 0$ ther exists open $U$ such that $\Cap(U, \Omega) < \epsilon$ and $u = u_0$, $v = v_0$ on $\Omega \setminus U$ where $u_0, v_0 \in C^0(\Omega)$. Take regularizations $u_k \downarrow u$ and $u_k \downarrow v$. Then $\forall \delta > 0$ we have,
\[ E_k(\delta) = \{ u_k + \delta < v_k \} \] 
Recall that $u_k \to u$ and $v_k \to v$ converge uniformly on $\Omega \setminus U$. Therefore 
\[ E_0(2 \delta) \setminus U \supset \bigcap_{k \gg 1} E_k(\delta) \setminus U \]
Furthermore,
\[ \bigcup_{k \gg 1} E_k(\delta) \setminus U \subset E_0(0) \setminus U \]
Then 
\[ \int_{E(2 \delta) \setminus U} (\ddc{v})^n = \int_{E_0(2 \delta \setminus U} (\ddc{v})^n \]
By continuity, $E_0(2 \delta)$ is open. Using the above inclusions,
\[ \int_{E(2 \delta \setminus U} (\ddc{v})^n = \lim_{k \to \infty} \int_{E_0(2 \delta) \setminus U} (\ddc{v_k})^n \le \lim_{k \to \infty} \int_{E_k(2 \delta) \setminus U} (\ddc{v_k})^n \le \lim_{k \to \infty} \int_{E_k(2 \delta)} (\ddc{v_k})^n \]
Then by the first part,
\[ \lim_{k \to \infty} \int_{E_k(2 \delta)} (\ddc{v_k})^n \le \lim_{k \to \infty} \int_{E_k(\delta)} (\ddc{v_k})^n = \lim_{k \to \infty} \int_{E_k(\delta) \setminus U} (\ddc{v_k})^n + \int_U (\ddc{v_k})^n \]
However, 
\[ \int_U (\ddc{v_k})^n \le \Cap(U, \Omega) <\epsilon \]
Therefore, 
\[ \lim_{k \to \infty} \int_{E_k(2 \delta)} (\ddc{v_k})^n \le \lim_{k \to \infty} \int_{E_0(0) \setminus U} (\ddc{u_k})^n + \epsilon = \int_{E_0(0) \setminus U} (\ddc{u})^n + \epsilon \]
Finally,
\[ \int_{E_0(2 \delta) \setminus U} (\ddc{v})^n \le \int_{\{ u < v \}} (\ddc{u})^n + \epsilon \]
\end{proof}

\begin{corollary}
Suppose that $(\ddc{u})^n \le (\ddc{v})^n$ in $\Omega$ and $u \ge v$ on $\partial \Omega$ then $u \ge v$ in $\Omega$.
\end{corollary}

\begin{proof}
Suppose not $\{ u < v \}$ is nonempty then by upper semicontinuity there exists $\epsilon > 0$ such that $\{ u + \epsilon < v \}$. Fix a strictly PSH function $\rho$ on $\Omega$ for example,
\[ \rho(z) = \left( \frac{|z|^2}{\diam{\Omega}^2} - 1 \right) \epsilon \implies - \epsilon < \rho < 0 \]
Then $\{ u < v - \epsilon \} \subset \{ u < v + \rho \}$ is nonempty. Then consider the integral,
\[ \int_{\{u < v + \rho \}} (\ddc{v})^n < \int_{\{u < v + \rho \}} (\ddc{(v + \rho)})^n \] 
Now by the comparison principle, 
\[ \int_{\{u < v + \rho \}} (\ddc{(v + \rho)})^n \le \int_{\{u < v + \rho\}} (\ddc{u})^n \le \int_{\{u < v + \rho \}} (\ddc{v})^n \] 
which is a contradiction. 
\end{proof}

\begin{theorem}
Let $\Omega \subset \C^n$ be a bounded domain and $u,v \in \PSH{\Omega} \cap L^\infty$. We have $\max(u,v) \in \PSH{\Omega} \cap L^\infty$ and then,
\[ (\ddc{\max(u,v})^n) \ge (\ddc{u})^n \chi_{\{ u \ge v\}} + (\ddc{v})^n \chi_{\{u < v \}} \]
\end{theorem}

\begin{proof}
WLOG take $-1 \le u,v \le 0$. Take compact $K \subset \{ u \ge v \}$. For any $\epsilon > 0$ there exists open $U$ s.t. $\Cap(U, \Omega) < \epsilon$ and $u = u_0$, $v = v_0$ on $\Omega \setminus U$ where $u_0, v_0 \in C^0(\Omega)$. Take $u_J \downarrow u$ in $\Omega$ a convergent sequence of smooth PSH functions. Now let,
\[ V_0 = \{ v_0 < u_0 + \delta \} \] 
on $V_0 \setminus U$ we know that $v < u + \delta$ (since $u = u_0$ and $v = v_0$). In particular, $K \subset V_0(\delta) \cup U$. Then we find,
\[ \int_K (\ddc{u})^n = \lim_{j \to \infty} \int (\ddc{u_j})^n \le \lim_{j \to \infty} \left[ \int_{V_0(\delta) \setminus U} (\ddc{v_j})^n + \int_U (\ddc{u_j})^n \right]  \] 
However, 
\[ \int_U (\ddc{u_j})^n \le \Cap(U, \Omega) \le \epsilon \]
Therefore, 
\[ \int_K (\ddc{u})^n \le \lim_{j \to \infty} \int_{V_0(\delta) \setminus U} (\ddc{\max(u_j + \delta, v)})^n + \epsilon \]
However, $\max(u_j + \delta, v) \to \max(u + \delta, v)$ and thus,
\begin{align*}
\lim_{j \to \infty} \int_{V_0(\delta) \setminus U} (\ddc{\max(u_j + \delta, v)})^n + \epsilon & = \int_{V_0(\delta) \setminus U} (\ddc{\max(u + \delta, v)})^n + \epsilon 
\\
& \le \int_{\{v < u + \delta\}} (\ddc{\max(u + \delta, v)})^n + \epsilon
\end{align*}
Then let $\delta \to 0$ which implies that $\{ v < u + \delta \} \to \{ v < u \}$ and $\ddc{\max(u + \delta, v)} \to \ddc{\max(u , v)}$. Then dominated convergence theorm implies that this converges to,
\[ \int_{\{ v \le u \}} (\ddc{\max(u,v)})^n + \epsilon \]
Letting $\epsilon \to 0$ we have the result. 
\end{proof}

\subsection{Relative Extremal Functions}

\begin{definition}
Given $E \subset \Omega$ we define its relative extremal function,
\[ u_{E, \Omega} = \sup \{ u \in \PSH{\Omega} \mid u|_\Omega < 0 \text{ and } u|_E < - 1\} \]
However, the resulting function may not be PSH. In this case, we take instead $u_{E,\Omega}^*(z) = \limsup\limits_{w \to z} u_{E, \Omega}(w)$, the upper-semicontinuous regularization of $u_{E, \Omega}$, is PSH. Thus,
\[ u_{E, \Omega}^* \in \PSH{\Omega} \cap L^\infty \] 
\end{definition}

\begin{remark}
The relative extremal function is bounded $-1 \le u_E \le 0$ by construction. 
\end{remark}

\begin{remark}
Choquet's lemmma gives a sequence $u_j \in \PSH{\Omega}$ such that $u_j \uparrow u_E$ pointwise.
\end{remark}

\begin{proposition}
We have,
\begin{enumerate}
\item if $E_1 \subset E_2 \subset \subset \Omega$ then $u_{E_2}^{*} \le u_{E_1}^*$
\item if $E_j \downarrow E$ and $E_j$ are compact then,
\[ \left( \lim_{j \to \infty} u_{E_j}^* \right)^* = u_E^8 \]
\end{enumerate}
\end{proposition}

\begin{proof}
Given such $E_j \downarrow E$ then $u_E^* \ge u_{E_j}^*$ implies that $\left( \lim_{j \to \infty} u^*_{E_j} \right)^* \le u_E^*$. Now, fix any $v \in \PSH{\Omega}$ such that $v < 0$ in $\Omega$ and $v < -1$ o $E$. Then the set,
\[ S_\epsilon = \{ z \in \Omega \mid v(z) < -1 + \epsilon \} \]
is open and contains $E$. Since $E_j$ is compact and $E_j \downarrow E$ then for sufficiently large $j$ we have $E_j \subset S_\epsilon$. Then $v - \epsilon$ is PSH and satisfies the requirements such that $v - \epsilon \le u_{E_j} \le u^*_{E_j}$. Then taking the limit,
\[ v - \epsilon \le \lim_{j \to \infty} u_{E_j}^* \]
Letting $\epsilon \to 0$ then we have,
\[ v \le \lim_{j \to \infty} u_{E_j}^* \]
Taking the suppremum over all such $v$ we find,
\[ u_E^* \le \left( \lim_{j \to \infty} u_{E_j}^* \right)^* \] 
\end{proof}

\begin{definition}
Let $E \subset \subset \Omega$  then the outer capacity is,
\[ \Cap^*(E, \Omega) = \inf \{ \Cap(U, \Omega) \mid U \supset E \text{ open} \} \]
\end{definition}

\begin{theorem}
if $E \subset \subset \Omega$ is relatively compact and $\Omega$ is pseudo-convext (i.e. there exists $h \in PSH{\Omega} \cap C^\infty$ s.t. $h|_\Omega = 0$) then,
\[ \Cap^*(E, \Omega) = \int_\Omega (\ddc{u_E^*})^n \] 
Moreover if $E_j \downarrow E$ and $E_j$ is compact then,
\[ \lim_{j \to \infty} \Cap(E_j, \Omega) = \Cap(E, \Omega) = \Cap^*(E, \Omega) \] 
\end{theorem}

\section{April 9}

\begin{definition}
Let $(M, \omega)$ be a Kahler manifold. Then define,
\[ \PSH{M,\omega} = \{ \varphi \in L^1(M) \mid \omega + i \partial \bar{\partial} \varphi > 0 \text{ and } \varphi \text{ is u.s.c} \} \]
\end{definition}


\begin{lemma}
There exists $\delta(M, \omega)$ and $C(M, \omega, ||e^f||_{L^p})$ such that for any $E \subset M$,
\[ \int_E (\omega + i \partial \bar{\partial} \varphi)^n \le C e^{-\delta \left( \frac{V}{\Cap_{\omega}(E)} \right)^{\frac{1}{n}}} \]
where $\varphi$ solves MA. 
\end{lemma}

\begin{corollary}
\[ \frac{1}{V} \int_E (\omega + i \partial \bar{\partial})^n \le C \left( \Cap_\omega(E) \right)^2 \]
\end{corollary}

\begin{proof}
Follows from the fact that $x^2 e^{-\delta x^2}$ is uniformly bounded in $x$.
\end{proof}

\begin{lemma}
$\forall u \in \PSH{M, \omega} \cap L^\infty(M)$ then $\forall \delta > 0 : \forall \gamma \in [0,1]$ we have,
\[ \gamma^n \Cap_{\omega}\left( \{ u < -\delta - \gamma \} \right) \le \int_{\{u < - \delta \}} (\omega + i \partial \bar{\partial} u) \]
\end{lemma}

\begin{lemma}
If $\varphi$ solve sMA then $\forall s > 1$,
\[ \frac{1}{V} \Cap_\omega\left( \{ \varphi < - s \} \right) \le \frac{c}{(s - 1)^{\frac{1}{q}}} \]
\end{lemma}

\begin{proof}
By the previous lemma,
\begin{align*}
 \Cap_{\omega}\left( \{ \varphi < - (s-1) - 1 \} \right) \le \int_{\{ \varphi < -(s-1) \}} (\omega + i \partial \bar{\partial} \varphi)^n & = \int_{\{ \varphi < - (s-1) \}} e^f \omega^n 
 \\
 & \le \int_{\{ \varphi < - (s-1) \}} \left( \frac{-\varphi}{s-1} \right)^{\frac{1}{q}} e^f \omega^n 
\\
& \le \frac{||e^f||_{L^p}}{(s - 1)^{\frac{1}{q}}} \int_M |\varphi| \omega^n 
\end{align*}
Then by Green's formula and $\sup{\varphi} = 0$ we have,
\[ \int |\varphi| \omega^n \le  \]
\end{proof}

\begin{lemma}
Let $F : [0, \infty) \to [0, \infty)$ be a decreasig and right-continuous function such that,
\[ \lim_{s \to \infty} F(s) = 0 \quad \quad \quad \forall \gamma \in [0, 1] \]
and,
\[ \gamma F(s + \gamma) \le A F(s)^{1 + \alpha} \]
for $\alpha > 0$ then $\exists E_\infty(A, \alpha, s0) > 0$ such that,
\[ F(s) \equiv 0 \quad \forall s \ge S_\infty \]
where $s_0$ is the first positive value such that,
\[ F(s)^\alpha < \frac{1}{2 A} \]
\end{lemma}

\section{$L^2$- Estimates for $\delta$-Equation}

Let $\Omega \subset \C^n$ be a bounded domain and $\varphi \in C^0(\Omega)$. Define,
\[ L^2(\Omega, \varphi) = \left\{ u \in L^1(\Omega) \: \middle| \: \int_\Omega u^2 e^{-\varphi} \d{V} < \infty \right\} \]
We denote this integral as,
\[ || u ||_{\varphi} = \int_\Omega u^2 e^{-\varphi} \d{V} \]
We also introduce the corresponding space for $(p,q)$-froms. Where $\omega \in L^2_{(p,q)}(\Omega, \varphi)$ if when written as,
\[ \omega = \sum'_{|I| = p, |J| = q} \omega_{I\bar{J}} \d{z_I} \wedge \d{\overline{z}_J} \]
we have $\omega_{I \bar{J}} \in L^2(\Omega, \varphi)$. Then define,
\[ |\omega|^2 = \sum' |\omega_{I \bar{J}} |^2 \quad \quad \quad || \omega ||_{\varphi} \sum || \omega_{I \bar{J}} ||_{\varphi} \]
Then $L^2(\Omega, \varphi)$ is a Hilbert space under the inner product,
\[ \inner{u}{v}_\varphi = \int_\Omega u \bar{v} e^{-\varphi} \d{V} \]
Now we introduct the space of testing forms,
\[ D_{(p,q)}(\Omega) = \{ \text{smooth } (p,q)-\text{forms with cpt support in } \Omega \} \]
Thus, $D_{(p,q)}(\Omega) \subset L^2_{(p,q)}(\Omega, \varphi)$ and is dense by smooth regularization under convolution by a smooth modifier.  Given $\varphi_1, \varphi_2 \in C^0(\Omega)$ consider the operator,
\[ T : L^2_{(p,q)}(\Omega, \varphi_1) \to L^2_{(p,q+1)}(\Omega, \varphi_2) \]
where $T = \bar{\partial}$ on $D_{(p,q)}(\Omega)$ but we will extend the operator to the full $L^2$ space. Consider $D_T$, the domain of $T$. Since it constains $D_{(p,q)}(\Omega)$ the domain is dense. $u \in D_T$ if $\bar{\partial} u$ is $L^2(\Omega, \varphi_2)$ in the sense of distributions meaning that,
\[ \int \bar{\partial} \varphi = \pm \int u \bar{\partial} \varphi \le C \left( \varphi^2 \right)^{\frac{1}{2}} \]
Then $T u \in L^2_{(p, q+1)}(\Omega, \varphi_2)$ and $T$ is closed.
\bigskip\\
\begin{remark}
For suitable $\varphi_1, \varphi_2 \in C^0(\Omega)$ we want to show that for any $f \in L^2_{(p,q)}(\Omega, \varphi_2)$ with $\bar{\partial} f= 0 $ then there exists $u \in L^2_{(p,q)}(\Omega, \varphi_1)$ s.t. $\bar{\partial} u = f$ with estimates on $u$.
\end{remark} 

\begin{definition}
Let $H_1$ and $H_2$ be Hilbert spaces with $D_T \subset H_1$ a subspace. Consider an operator,
\[ T : D_T \to H_2 \] 
then we define the graph of $T$ a subset $G(T) \subset H_1 \times H_2$ via,
\[ G(T) = \{ (x, Tx) \mid x \in D_T \} \] 
\end{definition}

\begin{definition}
For $T : D_T \to H_2$ we hay that $T$ is
\begin{enumerate}
\item \textit{densely defined} if $D_T \subset H_1$ is dense

\item closed if $G(T) \subset H_1 \times H_2$ is closed.
\end{enumerate}
\end{definition}

\begin{definition}
We define $T^* : D_{T^*} \to H_1$ where for $\eta \in H_2$ we say $\eta \in D_{T^*}$ if the map $L_\eta : D_T \to \C$ via $\xi \mapsto \inner{T \xi}{\eta}_{H_2}$ is a continuous bouned linear functional. By Riesz representation then $L_\eta(\xi) = \inner{\xi}{u}_{H_1}$ for some $u \in H_1$ then for $\xi \in D_T$ we define $T^* \eta = u$ i.e.
\[ \inner{T \xi}{\eta}_{H_2} = \inner{\xi}{T^* \eta}_{H_1} \] 
\end{definition}

\begin{proposition}
Suppose that $T : H_1 \to H_2$ is densly defined and closed then $T^* : H_2 \to H_1$ is also densely defined and closed.
\end{proposition}

\begin{proof}
Definem $F : H_1 \times H_2 \to H_2 \times H_1$ via $(\xi, \eta) \mapsto (-\eta, \xi)$. Note that for $(\eta,\xi) \in H_2 \times H_1$,
\[ (\eta, \xi) \perp F(G(T)) \iff (\eta, \xi) \cdot (-Tx, x) = \inner{\eta}{-Tx}_{H_2} + \inner{\xi}{x}_{H_1} = 0 \iff \inner{x}{\xi}_{H_1} = \inner{Tx}{\eta}_{H2} \] 
This condition then implies that $\eta \in D_{T^*}$ and $\xi = T^* \eta$ because,
\[ \inner{x}{\xi}_{H_1} = \inner{x}{T^* \eta}_{H_1} \]
which implies that $\xi = T^* \eta$ because $x$ can be chosen arbitrarily in a dense set. Therefore,
\[ F(G(T))^\perp = G(T^*) \implies T^* \text{ is closed} \]
To see that $D_{T^*}$ is dense, take $\eta \perp D_{T^*}$ and observe that $(\eta, 0) \perp G(T^*)$. Thus $(\eta, 0) \in G(T^*)^\perp = F(G(T))$ so $\exists x \in D_T$ such that,
\[ F(x, Tx) = (-Tx, x) = (\eta, 0) \implies \eta = 0 \]
Therefore, $D_{T^*} \subset H_2$ is dense. 
\end{proof}

\begin{proposition}
If $T$ is densely defined and closed then $T^{**} = T$. 
\end{proposition}

\begin{proof}
By above, $T^*$ is densely defined and closed and $\forall \eta \in D_{T^*}, x \i D_T$
\[ \inner{x}{T^* \eta} = \inner{Tx}{\eta} \]
For any given $x \in D_T$, consider the linear functional given by, $\eta \mapsto \inner{x}{T^* \eta}_{H_1} = \inner{Tx}{\eta}_{H_1}$ which is thus bounded and thus continuous. This $x \in D_{T^**}$ and $Tx = T^{**} x$. Because $T^{**}$ is densly defined and closed then $T^{**} = T$. 
\end{proof}

\begin{lemma}
Suppose $\alpha \in H_2$ satisfies,
\[ | \inner{\alpha}{\beta}_{H_2} |^2 \le C_0 || T^* \beta ||_{H_1}^2 \]
for any $\beta \in D_{T^*}$ there $\exists u \in D_T$ such that $T u = \alpha$ and $|| \alpha ||_{H_1} \le C_0$.
\end{lemma}

\begin{proof}
Consider a linear functional,
\[ L : R(T^*) \to \C \quad \quad L(T^* \beta) = \inner{\alpha}{\beta}_{H_2} \]
$L$ is well-defined since $\alpha \perp \ker{T^*}$. Then $L$ is bounded/continuous by the innequality. Extend $L : H_1 \to \C$ by defining $L(v) = 0$ for any $v \perp R(T^*)$. Now,
\[ |L(u)| \le \sqrt{C_0} || u ||_{H_1} \]
by Riesz representation there exists $u \in H_1$ with $|| u ||_{H-1} \le \sqrt{C_0}$ such that $L(v) = \inner{u}{v}_{H_1}$ for $v = T^* \beta$. Then 
\[ \inner{u}{T^*\beta}_{H_1} = \inner{\alpha}{\beta}_{H_2} \]
The innequality implies that $u \in D_{T^{**}} = D_T$. However,
\[ \inner{u}{T^* \beta}_{H_1} = \inner{Tu}{\beta}_{H_2} \]
which implies that $Tu = \alpha$ since $T$ is densely defined. 
\end{proof}


\section{April 16}

\begin{remark}
Recall that if $T : H_1 \to H_2$ is a densly defined and closed operator then $\alpha \in H_2$ is in the image if 
\[ | \inner{\alpha}{\beta}_{H_2} | \le C_0 || T^* \beta ||_{H_1} \]
for all $\beta \in D_{T^*}$ thus there exists $u \in D_T$ such that $Tu = \alpha$ and $|| u ||_{H_1} \le C_0$. Then we need,
\[ || \beta ||_{H_2} \le C_0 || T^* \beta ||{H_1} \]
for all $\beta \in D{T^*}$. 
\end{remark}

In particular, let $H_1 = L^2_{(p,q)}(\Omega, \varphi_1)$ and $H_2 = L^2_{(p, q+1)}(\Omega, \varphi_2)$ where $T = \bar{\partial} : H_1 \to H_2$ and $H_3 = L^2_{(p, q+2)}(\Omega, \varphi_3)$. Let $S = \bar{\partial} : H_2 \to H_3$ then $S \circ T = 0$. We hope to derive,
\[ || f||_{\varphi_2}^2 \le C_0 \left( ||T^* f ||_{\varphi_1}^2 + || Sf ||_{\varphi_3}^2 \right) \]
for any $f \in D_{T^*} \cap D_S$.
If this holds then for any $f$ with $S f = 0$ then we can solve the $\bar{\partial}$-equation $\bar{\partial} u = f$. 
\bigskip\\
Consider $\{ \Omega_\nu \}$ a compact exhaustion of $\Omega$ and define,
\[ \eta_\nu(z) = \begin{cases}
1 & z \in \Omega_\nu 
\\
0 & z \in \Omega \setminus \Omega_{\nu+1} 
\end{cases} \]
such that $\eta_\nu \in C_0^\infty(\Omega)$. 

\begin{lemma}
If $\varphi_1, \varphi_2, \varphi_3$ satisfy,
\[ e^{-\varphi_{j+1}} | \nabla \eta\nu |^2 \le e^{-\varphi_j} \]
for each $\nu$ and $j = 1,2$ then $D_{(p,q+1)}$ is dense in $L^2_{(p,q+1)}(\Omega)$ under the graphi norm,
\[ f \mapsto || f ||_{\varphi_2} + || T^* f ||_{\varphi_1} + || S f ||{\varphi_3} \]
\end{lemma}

\begin{proof}
Given $f \in D_S$ take,
\[ S(\eta_\nu f)  \eta_\nu S(f) = \bar{\partial} \eta_\nu \wedge f \]
thus,
\[ | S(\eta_\nu f) - \eta_\nu S(f) |^2 e^{-\varphi_3} \le e^{-\varphi_3} | \bar{\partial} \eta_\nu |^2 \cdot |f|^2 \le |f|^2 e^{-\varphi_2} \]
Then by dominated convergence theorem,
\[ || S(\eta_\nu f) - \eta_\nu S(f) ||_{\varphi_3} \to 0 \]
Now assume $f \in D_{T^*}$ first claim that $\eta_\nu f \in D_{T^*}$ because that for any $u \in D_T$ that $\inner{T u}{\eta_\nu f}_{\varphi_2}$ is bounded in the norm of $u$. By definition,
\begin{align*}
\inner{Tu}{\eta_\nu f}_{\varphi_2} & = \inner{\eta_\nu T u}{f}_{\varphi_2} + \inner{\eta_\nu T u - T(\eta_\nu u)}{f}_{\varphi_2} 
\\
& = \inner{\eta_\nu u}{T^* f}_{\varphi_1} \pm \inner{\bar{\partial} \eta_\nu \wedge u}{f}_{\varphi_2} 
\\
& = \inner{u}{\eta_\nu T^* f}_{\varphi_1} \pm \inner{\bar{\partial} \eta_\nu \wedge u}{f}_{\varphi_2} 
\end{align*}
Therefore, $\eta_\nu f \in D_{T^*}$. 
Therefore,
\[ T^*(\eta_\nu f) = \eta_\nu T^* f + \bar{\partial} \eta_\nu * f e^{-\varphi_2 + \varphi_1} \]
which implies that,
\[ | T^* (\eta_\nu f) - \eta_\nu T^* f |^2 e^{-\varphi_1} \le | \nabla \eta_\nu |^2 |f|^2 e^{-2 \varphi_2 + 2 \varphi_1} e^{-\varphi_1} \le e^{-\varphi_2} |f|^2 \]
Thus by dominated convergence theorem,
\[ || T^*(\eta_\nu f) - \eta_\nu T^8 f ||_{\varphi_1} \to 0 \]
Now assume that $f \in D_{T^*} \cap D_S$ and $\supp{f} \subset \subset \Omega$. Consider the standard regularization $\rho_\epsilon$. Define, $f_\epsilon = f * \rho_\epsilon$ then we have $f_\epsilon \to f$ as $\epsilon \to 0$ in the $L^2(\Omega)$ sense. Furthermore,
\[ S(f_\epsilon) = \bar{\partial} f_{\epsilon} = (S f) * \rho_\epsilon \]
which implies that $S f_\epsilon \to S f$ converges in $L^2(\Omega)$ space. For $T^*$, we know
\[ e^{-\varphi_1 + \varphi_2}  T^* =  D + a \]
where $D$ is the first-order operator with constan coefficients. Then,
\begin{align*}
T^* f_\epsilon & = e^{\varphi_1 - \varphi_2} (D + a) f_\epsilon 
\\
& = e^{\varphi_1 - \varphi2} (D f) * \rho_\epsilon + e^{\varphi_1 - \varphi_2} a f_\epsilon 
\\
& = e^{\varphi_1 - \varphi_2} ((D + a) f) * \rho_\epsilon + e^{\varphi_1 - \varphi_2} a f_\epsilon - e^{\varphi_1 - \varphi_2} (af) * f_\epsilon  
\end{align*} 
The second term vanishes in the limt $\epsilon \to 0$ because $a f_\epsilon = a (f * \rho_\epsilon)$. 
The first term is approximatly,
\[ (e^{\varphi_1 -\varphi_2} (D + a) f) * \rho_\epsilon = (T^* f) * \rho_\epsilon \to T^* f \]
Now we will calculate $T^*$ in local coordinates expanding,
\[ f = \sum'_{\substack{|I| = p \\ |J| = q + 1}} f_{I\bar{J}} \d{z_I} \wedge \d{\overline{z}_J} \]
We also expand,
\[ u = \sum'_{\substack{|I| = p \\ |J| = q}} u_{I\bar{J}} \d{z_I} \wedge \d{\overline{z}_J} \]
for some element $u \in D_{(p,q)}$. Then the derivative becomes,
\[ \bar{\partial} u = \sum'_{I,J} \sum_{j = 1}^n \pderiv{u}{z_j} \d{\overline{z}_j} \wedge \d{z_I} \wedge \d{\overline{z}_J} \]
and the inner product is,
\[ \inner{T^* f}{u}_{\varphi_1} = \int \sum_{I,J}' (T^* f)_{I \bar{J}} \overline{u_{I \bar{J}}} e^{-\varphi_1} \]
However,
\begin{align*}
\inner{T^* f}{u}_{\varphi_1} & = \inner{f}{Tu} = \int \sum'_{\substack{|I| = p \\ |J| = q}} (-1)^p \sum_{j = 1}^n f_{I, \overline{jJ}} \overline{\pderiv{u}{\overline{z}_j}} e^{-\varphi_1}
\\
(-1)^p \int \sum'_{\substack{|I| = p \\ |J| = q}} (-1)^p \pderiv{}{z_j} \left( f_{I, \overline{jJ}} e^{-\varphi_1} \right) \overline{u_{I \overline{J}}} 
\end{align*}
Therefore, comparing coefficients, we find,
\[ T^* f = (-1)^{p+1} \sum'_{I,J} \pderiv{}{z_j} \left( f_{I, \overline{j J}} e^{-\varphi_2} \right) e^{\varphi_1} \]
\end{proof}

\begin{remark}
Our goal is, $|| f ||_{\varphi_2} \le C \left( || T^* f ||_{\varphi_1} + || S f ||_{\varphi_3} \right)$ for some suitable $\varphi_1, \varphi_2, \varphi_3$.
\end{remark}

Fix a smooth $\psi \in C^\infty(\Omega)$ such that,
$| \nabla \eta_\nu |^2 \le e^\psi$ on $\Omega$ for each $\nu$. Then for $\varphi \in C^2(\Omega)$ define,
\begin{align*}
\varphi_1 & = \varphi - 2 \psi
\\
\varphi_2 & = \varphi - \psi
\\
\varphi_3 & = \varphi
\end{align*}
For $f \in D_{(p, q+1)}(\Omega)$ we have $Sf = \bar{\partial} f$ where,
\[ \bar{\partial} f = \sum'_{\substack{|I| = p \\ |J| = q+1}} \sum_{j = 1}^n \pderiv{f_{I \bar{J}}}{\overline{z}_j} \d{\overline{z}_j} \wedge \d{z_I} \wedge \d{\overline{z}_J} \]
We need to antisymmetrize this expression after which we see,
\[ | \bar{\partial} f |^2 = \sum'_{\substack{|I| = p \\ |J| = |L| = q+1}} \sum_{j,l = 1}^n \pderiv{f_{I \bar{J}}}{\overline{z}_j} \pderiv{f_{I\bar{L}}}{\overline{z}_\ell} \epsilon^{j J}_{\ell L} \]
Rearanging,
\[ | \bar{\partial} f |^2 = \sum'_{\substack{|I| = p \\ |J| = q + 1}} \sum_{j = 1}^n \left| \pderiv{f_{I \bar{J}}}{\overline{z}_j} \right|^2  \sum'_{\substack{|I| = p \\ |K| = q}} \sum_{j \neq \ell} \pderiv{f_{I, \overline{\ell K}}}{\overline{z}_j} \overline{\pderiv{f_{I, \overline{jK}}}{\overline{z}_{\ell}}} \]
Furthermore, we can expand,
\begin{align*}
T^* f & = (-1)^{p+1} \sum'_{\substack{|I| = p \\ |K| = q+1}} \sum_{j = 1}^n \pderiv{}{z_j} \left( f_{I, \overline{jK}}  e^{-\varphi_2} \right) e^{\varphi_1} 
\\
& = (-1)^{p+1} \sum'_{\substack{|I| = p \\ |K| = q+1}} \sum_{j = 1}^n \pderiv{}{z_j} \left( \delta_j(f_{I, \overline{jK}})  e^{-\psi} + f_{I, \overline{jK}} \partial_j \psi e^{- \psi} \right) 
\end{align*}
where,
\[ \delta_j u = e^\varphi \pderiv{}{z_j} (u e^{-\varphi})  \]
and
\[ 
\inner{\delta_j u}{v}_{\varphi} = - \inner{u}{\partial v}_\varphi \]
By the formulas,
\begin{align*}
\int & \left( \sum_{I,J}' \sum_{j,k = 1} \delta_j f_{I, \overline{jK}} \overline{\delta_k f_{I, \overline{kK}}} - \partial_{\bar{k}} f_{I, \overline{jK}} \overline{\partial_{\bar{j}} f_{I, \overline{kK}}} \right) e^{-\varphi} 
+ \int \sum_{I,J}' \sum^n_{j = 1} \left| \pderiv{f_{I,\bar{J}}}{\overline{z}_j} \right|^2 e^{-\varphi} 
\\
& \le 2 || T^* f||^2_{\varphi_1} + || S f ||^2_{\varphi_3} + 2 \int |f|^2 |\nabla \psi|^2 e^{-\varphi} 
\end{align*}

\section{April 18}

If $\partial^2_{i \bar{k}} \varphi > C(z) \omega_{\C^n}$ where $C > 0$ then,
\[ \int_\Omega (c  - 2 |\partial \psi|^2 )|f|^2 e^{-\varphi} + \int_{\Omega} \sum_{IJ}' \sum_j \left| \pderiv{f_{I \bar{J}}}{\bar{z}_j} \right|^2 \le 2 || T^* f ||^2_{\varphi_1} + ||S f ||^2_{\varphi_3} \]

\begin{lemma}
If $C \ge 2 |\partial \psi|^2 + e^{\psi}$ then,
\[ \int_\Omega |f|^2 e^{-\varphi + \psi} \le 2 ||T^* f ||^2_{\varphi_1} + || Sf ||_{\varphi_3}^2 \]
and therefore,
\[ || f ||_{\varphi_2}^2 \le C \left( || T^* f ||^2_{\varphi_1} + || S f ||_{\varphi_3}^2 \right) \]
\end{lemma}

\begin{theorem}
Let $\Omega \subset \C^n$ be a strongly pseudo-convex domain (i.e. there exists $\rho$ strictly PSH s.t. $\Omega = \{ \rho < 0 \}$ and $\rho$ is not degenerate on $\partial \Omega$). Then for any $f \in L_{(p, q + 1)}^2(\Omega, \text{loc})$ with $\bar{\partial} f = 0$ then  there exists $u \in L^2_{(p,q)}(\Omega, \text{loc})$ such that,
\[ \bar{\partial} u = f \] 
\end{theorem}

\begin{proof}

\end{proof}

\begin{remark}
What is the regularity of $u$. The space of possible solutions is the coset $u + \ker{T}$ so if we choose the solution $u$ such that $u \perp \ker{T}$ in $J_1$ then this $u \in \Im{T^*}$ because $(\ker{T})^\perp = \overline{\Im{T^*}}$. This gives an additional constraint on $u$, namely, $T^* u = 0$. 
\end{remark}

\begin{definition}
$W^{k,2}(\Omega) = \{ u \in L^2 \mid \forall \ell \in \Zplus : D^\ell u \in L^2 \}$
\end{definition}

\begin{remark}
Our goal is, if $\bar{\partial} u = f$ and $\bar{\partial} f = 0$ for $f \in W^{k,2}_{(p, q+1)}(\Omega, \text{loc})$ then $\exists u \in W^{k+1,2}_{(p,q)}(\Omega, \text{loc})$. 
\end{remark}

\begin{definition}
If $f \in W^{k+1,2}_{(p,q)}(\Omega, \text{loc})$ the define,
\[ \theta_i f = \sum'_{IJ} \sum_{j = 1}^n \pderiv{f_{I, \overline{jJ}}}{z_j} \d{z_I} \wedge \overline{\d{z_{\bar{j}}}} \]
\end{definition}

\begin{lemma}
If $f \in L^2_{(p,q+1)}(\Omega)$ and $\supp{f} \subset \subset \Omega$ and $\bar{\partial} f \in L^2_{(p,q+2)}(\Omega)$ and $\theta f \in L^2_{(p,q)}(\Omega)$ then $f \in W^{1,2}_{(p,q)}(\Omega)$.
\end{lemma}

\begin{proof}
First, $f \in D_{(p, q+1)}(\Omega)$ and take $\varphi = \psi = 0$ then our previous computation gives,
\[ \int \sum_{IJ} \sum^n_{j = 1} \left| \pderiv{f_{I \bar{J}}}{\overline{z_j}} \right|^2 \d{V} \le || T^* f ||^2 + || S f ||^2 \]
If $f$ is not smooth then take the convolution $f_\epsilon = f * \rho_\epsilon$. Then we have,
\begin{align*}
\bar{\partial} f_\epsilon = (\bar{\partial} f) * \rho_\epsilon \xrightarrow{L^2} & \bar{\partial} f 
\\
\theta f_\epsilon = (\theta f) * \rho_\epsilon \xrightarrow{L^2} & \theta f 
\end{align*}
Now apply (*) to show,
\[ \left|\left| \pderiv{f_{\epsilon, I \bar{j}}}{\overline{z_j}} - \pderiv{f_{\epsilon, I \bar{J}}}{\overline{z_j}} \right| \right|_{L^2} \xrightarrow{L^2} 0 \]
Thus,
\[ \pderiv{f_{\epsilon, I \bar{J}}}{\overline{z_j}} \xrightarrow{L^2} \pderiv{f_{I\bar{J}}}{\overline{z_j}} \in L^2 \]
therefore,
\[ \pderiv{f_{I \bar{J}}}{z_j} \in L^2 \]
\end{proof}

\begin{remark}
Now we can prove the required regularity.
\end{remark}

\begin{proof}
First, for $q = 0$ consider $\bar{\partial} u = 0$ where $\bar{\partial} f = 0$. If we write,
\[ u = \sum' u_I \d{z_I} \]
then this equation is equivalent to,
\[ \pderiv{u_I}{\overline{z_j}} = f_{I \bar{J}} \in L^2(\Omega, \text{loc}) \cap W^{k,2}(\Omega, \text{loc}) \]
Now choose a cutoff function $\chi \in C^\infty_0(\Omega)$. Then,
\[ \pderiv{}{\overline{z_j}}(u_I \chi) = f_{I \bar{j}} + u_I \pderiv{\chi}{\overline{z_j}} \]
Assue for induction that $u_I \in W^{\ell, 2}(\Omega, \text{loc})$ for some $0 \le \ell \le k$ then I claim that $u_i \in W^{\ell + 1, 2}(\Omega)$. Differentiating the above equation,
\[ \pderiv{}{\overline{z_j}} (D^\ell (u_I, \chi)) = D^\ell (f_{I \bar{J}} + u_I \deriv{\chi}{\overline{z}_j} \]
\end{proof}


\section{April 23}

\begin{proposition}
Let $\Omega \subset \subset \C^n$ be pseudoconvex and $\varphi \in C^2(\Omega)$ such that $i \partial \bar{\partial} \varphi \ge c \omega_{\C^n}$ in $\Omega$ where $c : \Omega \to \R_+$. If $g \in L^2_{(p, q+1)}(\Omega, \varphi)$ and $\bar{\partial} g = 0$ then there exists $u \in L^2_{(p,q)}(\Omega, \varphi)$ such that $\bar{\partial} u = g$. Moreover,
\[ \int_\Omega |u|^2 e^{-\varphi} \d{V} \le K \int_\Omega \frac{|g|^2}{c} e^{-\varphi} \d{V} \]
provided the RHS is finite with a uniform constant $K$.
\end{proposition}

\begin{proof}
$\Omega$ is pseudo-convex means that there exists $p \in \PSH{\Omega} \cap C^\infty(\Omega)$ and $a \in \R$ s.t. $\Omega_a = \{ x \in \Omega \mid p(x) < a \} \subset \subset \Omega$. Fix an $a$ and $\eta_\nu$ to be $1$ on $\Omega_{a + 1}$ if $\nu \ge \ge 1$ we can find $\psi \ge 0$ such that, $\phi = 0$ on $\Omega_{a+1}$ and $|\nabla \eta_\nu|^2 \le e^\psi$ on $\Omega$. 
\bigskip\\
Let $\tilde{\phi} = \varphi + \chi \cdot p$ where $\chi$ is convex and increasing very fast. Then we may further write,
\begin{align*}
\tilde{\varphi} - 2 \psi & = \varphi + \chi(p) - 2 \psi \ge \varphi 
\\
i \partial \bar{\partial} \chi(p) & = \chi'(p) i \partial \bar{\partial} p + \chi'' i \partial p \wedge \bar{\partial} p 
\\
& \ge \chi'(p) i \partial \bar{\partial} p \ge 2 | \nabla \psi |^2 \omega_{\C^n} 
\end{align*}
Define,
\begin{align*}
\varphi_1 & = \tilde{\varphi} - 2 \psi
\\
\varphi_2 & = \tilde{\varphi} - \psi
\\
\varphi_3 & = \tilde{\varphi}
\end{align*}
Then by a previous calculation,
\[ \int_\Omega c |f|^2 e^{-\tilde{\varphi}} \d{V} \le 2 || T^* f ||^2_{\varphi_1} + || S f ||^2_{\varphi_3} \]
$\forall f \in D_{(p, q+1)}(\Omega)$ and hence $f \in D{T^*} \cap D_S$. By the Holder innequality,
\begin{align*}
\left| \inner{g}{f} \right|^2 = \left| \int_\Omega g \bar{f} e^{-\varphi_2} \d{V} \right|^2 \le \left( \int_\Omega \frac{|g|^2}{c} e^{-\varphi} \right) \left( \int_\Omega c | f |^2 e^{\varphi - 2 \varphi_2} \right) 
\end{align*}
Define,
\[ A = \int_\Omega \frac{|g|^2}{c} e^{-\varphi}  \]
Then we find,
\[ \left| \inner{g}{f} \right|^2 \le A \int_\Omega c |f|^2 e^{\varphi - 2 \varphi_2} \le 2 A \left( || T^* f ||_{\varphi_1} + || S f ||^2_{\varphi_3} \right) \]
Now I claim that,
\[ \left| \inner{g}{f}_{\varphi_2} \right|^2 \le 2 A || T^* f ||_{\varphi_1}^2 \]
for all $f \in D_{T^*}$. If $f \in \ker{S}$ then $S f = 0$ if $f \perp \ker{S}$ then $\Im{T} \subset \ker{S} \implies f \perp \Im{T}$ and thus $T^* f = 0$ since $(\Im{T^*})^\perp = \ker{T^*}$. Thus $S \circ T = 0$. We then need to check that for $g \in L^2(\Omega, e^{-\varphi_2} = H_2$ then,
\[ \int |g|^2 e^{-\varphi_2} \le \int |g|^2 e^{-\varphi} < \infty \]
By functional analysis lemma, we need to find $u = u_a \in L^2_{(p,q)}(\Omega, e^{-\varphi_1}) = H_1$ s.t $\bar{\partial} u_a = g$ and 
\[ || u_a ||^2_{H_1} \le 2 A \] 
\end{proof}

\begin{theorem}
Let $\Omega \subset \subset \C^n$ be a pseudo-convex domain with $p \in \PSH{\Omega}$ (not necessarily smooth). Let $g \in L^2_{(p, q+1)}(\Omega, e^{-\varphi})$ and $\bar{\partial} g = 0$ then $\exists u \in L^2_{(p,q)}(\Omega, e^{-\varphi})$ s.t. $\bar{\partial} u = g$ and,
\[ \int_\Omega |u|^2 \frac{e^{-\varphi}}{(1 + |z|^2)^2} \d{V} \le \int_\Omega |g|^2 e^{-\varphi} \d{V} \]
\end{theorem}

\begin{proof}
First, assume that $\varphi \in C^2(\Omega) \cap \PSH{\Omega}$ and define $\tilde{\varphi} = \varphi + 2 \log{(1 + |z|^2)}$. Then,
\[ i \partial \bar{\partial} \tilde{\varphi} \ge 2 \left( \frac{\delta_{ij}}{1 + |z|^2} - \frac{\bar{z}_i z_j}{(1 + |z|^2)^2} \right) i \d{z_i} \wedge \d{\overline{z}_j} \]
Then we may take,
\[ c = \frac{1}{(1 + |z|^2)^2} \]
By the previous proposition with $\tilde{\varphi}$,
\[ \int_\Omega |u|^2 e^{-\tilde{\varphi}} \le \int_\Omega |g|^2 (1 + |z|^2)^2 e^{-\varphi - 2 \log{(1 + |z|^2)}} = \int_\Omega |g|^2 e^{-\varphi} \]
Now for the general case if $\varphi$ is not $C^2$. For a general $\varphi \in \PSH{\Omega}$ let $p$ be strictly PSH and,
\[ \Omega_a = \{ x \in \Omega \mid p(x) < a \} \subset \subset \Omega \]
Take a regularization of $\varphi_\epsilon$ of $\varphi$ on $\Omega_{a(\epsilon)}$ with $a(\epsilon) \to \infty$ as $\epsilon \to 0$ s.t. $\varphi_\epsilon \downarrow \varphi$ as $\epsilon \to 0$. We have $\varphi_\epsilon \in C^\infty(\Omega_{a(\epsilon)}) \cap \PSH{\Omega_{a(\epsilon)}}$. Furthermore, $\Omega_{a(\epsilon)}$ is pseudo-convex and thus we may apply the previous case. Then there exists $u_\epsilon \in L^2(\Omega_{a(\epsilon)}, \text{loc})$ s.t. $\bar{\partial} u_\epsilon = g$ in $\Omega_{a(\epsilon)}$ and
\[ \int_\Omega | u_\epsilon |^2 \frac{e^{-\varphi_\epsilon}}{(1 + |z|^2)^2} \le \int_{\Omega_{a(\epsilon)}} |g|^2 e^{-\varphi_\epsilon} \le \int_\Omega |g|^2 e^{-\varphi} < \infty \]
Now on any compact $K \subset \subset \Omega$ we have,
\[ || u_\epsilon ||_{L^2(K)} \le C \]
Now as $\epsilon_j \to 0$ we have $u_{\epsilon_j} \to u$ in $L^2(\Omega, \text{loc})$. This implies that $\bar{\partial} u = g$ in $\Omega$. Then $\Omega_a$ for any fixed $a$ and $\epsilon'$ we have,
\[ \int_{\Omega_a} |u|^2 \frac{e^{-\varphi_{\epsilon'}}}{(1 + |z|^2)^2} \le \liminf_{\epsilon_j \to 0} \int_{\Omega_{a(\epsilon_j)}} |u_{\epsilon_j} |^2 \frac{e^{-\varphi_{\epsilon'}}}{(1 + |z|^2)^2} \d{V} \le \int_\Omega |g|^2 e^{-\varphi} \]
Now let $\epsilon' \to 0$ and $a \to \infty$ such that $\Omega_a \uparrow \Omega$ which gives the result by the monotone convergence theorem. 
\end{proof}

\section{April 26}

\begin{theorem}[Donnelly-Fetterman]
Assume $\psi \in \PSH{\Omega}$ such that 
\[ i \partial \psi \wedge \bar{\partial} \psi \le i \partial \partial \psi \]
in $\Omega$. Then $\exists u \in L^2(\Omega, \text{loc})$ with $\bar{\partial} u = \alpha$ and,
\[ \int_\Omega |u|^2 e^{-\varphi} \le C_1 \int_\Omega |\alpha|^2_{i \partial \bar{\partial} \psi} e^{-\varphi} \d{V} \]
\end{theorem}

\begin{remark}
Suppose $v < 0$ for $v \in \PSH{\Omega}$ with $\psi = - \log{(-v)}$ satisfies the hypothesis of the theorem.
\end{remark}

\begin{proof}
By regularization we may assume that all functions are smooth and $\Omega$ has smooth boundary. Let $u$ be a minimal solutio to $\bar{\partial} u = \alpha$ in $L^2(\Omega, e^{-(\varphi + \psi / 2)})$. This implies that $u \perp \ker{\bar{\partial}}$ i.e. $\forall f \in L^2(\Omega, e^{-(\varphi + \psi / 2)})$ with $\bar{\partial} f = 0$ (i.e. $f$ is holomorphic) we have,
\[ \int_\Omega u \bar{f} e^{-(\varphi + \psi / 2)} \d{V} = 0 \]
We may write this as,
\[ \int_\Omega (u e^{\psi/2}) \bar{f} e^{-(\varphi + \psi)} \d{V} = 0 \]
and thus $v = (u e^{\psi / 2} \perp \ker{\bar{\partial}}$ in $L^2(\Omega, e^{-(\varphi + \psi)})$. Thus $v$ is a minimal solution to $\bar{\partial} v = \bar{\partial} u e^{\psi/2} + \tfrac{1}{2} e^{\psi / 2} u \bar{\partial} \psi = \beta$ in $L^2(\Omega, e^{-(\varphi + \psi)})$. However, by H\"{o}rmander's estimate, (using the fact that $v$ is minimal and thus less than anyy solution given by the estimate),
\[ \int_\Omega |v|^2 e^{-(\varphi + \psi)} \le C_0 \int_\Omega |\beta|^2_{i \partial \bar{\partial} (\varphi + \psi)} e^{-(\varphi + \psi)} \d{V} \]
However, $|v|^2 = |u|^2 e^{\psi}$ and thus,
\[ \int_\Omega |v|^2 e^{-(\varphi + \psi)} = \int_\Omega |u|^2 e^{- \varphi} \]
Furthermore, $\beta = (\alpha + \tfrac{1}{2} u \bar{\partial} \psi)^2 e^{\psi}$ and thus,
\[ \int_\Omega |\beta|^2_{i \partial \bar{\partial} (\varphi + \psi)} e^{-(\varphi + \psi)} \d{V} \le \int_\Omega |\alpha + \tfrac{1}{2} u \bar{\partial} \psi|^2_{i \partial \bar{\partial} \psi} e^{-\varphi} \d{V} \]
Using the fact that $i \partial \bar{\partial} (\varphi + \psi) \ge i \partial \bar{\partial} \psi$. 
Consider, 

\end{proof}

\begin{theorem}[Berndtsson]
Suppose that 
\[ |\bar{\partial} |^2_{i \partial \bar{\partial} \psi} < a < 1 \]
in the support of $\alpha$. Then there exists $u \in L^2(\Omega, \text{loc})$ such that $\bar{\partial} u = \alpha$ such that,
\[ \int_\Omega \left( 1 - | \bar{\partial} \psi |^2_{i \partial \bar{\partial} \psi} \right) |u|^2 e^{\psi - \varphi} \le \frac{1 + \sqrt{a}}{1 - \sqrt{a}} \int_\Omega |\alpha|^2_{i \partial \bar{\partial} \psi} e^{\psi - \varphi} \]
\end{theorem}

\begin{proof}
Assume that all the functions are smooth by regularization. Let $u \in L^2(\Omega, e^{-\varphi})$ be a minimal solution to $\bar{\partial} u = \alpha$ i.e. $u \perp \ker{\bar{\partial}}$ i.e.
\[ \int_\Omega u \bar{f} e^{-\varphi} \d{V} = 0 \]
for each holomorphic $f$. However, 
\[ \int_\Omega u \bar{f} e^{-\varphi} \d{V} = \int (u e^\psi) \bar{f} e^{(-\varphi + \psi)} \]
So $v = u e^\psi$ s the minimal solution to,
\[ \bar{\partial} v = \bar{\partial} u e^\psi + u e^{\psi} \bar{\partial} \psi = (\alpha + \bar{\partial} \psi) e^\psi \]
H\"{o}rmander's estimate then gives,

\end{proof}

\begin{theorem}[Ohsawa-Takegoshi Extension]
Let $\Omega \subset \C^n$ is a pseudo-convext domain, $H  \subset \C^n$ an affine subspace, and $\Omega' = \Omega \cap H$. Suppose $\varphi \in \PSH{\Omega}$ and $f \in \mathcal{O}(\Omega') \cap L^2(\Omega, e^{-\varphi})$ then there exists $F \in \mathcal{O}(\Omega)$ s.t. $F |_{\Omega'} = f$ and,
\[ \int_\Omega |F|^2 e^{-\varphi} \d{V} \le C(n, \Omega) \]
\end{theorem}

\begin{remark}
We will prove a slightly stronger theorem.
\end{remark}

\begin{theorem}
Let $\Omega \subset \C^n$ be a pseudo-convex domain, $H = \{z_n = 0 \}$ and $\Omega' = H \cap \Omega$ and $\varphi \in \PSH{\Omega}$ and $f \in \mathcal{O}(\Omega')$ then $\exists F \in \mathcal{O}(\Omega)$ s.t. $F |_{\Omega'} = f$ and,
\[ \int_\Omega \frac{|F|^2 e^{-\varphi}}{|z_n|^2 (\log{|z_n|^2})^2} \d{V} \le C(n) \int_{\Omega'} |f|^2 e^{-\varphi} \d{V} \]
\end{theorem}

\begin{lemma}[Chen]
Take $\xi \in \C$ and sufficiently small $\epsilon > 0$ then,
\[ \psi(\xi) = - \log{\left[ - \log{( |\xi|^2 + \epsilon^2)} + \log{\left( - \log{(|\xi^2 + \epsilon^2)} \right)} \right]} \]
satisfies on $|\xi| \le (2 e)^{-\frac{1}{2}}$,
\begin{enumerate}
\item 
\[ \left( 1 - \frac{|\psi_\xi|^2}{\psi_{\xi \bar{\xi}}} \right) e^{\psi} \ge \frac{1}{C_1 \log{( |\xi|^2 + \epsilon^2)}} \]
\item
\[ \frac{|\psi_\xi|^2}{\psi_{\xi \bar{\xi}}} \le - \frac{C_2}{\log{\epsilon}} \quad \quad \text{on} \{ \xi \in \C \mid |\xi| \le \epsilon \} \]
\item
\[ \frac{e^\psi}{|\zeta|^2 \psi_{\xi \xi}} \le C_3 \quad \quad \text{on} \{ \xi \in \C \mid \epsilon / 2 \le |\xi| \le \epsilon \} \]
\end{enumerate}
\end{lemma}


\end{document}


