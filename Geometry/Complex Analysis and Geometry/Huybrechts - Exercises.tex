\documentclass[12pt]{article}
\usepackage{hyperref}
\hypersetup{
    colorlinks=true,
    linkcolor=blue,
    filecolor=magenta,      
    urlcolor=cyan,
}
 
\urlstyle{same}
\usepackage{import}
\import{../}{GeometryCommands}


\begin{document}

\tableofcontents

\section{Chapter 1}


\section{Chapter 2}

\subsection{Section 2.1}

\subsubsection{2.2.2}

Given an exact sequence of vector bundles,
\begin{center}
\begin{tikzcd}
0 \arrow[r] & L \arrow[r, "\varphi"] & E \arrow[r, "\psi"] & F \arrow[r] & 0
\end{tikzcd}
\end{center}
where $L$ is a line bundle. Consider the exact sequence,
\begin{center}
\begin{tikzcd}
0 \arrow[r] & L \otimes \bigwedge^{i-1} F \arrow[r] & \bigwedge^i E \arrow[r] & \bigwedge^i F \arrow[r] & 0
\end{tikzcd}
\end{center}
where the first map is defined by taking sections $s_1 \otimes (f_2 \wedge \cdots \wedge f_i)$ and lifing each $f_i$ to a section $e_j$ of $E$ up to a section $s_j$ of $L$ and then map $s_1 \otimes (f_2 \wedge \cdots \wedge f_i) \mapsto s_1 \wedge e_2 \wedge \cdots \wedge e_i$. This map is well-defined because,
\begin{align*}
s_1 \wedge e'_2 \wedge \cdots \wedge e'_{i} & = s_1 \wedge (e_2 + \varphi(s_2)) \wedge \cdots \wedge (e_{i} + \varphi(s_{i})) = s_1 \wedge e_2 \wedge \cdots \wedge e_n + s_1 \wedge s_2 \wedge \cdots \wedge e_n + \cdots 
\\
& = s_1 \wedge e_2 \wedge \cdots \wedge e_n
\end{align*} 
because $s_1 \wedge s_j = 0$ since $L$ is a line bundle. Thus this map is well-defined and clearly $\ker{\wedge^i \psi}$ is the image of this map because $L \subset E$ maps to zero under $\psi$ thus the kernel is exterior products where one factor is in $L$. Furthermore, dualizing if we have an exact sequence of vector bundles,
\begin{center}
\begin{tikzcd}
0 \arrow[r] & F \arrow[r] & E \arrow[r] & L \arrow[r] & 0
\end{tikzcd}
\end{center}
where $L$ is a line bundle then there is an exact sequence,
\begin{center}
\begin{tikzcd}
0 \arrow[r] & \bigwedge^i F \arrow[r] & \bigwedge^i E \arrow[r] & L \otimes \bigwedge^{i-1} F \arrow[r] & 0
\end{tikzcd}
\end{center}

\subsubsection{2.2.3}

Let $E$ be a holomorphic vector bundle $E$ of rank $r$ there exists a non-degenerate pairing,
\[ \bigwedge^k E \times \bigwedge^{r-k} E \to \det{E} \]
via $(e_1 \wedge \cdots \wedge e_k, e_{k+1} \wedge \cdots \wedge e_{r}) \mapsto e_1 \wedge \cdots \wedge e_r$. Locally $E \cong \struct{X}^{\oplus r}$ and thus the pairing is nondegenerate because we can take $e_i$ to be the standard basis of $\struct{X}^{\oplus r}$. Then $\det{E} \cong \struct{X}$ and $(e_1 \wedge \cdots \wedge e_k, e_{k+1} \wedge \cdots \wedge e_{r}) \mapsto e_1 \wedge \cdots \wedge e_r$ is a generator of $\det{E} \cong \struct{X}$.
\bigskip\\
Since the above pairing is nondegenerate, we get an isomorphism $\bigwedge^k E \iso \bigwedge^{r-k} E^* \otimes \det{E}$.

\subsubsection{2.2.5}


Let $L, L^*$ be holomorphic line bundles on a compact complex manifold $X$. Suppose that $L$ and $L^*$ admit nonzero global holomorphic sections $s, s'$. Then consider $s \otimes s'$ a global section of $L \otimes L^* \cong \struct{X}$. However, all nonzero sections of $\struct{X}$ are nonvanishing because $X$ is compact and thus $H^0(X, \struct{X}) = \C$. Therefore, $s$ and $s'$ are nonvanishing meaning that $L \cong L^* \cong \struct{X}$.

\subsubsection{2.2.6 DO!!}

\subsection{Section 2.6}

\subsubsection{2.6.1}

I think $f$ is holomorphic iff $\d{f}(Iv) = i \d{f} (v)$

\subsubsection{2.6.2 DO!!}

\subsubsection{2.6.3 DO!!}

\subsubsection{2.6.4 CHECK!!}

Let $f : X \to Y$ be a surjective holomorphic map between connected xomplex manifolds. We want to look at the smooth locus of $f$.

I claim the following is true: for a morphism of vector budles (not necessarily constant rank) $\phi : \E_1 \to \E_2$ then $\phi$ has full rank $k = \min{\{m, n \}}$ iff the morphism $\phi' : \bigwedge^k \E_1 \to \bigwedge^k \E_2$ is nonzero (is this true).
\bigskip\\
Therefore, the locus where $\phi$ is not full rank is the vanishing the section
\[ \phi' \in \mathcal{HOM}_{\struct{X}}\left(\bigwedge^k \E_, \bigwedge^k \E_2\right) \]
Now apply this to the map $f^* \Omega_Y \to \Omega_X$ to get the nonsmooth 
locus.

\subsubsection{2.6.5 CHECK!!}

The cousins' problem has a solution because $H^1(X, \struct{X}) = 0$. Question: why is every hypersurface defined by a $H^0(K^\times / \struct{X}^\times)$. Question: how are we supposed to use the poincare lemma.

\subsubsection{2.6.5 DO!!}

\subsubsection{2.6.7}

\renewcommand{\A}{\mathcal{A}}

We define,
\[ H^{p,q}_{\mathrm{BC}}(X) = \frac{\{ \alpha \in \A^{p,q}(X) \mid \d{\alpha} = 0 \}}{ \partial \bar{\partial} \A^{p-1,q-1}(X)} \] 
This makes sense because if $\alpha = \partial \bar{\partial} \gamma$ then \[ \d{\alpha} = \partial^2 \bar{\partial} \gamma - \bar{\partial}^2 \partial \gamma = 0 \]
Now, the inclusion of $\d{}$-closed forms into $\bar{\partial}$-closed forms induces a map,
\[ H^{p,q}_{\mathrm{BC}}(X) \to H^{p,q}(X) \]
which is well-defined because if $\alpha = \partial \bar{\partial} \gamma$ then $\alpha = - \bar{\partial} \partial \gamma$ and is thus $\bar{\partial}$-exact. If $X$ is furthermore compact Kahler then by the $\partial \bar{\partial}$-lemma we see if $\alpha$ maps to zero i.e. $\alpha = \partial \bar{\partial} \beta$ and $\d{\alpha} = 0$ then $\alpha = \d{\gamma}$ so the map is injective. Furthermore, by the Hodge decomposition, $H^{p,q}(X)$ can be represented by Harmonic forms which are $\d$-closed and thus this map is surjective as well.

\subsubsection{2.6.8 ASK RON!!}

Is this just because we can take $M \to M$ via complex conjugation.

\subsubsection{2.6.9 DO!!}

\subsubsection{2.6.10 DO!!}

\subsubsection{2.6.11 ASK RON!!}

\section{Chapter 3}

\subsection{Section 3.1}

\subsubsection{3.1.1 DO!!}

Let $X$ be a complex manifold with an almost complex structure $(M, I)$. We need to find a Riemannian structure $g$ on $M$ such that $g$ is compatible with $I$ meaning $g(I-,I-) = g(-,-)$.
(FINISH)

\subsubsection{3.1.2}

Let $X$ be a connected complex manifold of dimension $n > 1$ and let $g$ be a \kahler metric. Suppose that $g' = e^f \cdot g$ for some real smooth function $f \in \A^0(X)$ is also a \kahler metric. Then the associated \kahler forms satisfy $\omega' = e^f \cdot \omega$. Since both are \kahler forms, we must have $\d{\omega'} = 0$  and $\d{\omega} = 0$. However, 
\[ \d{\omega'} = e^f \d{f} \wedge \omega + e^f \d{\omega} = e^f \d{f} \wedge \omega \]
Therefore, $\d{f} \wedge \omega = 0$ and thus $L(\d{f}) = 0$. However, since $n > 1$ the Lefschetz operator is injective on $k$-forms for $k < n$ and thus $\d{f} = 0$. Since $X$ is connected then $f$ is constant so $\omega' = c \omega$.

\subsubsection{3.1.3 DO!!}

\subsubsection{3.1.4 DO!!}

\subsubsection{3.1.5 DO!!}


\subsubsection{3.1.7 DO!!}


\subsubsection{3.1.8 DO!!}

\subsubsection{3.1.12 DO!!}

\subsubsection{3.1.13 DO!!}

\subsection{Section 3.2}

\subsubsection{3.2.1} 
 
\subsubsection{3.2.2}

\subsubsection{3.2.4}
What does this really mean?? Ask Ron.

\subsubsection{3.2.5 DO!!}

\subsubsection{3.2.6}

Let $X$ be a compact \kahler manifold. Then,
\[ H^{n}(X, \C) = \bigoplus_{p + q = n} H^{p,q}(X) \]
Furthermore, $H^{q,p} = \overline{H^{p,q}}$. Therefore,
\[ b_{2k + 1} = \sum_{p + q = 2k + 1} h^{p, q} = \sum_{i = 0}^k (h^{2 k + 1 - i, i} + h^{i, 2k + 1 - i}) = 2 \sum_{i = 0}^k h^{2k + 1 - i, i} \]
is even.

\subsubsection{3.2.7 DO!!}

No! (PROVE IT)

\subsubsection{3.2.8}

Let $X$ be a compact \kahler manifold. Let $\omega \in H^0(X, \Omega^p_X)$. Clearly, $\bar{\partial} \omega = 0$ since $\omega$ is a holomorphic $(p, 0)$-form. Furthermore, 
\[ \bar{\partial}^* \omega = - (\bar{\star} \circ \bar{\partial} \circ \bar{\star}) \, \omega \]
but $\bar{\star} \omega$ is a $(n - p, n)$-form and thus $\bar{\partial} \bar{\star} \omega = 0$. Therefore, $\bar{\partial} \omega = 0$ and $\bar{\partial}^* \omega = 0$ and thus $\Delta_{\bar{\partial}} \, \omega = 0$.


\subsubsection{3.2.9 DO!!}

\subsubsection{3.2.10 CHECK!!}

Let $(X, g)$ be a compact hermitian manifold. Show that any $\d$-harmonic $(p,q)$-form is also $\bar{\partial}$-harmonic.
\bigskip\\
Since $\alpha$ is $\d$-harmonic, we have $\d{\alpha} = 0$ and $\d^*{\alpha} = 0$. Therefore, $\partial \alpha = 0$ and $\bar{\partial} \alpha = 0$ and $\partial^* \alpha = 0$ and $\bar{\partial}^* \alpha = 0$. Therefore, $\Delta_{\partial} (\alpha) = 0$ and $\Delta_{\bar{\partial}} (\alpha) = 0$.

\subsubsection{3.2.11}

Let $X = \P^n$. Consider the Euler sequence,
\begin{center}
\begin{tikzcd}
0 \arrow[r] & \Omega_{X} \arrow[r] & \struct{X}(-1)^{\oplus (n+1)} \arrow[r] & \struct{X} \arrow[r] & 0 
\end{tikzcd}
\end{center}
Then we may apply exterior powers to get the following sequence,
\begin{center}
\begin{tikzcd}
0 \arrow[r] & \Omega^p_{X} \arrow[r] & \bigwedge^p \struct{X}(-1)^{\oplus (n+1)} \arrow[r] & \struct{X} \otimes_{\struct{X}} \Omega^{p-1}_{X} \arrow[r] & 0
\end{tikzcd}
\end{center}
However, 
\[ \bigwedge^p \struct{X}(-1)^{\oplus (n+1)} = \struct{X}(-p)^{\oplus { n + 1 \choose p }} \] 
and thus we have the sequence,
\begin{center}
\begin{tikzcd}
0 \arrow[r] & \Omega^p_{X} \arrow[r] & \struct{X}(-p)^{\oplus { n + 1 \choose p }} \arrow[r] & \Omega^{p-1}_{X} \arrow[r] & 0
\end{tikzcd}
\end{center}
Now applying the cohomology sequence we find,
\begin{center}
\begin{tikzcd}
H^{q-1}(X, \struct{X}(-p))^{\oplus {n + 1 \choose p }} \arrow[r] & H^{q-1}(X, \Omega_{X}^{p-1}) \arrow[r] & H^{q}(X, \Omega_{X}^p) \arrow[r] & H^{q}(X, \struct{X}(-p))^{\oplus {n + 1 \choose p }}
\end{tikzcd}
\end{center}
Therefore, if $0 < q < n$ and $p > 0$ then $H^{q-1}(X, \Omega^{p-1}_{X}) \iso H^q(X, \Omega^p_{X})$. Furthermore, if $q = 0$ and $p > 0$ then $H^0(X, \Omega^p_{X}) = 0$ because we get an exact sequence,
\begin{center}
\begin{tikzcd}
0 \arrow[r] & H^0(X, \Omega^p_X) \arrow[r] & H^0(X, \struct{X}(-p))^{\oplus {n + 1 \choose p }}
\end{tikzcd}
\end{center}
and $H^0(X, \struct{X}(-p)) = 0$. Finally, if $q = n$ and $p < n + 1$ (which it must) then $H^{q-1}(X, \Omega_X^{p-1}) \iso H^q(X, \Omega_X^p)$ because $H^n(X, \struct{X}(-p)) = H^0(X, \struct{X}(p - n - 1))$ by Serre duality.
\bigskip\\
To finish the base case $p = 0$ we know,
\[ H^q(X, \struct{X}) = 
\begin{cases}
\C & q = 0
\\
0 & q > 0
\end{cases} \]
Therefore, by induction, $H^p(X, \Omega_X^p) = H^{p-1}(X, \Omega_X^{p-1}) = \C$ for $p \le n$. Furthermore, if $p \neq q$ then reducing via $H^{q-1}(X, \Omega^{p-1}_{X}) \iso H^q(X, \Omega^p_{X})$ we get to either $H^q(X, \Omega^0_X) = 0$ with $q > 0$ or $H^0(X, \Omega^p_X) = 0$ with $p > 0$. Therefore,
\[ H^q(X, \Omega^p_X) = 
\begin{cases}
\C & p = q \le n
\\
0 & p \neq q
\end{cases} \]
Now consider the exponential sequence,
\begin{center}
\begin{tikzcd}
H^0(X, \struct{X}^\times) \arrow[r] & H^1(X, \Z) \arrow[r] & H^1(X, \struct{X}) \arrow[r] & H^1(X, \struct{X}^\times) \arrow[r] & H^2(X, \Z) \arrow[r] & H^2(X, \struct{X})
\end{tikzcd}
\end{center}
but $H^1(X, \struct{X}) = H^2(X, \struct{X}) = 0$ and $H^1(X, \struct{X}^\times) = \Pic{X}$. Therefore, the exponential sequence defines an isomorphism $\Pic{X} \iso H^2(X, \Z)$. By the \kahler decomposition $H^2(X, \C) = H^{1,1}(X) = H^1(X, \Omega^1_X)$ and thus $\dim_\C H^2(X, \C) = 1$. Therefore, $H^2(X, \Z) = \Z$ because $H^2(X, \Z)$ is torsion-free since $\P^n$ is simply connected and therefore $H_1(X, \Z) = 0$.

\subsubsection{3.2.12 DO!!}

\subsubsection{3.2.13}

Let $X$ be a compact \kahler manifold and $\alpha \in \A^k(X)$ which is $\d$-closed and $\dc$-exact where $\dc = i (\bar{\partial} - \partial)$. Notice that $\d \dc = 2 i \partial \bar{\partial}$. Write $\alpha = \alpha^{k, 0} + \cdots + \alpha^{0, k}$. Since $\d{\alpha} = 0$ and $\dc{\alpha} = 0$ we see that $\partial \alpha = 0$ and $\bar{\partial} \alpha = 0$ and this must be true for each $\alpha^{p,q}$ because $\Pi^{p+1,q} \partial \alpha = \partial \alpha^{p,q}$ etc. Then by the $\partial \bar{\partial}$-lemma we know $\alpha^{p,q} = \partial \bar{\partial} \beta$ for $\beta \in \A^{p-1,q-1}(X)$. Therefore, 
\[ \alpha = \alpha^{k, 0} + \cdots + \alpha^{0, k} = \partial \bar{\partial} (\beta^{k, 0} + \cdots + \beta^{0, k}) = - \frac{i}{2} \d \dc \beta \]
where $\beta = \beta^{k, 0} + \cdots + \beta^{0, k}$. 

\subsubsection{3.2.14 DO!!}

DO! LOOK AT PREVIOUS BC PROBLEM AND CORRESPOND

\subsubsection{3.2.15}

Let $(X, g)$ be a compact hermitian manifold and let $[\alpha] \in H^{p,q}(X)$ be a cohomology class. If we deform $\alpha' = \alpha + t \bar{\partial} \beta$ by an exact form such that $[\alpha'] = [\alpha]$. Then consider the norm,
\[ || \alpha' ||^2 = \inner{\alpha + t \bar{\partial} \beta}{\alpha + t \bar{\partial}  \beta} = || \alpha ||^2 + 2 t \Re{\inner{\alpha}{\bar{\partial} \beta}} + t^2 || \bar{\partial} \beta ||^2 \]
Suppose that $\alpha$ has minimal norm then the linear term must be zero so $\Re{\inner{\alpha}{\bar{\partial} \beta}} = 0$. Likewise, replacing $\beta$ by $i \beta$ then $\Re{\inner{\alpha}{\bar{\partial} i \beta}} = \Im{\inner{\alpha}{\bar{\partial} \beta}}$ so we must have $\inner{\alpha}{\bar{\partial} \beta} = 0$. However, $\inner{\alpha}{\bar{\partial} \beta} = \inner{\bar{\partial}^* \alpha}{\beta} = 0$ for arbitrary $(p, q-1)$-forms $\beta$. In particular, we can take,
\[ || \bar{\partial}^* \alpha ||^2 = \inner{\bar{\partial}^* \alpha}{\bar{\partial}^* \alpha} = 0 \]
so $\bar{\partial}^* \alpha = 0$. Furthermore $\bar{\partial} \alpha = 0$ since it defines a cohomology class. Thus $\Delta_{\bar{\partial}} \alpha = 0$ so the representatives of minimal norm are exactly the harmonic representatives.

\subsubsection{3.2.16}

Let $X$ be a compact \kahler manifold. Let $\omega$ and $\omega'$ be \kahler forms such that $[\omega] = [\omega'] \in H^2(X, \R)$. Then $\eta = \omega - \omega' = \d{\alpha}$ for some real $1$-form $\alpha$. Thus $\eta$ is a closed real $(1,1)$-form which is $\d$-exact and thus by the $\partial \bar{\partial}$-lemma $\eta = i \partial \bar{\partial} f$ for some $f \in \A^{0,0}$. Notice,
\[ \bar{\eta} = - i \bar{\partial} \partial \bar{f} = i \partial \bar{\partial} \bar{f} \]
however $\eta$ is real so $\bar{\eta} = \eta$ and thus $\bar{f} = f$ so $f \in \A^0_\R$ is a real function and,
\[ \omega = \omega' + i \partial \bar{\partial} f \]

\subsection{Section 3.3}

\subsubsection{3.3.1 DO!!}

\subsubsection{3.3.2 DO!!}


\subsubsection{3.3.3 DO!!}


\section{Chapter 4}

\subsection{Section 4.1}

\subsubsection{4.1.1}

Let $L$ be a holomorphic line bundle globally generated by sections $s_1, \dots, s_k \in H^0(X, L)$. Then $L$ admits a canonical hermitian structure $h$ defined by locally choosing a trivialization, $\phi : L|_U \iso \struct{U}$ then for any local sections $\alpha, \beta \in \L(U)$,
\[ h(\alpha, \beta) = \frac{\psi(\alpha) \cdot \overline{\psi(\beta)}}{\sum_i |\psi(s_i)|^2} \]
This is well-defined because for any other choice of local trivialization $\psi' : L|_u \iso \struct{U}$ gives a transition function $t = \psi' \circ \psi^{-1} : \struct{U} \to \struct{U}$ which is a holomorphic function on $U$. Then $\psi' = (\psi' \circ \psi^{-1}) \circ \psi = t \psi$ and therefore,
\[ \frac{\psi'(\alpha) \cdot \overline{\psi'(\beta)}}{\sum_i |\psi'(s_i)|^2} = \frac{t \psi(\alpha) \cdot \overline{t \psi(\beta)}}{\sum_i |t \psi(s_i)|^2} = \frac{|t|^2 \psi(\alpha \cdot \overline{\psi(\beta)}}{|t|^2 \sum_i |\psi(s_i)|^2} = \frac{\psi(\alpha) \cdot \overline{\psi(\beta)}}{\sum_i |\psi(s_i)|^2} \]
Then the dual bundle $L^*$ obtains a natural hermitian structure $h^*$ defined on $\alpha, \beta \in L^*(U)$ i.e. maps $L|_U \to \struct{U}$ via,
\[ h^*(\alpha, \beta) = \alpha(h^{-1}(\beta)) \]
viewing $h$ as an $\C$-antilinear isomorphism $h : \L \iso \L^*$. Furthermore, there is an inclusion $L^* \embed \struct{X}^{\oplus k}$ dual to $\struct{X}^{\oplus k} \onto \L$ defined by applying $\alpha \in L^*(U)$ to $s_1, \dots, s_n$. By restricting the standard hermitian structure on $\struct{X}^{\oplus k}$ to $L^*$ we get a hermitian structure $h'$ on $\L$. Explicitly, for $\alpha, \beta \in L^*(U)$,
\[ h'(\alpha, \beta) = \sum_i \alpha(s_i) \overline{\beta(s_i)} \]
However, chooseing a local trivialization $\psi : \L|_U \iso \struct{U}$ then $\psi^* : \struct{U} \iso \L^*|_U$ given by $\psi^*(f) = f \psi$ is an isomorphism. We explicity find,
\[ h^{-1}(\psi^*(f)) = \bar{f} h^{-1}(\psi) = f \sum_i s_i \cdot \overline{\psi(s_i)} \]
Therefore, 
\[ h^*(\psi^*(f), \psi^*(g)) = f \bar{g} \psi(h^{-1}(\beta)) = f \bar{g} \sum_i |\psi(s_i)|^2 = h'(\psi^*(f), \psi^*(g)) \]
Thus $h^* = h'$.

\subsubsection{4.1.2}

Let $L$ be a holomorphic line bundle of degree $d > 2 g(C) - 2$ on a curve $C$ where $\deg{K_C} = 2 g(C) - 2$. By Serre duality $H^1(C, L) = H^0(X, L^* \otimes K_C) = 0$ since $\deg{(L^* \otimes K_C)} = 2 g(C) - 2 - d < 0$. In particular if $L$ is a line bundle with $\deg{L} > 0$ then $H^1(C, K_C \otimes L) = 0$.

\subsubsection{4.1.3 DO!!}

Let $X = \P^n$.
To compute the cohomology $H^q(X, \struct{X}(k))$ consider the  

\subsubsection{4.1.4}

Let $E$ be a hermitian holomorhic vector bundle on a compact \kahler manifold $X$ Consider a global holomorphic section $s \in H^0(X, \Omega^p \otimes E)$. Then we know,
\[ \Delta_E(s) = 0 \iff \bar{\partial}_E s = 0 \text{ and } \bar{\partial}_E^* s = 0 \]
Notice that $\Omega^p \otimes E$ is the kernel of $\bar{\partial}_E : \A^{p,0}(X, E) \to \A^{p, 1}(X, E)$ therefore $\bar{\partial}_E s = 0$ automatically. Furthermore, 
\[ \bar{\partial}^*_E s = - \bar{\star}_E \bar{\partial}_{E^*} \bar{\star}_E s \]
However $\bar{\star}_E s \in \A^{n-p,n}(X, E^*)$ and thus $\bar{\partial}_{E^*} \bar{\star}_E s \in \A^{n-p,n+1}(X, E^*) = 0$. Therefore, $\bar{\partial}^*_E s = 0$ so $\nabla_E s = 0$ and thus $s$ is Harmonic.

\subsubsection{4.1.5 DO!!}

\subsection{Section 4.2}

\subsubsection{4.2.1 DO!!}

\subsubsection{4.2.2 DO!!}

\subsubsection{4.2.3 DO!!}

\subsubsection{4.2.4 DO!!}

\subsubsection{4.2.5 DO!!}

Let $(E, h)$ be a hermitian vector bundle and suppose $E = E_1 \oplus E_2$. Then $E_1, E_2$ are hermitian with hermitian structures $h_1$ and $h_2$ induced by the inclusions. Now let $\nabla_1$ and $\nabla_2$ be the induced connections.

\subsubsection{4.2.6}

Let $E$ be a vector bundle and $\nabla$ a connection on $E$. Then there is an induced connection $\nabla' : \bigwedge^2 E \to \Omega^1_X \otimes \bigwedge^2$ as the quotient of the induced connection on $E^{\otimes 2}$. Explicitly,
\[ \nabla' (s_1 \wedge s_2) = \nabla s_1 \wedge s_2 + s_1 \wedge \nabla s_2 \]
This is well-defined because,
\[ \nabla' (s \wedge s) = \nabla s \wedge s + s \wedge \nabla s = 0 \]
Furthermore, there is an induced connection $\nabla' : \det{E} \to \Omega^1_X \otimes \det{E}$ as the quotient of the induced connection on $E^{\otimes n}$. Explicitly,
\[ \nabla' (s_1 \wedge \cdots \wedge s_n) = \nabla s_1 \wedge \cdots \wedge s_n + \cdots + s_1 \wedge \cdots \wedge \nabla s_n \]
which is well-defined as above. Let $e_i$ be a local frame of $E$ then we can write $\nabla = \d + A$ where $A$ is a matrix of $1$-forms which really means,
\[ \nabla(f_j e_j) = \d{f_j} \otimes e_j + f_j \nabla e_j = \d{f_j} \otimes e_j + f_j A_{ij} \otimes e_i = (\d{f}_i + A_{ij} f_j) \otimes e_i \]
where $A_{ij} \otimes e_i = \nabla e_j$. Then we find that,
\begin{align*}
\nabla' (e_1 \wedge \cdots \wedge e_n) & = A_{1k} \otimes e_k \wedge \cdots \wedge e_n + \cdots + e_1 \wedge \cdots \wedge A_{nk} \otimes e_k = (A_{11} + \cdots + A_{nn}) \otimes (e_1 \wedge \cdots \wedge e_n) 
\\
& = \tr{A} \otimes (e_1 \wedge \cdots \wedge e_n) 
\end{align*}
Therefore, the connection $\nabla' : \det{E} \to \Omega^1_X \otimes \det{E}$ on the line bundle $\det{E}$ is locally given by the $1$-form $\tr{A}$.

\subsubsection{4.2.7 DO!!}

\subsubsection{4.2.8}

First note that if $\nabla_1$ and $\nabla_2$ are connections on $E_1$ and $E_2$ we define a connection $\nabla : E_1 \otimes E_2 \to \Omega \otimes (E_1 \otimes E_2)$ via $\nabla (s_1 \otimes s_2)  = \nabla_1 s_1 \otimes s_2 + s_1 \otimes \nabla_2 s_2$ and also $\nabla : \Hom{\struct{X}}{E_1}{E_2} \to \Hom{\struct{X}}{E_1}{E_2}$ via $\varphi \mapsto \nabla \varphi$ such that $(\nabla \varphi)(s) = \nabla_2 \varphi(s) - \varphi(\nabla_1(s))$ for any $s \in \Gamma(U, E)$.
\bigskip\\
Therefore, if $\nabla$ is a connection on $E$ define the dual connection $\nabla^* : E^* \to \Omega^1_X \otimes E^*$ on $E^*$ via sending $\varphi : E \to \struct{X}$ so $(\nabla^* \varphi)(s) = \d{\varphi(s)} - \varphi(\nabla s)$.
\bigskip\\
Consider the induced connection $\nabla$ on $(E \otimes \overline{E})^*$. Then for any section $h \in \Gamma(U, (E \otimes \overline{E})^*)$, for instance a hermitian metric, we get $\nabla h$ such that for any $s_1, s_2 \in E$,
\begin{align*}
(\nabla h)(s_1 \otimes \bar{s}_2) & = \d{h(s_1 \otimes \bar{s}_2)} - h(\nabla (s_1 \otimes \bar{s}_2)) = \d{h(s_1 \otimes \bar{s}_2)} - h(\nabla s_1 \otimes \bar{s}_2) - h(s_1 \otimes \bar{\nabla} \bar{s}_2) 
\\
& = \d{h(s_1 \otimes \bar{s}_2)} - h(\nabla s_1 \otimes \bar{s}_2) - h(s_1 \otimes \overline{\nabla s_2}) 
\end{align*}
Therefore, as a metric,
\[ (\nabla h)(s_1, s_2) = \d{h(s_1, s_2)} - h(\nabla s_1, s_2) - h(s_1, \nabla s_2) \]
Therefore, $\nabla$ is hermitian with respect to the hermitian structure $(E, h)$ iff $\nabla h = 0$.

\subsubsection{4.2.9 DO!!}

\subsection{Section 4.3}

\subsubsection{4.3.1 DO!!}

Consider $\nabla^2 : \A^k(E) \to \A^{k+2}(E)$ on $\omega \otimes s$ where $\omega$ is a $k$-form and $s$ is a section of $E$. Then,
\[ \nabla (\omega \otimes s) = \d{\omega} \otimes s + (-1)^k \omega \wedge \nabla(s) \]
and thus,
\begin{align*}
\nabla^2(\omega \otimes s) & = \d{\d{\omega}} \otimes s + (-1)^{k+1} \d{\omega} \wedge \nabla s + (-1)^k \d{\omega} \wedge \nabla(s) + (-1)^{2k} \omega \wedge \nabla^2(s)
\\
& = \omega \wedge \nabla^2 (s) 
\end{align*}
since the $\d{\d{\omega}} = 0$ and the middle terms cancel. Here we use the generalized Leibniz formula,
\[ \nabla (\omega \wedge \alpha) = \d{\omega} \wedge \alpha + (-1)^k \omega \wedge \nabla \alpha \]
where $\omega$ is a $k$-form and $\alpha \in \A^\ell(E)$. Furthermore $\nabla^2(s) = F_\nabla(s)$ so we find,
\[ \nabla^2 (\omega \otimes s) = \omega \wedge F_\nabla(s) \]
and therefore for a general $E$-valued $k$-form $\alpha$ we see that $\nabla^2(\alpha) = \tr{\omega \wedge F_\nabla}$ where we view $\omega \in H^0(X, \Omega^k_X \otimes E)$ and $F_\nabla \in H^0(X, \Omega^2_X \otimes E \otimes E^*)$ and taking the map $\wedge : \Omega^k_X \otimes \Omega^2_X \to \Omega^{k+2}_X$ and contracting the $E^*$ from $F_\nabla$ with the $E$ from $\omega$.

\subsubsection{4.3.2 DO!!}

\subsubsection{4.3.3 DO!!}

Let $(E_1, h_1)$ and $(E_2, h_2)$ be two hermitian holomorphic vector bundles endowed with hermitian connections $\nabla_1, \nabla_2$ such that the curvature of both is (semi)-positive. 

\begin{enumerate}
\item The curvature of $\nabla^*$ on $E^*$ is $F_{\nabla^*} = - F_\nabla^\top$ which is (semi)-negative since $F_{\nabla}$ is (semi)-positive.

\item The curvature of $\nabla$ on $E_1 \otimes E_2$ is $F_{\nabla} = F_{\nabla_1} \otimes \id + \id \otimes F_{\nabla_2}$ which is (semi)-positive since $F_{\nabla_1}$ and $F_{\nabla_2}$ are. Furthermore if one of $F_{\nabla_i}$ is positive then $F_{\nabla}$ is positive since the other term is nonegative.

\item The curvature of $\nabla$ on $E_1 \oplus E_2$ is $F_{\nabla} = F_{\nabla_1} \oplus F_{\nabla_2}$ which is (semi)-positive since $F_{\nabla_1}$ and $F_{\nabla_2}$ are. 
\end{enumerate}

\subsubsection{4.3.4 DO!!}

\subsubsection{4.3.5 CHECK!!}

Let $X$ be complex manifold. Let $L$ be a holomorphic line bundle with a hermitian structure $h$ whose Chern connection has positive curvature. Then $F_\nabla \in \A^{1,1}(X)$ is an imaginary $(1,1)$-form. Furthermore, note that $F_\nabla = \bar{\partial} \partial \log{h}$ and thus,
\[ \d{F_\nabla} = (\partial + \bar{\partial}) \bar{\partial} \partial \log{h} = 0 \]
because $\bar{\partial}^2 = 0$ and $\partial \bar{\partial} \partial = - \partial^2 \bar{\partial} = 0$. Since $\omega = i F_\nabla$ is positive, it is a \kahler form. Furthermore if $X$ is compact then,
\[ \int_X A(L)^n = \int_X F_\nabla^n = \int_X \omega^n = n! \int_X \mathrm{vol}_\omega > 0 \]
(CHECK THIS! FACTORS OF I)

\subsubsection{4.3.6}

Let $X = \P^n$ and $\omega_X = \struct{X}(-n-1)$ be the canonical bundle. The sections $x_0, \dots, x_n \in \Gamma(X, \struct{X}(1))$ define a canonical hermitian structue $h$ on $\struct{X}(1)$ which has the property that $\frac{i}{2 \pi} F_\nabla = \omega_{\mathrm{FS}}$ which is positive. Then $\omega_X = \struct{X}(-n-1)$ has a canonical hermitian structure $(h^*)^{\otimes n + 1}$ which has curvature $\frac{i}{2 \pi} F_{\nabla'} = - (n + 1) \omega_{\mathrm{FS}}$ which is therefore negative. 

\subsubsection{4.3.7 DO!!}

\subsubsection{4.3.8 DO!!}

\subsubsection{4.3.9}

Let $X$ be a compact \kahler manifold with $b_1(X) = 0$. Suppose that $\nabla$ is a flat connection on $\struct{X}$ with $\nabla^{0,1} = \bar{\partial}$. Then $\nabla = \d + \omega$ where $\omega : \A^0(X) \to \A^1(X)$ is $\A^0(X)$-linear and thus $\omega \in \A^1(X)$. Futhermore, $\nabla^{0,1} = \bar{\partial}$ so $\omega$ is a smooth $(1,0)$-form. Now consider the curvature,
\[ F_\nabla = \nabla \circ \nabla(1) = \nabla (\omega \otimes 1) = \d{\omega} \otimes 1 - \omega \wedge \nabla(1) = \d{\omega} \otimes 1 - \omega \wedge \omega \otimes 1 = \d{\omega} \]
Since $\nabla$ is flat we must have $\d{\omega} = 0$. Thus $\omega$ defines a de Rham cohomology class $[\omega] \in H^1(X, \C)$ but $b_1(X) = 0$ so $\omega$ is exact. Take $\omega = \d{f}$ for some smooth function $f$. However, $\omega$ is a $(1,0)$-form so $f$ is holomorphic. But $X$ is compact so $f$ is constant and thus $\omega = 0$ showing that $\nabla = \d$.
\bigskip\\
Now suppose that $L$ is a line bundle on $X$ with $c_1(L) = 0$. From the exponential sequence,
\begin{center}
\begin{tikzcd}
H^1(X, \struct{X}) \arrow[r] & \Pic{X} \arrow[r, "c_1"] & H^2(X, \Z) 
\end{tikzcd}
\end{center}
and thus $\ker{c_1} = \Im{H^1(X, \struct{X}) \to \Pic{X}}$. However, $b_1(X) = 0$ so by the \kahler decomposition, $H^1(X, \struct{X}) = 0$. Therefore, $\ker{c_1}$ is trivial so $L = \struct{X}$.

\subsubsection{4.3.10 DO!!}

Let $\nabla$ be a connection on a complex vector bundle $E$. We want to show that $E$ locally has parallel frames iff $F_\nabla = 0$.
\bigskip\\
Suppose that $E$ has a local frame $e_1, \dots, e_n$ of parallel sections over $U$ i.e. $\nabla e_i = 0$ and these are independent on each fiber. Since the curvature form $\omega_\nabla(s) = \nabla_1 \circ \nabla(s)$ is $\struct{X}$-linear, writing $s = f_i e_i$ we get, 
\[ \omega_\nabla(f_i e_i) = f_i \omega_\nabla(e_i) = f_i \nabla_1 \circ \nabla e_i = 0 \]
Therefore, $\omega_\nabla = 0$ so $\nabla$ must be flat.
\bigskip\\
Locally write $E|_U \cong \struct{U}^{\oplus n}$ write $e_i$ for a local frame of $E|_U$. Now write $\nabla e_j = \omega_{ij} \otimes e_i$ thus we see,
\[ \nabla (f_j e_j) = \d{f_j} \otimes e_j + \omega_{ij} f_j \otimes e_i = ( \d{f_i} + \omega_{ij} f_j) \otimes e_i \] 
Now, applying $\nabla_1 : \Omega^1_X \otimes E \to \Omega^2_X \otimes E$ we get,
\begin{align*}
\nabla_1 \circ \nabla (f_j e_j) & = \nabla_1 (\d{f_i} + \omega_{ij} f_j) \otimes e_i = \d{\d{f_i}} \otimes e_i + \d{(\omega_{ij} f_j)} \otimes e_i - (\d{f_i} + \omega_{ij} f_j) \wedge \nabla e_i 
\\
& = (\d{\omega_{ij}} \, f_j - \omega_{ij} \wedge \d{f_j}) \otimes e_i - (\d{f_i} + \omega_{ij} f_j) \wedge \omega_{ki} \otimes e_k
\\
& = \d{\omega_{ij}} f_j \otimes e_i + \d{f_j} \wedge \omega_{ij} \otimes e_i - \d{f_i} \wedge \omega_{ki} \otimes e_k + \omega_{ki} \wedge \omega_{ij} f_j \otimes e_k
\\
& = (\d{\omega_{ij}} + \omega_{ik} \wedge \omega_{kj}) f_j \otimes e_i
\end{align*}
Therefore,
\[ \omega_\nabla(f_j e_j) = (\d{\omega_{ij}} + \omega_{ik} \wedge \omega_{kj}) f_j \otimes e_i \]
is linear as it should be. Now assume $\nabla$ is flat i.e. $\omega_\nabla = 0$. Thus,
\[ \d{\omega_{ij}} + \omega_{ik} \wedge \omega_{kj} = 0 \] 
First, in the case $n = 1$ the connection is given by a $1$-form $\omega$. Then $\omega_\nabla = 0 \iff \d{\omega} = 0$ in which case locally $\omega = - \d{f}$ and thus $\nabla (fe) = \d{f} \otimes e + \omega \otimes e = 0$ so we get a frame of parallel sections.
\bigskip\\
Now we proceed by induction for the general case. First, using a $\GL(n, \C)$ transformation we can 

  Assume we can find a frame $e_1, \dots, e_{n-1}, s$ such that $\nabla e_i = 0$. (FINISH) 

\subsection{Section 4.4}

\subsubsection{4.4.1 DO!!}

\subsubsection{4.4.2}

Let $X$ be a compact complex manifold and $L$ a basepoint-free line bundle. Then $L$ defines a map $f : X \to \P^N$ such that $f^* \struct{\P^N}(1) = L$. Let $h$ be the standard hermitian structure on $\struct{\P^N}(1)$ so $f^* h$ gives a hermitian structure on $L$. Taking the Chern connections $\nabla_{f^* h} = f^* \nabla_h$ and thus,
\[ F(L, f^* h) = F(f^* \struct{\P^N}(1), f^* h) = f^* F(\struct{\P^N}(1), h) = f^* \omega_{\text{FS}} \]
which is a positive form.
Therefore,
\[ c_1(L) = f^* [\omega_{\text{FS}}] \] so we see that,
\[ \int_{X} c_1(L)^{n} = \int_X (f^* \omega_{\text{FS}})^{n} = \int_X f^* \omega_{\text{FS}}^n \ge 0 \]

\subsubsection{4.4.3 ASK RON!!}

\subsubsection{4.4.4 ASK RON!!}

Ask Ron about interpretation!!

\subsubsection{4.4.5 DO!!}

\subsubsection{4.4.6 DO!!}

\subsubsection{4.4.7 DO!!}

\subsubsection{4.4.8 DO!!}

\subsubsection{4.4.9}

Note that $\End{E} \cong E^* \otimes E$ then,
\[ c_k(\End{E}) = \sum_{i + j = k} c_i(E^*) \cdot c_j(E) = \sum_{i + j} (-1)^i c_i(E) \cdot c_j(E) \]
In particular,
\[ c_1(\End{E}) = c_0(E) \cdot c_1(E) - c_1(E) \cdot c_0(E) = 0 \]
and likewise,
\[ c_2(\End{E}) = c_0(E) \cdot c_2(E) - c_1(E) \cdot c_1(E) + c_2(E) \cdot c_0(E) = 2 c_2(E) - c_1(E)^2 \]
Then if $E = L \oplus L$ where $L$ is a line bundle we have,
\[ c(L) = 1 + c_1(L) \]
and thus,
\[ c_1(E) = 2 c_1(L) \quad \text{and} \quad c_2(E) = c_1(L)^2 \]
Therefore, we see that,
\[ (4 c_2 - c_1^2)(E) = 4 c_1(E)^2 - (4 c_1(E))^2 = 0 \]
Furthermore, if $E \cong E^*$ then $c_{2k+1}(E) = c_{2k+1}(E^*) = (-1)^{2k + 1} c_{2k+1}(E) = - c_{2k + 1}(E)$ and thus $c_{2k + 1}(E) = 0$.

\subsubsection{4.4.10}

Let $L$ be a holomorphic line bundle on $X$ a compact K\"{a}hler manifold. Suppose that $c_1(L) = [\alpha]$ where $\alpha$ is closed a real $(1,1)$-form. Let $h_0$ be a Hermitian structure on $L$ then,
\[ c_1(L, h_0) = \frac{i}{2 \pi} \bar{\partial} \partial \log{h_0} \]
Now consider,
\[ \eta = \alpha - c_1(L, h_0) \]
is a real $(1,1)$-form and since $[\alpha] = [c_1(L, h_0)]$ also $\eta$ is $\d$-exact. Thus, by the $\partial \bar{\partial}$-lemma, we know,
\[ \eta = - \frac{i}{2 \pi} \partial \bar{\partial} f \]
for $f \in \A^{0, 0}_\R(X)$ i.e. $f$ is a real smooth function. Therefore,
\[ \alpha = \frac{i}{2 \pi} \bar{\partial} \partial \left[ f + \log{h_0} \right] = \frac{i}{2 \pi} \bar{\partial} \partial \log{e^f h_0} \]
Therefore, let $h = e^f h_0$ be annother Hermitian structure (since $f$ is real) then we see $c_1(L, h) = \alpha$.

\subsubsection{4.4.11}

Let $X$ be compact \kahler and $E$ a vector bundle with a Chern connection $\nabla$. If we let,
\[ \sum_{i = 0}^r \tilde{P}_i(B) = \tr{e^B} = \sum_{n = 0}^\infty \frac{1}{n!} \tr{B^n}  \]
so,
\[ \tilde{P}_k(B_1, \dots, B_k) = \frac{1}{k!} \tr{B_1 \cdots B_k} \] 
and then define,
\[ \ch_k(E,\nabla) := \tilde{P}_k \left( \frac{i}{2 \pi} F_\nabla \right) \in \A_\C^{2k}(M) \]
where $\tilde{P}_k$ acts on $\End{E}$-valued $2$-forms via,
\[ \tilde{P}_k(\alpha_1 \otimes \varphi_1, \dots, \alpha_k \otimes \varphi_k) = (\alpha_1 \wedge \cdots \wedge \alpha_k) \, \tilde{P}_k(\varphi_1, \dots, \varphi
_k) = (\alpha_1 \wedge \cdots \wedge \alpha_k) \, \frac{1}{k!} \tr{\varphi_1 \cdots \varphi_k} \]
This is the composition of $(\Omega_X^2)^{\otimes k} \to \Omega_X^{2k}$ via exterior product and $\End{E}^{\otimes k} \to \End{E}$ via composition and finally taking trace. We see that,
\[ \ch_k(E, \nabla) = \frac{1}{k!} \left( \frac{i}{2 \pi} \right)^k \tr{F_\nabla^{\otimes k}} \]
where $F_\nabla^{\otimes k}$ is the image under $(\Omega_X^2 \otimes \End{E})^{\otimes k} \to \Omega_X^{2k} \otimes \End{E}$. Now taking Dolbeault  cohomology classes via $\A^{k,k}_\C(\End{E}) \to H^k(X, \Omega^k \otimes \End{E})$,
\[ \ch_k(E) = \frac{1}{k!} \left( \frac{i}{2 \pi} \right)^k \tr{[F_\nabla]^{\otimes k}} \]
where $[F_\nabla]^{\otimes k}$ is the image under the map,
\[ H^1(X, \Omega^1_X \otimes \End{E}) \times \cdots \times H^1(X, \Omega^1_X \otimes \End{E}) \to H^{k}(X, \Omega^k \otimes \End{E}) \]
Furthermore $[F_\nabla] = A(E)$ so we get,
\[ \ch_k(E) = \frac{1}{k!} \left( \frac{i}{2 \pi} \right)^k \tr{A(E)^{\otimes k}} \]
as a class under the map $H^k(X, \Omega^k \otimes \End{E}) \xrightarrow{\mathrm{tr}} H^k(X, \Omega^k_X) \subset H^{2k}(X, \C)$.

\subsubsection{4.4.12}

Let $X$ be compact \kahler and $E$ a holomorphic vector bundle admitting a holomorphic connection. Then $A(E) = 0$ and therefore $c_k(E) = 0$. 

\section{Chapter 5}

\subsection{Section 5.1}

\subsubsection{5.1.1 DO!!}

\subsection{Section 5.2}

\subsubsection{5.2.1 DO!!}

\subsection{Section 5.3}

\subsubsection{5.3.1 DO!!}

\section{Chapter 6}

\subsection{Section 6.1}

\subsubsection{6.1.1}

\begin{enumerate}
\item Let $X = \P^n$ then by the Euler sequence,
\begin{center}
\begin{tikzcd}
0 \arrow[r] & \struct{X} \arrow[r] & \struct{X}(1)^{\oplus (n+1)} \arrow[r] & \T_X \arrow[r] & 0
\end{tikzcd}
\end{center}
giving a long exact sequence,
\begin{center}
\begin{tikzcd}
H^1(X, \struct{X}) \arrow[r] & H^1(X, \struct{X}(1))^{\oplus (n+1)} \arrow[r] & H^1(X, \T_X) \arrow[r] & H^2(X, \struct{X})
\end{tikzcd}
\end{center}
However, $H^i(X, \struct{X}(k)) = 0$ for $i > 0$ when $k \ge 0$ and thus we find $H^1(X, \T_X) = 0$.

\item Let $X = \C^n / \Gamma$ be a complex torus. Then we know $\T_X = \struct{X}^{\oplus n}$ and we need to compute $H^1(X, \T_X) = H^1(X, \struct{X})^{\oplus n}$. Then $h^{0,1} = H^1(X, \struct{X})$ and $b_1 = h^{0,1} + h^{1,0} = 2 h^{0,1}$ by Serre duality. However, $b_1 = 2n$ because $H^1(X, \Z) \cong \Hom{\Z}{\Gamma}{\Z}$ and $\Gamma$ has rank $2n$. Therefore $h^{0,1} = n$ and thus $H^1(X, \T_X) = H^1(X, \struct{X})^{\oplus n}$ meaning that $\dim_\C H^1(X, \T_X) = n^n$. 

\item Let $X$ be a compact curve of genus $g$. Then $\T_X$ is a line bundle of degree $2 - 2 g$. Then, by Serre duality, $H^1(X, \T_X) = H^0(X, \Omega_X^{\otimes 2})$. However if $g = 0$ then $\Omega_X$ is negative so $H^1(X, \T_X) = 0$. If $g = 1$ then $\T_X \cong \Omega_X \cong \struct{X}$ in which case $\dim_\C H^1(X, \T_X) = 1$. Finally, if $g > 1$ then $\Omega_X$ is positive and $\Omega_X^{\otimes 2}$ has degree $4g - 4 > 2g - 2$ so $H^1(X, \Omega_X^{\otimes 2}) = 0$ and thus, by Riemann-Roch,
\[ \dim_\C H^1(X, \T_X) = \deg{\Omega_X^{\otimes 2}} + 1 - g = 3g - 3 \] 
\end{enumerate}

\subsubsection{6.1.2 DO!!}

Let $X$ be a compact complex manifold (not necessarily \kahler) and $\sigma \in H^0(X, \Omega_X^2)$ an everywhere non-degenerate holomorphic two-form meaning the map $\sigma : \T_X \to \Omega_X$ is an isomorphism. Since $\sigma$ is nondegenerate fiberwise, we see that $\dim{X} = 2r$ is even. Furthermore, 
\[ \sigma^r = \sigma \wedge \cdots \wedge \sigma \in H^0(X, \Omega_X^{2r}) \]
is everywhere nonvanishing and thus $\sigma^r \in H^0(X, K_X)$ trivializes the canonical bundle $K_X$ so $X$ is a Calabi-Yau. Therefore, applying the results of this section with $\Omega = \sigma^r$, if $v \in H^1(X, \T_X)$ is a cohomology class, then there exists a $\bar{\partial}$-closed lift $\phi_1 \in \A^{0,1}(\T_X)$ representing $H^1(X, \T_X)$ such that the Maurer-Cartan equation admits a formal solution $\sum_i \phi_i t^i$ extending $\phi_1$.

\subsubsection{6.1.3}

Let $X$ be a compact complex manifold with $H^2(X, \T_X) = 0$. Take a cohomology class $v \in H^1(X, \T_X)$ and lift to some $\bar{\partial}$-closed $\phi_1 \in \A^{0,1}(\T_X)$ such that $[\phi_1] = v$. Now for induction suppose we have classes $\phi_1, \dots, \phi_n \in \A^{0,1}(\T_X)$ satisfying the Maurer-Cartan equations up to degree $n$,
\begin{align*}
\bar{\partial} \phi_1 & = 0
\\
\bar{\partial} \phi_2 & = - \sum_{0 < i < 2} [\phi_i, \phi_{2-i}] 
\\
\vdots &
\\
\bar{\partial} \phi_n & = - \sum_{0 < i < n} [\phi_i, \phi_{n-i}]
\end{align*}
Now consider,
\[ \omega =  - \sum_{0 < i < n+1} [\phi_{i}, \phi_{n+1-i}] \]
and we have,
\begin{align*}
\bar{\partial} \omega & = - \sum_{0 < i < n+1} \bar{\partial} [\phi_i, \phi_{n+1 - i}] = - \sum_{0 < i < n+1} \left( [ \bar{\partial} \phi_i, \phi_{n+1-i}] + [\phi_i, \bar{\partial} \phi_{n+1-i}] \right)
\\
& = \sum_{0 < i < n+1} \left( \sum_{0 < j < i} [[\phi_j, \phi_{i-j}], \phi_{n+1-i}] + \sum_{0 < j < n+1 - i} [\phi_i, [\phi_j, \phi_{n+1 - i - j}]] \right)
\\
& = \sum_{0 < i < n+1} \sum_{0 < j < i} [[\phi_j, \phi_{i-j}], \phi_{n+1 - j}] + \sum_{0 < i < n+1} \sum_{0 < j < i} [\phi_{n+1-i}, [\phi_j, \phi_{i}]] = 0
\end{align*}
replacing $i$ by $n+1 - i$ in the second sum. Vanishing follows from $[\alpha, \beta] = (-1)^{k\ell+1} [\beta, \alpha]$ for $\alpha \in \A^{0,k}(\T_X)$ and $\beta \in \A^{0, \ell}(\T_X)$ and in our case $k = 2$. Therefore $\omega \in \A^{0,2}(\T_X)$ is $\bar{\partial}$-closed but $H^2(X, \T_X) = 0$ so every $\bar{\partial}$-closed $\T_X$-valued $2$-form is $\bar{\partial}$-exact meaning there exists $\phi_{n+1} \in \A^{0,1}(\T_X)$ such that $\omega = \bar{\partial} \phi_{n+1}$. Explicitly,
\[ \bar{\partial} \phi_{n+1} = - \sum_{0 < i < n+1} [\phi_{i}, \phi_{n+1-i}] \]
solving the Maurer-Cartan equation recursively.

\section{Extra Questions for Ron}

\subsubsection{1}

Kodaira embedding says that every positive line bundle is ample in the sense of having some power very ample. Does the algebraic geometry definition work here? I.e. $L$ is ample iff for each bundle $Q$ we have $Q \otimes L^n$ generated by global sections for $n \gg 0$. Do we need $Q$ to be arbitrary coherent sheaf.
\\
Yes, in fact we only need this for vector bundles because it then follows by resolution for all coherent sheaves.

\subsubsection{2}

If we have a big line bundle $H^0(X, L^{\otimes m}) \sim m^n$ then does it follow there is an ample line bundle i.e. $X$ is projective. I am guessing not. This is similar to asking if there are non algebraic examples of compact Moishezon manifolds $a(X) = \dim{X}$.

\end{document}


