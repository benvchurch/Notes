\documentclass[12pt]{article}
\usepackage[english]{babel}
\usepackage[utf8]{inputenc}
\usepackage[english]{babel}
\usepackage[a4paper, total={7.25in, 9.5in}]{geometry}
\usepackage{tikz-feynman}
\tikzfeynmanset{compat=1.0.0} 
\usepackage{subcaption}
\usepackage{float}
\floatplacement{figure}{H}
\usepackage{mathrsfs}  
\usepackage{dsfont}
\usepackage{relsize}
\DeclareMathAlphabet{\mathdutchcal}{U}{dutchcal}{m}{n}

\usepackage{revsymb}


\newcommand{\field}{\hat{\Phi}}
\newcommand{\dfield}{\hat{\Phi}^\dagger}
 
\usepackage{amsthm, amssymb, amsmath, centernot}
\usepackage{slashed}
\newcommand{\notimplies}{%
  \mathrel{{\ooalign{\hidewidth$\not\phantom{=}$\hidewidth\cr$\implies$}}}}
 
\renewcommand\qedsymbol{$\square$}
\newcommand{\cont}{$\boxtimes$}
\newcommand{\divides}{\mid}
\newcommand{\ndivides}{\centernot \mid}

\newcommand{\Integers}{\mathbb{Z}}
\newcommand{\Natural}{\mathbb{N}}
\newcommand{\Complex}{\mathbb{C}}
\newcommand{\Zplus}{\mathbb{Z}^{+}}
\newcommand{\Primes}{\mathbb{P}}
\newcommand{\Q}{\mathbb{Q}}
\newcommand{\R}{\mathbb{R}}
\newcommand{\ball}[2]{B_{#1} \! \left(#2 \right)}
\newcommand{\Rplus}{\mathbb{R}^+}
\renewcommand{\Re}[1]{\mathrm{Re}\left[ #1 \right]}
\renewcommand{\Im}[1]{\mathrm{Im}\left[ #1 \right]}
\newcommand{\Op}{\mathcal{O}}

\newcommand{\invI}[2]{#1^{-1} \left( #2 \right)}
\newcommand{\End}[1]{\text{End}\left( A \right)}
\newcommand{\legsym}[2]{\left(\frac{#1}{#2} \right)}
\renewcommand{\mod}[3]{\: #1 \equiv #2 \: \mathrm{mod} \: #3 \:}
\newcommand{\nmod}[3]{\: #1 \centernot \equiv #2 \: mod \: #3 \:}
\newcommand{\ndiv}{\hspace{-4pt}\not \divides \hspace{2pt}}
\newcommand{\finfield}[1]{\mathbb{F}_{#1}}
\newcommand{\finunits}[1]{\mathbb{F}_{#1}^{\times}}
\newcommand{\ord}[1]{\mathrm{ord}\! \left(#1 \right)}
\newcommand{\quadfield}[1]{\Q \small(\sqrt{#1} \small)}
\newcommand{\vspan}[1]{\mathrm{span}\! \left\{#1 \right\}}
\newcommand{\galgroup}[1]{Gal \small(#1 \small)}
\newcommand{\bra}[1]{\left| #1 \right>}
\newcommand{\Oa}{O_\alpha}
\newcommand{\Od}{O_\alpha^{\dagger}}
\newcommand{\Oap}{O_{\alpha '}}
\newcommand{\Odp}{O_{\alpha '}^{\dagger}}
\newcommand{\im}[1]{\mathrm{im} \: #1}
\renewcommand{\ker}[1]{\mathrm{ker} \: #1}
\newcommand{\ket}[1]{\left| #1 \right>}
\renewcommand{\bra}[1]{\left< #1 \right|}
\newcommand{\inner}[2]{\left< #1 | #2 \right>}
\newcommand{\expect}[2]{\left< #1 \right| #2 \left| #1 \right>}
\renewcommand{\d}[1]{\: \mathrm{d}#1 \:}
\newcommand{\dn}[2]{ \mathrm{d}^{#1} #2 \:}
\newcommand{\deriv}[2]{\frac{\d{#1}}{\d{#2}}}
\newcommand{\nderiv}[3]{\frac{\dn{#1}{#2}}{\d{#3^{#1}}}}
\newcommand{\pderiv}[2]{\frac{\partial{#1}}{\partial{#2}}}
\newcommand{\fderiv}[2]{\frac{\delta #1}{\delta #2}}
\newcommand{\parsq}[2]{\frac{\partial^2{#1}}{\partial{#2}^2}}
\newcommand{\topo}{\mathcal{T}}
\newcommand{\base}{\mathcal{B}}
\renewcommand{\bf}[1]{\mathbf{#1}}
\renewcommand{\a}{\hat{a}}
\newcommand{\adag}{\hat{a}^\dagger}
\renewcommand{\b}{\hat{b}}
\newcommand{\bdag}{\hat{b}^\dagger}
\renewcommand{\c}{\hat{c}}
\newcommand{\cdag}{\hat{c}^\dagger}
\newcommand{\hamilt}{\hat{H}}
\renewcommand{\L}{\hat{L}}
\newcommand{\Lz}{\hat{L}_z}
\newcommand{\Lsquared}{\hat{L}^2}
\renewcommand{\S}{\hat{S}}
\renewcommand{\empty}{\varnothing}
\newcommand{\J}{\hat{J}}
\newcommand{\lagrange}{\mathcal{L}}
\newcommand{\dfourx}{\mathrm{d}^4x}
\newcommand{\meson}{\phi}
\newcommand{\dpsi}{\psi^\dagger}
\newcommand{\ipic}{\mathrm{int}}
\newcommand{\tr}[1]{\mathrm{tr} \left( #1 \right)}
\newcommand{\C}{\mathbb{C}}
\newcommand{\CP}[1]{\mathbb{CP}^{#1}}
\newcommand{\Vol}[1]{\mathrm{Vol}\left(#1\right)}

\newcommand{\Tr}[1]{\mathrm{Tr}\left( #1 \right)}
\newcommand{\Charge}{\hat{\mathbf{C}}}
\newcommand{\Parity}{\hat{\mathbf{P}}}
\newcommand{\Time}{\hat{\mathbf{T}}}
\newcommand{\Torder}[1]{\mathbf{T}\left[ #1 \right]}
\newcommand{\Norder}[1]{\mathbf{N}\left[ #1 \right]}
\newcommand{\Znorm}{\mathcal{Z}}
\newcommand{\EV}[1]{\left< #1 \right>}
\newcommand{\interact}{\mathrm{int}}
\newcommand{\covD}{\mathcal{D}}
\newcommand{\conj}[1]{\overline{#1}}

\newcommand{\SO}[2]{\mathrm{SO}(#1, #2)}
\newcommand{\SU}[2]{\mathrm{SU}(#1, #2)}

\newcommand{\anticom}[2]{\left\{ #1 , #2 \right\}}


\newcommand{\pathd}[1]{\! \mathdutchcal{D} #1 \:}

\renewcommand{\theenumi}{(\alph{enumi})}


\renewcommand{\theenumi}{(\alph{enumi})}

\newcommand{\atitle}[1]{\title{% 
	\large \textbf{ASTR GR6001 Radiative Processes
	\\ Assignment \# #1} \vspace{-2ex}}
\author{Benjamin Church }
\maketitle}

\theoremstyle{definition}
\newtheorem{theorem}{Theorem}[section]
\newtheorem{definition}{definition}[section]
\newtheorem{lemma}[theorem]{Lemma}
\newtheorem{proposition}[theorem]{Proposition}
\newtheorem{corollary}[theorem]{Corollary}
\newtheorem{example}[theorem]{Example}
\newtheorem{remark}[theorem]{Remark}
\begin{document}

\newcommand{\lambdabar}{{\mkern0.75mu\mathchar '26\mkern -9.75mu\lambda}}

\atitle{5}

\section{Problem 1}

Consider a gas of nitrogen molecules with diameter $d$ that collide with mean free path $\ell$. 
\subsection*{(a)}

We suppose that at each collision, the mollecules come off with random velocities. The time between colisions will be distributed exponentially,
\[ P(t) \propto e^{-t/c_s \ell} \]
where $c_s$ is the sound speed. Then the time between colisions is, on average $t = \ell/c_s$. Therefore, the number of collisions is,
\[ N = \frac{t c_s}{\ell} \]
Now we can calculate the RMS distance traveled,
\begin{align*}
\EV{\vec{r}^{\, 2}} & = \EV{\left( \sum_{i = 1}^N \Delta \vec{r}_i^{\, 2} \right)}
\\
& = \sum_{i,j} \EV{\Delta \vec{r}_i \cdot \Delta \vec{r_j}}
\end{align*}
Since we have assumed that collisions erase all correlations in velocity,
\[ \EV{\Delta \vec{r}_i \cdot \Delta \vec{r_j}} = \ell^2 \delta_{ij} \]
and therefore,
\[ \EV{\vec{r}^{\, 2}} = \sum_{i,j} \EV{\Delta \vec{r}_i \cdot \Delta \vec{r_j}} = \sum_{i = 1}^N \ell^2 = N \ell^2 \]
and therefore the RMS distance traveled is,
\[ R_{\text{RMS}} = \ell \sqrt{N} = \ell \sqrt{\frac{t c_s}{\ell}} = \sqrt{\ell t c_s} \]
\bigskip\\
Since we know $c_s$ if we can measure the distance traveled in a time $t$ this determines $\ell$. This can be measured via the rate of diffusion of particles in the gas. For example, a highly volitile odorous chemical can be released on one side of a room and the time it takes before someobdy on the other side can smell it measured. Note that this this case the sound speed $c_s$ should be modified to be the RMS speed of the volitile particle from the Maxwell-Boltzman distribution. This corresponds simply to correcting $c_s$ by the squareroot of the mass ratio of nitrogen molecules to the volitile chemical. 


\subsection*{(b)}

We assume that in the liquid state that the distance between molecules is approximatly $d$. Therefore, we can compute the number of particles as,
\[ N = \frac{V_{\text{liquid}}}{\left( \frac{1}{6} \pi d^3 \right)} \]
Therefore,
\[ n = \frac{V_{\text{liquid}}}{V_{\text{gas}}} \cdot \frac{1}{\left( \frac{1}{6} \pi d^3 \right)} \]
Now we can compute $d$ from the mean free path. Let $\sigma$ be the scattering cross section for nitrogen molecules in the gas as,
\[ \ell = (\sigma n)^{-1} \]
Now we estimate,
\[ \sigma \approx \pi d^2 \]
Therefore, 
\[ d = \frac{1}{\sqrt{\pi \ell n}} \]
Now plugging in,
\[ n = \frac{V_{\text{liquid}}}{V_{\text{gas}}} \cdot \frac{6 (\pi \ell n)^{\frac{3}{2}}}{\pi} \]
Rearanging, we find that,
\[ n^{-\frac{1}{2}} = \frac{V_{\text{liquid}}}{V_{\text{gas}}} (6 \pi^{\frac{1}{2}} \ell^{\frac{3}{2}})  \]
and therefore,
\[ n = \left( \frac{V_{\text{gas}}}{V_{\text{liquid}}} \right)^{2}  \cdot \left( \frac{1}{36 \pi} \right) \cdot \frac{1}{\ell^3} \]

\section{Problem 2}

The classical treatment of cyclotron radiation is valid for non-relativistic velocities and orbital angular momentum $\ell$ much greater than $\hbar$.


\subsection*{(a)}

In the classical treatment we have an acceleration,
\[ a = \frac{v_\perp^2}{r} = \frac{e v_\perp}{mc} B \] 
and thus,
\[ r = \frac{mc v_\perp}{e B} \]
Furthermore, the orbital angular momentum is then,
\[ L = m v_\perp r = \frac{m^2 v_\perp^2 c}{e B} \] 
Therefore,
\[ B = \frac{m^2 v_\perp^2 c}{e L} \]
Now we must have $v_\perp \ll c$ and $L \gg \hbar$ for the classical approximation to be valid. This implies that for the classical approximation to hold we must have,
\[ B \ll B_{\text{max}} = \frac{m^2 c^3}{e \hbar} = \frac{e}{r_0 \lambdabar_e} = 4.9 \cdot 10^{13} \: \text{esu} \: \text{cm}^{-2} \]

\subsection*{(b)}

The frequency of the cyclotron radiation is,
\[ \omega = \frac{v_\perp}{r} = \frac{e B}{mc} \]
Now let,
\[ B = \frac{1}{10} B_{\text{max}} = \frac{1}{10} \cdot \frac{m^2 c^3}{e \hbar} \]
and thus,
\[ \omega = \frac{1}{10} \cdot \frac{m c^2}{\hbar} = \frac{1}{10} \cdot \frac{c}{\lambdabar_e} = 1.2 \cdot 10^{19} \: \text{Hz} \]

\section{Problem 3}

We know that, the spontaneous emission rate is given by radiation damping,
\[ A_{21} = 3 \Gamma = \frac{8 \pi^2 e^2}{m c \lambda_0^2} \: f_{21} =  \frac{8 \pi^2 r_0 c}{\lambda_0^2} \: f_{21} \quad \quad \quad g_1 f_{12} = g_2 f_{21} \]
given,
\[ r_0 = 2.82 \cdot 10^{-13} \: \mathrm{cm} \]
so we have,
\[ \Gamma = 2.23 \: \mathrm{GHz} \: \left( \frac{\lambda_0}{1000 \: \mathrm{\r{A}} } \right)^{-2} \: f_{21} \]
In the case of Ly$\alpha$ transition $n = 2 \mapsto 1$ we have $\lambda_0 = 1215.7 \: \mathrm{\r{A}}$ and $g_2 = 6$ and $g_1 = 2$ since there are three oribital angular momentum states at level $n = 2$. Thus,
\[ A_{21} = 6.26 \cdot 10^8 \: \mathrm{s}^{-1} \]
Furthermore,
\begin{align*}
B_{21} & = \frac{c^2}{2 h \nu_0^3} A_{21} = \frac{\lambda_0^3}{2 h c} A_{21}
\\
& = \frac{4 \pi^2 r_0}{h} \lambda_0 f_{21} = (2.33 \cdot 10^9 \: \mathrm{cm}^2 \: \mathrm{s} \: \mathrm{Hz} \: \mathrm{sr} \: \mathrm{erg}^{-1} \: \mathrm{s}^{-1}) \left( \frac{\lambda_0}{1000 \mathrm{\r{A}}} \right) 
\end{align*}
Therefore, in our case,
\[ B_{21} = 2.83 \cdot 10^9 \: \mathrm{cm}^2 \: \mathrm{s} \: \mathrm{Hz} \: \mathrm{sr} \: \mathrm{erg}^{-1} \: \mathrm{s}^{-1} \]
and finally,
\[ B_{12} = B_{21} \frac{g_2}{g_1} = 3 B_{21} = (7.00 \cdot 10^9 \: \mathrm{cm}^2 \: \mathrm{s} \: \mathrm{Hz} \: \mathrm{sr} \: \mathrm{erg}^{-1} \: \mathrm{s}^{-1}) \left( \frac{\lambda_0}{1000 \mathrm{\r{A}}} \right)  \]
so in the case of Ly$\alpha$ we have,
\[ B_{12} = 8.49 \cdot 10^8 \: \mathrm{cm}^2 \: \mathrm{s} \: \mathrm{Hz} \: \mathrm{sr} \: \mathrm{erg}^{-1} \: \mathrm{s}^{-1}  \]
\bigskip\\
Consider a Hdrogenic atom of atomic number $Z$. For the $n = 2 \mapsto 1$ transition we have an analogous line. The oscillator stengths $f_{12}$ and $f_{21}$ and the multiplicities $g_1$ and $g_2$ are determined by the geometry and quantum numbers which are unchanged. The only change in the above expressions is the energy difference and thus the wavelength $\lambda_0$. Now since the energy levels are,
\[ E_n = - \frac{m e^4 Z^2}{2 \hbar^2} \cdot \frac{1}{n^2} \] 
and thus we have,
\[ \lambda_0 = \frac{h c}{E_2 - E_1} \propto Z^{-2} \]
Therefore, we have,
\begin{align*}
A_{21} & = (6.26 \cdot 10^8 \: \mathrm{s}^{-1}) \: Z^{4}
\\
B_{21} & = (2.83 \cdot 10^9 \: \mathrm{cm}^2 \: \mathrm{s} \: \mathrm{Hz} \: \mathrm{sr} \: \mathrm{erg}^{-1} \: \mathrm{s}^{-1}) \: Z^{-2}
\\
B_{12} & = (8.49 \cdot 10^8 \: \mathrm{cm}^2 \: \mathrm{s} \: \mathrm{Hz} \: \mathrm{sr} \: \mathrm{erg}^{-1} \: \mathrm{s}^{-1}) \: Z^{-2}
\end{align*}

\section{Problem 4}

The cross section is given by,
\[ \sigma(\nu) = \frac{\pi e^2}{mc} \: f_{12} \: \phi(\nu) = \pi r_0 c \: f_{12} \: \phi(\nu)  \]
where $\phi(\nu)$ is the line width computed as a convolution of the Lorentizan natural line profile with the Gaussian dopler brodening,
\begin{align*}
\phi_{\text{nat}}(\nu) & = \frac{ \frac{\Gamma}{4 \pi^2}}{(\nu - \nu_0)^2 + \left( \frac{\Gamma}{4 \pi} \right)^2}
\\
\phi_D(\nu) & = \frac{1}{\Delta \nu \sqrt{\pi}} e^{- \left( \frac{\nu - \nu_0}{\Delta \nu} \right)^2}
\end{align*}
where,
\[ \Gamma = \frac{8 \pi^2 e^2}{3 m c \lambda_0^2} \: f_{21} =  \frac{8 \pi^2 r_0 c}{3 \lambda_0^2} \: f_{21} \quad \quad \quad \Delta \nu = \frac{b \nu_0}{c} = \sqrt{\frac{2 k T}{m_H}} \left( \frac{\nu_0}{c} \right) \quad \quad \quad g_1 f_{12} = g_2 f_{21} \]
Then,
\[ \phi = \phi_{\text{nat}} * \phi_D \]
First, for $T = 0$ we have $\Delta \nu \to 0$ and thus $\phi = \phi_{\text{nat}}$. In particular,
\[ \phi(\nu_0) = \phi_{\text{nat}}(\nu_0) = \frac{4}{\Gamma} \]
Therefore,
\[ \sigma(\nu_0) = \frac{4 \pi r_0 c}{\Gamma} f_{12} = \frac{3 \lambda_0^2}{2 \pi}  \cdot \frac{f_{12}}{f_{21}} = \frac{3 \lambda_0^2}{2 \pi} \cdot \frac{g_2}{g_1} \]
In our case, we have $\lambda_0 = 1215.7 \: \mathrm{\r{A}}$. For the $n = 2 \mapsto n = 1$ transition, there are $4$ states at $n = 2$ and $2$ at $n = 1$ so $g_2 / g_1 = 3$. Therefore,
\[ \sigma(\nu_0) = 2.12 \cdot 10^{-10} \: \mathrm{cm}^2 \]
\bigskip\\
For $T > 0$ we need to compute the convolution, 
\[ \phi = \phi_{\text{nat}} * \phi_D \]
and thus,
\begin{align*}
\phi(\nu_0) & = \frac{4 \pi}{\pi \Gamma \Delta \nu \sqrt{\pi}} \int_{-\infty}^{\infty} \frac{e^{- \left( \frac{\nu}{\Delta \nu} \right)^2}}{1 + \left( \frac{4 \pi \nu}{\Gamma} \right)^2} \d{\nu} = \frac{\sqrt{\pi}}{\Delta \nu} e^{\left( \frac{\Gamma}{4 \pi \Delta \nu} \right)^2} \left( 1 - \mathrm{erf}\left[ \frac{\Gamma}{4 \pi \Delta \nu}\right] \right)
\\
& = \frac{1}{\Delta \nu \sqrt{\pi}} \left( 1 - \frac{2}{\sqrt{\pi}}  \left[  \frac{\Gamma}{4 \pi \Delta \nu}\right] +  \left[  \frac{\Gamma}{4 \pi \Delta \nu}\right]^2 + O \left( \left[  \frac{\Gamma}{4 \pi \Delta \nu}\right]^3 \right) \right)
\end{align*}
Now we know that,
\[ r_0 = 2.82 \cdot 10^{-13} \: \mathrm{cm} \]
Therefore,
\[ \Gamma = 2.23 \: \mathrm{GHz} \: \left( \frac{\lambda_0}{1000 \: \mathrm{\r{A}} } \right)^{-2} \: f_{21} \quad \quad \quad \Delta \nu = 12.8 \: \mathrm{GHz} \: \left( \frac{T}{100 \: \mathrm{K}} \right)^{\frac{1}{2}} \cdot \left( \frac{\lambda_0}{1000 \: \mathrm{\r{A}} } \right)^{-1} \]
and so we have,
\[ y = \frac{\Gamma}{4 \pi \Delta \nu} = 1.39 \cdot 10^{-3} \cdot \left( \frac{T}{100 \: \mathrm{K}} \right)^{-\frac{1}{2}} \cdot \left( \frac{\lambda_0}{1000 \: \mathrm{\r{A}} } \right)^{-1} \: f_{21} \]
Now,
\[ \sigma(\nu_0) = \frac{\sqrt{\pi} r_0 c \: f_{12}}{\Delta \nu} \: \phi(\nu_0) = \frac{\sqrt{\pi} r_0 c \: f_{12}}{\Delta \nu} \left( 1 - \frac{2}{\sqrt{\pi}} \sqrt{\pi} \: y + y^2 + O(y^3) \right) \]
Plugging in,
\[ \sigma(\nu_0) = 1.17 \cdot 10^{-12} \: \mathrm{cm}^2 \: \left( \frac{T}{100 \: \mathrm{K}} \right)^{-\frac{1}{2}} \cdot \left( \frac{\lambda_0}{1000 \: \mathrm{\r{A}} } \right) f_{12} \left( 1 - \frac{2}{\sqrt{\pi}} \sqrt{\pi} \: y +  y^2 + O(y^3) \right) \]
In our case, 
\begin{align*}
\lambda_0 & = 1215.7 \: \mathrm{\r{A}} 
\\
T & = 1000 \: \mathrm{K}
\\
f_{12} & = 0.4216
\\
f_{21} & = 0.1388
\end{align*}
Therefore,
\[ y = 4.96 \cdot 10^{-4} \]
and thus,
\[ \sigma(\nu_0) = 1.86 \cdot 10^{-13} \: \mathrm{cm}^2 \]
\end{document}