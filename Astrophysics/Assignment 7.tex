\documentclass[12pt]{article}
\usepackage[english]{babel}
\usepackage[utf8]{inputenc}
\usepackage[english]{babel}
\usepackage[a4paper, total={7.25in, 9.5in}]{geometry}
\usepackage{tikz-feynman}
\tikzfeynmanset{compat=1.0.0} 
\usepackage{subcaption}
\usepackage{float}
\floatplacement{figure}{H}
\usepackage{mathrsfs}  
\usepackage{dsfont}
\usepackage{relsize}
\DeclareMathAlphabet{\mathdutchcal}{U}{dutchcal}{m}{n}

\usepackage{revsymb}


\newcommand{\field}{\hat{\Phi}}
\newcommand{\dfield}{\hat{\Phi}^\dagger}
 
\usepackage{amsthm, amssymb, amsmath, centernot}
\usepackage{slashed}
\newcommand{\notimplies}{%
  \mathrel{{\ooalign{\hidewidth$\not\phantom{=}$\hidewidth\cr$\implies$}}}}
 
\renewcommand\qedsymbol{$\square$}
\newcommand{\cont}{$\boxtimes$}
\newcommand{\divides}{\mid}
\newcommand{\ndivides}{\centernot \mid}

\newcommand{\Integers}{\mathbb{Z}}
\newcommand{\Natural}{\mathbb{N}}
\newcommand{\Complex}{\mathbb{C}}
\newcommand{\Zplus}{\mathbb{Z}^{+}}
\newcommand{\Primes}{\mathbb{P}}
\newcommand{\Q}{\mathbb{Q}}
\newcommand{\R}{\mathbb{R}}
\newcommand{\ball}[2]{B_{#1} \! \left(#2 \right)}
\newcommand{\Rplus}{\mathbb{R}^+}
\renewcommand{\Re}[1]{\mathrm{Re}\left[ #1 \right]}
\renewcommand{\Im}[1]{\mathrm{Im}\left[ #1 \right]}
\newcommand{\Op}{\mathcal{O}}

\newcommand{\invI}[2]{#1^{-1} \left( #2 \right)}
\newcommand{\End}[1]{\text{End}\left( A \right)}
\newcommand{\legsym}[2]{\left(\frac{#1}{#2} \right)}
\renewcommand{\mod}[3]{\: #1 \equiv #2 \: \mathrm{mod} \: #3 \:}
\newcommand{\nmod}[3]{\: #1 \centernot \equiv #2 \: mod \: #3 \:}
\newcommand{\ndiv}{\hspace{-4pt}\not \divides \hspace{2pt}}
\newcommand{\finfield}[1]{\mathbb{F}_{#1}}
\newcommand{\finunits}[1]{\mathbb{F}_{#1}^{\times}}
\newcommand{\ord}[1]{\mathrm{ord}\! \left(#1 \right)}
\newcommand{\quadfield}[1]{\Q \small(\sqrt{#1} \small)}
\newcommand{\vspan}[1]{\mathrm{span}\! \left\{#1 \right\}}
\newcommand{\galgroup}[1]{Gal \small(#1 \small)}
\newcommand{\bra}[1]{\left| #1 \right>}
\newcommand{\Oa}{O_\alpha}
\newcommand{\Od}{O_\alpha^{\dagger}}
\newcommand{\Oap}{O_{\alpha '}}
\newcommand{\Odp}{O_{\alpha '}^{\dagger}}
\newcommand{\im}[1]{\mathrm{im} \: #1}
\renewcommand{\ker}[1]{\mathrm{ker} \: #1}
\newcommand{\ket}[1]{\left| #1 \right>}
\renewcommand{\bra}[1]{\left< #1 \right|}
\newcommand{\inner}[2]{\left< #1 | #2 \right>}
\newcommand{\expect}[2]{\left< #1 \right| #2 \left| #1 \right>}
\renewcommand{\d}[1]{\: \mathrm{d}#1 \:}
\newcommand{\dn}[2]{ \mathrm{d}^{#1} #2 \:}
\newcommand{\deriv}[2]{\frac{\d{#1}}{\d{#2}}}
\newcommand{\nderiv}[3]{\frac{\dn{#1}{#2}}{\d{#3^{#1}}}}
\newcommand{\pderiv}[2]{\frac{\partial{#1}}{\partial{#2}}}
\newcommand{\fderiv}[2]{\frac{\delta #1}{\delta #2}}
\newcommand{\parsq}[2]{\frac{\partial^2{#1}}{\partial{#2}^2}}
\newcommand{\topo}{\mathcal{T}}
\newcommand{\base}{\mathcal{B}}
\renewcommand{\bf}[1]{\mathbf{#1}}
\renewcommand{\a}{\hat{a}}
\newcommand{\adag}{\hat{a}^\dagger}
\renewcommand{\b}{\hat{b}}
\newcommand{\bdag}{\hat{b}^\dagger}
\renewcommand{\c}{\hat{c}}
\newcommand{\cdag}{\hat{c}^\dagger}
\newcommand{\hamilt}{\hat{H}}
\renewcommand{\L}{\hat{L}}
\newcommand{\Lz}{\hat{L}_z}
\newcommand{\Lsquared}{\hat{L}^2}
\renewcommand{\S}{\hat{S}}
\renewcommand{\empty}{\varnothing}
\newcommand{\J}{\hat{J}}
\newcommand{\lagrange}{\mathcal{L}}
\newcommand{\dfourx}{\mathrm{d}^4x}
\newcommand{\meson}{\phi}
\newcommand{\dpsi}{\psi^\dagger}
\newcommand{\ipic}{\mathrm{int}}
\newcommand{\tr}[1]{\mathrm{tr} \left( #1 \right)}
\newcommand{\C}{\mathbb{C}}
\newcommand{\CP}[1]{\mathbb{CP}^{#1}}
\newcommand{\Vol}[1]{\mathrm{Vol}\left(#1\right)}

\newcommand{\Tr}[1]{\mathrm{Tr}\left( #1 \right)}
\newcommand{\Charge}{\hat{\mathbf{C}}}
\newcommand{\Parity}{\hat{\mathbf{P}}}
\newcommand{\Time}{\hat{\mathbf{T}}}
\newcommand{\Torder}[1]{\mathbf{T}\left[ #1 \right]}
\newcommand{\Norder}[1]{\mathbf{N}\left[ #1 \right]}
\newcommand{\Znorm}{\mathcal{Z}}
\newcommand{\EV}[1]{\left< #1 \right>}
\newcommand{\interact}{\mathrm{int}}
\newcommand{\covD}{\mathcal{D}}
\newcommand{\conj}[1]{\overline{#1}}

\newcommand{\SO}[2]{\mathrm{SO}(#1, #2)}
\newcommand{\SU}[2]{\mathrm{SU}(#1, #2)}

\newcommand{\anticom}[2]{\left\{ #1 , #2 \right\}}


\newcommand{\pathd}[1]{\! \mathdutchcal{D} #1 \:}

\renewcommand{\theenumi}{(\alph{enumi})}


\renewcommand{\theenumi}{(\alph{enumi})}

\newcommand{\atitle}[1]{\title{% 
	\large \textbf{ASTR GR6001 Radiative Processes
	\\ Assignment \# #1} \vspace{-2ex}}
\author{Benjamin Church }
\maketitle}

\theoremstyle{definition}
\newtheorem{theorem}{Theorem}[section]
\newtheorem{definition}{definition}[section]
\newtheorem{lemma}[theorem]{Lemma}
\newtheorem{proposition}[theorem]{Proposition}
\newtheorem{corollary}[theorem]{Corollary}
\newtheorem{example}[theorem]{Example}
\newtheorem{remark}[theorem]{Remark}
\begin{document}


\atitle{7}

\section{Problem 1}

Consider a mirror moving with velocity $v$ (let $\beta = \frac{v}{c}$) in the $x$-direction and a photon with frequency $\nu_0$ incident upon it at an angle $\theta$ is the Lab frame. The photon has four momentum,
\[ p^\mu = \frac{h \nu_0}{c}
\begin{pmatrix}
1
\\
-\cos{\theta}
\\
- \sin{\theta}
\\
0
\end{pmatrix} \]
Then we transform into the mirror frame via Lorentz boost by $\beta$ along $x$ to find,
\[ p'^\mu = \frac{h \nu_0}{c} 
\begin{pmatrix}
\gamma(1 + \beta \cos{\theta})
\\
-\gamma(\cos{\theta} + \beta)
\\
- \sin{\theta}
\\
0
\end{pmatrix}
\] 
Now the ray is reflected which, in the mirror frame, simply ammounts to reversing its $x$-momentum to give,
\[ p'^\mu_{\text{ref}} = \frac{h \nu_0}{c} 
\begin{pmatrix}
\gamma(1 + \beta \cos{\theta})
\\
\gamma(\cos{\theta} + \beta)
\\
- \sin{\theta}
\\
0
\end{pmatrix}
\] 
Now to get back to the lab frame we perform a Lorentz transformation in the opposite direction by $\beta$ to get,
\[ p^\mu_{\text{ref}} = \frac{h \nu_0}{c} 
\begin{pmatrix}
\gamma^2(1 + 2 \beta \cos{\theta} + \beta^2)
\\
\gamma^2(\cos{\theta} + 2 \beta + \beta^2 \cos{\theta})
\\
- \sin{\theta}
\\
0
\end{pmatrix}
\] 
Therefore, we find,
\[ \nu_{1} = \nu_0 \cdot \frac{1 + 2 \beta \cos{\theta} + \beta^2}{1 - \beta^2} \]
and furthermore,
\[ \cos{\theta_1} = \frac{(1 + \beta^2)\cos{\theta} + 2 \beta}{1 + \beta^2 + 2 \beta \cos{\theta}} \]
For small $\beta$ we can expand these formulae to first order,
\begin{align*}
\nu_1 & = \nu_0 (1 + 2 \beta \cos{\theta} + O(\beta^2)) 
\\
\cos{\theta_1} & = \cos{\theta} + 2 \beta \left( 1 - \cos{\theta} \right) + O(\beta^2)  
\end{align*}
Thus, since $1 - \cos{\theta}$ is positive, the reflection will increase $\cos{\theta}$ and thus decrease the angle between the photon ray and the normal.

\section{Problem 2}

\subsection*{(a)}

Consider the source moving between two events at which it emitts a light pulse. The time between, in the lab frame, is $\Delta t$ and in this time the source moves $(v \Delta t \cos{\theta}, v \Delta t \sin{\theta})$. When these two light pulses reach the observer, it will observe the sources apparent motion of $v \Delta t \sin{\theta}$ (the part perpendicular to the line of sight). However, the time between these pulses is not $\Delta t$ but rather the difference in light-travel times between the events plus the time delay between them. Let the first event be at a large distance $d$ from the observer then,
\[ t_0 = \frac{d}{c} \quad \quad \quad t_1 = \frac{\sqrt{(d - v \Delta t \cos{\theta})^2 + (v \Delta t \sin{\theta})^2}}{c} + \Delta t \]
Now, to first-order in $\Delta t$,
\begin{align*}
\Delta t_r = t_1 - t_0 & = \Delta t + \frac{d}{c} \left( \sqrt{\left[ 1 - \frac{ v \Delta t \cos{\theta}}{d} \right]^2 + \left[ \frac{v \Delta t \sin{\theta}}{d} \right]^2} - 1 \right)
\\
& = \Delta t + \frac{d}{c} \left( - \frac{v \Delta t \cos{\theta}}{d} \right) = \Delta t ( 1 - \beta \cos{\theta}) 
\end{align*}
Therefore,
\[ v_{\perp \text{app}} = \frac{r_{\text{app}}}{\Delta t_r} = \frac{v \sin{\theta}}{1 - \beta \cos{\theta}} \] 

\subsection*{(b)}

We have the apparent velocity,
\[ v_{\perp \text{app}} = \frac{v \sin{\theta}}{1 - \beta \cos{\theta}} \]
To maximize this take the derivative with respect to $\theta$,
\begin{align*}
\frac{v_{\perp \text{app}}'}{c} & = \frac{\beta \cos{\theta}}{1 - \beta \cos{\theta}} - \frac{\beta^2 \sin^2{\theta}}{(1 - \beta \cos{\theta})^2}
\\
& = \frac{\beta}{1 - \beta \cos{\theta}} \left( \cos{\theta} - \frac{\beta (1 - \cos^2{\theta})}{1 - \beta \cos{\theta}} \right)
\end{align*}
Therefore, we need to solve,
\[ \frac{\beta (1 - \cos^2{\theta})}{1 - \beta \cos{\theta}} = \cos{\theta} \]
which is equivalent to,
\[ \cos{\theta} = \beta \] 
Therefore, plugging in we find,
\[ v_{\perp \text{app, max}} = \frac{v \sqrt{1 - \beta^2}}{1 - \beta^2} = \frac{v}{\sqrt{1 - \beta^2}} = \gamma v \]

\section{Problem 3}

A quasar ejcts a pair of blobs from its core in opposite directions at equal speed $v$ and an angle $\theta$ from th line of sight. 

\subsection*{(a)}

Suppose that the distance $d$ to the source is known and that the proper motion,
\[ p = \frac{v_{\perp \text{app}}}{d} \]
is measured for each of the blobs. Multiplying by $d$ gives a measurement of both approaching and receeding apparent velocities,
\[ v_{\perp \text{app}}^{+} = \frac{v \sin{\theta}}{1 - \beta \cos{\theta}} \quad \quad  v_{\perp \text{app}}^{-} = \frac{v \sin{\theta}}{1 + \beta \cos{\theta}}  \] 
(where I choose positive motion as moving away from the Quasar for both jets).
Here we have two equations are two unknowns so we can solve for $\theta$ and $\beta$ as follows. Consider,
\[ - \frac{1}{v_{\perp \text{app}}^{+}} + \frac{1}{v_{\perp \text{app}}^{-}} = - \frac{1 - \beta \cos{\theta}}{v \sin{\theta}} + \frac{1 + \beta \cos{\theta}}{v \sin{\theta}} =  \frac{2 \beta \cos{\theta}}{v \sin{\theta}} =  \frac{2}{c \tan{\theta}} \]
Therefore,
\[ \tan{\theta} = \frac{2}{c} \cdot \frac{v_{\perp \text{app}}^{+} v_{\perp \text{app}}^{-}}{v_{\perp \text{app}}^{+} - v_{\perp \text{app}}^{-}} \]
Furthermore,
\[ \frac{1}{v_{\perp \text{app}}^{+}} + \frac{1}{v_{\perp \text{app}}^{-}} = \frac{1 - \beta \cos{\theta}}{v \sin{\theta}} + \frac{1 + \beta \cos{\theta}}{v \sin{\theta}} =  \frac{2}{v \sin{\theta}} \]
Therefore,
\[ v = \frac{2}{\sin{\theta}} \cdot \frac{v_{\perp \text{app}}^{+} v_{\perp \text{app}}^{-}}{v_{\perp \text{app}}^{+} + v_{\perp \text{app}}^{-}} \]
where,
\[ \sin{\theta} = \frac{\tan{\theta}}{\sqrt{1 + \tan^2{\theta}}} \]
and 
\[ \tan{\theta} = \frac{2}{c} \cdot \frac{v_{\perp \text{app}}^{+} v_{\perp \text{app}}^{-}}{v_{\perp \text{app}}^{+} - v_{\perp \text{app}}^{-}} \]
Therefore, we find,
\[ v = c \cdot \frac{v_{\perp \text{app}}^{+} - v_{\perp \text{app}}^{-}}{v_{\perp \text{app}}^{+} + v_{\perp \text{app}}^{-}} \cdot \left( 1 + \frac{4}{c^2} \left[ \frac{v_{\perp \text{app}}^{+} v_{\perp \text{app}}^{-}}{v_{\perp \text{app}}^{+} - v_{\perp \text{app}}^{-}} \right]^2 \right)^{\frac{1}{2}} \]

\subsection*{(b)}

Suppose we can measure the Doppler-shifted emission spectrum of the blobs to find a formula for the Doppler shift,
\[ 1 + z_{\pm} = \frac{\nu_1}{\nu_0} = \frac{1}{\gamma(1 \mp \beta \cos{\theta})} \]
(here I have written a formula for the redshift of each object but we shall see that we only need one of the two measurements.)
Now notice from the fomula we derived for the angle,
Notice that dividing our forumlas for $\tan{\theta}$ and $v \sin{\theta}$ we get,
\[ v \cos{\theta} = c \cdot \frac{v_{\perp \text{app}}^{+} - v_{\perp \text{app}}^{-}}{v_{\perp \text{app}}^{+} + v_{\perp \text{app}}^{-}} \]
is homogeneous in the apparent velocities so we may divide through by $d$ to get a formula just in terms of the directly measureable proper motion,
\[ v \cos{\theta} = c \cdot \frac{p^{+} - p^{-}}{p^{+} + p^{-}} \]
Then we can solve for $\gamma$ using the redshift formula,
\[ \gamma = \frac{1}{1 \mp \beta \cos{\theta}} \cdot \frac{1}{1 + z_{\pm}} = \frac{p^{+} + p^{-}}{2 p^{\mp}} \cdot \frac{1}{1 + z_{\pm}} \]
Thus we only need one redshift measurement to determine $\gamma$ and thus determine $v$ via,
\[ v = c \sqrt{1 - \gamma^{-2}} = c \: \sqrt{1 - (1 + z_{\pm})^2 \left( \frac{2 p^{\mp}}{p^{+} + p^{-}} \right)^2} \]
Combining with our above formula gives a formula for the angle via,
\[ \cos{\theta} = \frac{p^{+} - p^{-}}{p^{+} + p^{-}} \cdot \left( 1 - (1 + z_{\pm})^2 \left( \frac{2 p^{\mp}}{p^{+} + p^{-}} \right)^2 \right)^{-\frac{1}{2}} \]
Now, at last, we get a distance measurement via,
\[ d = \frac{ v_{\perp \text{app}}^{\pm}}{p^{\pm}} = \frac{v \sin{\theta}}{1 \mp \beta \cos{\theta}} \cdot \frac{1}{p^{\pm}} \]
Now recall that,
\[ 1 \mp \beta \cos{\theta} = \frac{2 p^{\mp}}{p^{+} + p^{-}} \]
and,
\begin{align*}
v \sin{\theta} & = \sqrt{v^2 - (v \cos{\theta})^2} = c \sqrt{ \left( 1 - (1 + z_{\pm})^2 \left( \frac{2 p^{\mp}}{p^{+} + p^{-}} \right)^2 \right) - \left( \frac{p^{+} - p^{-}}{p^{+} + p^{-}} \right)^2}
\end{align*}
Therefore,
\begin{align*}
d & = \frac{c}{p^{\pm}} \cdot \sqrt{ \left( \left( \frac{p^{+} + p^{-}}{2 p^{\mp}} \right)^2  - (1 + z_{\pm})^2 \right) - \left( \frac{p^{+} - p^{-}}{2 p^{\mp}} \right)^2} 
\\
& = \frac{c}{p^{\pm}} \sqrt{ \frac{p^{+} p^{-}}{(p^{\mp})^2} - (1 + z_{\pm})^2 }
\\
& = \frac{c}{p^{\pm}} \sqrt{ \frac{p^{\pm}}{p^{\mp}} - (1 + z_{\pm})^2 }
\end{align*}
Gives a formula for the distance. Notice, we need both proper motion measurements but only a measurement of one redshift then the other will be determined. 
\end{document}