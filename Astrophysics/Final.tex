\documentclass[12pt]{article}
\usepackage[english]{babel}
\usepackage[utf8]{inputenc}
\usepackage[english]{babel}
\usepackage[a4paper, total={7.25in, 9.5in}]{geometry}
\usepackage{tikz-feynman}
\tikzfeynmanset{compat=1.0.0} 
\usepackage{subcaption}
\usepackage{float}
\floatplacement{figure}{H}
\usepackage{mathrsfs}  
\usepackage{dsfont}
\usepackage{relsize}
\DeclareMathAlphabet{\mathdutchcal}{U}{dutchcal}{m}{n}

\usepackage{revsymb}


\newcommand{\field}{\hat{\Phi}}
\newcommand{\dfield}{\hat{\Phi}^\dagger}
 
\usepackage{amsthm, amssymb, amsmath, centernot}
\usepackage{slashed}
\newcommand{\notimplies}{%
  \mathrel{{\ooalign{\hidewidth$\not\phantom{=}$\hidewidth\cr$\implies$}}}}
 
\renewcommand\qedsymbol{$\square$}
\newcommand{\cont}{$\boxtimes$}
\newcommand{\divides}{\mid}
\newcommand{\ndivides}{\centernot \mid}

\newcommand{\Integers}{\mathbb{Z}}
\newcommand{\Natural}{\mathbb{N}}
\newcommand{\Complex}{\mathbb{C}}
\newcommand{\Zplus}{\mathbb{Z}^{+}}
\newcommand{\Primes}{\mathbb{P}}
\newcommand{\Q}{\mathbb{Q}}
\newcommand{\R}{\mathbb{R}}
\newcommand{\ball}[2]{B_{#1} \! \left(#2 \right)}
\newcommand{\Rplus}{\mathbb{R}^+}
\renewcommand{\Re}[1]{\mathrm{Re}\left[ #1 \right]}
\renewcommand{\Im}[1]{\mathrm{Im}\left[ #1 \right]}
\newcommand{\Op}{\mathcal{O}}

\newcommand{\invI}[2]{#1^{-1} \left( #2 \right)}
\newcommand{\End}[1]{\text{End}\left( A \right)}
\newcommand{\legsym}[2]{\left(\frac{#1}{#2} \right)}
\renewcommand{\mod}[3]{\: #1 \equiv #2 \: \mathrm{mod} \: #3 \:}
\newcommand{\nmod}[3]{\: #1 \centernot \equiv #2 \: mod \: #3 \:}
\newcommand{\ndiv}{\hspace{-4pt}\not \divides \hspace{2pt}}
\newcommand{\finfield}[1]{\mathbb{F}_{#1}}
\newcommand{\finunits}[1]{\mathbb{F}_{#1}^{\times}}
\newcommand{\ord}[1]{\mathrm{ord}\! \left(#1 \right)}
\newcommand{\quadfield}[1]{\Q \small(\sqrt{#1} \small)}
\newcommand{\vspan}[1]{\mathrm{span}\! \left\{#1 \right\}}
\newcommand{\galgroup}[1]{Gal \small(#1 \small)}
\newcommand{\bra}[1]{\left| #1 \right>}
\newcommand{\Oa}{O_\alpha}
\newcommand{\Od}{O_\alpha^{\dagger}}
\newcommand{\Oap}{O_{\alpha '}}
\newcommand{\Odp}{O_{\alpha '}^{\dagger}}
\newcommand{\im}[1]{\mathrm{im} \: #1}
\renewcommand{\ker}[1]{\mathrm{ker} \: #1}
\newcommand{\ket}[1]{\left| #1 \right>}
\renewcommand{\bra}[1]{\left< #1 \right|}
\newcommand{\inner}[2]{\left< #1 | #2 \right>}
\newcommand{\expect}[2]{\left< #1 \right| #2 \left| #1 \right>}
\renewcommand{\d}[1]{\: \mathrm{d}#1 \:}
\newcommand{\dn}[2]{ \mathrm{d}^{#1} #2 \:}
\newcommand{\deriv}[2]{\frac{\d{#1}}{\d{#2}}}
\newcommand{\nderiv}[3]{\frac{\dn{#1}{#2}}{\d{#3^{#1}}}}
\newcommand{\pderiv}[2]{\frac{\partial{#1}}{\partial{#2}}}
\newcommand{\fderiv}[2]{\frac{\delta #1}{\delta #2}}
\newcommand{\parsq}[2]{\frac{\partial^2{#1}}{\partial{#2}^2}}
\newcommand{\topo}{\mathcal{T}}
\newcommand{\base}{\mathcal{B}}
\renewcommand{\bf}[1]{\mathbf{#1}}
\renewcommand{\a}{\hat{a}}
\newcommand{\adag}{\hat{a}^\dagger}
\renewcommand{\b}{\hat{b}}
\newcommand{\bdag}{\hat{b}^\dagger}
\renewcommand{\c}{\hat{c}}
\newcommand{\cdag}{\hat{c}^\dagger}
\newcommand{\hamilt}{\hat{H}}
\renewcommand{\L}{\hat{L}}
\newcommand{\Lz}{\hat{L}_z}
\newcommand{\Lsquared}{\hat{L}^2}
\renewcommand{\S}{\hat{S}}
\renewcommand{\empty}{\varnothing}
\newcommand{\J}{\hat{J}}
\newcommand{\lagrange}{\mathcal{L}}
\newcommand{\dfourx}{\mathrm{d}^4x}
\newcommand{\meson}{\phi}
\newcommand{\dpsi}{\psi^\dagger}
\newcommand{\ipic}{\mathrm{int}}
\newcommand{\tr}[1]{\mathrm{tr} \left( #1 \right)}
\newcommand{\C}{\mathbb{C}}
\newcommand{\CP}[1]{\mathbb{CP}^{#1}}
\newcommand{\Vol}[1]{\mathrm{Vol}\left(#1\right)}

\newcommand{\Tr}[1]{\mathrm{Tr}\left( #1 \right)}
\newcommand{\Charge}{\hat{\mathbf{C}}}
\newcommand{\Parity}{\hat{\mathbf{P}}}
\newcommand{\Time}{\hat{\mathbf{T}}}
\newcommand{\Torder}[1]{\mathbf{T}\left[ #1 \right]}
\newcommand{\Norder}[1]{\mathbf{N}\left[ #1 \right]}
\newcommand{\Znorm}{\mathcal{Z}}
\newcommand{\EV}[1]{\left< #1 \right>}
\newcommand{\interact}{\mathrm{int}}
\newcommand{\covD}{\mathcal{D}}
\newcommand{\conj}[1]{\overline{#1}}

\newcommand{\SO}[2]{\mathrm{SO}(#1, #2)}
\newcommand{\SU}[2]{\mathrm{SU}(#1, #2)}

\newcommand{\anticom}[2]{\left\{ #1 , #2 \right\}}


\newcommand{\pathd}[1]{\! \mathdutchcal{D} #1 \:}

\renewcommand{\theenumi}{(\alph{enumi})}


\renewcommand{\theenumi}{(\alph{enumi})}

\newcommand{\atitle}[1]{\title{% 
	\large \textbf{ASTR GR6001 Radiative Processes
	\\ Assignment \# #1} \vspace{-2ex}}
\author{Benjamin Church }
\maketitle}

\theoremstyle{definition}
\newtheorem{theorem}{Theorem}[section]
\newtheorem{definition}{definition}[section]
\newtheorem{lemma}[theorem]{Lemma}
\newtheorem{proposition}[theorem]{Proposition}
\newtheorem{corollary}[theorem]{Corollary}
\newtheorem{example}[theorem]{Example}
\newtheorem{remark}[theorem]{Remark}
\begin{document}


\title{
	\large \textbf{ASTR GR6001 Radiative Processes
	\\ Final Exam}}
\author{Benjamin Church }

\maketitle

\newcommand{\DM}{\mathrm{DM}}
\newcommand{\RM}{\mathrm{RM}}
\newcommand{\cm}{\mathrm{cm}}
\newcommand{\pc}{\mathrm{pc}}

\section{Problem 1}

Consider when the following radiative transfer relations and approximations are valid:

\begin{enumerate}
\item $j_\nu = \alpha_\nu S_\nu$ this is the definition of $S_\nu$
\item $j_\nu = \alpha_\nu I_\nu$ this holds in the limit of large optical depth since,
\[ \deriv{I_\nu}{s} = - \alpha_\nu (I_\nu - S_\nu) \]
Gives $I_\nu \to S_\nu$ in the limit as $\tau_\nu \to \infty$. 
\item $j_\nu = \alpha_\nu B_\nu$ when the body is emitting in local thermodynamic equilibrium.
\item $S_\nu = J_\nu$ holds when we assume first that the source function $S_\nu$ is isotropic and second that the medium is in \textit{global} thermodynamic equilibrium such that the specific flux everywhere vanishes (there is no net energy transport in any frequency band). This follows from the first moment equation,
\[ \deriv{}{\tau_\nu} F_\nu = 4 \pi (J_\nu - S_\nu) \]
which assumes that $S_\nu$ is isotropic. 
\item $P_\nu = u_\nu  / 3$ this is the Eddington approximation which is valid when the specific intensity can be approximated by a linear function in $\mu = \cos{\theta}$,
\[ I_\nu(\mu) = a_\nu + b_\nu \mu \]

\end{enumerate}

\section{Problem 2}

Consider the two-stream approximation, with stream intensities,
\[ I_\nu^{\pm}(\tau) = J_\nu(\tau) \pm \frac{1}{\sqrt{3}} \pderiv{J_\nu}{\tau_\nu} \]
where the two streams are at angles $\mu = \pm \frac{1}{\sqrt{3}}$. From the fixed angular distribution, we can easily compute the moments inside our medium,
\begin{align*}
J_\nu &= \frac{1}{4 \pi} \int I_\nu \d{\Omega} = \tfrac{1}{2} (I_\nu^+ + I_\nu^-)
\\
F_\nu & = \int \mu I_\nu \d{\Omega}= \frac{2 \pi}{\sqrt{3}} (I_\nu^+ - I_\nu^-) = \frac{4 \pi}{3} \deriv{J_\nu}{\tau_\nu} 
\\
P_\nu & = \frac{1}{c} \int \mu^2 I_\mu \d{\Omega} = \frac{2 \pi}{3c} (I_\nu^+ + I_\nu^-) = \frac{4 \pi}{3c} J_\nu
\end{align*} 
Therefore, we see immediately that $P_\nu = \frac{1}{3} u_\nu$ and thus our two-stream approximation satisfies the Eddington approximation. 

\section{Problem 3}

Consider a semi-infinite atmosphere with absorption coefficient $\alpha_\nu$ and scattering coefficients $\sigma_\nu$. We further assume that the atmosphere has uni is in local thermodynamic equilibrium at a uniform temperature $T$. Finally I will also assume that the scattering is approximately isotropic and that the coefficients for scattering and absorption are uniform since the atmosphere has uniform composition.

\subsection*{(a)}

The radiative diffusion equation for absorption and scattering under the Eddington approximation gives,
\[ \frac{1}{3} \nderiv{2}{J_\nu}{\tau_\nu} = \epsilon_\nu (J_\nu - B_\nu) \]
where,
\[ \epsilon_\nu = \left( \frac{\alpha_\nu}{\sigma_\nu + \alpha_\nu} \right) \]
Since the blackbody spectrum $B_\nu(T)$ is isotropic and a function only of $T$ which we assume is uniform, then we may take $B_\nu$ to be a constant. Furthermore since $\alpha_\nu$ and $\sigma_\nu$ are constant we may solve this differential equation as follows. Consider the difference $D_\nu = B_\nu - J_\nu$. Using the homogeneity of $B_\nu$,
\[ \nderiv{2}{D_\nu}{\tau_\nu} = - \nderiv{2}{J_\nu}{\tau_\nu} = - 3 \epsilon_\nu (J_\nu - B_\nu) = 3 \epsilon_\nu D_\nu \]
This has solutions,
\[ D_\nu = C_1 e^{\sqrt{3 \epsilon_\nu} \tau_\nu} + C_2 e^{-\sqrt{3 \epsilon_\nu} \tau_\nu} \]
Since the atmosphere is semi-infinite we must have $C_1 = 0$ since otherwise in the limit $\tau_\nu \to \infty$ we would have $J_\nu \to \infty$ which is unphysical. We simply need to consider the boundary conditions at $\tau_\nu = 0$ to get a full solution. 
\bigskip\\
Outside the atmosphere there may be an emitted ray $I_E$ but we assume there is no radiation incident on the atmosphere (else it can certainly be possible for the radiation intensity to exceed $B_\nu$).
At the surface $\tau_\nu = 0$ we impose continuity of the specific intensity so the outgoing intensities match, $I_E = I^+_\nu(0)$, and the incoming intensities match, $I^-_\nu(0) = 0$ since we assume there is no radiation incident on the atmosphere. Thus,
\[ I^{-}_\nu(0) = J_\nu(0) - \frac{1}{\sqrt{3}} \deriv{J_\nu}{\tau_\nu} \bigg|_{\tau_\nu = 0} = 0 \]
which implies that,
\[ J_\nu(0) = \frac{1}{\sqrt{3}} \deriv{J_\nu}{\tau_\nu} \bigg|_{\tau_\nu = 0} \]
Now we use our solution,
\[ J_\nu(\tau_\nu) = C_\nu e^{-\sqrt{3 \epsilon_\nu} \tau_\nu} + B_\nu \]
so we must match,
\[ J_\nu(0) = C_\nu + B_\nu = \frac{1}{\sqrt{3}}  \deriv{J_\nu}{\tau_\nu} \bigg|_{\tau_\nu = 0} = - \sqrt{\epsilon_\nu} C_\nu \]
which implies that,
\[ C_\nu = - \frac{B_\nu}{1 + \sqrt{\epsilon_\nu}} \]
In particular,
\[ D_\nu = B_
\nu - J_\nu = - C_\nu e^{-\sqrt{3 \epsilon_\nu} \tau_\nu} = \frac{B_\nu}{1 + \sqrt{\epsilon_\nu}} e^{-\sqrt{3 \epsilon_\nu} \tau_\nu}  \]
Since $D_\nu > 0$ for all $\tau_\nu$ we have demonstrated that $J_\nu < B_\nu$ at all optical depths and that $D_\nu \to 0$ as $\tau_\nu \to \infty$ (since $\sqrt{3 \epsilon_\nu} > 0$ for positive absorption coefficient $\alpha_\nu > 0$) thus $J_\nu \to B_\nu$ in the limit of large optical depth.  

\subsection*{(b)}

Note that from the previous solution, $D_\nu \sim e^{-\sqrt{3 \epsilon_\nu} \tau_\nu}$. Therefore, $J_\nu$ approaches $B_\nu$ on a thermalization scale of approximately,
\[ \tau_{\text{th}} = \frac{1}{\sqrt{3 \epsilon_\nu}} = \sqrt{\frac{\alpha_\nu + \sigma_\nu}{3 \alpha_\nu}} \]

\section{Problem 4}

Consider the ZnII $\lambda = 2026 \: \mathrm{\r{A}}$ line which has an oscillator strength of $f = 0.489$. Furthermore we have an effective Doppler parameter of $b = 115 \: \mathrm{km} \: \mathrm{s}^{-1}$.

\subsection*{(a)}

The detected ZnII line is within the linear part of the curve of growth and we can read off the combination, $N \lambda f = 10^{8.8} \: \mathrm{cm}^{-1}$ from position of the ZnII species on the curve of growth. Therefore, using the parameters,
\begin{align*}
\lambda & = 2026 \: \mathrm{\r{A}}
\\
f & = 0.489
\end{align*}
we get an estimate for the column density of,
\[ N = 6.37 \cdot 10^{13} \: \mathrm{cm}^{-2} \] 

\subsection*{(b)}

Assuming that the near-center line profile is dominated by Doppler broadening, the optical depth is,
\[ \tau_\nu = N \sigma(\nu) = \frac{\pi e^2 N}{mc} f_{12} \: \phi(\nu) = \frac{\lambda_0 e^2 N \sqrt{\pi}}{mc b} f_{12} e^{-\left[ \frac{c(\nu - \nu_0)}{b \nu_0} \right]^2} \]
Therefore, we get an expression for the optical depth at line center in terms of the measured column density,
\[ \tau_0 = \tau_{\nu_0} = \frac{\pi e^2 N}{mc} f_{12} \phi(\nu_0) = \frac{e^2 \sqrt{\pi}}{m c^2} \cdot \left( \frac{c}{b} \right) \cdot (N \lambda_0 f_{12}) = \sqrt{\pi} \: r_0 \cdot \left( \frac{c}{b} \right) \cdot (N \lambda_0 f_{12})  \]
Plugging in constants gives,
\[ \tau_0 = (1.49 \cdot 10^{-7}) \cdot \left( \frac{b}{1 \: \mathrm{km} \: \mathrm{s}^{-1}} \right)^{-1} \cdot \left( \frac{N \lambda_0 f_{12}}{1 \: \mathrm{cm}^{-1}} \right) \]
Now using in our parameters,
\begin{align*}
\lambda & = 2026 \: \mathrm{\r{A}}
\\
f & = 0.489
\\
b & = 115 \: \mathrm{km} \: \mathrm{s}^{-1}
\end{align*}
we estimate,
\[ \tau_0 = 0.816 \]

\section{Problem 5}

Consider a mirror in the $x$-$y$ plane which moves at velocity $v$ in the $+x$ direction. Then consider a photon of frequency $\nu_0$ above the mirror in the $x$-$z$ plane incident upon the mirror with an angle of $\theta_0$ from the normal as viewed in the lab frame. In the lab frame the photon has four-momentum,
\[ p^\mu = \frac{h \nu_0}{c}
\begin{pmatrix}
1
\\
\sin{\theta_0}
\\
0
\\
-\cos{\theta_0}
\end{pmatrix}
\]
However, in this frame, the mirror is moving so we need to boost into the mirror's frame by performing a Lorentz transformation in the $+x$-direction by a velocity $v$. The photon's four momentum then becomes,
\[ p'^\mu = \frac{h \nu_0}{c}
\begin{pmatrix}
\gamma (1 - \beta \sin{\theta_0} )
\\
\gamma ( \sin{\theta_0} - \gamma)
\\
0
\\
-\cos{\theta_0}
\end{pmatrix}
\]
Now the ray is reflected which, in the mirror frame, simply amounts to reversing its $z$-momentum to give,
\[ p'^\mu_{\text{ref}} = \frac{h \nu_0}{c}
\begin{pmatrix}
\gamma (1 - \beta \sin{\theta_0} )
\\
\gamma ( \sin{\theta_0} - \gamma)
\\
0
\\
\cos{\theta_0}
\end{pmatrix}
\]
Now to get back to the lab frame we perform a Lorentz transformation in the opposite direction by $\beta$ to get,
\[ p^\mu_{\text{ref}} = \begin{pmatrix}
1
\\
\sin{\theta_0}
\\
0
\\
\cos{\theta_0}
\end{pmatrix}
\]
which is exactly the four-momentum of the original photon with its $z$-momentum reflected. Therefore $\theta' = \theta_0$ and $\nu' = \nu_0$. 

\section{Problem 6}

Consider the energy-level diagrams for OIII.

\subsection*{(a)}

The $3s$, $3p$, $3d$ states are increasing in energy. This can be explained by considering the effect of shielding from the inner $2 p^2$ electrons. The higher the orbital angular momentum the further the electron is kept from the nucleus by the centrifugal barrier. The greater degree to which the outer electron's wavefunction penetrates the inner electron cloud towards the nucleus, the greater effective nuclear charge it experiences i.e. the less effective the shielding from the inner electrons. Therefore, we expect that increasing the orbital angular momentum of the outer electron should increase its orbital energy by making shielding more effective and thus effectively reducing the nuclear charge it experiences. 

\subsection*{(b)}

In the case of the ground state $2p^2$ levels the states split up into two groups, the singlets $^1 D_2$, $^1 S_0$ (the $^1 P_1$ state is forbidden by the exclusion principle since the $P$ state wavefunction of two equivalent $\ell = 1$ electrons is anti-symmetric so the their spin states must be symmetric i.e. a triplet state) and the triplets $^3 P_{0,1,2}$ (again the $^3 S$ and $^3 D$ states are forbidden by the exclusion principle since the $S$ and $D$ state wavefunction for equivalent $\ell = 1$ electrons are symmetric so the spin states must be antisymmetric i.e. a singlet). Due to the symmetric combination leading to increased proximity and thus higher repulsion energy, the singlet states have higher energy than the triplet states whose energy is lowered by this ``exchange term''. This explains why the singlets $^1 D_2$, $^1 S_0$ have higher energy than the triplets $^3 P_{0,1,2}$. The ordering within these categories is due to spin-orbit coupling.
\bigskip\\
Among the triplet  states $^3 P_{0,1,2}$ we have fixed $\ell = 1$ for and increasing total $J = L + S$. The formula for spin-orbit coupling gives an increasing (positive) energy shift in increasing $J$ giving the ordering of the $^3 P_{0,1,2}$ states increasing in energy with increasing $J$.
\bigskip\\
On the other hand, the two singlet states $^1 D_2$, $^1 S_0$ have zero total spin and thus $J = L$ so spin-orbit effects are not important. However, there is a ``residual electrostatic'' term which arises from the electrostatic repulsion between the paired electrons in an orbital. This repulsion prefers orbitals with higher total orbital angular momentum explaining the ordering of $^1 D_2$ below $^1 S_0$ in energy.

\section{Problem 7}

In the Crab spectrum, we consider the Luminosity of Synchrotron and Inverse Compton radiation. The power density emitted by inverse Compton processes takes the form,
\[ P_{\text{IC}} = \tfrac{4}{3} \sigma_T c \gamma^2 \beta^2 U_{\text{ph}} \]
where $U_{\text{ph}}$ is the energy density in the photon to be up-scattered. Now the total inverse Compton luminosity from the Crab comes from integrating this power over the spectrum of electron energies. Since the main emitting electrons in the Crab are highly relativistic, we may take $\beta = 1$. Thus,
\[ L_{\text{IC}} = V \int_{0}^{\infty} n(\gamma) P_{\text{IC}}(\gamma) \: \d{\gamma} = \tfrac{4}{3} \sigma_T c V U_{\text{ph}} \int_0^{\infty} n(\gamma) \gamma^2 \: \d{\gamma} \]
Similarly, the power emitted in synchrotron radiation takes the form,
\[ P_{\text{synch}} = \tfrac{4}{3} \sigma_T c \gamma^2 \beta^2 U_{\text{mag}} \]
where $U_{\text{mag}}$ is the energy density in the magnetic field,
\[ U_{\text{mag}} = \frac{B^2}{8 \pi} \]
Similarly, integrating over electron energies we find the total synchrotron luminosity of the Crab,
\[ L_{\text{sync}} = V  \int_{0}^{\infty} n(\gamma) P_{\text{sync}}(\gamma) \: \d{\gamma} = \tfrac{4}{3} \sigma_T c V U_{\text{mag}} \int_0^{\infty} n(\gamma) \gamma^2 \: \d{\gamma} \]
Therefore, we find simply that,
\[ \frac{L_{\text{sync}}}{L_{\text{IC}}} = \frac{U_{\text{mag}}}{U_{\text{ph}}} \]
Using the parameters of previous problems, we know the Crab Nebula has a uniform magnetic field of $B = 5 \times 10^{-4} \: \mathrm{G}$ and a volume of $3 \times 10^{56} \: \mathrm{cm}^3$ giving a magnetic energy density of,
\[ U_{\text{mag}} = 9.9 \cdot 10^{-9} \: \mathrm{erg} \: \mathrm{cm}^{-3} \]
Now we need to address the source of the photons for inverse Compton scattering which primary comes from the synchrotron photons emitted by the Crab itself. If we assume that the radiation field inside the Crab is isotropic then the emitted flux from the surface satisfies, 
\[ U_\nu = \frac{4}{c} F_\nu \] 
Furthermore, integrating over frequency, the total luminosity must be the surface flux integrated over the surface area $S$. Thus we find that,
\[ L = FS = PV \]
which implies that,
\[ U_{\text{sync}} = \frac{4 L_{\text{sync}}}{c S} \]
if we assume that the nebula is approximately spherical then,
\[ S \approx V^{\frac{2}{3}} = 1.38 \cdot 10^{18} \: \mathrm{cm} \]
Furthermore, on HW 10 we computed the total synchrotron Luminosity of the Crab nebula to be,
\[ L_{\text{sync}} = 3.858 \cdot 10^{38} \: \mathrm{erg} \: \mathrm{s}^{-1} \]
Therefore, putting this together we find,
\[ U_{\text{sync}} = 1.15 \cdot 10^{-10} \: \mathrm{erg} \: \mathrm{cm}^{-3} \]
Thus, I estimate that,
\[ \frac{L_{\text{sync}}}{L_{\text{IC}}} = \frac{U_{\text{mag}}}{U_{\text{ph}}} \approx 86  \]
about a factor of two smaller than the true value of $\approx 200$. 

\section{Problem 8}

The spin temperature in the ISM is determined by an interplay between different excitational and deexicational mechanisms of the hydrogen spin-spin transition. The three main mechanisms that the referenced paper discusses are collisional excitation which is controlled by the kinetic temperature $T_K$ and the density $n(H)$ of the gas, background flux in the $21 \: \mathrm{cm}$-line spectrum (given primarily from the CMB spectrum), and background Ly-$\alpha$ flux from starlight.
\bigskip\\
In the homework, we approximated the spin temperature excluding the effects of Ly-$\alpha$ flux taking into account both collisional and CMB photo-excitation. We found a relation, 
\[ T_S = \frac{T_{\text{CMB}} + y T_K}{1 + y} \]  
where we define,
\[ y = \frac{h \nu}{k T_K} \frac{n}{n_{\text{crit.}}} \]
and,
\[ n_{\text{crit.}} = \frac{A_{10}}{k_{10}} \]
is the critical density for collisional excitation.
\bigskip\\
At high density, the spin states are rapidly repopulated by collisions and thus the distribution of spin populations follows closely the detailed balance of collisional excitation coefficients meaning that the spin temperature is controlled purely by the kinetic temperature. We see this effect in our toy model when $y \to \infty$ we get $T_S = T_K$. This corresponds to the independence of the graph on pressure at fixed temperature (i.e. on density) in the high-density region (note how the lines of constant $T_K$ are flat i.e. give constant spin temperature). 
\bigskip\\
Furthermore, at low density, for large kinetic temperatures $T_K \gg T_{\text{CMB}}$ (which are expected in the ISM) the spin temperature becomes highly sensitive to density since the level populations are largely determined by the low CMB temperature so small fluctuations in the efficacy of collisional excitation produce large effects on the spin temperature. On the left-hand graph this corresponds to the lines of constant $T_K$ tilting upwards in the low-density regime. 
\bigskip\\
The right-hand panel shows the effect of Ly-$\alpha$ flux in addition to the two factors we have already discussed. The effect of adding the starlight Ly-$\alpha$ flux on the density - spin temperature plot is to raise the spin temperature and flatten the lines of constant $T_K$ in the low-density regime. Importantly, one should note that the plots on the right are generated assuming that the Ly-$\alpha$ temperature $T_L$ equals the kinetic temperature $T_K$ of the gas. Therefore, the effects of populating the upper spin levels via these photons is to make the spin population distribution more similar to the population levels that would determined by collisional excitation if the levels were effectively replenished by these collisions. This means that the effects of the starlight is to push the spin temperature $T_S$ closer to the kinetic temperature $T_K$ which explains the flattening of the constant $T_K$ curves even in the low density regime. Since the radiative coefficients for excitation may be significantly more effective than the collisional ones, this effect can be interpreted as lowering the effective critical density above which $T_S$ becomes highly coupled to $T_K$. 






\end{document}