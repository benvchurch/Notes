\documentclass[12pt]{article}
\usepackage[english]{babel}
\usepackage[utf8]{inputenc}
\usepackage[english]{babel}
\usepackage[a4paper, total={7.25in, 9.5in}]{geometry}
\usepackage{tikz-feynman}
\tikzfeynmanset{compat=1.0.0} 
\usepackage{subcaption}
\usepackage{float}
\floatplacement{figure}{H}
\usepackage{mathrsfs}  
\usepackage{dsfont}
\usepackage{relsize}
\DeclareMathAlphabet{\mathdutchcal}{U}{dutchcal}{m}{n}

\usepackage{revsymb}


\newcommand{\field}{\hat{\Phi}}
\newcommand{\dfield}{\hat{\Phi}^\dagger}
 
\usepackage{amsthm, amssymb, amsmath, centernot}
\usepackage{slashed}
\newcommand{\notimplies}{%
  \mathrel{{\ooalign{\hidewidth$\not\phantom{=}$\hidewidth\cr$\implies$}}}}
 
\renewcommand\qedsymbol{$\square$}
\newcommand{\cont}{$\boxtimes$}
\newcommand{\divides}{\mid}
\newcommand{\ndivides}{\centernot \mid}

\newcommand{\Integers}{\mathbb{Z}}
\newcommand{\Natural}{\mathbb{N}}
\newcommand{\Complex}{\mathbb{C}}
\newcommand{\Zplus}{\mathbb{Z}^{+}}
\newcommand{\Primes}{\mathbb{P}}
\newcommand{\Q}{\mathbb{Q}}
\newcommand{\R}{\mathbb{R}}
\newcommand{\ball}[2]{B_{#1} \! \left(#2 \right)}
\newcommand{\Rplus}{\mathbb{R}^+}
\renewcommand{\Re}[1]{\mathrm{Re}\left[ #1 \right]}
\renewcommand{\Im}[1]{\mathrm{Im}\left[ #1 \right]}
\newcommand{\Op}{\mathcal{O}}

\newcommand{\invI}[2]{#1^{-1} \left( #2 \right)}
\newcommand{\End}[1]{\text{End}\left( A \right)}
\newcommand{\legsym}[2]{\left(\frac{#1}{#2} \right)}
\renewcommand{\mod}[3]{\: #1 \equiv #2 \: \mathrm{mod} \: #3 \:}
\newcommand{\nmod}[3]{\: #1 \centernot \equiv #2 \: mod \: #3 \:}
\newcommand{\ndiv}{\hspace{-4pt}\not \divides \hspace{2pt}}
\newcommand{\finfield}[1]{\mathbb{F}_{#1}}
\newcommand{\finunits}[1]{\mathbb{F}_{#1}^{\times}}
\newcommand{\ord}[1]{\mathrm{ord}\! \left(#1 \right)}
\newcommand{\quadfield}[1]{\Q \small(\sqrt{#1} \small)}
\newcommand{\vspan}[1]{\mathrm{span}\! \left\{#1 \right\}}
\newcommand{\galgroup}[1]{Gal \small(#1 \small)}
\newcommand{\bra}[1]{\left| #1 \right>}
\newcommand{\Oa}{O_\alpha}
\newcommand{\Od}{O_\alpha^{\dagger}}
\newcommand{\Oap}{O_{\alpha '}}
\newcommand{\Odp}{O_{\alpha '}^{\dagger}}
\newcommand{\im}[1]{\mathrm{im} \: #1}
\renewcommand{\ker}[1]{\mathrm{ker} \: #1}
\newcommand{\ket}[1]{\left| #1 \right>}
\renewcommand{\bra}[1]{\left< #1 \right|}
\newcommand{\inner}[2]{\left< #1 | #2 \right>}
\newcommand{\expect}[2]{\left< #1 \right| #2 \left| #1 \right>}
\renewcommand{\d}[1]{\: \mathrm{d}#1 \:}
\newcommand{\dn}[2]{ \mathrm{d}^{#1} #2 \:}
\newcommand{\deriv}[2]{\frac{\d{#1}}{\d{#2}}}
\newcommand{\nderiv}[3]{\frac{\dn{#1}{#2}}{\d{#3^{#1}}}}
\newcommand{\pderiv}[2]{\frac{\partial{#1}}{\partial{#2}}}
\newcommand{\fderiv}[2]{\frac{\delta #1}{\delta #2}}
\newcommand{\parsq}[2]{\frac{\partial^2{#1}}{\partial{#2}^2}}
\newcommand{\topo}{\mathcal{T}}
\newcommand{\base}{\mathcal{B}}
\renewcommand{\bf}[1]{\mathbf{#1}}
\renewcommand{\a}{\hat{a}}
\newcommand{\adag}{\hat{a}^\dagger}
\renewcommand{\b}{\hat{b}}
\newcommand{\bdag}{\hat{b}^\dagger}
\renewcommand{\c}{\hat{c}}
\newcommand{\cdag}{\hat{c}^\dagger}
\newcommand{\hamilt}{\hat{H}}
\renewcommand{\L}{\hat{L}}
\newcommand{\Lz}{\hat{L}_z}
\newcommand{\Lsquared}{\hat{L}^2}
\renewcommand{\S}{\hat{S}}
\renewcommand{\empty}{\varnothing}
\newcommand{\J}{\hat{J}}
\newcommand{\lagrange}{\mathcal{L}}
\newcommand{\dfourx}{\mathrm{d}^4x}
\newcommand{\meson}{\phi}
\newcommand{\dpsi}{\psi^\dagger}
\newcommand{\ipic}{\mathrm{int}}
\newcommand{\tr}[1]{\mathrm{tr} \left( #1 \right)}
\newcommand{\C}{\mathbb{C}}
\newcommand{\CP}[1]{\mathbb{CP}^{#1}}
\newcommand{\Vol}[1]{\mathrm{Vol}\left(#1\right)}

\newcommand{\Tr}[1]{\mathrm{Tr}\left( #1 \right)}
\newcommand{\Charge}{\hat{\mathbf{C}}}
\newcommand{\Parity}{\hat{\mathbf{P}}}
\newcommand{\Time}{\hat{\mathbf{T}}}
\newcommand{\Torder}[1]{\mathbf{T}\left[ #1 \right]}
\newcommand{\Norder}[1]{\mathbf{N}\left[ #1 \right]}
\newcommand{\Znorm}{\mathcal{Z}}
\newcommand{\EV}[1]{\left< #1 \right>}
\newcommand{\interact}{\mathrm{int}}
\newcommand{\covD}{\mathcal{D}}
\newcommand{\conj}[1]{\overline{#1}}

\newcommand{\SO}[2]{\mathrm{SO}(#1, #2)}
\newcommand{\SU}[2]{\mathrm{SU}(#1, #2)}

\newcommand{\anticom}[2]{\left\{ #1 , #2 \right\}}


\newcommand{\pathd}[1]{\! \mathdutchcal{D} #1 \:}

\renewcommand{\theenumi}{(\alph{enumi})}


\renewcommand{\theenumi}{(\alph{enumi})}

\newcommand{\atitle}[1]{\title{% 
	\large \textbf{ASTR GR6001 Radiative Processes
	\\ Assignment \# #1} \vspace{-2ex}}
\author{Benjamin Church }
\maketitle}

\theoremstyle{definition}
\newtheorem{theorem}{Theorem}[section]
\newtheorem{definition}{definition}[section]
\newtheorem{lemma}[theorem]{Lemma}
\newtheorem{proposition}[theorem]{Proposition}
\newtheorem{corollary}[theorem]{Corollary}
\newtheorem{example}[theorem]{Example}
\newtheorem{remark}[theorem]{Remark}
\begin{document}


\atitle{8}

\section{Problem 1}

Consider mono-energetic electrons of a very large $\gamma_0$ are injected at a constant rate into a region with uniform $B$ field. Assume that there is sufficient time for these electrons to radiate away all their energy so we reach a steady state. 
\bigskip\\
We have shown that the Lorentz factor for an electron emitting synchrotron radiation will decrease as,
\[ \deriv{}{t} \gamma = - A \gamma^2 \]
where,
\[ A = \frac{2 e^4}{3 m^2 c^5} B_\perp^2 \]
Let $n(\gamma)$ be the number of electrons at a given energy (and thus Lorentz factor $\gamma$). There is a ``current'' $n(\gamma) \dot{\gamma}$ of electrons of energy $\gamma$ flowing downwards in energy. For steady state to be reached, this current must be constant else the electron density changes at some energy. Therefore,
\[ \deriv{n}{\gamma} \dot{\gamma} + n(\gamma) \pderiv{\dot{\gamma}}{\gamma} = 0 \]
However, we know that,
\[ \deriv{}{t} \gamma = - A \gamma^2 \]
and thus,
\[ \pderiv{\dot{\gamma}}{\gamma} = - 2 A \gamma \]
This implies that,
\[ (- A \gamma^2) \deriv{n}{\gamma} - 2 A \gamma \: n(\gamma)  = 0 \]
Thus,
\[ \deriv{n}{\gamma} = - 2 \gamma^{-1} n(\gamma) \]
and thus,
\[ \deriv{\log{n}}{\log{\gamma}} = - 2 \]
which implies that,
\[ n(\gamma) = n_0 \left( \frac{\gamma}{\gamma_0} \right)^{-2} \]
so the power law of electron energies has index $p = 2$. Now we need to compute the spectrum of synchrotron radiation emitted by this distribution. This is,
\begin{align*}
P(\omega) & = \frac{\sqrt{3}}{2 \pi} \cdot \frac{e^3 B \sin{\alpha}}{m c^2} \int_{1}^{\gamma_0} n(\gamma) F(\omega / \omega_c) \d{\gamma}  
\end{align*}
where,
\[ \omega_c = \tfrac{3}{2} \gamma^3 \left( \frac{e B}{\gamma m c} \right) \sin{\alpha} \]
We define,
\[ \omega_0 = \frac{3 \gamma_0^2 e B}{2 m c} \sin{\alpha} \]
and thus,
\[ \omega_c = \left( \frac{\gamma}{\gamma_0} \right)^2 \omega_0 \]
Therefore, we can write this integral in terms of,
\[ x = \frac{\omega}{\omega_c} = \frac{\omega}{\omega_0} \left( \frac{\gamma}{\gamma_0} \right)^{-2} \]
Now,
\[ \d{x} = - \frac{\omega}{\omega_0} \cdot \frac{2}{\gamma_0} \left( \frac{\gamma}{\gamma_0} \right)^{-3} \d{\gamma} \]
and thus,
\[ \d{\gamma} = - \frac{\gamma_0}{2} \cdot \left( \frac{\omega}{\omega_0} \right)^{\frac{1}{2}} \cdot \frac{\d{x}}{x^{\frac{3}{2}}} \]
Therefore,
\begin{align*}
P(\omega) & = \frac{\sqrt{3}}{2 \pi} \cdot \frac{n_0 e^3 B \sin{\alpha}}{m c^2} \int_{1}^{\gamma_0} \left( \frac{\gamma}{\gamma_0} \right)^{-p}  F(\omega / \omega_c) \d{\gamma}  
\\
& = \frac{\sqrt{3}}{2 \pi} \cdot \frac{n_0 e^3 B \sin{\alpha}}{m c^2} \left( \frac{\omega}{\omega_0} \right)^{-\frac{p}{2}} \int_{1}^{\gamma_0} x^{\frac{p}{2}}   F(x) \d{\gamma}  
\\
& = \frac{\sqrt{3}}{4 \pi} \cdot \frac{n_0 \gamma_0 e^3 B \sin{\alpha}}{m c^2} \left( \frac{\omega}{\omega_0} \right)^{-\frac{p - 1}{2}} \int_{x_1}^{x_2} x^{\frac{p - 3}{2}}  F(x) \d{x} 
\end{align*} 
If $\omega_0$ is very large then $x_1$ and $x_2$ can be taken to be approximately constant in $\omega$ and thus the integral is constant. Therefore, $P$ satisfies a power law,
\[ P(\omega) \propto \left( \frac{\omega}{\omega_0} \right)^{-\frac{p - 1}{2}} \]
so the spectral index of the emitted synchrotron radiation is $s = \tfrac{1}{2}(p - 1) = \tfrac{1}{2}$ since $p = 2$. 


\section{Problem 2}

\subsection*{(a)}

Relativistic electrons satisfy the Lorentz force law,
\[ F = \deriv{}{t} (\gamma m \vec{v}) = e \left( E + \frac{\vec{v}}{c} \times B \right) \]
Then, for circular motion,
\[ a_\perp = \frac{v_\perp^2}{r_g} = \omega_B v_\perp \]
For the case $E = 0$ then the force and the acceleration are perpendicular so,
\[ a_\perp = \frac{F_\perp}{\gamma m} = \frac{e v_\perp B}{\gamma m c} \]
and $a_\parallel = 0$. Therefore, 
\[ \omega_B = \frac{e B}{\gamma m c} \]
and 
\[ r_g = \frac{\gamma m c v_\perp}{e B} \] 
Now note that for a pitch angle $\alpha$ we have $v_\perp = v \sin{\alpha}$. We must have $L \gg \hbar$ for the classical approximation to be valid. The orbital angular momentum is,
\[ L = \gamma m v_\perp r_g = \frac{\gamma^2 m^2 c v_\perp^2}{e B}  \] 
Furthermore, if the electron energy becomes too great then the electron will radiate energy at a rate much faster than the orbital period which makes synchrotron radiation possible. The relativistic Larmor formula gives, 
\[ P = \frac{2}{3} \frac{e^2}{c^3} \gamma^4 ( a_\perp^2 + \gamma^2 a_{\parallel}^2) = \frac{2 e^2}{3 c^3} \gamma^4 \left( \frac{e v_\perp B}{\gamma m c} \right)^2 = \frac{2 e^2}{3 c^3} \gamma^2 v_\perp^2 \left( \frac{e B}{m c} \right)^2 \]
We must have $P \ll \omega_B E$ where $E = \gamma m c^2$ is the energy of the orbiting electron and,
\[ \omega_B = \frac{e B}{\gamma m c} \] 
and thus,
\[ \omega_B E = \left( \frac{e B}{\gamma m c} \right) \gamma m c^2 = e B c \]
is the orbital period. Therefore,
\[ \frac{2 e^2}{3 c^3} \gamma^2 v_\perp^2 \left( \frac{e B}{m c} \right)^2 \ll e B c \]
which implies that,
\[ \frac{3 m^2 c^6}{2 e^3 B} \gg \gamma^2 v_\perp^2 \]
Furthermore, 
\[ \gamma^2 v_\perp^2 = \frac{e B L}{m^2 c} \]
and thus,
\[ \frac{3 m^2 c^6}{2 e^3 B} \gg \frac{e B L}{m^2 c} \]
which implies that,
\[ B^2 \ll \frac{3 m^4 c^7}{2 e^4 L} \]
However, we also require $L \gg \hbar$ and thus,
\[ B \ll B_{\text{max}} = \left( \frac{3 m^4 c^7}{2 e^4 \hbar} \right)^{\frac{1}{2}} = \left( \frac{3 m^2 c^3}{2 r_0^2 \hbar} \right)^{\frac{1}{2}} = 6.32 \cdot 10^{14} \: \mathrm{G} \]
For magnetic fields greater than than maximum field, synchrotron radiation is not possible. Notice that this limit does not depend on the electron pitch angle.

\subsection*{(b)}

Now we fix the magnetic field $B$ and consider the maximum frequency,
\[ \omega_B = \frac{e B}{\gamma m c} \]
However, recall that the peak of synchrotron radiation occurs not at a frequency of $\omega_B$ but rather at approximately,
\[ \omega_C = \tfrac{3}{2} \gamma^3 \omega_B \sin{\alpha} = \gamma^2 \frac{3 e B}{2m c} \sin{\alpha} \]
Therefore, finding maximum $\omega_S$ is equivalent to asking for the maximum value of $\gamma$ for which our approximations are valid since the other parameters are fixed. We have shown that,
\[ \frac{3 m^2 c^6}{2 e^3 B} \gg \gamma^2 v_\perp^2 \]
Furthermore, 
\[ v_\perp = v \sin{\alpha} \]
and thus,
\[ \gamma^2 v_\perp^2 = c^2  (\gamma^2 - 1) \sin^2{\alpha} \]
Therefore, we must have,
\[ \gamma^2 \ll \frac{3 m^2 c^4}{2 e^3 B \sin^2{\alpha}} \]
and thus,
\[ \omega_C \ll \frac{9 m c^3}{4 e^2 \sin{\alpha}} = \frac{9 c}{4 r_0 \sin{\alpha}} \]
Therefore, we find an upper limit for the peak emission frequency of synchrotron radiation,
\[ \omega_{\text{max}} = \frac{9 c}{4 r_0 \sin{\alpha}} = \left( 2.39 \cdot 10^{23} \: \mathrm{s}^{-1} \right) \cdot (\sin{\alpha})^{-1} \]
which notice is independent of the magnetic field $B$. This peak frequency corresponds to a photon energy of,
\[ E_{\text{max}} = \hbar \omega_{\text{max}} = \frac{9 \hbar c}{4 r_0 \sin{\alpha}} = (157.4 \: \mathrm{MeV}) \cdot (\sin{\alpha})^{-1} \] 

\section{Problem 3}

The Crab Nebula has a uniform magnetic field of $B = 5 \times 10^{-4} \: \mathrm{G}$ and a volume of $3 \times 10^{56} \: \mathrm{cm}^3$. 

\subsection*{(a)}

From the formula,
\[ \omega_c = \tfrac{3}{2} \: \gamma^3 \omega_B  \: \sin{\alpha} \]
where,
\[ \omega_B = \frac{e B}{\gamma m c} \]
Assuming the electron velocities are randomly distributed in angle, we can average $\sin{\alpha}$ over a sphere to find,
\[ \EV{\sin{\alpha}} = \frac{1}{4 \pi} \int_{0}^{\pi} \sin{\alpha} (2 \pi \sin{\alpha} \d{\alpha}) = \frac{\pi}{4} \]
Therefore,
\[ \omega_c = \tfrac{3 \pi}{8} \gamma^2 \omega_0 \]
with,
\[ \omega_0 = \frac{e B}{m c} = 8.79 \times 10^3 \: \mathrm{s}^{-1} \]
Thus, we find that the average electron must have a $\gamma$-factor of,
\[ \gamma = \left( \frac{8}{3 \pi} \right)^{\frac{1}{2}} \cdot \left( \frac{\omega_c}{\omega_B} \right)^{\frac{1}{2}} \]
Furthermore, the gyroradius,
\[ r_g = \frac{\gamma m c v_\perp}{eB} = \frac{\gamma c \beta}{\omega_0}  \]
For highly relativistic electrons $\beta \approx 1$ and thus,
\[ r_g = \frac{\gamma c}{\omega_0} = \gamma \cdot (3.4 \cdot 10^{6} \: \mathrm{cm}) \]
Furthermore, the synchrotron lifetime is,
\[ t_{\frac{1}{2}} = \left( \frac{2 e^4 B^2}{3 m^3 c^5} \cdot \gamma \right)^{-1} \]
Therefore,
\[ t_{\frac{1}{2}} = (5.1 \cdot 10^8 \: \mathrm{s}) \cdot \left( \frac{B}{1 \: \mathrm{G}} \right)^{-2} \cdot \gamma^{-1} \] 
For $B = 5 \cdot 10^{-4} \: \mathrm{G}$ we have,
\[ t_{\frac{1}{2}} = (2.0 \cdot 10^{15} \: \mathrm{s}) \cdot \gamma^{-1} \] 
Now we can calculate the table,
\begin{center}
 \begin{tabular}{||c c c c||} 
 \hline
 $\nu \: (\mathrm{Hz})$ & $\gamma$ & $r_g \: (\mathrm{cm})$ & $t_{\frac{1}{2}} \: (\mathrm{s})$ 
 \\ [0.5ex] 
 \hline\hline
 $10^{8}$ & $250$ & $8.5 \cdot 10^{8}$ & $8.0 \cdot 10^{12}$ \\ 
 \hline
 $10^{14}$ & $2.5 \times 10^5$ & $8.5 \cdot 10^{13}$ & $8.0 \cdot 10^{9}$ \\
 \hline
 $10^{22}$ & $2.5 \cdot 10^{9}$ & $8.5 \cdot 10^{17}$ & $8.0 \cdot 10^{5}$ \\
 \hline
\end{tabular}
\end{center}
Furthermore, the radius of the Nebula,
\[ r_N = \left( \frac{3 V}{4 \pi} \right)^{\frac{1}{3}} = 4.15 \cdot 10^{18} \]
Therefore, the gyroradius for $\gamma$-rays ($10^{22} \: \mathrm{Hz}$) is on the order of the radius of the nebula $r_N \sim 10^{18} \: \mathrm{cm}$ although smaller by a factor of $5$. 

\subsection*{(b)}

The spectrum of electron energies can be computed from the synchrotron emission spectrum of the nebula. The emission spectrum is computed in the notes as,
\[ \mathcal{E}_\nu = (1.7 \cdot 10^{-21} \: \mathrm{erg} \: \mathrm{cm}^{-3} \: \mathrm{s}^{-1} \: \mathrm{Hz}^{-1})  \: \left( \frac{n_0}{1 \: \mathrm{cm}^{-3}} \right) a(p) \left( \frac{B}{1 \: \mathrm{G}} \right)^{s + 1} \cdot \left( \frac{\nu}{4 \cdot 10^6 \: \mathrm{Hz}} \right)^{-s} \]
where $s$ is the spectral index of the radiation and $a(p)$ is a numerical value depending on the index $p = 2 s + 1$. Furthermore, the $n_0$ which appears above is the specific number density appearing in the power-law distribution for electron energy distribution,
\[ n(\gamma) = n_0 \gamma^{-p} \]
and therefore, the total number of emitting electrons is,
\[ n = \int_{\gamma_{\text{min}}}^{\gamma_{\text{max}}} n_0 \gamma^{-p} \d{\gamma} = \frac{n_0}{p - 1} \left[ \frac{1}{\gamma_{\text{min}}^{p - 1}} - \frac{1}{\gamma_{\text{max}}^{p-1}} \right] \]
Therefore, it suffices to find $p$, $n_0$, and $\gamma_{\text{min}}$ and $\gamma_{\text{max}}$. 
\bigskip\\
Our available information is the observed specific flux spectrum which is related to $\mathcal{E}_\nu$ via the distance to the cluster and the total volume via,
\[ f_{\nu} = \frac{\mathcal{E}_\nu V}{4 \pi d^2} \]
since $\mathcal{E}_\nu V$ is the total power emitted by the nebula. Therefore, we have,
\[ n_0 = (1 \: \mathrm{cm}^{-3}) \cdot \frac{4 \pi d^2 f_\nu}{V} \cdot (1.7 \cdot 10^{-21} \: \mathrm{erg} \: \mathrm{cm}^{-3} \: \mathrm{s}^{-1} \: \mathrm{Hz}^{-1})^{-1} a(p)^{-1} \cdot \left( \frac{B}{1 \: \mathrm{G}} \right)^{-(s + 1)} \cdot \left( \frac{\nu}{4 \cdot 10^6 \: \mathrm{Hz}} \right)^{s} \]
Computing this we will find the total number of electrons,
\begin{align*}
N & = V n = V n_0 \int_{\gamma_{\text{min}}}^{\gamma_{\text{max}}} \gamma^{-p} \d{\gamma} 
\\
& = (1 \: \mathrm{cm}^{-3}) \cdot (4 \pi d^2 f_\nu) \cdot (1.7 \cdot 10^{-21} \: \mathrm{erg} \: \mathrm{cm}^{-3} \: \mathrm{s}^{-1} \: \mathrm{Hz}^{-1})^{-1} a(p)^{-1} \cdot \left( \frac{B}{1 \: \mathrm{G}} \right)^{-(s + 1)} \cdot \left( \frac{\nu}{4 \cdot 10^6 \: \mathrm{Hz}} \right)^{s} 
\\
& \quad \cdot \frac{1}{p - 1} \left[ \frac{1}{\gamma_{\text{min}}^{p - 1}} - \frac{1}{\gamma_{\text{max}}^{p-1}} \right] 
\end{align*} 
\bigskip\\
We split the Crab spectrum into a broken power law with two segments in frequency, $10^{7} \: \mathrm{Hz} - 10^{14} \: \mathrm{Hz}$ with spectral index $s_1 \approx 0.25$ and $10^{14} \: \mathrm{Hz} - 10^{24} \: \mathrm{Hz}$. The end points of these segments gives the minimum and maximum $\gamma$-factors and their slopes give the values of $s$ and $p$. We need to do a linear (in log-log) fit to each segment. Define the parameter,
\[ g = \left( \frac{f_\nu}{10^{-20} \: \mathrm{erg} \: \mathrm{cm}^{-2} \: \mathrm{s}^{-1} \: \mathrm{Hz}^{-1}} \right) \cdot \left( \frac{\nu}{4 \cdot 10^6 \: \mathrm{Hz}} \right)^{s} \]  
with $s$ chosen to fit a segment. This number is approximately constant on each segment. In terms of the parameters, $g$ and $s$ and plugging in for $B$ and $d$,
\[ N = (2.8 \cdot 10^{45}) \left( 5 \cdot 10^{-4} \right)^{-(s+1)} \cdot \frac{g}{a(p)(p-1)} \cdot \left[ \frac{1}{\gamma_{\text{min}}^{p - 1}} - \frac{1}{\gamma_{\text{max}}^{p-1}} \right]  \]
From linear fits to the data (note that the data is given in $\mathrm{W} \: \mathrm{m}^{-2} \: \mathrm{Hz}^{-1} = 10^{-3} \: \mathrm{erg} \: \mathrm{cm}^{-2} \: \mathrm{Hz}^{-1}$), I find,
\begin{center}
 \begin{tabular}{||c c c c c c c||} 
 \hline
 segment & $s$ & $p$ & $a(p)$ & $g$ & $\gamma_{\text{min}}$ & $\gamma_{\text{max}}$
 \\ [0.5ex] 
 \hline\hline
 $1$ & $0.25$ & $1.5$ & $0.085$ & $7.07$ & $79$ & $2.5 \cdot 10^5$ \\ 
 \hline
 $2$ & $1.20$ & $3.4$ & $0.074$ & $7.54 \cdot 10^7$ & $2.5 \cdot 10^5$ & $2.5 \cdot 10^{10}$ \\
 \hline
\end{tabular}
\end{center}
Therefore, for the two segments, we find,
\[ N_1 = 6.9 \cdot 10^{50} \quad \quad \quad N_2 = 2.4 \cdot 10^{48} \]
Therefore,
\[ N_{\text{tot}} = 6.9 \cdot 10^{50} \]

\section{Problem 4} 

Consider a compact extra-galactic radio source at $d = 300 \: \mathrm{Mpc}$, with an angular diameter $\theta = 2 \cdot 10^{-3} \: \mathrm{arcsec}$ with observed flux density $f_{s} = 1 \cdot 10^{-25} \: \mathrm{erg} \: \mathrm{cm}^{-2} \: \mathrm{s}^{-1} \: \mathrm{Hz}^{-1}$ at a synchrotron self-absorption frequency of $\nu_s = 10^{8} \: \mathrm{Hz}$, and a spectral index $s = 0.75$ in the optically thin regime. We assume the source is spherical and homogeneous.

\subsection*{(a)}

At the self-absorption frequency $\nu_s$ we can compute the brightness temperature,
\[  T_B = \frac{2 m c^2}{k_B} \left( \frac{\nu m c}{e B} \right)^{\frac{1}{2}}  \]
However, at the self-absorption limit, the emission is given by,
\[  B_\nu = 4 m \nu^2 \left( \frac{\nu m c}{e B} \right)^{\frac{1}{2}} \]
Therefore,
\[ T_B = \frac{B_\nu c^2}{2 k_B \nu^2} \]
Now, we can compute $\mathcal{E}_\nu$ from the observed flux $f_\nu$. The total specific power is,
\[ P_\nu = 4 \pi R^2 (\pi B_\nu) \]
and therefore,
\[ f_s = \left( \frac{R}{d} \right)^2 \pi B_\nu = \left( \frac{\theta}{2} \right)^2 \pi B_\nu \]
Thus we have,
\[ T_B = \frac{f_s c^2}{2 \pi k_B \nu^2} \cdot \left( \frac{\theta}{2} \right)^{-2} = 4.4 \cdot 10^{11} \: ^\circ \mathrm{K} \]

\subsection*{(b)}

We can compute the magnetic field from the above equations since,
\[ B = \left( \frac{2 m c^2}{k_B T_B} \right)^2 \cdot \frac{\nu m c}{e} \]
Using the previous values,
\[ B = 4.1 \cdot 10^{-3} \: \mathrm{G} \]
Then the total magnetic energy is,
\[ U_{\text{mag}} = V \left( \frac{B^2}{8 \pi} \right) = \frac{4 \pi R^3}{3} \left( \frac{B^2}{8 \pi} \right) = \frac{4 \pi d^3}{3} \cdot \left( \frac{\theta}{2} \right)^3 \cdot \left( \frac{B^2}{8 \pi} \right) = \left( \frac{\theta}{2} \right)^3 \cdot \left( \frac{B^2 d^3}{6} \right) \]
Plugging in,
\[ U_{\text{mag}} = 2.5 \cdot 10^{50} \: \mathrm{erg}  \]

\subsection*{(c)}

The frequency of synchrotron emitting electrons is,
\[ \nu = \tfrac{3}{16} \gamma^2 \omega_0 \quad \quad \omega_0 = \frac{e B}{m c} \]
From the previous calculation of $B$ we have,
\[ \omega_0 = 7.3 \cdot 10^{4} \: \mathrm{Hz} \] 
At the frequency $\nu_s = 10^{8} \: \mathrm{Hz}$ we have,
\[ \gamma_s = \sqrt{\frac{16}{3}} \cdot \left( \frac{\nu_s}{\omega_0} \right)^{\frac{1}{2}} = 85.5 \]

\subsection*{(d)}

To compute the energy density in relativistic electrons,
\[ \rho = \int_{\gamma_{\text{min}}}^{\gamma_{\text{max}}} n(\gamma) (\gamma m c^2) \d{\gamma} = \int_{\gamma_{\text{min}}}^{\gamma_{\text{max}}} n_0 \gamma^{-p} (\gamma m c^2) \d{\gamma} = n_0 m c^2 \int_{\gamma_{\text{min}}}^{\gamma_{\text{max}}} \gamma^{1 - p} \d{\gamma} = \frac{n_0 m c^2}{p - 2} \left[ \frac{1}{\gamma_{\text{min}}^{p - 2}} - \frac{1}{\gamma_{\text{max}}^{p - 2}} \right]  \]
we need to find $n_0$ and $p$ and the limits of the $\gamma$-factor. The total energy is then,
\[ E = \rho V = \frac{4 \pi R^3}{3} \rho = \frac{N_0 m c^2}{p - 2} \left[ \frac{1}{\gamma_{\text{min}}^{p - 2}} - \frac{1}{\gamma_{\text{max}}^{p - 2}} \right]   \]
\bigskip\\
We assume that the emission spectrum of the compact source is a power law with spectral index $s = 0.75$ starting at $\nu_s = 10^{8} \: \mathrm{Hz}$ and thus $\gamma_{\text{min}} = \gamma_s = 85.5$. We assume that $\gamma_{\text{max}}$ is much larger than $\gamma_{\text{min}}$ so we may drop the second term. The spectral index $s = 0.75$ gives $p = 2 s + 1 = 2.5$. Finally, to compute $n_0$ we use the measured flux and compare it to the emission,
\[ \mathcal{E}_\nu = (1.7 \cdot 10^{-21} \: \mathrm{erg} \: \mathrm{cm}^{-3} \: \mathrm{s}^{-1} \: \mathrm{Hz}^{-1})  \: \left( \frac{n_0}{1 \: \mathrm{cm}^{-3}} \right) a(p) \left( \frac{B}{1 \: \mathrm{G}} \right)^{s + 1} \cdot \left( \frac{\nu}{4 \cdot 10^6 \: \mathrm{Hz}} \right)^{-s} \]
Plugging in for $B$ and computing at $\nu_s$,
\[ \mathcal{E}_s = (8.58 \cdot 10^{-28} \: \mathrm{erg} \: \mathrm{cm}^{-3} \: \mathrm{s}^{-1} \: \mathrm{Hz}^{-1}) \cdot \left( \frac{n_0}{1 \: \mathrm{cm}^{-3}} \right) \]
Furthermore,
\[ f_s = \frac{4 \pi R^3}{3} \cdot \frac{\mathcal{E}_s}{4 \pi d^2}  = (4 \pi d^2)^{-1} \cdot (8.58 \cdot 10^{-28} \: \mathrm{erg}  \: \mathrm{s}^{-1} \: \mathrm{Hz}^{-1}) \cdot N_0 \]
Therefore,
\[ N_0 = \frac{4 \pi d^2 f_s}{8.58 \cdot 10^{-28} \: \mathrm{erg}  \: \mathrm{s}^{-1} \: \mathrm{Hz}^{-1}} = 1.25 \cdot 10^{57} \]
Finally, we can compute the total energy,
\[ E = \frac{N_0 m c^2}{(p - 2) \gamma_{\text{min}}^{p - 2}} = 2.23 \cdot 10^{50} \: \mathrm{erg} \]
This value is very close to what we computed for the magnetic energy so the energy is in equipartition between the radiating electrons and the magnetic field. 

\end{document}