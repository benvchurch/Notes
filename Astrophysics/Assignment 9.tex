\documentclass[12pt]{article}
\usepackage[english]{babel}
\usepackage[utf8]{inputenc}
\usepackage[english]{babel}
\usepackage[a4paper, total={7.25in, 9.5in}]{geometry}
\usepackage{tikz-feynman}
\tikzfeynmanset{compat=1.0.0} 
\usepackage{subcaption}
\usepackage{float}
\floatplacement{figure}{H}
\usepackage{mathrsfs}  
\usepackage{dsfont}
\usepackage{relsize}
\DeclareMathAlphabet{\mathdutchcal}{U}{dutchcal}{m}{n}

\usepackage{revsymb}


\newcommand{\field}{\hat{\Phi}}
\newcommand{\dfield}{\hat{\Phi}^\dagger}
 
\usepackage{amsthm, amssymb, amsmath, centernot}
\usepackage{slashed}
\newcommand{\notimplies}{%
  \mathrel{{\ooalign{\hidewidth$\not\phantom{=}$\hidewidth\cr$\implies$}}}}
 
\renewcommand\qedsymbol{$\square$}
\newcommand{\cont}{$\boxtimes$}
\newcommand{\divides}{\mid}
\newcommand{\ndivides}{\centernot \mid}

\newcommand{\Integers}{\mathbb{Z}}
\newcommand{\Natural}{\mathbb{N}}
\newcommand{\Complex}{\mathbb{C}}
\newcommand{\Zplus}{\mathbb{Z}^{+}}
\newcommand{\Primes}{\mathbb{P}}
\newcommand{\Q}{\mathbb{Q}}
\newcommand{\R}{\mathbb{R}}
\newcommand{\ball}[2]{B_{#1} \! \left(#2 \right)}
\newcommand{\Rplus}{\mathbb{R}^+}
\renewcommand{\Re}[1]{\mathrm{Re}\left[ #1 \right]}
\renewcommand{\Im}[1]{\mathrm{Im}\left[ #1 \right]}
\newcommand{\Op}{\mathcal{O}}

\newcommand{\invI}[2]{#1^{-1} \left( #2 \right)}
\newcommand{\End}[1]{\text{End}\left( A \right)}
\newcommand{\legsym}[2]{\left(\frac{#1}{#2} \right)}
\renewcommand{\mod}[3]{\: #1 \equiv #2 \: \mathrm{mod} \: #3 \:}
\newcommand{\nmod}[3]{\: #1 \centernot \equiv #2 \: mod \: #3 \:}
\newcommand{\ndiv}{\hspace{-4pt}\not \divides \hspace{2pt}}
\newcommand{\finfield}[1]{\mathbb{F}_{#1}}
\newcommand{\finunits}[1]{\mathbb{F}_{#1}^{\times}}
\newcommand{\ord}[1]{\mathrm{ord}\! \left(#1 \right)}
\newcommand{\quadfield}[1]{\Q \small(\sqrt{#1} \small)}
\newcommand{\vspan}[1]{\mathrm{span}\! \left\{#1 \right\}}
\newcommand{\galgroup}[1]{Gal \small(#1 \small)}
\newcommand{\bra}[1]{\left| #1 \right>}
\newcommand{\Oa}{O_\alpha}
\newcommand{\Od}{O_\alpha^{\dagger}}
\newcommand{\Oap}{O_{\alpha '}}
\newcommand{\Odp}{O_{\alpha '}^{\dagger}}
\newcommand{\im}[1]{\mathrm{im} \: #1}
\renewcommand{\ker}[1]{\mathrm{ker} \: #1}
\newcommand{\ket}[1]{\left| #1 \right>}
\renewcommand{\bra}[1]{\left< #1 \right|}
\newcommand{\inner}[2]{\left< #1 | #2 \right>}
\newcommand{\expect}[2]{\left< #1 \right| #2 \left| #1 \right>}
\renewcommand{\d}[1]{\: \mathrm{d}#1 \:}
\newcommand{\dn}[2]{ \mathrm{d}^{#1} #2 \:}
\newcommand{\deriv}[2]{\frac{\d{#1}}{\d{#2}}}
\newcommand{\nderiv}[3]{\frac{\dn{#1}{#2}}{\d{#3^{#1}}}}
\newcommand{\pderiv}[2]{\frac{\partial{#1}}{\partial{#2}}}
\newcommand{\fderiv}[2]{\frac{\delta #1}{\delta #2}}
\newcommand{\parsq}[2]{\frac{\partial^2{#1}}{\partial{#2}^2}}
\newcommand{\topo}{\mathcal{T}}
\newcommand{\base}{\mathcal{B}}
\renewcommand{\bf}[1]{\mathbf{#1}}
\renewcommand{\a}{\hat{a}}
\newcommand{\adag}{\hat{a}^\dagger}
\renewcommand{\b}{\hat{b}}
\newcommand{\bdag}{\hat{b}^\dagger}
\renewcommand{\c}{\hat{c}}
\newcommand{\cdag}{\hat{c}^\dagger}
\newcommand{\hamilt}{\hat{H}}
\renewcommand{\L}{\hat{L}}
\newcommand{\Lz}{\hat{L}_z}
\newcommand{\Lsquared}{\hat{L}^2}
\renewcommand{\S}{\hat{S}}
\renewcommand{\empty}{\varnothing}
\newcommand{\J}{\hat{J}}
\newcommand{\lagrange}{\mathcal{L}}
\newcommand{\dfourx}{\mathrm{d}^4x}
\newcommand{\meson}{\phi}
\newcommand{\dpsi}{\psi^\dagger}
\newcommand{\ipic}{\mathrm{int}}
\newcommand{\tr}[1]{\mathrm{tr} \left( #1 \right)}
\newcommand{\C}{\mathbb{C}}
\newcommand{\CP}[1]{\mathbb{CP}^{#1}}
\newcommand{\Vol}[1]{\mathrm{Vol}\left(#1\right)}

\newcommand{\Tr}[1]{\mathrm{Tr}\left( #1 \right)}
\newcommand{\Charge}{\hat{\mathbf{C}}}
\newcommand{\Parity}{\hat{\mathbf{P}}}
\newcommand{\Time}{\hat{\mathbf{T}}}
\newcommand{\Torder}[1]{\mathbf{T}\left[ #1 \right]}
\newcommand{\Norder}[1]{\mathbf{N}\left[ #1 \right]}
\newcommand{\Znorm}{\mathcal{Z}}
\newcommand{\EV}[1]{\left< #1 \right>}
\newcommand{\interact}{\mathrm{int}}
\newcommand{\covD}{\mathcal{D}}
\newcommand{\conj}[1]{\overline{#1}}

\newcommand{\SO}[2]{\mathrm{SO}(#1, #2)}
\newcommand{\SU}[2]{\mathrm{SU}(#1, #2)}

\newcommand{\anticom}[2]{\left\{ #1 , #2 \right\}}


\newcommand{\pathd}[1]{\! \mathdutchcal{D} #1 \:}

\renewcommand{\theenumi}{(\alph{enumi})}


\renewcommand{\theenumi}{(\alph{enumi})}

\newcommand{\atitle}[1]{\title{% 
	\large \textbf{ASTR GR6001 Radiative Processes
	\\ Assignment \# #1} \vspace{-2ex}}
\author{Benjamin Church }
\maketitle}

\theoremstyle{definition}
\newtheorem{theorem}{Theorem}[section]
\newtheorem{definition}{definition}[section]
\newtheorem{lemma}[theorem]{Lemma}
\newtheorem{proposition}[theorem]{Proposition}
\newtheorem{corollary}[theorem]{Corollary}
\newtheorem{example}[theorem]{Example}
\newtheorem{remark}[theorem]{Remark}
\begin{document}


\atitle{9}

\section{Problem 1}

\subsection*{(a)}

The radiation produced through magnetic field curvature is also produced by (instantaneous) circular motion of the electron and therefore should have an identical form to that of synchrotron radiation. For synchrotron radiation we have a characteristic power and frequency,
\begin{align*}
P_{\text{sync}} & = \frac{2 e^2}{3 c^3} \gamma^4 \left( \frac{e v_\perp B}{\gamma m c} \right)^2 
\\
\nu_c & = \frac{3 \gamma^3}{4 \pi} \omega_B \sin{\alpha} 
\\
\omega_B & = \frac{e B}{\gamma m c}
\end{align*}
In the case of synchrotron radiation, the gyroradius is,
\[ r_g = \frac{\gamma m v_\perp c}{e B} \]
which should be replaced by the radius of curvature $r_B$ of the field lines. Thus, the characteristic quantity,
\[ \omega_B = \frac{e B}{\gamma m c} \]
should be replaced by,
\[ \omega_0 = \frac{v_{\parallel}}{r_B} \]
Making this substitution, we find,
\begin{align*}
P_{\text{curv}} & = \frac{2 e^2}{3 c^3} \frac{\gamma^4 v_{\parallel}^4}{r^2} 
\\
\nu_c & = \frac{3 \gamma^3}{4 \pi} \omega_0
\\
\omega_0 & = \frac{v_{\parallel}}{r}
\end{align*}

\subsection*{(b)}

For magnetic field line above a neutron star with a constant radius of curvature of,
\[ r_B = 5 \times 10^{8} \: \mathrm{cm} \]
and an electron  of $\gamma = 10^{7}$ we have $v_\parallel \approx c$ and therefore,
\begin{align*}
P_{\text{curv}} & = \frac{2 e^2 \gamma^4 c}{3 r^2}
\\
\nu_c & = \frac{3 \gamma^3}{4 \pi} \omega_0
\\
\omega_0 & = \frac{c}{r}
\end{align*}
Rearranging,
\begin{align*}
P_{\text{curv}} =  \frac{2 e^2}{3 m c^2 r} \cdot \frac{\gamma^4 m c^3}{r} = \frac{2}{3} \cdot \left( \frac{r}{r_0} \right)^{-1} \omega_0 \cdot \gamma^4 m c^2
\end{align*}
Plugging in we find,
\[ \omega_0 = 60.0 \: \mathrm{s}^{-1} \]
Therefore,
\[ \nu_0 = 1.43 \times 10^{22} \: \mathrm{Hz} \]
Finally,
\[ P_{\text{curv}} = 184.5 \: \mathrm{erg} \: \mathrm{s}^{-1} \]

\section{Problem 2}

\subsection*{(a)}

The average frequency of a photon after inverse Compton scattering,
\[ \EV{\nu_1} = \tfrac{4}{3} \gamma^2 \nu_0 \]
Equivalently we can express this in photon energy,
\[ \frac{E_1}{E_0} = \tfrac{4}{3} \gamma^2 \]
Therefore, starting with a CMB photon with, 
\[ E_0 = k_B T_{\mathrm{CMB}} = 2.35 \cdot 10^{-4} \: \mathrm{eV}  \]
and scattering into a $\gamma$-ray with,
\[ E_1 = 10^{12} \: \mathrm{eV} \]
Therefore,
\[ \gamma = \sqrt{\frac{3 E_1}{4 E_0}} = 5.65 \cdot 10^{7} \]
Such an electron has energy,
\[ E_{e^{-}} = \gamma m c^2 = 2.89 \cdot 10^{13} \: \mathrm{eV} \]
The fraction of energy lost by the electron in this scattering event is,
\[ \frac{\Delta E}{E} = \frac{E_1 - E_0}{E^{e^{-}}} = 0.034 \]
Finally, the lifetime of an electron undergoing inverse Compton scattering,
\[ t_{\mathrm{IC}} = \frac{3 \cdot 10^7 \: \mathrm{s}}{\gamma U_{\text{ph}}} \]
Now the energy density in the CMB is,
\[ U_{\text{ph}} = \left( \frac{8 \pi^5 k^4}{15 c^3 h^3} \right) T_{\text{CMB}}^4 \]
where, $T_{\text{CMB}} = 2.73 \: ^\circ \mathrm{K}$ then,
\[ U_{\text{ph}} = 4.20 \cdot 10^{-13} \: \mathrm{erg} \: \mathrm{cm}^{-3} \] 
Therefore,
\[ t_{\mathrm{IC}} = (7.14 \cdot 10^{19} \: \mathrm{s}) \: \gamma^{-1} = 1.26 \cdot 10^{12} \: \mathrm{s} \]

\subsection*{(b)}

Consider now the Galactic magnetic field $B = 3 \cdot 10^{-6} \: \mathrm{G}$. Electrons emit via synchrotron radiation at a characteristic frequency,
\[ \omega_c = \tfrac{3}{2} \gamma^3 \omega_B \quad \quad \quad \omega_B = \frac{e B}{\gamma m c} \]
Then the characteristic photon energy is,
\[ E_c = \frac{3 \hbar \gamma^2 e B}{2 m  c} = 2.94 \cdot 10^{-6} \: \mathrm{eV} \]
The lifetime under synchrotron radiation is,
\[ t_{\frac{1}{2}} = \left( \frac{2 e^4 B^2}{3 m^3 c^5} \cdot \gamma \right)^{-1} \]
Therefore,
\[ t_{\frac{1}{2}} = (5.1 \cdot 10^8 \: \mathrm{s}) \cdot \left( \frac{B}{1 \: \mathrm{G}} \right)^{-2} \cdot \gamma^{-1} \]  
Plugging in,
\[ t_{\frac{1}{2}} = 1.00 \cdot 10^{12} \: \mathrm{s} \]

\section{Problem 3}

A supernova remnant of radius $R = 3 \: \mathrm{pc}$ and age $T_{\text{age}} = 1000 \: \mathrm{yr}$ is detected in $E_{X} = 10 \: \mathrm{keV}$ energy X-rays and $E_{\gamma} = 10^{13} \: \mathrm{eV}$ energy $\gamma$-rays.

\subsection*{(a)}


From inverse Compton scattering,
\[ \gamma^2 = \frac{3 E_{\mathrm{IC}}}{4 E_{\text{CMB}}} \]
and from synchrotron radiation,
\[ E_{\mathrm{sync}} = \frac{3 \hbar \gamma^2 e B}{2 m c} \]
Therefore,
\[ E_{\mathrm{sync}} = \frac{9 \hbar e B}{8 m c} \left( \frac{E_{\mathrm{IC}}}{E_{\mathrm{CMB}}} \right) = \tfrac{9}{8} \lambdabar e B \left( \frac{E_{\mathrm{IC}}}{E_{\mathrm{CMB}}} \right) \]
where $\lambdabar$ is the Compton wavelength,
\[ \lambdabar = \frac{\hbar}{mc} = 3.86 \cdot 10^{-11} \: \mathrm{cm} \]
For the CMB at $T_{\text{CMB}} = 2.73 \:^\circ \mathrm{K}$ we have, $E_{\text{CMB}} = 2.35 \cdot 10^{-4} \: \mathrm{eV}$. Then,
\[ \frac{E_{\text{sync}}}{E_{\text{IC}}} = \frac{9 \lambdabar e B}{8 E_{\text{CMB}}} = \left( \frac{B}{18070 \: \mathrm{G}} \right) \]
In this case, we want the inverse Compton CMB scattering to emit in the X-ray and synchrotron radiation to emit in the $\gamma$-ray. Using our detection,
\[ \frac{E_{\text{sync}}}{E_{\text{IC}}} = \frac{E_{\gamma}}{E_X} = 10^9 \]
and therefore we require,
\[ B = 1.81 \cdot 10^{13} \: \mathrm{G} \]
Then we require that,
\[ \gamma = \sqrt{\frac{3 E_{\mathrm{IC}}}{4 E_{\mathrm{CMB}}}} = 5650 \]
However, this does not make much sense so we instead suppose that inverse Compton CMB scattering emits the $\gamma$-rays and sychrotron radiation is responsible for the X-ray emission. Then we require,
\[ \frac{E_{\text{sync}}}{E_{\text{IC}}} = \frac{E_{\gamma}}{E_X} = 10^{-9} \]
and thus,
\[ B = 1.81 \cdot 10^{-5} \: \mathrm{G} \]
Then we require that,
\[ \gamma = \sqrt{\frac{3 E_{\mathrm{IC}}}{4 E_{\mathrm{CMB}}}} = 1.79 \cdot 10^{8} \]

\subsection*{(b)}

Consider the electrons of part (a) in which we have computed their $\gamma$-factor and the magnetic field $B$ they experience. The gyroradius of these electrons is,
\[ r_g = \frac{\gamma m v_\perp c}{e B} \]
In the ultra-relativistic limit, $v_\perp \to c$ and thus,
\[ r_g = \frac{\gamma m c^2}{e B} = (1.70 \cdot 10^3 \: \mathrm{cm}) \cdot \gamma \left( \frac{B}{1 \: \mathrm{G}} \right)^{-1} \]
Therefore, in this case,
\[ r_g = 26 \: \mathrm{cm} \]
Furthermore, the lifetime for synchrotron radiation is,
\[ t_{\frac{1}{2}} = (5.1 \cdot 10^8 \: \mathrm{s}) \cdot \left( \frac{B}{1 \: \mathrm{G}} \right)^{-2} \cdot \gamma^{-1} \]  
which for our values is,
\[ t_{\frac{1}{2}} = 7.5 \cdot 10^{-7} \: \mathrm{s}  \]
The radius is easily consistent with the size of the nebula $R = 3 \: \mathrm{pc}$ but the lifetime is far too short for the remnant to still be emitting synchrotron radiation after $T_{\text{age}} = 1000 \: \mathrm{yr}$. 
\bigskip\\
This shows our assumption that the X-ray emission derives from inverse Compton scattering and the $\gamma$-ray emission from synchrotron radiation must be false. We now consider the opposite assumption. Then,
\[ \frac{E_{\text{sync}}}{E_{\text{IC}}} = \frac{E_X}{E_{\gamma}} = 10^{-9} \]
and therefore we require,
\[ B = 3.47 \cdot 10^{-13} \: \mathrm{G} \]
Then we further require that,
\[ \gamma = \sqrt{\frac{3 E_{\mathrm{IC}}}{4 E_{\mathrm{CMB}}}} = 1.79 \cdot 10^8 \]
Going back to our formulas for the gyroradius and lifetime we find,
\[ r_g = 8.74 \cdot 10^{23} \: \mathrm{cm} = 2.83 \cdot 10^{5} \: \mathrm{pc} \quad \quad \quad t_{\frac{1}{2}} = 2.37 \cdot 10^{26} \: \mathrm{s} \]
This time the lifetime is more than sufficient given the age of the remnant. However, now the gyroradius is far too large to fit within a remnant with size $R = 3 \: \mathrm{pc}$. 

\end{document}