\documentclass[12pt]{article}
\usepackage[english]{babel}
\usepackage[utf8]{inputenc}
\usepackage[english]{babel}
\usepackage[a4paper, total={7.25in, 9.5in}]{geometry}
\usepackage{tikz-feynman}
\tikzfeynmanset{compat=1.0.0} 
\usepackage{subcaption}
\usepackage{float}
\floatplacement{figure}{H}
\usepackage{mathrsfs}  
\usepackage{dsfont}
\usepackage{relsize}
\DeclareMathAlphabet{\mathdutchcal}{U}{dutchcal}{m}{n}

\usepackage{revsymb}


\newcommand{\field}{\hat{\Phi}}
\newcommand{\dfield}{\hat{\Phi}^\dagger}
 
\usepackage{amsthm, amssymb, amsmath, centernot}
\usepackage{slashed}
\newcommand{\notimplies}{%
  \mathrel{{\ooalign{\hidewidth$\not\phantom{=}$\hidewidth\cr$\implies$}}}}
 
\renewcommand\qedsymbol{$\square$}
\newcommand{\cont}{$\boxtimes$}
\newcommand{\divides}{\mid}
\newcommand{\ndivides}{\centernot \mid}

\newcommand{\Integers}{\mathbb{Z}}
\newcommand{\Natural}{\mathbb{N}}
\newcommand{\Complex}{\mathbb{C}}
\newcommand{\Zplus}{\mathbb{Z}^{+}}
\newcommand{\Primes}{\mathbb{P}}
\newcommand{\Q}{\mathbb{Q}}
\newcommand{\R}{\mathbb{R}}
\newcommand{\ball}[2]{B_{#1} \! \left(#2 \right)}
\newcommand{\Rplus}{\mathbb{R}^+}
\renewcommand{\Re}[1]{\mathrm{Re}\left[ #1 \right]}
\renewcommand{\Im}[1]{\mathrm{Im}\left[ #1 \right]}
\newcommand{\Op}{\mathcal{O}}

\newcommand{\invI}[2]{#1^{-1} \left( #2 \right)}
\newcommand{\End}[1]{\text{End}\left( A \right)}
\newcommand{\legsym}[2]{\left(\frac{#1}{#2} \right)}
\renewcommand{\mod}[3]{\: #1 \equiv #2 \: \mathrm{mod} \: #3 \:}
\newcommand{\nmod}[3]{\: #1 \centernot \equiv #2 \: mod \: #3 \:}
\newcommand{\ndiv}{\hspace{-4pt}\not \divides \hspace{2pt}}
\newcommand{\finfield}[1]{\mathbb{F}_{#1}}
\newcommand{\finunits}[1]{\mathbb{F}_{#1}^{\times}}
\newcommand{\ord}[1]{\mathrm{ord}\! \left(#1 \right)}
\newcommand{\quadfield}[1]{\Q \small(\sqrt{#1} \small)}
\newcommand{\vspan}[1]{\mathrm{span}\! \left\{#1 \right\}}
\newcommand{\galgroup}[1]{Gal \small(#1 \small)}
\newcommand{\bra}[1]{\left| #1 \right>}
\newcommand{\Oa}{O_\alpha}
\newcommand{\Od}{O_\alpha^{\dagger}}
\newcommand{\Oap}{O_{\alpha '}}
\newcommand{\Odp}{O_{\alpha '}^{\dagger}}
\newcommand{\im}[1]{\mathrm{im} \: #1}
\renewcommand{\ker}[1]{\mathrm{ker} \: #1}
\newcommand{\ket}[1]{\left| #1 \right>}
\renewcommand{\bra}[1]{\left< #1 \right|}
\newcommand{\inner}[2]{\left< #1 | #2 \right>}
\newcommand{\expect}[2]{\left< #1 \right| #2 \left| #1 \right>}
\renewcommand{\d}[1]{\: \mathrm{d}#1 \:}
\newcommand{\dn}[2]{ \mathrm{d}^{#1} #2 \:}
\newcommand{\deriv}[2]{\frac{\d{#1}}{\d{#2}}}
\newcommand{\nderiv}[3]{\frac{\dn{#1}{#2}}{\d{#3^{#1}}}}
\newcommand{\pderiv}[2]{\frac{\partial{#1}}{\partial{#2}}}
\newcommand{\fderiv}[2]{\frac{\delta #1}{\delta #2}}
\newcommand{\parsq}[2]{\frac{\partial^2{#1}}{\partial{#2}^2}}
\newcommand{\topo}{\mathcal{T}}
\newcommand{\base}{\mathcal{B}}
\renewcommand{\bf}[1]{\mathbf{#1}}
\renewcommand{\a}{\hat{a}}
\newcommand{\adag}{\hat{a}^\dagger}
\renewcommand{\b}{\hat{b}}
\newcommand{\bdag}{\hat{b}^\dagger}
\renewcommand{\c}{\hat{c}}
\newcommand{\cdag}{\hat{c}^\dagger}
\newcommand{\hamilt}{\hat{H}}
\renewcommand{\L}{\hat{L}}
\newcommand{\Lz}{\hat{L}_z}
\newcommand{\Lsquared}{\hat{L}^2}
\renewcommand{\S}{\hat{S}}
\renewcommand{\empty}{\varnothing}
\newcommand{\J}{\hat{J}}
\newcommand{\lagrange}{\mathcal{L}}
\newcommand{\dfourx}{\mathrm{d}^4x}
\newcommand{\meson}{\phi}
\newcommand{\dpsi}{\psi^\dagger}
\newcommand{\ipic}{\mathrm{int}}
\newcommand{\tr}[1]{\mathrm{tr} \left( #1 \right)}
\newcommand{\C}{\mathbb{C}}
\newcommand{\CP}[1]{\mathbb{CP}^{#1}}
\newcommand{\Vol}[1]{\mathrm{Vol}\left(#1\right)}

\newcommand{\Tr}[1]{\mathrm{Tr}\left( #1 \right)}
\newcommand{\Charge}{\hat{\mathbf{C}}}
\newcommand{\Parity}{\hat{\mathbf{P}}}
\newcommand{\Time}{\hat{\mathbf{T}}}
\newcommand{\Torder}[1]{\mathbf{T}\left[ #1 \right]}
\newcommand{\Norder}[1]{\mathbf{N}\left[ #1 \right]}
\newcommand{\Znorm}{\mathcal{Z}}
\newcommand{\EV}[1]{\left< #1 \right>}
\newcommand{\interact}{\mathrm{int}}
\newcommand{\covD}{\mathcal{D}}
\newcommand{\conj}[1]{\overline{#1}}

\newcommand{\SO}[2]{\mathrm{SO}(#1, #2)}
\newcommand{\SU}[2]{\mathrm{SU}(#1, #2)}

\newcommand{\anticom}[2]{\left\{ #1 , #2 \right\}}


\newcommand{\pathd}[1]{\! \mathdutchcal{D} #1 \:}

\renewcommand{\theenumi}{(\alph{enumi})}


\renewcommand{\theenumi}{(\alph{enumi})}

\newcommand{\atitle}[1]{\title{% 
	\large \textbf{ASTR GR6001 Radiative Processes
	\\ Assignment \# #1} \vspace{-2ex}}
\author{Benjamin Church }
\maketitle}

\theoremstyle{definition}
\newtheorem{theorem}{Theorem}[section]
\newtheorem{definition}{definition}[section]
\newtheorem{lemma}[theorem]{Lemma}
\newtheorem{proposition}[theorem]{Proposition}
\newtheorem{corollary}[theorem]{Corollary}
\newtheorem{example}[theorem]{Example}
\newtheorem{remark}[theorem]{Remark}
\begin{document}


\atitle{10}

\section{Problem 1}

\subsection*{(a)}

Recall that we split the Crab spectrum into a broken power law with two segments in frequency, $10^{7} \: \mathrm{Hz} - 10^{14} \: \mathrm{Hz}$ with spectral index $s_1 \approx 0.25$ and $10^{14} \: \mathrm{Hz} - 10^{24} \: \mathrm{Hz}$. The end points of these segments gives the minimum and maximum $\gamma$-factors and their slopes give the values of $s$ and $p$. We need to do a linear (in log-log) fit to each segment. Define the parameter,
\[ g = \left( \frac{f_\nu}{10^{-20} \: \mathrm{erg} \: \mathrm{cm}^{-2} \: \mathrm{s}^{-1} \: \mathrm{Hz}^{-1}} \right) \cdot \left( \frac{\nu}{4 \cdot 10^6 \: \mathrm{Hz}} \right)^{s} \]  
with $s$ chosen to fit a segment. This number is approximately constant on each segment.
From linear fits to the data (note that the data is given in $\mathrm{W} \: \mathrm{m}^{-2} \: \mathrm{Hz}^{-1} = 10^{-3} \: \mathrm{erg} \: \mathrm{cm}^{-2} \: \mathrm{Hz}^{-1}$), I find,
\begin{center}
 \begin{tabular}{||c c c c c c c||} 
 \hline
 segment & $s$ & $p$ & $a(p)$ & $g$ & $\gamma_{\text{min}}$ & $\gamma_{\text{max}}$
 \\ [0.5ex] 
 \hline\hline
 $1$ & $0.25$ & $1.5$ & $0.085$ & $7.07$ & $79$ & $2.5 \cdot 10^5$ \\ 
 \hline
 $2$ & $1.20$ & $3.4$ & $0.074$ & $7.54 \cdot 10^7$ & $2.5 \cdot 10^5$ & $2.5 \cdot 10^{10}$ \\
 \hline
\end{tabular}
\end{center}

Then the total luminosity of the Crab is,
\begin{align*}
L & = 4 \pi d^2 \int_{0}^{\infty} f_\nu \d{\nu}
\\
& = 4 \pi d^2 \:  (10^{-20} \: \mathrm{erg} \: \mathrm{cm}^{-2} \: \mathrm{s}^{-1} \: \mathrm{Hz}^{-1})  \: \sum_{\text{segment}} g \left[ \int_{\nu_{\text{min}}}^{\nu_{\text{max}}}  \left( \frac{\nu}{4 \cdot 10^6 \: \mathrm{Hz}} \right)^{-s} \d{\nu} \right]
\end{align*}
The Crab is at a distance of $d = 2000 \: \mathrm{pc}$ so define,
\[ x = \frac{\nu}{4 \cdot 10^6 \: \mathrm{Hz}} \]
and then,
\begin{align*}
L & = 4 \pi d^2 \:  (10^{-20} \: \mathrm{erg} \: \mathrm{cm}^{-2} \: \mathrm{s}^{-1} \: \mathrm{Hz}^{-1})  \: \sum_{\text{segment}} g \left[ \int_{x_{\text{min}}}^{x_{\text{max}}}  x^{-s} \d{x} \right]
\\
& = (1.91 \cdot 10^{31} \: \mathrm{erg} \: \mathrm{s}^{-1}) \sum_{\text{segment}} \frac{g}{s - 1} \left[ \frac{1}{x_{\text{min}}^{s - 1}} - \frac{1}{x_{\text{max}}^{s - 1}} \right]  
\\
& = 3.858 \cdot 10^{38} \: \mathrm{erg} \: \mathrm{s}^{-1}
\end{align*}

\subsection*{(b)}

The spectrum of electron energies can be computed from the synchrotron emission spectrum of the nebula. The emission spectrum is computed in the notes as,
\[ \mathcal{E}_\nu = (1.7 \cdot 10^{-21} \: \mathrm{erg} \: \mathrm{cm}^{-3} \: \mathrm{s}^{-1} \: \mathrm{Hz}^{-1})  \: \left( \frac{n_0}{1 \: \mathrm{cm}^{-3}} \right) a(p) \left( \frac{B}{1 \: \mathrm{G}} \right)^{s + 1} \cdot \left( \frac{\nu}{4 \cdot 10^6 \: \mathrm{Hz}} \right)^{-s} \]
where $s$ is the spectral index of the radiation and $a(p)$ is a numerical value depending on the index $p = 2 s + 1$. Furthermore, the $n_0$ which appears above is the specific number density appearing in the power-law distribution for electron energy distribution,
\[ n(\gamma) = n_0 \gamma^{-p} \]
and therefore, the energy in emitting electrons is,
\[ E_{\text{seg}} = V \int_{\gamma_{\text{min}}}^{\gamma_{\text{max}}} (\gamma m c^2) n(\gamma) \d{\gamma} = V \int_{\gamma_{\text{min}}}^{\gamma_{\text{max}}} (\gamma m c^2) n_0 \gamma^{-p} \d{\gamma} = V \frac{n_0 m c^2}{p - 2} \left[ \frac{1}{\gamma_{\text{min}}^{p - 2}} - \frac{1}{\gamma_{\text{max}}^{p - 2}} \right] \]
Therefore, it suffices to find $p$, $n_0$, and $\gamma_{\text{min}}$ and $\gamma_{\text{max}}$. 
\bigskip\\
Our available information is the observed specific flux spectrum which is related to $\mathcal{E}_\nu$ via the distance to the cluster and the total volume via,
\[ f_{\nu} = \frac{\mathcal{E}_\nu V}{4 \pi d^2} \]
since $\mathcal{E}_\nu V$ is the total power emitted by the nebula. Therefore, we have,
\[ n_0 = (1 \: \mathrm{cm}^{-3}) \cdot \frac{4 \pi d^2 f_\nu}{V} \cdot (1.7 \cdot 10^{-21} \: \mathrm{erg} \: \mathrm{cm}^{-3} \: \mathrm{s}^{-1} \: \mathrm{Hz}^{-1})^{-1} a(p)^{-1} \cdot \left( \frac{B}{1 \: \mathrm{G}} \right)^{-(s + 1)} \cdot \left( \frac{\nu}{4 \cdot 10^6 \: \mathrm{Hz}} \right)^{s} \]
Computing this we will find the total energy in emitting electrons,
\begin{align*}
E & = V \sum_{\text{segments}} \frac{n_0 m c^2}{p - 2} \left[ \frac{1}{\gamma_{\text{min}}^{p - 2}} - \frac{1}{\gamma_{\text{max}}^{p - 2}} \right] 
\\
& = (1 \: \mathrm{cm}^{-3}) \cdot (4 \pi d^2 f_\nu) \cdot (1.7 \cdot 10^{-21} \: \mathrm{erg} \: \mathrm{cm}^{-3} \: \mathrm{s}^{-1} \: \mathrm{Hz}^{-1})^{-1} a(p)^{-1} \cdot \left( \frac{B}{1 \: \mathrm{G}} \right)^{-(s + 1)} \cdot \left( \frac{\nu}{4 \cdot 10^6 \: \mathrm{Hz}} \right)^{s} 
\\
& \quad \cdot \sum_{\text{segments}} \frac{m c^2}{p - 2} \left[ \frac{1}{\gamma_{\text{min}}^{p - 2}} - \frac{1}{\gamma_{\text{max}}^{p - 2}} \right] 
\end{align*} 
\bigskip\\
Again, we split the Crab spectrum into a broken power law with two segments in frequency, $10^{7} \: \mathrm{Hz} - 10^{14} \: \mathrm{Hz}$ with spectral index $s_1 \approx 0.25$ and $10^{14} \: \mathrm{Hz} - 10^{24} \: \mathrm{Hz}$. The end points of these segments gives the minimum and maximum $\gamma$-factors and their slopes give the values of $s$ and $p$. We need to do a linear (in log-log) fit to each segment. Define the parameter,
\[ g = \left( \frac{f_\nu}{10^{-20} \: \mathrm{erg} \: \mathrm{cm}^{-2} \: \mathrm{s}^{-1} \: \mathrm{Hz}^{-1}} \right) \cdot \left( \frac{\nu}{4 \cdot 10^6 \: \mathrm{Hz}} \right)^{s} \]  
with $s$ chosen to fit a segment. This number is approximately constant on each segment. In terms of the parameters, $g$ and $s$ and plugging in for $B = 5 \cdot 10^{-4} \: \mathrm{G}$ and $d = 2000 \: \mathrm{pc}$ we find,
\begin{align*}
E & = (2.8 \cdot 10^{45}) \sum_{\text{segments}} \left( 5 \cdot 10^{-4} \right)^{-(s+1)} \cdot \frac{g mc^2}{a(p)(p-2)} \cdot \left[ \frac{1}{\gamma_{\text{min}}^{p - 2}} - \frac{1}{\gamma_{\text{max}}^{p - 2}} \right]
\\
& = (2.29 \cdot 10^{39}) \sum_{\text{segments}} \left( 5 \cdot 10^{-4} \right)^{-(s+1)} \cdot \frac{g}{a(p)(p-2)} \cdot \left[ \frac{1}{\gamma_{\text{min}}^{p - 2}} - \frac{1}{\gamma_{\text{max}}^{p - 2}} \right]
\end{align*}
From linear fits to the data (note that the data is given in $\mathrm{W} \: \mathrm{m}^{-2} \: \mathrm{Hz}^{-1} = 10^{-3} \: \mathrm{erg} \: \mathrm{cm}^{-2} \: \mathrm{Hz}^{-1}$), I find,
\begin{center}
 \begin{tabular}{||c c c c c c c||} 
 \hline
 segment & $s$ & $p$ & $a(p)$ & $g$ & $\gamma_{\text{min}}$ & $\gamma_{\text{max}}$
 \\ [0.5ex] 
 \hline\hline
 $1$ & $0.25$ & $1.5$ & $0.085$ & $7.07$ & $79$ & $2.5 \cdot 10^5$ \\ 
 \hline
 $2$ & $1.20$ & $3.4$ & $0.074$ & $7.54 \cdot 10^7$ & $2.5 \cdot 10^5$ & $2.5 \cdot 10^{10}$ \\
 \hline
\end{tabular}
\end{center}
Therefore, plugging in,
\[ E_{\text{el}} = 3.35 \cdot 10^{48} \: \mathrm{erg} \]
Furthermore, the energy in the magnetic field is,
\[ E_{\text{mag}} = V \cdot \left( \frac{B^2}{8 \pi} \right) \]
Plugging in,
\[ E_{\text{mag}} = 2.98 \cdot 10^{48} \: \mathrm{erg} \]
and therefore the cluster is approximately in equipartition between the emitting electrons and the magnetic field energy. 

\section{Problem 2}

Consider a cluster of galaxies containing $10^{13} \: M_{\odot}$ of ionized plasma in the intracluster medium, in a volume of $0.1 \: \mathrm{Mpc}^3$, at a temperature of $5 \times 10^7 \: \mathrm{K}$. We assume the composition is $90 \% \: \mathrm{H}$ and $10 \% \: \mathrm{He}$. 

\subsection*{(a)}

We can use the formula for Bremsstrahlung luminosity,
\[ \mathcal{E} = (1.4 \times 10^{-27} \: \mathrm{erg} \: \mathrm{cm}^{-3} \: \mathrm{s}^{-1}) \: \left( \frac{T}{1 \: \mathrm{K}} \right)^{\frac{1}{2}} \: (1 \: \mathrm{cm}^6) \sum_i n_e n_i Z^2 \overline{g}_{ff}(T) \] 
For $T > 10^{6} \: \mathrm{K}$ we may take the Gaunt factor to be $\overline{g}_{ff}(T) \approx 1.2$. Furthermore, for the given abundances,
\[ \sum_i n_e n_i Z^2 = 1.4 \: n_H^2 \]
Therefore, we find,
\[ \mathcal{E} = (2.4 \cdot 10^{-27} \: \mathrm{erg} \: \mathrm{cm}^{-3} \: \mathrm{s}^{-1}) \: \left( \frac{T}{1 \: \mathrm{K}} \right)^{\frac{1}{2}} \cdot \left( \frac{n_H}{1 \: \mathrm{cm}} \right)^2 \]
Similarly, the mass density of the plasma is $\rho = 1.4 \: m_H n_H$. Therefore,
\[ n_H = \frac{M}{1.4 \: V m_H} \]
Plugging in for the mass and volume of the plasma,
\[ n_H = 2.89 \cdot 10^{-3} \: \mathrm{cm}^{-3} \]
Then plugging in to our formula,
\[ L = \mathcal{E} V = 2.43 \cdot 10^{44} \: \mathrm{erg} \: \mathrm{s}^{-1} \] 

\subsection*{(b)}

The Luminosity is proportional to $T^{\frac{1}{2}}$ i.e. we have,
\[ L = A T^{\frac{1}{2}} \]
Note that the energy for a relativistic gas is,
\[ E = 3 N k T \]
and thus,
\[ L = \dot{E} = 3 N k \dot{T} = - A T^{\frac{1}{2}} \]
This has a solution,
\[ T(t) = \left( T_0^{\frac{1}{2}} - \frac{A t}{6 N k}  \right)^2 \]
Therefore, the lifetime is,
\[ t = \frac{6 N k T_0^{\frac{1}{2}}}{A} = \frac{6 N k T_0}{A T_0^{\frac{1}{2}}} = \frac{2 E}{L} \]
Plugging in we find,
\[ t = 4.59  \cdot 10^{10} \: \mathrm{yr} \]

\subsection*{(c)}

The Compton $y$ parameter is defined as,
\[ y = \int n_e \frac{k T_e}{m c^2} \sigma_\tau \d{\ell} \approx \frac{k T}{m c^2} n_e \sigma_\tau L  \]
where $L$ is the line of sight, $\sigma_\tau$ is the Thomson cross section, $T_e$ is the electron density, and $n_e$ is the electron density. The density of electrons is, $n_e = 1.1 \, n_H$. Assuming the cloud is approximately spherical, the line of sight is,
\[ L = \left( \frac{3 V}{4 \pi} \right)^{\frac{1}{3}} \] 
Therefore, plugging in, we find,
\[ y = 4.02 \cdot 10^{-5} \]
The magnitude of the Sunyaev-Zel’dovich effect is given by,
\[ \frac{\Delta T_{\text{SZE}}}{T_{\text{CMB}}} = \left( x \left( \frac{e^x + 1}{e^x - 1} \right) - 4 \right) \int n_e \frac{k T_e}{m c^2} \sigma_\tau \d{\ell} \]
Where,
\[ x = \frac{h \nu}{k T_{\text{CMB}}} = \left( \frac{\nu}{57 \: \mathrm{GHz}} \right) \]
Then consider a frequency of $\nu = 120 \: \mathrm{GHz}$ we have $x = 2.11$. Therefore,
\[ \frac{\Delta T_{\text{SZE}}}{T_{\text{CMB}}} = -2.30 \cdot 10^{-4} \]


\section{Problem 3}

Consider Photo-ionized broad emission-line ``clouds'' in quasars are expected to have density $n \approx 10^{10} \: \mathrm{cm}^{-3}$, column density $N \approx 10^{22} \: \mathrm{cm}^{-2}$, and temperature $T = 15,000 \: ^\circ \mathrm{K}$. Now we consider the effects of free-free absorption. The specific optical depth is approximately,
\[ \tau_\nu = N \alpha_\nu \] 
Furthermore, in the Rayleigh - Jeans limit we can compute the absorption from free-free processes from the bremsstrahlung emission. This gives,
\begin{align*}
\alpha_\nu & = \frac{4 e^6}{3 m h c} \left( \frac{2 \pi}{3 k m} \right)^{\frac{1}{2}} \cdot \left( \frac{T}{1 \: \mathrm{K}} \right)^{- \frac{1}{2}} \sum_i \frac{n_e n_i Z^2}{\nu^3} \left( 1 - e^{-h \nu / k T} \right) \overline{g}_{ff}
\\
& = (0.018 \: \mathrm{cm}^2 ) \: \left( \frac{T}{1 \: \mathrm{K}} \right)^{- \frac{3}{2}} \: \left( \frac{\nu}{1 \: \mathrm{Hz}} \right)^{-2} \: (1 \: \mathrm{cm}^6) \: \sum_i n_e n_i Z^2 \: \overline{g}_{ff}  
\end{align*}
We know that $n_e = 0.9 \, n + 0.2 \, n = 1.1 \, n$ and we have,
\[ \sum_i n_e n_i Z^2 = 1.4 \, n^2 \]
Furthermore, since the temperature is much higher than $10^6 \: ^\circ \mathrm{K}$ then $\overline{g}_{ff} \approx 1.2$. Therefore, we find,
\[ \tau_\nu = (1.65 \cdot 10^{34}) \: \left( \frac{\nu}{1 \: \mathrm{Hz}} \right)^{-2} \]
This implies that the cloud is optically thick due to free-free absorption for frequencies less than,
\[ \nu_0 = 1.28 \cdot 10^{17} \: \mathrm{Hz}  \]

\end{document}