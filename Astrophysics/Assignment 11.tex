\documentclass[12pt]{article}
\usepackage[english]{babel}
\usepackage[utf8]{inputenc}
\usepackage[english]{babel}
\usepackage[a4paper, total={7.25in, 9.5in}]{geometry}
\usepackage{tikz-feynman}
\tikzfeynmanset{compat=1.0.0} 
\usepackage{subcaption}
\usepackage{float}
\floatplacement{figure}{H}
\usepackage{mathrsfs}  
\usepackage{dsfont}
\usepackage{relsize}
\DeclareMathAlphabet{\mathdutchcal}{U}{dutchcal}{m}{n}

\usepackage{revsymb}


\newcommand{\field}{\hat{\Phi}}
\newcommand{\dfield}{\hat{\Phi}^\dagger}
 
\usepackage{amsthm, amssymb, amsmath, centernot}
\usepackage{slashed}
\newcommand{\notimplies}{%
  \mathrel{{\ooalign{\hidewidth$\not\phantom{=}$\hidewidth\cr$\implies$}}}}
 
\renewcommand\qedsymbol{$\square$}
\newcommand{\cont}{$\boxtimes$}
\newcommand{\divides}{\mid}
\newcommand{\ndivides}{\centernot \mid}

\newcommand{\Integers}{\mathbb{Z}}
\newcommand{\Natural}{\mathbb{N}}
\newcommand{\Complex}{\mathbb{C}}
\newcommand{\Zplus}{\mathbb{Z}^{+}}
\newcommand{\Primes}{\mathbb{P}}
\newcommand{\Q}{\mathbb{Q}}
\newcommand{\R}{\mathbb{R}}
\newcommand{\ball}[2]{B_{#1} \! \left(#2 \right)}
\newcommand{\Rplus}{\mathbb{R}^+}
\renewcommand{\Re}[1]{\mathrm{Re}\left[ #1 \right]}
\renewcommand{\Im}[1]{\mathrm{Im}\left[ #1 \right]}
\newcommand{\Op}{\mathcal{O}}

\newcommand{\invI}[2]{#1^{-1} \left( #2 \right)}
\newcommand{\End}[1]{\text{End}\left( A \right)}
\newcommand{\legsym}[2]{\left(\frac{#1}{#2} \right)}
\renewcommand{\mod}[3]{\: #1 \equiv #2 \: \mathrm{mod} \: #3 \:}
\newcommand{\nmod}[3]{\: #1 \centernot \equiv #2 \: mod \: #3 \:}
\newcommand{\ndiv}{\hspace{-4pt}\not \divides \hspace{2pt}}
\newcommand{\finfield}[1]{\mathbb{F}_{#1}}
\newcommand{\finunits}[1]{\mathbb{F}_{#1}^{\times}}
\newcommand{\ord}[1]{\mathrm{ord}\! \left(#1 \right)}
\newcommand{\quadfield}[1]{\Q \small(\sqrt{#1} \small)}
\newcommand{\vspan}[1]{\mathrm{span}\! \left\{#1 \right\}}
\newcommand{\galgroup}[1]{Gal \small(#1 \small)}
\newcommand{\bra}[1]{\left| #1 \right>}
\newcommand{\Oa}{O_\alpha}
\newcommand{\Od}{O_\alpha^{\dagger}}
\newcommand{\Oap}{O_{\alpha '}}
\newcommand{\Odp}{O_{\alpha '}^{\dagger}}
\newcommand{\im}[1]{\mathrm{im} \: #1}
\renewcommand{\ker}[1]{\mathrm{ker} \: #1}
\newcommand{\ket}[1]{\left| #1 \right>}
\renewcommand{\bra}[1]{\left< #1 \right|}
\newcommand{\inner}[2]{\left< #1 | #2 \right>}
\newcommand{\expect}[2]{\left< #1 \right| #2 \left| #1 \right>}
\renewcommand{\d}[1]{\: \mathrm{d}#1 \:}
\newcommand{\dn}[2]{ \mathrm{d}^{#1} #2 \:}
\newcommand{\deriv}[2]{\frac{\d{#1}}{\d{#2}}}
\newcommand{\nderiv}[3]{\frac{\dn{#1}{#2}}{\d{#3^{#1}}}}
\newcommand{\pderiv}[2]{\frac{\partial{#1}}{\partial{#2}}}
\newcommand{\fderiv}[2]{\frac{\delta #1}{\delta #2}}
\newcommand{\parsq}[2]{\frac{\partial^2{#1}}{\partial{#2}^2}}
\newcommand{\topo}{\mathcal{T}}
\newcommand{\base}{\mathcal{B}}
\renewcommand{\bf}[1]{\mathbf{#1}}
\renewcommand{\a}{\hat{a}}
\newcommand{\adag}{\hat{a}^\dagger}
\renewcommand{\b}{\hat{b}}
\newcommand{\bdag}{\hat{b}^\dagger}
\renewcommand{\c}{\hat{c}}
\newcommand{\cdag}{\hat{c}^\dagger}
\newcommand{\hamilt}{\hat{H}}
\renewcommand{\L}{\hat{L}}
\newcommand{\Lz}{\hat{L}_z}
\newcommand{\Lsquared}{\hat{L}^2}
\renewcommand{\S}{\hat{S}}
\renewcommand{\empty}{\varnothing}
\newcommand{\J}{\hat{J}}
\newcommand{\lagrange}{\mathcal{L}}
\newcommand{\dfourx}{\mathrm{d}^4x}
\newcommand{\meson}{\phi}
\newcommand{\dpsi}{\psi^\dagger}
\newcommand{\ipic}{\mathrm{int}}
\newcommand{\tr}[1]{\mathrm{tr} \left( #1 \right)}
\newcommand{\C}{\mathbb{C}}
\newcommand{\CP}[1]{\mathbb{CP}^{#1}}
\newcommand{\Vol}[1]{\mathrm{Vol}\left(#1\right)}

\newcommand{\Tr}[1]{\mathrm{Tr}\left( #1 \right)}
\newcommand{\Charge}{\hat{\mathbf{C}}}
\newcommand{\Parity}{\hat{\mathbf{P}}}
\newcommand{\Time}{\hat{\mathbf{T}}}
\newcommand{\Torder}[1]{\mathbf{T}\left[ #1 \right]}
\newcommand{\Norder}[1]{\mathbf{N}\left[ #1 \right]}
\newcommand{\Znorm}{\mathcal{Z}}
\newcommand{\EV}[1]{\left< #1 \right>}
\newcommand{\interact}{\mathrm{int}}
\newcommand{\covD}{\mathcal{D}}
\newcommand{\conj}[1]{\overline{#1}}

\newcommand{\SO}[2]{\mathrm{SO}(#1, #2)}
\newcommand{\SU}[2]{\mathrm{SU}(#1, #2)}

\newcommand{\anticom}[2]{\left\{ #1 , #2 \right\}}


\newcommand{\pathd}[1]{\! \mathdutchcal{D} #1 \:}

\renewcommand{\theenumi}{(\alph{enumi})}


\renewcommand{\theenumi}{(\alph{enumi})}

\newcommand{\atitle}[1]{\title{% 
	\large \textbf{ASTR GR6001 Radiative Processes
	\\ Assignment \# #1} \vspace{-2ex}}
\author{Benjamin Church }
\maketitle}

\theoremstyle{definition}
\newtheorem{theorem}{Theorem}[section]
\newtheorem{definition}{definition}[section]
\newtheorem{lemma}[theorem]{Lemma}
\newtheorem{proposition}[theorem]{Proposition}
\newtheorem{corollary}[theorem]{Corollary}
\newtheorem{example}[theorem]{Example}
\newtheorem{remark}[theorem]{Remark}
\begin{document}


\atitle{11}

\newcommand{\DM}{\mathrm{DM}}
\newcommand{\RM}{\mathrm{RM}}
\newcommand{\cm}{\mathrm{cm}}
\newcommand{\pc}{\mathrm{pc}}

\section{Problem 1}

We have computed the Faraday rotation from an electron plasma to be,
\[ \Delta \theta = \frac{2 \pi e^3}{(m_e c)^2 \omega^2} \int_0^d n_e B_\parallel \d{s} \]
Then,
\[ \omega = \frac{2 \pi c}{\lambda} \] 
and thus,
\[ \Delta \theta = \frac{e^3 \lambda^2}{2 \pi m_e^2 c^4} \int_0^d n_e B_\parallel \d{s} \]
Recall the classical electron radius is,
\[ r_0 = \frac{e^2}{m_e c^2} \]
Therefore, we find,
\[ \Delta \theta = \frac{r_0 \lambda^2 e}{2 \pi m_e c^2} \int_0^d n_e B_\parallel \d{s} \]
Therefore, plugging in constants we find,
\[ \Delta \theta = (0.811 \: \mathrm{rad})  \left( \frac{\lambda}{1 \: \mathrm{m}} \right)^2 \int_0^d \left( \frac{n_e}{1 \: \cm^{-3}} \right) \left( \frac{B}{10^{-6} \: \mathrm{G}} \right) \frac{\d{s}}{1 \: \pc} \] 
Therefore, if we write,
\[ \Delta \theta = \RM \: \lambda^2 \]
expressing $\RM$ in units $\mathrm{rad} \: \mathrm{m}^{-2}$. Then we find,
\[ \RM = (0.811 \: \mathrm{rad} \: \mathrm{m}^{-2})  \int_0^d \left( \frac{n_e}{1 \: \cm^{-3}} \right) \left( \frac{B}{10^{-6} \: \mathrm{G}} \right) \frac{\d{s}}{1 \: \pc} \] 

\section{Problem 2}

The Crab pulsar has $\DM = 56.7 \: \cm^{-3} \: \pc$ and $\RM = 42.3 \: \mathrm{rad} \: \mathrm{m}^{-2}$.

\subsection*{(a)}

The definitions of the dispersion and rotation measures are as follows,
\begin{align*}
\DM & =  \int_0^d n_e \d{s}
\\
\RM & = (0.811 \: \mathrm{rad} \: \mathrm{m}^{-2})  \int_0^d \left( \frac{n_e}{1 \: \cm^{-3}} \right) \left( \frac{B}{10^{-6} \: \mathrm{G}} \right) \frac{\d{s}}{1 \: \pc}
\end{align*}
Therefore, the average component of the magnetic field along the line of sight is,
\begin{align*}
\left< B_\parallel \right> = \frac{\int_0^d n_e B_\parallel \d{s}}{\int_0^d n_e \d{s}} = \frac{\RM}{\DM} \cdot (1 \: \cm^{-3}) \cdot (10^{-6} \: \mathrm{G}) \cdot (1 \: \mathrm{pc}) \cdot (0.811 \: \mathrm{rad} \: \mathrm{m}^{-2}) ^{-1}
\end{align*}
Thus, plugging in, we find,
\[ \left< B_\parallel \right> = 9.20 \cdot 10^{-7} \: \mathrm{G} \]

\subsection*{(b)}

The dispersion delay is given by,
\[ t_D = \frac{2 \pi e^2}{m c \omega^2} \DM = \frac{r_0 c}{2 \pi \nu^2} \DM \]
Therefore, for $\nu = 400 \: \mathrm{MHz}$ we find a delay,
\[ t_D = 1.47 \: \mathrm{s} \]
which is the delay of a pulse at frequency $\nu = 400 \: \mathrm{MHz}$ as compared to a pulse at infinite frequency which under this formula corresponds to zero frequency shift. 

\subsection*{(c)}

The differential dispersion for some frequency spacing $\Delta \nu$ can be computed from the derivative of the dispersion delay as follows,
\[ \Delta t_{D} = \frac{2 r_0 c}{\nu^2} \left( \frac{\Delta \nu}{\nu} \right) \DM = \left( \frac{\Delta \nu}{\nu} \right) 2 t_D \]
Therefore, for a frequency band centered at $\nu = 400 \: \mathrm{MHz}$, a differential dispersion delay of $t_D = 0.016 \: \mathrm{s}$ (half the $0.033 \: \mathrm{s}$ period of the Crab pulsar) corresponds to frequency band of width,
\[ \frac{\Delta \nu}{\nu} = \frac{\Delta t_D}{2 t_D} = 5.61 \cdot 10^{-3} \]
which gives the width of the spacing as,
\[ \Delta \nu = 2.24 \: \mathrm{MHz} \]

\subsection*{(d)}

We have computed the total rotation by Faraday rotation as follows,
\[ \theta = \RM \: \lambda^2 = c^2 \RM \: \nu^{-2} \]
Then the difference between the Faraday rotation angle across a frequency spacing $\Delta \nu$ is,
\[ \Delta \theta = 2 c^2 \RM \: \nu^{-2} \: \left( \frac{\Delta \nu}{\nu} \right) \]
Therefore, we find a frequency spacing corresponding to a differential rotation of $1 \: \mathrm{rad}$ of,
\[ \frac{\Delta \nu}{\nu} = \frac{\Delta \theta \nu^2}{2 c^2 \RM} = 2.10 \cdot 10^{-2} \]
Therefore,
\[ \Delta \nu = 8.42 \: \mathrm{MHz} \]

\section{Problem 3}

For an electron plasma we have a conductivity,
\[ \sigma = \frac{i n_e e^2}{\omega m_e} \]
from a current $J = - n_e e v$ produced by moving electrons. For an electron-positron plasma, we also have a component from positrons with $J_{+} = n_e e v_{+}$. Furthermore, since the electric force on the positrons is opposite to the force on the electrons, we have, $v_{+} = - v_{-}$ and thus $J_{+} = J_{-}$ so we have $J = - 2 n_e e v$. This implies that for an electron-positron plasma the conductivity is,
\[ \sigma = \frac{2 i n_e e^2}{\omega m_e} \]
This corresponds to modifying the plasma frequency by sending $n_e \mapsto 2 n_e$ i.e.
\[ \omega_p^2 = \frac{8 \pi n_e e^2}{m_e} \]
Then from the group velocity formula,
\[ v_{\mathrm{gr}} = c \sqrt{1 - \left( \frac{\omega_p}{\omega} \right)^2} \approx c \left( 1 - \frac{1}{2} \left( \frac{\omega_p}{\omega} \right)^2 \right) \]
we can compute the time delay,
\[ t_r = \int_0^d \frac{\d{s}}{v_{\text{gr}}} =  \frac{1}{2c} \int_0^d\left( \frac{\omega_p}{\omega} \right) \d{s} = \frac{4 \pi e^2}{m c \omega^2} \int_0^d n_e \d{s} = \frac{2 \pi e^2}{m c \omega^2} \DM \]
for,
\[ \DM = 2 \int_0^d n_e \d{s} \]
so to compute the dispersion measure for an electron-positron plasma we replace $n_e \mapsto 2 n_e$ which makes sense because we now have two light charged particles for each electron.
\bigskip\\
However, since the Lorentz force law,    
\[ \deriv{\vec{v}}{t} = q \vec{E} + q \, \frac{\vec{v}}{c} \times \vec{B} \]
for the magnetic term is even in $q \mapsto - q$ and $v \mapsto - v$ while the first term is odd, the effects of positrons is more complex than simply negating the velocities for the electrons. However, the effect of sending $q \mapsto -q$ and $\vec{B} \mapsto - \vec{B}$ and $\vec{v} \mapsto - \vec{v}$ in combination preserves the Lorentz force law and therefore positrons follow the negated trajectories of the electrons would in the magnetic field $- \vec{B}$. In the approximation of constant $\vec{B}$ field and oscillating radiative $\vec{E}$ field we have computed for the electrons,
\[ \vec{v}(t) = - \frac{i e \vec{E}(t)}{m (\omega \pm \omega_B)} \]
and using,
\[ \vec{J} = - e n_e \vec{v} \]
we find,
\[ \sigma_{e} = \frac{i e^2 n_e}{m (\omega \pm \omega_B)} \]
Then for the positrons we have shown that,
\[ \vec{v}(t) =  \frac{i e \vec{E}(t)}{m (\omega \mp \omega_{B})} \]
and using,
\[ \vec{J} = e n_p \vec{v} \]
(where $n_p = n_e$)
we find,
\[ \sigma_{p} = \frac{i e^2 n_e}{m(\omega \mp \omega_B)} \]
Therefore,
\begin{align*}
\sigma & = \sigma_{p} + \sigma_{e} = \frac{2 i e^2 n_e}{m \omega} \left[ \frac{1}{1 \pm \frac{\omega_B}{\omega}} + \frac{1}{1 \mp \frac{\omega}{\omega_B}} \right]
\\
& = \frac{2 i n_e e^2}{m \omega} \left[1 + 2 \left( \frac{\omega_B}{\omega} \right)^2 + O(B^4) \right]
\end{align*}
Firstly, this recovers the above result that for an electron-positron plasma we simply double $\DM$ for a fixed electron density $n_e$. However, the the first-order term in $B$ cancels when equal quantities of positrons are added to the plasma the magnetic effects appear at quadratic order in the conductivity and thus plasma frequency. Furthermore, the conductivity is now independent of polarization so there is no Faraday rotation. 

\end{document}