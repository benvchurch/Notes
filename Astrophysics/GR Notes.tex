\documentclass[11pt, a4paper]{article}
\usepackage[utf8]{inputenc}
\usepackage[english]{babel}
\usepackage[a4paper, total={7.25in, 9.5in}]{geometry}
\usepackage{tikz-feynman}
\tikzfeynmanset{compat=1.0.0} 
\usepackage{subcaption}
\usepackage{float}
\floatplacement{figure}{H}
\usepackage{simpler-wick}
\usepackage{mathrsfs}  
\usepackage{dsfont}
\usepackage{relsize}
\DeclareMathAlphabet{\mathdutchcal}{U}{dutchcal}{m}{n}


\newcommand{\field}{\hat{\Phi}}
\newcommand{\dfield}{\hat{\Phi}^\dagger}
 
\usepackage{amsthm, amssymb, amsmath, centernot}
\usepackage{slashed}
\newcommand{\notimplies}{%
  \mathrel{{\ooalign{\hidewidth$\not\phantom{=}$\hidewidth\cr$\implies$}}}}
 
\renewcommand\qedsymbol{$\square$}
\newcommand{\cont}{$\boxtimes$}
\newcommand{\divides}{\mid}
\newcommand{\ndivides}{\centernot \mid}

\newcommand{\Integers}{\mathbb{Z}}
\newcommand{\Natural}{\mathbb{N}}
\newcommand{\Complex}{\mathbb{C}}
\newcommand{\Zplus}{\mathbb{Z}^{+}}
\newcommand{\Primes}{\mathbb{P}}
\newcommand{\Q}{\mathbb{Q}}
\newcommand{\R}{\mathbb{R}}
\newcommand{\ball}[2]{B_{#1} \! \left(#2 \right)}
\newcommand{\Rplus}{\mathbb{R}^+}
\renewcommand{\Re}[1]{\mathrm{Re}\left[ #1 \right]}
\renewcommand{\Im}[1]{\mathrm{Im}\left[ #1 \right]}
\newcommand{\Op}{\mathcal{O}}

\newcommand{\invI}[2]{#1^{-1} \left( #2 \right)}
\newcommand{\End}[1]{\text{End}\left( A \right)}
\newcommand{\legsym}[2]{\left(\frac{#1}{#2} \right)}
\renewcommand{\mod}[3]{\: #1 \equiv #2 \: \mathrm{mod} \: #3 \:}
\newcommand{\nmod}[3]{\: #1 \centernot \equiv #2 \: mod \: #3 \:}
\newcommand{\ndiv}{\hspace{-4pt}\not \divides \hspace{2pt}}
\newcommand{\finfield}[1]{\mathbb{F}_{#1}}
\newcommand{\finunits}[1]{\mathbb{F}_{#1}^{\times}}
\newcommand{\ord}[1]{\mathrm{ord}\! \left(#1 \right)}
\newcommand{\quadfield}[1]{\Q \small(\sqrt{#1} \small)}
\newcommand{\vspan}[1]{\mathrm{span}\! \left\{#1 \right\}}
\newcommand{\galgroup}[1]{Gal \small(#1 \small)}
\newcommand{\bra}[1]{\left| #1 \right>}
\newcommand{\Oa}{O_\alpha}
\newcommand{\Od}{O_\alpha^{\dagger}}
\newcommand{\Oap}{O_{\alpha '}}
\newcommand{\Odp}{O_{\alpha '}^{\dagger}}
\newcommand{\im}[1]{\mathrm{im} \: #1}
\renewcommand{\ker}[1]{\mathrm{ker} \: #1}
\newcommand{\ket}[1]{\left| #1 \right>}
\renewcommand{\bra}[1]{\left< #1 \right|}
\newcommand{\inner}[2]{\left< #1 | #2 \right>}
\newcommand{\expect}[2]{\left< #1 \right| #2 \left| #1 \right>}
\renewcommand{\d}[1]{ \mathrm{d}#1 \:}
\newcommand{\dn}[2]{ \mathrm{d}^{#1} #2 \:}
\newcommand{\deriv}[2]{\frac{\d{#1}}{\d{#2}}}
\newcommand{\nderiv}[3]{\frac{\dn{#1}{#2}}{\d{#3^{#1}}}}
\newcommand{\pderiv}[2]{\frac{\partial{#1}}{\partial{#2}}}
\newcommand{\fderiv}[2]{\frac{\delta #1}{\delta #2}}
\newcommand{\parsq}[2]{\frac{\partial^2{#1}}{\partial{#2}^2}}
\newcommand{\topo}{\mathcal{T}}
\newcommand{\base}{\mathcal{B}}
\renewcommand{\bf}[1]{\mathbf{#1}}
\renewcommand{\a}{\hat{a}}
\newcommand{\adag}{\hat{a}^\dagger}
\renewcommand{\b}{\hat{b}}
\newcommand{\bdag}{\hat{b}^\dagger}
\renewcommand{\c}{\hat{c}}
\newcommand{\cdag}{\hat{c}^\dagger}
\newcommand{\hamilt}{\hat{H}}
\renewcommand{\L}{\hat{L}}
\newcommand{\Lz}{\hat{L}_z}
\newcommand{\Lsquared}{\hat{L}^2}
\renewcommand{\S}{\hat{S}}
\renewcommand{\empty}{\varnothing}
\newcommand{\J}{\hat{J}}
\newcommand{\lagrange}{\mathcal{L}}
\newcommand{\dfourx}{\mathrm{d}^4x}
\newcommand{\meson}{\phi}
\newcommand{\dpsi}{\psi^\dagger}
\newcommand{\ipic}{\mathrm{int}}
\newcommand{\tr}[1]{\mathrm{tr} \left( #1 \right)}
\newcommand{\C}{\mathbb{C}}
\newcommand{\CP}[1]{\mathbb{CP}^{#1}}
\newcommand{\Vol}[1]{\mathrm{Vol}\left(#1\right)}

\newcommand{\Tr}[1]{\mathrm{Tr}\left( #1 \right)}
\newcommand{\Charge}{\hat{\mathbf{C}}}
\newcommand{\Parity}{\hat{\mathbf{P}}}
\newcommand{\Time}{\hat{\mathbf{T}}}
\newcommand{\Torder}[1]{\mathbf{T}\left[ #1 \right]}
\newcommand{\Norder}[1]{\mathbf{N}\left[ #1 \right]}
\newcommand{\Znorm}{\mathcal{Z}}
\newcommand{\EV}[1]{\left< #1 \right>}
\newcommand{\interact}{\mathrm{int}}
\newcommand{\covD}{\mathcal{D}}
\newcommand{\conj}[1]{\overline{#1}}

\newcommand{\SO}[2]{\mathrm{SO}(#1, #2)}
\newcommand{\SU}[2]{\mathrm{SU}(#1, #2)}

\newcommand{\anticom}[2]{\left\{ #1 , #2 \right\}}


\newcommand{\pathd}[1]{\! \mathdutchcal{D} #1 \:}

\renewcommand{\theenumi}{(\alph{enumi})}


\renewcommand{\theenumi}{(\alph{enumi})}

\newcommand{\atitle}[1]{\title{% 
	\large \textbf{Physics GR8040 General Relativity
	\\ Assignment \# #1} \vspace{-2ex}}
\author{Benjamin Church }
\maketitle}

\theoremstyle{definition}
\newtheorem{theorem}{Theorem}[section]
\newtheorem{definition}{definition}[section]
\newtheorem{lemma}[theorem]{Lemma}
\newtheorem{proposition}[theorem]{Proposition}
\newtheorem{corollary}[theorem]{Corollary}
\newtheorem{example}[theorem]{Example}
\newtheorem{remark}[theorem]{Remark}

\begin{document}
\author{Benjamin Church}
\title{\Huge General Relativity}


\maketitle
\tableofcontents
\newpage

\section{Introduction}

We have Newton's law:
\[ F = - \frac{G M_1 M_2}{R^2} \]
for universal gravitation. However, this cannot be consistent with Lorentz-invariance in special relativity because it does not transform like a force under a Lorentz transformation. Furthermore, the propagation is instantaneous which violates causality. We need a Lorentz-consistent theory. 
\par
The first hints of this theory come from the equivalence principal. All nearby objects have the same gravitational acceleration i.e. it is a property of the spacetime not the objects. We may always choose a frame in which gravity vanishes locally but not globally. Because of this lack of global inertial frames in the presence of gravity, we search for generally covariant theories, i.e. theories which have general diffeomorphism invariance i.e. coordinate independence. 


\section{Special Relativity}

\begin{definition}
The four velocity and four acceleration are defined via,
\[ u^\alpha = \deriv{u^\alpha}{\tau} \quad \quad a^\alpha = \deriv{u^\alpha}{\tau} \]
\end{definition}

\begin{proposition}
$u^\alpha u_\alpha = - c^2$ and $u^\alpha a_\alpha = 0$ and $a^\alpha a_\alpha = a^2$ where $\vec{a}$ is the proper acceleration (acceleration in the instantaneous rest frame). 
\end{proposition}

\begin{theorem}
Straight lines are the paths between events which maximize proper time.
\end{theorem}

\begin{proof}

\end{proof}
 
\begin{remark}
We have found a quantity which is maximized for the paths of free particles. This motivates the following definition seting the action proportional to the negative of the proper time. The constant is choosen to match the standard Newtonian action in the non-relativistic limit.
\end{remark} 

\begin{definition}
The action of a free particle is defined,
\[ S = - mc^2 \int \d{\tau} = - mc^2 \int \sqrt{1 - \beta^2} \: \d{t} \]
Therefore, we find the canonical momenta,
\[ p_\alpha = \deriv{L}{v^\alpha} = \frac{m v_\alpha}{\sqrt{1 - \beta^2}} = \gamma m v_\alpha \]
and the Hamiltonian,
\[ H = p_\alpha v^\alpha - L = \frac{m v^2 + m c^2(1 - \beta^2)}{\sqrt{1 - \beta^2}} = \frac{mc^2}{\sqrt{1 - \beta^2}} = \gamma m c^2 \]
Therefore we can identify,
\[ p^\alpha = m \deriv{u^\alpha}{\tau} \]
to contain the energy and momenta. 
\end{definition}

\section{Maxwell Theory}

We modify the action of a free particle to have the electromagnetic interaction,
\[ S = S_{\text{free}} + S_{\text{EM-int}} = - mc^2 \int \d{\tau} - \int q A_\mu \d{x^\mu} \]
This produces a canonical momentum,
\[ p_\mu = m \deriv{u_\mu}{\tau} + q A_\mu \]
If we vary the path under the action $x^\mu \mapsto x^\mu + \delta x^\mu$ then we find,
\begin{align*}
\delta S & = - m \int \delta \sqrt{- \d{x^\mu} \d{x_\mu}} - \int q \delta(A_\mu \d{x^\mu}) = m \int \frac{\d{\delta x^\mu} \d{x_\mu}}{\sqrt{- \d{x^\mu} \d{x_\mu}}} - \int [q A_\mu \d{\delta x^\mu} + q \partial_\nu A_\mu \d{x^\nu} \delta x^\mu]
\\
& = m \int \d{\delta x^\mu} u_\mu - \int [q \d{(A_\mu \delta x^\mu)} - q \partial_\nu A_\mu \d{x^\nu} \delta x^\mu + q \partial_\nu A_\mu \d{x^\nu} \delta x^\mu]
\\
& = [\delta x^\mu u_\mu - q A_\mu \delta x^\mu]_1^2 - \int \delta x^\mu \left[ m \d{u_\mu} + q (\partial_\nu A_\mu - \partial_\mu A_\nu) \d{x^\nu} \right]
\end{align*}
Therefore, if $\delta S = 0$ for all $\delta x^\mu$ then we must have,
\[ m \deriv{u_\mu}{\tau} = q \left( \partial_\mu A_\nu - \partial_\nu A_\mu \right) \deriv{x^\nu}{\tau} = F_{\mu \nu} u^\nu \]
where,
\[ F_{\mu \nu} = \partial_\mu A_\nu - \partial_\nu A_\mu \]
\begin{definition}
The electromagentic 3-fields are $E_i = c F_{0i}$ and $B_i = \epsilon_{ijk} F^{jk}$. 
\end{definition}
\begin{proposition}
We have the equivalent equations,
\[ \partial_\mu * F^{\mu \nu} = 0 \iff \frac{1}{c} \pderiv{\vec{B}}{t} + \nabla \times \vec{E} = 0 \]
\end{proposition}
\begin{proof}
\begin{align*}
\partial_\mu * F^{\mu \nu} = \partial_\mu \epsilon^{\mu \nu \alpha \beta} (\partial_\alpha A_\beta - \partial_\beta A_\beta) = \epsilon^{\mu \nu \alpha \beta} \left[ \partial_\mu \partial_\alpha A_\beta - \partial_\mu \partial_\beta A_\alpha \right] = 0
\end{align*}
The first term is zero because it is symmetric and antisymmetric under $\mu \iff \alpha$ and the second is zero because it is symmetric and antisymmetric under $\mu \iff \beta$. 
\end{proof}

\begin{remark}
This equation is purely topological in the sense that it does  not depend on the choice of metric. We can identify $F_{\mu \nu}$ with a $2$-form $\bf{F} = F_{\mu \nu} \d{x^\mu} \d{x^\nu}$ and the vector field $A_\mu$ with a 1-form $\bf{A} = A_\mu \d{x^\mu}$. Then we have,
\[ \bf{F} = \d{\bf{A}} \]
which implies that,
\[ \d{\bf{F}} =\d{\d{\bf{A}}} = 0 \implies \partial_\mu * F^{\mu \nu} = 0 \]
To find a geometric equation of motion for the electromagnetic field requires introducing a field action.
\end{remark}

\begin{definition}
The Maxwell action can be written,
\[ S = S_{\text{M}} + S_{\text{EM-int}} + S_{\text{EM}} \]
where the electromagnetic action is,
\[ S_{\text{EM}} = - \frac{1}{16 \pi} \int F_{\mu \nu} F^{\mu \nu} \: \dn{4}{x} \]
\end{definition}

\section{Tensor Analysis}


\subsection{Geodesic Equation}

A curve $\gamma : I \to M$ is a geodesic if it parallel transports its own tangent vector $u$ i.e. if it satisfies the geodesic equation,
\[ \frac{\delta u^\alpha}{\delta \lambda} = u^\mu \nabla_\mu u^\alpha = u^\mu \left( \partial_\mu u^\alpha + \Gamma^\alpha_{\mu \sigma} u^\sigma \right) = 0 \]
Equivalently,
\[ \nderiv{2}{x^\alpha}{\lambda} + \Gamma^\alpha_{\mu \sigma} \deriv{x^\mu}{\lambda} \deriv{x^\sigma}{\lambda} = 0 \]
Furthermore we may define the distance function which locally agrees with the Euclidean (Minkowkian) notion,
\[ S = \int \sqrt{g_{\mu \nu} \d{x^\mu} \d{x^\nu}} = \int \sqrt{g_{\mu \nu} u^\mu u^\nu} \: \d{\lambda} \]
To find the curves which extremalize this distance function, we need to consider perturbations to this curve $x^\mu(\lambda) \mapsto x^\mu(\lambda) + \delta x^\mu(\lambda)$. Then,
\begin{align*}
\delta S &= \int \delta \sqrt{g_{\mu \nu} x^\mu x^\nu} = \int \frac{\delta(g_{\mu \nu} u^\mu u^\nu)}{\sqrt{g_{\mu \nu} u^\mu u^\nu}} \: \d{\lambda} 
\end{align*}
Because $\sqrt{g_{\mu \nu} u^\mu u^\nu}$ is positive

\section{Feb. 28}

Consider the action for a single free particle,
\[ S = - mc^2 \int \d{\tau} = - mc^2 \int \frac{\d{t}}{u^0} \]
The free particle Geodesic motion is,
\[ u^\alpha \nabla_\alpha u_\beta = 0 \implies u^\alpha\partial_\alpha u_\beta = \Gamma^\mu_{\alpha \beta} u^\alpha u_\mu \]
However,
\[ \Gamma^{\mu}_{\alpha \beta} u^\alpha u_\mu = \tfrac{1}{2} u^\alpha u_\mu g^{\mu \nu} \left( \partial_\alpha g_{\beta \nu} + \partial_{\beta} g_{\alpha \nu} - \partial_\nu g_{\alpha \beta} \right) = \tfrac{1}{2} u^\alpha u^\nu \left( \partial_\alpha g_{\beta \nu} + \partial_{\beta} g_{\alpha \nu} - \partial_\nu g_{\alpha \beta} \right) = \tfrac{1}{2} u^\alpha u^\nu \partial_\beta g_{\alpha \nu} \]
Thus,
\[ u^\alpha\partial_\alpha u_\beta = \tfrac{1}{2} u^\alpha u^\nu \partial_\beta g_{\alpha \nu} \]
Therefore, if $\partial_\beta g_{\alpha \nu} = 0$ for some fixed coordinate direction $\beta$ then $u^\alpha \partial_\alpha u_\beta = 0$ so we have a conserved momentum whenever the metric has a cyclic coordinate. This is exactly the conserved charge from Noether's theorem under the translation by $x^\beta$ symmetry. 

\subsection{Killing Fields}

Suppose there exists a chart $x^\alpha_*$ in which $g_{\alpha \beta}^*$ is independent of $x^0_*$. Define $K^\alpha_* = (1, 0, 0, 0)$. Then the fact that $\partial_0 g_{\alpha \beta}^* = 0$ is equivalent to,
\[ \nabla_\alpha K_\alpha + \nabla_\beta K_\alpha = 0 \] 
To see this, consider,
\begin{align*}
\nabla_\alpha K_\beta + \nabla_\beta K_\alpha & = g_{\beta \mu} \nabla_\alpha K^\mu + g_{\alpha \mu} \nabla_\beta K^\mu 
\\
& = g_{\beta \mu} \left( \partial_\alpha K^\mu + \Gamma_{\alpha \lambda}^\mu K^\lambda \right) + g_{\alpha \mu} \left( \partial_\beta K^\mu + \Gamma^\mu_{\beta \lambda} K^\lambda \right)
\\
& = \partial_\alpha K^\mu g_{\beta \mu} + \partial_\beta K^\mu g_{\alpha \mu} + \left( g_{\beta \mu} \Gamma^\mu_{\alpha \lambda} + g_{\alpha \mu} \Gamma^\mu_{\beta \lambda} \right) K^\lambda
\end{align*}
Furthermore,
\[ \nabla_\lambda g_{\alpha \beta} = \partial_\lambda g_{\alpha \beta} = \partial_\lambda g_{\alpha \beta} - \Gamma^\mu_{\alpha \lambda} g_{\mu \beta} - \Gamma^\mu_{\lambda \beta} g_{\alpha \mu} = 0 \]
Therefore,
\[ \nabla_\alpha K_\beta + \nabla_\beta K_\alpha  = \partial_\alpha K^\mu g_{\beta \mu} + \partial_\beta K^\mu g_{\alpha \mu} +  K^\lambda \partial_\lambda g_{\alpha \beta} \]
In particular, in the special chart $x_*^\mu$ such that $K^\mu_* = (1, 0, 0, 0)$ the derivates $\partial_\alpha K^\mu = 0$ all vanish so we see that,
\[ \nabla_\alpha K_\beta + \nabla_{\beta} K_\alpha = 0 \iff \partial^*_0 g_{\alpha \beta}^* = 0 \]

\subsection{Lie Derivatives}

\newcommand{\Lie}[1]{\mathcal{L}_{#1}}

Given a vector field $X$ we define the Lie derivative along $X$ to be,
\[ \Lie{X} a_{\alpha \beta} = X^\mu \partial_\mu a_{\alpha \beta} + \partial_\alpha X^\mu a_{\mu \beta} + \partial_\beta X^\mu a_{\alpha \mu} = X^\mu \nabla_\mu a_{\alpha \beta} + \nabla_\alpha X^\mu a_{\mu \beta} + \nabla_\beta X^\mu a_{\alpha \mu} \]
Therefore, the Lie derivative is a tensorial derivative object with is independent of the metric since we may replace the covariant derivatives with ordinary derivatives without breaking the tensor property. For a vector field we have,
\[ \Lie{X} Y^\alpha = X^\mu \partial_\mu Y^\alpha - \partial_\mu X^\alpha Y^\mu = [X, Y] = - [Y, X] \]
Therefore, 
\[ \Lie{X} Y + \Lie{Y} X = 0 \]
Using the previous discussion,
\[ \Lie{K} g_{\alpha \beta} = \nabla_\alpha K_\beta + \nabla_\beta K_\alpha \]
Therefore $K$ is a Killing vector iff $\Lie{K} g = 0$. In this case there is a conserved quanitiy,
\[ Q = K_\mu u^\mu \]
such that $Q$ is conserved along the geodesics,
\[ \deriv{}{\tau} Q = u^\alpha \nabla_\alpha (K_\mu u^\mu) = 0 \]

\section{Newtonian Limit}

Consider a metric of the form $g_{\alpha \beta} = \eta_{\alpha \beta} + h_{\alpha \beta}$ with small $h_{\alpha \beta}$. To take the Newtonian limit, we drop higher-order terms in $h_{\alpha \beta}$ and consider velocities much less than the speed of light. Consider the Geodesic equation,
\begin{align*}
\nderiv{2}{x^\alpha}{\tau} + \Gamma^\alpha_{\mu \nu} u^\mu u^\nu = 0 
\end{align*}
Since $u^i \ll c$ we can ignore then in comparison with $u^0 = c \gamma$. Furthermore $\gamma \approx 1$ since it is second-order in $u^i/c$. Thus we take $\d{\tau} = \gamma^{-1} \d{t} = \d{t}$. This reduces the geodesic equation to the form,
\[ a^i = \nderiv{2}{x^\alpha}{t} = - \Gamma^i_{\mu \nu} u^\mu u^\nu = - \Gamma^i_{00} c^2 \]
Now we must expand the Christoffel symbols,
\[ \Gamma^i_{00} = \tfrac{1}{2} g^{i \alpha} \left( \partial_0 g_{\alpha 0} + \partial_0 g_{0 \alpha} - \partial_\alpha g_{00} \right) \]
Since $g_{\alpha \beta} = \eta_{\alpha \beta} + h_{\alpha \beta}$ with $\eta_{\alpha \beta}$ constant this becomes,
\[ \Gamma^i_{00} = \tfrac{1}{2} \eta^{i \alpha} \left( \partial_0 h_{\alpha 0} + \partial_0 h_{0 \alpha} - \partial_\alpha h_{00} \right) \]
where I have dropped the higher-order terms in the inverse metric because the given term is already first-order in $h_{\alpha \beta}$. Let us now futher assume that the metric is stationary, i.e. $\partial_0 g_{\alpha \beta} = 0$. Then we find,
\[ \Gamma^i_{00} = - \frac{1}{2} \partial_i h_{00} \]
Plugging in, we find,
\[ a^i = \tfrac{1}{2} \pderiv{}{x^i} g_{00} c^2 \]
We define the Newtonian potential $\Phi$ via,
\[ g_{00} = - 1 - \frac{2 \Phi}{c^2} \]
such that we recover,
\[ \vec{a} = - \nabla \Phi \]

\section{Matter}

Consider dust which has proper mass density $\rho_0$ i.e. density in the local rest frame. Then $\rho_{\text{lab}} = \gamma \rho_0$. Therefore we have the mass current density $j^\mu = \rho_0 u^\mu$ where $u^\mu$ is the local four velocity of the dust. Then lab the energy density can be expressed as,
\[ w_{\text{lab}} = n_{\text{lab}} \gamma m c^2 = n_0 \gamma^2 m c^2 = n_0 m u^0 u^0 = \rho_0 u_0 u _0 \]
Therefore, we should expect this object to be part of the symmetric tensor,
\[ T^{\mu \nu} = \rho_0 u^\mu u^\nu \]
with $w = T^{00}$. This is the stress-energy tensor. The conservation of particle number requires that,
\[ \pderiv{n_{\text{lab}}}{t} + \pderiv{}{x^i} (n_{\text{lab}} v^i) = 0 \implies \frac{1}{c} \pderiv{}{t} (c \gamma n_0) + \pderiv{}{x^i}(n_0 \gamma v^i) = 0 \implies \partial_\mu j^\mu = 0 \]
Requiring energy and momentum conservation requires,
\[ \partial_\mu T^{\mu \nu} = 0 \]

\subsection{Perfect Fluid}

We define a perfect fluid to be a fluid having stress-energy tensor in the rest frame equal to,
\[ T^{\mu \nu} =
\begin{pmatrix}
w & 0 & 0 & 0
\\
0 & P & 0 & 0
\\
0 & 0 & P & 0
\\
0 & 0 & 0 & P
\end{pmatrix} \]
where $w$ is the energy density and $P$ the pressue. Then in any reference frame,
\[ T^{\mu \nu} = \left( \rho + \frac{P}{c^2} \right) u^\mu u^\nu + P \eta^{\mu \nu} \]
which is the unique tensor having the correct components in the rest frame. 

\section{The Einstein Equations}

\begin{theorem}[Bianchi]
\[ \nabla_{[\mu} R_{\alpha \beta ] \gamma \delta} = 0 \]
\end{theorem}


\begin{proof}

\end{proof}

\begin{corollary}
\[ \nabla^\mu R_{\mu \nu} = \tfrac{1}{2} \nabla_\nu R \]
\end{corollary}

\begin{proof}
By the Bianci identity,
\begin{align*}
\nabla_\mu R_{\alpha \beta \gamma \delta} + \nabla_\alpha R_{\beta \mu \gamma \delta} + \nabla_\beta R_{\mu \alpha \gamma \delta} = 0
\end{align*}
Now contract terms,
\begin{align*}
\nabla_\mu g^{\alpha \gamma} g^{\beta \delta} R_{\alpha \beta \gamma \delta} + \nabla_\alpha g^{\alpha \gamma} g^{\beta \delta} R_{\beta \mu \gamma \delta} + \nabla_\beta g^{\alpha \gamma} g^{\beta \delta} R_{\mu \alpha \gamma \delta} = 0
\end{align*}
Identifying terms we find,
\[ \nabla_\mu R - \nabla^\gamma R_{\mu \gamma} - \nabla^\delta R_{\mu \delta} = 0  \]
Relabeling indicies, we find,
\[ \nabla^\mu R_{\mu \nu} = \tfrac{1}{2} \nabla_\nu R \]
\end{proof}

\begin{corollary}
The tensor, $G_{\mu \nu} = R_{\mu \nu} - \tfrac{1}{2} R g_{\mu \nu}$ is divergenceless, $\nabla^\mu G_{\mu \nu} = 0$. 
\end{corollary}

\subsection{The Form of the Equations and the Newtonian Limit}

We postulate a field equation $G_{\mu \nu} = \kappa T_{\mu \nu}$ (DO NEWTONIAN LIMIt)


\subsubsection{The Einstein-Hilbert Action}

Consider the action:
\[ S_G = \alpha \int R \sqrt{-g} \dn{4}{x} \]

\begin{theorem}
The extermal action condition $\delta S_G = 0$ implies the Einstein Field Equation is vacuum.
\end{theorem}

\begin{proof}
Under a metric perturbation $g_{\alpha \beta} \mapsto
 g_{\alpha \beta} + \delta g_{\alpha \beta}$, consider the varition,
\[ \delta S_G = \alpha \int \left( \delta R \sqrt{-g} + R \delta \sqrt{-g} \right) \dn{4}{x} \]
However,
\[ \delta \sqrt{g} = - \frac{\delta g}{2 \sqrt{- g}} = + \tfrac{1}{2} g g^{\mu \nu} \delta g_{\mu \nu} \]
Therefore,
\[ \delta S_G = \alpha \int \left( \delta R \sqrt{-g} + \tfrac{1}{2} R \sqrt{-g} g^{\alpha \beta} \delta g_{\alpha \beta} \right) \dn{4}{x} \]
Furthermore,
\[ \delta R = \delta (g^{\alpha \beta} R_{\alpha \beta} = -  R^{\alpha \beta} \delta g_{\alpha \beta} + g^{\alpha \beta} \delta R_{\alpha \beta} \]
where I have used the identity,
\[ \delta (g^{\alpha \beta} g_{\beta \gamma}) = 0 \implies \delta g^{\alpha \beta} = - g^{\alpha \gamma} g^{\beta \delta} \delta g_{\gamma \delta} \]
Therefore,
\[ \delta S_G = \alpha \int \left( - R^{\alpha \beta} + \tfrac{1}{2} R g^{\alpha \beta} \right) \delta g_{\alpha \beta} \sqrt{-g} \dn{4}{x} + \alpha \int g^{\alpha \beta} \delta R_{\alpha \beta} \sqrt{-g} \dn{4}{x} \]
We need to examine the second term in more detail. Consider,
\[ g^{\alpha \beta} \delta R^\mu_{\alpha \mu \beta} = \delta \left[ \partial_\mu \Gamma^\mu_{\alpha \beta} - \partial_\beta \Gamma^\mu_{\alpha \beta} + \Gamma^\mu_{\alpha \mu} \Gamma^\sigma_{\alpha \beta} - \Gamma^\mu_{\beta \sigma} \Gamma^\sigma_{\alpha \beta} \right] \]
If we choose local ineratial coordinates, we may drop the last two terms since they are proportional to $\Gamma^\mu_{\alpha \beta}$ not just the variation of $\Gamma^\mu_{\alpha \beta}$. Furthermore, using the fact that $\partial_\mu g_{\alpha \beta} = 0$ in local ineratial coorinates,
\begin{align*}
g^{\alpha \beta} \delta R^\mu_{\alpha \mu \beta} & = \partial_\mu \left(g^{\alpha \beta} \delta \Gamma^\mu_{\alpha \beta} \right) - \partial_\beta \left( g^{\alpha \beta} \delta \Gamma^\mu_{\alpha \beta} \right)  
\\
& = \partial_\mu \left[ g^{\alpha \beta} \delta \Gamma^\mu_{\alpha \beta} - g^{\alpha \mu} \delta \Gamma^\sigma_{\alpha \sigma} \right] 
\end{align*}
However, the difference of covariant derivatives with respect to the metrics $g_{\alpha \beta}$ and $g_{\alpha \beta} + \delta g_{\alpha \beta}$ simply give $\delta \Gamma^\mu_{\alpha \beta}$ which is therefore a tensor quantity. Thus, combination,
\[ g^{\alpha \beta} \delta \Gamma^\mu_{\alpha \beta} - g^{\alpha \mu} \delta \Gamma^\sigma_{\alpha \sigma} \]
is a vector field. Thus, in all coordinate systems,
\[ g^{\alpha \beta} \delta R^\mu_{\alpha \mu \beta} = \nabla_\mu \left[ g^{\alpha \beta} \delta \Gamma^\mu_{\alpha \beta} - g^{\alpha \mu} \delta \Gamma^\sigma_{\alpha \sigma} \right] = \nabla_\mu B^\mu \]
Therefore, the term, $g^{\alpha \beta} \delta R^\mu_{\alpha \mu \beta}$ is a total derivative so,
\[ \alpha \int g^{\alpha \beta} \delta R_{\alpha \beta} \sqrt{-g} \dn{4}{x} = \alpha \int \nabla_\mu B^\mu \sqrt{-g} \dn{4}{x} = \alpha \int \partial_\mu \left( \sqrt{-g} B^\mu \right) \dn{4}{x}  \]
is a boundary term which is zero because we require that all variations vanish on the bounary. 
Thus,
\[ \delta S_G = \alpha \int \left( - R^{\alpha \beta} + \tfrac{1}{2} R g^{\alpha \beta} \right) \delta g_{\alpha \beta} \sqrt{-g} \dn{4}{x} \]
The condition $\delta S_G = 0$ for any $\delta g_{\alpha \beta}$ now implies,
\[  R^{\alpha \beta} - \tfrac{1}{2} R g^{\alpha \beta}  = 0 \]
\end{proof}

\begin{remark}
Now we must add matter fields and consider the full action to get the nonvacuum Einstein field equations.
\end{remark}

\begin{definition}
Let $\lagrange_{M}$ be the Lagrangian for all matter fields in the theory. Then the gravitational stress-energy tensor takes the form,
\[ T^{\mu \nu} = \frac{2}{\sqrt{-g}} \frac{\delta \left(\sqrt{-g} \lagrange_M \right)}{\delta g_{\mu \nu}} = \frac{2}{\sqrt{-g}} \pderiv{\sqrt{-g} \lagrange_M}{g_{\mu \nu}} - \frac{2}{\sqrt{-g}} \partial_\alpha \pderiv{\sqrt{-g} \lagrange_M}{ \partial_\alpha g_{\mu \nu}} \] 
where the second equality follows from integration by parts. This definition is manefestly symmetric. 
\end{definition}

\begin{theorem}
The full action $S = S_G + S_M$ gives the nonvaccum Einstein field equations.
\end{theorem}

\begin{proof}
Consider,
\begin{align*}
\delta S & = \delta S_G + \delta S_M = - \alpha \int G^{\alpha \beta} \delta g_{\alpha \beta} \sqrt{-g} \dn{4}{x} + \int \frac{\delta \left(\sqrt{-g} \lagrange_M \right)}{\delta g_{\mu \nu}} \delta g_{\alpha \beta} \dn{4}{x}
\\
& = -\int \left[ \alpha G^{\alpha \beta} - \frac{1}{\sqrt{-g}} \frac{\delta \left(\sqrt{-g} \lagrange_M \right)}{\delta g_{\mu \nu}} \right] \sqrt{-g} g_{\alpha \beta} \dn{4}{x}  
\end{align*}
Therefore, $\delta S = 0$ implies that,
\[ G^{\mu \nu} = \tfrac{1}{2} \alpha^{-1} T^{\mu \nu} \]
which gives the Einstein field equations if we identify,
\[ \alpha = 2 \kappa^{-1} = \frac{c^4}{16 \pi G} \]
\end{proof}

\section{Spherically Symmetric Spaces}

For a space with spherical symmetry, We can force the metric to be of the form,
\[ \d{s^2} = g_{tt}(t, r) \d{t^2} + g_{rr}(t, r)  \d{r^2} + r^2 \d{\Omega^2} \] 
We write this in the canonical form,
\[ \d{s^2} = - e^{2 \alpha(t,r)} \d{t^2} + e^{2 \beta(t, r)} \d{r^2} +  r^3 \d{\Omega^2} \]
Then we may take the nonzero Christoffel symbols,
\begin{align*}
\Gamma^t_{tt} & = \dot{\alpha} \quad \quad \Gamma^t_{tr} = \alpha' \quad \quad \Gamma^t_{rr} = e^{2(\beta - \alpha)} \dot{\beta}
\\
\Gamma^r_{tt} & = \alpha' e^{2(\alpha - \beta)} \quad \quad \Gamma^r_{rr} = \beta' \quad \quad \Gamma^r_{\theta \theta} = - e^{-2 \beta} r \quad \quad \Gamma^r_{rt} = \dot{\beta} \quad \quad \Gamma^r_{\phi \phi} = e^{-2 \beta} r \sin^2{\theta} 
\\
\Gamma^\theta_{r \theta} & = \frac{1}{r} \quad \quad \Gamma^\theta_{\phi \phi} = - \sin{\theta} \cos{\theta} \quad \quad \Gamma^\phi_{r\phi} = \frac{1}{r} \quad \quad \Gamma^\phi_{\theta \phi} = \cot{\theta}
\end{align*}

\begin{theorem}[Birkhoff]
Every spherically symmetric vacuum spacetime has a time-like killing vector i.e. is static and stationary. 
\end{theorem}

\section{Gravitational Waves}

Consider spacetime far from the source of gravitational waves such that the gravitational waves are a small perturbation above Minknowski space. We write,
\[ g_{\mu \nu} = \eta_{\mu \nu} + h_{\mu \nu} \]
where $| h_{\mu \nu} | \le \le 1$. If we compute the Riemann tensor dropping terms above first-order in $h_{\mu \nu}$ we find,
\[ R_{\lambda \beta \mu \nu} =  \tfrac{1}{2} \left\{ \partial_{\beta \mu} h_{\lambda \nu} + \partial_{\lambda \nu} h_{\beta \mu} - \partial_{\lambda \mu} h_{\beta \nu} - \partial_{\beta \nu} h_{\lambda \mu} \right\} \]
Because all such terms are already first-order in $h_{\mu \nu}$, we may raise and lower indices with $\eta_{\mu \nu}$ since the corrections are then higher-order in $h_{\mu \nu}$. In particular, partial derivatives commute with index juggling. 
\bigskip\\
Now consider a small coordinate transformation,
\[ x^\mu \mapsto x'^\mu = x^\mu + \xi^\mu \]
Then,
\[ \pderiv{x'^\mu}{x^\nu} = \delta^\mu_\nu + \partial_\nu \xi^\mu \]
which implies that,
\[ g'^{\mu \nu} =  g^{\mu \nu} +  \partial_\alpha \xi^\mu g^{\alpha \nu} +  \partial_\beta \xi^\nu g^{\mu \beta}  \]
Now as a function of the old coordinates,
\[ g'^{\mu \nu}(x) = g^{\mu \nu}(x') - \xi^\alpha \partial_\alpha g'^{\mu \nu} = g^{\mu \nu}(x) +  \partial_\alpha \xi^\mu g^{\alpha \nu} +  \partial_\beta \xi^\nu g^{\mu \beta} - \xi^\alpha \partial_\alpha g^{\mu \nu} \]
That is,
\[ g' = g - \Lie{\xi} g \]
Now the Ricci tensor becomes,
\[ R_{\mu \nu} = \tfrac{1}{2} \left\{ \partial_{\mu} \partial^\lambda h_{\lambda \nu} + \partial_\nu \partial^\lambda h_{\lambda \mu} - \partial_\lambda \partial^\lambda h_{\mu \nu} - \partial_{\mu} \partial_\nu h^\lambda_\lambda \right\} \]
Define,
\[ \bar{h}_{\mu \nu} = h_{\mu \nu} - \tfrac{1}{2} \eta_{\mu \nu} h^\lambda_\lambda \]
Then,
\[ R_{\mu \nu} = \tfrac{1}{2} \left\{ \partial_{\mu \lambda} \bar{h}^\lambda_\nu + \partial_{\nu \lambda} \bar{h}^\lambda_\mu - \partial_\lambda \partial^\lambda h_{\mu \nu} \right\} \]
Because we have the gauge freedom of $\xi^\mu$ we can choose the gauge condition,
\[ \partial_\alpha \bar{h}^{\alpha \beta} = 0 \]
which is exactly four conditions. Then the Riemann tensor is simply,
\[ R_{\mu \nu} = - \tfrac{1}{2} \partial_\alpha \partial^\alpha h_{\mu \nu} \] 
The Vacuum Einstein equation,
\[ R_{\mu \nu} = 0 \]
then implies,
\[ \partial_\alpha \partial^\alpha h_{\mu \nu} = 0 \]
which is a wave equation. However, there is residual gauge freedom. We require that the gauge condition remains satisfied under the residual gauge transformation. If we take $\xi^\mu$ and $h^{\alpha \beta}$ to be small, then,
\[ \tilde{h}^{\alpha \beta} = h^{\alpha \beta} + \partial^\beta \xi^\alpha + \partial^\alpha \xi^\beta \]
Now we require,
\[ \partial_\alpha \bar{\tilde{h}}^{\alpha \beta} = \partial_\alpha \tilde{h}^{\alpha \beta} - \tfrac{1}{2} \partial_\alpha \tilde{h} \eta^{\mu \nu} = 0 \]
However,
\[ \partial_\alpha \bar{\tilde{h}}^{\alpha \beta} = \partial_\alpha \bar{h}^{\alpha \beta} + \partial_\alpha \partial^\alpha \xi^\beta + \partial_\alpha \partial^\beta \xi^\alpha -  \partial^\beta \partial_\alpha \xi^\alpha \]
And since we require,
\[ \partial_\alpha \bar{h}^{\alpha \beta} = 0 \]
this requires that,
\[ \partial_\alpha \partial^\alpha \xi^\beta + \partial_\alpha \partial^\beta \xi^\alpha -  \partial^\beta \partial_\alpha \xi^\alpha = 0 \]
Since partial derivatives commute, this reduces to,
\[ \partial_\alpha \partial^\alpha \xi^\beta = 0 \]

\section{Production of Gravitational Waves}

Recall the form of the Ricci tensor,
\[ R_{\mu \nu} = \tfrac{1}{2} \left\{ \partial_{\mu} \partial^\lambda h_{\lambda \nu} + \partial_\nu \partial^\lambda h_{\lambda \mu} - \partial_\lambda \partial^\lambda h_{\mu \nu} - \partial_{\mu} \partial_\nu h^\lambda_\lambda \right\} \]
Now the Ricci scalar becomes,
\[ R = \left\{ \partial_{\mu} \partial_\lambda h^{\lambda \mu} - \partial_\lambda \partial^\lambda h \right\} \]
Now recall,
\[ \bar{h}_{\mu \nu} = h_{\mu \nu} - \tfrac{1}{2} \eta_{\mu \nu} h \]
Then the Ricci tensor becomes,
\[ R_{\mu \nu} = \tfrac{1}{2} \left\{ \partial_{\mu} \partial^\lambda \bar{h}_{\lambda \nu} + \partial_\nu \partial^\lambda \bar{h}_{\lambda \mu} - \partial_\lambda \partial^\lambda h_{\mu \nu} \right\} \]
Now the Einstein tensor is,
\[ G_{\mu \nu} = R_{\mu \nu} - \tfrac{1}{2} R g_{\mu \nu} = - \tfrac{1}{2} \left\{ \partial^2 \bar{h}_{\mu \nu} + \eta_{\mu \nu} \partial_\alpha \partial_\beta \bar{h}^{\alpha \beta} - \partial_\mu \partial^\alpha \bar{h}_{\alpha \nu} - \partial_\nu \partial^\alpha \bar{h}_{\alpha \mu} \right\} \]
We nay always impose the Lorentz gauge condition,
\[ \partial_\alpha \bar{h}^\alpha_{\nu} = 0 \]
In which the Einstein tensor becomes,
\[ G_{\mu \nu} = -\tfrac{1}{2} \partial^2 \bar{h}_{\mu \nu} \]
Now the Einstein field equation with source gives,
\[ G_{\mu \nu} = \frac{8 \pi G}{c^4} T_{\mu \nu} \]
Therefore, in the weak field limit,
\[ \partial^2 \bar{h}_{\mu \nu} = - \frac{16 \pi G}{c^4} T_{\mu \nu} \]
This is a wave equation with source $T_{\mu \nu}$. This wave equation has a Green's function,
\[ G(x^\mu, x'^\mu) = - \frac{1}{4 \pi c} \frac{1}{|\vec{r} - \vec{r}'|} \delta(|\vec{r} - \vec{r}'| - c(t - t')) \theta(t - t') \]
Therefore, convolving with the source term we find,
\[ \bar{h}_{\alpha \beta}(t, \vec{r}) = \frac{4 G}{c^4} \int \frac{T_{\alpha \beta}(t - c^{-1} |\vec{r} - \vec{r}'|, \vec{r}') \dn{3}{r'}}{|\vec{r} - \vec{r}'|} \]
Now we assume that the stress-energy tensor changes periodically in time,
\[ T_{\alpha \beta} = a_{\alpha \beta}(\vec{r}) e^{- i \omega t} \]
Now, in the far zone, $r \gg r'$ we have,
\begin{align*}
\bar{h}_{\alpha \beta}(t, \vec{r}) & = \frac{4 G}{c^4} \int \frac{a_{\alpha \beta}(\vec{r}') \exp{[- i \omega (t - c^{-1} r)]} \dn{3}{r'}}{r}
\\
& = \frac{4 G}{c^4} \frac{e^{i \omega r / c}}{r} \int a_{\alpha \beta}(\vec{r}') e^{- i \omega t} \dn{3}{r'}
\\
& = \frac{4 G}{c^4} \frac{e^{i \omega r / c}}{r} \int T_{\alpha \beta} \d{V} 
\end{align*}
Now we assume that $T_{\alpha \beta}$ is small of the same order as $h_{\mu \nu}$ and thus we may use Minkowsi derivatives,
\[ \partial_\alpha T^{\alpha \beta} = 0 \]
Because $T^{\alpha \beta}$ changes harmonically in time,
\[ \frac{i \omega}{c} T^{0 \beta} = \partial_i T^{i \beta}  \] 
Then we use the identity,
\[ \int T^{ij} \d{V} = \frac{1}{2} \int x^i x^j \partial_0^2 T^{00} \d{V} \]
Which, in our case, gives,
\[ \int T^{ij} \d{V} = - \frac{\omega^2}{2c^2} \int x^i x^j T^{00} \: \d{V} \]
which we may write as,
\[ \int T^{ij} \d{V} = - \frac{\omega^2}{2} \int \rho x^i x^j  \d{V} \]
Therefore,
\[ \bar{h}^{ij}(t, \vec{r})  = - \frac{2 G \omega^2}{c^4} \frac{e^{i \omega r /c}}{r} \int \rho x^i x^j \d{V} \]
This is the solution for a single fourier component $\tilde{\rho}(-\omega) = a_{00}$ thus to find the general solution we need to take the Fourier transform,
\begin{align*}
\bar{h}^{ij}(t, \vec{r}) & = - \frac{2 G}{c^4} \int \frac{e^{-i \omega r / c}}{r} \omega^2 \left[ \int \tilde{\rho}(\omega) x^i x^j \d{V} \right] e^{i \omega t} \d{\omega} 
\\
& = \frac{2 G}{c^4} \frac{1}{r} \nderiv{2}{}{t} \int \left[ \int \tilde{\rho}(\omega) x^i x^j \d{V} \right] e^{i \omega (t - r/c)} \d{\omega}
\end{align*}
Now define,
\[ I^{ij}(t) =  \int \rho(t) x^i x^j \d{V} \]
and thus,
\[ \tilde{I}^{ij}(\omega) =  \int \tilde{\rho}(\omega) x^i x^j \d{V} \]
Then,
\[ \bar{h}^{ij}(t, \vec{r}) = \frac{2 G}{c^4} \frac{1}{r} \nderiv{2}{}{t} \int \tilde{I}^{ij}(\omega) e^{i \omega (t - r/c)} \d{\omega}  \]
Which is just the inverse Fourier transform shifted by the light travel time $r/c$ giving,
\[ \bar{h}^{ij}(t, \vec{r}) = \frac{2 G}{c^4} \frac{1}{r} \left( \nderiv{2}{I^{ij}}{t} \right)_{t' = t - r / c} \]
so we see that the gravitational waves are generated by the second derivative of the quadropole moment. 
\end{document}