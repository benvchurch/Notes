\documentclass[12pt]{article}
\usepackage[english]{babel}
\usepackage[utf8]{inputenc}
\usepackage[english]{babel}
\usepackage[a4paper, total={7.25in, 9.5in}]{geometry}
\usepackage{tikz-feynman}
\tikzfeynmanset{compat=1.0.0} 
\usepackage{subcaption}
\usepackage{float}
\floatplacement{figure}{H}
\usepackage{mathrsfs}  
\usepackage{dsfont}
\usepackage{relsize}
\DeclareMathAlphabet{\mathdutchcal}{U}{dutchcal}{m}{n}

\usepackage{revsymb}


\newcommand{\field}{\hat{\Phi}}
\newcommand{\dfield}{\hat{\Phi}^\dagger}
 
\usepackage{amsthm, amssymb, amsmath, centernot}
\usepackage{slashed}
\newcommand{\notimplies}{%
  \mathrel{{\ooalign{\hidewidth$\not\phantom{=}$\hidewidth\cr$\implies$}}}}
 
\renewcommand\qedsymbol{$\square$}
\newcommand{\cont}{$\boxtimes$}
\newcommand{\divides}{\mid}
\newcommand{\ndivides}{\centernot \mid}

\newcommand{\Integers}{\mathbb{Z}}
\newcommand{\Natural}{\mathbb{N}}
\newcommand{\Complex}{\mathbb{C}}
\newcommand{\Zplus}{\mathbb{Z}^{+}}
\newcommand{\Primes}{\mathbb{P}}
\newcommand{\Q}{\mathbb{Q}}
\newcommand{\R}{\mathbb{R}}
\newcommand{\ball}[2]{B_{#1} \! \left(#2 \right)}
\newcommand{\Rplus}{\mathbb{R}^+}
\renewcommand{\Re}[1]{\mathrm{Re}\left[ #1 \right]}
\renewcommand{\Im}[1]{\mathrm{Im}\left[ #1 \right]}
\newcommand{\Op}{\mathcal{O}}

\newcommand{\invI}[2]{#1^{-1} \left( #2 \right)}
\newcommand{\End}[1]{\text{End}\left( A \right)}
\newcommand{\legsym}[2]{\left(\frac{#1}{#2} \right)}
\renewcommand{\mod}[3]{\: #1 \equiv #2 \: \mathrm{mod} \: #3 \:}
\newcommand{\nmod}[3]{\: #1 \centernot \equiv #2 \: mod \: #3 \:}
\newcommand{\ndiv}{\hspace{-4pt}\not \divides \hspace{2pt}}
\newcommand{\finfield}[1]{\mathbb{F}_{#1}}
\newcommand{\finunits}[1]{\mathbb{F}_{#1}^{\times}}
\newcommand{\ord}[1]{\mathrm{ord}\! \left(#1 \right)}
\newcommand{\quadfield}[1]{\Q \small(\sqrt{#1} \small)}
\newcommand{\vspan}[1]{\mathrm{span}\! \left\{#1 \right\}}
\newcommand{\galgroup}[1]{Gal \small(#1 \small)}
\newcommand{\bra}[1]{\left| #1 \right>}
\newcommand{\Oa}{O_\alpha}
\newcommand{\Od}{O_\alpha^{\dagger}}
\newcommand{\Oap}{O_{\alpha '}}
\newcommand{\Odp}{O_{\alpha '}^{\dagger}}
\newcommand{\im}[1]{\mathrm{im} \: #1}
\renewcommand{\ker}[1]{\mathrm{ker} \: #1}
\newcommand{\ket}[1]{\left| #1 \right>}
\renewcommand{\bra}[1]{\left< #1 \right|}
\newcommand{\inner}[2]{\left< #1 | #2 \right>}
\newcommand{\expect}[2]{\left< #1 \right| #2 \left| #1 \right>}
\renewcommand{\d}[1]{\: \mathrm{d}#1 \:}
\newcommand{\dn}[2]{ \mathrm{d}^{#1} #2 \:}
\newcommand{\deriv}[2]{\frac{\d{#1}}{\d{#2}}}
\newcommand{\nderiv}[3]{\frac{\dn{#1}{#2}}{\d{#3^{#1}}}}
\newcommand{\pderiv}[2]{\frac{\partial{#1}}{\partial{#2}}}
\newcommand{\fderiv}[2]{\frac{\delta #1}{\delta #2}}
\newcommand{\parsq}[2]{\frac{\partial^2{#1}}{\partial{#2}^2}}
\newcommand{\topo}{\mathcal{T}}
\newcommand{\base}{\mathcal{B}}
\renewcommand{\bf}[1]{\mathbf{#1}}
\renewcommand{\a}{\hat{a}}
\newcommand{\adag}{\hat{a}^\dagger}
\renewcommand{\b}{\hat{b}}
\newcommand{\bdag}{\hat{b}^\dagger}
\renewcommand{\c}{\hat{c}}
\newcommand{\cdag}{\hat{c}^\dagger}
\newcommand{\hamilt}{\hat{H}}
\renewcommand{\L}{\hat{L}}
\newcommand{\Lz}{\hat{L}_z}
\newcommand{\Lsquared}{\hat{L}^2}
\renewcommand{\S}{\hat{S}}
\renewcommand{\empty}{\varnothing}
\newcommand{\J}{\hat{J}}
\newcommand{\lagrange}{\mathcal{L}}
\newcommand{\dfourx}{\mathrm{d}^4x}
\newcommand{\meson}{\phi}
\newcommand{\dpsi}{\psi^\dagger}
\newcommand{\ipic}{\mathrm{int}}
\newcommand{\tr}[1]{\mathrm{tr} \left( #1 \right)}
\newcommand{\C}{\mathbb{C}}
\newcommand{\CP}[1]{\mathbb{CP}^{#1}}
\newcommand{\Vol}[1]{\mathrm{Vol}\left(#1\right)}

\newcommand{\Tr}[1]{\mathrm{Tr}\left( #1 \right)}
\newcommand{\Charge}{\hat{\mathbf{C}}}
\newcommand{\Parity}{\hat{\mathbf{P}}}
\newcommand{\Time}{\hat{\mathbf{T}}}
\newcommand{\Torder}[1]{\mathbf{T}\left[ #1 \right]}
\newcommand{\Norder}[1]{\mathbf{N}\left[ #1 \right]}
\newcommand{\Znorm}{\mathcal{Z}}
\newcommand{\EV}[1]{\left< #1 \right>}
\newcommand{\interact}{\mathrm{int}}
\newcommand{\covD}{\mathcal{D}}
\newcommand{\conj}[1]{\overline{#1}}

\newcommand{\SO}[2]{\mathrm{SO}(#1, #2)}
\newcommand{\SU}[2]{\mathrm{SU}(#1, #2)}

\newcommand{\anticom}[2]{\left\{ #1 , #2 \right\}}


\newcommand{\pathd}[1]{\! \mathdutchcal{D} #1 \:}

\renewcommand{\theenumi}{(\alph{enumi})}


\renewcommand{\theenumi}{(\alph{enumi})}

\newcommand{\atitle}[1]{\title{% 
	\large \textbf{ASTR GR6001 Radiative Processes
	\\ Assignment \# #1} \vspace{-2ex}}
\author{Benjamin Church }
\maketitle}

\theoremstyle{definition}
\newtheorem{theorem}{Theorem}[section]
\newtheorem{definition}{definition}[section]
\newtheorem{lemma}[theorem]{Lemma}
\newtheorem{proposition}[theorem]{Proposition}
\newtheorem{corollary}[theorem]{Corollary}
\newtheorem{example}[theorem]{Example}
\newtheorem{remark}[theorem]{Remark}

\usepackage{amsmath, amsthm, amssymb}
\usepackage{fullpage,todonotes,enumitem,amsmath}
\usepackage{braket}
\usepackage{tikz-cd}
\usepackage{float}

\renewcommand{\d}[1]{\mathrm{d} #1 \;}

\begin{document}

\atitle{4}

\section{Problem 1}

Consider an atmosphere of uniform temperature and uniform composition which is emitting, absorbig, and scattering. We write the equation of radiative diffusion,
\[ \frac{1}{3} \nderiv{2}{J_\nu}{\tau_\nu} = \epsilon_\nu (J_\nu - B_\nu) \]
where,
\[ \epsilon_\nu = \left( \frac{\alpha_\nu}{\sigma_\nu + \alpha_\nu} \right) \]


\subsection*{(a)}

Consider the difference $D_\nu = B_\nu - J_\nu$. Since we assume that the atmosphere has uniform temperature,
\[ \nderiv{2}{D_\nu}{\tau_\nu} = - \nderiv{2}{J_\nu}{\tau_\nu} = - 3 \epsilon_\nu (J_\nu - B_\nu) = 3 \epsilon_\nu D_\nu \]
This has solutions,
\[ D_\nu = C_1 e^{\sqrt{3 \epsilon_\nu} \tau_\nu} + C_2 e^{-\sqrt{3 \epsilon_\nu} \tau_\nu} \]
Since the atmosphere is semi-infinite we must have $C_1 = 0$ since otherwise in the limit $\tau_\nu \to \infty$ we would have $J_\nu \to \infty$ which is unphysical. Therefore, we simply need to consider the boundary conditions at $\tau_\nu = 0$.
In the two stream approximation, we have an incoming ray $I_0$, reflected ray $I_R$ and a pair of internal rays $I^{\pm}_\nu$ at angles $\mu = \pm \frac{1}{\sqrt{3}}$. 
Then, internally, we have,
\begin{align*}
J_\nu &= \tfrac{1}{2} (I_\nu^+ + I_\nu^-)
\\
F_\nu & = \frac{2 \pi}{\sqrt{3}} (I_\nu^+ - I_\nu^-)
\\
P_\nu & = \frac{2 \pi}{3c} (I_\nu^+ + I_\nu^-) = \frac{4 \pi}{3c} J_\nu
\end{align*} 
We can solve these equations to give,
\[ I^\pm_\nu = J_\nu \pm \frac{\sqrt{3}}{4 \pi} F_\nu \] 
Furthermore, from the first moment equation,
\[ c \deriv{P_\nu}{\tau_\nu} = F_\nu \]
however, in the Eddington approximation,
\[ P_\nu = \frac{4 \pi}{3 c} J_\nu \]
which implies that,
\[ F_\nu = \frac{4 \pi}{3} \deriv{J_\nu}{\tau_\nu} \]
Thus the two-stream rays take the form,
\[ I^\pm_\nu = J_\nu \pm \frac{1}{\sqrt{3}} \deriv{J_\nu}{\tau_\nu} \] 
Now at the surface $\tau_\nu = 0$ we impose continuity of the specific intensity so the outgoing intensities match, $I_R = I^+_\nu(0)$, and the incoming intensities match, $I^-_\nu(0) = 0$, which must be zero since we assume there is no radiation incident on the atmosphere. Thus,
\[ I^{-}_\nu(0) = J_\nu(0) - \frac{1}{\sqrt{3}} \deriv{J_\nu}{\tau_\nu} \bigg|_{\tau_\nu = 0} = 0 \]
which implies that,
\[ J_\nu(0) = \frac{1}{\sqrt{3}} \deriv{J_\nu}{\tau_\nu} \bigg|_{\tau_\nu = 0} \]
Now we use our solution,
\[ J_\nu(\tau_\nu) = C_\nu e^{-\sqrt{3 \epsilon_\nu} \tau_\nu} + B_\nu \]
so we must match,
\[ J_\nu(0) = C_\nu + B_\nu = \frac{1}{\sqrt{3}}  \deriv{J_\nu}{\tau_\nu} \bigg|_{\tau_\nu = 0} = - \sqrt{\epsilon_\nu} C_\nu \]
which implies that,
\[ C_\nu = - \frac{B_\nu}{1 + \sqrt{\epsilon_\nu}} \]
In particular,
\[ D_\nu = B_\nu - J_\nu = - C_\nu e^{-\sqrt{3 \epsilon_\nu} \tau_\nu} = \frac{B_\nu}{1 + \sqrt{\epsilon_\nu}} e^{-\sqrt{3 \epsilon_\nu} \tau_\nu} \]
Since $D_\nu > 0$ for all $\tau_\nu$ we have demonstrated that $J_\nu < B_\nu$ at all optical depths.

\subsection*{(b)}

The outgoing intensity at the surface is given by,
\[ I^{+}_\nu(0) = J_\nu(0) + \frac{1}{\sqrt{3}} \deriv{J_\nu}{\tau_\nu} \bigg|_{\tau_\nu = 0} \]
Our previous solution gives,
\[ J_\nu(\tau_\nu) = C_\nu e^{-\sqrt{3 \epsilon_\nu} \tau_\nu} + B_\nu = B_\nu \left[ 1 - \frac{e^{-\sqrt{3 \epsilon_\nu} \tau_\nu}}{1 + \sqrt{\epsilon_\nu}} \right]  \]
Thus, using the boundary conditions,
\[ J_\nu(0)= \frac{1}{\sqrt{3}} \deriv{J_\nu}{\tau_\nu} \bigg|_{\tau_\nu = 0} = B_\nu \left[ \frac{\sqrt{\epsilon_\nu}}{1 + \sqrt{\epsilon_\nu}} \right] \]
\begin{align*}
I^{+}_\nu(0) & = B_\nu \left[ \frac{2 \sqrt{\epsilon_\nu}}{1 + \sqrt{\epsilon_\nu}} \right]
\end{align*}
Note that in the limit of weak scattering $\sigma_\nu \to 0$ thus $\epsilon_\nu \to 0$ and then ,
\[ I^+_\nu(0) \to B_\nu \]
so the atmosphere becomes a blackbody in the limit of no scattering. Furthermore, for strong scattering, $\sigma_\nu \gg \alpha_\nu$ then $\epsilon_\nu \ll 1$ which implies that,
\[ I^{+}_\nu(0) \approx 2 \sqrt{\epsilon_\nu} B_\nu \]
is highly suppressed from an ideal blackbody. 

\subsection*{(c)}

We have derived a formula for the total flux in the Eddington approximation,
\[ F_\nu = \frac{4 \pi}{3} \deriv{J_\nu}{\tau_\nu} \]
However, at the boundary $\tau_\nu = 0$, in our two-stream approximation, we have shown that the incoming ray vanishes so the entirety of the flux at the boundary is from the outgoing ray and thus is emergent so,
\[ F_\nu^+ = F_\nu(0) = \frac{4 \pi}{3} \deriv{J_\nu}{\tau_\nu} \bigg|_{\tau_\nu = 0} = \frac{4 \pi}{\sqrt{3}} \cdot  \left[ \frac{\sqrt{\epsilon_\nu}}{1 + \sqrt{\epsilon_\nu}} \right] B_\nu \]
In the strong scattering limit we have,
\[ F^+_\nu = \frac{4 \pi}{\sqrt{3}} \sqrt{\epsilon_\nu} B_\nu \]
which is much less than the blackbody flux $F^B_\nu = \pi B_\nu$. However, in the weak scattering limit we have $\epsilon_\nu \to 1$ and so we find,
\[ F^+_\nu \to \frac{2\pi}{\sqrt{3}} B_\nu \]
which slightly exceeds the blackbody flux $F^B_\nu = \pi B_\nu$. This cannot be correct since we know that a blackbody emitts the greatest possible intensity in each part of the spectrum of any body in thermal equilibirum at a given temperature. Since this does not seem right I may have made a mistake or it may be related to the two-stream approximation since the outgoing radiation is fixed with angle $\mu = \frac{1}{\sqrt{3}}$ unlike that of a blackbody with isotropic emergent radiation giving the $F^B_\nu = \pi B_\nu$ relation. In fact, the factor $\frac{2 \pi}{\sqrt{3}}$ is exactly the integration factor computing the flux from a cone of rays at fixed angle $\mu = \frac{1}{\sqrt{3}}$ while $\pi$ is the integration factor for isotropic emerging radiation given that both have the same (constant) specific intensity (over angles where it is nonzero). 

\section{Problem 2}

Consider a plane EM wave propagating in the $z$-direction of the form,
\[ \vec{E}(z, t) = (\varepsilon_1 e^{i \phi_1} \hat{x} + \varepsilon_2 e^{i \phi_2} \hat{y}) e^{i (kz - \omega t)} \]
Now, evaluating near an observer at $z = 0$ the electric field is,
\[ \vec{E}(0, t) = (\varepsilon_1 e^{i \phi_1} \hat{x} + \varepsilon_2 e^{i \phi_2} \hat{y}) e^{-i\omega t} \] 
Now we define the following Stokes parameters,
\begin{align*}
I & = \varepsilon_1^2 + \varepsilon_2^2
\\
Q & = \varepsilon_1^2 - \varepsilon_2^2 
\\
U & = 2 \varepsilon_1 \varepsilon_2 \cos{(\phi_1 - \phi_2)}
\\
V & = 2 \varepsilon_1 \varepsilon_2 \sin{(\phi_1 - \phi_2)}
\end{align*}

We need to relate these quantities to the intensities which pass through various polarizers. First note that we measure the energy flux via the Poynting vector,
\[ S = c ( E \times B) = c \hat{z} E^2 \]
so we simply need to consider the time-averaged square of the field to get,
\[ \left< |S| \right> = c \left< E^2 \right> \] 
However, we need to be somewhat careful to use the \textit{real} fields in this calculation since taking real parts does not commute with complex multiplication. Therefore, consider,
\begin{align*}
\left< E^2 \right> & = \tfrac{1}{4} \left< (E_\C + \bar{E}_\C) \cdot (E_\C + \bar{E}_\C) \right>
\\
& = \tfrac{1}{4}  \left< E_\C^2 \right> + \tfrac{1}{2}  \left< E_\C \cdot \bar{E}_\C \right> + \tfrac{1}{4}  \left< \bar{E}_\C^2 \right>
\end{align*}
where $E_\C$ is the complex field given above. However, the first and last terms will have pure phase time dependence and therefore time-average to zero. Only the middle term remains,
\begin{align*}
\left< E^2 \right>  & = \tfrac{1}{2}  \left< E_\C \cdot \bar{E}_\C \right>
\\
& = \tfrac{1}{2} (\varepsilon_1 e^{i \phi_1} \hat{x} + \varepsilon_2 e^{i \phi_2} \hat{y}) \cdot (\varepsilon_1 e^{-i \phi_1} \hat{x} + \varepsilon_2 e^{-i \phi_2} \hat{y})
\\
& = \tfrac{1}{2} (\varepsilon_1^2 + \varepsilon_2^2)
\end{align*}
and thus,
\[ I = 2 \left< E^2 \right> \]
is proportional to the total intensity. 
\bigskip\\
Now we consider various polarizing filters $L_\theta$ which only permit radiation linearly polarized at $\theta$ pass through and $C_\pm$ which only permit circularly polarized light of a certain handedness to pass. To accomplish such a filtering we simply need to expand the electric field in the corresponding polarization basis. For linear polarization, this corresponds to finding the component of $\vec{E}$ at an angle $\theta$ in the $x$-$y$ plane. Thus,
\[ I_\theta \vec{E} = \vec{E} \cdot \hat{n}_\theta = (\varepsilon_1 e^{i \phi_1} \cos{\theta} + \varepsilon_2 e^{i \phi_2} \sin{\theta}) e^{-i \omega t} \]
Thus, the corresponding intensity is proportional to,
\begin{align*}
P_\theta = \left< (L_\theta \vec{E})^2 \right> & = \tfrac{1}{2} \left< (L_\theta E_\C)(L_\theta \bar{E}_\C) \right> 
\\
& = \tfrac{1}{2} (\varepsilon_1 e^{i \phi_1} \cos{\theta} + \varepsilon_2 e^{i \phi_2} \sin{\theta}) (\varepsilon_1 e^{-i \phi_1} \cos{\theta} + \varepsilon_2 e^{-i \phi_2} \sin{\theta})
\\
& = \tfrac{1}{2} \left[ \varepsilon_1^2 \cos^2{\theta} + \varepsilon_2^2 \sin^2{\theta} + 2 \varepsilon_1 \varepsilon_2 \sin{\theta} \cos{\theta} \cos{(\phi_1 - \phi_2)} \right] 
\end{align*}
Consider now,
\[ 2(I_0 - I_{\frac{\pi}{2}}) = \varepsilon_1^2 - \varepsilon_2^2 = Q \]
so $Q$ is the difference in intensity between vertical and horizontal polarization. Furthermore, consider,
\[ 2 (I_{\frac{\pi}{4}} - I_{\frac{3 \pi}{4}}) = 2 \varepsilon_1 \varepsilon_2 \cos{(\phi_1 - \phi_2)} = U \]
so $U$ is the difference in intensity between $45^\circ$ and $135^\circ$ polarization.
Finally, we need to consider circular polarizers. We expand in the basis,
\[ \hat{e}_\pm = \tfrac{1}{\sqrt{2}} (\hat{x} \pm i \hat{y}) \]
which are pure circular polarization states. We can write,
\[ \hat{x} = \tfrac{1}{\sqrt{2}} (\hat{e}_+ + \hat{e}_-) \quad \quad \hat{y} = \tfrac{1}{i \sqrt{2}} (\hat{e}_+ - \hat{e}_-) \]
and therefore,
\begin{align*}
\vec{E} = \left[ \tfrac{1}{\sqrt{2}} (\varepsilon_1 e^{i \phi_1} - i \varepsilon_2 e^{i \phi_2}) \hat{e}_+ + \tfrac{1}{\sqrt{2}} (\varepsilon_1 e^{i \phi_1} + i \varepsilon_2 e^{i \phi_2}) \hat{e}_-  \right] e^{-i \omega t}
\end{align*}
Therefore,
\[ \vec{E}_{\pm} = \tfrac{1}{\sqrt{2}} (\varepsilon_1 e^{i \phi_1} \mp i \epsilon_2 e^{i \phi_2} ) \hat{e}_{\pm} e^{- i \omega t} \]
so the power in each circularly polarized component is,
\begin{align*}
2 I_{\pm} & = 2 \left< E_{\pm}^2 \right> = \left< E_{\pm} \cdot \bar{E}_{\pm} \right>
\\
& = \tfrac{1}{2} (\varepsilon_1 e^{i \phi_1} \mp i \varepsilon_2 e^{i \phi_2}) (\varepsilon_1 e^{-i \phi_1} \pm i \varepsilon_2 e^{-i \phi_2}) \hat{e}_{\pm} \cdot \bar{\hat{e}}_{\pm}
\end{align*}
Now,
\[ \hat{e}_{\pm} \cdot \bar{\hat{e}}_{\pm} = \tfrac{1}{2} (\hat{x} \pm i \hat{y}) \cdot (\hat{x} \mp i \hat{y}) = \tfrac{1}{2} (1 + 1) = 1 \]
Therefore,
\begin{align*}
2 I_{\pm} & = \tfrac{1}{2} \left( \varepsilon_1^2 + \varepsilon_2^2 \mp i \varepsilon_1 \varepsilon_2 e^{i (\phi_2 - \phi_1)} \pm i \varepsilon_1 \varepsilon_2 e^{i (\phi_1 - \phi_2)} \right)
\\
& = \tfrac{1}{2} \left( \varepsilon_1^2 + \varepsilon_2^2  \mp 2 \varepsilon_1 \varepsilon_2 \sin{(\phi_1 - \phi_2)}  \right)
\end{align*}
Therefore,
\[ 2 (I_{-} - I_{+}) = 2 \varepsilon_1 \varepsilon_2 \sin{(\phi_1 - \phi_2)} = V \]
so $V$ measures the difference in intensity between the two circular polarizations. 

\section{Problem 3}

The differential Thomson cross section for unpolarized light is,
\[ \deriv{\sigma}{\Omega} = r_0^2 \cdot \frac{1 + \cos^2{\phi}}{2} \]
where $r_0$ is the effective electron radius,
\[ r_0 = \frac{e^2}{mc^2} \]
We can compute the total cross section integrating in spherical coordinates alinged along the incident directon,
\begin{align*}
\sigma_T & = r_0^2  \int  \frac{1 + \cos^2{\phi}}{2} \d{\Omega}
\\
& = r_0^2 \int_0^{\pi} \int_0^{2 \pi} \frac{1 + \cos^2{\phi}}{2} \sin{\phi} \: \d{\gamma} \d{\phi}
\\
& = \pi r_0^2 \int_0^\pi (1 + \cos^2{\phi}) \sin{\phi} \: \d{\phi}
\\
& = \pi r_0^2 \int_{-1}^1 (1 + \mu^2) \: \d{\mu} = \frac{8 \pi r_0^2}{3}
\end{align*}
For totally linearly polarized light with polarization vector $\hat{E}$, the differential cross section for Thomson scattering is,
\[ \deriv{\sigma}{\Omega} = r_0^2 (1 - (\hat{r} \cdot \hat{E})^2) \]
where $\hat{r}$ is the unit vector in the direction of scattering. We can compute the total scattering cross section by integrating in spherical coordinates aligned along the direction of polarization,
\begin{align*}
\sigma_T & = r_0^2 \int (1 - (\hat{r} \cdot \hat{E})^2) \: \d{\Omega}
\\
& = r_0^2 \int_0^\pi \int_0^{2 \pi} (1 - \cos^2{\theta}) \sin{\theta} \: \d{\phi} \d{\theta}
\\
& = 2 \pi r_0^2 \int_{-1}^1 (1  - \mu^2) \: \d{\mu}
\\
& = \frac{8 \pi r_0^2}{3}
\end{align*}
which agrees with the earlier computation of the total cross section. 

\end{document}
