\documentclass[12pt]{article}
\usepackage[english]{babel}
\usepackage[utf8]{inputenc}
\usepackage[english]{babel}
\usepackage[a4paper, total={7.25in, 9.5in}]{geometry}
\usepackage{tikz-feynman}
\tikzfeynmanset{compat=1.0.0} 
\usepackage{subcaption}
\usepackage{float}
\floatplacement{figure}{H}
\usepackage{mathrsfs}  
\usepackage{dsfont}
\usepackage{relsize}
\DeclareMathAlphabet{\mathdutchcal}{U}{dutchcal}{m}{n}

\usepackage{revsymb}


\newcommand{\field}{\hat{\Phi}}
\newcommand{\dfield}{\hat{\Phi}^\dagger}
 
\usepackage{amsthm, amssymb, amsmath, centernot}
\usepackage{slashed}
\newcommand{\notimplies}{%
  \mathrel{{\ooalign{\hidewidth$\not\phantom{=}$\hidewidth\cr$\implies$}}}}
 
\renewcommand\qedsymbol{$\square$}
\newcommand{\cont}{$\boxtimes$}
\newcommand{\divides}{\mid}
\newcommand{\ndivides}{\centernot \mid}

\newcommand{\Integers}{\mathbb{Z}}
\newcommand{\Natural}{\mathbb{N}}
\newcommand{\Complex}{\mathbb{C}}
\newcommand{\Zplus}{\mathbb{Z}^{+}}
\newcommand{\Primes}{\mathbb{P}}
\newcommand{\Q}{\mathbb{Q}}
\newcommand{\R}{\mathbb{R}}
\newcommand{\ball}[2]{B_{#1} \! \left(#2 \right)}
\newcommand{\Rplus}{\mathbb{R}^+}
\renewcommand{\Re}[1]{\mathrm{Re}\left[ #1 \right]}
\renewcommand{\Im}[1]{\mathrm{Im}\left[ #1 \right]}
\newcommand{\Op}{\mathcal{O}}

\newcommand{\invI}[2]{#1^{-1} \left( #2 \right)}
\newcommand{\End}[1]{\text{End}\left( A \right)}
\newcommand{\legsym}[2]{\left(\frac{#1}{#2} \right)}
\renewcommand{\mod}[3]{\: #1 \equiv #2 \: \mathrm{mod} \: #3 \:}
\newcommand{\nmod}[3]{\: #1 \centernot \equiv #2 \: mod \: #3 \:}
\newcommand{\ndiv}{\hspace{-4pt}\not \divides \hspace{2pt}}
\newcommand{\finfield}[1]{\mathbb{F}_{#1}}
\newcommand{\finunits}[1]{\mathbb{F}_{#1}^{\times}}
\newcommand{\ord}[1]{\mathrm{ord}\! \left(#1 \right)}
\newcommand{\quadfield}[1]{\Q \small(\sqrt{#1} \small)}
\newcommand{\vspan}[1]{\mathrm{span}\! \left\{#1 \right\}}
\newcommand{\galgroup}[1]{Gal \small(#1 \small)}
\newcommand{\bra}[1]{\left| #1 \right>}
\newcommand{\Oa}{O_\alpha}
\newcommand{\Od}{O_\alpha^{\dagger}}
\newcommand{\Oap}{O_{\alpha '}}
\newcommand{\Odp}{O_{\alpha '}^{\dagger}}
\newcommand{\im}[1]{\mathrm{im} \: #1}
\renewcommand{\ker}[1]{\mathrm{ker} \: #1}
\newcommand{\ket}[1]{\left| #1 \right>}
\renewcommand{\bra}[1]{\left< #1 \right|}
\newcommand{\inner}[2]{\left< #1 | #2 \right>}
\newcommand{\expect}[2]{\left< #1 \right| #2 \left| #1 \right>}
\renewcommand{\d}[1]{\: \mathrm{d}#1 \:}
\newcommand{\dn}[2]{ \mathrm{d}^{#1} #2 \:}
\newcommand{\deriv}[2]{\frac{\d{#1}}{\d{#2}}}
\newcommand{\nderiv}[3]{\frac{\dn{#1}{#2}}{\d{#3^{#1}}}}
\newcommand{\pderiv}[2]{\frac{\partial{#1}}{\partial{#2}}}
\newcommand{\fderiv}[2]{\frac{\delta #1}{\delta #2}}
\newcommand{\parsq}[2]{\frac{\partial^2{#1}}{\partial{#2}^2}}
\newcommand{\topo}{\mathcal{T}}
\newcommand{\base}{\mathcal{B}}
\renewcommand{\bf}[1]{\mathbf{#1}}
\renewcommand{\a}{\hat{a}}
\newcommand{\adag}{\hat{a}^\dagger}
\renewcommand{\b}{\hat{b}}
\newcommand{\bdag}{\hat{b}^\dagger}
\renewcommand{\c}{\hat{c}}
\newcommand{\cdag}{\hat{c}^\dagger}
\newcommand{\hamilt}{\hat{H}}
\renewcommand{\L}{\hat{L}}
\newcommand{\Lz}{\hat{L}_z}
\newcommand{\Lsquared}{\hat{L}^2}
\renewcommand{\S}{\hat{S}}
\renewcommand{\empty}{\varnothing}
\newcommand{\J}{\hat{J}}
\newcommand{\lagrange}{\mathcal{L}}
\newcommand{\dfourx}{\mathrm{d}^4x}
\newcommand{\meson}{\phi}
\newcommand{\dpsi}{\psi^\dagger}
\newcommand{\ipic}{\mathrm{int}}
\newcommand{\tr}[1]{\mathrm{tr} \left( #1 \right)}
\newcommand{\C}{\mathbb{C}}
\newcommand{\CP}[1]{\mathbb{CP}^{#1}}
\newcommand{\Vol}[1]{\mathrm{Vol}\left(#1\right)}

\newcommand{\Tr}[1]{\mathrm{Tr}\left( #1 \right)}
\newcommand{\Charge}{\hat{\mathbf{C}}}
\newcommand{\Parity}{\hat{\mathbf{P}}}
\newcommand{\Time}{\hat{\mathbf{T}}}
\newcommand{\Torder}[1]{\mathbf{T}\left[ #1 \right]}
\newcommand{\Norder}[1]{\mathbf{N}\left[ #1 \right]}
\newcommand{\Znorm}{\mathcal{Z}}
\newcommand{\EV}[1]{\left< #1 \right>}
\newcommand{\interact}{\mathrm{int}}
\newcommand{\covD}{\mathcal{D}}
\newcommand{\conj}[1]{\overline{#1}}

\newcommand{\SO}[2]{\mathrm{SO}(#1, #2)}
\newcommand{\SU}[2]{\mathrm{SU}(#1, #2)}

\newcommand{\anticom}[2]{\left\{ #1 , #2 \right\}}


\newcommand{\pathd}[1]{\! \mathdutchcal{D} #1 \:}

\renewcommand{\theenumi}{(\alph{enumi})}


\renewcommand{\theenumi}{(\alph{enumi})}

\newcommand{\atitle}[1]{\title{% 
	\large \textbf{ASTR GR6001 Radiative Processes
	\\ Assignment \# #1} \vspace{-2ex}}
\author{Benjamin Church }
\maketitle}

\theoremstyle{definition}
\newtheorem{theorem}{Theorem}[section]
\newtheorem{definition}{definition}[section]
\newtheorem{lemma}[theorem]{Lemma}
\newtheorem{proposition}[theorem]{Proposition}
\newtheorem{corollary}[theorem]{Corollary}
\newtheorem{example}[theorem]{Example}
\newtheorem{remark}[theorem]{Remark}
\begin{document}

\newcommand{\Kel}{\: ^\circ\mathrm{K}}

\atitle{3}

\section*{1.}

The 21 cm hyperfine transition line is observed from a cloud of uniform spin temperature $T_S$, on and off the position of a background compact continuum radio source of brightness temperature $T_0$. Radiative transfer gives, 
\begin{align*}
T_B^{\text{on}}(\nu) & = T_0 e^{-\tau_\nu} + T_S (1 - e^{-\tau_\nu})
\\
T_B^{\text{off}}(\nu) & = T_S (1 - e^{-\tau_\nu})
\end{align*}
where,
\[ \frac{c^2}{2 k \nu^2} B_\nu(T_S) \approx T_S \]
because the frequency of the 21cm line is well within the Rayleigh-Jeans regime.
which we solve as follows:
\[ T_B^{\text{on}}(\nu) - T_B^{\text{off}}(\nu) = T_0 e^{-\tau_\nu} \]
and thus,
\[ \tau_\nu = \log{\left( \frac{T_0}{T_B^{\text{on}}(\nu) - T_B^{\text{off}}(\nu)} \right)} \]
Furthermore,
\[ T_B^{\text{off}}(\nu) = T_S - \frac{T_S}{T_0} [T_B^{\text{on}}(\nu) - T_B^{\text{off}}(\nu)] \]
and thus,
\[ T_S  = \frac{T_0 T_B^{\text{off}}(\nu)}{T_0 - [T_B^{\text{on}}(\nu) - T_B^{\text{off}}(\nu)]} 
\]
Now, consider each of the following cases where all brightness temperatures will be given at the line center of 21 cm. 

\subsection*{(a)}

Here $T^{\text{on}} = 300 \Kel$ and $T^{\text{off}} = 200 \Kel$ and $T_0 = 200 \Kel$. Therefore,
\[ \tau_0 = 0.69 \]
and 
\[ T_S = 400 \Kel \]

\subsection*{(b)}

Here $T^{\text{on}} = 300 \Kel$ and $T^{\text{off}} = 200 \Kel$ and $T_0 = 400 \Kel$. Therefore,
\[ \tau_0 = 1.386 \]
and 
\[ T_S = 267 \Kel \]

\subsection*{(c)}

Here $T^{\text{on}} = 200 \Kel$ and $T^{\text{off}} = 200 \Kel$ and $T_0 = 200 \Kel$. Therefore,
\[ \tau_0 = \infty \]
and 
\[ T_S = 200 \Kel \]

\section*{2.}

Consider a neutral hydrogen cloud of thin optical depth $\tau_\nu \ll 1$ over the 21 cm line.


\subsection*{(a)}

From radiative transfer, we have,
\[ I_\nu = S_\nu (1 - e^{-\tau_\nu}) \]
Therefore, computing the brightness temperature,
\[ T_B(\nu) = T_S (1 - e^{- \tau_\nu}) \]
where,
\[ \frac{c^2}{2 k \nu^2} B_\nu(T_S) \approx T_S \]
because the frequency of the 21cm line is well within the Rayleigh-Jeans regime. Then, for small optical depth,
\[ e^{-\tau_\nu} \approx 1 - \tau_\nu \]
and thus,
\[ T_B(\nu) = T_S \tau_\nu \]

\subsection*{(b)}

We have a line function $\phi(\nu)$ in terms of frequency. To convert this to a line profile in terms of radial velocity we need to relate the observer and source frequencies which are related via the dopler shift. We assume nonrelativistic veloctities such that,
\[ \frac{\nu}{\nu_0} = 1 + \frac{v_r}{c} \]
Then,
\[ \d{\nu} = \frac{\nu_0}{c} \d{v_r} \]
Now, by definition,
\[ \phi(\nu) \d{\nu} = \phi(v_r) \d{v_r} \]
and therefore,
\[ \phi(v_r) = \frac{\nu_0}{c} \phi(\nu_0(1 + \tfrac{v_r}{c})) \]
We can rewrite this as,
\[ \phi(\nu(v_r)) = \lambda_0 \phi(v_r) \]

\subsection*{(c)}

We know that,
\[ \tau_\nu = \frac{3 N_{HI} A_{10}}{32 \pi} \left( \frac{h c \lambda}{k T_S} \right) \phi(\nu)  \]
Therefore,
\[ T_B(\nu) = T_S \tau_\nu = \frac{3 N_{HI} A_{10}}{32 \pi} \left( \frac{h c \lambda}{k} \right) \phi(\nu) \]
Now, we can integrate $T_B(v_r)$ over radial velocity,
\[ \int T_B(v_r) \d{v_r} = \frac{3 N_{HI} A_{10}}{32 \pi} \left( \frac{h c \lambda_0}{k} \right) \int  \phi(\nu) \d{v_r} = \frac{3 N_{HI} A_{10}}{32 \pi} \left( \frac{h c \lambda_0^2}{k} \right)  \int \phi(v_r) \d{v_r} = \frac{3 N_{HI} A_{10}}{32 \pi} \left( \frac{h c \lambda_0^2}{k} \right) \]
Plugging in constants, we find that,
\[ N_{HI} = (1.82 \cdot 10^{18} \: \text{cm}^{-2} ) \cdot \left( \frac{\int T_B(v_r) \d{v_r}}{\Kel \cdot \text{km} \: \text{s}^{-1}}\right)
\]

\section*{3.}

We will compute the spin temperature of the 21 cm line assuming that the hyperfine levels are populated by atomic collisions as well as by CMB photons. Recall that the spin temperature is defined as the population temperature of hyperfine levels,
\[ \frac{n_1}{n_0} = \frac{g_1}{g_0} e^{-h \nu / k T_S} \]
In equilibrium we have detailed balance,
\[ n_0 (B_{01} B_\nu + n k_{01}) = n_1 (A_{10} + B_{10} B_\nu + n k_{10}) \]
where $n = n_0 + n_1$ and $k_{01}$ and $k_{10}$ are the rate coefficients for kinetic excitation and deexication. These rates are related by detialed ballance for, if the system were in collisional equilibrium alone (i.e. without the influence of the CMB) then we would have,
\[ n_0 n k_{01} = n_1 n k_{10} \]
which implies that,
\[ \frac{k_{01}}{k_{10}} = \frac{n_1}{n_0} = \frac{g_1}{g_0} e^{- h\nu / k T_K} \]
where $T_K$ is the kinetic temperature of the atoms. 

\subsection*{(a)}

Consider the detailed balance equation,
\[ n_0 (B_{01} B_\nu + n k_{01}) = n_1 (A_{10} + B_{10} B_\nu + n k_{10}) \]
Thus we find,
\[ e^{h \nu / k T_S} = \frac{n_0 g_1}{n_1 g_0} = \frac{g_1}{g_0} \cdot \frac{A_{10} + B_{10} B_\nu + n k_{10}}{B_{01} B_\nu + n k_{01}} \]
Then we can write,
\[ \frac{g_1}{g_0} \cdot \frac{A_{10} + B_{10} B_\nu + n k_{10}}{B_{01} B_\nu + n k_{01}} = \frac{1 + B_{10}/A_{10} B_\nu + n k_{10} / A_{10}}{g_0 B_{01} / g_1 A_{10} B_\nu + n g_0 k_{01} / g_1 A_{10}} \]
Applying the Einstein relations,
\[  \frac{B_{10}}{B_{01}} = \frac{g_0}{g_1}  \quad \text{and} \quad \frac{A_{10}}{B_{10}} = \frac{2 h \nu^3}{c^2} \]
we find that,
\[ \frac{B_{10}}{A_{10}} B_\nu = \frac{1}{e^{h\nu / k T_\text{CMB}} - 1} \quad \quad \frac{g_0 B_{01}}{g_1 A_{10}} = \frac{B_{10}}{A_{10}} \]
Defining the critical density,
\[ n_{\text{crit.}} = \frac{A_{10}}{k_{10}} \approx 3 \times 10^{-5} \text{cm}^{-3} \]
and noting that,
\[ \frac{g_0 k_{01}}{g_1 A_{10}} = \frac{k_{10}}{A_{10}} e^{-h \nu / k T_K} \]
we can write the above expression in the form,
\begin{align*}
\frac{g_1}{g_0} \cdot \frac{A_{10} + B_{10} B_\nu + n k_{10}}{B_{01} B_\nu + n k_{01}} & = \frac{1 + B_{10}/A_{10} B_\nu + n k_{10} / A_{10}}{g_0 B_{01} / g_1 A_{10} B_\nu + n g_0 k_{01} / g_1 A_{10}}
\\
& = \frac{1 + (e^{h \nu / k T_{\text{CMB}}} - 1)^{-1} + n/n_{\text{crit.}}}{(e^{h \nu / k T_{\text{CMB}}} - 1)^{-1} + n/n_{\text{crit.}} e^{- h\nu / k T_K}} =  \frac{e^{h \nu / k T_{\text{CMB}}} + n / n_{\text{crit.}} (e^{h \nu / k T_{\text{CMB}}} - 1)}{1 + n / n_{\text{crit.}} (e^{h \nu / k T_{\text{CMB}}} - 1)e^{- h\nu / k T_K} } 
\end{align*}
Now we assume that $h \nu \ll k T_S$ an thus,
\[ e^{h \nu / k T_S} = 1 + \frac{h \nu}{k T_S} \]
Therefore,
\begin{align*}
\frac{h \nu}{k T_S} & = \frac{e^{h \nu / k T_{\text{CMB}}} + n / n_{\text{crit.}} (e^{h \nu / k T_{\text{CMB}}} - 1)}{1 + n / n_{\text{crit.}} (e^{h \nu / k T_{\text{CMB}}} - 1)e^{- h\nu / k T_K} }  - 1 = \frac{(e^{h \nu / k T_{\text{CMB}}} - 1) + n / n_{\text{crit.}} (e^{h \nu / k T_{\text{CMB}}} - 1)(1 - e^{- h\nu / k T_K})}{1 + n / n_{\text{crit.}} (e^{h \nu / k T_{\text{CMB}}} - 1)e^{- h\nu / k T_K} }
\\
& = \frac{1 + n / n_{\text{crit.}} (1 - e^{- h\nu / k T_K})}{(e^{h \nu / k T_{\text{CMB}}} - 1)^{-1} + n / n_{\text{crit.}} e^{- h\nu / k T_K} }
\end{align*}
Now we further assume that $h \nu \ll k T_K, k T_{\text{CMB}}$ (which is true in the current universe) so that,
\[ (e^{h \nu/ k T_{\text{CMB}}} - 1) \approx \frac{k T_{\text{CMB}}}{h \nu} \quad \quad (1 - e^{-h\nu/k T_K}) \approx \frac{h \nu}{k T_K} \]
Therefore, we find,
\[ \frac{h \nu}{k T_S} = \frac{1 + \frac{n}{n_{\text{crit}.}} \frac{h \nu}{k T_K}}{\frac{k T_{\text{CMB}}}{h \nu} + \frac{n}{n_{\text{crit}.}} \left( 1 - \frac{h \nu}{k T_K} \right)} \]
which implies that,
\[ T_S = \frac{T_{\text{CMB}} + \frac{n}{n_{\text{crit}.}} \frac{h \nu}{k T_K} \left(T_K - \frac{h \nu}{k} \right)}{1 + \frac{n}{n_{\text{crit}.}} \frac{h \nu}{k T_K}} \]
Define,
\[ y = \frac{h \nu}{k T_K} \frac{n}{n_{\text{crit.}}} \]
Now, we can drop $h \nu / k$ compared with $T_K$ (assuming that $k T_{\text{CMB}}  \gg h \nu y$) to find,
\[ T_S = \frac{T_{\text{CMB}} + y T_K}{1 + y} \] 

\subsection*{(b)}

Recall that,
\[ T_L = \frac{h \nu}{k} = 0.0682 \Kel \]
and
\[ T_{\text{CMB}} = 2.73 \Kel \]
Then,
\[ y = \frac{h \nu}{k T_K} \frac{n}{n_{\text{crit.}}} = 22.7 \left( \frac{T_K}{100 \: \text{K}} \right)^{-1} \cdot \left( \frac{n}{\text{cm}^{-3}} \right) \]
\bigskip\\
Consider a gas with $n = 20 \: \text{cm}^{-3}$ and $T_K = 100 \Kel$ then $y = 454$ and then,
\[ T_S = 99.8 \Kel \]
\bigskip\\
Consider a gas with $n = 0.25 \: \text{cm}^{-3}$ and $T_K = 8000 \Kel$ then $y = 0.071$ and then,
\[ T_S = 532.5 \Kel \]
\end{document}