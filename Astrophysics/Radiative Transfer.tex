\documentclass[11pt, a4paper]{article}
\usepackage[utf8]{inputenc}
\usepackage[english]{babel}
\usepackage[a4paper, total={7.25in, 9.5in}]{geometry}
\usepackage{tikz-feynman}
\tikzfeynmanset{compat=1.0.0} 
\usepackage{subcaption}
\usepackage{float}
\floatplacement{figure}{H}
\usepackage{mathrsfs}  
\usepackage{dsfont}
\usepackage{relsize}
\DeclareMathAlphabet{\mathdutchcal}{U}{dutchcal}{m}{n}

\usepackage{revsymb}


\newcommand{\field}{\hat{\Phi}}
\newcommand{\dfield}{\hat{\Phi}^\dagger}
 
\usepackage{amsthm, amssymb, amsmath, centernot}
\usepackage{slashed}
\newcommand{\notimplies}{%
  \mathrel{{\ooalign{\hidewidth$\not\phantom{=}$\hidewidth\cr$\implies$}}}}
 
\renewcommand\qedsymbol{$\square$}
\newcommand{\cont}{$\boxtimes$}
\newcommand{\divides}{\mid}
\newcommand{\ndivides}{\centernot \mid}

\newcommand{\Integers}{\mathbb{Z}}
\newcommand{\Natural}{\mathbb{N}}
\newcommand{\Complex}{\mathbb{C}}
\newcommand{\Zplus}{\mathbb{Z}^{+}}
\newcommand{\Primes}{\mathbb{P}}
\newcommand{\Q}{\mathbb{Q}}
\newcommand{\R}{\mathbb{R}}
\newcommand{\ball}[2]{B_{#1} \! \left(#2 \right)}
\newcommand{\Rplus}{\mathbb{R}^+}
\renewcommand{\Re}[1]{\mathrm{Re}\left[ #1 \right]}
\renewcommand{\Im}[1]{\mathrm{Im}\left[ #1 \right]}
\newcommand{\Op}{\mathcal{O}}

\newcommand{\invI}[2]{#1^{-1} \left( #2 \right)}
\newcommand{\End}[1]{\text{End}\left( A \right)}
\newcommand{\legsym}[2]{\left(\frac{#1}{#2} \right)}
\renewcommand{\mod}[3]{\: #1 \equiv #2 \: \mathrm{mod} \: #3 \:}
\newcommand{\nmod}[3]{\: #1 \centernot \equiv #2 \: mod \: #3 \:}
\newcommand{\ndiv}{\hspace{-4pt}\not \divides \hspace{2pt}}
\newcommand{\finfield}[1]{\mathbb{F}_{#1}}
\newcommand{\finunits}[1]{\mathbb{F}_{#1}^{\times}}
\newcommand{\ord}[1]{\mathrm{ord}\! \left(#1 \right)}
\newcommand{\quadfield}[1]{\Q \small(\sqrt{#1} \small)}
\newcommand{\vspan}[1]{\mathrm{span}\! \left\{#1 \right\}}
\newcommand{\galgroup}[1]{Gal \small(#1 \small)}
\newcommand{\bra}[1]{\left| #1 \right>}
\newcommand{\Oa}{O_\alpha}
\newcommand{\Od}{O_\alpha^{\dagger}}
\newcommand{\Oap}{O_{\alpha '}}
\newcommand{\Odp}{O_{\alpha '}^{\dagger}}
\newcommand{\im}[1]{\mathrm{im} \: #1}
\renewcommand{\ker}[1]{\mathrm{ker} \: #1}
\newcommand{\ket}[1]{\left| #1 \right>}
\renewcommand{\bra}[1]{\left< #1 \right|}
\newcommand{\inner}[2]{\left< #1 | #2 \right>}
\newcommand{\expect}[2]{\left< #1 \right| #2 \left| #1 \right>}
\renewcommand{\d}[1]{\: \mathrm{d}#1 \:}
\newcommand{\dn}[2]{ \mathrm{d}^{#1} #2 \:}
\newcommand{\deriv}[2]{\frac{\d{#1}}{\d{#2}}}
\newcommand{\nderiv}[3]{\frac{\dn{#1}{#2}}{\d{#3^{#1}}}}
\newcommand{\pderiv}[2]{\frac{\partial{#1}}{\partial{#2}}}
\newcommand{\fderiv}[2]{\frac{\delta #1}{\delta #2}}
\newcommand{\parsq}[2]{\frac{\partial^2{#1}}{\partial{#2}^2}}
\newcommand{\topo}{\mathcal{T}}
\newcommand{\base}{\mathcal{B}}
\renewcommand{\bf}[1]{\mathbf{#1}}
\renewcommand{\a}{\hat{a}}
\newcommand{\adag}{\hat{a}^\dagger}
\renewcommand{\b}{\hat{b}}
\newcommand{\bdag}{\hat{b}^\dagger}
\renewcommand{\c}{\hat{c}}
\newcommand{\cdag}{\hat{c}^\dagger}
\newcommand{\hamilt}{\hat{H}}
\renewcommand{\L}{\hat{L}}
\newcommand{\Lz}{\hat{L}_z}
\newcommand{\Lsquared}{\hat{L}^2}
\renewcommand{\S}{\hat{S}}
\renewcommand{\empty}{\varnothing}
\newcommand{\J}{\hat{J}}
\newcommand{\lagrange}{\mathcal{L}}
\newcommand{\dfourx}{\mathrm{d}^4x}
\newcommand{\meson}{\phi}
\newcommand{\dpsi}{\psi^\dagger}
\newcommand{\ipic}{\mathrm{int}}
\newcommand{\tr}[1]{\mathrm{tr} \left( #1 \right)}
\newcommand{\C}{\mathbb{C}}
\newcommand{\CP}[1]{\mathbb{CP}^{#1}}
\newcommand{\Vol}[1]{\mathrm{Vol}\left(#1\right)}

\newcommand{\Tr}[1]{\mathrm{Tr}\left( #1 \right)}
\newcommand{\Charge}{\hat{\mathbf{C}}}
\newcommand{\Parity}{\hat{\mathbf{P}}}
\newcommand{\Time}{\hat{\mathbf{T}}}
\newcommand{\Torder}[1]{\mathbf{T}\left[ #1 \right]}
\newcommand{\Norder}[1]{\mathbf{N}\left[ #1 \right]}
\newcommand{\Znorm}{\mathcal{Z}}
\newcommand{\EV}[1]{\left< #1 \right>}
\newcommand{\interact}{\mathrm{int}}
\newcommand{\covD}{\mathcal{D}}
\newcommand{\conj}[1]{\overline{#1}}

\newcommand{\SO}[2]{\mathrm{SO}(#1, #2)}
\newcommand{\SU}[2]{\mathrm{SU}(#1, #2)}

\newcommand{\anticom}[2]{\left\{ #1 , #2 \right\}}


\newcommand{\pathd}[1]{\! \mathdutchcal{D} #1 \:}

\renewcommand{\theenumi}{(\alph{enumi})}


\renewcommand{\theenumi}{(\alph{enumi})}

\newcommand{\atitle}[1]{\title{% 
	\large \textbf{ASTR GR6001 Radiative Processes
	\\ Assignment \# #1} \vspace{-2ex}}
\author{Benjamin Church }
\maketitle}

\theoremstyle{definition}
\newtheorem{theorem}{Theorem}[section]
\newtheorem{definition}{definition}[section]
\newtheorem{lemma}[theorem]{Lemma}
\newtheorem{proposition}[theorem]{Proposition}
\newtheorem{corollary}[theorem]{Corollary}
\newtheorem{example}[theorem]{Example}
\newtheorem{remark}[theorem]{Remark}

\begin{document}
\author{Benjamin Church}
\title{\Huge Radiative Transfer}


\maketitle
\tableofcontents
\newpage

\section{Limb Darkening}

Consier the equation for radiative transfer along a ray,
\[ \deriv{I_\nu}{s} = - \alpha_\nu (I_\nu - S_\nu) \]
where $S_\nu = j_\nu / \alpha_\nu$ is the specific emittance (CHECK).
Now consider a ray at an angle $\theta$ to the surface. We have $\d{s} = \d{z} \cos{\theta} = \mu \d{z}$ where $\mu = \cos{\theta}$. Then we have,
\[ \mu \pderiv{I_\nu}{z} = - \alpha_\nu (I_\nu - S_\nu) \]
Then we define the optical depth via $\d{\tau_\nu} = - \alpha_\nu \d{z}$ 
giving,
\[ \mu \pderiv{I_\nu}{\tau_\nu} = (I_\nu - S_\nu) \]
This differential equation integrates to,
\[ I_\nu(\tau_\nu, \mu) = \int_{\tau_\nu}^\infty S_\nu(\tau_\nu') e^{-(\tau_\nu' - \tau_\nu) / \mu} \frac{\d{\tau_\nu'}}{\mu} \]
However, we don't know the source function everywhere inside the star such knowledge would require detailed modeling. However, we can learn something by taking the moments of the radiative transfer equation. Integrating over solid angles,
\[ \frac{1}{4 \pi} \mu \pderiv{I_\nu}{\tau_\nu} \d{\Omega} = \frac{1}{4 \pi} \int (I_\nu - S_\nu) \d{\Omega} \]
we get, noting the lack of $\phi$ dependence,
\[ \frac{1}{2} \int_{-1}^1 \mu \pderiv{I_\nu}{\tau_\nu} \d{\mu} = \frac{1}{2} \int_{-1}^1 I_\nu \d{\mu}  - \frac{1}{2} \int^1_{-1} S_\nu \d{\mu} = J_\nu - S_\nu \]
assuming that $S_\nu$ is isotropic i.e. the sources are emitting isotropically. Now we can rearrange,
\[ \frac{1}{2} \int_{-1}^1 \mu \pderiv{I_\nu}{\tau_\nu} \d{\mu} = \frac{1}{2} \pderiv{}{\tau_\nu} \mu I_\nu \d{\mu} = \frac{1}{4 \pi} \pderiv{F_\nu}{\tau_\nu} \] 
Therefore, we find,
\[ \deriv{}{\tau_\nu} F_{\nu} = 4 \pi ( J_\nu - S_\nu) \]
Now, if the star is in thermal equilibrium, we expect the total flux to be constant along each surface. If we assume the radiation is isothermal then the specific flux is also constant. Therefore we find, $J_\nu = S_\nu$. 
Now, taking the first moment of the radiative transfer equation, we find,
\[ \int \mu^2 \pderiv{I_\nu}{\tau_\nu} \d{\Omega} = \int \mu (I_\nu - S_\nu) \d{\Omega} \]
We can simplify this to,
\[ c \pderiv{P_\nu}{\tau_\nu} = F_\nu  \]
because $S_\nu$ is assumed to be isotropic. Now we assume that $I_\nu$ only depends on $\mu$ linearly (the Eddington Approximation) then,
\[ P_\nu = \frac{1}{c} \int \mu^2 I_\nu(\mu) \d{\Omega} = \frac{1}{c} \int \mu^2 [I^0_\nu + \mu I^1_\nu ] \d{\Omega} = \frac{4 \pi}{3 c} I^0_\mu = \frac{4 \pi}{3 c} J_\nu \] 
Therefore, in our situation,
\[ P_\nu = \frac{4 \pi}{3c} S_\nu \]
Now solving our differential equation (recall that $F_\nu$ is constant),
\[ S_\nu  = \frac{3}{4 \pi} F_\nu (\tau_\nu + \tau_\nu^0) \]
where $\tau_\nu^0$ is some integration constant. Now we plug this approximation into our general solution to the radiative transfer equation,
\[ I_\nu(\tau_\nu, \mu) = \frac{3}{4 \pi} F_\nu \int^{\infty}_{\tau_\nu} (\tau_\nu' + \tau_\nu^0) e^{-(\tau_\nu' - \tau_\nu) / \mu} \frac{\d{\tau_\nu}}{\mu} = \frac{3}{4 \pi} F_\nu (\mu + \tau_\nu + \tau_\nu^0) \]
Now to determine our integration constant, we fix the outward flux at the surface to equal the constant $F_\nu$,
\[ F_\nu = 2 \pi \int_0^1 \mu I_\nu(0, \mu) \d{\mu} = \frac{3}{2} F_\nu \left( \frac{1}{3} + \frac{1}{2} \tau_\nu^0 \right) = F_\nu \left( \frac{1}{2}  + \frac{3}{4} \tau_\nu \right) \]
which implies that $\tau_\nu^0 = \frac{2}{3}$. Putting everything together, we find that,
\[ I_\nu(\tau_\nu, \mu) = \frac{3}{4 \pi} F_\nu \left( \mu + \tau_\nu \right) + \frac{1}{2 \pi} F_\nu \]
Therefore,
\begin{align*}
I_\nu(0, \mu) & = \frac{3}{4 \pi} F_\nu \mu + \frac{1}{2 \pi} F_\nu
\\
I_\nu(0, 1) & = \frac{5}{4\pi} F_\nu
\\
I_\nu(0, 0) & = \frac{1}{2 \pi} F_\nu
\end{align*}
so we find,
\[ \frac{I_\nu(0, 0)}{I_\nu(1,1)} = \frac{2}{5} \]

\section{Black Bodies}

A perfect blackbody emitts a uniform spectrum,
\[ I_\nu = B_\nu = \frac{2 h \nu^3}{c^2} \frac{1}{e^{\frac{h\nu}{h T}} - 1} \]
A blackbody is a perfect absorber and also consequently a perfect emitter (any other object at the same temperature emitts at most $B_\nu$ at the frequency $\nu$). 



\section{Einstein Coefficients}

There are three processes, absorption, stimulated emission, and spontaneous emission with rates $B_{01}$ $B_{10}$ and $A_{10}$ which are defined via,
\[ \deriv{n_1}{t} = - n_1 A_{10} - n_1 B_{10} J_\nu + n_0 B_{01} J_\nu \]
For detailed ballance we must have,
\[ \deriv{n_1}{t} = 0 \implies J_\nu = \frac{n_1 A_{10}}{n_0 B_{01} - n_1 B_{10}} = \frac{A_{10} / B_{10}}{\frac{n_0}{n_1} \frac{B_{01}}{B_{10}} - 1} \]
Furthermore, in local thermal equilibrium,
\[ \frac{n_1}{n_0} = \frac{g_1}{g_0} e^{-\frac{h\nu}{kT}} \]
therefore,
\[ J_\nu = \frac{n_1 A_{10}}{n_0 B_{01} - n_1 B_{10}} = \frac{A_{10} / B_{10}}{\frac{g_0}{g_1} \frac{B_{01}}{B_{10}} e^{h \nu / kT}  - 1} \]
and furthermore we must have a blackbody spectrum,
\[ J_\nu = B_\nu = \frac{2 h \nu^3}{c^2} \frac{1}{e^{h \nu / k T} - 1} \]
Therefore, the Einstein coefficients must be related by,
\[  \frac{B_{10}}{B_{01}} = \frac{g_0}{g_1}  \quad \text{and} \quad \frac{A_{10}}{B_{10}} = \frac{2 h \nu^3}{c^2} \]
Suppose the transition has some line width profile $\phi(\nu)$ with,
\[ \int \phi(\nu) \d{\nu} = 1 \]
Then the spontaneous emission is,
\[ j_\nu = \frac{h \nu}{4 \pi} n_1 A_{10}  \phi(\nu) \]
For pure absorption given by,
\[ \deriv{I_\nu}{s} = -\alpha_\nu I_\nu \]
an thus,
\[ \alpha_\nu = -\frac{\d{E}}{\d{s} \d{A} \d{\Omega} \d{t} \d{\nu} I_\nu} \] 
However, we know that,
\[ - \frac{\d{E}}{\d{V} \d{t} \d{\nu} } = h \nu \: n_0 B_{01} \frac{1}{4\pi} \left( \int I_\nu \d{\Omega} \right) \phi(\nu) \]
we have,
\[ \alpha_\nu = \frac{h \nu}{4 \pi} n_0 B_{01}  \phi(\nu) \]
However, we can correct this absorption coefficient for stimulated emission via,
\[ \alpha_\nu = \frac{h \nu}{4 \pi} (n_0 B_{01} - n_1 B_{10}) \phi(\nu) \]
Now note that,
\[ \frac{n_1 B_{10}}{n_0 B_{01}} = e^{-h \nu / k T} \]
and thus,
\[ \alpha_\nu = \frac{h \nu}{4 \pi} n_0 B_{01}[1 - e^{- h \nu / k T}] \phi(\nu) \]
In particular, this implies that the source function,
\[ S_\nu = \frac{j_\nu}{\alpha_\nu} = \frac{n_1 A_{10}}{n_0 B_{01}} \frac{1}{1 - e^{-h\nu / kT}} = \frac{2 \nu^3}{c^2} \frac{1}{e^{h \nu / kT} - 1} \]
which is exactly the blackbody spectrum at the population temperature $T$.
Furthermore,
\[ A_{10} = \frac{2 h \nu^3}{c^2} B_{10} = \frac{2 h \nu^3}{c^2} \frac{g_0}{g_1} B_{01} \] 
Therefore,
\[ \alpha_\nu = \frac{h \nu}{4 \pi} \frac{c^2}{2 h \nu^3} \frac{g_1}{g_0} n_0 A_{10} [1 - e^{- h \nu / k T}] \phi(\nu) \]
which, using $\lambda \nu = c$ we can rewrite as,
\[ \alpha_\nu =  n_0 A_{10} \frac{g_1}{g_0} \frac{\lambda^2}{8 \pi} [1 - e^{- h \nu / k T}] \phi(\nu) \]

\section{Hyperfine 21cm Radiation}

For a line with very small $\nu_0$ such as 21 cm radiation with $T_L = h \nu_0 / k = 0.0682 K$ we can approximate,
\[ e^{-k \nu / kT_S} = 1 - \frac{k \nu}{k T_S} \]
and thus,
\[ \alpha_\nu \approx n_0 A_{10} \frac{g_1}{g_0} \frac{\lambda^2}{8 \pi} \left( \frac{h \nu}{k T_S} \right) \phi(\nu) = \frac{n_0 A_{10}}{8 \pi} \frac{g_1}{g_0} \frac{h c \lambda}{k T_S} \phi(\nu) 
\]
Now, let $n = n_0 + n_1$. For $T_S \gg T_L$ we have,
\[ \frac{n_1}{n_0} \approx \frac{g_1}{g_0} \]
and thus,
\[ n = n_0 \left( 1 + \frac{g_1}{g_0} \right) \]
which implies that,
\[ \alpha_\nu = \frac{g_1}{g_0 + g_1} \frac{n A_{10}}{8 \pi} \left( \frac{h c \lambda}{k T_S} \right) \phi(\nu) \]
For the hyperfine transition $g_0 = 1$ and $g_1 = 3$ so,
\[ \alpha_\nu = \frac{3 n A_{10}}{32 \pi} \left( \frac{h c \lambda}{k T_S} \right) \phi(\nu) \]
However, the spin temperature is unknown. First, consider a Gaussian line profile,
\[ \phi(\nu) = \frac{1}{\Delta \nu_0 \sqrt{\pi}} e^{- \left( \frac{ \nu - \nu_0 }{\Delta \nu_0} \right)} \]
We can define,
\[ \frac{b}{c} = \frac{\Delta \nu_0}{\nu_0} \]
for thermal line broadening,
\[ b = \sqrt{ \frac{2 k T_K}{m_H}} \]
Then,
\[ \phi(\nu) = \frac{\lambda_0}{b \sqrt{\pi}} e^{- \left( \frac{c(\nu - \nu_0)}{b \nu_0} \right)^2} \]
Then,
\[ \alpha_{\nu_0} = \frac{3 n A_{10}}{32 \pi} \left( \frac{h c}{k T_S} \right) \frac{\lambda_0^2}{b \sqrt{\pi}} \]  
Now, the column density of hydrogen is,
\[ N = \int n \d{x} \]
and the optical depth is,
\[ \tau_0 = \int \alpha_{\nu_0} \d{x} \]
Therefore,
\[ \tau_0 = \frac{3 N A_{10}}{32 \pi} \left( \frac{h c}{k T_S} \right) \frac{\lambda_0^2}{b \sqrt{\pi}} = 3.1 \left( \frac{N}{10^{21} \text{cm}^{-2}} \right) \left( \frac{b}{\text{km} \: \text{s}^{-1}}\right)^{-1} \left( \frac{T_S}{100 \text{K}} \right)^{-1} \]

\subsection{Measurement of Spin Temperature}

Suppose we observe a compact continuum source with intensity $I_0$ (over the frequency range of the line). We observe the intensity from the cloud on and off the radio source which we denote $I^{\text{on}}_\nu$ and $I^{\text{off}}_\nu$. Suppose that $T_S$ of the cloud is constant and $\tau_\nu$ along the two optical paths is approximatly the same. Then, from the equation for radiative transfer,
\[ I^{\text{on}}_\nu = I_0 e^{-\tau_\nu} + S_\nu (1 - e^{-\tau_\nu}) \]
and 
\[ I^{\text{off}}_\nu = S_\nu (1 - e^{-\tau_\nu}) \]
The source function is a blackbody spectrum at the population temperature, in this case spin temperature, $S_\nu = B_\nu(T_S)$. Now, for any intensity we can define an associated brightness temperature,
\[ T_B(\nu) = \frac{c^2}{2 k \nu^2} I_\nu \]  
In the Rayleigh-Jeans regime $h \nu \ll k T$ we know that,
\[ B_\nu \approx \frac{2 \nu^2}{c^2} kT \]
and thus $T_B(\nu) \approx T$ for low frequency. Rewriting our equations in terms of brightness temperatures,
\begin{align*}
T_B^{\text{on}}(\nu) & = T_0 e^{-\tau_\nu} + T_S (1 - e^{-\tau_\nu})
\\
T_B^{\text{off}}(\nu) & = T_S (1 - e^{-\tau_\nu})
\end{align*}
where,
\[ \frac{c^2}{2 k \nu^2} B_\nu(T_S) \approx T_S \]
because the frequency of the 21cm line is well within the Rayleigh-Jeans regime.
Now,
\[ T_B^{\text{on}}(\nu) - T_B^{\text{off}}(\nu) = T_0 e^{-\tau_\nu} \]
and thus,
\[ \tau_\nu = \log{\left( \frac{T_0}{T_B^{\text{on}}(\nu) - T_B^{\text{off}}(\nu)} \right)} \]
Furthermore,
\[ T_B^{\text{off}}(\nu) = T_S - \frac{T_S}{T_0} [T_B^{\text{on}}(\nu) - T_B^{\text{off}}(\nu)] \]
and thus,
\[ T_S  = \frac{T_0 T_B^{\text{off}}(\nu)}{T_0 - [T_B^{\text{on}}(\nu) - T_B^{\text{off}}(\nu)]} 
\]
Note, in the limit of thin optical depth $\tau_\nu \ll 1$ then,
\[ 1 + \tau_\nu = \frac{T_0}{T_B^{\text{on}}(\nu) - T_B^{\text{off}}(\nu)} \]
If $\tau_\nu \gg 1$ then $T_B^{\text{on}}(\nu) \approx T_B^{\text{off}}(\nu) \approx T_S$.

\section{Scattering}

We can describe an emission coefficient for scattering,
\[ j_\nu = \int \deriv{\sigma_\nu}{\Omega'} I'_\nu \d{\Omega'} \]
where $\sigma_\nu$ is the cross section as as function of the direction of the incoming and outgoing rays. Then the equation of radiative transfer becomes,
\[ \deriv{I}{s} = - \sigma_\nu I_\nu + j_\nu \]
Consider a surface of pure scattering. We will first approach this problem with isotropic scattering. Then,
\[ j_\nu = \int \deriv{\sigma_\nu}{\Omega'} I' \nu \d{\Omega'} = \sigma_\nu J_\nu \] 
In this case we find,
\[ \deriv{I_\nu}{s} = - \sigma_\nu (I_\nu - J_\nu) \]
Introducing the optical depth $\d{\tau} = \sigma_\nu \d{s}$ we find,
\[ \deriv{I_\nu}{\tau} = - (I_\nu - J_\nu) \]

\section{WUDA MISS}

Consider an incident ray $I_0$ and with reflected ray $I_R$ which corresponds inside the medium to two rays $I^\pm$. Then,
\[ J_\nu = \tfrac{1}{2} (I^+ + I^-) \]
and then,
\[ F_\nu = \frac{2 \pi}{\sqrt{3}} (I^+ - I^-) \]
and finally,
\[ P_\nu = \frac{2 \pi}{3c} (I^+ + I^-) = \frac{4 \pi}{3 c} J_\nu = \frac{u}{3} \]
We have the equation for scattering,
\[ \frac{1}{3} \nderiv{2}{J_\nu}{\tau_\nu} = \left( \frac{\alpha_\nu}{\sigma_\nu + \alpha_\nu} \right) \left(J_\nu - B_\nu \right) \]
We define,
\[ \epsilon_\nu =  \frac{\alpha_\nu}{\sigma_\nu + \alpha_\nu} \] 
This equation has a general solution,
\[ J_\nu = C_1 e^{\sqrt{3 \epsilon_\nu} \tau_\nu} + C_2 e^{- \sqrt{3 \epsilon_\nu} \tau_\nu} \]
Now we have boundary conditions,
\begin{align*}
I^+(0) & = I_R
\\
I^{-}(0) & = I_0
\end{align*}

\section{Stokes Parameters}

Consider a plane EM wave propagating in the $z$-direction of the form,
\[ \vec{E}(z, t) = (\varepsilon_1 e^{i \phi_1} \hat{x} + \varepsilon_2 e^{i \phi_2} \hat{y}) e^{i (kz - \omega t)} \]
Now, evaluating near an observer at $z = 0$ the electric field is,
\[ \vec{E}(0, t) = (\varepsilon_1 e^{i \phi_1} \hat{x} + \varepsilon_2 e^{i \phi_2} \hat{y}) e^{-i\omega t} \] 
Now we define the following Stokes parameters,
\begin{align*}
I & = \varepsilon_1^2 + \varepsilon_2^2
\\
Q & = \varepsilon_1^2 - \varepsilon_2^2 
\\
U & = 2 \varepsilon_1 \varepsilon_2 \cos{(\phi_1 - \phi_2)}
\\
V & = 2 \varepsilon_1 \varepsilon_2 \sin{(\phi_1 - \phi_2)}
\end{align*}
These are measured via,
\begin{itemize}
\item $I$ is the total intensity
\item $Q$ is the difference in intensity passing $0^\circ$ and $90^\circ$ polarizing filters
\item $Q$ is the difference in intensity passing $45^\circ$ and $135^\circ$ polarizing filters 
\item $V$ is the difference in intensity passing right and left-handed circular polarizing filters
\end{itemize}
Now we define the fraction of polarization,
\[ \Pi = \frac{I_{\text{pol}}}{I_{\text{tot}}} \]
where,
\[ I_{\text{pol}}^2 = Q^2 + U^2 + V^2 \]
Thus,
\[ \Pi = \frac{\sqrt{Q^2 + U^2 + V^2}}{I} \]
Furthermore, if we measure the intensity passing a polarizing filter as a function of angle, then,
\begin{align*}
I_{\text{max}} & = I_{\text{pol}} + \tfrac{1}{2} I_{\text{unpol}} 
\\
I_{\text{min}} & = \tfrac{1}{2} I_{\text{unpol}}
\end{align*}
Therefore,
\[ \Pi = \frac{I_{\text{max}} - I_{\text{min}}}{I_{\text{max}} + I_{\text{min}}} \]


\section{Cyclotron Radiation}

Consider an electron orbiting in a uniform magnetic field. Decompose the velocity $v$ into $v_{\parallel}$ and $v_\perp$ about the direction of $B$. Then by the Lorentz force law,
\[ F = q v \times B = q \frac{v_\perp}{c} \times B \]
and thus, 
\[ a^2 = \frac{q^2}{m^2 c^2} v_\perp^2 B^2 \]
Furthrermore, the electron moves in a circle of radius $r$ so,
\[ |a| = \frac{v_\perp^2}{r} = \omega^2 r \]
Therefore,
\[ \omega = \frac{q B}{mc} \]
Now, the electron radiates a power,
\[ P = \frac{q^2 a^2}{6 \pi c^3} = \frac{q^4 \omega^2 v_\perp^2}{6 m^2 c^5}  \]
Furthermore, the radiation fields measured by an observer in the direction $\hat{n}$ and distance $R$ are,
\begin{align*}
\vec{E} & = \frac{q}{R c^2} \hat{n} \times (\hat{n} \times \vec{a}) 
\\
\vec{B} &= - \hat{n} \times \vec{E}
\end{align*}
Therefore,
\[ \vec{S} = \frac{c}{4 \pi} \vec{E} \times \vec{B} = \frac{c}{4 \pi} \hat{n} E^2 \]
Then if we write,
\[ \hat{n} = \sin{i} \hat{y} + \cos{i} \hat{z} \]
for an observer with inclination $i$ to the magnetic field in the $\hat{z}$-direction. Then,
\[ \vec{a} = - \omega v_\perp [ \cos{(\omega t)} \hat{x} + \sin{(\omega t)} \hat{y} ]\]
so we may compute,
\begin{align*}
S & = \frac{c}{4 \pi} E^2 = \frac{q^2}{4 \pi R^2 c^3} [\hat{n} \times (\hat{n} \times \vec{a})]^2
\\
& = \frac{(q \omega v_\perp)^2}{4 \pi R^2 c^3} [\vec{a}^2 - (\hat{n} \cdot \vec{a})^2 ]
\\
& = \frac{(q \omega v_\perp)^2}{4 \pi R^2 c^3} [1 - \sin^2{(\omega t)} \sin^2{i} ]
\\
& = \frac{(q \omega v_\perp)^2}{4 \pi R^2 c^3} [ \cos^2{(\omega t)} + \sin^2{(\omega t)} \cos^2{i} ] 
\end{align*}
Therefore, the time averaged angular power gives, 
\[ \left< \pderiv{P}{\Omega} \right> = \frac{(q \omega v_\perp)^2}{4 \pi c^3} \cdot \frac{1 + \cos^2{i}}{2} \]
The observer basis is given by $\hat{x}$, $\hat{y}'$ and $\hat{n}$
so, we may compute the polarization, via computing,
\begin{align*}
\hat{x} \cdot \vec{E} & = \frac{q}{c^2 R} \: \hat{x} \cdot (\hat{n} \times (\hat{n} \times \vec{a})) = \frac{q}{c^2 R} \: (\hat{n} \times \vec{a}) \cdot (\hat{x} \times \hat{n})
\\
& = \frac{q}{c^2 R} \: (\hat{n} \times \vec{a}) \cdot \hat{y}'
\\
& = \frac{q}{c^2 R} \: \vec{a} \cdot (\hat{y}' \times \hat{n})
\\
& = - \frac{q}{c^2 R} \: \vec{a} \cdot \hat{x}
\\
& = \frac{q \omega v_\perp}{c^2 R} \cos{(\omega t)}
\end{align*}
Likewise,
\begin{align*}
\hat{y}' \cdot \vec{E} & = \frac{q}{c^2 R} \: \hat{y}' \cdot (\hat{n} \times (\hat{n} \times \vec{a})) = \frac{q}{c^2 R} \: (\hat{n} \times \vec{a}) \cdot (\hat{y}' \times \hat{n})
\\
& = - \frac{q}{c^2 R} \: (\hat{n} \times \vec{a}) \cdot \hat{x}
\\
& = - \frac{q}{c^2 R} \: \vec{a} \cdot (\hat{x} \times \hat{n})
\\
& = - \frac{q}{c^2 R} \: \vec{a} \cdot \hat{y}'
\\
& = \frac{q \omega v_\perp}{c^2 R} \sin{(\omega t)} \cos{i}
\end{align*}
Thus, we can write the complexified electric field as,
\[ \vec{E} = \frac{q \omega v_\perp}{c^2 R} [ \hat{x}  - i \hat{y} \cos{i} ] e^{i \omega t} \]
Thus, the stokes parameters are,
\begin{align*}
I & \propto 1 + \cos^2{i}
\\
\frac{Q}{I} & = \frac{1 - \cos^2{i}}{1 + \cos^2{i}} = \frac{\sin^2{i}}{1 + \cos^2{i}}
\\
\frac{U}{I} & = 0 
\\
\frac{V}{I} & = \frac{2 \cos{i}}{1 + \cos^2{i}}
\end{align*}
Furthermore $Q^2 + U^2 + V^2 = I^2$ so the light is completely elliptically polarized. 

\section{Rayleigh Scattering}

We have the cross section for a diellectric sphere with $r \ll \lambda$,
\[ \sigma_R = \frac{125 \pi^5 r^6}{3 \lambda^4} \left( \frac{\epsilon - 1}{\epsilon + 2} \right)^2 \]
and $\epsilon$ is the ratio of the diellectric constant inside to outside the sphere. Now the optical depth in the atmosphere is,
\[ \tau = \int_0^\infty n \sigma_R \d{z} = 0.23 \text{ at } 450 \: \text{nm} \]


\section{Effective Line Widths}

\subsection{Linear Part}

If the optical depth is thin along the entire line then we have,
\[ W_\lambda = \int_0^\infty [1 - e^{-\tau_\nu}] \d{\lambda} = \int_0^{\infty} \tau_\nu \d{\lambda} \approx \frac{\lambda_0^2}{c} \int_0^\infty \tau_\nu \d{\nu} \]
Furthermore, we know that,
\[ \tau_\nu = N \sigma(\nu) = \frac{\pi e^2 N}{mc} f_{12} \: \phi(\nu) \]
Then,
\[ \int \tau_\nu \d{\nu} = \frac{\pi e^2 N}{mc} f_{12} \]
since the line profile is normalized. Therefore,
\[ \frac{W_\lambda}{\lambda_0} = \frac{\lambda_0}{c} \frac{\pi e^2 N}{mc} f_{12} = \frac{\pi e^2}{m c^2} (N \lambda_0 f_{12}) = \pi r_0 (N \lambda_0 f_{12})  \]

\subsection{Flat Part}

For dominant doppler broadening,
\[ \phi(\nu) = \frac{\lambda_0}{\sqrt{\pi} b} e^{-\left[ \frac{c(\nu - \nu_0)}{b \nu_0} \right]^2} \]
Now the effecitve line width is,
\[ W_\lambda = \int_{0}^\infty (1 - e^{-\tau_\nu}) \d{\lambda} \]
and the optical depth is,
\[ \tau_\nu = N \sigma(\nu) = \frac{\pi e^2 N}{mc} f_{12} \: \phi(\nu) = \frac{\lambda_0 e^2 N \sqrt{\pi}}{mc b} f_{12} e^{-\left[ \frac{c(\nu - \nu_0)}{b \nu_0} \right]^2} \]
Now define,
\[ u = \frac{c(\nu - \nu_0)}{b \nu_0} = \frac{(\nu - \nu_0)}{b} \lambda_0 \]
and thus, 
\[ \d{u} = \d{\nu} \frac{\lambda_0}{b} \]
Furthermore,
\[ \d{\lambda} = - \frac{c}{\nu^2} \d{\nu} = - \frac{\lambda^2}{c} \d{\nu} = - \frac{\lambda^2 b}{c \lambda_0} \d{u} \]
further define,
\[ \tau_0 = \tau_{\nu_0} = \frac{\pi e^2 N}{mc} f_{12} \phi(\nu_0) = \frac{e^2 \sqrt{\pi}}{m c^2} \cdot \left( \frac{c}{b} \right) \cdot (N \lambda_0 f_{12}) = \sqrt{\pi} \: r_0 \cdot \left( \frac{c}{b} \right) \cdot (N \lambda_0 f_{12})  \]
Thus we have,
\begin{align*}
W_\lambda & = \int_{0}^{\infty} (1 - e^{-\tau_0 e^{-u^2}}) \d{\lambda} 
\\
& = \int_{-\infty}^{\infty} \frac{\lambda^2 b}{c \lambda_0} (1 - e^{-\tau_0 e^{-u^2}}) \d{u}
\\
& \approx \frac{2 \lambda_0 b}{c} \int_{0}^{\infty}  (1 - e^{-\tau_0 e^{-u^2}}) \d{u}
\end{align*}
Therefore,
\[ \frac{W_\lambda}{\lambda_0} = \frac{2 b}{c} \int_{-\infty}^{\infty}  (1 - e^{-\tau_0 e^{-u^2}}) \d{u} \]
then we can approximate this for large $\tau_0$ as a step function with cutoff at,
\[ \tau_0 e^{-u^2} = \log{2} \]
and thus,
\[ u_{\text{crit}} = \sqrt{\log{(\tau_0 / \log{2})}} \]
Therefore,
\[ \frac{W_\lambda}{\lambda_0} = \frac{2 b}{c} \sqrt{\log{(\tau_0 / \log{2})}} \]

\subsection{Square Root Part}

\section{Synchrotron Radiation}

We have the relativistic power,
\[ P = \frac{2}{3} \frac{q^2}{4 \pi c^3} \gamma^4 ( a_\perp^2 + \gamma^2 a_{\parallel}^2) \]
Then from the Lorentz force law,
\[ F = \deriv{}{t} (\gamma m \vec{v}) = q \left( E + \frac{\vec{v}}{c} \times B \right) \]
Then,
\[ a_\perp = \frac{v_\perp^2}{r_g} = \omega_B v_\perp \]
For the case $E = 0$ then the force and the acceleration are perpendicular so,
\[ a_\perp = \frac{F_\perp}{\gamma m} = \frac{q v_\perp B}{\gamma m c} \]
and $a_\parallel = 0$. Therefore, 
\[ \omega_B = \frac{q B}{\gamma m c} \]
and 
\[ r_g = \frac{\gamma m c v_\perp}{q B} \]
Furthermore,
\[ P = \frac{2}{3} \frac{q^2}{4 \pi c^3} \gamma^4 a_\perp^2 = \frac{2}{3} \frac{q^2}{4 \pi c^3} \gamma^4 \left( \frac{q v_\perp B}{\gamma m c} \right)^2 \] 
For randomly oriented particles,
\[ \EV{v_\perp^2} = \tfrac{2}{3} \EV{v^2} \]
Therefore,
\[ P = \frac{4}{9} \left( \frac{q^4}{4 \pi m^2 c^5} \right) \gamma^2 B^2 = \frac{16 \pi}{9} r_0^2 c^{-1} \gamma^2 B^2 \EV{v^2} = \tfrac{4}{3} \sigma_T \: (\tfrac{1}{2} B^2) \:  \gamma^2 c \EV{\beta^2}  \]
Thus,
\[ P = \tfrac{4}{3} \sigma_T U_{\text{mag}} \: \gamma^2 c \EV{\beta^2}  \]
Then for highly relativistic particles $\EV{\beta^2} \approx 1$ so $P \propto \gamma^2$. Then since,
\[ \deriv{E}{t} = \deriv{\gamma}{t} m c^2 \]
then,
\[ \deriv{\gamma}{t} = - A \gamma^{2} \]
implies that,
\[ \gamma(t) = \frac{\gamma_0}{1 + A \gamma_0 (t - t_0)} \]
where,
\[ A = \frac{4 \sigma_T U_{\text{mag}} }{3 mc} \]

\section{11/11}

We require that,
\[ \frac{B^2}{8 \pi} > \frac{L}{4 \pi R^2 c}  = \frac{4 \pi d^2 \int F_{\nu_{\nu}}}{4 R^2 c} \approx \frac{4 \nu_a F_{\nu_a} d^2}{R^2 c} \]
Therefore,
\[ F_{\nu} = \frac{\pi B_\nu R^2}{d^2} \]
In the R-J regime,
\[ B_\nu = 2 k T_B \frac{\nu_a^2}{c^2} \]
Therefore,
\[ T_B < 3 \cdot 10^{12} K \: \left( \frac{B}{10^{-4} G} \right)^2 \left( \frac{\nu_a}{10^8} \right)^{-3} \]
For inverse compton,
\[ k T_B = \gamma m c^2 \]
and thus, 
\[ k T_B = \left( \frac{4 \pi}{3} \nu \frac{m c}{e B} \right)^{\frac{1}{2}} m c^2 = 2 m c^2 \left( \frac{\nu m c}{e B} \right)^{\frac{1}{2}} \]
Thus we find,
\[ B^2 \propto T_B^{-4} \nu_a^2 \]
and,
\[ T_B < 1.3 \cdot 10^{12} \: \left( \frac{\nu_a}{10^8} \right)^{-\frac{1}{5}} \]

\subsection{Lorentz Invariants}

The specific intensity is not invariant under Lorentz boosts. However,
\[ \frac{I_\nu}{\nu^3} \]
is invariant which gives the formula,
\[ I_{\nu} = I'_{\nu'} \left( \frac{\nu}{\nu'} \right)^3 \]
This comes from the formula,
\[ \deriv{P_r}{\Omega} = \frac{1}{\gamma^4 (1 - \beta \cos{\theta})} \deriv{P}{\Omega} \] 
However,
\[ \frac{\nu}{\nu'} = \frac{1}{\gamma ( 1 - \beta \cos{\theta})} \]
and therefore,
\[ \frac{\d{P}}{\d{\nu} \d{\Omega}} = \frac{1}{\gamma^3  (1 - \beta \cos{\theta})^3 } \frac{\d{P'}}{\d{\nu'} \d{\Omega'}} = \left( \frac{\nu}{\nu'} \right)^3 \frac{\d{P'}}{\d{\nu'} \d{\Omega'}}  \]
Therefore,
\[ I_{\nu} = I'_{\nu'} \left( \frac{\nu}{\nu'} \right)^3 \]
Now suppose that the intensity is a powerlaw,
\[ I'_{\nu'} \propto (\nu')^{\alpha} \]
Then using,
\[ \frac{I_\nu}{\nu^3} = \frac{I'_{\nu'}}{(\nu')^3} = \frac{I'_{\nu}}{(\nu')^3} \left( \frac{\nu'}{\nu} \right)^\alpha \]
This gives,
\[ I_\nu = I'_\nu \left( \frac{\nu}{\nu'} \right)^{3 - \alpha} \]

\subsection{Microwave Background}

Consider an observer in the $S'$ frame and a ray in $S'$ at an angle $\theta'$ to the direction of relative motion of $S$ and $S'$ where $S$ is the rest frame of the CMB. Then $\nu = \nu' \gamma(1 + \beta \cos{\theta'})$. In it's rest frame the CMB has a spectrum,
\[ I_\nu = \frac{2 h \nu^3}{c^2} \cdot \frac{1}{e^{\frac{h \nu}{k T_{\text{CMB}}} - 1}} \]
Then using the above transformation properties,
\[ I'_{\nu'} = I_\nu \left( \frac{\nu'}{\nu} \right)^3 \]
we find,
\[ I'_{\nu'} = \frac{2 h (\nu')^3}{c^2} \left( \frac{1}{e^{h \nu' \gamma (1 + \beta \cos{\theta'}) / k T_{\text{CMB}}} - 1} \right) \]
This is a blackbody but at a new temperatue of,
\[ T_{\text{app}} = \frac{T_{\text{CMB}}}{\gamma (1 + \beta \cos{\theta'})} \]
For small velocities, $\gamma \approx 1$ and thus,
\[ T_{\text{app}} = T_{\text{CMB}} (1 - \beta \cos{\theta'}) \]
which is a dipole anisotropy. 

\section{Nov 17}

The change in brightness temperature due to Soynia-Zelevich effect,
\[ \frac{\Delta T_{\text{SZE}}}{T_{\text{CMB}}} = \left( x \left( \frac{e^x + 1}{e^x - 1} \right) - 4 \right) \int n_e \frac{k T_e}{m c^2} \sigma_\tau \d{\ell} \]
Where,
\[ x = \frac{h \nu}{k T_{\text{CMB}}} = \left( \frac{\nu}{57 \: \mathrm{GHz}} \right) \]
When $x = 3.83$ then $\Delta T = 0$. This occurs for a frequency,
\[ \nu_c = 218 \: \mathrm{GHz} \]
In the limit $x \le 1$ we have,
\[ x \left( \frac{e^x + 1}{e^x - 1} \right) \approx 2 \]
and therefore,
\[  \frac{\Delta T_{\text{SZE}}}{T_{\text{CMB}}} = - 2 y \]
where $y$ is the Compton $y$ parameter. 

\section{Nov 24}

In a material with conductivity $\sigma$ s.t. $J = \sigma E$ tbe dispersion relation becomes,
\[ k^2 = \frac{\omega^2}{c^2} + i \left( \frac{4 \pi \sigma \omega}{c^2} \right) \]
For plasma, the conductivity is purely imaginary,
\[ \sigma = \frac{i n_e e^2}{\omega m_e} \]
and therefore,
\[ k^2 = \frac{\omega^2 - \omega_0^2}{c^2} \]
where,
\[ \omega_0^2 = \frac{4 \pi n_e e^2}{m_e} \]
Then the solution,
\[ \vec{E} = E_0 e^{i (kx - \omega t)} \]
can only propagate when $\omega  > \omega_0$. Then the phase velocity is,
\[ v_{\text{ph}} = \frac{c}{\sqrt{1 - \left( \frac{\omega_0}{\omega} \right)^2}} \]
and similarly, the group velocity is,
\[ v_{\text{gp}} = c \sqrt{1 - \left( \frac{\omega_0}{\omega} \right)^2} \]
Now as the radiation passes through a electron gas this modified group velocity produces a time delay,
\begin{align*}
t_r & = \int_0^d \frac{\d{\ell}}{v_{\text{gp}}} 
\\
& \approx \int_0^d \left[ 1 + \frac{1}{2} \left( \frac{\omega_0}{\omega} \right)^2 \right] \frac{\d{\ell}}{c}
\\
& = \frac{d}{c} + \frac{2 \pi e^2}{m c \omega^2} \int_0^d n \d{\ell} 
\end{align*}
Thefore the delay is,
\[ \Delta t_r =  \frac{2 \pi e^2}{m c \omega^2} \left( \int_0^d n \d{\ell} \right) \]
Then we find,
\[ \deriv{\Delta t_r}{\nu} = 2 \pi \deriv{\Delta t_r}{\omega} = - \frac{8 \pi^2 e^2}{m c \omega^3} \mathrm{DM} = - \frac{e^2 \mathrm{DM}}{\pi m c \nu^3}  \]
where the dispersion measure is,
\[ \mathrm{DM} = \left( \int_0^d n \d{\ell} \right) \]
\end{document}
