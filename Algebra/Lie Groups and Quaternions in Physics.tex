\documentclass[12pt]{extarticle}
\usepackage[utf8]{inputenc}
\usepackage[english]{babel}
\usepackage[a4paper, total={6in, 9in}]{geometry}
\usepackage{tikz-cd}
 
\usepackage{amsthm, amssymb, amsmath, centernot}
\usepackage{mathrsfs} 

\newcommand{\notimplies}{%
  \mathrel{{\ooalign{\hidewidth$\not\phantom{=}$\hidewidth\cr$\implies$}}}}
 
\renewcommand\qedsymbol{$\square$}
\newcommand{\cont}{$\boxtimes$}
\newcommand{\divides}{\mid}
\newcommand{\ndivides}{\centernot \mid}
\newcommand{\Z}{\mathbb{Z}}
\newcommand{\R}{\mathbb{R}}
\newcommand{\N}{\mathbb{N}}
\newcommand{\Zplus}{\mathbb{Z}^{+}}
\newcommand{\Primes}{\mathbb{P}}
\newcommand{\colim}[1]{\mathrm{colim}(#1)}
\newcommand{\Ob}[1]{\mathrm{Ob}(#1)}
\newcommand{\cat}[1]{\mathcal{#1}}
\newcommand{\id}{\mathrm{id}}
\newcommand{\Hom}[2]{\mathrm{Hom}\left( #1, #2 \right)}
\newcommand{\catHom}[3]{\mathrm{Hom}_{#1}\left( #2, #3 \right)}
\newcommand{\End}[1]{\mathrm{End}\left(#1\right)}
\newcommand{\Top}{\mathbf{Top}}
\newcommand{\pTop}{\mathbf{Top}_{\bullet}}
\newcommand{\Set}{\mathbf{Set}}
\newcommand{\pSet}{\mathbf{Set}_\bullet}
\newcommand{\hTop}{\mathbf{hTop}}
\newcommand{\phTop}{\mathbf{hTop}_{\bullet}}
\renewcommand{\Im}[1]{\mathrm{Im}(#1)}
\newcommand{\homspace}[2]{\left< #1, #2 \right>}
\newcommand{\rp}{\mathbb{RP}}
\newcommand{\coker}[1]{\mathrm{coker}\: #1}

\renewcommand{\d}[1]{ \mathrm{d}#1 \:}
\newcommand{\dn}[2]{ \mathrm{d}^{#1} #2 \:}
\newcommand{\deriv}[2]{\frac{\d{#1}}{\d{#2}}}
\newcommand{\nderiv}[3]{\frac{\dn{#1}{#2}}{\d{#3^{#1}}}}
\newcommand{\pderiv}[2]{\frac{\partial{#1}}{\partial{#2}}}
\newcommand{\fderiv}[2]{\frac{\delta #1}{\delta #2}}

\theoremstyle{definition}
\newtheorem{theorem}{Theorem}[section]
\newtheorem{lemma}[theorem]{Lemma}
\newtheorem{proposition}[theorem]{Proposition}
\newtheorem{example}[theorem]{Example}
\newtheorem{corollary}[theorem]{Corollary}
\newtheorem{remark}{Remark}

\newenvironment{definition}[1][Definition:]{\begin{trivlist}
\item[\hskip \labelsep {\bfseries #1}]}{\end{trivlist}}


\newenvironment{lproof}{\begin{proof} \renewcommand{\qedsymbol}{}}{\end{proof}}
\renewcommand{\mod}[3]{\: #1 \equiv #2 \: mod \: #3 \:}
\newcommand{\nmod}[3]{\: #1 \centernot \equiv #2 \: mod \: #3 \:}
\newcommand{\ndiv}{\hspace{-4pt}\not \divides \hspace{2pt}}
\newcommand{\gen}[1]{\langle #1 \rangle}
\newcommand{\hook}{\hookrightarrow}
\newcommand{\Tor}[4]{\mathrm{Tor}^{#1}_{#2} \left( #3, #4 \right)}
\newcommand{\Ext}[4]{\mathrm{Ext}^{#1}_{#2} \left( #3, #4 \right)}

\tikzset{
    labl/.style={anchor=south, rotate=90, inner sep=.5mm}
}

\renewcommand{\bf}[1]{\mathbf{#1}}
\newcommand{\res}{\mathrm{res}}
\newcommand{\F}{\mathcal{F}}
\newcommand{\G}{\mathcal{G}}
\renewcommand{\O}{\mathcal{O}}
\newcommand{\m}{\mathfrak{m}}

\newcommand{\GL}[1]{\mathrm{GL}\left(#1\right)}
\newcommand{\SL}[1]{\mathrm{SL}\left(#1\right)}
\newcommand{\PGL}[1]{\mathrm{PGL}\left(#1\right)}
\newcommand{\PSL}[1]{\mathrm{PSL}\left(#1\right)}

\newcommand{\Orth}[1]{\mathrm{O}\left(#1\right)}
\newcommand{\U}[1]{\mathrm{U}\left(#1\right)}
\newcommand{\SO}[1]{\mathrm{SO}\left(#1\right)}
\newcommand{\SU}[1]{\mathrm{SU}\left(#1\right)}
\newcommand{\g}{\mathfrak{g}}
\newcommand{\h}{\mathfrak{h}}
\newcommand{\gl}[1]{\mathfrak{gl}\left(#1\right)}
\newcommand{\Lie}[1]{\mathrm{Lie}\left(#1 \right)}
\newcommand{\Aut}[1]{\mathrm{Aut}\left(#1 \right)}

\newcommand{\C}{\mathbb{C}}
\renewcommand{\H}{\mathbb{H}}
\newcommand{\Hil}{\mathcal{H}}
\newcommand{\inner}[2]{\left< #1, #2 \right>}
\renewcommand{\P}{\mathbb{P}}
\newcommand{\PAU}[1]{\mathrm{PAU}\left( #1 \right)}
\newcommand{\CP}{\mathbb{CP}}
\newcommand{\Cl}{\mathrm{C} \ell}
\newcommand{\fchar}[1]{\mathrm{char}(#1)}



\begin{document}

\section{Some Algebra}

\begin{definition}
A group is a set $G$ with a binary operation $\circ : G \times G \to G$ such that,
\begin{enumerate}
\item $\forall x,y,z \in G : x \circ (y \circ z) = (x \circ y) \circ z$
\item $\exists e \in G : \forall x \in G : x \circ e = e \circ x = x$
\item $\forall x \in G : \exists x^{-1} \in G : x \circ x^{-1} = x^{-1} \circ x = e$ 
\end{enumerate}
A group is abelian if $\forall x,y \in G: x \circ y = y \circ x$.
\end{definition}

\begin{definition}
A ring is a set $R$ with binary operations $+, \cdot$ such that $(R, +)$ is an abelian group such that,
\begin{enumerate}
\item $\forall x,y,z \in R : x \cdot (y \cdot z) = (x \cdot y) \cdot z$
\item $\forall x,y,z \in R : x \cdot (y + z) = x \cdot y + x \cdot z$  and $(x + y) \cdot z = x \cdot z + y \cdot z$
\item $\exists 1 \in R : 1 \cdot x = x \cdot 1 = x$
\end{enumerate}
\end{definition}

\begin{definition}
A field is a commutative division ring meaning that multiplication is commutative and all nonzero elements have inverses. 
\end{definition}

\begin{definition}
An $R$-module over a ring $R$ is an abelian group $M$ equiped with a bilinear map $R \times M \to M$ written $(r, m) \mapsto r \cdot m$ such that,
\begin{enumerate}
\item $r \cdot (m_1 + m_2) = r \cdot m_1 + r \cdot m_2$
\item $(r_1 + r_2) \cdot m = r_1 \cdot m + r_2 \cdot m$
\item $(r_1 r_2) \cdot m = r_1 \cdot (r_2 \cdot m)$
\item $1_R \cdot m = m$
\end{enumerate}
If $R$ is a field then we call $M$ a $k$-vectorspace. An $R$-linear map of $R$-modules is a group homomorphism $\phi : M_1 \to M_2$ such that $\phi(r \cdot m) = r \cdot \phi(m)$.
\end{definition}

\begin{definition}
An $k$-algebra over a field $k$ is a $k$-vectorspace $V$ with a bilinear operation $B : V \times V \to V$. A morphism of $k$-algebras is an $k$-linear map $\Phi : V_1 \to V_2$ preserving the bilinear operation, i.e. $B_2(\Phi(v_1), \Phi(v_2)) = \Phi(B_1(v_1, v_2))$. 
\end{definition}

\begin{definition}
Given a $k$-algebra $V$ and a bais $\{ \bf{e}_i \}$ of $V$ the structure constants are coefficients $C_{ijk} \in V$ defined by,
\[ B(\bf{e}_i, \bf{e}_j) = \sum_{k = 1}^{\dim{V}} C_{ijk} \: \bf{e}_k \] 
\end{definition}

\begin{example}
The complex numbers $\C$ are a $2$-dimensional (associative unital commutative division) $\R$-algebra. We can write $z = x + i y$ and the product is defined,
\[ z_1 \cdot z_2 = x_1 x_2 - y_1 y_2 + i (x_1 y_1 + x_2 y_2)  \]
Thus the structure constants in the basis $\{ 1, i \}$ are,
\[ C_{\cdot \cdot 1} = \begin{pmatrix}
1 & 0
\\
0 & -1
\end{pmatrix} \quad \quad C_{\cdot \cdot i} = \begin{pmatrix}
0 & 1
\\
1 & 0
\end{pmatrix}  \]
\end{example}

\begin{remark}
Willaim Rowan Hamilton famously searched for a $3$-dimensional version of the complex numbers. However he soon discovered the following.
\end{remark}

\begin{proposition}
There does not exist a $3$-dimensonal associative unital $\R$-algebra with two independent elements which square to $-1$.
\end{proposition}

\begin{proof}
Suppose we have $i^2 = j^2 = -1$. By associativity,
\[ i \cdot (i \cdot j) = i^2 \cdot j = - j \quad \text{and} \quad (i \cdot j) \cdot j = i \cdot j^2 = - i \]
Now expand,
\[ i \cdot j = a + i b + j c \]
However,
\[ i \cdot (i \cdot j) = i a - b + (a + i b + j c)c = ac - b + i(a  + bc) + j c^2 = - j \]
meaning that $c^2 = - 1$ which is impossible.  
\end{proof}

\begin{remark}
However, all was not lost. He realized that if he added a fourth dimension then the construction would work as long as he gave up the requirement of commutativity leading to the Quaternions.
\end{remark}

\begin{example}
The Quaternions $\H$ are a $4$-dimensional (associative unital division) $\R$-algebra,
\[ \H = \{ w + \bf{i} x + \bf{j} y + \bf{k} z \mid w,x,y,z \in \R \} \]
with bilear multiplication defined by,
\[ \]
\end{example}

\begin{proposition}
We can write any quaternion in vector and scalar part $q = s + \vec{v}$ where $\vec{v} = \bf{i} x + \bf{j} y + \bf{k} z$. Furthermore, define the conjugte $\bar{q} = s - \vec{v}$. Now multiplication becomes,
\[ q \cdot q' = ss' - \vec{v} \cdot \vec{v}' + s \vec{v}' + s' \vec{v} + \vec{v} \times \vec{v}' \]
where $\vec{v} \cdot \vec{v}'$ is the dot product of vectors and $\vec{v} \times \vec{v}'$ is the cross product of vectors. Furthermore, we define the norm,
\[ |q|^2 = q \bar{q} = w^2 + x^2 + y^2 + z^2 \]
which is multiplicative,
\[ |q q'|^2 = |q|^2 \cdot |q'|^2 \]
\end{proposition}

\begin{proof}

\end{proof}

\begin{remark}
HISTROY OF PRODUCTS
\end{remark}


\section{Lie Groups and Lie Algebras}

\begin{definition}
A Lie Group is a group object in the category of smooth manifolds. 
\end{definition}

\begin{definition}
A Lie Algebra is a $k$-algebra $\g$ over a field $k$ with its bilinear bracket written as $[ \bullet, \bullet] : \g \times \g \to \g$ and satisfying,
\begin{enumerate}
\item $[x,x] = 0$
\item $[x, [y, z]] + [y, [z, x]] + [z, [x, y]] = 0$
\end{enumerate}
\end{definition}

\begin{example}
The $\R$-vectorspace $V = \R^3$ with the cross product $\times$ is a Lie algebra. In the standard basis it has structure coefficients $\epsilon_{ijk}$. 
\end{example}


\begin{theorem}
Every Lie group $G$ has an associated Lie algebra $\g = \Lie{G} = T_e G$. Furthermore, if $f : G \to H$ is a Lie group homomorphism then $\d{f} : \g \to \h$ is a Lie algebra homomorphism.
\end{theorem}

\begin{proof}
The Lie bracket is the second-order term in the expansion of $g_1 g_2 g_1^{-1} g_2^{-1}$ about the identity.
\end{proof}

\begin{example}
The standard example of a Lie group is $\GL{n}$ the group of invertible real $n \times n$ matrices. In particular, for any real vectorspace $V$ then $\Aut{V} = \GL{\dim{V}}$ is a Lie group. We can write elements of $\GL{n}$ close to the identity as $M = I + \Omega$ for small matrix $\Omega$. So $\gl{n} = \End{\R^n}$ all $n \times n$ matrices. Then, to second-order,
\begin{align*}
M_1 M_2 M_1^{-1} M_2^{-1} & = (I + \Omega_1)(I + \Omega_2)(I - \Omega_1 + \Omega_1^2)(I - \Omega_2 + \Omega_2^2)
\\
& = I + \Omega_1 \Omega_2 - \Omega_2 \Omega_1 + O(\Omega^3) = I + [\Omega_1, \Omega_2] + O(\Omega^3)
\end{align*}
So the Lie bracket on $\gl{n} = \Lie{\GL{n}}$ is the standard commutator. 
\end{example}

\subsection{Main Theorems of Lie}

\begin{definition}
We say a connected topological space is simply connected if every loop may be continuously contracted to a point. 
\end{definition}

\begin{theorem}
Every finite-dimensional real Lie algebra is $\Lie{G}$ for some simply-connected Lie group $G$. Furthermore, if $\h \subset \g$ is a Lie subalgebra then there exists a unique connected Lie subgroup $H \subset G$ with $\Lie{H} = \h$.
\end{theorem}

\begin{theorem}
Let $G$ and $H$ be Lie groups with $G$ simply-connected. Any Lie algebra homomorphism $\phi : \g \to \h$ lifts to a unique Lie group homomorphism $f : G  \to H$ such that $\phi = \d{f}$. 
\end{theorem}


\subsection{Representations of Lie Groups and Algebras}

\begin{definition}
A Lie group representation is a vectorspace $V$ and a Lie group homomorphism $G \to \Aut{V}$. And a representation of Lie Algebras is a Lie algebra homomorphism $\g \to \gl{V}$. That is a linear map $\rho : \g \to \End{V}$ which preserves the bracket i.e.
\[ \rho([X, Y]) = \rho(X) \rho(Y) - \rho(Y) \rho(X) \]
\end{definition}

\begin{proposition}
Every Lie group representation induces a Lie algebra representation. If $G$ is simply connected then every representation of $\g = \Lie{G}$  gives rise to a unique Lie group representation of $G$.
\end{proposition}

\begin{proof}
Immediate from the correspondence between homomorphisms of Lie groups and their Lie algebras for simply connected domains.
\end{proof}

\section{Representation Theory of $\SU{2}$ and $\SO{3}$}

\begin{definition}
The $n$-sphere is the $n$-dimensional manifold,
\[ S^n = \{ \vec{x} \in \R^{n+1} \mid |\vec{x}| = 1 \} \]
\end{definition}

\begin{remark}
One of the most important examples of a Lie group is the group $\SO{n}$ of rotations of $\R^n$ or equivalently orientation preserving isometries of $S^{n-1}$. This group can be constructed as the group of $n \times n$ determinant $1$ (special) orthogonal matricies (hence the name). We want to understand this group in detail. First we will start with the simplier question of which spheres can be Lie groups.
\end{remark}

DEFINE ISOMETRY GROUPS

\begin{theorem}
$S^n$ can be given a Lie group structure only when $n = 1,3$. 
\end{theorem}

\begin{definition}
We give $S^1$ a Lie group structure by identifing $S^1 = \{ z \in \C \mid |z| = 1 \}$. Since $|z \cdot z'| = |z| \cdot |z'| = 1$ and $z \bar{z} = |z|^2 = 1$ this is indeed a group. By analogy, we define a Lie group structure on $S^3$ by identifing,
\[ S^3 = \{ q \in \H \mid |q| = 1 \} \]
which is a Lie group because $|q \cdot q'| = |q| \cdot |q'| = 1$ and $q \bar{q} = |q|^2 = 1$. From now on, let $S^3$ refer to this group. 
\end{definition}

\begin{theorem}
The group $S^3$ is isomorphic to the group $\SU{2}$. 
\end{theorem}

\begin{proof}

\end{proof}


\begin{theorem}
There is a 2-fold covering map $\pi : \SU{2} \to \SO{3}$ making $\SU{2}$ the univeral cover of $\SO{3}$. 
\end{theorem}

\begin{proof}
We will give two equivalent descriptions of this map in terms of quaterions and complex matrices. This map turns out to be extremely practically useful in physics and physics applications for parametrizing rotations. Therefore, it is worth understanding it in detail.
\end{proof}

\begin{remark}
Topologically this is the covering map $\pi : S^3 \to \mathbb{RP}^3$. 
\end{remark}



\section{The Hopf Fibration}

\begin{remark}
There is a notion of a fibration and of a fibre bundle. In this case they are equivalent and the concept of a fibre bundle is easier to understand.
\end{remark}

\begin{definition}
A fibre bundle is a morphism $\pi : E \to B$ with fibre $F$ usually witten as
\begin{center}
\begin{tikzcd}
F \arrow[r, hook] & E \arrow[r] & B
\end{tikzcd}
\end{center}
is a map such that there are local trivializations, i.e an open cover $U_\alpha \subset B$ with isomorphisms $\varphi_\alpha : \pi^{-1}(U_\alpha) \to U_\alpha \times F$ such that the following diagram commutes,
\begin{center}
\begin{tikzcd}[row sep = large, column sep = small]
\pi^{-1}(U_\alpha) \arrow[dr, "\pi"] \arrow[rr, "\varphi_\alpha"] & & U_\alpha \times F \arrow[dl]
\\
& U 
\end{tikzcd}
\end{center}
\end{definition}

\begin{definition}
A section of a bundle $\pi : E \to B$ is a morphism $s : B \to E$ such that $\pi \circ s = \id_B$. 
\end{definition}

\begin{example}
The trivial bundle with fiber $F$ over $X$ is simply $\pi : X \times F \to X$ given by projection by the first factor. For example, the tivial bundle of $\R$ over $S^1$ is a cylinder $\R \times S^1$.  
\end{example}

\begin{example}
The m\"{o}bius bundle is $\R \hookrightarrow M \to S^1$ is given by a twist of $\R$ over $S^1$ such that $M$ is a m\"{o}bius strip. $M$ is nontrivial in the sense that $M \not\cong \R \times S^1$ because $M$ does not admit a nonvanishing section.
\end{example}

\begin{remark}
A fibre bundle $F \hookrightarrow E \to B$ is morally a way of building $E$ up fibers $F$ parametrized by the space $B$ along the map $\pi : E \to B$. 
\end{remark}

\begin{proposition}
The Hopf Fibration or Hopf Bundle is a fibre bundle $S^1 \hookrightarrow S^3 \to S^2$ of spheres with the structure map $\pi : S^3 \to S^2$ defined by $q \mapsto q \cdot \vec{v} \cdot q^{-1}$ for $q \in S^3 \subset \H$ and $\vec{v}$ a fixed unit vector in $\H$ i.e. $\vec{v} \in S^2 \subset \R^3 \subset \H$.  
\end{proposition}

\begin{proof}
Suppose that $q_1 \vec{v} q_1^{-1} = q_2 \vec{v} q_2^{-1}$ then $(q_2^{-1} q_1) \vec{v} (q_1 q_2^{-1})^{-1} = \vec{v}$. Thus $(q_2^{-1} q_1)$ maps to a rotation fixing $\vec{v}$ so it must have axis parallel to $\vec{v}$ so $(q_2^{-1} q_1) = e^{a \vec{v}}$ for some $a \in S^1$. Thus $F = \pi^{-1}(\vec{u}) \cong S^1$. The map is a fibre bundle because for sufficiently small open balls $U \subset S^2$ we have $\pi^{-1}(U) \cong U \times S^1$ which is a torus. Thus the Hopf fibration builds $S^3$ out of interlocking tori.
\end{proof}

\begin{remark}
The Hopf fibration is simply the covering map $\pi : S^3 \to \SO{3}$ composed with the canonical action of $\SO{3}$ on $S^2$ by rotation. 
\end{remark}

\begin{remark}
Steriographic projection.
\end{remark}

\section{Clifford Algebras}


\begin{definition}
A quadratic form over a field $K$ is a map $q : V \to K$ where $V$ is a finite-dimensional $K$-vectorspace which satisfies,
\begin{enumerate}
\item $\forall a \in K, v \in V : q(a v) = a^2 q(v)$

\item The map $B(u,v) = q(u + v) - q(u) - q(v)$ is bilinear.
\end{enumerate}
If $\fchar{K} \neq 2$ then we may define the associated bilinear form,
\[ B(u,v) = \tfrac{1}{2} (q(u + v) - q(u) - q(v)) \]
such that $q(v) = B(v, v)$. Thus, for $\fchar{K} \neq 2$ the theory of quadratic forms is equivalent to that of bilinear forms. 
\end{definition}

\begin{definition}
Let $V$ be a $K$-vectorspace and $Q : V \to K$ a quadratic form on $V$. A Clifford algebra $\Cl(V, Q)$ is a unital associative algebra over $K$ with a linear map $i : V \to \Cl(V, Q)$ which satisfies,
\[ \forall v \in V : i(v)^2 = Q(v) \cdot 1 \]
and the following universal property. Let $A$ be any unital associative $K$-algebra and $j : V \to A$ any linear map satisfying the condition,
\[ \forall v \in V : j(v)^2 = Q(v) \cdot 1_A \]
then there exists a unique $K$-algebra homomorphism $f : \Cl(V, Q) \to A$ such that,
\begin{center}
\begin{tikzcd}[row sep = huge]
V  \arrow[rd, "j"'] \arrow[r, "i"] & \Cl(V, Q) \arrow[d, dashed, "f"]
\\
& A
\end{tikzcd}
\end{center} 
commutes i.e. $f \circ i = j$. Thus $\Cl(V, Q)$ is the ``freest'' such algebra. It is clear that $T(V) / I_Q$ where $I_Q$ is the two-sidded ideal of $T(V)$ generated by 
\[ \{ v \otimes v - Q(v) \mid v \in V \} \]
satisfies this universal property and thus $\Cl(V, Q) = T(V) / I_Q$ up to unique isomorphism making this \textit{the} Clifford algebra over $V$ with respect to $Q$. 
\end{definition}

\begin{theorem}
There is a functor $\Cl$ from the category of $K$-vectorspaces endowed with quadratic forms to the category of unital associative $K$-algebras.
\end{theorem}

\begin{proof}
Let $(V, Q_V)$ and $(W, Q_W)$ be vectorspaces paired with quadratic forms and $h : (V, Q_V) \to (W, Q_W)$ a morphism, that is a $K$-linear map $h : V \to W$ such that $Q_W \circ h = Q_V$. Then consider the diagram,
\begin{center}
\begin{tikzcd}[row sep = huge]
V  \arrow[d, "h"'] \arrow[r, "i_V"] & \Cl(V, Q_V) \arrow[d, dashed, "f"]
\\
W \arrow[r, "i_W"] & \Cl(W, Q_W)
\end{tikzcd}
\end{center}
Because $(i_W \circ h(v))^2 = Q_W(h(v)) \cdot 1_W = Q_V(v) \cdot 1_W$ then the map $i_W \circ h(v)$ factors uniquely through $\Cl(V, Q_V)$ to give a unique map $f : \Cl(V, Q_V) \to \Cl(W, Q_W)$ lifting $h$. By the uniqueness of this lift this proceedure automatically satisfies functoriality. 
\end{proof}

\begin{definition}
Let $A$ be a unital associative $K$-algebra. An $A$-representation is a $K$-vectorspace $V$ with a $K$-algebra map $\rho : A \to \End{V}$. Equivalently, a representation $V$ is simply an $A$-module with $K$-vectorsapce structure induced by the inclusion $K \to A$ making $V$ a $K$-module.
\end{definition}

\begin{proposition}
Let $\fchar{K} \neq 2$ and $Q : V \to K$ be a quadratic form with associated bilinear form $B$. Then the Clifford condition is equivalent to,
\[ \forall u,v \in V : i(v) i(u) + i(u) i(v) = 2 B(v,u) \cdot 1 \]
\end{proposition}

\begin{proof}
First, note that $B(v, v) = Q(v)$ and thus setting $u = v$ gives,
\[ i(v)^2 + i(v)^2 = 2 Q(v) \cdot 1 \]
which for $\fchar{K} \neq 2$ implies the defining condition. Convsersely, if for each $v \in V$ we have,
\[ i(v)^2 = Q(v) \cdot 1 \]
then consider,
\[ i(v + u)^2 = (i(v) + i(u))^2 = i(v)^2 + i(v) i(u) + i(u) i(v) + i(u)^2 = Q(v) \cdot 1 + i(v) i(u) + i(u) i(v) + Q(u) \cdot 1 \]
However, we also have,
\[ i(v + u)^2 = Q(v + u) \cdot 1 \]
which implies that,
\[ i(v) i(u) + i(u) i(v) = [Q(v + u) - Q(v) - Q(u)] \cdot 1 = 2 B(v, u) \cdot 1 \]
\end{proof}

\begin{remark}
From now on we will assume that $\fchar{K} \neq 2$ and $V$ is a vectorspace of dimension $n$ over $K$ with quadratic form $Q : V \to K$ and associated bilinear form $B : V \times V \to K$. 
\end{remark}

\begin{proposition}
Let $\{ e_1, \dots, e_n \}$ be an orthogonal basis for $V$ with respect to $B$ i.e. $B(e_i, e_j) = \delta_{ij} Q(e_i)$. Then $\Cl(V, Q)$ has a basis,
\[ \{ e_{i_1} e_{i_2} \cdots e_{i_k} \mid 1 \le i_1 < i_2 \cdots < i_k  \le n \text{ and }  1 \le k \le n \} \]
and thus,
\[ \dim_K \Cl(V, Q) = \sum_{k = 1}^n { n \choose k } = 2^n \]
\end{proposition}

\begin{proposition}
Any vectorspace over $K$ with a nondegenerate quadratic form $Q : V \to K$ is isomorphic to $K^n$ with the stanard quadratic form,
\[ Q_{p,q}(v) = v_1^2 + \cdots + v_p^2 - v_{p+1}^2 - \cdots - v_{p + q}^2 \]
where $p$ is the number of positive eigenvalues and $q$ the number of negative eigenvalues. We then denote the Clifford algebra in this case,
\[ \Cl_{p+q}(K) = \Cl(K^n, Q_{p,q}) = \Cl(V, Q) \]
\end{proposition}


(EXAMPLES)

(PROPOSITIONS)

(Cl(1,3)(C) = C04(C) is WICK ROTATION)

\subsection{The Pin and Spin Groups}

\subsection{Representations of Clifford Algebras}

\section{General Theory of Transformations in Physics}

\begin{remark}
Quantum states are not defined up to scaling which motivates the following definition.
\end{remark}

\begin{definition}
Let $\Hil$ be a quantum Hilbert space then the projective space \textit{state space} or \textit{ray space} is $\P(\Hil)$ with the metric,
\[ d(\underline{\psi}, \underline{\phi}) = \cos^{-1}{\sqrt{\frac{\inner{\psi}{\phi} \inner{\phi}{\psi}}{\inner{\psi}{\psi} \inner{\psi}{\phi}}}} \]
where we define the \textit{ray product}
\[ \underline{\psi} \cdot  \underline{\phi} = \sqrt{\frac{\inner{\psi}{\phi} \inner{\phi}{\psi}}{\inner{\psi}{\psi} \inner{\psi}{\phi}}} = \frac{| \inner{\psi}{\phi} |}{|\psi| |\phi|} \]
which is the Fubini-Study metric on $\CP^n$. 
\bigskip\\
A \textit{symmetry transformation} is a ray transformation, that is a homeomorphism \[ T : \P(\Hil) \to \P(\Hil') \]
which is an isometry i.e. preserves the ray product. Explicitly,
\[ T \underline{\Psi} \cdot T \underline{\Phi} = \underline{\Psi} \cdot \underline{\Phi} \]
\end{definition}

\begin{theorem}[Wigner]
Any symmetry transformation $T : \P(\Hil) \to \P(\Hil')$ lifts to a $\R$-linear map $T : \Hil \to \Hil'$ which is either unitary (and thus $\C$-linear) or anti-unitary (and thus $\C$-antilinear). 
\end{theorem}

\begin{theorem}
Let $T : \P(\Hil) \to \P(\Hil')$ be a ray transformation and $U, V : \Hil \to \Hil'$ are lifts of $T$ i.e. linear maps compatible with $T$. Assume that $\dim{\Hil} \ge 2$ then $V = U e^{i \alpha}$ for some $\alpha \in \R$. 
\end{theorem}

\begin{definition}
The \textit{symmetry group} of a Hilbert space $\Hil$ is the group $\Gamma(\Hil)$ of symmetry transformations of the ray space $\P(\Hil) \to \P(\Hil)$. A a \textit{quantum representation} of a group $G$ on $\Hil$ is a projective representation $G \to \PAU{\Hil} = \Gamma(\Hil)$ by Wigner's theorem. 
\bigskip\\
In general, a quantum field is an operator valued section of the associated bundle $E = P \times_\rho \P(V)$ to a projective representation $\rho : G \to \PGL{V}$ where $P \to M$ is a principle $G$-bundle over $M$ (usually the frame bundle or the bundle of spin frames). 
\end{definition}

\section{The Wigner-Eckart Theorem}

\begin{lemma}
Let $G$ be a compact Lie group and $\rho : G \to \Aut{V}$ a $G$-representation. There is a projection map $s :V \to V^G$ given by,
\[ s(v) = \frac{1}{|G|} \int_{g \in G} \rho(g) \cdot v \]
\end{lemma}

\begin{proof}
First, this map is well-defined because for any $v \in V$ and $g \in G$ we have,
\begin{align*}
\rho(g) \cdot s(v) = \frac{1}{|G|} \int_{h \in G} \rho(g) \cdot \rho(h) \cdot v = \frac{1}{|G|} \int_{h \in G} \rho(gh) \cdot v = \frac{1}{|G|} \int_{h \in G} \rho(h) \cdot v = s(v) 
\end{align*}
Thus $s(v) \in V^G$.
Consider the inclusion $\iota : V^G \to V$. Then,
\[ s \circ \iota(v) = \frac{1}{|G|} \int_{g \in G} \rho(g) \cdot v = \frac{1}{|G|} \int_{g \in G} v = v \]
thus $p \circ \iota = \id_{V^G}$. 
\end{proof}

\begin{lemma}
Let $G$ be a compact Lie group, $V$ a Hilbert space, and $\rho : G \to \Aut{V}$ a unitary representation. If $V$ is finite-dimensional then for any $G$-subrep $W \subset V$ there exists $G$-equivariant projection $p : V \to W$ satisfing $p^\dagger = \iota_W$. 
\end{lemma}

\begin{proof}
As inner product spaces, there automatically exists a projection $q : V \to W$ because $V$ is finite-dimensional satisfing $q \circ \iota_W = \id_W$ and $q^\dagger = \iota_W$ i.e.  is orthogonal. Then apply the previous lemma to the $G$-rep $\Hom{V}{W}$ to get a morphism $p = s(q) \in \catHom{G}{V}{W}$. It remains to show that $p \circ \iota_W = \id_W$ and $p^\dagger = \iota_W$. Since $\iota_W$  is $G$-equivariant we have,
\begin{align*}
 p \circ \iota_W & = \frac{1}{|G|} \int_{g \in G} \rho(g) \circ q \circ \rho(g)^{-1} \circ \iota_W = \frac{1}{|G|} \int_{g \in G} \rho(g) \circ q  \circ \iota_W \circ \rho(g)^{-1}
\\
& = \frac{1}{|G|} \int_{g \in G} \rho(g) \circ \id_W \circ \rho(g)^{-1} = \id_W
\end{align*}
Finally, because $\rho : G \to \Aut{V}$ is a unitary rep,
\begin{align*}
p^\dagger & = \left( \frac{1}{|G|} \int_{g \in G} \rho(g) \circ q \circ \rho(g)^{-1} \right)^\dagger =  \frac{1}{|G|}  \int_{g \in G} (\rho(g)^{-1})^{\dagger} \circ q^\dagger \circ \rho(g)^\dagger
\\
& = \frac{1}{|G|} \int_{g \in G} \rho(g) \circ q \circ \rho(g)^{-1} = p
\end{align*}
\end{proof}

\begin{lemma}
Let $G$ be a compact Lie group then any finite-dimensional unitary $G$-representation is decomposable as Hilbert space $G$-reps into sum of irreducible unitary $G$-reps.
\end{lemma}

\begin{proof}
Let $\rho : G \to \Aut{V}$ be the lowest-dimension unitary $G$-rep which is not decomposable in the desired maner. In particular, $V$ is not irreducible. Thus, there is a nontrivial submodule $W \subset V$ which then has a $G$-equivariant hermitian projection $p : V \to W$. Therefore, $\ker{p}$ and $W$ are orthogonal since,
\[ \inner{v}{w}_W = \inner{v}{\iota_W w}_W = \inner{\iota_W^\dagger v}{w}_V = \inner{p(v)}{w}_V = 0 \]
Then there is an exact sequence of $\C[G]$-modules,
\begin{center}
\begin{tikzcd}
0 \arrow[r] & \ker{p} \arrow[r] & V \arrow[r, "p"] & W \arrow[l, "\iota_W", bend left] \arrow[r] & 0
\end{tikzcd}
\end{center}
Since $p \circ \iota_W = \id_W$ the sequence splits so we find that,
\[ V = W \oplus \ker{p} \]
as $G$-representations and Hilbert spaces. Since the dimension of $V$ is minimal, these factors must be decomposable in the proper way which implies that $V$ is as well.
\end{proof}

\begin{corollary}
Let $G$ be a compact Lie group. Then any finite-dimensional $G$-representation is semisimple.
\end{corollary}

\begin{proof}
Apply the unitarian trick: every finite-dimensional $G$-rep is unitary. To show this consider a finite-dimensional $G$-rep $\rho : G \to \Aut{V}$ and any inner product $\inner{\bullet}{\bullet}$ on $V$. Then define,
\[ \inner{v}{u}_G = \frac{1}{|G|} \int_{g \in G} \inner{\rho(g) \cdot v}{ \rho(g) \cdot u} \]
which is an inner products and $r\rho : G \to \Aut{V}$ is unitary with respect to it.
\end{proof}

\begin{definition}
We say a connected compact Lie group $G$ has the \textit{Clebsch-Gordon property} if for every pair of finite-dimensional irreducible complex representations $V_1$ and $V_2$ then the tensor product representation $V_1 \otimes_\C V_2$ is decomposed into \textit{distinct} irreducible $G$-representations i.e. $V_1 \otimes_\C V_2$ does not contain two isomorphic copies of the same irreducible representation.
\end{definition}

\begin{lemma}
Let $G$ be a connected compact Lie group and $\rho : G \to \Aut{\Hil}$ a unitary representation. Let $W_1, W_2 \subset \Hil$ be irreducible $G$-submodules. Then either $W_1 \cong W_2$ or $W_1 \perp W_2$. 
\end{lemma}

\begin{proof}
There exists a $G$-invariant orthogonal projection map $p : \Hil \to V_2$. Then $p |_{W_1} : W_1 \to W_2$ gives a $G$-morphism. By Schur, either $W_1 \cong W_2$ or $p|_{W_1} = 0$. The latter implies that for each $w \in W_1$ we have $p(w) = 0$ i.e. $w \in W_2^\perp$ so $W_1 \perp W_2$. 
\end{proof}

\begin{definition}
Let $V$ be a $\C$-vector space and $\overline{V} = V \otimes_\sigma \C$ where $\sigma : \C \to \C$ is complex conjugation. The canonical map $\sigma : V \to \overline{V}$ is then $\C$-antilinear (real) isomorphism since, 
\[ \sigma(\alpha v) = \alpha v \otimes 1 =  v \otimes \bar{\alpha} = \bar{\alpha} \cdot (v \otimes 1) = \bar{\alpha} \sigma(\alpha v) \]
\bigskip\\
If $V$ is a $G$-rep then $V \otimes_\sigma \C$ is canonically a $G$-rep via,
\[ \overline{\rho}(g) = \sigma \circ \rho(g) \circ \sigma^{-1} \] 
\end{definition}

\begin{lemma}
Let $V$ be a finite-dimensional $G$-rep then,
\[ V \text{ is irreducible} \iff \overline{V} \text{ is irreducible} \iff V^* \text{ is irreducible} \]
\end{lemma}

\begin{proof}
If $V^*$ is irreducible take a $G$-invariant subspace $W \subset V$ and conside,
\[ W^\perp = \{ \phi \in V^* \mid \forall w \in W : \phi(w) = 0 \} \]
This is $G$-invariant because $\rho^*(g) \cdot \phi = \phi \circ \rho(g^{-1})$ and $\phi(\rho(g^{-1}) w) = 0$ since $W$ is $G$-invariant. Thus, either $W^\perp = (0)$ or $W^\perp = V^*$. In the first case $W = V$ otherwise there would be a nontrial projection onto its complement sending it to zero, in the second case, $W = (0)$ otherwise the projection onto any vector in $W$ would not vanish on $W$ contradicting $W^\perp = V^*$. Therefore $V$ is irreducible. Furthermore $V^{**} \cong V$ as $G$-modules so if $V$ is irreducble then so is $V^{**}$ which implies, by the above, that $V^*$ is irreducible. 
\bigskip\\
Now suppose that $\overline{V}$ is irreducible take a $G$-invariant subspace $W \subset V$ and consider $W' = \sigma(W) \subset \overline{V}$. This is a subspace because it is additive and $\alpha \sigma(W) = \sigma(\bar{\alpha} W) = \sigma(W)$. Since $\overline{V}$ is irrducible $W' = (0)$ or $W' = \overline{V}$. Since $\sigma$ is an isomorphism this implies that $W = (0)$ or $W' = \overline{V}$ and thus $V$ is irreducible. Furthermore, $\overline{\overline{V}} = V$ so if $V$ is irreducible then so is $\overline{\overline{V}}$ which implies, by the above, that $\overline{V}$ is irreducible. 
\end{proof}

\begin{lemma}
A sesquilinear form on $V$, equivalently a linear map $S : \overline{V} \otimes_\C V \to \C$ is the same as a linear map $\overline{V} \to V^*$. If $V$ is a $G$-rep then a sequilinear form on $V$ is $G$-invariant iff the associated linear map $\overline{V} \to V^*$ is a $G$-morphism. 
\end{lemma}

\begin{proof}
Consider the map $\phi_S : \overline{V} \to V^*$ via $\psi_S(v \otimes 1) = \inner{v}{-}$ which is a linear functional on $V$ and thus in $V^*$. Furthermore $\psi_S$ is $\C$-linear since,
\[ \phi_S(v \otimes \alpha) = \phi_S(\bar{\alpha} v \otimes 1) = \inner{\bar{\alpha} v}{-}_S = \alpha \inner{v}{-}_S = \alpha \phi_S(v) \]
and it is clearly additive. Conversely, any $\C$-linear map $\phi : \overline{V} \to V^*$ defined a sequilinear form via $\inner{v}{w}_{\phi} = \phi(\sigma(v))(w)$. This is sesquilinear since,
\[ \inner{\alpha v + \beta u}{w}_\phi = \phi(\sigma(\alpha v + \beta u))(w) = \bar{\alpha} \phi(\sigma(v))(w) + \bar{\beta} \phi(\sigma(u))(w) = \bar{\alpha} \inner{v}{w}_\phi + \bar{\beta} \inner{u}{w}_\phi \]
Furthermore, if the sequlinear map is $G$-invariant the,
\begin{align*}
\phi(\overline{\rho}(g) \bar{v}) & = \inner{\rho(g) v}{w} = \inner{\rho(g) v}{\rho(g) \circ \rho(g^{-1}) w} = \inner{v}{\rho^{-1}(g)w}
\\
& = \phi(v)(\rho^{-1}(g)w) = [\rho^*(g) \cdot \phi(v)](w)
\end{align*}
and thus $\phi \circ \bar{\rho}(g) = \rho^*(g) \circ \phi$. Likewise if $\phi$ is a $G$-morhpism then,
\begin{align*}
\inner{\rho(g) \cdot v}{\rho(g) \cdot w} & = \phi(\sigma \circ \rho(g) \cdot v)(\rho(g) \cdot w) = \phi(\overline{\rho}(g) \cdot \sigma(v))(\rho(g) \cdot w) 
\\
& = [ \phi \circ \overline{\rho}(g)](v)(\rho(g) \cdot w) = [\rho^*(g) \circ \phi](\sigma(v))(\rho(g) \cdot w) 
\\
& = \phi(\sigma(v))(\rho(g^{-1}) \circ \rho(g) \cdot w) = \phi(\sigma(v))(w) = \inner{v}{w}
\end{align*}
and thus $\inner{-}{-}$ is $G$-invariant.
\end{proof}

\begin{lemma}
Let $V$ be a finite dimensional irreducible $G$-representation then, up to scale, there is a unique $G$-invariant sesquilinear form on $V$.
\end{lemma}

\begin{proof}
We know that $G$-invariant sesquilinear forms are the same as $G$-morphisms $\overline{V} \to V^*$ which, since $V$ is irreducible, are irreducible. Therefore, by Schur's lemma, $G$-morphisms $\overline{V} \to V^*$ are unique up to scale. 
\end{proof}

\begin{lemma}
Let $G$ be a connected compact Lie group and $\rho : G \to \Aut{\Hil}$ a unitary representation. Let $W_1, W_2 \subset \Hil$ be isomorphic irreducible $G$-submodules. Then, for any isomorphism $f : W_1 \to W_2$, there exists a constant $c$ such that,
\[ \forall w_1, w_2 \in W_1 : \inner{w_1}{f(w_2)} = c \inner{w_1}{w_2} \]
\end{lemma}

\begin{proof}
The sesquilinear form $\inner{w_1}{f(w_2)}$ is $G$-invariant because $f$ is a $G$-morphism meaning that,
\[ \inner{\rho(g) \cdot w_1}{f(\rho(g) \cdot w_2)} = \inner{\rho(g) \cdot w_1}{\rho(g) \cdot f(w_2)} = \inner{w_1}{f(w_2)} \]
where I have used the fact that $\inner{-}{-}$ is $G$-invariant because $\rho : G \to \Aut{V}$ is unitary. Therefore, by the lemma, the $G$-invariant sesquilinear forms $\inner{-}{-}$ and $\inner{-}{f-}$ are equal up to scale. Since $\inner{-}{-}$ is a Hermitian inner product, $\forall v \neq 0 : \inner{v}{v} > 0$, so it is nondegenerate (the associted $G$-morphism $\overline{V} \to V^*$ is an isomorphism) which implies that any other sesquilinear form (equivalently isomorphism $\overline{V} \to V^*$), in particular $\inner{-}{f-}$, must be a constant multiple of it. 
\end{proof}

\begin{lemma}
Let $V$ be an irreducible $G$-rep and $f : V \to W$ a $G$-morphism. Then either $f$ is zero or induces an isomorphism onto its image making $V$ a irreducible factor of $W$.
\end{lemma}

\begin{proof}
Since $\ker{f} \subset V$ is $G$-invariant we must have $\ker{f} = V$ or $\ker{f} = 0$. In the first case $f = 0$, in the second, $f : V \to W$ is injective and thus $f : V \to \Im{f}$ is an isomorphism onto a $G$-submodule.
\end{proof}

\begin{theorem}[Wigner-Eckart]
Let $G$ be a connected compact Lie group with the Clebsch-Gordon property. Let $\Hil$ be a Hilbert space and $\rho : G \to \Aut{\Hil}$ a unitary representation. Let $V_1, V_2 \subset \Hil$ be finite-dimensional irreducible $G$-subrepresentations. Let $\mathcal{O}_1, \mathcal{O}_2 \subset \End{\Hil}$ be finite-dimensional irreducible $G$-submodules and $f : \mathcal{O}_1 \to \mathcal{O}_2$ an isomorphism. Then there exists a constant $c_f \in \C$ such that either,
\[ \forall v_1 \in V_1, v_2 \in V_2, T \in \mathcal{O}_1 : \inner{v_1}{f(T) v_2} = c \inner{v_1}{T v_2} \]
or $\inner{v_1}{T v_2} = 0$ for all $V_1 \in V_1$, $w_2 \in W_2$, and $T \in \mathcal{O}_1$. 
\end{theorem}

\begin{proof}
Consider diagram of $G$-morphisms,
\begin{center}
\begin{tikzcd}[column sep = large, row sep = large]
\mathcal{O}_1 \otimes V_2 \arrow[d, "f \otimes \id"] \arrow[r, "e"] & \mathcal{O}_1(V_2) 
\\
\mathcal{O}_2 \otimes V_2 \arrow[r, "e"] & \mathcal{O}_2(V_2)
\end{tikzcd}
\end{center}
By the Clebsch-Gordon property, $\mathcal{O}_1 \otimes V_2$ and $\mathcal{O}_2 \otimes V_2$ decompose into irreducible $G$-reps with at most one factor of $V_1$. Since the evaulation morphism is surjective, this implies that $\mathcal{O}_1(V_2)$ and $\mathcal{O}_2(V_2)$ can have have at most one copy of $V_1$ in their decomposition. If none of its irreducible representations are isomorphic to $V_1$ then we have seen that the inner products $\inner{v_1}{T v_2}$ vanish. Likewise, if $V_1$ does not appear in the decomposition of $\mathcal{O}_2(V_2)$ then all $\inner{v_1}{f(T) v_2}$ vanish. Therefore, we may assume that both $\mathcal{O}_1(V_2)$ and $\mathcal{O}_2(V_2)$ contain a copy of $V_1$ which we denote as $W_1, W_2$ reprectivly. Restricting the diagram, we find,
\begin{center}
\begin{tikzcd}[column sep = large, row sep = large]
W_1^\otimes \subset \mathcal{O}_1 \otimes V_2 \arrow[d, "f \otimes \id"] \arrow[r, "e"] & W_1 \arrow[d, dashed, "\tilde{f}"] \subset \mathcal{O}_1(V_2) 
\\
W_2^\otimes \subset \mathcal{O}_2 \otimes V_2 \arrow[r, "e"] & W_2 \subset \mathcal{O}_2(V_2)
\end{tikzcd}
\end{center}
These arrows become isomorphisms since they are nontrivial $G$-morphisms which alows the construction of $\tilde{f}$. As we have shown, nonisomorphic irreducible $G$-submodules are perpendicular so to compute these inner products we need only to consider $T v_2 \in W_1 \subset \mathcal{O}_1(V_2)$ and $T' v_2 \in W_2 \subset \mathcal{O}_2(V_2)$ since all other components product to zero with $v_1 \in V_1$. Now, consider,
\[ f'(T v_2) = e \circ (f \otimes \id) \circ e^{-1}(T v_2) = e \circ (f \otimes \id)(T \otimes v_2) = e(f(T) \otimes v_2) = f(T) v_2 \]
Now choose an isomorphism $g : V_1 \to W_1$ then $f' \circ f : V_1 \to W_2$ is an isomorphism. Therefore, by the previous lemma, there exist constants, $c_1$ and $c_2$ such that,
\[ \forall v_1, v_2' \in V_1 : \inner{v_1}{g(v_2')} = c_1 \inner{v_1}{v_2'} \quad \quad \inner{v_1}{f' \circ g(v_2')} = c_2 \inner{v_1}{v_2'} \]
In paricular, choosing $v_2' = g^{-1}(T v_2)$ for any $T v_2 \in W_2$ gives,
\[  \inner{v_1}{Tv_2} = c_1 \inner{v_1}{g^{-1}(T v_2)} \quad \quad \inner{v_1}{f(T) v_2} = c_2 \inner{v_1}{g^{-1}(T v_2)} \]
which, if $c_1 \neq 0$ taking $c_f = c_2 / c_1$ and otherwise $\inner{v_1}{T v_2} = 0$ proves the theorem. 
\end{proof}

\begin{remark}
In practice, if we have a $G$-representation on a set of operators this theorem reduces computing $\inner{v_1}{T v_2}$ for each operator and vector in the representation, to computing a single constant as long as we know these inner products for any other set of operators in the same representation. In fact, we have shown that these inner products are up to scale the same as the overlap coefficients in the tensor product decomposition.
\end{remark}

\section{Proof of the Main Theorem}

\subsection{Setup}

\newcommand{\ph}{\hat{\phi}}
\newcommand{\phd}{\hat{\phi}^\dagger}
\newcommand{\Hilbert}{\mathcal{H}}
\newcommand{\Spin}[1]{\mathrm{Spin}\left( #1 \right)}
\newcommand{\ket}[1]{\left| #1 \right>}
\newcommand{\bra}[1]{\left< #1 \right|}

Consider the Hilbert space $\Hilbert$ of our possibly interacting QFT which has unitary projective representation of the Poincare group or equivalently  a unitary representation of its double cover. This representation gives a representation of the Lie algebra which has generators,
\begin{subequations}
\begin{align}
\hat{P}_i
\quad \quad
\hat{K}_i
\quad \quad
\hat{J}_i
\end{align}
\end{subequations}
acting as operators on $\Hilbert$ and satisfiyng the commutation relations, 
\begin{subequations}
\begin{align}
[\hat{J}_i, \hat{J}_j] = i \epsilon_{ijk} \hat{J}_k
\end{align}
\end{subequations}
We assume there is an (interacting) vacuum state $\ket{\Omega}$ is is invariant under the action of the Poincar\'{e} group (AND UNDER CPT?).
A particle is given by a multiplet of field operators $\ph_i(x)$ in a finite-dimensional irreducible $\Spin{1,3}$-representation $S : \Spin{1,3} \to W$. Explicitly,  this multiplet transforms under a Lorentz transformation $\Lambda$ (or more generally an element of $\Spin{1,3}$) via,
\begin{subequations}
\begin{align}
U(\Lambda) \ph_i(x) U(\Lambda)^{-1} & = S_{ij}(\Lambda) \ph_j(\Lambda^{-1} x)
\end{align}
\end{subequations}
where $\Lambda \in \Spin{1,3}$ acts on $x$ via the covering projection $\pi : \Spin{1,3} \to \SO{1,3}$ which is thus acts as a matrix. Note that the representation $W$ must be of the form $(\mu, \nu)$ by the classification of projetive $\SO{1,3}$-reps and cannot be unitary because $W$ is finite dimensional. Furthermore, the Poincare group translations act via,
\[ e^{i \hat{P}_\mu x^\mu} \ph(y^\mu) e^{- i \hat{P}_\mu x^\mu} = \ph(x^\mu + y^\mu)  \]

\subsection{The K\"{a}ll\'{e}n-Lehmann Spectral Resolution}

We must compute the two-point correlation functions,
\begin{subequations}
\begin{align}
\bra{\Omega} \ph_i(x) \phd_j(y) \ket{\Omega} &
\\
\bra{\Omega} \phd_j(y) \ph_i(x) \ket{\Omega} &
\end{align}
\end{subequations}
We now use a resolution of unity. However, since $\ph$ is a charged field, by selection rules, we need only consider single particle states in the resolution,
\begin{subequations}
\begin{align}
\bra{\Omega} \ph_i(x) \phd_j(y) \ket{\Omega} & = \sum_{\lambda, s} \int\limits_{p^2 = m_\lambda^2} \frac{\dn{3}{p}}{(2 \pi)^3} \frac{1}{2 p_0} \bra{\Omega} \ph_i(x) \ket{\lambda_{p,s}} \bra{\lambda_{p,s}} \phd_j(y) \ket{\Omega}
\end{align}
\end{subequations}
where $\ket{\lambda_{s}}$ is a $W$-multiplet of single particle rest states of mass $m_\lambda$ and $\ket{\lambda_{p,s}} = U(B_p) \ket{\lambda_s}$ where $B_p$ is the Lorentz transformation which performs a bost to momentum $p$. Since $\hat{P}_i \ket{\lambda_s} = 0$ and $[\hat{P}_i, U(B_p)] = p$, (WHY?)
Therefore,
\begin{subequations}
\begin{align}
\bra{\Omega} \ph_i(x) \ket{\lambda_{p,s}} & = \bra{\Omega} e^{i \hat{P}_\mu x^\mu} \ph_i(0) e^{-i \hat{P}_\mu x^\mu} \ket{\lambda_{p,s}}
\\
& = \bra{\Omega} \ph_i(0) \ket{\lambda_{p,s}} e^{- i p \cdot x}
\end{align}
\end{subequations}
using the fact that $e^{i \hat{P}_\mu x^\mu} \ket{\lambda_{p,s}} = e^{i p_\mu x^\mu} \ket{\lambda_{p,s}}$ and the vacuum is invariant under the Poincare group. Now,
\begin{subequations}
\begin{align}
\bra{\Omega} \ph_i(x) \ket{\lambda_{p,s}} & = 
\bra{\Omega} \ph_i(0) U(B_p) \ket{\lambda_{s}} e^{- i p \cdot x}
\\
& = \bra{\Omega} \ph_i(0) U(B_p) \ket{\lambda_{s}} e^{- i p \cdot x}
\end{align}
\end{subequations}
Again, using the fact that $\ket{\Omega}$ is Poincar\'{e} invariant we find,
\begin{subequations}
\begin{align}
\bra{\Omega} \ph_i(x) \ket{\lambda_{p,s}}
& = \bra{\Omega} U^\dagger(B_p) \ph_i(0) U(B_p) \ket{\lambda_{s}} e^{- i p \cdot x}
\\
& =  S_{ik}(B_p^{-1}) \bra{\Omega} \ph_k(0) \ket{\lambda_{s}} e^{-i p \cdot x}
\end{align}
\end{subequations}
Furthermore, since $\ph_k$ and $\ket{\lambda_s}$ are both $W$-multiplets, by the Wigner-Ekart theorem we have,
\begin{equation}
\bra{\Omega} \ph_k(0) \ket{\lambda_{s}} = C_\lambda \delta_{ks}
\end{equation}
for some constant $C_\lambda \in \C$. Thus,
\begin{equation}
\bra{\Omega} \ph_i(x) \ket{\lambda_{p,s}} = C_\lambda S_{is}(B_p^{-1})  e^{-i p \cdot x}
\end{equation}
Notting that,
\begin{equation}
\bra{\lambda_{p,s}} \phd_j(y) \ket{\Omega} = \overline{\bra{\Omega} \ph_j(y) \ket{\lambda_{p,s}}} = C^*_\lambda S^*_{js}(B_p^{-1})  e^{i p \cdot y}
\end{equation}
Therefore, defining $Z_\lambda = |C_\lambda|^2 \ge 0$ we have,
\begin{subequations}
\begin{align}
\bra{\Omega} \ph_i(x) \phd_j(y) \ket{\Omega} &= \sum_{\lambda} \int\limits_{p^2 = m_\lambda^2} \frac{\dn{3}{p}}{(2 \pi)^3} \frac{e^{-i p \cdot (x - y)}}{2 p_0} Z_\lambda \sum_s S_{is}(B_p^{-1}) S^*_{js}(B_p^{-1})
\\
& = \sum_{\lambda} \int\limits_{p^2 = m_\lambda^2} \frac{\dn{3}{p}}{(2 \pi)^3} \frac{e^{-i p \cdot (x - y)}}{2 p_0} Z_\lambda \: (S(B_p^{-1}) S^\dagger(B_p^{-1}))_{ij}
\end{align}
\end{subequations}
A similar computation shows that,
\begin{subequations}
\begin{align}
\bra{\Omega} \phd_j(y) \ph_i(x) \ket{\Omega} 
&= \sum_{\bar{\lambda}, s} \int\limits_{p^2 = m_{\bar{\lambda}}^2} \frac{\dn{3}{p}}{(2 \pi)^3} \frac{1}{2 p_0} \bra{\Omega} \phd_j(y) \ket{\bar{\lambda}_{p,s}} \bra{\bar{\lambda}_{p,s}} \ph_i(x) \ket{\Omega}
\\
& = \sum_{\bar{\lambda}, s} \int\limits_{p^2 =m_{\bar{\lambda}}^2} \frac{\dn{3}{p}}{(2 \pi)^3} \frac{e^{-i p \cdot (y - x)}}{2 p_0} \bra{\Omega} S^*_{jk}(B_p^{-1}) \phd_k(0) \ket{\bar{\lambda}_{s}} \bra{\bar{\lambda}_{s}} S_{i\ell}(B_p^{-1}) \ph_\ell(0) \ket{\Omega}
\\
& = \sum_{\bar{\lambda}} \int\limits_{p^2 = m_{\bar{\lambda}}^2} \frac{\dn{3}{p}}{(2 \pi)^3} \frac{e^{-i p \cdot (y - x)}}{2 p_0} Z_{\bar{\lambda}} \: (S(B_p^{-1}) S^\dagger(B_p^{-1}))_{ij}
\end{align}
\end{subequations}

\subsection{The Representation Function}

Let $H_{m_\lambda}$ denote the hyperbola $\{ p^2 = m_\lambda^4 \} \subset \R^{1,3}$. Then we define a matrix of functions $f_{ij} : H_{m_\lambda} \to \C$ via,
\begin{equation}
f_{ij}(p) = (S(B_p) S^\dagger(B_p))_{ij}
\end{equation}
I claim that $W = \mathrm{Span}(f_{ij}) \subset \mathcal{C}^\infty(H_{m_\lambda})$ is a finite-dimensional $\Spin{1,3}$-representation. Consider,
\begin{equation}
f(\Lambda p) = S(B_{\Lambda p}) S^\dagger(B_{\Lambda p})
\end{equation}
and the Lorentz transformation $T = B_p^{-1} \Lambda^{-1} B_{\Lambda p}$ which acts on the time Killing vector as follows,
\begin{equation}
B_p^{-1} \Lambda^{-1} B_{\Lambda p} m_\lambda e_t 
= B_p^{-1} \Lambda^{-1} (\Lambda p) = B_p^{-1} p = m_\lambda e_t
\end{equation}
so $T$ preserves the time direction. Therefore, $T$ must be a rotation (see Lemma \ref{preserves_time_rotation}) so $T = R_{\Lambda}$. Therefore,
\begin{equation}
B_{\Lambda p} = \Lambda B_p R_{\Lambda} 
\end{equation}
This implies that,
\begin{subequations}
\begin{align}
f(\Lambda p) & = S(\Lambda) S(B_p) S(R_{\Lambda}) S^\dagger(R_{\Lambda}) S^\dagger(B_p) S^\dagger(\Lambda) 
\\
& = S(\Lambda) S(B_p) S^\dagger(B_p) S^\dagger(\Lambda) =  S(\Lambda) \cdot f(p) \cdot S^\dagger(\Lambda) 
\end{align}
\end{subequations}
Where I have used the fact that $\SO{3} \subset \SO{1,3}$ is compact and thus we may choose our representation restricted to this compact subgroup $S|_{\SO{3}}$ to be unitary. We have therefore, shown that $V$ transforms in the $W \otimes \overline{W}$ representation which is finite and decompses into irreducible $\SO{1,3}$-reps as,
\begin{equation}
W \otimes \overline{W} = (\mu, \nu) \otimes (\nu, \mu) = \sum_{| \mu - \nu | \le j_{+}, j_{+} \le \mu + \nu} (j_{+}, j_{-})
\end{equation} 
However, $V$ is a representation consisting of Lorentz covariant functions and therefore by general theory (CITE THIS) $V$ decomposes as a sum of irreducible harmonic functions on the hyperbola in representations $(J, J)$ for $J \in \{ 0, \tfrac{1}{2}, 1, \tfrac{3}{2}, \cdots \}$ so we in fact have only the following direct summands in the decompositon of $V$,
\begin{equation}
V \hookrightarrow \bigoplus_{\mu - \nu \le J \le \mu + \nu} (J, J) 
\end{equation}
This implies the following harmonic decomposition,
\newcommand{\Y}{\mathcal{Y}_{m_{+},m_{-}}}
\begin{equation}
f_{ij}(p^\mu) = \sum_{J = |\mu - \nu|}^{\mu + \nu} f_J \sum_{- J \le m_+, m_- \le J} { i \quad j \choose J \: m_{+} \: m_{-}} \:  m_{\lambda}^{2 J} \Y^J(p^\mu / m_{\lambda}) 
\end{equation}
where these symbols are the generalizations of the Clebsch-Gordon decomposition coefficients. The harmonics $\Y^J(x)$ are homogeneous polynomials of degree $2J$ (CITE THIS FACT). Therefore, we may analytically continue $p$ off-shell i.e. over $H_{m_\lambda} \subset \R^{1,3} \subset \C^4$ to both arbitrary real and complex momenta by the definition,
\begin{equation}
f_{ij}(p) = \sum_{J = |\mu - \nu|}^{\mu + \nu} f_J \sum_{- J \le m_+, m_- \le J} { i \quad j \choose J \: m_{+} \: m_{-}} \: \Y^J(p) 
\end{equation} 
Now, since $\Y^J$ has degree $2J$ we have $\Y^J(-p) = (-1)^{2J}(p)$ and furhtermore, since $|\mu - \nu| \le J \le \mu + \nu$ indexes by one, each possible $2 J$ has the same parity as $2 (\mu + \nu) = 2s$ where $s$ is the spin of the representation $W$. This follows from the fact that,
\begin{equation}
\hat{J}_{+} + \hat{J}_{-} = \hat{J} 
\end{equation}
so when $W$ restricts to $\SO{3} \hookrightarrow \SO{1,3}$ it decomposes into irreps with spin $|\mu - \nu| \le s \le \mu + \nu$. Therefore, we again have that $(-1)^{2J} = (-1)^{2 s} = (-1)^{2(\mu + \nu)}$ where $2 s$ gives the parity for the possible spin states of the particle. Applying this discussion, 
\begin{subequations}
\begin{align}
f_{ij}(-p) & = \sum_{J = |\mu - \nu|}^{\mu + \nu} f_J \sum_{- J \le m_+, m_- \le J} { i \quad j \choose J \: m_{+} \: m_{-}} \: \Y^J(-p) 
\\
& = \sum_{J = |\mu - \nu|}^{\mu + \nu} f_J \sum_{- J \le m_+, m_- \le J} { i \quad j \choose J \: m_{+} \: m_{-}} \: (-1)^{2 J} \: \Y^J(p)
\\
& = (-1)^{2s} f_{ij}(p)
\end{align}
\end{subequations}

\subsection{Completing the Proof}

The derived relations hold for $f_{ij}(p)$ off-shell so we need to modify the two-point correlation functions to take $p^\mu$ off-shell. Using the polynomial nature of the funtions $f_{ij}(p)$,
\begin{subequations}
\begin{align}
\bra{\Omega} \ph_i(x) \phd_j(y) \ket{\Omega} 
& = \sum_{\lambda} Z_\lambda \: f_{ij}(+ i \partial_x)  D_\lambda(x - y)
\\
\bra{\Omega} \phd_j(y) \ph_i(x) \ket{\Omega} 
& = \sum_{\bar{\lambda}} Z_{\bar{\lambda}} \: f_{ij}(-i \partial_x)  D_{\bar{\lambda}}(y - x)
\end{align}
\end{subequations}
where,
\begin{equation}
D_\lambda(x - y) = \int\limits_{p^2 = m_\lambda^2} \frac{\dn{3}{p}}{(2 \pi)^3} \frac{e^{-i p \cdot (x - y)}}{2 p_0} 
\end{equation}
When $x$ and $y$ are spacelike seperated i.e. $(x - y)^2 < 0$ then $D(x - y) = D(y - x)$ (CITE PESKIN 28). In this case, applying our result of the previous section,
\begin{subequations}
\begin{align}
\bra{\Omega} \phd_j(y) \ph_i(x) \ket{\Omega} 
& = \sum_{\lambda} Z_\lambda \: f_{ij}(- i \partial_x)  D_\lambda(x - y)
\\
& = (-1)^{2s} \sum_{\bar{\lambda}} Z_{\bar{\lambda}} \: f_{ij}(+i \partial_x)  D_{\bar{\lambda}}(x - y)
\end{align}
\end{subequations}
From relativistic causality arguments we know that these field operators must either commute or anti-commute. Since $Z_\lambda \ge 0$ the value of $(-1)^{2s}$ fixes the sign of this relation proving our theorem. In fact, we see slightly more. In order that these correlation functions actually commute or anti-commute, there must be a spectral symmetry $\lambda \leftrightarrow \bar{\lambda}$ reversing charge but preserving mass and spin. Thus, the spin-statistics relationship also establishes the existence and spectral symmetry of anti-particles.


\section{Lemmas}

\begin{lemma}
If a Lorentz transformation $\Lambda \in \SO{1,3}$ preseves the temporal Killing vector $e_t$ then $\Lambda$ is a rotation. 
\end{lemma}

\begin{proof}
We can write any four vector in the form $x^\mu = x^0 e_t + \vec{x}$. Then, $\Lambda x^\mu = x^0 e_t + \Lambda \vec{x}$. Furthermore $e_t \cdot \vec{x} = 0$ and $\Lambda$ preserves the Minkowski inner product so $e_t \cdot \Lambda \vec{x}$ and thus $\vec{x}$ is also a vector. Furthermore, $\Lambda$ preserves $(x^0)^2 - \vec{x}^2$ so $\Lambda \vec{x}$ must be a vector with $|\Lambda \vec{x}|^2 = |\vec{x}|^2$ so $\Lambda$ must be a rotation since $\det{\Lambda} = 1$. 
\end{proof}

\end{document}


