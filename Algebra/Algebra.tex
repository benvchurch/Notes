\documentclass[12pt]{extarticle}
\usepackage[utf8]{inputenc}
\usepackage[english]{babel}
\usepackage[a4paper, total={6in, 9in}]{geometry}
 
\usepackage{amsthm, amssymb, amsmath, centernot}

\newcommand{\notimplies}{%
  \mathrel{{\ooalign{\hidewidth$\not\phantom{=}$\hidewidth\cr$\implies$}}}}
 
\renewcommand\qedsymbol{$\square$}
\newcommand{\cont}{$\boxtimes$}
\newcommand{\divides}{\mid}
\newcommand{\ndivides}{\centernot \mid}
\newcommand{\Z}{\mathbb{Z}}
\newcommand{\N}{\mathbb{N}}
\newcommand{\Zplus}{\mathbb{Z}^{+}}
\newcommand{\Primes}{\mathbb{P}}

\theoremstyle{definition}
\newtheorem{theorem}{Theorem}[section]
\newtheorem{lemma}[theorem]{Lemma}
\newtheorem{proposition}[theorem]{Proposition}
\newtheorem{corollary}[theorem]{Corollary}


\newenvironment{definition}[1][Definition:]{\begin{trivlist}
\item[\hskip \labelsep {\bfseries #1}]}{\end{trivlist}}


\newenvironment{lproof}{\begin{proof} \renewcommand{\qedsymbol}{}}{\end{proof}}
\renewcommand{\mod}[3]{\: #1 \equiv #2 \: mod \: #3 \:}
\newcommand{\nmod}[3]{\: #1 \centernot \equiv #2 \: mod \: #3 \:}
\newcommand{\ndiv}{\hspace{-4pt}\not \divides \hspace{2pt}}
\newcommand{\gen}[1]{\langle #1 \rangle}

\begin{document}

\author{Benjamin Church}
\title{\Huge Introduction to Abstract Algebra}
\date{}

\maketitle
\tableofcontents
\newpage

\section{Introduction to Algebraic Structures}
Abstract algebra is the study of algebraic structures and mappings between structures. An algebraic structure is an underlying set along with one or more (usually binary) operations that satisfy certain axioms. The most fundamental axiom is that of closure: operations over the underlying set have a range contained within the underlying set. Abstract algebra is the study of fundamental questions about algebraic structures: which structures satisfy certain axioms, how can structures be classified, what kind of substructures exist, and what relationships, mappings, and equivalences can be drawn between structures. Furthermore, there is a rich study of representing algebraic structures and using algebra to determine which properties are necessary and sufficient for properties of a given structure to hold. Algebraic structures are essential for rigours generalization in many fields of mathematics especially number theory, combinatorics, symmetry, construction, and the study of solutions to polynomials.  

\section{Functions and Relations}

\section{Group Axioms}

\subsection{Definition of a Group}


\begin{definition}
A group is an ordered pair $(G, \circ)$ where $G$ is a set equipped with a binary operation or group law $\circ : G \times G \rightarrow G$ where $\circ(x,y)$ is denoted as $x \circ y$ satisfying:
\begin{enumerate}
\item \textbf{Closure:} \\ $\forall x,y \in G$ : $x \circ y \in G$ \: i.e. the image of $\circ$ is contained within $G$
\item \textbf{Associativity:} \\ $\forall x,y,z \in G$ : $x \circ (y \circ z) = (x \circ y) \circ z$
\item \textbf{Identity Element:} \\ $\exists e \in G$ s.t. $\forall x \in G$ : $e \circ x = x \circ e = x$
\item \textbf{Inverse Elements:} \\ $\forall x \in G$ : $\exists x^{-1} \in G$ s.t. $x \circ x^{-1} = x^{-1} \circ x = e$
\end{enumerate}
\end{definition}

\subsection{Basic Group Properties}

\begin{lemma}
The identity element is unique.
\end{lemma}

\begin{lproof}
Suppose that both $e$ and $e' \in G$ satisfy the identity property. \\ Because $\forall x \in G$ : $e \circ x = x$ we have $e \circ e' = e'$. Likewise, because $\forall x \in G$ : $x \circ e' = x$ we have $e \circ e' = e$. Thus, $e = e'$
\end{lproof}

\begin{lemma}
Each inverse element is unique.
\end{lemma}

\begin{lproof}
Consider $x \in G$ and let both $a$ and $b$ satisfy the inverse element property. Thus, $x \circ b = e$ so $a \circ (x \circ b) = a \circ e$ by associativity and the identity property, $(a \circ x) \circ b = a$ However, $a \circ x = e$ so $(a \circ x) \circ b = b$ thus $a = b$. 
\end{lproof}

\begin{lemma}
Let $x,y \in G$ then $(x^{-1})^{-1} = x$ and $(x \circ y)^{-1} = y^{-1} \circ x^{-1}$
\end{lemma}

\begin{lproof}
Since by definition, $x \circ x^{-1} = x^{-1} \circ x = e$ then $x$ satisfies the inverse property for $x^{-1}$ and because inverses are unique, $(x^{-1})^{-1} = x$. \\\\ $(y^{-1} \circ x^{-1}) \circ (x \circ y) = y^{-1} \circ (x^{-1} \circ x) \circ y = y^{-1} \circ y = e$. Likewise, $(x \circ y) \circ (y^{-1} \circ x^{-1}) = e$  so by the uniqueness of inverses, $(x \circ y)^{-1} = y^{-1} \circ x^{-1}$
\end{lproof}

\begin{lemma}[The Cancellation Law]
If $a \circ b = a \circ c$ or $b \circ a = c \circ a$ then $b = c$
\end{lemma}

\begin{lproof}
Let $a \circ b = a \circ c$ then since there exist inverses, $a^{-1} \circ (a \circ b) = a^{-1} \circ (a \circ c)$ therefore $(a^{-1} \circ a) \circ b = (a^{-1} \circ a) \circ c$ thus $b = c$. Similarly for right cancellation.
\end{lproof}

\subsection{Sufficient Conditions for Group Axioms}

\subsection{Basic Properties of Finite Groups}

\section{Subgroups and Cosets}
\begin{definition}
A subset $H$ of a group $(G, \circ)$ is a subgroup of $G$ denoted as $H < G$ if $(H, \circ)$ is a group with the same operation.
\end{definition}

\begin{proposition}
If $e \in H$ and $\forall x,y \in H : x \circ y \in H$ and $x^{-1} \in H$ then $H < G$.
\end{proposition}

\begin{proof}
Because $H \subset G$, $\forall x,y,z \in H : x,y,z \in G$ therefore, $(x\circ y)\circ z = x \circ (y \circ z)$. And $\forall x \in H : x \in G$ so $e \circ x = x \circ e = x$ thus $e$ is the identity of $H$. Closure and the existence of inverses are given by hypothesis. 
\end{proof}

Standard group multiplication will from now on be denoted by juxtaposition. 

\begin{proposition}
If $H, K < G$ then $H \cap K < G$
\end{proposition}

\begin{lproof}
If $x,y \in H \cap K$ then $x, y \in H$ and $x, y \in K$ so by closure, $xy \in H$ and $xy \in K$ so $xy \in H \cap K$. Also, $x^{-1} \in H$ and $x^{-1} \in K$ so $x^{-1} \in H \cap K$. Finally, $e \in H$ and $e \in K$ so $e \in H \cap K$.
\end{lproof}

\begin{definition}
Given $S \subset G$ then $\gen{S} = \{s_1 \cdots s_n \mid s_i \in S \cup S^{-1} \} \cup \{e\}$ is the subgroup generated by $S$ where $S^{-1} = \{s^{-1} \mid s \in S\}$.
\end{definition}

\begin{proposition}
Take $S \subset G$ then $\gen{S} < G$.
\end{proposition}

\begin{lproof}
By definition, $e \in \gen{S}$. If $x, y \in \gen{S}$ then $x = s_1 \cdots s_n$ and $y = s_1' \cdots s_k'$ so $xy = s_1 \cdots s_n s_1' \cdots s_k' \in \gen{S}$. Also, $x^{-1} = s_n^{-1} \cdots s_1^{-1} \in \gen{S}$ because if $s \in S \cup S^{-1}$ then $s^{-1} \in S \cup S^{-1}$.
\end{lproof}

\begin{proposition}
For any $S \subset G$, $\gen{S}  = \bigcap \{H \mid H < G \text{ and } S \subset H\}$
\end{proposition}

\begin{lproof}
If $S \subset H$ then $\forall s_i \in S \cup S^{-1} : s_i \in H$ because if $s \in S^{-1}$ then $s^{-1} \in S \subset H$ so $s \in H$. By closure, $s_1 \cdots s_n \in H$ so $\gen{S} \subset H$ thus $\gen{S} \subset \bigcap \{H \mid H < G \text{ and } S \subset H\}$. \\ \\ However, $\gen{S} < G$ with $S \subset \gen{S}$ so $\bigcap \{H \mid H < G \text{ and } S \subset H\} \subset \gen{S}$. \\ Therefore, $\gen{S}  = \bigcap \{H \mid H < G \text{ and } S \subset H\}$.
\end{lproof}

\begin{definition}
For $H < G$ and some $g \in G$ the left coset of $H$ is $gH = \{g h \mid h \in H \}$.
\end{definition}

\begin{definition}
For $H < G$ and some $g \in G$ the right coset of $H$ is $Hg = \{hg \mid h \in H \}$.
\end{definition}

\begin{definition}
Let $H < G$ then $G / H = \{gH \mid g \in G\}$ and $[G : H] = \left| G/H \right|$.
\end{definition}

\begin{proposition}
Given $H < G$, the relation given by $x ~_H y \iff x^{-1}y \in H$ is an equivalence relation and $[x]_H = xH$.
\end{proposition}

\begin{proposition}
$xH = yH \iff x^{-1}y \in H$ and $Hx = Hy \iff xy^{-1} \in H$. 
\end{proposition}

\begin{lproof}
If $xH = yH$ then $\exists h_1, h_2 \in H : x h_1 = y h_2$ so 
\end{lproof}

\begin{proposition}
$[G : G] = 1$ in partiuclar, $\forall g \in G : gG = Gg = G$.
\end{proposition}

\begin{lproof}
By closure, $gG \subset G$ and $Gg \subset G$. For any $x \in G$, by closure and inverses, $g^{-1}x, xg^{-1} \in G$ thus, $g(g^{-1}x) = x \in gG$ and $(x g^{-1})g = x \in Gg$ so $G \subset gG$ and $G \subset Gg$. Therefore, $gG = Gg = G$. 
\end{lproof}

\begin{proposition}
$[G : H] = \left| \{ Hg \mid g \in G \} \right|$ i.e. there are the same number of left and right cosets. 
\end{proposition}

\begin{lproof}
Let $f : G/H \rightarrow \{ Hg \mid g \in G \}$ and take $f : gH \mapsto Hg^{-1}$.
\end{lproof}

\begin{theorem}[Lagrange]
Let $G$ be a finite group and $H < G$ then $|G| = [G:H] \cdot |H|$.
\end{theorem}

\begin{proof}

\end{proof}


\section{Cyclic Groups and Group Order}

\section{Normal Subgroups and Quotient Groups}

\section{Homomorphisms and Isomorphisms}

\section{Group Products}

\section{The Isomorphism Theorems}

\section{Group Actions}

\section{The Sylow Theorems}

\section{Simple Groups}

\section{Normal Series}

\section{Basic Ring Theory}

\section{Subrings and Ideals}

\section{Multiplicative Groups and the Chinese Remainder Theorem}

\section{Domains}

\section{Unique Factorization}

\section{Fields}

\section{Polynomial Rings}

\section{Field Extensions}

\section{Galois Theory}

\section{Modules}

\section{Noetherian Rings}

\section{P-Adic Rings and Fields}




\end{document}
