\documentclass[12pt]{extarticle}
\usepackage{import}
\import{./}{Includes}


\begin{document}
\atitle{6}

\section*{Page 163.} 
\subsection*{Problem 3.}

$\sqrt{2} + \sqrt{3} \in \Q(\sqrt{2}, \sqrt{3})$ which is generated by a finite number of algebraic elements (since $\sqrt{2}$ and $\sqrt{3}$ solve $X^2 - 2$ and $X^2 - 3$ respectively) so it is an algebraic extension. Therefore, every element of $\Q(\sqrt{2}, \sqrt{3})$ is algebraic including $\sqrt{2} + \sqrt{3}$.  \\\\
Consider the polynomial,
\begin{align*}
(X - (\sqrt{2} + \sqrt{3})) & (X + (\sqrt{2} + \sqrt{3}))(X - (\sqrt{2} - \sqrt{3}))(X + (\sqrt{2} - \sqrt{3})) \\  & = (X^2 - (5 + 2 \sqrt{6}))(X^2 - (5 - 2 \sqrt{6})) = ([X^2 - 5] - 2 \sqrt{6}))([X^2 - 5] + 2 \sqrt{6})) \\ & = [X^2 - 5]^2 - 4 \cdot 6 = X^4 - 10 X^2 + 1
\end{align*}
Clearly, $\sqrt{2} + \sqrt{3}$ is a root of $X^4 - 10 X^2 + 1$ and this must be the minimal polynomial because it has degree $4$ which is the degree of $[\Q(\sqrt{2} + \sqrt{3}) : \Q] = [\Q(\sqrt{2} + \sqrt{3}) : \Q(\sqrt{2})][\Q(\sqrt{2}) : \Q] = 2 \cdot 2$

\subsection*{Problem 5.}
Let $E = \Q(\sqrt[6]{2})$ then because $\sqrt{2} = (\sqrt[6]{2})^3 \in \Q(\sqrt[6]{2})$ we have $\Q(\sqrt{2}) \subset E$. However, $\sqrt{2} \notin \Q$ and $\sqrt[6]{2}$ is not of degree $2$ so $\sqrt[6]{2} \notin \Q(\sqrt{2})$. Thus, $\Q \subsetneq \Q(\sqrt{2}) \subsetneq E$. 

\section*{Page 165.} 

\subsection*{Problem 3.}
Let $K \subset E \subset F$ be fields with $F$ algebraic over $E$ and $E$ algebraic over $K$. Let $\alpha \in F$ then $\alpha$ satisfies some $f \in E[X]$ with $f(X) = a_0 + a_1 X + \cdots + a_n X^n$ where $a_i \in E$. Thus, $f \in K(a_0, \dots, a_n)[X]$ so $\alpha$ is algebraic over $K(a_0, \dots, a_n)$ and therefore, $K(a_0, \dots, a_n)(\alpha)$ is a finite extension of $K(a_0, \dots, a_n)$. Finally, \[[K(a_0, \dots, a_n)(\alpha) : K] = [K(a_0, \dots, a_n)(\alpha) : K(a_0, \dots, a_n)][K(a_0, \dots, a_n) : K]\] and the two factors on the right hand side are finite. Therefore, $[K(a_0, \dots, a_n)(\alpha) : K]$ is finite so the fact that \[[K(a_0, \dots, a_n)(\alpha) : K] = [K(a_0, \dots, a_n)(\alpha) : K(\alpha)][K(\alpha) : K]\] gives that $[K(\alpha) : K]$ is finite so $\alpha$ is algebraic over $K$. Thus, $F$ is an algebraic extension of $K$.     

\section*{Additional Problem 1.}

Let $E / K$ be a field extension and $\alpha \in E$ be algebraic over $K$. Let $q \in K[X]$ be the minimal polynomial of $\alpha$. Suppose that $f \in K[X]$ is a monic polynomial with degree equal to the degree of $q$ such that $f(\alpha) = 0$. Now, let $ev_\alpha : K[X] \to K$ be the homomorphism given by $ev_\alpha(f) = f(\alpha)$. By definition, $\ker{ev_\alpha} = (q)$ and $f \in \ker{ev_\alpha}$. Therefore, $f = kq$ so $\deg{f} = \deg{k} + \deg{q}$ and thus, $\deg{k} = 0$ because $\deg{f} = \deg{q}$. Now, $k \in K$ but both polynomials are monic so $k = 1$. Thus, $f = q$. \\\\
Alternatively, $(f - q)(\alpha) = 0$ but $f$ and $q$ are monic of equal degree so $\deg{(f - q)} < \deg{q}$. However, $q$ is the minimal polynomial so we must have $f - q = 0$ and thus $f = q$.

\section*{Additional Problem 2.} 
Let $E / K$ be a field extension and $\alpha \in E$ be algebraic over $K$. Let $q \in K[X]$ be the minimal polynomial of $\alpha$ with $d = \deg{q}$. We introduce the homomorphism $ev_\alpha : K[X] \to K(\alpha)$ given by $ev_\alpha(f) = f(\alpha)$. Now, $\ker{ev_\alpha} = (q)$ and $q$ is irreducible so $(q)$ is a maximal ideal. Since $(q)$ is maximal, $K[X]/(q)$ is a field. Also, $K[X] / (q) \cong \Im{ev_\alpha} \subset K(\alpha)$. However, by the isomorphism, $\Im{ev_\alpha}$ is a field containing $\alpha$ and $K$ contained in $K(\alpha)$ so $\Im{ev_\alpha} = K(\alpha)$ by minimality. The map $ev_\alpha$ factors through $K[X]/(q)$ by $ev_\alpha = f \circ \pi$ with unique isomorphism $f$. Since $f$ is a surjection, given any element $k \in K(\alpha)$ we can write $f(p + (q)) = k$ for some $p \in K[X]$. Now write $p = qs + r$ with $s, r \in K[X]$ and $r = 0$ or $\deg{r} < \deg{q} = d$. Therefore, we can write \[r(X) = a_0 + a_1 X + \dots + a_l X^l\]
with $a_i \in K$ and $l < d$. Now, $p + (q) = qs + r + (p) = r + (p) = \pi(r)$. 
Thus, we have, \[k = f \circ \pi(r) = ev_\alpha(r) = a_0 + a_1 \alpha + \dots + a_{l} \alpha^{l}\] 
thus $k \in \vspan{1, \alpha, \alpha^2, \dots, \alpha^{d-1}}$ so the set $\{1, \alpha, \alpha^2, \dots, \alpha^{d-1} \}$ spans all of $K(\alpha)$. Also, suppose that for some constants $a_i \in K$ we have,
\[a_0 + a_1 \alpha + a_2 \alpha^2 + \dots + a_{d-1} \alpha^{d-1} = 0\]
Then, the polynomial $p \in K[X]$ given by $p(X) = a_0 + a_1 X + \dots a_{d-1} X^{d-1}$ has $\alpha$ as a root. However, $\deg{p} = d - 1 < d = \deg{q}$ contradicting the minimality of $q$ unless $p = 0$. Therefore, each $a_i = 0$ so the set $\{1, \alpha, \alpha^2, \dots, \alpha^{d-1}\}$ is linearly independent and thus a basis.
   

\end{document}