\documentclass[12pt]{extarticle}
\usepackage{import}
\import{./}{Includes}


\begin{document}
\section{Splitting Fields}
\begin{lemma}
If $\alpha \in E$ is transecendental over $K$ then $K(\alpha) \cong K(X) = Q_{K[X]}$
\end{lemma}
\begin{proof}
The map $ev_\alpha : K[X] \mapsto K(\alpha)$ is injective because $\alpha$ is transecendental over $K$. Thus, because $K(\alpha)$ is a field and $K[X]$ is a domain, $ev_\alpha$ factors into $ev_\alpha = \iota \circ f$ where $\iota : K[X] \to Q_{K[X]}$ is given by $\iota : f \mapsto (f,1)$ and $f$ is an injective homomorphism. 
\begin{center}
\begin{tikzcd}
K[X] \arrow[r, hook, "\iota"] \arrow[rd, hook, "ev_\alpha"] & K(X) \arrow[d, dashed, "f"] \\
& K(\alpha) 
\end{tikzcd}
\end{center}

Then, $f$ is an embedding of $K(X)$ into $K(\alpha)$ but $K(\alpha)$ is the subfield of $E$ generated by $K[\alpha] = \Im{ev_\alpha}$ so $K(X) = K(X)$ 
\end{proof}

\begin{theorem}[Embedding]
Let $K$ be a field and $E, E'$ be extensions with $\alpha \in E$ and $\alpha' \in E'$ with both $\alpha$ and $\alpha'$ algebraic over $K$. Suppose that $\minimal{\alpha}{K} = \minimal{\alpha'}{K}$ then there exists a $K$-preserving isomorphism $\phi : K(\alpha) \to K(\alpha')$ such that $\phi : \alpha \mapsto \alpha'$. 
\end{theorem}

\begin{proof}
There exist $K$-preserving isomorphisms $\beta : K[X]/(\minimal{\alpha}{K}) \to K(\alpha)$ and $\gamma : K[X]/(\minimal{\alpha'}{K}) \to K(\alpha')$ but $\minimal{\alpha}{K} = \minimal{\alpha'}{K}$ so let $\phi = \gamma \circ \beta^{-1}$ which is a $K$-preserving isomorphism. Now, 
\begin{center}
\begin{tikzcd}[column sep=small]
& K[X]/(\minimal{\alpha}{K}) \arrow[dl, "\beta"] \arrow[dr, "\gamma"] & \\
K(\alpha) \arrow[rr, "\phi"] & & K(\alpha')
\end{tikzcd}
\end{center}
Finally, $\beta(\bar{X}) = ev_\alpha(X) = \alpha$ and $\gamma(\bar{X}) = \alpha'$ so $\phi \circ \beta(\bar{X}) = \gamma(\bar{X})$ thus $\phi(\alpha) = \alpha'$.  
\end{proof}

\begin{corollary}
If $E/K$ and $p \in K[X]$ is irreducible over $K$ with roots $\alpha_1, \alpha_2 \in E$ then there exists a $K$-isomorphism from $K(\alpha)$ to $K(\alpha')$ which takes $\alpha$ to $\alpha'$.  
\end{corollary}

\begin{corollary}
If $\alpha_1, \alpha_2 \in E$ are algebraic over $K$ with equal minimal polynomials and $E = K(\alpha_1) = K(\alpha_2)$ but $\alpha_1 \neq \alpha_2$ then there exists an automorphism of $E$ which preserves $K$ and sends $\alpha_1$ to $\alpha_2$.  
\end{corollary}

\begin{corollary}
If $\alpha \in E$ is algebraic over $K$ and $E'$ contains a root of $\minimal{\alpha}{K}$ then there exists a field embedding $\phi : K(\alpha) \to E'$. 
\end{corollary}

\begin{definition}
$E/K$ is an algebraic extension if $\forall \alpha \in E : \alpha$ is algebraic over $K$.
\end{definition} 

\begin{proposition}
If $[E : K]$ is finite then $E/K$ is algebraic.
\end{proposition}

\begin{proof}
If $[E : K]$ is finite then for any $\alpha \in E$, the identity, 
\[[E : K] = [E : K(\alpha)][K(\alpha) : K]\] gives that $[K(\alpha) : K]$ is finite so $\alpha$ is algebraic over $K$.
\end{proof}

\begin{proposition}
$[E : K]$ is finite if and only if $\exists \alpha_1, \dots \alpha_n \in E$ s.t. $E = K(\alpha_1, \dots, \alpha_n)$ 
\end{proposition}

\begin{proof}

\end{proof}
 
\begin{definition}
Let $K$ be a field and $f \in K[X]$ with $L/K$ a field extension, $L$ is the splitting field of $f$ if,
\begin{enumerate}
\item $f \in L[X]$ is split into linear factors, i.e. $f(X) = a (X - \alpha_1) \cdots (X - \alpha_n)$ with $a \in K$ and $\alpha_i \in L$.
\item $L = K(\alpha_1, \dots, \alpha_n)$
\end{enumerate}
\end{definition}

\begin{lemma}
Let $p \in K[X]$ be irreducible, then there exists a field extension $L/ K$ such that $\tilde{p} \in L[X]$ has a root in $L$.
\end{lemma}
\begin{proof}
Define $L = K[X]/(p)$ which is a field because $(p)$ is a maximal ideal since $p$ is irreducible in a PID. Now, $\iota : K \to K/(p)$ given by $\iota : r \mapsto r + (p)$ is a field homomorphism and thus an injection. Thus, $K \cong \Im{\iota}$ so we have an embedding of $K$ in $K/(p)$. We extend $\iota : K[X] \to L[X]$ by acting on coefficients and $\tilde{p} = \iota(p)$ Consider the map $\pi : K[X] \to K[X]/(p)$ given by $\pi : a \mapsto x + (p)$ is a homomorphism so $\tilde{p}(\pi(X)) = \pi(p(X)) = p(X) + (p) = (p)$ so $\pi(X)$ is a root of $\tilde{p}$ in $L$.   
\end{proof}

\begin{corollary}
For any $p \in K[X]$, there exists a field extension $L/ K$ such that $\tilde{p} \in L[X]$ has a root in $L$.
\end{corollary}
\begin{proof}
If $p$ is irreducible, we are done. Otherwise, because $K[X]$ is a UFD, take some irreducible $g \divides p$ with $\deg{g} > 0$ then there exists a field extension in which $g$ has a root and therefore, $p$ has a root. 
\end{proof}

\begin{theorem}
For any nonconstant $f \in K[X]$ there exists a splitting field of $f$.
\end{theorem}

\begin{proof}
Let $\deg{f} = n$. Construct $K_1 \supset K$ with $\alpha \in K_1$ s.t. $f(\alpha_1) = 0$ so in $K_1[X]$ we have $f(X) = (X - \alpha_1) g_1(X)$ with $g_1(X) \in K_1[X]$ and $\deg{g_n} = n-1$. By Induction, we get a chain $K_n \supset K_{n-1} \supset \dots \supset K$ such that \[f(X) = (X - \alpha_1)(X - \alpha_2) \cdots (X - \alpha_n) g_n(X)\] with $\deg{g} = 0$ so $g(X) = c \in K_n[X]$. Then, take $L = K(\alpha_1, \dots, \alpha_n) \subset K_n$. 
\end{proof}

\begin{theorem}
For nonconstant $f \in K[X]$, if $L_1$ and $L_2$ are both splitting fields of $f$ over $K$ then $L_1$ and $L_2$ are $K$-isomorphic.
\end{theorem}

\begin{proof}
Induction on $\deg{f} = n$. For $n = 1$, $f(X) = x - r$ with $r \in K$ so $L_1 = L_2 = K$. Suppose the theorem holds for $\deg{g} < n$. Then, if $f$ is reducible, $f(X) = f_1(X) f_2(X)$ with strictly smaller degrees. 
\end{proof}

\begin{definition}
An algebraic extension $E/K$ is normal if for all $\alpha \in E$, the minimal polynomial splits completely in $E$.
\end{definition} 

\begin{lemma}
If $E/K$ is normal and $K \subset L \subset E$ then $E/L$ is normal.
\end{lemma}

\begin{proof}
Take $\alpha \in E$ with minimal polynomial over $K$ given by $\minimal{\alpha}{K}$. Then, $\minimal{\alpha}{K} \in L[X]$ and has $\alpha$ as a root. Thus, $\minimal{\alpha}{L} \divides \minimal{\alpha}{K}$ but $\minimal{\alpha}{K}$ splits in $E$ so $\minimal{\alpha}{L}$ splits in $E$. 
\end{proof}

\begin{theorem}
Let $E/K$ be a finite extension then the following are equivalent:
\begin{enumerate}
\item $E/K$ is normal
\item $E$ is the splitting field of some $f \in K[X]$
\item If $K \subset K_1 \subset E \subset E_1$ and $\sigma : K_1 \to E_1$ is a $K$-homomorphisms then $\Im{\sigma} \subset E$
\end{enumerate}
\end{theorem}

\begin{proof}
Since $E/K$ is a finite extension, $E = K(\alpha_1, \dots, \alpha_n)$ for $\alpha_i \in K$. Then take $f = \minimal{\alpha_1}{K} \dots \minimal{\alpha_n}{K}$. By normality, $f$ must split into linear factors in $E$. Furthermore, $E = K(\alpha_1, \dots, \alpha_n) \subset K(r_1, \dots r_s)$ where there are the roots of $f$. However, $E$ contains every root by normality so $E = K(r_1, \dots, r_s)$. Thus, $E$ is the splitting field of $f$. \\\\
Take some $f$ with $E$ the splitting field of $f$. Take $f(X) = a(X - \alpha_1) \dots (X - \alpha_n)$ with $\alpha \in E$ and $E = K(\alpha_1, \dots, \alpha_n)$. Take any $K \subset K_1 \subset E \subset E_1$ and $\sigma : K_1 \to E_1$. Define $K_1' = \sigma(K_1) \subset E_1$ and $E' = K_1'(\alpha_1, \dots, \alpha_n)$.

\begin{center}
\begin{tikzcd}[column sep=small]
& E_1 & \\
E \arrow[ur, "\iota"] \arrow[rr, dashed, "\tilde{\sigma}"] & & E' \arrow[ul, "\iota"] \\
K_1 \arrow[u, "\iota"] \arrow[rr, "\sigma"] \arrow[rru, dotted, ""] &  & K_1' \arrow[u, "\iota"]\\
& \arrow[ul, "\iota"] K \arrow[ur, "\iota"] &
\end{tikzcd}
\end{center}
$K$ is embedded in $E$ and and $E'$  $\iota \circ \sigma$ and both fields contain every root of $f$ so they 
$\sigma$ takes $f$ to $f$ and $\tilde{\sigma} : E \to E'$ extends $\sigma$ because both are the splitting field of $f$ over $K$. For each $\sigma : \alpha_i \to \alpha_j$ because $\tilde{\sigma}(f(\alpha)) = f(\tilde{\sigma}(\alpha))$ and it maps into a field containing the splitting field. Because $\tilde{\sigma}$ is a $K$-homomorphism, it fixes $K$ and also preserves the set $\{\alpha_i\}$ so $\sigma(E) = \sigma(K(\alpha_1, \dots, \alpha_n)) = K(\alpha_1, \dots, \alpha_n) = E'$. Thus, $E = E'$ so $K_1' = \Im{\sigma} \subset E' = E$. \\\\
Let $\alpha \in E$ and let $E_1$ be the splitting field of $\minimal{\alpha}{K}$. Then take $K \subset K_1 = K(\alpha) \subset E \subset E_1$. By the embedding theorem, there exits a $K$-homomorphism $\sigma : K(\alpha) \mapsto E_1$ sending $\alpha$ to any root of $\minimal{\alpha}{K}$. However by $(2)$, we have $\Im{\sigma} \subset E$ so $E$ contains every root of $\minimal{\alpha}{K}$ so $E$ is normal. 
\end{proof}

\begin{proposition}
If $[E : K] = 2$ then $E/K$ is a normal extension.
\end{proposition}

\begin{proof}
Take $\alpha \in E \setminus K$ then $\{1_K, \alpha \}$ is a basis for $E$ over $K$ because $\alpha \neq k \cdot 1_K$ because $\alpha \notin K$. Thus, $E = K(\alpha)$ so the polynomial $q = \minimal{\alpha}{K}$ has degree $2$. Since $\alpha \in E$ the minimal polynomial has one root in $E$ but $q(X) = (X - \alpha)g(X)$ and $\deg{f} = 2$ implies that $g$ is a linear factor so $q$ is split. 
\end{proof}

\begin{proposition}
let $f \in K[X]$ and $L$ be the splitting field of $f$ over $K$ then $[L : K] \le n!$.
\end{proposition}

\begin{proof}
Let $f(X) = c (X - \alpha_1) \cdots (X - \alpha_n) \in L[X]$ and $L = K(\alpha_1, \dots, \alpha_n)$. Now, let $L_i = K(\alpha_1, \dots, \alpha_i)$ such that $L_{i+1} = L_i ( \alpha_{i+1})$. Therefore, \[[L_{i+1} : L_i] = \deg{\minimal{\alpha_{i+1}}{L_i}}\]
However $L_i = K(\alpha_1, \dots, \alpha_n)$, so 
\[f(X) = (X - \alpha_1) \cdots (X - \alpha_i) g_i(X)\]
with $g_i \in L_i[X]$ but $g_i(\alpha_{i+1}) = 0$ because in $L[X]$, \[g_i(X) = \frac{f(X)}{(X - \alpha_1) \cdots (X - \alpha_n)} = c (X - \alpha_{i+1}) \cdots (X - \alpha_n)\]
Therefore, $\minimal{\alpha_{i+1}}{L_{i}} \divides g_i$ so $[L_{i+1} : L_i] \le \deg{g_i} = n - i$. Thus, \[[L : K] = [L : L_{n-1}] [L_{n-1} : L_{n-2}] \cdots  [L_{1} : L_{0}] \le 1 \cdot 2 \cdots n = n!\]
because $L = L_n$ and $K = L_0$. 
\end{proof}

\begin{proposition}
If $E$ and $K$ are finite fields then $E/K$ is a normal extension.
\end{proposition}

\begin{proof}
Let $|E| = q$ then $E$ is the splitting field of $X^q - X$ over $K$ so it is a normal extension of $K$. 
\end{proof}

\section{Seperable Extensions}

\begin{definition}
A polynomial $f \in K[X]$ is seperable if $f$ does not have multiple roots in $E$, the splitting field of $f$ over $K$. 
\end{definition}

\begin{lemma}
If $f \in K[X]$ is irreducible and $f' \neq 0_{K[X]}$ then $f$ is seperable.
\end{lemma}

\begin{proof}
Because $f' \neq 0$ we have that $\deg{f'} < \deg{f}$ so $f \ndivides f'$. Now consider the ideal $(f, f')$. Because $K[X]$ is a PID, we have that $(f, f') = (g)$ so $g \divides f$. However, $f$ is irreducible so $g = uf$ or $g = u$ with $u \in K[X]^\times$. However, $g \divides f'$ and $f \ndivides f'$ so $g = u$. Therefore, $(f, f') = K[X]$. In particular, there exist $a, b \in K[X]$ such that $af + bf' = 1$. Take any field extension $E/K$. If $f$ had a multiple root in $E$ then there would be some $\alpha \in E$ such that $f(\alpha) = f'(\alpha) = 0$. However, then $a(\alpha) f(\alpha) + b(\alpha) f'(\alpha) = 0$ which contraicts the fact that $a f + b f' = 1$. Therefore, $f$ has no multiple roots in any field extension of $K$ and, in particular, none in its splitting field.  
\end{proof} 

\begin{proposition}
Let $K$ have characteristic zero, then any irreducible polynomial in $K[X]$ is seperable. 
\end{proposition}

\begin{proof}
Let $f(X) = a_n X^n + \cdots + a_1 X + a_0$ be an irreducible polynomial over $K$. Now, \[f'(X) = n \cdot a_n X^{n-1} + \cdots a_1\] If $f' \neq 0$ then $f$ is $f$ is seperable by above. Otherwise, because $K$ has characteristic zero, the unique homomorpihsm $\Z \to K$ given by repeated addition is injective so $f' = 0$ implies that $a_n = \cdots= a_1 = 0$. Therefore, $f(X) = a_0$ which is already split in $K$ and has no roots. Thus, $f$ is vacuously seperable.   
\end{proof}

\begin{lemma}
Let $K$ have characteristic $p$ and $f \in K[X]$ be irreducible. Then, there exists a seperable polynomial $g \in K[X]$ and some $k \in \Z$ such that $f(X) = g(X^k)$. 
\end{lemma}

\begin{proof}
If $f$ is seperable, then let $k = 1$ and $g = f$. Otherwise, because $f$ is inseperable and irreducible, $f' = 0$. Let \[f(X) = \sum\limits_{k = 0}^n a_k X^k \quad \text{so} \quad f'(X) = \sum\limits_{k = 1}^n k \cdot a_k X^{k - 1} = 0\]
Therefore, $k \cdot a_k = 0$ for each $k \ge 1$. Therefore, either $a_k = 0$ or $k \in \ker{\varphi}$ with $\varphi : \Z \to K$. Thus, $p \divides k$. Thus, the only nonzero terms are divisible by $k$. Therefore,
\[f(X) = \sum\limits_{i = 0}^r a_{ip} X^{ip} = g_1(X^p) \quad \text{where} \quad g_1(X) = \sum\limits_{i = 0}^n a_{ip} x^i\]
Now, $g_1$ is irreducible because $f(X) = g_1(X^p)$ and $f$ is irreducilbe. If $g_1$ is seperable, we are done. Else, by the same argument, $g_1(X) = g_2(X^p)$ and thus $f(X) = g_2(X^{p^2})$. At each stage, the degree is reduced so either the process terminates because $g_k' \neq 0$ and then $g_k$ is seperable with $f(X) = g_k(X^{p^k})$ or we reach $\deg{g_k} < p$. However, then $g_k' = 0$ implies that $g = 0$ because no power is an element in the kernel. Thus, $g_k' \neq 0$ which reduces to the earlier case.  
\end{proof}

\begin{definition}
For a field extension $E/K$ an element $\alpha \in E$ is seperable over $K$ if $\minimal{\alpha}{K}$ is seperable. 
\end{definition}

\begin{definition}
An extension $E/K$ is seperable if $\forall \alpha \in E : \alpha$ is seperable over $K$.
\end{definition}

\begin{definition}
$K$ is perfect is every algebraic extension is seperable.
\end{definition}

\begin{proposition}
If $K$ has characteristic zero then $K$ is perfect.
\end{proposition}

\begin{proof}
Because $K$ has characteristic zero, every irreducible polynomial over $K$ is seperable including the minimal polynomial of any $\alpha \in E$.  
\end{proof}

\begin{definition}
Let $\ch{K} = p$, the Frobenius map $\sigma_F : K \to K$ is given by $\sigma_F : x \to x^p$.
\end{definition}

\begin{lemma}
The Frobenius map is a field endomorphism. 
\end{lemma}

\begin{proof}
Take $x,y \in K$ then, 
\[\sigma_F(x + y) = (x + y)^p = \sum\limits_{k = 0}^p \frac{p!}{k! (p - k)!} x^k y^{p - k}\]
However, if $k < p$ and $p - k < p$ i.e. $0 < k < p$ then $p \divides \frac{p!}{k! (p - k)!}$ so because $\ch{K} = p$, the term $\frac{p!}{k! (p - k)!} x^k y^{p - k} = 0$. Therefore, $(x + y)^p = x^p + y^p$. Thus, $\sigma_F(x + y) = \sigma_F(x) + \sigma_F(y)$. Furthermore, $\sigma_F(xy) = (xy)^p = x^p y^p = \sigma_F(x) \sigma_F(y)$ because field multiplication is commutative. 
\end{proof}

\begin{theorem}
Let $\ch{K} = p$, then $K$ is perfect if and only if the Frobenius map is surjective and therefore an isomorphism because field homomorphisms are injective. 
\end{theorem}

\begin{proof}
Let $\sigma_F$ be an isomorphism and suppose that $E/K$ is not seperable. Then, $\exists \alpha \in E$ such that $q =\minimal{\alpha}{K}$ is not seperable. Therefore, $q(X) = g(X^{p^k})$ for some $k \in \Z$ and some irreducible $g \in K[X]$. Write, 
\[q(X) = g(X^p) = \sum_{k = 0}^r a_k X^{pk}\]
However, because $\sigma$ is an automorphism, there exists $b_k \in K$ such that $\sigma(b_k) = a_k$. Thus, we have $a_k = (b_k)^p$ and then,
\[q(X) = g(X^p) = \sum_{k = 0}^r a_k X^{pk} = \sum_{k = 0}^r (b_k)^p X^{pk} = \sum_{k = 0}^r \sigma_F(b_k X^k) = \sigma_F \left(\sum_{k = 0}^r b_k X^k\right) = R(X)^p \]
which contradicts the irreduciblility of $q$. \bigskip \\
Conversely, suppose that $\sigma_F$ is not sujective. Then, there exists $a \in K$ such that $a \notin \Im{\sigma_F}$. Therefore, $X^p - a$ has no roots in $K$. Let $\alpha$ be a root of $X^p - a$ in the splitting field such that $\alpha^p = a$ so $X^p - a = X^p - \alpha^p = (X - \alpha)^p$ because the Frobenius is a homomorphism. However, $\minimal{\alpha}{K}$ divides $X^p - a = (X - \alpha)^p$ so $\minimal{\alpha}{K} = (X - \alpha)^k$ by unique factorization. Also, $k > 1$ because $\alpha \notin K$ since $\alpha$ is a root of $X^p - a$ which has no roots in $K$. Thus, the minimal polynomial of $\alpha$ has multiple roots and therefore, $K$ is not perfect.   
\end{proof}

\section{Classification of Finite Fields}

If $K$ is a finite field then $\ch{K} = p > 0$ else we would have an injection $\varphi : \Z \to K$ and $\finfield{p} \cong \Z/p\Z$ and $[K : \finfield{p}] = n < \infty \implies |K| = p^n$. Since $K$ is a field and $K^\times$ is a finite group of order $p^n - 1$ so $K^\times$ is cyclic. Thus, $\forall x \in K : x^{p^n} - x = 0$ because $0$ satisfies this and for $x \in K^\times = K \setminus \{0 \}$ we know that $x^{p^n - 1} = 1$ so $x^{p^n} - x = 0$. Thus, the polynomial $P(X) = X^{p^n} - X$ has exactly $p^n$ roots in $K$. Thus, $K$ is the splitting field of $P$ over $\finfield{p}$. Therefore, any two extensions of $\finfield{p}$ of equal degree are isomorphic. In particular, a unique $K$ exists by the existence of the splitting field of $P$ over $\finfield{p}$. We must check that $|K| = p^n$. Then $P'(X) = p^n X^{p^n - 1} - 1 = -1$ is always nonzero. Therefore, $P$ cannot have multiple roots in $\finfield{p}$. However, $K$ is the spliting field of a degree $p^n$ polynomial so $P$ splits into $p^n$ factors which are all distict. Thus, $|K| \ge p^n$. However, if $\alpha, \beta$ are roots of $P$ then 
\[(\alpha \beta)^{p^n} - \alpha \beta = (\alpha^{p^n} - \alpha + \alpha) \beta^{p^n} - \alpha \beta = \alpha(\beta^{p^n} - \beta) = 0\]
and likewise,    
\[(\alpha +  \beta)^{p^n} - (\alpha +  \beta) = \alpha^{p^n} + \beta^{p^n} - (\alpha +  \beta) = 0\]
because $K$ has characteristic $p$. Thus, the roots of $P$ form a subfield of $K$ but $K$ is the splitting field of $P$ so this subfield cannot be proper. Thus, $|K| = p^n$.

\section{Galois Theory}

\begin{definition}
$E/K$ is Galois if $E/K$ is normal and seperable. 
\end{definition}

\begin{definition}
$\galgroup{E/K} = H < \Aut{E}$ where $\sigma \in H \iff \forall x \in K : \sigma(x) = x$. 
\end{definition}

\begin{proposition}
Let $F/K$ be Galois and $K \subset E \subset F$ then $F/E$ is Galois.
\end{proposition}

\begin{proof}

\end{proof}

\begin{proposition}
Let $F/K$ be Galois and $K \subset E \subset F$ then $E/K$ is Galois iff $E/K$ is normal.
\end{proposition}

\begin{proof}

\end{proof}

\begin{proposition}
For $K'/K$ and $E, K \subset F'$ and $E/K$ is Galois then $EF/KF$ is Galois. 
\end{proposition}

\begin{proof}

\end{proof}

\begin{theorem}
A field extension $E/K$ is Galois if and only if $[E : K] = |\galgroup{E/K}|$  
\end{theorem}

\begin{proof}

\end{proof}

\begin{definition}
For a field extension $E/K$ and $H < \galgroup{E/K}$ then, \[E^H = \fix{E}{H} = \{\alpha \in E \mid \forall \sigma \in H : \sigma(\alpha) = \alpha\}\]
\end{definition}

\begin{proposition}
Let $E/K$ be fintie Galois and $\alpha \in E$ then by normality, \[q(X) = \minimal{\alpha}{K}(X) = a (X - \alpha_1) \cdots (X - \alpha_n)\] and $\galgroup{E/K}$ acts on the set $\{\alpha_1, \dots, \alpha_n\}$ transitively. 
\end{proposition}

\begin{proof}
Let $G = \galgroup{E/K}$ and take $\sigma \in G$ then $q(X) = a_0 + a_1 X + \cdots + a_n X^n$ for $a_k \in K$. Now, $q(\alpha_i) = a_0 + a_1 \alpha_i + \cdots + a_n \alpha_i^n = 0$. Thus,
\[\sigma(q(\alpha_i)) = \sigma(a_0) + \sigma(a_1) \sigma(\alpha_1) + \cdots + \sigma(a_n) \sigma(\alpha_i)^n = a_0 + a_1 \sigma(\alpha_1) + \cdots + a_n \sigma(\alpha_i)^n = q(\sigma(\alpha_i))\]
because $\sigma$ preserves $K$. Thus, $q(\sigma(\alpha_i)) = \sigma(q(\alpha_i)) = 0$ so $\sigma(\alpha_i)$ is a root of $q$ is is split so $\sigma(\alpha_i) = \alpha_j$ for some $j$. \\\\
By hypothesis, $E$ is normal and thus $E$ is the splittig field of some $f \in K[X]$. Let $s = f \cdot q$ then $E$ is also the splitting field of $s$ over $K$. Take $\alpha, \beta$ th (STILL PROVE THIS)
\end{proof}

\begin{proposition}
Let $E/K$ be normal and therefore the splitting field of a monic $f \in K[X]$ with $\deg{f} = n$. Then, there exists an embedding $\phi : \galgroup{E/K} \to S_n$ which is a transitive subgroup of $S_n$ if $f$ is irreducible.  
\end{proposition}

\begin{proof}
Write $f(X) = (X - \alpha_1) \cdots (X - \alpha_n)$. Then, $\galgroup{E/K}$ acts on the set of roots $\{\alpha_1, \cdots, \alpha_n\}$ and this action is a homomorphism $\phi : \galgroup{E/K} \to S_n$. If $\phi(\sigma) = \mathrm{id}$ then $\sigma(\alpha_i) = \alpha_i$ for every $i$. However, $\sigma$ is a $K$-preserving homomorphism so $\sigma$ preserves $E = K(\alpha_1, \dots, \alpha_n)$ and is thus the identitiy element of $\galgroup{E/K}$. Suppose $f$ is irreducible and monic, then, $f = \minimal{\alpha_i}{K}$ and therefore, $\galgroup{E/K}$ acts transitivly on $\{\alpha_1, \cdots, \alpha_n\}$ so by definition, its image in $S_n$ is a transitive subgroup.
\end{proof}


\begin{theorem}[Fundamental Theorem of Galois Theory]
Let $E/K$ be a finite Galois extension then there is a bijection $\varphi_G$ (a Galois correspondence) between the subgroups $H < \galgroup{E/K}$ and subfields $K \subset F \subset E$ such that:

\begin{enumerate}
\item $\varphi : H \mapsto E^H$
\item $\varphi^{-1} : F \mapsto \galgroup{E/F}$ where $E/F$ is also Galois. 
\item $\varphi^{-1} \circ \varphi : H \mapsto \galgroup{E/E^H} = H$
\item $\varphi \circ \varphi^{-1} : F \mapsto E^{\galgroup{E/F}} = F$
\item $H_1 < H_2 \iff E^{H_1} \supset E^{H_2}$
\item $|H| = [E : E^H]$ and $[\galgroup{E/K} : H] = [E^H : K]$ 
\item $H \triangleleft \galgroup{E/K} \iff E^H/K$ is Galois and then $\galgroup{E^H/K} \cong \galgroup{E/K} / H$
\end{enumerate}
\end{theorem}

\begin{proof}
\begin{enumerate}
\item

\item

\item

\item

\item

\item

\item For $\sigma \in G$, $\varphi(\sigma H \sigma^{-1}) = \sigma(E^H)$ so if $H$ is normal iff $\sigma(E^H)$ for every $\sigma \in G$. 

\end{enumerate}

\end{proof}

\begin{proposition}
Let $E/K$ be a finite Galois extension with $G = \galgroup{E/K}$ and let $H \subset G$ with $F = E^H$. Then $F/K$ is Galois iff $H \triangleleft G$ and then $\galgroup{F/K} \cong G/H$. 
\end{proposition}

\begin{proof}
Since $E/K$ is seperable, so is $F/K$.  Now, $g(F) = E^{g H g^{-1}}$ because for $g \in G$, \[x \in g(F) \iff \exists y \in F = E^H  : g(y) = x \iff \forall \sigma \in H : \sigma(y) = y \text{ and } g(y) = x \]
if and only if for any $\sigma \in H$, $g \sigma g^{-1}(x) = g \sigma(y) = g(y) = x$ i.e. $x \in E^{g H g^{-1}}$. Suppose that $H$ is normal then $gHg^{-1} = H$ so $g(F) = F$ and thus $F$ contains all the roots of any minimal polynomial of its elements because the set of $g$ acts transitivly on the roots but $g(F) = F$ so $F$ contains all the images. Explicitly, for any $\alpha \in F$ we have $g(\alpha) \in g(F) = F$ so $F$ contains all the conjuates of $\alpha$. Therefore, $F/E$ is normal. Now, let $F/K$ be normal and therefore the splitting field of some polynomial $f \in K[X]$. Therefore, 
\[f(X) = a (X - \beta_1) \cdots (X - \beta_n)\]
and $F = K(\beta_1, \dots, \beta_n)$. However, $g \in \galgroup{E/K}$ so $g$ acts on the set of roots of $f$. However, $g$ preserves $K$ so $g(F) = F$ because $g(\beta_i) = \beta_j$ and $F = K(\beta_1, \dots, \beta_n)$. Therefore, $E^H = E^{g H g^{-1}}$ so $H = gHg^{-1}$ thus $H \triangleleft G$. In this case, define the homomorphism $\eta : G \to \galgroup{F/K}$ by $\eta : \sigma \to \sigma |_F = \sigma \circ \iota_F$. Now, 
\[\sigma \in \ker{\eta} \iff \sigma|_F = \id_F \iff \sigma \in \galgroup{E/F} \iff \sigma \in H\]
Thus, $\ker{\eta} = H$ so $G/H \cong \Im{\eta}$. However, $[F : K] = |G/H| = |\galgroup{F/K}|$ so $|\Im{\eta}| = |\galgroup{F/K}|$ and thus $\Im{\eta} = \galgroup{F/K}$. Finally, $\galgroup{F/K} \cong G/H$.  
\end{proof}

\begin{definition}
In $\Q(Y_1, \cdots, Y_n)$, the fraction field of $\Q[Y_1, \cdots, Y_n]$, the elementary symmetric polynomials are, 
\[u_i = \sum_{k_1 < k_1 < \cdots < k_i} Y_{k_1} Y_{k_2} \cdots Y_{k_i}\]
\end{definition}


\begin{proposition}
Let $K_0 = \Q(u_1, \cdots, u_n) \subset \Q(Y_1, \cdots, Y_n) = E_0$ then let $f_0 \in K_0[X]$ be the polynomial $x^n - u_1 X^{n-1} + u_2 X^{n-2} + \cdots + (-1)^n u_n$. Then $E_0$ is the splitting field of $f_0$ and $\galgroup{E_0/K_0} \cong S_n$.
\end{proposition}

\begin{proof}
By Vieta, $f_0(X) = (X - Y_1) \cdots (X - Y_n)$ so because $E_0 = \Q(Y_1, \cdots Y_n)$ we have that $E_0$ is the splitting field of $f_0$ over $K_0$. Therefore, $E_0 / K_0$ is a normal extension and also a seperable extension because $\Q$ is perfect and $\Q \subset K_0 \subset E_0$. Therefore, $E_0/K_0$ is Galois. Futhermore, consider $G =  S(\{Y_1, \cdots, Y_n\}) \cong S_n$. Any $\sigma \in G$ satisfies $\sigma(u_i) = u_i$ because they are unique with respect to reordering. We extend $\sigma : E_0 \to E_0$ by fixing it on $\Q$. Then $\sigma|_{K_0} = \id_{K_0}$ because $K_0 = \Q(u_1, \cdots u_n)$. Then $G \hookrightarrow \galgroup{E_0/K_0} \hookrightarrow S_n \cong G$. Therefore, $G \cong \galgroup{E_0/K_0}$. 
\end{proof}

\begin{corollary}
Any symmetric polynomial is generated by elementary symmetric polynomials.
\end{corollary}

\begin{proof}
Let $f \in \Q(Y_1, \cdots, Y_n)$ be symmetric. For any automorphism $\sigma \in \galgroup{E_0/K_0}$, $\sigma(f) = f$ because the variables are symmetric under exchange. Therefore, $f$ is fixed by every Galois automorphism so $f \in E_0^{\galgroup{E_0/K_0}} = K_0 = \Q(u_1, \cdots, u_n)$ by the Galois correspondence. Thus, $f$ is a fraction of elements of $\Q[u_1, \dots, u_n]$ but since $f \in \Q[Y_1, \dots, Y_n]$ then it mut lie in $\Q[u_1, \dots, u_n]$. 
\end{proof}


\begin{corollary}
Let $K$ be a field and $f \in K[X]$ with splitting field $E$ such that $f(X) = a(X - \alpha_1) \cdots (X - \alpha_n)$ then any symmetric polynomial in the roots is given by a universal polynomial in the coefficients of $f$.  
\end{corollary}

\begin{definition}
$\Disc{f} = \Delta = \prod\limits_{i < j} (\alpha_i - \alpha_j)^2$ is a symmetric polynomial in the roots of $f$ and therefore expressible as a polynomial of the coefficients of $f$. 
\end{definition}

\begin{proposition}
$f \in K[X]$ is seperable if and only if $\Disc{f} \neq 0$
\end{proposition}

\begin{proof}
$\Disc{f} = \prod\limits_{i < j} (\alpha_i - \alpha_j)^2 = 0$ if and only if one of the factors is zero i.e. for $i < j$ we must have $\alpha_i - \alpha_j = 0$ so $\alpha_i = \alpha_j$. Thus, $f$ has multiple roots in its splitting field and is thus non-seperable if and only if $\Disc{f} = 0$. 
\end{proof}

\begin{corollary}
Let $\ch{K} = 0$, if $\Disc{f} = 0$ then $f$ is not irreducible. 
\end{corollary}

\begin{proof}
Suppose that $\Disc{f} = 0$ then $f$ must be non-seperable but because $\ch{K} = 0$ every irreducible polynomial over $K$ is seperable so $f$ is not irreducible. 
\end{proof}

\begin{definition}
For $\sigma \in \galgroup{E/K}$ embdded in $S_n$, define $\mathrm{sgn} : \galgroup{E/K} \to \{\pm 1\}$ where $\sgn{\sigma}$ is the sign of the permutation defined by $\sigma$ acting on the $n$ roots.
\end{definition}

\begin{lemma}
Let $\sigma \in \galgroup{E/K}$ then $\sigma(d) = d \cdot \sgn{\sigma}$ where $d = \prod\limits_{i < j} (\alpha_i - \alpha_j)$
\end{lemma}

\begin{theorem}
Let $f \in K[X]$ with $\deg{f} = n$ and $\ch{K} = 0$ and $\Delta = \Disc{f} \neq 0$. Suppose that $E$ is the splitting field of $f$ over $K$. Now, $\galgroup{E/K}$ is embedded in $A_n$ if and only if $\Delta$ is a square in $K$.
\end{theorem}

\begin{proof}
$\pm d \in E$ are the only square roots of $\Delta$ (since $X^2 - \Delta$ has exactly two solutions) therefore if $\Delta$ is a square in $K$ then $d \in K$. However, $K$ is fixed by every Galois automorphism so $\sigma(d) = d$. Since $d \neq 0$, $\sgn{\sigma} = 1$ because $\sigma(d) = d \cdot \sgn{\sigma}$. Therefore, $\sigma \in A_n$. Conversely, if $\galgroup{E/K} \hookrightarrow A_n$ then every $\sigma \in \galgroup{E/K}$ satisfies $\sgn{\sigma} = 1$.  
\end{proof}

\subsection{Candano's Formula}

\begin{theorem}
Let $K$ have characteristic $0$ then if $f \in K[X]$ is $f(X) = X^3 + px + q$ then 
\end{theorem}

\begin{proof}

\end{proof}

\subsection{Cyclic Extensions}

\begin{definition}
Let $K$ be a field, then $\mu_n(K) = \{ \alpha \in K \mid \alpha^n = 1 \}$. 
\end{definition} 

\begin{proposition}
Let $\ch{K} = 0$ or $\ch{K} = p$ which is coprime with $n$ then if $f(X) = X^n - 1$ splits in $K$ then $|\mu_n(K)| = n$.
\end{proposition}

\begin{proof}
In this case, $f'(X) = n X^{n-1} \neq 0$ because $n \notin (p)$. Thus, $f$ has no double roots in $K$ but $f$ splits in $K$ so it has exactly $n$ roots. Therefore, $|\mu_n(K)| = n$. 
\end{proof}

\begin{proposition}
$\mu_n(K)$ is a cyclic group under multiplication. Its generators are the \textit{primitive} $n^{\mathrm{th}}$ roots of unity. 
\end{proposition}

\begin{proof}
Take $\alpha, \beta \in \mu_n(K)$ then $(\alpha \beta)^n = \alpha^n \beta^n = 1$ thus $\alpha \beta \in \mu_n(K)$. Futhermore, since $\alpha^n = 1$ then $\alpha \neq 0$ therefore $\alpha^{-1} \in K$ and $(\alpha^{-1})^n = (\alpha^n)^{-1} = 1$ so $\alpha^{-1} \in \mu_n(K)$. Lastly, $1^n = 1$ so $1 \in \mu_n(K)$. Therefore, $\mu_n(K)$ is a finite subgroup of $K^\times$ so $\mu_(K)$ is cyclic.  
\end{proof}

\begin{proposition}
If $|\mu_n(K)| = n$ and $m \divides n$ then $|\mu_m(K)| = m$.
\end{proposition}

\begin{proof}
Since $\mu_n(K)$ is cyclic and $m \divides n$ there is a unique subgroup $H < \mu_n(K)$ of order $m$. By Lagrange, $\forall h \in H : h^m = 1$ so $H \subset \mu_m(K)$. Therefore, $|\mu_m(K)| \ge m$ however, by the maximum number of roots of a polynomial, $|\mu_m(K)| \le m$ so $|\mu_m(K)| = m$.  
\end{proof}

\begin{definition}
$E/K$ is a cyclic extension if it is Galois and $\galgroup{E/K}$ is a cyclic group.
\end{definition}

\begin{lemma}
Let G be a group and $K$ a field. Let $\phi_1, \dots, \phi_n : G \to K^\times$ be distinct homomorphisms then $\forall \lambda_1, \dots, \lambda_n \in K$ not all zero, the map $\lambda_1 \phi_1 + \cdots + \lambda_n \phi_n : G \to K$ is not identically zero. 
\end{lemma}

\begin{proof}
Suppose that $n$ is the least positive $n \in \Zplus$ such that $\exists \lambda_1, \dots, \lambda_n \in K$ not all zero such that the map $\lambda_1 \phi_1 + \cdots + \lambda_n \phi_n : G \to K$ is identically zero. However, because $n$ is minimal, every $\lambda_i \neq 0$ since if $\lambda_1 \phi_1 + \cdots + \lambda_n \phi_n = 0$ then $\lambda_1 \phi_1 + \cdots  + \lambda_{i-1} \phi{i -1} + \lambda_{i+1} \phi_{i+1} + \cdots + \lambda_n \phi_n = 0$ so we have a smaller counterexample. Since $\phi_1 \neq \phi_2$ then $\exists g \in G : \phi_1(g) \neq \phi_2(g)$. Then, $\forall x \in G$, 
\[\lambda_1 \phi_1(gx) + \cdots + \lambda_n \phi_(gx) = \lambda_1 \phi_1(g) \phi_1(x) + \cdots + \lambda_n \phi_n(g) \phi_n(x) = 0\]   
but also,
\[\lambda_1 \phi_1(x) + \cdots + \lambda_n \phi_n(x) = 0 \implies \phi_1(g) (\lambda_1 \phi_1(x) + \cdots + \lambda_n \phi_n(x)) = 0\]   
Therefore, subtracting the expressions,
$\lambda_1 ( \phi_1(g) - \phi_1(g)) \phi_1(x) + \lambda_2 ( \phi_2(g) - \phi_1(g)) \phi_2(x) + \cdots + \lambda_n ( \phi_n(g) - \phi_2(g)) \phi_1(x) = 0$ 
so define $\mu_i = \lambda_i (\phi_i(g) - \phi_1(g))$ then $\mu_1 = 0$ but $\mu_2 \neq 0$ because $\lambda_2 \neq 0$ and $\phi_1(g) \neq \phi_2(g)$. Thus, for any $x \in G$,
\[ \mu_2 \phi_2(x) + \cdots + \mu_n \phi_n(x) = 0\]
where not all $\mu_i$ are zero. Thus, we have found a counterexample of size $n - 1$ contradicting the minimality of $n$. 
\end{proof}

\begin{theorem}
If $\ch{K} = 0$ or $\ch{K} = p$ is coprime to $n$ and $|\mu_n(K)| = n$ then,
\begin{enumerate}
\item If $E/K$ is cyclic of degree $n$ then $\exists \alpha \in E$ such that $E = K(\alpha)$ and $\alpha^n \in K$.
\item If $E = K(\alpha)$ where $\alpha \in E$ such that $\alpha^n \in K$, where $m$ is the least positive integer such that this holds, then $E/K$ is cyclic of degree $m$.
\end{enumerate}
\end{theorem}

\begin{proof}
Suppose that $E/K$ is cyclic then let $\sigma \in \galgroup{E/K}$ be a generator. Let $\omega \in \mu_n(K)$ be a primitive $n^{\mathrm{th}}$ root of unity. Applying Dedekind's Lemma to the set of homomorphisms,
\[\phi_1 = \sigma, \phi_2 = \sigma^2, \dots, \psi_{n-1} = \sigma^{n-1}, \phi_n =  \sigma^n = \id : E^{\times} \to E^{\times}\] Also, take the element,
\[\lambda_1 = \omega, \lambda_2 = \omega^2, \dots, \lambda^n = \omega^n = 1\]
Then, the map $\phi = \sum\limits_{i = 1}^n \lambda_i \phi_i = \sum\limits_{i = 1}^n \omega^i \sigma^i : E^\times \to E$ is not the zero map. Therefore, $\exists \beta \in E^\times$ such that $\alpha = \phi(\beta) = \omega \sigma(\beta) + \omega^2 \sigma^2(\beta) + \dots + \omega^n \sigma^n(\beta) \neq 0$. Now, consider,
\[\sigma(\alpha) = \omega \sigma^2(\beta) + \omega \sigma^3(\beta) + \cdots + \omega^n \sigma^{n+1}(\beta) = \sigma(\beta) + \omega \sigma^2(\beta) + \cdots + \omega^{n-1} \sigma^n(\beta) = \omega^{-1} \alpha\] 
Therefore, $\sigma(\alpha^n) = \sigma(\alpha)^n = \omega^{-n} \alpha^n = \alpha^n$. Because $\sigma$ generates $\galgroup{E/K}$ we have that $\alpha^n$ is fixed under every Galois automorphism so $\alpha^n \in K$. Futhermore, $\sigma^i(\alpha) = \omega^{-i} \alpha$ which are all distinct because $\omega$ is primitive and $\alpha \neq 0$ so if $\omega^i \alpha = \omega^j \alpha$ then $\omega^{i-j} = 1$ which means $n \divides i - j$ so $i = j$ because both are reduced modulo $n$. Therefore, $\minimal{\alpha}{K}$ has $n$ distinct roots in $E$ so $\deg{\alpha} \ge n$ but \[[E : K] = [E : K(\alpha)] [K(\alpha) : K] \ge n [E : K(\alpha)]\] which implies that $[E : K(\alpha)] = 1$ since $[E : K] = n$. Therefore, $E = K(\alpha)$. \bigskip \\\\
Suppose that $E = K(\alpha)$ and $m$ is the lest positive integer such that $\alpha^m \in K$. Then $m \divides n$ and take $b = \alpha^n \in K$. Then, $\alpha$ is a root of the polynomial $f(X) = X^m - b \in K[X]$. Let $\zeta = \omega^{n/m}$ which is a primitve $m^{\mathrm{th}}$ root of unity in $K$ and then $f(X) = (X - \alpha)(X - \zeta \alpha) \cdots (X - \zeta^{m-1} \alpha)$ so $E = K(\alpha)$ is the splitting field of $f$ which is therefore seperable because $\zeta$ is primitive. Therefore, $E/K$ is a Galois extension. Consider the map $\phi : \galgroup{E/K} \to \left< \zeta \right> \subset K^\times$ given by $\phi : \sigma \mapsto \sigma(\alpha) \alpha^{-1}$ which is a power of $\zeta$ because $\sigma$ maps $\alpha$ to a root of $f$ which is of the form $\zeta^i \alpha$. Also, \[\phi(\sigma \tau) = \sigma \circ \tau(\alpha) \alpha^{-1} = \sigma(\alpha \tau(\alpha) \alpha^{-1}) \alpha^{-1} = \sigma(\alpha \phi(\tau)) \alpha^{-1} = \sigma(\alpha) \alpha^{-1} \sigma(\phi(\tau)) = \phi(\sigma) \phi(\tau)\] because $\phi(\tau) \in K$ and is thus fixed by the Galois group. Furthermore, if $\phi(\sigma) = 1$ then $\sigma(\alpha) = \alpha$ so $\sigma(\zeta^i \alpha) = \zeta^i \alpha$ so $\sigma$ acts trivially on the roots of $f$. Therefore, $\sigma = \id $ because $E$ is the splitting field of $f$. Thus, $\phi$ is injective so the Galois group is embedded in $\left< \zeta \right> \cong C_m$. 
\end{proof}

\begin{definition}
A finite extension $E/K$ is solvable if $E = K(\alpha_1, \dots, \alpha_n)$ such that $\forall i$, $\alpha_i^{n_i} \in K(\alpha_1, \dots, \alpha_{i-1})$. $E/K$ is $N$-solvable if $n_i \divides N$ for each $i$. 
\end{definition}

\begin{definition}
The normal closure $E'$ of $E/K$ is the splitting field of $\alpha_i$ for all $i$.
\end{definition}

\begin{proposition}
If $E/K$ is solvable then its normal closure is also solvable.
\end{proposition}

\begin{proposition}
If $K \subset K' \subset E$ and $E/K$ is solvable then $E/K'$ is also solvable.
\end{proposition}

\begin{definition}
A polynomial $f \in K[X]$ is solvable if its splitting field over $K$ is contained in a solvable extension of $K$.
\end{definition}

\begin{definition}
A finite group $G$ is solvable if there is a tower $G = G_0 \triangleright G_1 \triangleright \dots \triangleright G_k = \{e\}$ such that for each $i$ the factor group $G_{i}/G_{i+1}$ is abelian. 
\end{definition}

\begin{lemma}
A finite group $G$ is solvable if and only if there is a tower $G = G_0 \triangleright G_1 \triangleright \dots \triangleright G_k = \{e\}$ such that for each $i$ the factor group $G_{i}/G_{i+1}$ is cyclic. 
\end{lemma}

\begin{theorem}[Galois]
$f \in K[X]$ is solvable if and only if $\galgroup{E/K}$ is solvable where $E$ is the splitting field of $f$ and $\ch{K} = 0$.
\end{theorem}

\begin{proof}
Let $f \in K[X]$ be solvable and let $E$ be the splitting field of $f$ over $K$. Then, there exists a solvable extension $F$ of $K$ such that $K \subset E \subset L$. Let $N$ be the $\mathrm{lcm}$ of the degrees of the extensions from $K$ to $F$ so $F$ is $N$-solvable over $K$. Now let $\omega$ be a primitive $n^{\mathrm{th}}$ root of unity. Because $\omega^N = 1 \in E$ then the extension $F(\omega)$ is $N$-solvable over $K$. Finally, let $L$ be the normal closure of $F(\omega)$ which is still solvable over $K$. Futhermore because $K \subset K(\omega) \subset L$ then $L$ is solvable over $K(\omega)$. 

\begin{center}
\begin{tikzcd}[column sep=small]
L \arrow[dash, d] \\
F(\omega) \arrow[dash, d] \\
F \arrow[dash, d] \\
E \arrow[dash, d] & K(\omega) \arrow[dash, ld] \arrow[dash, luu] \\
K
\end{tikzcd}
\end{center}
Because $L/K(\omega)$ is solvable, there exists $K \subset K(\omega) = L_0 \subset L_1 \subset L_2 \subset \cdots \subset L_k = L$ such that $L_{i + 1} = L_i(\alpha_i)$ such that $\alpha_i^N \in L_i$. Because $\omega \in L_i$ and $\ch{L_i} = 0$, by the main theorem on cyclic extensons, $L_{i+1}/L_i$ is a cyclic extension. This holds because $\omega$ is a primitive root so $|\mu_N(L_i)| = N$. Futhermore, $K(\omega)/K$ is a cyclotomic extension and thus abelian. Now, by the Galois correspondence, the chain,
\[K \subset K(\omega) \subset L_1 \subset L_2 \subset \cdots \subset L_k = L\]
corresponds to the subgroups,
\[\galgroup{L/K} \supset \galgroup{L/K(\omega)} \supset \galgroup{L/L_1} \supset \cdots \supset \galgroup{L/L_{k-1}} \supset \galgroup{L/L_k} = \{e\}\]
however, each extension is Galois and therefore, each group is a normal subgroup of the previous. Furthermore, $\galgroup{L_{i+1}/L_i} \cong \galgroup{L/L_{i}}/\galgroup{L/L_{i+1}}$ but $L_{i+1}/L_i$ is a cyclic extension so these quotient groups are cyclic. Likewise, $K(\omega)/K$ is an abelian extension and thus Galois so $\galgroup{K(\omega)/K} \cong \galgroup{L/K}/\galgroup{L/K(\omega)}$ but $\galgroup{K(\omega)/K}$ is abelian so the quotient is also abelian. Therefore, this is a solvable series for $\galgroup{L/K}$. Therefore, because $E/K$ is Galois, then, 
\[\galgroup{E/K} \cong \galgroup{L/K} /\galgroup{L/E}\] which is a quotient group of a solvable group and therefore solvable. \bigskip \\\\
Conversely, let $E$ be the splitting field of $f \in K[X]$ and suppose that $\galgroup{E/K}$ is solvable. Let $L$ be the normal closure of $E(\omega)$ where $\omega$ is a primitve $N^{\mathrm{th}}$ root of unity. Consider $L/K(\omega)$ and define the map,
\begin{align*}
\psi : \galgroup{L/K(\omega)} \subset \galgroup{L/K} & \to \galgroup{E/K} = G \\
\psi : \sigma & \mapsto  \sigma |_E
\end{align*} 
Now, $L$ is the splitting field of $f$ over $K(\omega)$ so if $\sigma \in \galgroup{L/K(\omega)}$ then $\sigma$ is determined by its action on the roots of $f$. However, all the roots of $f$ are contained in $E$ so $\sigma$ is determined by its action of $E$. Therfore, $\psi$ is an injective map. Thus, $\galgroup{L/K(\omega)}$ is embedded in $\galgroup{E/K} = G$ a solvable group. Therefore, $\galgroup{L/K(\omega)}$ is also solvable. By the lemma, $\galgroup{L/K(\omega)}$ admits a normal series with cyclic factors,
\[ G = G_0 \triangleleft G_1 \triangleleft G_2 \triangleleft \cdots \triangleleft G_0 = \{e\}\]  
By the Galois correspondence, we obtain a tower of subextensions,
\[ K(\omega) = L_0 \subset L_1 \subset L_2 \subset \cdots \subset L_k = L\]
such that each extension is cyclic. Because $K(\omega) \subset L_i$ then $\omega \in L_i$ and $[L_{i+1}/L_i] = n$ where $n \divides N$ so $|\mu_n(L_i)| = n$. Therefore, by the main theorem on cyclic extensions, $L_{i + 1} = L_i(\alpha_i)$ such that $\alpha_i^n \in L_i$. Therefore, $L$ is solvable over $K(\omega)$ and thus solvable over $K$ because $\omega^N  = 1 \in K$. Therefore, $f$ is solvable because $E \subset L$ where $L/K$ is a solvable extension.  
\end{proof}

\begin{corollary}

\end{corollary}

\end{document}

