\documentclass[12pt]{extarticle}
\usepackage{import}
\import{./}{Includes}


\begin{document}
\atitle{2}
\textbf{Page 115.} 
\begin{enumerate}
\item Let $\phi : R \rightarrow S$ be a ring homomorphism and let $A \subset R$ be a subring. For $x, y \in \phi(A)$ we have $\exists a,b \in A$ s.t. $\phi(a) = x$ and $\phi(b) = y$. \\ \\ Now, $a + b, ab, 1_R \in A$ so $\phi(a+b) = \phi(a) + \phi(b) = x + y \in \phi(A)$ and $\phi(ab) = \phi(a) \phi(b) = xy \in  \phi(A)$. Also $\phi(1_R) = 1_S \in \phi(A)$. Finally, $-x = -\phi(a) = \phi(-a) \in \phi(A)$ because $a \in A \implies -a \in A$. Therefore, $\phi(A)$ is a subring of $S$.

\item Let $\phi : R \rightarrow S$ be a ring homomorphism and let $B \subset S$ be a subring. Let $x, y \in \invI{\phi}{B}$ then $\phi(x), \phi(y) \in B$ thus, $\phi(x) + \phi(y) \in B$ so $\phi(x+y) \in B$ equivalently, $x+y \in \invI{\phi}{B}$. Also, $\phi(x)\phi(y) \in B$ so $\phi(xy) \in B$. Therefore, $xy \in \invI{\phi}{B}$. Also $\phi(1_R) = 1_S \in B$ so $1_S \in \invI{\phi}{B}$. Finally, $\phi(-x) = - \phi(x) \in B$ so $-x \in \invI{\phi}{B}$. Therefore, $\invI{\phi}{B}$ is a subring of $R$.

\item Let $\phi : R \rightarrow S$ be a sujective ring homomorphism and let $I \subset R$ be an ideal. By problem 1, since $I$ is an additive subgroup of $R$ then $\phi(I)$ is an additive subgroup of $S$. Take $x \in \phi(I)$ and $s \in S$ thus $\exists a \in I$ s.t. $x = \phi(a)$. Since $\phi$ is surjective, $\exists r \in R$ s.t. $s = \phi(r)$. Now $sx = \phi(r) \phi(a) = \phi(ra) \in \phi(I)$ because $ra \in I$ by absorption. Similarly, $xs = \phi(a) \phi(r) = \phi(ar) \in \phi(I)$ because $ar \in I$ by absorption. Therefore, $\forall x \in \phi(I), \forall s \in S : xs, sx \in \phi(I)$ so $\phi(I)$ is an ideal of $S$. 

\item Let $\phi : \Z \rightarrow \Z/ 2\Z \times \Z/2\Z$ be the unique ring homomorphism from $\Z$. $\phi(1) = (1,1)$ so $\phi(n) = ([n]_2, [n]_2)$. Now consider the ideal $(3) \subset \Z$. However, $\phi((3)) = \{(0, 0), (1, 1) \}$ is not an ideal because $(0, 1) \in \Z/ 2\Z \times \Z/2\Z$ and $(1, 1) \in \phi((3))$ but $(0, 1) \cdot (1, 1) = (0, 1) \notin \phi((3))$.

\item Let $\phi : R \rightarrow S$ be a ring homomorphism and let $J \subset S$ be an ideal. By problem 2, since $J$ is an additive subgroup of $S$ then $\invI{\phi}{J}$ is an additive subgroup of $R$. Take $x \in \invI{\phi}{J}$ then $\phi(x) \in J$ and take $r \in R$. Now $\phi(rx) = \phi(r) \phi(x) \in J$ and $\phi(xr) = \phi(x) \phi(r) \in J$ by the absorption property of $J$. Thus, $rx, xr \in \invI{\phi}{J}$ so $\invI{\phi}{J}$ is an ideal of $R$.

\item Let $F = \{a + ib \mid a, b \in \Q \} \subset \C$. Now $F$ is a subring of $\C$ if it is closed under addition and multiplication and contains inverses and identities.\\

Let $x, y \in F$ then $x = a_x + ib_x$ and $y = a_y + ib_y$ thus $x + y = (a_x + a_y) + i (b_x + b_y) \in F$ because $a_x + a_y, b_x + b_y \in \Q$. Also, $xy = (a_x a_y - b_x b_y) + i (a_x b_y + b_x a_y) \in F$ because $a_x a_y - b_x b_y \in \Q$ and  $a_x b_y + b_x a_y \in \Q$. \\

Also, $1 \in F$ since $1 = 1+i0$ and $-x = -a - ib \in F$ because $1,0, -a, -b \in \Q$. Since $x + (-x) = 0$ and $-x + x = 0$, inverses hold. Furthermore, $F$ is a field because for any $x \in F \sm \{0\}$, $x = a + ib$ then take $y = \frac{1}{a^2 + b^2} (a-ib)$. Now, $xy = yx = (a+ib)(a-ib)\frac{1}{a^2 + b^2} = \frac{a^2 + b^2}{a^2 + b^2} = 1$. This exists because $a^2 + b^2 = 0$ only when $a = b = 0$ i.e. $x = 0$ which is the case we excluded. \\ \\

Let $\phi : \Z[i] \rightarrow F$ be the ring homomorphism given by $\phi(x) = x$ which is trivially injective. Now any $x \in F$ can be writen as $x = \frac{p_1}{q_1} + i \frac{p_2}{q_2} = \frac{p_1 q_2}{q_1 q_2} + i \frac{p_2 q_1}{q_1 q_2}$ with $p_1, p_2, q_1, q_2 \in \Z$. Then, $x = \phi(p_1 q_2 + i p_2 q_1) \phi(q_1 q_2)^{-1}$ which is defined because $q_1 \neq 0$ and $q_2 \neq 0$ and since $\Z$ is a domain, $q_1 q_2 \neq 0$. Also, $\phi$ is injective so $\phi(q_1 q_2) \neq 0$. Therefore, by the universal mapping property of the field of fractions, $F \cong Q_{Z[i]}$. 

\end{enumerate}


\end{document}