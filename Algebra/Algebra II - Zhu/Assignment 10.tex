\documentclass[12pt]{extarticle}
\usepackage{import}
\import{./}{Includes}


\begin{document}
\atitle{10}

\section*{Page 210.} 
\subsection*{Problem 1.}
Let $K = \Q$ and $E = \Q(\sqrt[3]{2})$ and $F = \Q(\zeta_3, \sqrt[3]{2})$ the splitting field of $X^3 - 2$ over $\Q$. Thus, $K \subset E \subset F$ and $F/K$ us normal because $F$ is a splitting field over $K$. However, $E/K$ is not normal because $\sqrt[3]{2} \in E = \Q(\sqrt[3]{2})$ but $\minimal{\sqrt[3]{2}}{\Q} = X^3 - 2$ does not split in $E$ because it has complex roots in $F$ which are not contained in $E$ and any root in $E$ must also be a root in $F$. Therefore, there are not enough roots in $E$ so $E/K$ is not normal.
\subsection*{Problem 4.}
Let $f(X) = X^3 - X - 1$ then $\Disc{f} = \Delta = - 4 p^3 - 27 q^2$ where $p = -1$ and $q = -1$ so $\Delta = -23$. Because $-23$ is not a square in $\Q$ the Galois group of $f$ over $\Q$ is $S_3$. We will now consider an arbitrary irreducible monic cubic polynomial $f$ with a non-square discriminant over $\Q$. Enumerate the roots of $f$ in $\C$ as $\alpha_1, \alpha_2, \alpha_3$. The nontrivial subgroups of $S_3 = \left< \sigma, \tau \mid \sigma^3 = e, \tau^2 = e, \sigma \tau = \tau \sigma^2 \right>$ are, $R = \left< \sigma \right>$, $F_0 = \left< \tau \right>$, $F_1 = \left< \tau \sigma \right>$, and $F_2 = \left< \tau \sigma^2 \right>$. These must correspond to the nontrivial intermediate fields of $E/\Q$, the splitting field of $f$. The action of these elements must be faithful. One choice of representation is $\sigma = (1 \: 2 \: 3)$ and $\tau = (1 \: 2)$. These fixed fields are, 
\[E^{R} = \Q(\sqrt{\Disc{f}}) = \Q(\sqrt{\Delta}) \] 
which holds because $\sqrt{\Disc{f}} = \pm \prod_{i < j} (\alpha_i - \alpha_j) \in E = \Q(\alpha_1, \alpha_2, \alpha_3)$ so $\Q(\sqrt{\Delta}) \subset E$ however, the degree, $[E : \Q(\sqrt{\Delta})] = 2$ because $X^2 - \Delta$ is irreducible if $\Delta$ is not a square in $\Q$. Therefore, $\Q(\sqrt{\Delta})$ corresponds to a subgroup of $S_3$ with index $2$. However, there is only one such subgroup, namely, $R$. By the Galois correspondence, $E^R = \Q(\sqrt{\Delta})$. Next,  
\[E^{F_0} = \Q(\alpha_3)\] 
because $r_3$ is fixed under the permutation $(1 \: 2)$ and no other root. Also, $[\Q(r_3) : \Q] = \deg f = 3$ which is the minimal polynomial because $f$ is irreducible and monic. However, by the Galois correspondence, 
\[[E^{F_0} : \Q] = [S_3 : F_0] = 6/2 = 3\]
Thus, $\Q(\alpha_3) \subset E^{F_0}$ has the degree over $\Q$ of the entire field $E^{F_0}$ and thus must equal $E^{F_0}$. Similarly,
\[E^{F_1} = \Q(\alpha_2)\]   
because $(1 \: 2 \: 3) (1 \: 2)$ fixes $\alpha_2$ and the previous argument ensures that $\Q(r_2)$ is the entire fixed field. Likewise,
\[E^{F_2} = \Q(\alpha_1)\]   
since $(1 \: 2 \: 3)^2 (1 \: 2)$ fixes $\alpha_1$. \bigskip\\
For the polynomial $f(X) = X^3 - X - 1$, which does in fact satisfy the necessary conditions, we have the four intermediate fields of its splitting field, $\Q(\sqrt{\Delta}) = \Q(\sqrt{-23})$ and $\Q(\alpha_1)$ and $\Q(\alpha_2)$ and $\Q(\alpha_3)$. These roots can be found via Cardano’s formula but are sufficiently terrible to not warrant space on this paper.  

 
\subsection*{Problem 5.}

Let $f(X) = X^3 - 2$ which is irreducible, monic and has $\Delta = -27 \cdot 2^2 = -108$ which is not a square in $\Q$. Therefore, the earlier discussion applies. Letting $E$ be the splitting field of $f$, we may immediately conclude that $\galgroup{E/\Q} \cong S_3$ and that the only intermediate fields are, $\Q(\sqrt{\Delta}) = \Q(\sqrt{-108}) = \Q(\sqrt{-3}) = \Q(\zeta_3)$ and $\Q(\alpha_1) = \Q(\sqrt[3]{2})$ and $\Q(\alpha_2) = \Q(\zeta_3 \sqrt[3]{2})$ and $\Q(\alpha_1) = \Q(\zeta_3^2 \sqrt[3]{2})$.

\subsection*{Problem 8.}
Let $f(X) = X^3 - X - 1$ so as before $\Delta = -23$. However, $\Delta$ is a square in $\Q(\sqrt{-23})$ since $\sqrt{-23} \in \Q$. Since $f$ is irredcible over $\Q$, by Lemma \ref{cubic}, $f$ is an irreducible seperable (because $\Q$ is perfect) cubic with square discriminant over $\Q(\sqrt{-23})$ so $\galgroup{E/\Q(\sqrt{-23})} \cong A_3 \cong \Z/3\Z$. However, $\Z/3\Z$ has no nontrivial subgroups and thus the extension $E/\Q(\sqrt{-23})$ has no nontrivial intermediate fields.


\subsection*{Problem 9.}
The discriminant of $f = X^3 - 10$ is $\Delta = -27 \cdot 10^2 = -2700$ which is not a square in $\Q(\sqrt{2})$ because $\Q(\sqrt{2}) \subset \R$ and therefore does not contain the root of any negtive number. Since $f$ is irredcible over $\Q$, by Lemma \ref{cubic}, $f$ is an irreducible seperable (because $\Q$ is perfect) cubic with non-square discriminant over $\Q(\sqrt{2})$ so $\galgroup{E/\Q(\sqrt{2})} \cong S_3$ where $E$ is the splitting field of $f$ over $\Q(\sqrt{2})$. 

\subsection*{Problem 10.}
Again, the discriminant of $f = X^3 - 10$ is $\Delta = -27 \cdot 10^2 = -2700$ which, this time, is a square in $\Q(\sqrt{-3})$ because $-2700 = (30 \sqrt{-3})^2$. Since $f$ is irredcible over $\Q$, by Lemma \ref{cubic}, $f$ is an irreducible seperable (because $\Q$ is perfect) cubic with square discriminant over $\Q(\sqrt{-3})$ so $\galgroup{E/\Q(\sqrt{-3})} \cong A_3$ where $E$ is the splitting field of $f$ over $\Q(\sqrt{2})$. 

\section*{Additional Problem.}

Let $f \in \Q[Y_1, Y_2, Y_3]$ be $f(Y_1, Y_2, Y_3) = Y_1^3 + Y_2^3 + Y_3^3 = u_1^3 - 3 u_1 u_2 + 3 u_3$. This is most easily checked by direct computation, 
\begin{align*}
u_1^3 - 3 u_1 u_2 + 3 u_3 & = (Y_1 + Y_2 + Y_3)^3 - 3 (Y_1 + Y_2 + Y_3) ( Y_1 Y_2 + Y_2 Y_3 + Y_1 Y_3) + 3 Y_1 Y_2 Y_3 \\ 
 & = Y_1^3 + Y_2^3 + Y_3^3 + 3 Y_1^2 Y_2 + 3 Y_1 Y_2^2 + 3 Y_2^2 Y_3 + 3 Y_2 Y_3^2 + 3 Y_1^2 Y_3 + 3 Y_1 Y_3^2 + 6 Y_1 Y_2 Y_3 \\
 & - 3(Y_1^2 Y_2 + Y_1 Y_2^2 + Y_2^2 Y_3 + Y_2 Y_3^2 + Y_1^2 Y_3 + Y_1 Y_3^2 + 3 Y_1 Y_2 Y_3) + 3 Y_1 Y_2 Y_3 \\
 & = Y_1^3 + Y_2^3 + Y_3^3    
\end{align*}
Thus, given a cubic $g(X) = X^3 + a X^2 + b X + c$ with roots $\alpha_1, \alpha_2, \alpha_3$, by Vieta, 
\begin{align*}
-a & = \alpha_1 + \alpha_2 + \alpha_3 = u_1(\alpha_1, \alpha_2, \alpha_3) \\ 
 b & = \alpha_1 \alpha_2 + \alpha_2 \alpha_3 + \alpha_1 \alpha_3 = u_2(\alpha_1, \alpha_2, \alpha_3) \\
-c & = \alpha_1 \alpha_2 \alpha_3 = u_3(\alpha_1, \alpha_2, \alpha_3)
\end{align*}
Since $f = u_1^3 - 3 u_1 u_2 + 3 u_3$, plugging in the roots, we get, 
\[\alpha_1^3 + \alpha_2^3 + \alpha_3^3 = f(\alpha_1, \alpha_2, \alpha_3) = (-a)^3 - 3 (-a) b + 3(-c) = -a^3 + 3ab - 3c\] 
Therefore, for $g(X) = X^3 - 2$ we have $a = b = 0$ and $c = -2$ so $\alpha_1^3 + \alpha_2^3 + \alpha_3^3 = (-3) \cdot (-2) = 6$. 

\section*{Lemmas}

\begin{lemma} \label{cubic}
An irreducible cubic over $\Q$ is irreducible over any quadratic extension $\Q(\sqrt{d})$. 
\end{lemma}
\begin{proof}
Suppose that $\Q(\sqrt{d})$ contained $\alpha$, a root of $f$ then $\Q(\alpha) \subset \Q(\sqrt{d})$ but since $f$ is irreducible over $\Q$, then $[\Q(\alpha) : \Q] = 3$ but $[\Q(\sqrt{d}) : \Q] = [\Q(\sqrt{d}) : \Q(\alpha)][\Q(\alpha) : \Q] \ge 3$. This contradicts the fact that $X^2 + d$ is the minimal polynomial of $\sqrt{d}$ and thus $[\Q(\sqrt{d}) : \Q] = 2$. Therefore, $f$ has no roots in $\Q(\sqrt{d})$ so $f$ is irreducible over $\Q(\sqrt{d})$ because otherwise it would split into a linear and a quadratic factor which would imply the existence of a root.
\end{proof}

\end{document}