\documentclass[12pt]{extarticle}
\usepackage{import}
\import{./}{Includes}


\begin{document}
\atitle{8}

\section*{Page 200. Problem 6.}
Let $\zeta$ be a primitive $n^{\mathrm{th}}$ root of unity in a field extension $E/K$. That is, $\zeta$ is a root of $X^n - 1$ and not of $X^k - 1$ for any $k < n$. Consider $G\subset E$, the set of all roots of the polynomial $X^n - 1$. Now, if $\alpha, \beta \in G$ then $(\alpha \beta)^n - 1 = \alpha^n \beta^n - 1 = 0$ because $\alpha^n = \beta^n = 1$. Furthermore, $\alpha \neq 0$ (since $0^n = 0 \neq 1$) so $\alpha^{-1} \in E$ and $(\alpha^{-1})^n = (\alpha^n)^{-1} = 1^{-1} = 1$ so $\alpha^{-1} \in E$. Therefore, $(G, \cdot)$ is a group and because each element satisfies $X^n - 1$ which has at most $n$ roots in $E$ then $|G| \le n$ so $G$ is finite. Therefore, as a finite multiplicative subgroup of a field, $G$ is cyclic. Suppose that $\zeta^i = \zeta^j$ with $i > j$, then $\zeta^{i - j} = 1$ but because $\zeta$ is primitive, $i - j > n$. Therefore, for $0 \le i, j < n$, the $n$ powers of $\zeta$ are all distinct meaning that the entire group is only powers of $\zeta$. Thus, $\zeta$ is a generator of $G$. \\\\
Consider the extension $K(\zeta)/K$. Because $G = \left< \zeta \right>$ and $K(\zeta)$ is a subfield of $E$, then $G \subset K(\zeta)$ so $K(\zeta)$ contains all the roots of $X^n - 1$. Furthermore, $K(\zeta) = K(G)$ because $\zeta \in G$ and $G \subset K(\zeta)$. Therefore, $K(\zeta)$ is the splitting field of $X^n - 1$ over $K$. Therefore, $K(\zeta)/K$ is a normal extension. The extension is also seperable because $K$ has characteristic zero. Therefore, $K(\zeta)/K$ is a Galois extension. Finally, consider $\galgroup{K(\zeta)/K}$ which is embedded in $S_n$ because $\deg{(X^n - 1)} = n$. Any Galois automorphism restricted to $G$ is a group automorphism of $G$ because it is multiplicative and permutes roots. Because the group of automorphisms of any cyclic group is abelian (see Lemma \ref{abelian}), for any $\sigma, \tau \in \galgroup{K(\zeta)/K}$ then $\sigma|_G \circ \tau|_G = \tau|_G \circ \sigma|_G$. However, the map from $\galgroup{E/K}$ into $S_n$ given by the action of $\galgroup{E/K}$ on $G$ (the roots of $X^n - 1$) is injective since $E = K(G)$ so if $\sigma$ fixes $G$ and $K$ then it is the idenitity on $E$. Because $\sigma \circ \tau$ and $\tau \circ \sigma$ acts indentically on $G$ by injectivity $\sigma \circ \tau = \tau \circ \sigma$. Therefore $\galgroup{E/K}$ is abelian.        

\section*{Probelm 2.}
Let $E$ be the splitting field of $X^3 - 2$ over $\Q$. Take $\alpha = \sqrt[3]{2}$ and $\zeta = \frac{-1 + \sqrt{-3}}{2}$. The Galois group $\galgroup{E/\Q}$ is isomorphic to $S_3 = \left< \sigma, \tau \right>$ with automorphisms acting as follows,
\begin{align*}
\sigma & : \zeta^i \alpha \mapsto \zeta^{i + 1} \alpha \\
\tau & : x \mapsto \bar{x}
\end{align*} 
which permute the roots of $X^3 - 2$. Now, we consider the element $AB \in E$ where $A = \alpha = \sqrt[3]{2}$ and $B = \sqrt{-3} = 2 \zeta + 1$. The six elements of $S_3$ are presented as $\sigma^i \tau^j$ for $0 \le i \le 2$ and $0 \le j \le 1$. Therefore, we can calculate the conjugtes of $AB$ by acting on the element $AB = (2 \zeta + 1)\alpha$ with the elements of the Galois group:
\begin{align*}
\sigma^0 \tau^0 ((2 \zeta + 1)\alpha) &= \id((2 \zeta + 1)\alpha) = (2 \zeta + 1)\alpha = \sqrt{-3} \cdot \sqrt[3]{2} \\
\sigma^0 \tau^1 ((2 \zeta + 1)\alpha) &= \tau((2 \zeta + 1)\alpha) = (2 \bar{\zeta} + 1)\alpha = - \sqrt{-3} \cdot \sqrt[3]{2} \\
\sigma^1 \tau^0 ((2 \zeta + 1)\alpha) &= \sigma((2 \zeta + 1)\alpha) = (2 \zeta + 1)\zeta \alpha = \zeta \sqrt{-3} \cdot \sqrt[3]{2} \\
\sigma^1 \tau^1 ((2 \zeta + 1)\alpha) &= \sigma((2 \bar{\zeta} + 1)\alpha) = (2 \bar{\zeta} + 1) \zeta \alpha = - \zeta \sqrt{-3} \cdot \sqrt[3]{2} \\
\sigma^2 \tau^0 ((2 \zeta + 1)\alpha) &= \sigma^2((2 \zeta + 1)\alpha) = (2 \zeta + 1) \zeta^2 \alpha = \zeta^2 \sqrt{-3} \cdot \sqrt[3]{2} \\
\sigma^2 \tau^1 ((2 \zeta + 1)\alpha) &= \sigma((2 \bar{\zeta} + 1)\alpha) = (2 \bar{\zeta} + 1) \zeta^2 \alpha = - \zeta^2 \sqrt{-3} \cdot \sqrt[3]{2} \\
\end{align*}

\section*{Lemmas}

\begin{lemma} \label{abelian}
Let $G$ be cyclic, then $\mathrm{Aut}(G)$ is abelian.
\end{lemma}
\begin{proof}
Take a generator $g \in G$. Now, let $\sigma, \tau : G \to G$ be automorphisms. Then $\sigma(g) = g^{k_\sigma}$ and $\tau(g) = g^{k_\tau}$ because every element in $G$ is a power of $g$. Thus, for any element $g^r \in G$,
\[\sigma \circ \tau(g^r) = \sigma(\tau(g)^r) = \sigma(g^{k_\tau r}) = \sigma(g)^{k_\tau r} = (g^{k_\sigma})^{k_\tau r} = g^{k_\sigma k_\tau r}\] 
\[\tau \circ \sigma(g^r) = \tau(\sigma(g)^r) = \tau(g^{k_\sigma r}) = \tau(g)^{k_\sigma r} = (g^{k_\tau})^{k_\sigma r} = g^{k_\tau k_\sigma r}\] 
However, $k_\sigma k_\tau = k_\tau k_\sigma$ because integer multiplication is commutative. Therefore, $\sigma \circ \tau(g^r) = \tau \circ \sigma(g^r)$. However, every element of $G$ is of the form $g^r$ so $\sigma \circ \tau = \tau \circ \sigma$. 
\end{proof}

\end{document}