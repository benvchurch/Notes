\documentclass[12pt]{extarticle}
\usepackage{import}
\import{./}{Includes}


\begin{document}
\atitle{8}

\section*{Page 196.} 
\subsection*{Problem 1.}
The minimal polynomial of $\sqrt[3]{s}$ is $X^3 - 2$. This must be minimal because any polynomial of lower degree is linear or quadratic which can always be solved by square roots. However, if $\sqrt[3]{2} = a + b \sqrt{d}$ with $a, b \in \Q$ then $(a + b \sqrt{d})^3 = 2$ so $a^3 + 3 a^2 b \sqrt{d} + 3 ab^2 d + b^3 d \sqrt{d} \in \Q$ which implies that $\sqrt{d} \in \Q$ so $\sqrt[3]{2} \in \Q$ which is a contradiction. One could also argue that $X^3 - 2$ is irreducible by Eisenstein's criterion because $2$ divides every subleadiing term but $4$ does not divide the constant term or leading. Thus, $X^3 - 2$ is the minimal polynomial of $\sqrt[3]{2}$ and we showed on assignment \# 7 that the splitting field of $X^3 - 2$ is $\Q(\sqrt[3]{2}, \zeta_3)$ in which $X^3 - 2 = (X - \sqrt[3]{2})(X - \zeta_3 \sqrt[3]{2})(X - \zeta_3^2 \sqrt[3]{2})$ so the conjugates of $\sqrt[3]{2}$ are $\sqrt[3]{2}$, $\zeta_3 \sqrt[3]{2}$, and $\zeta_3^2 \sqrt[3]{2}$.   

\subsection*{Problem 2.}
On assignment \# 6, I showed that the minimal polynomial of $\sqrt{2} + \sqrt{3}$ is,
\begin{align*}
X^4 - 10 X^2 + 1 = (X - (\sqrt{2} + \sqrt{3})) & (X + (\sqrt{2} + \sqrt{3}))(X - (\sqrt{2} - \sqrt{3}))(X + (\sqrt{2} - \sqrt{3})) 
\end{align*}
with roots: $\sqrt{2} + \sqrt{3}$, $\sqrt{2} - \sqrt{3}$, $-\sqrt{2} + \sqrt{3}$, $-\sqrt{2} - \sqrt{3}$ which are the conjugates of $\sqrt{2} + \sqrt{3}$. 

\subsection*{Problem 3.}
Let $F$ be normal over $K$ and $K \subset E \subset F$. Since $F/K$ is normal, for every $\alpha \in F$ the minimal polynomial $\minimal{\alpha}{K}$ splints in $F$. Now, let $q = \minimal{\alpha}{E}$ then because $K \subset E$ we have $\minimal{\alpha}{K} \in E[X]$ and has $\alpha$ as a root so $q \divides \minimal{\alpha}{K}$ in the ring $E[X]$. By unique factorization, since $\minimal{\alpha}{K}$ splits in $F$ then $q$ splits in $F$. Therefore, for any $\alpha \in F$ we have $\minimal{\alpha}{E}$ splits in $F$ so $F/E$ is normal. 

\subsection*{Problem 5.}
Suppose that $K$, $F$, and $E$ are contained in a larger field $L$. Let $E$ and $F$ be normal over $K$. Take $\alpha \in E \cap F$ then $\alpha \in E$ and $\alpha \in F$. Because both are normal extensions, the minimal polynomial $q =\minimal{\alpha}{K}$ splits in both $E$ and $F$ and thus splits in $L$. Thus, $q(X) = c(X - \alpha_1) \cdots (X - \alpha_n)$ with $\alpha_i \in L$. However, $q$ splits in both $F$ and $E$ and the roots cannot be different because then $q$ would spit in multiple ways inside $L$ which would contradict unqiue factorization. Thus, $\alpha_i \in E$ and $\alpha_i \in F$ so $\alpha_i \in E \cap F$. Thus, all the roots of $q$ are contained in $E \cap F$ so $q$ splits in $E \cap F$. Thus, $E \cap F$ is normal over $K$. 

\section*{Page 200.} 
\subsection*{Problem 1.} 
Let $F_1$ and $F_2$ be intermediate fields of a Galois extension $E/K$ with corresponding subgroups of $\galgroup{E/K}$ given by $H_1$ and $H_2$. Suppose that $F_1 \subset F_2$ then take $\sigma \in H_2$. Now, $\sigma$ fixes $F_2$ and therefore fixes $F_1 \subset F_2$. Thus, $\sigma \in H_1$ the subgroup of automorphisms fixing $F_1$. Thus, $H_2 \subset H_1$. Conversely, suppose that $H_2 \subset H_1$ then take $\alpha \in F_1$. Now, $\forall \sigma \in H_1 : \sigma(\alpha) = \alpha$ and $H_2 \subset H_1$ so $\forall \sigma \in H_2 : \sigma(\alpha) = \alpha$ so $\alpha \in E^{H_2} = F_2$. Thus, $F_1 \subset F_2$.   
\subsection*{Problem 2.}
Let $F_1$, $F_2$, and $F_3$ be intermediate fields of a Galois extension $E/K$ with corresponding subgroups of $\galgroup{E/K}$ given by $H_1$, $H_2$, and $H_3$. Suppose that $F_1 = F_2 F_3$ then $F_2, F_3 \subset F_1$ so $H_2, H_3 \supset H_1$ so $H_1 \subset H_2 \cap H_3$. Now, the subgroup $H' = H_1 \cap H_2 \subset H_2, H_3$ so $E^{H'} \supset F_2$ and $E^{H'} \supset F_3$. Therefore, $E^{H'} \supset F_2 F_3 = F_1$ and thus $H' \subset H_1$ so $H_1 = H_2 \cap H_3$. \\\\
Conversely, let $H_1 = H_2 \cap H_3$. Then, $H_1 \subset H_2, H_3$ so $F_2, F_3 \subset F_1$ and thus $F_2 F_3 \subset F_1$. Now, take $L = F_2 F_3$ which satisfies $L \supset F_2$ and $L \supset F_3$ so $H_L = \galgroup{E/L}$ satisfies $H_L \subset H_2$ and $H_L \subset H_3$ so $H_L \subset H_2 \cap H_3 = H_1$ so $L \supset F_1$ and thus $F_1 = F_2 F_3$. 

\subsection*{Problem 3.}
Let $F_1$, $F_2$, and $F_3$ be intermediate fields of a Galois extension $E/K$ with corresponding subgroups of $\galgroup{E/K}$ given by $H_1$, $H_2$, and $H_3$. Suppose that $F_1 = F_2 \cap F_3$ then $F_1 \subset F_2, F_3$ so $H_1 \supset H_2, H_3$ so $H_1 \supset \left< H_2 \cup H_3 \right>$. Now, the subgroup $H' = \left< H_2 \cup H_3 \right> \supset H_2, H_3$ so $E^{H'} \subset F_2$ and $E^{H'} \subset F_3$. Therefore, $E^{H'} \subset F_2 \cap F_3 = F_1$ and thus $H' \supset H_1$ so $H_1 = \left< H_2 \cup H_3 \right>$. \\\\ 
Conversely, let $H_1 = \left< H_2 \cup H_3 \right>$. Then, $H_1 \supset H_2, H_3$ so $F_1 \subset F_2, F_3$ and thus $F_1 \subset F_2 \cap F_3$. Now, take $L = F_2 \cap F_3$ which satisfies $L \subset F_2$ and $L \subset F_3$ so $H_L = \galgroup{E/L}$ satisfies $H_L \supset H_2$ and $H_L \supset H_3$ so $H_l \supset \left< H_2 \cup H_3 \right> = H_1$ so $L \subset F_1$ and thus $F_1 = F_2 \cap F_3$. 

\subsection*{Problem 5.}   

The extension $\finfield{p^n}/\finfield{p}$ is normal because $\finfield{p^n}$ is the splitting field of $X^{p^n} - X$ over $\finfield{p}$. consider the Frobenius map, $\sigma : \finfield{p^n} \to \finfield{p^n}$ given by $\sigma : x \mapsto x^p$. Because $\finfield{p^n}$ has characteristic $p$, this is a field homomorphism and therefore is injective. However, because the field is finite, the map is also surjective and thus an automorphism. The extension is seperable because $\finfield{p}$ is perfect since the Frobenius is surjective. Thus, the extension is Galois so $|\galgroup{\finfield{p^n} /\finfield{p}}| = [\finfield{p^n} : \finfield{p}] = n$. However, by Lagrange, $\forall x \in \finunits{p} : x^{p-1} = 1$ so $x^p = x$. This equation is also satisfied by $x = 0$ so for any $x \in \finfield{p}$ we have $\sigma(x) = x$. Thus, $\sigma \in \galgroup{\finfield{p^n}/\finfield{p}}$. Suppose that $\sigma^k = \id$. Then, $\forall x \in \finfield{p^n} : \sigma^k(x) = x^{p^k} = x$. Therefore, every element of $\finfield{p^n}$ is a root of $X^{p^k} - X$. However, in any field, this polynomial has at most $p^k$ roots. Thus, $p^n \le p^k$ so $n \le k$. Therefore, the order of $\sigma$ is at least $n$. However, the order of $\galgroup{\finfield{p^n}/\finfield{p}}$ is $n$ so $\sigma$ must be a generator of the group and therefore the Galois group is cyclic with order $n$.  
        

\end{document}