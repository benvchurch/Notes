\documentclass[12pt]{extarticle}
\usepackage{import}
\import{./}{Includes}


\begin{document}
\atitle{5}
\textbf{Page 159.} 
\begin{enumerate}
\item[1.] Let $\{F_i \mid i \in I \}$ be an indexed family of subfields of $K$. The set 
\[ F = \bigcap_{i \in I} F_i\] is contained in $K$ so the associative, commutative,  and distributive properties are inherited from the field properties of $K$. It suffices to check that $F$ contains $0$ and $1$ and is closed under addition, multiplication, and inverses. Since every $F_i$ is a field, each $F_i$ satisfies these properties. In particular, $0, 1 \in F_i$. Also if $x, y \in F$ then by the definition of the intersection, $x, y \in F_i$ for every $i \in I$. Thus, $x + y, xy , -x , x^{-1} \in F_i$. Since this holds for every $i \in I$ then these elements also appear in the intersection. Thus, $x + y, xy, -x, x^{-1} \in F$ so $F$ is a field. 

\item[3.] Let $K, L, M$ be subfields of $F$. Consider the subfield $K(LM)$ which is the smallest subfield of $F$ which contains $K$ and $LM$. Similarly, $LM \supset L$ and $LM \supset M$ therefore, $K(LM)$ contains $K$, $L$, and $M$. By the definition of the compositum, any subfield that contains $K$ and $L$ must contain $KL$ because $KL$ is the intersection of all such subfields. Therefore, $K(LM) \supset KL$ but $K(LM)$ also contains $M$ so by identical reasoning, $K(LM) \supset (KL)M$. \\ \\ The converse proceeds identically. The field $(KL)M$ contains both the fields $KL$ and $M$. Also, $KL$ contains $K$ and $L$. Thus, $(KL)M$ contains $K$, $L$, and $M$. Thus, because $LM$ is the minimal field containing $L$ and $M$, $(KL)M \supset LM$. However, $(KL)M \supset K$ so $(KL)M \supset K(LM)$. Therefore, $(KL)M = K(LM)$.        
\end{enumerate}

\textbf{Page 163.} 
\begin{enumerate}
\item[7.] Let $F = \Z/2\Z$ which is a field because $2$ is a prime. I claim that $f = X^2 + X + 1$ has no roots in $F$. This is easily checked because $F$ is finite: $f(0) = \mod{0^2 + 0 + 1}{1}{2}$ and $f(1) = \mod{1^2 + 1 + 1}{1}{2}$. From problem \# 9 on assignment \# 3, we know that any degree two polynomial over $F$ is irreducible iff it has no roots in $F$. Thus, $f$ is irreducible over $F$ and therefore, $E = F[X]/(X^2 + X + 1)$ is a field. We know that $[E : F] = 2$ because $\deg{f} = 2$ with $\{1, X\}$ forming a basis of $E$ over $F$. Since $F$ contains $2$ elements, $E$ contains 4 elements, namely, $0, 1, X, 1 + X$. We explicitly exhibit their addition and multiplication tables below. \\ \\ \\
\begin{center}
\textbf{Addition} \\
\begin{tabular}{ c | c c c c}
  & $0$ & $1$ & $X$ & $1 + X$ \\
\hline
 $0$ & $0$ & $1$ & $X$ & $1 + X$ \\ 
 $1$ & $1$ & $0$ & $1 + X$ & $X$ \\  
 $X$ & $X$. & $1 + X$ & $0$ & $1$ \\
 $1 + X$ & $1 + X$ & $X$ & $1$ & $0$ 
\end{tabular}
\end{center}

\begin{center}
\textbf{Multiplication} \\
\begin{tabular}{ c | c c c c}
  & $0$ & $1$ & $X$ & $1 + X$ \\
\hline
 $0$ & $0$ & $0$ & $0$ & $0$ \\ 
 $1$ & $0$ & $1$ & $X$ & $1 + X$ \\  
 $X$ & $0$ & $X$ & $1 + X$ & $1$ \\
 $1 + X$ & $0$ & $1 + X$ & $1$ & $X$ 
\end{tabular}
\end{center}
$(E, +) \cong \Z/2\Z \times \Z/2\Z$ and $(E^\times, \cdot) \cong \Z/3\Z$.

\item[4. ] Let $S = \{i, \sqrt{2}\}$ and consider $\Q(S) \subset \C$. By definition, $\Q(S)$ must contain $\Q, i$ and $\sqrt{2}$ and thus by closure, $\Q(S)$ must contain $i \sqrt{2}$ and all $\Q$-linear combinations of the elements $\{1, i, \sqrt{2}, i \sqrt{2}\}$. If we prove that the set of all such combinations, $L$, is a field, then it must be $\Q(S)$ because $\Q(S)$ is the smallest field containing this set. \\ \\
Clearly, $L$ is closed under addition and contains additive inverses. It remains to check multiplicative closure and inverses. Take $x, y \in L$ then $x = a + ib + c \sqrt{2} + i d  \sqrt{2}$ and $x = a' + i b' + c' \sqrt{2} + i d' \sqrt{2}$ with constants in $\Q$. Now, 
\begin{align*}
xy & = aa' + i ab' + a c' \sqrt{2} + i a d' \sqrt{2} + i b a' - b b' + i b c' \sqrt{2} - b d \sqrt{2} \\  & + ca' \sqrt{2} + i cb' \sqrt{2} + 2 c c' + 2 i c d' + i d a' \sqrt{2} - d b' \sqrt{2} + 2i d c' - 2 dd' \\
 & = (aa' - bb'+ 2cc' - dd') + i(ab' + ba' + 2 cd' + 2idc') \\ & + (ac' - bd + ca'- db') \sqrt{2} + i (ad' + bc' + cb' + da') \sqrt{2} \in L 
\end{align*}
Thus, $L$ is closed under multiplcation. It remains to prove that $L$ contains multiplicative inverses.  

\begin{align*}
x^{-1} & = \frac{1}{(a + c \sqrt{2}) + i (b +  d \sqrt{2})} = \frac{(a + c \sqrt{2}) - i (b +  d \sqrt{2})}{(a + c \sqrt{2})^2 + (b +  d \sqrt{2})^2} \\ 
& = \frac{(a + c \sqrt{2}) - i (b +  d \sqrt{2})}{(a^2 + b^2 + c^2 + d^2) + (2ac + 2bd)\sqrt{2}} \\ 
& = \frac{\left[(a + c \sqrt{2}) - i (b +  d \sqrt{2})\right] \left[ (a^2 + b^2 + c^2 + d^2) - (2ac + 2bd)\sqrt{2} \right] }{(a^2 + b^2 + c^2 + d^2)^2 - 2 (2ac + 2bd)^2} \in L
\end{align*}
The final inclusion holds because by closure under multiplication the numerator is in $L$ and the denominator is in $\Q$. Futhermore, the denominator cannot be zero unless \[(a^2 + b^2 + c^2 + d^2)/(2ac + 2bd) = \sqrt{2}\] which is impossible because $\sqrt{2}$ is irrational. Thus, $L$ is a field contaning $\Q, i, \sqrt{2}$ with $L \subset \Q(S)$ and therefore $\Q(S) = L$. Furthermore, the set we have exhibited is a basis of $\Q(S)$ over $\Q$. By the definition of $L = \Q(S)$, the set $B = \{1, i, \sqrt{2}, i \sqrt{2}\}$ spans $\Q(S)$. Suppose that \[a + ib + c \sqrt{2} + i d \sqrt{2} = 0\] then by properties of complex numbers, \[a + c \sqrt{2} = 0 \text{ and } b + d \sqrt{2} = 0\] However, if $c \neq 0$ then $\sqrt{2} = - \frac{a}{c} \in \Q$ contradicting its irrationality. Thus, $c = 0$ so $a  = 0$. Similarly, if $d \neq 0$ then $\sqrt{2} = - \frac{b}{d} \in \Q$ so $b = d = 0$. Thus, $B$ is linearly independent. Therefore, $B$ is a basis of $\Q(S)$ over $\Q$ so $[\Q(S) : \Q] = 4$.    
\end{enumerate}

\end{document}