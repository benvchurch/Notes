\documentclass[12pt]{extarticle}
\usepackage{import}
\import{./}{Includes}


\begin{document}
\atitle{3}
\textbf{Page 138.} 
\begin{enumerate}
\item[2.] It suffices to show that $\Z[i]$ is a Euclidean Domain and then apply problem 3 to conclude that $\Z[i]$ is a PID. Define $N : \Z[i] \rightarrow \N$ by $N(a + ib) = |a+ib|^2 = a^2 + b^2$. By Lemma \ref{normprop}, $N$ extends to a function $\Q[i] \rightarrow \Q^+ \cup \{0 \}$ which is totally multiplicative.  


Take $\alpha, \beta \in \Z[i]$ with $\beta \neq 0$. Then since $\Q[i] \cong Q_{Z[i]}$ we have $\frac{\alpha}{\beta} \in \Q[i]$ so $\frac{\alpha}{\beta} = p + iq$ with $p, q \in \Q$. Now, consider the best integer approximations of $p$ and $q$, namely, $n$ and $k$ s.t. $|p - n| \le \frac{1}{2}$ and $|q - n| \le \frac{1}{2}$. These approximations exist by Lemma \ref{approx}. Let $\gamma = n + ik \in \Z[i]$ and $\delta = \alpha - \beta \gamma \in \Z[i]$. Consider,
\[N\left(\delta \right) = N(\alpha - \beta \gamma) = N(\beta) N\left(\frac{\alpha}{\beta} - \gamma \right) = N(\beta) \left[(p - n)^2 + (q - k)^2 \right] \le N(\beta) \left(\frac{1}{4} + \frac{1}{4} \right) < N(\beta) \] 
Thus, $\alpha = \beta \gamma + \delta$ with $N(\delta) < N(\beta)$ and $\gamma, \delta \in \Z[i]$ so $\Z[i]$ is a Euclidean Domain.

\item[3.] Let $R$ be a Euclidean Domain with a function $\varphi : R \sm \{0_R\} \rightarrow \N$. Explicitly, for every $a, b \in R$ with $b \neq 0$ there exists $q, r \in R$ s.t. $a = bq + r$ and either $r = 0$ or $\varphi(r) < \varphi(q)$. \smallskip

Consider an ideal $I \subset R$. If $I = (0)$ then it is principal. Otherwise, $\varphi(I \sm \{0_R\}) \subset \N$ is not empty so by well-ordering, it has a least element $k$. Since $g \in \varphi(I \sm \{0_R\})$, $\exists g \in I$ s.t. $\varphi(g) = k$ and $g \neq 0_R$. Thus, for any $a \in I$, by the Euclidean property, $\exists q, r \in R$ s.t. $a = gq + r$. Now $r = a - gq \in I$ because $a, g \in I$ and $I$ is an ideal. However, unless $r = 0$, $\varphi(r) < \varphi(g)$ which is a contradiction because $r \in I$ and $\varphi(g) = k$ the least element of $\varphi(I \sm \{0_R\})$. Therefore, $r = 0$ so $g \divides a$. Thus, $\forall a \in I : a \in (g)$ so $I \subset (g)$ and because $g \in I$ by closure and absorption we also have that $(g) \subset I$. Therefore, $I = (g)$ so every ideal is principal.  

\item[9.] Suppose $K$ is a field and a polynomial $f \in K[X]$ has degree $2$ or $3$.
If $f$ has a root $\alpha$ in $K$ then $X - \alpha \divides f$. Thus, $f = (X - \alpha) g$. Also, $\deg{f} = \deg{\left(X-a\right)} + \deg{g} = 2$ or $3$ so $\deg g = 1$ or $2$. Therefore, $g$ is not a unit because $K$ is a domain and thus the only units of $K[X]$ are the units of $K$ which have degree $0$. Thus, $f$ is reducible in $K[X]$. \smallskip

Consider $\deg{f} = 2$ or $3$ and $f$ is reducible. Then $f = gh$ for $g, h \in K[X]$ and $g,h \notin K[X]^\times$. However, $K$ is a field so $K[X]^\times = K^\times = K \sm \{0\}$. Therefore, $\deg{g}, \deg{h} \ge 1$ but $\deg{g} + \deg{h} = \deg{f} = 2$ or $3$. The only solutions are $\deg{g} = 1$, $\deg{h} = 2$ or $\deg{g} =  2$, $\deg{h} = 1$. WLOG take $\deg{g} = 1$ and $\deg{h} = 2$. Thus $g = aX + b$ for $a, b \in K$ and $a \neq 0$. Now since $K$ is a field and $a \neq 0$ then $-\frac{b}{a} \in K$ so consider \[f \left(-\frac{b}{a} \right) = \left( -\frac{b}{a} a + b \right) h \left(- \frac{b}{a} \right) = 0 \cdot h \left( -\frac{b}{a} \right) = 0\] Thus, $f$ has a root in $K$. Therefore, $f$ has a root in $K$ iff $f$ is reducible in $K[X]$. Equivalently, $f$ is irreducible in $K[X]$ iff $f$ has no roots in $K$. \smallskip

\item[10.] $X^5 + X^3 - X^2 - 1 = X^3(X^2 + 1) - (X^2 + 1) = (X^3 - 1)(X^2 + 1) = (X - 1)(X^2 + X + 1)(X^2 + 1)$ \\ The first factor is irreducible over $\R$ because it has degree $1$. The other two factors are irreducible over $\R$ by problem 9 because they have degree $2$ and no roots in $\R$. This can be shown by considering the discriminant $\Delta = b^2 - 4ac$ which cannot be negative if the quadratic equation has roots in $\R$. However, $\Delta_{X^2 + X + 1} = 1 - 4 = -3 < 0$ and $\Delta_{X^2 + 1} = 0 - 4 = - 4 < 0$.

\end{enumerate}

\section*{Lemmas}

\begin{lemma} \label{normprop} $N : \Q[i] \rightarrow \Q^+ \cup \{0\}$ given by $N(p + iq) = p^2 + q^2$ is multiplicative and $\Im{N|_{Z[i]}} \subset \N$.
\end{lemma}
\begin{proof}
Take $\alpha = 1 + i q_1, \beta = p_2 + i q_2 \in \Q[i]$ then $N(p_1 + i q_1) = p_1^2 + q_1^2 \in \Q^+ \cup \{0\}$. Thus, 
\begin{align*}
N(\alpha \beta) &= N(p_1 p_2 - q_1 q_2 + i(p_1 q_2 + p_2 q_1)) = (p_1 p_2 - q_1 q_2)^2 + (p_1 q_2 + p_2 q_1)^2 \\ &= p_1^2 p_2^2 - 2 p_1 p_2 q_1 q_2 + q_1^2 q_2^2 + p_1^2 q_2^2 + 2 p_1 q_2 p_2 q_1 + p_2^2 q_1^2 \\ &= p_1^2 p_2^2 + q_1^2 q_2^2 + p_1^2 q_2^2 + p_2^2 q_1^2 = (p_1^2 + q_1^2)(p_2^2 + q_2^2) = N(\alpha) N(\beta)
\end{align*}
Finally, if $\alpha \in \Z[i]$ then $\alpha = a + ib$ with $a, b \in \Z$ so $N(\alpha) = a^2 + b^2 \in \N$.
\end{proof}

\begin{lemma} \label{approx} $\forall r \in \R : \exists z \in \Z$ s.t. $|z - r| \le \frac{1}{2}$. In particular, this holds for $r \in \Q$.
\end{lemma}
\begin{proof}
Consider $S = \{n \in \Z \mid r < n + 1 \}$. $S$ is non-empty because $\Z$ is unbounded but $S$ is bounded below by $r$ so by well ordering, $S$ has a least element $z$. Since $z \in S$, $r < z + 1$. Suppose that $r < z$ then $z - 1 \in S$ contradicting the fact that $z$ is the least element. Thus, $z \le r < z + 1$. \bigskip \\
Now if $|r - z| < \frac{1}{2}$ then we are done. Else, $|r - z| = r - z \ge \frac{1}{2}$ so $1 -  \frac{1}{2} \ge z + 1 - r$ so $(z + 1) - r \le \frac{1}{2}$. However, $z + 1 > r$ so $|(z + 1) - r| \le \frac{1}{2}$ and $z + 1 \in \Z$. 
\end{proof}

\end{document}