\documentclass[12pt]{extarticle}
\usepackage{import}
\import{./}{Includes}

\begin{document}
\atitle{1}
 
\begin{enumerate}
\item[p.108 - 109]
\begin{enumerate}
\item[2] Let $A$ be an abelian group and $\End{A} = \{ f : A \rightarrow A \mid f \text{ is a homomorphism} \}$. \\ \\
Now take $f,g \in \End{A}$ then $(f+g)(x + y) = f(x+y) + g(x+y) = f(x) + f(y) + g(x) + g(y) = (f(x) + g(x)) + (f(y) + g(y))$ since $A$ is abelian and $f,g$ are homomorphisms. \\ \\
Let $0_{End} \in \End{A}$ given by $0_{End}(x) = 0_A$ is a homomorphism because $0_{End}(x + y) = 0_A = 0_A + 0_A = 0_{End}(x) + 0_{End}(y)$ and $(0_{End} + f)(x) = 0_{End}(x) + f(x) = 0_A + f(x) = f(x)$ and $(f + 0_{End})(x) = f(x) + 0_{End}(x) = f(x) + 0_A = f(x)$ (Identity) \\ \\ 
$(f + g)(x) = f(x) + g(x) = g(x) + f(x) = (g + f)(x)$ (Commutativity) \\ \\
Now define $-f$ by $(-f)(x) = -f(x)$, $(-f)(x + y) = -f(x+y) = -(f(x) + f(y)) = -f(x) + (-f(y))$ so $-f \in \End{A}$ also $(-f + f)(x) = -f(x) + f(x) = 0_A$ and $(f + (-f))(x) = f(x) + (-f(x)) = 0_A$.  (Inverses) \\ \\
Let $f,g,h \in \End{A}$ then $((f + g) + h)(x) = (f + g)(x) + h(x) =  f(x) + g(x) + h(x) = f(x) + (g + h)(x) = (f + (g + h))(x)$ \\ \\
$((f \circ g) \circ h)(x) = (f \circ g)(h(x)) = f(g(h(x))) = f((g \circ h)(x)) = (f \circ (g \circ h))(x)$ (Associativity) \\ \\ 
Let $1_{End} \in \End{A}$ given by $1_{End}(x) = x$ is a homomorphism because $1_{End}(x + y) = x + y = 1_{End}(x) + 1_{End}(y)$ (Identity) also $(1_{End} \circ f)(x) = 1_{End}(f(x)) = f(x)$ and $(f \circ 1_{End})(x) = f(x)$ thus $1_{End} \circ f = f \circ 1_{End} = f$ 
(Multiplicative Identity) \\ \\
Let $f,g,h \in \End{A}$ then $(f \circ (g + h))(x) = f((g + h)(x)) = f(g(x) + h(x)) = f(g(x)) + f(h(x)) = (f \circ g)(x) + (f \circ h)(x)$. Also $((f + g) \circ h)(x) = (f + g)(h(x)) = f(h(x)) + g(h(x)) = (f \circ h)(x) + (g \circ h)(x)$. (Distributive Law) Thus $(\End{A}, + , \circ )$ is a ring.

\item[3] Let $R$ be a ring. Then $U(R) = \{ u \in R \mid \exists v \in R : uv = vu = 1_R \}$. \\ Now $1_R \cdot 1_R = 1_R$ so $1_R \in U(R)$ (Identity) \\ \\ 
If $u, v \in U(R)$ then $\exists u', v' \in R : uu' = u'u = 1_R = vv' = v'v$ Thus $uv \cdot (v'u') = u(vv')u' = uu' = 1_R$ and $(v'u') \cdot uv = v'(u'u)v = 1_R$ so $uv \in R$. (Closure) \\ \\
If $u \in U(R)$ then $\exists v \in R : uv = vu = 1_R$ so $v \in U(R)$ and $vu = uv = 1_R$ (Inverses) \\ \\
Furthermore, $U(R)$ is a subset of $R$ and therefore inherents associativity.  

\item[4] Let $u \in R$ be a unit then $\exists v \in R : uv = vu = 1_R$ so take $ x = y = v$. \\ \\ Let $\exists x, y \in R : xu = uy = 1_R$ then $x(uy) = x$ but $x(uy) = (xu)y = y$ so $x = y$ thus $ux = xu = 1_R$ so $u \in U(R)$. 

\item[7] $\Z[i] = \{a + ib \mid a, b \in \Z \}$. Then for $z_i, z_2 \in \Z[i]$ we must check that $z_1 + z_2, z_1 \cdot z_2 \in \Z[i]$ and $-z_1, 1_{\Z[i]}, 0_{\Z[i]} \in \Z[i]$. Associativity (of both addition and multiplication), Distributivity, and Commutativity of addition are inherited from $\C$. \\ \\
If $z_1, z_2 \in \Z[i]$ then $z_1 = a_1 + ib_1$ and $z_2 = a_2 + ib_2$ with $a_1, a_2, b_1, b_2 \in \Z$. Then $z_1 + z_2$ = $(a_1 + a_1) + i(b_1 + b_2) \in \Z[i]$ because $a_1 + a_2, b_1 + b_2 \in \Z[i]$ Also, $z_1 \cdot z_2 = (a_1 a_2 - b_1 b_2) + i(a_1b_2  + a_2 b_1) \in \Z[i]$ because $a_1 a_2 - b_1 b_2 \in \Z$ and $a_1b_2  + a_2 b_1 \in \Z$. Therefore, $1 = 1+i0 \in \Z[i]$ takes $1 \cdot z_1 = z_1 \cdot 1 = z_1$ and $0 = 0 + i0 \in \Z[i]$ takes $0 + z_1 = z_1 + 0 = z_1$. Also $-z_1 = -a_1 - i b_1 \in \Z[i]$ then $z_1 + (-z_1) = a_1 - a_1 + i(b_1 - b_1) = 0$. By Commutativity, we don't neet to check the other direction. \\ \\
If $z \in U(\Z[i])$ then $zz' = 1$ so $|z|^2 |z'|^2 = 1$ i.e. $(a^2 + b^2)(a'^2 + b'^2) = 1$ so $a^2 + b^2 \mid 1$ and thus $a^2 + b^2 = 1$ since both are in $\mathbb{N}$. If $|a| > 1$ then $b^2 < 0$ so $a = 0, \pm 1$ and $b = \pm 1, 0$ so the units are $1,-1, i,-i$. 

\end{enumerate}

\item[p.112]
\begin{enumerate}
\item[10] Let $A, B$ be ideals in $R$. Then $AB = \left(\{ ab \mid a \in A \text{ and } b \in B \}\right) = $ \[ \left\{\sum_{i = 1}^n x_i (a_ib_i) y_i \: \Big| \: a_i \in A \text{ and } b_i \in B \text{ and } x_i,y_i \in R \right\} \] If $r \in AB$ then $r = \sum_{i = 1}^n x_i(a_i b_i)y_i$ but $x_i (a_i b_i) y_i = (x_i a_i) (b_i y_i)$ and since $A$ and $B$ are ideals then $(x_i a_i) = a_i' \in A$ and $(b_i y_i) = b_i' \in B$. Thus, $r = \sum_{i = 1}^n a_i' b_i'$. \\ \\ Also for $r = \sum_{i = 1}^n a_i b_i$ take $x_i = y_i = 1_R$ then $r = \sum_{i = 1}^n x_i(a_i b_i)y_i$ so \[AB = \left\{\sum_{i = 1}^n a_i b_i \: \Big| \: a_i \in A \text{ and } b_i \in B \right\}\]

\item[11] Let $x \in (AB)C$ then $\sum_{i = 1}^n g_i c_i$ s.t. $g_i \in AB \text{ and } c_i \in C$ with $g_i = \sum_{j = 1}^{k_i} a_{ij} b_{ij}$ so $x = \sum_{i = 1}^n \sum_{j = 1}^{k_i} a_{ij}b_{ij}c_i $ distributing and reparametrizing, $x = \sum_{k = 1}^r a_k b_k c_k$ so $x \in \{ \sum_{i = 1}^r a_i b_i c_i \mid a_i \in A, b_i \in B, c_i \in C\} = S$. Also if $x \in S$ then $x = \sum_{i = 1}^r (a_i b_i) c_i$ but $a_i b_i \in AB$ so $x \in (AB)C$ thus $(AB)C = S$. \\ \\
Let $x \in A(BC)$ then $\sum_{i = 1}^n a_i g_i$ s.t. $g_i \in BC \text{ and } a_i \in A$ with $g_i = \sum_{j = 1}^{k_i} b_{ij} c_{ij}$ so $x = \sum_{i = 1}^n \sum_{j = 1}^{k_i} a_{i}b_{ij}c_{ij} $ distributing and reparametrizing, $x = \sum_{k = 1}^r a_k b_k c_k$ so $x \in S$. Also if $x \in S$ then $x = \sum_{i = 1}^r a_i (b_i c_i)$ but $b_i c_i \in BC$ so $x \in A(BC)$ thus $A(BC) = S = (AB)C$.

\item[12] Let $A$, $B$, and $C$ be ideals in $R$ and $x \in A(B+C)$ \\ then $x = \sum_{i = 1}^n a_i (b_i + c_i) = \sum_{i = 1}^n (a_i b_i + a_i c_i) = \sum_{i = 1}^n a_i b_i + \sum_{i = 1}^n a_i c_i$ therefore, $x \in AB + AC$. \\ \\
Now let $x \in AB + AC$ then $x = \sum_{i = 1}^n a_i b_i + \sum_{i = 1}^{n'} a_i' c_i'$ \\ \\ Define: \\ \\ $\tilde{a}_i = \begin{cases} a_i & 1 \leq i \leq n \\ a_{i-n}' & n < i \leq n' \end{cases}$ \quad $\tilde{b}_i = \begin{cases} b_i & 1 \leq i \leq n \\ 0_R & n < i \leq n' \end{cases}$ \quad $\tilde{c}_i = \begin{cases} 0_R & 1 \leq i \leq n \\ c_{i-n}' & n < i \leq n' \end{cases}$ \\ \\ \\ then  $\sum_{i_1}^{n+n'} \tilde{a}_i (\tilde{b}_i + \tilde{c}_i) = \sum_{i = 1}^{n} a_i (b_i + 0_R) + \sum_{i = n + 1}^{n+n'} a_{i-n}' (0 + c_{i-n}') = \\ \sum_{i = 1}^{n} a_i b_i + \sum_{i = 1}^{n'} a_i' c_i' = x$ thus $x \in A(B+C)$ \\ \\ Therefore $A(B+C) = AB + AC$ \\ \\
Take $x \in (A+B)C$ then $x = \sum_{i = 1}^n (a_i  + b_i) c_i = \sum_{i = 1}^n (a_i c_i + b_i c_i) = \sum_{i = 1}^n a_i c_i + \sum_{i = 1}^n b_i c_i$ therefore, $x \in AC + BC$. \\ \\
Now let $x \in AC + BC$ then $x = \sum_{i = 1}^n a_i c_i + \sum_{i = 1}^{n'} b_i' c_i'$. \\ \\ Define: \\ \\
$\tilde{c}_i = \begin{cases} c_i & 1 \leq i \leq n \\ c_{i-n}' & n < i \leq n' \end{cases}$ \quad $\tilde{a}_i = \begin{cases} a_i & 1 \leq i \leq n \\ 0_R & n < i \leq n' \end{cases}$ \quad $\tilde{b}_i = \begin{cases} 0_R & 1 \leq i \leq n \\ b_{i-n}' & n < i \leq n' \end{cases}$ \\ \\ \\
then  $\sum_{i_1}^{n+n'} (\tilde{a}_i  + \tilde{b}_i) \tilde{c}_i = \sum_{i = 1}^{n} (a_i + 0_R) c_i + \sum_{i = n + 1}^{n+n'} (0 + b_{i-n}') c_{i-n}' = \\ \sum_{i = 1}^{n} a_i c_i + \sum_{i = 1}^{n'} b_i' c_i' = x$ thus $x \in (A+B)C$\\ \\ Therefore $(A+B)C = AC + BC$
\end{enumerate}

\end{enumerate}


\end{document}