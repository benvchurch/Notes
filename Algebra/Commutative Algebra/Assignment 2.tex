\documentclass[12pt]{extarticle}
\usepackage[utf8]{inputenc}
\usepackage[english]{babel}
\usepackage[a4paper, total={6in, 9in}]{geometry}
\usepackage{tikz-cd}
 
\usepackage{amsthm, amssymb, amsmath, centernot}


\newcommand{\notimplies}{%
  \mathrel{{\ooalign{\hidewidth$\not\phantom{=}$\hidewidth\cr$\implies$}}}}

\renewcommand\qedsymbol{$\square$}
\newcommand{\cont}{$\boxtimes$}
\newcommand{\divides}{\mid}
\newcommand{\ndivides}{\centernot \mid}
\newcommand{\Z}{\mathbb{Z}}
\newcommand{\N}{\mathbb{N}}
\newcommand{\C}{\mathbb{C}}
\newcommand{\Zplus}{\mathbb{Z}^{+}}
\newcommand{\Primes}{\mathbb{P}}
\newcommand{\ball}[2]{B_{#1} \! \left(#2 \right)}
\newcommand{\Q}{\mathbb{Q}}
\newcommand{\R}{\mathbb{R}}
\newcommand{\Rplus}{\mathbb{R}^+}
\newcommand{\invI}[2]{#1^{-1} \left( #2 \right)}
\newcommand{\End}[1]{\text{End}\left( A \right)}
\newcommand{\legsym}[2]{\left(\frac{#1}{#2} \right)}
\renewcommand{\mod}[3]{\: #1 \equiv #2 \: \mathrm{mod} \: #3 \:}
\newcommand{\nmod}[3]{\: #1 \centernot \equiv #2 \: \mathrm{mod} \: #3 \:}
\newcommand{\ndiv}{\hspace{-4pt}\not \divides \hspace{2pt}}
\newcommand{\finfield}[1]{\mathbb{F}_{#1}}
\newcommand{\finunits}[1]{\mathbb{F}_{#1}^{\times}}
\newcommand{\ord}[1]{\mathrm{ord}\! \left(#1 \right)}
\newcommand{\quadfield}[1]{\Q \small(\sqrt{#1} \small)}
\newcommand{\vspan}[1]{\mathrm{span}\! \left\{#1 \right\}}
\newcommand{\galgroup}[1]{Gal \small(#1 \small)}
\newcommand{\Aut}[1]{\mathrm{Aut} \small(#1 \small)}
\newcommand{\ints}[1]{\mathcal{O}_{#1}}
\newcommand{\sm}{\! \setminus \!}
\newcommand{\norm}[3]{\mathrm{N}^{#1}_{#2}\left(#3\right)}
\newcommand{\qnorm}[2]{\mathrm{N}^{#1}_{\Q}\left(#2\right)}
\newcommand{\quadint}[3]{#1 + #2 \sqrt{#3}}
\newcommand{\pideal}{\mathfrak{p}}
\newcommand{\inorm}[1]{\mathrm{N}(#1)}
\newcommand{\tr}[1]{\mathrm{Tr} \! \left(#1\right)}
\newcommand{\delt}{\frac{1 + \sqrt{d}}{2}}
\newcommand{\ch}[1]{\mathrm{char} \: #1}
\renewcommand{\Im}[1]{\mathrm{Im}(#1)}
\newcommand{\minimal}[2]{\mathrm{Min}(#1;#2)}
\newcommand{\fix}[2]{\mathrm{Fix}_{#1} (#2)}
\newcommand{\id}{\mathrm{id}}
\renewcommand{\empty}{\varnothing}
\newcommand{\Tor}[4]{\mathrm{Tor}^{#1}_{#2} \left( #3, #4 \right)}
\newcommand{\Ext}[4]{\mathrm{Ext}^{#1}_{#2} \left( #3, #4 \right)}
\newcommand{\Homover}[3]{\mathrm{Hom}_{#1} \left( #2, #3 \right)}
\newcommand{\Frac}[1]{\mathrm{Frac}\left(#1\right)}

\newcommand{\U}[1]{\mathrm{U}(#1)}
\renewcommand{\O}[1]{\mathrm{O}(#1)}
\newcommand{\SU}[1]{\mathrm{SU}(#1)}
\newcommand{\SO}[1]{\mathrm{SO}(#1)}
\newcommand{\GL}[2]{\mathrm{GL}_{#1}(#2)}
\newcommand{\SL}[2]{\mathrm{SL}_{#1}(#2)}
\newcommand{\PGL}[2]{\mathrm{PGL}_{#1}(#2)}
\newcommand{\PSL}[2]{\mathrm{PSL}_{#1}(#2)}


\newcommand{\Hom}[2]{\mathrm{Hom}\left(#1, #2 \right)}
\newcommand{\Mod}[1]{\mathbf{Mod}_{#1}}
\newcommand{\Grp}{\mathbf{Grp}}
\newcommand{\AbGrp}{\mathbf{AbGrp}}
\newcommand{\Ring}{\mathbf{Ring}}

\newcommand{\Ann}[2]{\mathrm{Ann}_{#1}\left(#2\right)}
\newcommand{\Ass}[2]{\mathrm{Ass}_{#1}\left( #2 \right)}
\newcommand{\supp}[2]{\mathrm{Supp}_{#1} \left( #2 \right) }
\newcommand{\Supp}[2]{\mathrm{Supp}_{#1}\left(#1\right)}
\newcommand{\spec}[1]{\mathrm{Spec}\left( #1 \right)}
\newcommand{\Spec}[1]{\mathrm{Spec}\left( #1 \right)}
\newcommand{\rad}[1]{\mathrm{rad}\left( #1 \right)}
\newcommand{\nilrad}[1]{\mathrm{nilrad}\left( #1 \right)}
\newcommand{\gr}[2]{\mathbf{gr}_{#1}\left(#2\right)}

\newcommand{\ev}{\mathrm{ev}}
\newcommand{\p}{\mathfrak{p}}
\renewcommand{\P}{\mathfrak{P}}
\newcommand{\q}{\mathfrak{q}}
\newcommand{\m}{\mathfrak{m}}

\theoremstyle{remark}
\newtheorem*{remark}{Remark}

\theoremstyle{definition}
\newtheorem{theorem}{Theorem}[section]
\newtheorem{lemma}[theorem]{Lemma}
\newtheorem{proposition}[theorem]{Proposition}
\newtheorem{corollary}[theorem]{Corollary}


\newenvironment{definition}[1][Definition:]{\begin{trivlist}
\item[\hskip \labelsep {\bfseries #1}]}{\end{trivlist}}


\newenvironment{lproof}{\begin{proof} \renewcommand{\qedsymbol}{}}{\end{proof}}

\newcommand{\atitle}[1]{\title{% 
	\large \textbf{Mathematics GR6261 Commutative Algebra
	\\ Assignment \# #1} \vspace{-2ex}}
\author{Benjamin Church }
\maketitle}



\begin{document}
\atitle{2}
 
\section{Problem 1}
Let $A$ be a Noetherian ring. 
\begin{enumerate}
\item[$(a) \implies (b)$]
If $A$ is Artinian then $A$ has finitely many maximal ideals and every prime ideal is maximal. Thus, $\Spec{A}$ is finite and if $\p \in \Spec{A}$ then $\p$ is maximal so $V(\p) = \{ \p \}$ since there are no proper ideals above it. Therefore each point is closed so $\Spec{A}$ is discrete. 
\item[$(b) \implies (c)$]
Trivial.
\item[$(c) \implies (a)$]
Suppose that $\Spec{A}$ is discrete. Then for each prime $\p \in \Spec{A}$ the set $\{ \p \}$ must be closed. Therefore, $\{ \p \} = V(I)$ for some ideal $I$ so $\p$ is the only prime ideal above $I$. However, every ideal is contained in a maximal ideal which is prime so $\p$ must be maximal. Therefore $A$ is an Noetherian ring in which every prime ideal is maximal. This implies that $A$ is Artinian. 
\end{enumerate}

\section{Problem 2}

Let $k$ be a field and $A$ a $k$-algebra. If $A$ is finite-dimensional then, as $k$-modules, $A \cong k^n$ which is Artinian because $k$ a field and thus Artinian as a $k$-module. Thus $A$ is Artinian as a $A$-module since every ideal is a $k$-submodule. Conversely, if $A$ is Artinian then it must be Noetherian as well. Take any $v_0 \in A$ and let $A_1 = k \cdot v_0$ and define a sequence of submodules inductively. Define $A_{n+1} = A_n + k \cdot v_n$ where $v_n$ is some element of $A \setminus A_n$ which we assume is nonempty. Clearly, we have a strictly increasing infinite chain,
\[ A_0 \subsetneq A_1 \subsetneq A_2 \subsetneq \cdots \]
contradicting the fact that $A$ is Noetherian. Thus, $A = A_n = k \cdot v_0 + \cdots + k \cdot v_n$ for some $n$. Therefore $A$ is finite-dimensional. 


\section{Problem 3}

Let $M$ be an $A$-module and $f : M \to M$ a morphism of $A$-modules.
\begin{enumerate}
\item[a)] Suppose that $f$ is surjective and $M$ is Noetherian. Consider the ascending chain of submodules,
\[ \ker{f} \subset \ker{f^2} \subset \ker{f^3} \subset \cdots \]
Because $M$ is Noetherian, the chain must stabilize at some $k$. Suppose $f(x) = 0$. By surjectivity, there must exist $y \in M$ such that $f^k(y) = x$. Thus $f(x) = f(f^k(y)) = f^{k+1}(y) = 0$ so $y \in \ker{f^{k+1}} = \ker{f^k}$ so $x =f^k(y) = 0$. Thus, $f$ is injective. Because $f$ is also a surjective morphism it is an isomorphism.

\item[b)] Suppose that $f$ is injective and $M$ is Artinian. Consider the descending chain of submodules,
\[ \Im{f} \supset \Im{f^2} \supset \Im{f^3} \supset \cdots \]
Because $M$ is Artinian, this chain must stabilize at some $k$. Thus, for any $y \in M$ we know that $f^{k}(y) \in \Im{f^{k}} = \Im{f^{k+1}}$ so $f^{k}(y) = f^{k+1}(x) = f^k(f(x))$ for some $x \in M$. However, $f$ is injective and thus $f^k$ is also injective so $y = f(x)$ since $f^k(y) = f^k(f(x))$. Therefore $y \in \Im{f}$ so $f$ is surjective. Since $f$ is also an injective morphism, $f$ is an isomorphism.    
\end{enumerate}

\section{Problem 4}

\begin{enumerate}
\item[a)]
Let $\q_1, \dots, \q_r$ be $\p$-primary ideals. Consider the ideal,
\[ I = \q_1 \cap \cdots \cap \q_r \]
By lemma \ref{radical_intersection}, we have $\sqrt{I} = \sqrt{\q_1} \cap \cdots \cap \sqrt{\q_r} = \p \cap \cdots \cap \p = \p$. Furthermore, I claim that $I$ is primary. If $xy \in I$ then $xy \in \q_i$ for each $i$ so either $x \in \q_i$ or $y \in \p = \sqrt{\q_i}$. If $x \in \q_i$ for each $i$ then $x \in I$ so we are done. Otherwise, $y \in \p$ for some $i$. However, for every $i$, $\p = \sqrt{\q_i}$ so there is some power such that $y^{n_i} \in \q_i$ for each $i$. Take $N = n_1 + \cdots + n_r$ then $y^{N} \in \q_i$ for all $i$ so $y^{N} \in I$. Thus, $I$ is primary and $\sqrt{I} = \p$ so $I$ is $\p$-primary.
\item[b)]
Let $\q$ be $\p$-primary and $x \in A \setminus \q$. Consider $I = (\q : Ax) = \{r \in A \mid rx \in \q \}$. Since $\q \subset I$ we have $\sqrt{\q} = \p \subset \sqrt{I}$. Suppose that $r \in \sqrt{I}$ then $r^n \in I$ for some $n$ so $r^n x \in \q$ but $x \notin \q$ and $\q$ is primary so $r \in \sqrt{\q} = \p$. Furthermore, if $ab \in I$ then $abx \in \q$ then $a^r \in \q$ or $bx \in \q$ since $\q$ is primary. Thus, $a^n \in I$ or $b \in I$ and thus $I$ is $\p$-primary.

\item[c)]

Suppose that $\a = \q_1 \cap \cdots \cap \q_r$ be a minimal primary decomposition. Then, for any $x \in A$, by Lemma \ref{intersect},
\[ (\a : x) = \bigcap_i (\q_i : x) \]
Since $\q_i$ is $\p_i$-primary with $\p_i = \sqrt{\q_i}$ by (b) we know that $(\q_i : x)$ is $\p_i$-primary when $x \in A \setminus \q_i$ and otherwise $(\q_i : x) = A$. Then, by lemma \ref{radical_intersection}, we know that,
\[ \sqrt{(\a : x)} = \bigcap_i \sqrt{(\q_i : x)} = \bigcap_{i \mid x \notin \q_i} \p_i \]
Suppose that $\p = \sqrt{(\a : x)}$ is a prime ideal. By corollary \ref{one_of_intersect}, we know that $\p = \p_i$ for some $i$. Thus every prime of the form $\sqrt{(\a : x)}$ is one of $\p_i$, the associated primes. The decomposition is minimal so for each $i$ there must exist some $x_i \notin \q_i$ which is an element of all $\q_j$ for $i \neq j$. Then, $(\q_i : x_i)$ is $\p_i$-primary and $(\q_j : x_i) = A$ for $i \neq j$. Therefore,
\[ \sqrt{(\a : x_i)} = \bigcap_j \sqrt{(\q_j : x_i)} = \p_i \]
so each associated prime is of the form $\sqrt{(\a : x)}$. We have shown that the primes associated to $\a$ are exactly those primes which can be written as $\p = \sqrt{(\a : x)}$. Furthermore, this proves that the set of associate primes $\sqrt{\q_i}$ is independent of the primary decomposition $\{ \q_i \}$ since ``all primes of the form $\p = \sqrt{(\a : x)}$'' is a categorization with is manifestly independent of the choice of decomposition.   

\item[d)]
Clearly, if $\p = (\a : x)$ is prime then $\p = \sqrt{\p} = \sqrt{(\a : x)}$ so $\p$ is an $\a$-associate. However, I believe the converse is false. All $\a$-associates are primes of the form $\p = \sqrt{(\a : x)}$ but not necessarily of the form $(\a : x)$. 

\item[e)]
Let $S$ be a multiplicative subset and $\iota : A \to S^{-1} A$ the localization map. Suppose that $\a$ has minimal primary decomposition,
\[ \a = \q_1 \cap \cdots \cap \q_r \]
Now,
\[ \iota^{-1}(S^{-1} \a) = \iota^{-1} \left( S^{-1} \q_1 \cap \cdots \cap S^{-1} \q_r \right) = \iota^{-1}(S^{-1} \q_1) \cap \cdots \cap \iota^{-1}(S^{-1} \q_r) \]
Therefore we need to investigate $\iota^{-1}(S^{-1} \q_i)$ for each $i$. We have,
\begin{align*}
x \in \iota^{-1}(S^{-1} \q) \iff (x,1) \in S^{-1} \q \iff \exists s \in S : s x \in \q \iff \exists s \in S : x \in (\q : s) 
\end{align*}
If $S \cap \q \neq \varnothing$ then take $s \in S \cap \q$ and $(\q : s) = A$ so $\iota^{-1}(S^{-1} \q) = A$. However, if $S \cap \q = \varnothing$ then for any $s \in S$ we have $s \notin \q$ but if $s \in \p = \sqrt{\q}$ then $s^n \in \q$. However, $S$ is multiplicative so $s^n \in S$ and thus $s \notin \q$ since $S$ and $\q$ have trivial intersection. Therefore, $s \notin \p$. Applying Lemma \ref{quotient_primary}, $(\q : s) = \q$ so we have,
\[ x \in \iota^{-1}(S^{-1}\q) \iff \exists s \in S : x \in (\q : s) \iff x \in \q \]
Therefore, $\iota^{-1}(S^{-1}\q) = \q$ if $S \cap \q = \varnothing$ and $\iota^{-1}(S^{-1}\q) = A$ otherwise. Finally,
\[ \iota^{-1}(S^{-1} \a) = \bigcap_{i \mid S \cap \q_i = \varnothing} \q_i \]
since when $S \cap \q_i \neq \varnothing$ we have $\iota^{-1}(S^{-1} \q_i) = A$ which does not contribute to the intersection. 
\bigskip\\
Now let $\Sigma$ be a set of primes associated to $\a$ which are isolated in $\Spec{A}$. I take this to mean that $\Sigma$ is separated from every $\a$-associated prime it does not contain. In particular, if $\p$ is $\a$-associated and $\overline{\{ \p \}} = V(\p)$ intersects $\Sigma$ then $\p \in \Sigma$. That is, if $\p$ is an $\a$-associate prime such that there exists $\p' \in \Sigma$ with $\p \subset \p'$ then $\p \in \Sigma$.
Now take,
\[ S = A \setminus \bigcup_{\p \in \Sigma} \p \]
Clearly, $1 \in S$ and if $x, y \in S$ then $x,y \notin \p$ for any $\p \in \Sigma$ so $xy \notin \p$ for each $\p$ and thus $x y \in S$. Therefore, $S$ is multiplicative. Take,
\[ \a_{\Sigma} = \iota^{-1}(S^{-1} \a) = \bigcap_{i \mid S \cap \q_i = \varnothing} \q_i \] 
However, $S \cap \q_i = \varnothing$ exactly when,
\[ 
\q_i \subset \bigcup_{\p \in \Sigma} \p = U  \]
Clearly, if $\p_i  = \sqrt{\q_i} \in \Sigma$ then $\q_i \subset \p_i \subset U$ so $S \cap \q_i = \varnothing$. Conversely, if $\q_i \subset U$ then $\q_i \subset \p'$ for some $\p' \in \Sigma$. Thus, $\p_i = \sqrt{\q_i} \subset \p'$ so, since $\Sigma$ is isolated, $\p_i \in \Sigma$. Therefore,
\[ \a_{\Sigma} = \iota^{-1}(S^{-1} \a) = \bigcap_{i \mid \p_i \in \Sigma} \q_i \] 
However, $\iota^{-1}(S^{-1} \a)$ is independent of the primary decomposition and thus,
\[ \a_{\Sigma} = \bigcap_{i \mid \p_i \in \Sigma} \q_i \] 
is also independent of the decomposition. 
\end{enumerate}

\section{Lemmas}

\begin{lemma} \label{radical_intersection}
Let $I, J \subset A$ be ideals. Then $\sqrt{I \cap J} = \sqrt{I} \cap \sqrt{J}$ 
\end{lemma}

\begin{proof}
Note that $I, J \supset I \cap J$ so $\sqrt{I}$ and $\sqrt{J}$ lie above $\sqrt{I \cap J}$ so $\sqrt{I} \cap \sqrt{J} \supset \sqrt{I \cap J}$. Furthermore, if $x \in \sqrt{I} \cap \sqrt{J}$ then $x^n \in I$ and $x^m \in J$ for some $m$ and $n$. Then, $x^{m + n} \in I \cap J$ so $x \in \sqrt{I \cap J}$. Thus, $\sqrt{I \cap J} = \sqrt{I} \cap \sqrt{J}$.  
\end{proof}

\begin{lemma} \label{intersect}
For ideals $I_i, J \subset A$,
\[ \left( \bigcap_i I_i : J \right) = \bigcap_i (I_i : J ) \]
\end{lemma}

\begin{proof}
We know that,
\[   \left( \bigcap_i I_i : J \right) = \{ x \in A \mid x J \subset \bigcap_i I_i \} \]
However, $x J$ is a subset of $\cap I_i$ iff $x J \subset I_i$ for each $i$. Thus,
\[ \left( \bigcap_i I_i : J \right) = \bigcap_i (I_i : J ) \]
\end{proof}

\begin{lemma}
If $\p \supset \p_1 \cap \cdots \cap \p_r$ are all prime ideals then $\p \supset \p_i$ for some $i$. 
\end{lemma}

\begin{proof}
Suppose not. Then there exists $x_i \in \p_i$ such that $x_i \notin \p$ for each $i$. Therefore, we have elements not in $\p$ such that the product $x_1 \cdots x_r \in \p_1 \cap \cdots \cap \p_r = \p$ is in $\p$ which contradicts the primality of $\p$. 
\end{proof}

\begin{corollary} \label{one_of_intersect}
If $\p = \p_1 \cap \cdots \cap \p_r$ are all prime ideals then $\p = \p_i$ for some $i$. 
\end{corollary}

\begin{proof}
We know that $\p \supset \p_i$ for some $i$ but $\p = \p_1 \cap \cdots \cap \p_r \subset \p_i$ so $\p = \p_i$. 
\end{proof}

\begin{lemma} \label{quotient_primary}
Let $\q$ be $\p$-primary and $s \notin \p$ then $(\q : s) = \q$. 
\end{lemma}

\begin{proof}
If $x \in (\q : s)$ then $xs \in \q$ thus either $x \in \q$ or $s^n \in \q$ so $s \in \p = \sqrt{\q}$ which is false. Thus $x \in \q$. Conversely, if $x \in \q$ then $xs \in \q$ so $(\q : s) = \q$.
\end{proof}




\end{document}