\documentclass[12pt]{extarticle}
\usepackage[utf8]{inputenc}
\usepackage[english]{babel}
\usepackage[a4paper, total={6in, 9in}]{geometry}
\usepackage{tikz-cd}
 
\usepackage{amsthm, amssymb, amsmath, centernot}


\newcommand{\notimplies}{%
  \mathrel{{\ooalign{\hidewidth$\not\phantom{=}$\hidewidth\cr$\implies$}}}}

\renewcommand\qedsymbol{$\square$}
\newcommand{\cont}{$\boxtimes$}
\newcommand{\divides}{\mid}
\newcommand{\ndivides}{\centernot \mid}
\newcommand{\Z}{\mathbb{Z}}
\newcommand{\N}{\mathbb{N}}
\newcommand{\C}{\mathbb{C}}
\newcommand{\Zplus}{\mathbb{Z}^{+}}
\newcommand{\Primes}{\mathbb{P}}
\newcommand{\ball}[2]{B_{#1} \! \left(#2 \right)}
\newcommand{\Q}{\mathbb{Q}}
\newcommand{\R}{\mathbb{R}}
\newcommand{\Rplus}{\mathbb{R}^+}
\newcommand{\invI}[2]{#1^{-1} \left( #2 \right)}
\newcommand{\End}[1]{\text{End}\left( A \right)}
\newcommand{\legsym}[2]{\left(\frac{#1}{#2} \right)}
\renewcommand{\mod}[3]{\: #1 \equiv #2 \: \mathrm{mod} \: #3 \:}
\newcommand{\nmod}[3]{\: #1 \centernot \equiv #2 \: \mathrm{mod} \: #3 \:}
\newcommand{\ndiv}{\hspace{-4pt}\not \divides \hspace{2pt}}
\newcommand{\finfield}[1]{\mathbb{F}_{#1}}
\newcommand{\finunits}[1]{\mathbb{F}_{#1}^{\times}}
\newcommand{\ord}[1]{\mathrm{ord}\! \left(#1 \right)}
\newcommand{\quadfield}[1]{\Q \small(\sqrt{#1} \small)}
\newcommand{\vspan}[1]{\mathrm{span}\! \left\{#1 \right\}}
\newcommand{\galgroup}[1]{Gal \small(#1 \small)}
\newcommand{\Aut}[1]{\mathrm{Aut} \small(#1 \small)}
\newcommand{\ints}[1]{\mathcal{O}_{#1}}
\newcommand{\sm}{\! \setminus \!}
\newcommand{\norm}[3]{\mathrm{N}^{#1}_{#2}\left(#3\right)}
\newcommand{\qnorm}[2]{\mathrm{N}^{#1}_{\Q}\left(#2\right)}
\newcommand{\quadint}[3]{#1 + #2 \sqrt{#3}}
\newcommand{\pideal}{\mathfrak{p}}
\newcommand{\inorm}[1]{\mathrm{N}(#1)}
\newcommand{\tr}[1]{\mathrm{Tr} \! \left(#1\right)}
\newcommand{\delt}{\frac{1 + \sqrt{d}}{2}}
\newcommand{\ch}[1]{\mathrm{char} \: #1}
\renewcommand{\Im}[1]{\mathrm{Im}(#1)}
\newcommand{\minimal}[2]{\mathrm{Min}(#1;#2)}
\newcommand{\fix}[2]{\mathrm{Fix}_{#1} (#2)}
\newcommand{\id}{\mathrm{id}}
\renewcommand{\empty}{\varnothing}
\newcommand{\Tor}[4]{\mathrm{Tor}^{#1}_{#2} \left( #3, #4 \right)}
\newcommand{\Ext}[4]{\mathrm{Ext}^{#1}_{#2} \left( #3, #4 \right)}
\newcommand{\Homover}[3]{\mathrm{Hom}_{#1} \left( #2, #3 \right)}
\newcommand{\Frac}[1]{\mathrm{Frac}\left(#1\right)}

\newcommand{\U}[1]{\mathrm{U}(#1)}
\renewcommand{\O}[1]{\mathrm{O}(#1)}
\newcommand{\SU}[1]{\mathrm{SU}(#1)}
\newcommand{\SO}[1]{\mathrm{SO}(#1)}
\newcommand{\GL}[2]{\mathrm{GL}_{#1}(#2)}
\newcommand{\SL}[2]{\mathrm{SL}_{#1}(#2)}
\newcommand{\PGL}[2]{\mathrm{PGL}_{#1}(#2)}
\newcommand{\PSL}[2]{\mathrm{PSL}_{#1}(#2)}


\newcommand{\Hom}[2]{\mathrm{Hom}\left(#1, #2 \right)}
\newcommand{\Mod}[1]{\mathbf{Mod}_{#1}}
\newcommand{\Grp}{\mathbf{Grp}}
\newcommand{\AbGrp}{\mathbf{AbGrp}}
\newcommand{\Ring}{\mathbf{Ring}}

\newcommand{\Ann}[2]{\mathrm{Ann}_{#1}\left(#2\right)}
\newcommand{\Ass}[2]{\mathrm{Ass}_{#1}\left( #2 \right)}
\newcommand{\supp}[2]{\mathrm{Supp}_{#1} \left( #2 \right) }
\newcommand{\Supp}[2]{\mathrm{Supp}_{#1}\left(#1\right)}
\newcommand{\spec}[1]{\mathrm{Spec}\left( #1 \right)}
\newcommand{\Spec}[1]{\mathrm{Spec}\left( #1 \right)}
\newcommand{\rad}[1]{\mathrm{rad}\left( #1 \right)}
\newcommand{\nilrad}[1]{\mathrm{nilrad}\left( #1 \right)}
\newcommand{\gr}[2]{\mathbf{gr}_{#1}\left(#2\right)}

\newcommand{\ev}{\mathrm{ev}}
\newcommand{\p}{\mathfrak{p}}
\renewcommand{\P}{\mathfrak{P}}
\newcommand{\q}{\mathfrak{q}}
\newcommand{\m}{\mathfrak{m}}

\theoremstyle{remark}
\newtheorem*{remark}{Remark}

\theoremstyle{definition}
\newtheorem{theorem}{Theorem}[section]
\newtheorem{lemma}[theorem]{Lemma}
\newtheorem{proposition}[theorem]{Proposition}
\newtheorem{corollary}[theorem]{Corollary}


\newenvironment{definition}[1][Definition:]{\begin{trivlist}
\item[\hskip \labelsep {\bfseries #1}]}{\end{trivlist}}


\newenvironment{lproof}{\begin{proof} \renewcommand{\qedsymbol}{}}{\end{proof}}

\newcommand{\atitle}[1]{\title{% 
	\large \textbf{Mathematics GR6261 Commutative Algebra
	\\ Assignment \# #1} \vspace{-2ex}}
\author{Benjamin Church }
\maketitle}



\begin{document}
\atitle{1}
 
\section{Problem 1}

\subsection*{(a)}
Let $A$ be a ring and $M$ an $A$-module. We know that $M_{\p} = 0$ if and only if for each $m \in M$ then $(m, s) \sim (0, 1)$ iff there exists $u \in S = A \setminus \p$ such that $u \cdot m = 0$. Therefore if there exists $u \in \Ann{A}{M} \cap S$ i.e. if $\Ann{A}{M} \not\subset \p$ then $u \cdot m = 0$ for all $m$ so $M_{\p} = 0$. Suppose that $\p \in \Supp{A}{M}$ then we know that $M_{\p} \neq 0$ and thus $\Ann{A}{M} \subset \p$ which means that $\p \in V(\Ann{A}{M})$. Therefore,
\[ \Supp{A}{M} \subset V(\Ann{A}{M}) \]
Now, suppose that $M$ is of finite type so that we may write $M = A m_1 + \cdots + A m_r$ for constants $m_i \in M$. If $M_{\p}$ then we know that for each $m \in M$, in particular for $m_i$ there exist elements $u_i \in S$ such that $u_i \cdot m_i = 0$ for each $i$. Take $u = u_1 \dots u_r$. Clearly, for any $m \in M$ we can write,
$m = a_1 \cdot m_1 + \cdots + a_i \cdot m_i$ and thus
\[ u \cdot m = (u_2 \cdots u_r a_1) \cdot (u_1 \cdot m_1) + \cdots + (u_1 \cdots u_{r-1} \cdot a_r) \cdot (u_r \cdot m_r) = 0 \]
so $u \in \Ann{A}{M}$ but $u \in S$ so $\Ann{A}{M} \not\subset \p$. Therefore, if $\p \notin \Supp{A}{M}$ then $\p \notin V(\Ann{A}{M})$ which implies that,
\[ \Supp{A}{M} = V(\Ann{A}{M}) \]

\subsection*{(b)}
Assume that $A$ is Noetherian and $M$ is of finite type. By the above, since $M$ has finite type,
\[ \Supp{A}{M} = V(\Ann{A}{M}) \]
Now suppose that $\Supp{A}{M} \subset V(I)$ for some ideal $I \subset A$ then we have that $V(\Ann{A}{M}) \subset V(I)$ and thus if $\Ann{A}{M} \subset \p$ then we know that $I \subset \p$. This implies that $I \subset \sqrt{\Ann{A}{M}}$ because it is contained in every prime ideal above $\Ann{A}{M}$ and thus also their intersection. Since $A$ is Noetherian, the ideal $I$ is finitely generated so by Lemma \ref{fin_gen_rad_power} we know that there exists a power $N$ such that $I^N \subset \Ann{A}{M}$ and thus $I^N \cdot M = 0$. 

\section{Problem 2}

\renewcommand{\a}{\mathfrak{a}}
\renewcommand{\b}{\mathfrak{b}}

\subsection*{(a)}
Let $\a$ and $\b$ be ideals of $A$ such that $\a$ is finitely generated and both $A / \a$ and $A / \b$ are Noetherian. Since $\a$ is finitely generated we may write it as $\a = A a_1 + \cdots + A a_n$ for constants $a_i \in \a$. Consider the map $A^n \to \a / \a \b$ given by $(c_i) \mapsto \left[ \sum_{i = 1}^n a_i c_i \right]_{\a \b}$. Clearly, this map is surjective because $\a$ is generated by the set $\{ a_i \}$ over $A$. Furthermore, the ideal $\b^n \subset A^n$ is sent to $(0)$ because if $(b_i) \subset \b^n$ then $\sum_{i = 1}^n a_i b_i \in \a \b$. Therefore, this map factors through the quotient as,
\begin{center}
\begin{tikzcd}
A^n \arrow[rr, two heads] \arrow[rd] & & \a / \a \b 
\\
& (A / \b)^n \arrow[ru, two heads]
\end{tikzcd}
\end{center}
where the map $(A / \b)^n \to \a / \a \b$ is still surjective. Furthermore, $A / \b$ is Noetherian so, by Lemma \ref{powers_are_noetherian}, $(A / \b)^n$ is also Noetherian and then surjectivity forces the image $\a / \a \b$ to be Noetherian as well. Next, consider the exact sequence given by the third isomorphism theorem,
\begin{center}
\begin{tikzcd}
0 \arrow[r] & \a / \a \b \arrow[r] & A / \a \b \arrow[r] & A / \a \arrow[r] & 0 
\end{tikzcd}
\end{center}
Because $\a / \a \b$ and $A / \a$ are Noetherian we must have that $A / \a \b$ is Noetherian as well. 


\subsection*{(b)}

Suppose that $A$ is not Noetherian and consider the poset $\Sigma$ of all ideals of $A$ which are not finitely generated ordered by inclusion. Since $A$ is not Noetherian $\Sigma$ is nonempty. Let $\mathcal{I}$ be a chain inside $\Sigma$ then consider the ideal,
\[ U = \bigcup_{I \in \mathcal{I}} I \]
which is, in fact, an ideal because $\mathcal{I}$ is totally ordered so all pairs of elements in $U$ lie in a common ideal in $\mathcal{I}$ (c.f. problem 3). Suppose that $U$ is finitely generated then there must exist some $x_1, \dots, x_n \in U$ such that $U = A x_1 + \cdots + A x_n$. However, each generator must lie in some element of the union, $x_i \in I_i$. Since $\mathcal{I}$ is totally ordered and $\{ I_i \}$ is a finite set of ideals, there must exists a maximum $I_m$ of the $I_i$. Then, $x_i \in I_m$ for all $i$ so $U = A x_1 + \cdots + A x_n \subset I_m$ which implies that $U = I_m$ so $I_m$ is finitely generated which is a contradiction. Therefore, $U \in \Sigma$ so every chain has a maximum. Thus, by Zorn's lemma, $\Sigma$ has maximal elements.
\bigskip\\
Let $\m \in \Sigma$ be a maximal element. Consider $x,y \notin \m$. Then $\m + (x)$ and $\m + (y)$ lie above $\m$ so by maximality neither $\m + (x)$ nor $\m + (y)$ can be in $\Sigma$ and thus are finitely generated. Every ideal of $A / \m$ is the image of an ideal in $A$ above $\m$ which must be finitely generated because $\m$ is maximal in $\Sigma$. However, the image of a finitely generated ideal is finitely generated and thus every ideal of $A / \m$ is finitely generated so $A / \m$ is Noetherian. Furthermore we have surjective maps, $A / \m \to A / (\m + (x))$ and $A / \m \to A / (\m + (y))$ so both quotients are Noetherian as well. Let $K = (\m + (x))(\m + (y))$ which is finitely generated because both $\m + (x)$ and $\m + (y)$ are. By the previous problem, since both are finitely generated, $A / K$ is Noetherian as well. However, $K = \m^2 + \m x + \m y + (x y)$. If $xy \in \m$ then $K \subset \m$. Then by Lemma \ref{finitely_generated_noetherian_quotient}, since $A / K$ is Noetherian, $K$ is finitely generated, and $K \subset \m$ we would have that $\m$ is finitely generated contradincting $\m \in \Sigma$. Therefore, $xy \notin \m$ so $\m$ is a prime ideal (since $\m \neq A$ since $A$ is obviously finitely generated). Therefore, if $A$ is not Noetherian then there must exist a prime ideal of $A$ is not finitely generated. Equivalently, if every prime ideal of $A$ is finitely generated then $A$ is Noetherian.       

\section{Problem 3}

Let $A$ be a ring and $\Sigma$ the set of ideals of $A$ containing only zero-divisors. The set $\Sigma$ is naturally a poset under set inclusion. Let $\mathcal{I}$ be a chain in $\Sigma$. Then consider the ideal,
\[ U = \bigcup_{I \in \mathcal{I}} I \]
This is an ideal because if $x, y \in U$ then $x \in I$ and $y \in I'$ for $I, I' \in \mathcal{I}$ but $\mathcal{I}$ is totally ordered so either $I \subset I'$ or $I' \subset I$ and thus the larger contains both $x$ and $y$ and therefore the sum and the multiples are in $U$. Also each $x \in U$ is contained in some ideal $I \in \Sigma$ so $x$ is a zero-divisor and thus $U \in \Sigma$. However, $\forall I \in \mathcal{I} : I \subset U$. Since every chain has a maximum, by Zorn's Lemma there exist maximal elements of $\Sigma$ above every ideal $I \in \Sigma$. Suppose $\m \in \Sigma$ is maximal and $xy \in \m$. Since $xy$ is a zero-divisor we must have $a xy = 0$ for some $a \neq 0$ so either $ax = 0$ or $y$ is a zero-divisor. Thus either $x$ or $y$ is a zero-divisor. Without loss of generality, suppose that $x$ is a zero-divisor then so is $ax$ for any $a \in A$ so $(x) \in \Sigma$. Since the sum of zero-divisors is a zero divisor, $\m + (x) \in \Sigma$ but $\m \subset \m + (x)$ contradicting maximality unless $\m + (x) = \m$ and thus $x \in \m$ so $\m$ is prime ($\m \neq A$ since $1 \in A$ is not a zero-divisor but $\m \in \Sigma$). 
\bigskip\\
Now, if $x$ is a zero-divisor then so is $ax$ for any $a \in A$ so $(x) \in \Sigma$. By Zorn's Lemma, there exists a maximal element $\m_x \in \Sigma$ above $(x)$ so $x \in \m_x$ and we know that $\m_x$ is prime. Therefore, each zero-divisor is contained in a prime ideal containing only zero-divisors. Let $Z \subset A$ be the zero divisors of $A$. Then,
\[ Z = \bigcup_{x \in Z} \m_x \]
because if $x \in Z$ then $x \in \m_x$ and thus in the union and each ideal of the union is contained in $Z$ so the enite union is as well. Therefore, the zero-divisors of $A$ are a union of prime ideals of $A$. 

\section{Problem 4}

Let $N = \sqrt{(0)}$ the nilradical of $A$ which is the ideal of all nilpotent elements and the intersection of all prime ideals. Suppose that $N$ is prime. Leet $I \subset A$ be any ideal then 
\[N \in V(I) \iff I \subset N \iff \forall \p \in \Spec{A} : I \subset \p \iff V(I) = \Spec{A}\]
Suppose that $U \subset \Spec{A}$ is open and nonempty. Then $U^C$ is closed and proper so $U^C = V(I) \neq \Spec{A}$ for some $I$ which implies that $N \notin V(I)$ so $N \in U$. However, any closed set containing $U$ must be of the form $V(J)$ for some ideal $J$. Since $N \in U \subset V(J)$ we know that $V(J) = \Spec{A}$. Thus, $\overline{U} = \Spec{A}$ so $U$ is dense. Therefore, $\Spec{A}$ is irreducible.
\bigskip\\
Conversely, suppose that $N$ is not prime. Then there exist elements $x,y \notin N$ such that $xy \in N$. Now, $V(xy) = V(x) \cup V(y)$ where $V(x)$ and $V(y)$ are closed proper sets because $x, y \notin N$ so $(x)$ and $(y)$ both cannot be contained in every prime ideal else they would be elements of the intersection $N$. However, $xy \in N$ so $(xy) \subset \p$ for each prime ideal so $V(xy) = \Spec{A}$. Thus we have written $\Spec{A} = V(x) \cup V(y)$ as the union of closed proper subsets so $\Spec{A}$ is not irreducible.
\bigskip\\
Therefore, $\Spec{A}$ is irreducible if and only if the nilradical is prime. 

\section{Problem 5}



\section*{Lemata}

\begin{lemma} \label{fin_gen_rad_power}
Let $A$ be a ring with a finitely generated $I \subset A$ and any ideal $J \subset A$ such that $I \subset \sqrt{J}$. Then there exists an integer $N$ such that $I^N \subset J$.
\end{lemma}

\begin{proof}
$I$ is finitely generated so let $I = (x_1, \dots, x_{\ell})$. We know that $I \subset \sqrt{J}$ so for each $1 \le i \le \ell$ there exists a positive integer $n_i$ such that $x_i^{n_i} \in J$. Take $N = n_1 + \cdots + n_{\ell}$. The ideal $I^N$ is generated by monomials of the form $x_1^{r_1} \cdots x_{\ell}^{r_{\ell}}$ such that $r_1 + \cdots + r_{\ell} = N = n_1 + \cdots + n_{\ell}$. However, because these are all positive integers there must exist at least one $r_i$ such that $r_i \ge n_i$ and thus $x_i^{r_i} = x_i^{n_i} \cdot x_i^{r_i - n_i} \in J$ and thus, by ideal absorption, the entire monomial $x_1^{r_1} \cdots x_{\ell}^{r_{\ell}} \in J$. Therefore $I^N \subset J$ because each generator is contained in $J$. 
\end{proof}

\begin{lemma} \label{powers_are_noetherian}
Let $A$ be a Noetherian ring then $A^n$ is Noetherian also.
\end{lemma}

\begin{proof}
Proceede by induction on $n$. The case $n = 1$ is trivial. Assume true for $n -1$ then consider the exact sequence,
\begin{center}
\begin{tikzcd}
0 \arrow[r] & A \arrow[r] & A^n \arrow[r] & A^{n-1} \arrow[r] & 0 
\end{tikzcd}
\end{center}  
since both $A$ and $A^{n-1}$ are Noetherian by assumption we get that $A^n$ is Noetherian. Thus, the result holds by induction.
\end{proof}

\begin{lemma} \label{finitely_generated_noetherian_quotient}
Let $K \subset A$ be a finitely generated ideal such that $A / K$ is Noetherian then any ideal of $A$ above $K$ is finitely generated.
\end{lemma}

\begin{proof}
Let $I \supset K$ be an ideal and $\bar{I}$ its image under $\pi : A \to A / K$. Since $A / K$ is Noetherian we know that $\bar{I}$ is finitely generated. Write $\bar{I} = (A/K) x_1 + \cdots + (A / K) x_n$ for $x_n \in I$ (i.e. I choose some lifts to $A$ of the generators). Let $J = A x_1 + \cdots + A x_n$. I claim that $I = J + K$. Clearly, $J \subset I$ and $K \subset I$ so $J + K \subset I$. Furthermore, if $x \in I$ then $\bar{x} \in \bar{I}$ so we can write,
\[ \bar{x} = \bar{a}_1 x_1 + \cdots + \bar{a}_n x_n \]
and therefore, $\pi(a_1 x_1 + \cdots + a_n x_n - x) = 0$ which implies that,
\[ a_1 x_1 + \cdots + a_n x_n - x \in \ker{\pi} = K \]
proving the claim. However, both $J$ and $K$ are finitely generated and thus $I$ is as well. 
\end{proof}

\begin{lemma} \label{irreducibility_criterion}
A topological space $X$ is irreducible if and only if it cannot be written as the union of proper closed sets.
\end{lemma}

\begin{proof}
First, note that $Z_1 \cup Z_2 = X \iff (X \setminus Z_1) \subset Z_2$ because both express the condition that if $x \notin Z_1$ then $x \in Z_2$. 
\bigskip\\
Let $Z_1, Z_2 \subset X$ be proper closed sets. If $Z_1 \cup Z_2 = X$ then $(X \setminus Z_1) \subset Z_2$ which is proper so $(X \setminus Z_1)$ is not dense and thus $X$ is not irreducible. Suppose that for every choice of $Z_1, Z_2 \subset X$ that $Z_1 \cup Z_2 \neq X$. For an nonempty open set $U$ take $Z_1 = X \setminus U$ which is proper and closed. Then if $U \subset Z_2$ for closed $Z_2$ we know that $Z_1 \cup Z_2 = X$ so $Z_2$ must be non proper and thus $\overline{U} = X$ so $X$ is irreducible.  
\end{proof}

\end{document}