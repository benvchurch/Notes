\documentclass[12pt]{extarticle}
\usepackage[utf8]{inputenc}
\usepackage[english]{babel}
\usepackage[a4paper, total={6in, 9in}]{geometry}
\usepackage{tikz-cd}
 
\usepackage{amsthm, amssymb, amsmath, centernot}


\newcommand{\notimplies}{%
  \mathrel{{\ooalign{\hidewidth$\not\phantom{=}$\hidewidth\cr$\implies$}}}}

\renewcommand\qedsymbol{$\square$}
\newcommand{\cont}{$\boxtimes$}
\newcommand{\divides}{\mid}
\newcommand{\ndivides}{\centernot \mid}
\newcommand{\Z}{\mathbb{Z}}
\newcommand{\N}{\mathbb{N}}
\newcommand{\C}{\mathbb{C}}
\newcommand{\Zplus}{\mathbb{Z}^{+}}
\newcommand{\Primes}{\mathbb{P}}
\newcommand{\ball}[2]{B_{#1} \! \left(#2 \right)}
\newcommand{\Q}{\mathbb{Q}}
\newcommand{\R}{\mathbb{R}}
\newcommand{\Rplus}{\mathbb{R}^+}
\newcommand{\invI}[2]{#1^{-1} \left( #2 \right)}
\newcommand{\End}[1]{\text{End}\left( A \right)}
\newcommand{\legsym}[2]{\left(\frac{#1}{#2} \right)}
\renewcommand{\mod}[3]{\: #1 \equiv #2 \: \mathrm{mod} \: #3 \:}
\newcommand{\nmod}[3]{\: #1 \centernot \equiv #2 \: \mathrm{mod} \: #3 \:}
\newcommand{\ndiv}{\hspace{-4pt}\not \divides \hspace{2pt}}
\newcommand{\finfield}[1]{\mathbb{F}_{#1}}
\newcommand{\finunits}[1]{\mathbb{F}_{#1}^{\times}}
\newcommand{\ord}[1]{\mathrm{ord}\! \left(#1 \right)}
\newcommand{\quadfield}[1]{\Q \small(\sqrt{#1} \small)}
\newcommand{\vspan}[1]{\mathrm{span}\! \left\{#1 \right\}}
\newcommand{\galgroup}[1]{Gal \small(#1 \small)}
\newcommand{\Aut}[1]{\mathrm{Aut} \small(#1 \small)}
\newcommand{\ints}[1]{\mathcal{O}_{#1}}
\newcommand{\sm}{\! \setminus \!}
\newcommand{\norm}[3]{\mathrm{N}^{#1}_{#2}\left(#3\right)}
\newcommand{\qnorm}[2]{\mathrm{N}^{#1}_{\Q}\left(#2\right)}
\newcommand{\quadint}[3]{#1 + #2 \sqrt{#3}}
\newcommand{\pideal}{\mathfrak{p}}
\newcommand{\inorm}[1]{\mathrm{N}(#1)}
\newcommand{\tr}[1]{\mathrm{Tr} \! \left(#1\right)}
\newcommand{\delt}{\frac{1 + \sqrt{d}}{2}}
\newcommand{\ch}[1]{\mathrm{char} \: #1}
\renewcommand{\Im}[1]{\mathrm{Im}(#1)}
\newcommand{\minimal}[2]{\mathrm{Min}(#1;#2)}
\newcommand{\fix}[2]{\mathrm{Fix}_{#1} (#2)}
\newcommand{\id}{\mathrm{id}}
\renewcommand{\empty}{\varnothing}
\newcommand{\Tor}[4]{\mathrm{Tor}^{#1}_{#2} \left( #3, #4 \right)}
\newcommand{\Ext}[4]{\mathrm{Ext}^{#1}_{#2} \left( #3, #4 \right)}
\newcommand{\Homover}[3]{\mathrm{Hom}_{#1} \left( #2, #3 \right)}
\newcommand{\Frac}[1]{\mathrm{Frac}\left(#1\right)}

\newcommand{\U}[1]{\mathrm{U}(#1)}
\renewcommand{\O}[1]{\mathrm{O}(#1)}
\newcommand{\SU}[1]{\mathrm{SU}(#1)}
\newcommand{\SO}[1]{\mathrm{SO}(#1)}
\newcommand{\GL}[2]{\mathrm{GL}_{#1}(#2)}
\newcommand{\SL}[2]{\mathrm{SL}_{#1}(#2)}
\newcommand{\PGL}[2]{\mathrm{PGL}_{#1}(#2)}
\newcommand{\PSL}[2]{\mathrm{PSL}_{#1}(#2)}


\newcommand{\Hom}[2]{\mathrm{Hom}\left(#1, #2 \right)}
\newcommand{\Mod}[1]{\mathbf{Mod}_{#1}}
\newcommand{\Grp}{\mathbf{Grp}}
\newcommand{\AbGrp}{\mathbf{AbGrp}}
\newcommand{\Ring}{\mathbf{Ring}}

\newcommand{\Ann}[2]{\mathrm{Ann}_{#1}\left(#2\right)}
\newcommand{\Ass}[2]{\mathrm{Ass}_{#1}\left( #2 \right)}
\newcommand{\supp}[2]{\mathrm{Supp}_{#1} \left( #2 \right) }
\newcommand{\Supp}[2]{\mathrm{Supp}_{#1}\left(#1\right)}
\newcommand{\spec}[1]{\mathrm{Spec}\left( #1 \right)}
\newcommand{\Spec}[1]{\mathrm{Spec}\left( #1 \right)}
\newcommand{\rad}[1]{\mathrm{rad}\left( #1 \right)}
\newcommand{\nilrad}[1]{\mathrm{nilrad}\left( #1 \right)}
\newcommand{\gr}[2]{\mathbf{gr}_{#1}\left(#2\right)}

\newcommand{\ev}{\mathrm{ev}}
\newcommand{\p}{\mathfrak{p}}
\renewcommand{\P}{\mathfrak{P}}
\newcommand{\q}{\mathfrak{q}}
\newcommand{\m}{\mathfrak{m}}

\theoremstyle{remark}
\newtheorem*{remark}{Remark}

\theoremstyle{definition}
\newtheorem{theorem}{Theorem}[section]
\newtheorem{lemma}[theorem]{Lemma}
\newtheorem{proposition}[theorem]{Proposition}
\newtheorem{corollary}[theorem]{Corollary}


\newenvironment{definition}[1][Definition:]{\begin{trivlist}
\item[\hskip \labelsep {\bfseries #1}]}{\end{trivlist}}


\newenvironment{lproof}{\begin{proof} \renewcommand{\qedsymbol}{}}{\end{proof}}

\newcommand{\atitle}[1]{\title{% 
	\large \textbf{Mathematics GR6261 Commutative Algebra
	\\ Assignment \# #1} \vspace{-2ex}}
\author{Benjamin Church }
\maketitle}



\begin{document}
\atitle{3}
 
\section*{Problem 1}

Let $B$ be an $A$-algebra such that,
\[ \sum_{f \in \Homover{A}{B}{A}} f(B) = A \]
where $\Homover{A}{B}{A}$ are all $A$-linear maps as $A$-modules (not necessarily ring maps). Therefore, for each $x \in A$ there exists a finite list of maps $f_i \in \Homover{A}{B}{A}$ and elements $b_i \in B$ such that $f_1(b_1) + \cdots + f_n(b_n) = x$. Therefore, the evaluation map,
\[ \ev : \Homover{A}{B}{A} \otimes_A B \to A \]
given by $f \otimes b \mapsto f(b)$ is surjective. In particular, there exists some,
\[ e = \tilde{f}_1 \otimes \tilde{b}_1 + \cdots + \tilde{f}_n \otimes \tilde{b}_n \in \Homover{A}{B}{A} \otimes_A B \]
such that $\ev(e) = 1_A$ i.e.
\[ \tilde{f}_1(\tilde{b}_1) + \cdots \tilde{f}_n(\tilde{b}_n) = 1_A \]
Furthermore, $\Homover{A}{B}{A} \otimes_A B$ is a $B$-module via extension of scalars so the map $q : B \to \Homover{A}{B}{A} \otimes_A B$ via, 
\[ b \mapsto b \cdot e = \tilde{f}_1 \otimes (b \tilde{b}_1) + \cdots + \tilde{f}_n \otimes (b \tilde{b}_n) \]
is a $B$-linear map. Thus, $r = \ev \circ q$ is $A$-linear and,
\[ r(1_B) = \ev \circ q(1_B) = \ev(1_B \cdot e) = \ev(e) = 1_A\]
For $a \in A$ we have $r(a \cdot 1_B) = a \cdot r(1_B) = a \cdot 1_A = a$. Let $\iota : A \to B$ denote the canonical map induced by the $A$-algebra structure of $B$ given by $a \mapsto a \cdot 1_B$. Therefore we have $r \circ \iota = \id_A$ which is a retraction of $B \to A$. Firstly, this implies that $\iota : A \to B$ is injective so we may view $A$ as a subring of $B$ i.e. we can write $A \subset B$ and thus $r|_A = \id_A$. 
\bigskip\\
Now for an ideal $I \subset A$ consider,
\[ r(I \cdot B \cap A) \subset r(I \cdot B) \cap r(A) = I \cdot r(B) \cap A = I \cdot A \cap A = I \]
Thus, if $x \in I \cdot B \cap A$ then $x \in A$ so $r(x) = x$ and therefore $I \cdot B \cap A \subset I$. However, if $x \in I$ then $x \in A$ and $x = x \cdot 1_B \in I \cdot B$ so $x \in I \cdot B \cap A$. Therefore $I \cdot B \cap A = I$ proving property (i). Lemma \ref{intersection_ideal_property_implies_spec_surjection} shows that property (i) implies property (ii) i.e. the map $\iota^* : \Spec{B} \to \Spec{A}$ is surjective. 

\section*{Problem 2}

Let $k$ be a field and $A = k[t]_{(t)}$. Then consider the ideal $(tX - 1) \subset A[X]$. Since $k[t]_{(t)}$ is a local domain with maximal ideal $t k[t]_{(t)}$ we know that adjoining $t^{-1}$ gives the entire field of fractions $k(t)$. Thus, the map $\ev_{t^{-1}} : A[X] \to k(t)$ given by evaluation at $t^{-1}$ is surjective. Furthermore, the polynomial $p = tX - 1 \in A[X]$ is clearly in the kernel $\ker{\ev_{t^{-1}}}$ and if $p \in A[X]$ is in the kernel then $p \in k(t)[X]$ with $p(t^{-1}) = 0$ so because $k(t)$ is a field, we can write, $p(X) = (t X - 1) g(X)$ where $g \in k(t)[X]$. But then $p(X) = t X g(X) - g(X) \in A[X]$ so the linear coefficient must be in $A$ i.e. $g_0 \in A$. Suppose that $g_i \in A$ then $p_{i+1} = t g_i - g_{i+1}$ so $g_{i+1} = t g_i - p_{i + 1} \in A[X]$ and thus by induction $g \in A[X]$ since each coefficient is in $A$. Thus, $p \in (t X - 1) A[X]$. Therefore, $\ker{\ev_{t^{-1}}} = (t X - 1)$ so by the first isomorphism theorem,
\[ A[X] / (t X - 1) \cong k(t) \]
which implies that $(t X - 1)$ is a maximal ideal since $k(t)$ is a field. Suppose the ideal $(t X - 1) \cap A$ is not maximal then it is contained in some maximal ideal $\m \subset A$. However, if there existed a prime ideal $\P \supset (t X - 1) A[X]$ such that $\P \cap A = \m$ then, since $\m$ is strictly larger then $(t X - 1) \cap A$, we would have
$\P \supsetneq (t X - 1) A[X]$ contradicting the maximality of $(t X - 1) A[X]$. Therefore, if $(t X - 1) \cap A$ is not maximal then the going-up property fails. However, $(t X - 1) \cap A = (0)$ because every element of $(t X - 1)$ has $X$-degree at least $1$. Furthermore, $k[t]_{(t)}$ has a nonzero maximal ideal $t k[t]_{(t)}$ so $(t X - 1) \cap A = (0)$ is not maximal. Thus, the going-up property fails. 
However, $A[X]$ is always free over $A$ with a decomposition as a graded $A$-module,
\[ A[X] \cong \bigoplus_{n = 0}^{\infty} A x^n \]
with each factor naturally isomorphic to $A$. 
\section*{Problem 3}


Let $A \subset B$ be an integral extension of rings such that a unique prime $\P$ of $B$ lies above the prime $\p \subset A$. By Lemma \ref{integral_extensions_localizations}, the extension $A_{\p} \subset B_{\p} = B \otimes_A A_{\p}$ is integral. Notate the map $\iota : A_{\p} \to B \otimes_A A_{\p}$ then the map $\iota^* : \Spec{B \otimes_A A_{\p}} \to \Spec{A_{\p}}$ is surjective and has the property that the maximal primes of $B_{\p}$ are exactly those above the maximal ideals of $A_{\p}$ by Cohen's theorem. However, $A_{\p}$ is local with unique maximal ideal $\p A_{\p}$ and the ideal $\P B_{\p}$ lies above $\p A_{\p}$. By Lemma \ref{prime_ideals_of_localization}, the primes of $B_{\p}$ are exactly those primes $\P B_{\p} \in \spec{B}$ which are disjoint from $S = A \setminus \p$ and thus $\P \cap A \subset \p$. Therefore, any prime of $B_{\p}$ above $\p A_{\p}$ is of the form $\P B_{\p}$ with $\P \cap A = \p$ and there is a unique such $\P$ by assumption. Therefore, $B_{\p}$ has a unique maximal ideal $\P B_{\p}$. However, $B_{\p} = S^{-1} A$ where $S = A \setminus \p$ so if $B \to C$ maps all $B \setminus \P$ to units then it maps $A \cap (B \setminus \P) = A \setminus \p$ to units. Thus $B \to C$ factors through $B_{\p} \to C$ by the universal property of localization. Furthermore, the map $B \to B_{\p}$ sends $B \setminus \P$ to units because $B_{\p}$ is local with unique maximal ideal $\P B_{\p}$. Thus, $B_\p$ is the initial object in the category of $B$-algebras with canonical homomorphisms $B \to C$ sending $B \setminus \P$ to units which is the universal property of the localization $B_{\P}$. This universal property uniquely fixes $B_{\p} \cong B_{\P}$ up to unique $B$-isomorphism.   

\section*{Problem 4}

Let $A \subset B \subset Q$ be domains where $Q$ is the field of fractions of $A$ and thus also of $B$. If $A = B$ then clearly $B$ is faithfully-flat over $A$ since $(-) \otimes_A B = (-) \otimes_A A$ is the trivial functor.
\bigskip\\
Conversely, suppose that $B$ is faithfully flat over $A$. Consider the exact sequence of $A$-modules,
\begin{center}
\begin{tikzcd}
0 \arrow[r] & A \arrow[r, "\iota"] & B \arrow[r, "\pi"] & B / A \arrow[r] & 0
\end{tikzcd}
\end{center}
and apply the functor $(-) \otimes_A B$ which is exact since $B$ is flat to get the exact sequence,
\begin{center}
\begin{tikzcd}
0 \arrow[r] & A \otimes_A B \arrow[r, "\iota \otimes \id_B"] & B \otimes_A B \arrow[r, "\pi \otimes \id_B"] \arrow[l, "m", dashed, bend left] & B / A \otimes_A B \arrow[r] & 0
\end{tikzcd}
\end{center}
in which there exists the section $m : B \otimes_B B \to A \otimes_A B$ given by $b \otimes b' \mapsto 1 \otimes bb'$. This map is well-defined by the universal property of the tensor product since the multiplication on tensors is bilinear. Furthermore,
\[ m \circ (\iota \otimes \id_B) : a \otimes b \mapsto a \otimes b \mapsto 1 \otimes ab = a \otimes b \]
and thus $m \circ (\iota \otimes \id_B) = \id_{A \otimes_A B}$ so the sequence is left-split. Since $A \otimes_A B \cong B$, the splitting gives an isomorphism,
\[ B \otimes_A B \cong B \oplus \left( B/A \otimes_A B \right) \]
which implies that there is an embedding,
\begin{center}
\begin{tikzcd}[column sep = huge]
B / A \otimes_A B \arrow[r, hook] & B \otimes_A B 
\end{tikzcd}
\end{center}
However, I claim that $B / A \otimes_A B$ is a torsion $A$-module. Since $B \subset Q$ we can write an arbitrary element of $B$ as some fraction $\frac{x}{y}$ for $x, y \in A$. Then,
\[ y \cdot ([\tfrac{x}{y}] \otimes b) = [y \cdot \tfrac{x}{y}] \otimes b = [x] \otimes b = 0\]
since $x \in A$. Furthermore, we can clear the denominators of a sum of tensors by taking the products of the denominators. Thus, $B / A \otimes_A B$ is entirely $A$-torsion so it is embedded into the $A$-torsion submodule of $B \otimes_A B$. However, $B$ is $A$-flat so $B \otimes_A B$ is also $A$-flat since the composition of exact functors is exact. Therefore $B \otimes_A B$ is torsion-free by Corollary \ref{torsion_free} so we have,
\begin{center}
\begin{tikzcd}
B / A \otimes_A B \arrow[r, hook] & T_A(B \otimes_A B) = 0
\end{tikzcd}
\end{center}
and thus $B / A \otimes_A B = 0$. However, $B$ is faithfully-flat over $A$ so this implies that $B / A = 0$ and thus $A = B$. 


\section*{Problem 5}

Let $A$ be a ring and $B$ a Noetherian $A$-algebra such that the going-up property holds for the canonical map $\iota : A \to B$. Consider the map $\iota^* : \Spec{B} \to \Spec{A}$. An arbitrary closed set of $\Spec{B}$ takes the form $V(I)$ where $I \subset B$ is an ideal. Since $B$ is Noetherian, $I$ admits a primary decomposition,
\[ I = Q_1 \cap \cdots \cap Q_n \]
with associated primes $P_i = \sqrt{Q_i}$. Therefore,
\[ V(I) = V\left( \bigcap_{i = 1}^n Q_i \right) = \bigcup_{i = 1}^n V(Q_i) \]
Furthermore, by Lemma \ref{radical_generates_same_set}, $V(Q_i) = V(P_i)$ and thus,
\[ V(I) = \bigcup_{i = 1}^n V(P_i) \]
Since finite unions of closed sets are closed, it suffices to prove that $\iota^*(V(\P))$ is closed for $\P \subset B$ prime. If $\P' \supset \P$ then $\iota^*(\P') \supset \iota^*(\P)$ so $\iota^*(\P') \in V(\iota^*(\P))$. Therefore, $\iota^*(V(\P)) \subset V(\iota^*(\P))$. Furthermore, suppose we have $\p \in V(\iota^*(\P))$ i.e. we have $\p \supset \iota^*(\P)$. By the going-up property there exists a prime $\P' \supset \P$ such that $\iota^*(\P') = \p$ and thus $\p \in \iota^*(V(\P))$. Therefore, $\iota^*(V(\P)) = V(\iota^*(\P))$ and is closed. 

\section{Lemmata}

\begin{lemma} \label{intersection_ideal_property_implies_spec_surjection}
Let $B$ be an $A$-algebra where $\iota^{-1}(I \cdot B) = I$ for any ideal $I \subset A$. Then, the canonical map $\iota : A \to B$ induces a surjection $\iota^* : \Spec{B} \to \Spec{A}$. 
\end{lemma}

\begin{proof}
Consider the subset of ideals,
\[ \Sigma = \{ I \subset B \mid \iota^{-1}(I) = \p \} \]
which is a poset under inclusion. Furthermore, $\p \cdot B \in \Sigma$ since, by hypothesis, $\iota^{-1}(\p \cdot B) = \p$. Thus $\Sigma$ is nonempty. Suppose that $\mathcal{I} \subset \Sigma$ is a chain and consider,
\[ U = \bigcup_{I \in \mathcal{I}} I \]
This is an ideal because if $x, y \in U$ then $x \in I$ and $y \in I'$ for $I, I' \in \mathcal{I}$ but $\mathcal{I}$ is totally ordered so either $I \subset I'$ or $I' \subset I$ and thus the larger contains both $x$ and $y$ and therefore the sum and the multiples are in $U$. 
Furthermore,
\[ \iota^{-1}(I) = \iota^{-1} \left( \bigcup_{I \in \mathcal{I}} I \right) = \bigcup_{I \in \mathcal{I}} \iota^{-1}(I) = \bigcup_{I \in \mathcal{I}} \p = \p \]
so $U \in \Sigma$. However, $\forall I \in \mathcal{I} : I \subset U$. Since every chain has a maximum, by Zorn's Lemma there exist maximal elements of $\Sigma$ above every ideal $I \in \Sigma$. Suppose $\m \in \Sigma$ is maximal and $x,y \notin \m$. Then, by maximality, $(x) + \m \notin \Sigma$ and $(y) + \m \notin \Sigma$. Therefore, $\iota^{-1}(x + \m) \supsetneq \p$ and $\iota^{-1}(y + \m) \supsetneq \p$ since $\iota^{-1}(\m) = \p$ and otherwise these ideals would be elements of $\Sigma$ contradicting the maximality of $\m$. Thus, there exists $x', y' \in A \setminus \p$ such that $\iota(x') \in (x) + \m$ and $\iota(y') \in (y) + \m$. Since $\p$ is prime, $x'y' \in A \setminus \p$ and furthermore $\iota(x'y') \in [(x) + \m][(y) + \m] \subset (xy) + \m$ so $\iota^{-1}(xy + \m) \supsetneq \p$. Therefore $xy \notin \m$ so $\m$ is prime. Thus, there exists a prime $\m \in \Spec{B}$ such that $\iota^{-1}(\m) = \p$ for each prime $\p \in \Spec{A}$ so the map $\iota^* : \Spec{B} \to \Spec{A}$ is surjective. 
\end{proof}

\begin{lemma}
Let $A \subset B$ be an integral extension and $S \subset A$ be a multiplicative subset then $S^{-1} A \subset S^{-1} B$ is integral. 
\end{lemma}

\begin{proof}
For any $(b, s) \in S^{-1} B$ we have $b \in B$ so, because $B$ is integral over $A$, $b$ satisfies a monic polynomial $p \in A[x]$
given by,
\[ p(x) = x^n + a_{n-1} x^{n-1} + \cdots + a_0 \]
Then consider the monic polynomial $p' \in S^{-1}A[x]$ given by,
\[ p'(x) = x^n + \frac{a_{n-1}}{s} x^{n-1} + \cdots + \frac{a_0}{s^n} \]
Then we have,
\[ p'\left( \frac{b}{s} \right) = \left( \frac{b}{s} \right)^n + \frac{a_{n-1} b}{s^n} + \cdots + \frac{a_0}{s^n} = \frac{p(b)}{s^n} = 0 \]
Thus $S^{-1} B$ is integral over $S^{-1} A$.  
\end{proof}

\begin{lemma}
Take an extension of rings $A \subset B$. Let $\p \subset A$ be a prime and $S = A \setminus \p$. Then $B \otimes_A A_{\p} = B_{\p} = S^{-1}B$ as $A$-algebras where $A_{\p} = S^{-1} A$. 
\end{lemma}

\begin{proof}
Consider the map $B \otimes_A A_{\p} \to B_{\p}$ given by $b \otimes (a, s) \mapsto (ba, s)$. This map is clearly $A$-linear and a well-defined ring map since acting by $r \in A$ on either factor maps to $r \cdot (ba, s) = (rba, s)$. Furthermore, the map is injective since if $(ba, s) = 0$ then there exists $t \in S$ such that $tba = 0$ in which case $b \otimes (a, s) = b \otimes (at, st) = bat \otimes (1, st) = 0$ since these are $A$-modules and $at \in A$. Finally the map is surjective because $b \otimes (1, s) = (b, s)$ which is an arbitrary element of $B_{\p}$. 
\end{proof}

\begin{corollary} \label{integral_extensions_localizations}
If $B$ is integral over $A$ then $B_{\p} = B \otimes_A A_{\p}$ is integral over $A_{\p}$. 
\end{corollary}

\begin{proof}
Let $S = A \setminus \p$. Since $A \subset B$ is an integral extension then $A_{\p} = S^{-1} A \subset S^{-1} B = B_{\p} = B \otimes_A A_{\p}$ is also an integral extension by the above lemmas.
\end{proof}

\begin{lemma} \label{ideals_of_localization}
Let $S \subset A$ be multiplicative. The proper ideals $I \subset S^{-1} A$ are exactly $I = S^{-1} J$ with $J = \iota^{-1}(I)$ where $J \cap S = \varnothing$ and $\iota : A \to S^{-1} A$ is the canonical map. 
\end{lemma}

\begin{proof}
Let $I \subset S^{-1} A$ be an ideal. Consider the ideal $J = \iota^{-1}(I)$. If $(a, s) \in I$ then $(as, s) = (a, 1) \in I$ so $a \in J$. Thus, $(a, s) \in S^{-1} J$. Furthermore, $S^{-1} J \subset I$ since $\iota(J) \subset I$ and $I$ is an ideal. Therefore, $I = S^{-1} J$. However, if $s \in J \cap S$ then $\iota(s) = (s, 1) \in I$ but then $(1, s) \cdot (s, 1) = (s, s) = (1, 1) \in I$ so $I = S^{-1} A$. Thus if $I \subset S^{-1} A$ is proper then $J \cap S = \varnothing$. 
\end{proof}

\begin{lemma} \label{prime_ideals_of_localization}
Let $S \subset A$ be multiplicative. Then, the canonical map $\iota : A \to S^{-1}A$ induces an injection $\iota^* : \Spec{S^{-1}A} \to \Spec{A}$ which which is a homeomorphism onto its image,
\[ \iota^*(\Spec{S^{-1} A}) = \Spec{A} |_{S^{c}} = \{ \p \in \Spec{A} \mid \p \cap S = \varnothing \} \]
\end{lemma}

\begin{proof} 
Since any prime $\P \subset S^{-1} A$ is, by definition, proper, by Lemma \ref{ideals_of_localization}, its image under $\iota^*$ is disjoint from $S$. Thus, we may restrict $\iota^*$ to a map,
\[ \iota^* : \Spec{S^{-1} A} \to \Spec{A} |_{S^{c}} \]
I claim that $\iota_* : \Spec{A} |_{S^{c}} \to \Spec{S^{-1} A}$ given by $\iota_*(\p) = \iota(\p) S^{-1} A = S^{-1} \iota(\p)$ is a well-defined continuous inverse of $\iota^*$. First, if $\p \in \Spec{A} |_{S^{c}}$ then suppose that $(a_1, s_1), (a_2, s_2) \in S^{-1}A$ such that $(a_1 a_2, s_1 s_2) \in \iota_*(\p)$. This implies that $(a_1 a_2, s_1 s_2) = (x, s)$ for $x \in \p$ and thus $\exists u \in S$ such that,
\[ u \left( x s_1 s_2 - a_1 a_2 s \right) = 0 \implies a_1 s_2 su = x s_1 s_2 u \implies  a_1 a_2 su \in \p \]
However, $\p \cap S = \varnothing$ so $su \notin \p$. Since $\p$ is prime we have $a_1 a_2 \in \p$ which implies that $a_1 \in \p$ or $a_2 \in \p$. Thus, $(a_1, s_1) \in S^{-1} \p = \iota_*(\p)$ or $(a_2, s_2) \in \iota_*(\p)$. Therefore, $\iota_*(\p) \in \Spec{S^{-1} A}$. 
\bigskip\\
Furthermore, any closed set of $\Spec{B}$ is of the form $V(I)$ where $I \subset B$ is an ideal. Consider $(\iota_*)^{-1}(V(I))$. By Lemma \ref{ideals_of_localization}, we have $I = S^{-1} J$ with $J = \iota^{-1}(I)$. If $\p \in V(J)$ i.e. $\p \supset J$ then $\iota_*(\p) = S^{-1} \p \supset S^{-1} J = I$ so $\p \in (\iota_*)^{-1}(V(I))$. Conversely, if $\iota_*(\p) \in V(I)$ then $S^{-1} \p \supset S^{-1} J$ which implies that $\iota^{-1}(S^{-1} \p) \supset \iota^{-1}(S^{-1} J) \supset J$. However, if $\iota(a) \in S^{-1} \p$ then $\iota(a) = (a, 1)$ so $(a, 1) = (x, s)$ with $x \in \p$. Thus, there exists $u \in S$ such that,
\[ u \left( x - as \right) = 0 \implies asu = ux \in \p \]
but $\p \cap S = \varnothing$ and $\p$ is prime so $a \in \p$. Furthermore if $a \in \p$ then clearly $\iota(a) = (a, 1) \in S^{-1} \p$. Therefore, $\iota^{-1}(S^{-1} \p) = \p$ which, returning to the above, implies that $\p \supset J$ and thus $\p \in V(I)$. Therefore, 
\[ (\iota_*)^{-1}(V(I)) = V(J) = V(\iota^{-1}(I)) \]
which is closed. Thus the map $\iota_* : \Spec{A}|_{S^c} \to \Spec{S^{-1} A}$ is continuous. 
\bigskip\\
Consider $\iota_* \circ \iota^*(\P) = S^{-1} \iota^{-1}(\P) = \P$ by Lemma \ref{ideals_of_localization}. Also, $\iota^* \circ \iota_*(\p) = \iota^{-1}(S^{-1} \p) = \p$ as shown above. Therefore, $\iota_*$ and $\iota^*$ are continuous inverses.
\end{proof}

\begin{lemma} \label{torsion_tor}
Let $M$ be an $A$-module and $a \in A$ then the $a$-torsion is related to the Tor functor via,
\[ T_a(M) = \Tor{A}{1}{A / (a)}{M} \]
where the $a$-torsion submodule is defined as, 
\[ T_a(M) = \{ m \in M \mid a \cdot m = 0 \} \]
and for $i > 1$ the Tor vanishes i.e.
\[ \Tor{A}{i}{A /(a)}{M} = 0 \]
\end{lemma}

\begin{proof}
Consider the exact sequence,
\begin{center}
\begin{tikzcd}
0 \arrow[r] & A \arrow[r, "\times a"] & A \arrow[r] & A / (a) \arrow[r] & 0
\end{tikzcd}
\end{center}
Applying the long exact sequence of the derived functor $\Tor{A}{\ast}{-}{M}$ we get,
\begin{center}
\begin{tikzcd}
\Tor{A}{1}{A}{M} \arrow[r] \arrow[d] & \Tor{A}{1}{A / (a)}{M} \arrow[r] \arrow[d] & A \otimes_A M \arrow[r, "(\times a) \otimes \id_M"] \arrow[d] & A \otimes_A M \arrow[d] 
\\
0 \arrow[r] & \Tor{A}{1}{A / (a)}{M} \arrow[r] & M \arrow[r, "a \cdot"] & M 
\end{tikzcd}
\end{center} 
where the vertical arrows are isomorphisms since $A \otimes_A M \cong M$ and $\Tor{A}{1}{A}{M} = 0$ since $A$ is $A$-flat. Since $(\otimes a)\otimes \id_M$ sends $a' \otimes m \mapsto a a' \otimes m$ and thus we get the map $a' m \mapsto aa' m$ from $M \to M$ under this isomorphism. Therefore, $\Tor{A}{1}{A / (a)}{M}$ is the kernel of multiplication by $a$ since the rows are exact.  
\end{proof}

\begin{corollary} \label{torsion_free}
If $M$ is a flat $A$-module then $M$ is $A$-torsion-free.
\end{corollary}

\begin{proof}
This follows immediately from Lemma \ref{torsion_tor} using the fact that $M$ is flat so $\Tor{A}{1}{M}{-} = 0$ and thus $T_a(M) = \Tor{A}{1}{M}{A / (a)} = 0$ for each $a \in A$. 
\end{proof}


\begin{lemma} \label{radical_generates_same_set}
For any ideal $I \subset A$,
\[ V(I) = V(\sqrt{I}) \]
\end{lemma}

\begin{proof} 
Suppose that $\p \in V(I)$ and then $\p \supset I$ so clearly,
\[ \p \supset \bigcap_{\p \supset I} \p = \sqrt{I} \]
and thus $\p \in V(\sqrt{I})$. Conversely, since,
\[ \sqrt{I} = \bigcap_{\p \supset I} \p \supset I \]
we know that $V(\sqrt{I}) \subset V(I)$. Therefore, $V(I) = V(\sqrt{I})$. 
\end{proof}

\end{document}