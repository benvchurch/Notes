\documentclass[12pt]{article}
\usepackage[utf8]{inputenc}
\usepackage[english]{babel}
\usepackage[a4paper, total={6in, 9in}]{geometry}
\usepackage{tikz-cd}
 
\usepackage{amsthm, amssymb, amsmath, centernot}


\newcommand{\notimplies}{%
  \mathrel{{\ooalign{\hidewidth$\not\phantom{=}$\hidewidth\cr$\implies$}}}}

\renewcommand\qedsymbol{$\square$}
\newcommand{\cont}{$\boxtimes$}
\newcommand{\divides}{\mid}
\newcommand{\ndivides}{\centernot \mid}
\newcommand{\Z}{\mathbb{Z}}
\newcommand{\N}{\mathbb{N}}
\newcommand{\C}{\mathbb{C}}
\newcommand{\Zplus}{\mathbb{Z}^{+}}
\newcommand{\Primes}{\mathbb{P}}
\newcommand{\ball}[2]{B_{#1} \! \left(#2 \right)}
\newcommand{\Q}{\mathbb{Q}}
\newcommand{\R}{\mathbb{R}}
\newcommand{\Rplus}{\mathbb{R}^+}
\newcommand{\invI}[2]{#1^{-1} \left( #2 \right)}
\newcommand{\End}[1]{\text{End}\left( A \right)}
\newcommand{\legsym}[2]{\left(\frac{#1}{#2} \right)}
\renewcommand{\mod}[3]{\: #1 \equiv #2 \: \mathrm{mod} \: #3 \:}
\newcommand{\nmod}[3]{\: #1 \centernot \equiv #2 \: \mathrm{mod} \: #3 \:}
\newcommand{\ndiv}{\hspace{-4pt}\not \divides \hspace{2pt}}
\newcommand{\finfield}[1]{\mathbb{F}_{#1}}
\newcommand{\finunits}[1]{\mathbb{F}_{#1}^{\times}}
\newcommand{\ord}[1]{\mathrm{ord}\! \left(#1 \right)}
\newcommand{\quadfield}[1]{\Q \small(\sqrt{#1} \small)}
\newcommand{\vspan}[1]{\mathrm{span}\! \left\{#1 \right\}}
\newcommand{\galgroup}[1]{Gal \small(#1 \small)}
\newcommand{\Aut}[1]{\mathrm{Aut} \small(#1 \small)}
\newcommand{\ints}[1]{\mathcal{O}_{#1}}
\newcommand{\sm}{\! \setminus \!}
\newcommand{\norm}[3]{\mathrm{N}^{#1}_{#2}\left(#3\right)}
\newcommand{\qnorm}[2]{\mathrm{N}^{#1}_{\Q}\left(#2\right)}
\newcommand{\quadint}[3]{#1 + #2 \sqrt{#3}}
\newcommand{\pideal}{\mathfrak{p}}
\newcommand{\inorm}[1]{\mathrm{N}(#1)}
\newcommand{\tr}[1]{\mathrm{Tr} \! \left(#1\right)}
\newcommand{\delt}{\frac{1 + \sqrt{d}}{2}}
\newcommand{\ch}[1]{\mathrm{char} \: #1}
\renewcommand{\Im}[1]{\mathrm{Im}(#1)}
\newcommand{\minimal}[2]{\mathrm{Min}(#1;#2)}
\newcommand{\fix}[2]{\mathrm{Fix}_{#1} (#2)}
\newcommand{\id}{\mathrm{id}}
\renewcommand{\empty}{\varnothing}
\newcommand{\Tor}[4]{\mathrm{Tor}^{#1}_{#2} \left( #3, #4 \right)}
\newcommand{\Ext}[4]{\mathrm{Ext}^{#1}_{#2} \left( #3, #4 \right)}
\newcommand{\Homover}[3]{\mathrm{Hom}_{#1} \left( #2, #3 \right)}
\newcommand{\Frac}[1]{\mathrm{Frac}\left(#1\right)}

\newcommand{\U}[1]{\mathrm{U}(#1)}
\renewcommand{\O}[1]{\mathrm{O}(#1)}
\newcommand{\SU}[1]{\mathrm{SU}(#1)}
\newcommand{\SO}[1]{\mathrm{SO}(#1)}
\newcommand{\GL}[2]{\mathrm{GL}_{#1}(#2)}
\newcommand{\SL}[2]{\mathrm{SL}_{#1}(#2)}
\newcommand{\PGL}[2]{\mathrm{PGL}_{#1}(#2)}
\newcommand{\PSL}[2]{\mathrm{PSL}_{#1}(#2)}


\newcommand{\Hom}[2]{\mathrm{Hom}\left(#1, #2 \right)}
\newcommand{\Mod}[1]{\mathbf{Mod}_{#1}}
\newcommand{\Grp}{\mathbf{Grp}}
\newcommand{\AbGrp}{\mathbf{AbGrp}}
\newcommand{\Ring}{\mathbf{Ring}}

\newcommand{\Ann}[2]{\mathrm{Ann}_{#1}\left(#2\right)}
\newcommand{\Ass}[2]{\mathrm{Ass}_{#1}\left( #2 \right)}
\newcommand{\supp}[2]{\mathrm{Supp}_{#1} \left( #2 \right) }
\newcommand{\Supp}[2]{\mathrm{Supp}_{#1}\left( #2 \right)}
\newcommand{\spec}[1]{\mathrm{Spec}\left( #1 \right)}
\newcommand{\Spec}[1]{\mathrm{Spec}\left( #1 \right)}
\newcommand{\rad}[1]{\mathrm{rad}\left( #1 \right)}
\newcommand{\nilrad}[1]{\mathrm{nilrad}\left( #1 \right)}
\newcommand{\gr}[2]{\mathbf{gr}_{#1}\left(#2\right)}

\newcommand{\depth}[2]{\mathrm{depth}_{#1}\left(#2\right)}

\newcommand{\ev}{\mathrm{ev}}
\newcommand{\p}{\mathfrak{p}}
\renewcommand{\P}{\mathfrak{P}}
\newcommand{\q}{\mathfrak{q}}
\newcommand{\m}{\mathfrak{m}}
\renewcommand{\a}{\mathfrak{a}}

\theoremstyle{remark}
\newtheorem*{remark}{Remark}
\newtheorem*{example}{Example}

\theoremstyle{definition}
\newtheorem{theorem}{Theorem}[section]
\newtheorem{lemma}[theorem]{Lemma}
\newtheorem{proposition}[theorem]{Proposition}
\newtheorem{corollary}[theorem]{Corollary}


\newenvironment{definition}[1][Definition:]{\begin{trivlist}
\item[\hskip \labelsep {\bfseries #1}]}{\end{trivlist}}


\newenvironment{lproof}{\begin{proof} \renewcommand{\qedsymbol}{}}{\end{proof}}


\begin{document}


\section{Flatness}

\begin{definition}
An $A$-module $Q$ is said to to be $A$-flat if $(-) \otimes_A Q$ is exact. Thus, $Q$ is $A$-flat iff $\Tor{A}{n}{-}{Q} = 0$ for $n > 0$. Furthermore if $\Tor{A}{1}{-}{Q} = 0$ then $(-) \otimes_A Q$ is exact by the long exact sequence. Thus, $Q$ is $A$-flat iff $\Tor{A}{1}{-}{Q} = 0$.   
\end{definition}

\begin{proposition}
\[ \Tor{A}{n}{\varinjlim M_i}{P} = \varinjlim \Tor{A}{n}{M_i}{P} \]
\end{proposition}

\begin{proof}
The functor $\varinjlim$ is exact. Furthermore, 
\begin{align*}
\Homover{A}{(\varinjlim M_i) \otimes_A P}{N} & = \Homover{A}{\varinjlim M_i}{\Homover{A}{P}{N}} = \varprojlim \Homover{A}{M_i}{\Homover{A}{P}{N}} 
\\
& = \varprojlim \Homover{A}{M_i \otimes_A P}{N} = \Homover{A}{\varinjlim (M_i \otimes_A P)}{N}
\end{align*} 
Then since the Yonenda embedding is injective,
\[ (\varinjlim M_i) \otimes_A P = \varinjlim (M_i \otimes_A P) \]
\end{proof}

\begin{proposition}
If $Q$ is projective then $Q$ is $A$-flat. 
\end{proposition}

\begin{proof}
Since $Q$ is projective $\Tor{A}{n}{-}{Q} = 0$ for $n > 0$. 
\end{proof}

\begin{proposition}
Let $M$ be an $A$-module then the following are equivalent.
\begin{enumerate}
\item The $A$-module $M$ is $A$-flat.

\item The functor $(-) \otimes_A M$ preserves monomorphisms.

\item Every finitely generated ideal $I \subset A$ satisfies $I \otimes_A M = I M$. 

\item $\Tor{A}{1}{M}{A/I} = 0$ for all finitely generated ideals $I \subset A$. 

\item $\Tor{A}{1}{M}{N} = 0$ for any finitely generated $A$-module $N$.

\item For all $a_i \in A$ and $x_i \in M$ with $\sum_{i = 1}^r a_i x_i = 0$ there exists $b_{ij} \in A$ such that $\sum_{i = 1}^r b_{ij} = 0$ for all $j$ and there exist $y_i \in M$ such that $x_i = \sum_{j = 1}^s b_{ij} y_j$. 
\end{enumerate}
\end{proposition}

\begin{proposition}
Let $B$ be an $A$-algebra which is flat as an $A$-module and $M$ is a $B$-flat $B$-module then $M$ is an $A$-flat $A$-module.
\end{proposition}

\begin{proof}
Let $S$ be an $A$-module. Then,
\[ S \otimes_A M = S \otimes_A (B \otimes_B M) = (S \otimes_A B) \otimes_B M \]
However, $(-) \otimes_A B$ and $(-) \otimes_B M$ are exact so the composition $(-) \otimes_A M$ is exact. 
\end{proof}


\begin{proposition}
Suppose $B$ is an $A$-algebra then if $M$ is $A$-flat then $B \otimes_A M$ is $B$-flat.
\end{proposition}

\begin{proof}
Suppose $S$ is a $B$-module then,
\[ S \otimes_B (B \otimes_A M) = (S \otimes_B B) \otimes_A M = S \otimes_A M \]
However, $(-) \otimes_A M$ is exact so $(-) \otimes_B (B \otimes_A M)$ is exact. 
\end{proof}

\begin{proposition}
If $S \subset A$ is multiplicative then $S^{-1} A$ is $A$-flat.
\end{proposition}

\begin{proof}
Notice that if $M$ is an $A$-module then $S^{-1} M \cong M \otimes_A S^{-1} A$ and localization is exact so $(-) \otimes_A S^{-1} A$ is exact.
\end{proof}

\begin{proposition}
Let $M,N$ be $A$-modules and assume $B$ is a flat $A$-algebra then,
\[ \Tor{B}{i}{M \otimes_A B}{N \otimes_A B} \cong \Tor{A}{i}{M}{N} \otimes_A B \]
and similarly,
\[ \Ext{i}{B}{M \otimes_A B}{N \otimes_A B} \cong \Ext{i}{A}{M}{N} \otimes_A B \]
\end{proposition}

\begin{proof}
Let $\mathbf{P} \to N \to 0$ be a projective resolution of $N$. Because $B$ is $A$-flat then $\mathbf{P} \otimes_A B \to N \otimes_A B \to 0$ is a projective resolution. Thus,
\begin{align*}
\Tor{B}{i}{M \otimes_A B}{N \otimes_A B} & = H_i((M \otimes_A B) \otimes_B (\mathbf{P} \otimes_A B))
\\
& = H_i((M \otimes_A \mathbf{P}) \otimes_A B) = \Tor{A}{i}{M}{N} \otimes_A B 
\end{align*} 
where again I have used the exactness of $(-) \otimes_A B$ to pull it out of the homology since it preserves kernels and images. 
\end{proof}

\begin{proposition}
Let $A$ be a local ring and $M$ a finitely generated $A$-module. Then the following are equivalent,
\begin{enumerate}
\item $M$ is free

\item $M$ is projective

\item $M$ is flat
\end{enumerate}
\end{proposition}

\begin{proof}
The first and second implications are true in general. Suppose $\m \subset A$ is the maximal ideal and $k = A / \m$. Then $M \otimes_A k = M / (\m M)$ is a finite-dimensional $k$-vectorspace. There exist $x_1, \dots, x_r \in M$ such that their image $\bar{x}_1, \dots, \bar{x}_r \in M$ is a basis of $M \otimes_A k$. Consider the  span map $\phi : A^r \to M$ then $\phi \otimes \id : k^r \to M \otimes_A k = M / (\m M)$ is surjective so $\Im{\phi} + \m M = M$. By Nakayama, $M = \Im{\phi}$.  
\end{proof}

\begin{lemma}
Let $\phi : A \to B$ be a ring map. Take $\mathfrak{P} \in \spec{A}$ and $\p = \phi^{-1}(\mathfrak{P})$ and $N$ an $A$-module. Then,
\[ \Tor{A_{\p}}{i}{B_{\mathfrak{P}}}{N_{\p}} = \Tor{A}{i}{B}{N}_{\mathfrak{P}} \]
\end{lemma}

\begin{proposition}
Let $\phi : A \to B$ be a ring map then the following are equivalent,
\begin{enumerate}
\item $B$ is $A$-flat
\item $B_{\mathfrak{P}}$ is $A_{\p}$-flat for all primes $\p = \phi^{-1}(\mathfrak{P})$
\item $B_{\mathfrak{P}}$ is $A_{\p}$-flat for all maximal ideals $\p = \phi^{-1}(\mathfrak{P})$
\end{enumerate}
\end{proposition}

\begin{proof}
First, $B_\p = B \otimes_A A_\p$ which is clearly flat over $A_\p$ by change of base. Furthermore, $B_{\mathfrak{P}}$ is flat over $B_\p$ because $B_{\mathfrak{P}} = S^{-1} B_\p$ for $S = B_\p \setminus \mathfrak{P} B_\p$. By transitivity, $B_{\mathfrak{P}}$ is $A_\p$-flat. Clearly, the second implies the third. Take $Q = \Tor{A}{i}{B}{N}$ using the above lemma,
\[ Q_{\mathfrak{P}} = \Tor{A_\p}{i}{B_{\mathfrak{P}}}{N_\p} = 0 \]
because $B_{\mathfrak{P}}$ is $A_\p$-flat. Thus, $\forall \mathfrak{P} \in \spec{A}$ which are maximal we have $Q_{\mathfrak{P}} = 0$ which implies that $Q = 0$. 
\end{proof}

\begin{definition}
Let $M$ be an $A$-module. We say that $M$ is \textit{faithfully flat} over $A$ if the sequence,
\begin{center}
\begin{tikzcd}
N \arrow[r] & P \arrow[r] & Q
\end{tikzcd}
\end{center}
is exact if and only if the sequence,
\begin{center}
\begin{tikzcd}
N \otimes_A M \arrow[r] & P \otimes_A M \arrow[r] & Q \otimes_A M
\end{tikzcd}
\end{center}
is exact.  
\end{definition}

\begin{theorem}
Let $M$ be an $A$-module. Then the following are equivalent,
\begin{enumerate}
\item $M$ is faithfully flat over $A$
\item $M$ is $A$-flat and for any $A$-module $N \neq 0$ we have $N \otimes_A M \neq 0$. 
\item $M$ is $A$-flat and $\forall \m \subset A$ maximal we have $M \neq \m M$. 
\end{enumerate}
\end{theorem}

\begin{proof}
Faithfully flat implies flatness. Furhtermore, consider the sequence 
\[ 0 \to N \to 0 \]
If $M \otimes_A N = 0$ then clearly the sequence 
\[ 0 \to M \otimes_A N \to 0\]
is exact. Thus, 
\[ 0 \to N \to 0 \] must be exact so $N = 0$. 
\bigskip\\
Now suppose 2. and let,
\begin{center}
\begin{tikzcd}
N \arrow[r, "f"] & P \arrow[r, "g"] & Q
\end{tikzcd}
\end{center}
be a sequence such that,
\begin{center}
\begin{tikzcd}
N \otimes_A M \arrow[r] & P \otimes_A M \arrow[r] & Q \otimes_A M
\end{tikzcd}
\end{center}
is exact. However, $g \circ f = 0$ by exactness and the flatness of $M$. Furthermore, 
\[ \ker{g \otimes_A \id_M} = \ker{g} \otimes_A M \quad \quad \Im{f \otimes_A \id_M} = \Im{f} \otimes_A M \]
by flatness. However, exactness implies that $\ker{g} \otimes_A M = \Im{f} \otimes_A M$ which implies that $(\ker{g} / \Im{f}) \otimes_A M  = 0$ so $\ker{g} = \Im{f}$ because $(-) \otimes_A M$ is injective. Furthermore, assuming 2. take $\m \subset A$ maximal then $M \otimes A / \m \neq 0$ implies that $M \neq \m M$. Now assume 3. and take $N \neq 0$ with $x \in N$ nonzero. Let $I = \Ann{A}{x} \subset \m$ for some maximal ideal. Consider the map $\iota : A / I \xrightarrow{\sim} A x \subset N$. Then $A / \m \otimes_A M \neq 0$ implies tht $A / I \otimes M \neq 0$ so $A x \otimes_A M \neq 0$ by 3. Then $A x \otimes_A M$ embedds inside $N \otimes_A M$ because $M$ is $A$-flat. Thus $N \otimes_A M \neq 0$. 
\end{proof}

\begin{corollary}
Let $A$ and $B$ be local rings and $A \to B$ a local map. Let $M$ be a nontrivial finitely generated $B$-module, then $M$ if $A$-flat $\iff$ $M$ of faithfully flat over $A$.
\end{corollary}

\begin{proof}
Consider the maximal ideal $\m_B \subset B$. If $M \otimes_B (B / \m_B) = M / \m_B M = 0$. Then $\m_B M = M$ but $\m_B = \rad{A}$ so by Nakayama, $M = 0$. Thus, we have $M \otimes_B (B / \m_B) \neq 0$. Since the map is local $\m_A B \subset \m_B$ which gives a surjection,
\begin{center}
\begin{tikzcd}
B / \m_A B \arrow[r] & B / \m_B \arrow[r] & 0
\end{tikzcd}
\end{center}
since tensoring is right-exact we have the surjection
\begin{center}
\begin{tikzcd}
M \otimes_B (B / \m_A B) \arrow[r] & M \otimes_B (B / \m_B) \arrow[r] & 0
\end{tikzcd}
\end{center}
Therefore $M \otimes_B (B / \m_A B) \neq 0$. Furthermore, 
\[ M \otimes_B (B / \m_A B) = M \otimes_B (B \otimes_A A / \m_A) = M \otimes_A (A / \m_A) \] 
Therefore, $M \otimes_A (A / \m_A) = M / \m_A M$ is non-zero which implies that $M \neq \m_A M$. Therefore, by above, $M$ is faithfully flat over $A$ exactly when $A$ is flat.
\end{proof}

\begin{proposition}
Let $A \to B$ be a map of rings. If $M$ is faithfully flat over $A$ theb $M_B = M \otimes_A B$ is faithfully flat over $B$. 
\end{proposition}

\begin{proposition}
Let $M$ be a $B$-module and $A \to B$ a map of rings. Suppose that $M$ is faithfully flat over $B$ and faithfully flat over $A$ then $B$ is faithfully flat over $A$. 
\end{proposition}

\begin{proposition}
Let $\phi: A \to B$ be a map of rings with $B$ faithfully flat over $A$ then,
\begin{enumerate}
\item For any $A$-module $N$, the canonical map,
\[ N \to N \otimes_A B \]
is injective. In particular, $\phi$ is injective.
\item For any ideal $I \subset A$, we have $I B \cap A = I$.
\item $\phi^{-1} : \spec{B} \to \spec{A}$ is surjective. 
\end{enumerate} 
\end{proposition}

\begin{proof}
Let $x \neq 0$ take $x \otimes 1 \neq 0$ since $A x \otimes_A B \neq 0$ because $B$ is faithfully flat. Thus, $x \mapsto x \otimes 1$ is injective, Now consider the map,
\[ A / I \to A / I \otimes_A B = B / IB \]
which is injective by the above argument. Thus we have a diagram,
\begin{center}
\begin{tikzcd}
A / I \arrow[r, "\tilde{\phi}"] & A / I \otimes_A B \arrow[d]
\\
A \arrow[u] \arrow[r, "\bar{\phi}"] & B / IB 
\end{tikzcd}
\end{center}
Then $IB \cap A = \ker{ \bar{\phi}}$ and $\ker{\tilde{\phi}} = \ker{\bar{\phi}} / I = IB \cap A / I = 0$. Thus $I B \cap A = I$. 
Furthermore, consider $\phi^{-1} : \spec{B} \to \spec{A}$ and take $\p \in \spec{A}$. Consider,
\[ A_\p / \p A_\p \otimes_A B \neq 0 \]
which is nonzero because $B$ is faithfully flat. Thus $B_\p \supsetneq \p B_\p$ which implies that there exists $\m$ a maximal ideal of $B_\p$ containing $\p B_\p$. Furthermore, $\m \cap A_\p \supset \p A_\p$ which implies that $\m \cap A_\p = \p A_\p$. Then $\mathfrak{P} = \m \cap B$ so 
\[ \mathfrak{P} \cap A = \m \cap A = (\m \cap A_\p) \cap A = (\p A_\p) \cap A = \p \]
\end{proof}

\begin{proposition}
Let $B$ be a faithfully flat $A$-algebra and $M$ an $A$-module then,
\begin{enumerate}
\item $M$ is flat (resp. faithfully flat) over $A$ $\iff$ $M_B$ is flat (resp. faithfully flat) over $B$.
\item If $A$ is local and $M$ is a finitly generated $A$-module then $M$ is free over $A$ $\iff$ $M_B$ is free over $B$.   
\end{enumerate}
\end{proposition}

\begin{proof}
The 
\end{proof}

\begin{theorem}
Let $\varphi : A \to B$ be a ring map thn the following are equivalent,
\begin{enumerate}
\item $B$ is faithfully flat over $A$ i.e. $\varphi$ is faithfully flat.

\item $\varphi$ is flat and $\spec{B} \to \spec{A}$ is a surjection.

\item $\varphi$ is flat and for any maximal ideal $\m$ of $A$ there exists a maximal ideal $\m'$ of $B$ such that $\varphi^{-1}(\m') = \m$.
\end{enumerate}
\end{theorem}


\section{Integral Domains}

\begin{definition}
Take $A \subset B$ and $x \in B$. We say that $x$ is integral over $A$ if it satisfies a monic polynomial $p \in A[X]$ with $p(x) = 0$. 
\end{definition}

\begin{definition}
We say, for $A \subset B$, that $B$ is integral over $A$ if every $x \in B$ is integral over $A$.
\end{definition}

\begin{proposition}
The following are equivalent,
\begin{enumerate}
\item $x \in B$ is integral over $A$

\item $A[x] \subset B$ is a finitely generated $A$-module.

\item $A[x] \subset C \subset B$ where $C$ is finitely generated over $A$ and a subring of $B$.  

\item There exists a faithfull $A[x]$-module $M$ of finite type over $A$.
\end{enumerate}
\end{proposition}

\begin{proof}
If $x \in B$ is integral with $t \in A[X]$ then any polynomial $p(X) \in A[X]$ can be reduced to $p = tq + r$ with lower degree than $t$. Thus, we have a surjective map $A[x_0, \dots, x_d] \to A[x]$. Now suppose that $A[x] \subset C \subset B$ for some finitely generated $A$-module. Then $x \in A[x] \subset C$ so the map $m_x : C \to C$ given by multiplication by $x$ is given by some matrix $M$ in an $A$-basis of $C$. Then $(M - x I_n)$ is the zero map so $\det{(M - x I_n)} = 0$. This says that $x$ solves a monic polynomial $p(X) = \det{(M - X I_n)}$ which has coefficients in $A$. 
\end{proof}

\begin{corollary}
For $A \subset B$ if $y_1, \dots, y_n \in B$ are integral over $A$ then $A[y_1, \dots, y_n]$ is a finite $A$-module and is integral over $A$. 
\end{corollary}

\begin{corollary}
For $A \subset B$ then the set $C$ of elements of $B$ which are integral over $A$ is a subring of $B$ constaining $C$.
\end{corollary}

\begin{proof}
Clearly $C \supset A$. If $x, y \in C$ then $A[x,y]$ is a finite $A$-module so $A[x,y] \subset C$ and thus $xy, x + y \in C$. 
\end{proof}

\begin{corollary}
For $A \subset B \subset C$ if $C$ is integral over $B$ and $B$ is integral over $A$ then $C$ is integral over $A$. 
\end{corollary}

\begin{corollary}
Suppose $A \subset B$ is such that $B$ is a finitely generated $A$-algebra then $B$ is integral over $A$ if and only if $B$ is finite over $A$. 
\end{corollary}

\begin{proof}
If $B$ is a finite $A$-module then it is integral by above. Since $B$ is finitely generated over $A$ then $B = A[y_1, \dots, y_n]$ and $y_i$ are integral over $A$ so $B$ is finite over $A$. 
\end{proof}

\begin{definition}
Let $A \subset B$ then the set,
\[ C = \{ x \in B \mid x \text{ is integral over } A\} \]
is called the integral closure of $A$ inside $B$.
\end{definition}

\begin{definition}
If $A$ is a domain we say that $A$ is integrally closed if $x \in \Frac{A}$ is integral over $A$ implies that $x \in A$. That is, $A$ is equal to its integral closure inside $\Frac{A}$. 
\end{definition}

\begin{corollary}
Let $A \subset B$ be domains such that $A$ is integrally closed then the integral closure of $A$ inside $B$ is integrally closed.
\end{corollary}

\begin{proof}

\end{proof}

\begin{remark}
Let $A \subset B \subset C$ with $A$ Noetherian and $C$ finitely generated $A$-algebra ad $C$ is finitely generated over $B$ then $B$ is a finitely generated $A$-algebra. 
\end{remark}

\begin{lemma}
Suppose $B$ is integral over $A$ and $B$ is a domain then $A$ is a field if and only if $B$ is a field.
\end{lemma}

\begin{proof}
Suppose that $A$ is a field and take $x \in B$ with $x \neq 0$. We know that $x$ is integral over $A$ so $x$ solves some monic,
\[ x^n + a_{n-1} x^{n-1} + \cdots + a_1 x + a_0 = 0 \]
for $a_i \in A$. If $a_0 = 0$ then since $B$ is a domain we may eliminate a factor of $x$. Assume that $a_0 \neq 0$ then,
\[ x (x^{n-1} + a_{n-1} x^{n-2} + \cdots + a_1) = - a_0 \]
but $a_0 \neq 0$ and $A$ is a field so $a_0$ is invertible. Thus,
\[ x \tfrac{1}{-a_0} (x^{n-1} + a_{n-1} x^{n-2} + \cdots + a_1) = 1 \]
so $x$ is invertible. Thus $B$ is a field. 
\bigskip\\
Now assume that $B$ is a field. For any nonzero $a \in A$ we must have $a^{-1} \in B$ since $B$ is a field. However, $B$ is integral over $A$ so there must exist a monic polynomial such that,
\[ a^{-n} + a_{n-1} a^{-(n-1)} + \cdots + a_1 a^{-1} + a_0 = 0 \]
which implies that,
\[ a^{-1} + a_{n-1} + \cdots + a_1 a^{n-2} + a_0 a^{n-1} = 0 \]
However,
\[ a_{n-1} + \cdots + a_1 a^{n-2} + a_0 a^{n-1} \in A \]
and thus $a^{-1} \in A$ so $A$ is a field. 
\end{proof}

\begin{corollary}
Suppose $B$ is integral over $A$ with $\varphi : A \to B$ then the map $\phi^{-1} : \spec{B} \to \spec{A}$ satisfies the property that $\mathfrak{P} \subset B$ is maximal if and only if $\p = \varphi^{-1}(\mathfrak{P})$ is maximal. 
\end{corollary}

\begin{proof}
$B / \mathfrak{P}$ is integral over $A / \p$ with the map $A / \p \hookrightarrow B / \mathfrak{P}$ and both are domains. Therefore $A / \p$ is a field if and only if $B / \mathfrak{P}$ is a field and thus $\p$ is maximal if and only if $\mathfrak{P}$ if maximal. 
\end{proof}

\begin{definition}
Given a ring map $\varphi : A \to B$ we say that $\varphi$ has the going up property if whenever $\p \subset \p' \in \spec{A}$ and we have $\mathfrak{P} \mapsto \p$ then there exists $\mathfrak{P}' \in \spec{B}$ above $\mathfrak{P}$ such that $\mathfrak{P}' \mapsto \p'$. Similarly, $\varphi$ has the going down property if whenever we have $\mathfrak{P}' \mapsto \p'$ then there exists $\mathfrak{P} \in \spec{B}$ inside $\mathfrak{P}'$ such that $\mathfrak{P} \mapsto \p$. 
\end{definition}

\begin{theorem}
If $\varphi : A \to B$ is flat then $\varphi$ has the going down property. 
\end{theorem}

\begin{proof}
Take $\p \subset \p'$ in $\spec{A}$ and take $\mathfrak{P}' \in \spec{B}$ such that $\varphi^{-1}(\mathfrak{P}') = \p'$. Consider the localization $\spec{A_{\p'}}$ which is exactly the subset of $\spec{A}$ which lies below $\p'$. Furthermore, $B_{\p'}$ is flat over $A_{\p'}$ and $B_{\mathfrak{P}'}$ is flat over $A_{\p'}$ so $A_{\p'} \to B_{\mathfrak{P}'}$ is local and flat so faithfully flat. Thus the map $\spec{B_{\mathfrak{P}'}} \to \spec{A_{\p'}}$ is surjective. Thus for any prime below $\p'$ we get a prime below $\mathfrak{P}'$ mapping to it. 
\end{proof}

\begin{theorem}[Cohen]
Let $A \subset B$ and $B$ is integral over $A$ then the following hold,
\begin{enumerate}
\item The map $\spec{B} \to \spec{A}$ is surjective.
\item There is no inclusion relatons between two prime ideals of $B$ above the same ideal of $A$. 
\item The going up property holds. 
\item If $A$ is local with maximal ideal $\p$ the maximal ideals of $B$ are exactly those prime ideals with preimage $\p$. 
\item If $A$ and $B$ are integral domains with $A$ integrally closed then the going down property holds.  
\item If $A$ and $B$ are integral domains with $A$ integrally closed and $K = \Frac{A}$. Suppose $L$ is a normal extension of $K$ and $B$ is the integral closure of $A$ inside $L$ then any two prime ideals of $B$ lieing over the same prime ideal of $A$ are conjugated by an automorphism of $L / K$.  
\end{enumerate}
\end{theorem}


\begin{proof}
Part 4. follows directly from the lemma since $B / \mathfrak{P}$ is a field if and only if $A / \p$ is a field. Thus the maximal ideals are in correspondence.
\bigskip\\
Take $\p \in \spec{A}$ and consider $B_{\p} = B \otimes_A A_{\p}$ then let $\tilde{\P}$ be the maximal ideal of $B_{\p}$ above $\p A_\p$ then $\tilde{\P} \cap B = \P$ above $\p$. Use 4. with $B_{\p}$ integral over $\p$ to get that there can be no inclusions between these ideals because they are maximal proving 1 and 2
\bigskip\\
For part 3. suppose we have $\p \subset \p'$ and the prime $\P \subset B$ above $\p$. Furthermore, $B/\P$ is integral over $A / \p$ so we have shown that $\spec{B / \P} \to \spec{A / \p}$ is surjective. Thus every ideal of $A$ above $\p$ is the image of an ideal of $B$ above $\P$ which is exactly what we needed to show.
\bigskip\\
Now consider part 6. We may assume that $L / K$ is finite since we can always write $L$ as a union of finite extensions. Suppose we have prime ideals $\P$ and $\P'$ of $B$ both above $\p$. Assume that $\sigma_i(\P) \neq \P'$ for all $i$ running over the finite group $\Aut{L/K}$. By 2, $\P' \not\subset \sigma_i(\P)$ so there exists $x \in \P'$ such that $x \notin \sigma_i(\P)$. Take,
\[ y = \prod_{i = 1}^n \sigma_i(x) \]
and thus $\sigma(y) = y$ which implies that $y^{p^n} \in K$ for $\mathrm{char} \: K = p$. Since $x$ is integral over $A$ we know that $y^{p^n}$ is integral over $A$. But $A$ is integrally closed so $y^{p^n} \in A \cap \P' = \P$ then $y \in \p \subset \P$ which is a prime ideal so $\sigma_i(x) \in \P$ for some $i$ and thus $x \in \sigma_i^{-1}(\P)$ a contradiction. 
\bigskip\\
For part 5. we have integral domains $A \subset B$. Let $K = \Frac{A}$ and $L = \Frac{B}$ and let $L_1$ be the normal closure of $K$. Take $B_1$ to be the integral closure of $A$ inside $L_1$. Suppose we have a prime $\p \subset \p'$ in $A$ and $\P'$ above $\p'$. Furthermore, we can find $\P_1 \subset \P_1'$ in $B_1$ above $\p \subset \p'$ by surjectivity of the spec map and the going up property and also $\P_1''$ in $B_1$ above $\P'$ in $B$. Now $\P_1''$ and $\P_1'$ both lie above the same prime of $A$ so there is an automorphism $\sigma \in \Aut{L_1 / K}$ such that $\P_1'' = \sigma(\P_1')$. Thus,
\[ \sigma(\P_1) \subset \sigma(\P_1') = \P_1'' \]
Define $\P = \sigma(\P_1) \cap B \subset \sigma(\P_1') = \P_1''$. Thus, $\P \subset \P_1'' \cap B = \P'$. Finally,
\[ \P \cap A = \sigma(\P_1) \cap B \cap A = \sigma(\P_1) \cap A = \sigma(\P_! \cap A) = \sigma(\p) = \p \]
which satisfies the going down property.    
\end{proof}

\begin{corollary}
Let $A \subset B$ and $B$ be integral over $A$,
\begin{enumerate}
\item If $\P_0 \subsetneq \P_1 \subsetneq \cdots \subsetneq \P_n$ is a chain of prime ideals of $B$ then $\p_i = \P_i \cap A$ gives a strict chain of prime ideals of $A$.
\item If $\p_0 \subsetneq \p_1 \subsetneq \cdots \subsetneq \p_n$ is a chain of prime ideals of $A$ then there exists a chain of prime ideals of $B$ above it.
\item If $A$ and $B$ are domains and $A$ is integrally closed then for any $\P_n$ above $\p_n$ we can find a chain ending with $\P_n$.   
\end{enumerate}
\end{corollary}

\begin{proof} 
The inclusions must be strict otherwise we would have two prime ideals with an inclusion both above the same prime. The second part follows from the going up property and surjectivity of the spec map. The final part follows from the going down property. 
\end{proof}

\newcommand{\height}[1]{\mathbf{ht}\left( #1 \right)}


\begin{definition}
For a prime $\p \in \spec{A}$ the height $\height{\p}$ is the maximal length of any strict ascending chain of prime ideals ending with $\p$. Furthermore,
\[ \dim{A} = \max\{ \height{\p} \mid \p \in \spec{A} \} \] 
\end{definition}

\begin{corollary}
For $A \subset B$ and $B$ integral over $A$ with $\p \in \spec{A}$ and $\P \in \spec{B}$ above $\p$ we have,
\begin{enumerate}
\item $\height{\P} \le \height{\p}$
\item $\dim{A} = \dim{B}$
\item If $A$ and $B$ are domains with $A$ integrally closed then $\height{\P} = \height{\p}$. 
\end{enumerate}
\end{corollary}

\begin{proof}
The first and last follow immediately from the going up and going down properties. From the first, we have $\dim{B} \le \dim{A}$. However, if $\dim{A} = n$ then we have some $\p$ with $\height{\p} = n$ so there exists a chain in $B$ of length $n$ so $n \le \dim{B}$. Thus $\dim{A} = \dim{B}$.
\end{proof}

\section{Graded Rings}

\begin{definition}
A ring $A$ is graded if it is written as a direct sum of subgroups,
\[ A = \bigoplus_{n = 0}^\infty A_n \]
with $A_n \cdot A_m \subset A_{n + m}$ and thus $1 \in A_0$. An $A$-module $M$ is graded if we can write,
\[ M = \bigoplus_{n = 0}^\infty M_n \]
such that $A_n \cdot M_m \subset M_{n + m}$. We call an element homogeneous if it lies in a graded section $A_n$ or $M_n$. A submodule $N \subset M$ is a homogeneous submodule if,
\[ N = \bigoplus_{n = 0}^\infty N \cap M_n \]
\end{definition}

\begin{lemma}
Let $N \subset M$ be modules over a graded ring then the following are equivalent,
\begin{enumerate}
\item $N$ is a homogeneous submodule.
\item $N$ is generated over $A$ by homogeneous elements.
\item if a sum of homogenous elements of $M$ lies in $N$ then each term does also. 
\end{enumerate}
\end{lemma}

\begin{proposition}
Let $A$ be Noetherian and $M$ a graded $A$-module. Then,
\begin{enumerate}
\item If $\p \in \Ass{A}{M}$ then $\p$ is a graded homogenous ideal and $\p = \Ann{A}{x}$ for some homogenous $x \in M$.
\item For each homogeneous $\p \in \spec{A}$ there exists a graded submodule $Q(\p) \subset M$ such that $Q(\p)$ is $\p$-primary and
\[ \bigcap_{\p \in \Ass{A}{M}} Q(\p) = (0) \]
\end{enumerate}
\end{proposition}

\begin{proof}
We know that $\p = \Ann{A}{x}$ for some $x \in M$. We can write, 
\[ x = x_r + x_{r+1} + \cdots + x_s \]
with $x_i \in M_i$. We can also write,
\[ a = a_m + \cdots + a_n \]
with $a_i \in A_i$. Then consider,
\[ ax = a_m x_r + \left( a_{m+1} x_r + a_n x_{r+1}  \right) + \cdots = 0 \]
Thus the terms of each degree must individually be zero. Thus there exists some $i$ such that $a^i_m x = 0$ which implies that $a_m^i \in \Ann{A}{x} = \p$ so $a_m \in \p$. We can repeat this argument to find that all terms in the homogenous expansion of $a$ lie in $\p$. Thus, $\p$ is homogeneous. We have proven that $\p \subset \Ann{A}{x_i}$ for each $i$ and thus,
\[ \p = \bigcap_i \Ann{A}{x_i} \implies \exists i_0 \mid \Ann{A}{x_{i_0}} \subset \p \]
so $\p = \Ann{A}{x_{i_0}}$. 
\bigskip\\
For the second part we need the following lemma:
\begin{lemma}
Assume $M$ is graded and $\p \in \spec{A}$ is homogeneous and $Q \subset M$ is $\p$-primary. Let $Q' \subset Q$ be the submodule generated by the homogeneous elements of $Q$ then $Q'$ is $\p$-primary.
\end{lemma}
Now we can always choose $\tilde{Q}(\p)$ to be $\p$-primary such that
\[ \bigcap_{\p \in \Ass{A}{M}} \tilde{Q}(\p) = 0 \]
then take,
\[ Q(\p) = \left( \tilde{Q}(\p) \right)' \]
which is the submodule generated by the homogeneous elements of $\tilde{Q}(\p)$. Then $Q(\p)$ is $\p$-primary by the lemma and $Q(\p) \subset \tilde{Q}(\p)$.  
\end{proof}

\subsection{Filtrations}
Let $A$ be a ring and consider a filtration,
\[ A = J_0 \supset J_1 \supset J_2 \supset \cdots \]
such that $J_n \cdot J_m \subset J_{n + m}$. Then we can construct a graded ring,
\[ \gr{J}{A} = \bigoplus_{n = 0}^\infty J_m / J_{m+1} \]
For example, if $I \subset A$ is a proper ideal then $J_n = I^n$ is a filtration called the $I$-adic filtration. 

\begin{lemma}
Let $A$ be Noetherian, $I \subset A$ then, $\gr{I}{A}$ is Noetherian.
\end{lemma}

\begin{proof}
Since $I \subset A$ is an ideal $I$ is finitely generated so $I = (x_1, \dots, x_r)$. Then,
\[ I / I^2 = A \bar{x}_1 + \cdots + A \bar{x}_r \]
Thus the map,
\begin{center}
\begin{tikzcd}[column sep = large]
A / I [x_1, \dots, x_n] \arrow[two heads, r] & \gr{I}{A}
\end{tikzcd}
\end{center}
sending $p(x_1, \dots, x_r) \mapsto p(\bar{x}_1, \dots, \bar{x}_r)$
is surjective. However, $A / I$ is Noetherian so $A / I [x_1, \dots, x_r]$ is Noetherian so $\gr{I}{A}$ is Noetherian. 
\end{proof}

\subsection{Hilbert Polynomials}

Now suppose that $A$ is Artinian and $B = A[x_1, \dots, x_r]$ which is a graded Noetherian ring. Take $M$ a finitely generated graded $B$-module,
\[ M = \bigoplus_{n \in \Z} M_n \]
Define $B(d)$ to be the graded $B$-module such that $B(d)_m = B_{m - d}$ i.e. homogeneous polynomials of degree $m - d$. There exist $d_1, \dots, d_s$ and a surjection,
\begin{center}
\begin{tikzcd}
\bigoplus\limits_{i = 1}^s B(d_i) \arrow[r, two heads] & M 
\end{tikzcd}
\end{center}
which gives surjections,
\begin{center}
\begin{tikzcd}
\bigoplus\limits_{i = 1}^s B(d_i)_n \arrow[r, two heads] & M_m
\end{tikzcd}
\end{center}
Define the function $F_M(n) = \ell(M_n)$. However,
\[ B_{n - d} \cong A^{ n - d + r - 1 \choose r - 1 } \]
since it is all homogeneous polynomials of degree $n$. Therefore, 
\[ F_M(n) = \ell(M_n) \le \ell(B(b_1)_n) + \cdots + \ell(B(d_s)_n) \le \sum_{i = 1}^s {n - d_i + r - 1 \choose r - 1} \ell(A) \]
which is finite because $A$ is Artinian so $\ell(A)$ is finite. For example,
\[ F_{B(d)}(n) = {n - d + r - 1 \choose r - 1} \ell(A) \]
which is a polynomial in $n$.

\begin{theorem}
Let $A,B,M$ as before then there exists a polynomial $f_M(x) \in \Q[x]$ such that $F_M(n) = f_M(n)$ for sufficiently large $n$. 
\end{theorem}

\begin{proof}
Consider the set,
\[ \mathcal{J} = \{ N \subset M \mid \text{such that the theorem fails for } M / N \} \]
We want to show that $\mathcal{J}$ is empty. If $\mathcal{J}$ is not empty then it has a maximal element (since $M$ is finitely generated and $B$ Noetherian). For any $N' \supset N$ if $N' \notin \mathcal{J}$ then theorem holds for $M / N'$. We need to show that this implies that the theorem holds for $M / N$. 
\bigskip\\
First suppose $N = N_1 \cap N_2$. Then $N \subset N_1, N_2, N_1 + N_2$ so we may assume the theorem holds for each.
Define the notation $\ell_m(M) = \ell(M_m)$. Take the exact sequence,
\begin{center}
\begin{tikzcd}
0 \arrow[r] & N_1 / (N_1 \cap N_2) \arrow[r] & M / (N_1 \cap N_2) \arrow[r] & M / N_1 \arrow[r] & 0
\end{tikzcd}
\end{center}
Thus,
\[ \ell_n(M / (N_1 \cap N_2)) = \ell_n(M/N_1) + \ell_n(N_1 / (N_1 \cap N_2)) \]
However,
\[ N_1 / (N_1 \cap N_2) \cong (N_1 + N_2) / N_2 \] and we have an exact sequence,
\begin{center}
\begin{tikzcd}
0 \arrow[r] & (N_1 + N_2) / N_2  \arrow[r] & M / N_2 \arrow[r] & M / (N_1 + N_2) \arrow[r] & 0
\end{tikzcd}
\end{center}
which implies that,
\[ \ell_n((N_1 + N_2) / N_2) = \ell_n(M / N_2) - \ell_n(M / (N_1 + N_2)) \]
Putting these together we get,
\[ \ell_n(M/(N_1 \cap N_2)) = \ell_n(M / N_1) + \ell_n(M / N_2) - \ell_n(M / (N_1 + N_2)) \]
Therefore since $\ell_n(M / M_1)$ and $\ell_n(M / N_2)$ are given by polynomials in $n$ for sufficiently large $n$ then $\ell_n(M / (N_1 \cap N_2))$ is given by a polynomial in $n$. 
\bigskip\\
Secondly if $N$ is $\p$-primary with $\p \subset B$ homogeneous. Case 1, if $I = (x_1, \dots, x_r) \subset B$ is contained in $\p$ then $(M / N)_n = 0$ for sufficiently large $n$. 
\end{proof}

\begin{definition}
Let $I \subset A$ be an ideal and $M$ an $A$-module with filtration,
\[ M = M_0 \supset M_1 \supset M_2 \supset M_3 \supset \cdots \]  
we say the filtration is,
\begin{enumerate}
\item $I$-admissible if $I \cdot M_n \subset M_{n+1}$ for all $n \ge 0$,

\item $I$-adic if $I \cdot M_n = M_{n+1}$ for all $m \ge 0$,

\item essentially $I$-adic if $I \cdot M_n = M_{n+1}$ for sufficiently large $m$.
\end{enumerate} 
\end{definition}


\begin{definition}
Such a filtration defines a topology on $M$. For each $x \in M$ we declare $\{ x + M_n \}_{n \ge 0}$ to be a fundamentaly system of neighborhoods of $x$. The topology is Hausdorff if and only if $\bigcap\limits_{n \ge 0} M_n = 0$. Furthermore, we say this topology is $I$-adic when the filtration is (essentially) $I$-adic. 
\end{definition}


\begin{remark}
Let $M$ be an $A$-module and $I \subset A$ an ideal. We call the filtration $M_n = I^n M$ the $I$-adic filtration on $M$ which is clearly $I$-adic. Any essentially $I$-adic filtration defines the same topology as the $I$-adic toplogy. 
\end{remark}


\begin{lemma}
Take $A, I, M$ as before and consider an $I$-admssible filtration,
\[ M = M_0 \supset M_1 \supset M_2 \supset M_3 \supset \cdots \]  
of finitely generated $A$ modules and define,
\[ A' = \bigoplus_{n \ge 0} I^n x^n \subset A[x] \quad \text{and} \quad M' = \bigoplus_{n \ge 0} M_n x^n \subset M \otimes_A A[x]\]
Since the filtration in $I$-admissible, $M'$ is an $A'$-module. The filtration is essentially $I$-adic if and only if $M'$ is finitely generated over $A'$.
\end{lemma}

\begin{proof}
We have $A' M' \subset M'$ and $M'$ is a graded $A'$-module and take $M_n' = M_n \cdot x^n$. If $M'$ is finitely generated then for a finite set $S$ we can write,
\[ M' = \sum_{i \in S }^n A' m_i \]
where $m_i$ are homogeneous. Then $M'_{n+1} = (I x) \cdot M'_n$ when $n$ is larger than the degrees of all $m_i$. This holds because $M'_{n+1} = M_{n+1} \cdot x^{n+1} \subset Ix \cdot M_n x^n$ since the filtration is $I$-admissible. If $n$ is larger than the degrees of all $m_i$ then every element of $M_{n+1}'$ must have coeficients over $M_{n}'$ inside $I$ since they come from the product of $A'$ of degree at least $1$. Therefore, $M_{n+1} = I \cdot M_n$ so the filtration is essentially $I$-adic.
\bigskip\\
Conversely if $M_n = I^{n - n_0} M_{n_0}$ for $m \ge n_0$ then $M'$ is finitely genreated over $A'$ by a system of generators of $M_{n_0}$ because $M_n' = (I^{n - n_0} x^n) M_{n_0}$.  
\end{proof}

\begin{theorem}[Artin-Rees]
Let $A$ be Noetherian, and $M$ a finitely generated $A$-module. If $N \subset M$ and $I \subset A$ then there exists $r \ge 0$ such that,
\[ I^n \cdot M \cap N = I^{n - r} \left( I^r M \cap N \right) \]
for $n \ge r$. Thus the filtration on $N$ induced by the $I$-adic filtration on $M$ is essentially $I$-adic. Equivalently, the topology induced on $N$ by the standard $I$-adic topology on $M$ is $I$-adic. 
\end{theorem}

\begin{proof}
Set,
\[ A' = \bigoplus_{n \ge 0} I^n x^n \subset A[x] \quad \text{and} \quad M' = \bigoplus_{n \ge 0} I^n \cdot M x^n \subset M \otimes_A A[x]\]
because $A$ is Noetherian, $I$ is finitely generated so $A'$ is Noetherian. Since $M$ is finitely generated then the $I$-adic filtration is finitely generated and thus, by the preceeding lemma, $M'$ is a finitely generated $A'$-module. Now consider,
\[ N' = \bigoplus_{n \ge 0}(I^n M \cap N) x^n \subset M' \]
Since $M'$ is a finitely generated $A'$-module and $A'$ is Noetherian so $N'$ is a finitely genreated $A'$-module. By the lemma, the filtration $N_n = I^n M \cap N$ is essentially $I$-adic. 
\end{proof}

\begin{corollary}[Krull Intersection]
Let $A$ be Noetherian and $I \subset A$ and $M$ a finitely generated $A$-module. Define,
\[ N = \bigcap_{n \ge 0} I^n \cdot M \]
then we have,
\begin{enumerate}
\item $I \cdot N = N$.
\item
If $I \subset \rad{A}$ then $N = 0$. In particular, for $M = A$,
\[ \bigcap_{n \ge 0} I^n = (0) \]
\end{enumerate}
\end{corollary}

\begin{proof}
Apply the theorem for $N$ then,
\[ I^n M \cap N = I \cdot \left(I^{n-1} M \cap N \right) \]
but $I^n M \supset N$ so we have $N = I \cdot N$. For the second, use Nakayama's lemma.
\end{proof}

\begin{corollary}
Assume $A$ is a Noetherian integral domain with a proper ideal $I \subset A$. Then,
\[ \bigcap_{n \ge 0} I^n  = (0) \]
\end{corollary}

\begin{proof}
By the previous corolary for $M = A$,
\[ N = \bigcap_{n \ge 0} I^n \]
satisfies $I \cdot N = N$ so there exists $x \in I$ such that $(1 + x) \cdot N = 0$ by Nakayama. But since $I$ is proper, we cannot have $1 + x = 0$. Since $A$ is a domain, $(1 + x) \cdot N = (0)$ implies that $N = (0)$. 
\end{proof}

\begin{remark}
Let $A$ be Noetherian, $I$, $J$ ideals of $A$. Let $M$ be an $A$-module. Assume that $J$ is generated by $M$-regular elements then,
\[ (I^n M : J) = I^{n - r} (I^r M : J ) \]
where,
\[ (N : J) = \{m \in M \mid J m \subset N \} \] 
\end{remark}

\section{Dimension Theory}

\begin{definition}
For a prime $\p \in \spec{A}$ the height $\height{\p}$ is the maximal length of any strict ascending chain of prime ideals ending with $\p$,
\[ \p = \p_0 \supsetneq \p_1 \supsetneq \cdots \supsetneq \p_h \]
then $\height{\p} = \max{h}$.
Furthermore,
\[ \dim{A} = \max\{ \height{\p} \mid \p \in \spec{A} \} \] 
\end{definition}

\begin{lemma}
Let $A$ be a ring and $\p \subset A$ a prime. Then $\dim{A_{\p}} = \height{\p}$.
\end{lemma}

\begin{definition}
For any ideal $I \subset A$ we have,
\[ \height{I} = \inf\{ \height{\p} \mid \p \supset I \} \]
\end{definition}

\begin{proposition}
For an ideal $I \subset A$ we have $\dim{A} \ge \height{I} + \dim{A / I}$. 
\end{proposition}


\begin{definition}
Let $M$ be an $A$-module then,
\[ \dim{A} = \dim{\left( A / \Ann{A}{M} \right)} \]
\end{definition}

\begin{proposition}
Let $M$ be a finite $A$-module with $A$ a Noetherian ring. Then the following are equivalent,
\begin{enumerate}
\item $M$ is of finite length
\item $A / \Ann{A}{M}$ is Artinian
\item $\dim{M} = 0$
\end{enumerate}
\end{proposition}

\begin{proof}
We know $(2) \iff (3)$ becaues $A$ is Noetherian. We assume thal $\ell(M) < \infty$ we want to show that $\dim{(A / \Ann{A}{M})} = 0$. Assume otherwise and set $A' = A / \Ann{A}{M}$. Take $\p \in \Spec{A'}$ minimal but not maximal (assuming the dimension is nonzero). Thus, $\p \in \Supp{A'}{M}$ and thus $\p \in \Ass{A'}{M}$ since it is minimal. Thus we have an embedding of $A' / \p$ in $M$ and thus $\ell(A' / \p) \le \ell(M)$. However, $\dim{(A' / \p)} \ge 1$ since $\p$ is not maximal in $A'$ which implies that $A' / \p$ is not Artinian which implies that $\ell(A' / \p) = \infty$. Thus, $\ell(M) = \infty$. Conversely, if $\dim{M} = 0$ then $M$ is finitely generated over $A$ and thus over $A'$. Thus, $(A')^r \to M$ is a surjection so $\ell(M) \le r \ell(A') \le \infty$. Therefore, $\ell(M)$ is finite.
\end{proof}

\subsection{Hilbert-Samuel Polynomial}

\begin{definition}
A ring $A$ is semi-local if $A$ has finitely many maximal ideals.
\end{definition}

Let $A$ be Noetherian and semi-local. 

\begin{definition}
An ideal $I \subset A$ is called an ideal of definition if there exists $s \ge 1$ such that $\m^s \subset I \subset \m$ where $\m = \rad{A}$. 
\end{definition}

\begin{remark}
An ideal $I \subset \m$ is an ideal of definition if and only if $A / I$ is Artinian. 
\end{remark}

\begin{definition}
Let $I \subset A$ be an ideal of definition and $M$ a finitely-generated $A$-module. Then let,
\[ A^* = \gr{I}{A} = \bigoplus_{n = 0}^{\infty} I^n / I^{n+1} \]
and furthermore,
\[ M^* = \gr{I}{\m} = \bigoplus_{n = 0}^\infty I^n M / I^{n+1} M \]
then $M^*$ is a finitely generated $A^*$-module. Fix $x_1, \dots, x_r$ a system of generators of $I$. Then consider,
\begin{center}
\begin{tikzcd}[column sep = large]
B = (A/I) [x_1, \dots, x_r] \arrow[r, two heads] & A^*
\end{tikzcd}
\end{center}
Since $I$ is an ideal of definition then $A / I$ is Artinian ring. Thus using the Hilbert polynomial, we know that,
\[ F_n(M^*) = \ell(M_n^*) = \ell(I^n M / I^{n+1}M) \]
is a polynomial in $n$ of degree at most $r - 1$ for sufficiently large $n \ge n_0$. Therefore,
\begin{align*}
\ell(M / I^n M) & = \ell(M/IM) + \ell(IM/I^2M) + \cdots + \ell(I^{n-1}M/I^n M) 
\\
& = \ell(M / I^{n_0 + 1}M) + f_M(n_0) + f_M(n_0 + 1) + \cdots + f_M(n-1) 
\end{align*}
which is a polynomial in $n$ of degree at most $r$. We call this polynomial $\chi(M, I, n)$ the Hilbert-Semilocal Polynomial. We call $d(M) = \deg{\chi(M, I, x)}$ which is independent of the ideal of definition $I$. Furthermore, $d(M) \le r$.   
\end{definition}

\begin{lemma}
Assume that we have an exact sequence of Noetherian modules,
\begin{center}
\begin{tikzcd}
0 \arrow[r] & M' \arrow[r] & M \arrow[r] & M'' \arrow[r] & 0
\end{tikzcd}
\end{center}
then $d(M) = \max{\left\{ d(M'), d(M'') \right\}}$ and for an ideal of definition $I$ and for sufficiently large $n$ we have that
\[ \ell(M / I^n M) - \ell(M'/I^n M') - \ell(M''/I^n M'') \]
is a polynomial in $n$ of degree at most $d(M') - 1$. 
\end{lemma}

\begin{proof}
We have the exact sequence 
\begin{center}
\begin{tikzcd}
0 \arrow[r] & (M' + I^n M)/I^n M \arrow[r] & M/I^n M \arrow[r] & M''/I^n M'' \arrow[r] & 0
\end{tikzcd}
\end{center}
Compute,
\begin{align*}
Q = \ell(M / I^n M) - \ell(M'' / I^n M'') - \ell(M' / I^n M') = \ell((M' + I^n M) / I^n M) - \ell(M'/I^n M')
\end{align*}
Furthermore,
\[ \frac{M' + I^n M}{I^n  M} \cong \frac{M'}{I^n M \cap M'} \]
By Artin-Rees theorem we have $\exists n_0$ such that if $n > n_0$ then,
\[ I^n M \cap M' = I^{n - n_0} \left( I^n M \cap M' \right) \subset I^{n - n_0} M' \]
Therefore,
\[ \ell(M' / I^n M') \ge \ell\left( \frac{M'}{I^n M \cap M'} \right) \ge \ell(M' / I^{n-n_0} M') \]
so plugging in,
\[ 0 \ge  Q \ge \ell(M' / I^{n-n_0} M') - \ell(M'/I^n M') \]
Which is $f(n-n_0) - f(n)$ for a polynomial $f$ of degree $d(M')$. Thus, $f(n-n_0) - f(n)$ is a polynomial of degree at most $d(M') - 1$. Because $d(M) \ge d(M'')$ we have $d(M) = \max\left\{ d(M'), d(M'') \right\}$. 
\end{proof}

\begin{lemma}
Let $A$ be Noetherian and semi-local then,
\[ d(A) \ge \dim{A} \]
and, in particular, $\dim{A} \le \infty$. 
\end{lemma}

\begin{proof}
If $d(A) = 0$ then $\ell(A / I^n)$ is constant for sufficiently large $n$. Thus, $I^{n_0} = I^{n_0 + 1}$ so $I^{n_0} = 0$ by Nakayama. Thus, $A$ is Artinian so $\dim{A} = 0$. Now suppose that $d(A) > 0$ and $D = \dim{A} > 0$. We have a chain,
\[ \p_0 \supsetneq \p_1 \supsetneq \cdots \supsetneq \p_D = \p \]
Choose $x \in \p_{D - 1} \setminus \p$ then we have an exact sequence,
\begin{center}
\begin{tikzcd}
0 \arrow[r] & A / \p \arrow[r, "\times x"] & A / \p \arrow[r] & A / (\p + x A) \arrow[r] & 0
\end{tikzcd}
\end{center}  
since $[x]$ is nonzero and $A / \p$ is a domain. Applying the previous lemma for an ideal of definition we find that
\[ \chi(A / \p, I, n) - \chi(A / \p, I, n) - \chi(A / (\p + x A), I, n) \]
is a polynomial of degree at most $d(A/\p) -1$. Therefore, $d(A/(\p + xA)) \le d(A/\p) - 1 \le d(A) - 1$. We may apply the induction hypothesis for $A/(\p + x A)$ to get,
\[ \dim(A / (\p + x A)) \le d(A / (\p + x A)) \le d(A)  - 1 \]
However,
\[ \bar{p}_0 \supsetneq \bar{p}_1 \supsetneq \cdots \supsetneq \bar{p}_{D-1} \]
is a chain of ideals of $A / (\p + x A)$ so $D - 1 \le d(A) - 1$ and thus $\dim{A} \le d(A)$. 
\end{proof}

\begin{corollary}
Let $A$ be any Noetherian ring and $\p \in \Spec{A}$ then.
\[ \height{p} < \infty \]
\end{corollary}

\begin{proof}
Since $A_{\p}$ is local and Noetherian  we have $\height{\p} = \dim{A_{\p}} < \infty$. 
\end{proof}

\begin{lemma}
Let $A$ be Noetherian and semi-local and $M$ a finitely generated $A$-module. Let $X \in \m = \rad{A}$ then,
\[ d(M) \ge d(M/xM) \ge d(M) - 1 \]
\end{lemma}

\begin{proof}
\begin{align*}
Q = \ell\left( (M / x M) / I^n (M / xM) \right) = \ell(M / I^n M) - \ell\left( \frac{x M + I^n M}{I^n M} \right) 
\end{align*}
However, 
\[ \frac{(M / x M)}{ I^n (M / xM) } =  M / (x M + I^n M) = \frac{(M / I^n M)}{( x M + I^n M)/I^n M} \] 
\end{proof}
Furthermore, there is a surjection,
\begin{center}
\begin{tikzcd}
M / I^{n-1}M \arrow[r, two heads] & \frac{x M + I^n M}{I^n M} 
\end{tikzcd}
\end{center}
Therefore,
\[ Q = \ell(M / I^n M) - \ell(M / I^{n-1} M) = \chi(M, I, n) - \chi(M, I, n - 1) \]
which is a polynomial in $n$ of degree at most $d(M) - 1$ since the leading term cancels. Thus $d(M / xM) \ge d(M) - 1$.

\begin{lemma}
Let $A$ be Noetherian, semi-local and let $M$ be a finitely generated $A$-module with $r = \dim{M} \ge 1$. Then there exist $x_1, \dots, x_r \in \rad{A}$ such that,
\[ \ell(M / (x_1 M + \cdots + x_r M )) < \infty \] 
\end{lemma}

\begin{proof}
Consider the minimal prime ideals,
\[ \p_1, \dots, \p_s \supset \Ann{A}{M} \]
which is possible since there are finitely many such prime ideals since each is in $\Ass{A}{M}$ which is a finite set. Because $r \ge 1$ these ideals cannot be maximal and therefore $\rad{A} \not\subset \p_i$ for any $i$ otherwise $\p_i$, being prime, would lie above some maximal idea. Therefore,
\[ \rad{A} \not\subset \bigcup_{i = 1}^s \p_i \]
so there exists $x_1 \in \rad{A}$ but not in any $\p_i$. If we can show that $\dim{M / x_1 M} \le r - 1$ then we get the result by induction since we know that $\ell(M) < \infty$ if $\dim{M} = 0$. Take $\q \supset \Ann{A}{M/x_1 M} \supset \Ann{A}{M} + x_1 A$  a minimal prime. We knowthat $\q \neq \p_i$ since $x_1 \notin \p_i$ for each $i$. Therefore $\height{\q} \ge 1$ since it is not a minimal prime. Since the $\p_i$ enumerate all minimal primes we must have $\q \supsetneq \p_i$ for some $i$. Thus any chain above $\Ann{A}{M/x_1 M}$ can be augmented by appending $\p_i$. Therefore, $\dim{M / x_1 M} = \dim{A / \Ann{A}{M / x_1 M}}$ is exactly the maximal length of chains of primes containing the anhilator. Thus, $\dim{M / x_1 M} + 1 \le \dim{M} = r$ so $\dim{M / x_1 M} \le r - 1$. 
\end{proof}

\begin{theorem}
Let $A$ be semi-local and $M$ a finitely generated $A$-module. Then, $\dim{M} = d(M) = r$ where $r$ is the smallest integer such that here exist $x_1, \dots, x_r \in \rad{A}$ with,
\[ \ell(M / (x_1 M + \cdots + x_r M )) < \infty \] 
\end{theorem}

\begin{proof}
(DO THIS)
\end{proof}

\begin{corollary}
Let $A$ be Noetherian and $I = (x_1, \dots, x_r) \subset A$ then for any minimal prime $\p \supset I$ we have $\height{I} \le \tilde{r}$ in particular $\height{I} \le r$. 
\end{corollary}

\begin{proof}
Pick $\p \supset I$ then $A_\p / I A_p$ is local so $\p$ is minimal containing $I$ impliyng that $\bar{\p}$ is minimal and maximal and thus $A_{\p} / I A_{\p}$ is Artinian. Therefore, 
\[ \ell\left( A_{\p} / I A_{\p} \right) \le \infty \]
However,
\[ \ell \left(A_{\p} / (x_1 A_{\p} + \cdots + x_r A_{\p}) \right) \le \infty \implies r \ge d(A_{\p}) = \height{\p} \]
\end{proof}

\begin{proposition}
Let $I \subset A$ be an ideal of definition of a semi-local Artinian ring $A$ and $M$ a finitely generated $A$-module. Consider the $I$-adic completion of $M$,
\[ \hat{M} = \varprojlim M / I^n M \]
Which is a $\hat{A}$-module where,
\[ \hat{A} = \varprojlim A / I^n A \]
Then,
\[ \dim{\hat{M}} = \dim{M} \]
\end{proposition}

\begin{proof}
We have,
\[ \hat{M} / I^n \hat{M} = M / I^n M \implies d(\tilde{M}) = d(M) \implies \dim{\hat{M}} = \dim{M} \]
\end{proof}

\begin{lemma}
If $A$ is a semi-local Noetherian ring and $s = \dim{A}$ then there exists an ideal of definition $I$ generated by $s$ elements $\exists x_1, \dots, x_s \in \rad{A}$ such that,
\[ \ell\left(A / x_1 A + \cdots + x_s A \right) < 0 \]
\end{lemma}

\begin{corollary}
Let $A$ be Noetherian and $\p \subset A$ prime and $n$ an integer. Then the following are equivalent,
\begin{enumerate}
\item $\height{\p} \le n$

\item There exists $I \subset A$ generated by $n$ elements such that $\p$ in minimal in $I(V)$.
\end{enumerate}
\end{corollary}

\begin{proof}
First let $\height{\p} \le n$ then we have $\dim{A_{\p}} \le n$ so there exists an ideal of definition $J$ of $A_\p$ generated by $n$ elements because $A$ is semi-local and $\dim{A_{\p}} \le n$. Then,
\[ J = \left( \frac{x_1}{s_1}, \dots, \frac{x_n}{s_n} \right) \]
with $s_i \notin \p$. Then $I = (x_1, \dots, x_s) \subset \p$ and then $I A_\p = J$ and $\p A_{\p} \supset J$ so $\p \in V(I)$ and is minimal because $A_\p / I A_\p = A_\p / J$ is Artinian. 
\bigskip\\
Conversely, if we have $\p$ minimal in $V(I)$ such that $I = (x_1, \dots, x_s)$ then $A_{\p} / I A_{\p}$ is Artinian which implies that,
\[ \ell\left( A_\p / I A_\p \right) = \ell\left( A_\p / (x_1 A + \cdots x_ A \right) < 0 \]
and thus $n \ge \dim{A_{\p}} = \height{\p}$.
\end{proof}

\begin{definition}
A system of parameters for $M$, a finitely generated $A$-module over a semi-local ring $A$ is a set of elements $x_1, \dots, x_r \in \rad{A}$ such that,
\begin{enumerate}
\item $\ell(M / (x_1 M + \cdots + x_r M)) \le \infty$
\item $r = \dim{M}$
\end{enumerate}

\begin{proposition}
Let $x_1, \dots, x_r \in \rad{A}$ and let $A$ be a semilocal ring with $M$ a finitely genrated $A$-module. Then,
\[ \dim{\left( M / (x M_1 + \cdots + x_r M) \right)} \ge \dim{M} - r \]
with equality if and only if $x_1, \dots, x_r$ belong to a system of parameters for $M$.
\end{proposition}
\end{definition}

\begin{proof}
Take $x \in \rad{A}$ and $\dim{M / xM} \ge \dim{M} - 1$ the inequality follows by induction. Assume you have equality. Then,
\[ \dim{\left( M / (x M_1 + \cdots + x_r M) \right)} = \dim{M} - r \]
Pick $y_1, \dots, y_q$, a system of paramerters for $\bar{M} = M / (x_1 M + \cdots + x_r M)$ with $q = \dim{\bar{M}} = \dim{M} - r$. Therefore,
\[ \ell\left( \bar{M} / y_1 \bar{M} + \cdots + y_r \bar{M} \right) \le \infty \]
However,
\[ \ell\left( \bar{M} / y_1 \bar{M} + \cdots + y_r \bar{M} \right) = \ell\left( M / (x_1 M + \cdots + x_r M + y_1 M + \cdots + y_q M) \right) \]
Therefore, $x_1, \dots, x_r, y_1, \dots, y_q$ is a system of parameters for $M$. Conversely, if  $x_1, \dots, x_r, y_1, \dots, y_q$ is a system of parameters for $M$ then we have,
\[ \dim{\left( M / (x_1 M + \cdots + x_r M + y_1 M + \cdots + y_q M) \right)} \ge d(M / (x_1 M + \cdots + x_r M)) - p \]
However,
\[ \dim{\left( M / (x_1 M + \cdots + x_r M + y_1 M + \cdots + y_q M) \right)} = 0 \]
since its length is finite. Therefore,
\[ d(M / (x_1 M + \cdots + x_r M)) \le p = \dim{M} - r \] 
so the innequality is an equality. 
\end{proof}

\subsection{Dimension of Local Rings}

Let $(A, \m, k)$ be a local ring.

\begin{proposition}
$\dim{A} \le \text{number of set of generators of an ideal of definition of } A$
and there exists an ideal of definition generated by $\dim{A}$ number of elements a set of elements $x_1, \dots, x_n$ such that $I = (x_1, \dots, x_r)$ is an ideal of definition is called a system of parameters for $A$.
\end{proposition}

\begin{proposition}
We have $\m \otimes_A k = \m / \m^2$ and $\dim{A} \le \dim_k{\m / \m^2}$ as a $k$-vectorspace. 
\end{proposition}

\begin{definition}
We say that $A$ is regular if $\dim{A} = \dim_k{\m / \m^2}$. In that case, a minimal system of generators for $\m$ is called a regular system of paramters.
\end{definition}

\begin{example}
Let $k$ be a field and $A = k[[x_1, \dots, x_n]]$ then $\m = (x_1, \dots, x_n)$ is maximal and $\dim{A} = n$. Then there is a correspondence between maxomal ideals of the quotient $B = k[x_1, \dots, x_n] / I$ and points in $\bar{k}$ of the subariety of $\bar{k}^n$ defined by the vanishing of $I$. For a point $p$ it corresponds to a maximal ideal $\m$ and then $B_p = B_{\m}$ can be interpreted as the germ of regular functions at $p$. We say that the variety is regular or smooth at p if $B_{\m}$ is regular. 
\end{example}

\subsection{Relative Dimension}

Consider a map $\varphi : A \to B$ of rings. We have an induced map $\varphi^* : \Spec{B} \to \Spec{A}$. If we take $\p \in \Spec{A}$ then consider,
\[ (\varphi^*)^{-1}(\p) = \{ \P \in \Spec{B} \mid \varphi^{-1}(\P) = \p \} = \Spec{B / \p B_\p} = \Spec{B \otimes_A A_{\p}} \]

\begin{theorem}
Let $\varphi : A \to B$ be a map of Noetherian rings and $\P \in \Spec{B}$ and $\p = \varphi^{-1}(\P)$ then,
\begin{enumerate}
\item $\height{\P} \le \height{\p} + \height{\P / \p B}$ i.e. $\dim{B_{\P}} \le \dim{A_{\p}} + \dim{B_{\P}/ \p B_{\P}}$
\item Equality holds if the going down property holds for $\varphi$. In particular, if $\varphi$ is flat.

\item If $\varphi^* : \Spec{B} \to \Spec{A}$ is surjective and the going down property holds for $\varphi$ then,
\begin{enumerate}
\item $\dim{B} \ge \dim{A}$
\item $\height{I} = \height{IB}$ for any ideal $I \subset A$. 
\end{enumerate}
\end{enumerate}
\end{theorem} 

\begin{proof}
Let $r = \dim{A_\p}$ and since $A$ is Noetherian we may choose a system of parameters $x_1, \dots, x_r$ for $A$ such that $I = (x_1, \dots, x_r)$ is an ideal of definition of $A_{\p}$ so $\p^n \subset I \subset \p$. Therefore, $\p$, $\p^n$, and $I$ have the same nilradical and thus the same spectra.
Then, $\dim{B_\P / \p^n B_{\P}} = \dim{B / IB} = \dim{B / \p B} = s$
Let $\bar{y}_1, \dots, \bar{y}_s$ a system of perameters for $B / IB$ which implies that $x_1, \dots, x_r, y_1, \dots, r_s$ satisfies,
\[ \ell\left(B / (x_1, \dots, x_r, y_1, \dots, y_s) \right) \le \infty \]
Which implies that $r + s \ge \dim{B}$.
\bigskip\\
For the second, consider $\bar{\P}$ inside $B / \p B$. Take a chain of ideals of $B / \p B$,
\[ \bar{\P} = \bar{\P_0} \supset  \bar{\P_1} \supset \cdots \supset  \bar{\P_s} \]
and $\P_i$ is the preimage in $B$. Thus, $\p B \subset \P_i \subset \P$ so $\p \subset \varphi^{-1}(\P_i) \subset \varphi^{-1}(\P) = \p$ which implies that $\P_i$ is above $\p$. Now take a chain of primes in $A$,
\[ \p = \p_0 \supset \p_{1} \cdots \supset \p_r \]
so that $r = \height{\p}$. Then let $\varphi^{-1}(\P_s) = \p = \p_0$. By going down, there exists a decreasing chain of ideals above the sequence $\p_i$ which has length $r$. Therefore, $\height{\P} \ge r + s$ since we have found a descending chain begining at $\P$ of length $r + s$. Thus the equality holds.
\bigskip\\
Lastly, let $\p \in \Spec{A}$ and $\height{\p} = \dim{A}$ then since $\Spec{B} \to \Spec{A}$ is surjective there exists $\P \in \Spec{B}$ above $\p$. Using the second property and going down,
\[ \height{\P} = \height{\p} + \height{\P / \p B} \ge \height{\p}  = \dim{A} \]
which implies that $\dim{B} \ge \height{\P} \ge \dim{A}$. 
\bigskip\\
Furthermore, given $\P \supset IB$ a prime such that $\height{\P} = \height{IB}$. Consider $\p = \varphi^{-1}(\P)$ then $\p \supset I$. Thus, $B / \p B$ is a quotient of $B / IB$ which implies that,
\[ \height{\P / I B} \ge \height{\P / \p B} \]
but $\P$ is a minimal prime over $IB$ so there cannot exist primes of $B / IB$ below $\P$ implying that $\height{\P / I B} = 0$. Thus, $\height{\P / \p B} = 0$ which implies that,
\[ \height{\P} = \height{\p} + \height{\P / \p B} = \height{\p} \]
by going down. However, $\p \supset I$ so $\height{\p} \ge \height{I}$ and thus,
\[ \height{IB} = \height{\P} = \height{\p} \ge \height{I} \]
Furthermore, let $\p \subset I$ such that $\height{\p} = \height{I}$. By the surjectivity of $\Spec{B} \to \Spec{A}$ there exists $\P \in \Spec{B}$ above $\p$. Therefore, $\varphi^{-1}(\p) = \P$ so $\P \supset \p B \supset IB$. We can choose $\P$ minimal such that $\P \supset \p B \supset IB$ which implies that $\height{\P / I B} = 0$. Therefore, by going down,
\[ \height{\P} = \height{\p} + \height{\P / IB} = \height{\p} = \height{I} \]
Furthermore, $\height{\P} \ge \height{IB}$ so we have $\height{I} \ge \height{IB}$ completing the proof. 
\end{proof}

\begin{corollary}
Let $A \subset B$ be an integral extension of Noetherian rings then,
\begin{enumerate}
\item $\dim{A} = \dim{B}$
\item $\forall \P \in \Spec{B} : \height{\P} \le \height{\P \cap A}$
\item If going down holds then $\height{J} = \height{J \cap A}$ for all ideals $J \subset B$.
\end{enumerate}
\end{corollary}

\begin{proof}
The first we have already proven for integral extensions. For the second,
\[ \height{\P} \le \height{\P \cap A} + \height{\P / \p B} \]
where $\p = \P \cap A$. However, $B / \p B$ is integral over $A / \p A$ so $\height{\P / \p B} \le \height{\p / \p A} =  0$ which implies that $\height{\P} \le \height{\p \cap A}$. 
\bigskip\\
Suppose that going down holds. Take a minimal prime $\P \supset J$. By going gown and surjectivity of the spec map, $\height{\P} \ge \height{\P \cap A}$ thus by above $\height{\P} = \height{\P \cap A}$. Furthermore, $\P \cap A \supset J \cap A$ and thus 
\[ \height{J} = \height{\P} = \height{\P \cap A} \ge \height{J \cap A} \]
Likewise, take $\p$ a minimal prime above $J \cap A$. Then $B / J$ is integral over $A / (J \cap A)$ so by Cohen, $\Spec{B/J} \to \Spec{A / (J \cap A)}$ is surjective meaning that there exists $\P \in \Spec{B}$ above $\p$ such that $\P \supset J$. Therefore,
\[ \height{J} \le \height{\P} = \height{\p} = \height{J \cap A} \]  
\end{proof}

\subsection{Finitely Generated Extensions}

\begin{definition}
Let $A$ be a domain then $m$ is irreducible if $m = ab$ implies $a \in A^\times$ or $b \in A^\times$ and $m$ is a non-zero non-unit. 
\end{definition}

\begin{lemma}
If $R$ is a PID then 
\begin{enumerate}
\item $\m = (m)$ is maximal iff $m$ is irreducible. 
\item irreducibles and prime elements coincide. 
\end{enumerate}
\end{lemma}

\begin{proof}
Suppose that $\m = (m)$ is maximal. If $m = ab$ then $(m) \subset (a)$ so $(a) = R$ or $(a) = \m$ thus either $a$ or $b$ is a unit. Conversely, if $m$ is irreducible then suppose that $\m \subset I \subset R$. Since $R$ is a PID $I = (a)$ meaning that $m \in (a)$ so $m = ab$ which implies that either $a \in R^\times$ or $b \in R^\times$ so either $I = (a) = R$ or $\m = (m) = (a)$ meaning that $\m$ is maximal. 
\bigskip\\
If $p$ is a prime then $p = ab$ then $a \in (p)$ or $b \in (p)$. WLOG take $a \in (p)$ so $a = pk$ so $p = pkb$ which, because $R$ is a domain implies that $b$ is a unit so $p$ is irreducible. Converesely, if $m$ is irreducible then $(m)$ is maximal so $(m)$ is prime. 
\end{proof}

\begin{lemma}
If $R$ is a PID then $\dim{R} \le 1$. So if $R$ is not a field then $\dim{R} = 1$. 
\end{lemma}

\begin{proof}
Since primes and irreducibles coincide, the non-zero prime ideals are all maximal so every chain is of the form $(0) \subset \m$ whith $\m$ maximal. $R$, a domain, is a field of and only if $(0)$ is maximal if and only if $\dim{R} = 0$.   
\end{proof}

\begin{theorem}
If $A$ is Noetherian then $\dim{A[X]} = \dim{A} + 1$.
\end{theorem}

\begin{proof}
Let $B = A[X]$ and take $\P \in \Spec{B}$ and $\p = A \cap \P$. Then $B_{\p} = A_\p[X]$ so $B_\p / \p B_\p = (A_\p/\p A_\p)[X] = k(\p)[X]$. Therefore,
\[ 
\height{\P B_\p / \p B_\p} = \height{\P k(\p)[X]}
\]
However, $k(\p)[X]$ is a PID since $k(\p)$ is field ($A_\p$ is local with maximal ideal $\p$) and thus $\dim{k(\p)[X]} = 1$. Therefore,
\[ \height{\P B_\p / \p B_\p} \le 1 \]
and is equal to zero exactly $\P k(\p)[X]$ is the zero ideal exactly when $\P = \p B$. Since $B$ is free over $A$ it is $A$-flat then the localization $B_\p$ is $A_\p$-flat. Therefore, $A_\p \to B_\p$ satisfies the going down property so,
\[ \height{\P B_\p} = \height{\p A_\p} + \height{\P B_\p / \p B_\p} \]
However,
\[ \height{\P B_\p} = \dim{(B_\p)_{\P B_\p}} = \dim{B_\P} = \height{\P} \]
and similarly,
\[ \height{\p A_\p} = \height{\p} \]
Therefore, when $\P$ is maximal.
\[ \height{\P} = \height{\p} + 1 \]
Since $A[X]$ is faithfully flat over $A$ then $\Spec{B} \to \Spec{A}$ is surjective. Thus we may
pick $\P$ such that $\height{\p} = \dim{A}$ then 
$\dim{B} \ge \height{\P} = \dim{A} + 1$. 
Furthermore take $\height{\P} = \dim{B}$ then,
\[ \dim{B} = \height{P} = \height{\p} + 1 \le \dim{A} + 1 \]
Therefore,
\[ \dim{B} = \dim{A} + 1 \]
\end{proof}

\begin{corollary}
Let $A$ be Noetherian. Then,
\begin{enumerate}
\item $\dim{A[X_1, \dots, X_n]} = \dim{A} + n$ 
\item If $k$ is a field then $\dim{k[X_1, \dots, X_n]} = n$.
\end{enumerate}
\end{corollary}


\section{Random (NEED WORK)}

\begin{theorem}[Noetherian Normalization]
Let $A$ be a finitely generated $k$-algebra domain ($k$ is a field) and a chain of ideals,
\[ I_1 \subset \cdots \subset I_p \]
Then there exists $x_1, \cdots, x_m$ algebraically independent over $k$ such that,
\begin{enumerate}
\item
$A$ is integral over $k[x_1, \dots, x_n]$
\item $I_j \cap k[x_1, \dots, x_n] = (x_1, \cdots, x_{h(j)})$ where $h$ is an increasing function of $j$. 
\end{enumerate}
\end{theorem}

\newcommand{\trdeg}[2]{\mathrm{trdeg}_{#1} \: #2}

\begin{corollary}
Let $A$ be a finitely generated integral domain over a field $k$. Then,
\[ \dim{A} = \trdeg{k}{\Frac{A}} \]
where $\mathrm{trdeg}$ denotes the transcendence degree of $\Frac{A}$ over $k$. 
\end{corollary}

\begin{proof}
By Noetherian Normalization, we know that $A$ is integral over $B = k[x_1, \dots, x_n]$ for algebraically independent $x_1, \dots, x_n$. Thus, $\Frac{A}$ is algebraic over $k(x_1, \dots, x_n)$ which implies that $\trdeg{k}{\Frac{A}} = n = \dim{B} = \dim{A}$ where $B$ is a subring such that $A$ is integral over $B$.  
\end{proof}

\begin{corollary}
Let $A$ be a finitely generated $k$-algebra. Then if $\m \subset A$ is a maximal ideal then $A / \m$ is a finite extension of $k$.
\end{corollary}

\begin{proof}
Let $B = A / \m$ which is an integral domain over a field. By the previous corollary, $\trdeg{k}{A/\m} = \dim{A / \m} = 0$ since $A / \m$ is a field and thus has dimension zero. Therefore, $A / \m$ is algebraic over $k$. Furthermore, $A$ is a finitely generated $k$-algebra so $A / \m$ is a finitely generated $k$-algebra and algebraic and thus a finite extension. 
\end{proof}

\begin{proposition}
Let $A$ be a finitely generated $k$-integral domain and $\p \in \Spec{A}$ then,
\[ \height{\p} + \dim{A / \p} = \dim{A} \]
\end{proposition}

\begin{proof}
By Noetherian Normalization, there exist $x_1, \dots, x_n \in A$ such that $A$ is integral over $k[x_1, \dots, x_n] = B$ and $n = \dim{A}$ and $\p \cap B = (x_1, x_2, \dots, x_h)$ for some $h \le n$. Then, $\height{\p \cap B} = h$ and $B / \p \cap B = k[x_{h+1}, \dots, x_n]$ and $A / \p$ is integral over $B / \p \cap B$ so,
\[ \dim{A / \p} = \dim{B / \p \cap B} = n - h \]
Therefore,
\[ \height{\p} + \dim{A / \p} \ge h + n - h = \dim{A} \]
We can also say that because $A$ is integral over $B$ and $B$ is integrally closed then the going down property holds for $B \to A$ which implies that $\height{\p} = \height{\p \cap B}$. 
\end{proof}

\begin{theorem}[Strong Nullstellensatz]
Let $A$ a finitely generated $k$-algebra and $I \subset A$ then,
\[ \sqrt{I} = \bigcap_{\m \supset I} \m \]
over all maximal ideals.
\end{theorem}

\begin{proof}
Define,
\[ J = \bigcap_{\m \supset I} \m\]
since each maximal ideal is prime, $J \supset \sqrt{I}$. Assume $\exists a \in J \setminus \sqrt{I}$. Consider the set of powers $S_a = \{a^n \mid n \in \N \}$. Then $S_a \cap I = \varnothing$. Thus, $S^{-1} A / I$ is nonzero. Therefore there exists $\m_0$ a maximal ideal of $S^{-1} A$ containing $I$. Let $\m$ be the preimage of $\m_0$ under $A \to S^{-1} A$ which is prime. Therefore,
\[ k \subset A / \m \subset S^{-1} A / \m_0 \]
but $S^{-1} A / \m_0$ is a field. Therefore,
\[ \trdeg{k}{A / \m} \le \trdeg{k}{S^{-1} A / \m} = 0 \]
since $\dim{S^{-1} A / \m} = 0$. Therefore, $\dim{A / \m} = 0$ so $A / \m$ is an Artinian domain and thus a field. Therefore $\m$ is maximal. However, $\m \supset I$ and $S \cap \m = \varnothing$ so $a \notin \m$ so $a \notin J$. Thus, $J \setminus \sqrt{I}$ is empty. 
\end{proof}

\begin{remark}
Let $X = \Spec{A}$ with the Zariski topology. All closed subsets are of the form $V(I) = \{ \p \in \Spec{A} \mid \p \supset I \}$. Let $S \subset X$ and consider,
\[ \overline{S} = \bigcap_{C \supset S} C \]
where $C = V(I)$ is closed. Therefore, $\overline{S} = V(I_S)$ where I have defined,
\[ I_S = \bigcap_{\p \in S} \p \]
Applying the Strong Nullstellensatz to $I = (0)$ we find that,
\[ \bigcap_{\m \subset A} \m = \sqrt{(0)} = \bigcap_{\p \in \Spec{A}} \p \]
Thus, if $M \subset \Spec{A}$ are the maximal ideals then $\overline{M} = V(\sqrt{(0)}) = \Spec{A}$. 
Therefore, the Zariski colosure of all geometric points (points corresponding to maximal ideals) is all of $\Spec{A}$ i.e. the geometric points are dense. 
\end{remark}

\begin{proposition}
Let $A$ and $A'$ be two finitely generated $k$-algebras. Then for any minimal prime ideal $\p$ of $A \otimes_k A'$ we have,
\[ \dim{A \otimes_k A' / \p} = \dim{A} + \dim{A'} \]
\end{proposition}

\begin{proof}
Apply Noetherian Normalization to $A$ and $A'$. Let $B = k[x_1, \dots, x_n]$ and $B' = k[x_1, \dots, x_{n'}]$. Then $A$ is integral over $B$ and $A'$ is integral over $B'$. Then, I claim that $A \otimes_k A$ is a torsion-free $B \otimes_k B'$-module. Take $F = \Frac{A}$ and $F' = \Frac{A'}$ and $L = \Frac{B}$ and $L' = \Frac{B'}$. Then $F \otimes_k F'$ is isomorphic to finitely many copies of $L \otimes L'$. Then,
\begin{center}
\begin{tikzcd}
L \otimes_k L' \arrow[r] & F \otimes_k F'
\\
B \otimes_k B' \arrow[u] \arrow[r] & A \otimes_k A' \arrow[u]
\\
0 \arrow[u] & 0 \arrow[u]
\end{tikzcd}
\end{center}
is exact because $L$ and $F$ are $k$-flat as $k$-vectorspaces. 
Suppose that $\p$ minimal in $A \otimes_k A'$ then it is an associated prime so $\p$ only contains zero divisiors. Thus,
\[ \p \cap (B \otimes_k B') = (0) \]
from the fact, $A \otimes_k A'$ is integral over $B \otimes_k B'$ which implies that,
\[ \dim{A \otimes A'/\p} = \dim{B \otimes_k B'} = \dim{B} + \dim{B'} = \dim{A} + \dim{A'} \]
since,
\[ B \otimes_k B' = k[x_1, \dots, x_n, y_1, \dots, y_{n'}] \]
\end{proof}


\begin{remark}
Let $X = \Spec{A}$ and $X' = \Spec{A'}$ then $X \times X' = \Spec{A \otimes_k A'}$. If $\p \in \Spec{A \otimes_k A'}$ is minimal then $\Spec{A \otimes_k A'} \cong V(\p) \subset X \times X'$ is a minimal irreducible component of $X \times X'$.  
\end{remark}

\begin{proposition}
Let $A$ be a finitely generaed $k$-algeba. Take $\p_1$ and $\p_2$ two primes ideals of $A$. Let $\q$ be a minimal prime ideal inside $V(I)$ with $I = \p_1 + \p_2$. Then,
\[ \dim{A / \q} \ge \dim{A / \p_1} - \dim{A / \p_2} - \dim{A} \]
\end{proposition}

\begin{proof}
Assume that $A = k[x_1, \dots, x_n]$ and $A' = k[y_1, \dots, y_n]$ and $\p_2'$ is a copy of $\p_2$ in $A'$. Then,
\[ A \otimes_k A' = k[x_1, \dots, x_n, y_1, \dots, y_n] \supset I = \p_1 + \p_2' \]
Then take,
\[ \Delta = (x_1 - y_1, \dots, x_n - y_n) \]
such that,
\[ A \otimes_k A' / \Delta \cong A \quad \text{and} \quad A \otimes A' / (\Delta + I') = A / (\p_1 + \p_2) \]
Therefore,
\[ Q \supset I \text{ is minimal } \iff Q' \supset I' + \Delta \text{ is minimal} \]
Then,
\[ A \otimes A' / I' = A / \p_1 \otimes_k A' / \p'_2 \]
which implies that,
\[ \dim{ A \otimes_k A' / I' } = \dim{A / \p_1} + \dim{A' / \p'_2} = \dim{A / \p_1} + \dim{A / \p_2} \]
Then we have, 
\[ A / (\p_1 + \p_2) = (A \otimes A' / I') / (\Delta / (I' \cap \Delta)) = B / \overline{\Delta} \]
Where,
\[ \overline{\Delta} = (x_1 - y_1, \dots, x_n - y-n) \]
with $x_i$ and $y_i$ the image of $x_i$ and $y_i$ in $B$. Then $Q' \subset I + \Delta$ implies that the images $\overline{Q} \supset \overline{\Delta}$ is also minimal. Thus,
\[ \height{\overline{Q}} \le m \]
because $\overline{\Delta}$ is generated by $m$ elements. Thus,
\[ \dim{A / Q} = \dim{B / \overline{Q}} = \dim{B} - \height{\overline{Q}} \implies \height{\overline{Q}} = \dim{B} - \dim{A / Q} \]
so
\[ \dim{B} - \dim{A / Q} \le m = \dim{A} \]
Putting everything together,
\[ \dim{A / \p_1} + \dim{A / \p_2} - \dim{A / Q} \le \dim{A} \]
\end{proof}

\begin{remark}
We know,
\[ V(I) = V(\p_1) \cap V(\p_2) \]
then,
\[ \Spec{A / \p} \cong V(\q) \subset V(I) \]
Then $V(\q)$ is an irreducible component of $V(\p_1) \cap V(\p_2)$. 
\end{remark}

\begin{definition}
A local ring is called a discrete valuation ring if it is a principal ideal domain and there exits a nonzero prime ideal.
\end{definition}

\begin{proposition}
If $A$ is a DVR then $\dim{A} = 1$.
\end{proposition}

\begin{proof}
Any PID that is not a field has dimension one.
\end{proof}

\begin{proposition}
If $A$ is a DVR then we can define a valuation map $v : K^\times \to \Z$ with $K = \Frac{A}$. Let $v(x)$ be the minimal $n \in \Z$ such that $x \in (\varpi^n)$ and $v(\frac{a}{b}) = v(a) - v(b)$. Then $v(xy) = v(x) + v(y)$ and $v(x + y) \ge \min\{v(x), v(y)\}$. 
\end{proposition}


\begin{proposition}
Let $K$ be a field and $v : K^\times \twoheadrightarrow \Z$ be a valuation. Then,
\[ A = \{ x \in K^\times \mid v(x) \ge 0 \} \cup \{0\} \]
is a DVR.
\end{proposition}

\begin{proof}
Take $\varpi \in K$ such that $v(\varpi) = 1$. Then, $\m = (\varpi)$ is a maximal ideal since $A \setminus \m = \{ u \in A \mid v(u) = 0 \}$ so $u \in A^\times$ since $u^{-1} \in K^\times$ and $v(u u^{-1}) = 0$ and $v(u) = 0$ so $v(u^{-1}) = 0$ so $u^{-1} = A$. Clearly, $A$ is local.
\end{proof}


\begin{proposition}
A DVR has exactly two prime ideals, $(0)$ and $\m = (\varpi)$ such that all nonzero ideals are of the form $I = \m^n = (\varpi^n)$. 
\end{proposition}

\begin{proof}
Let $A$ be a DVR. Since $A$ is a domain, $(0)$ is a prime. Since $A$ is a PID every nonzero prime is maximal. However, $A$ is local so there is a unique nonzero prime $\p = \m = (\varpi)$ where we can take a uniformize $\varpi$ generating $\m$ since $A$ is a PID.
\bigskip\\
Let $x \in A$ be nonzero. Since $A$ is a Noetherian domain and $\m \subset A$ is proper we have,
\[ \bigcap_{n = 0}^\infty \m^n = (0) \]
therefore $x \in \m^n$ for some maximal $n$. Therefore we can write $x = u \varpi^n$ for $u \in A \setminus \m = A^\times$.
Since $A$ is a PID take $I = (x)$ be a nonzero ideal then $x$ can be factored as $x = u \varpi^n$ with $u \in A^\times$. Thus, $I = (\varpi^n) = \m^n$. 
\end{proof}


\begin{proposition}
Let $A$ be a ring then the following are equivalent.
\begin{enumerate}
\item $A$ is a DVR
\item $A$ is local, Noetherian, and $\m$ is generated by a non nilpotent element.
\end{enumerate}
\end{proposition}

\begin{proof}
Clearly if $A$ is a DVR then $A$ is local by definition, Noetherian since it is a PID, and there are no nilpotents since $A$ is a domain. Suppose the second. Take $\m = (\varpi)$ where $\varpi$ is not nilpotent. Since $A$ is Noetherian and local,
\[ \bigcap \m^n = (0) \]
so any $x \in A \setminus \{0\}$ is has a maximum $n$ such that $x \in \m^n$. Then $x = \varpi^n u$ for $u \notin \m$ since $n$ is largest which implies that $u$ is invertible. Then since $\varpi$ is not nilpotent $A$ is a domain since all elements can be written in terms of $\varpi$ and units.  
\end{proof}

\begin{theorem}
Let $A$ be a Noetherian domain. Then $A$ is a DVR if and only if the following hold,
\begin{enumerate}
\item $A$ is integrally closed
\item $A$ has a unique nonzero prime ideal.
\end{enumerate}
\end{theorem}

\begin{proof}
If $A$ is a DVR then we have shown that there is a unique nonzero prime. If $x \in K = \Frac{A}$ is integral over $A$ then,
\[ x^n + a_{n-1} x^{n-1} + \cdots + a_1 x + a_0 = 0 \]
for $a_i \in A$. Then,
\[ x^n = - \sum_{i = 0}^{n - 1} a_i x^i \]
which implies that,
\[ n v(x) \ge \inf\{(v(a_i) + i v(x))\} \]
and thus, for some $i_0$,
\[ (n - i_0) v(x) \ge v(a_{i_0}) \]
But $ n > i_0$ so $v(x) \ge 0$. Thus $x \in A$. Therefore, $A$ is integrally closed.
\bigskip\\
Now assume the properties. First $2.$ implies that $A$ is local with maximal ideal $\m$ which is the unique nonzero prime of $A$. Define,
\[ \m' = \{ x \in K \mid x \m \subset A \} \]
Since $\m' \supset A$ and $y \in \m'$ implies that $y' \m \subset A$ we find $\m \subset \m \m' \subset A$. Since $\m$ is maximal we have $\m \m' = \m$ or $\m \m' = A$. Assume that $\m \m' = \m$ then $x \in \m'$ implies that $x \m \subset \m$ so $x$ is integral over $A$ because $A$ is Noetherian and thus $x \in A$ so $\m' = A$. Pick $x \in \m \setminus \{0\}$. I claim that any element $y \in K$ can be written as $y = \frac{z}{x^n}$ for some $n \ge 0$. To see this, consider the set of powers $S_x = \{1, x, x^2, \cdots \}$ and $S^{-1} A \subset K$ but the unique prime ideal of $S^{-1} A$ is $(0)$ since $A$ has no pimes disjoint to $S$ except $(0)$. Thus, $S^{-1} A$ is a field so $S^{-1} A = K$. Then if $z \in A \setminus \{ 0 \}$ then $\frac{1}{z} = \frac{y}{x^n}$ for some $y \in A$ and $n \ge 0$. Then $yz = x^n$ some $x^n \in (z)$. Since $A$ is Noetherian, $\m = (x_1, \dots, x_r)$ so $x_i^{n_i} \in (z)$ for some $n_i$ and take,
\[ N = \sum_{i = 1}^r n_i \]
then $\m^N \subset (z)$. We pick $z \in \m$ and consider the smallest $N$ such that $\m^N \subset (z)$. Clearly, $N$ is positive since $z \in \m$ and $\m^{N - 1} \not\subset (z)$. There exists $y \notin \m^{N-1}$ such that $y \notin (z)$. Thus, $y \m \subset  \m^N \subset (z)$ so $\frac{y}{z} \cdot \m \subset A$ so $\frac{y}{z} \in \m'$ but $y \notin (z)$ so $\frac{y}{z} \notin A$ contradicting $\m \m' = \m$. Therefore, $\m \m' = A$ instead. There exists $x_i \in \m$ and $x_i' \in \m'$ such that,
\[ \sum_{i = 1}^r x_i x_i' = 1 \]
which implies that some $x_i x_i' \notin \m$ since $\m \subsetneq A$. Therefore, $x_i x_i'$ is a unit  so $1 \in x_i' \m$ so $x_i' \m = A$ and thus $\m = (x_i'^{-1})$ is principal. 
\end{proof}

\begin{definition}
A ring $A$ is a Dedekind domain if $A$ is a Noetherian domain which is integrall closed and $\dim{A} = 1$. 
\end{definition}

\begin{proposition}
Let $A$ be a Noetherian domain then the following are equivalent,
\begin{enumerate}
\item If $\p \subset A$ is a nonzero prime then $A_{\p}$ is a DVR
\item $A$ is integrally closed and $\dim{A} = 1$.
\end{enumerate}
\end{proposition}

\begin{proof}
Suppose that $A$ is Dedekind. Then $A_{\p}$ is a local ring with only one nonzero prime ideal since $\dim{A_{\p}} \le \dim{A} = 1$ but $\p A_{\p}$ is a nonzero prime so $\dim{A_{\p}} = 1$. Furthermore $A$ is integrally closed so $A_{\p}$, being its localization, is integrally closed. Thus $A_{\p}$ is a DVR. 
\bigskip\\
Now suppose that $A_{\p}$ is a DVR at each nonzero prime. Let $a \in K$ be integral over $A$ then $a$ is integral over $A_{\p}$ for each $\p \neq 0$ which are integral so $a \in A_{\p}$. Therefore,
\[ a \in \bigcap_{\p \neq 0} A_{\p} = \bigcap_{\p}  = A \]
so $A$ is integrally closed. Furthermore, if $\p$ is a nonzero maximal ideal then $\height{\p} = \dim{A_{\p}} = 1$ since $A_{\p}$ is a DVR. Thus, $\dim{A} = 1$. 
\end{proof}

\begin{example}
The following are Dedekind rings,
\begin{enumerate}
\item Any PID
\item If $A$ is Dedekind and $L / K = \Frac{A}$ is finite then the integral closure of $A$ in $L$ is Dedekind. 
\item If $K$ is a number field then $\mathcal{O}_K$, the integral closure of $\Z$, is Dedekind. 
\end{enumerate}
\end{example}

\begin{definition}
A fractional ideal $\a$ of a domain $A$ is a finitely generated $A$-submodule of $K = \Frac{A}$. 
\end{definition}

\begin{proposition}
If $A$ is a Dedekind ring then any fractional ideal of $A$ is invertible,
\[ \a^{-1} = \{ x \in K \mid x \a \subset A \} \]
then $\a^{-1} \a = A$. 
\end{proposition}

\begin{proof}
By localization, if $\a$ is a fractional ideal so is $\a_{\p} = (\varpi)^n$ then $\a^{-1}_{\p} = (\varpi)^{-n}$ so $\a^{-1}_{\p} \a_{\p} = A_{\p}$. Therefore,
\[ \a^{-1} \a = \bigcap_{\p} \a^{-1}_{\p} \a_{\p} = \bigcap_{\p} A_{\p} = A \] 
\end{proof}

\begin{corollary}
The set of fractional ideals of a Dedekind domain is a group.
\end{corollary}

\begin{proposition}
If $x \in A$ and $A$ is Dedekind then there are only finitely many prime ideals containing $x$.
\end{proposition}

\begin{proof}
\[ x \in \a \implies \a^{-1} \subset x^{-1} A \]
ideals containg $x$ then satisfy the descending chain condition since $\a \supset (x^{-1} A)^{-1}$. This implies that finitely may prime ideals contain $x$. Furthermore, only finitely many prime ideals contain an ideal $\a \subset A$. 
\end{proof}

\begin{corollary}
If $\a \subset A$ define $v_{\p}(\a) = v_{\p}(\a_{\p})$ for $\a_{\p} \subset A_{\p}$ for a nonzero prime ideal of $A$. Then $\a \subset \p \iff v_{\p}(\a) \ge 1$ so there are finitely many $\p$ with $v_{\p}(\a) \neq 0$. 
\end{corollary}

\begin{proposition}
If $\a$ is a fractional ideal then,
\[ \a = \prod_{\p \neq 0} \p^{v_{\p}(\a)} \]
is a unique primary decomposition of Dedekind rings. 
\end{proposition}

\section{Depth}

\begin{definition}
Let $A$ be a ring and $M$ an $A$-module then $a_1, \dots, a_n \in A$ let $\underline{a} = (a_1, \dots, a_n)$. We say that $a_1, \dots, a_n$ is an $M$-regular sequence if
\begin{enumerate}
\item multiplication by $a_i$ is injective on $M/(a_1 M + \cdots + a_{i-1} M)$
\item $\underline{a} M \neq M$.
\end{enumerate}
\end{definition}

\begin{remark}
$a$ is called $M$-regular exactly when the map $m \mapsto a m$ is injective. If $a_1, \dots, a_n$ is $M$ regular then so is $a_1, \dots, a_j$. And,
\[ M \supsetneq a_1 M \supsetneq a_1 M + a_2 M \supsetneq \cdots \supsetneq \underline{a} M \] 
because otherwise $M / (a_1 M + \cdots + a_i M) = M / (a_1 M + \cdots + a_{i+1} M)$ which implies that mutliplication by $a_{i+1}$ is zero in $M / (a_1 M + \cdots + a_i M)$. 
\end{remark}

\begin{lemma}
Assume that $a_1, \dots, a_n$ is $M$-regular and $m_1, \dots, m_n \in M$ such that,
\[ \sum_{i = 1}^n a_u m_n = 0 \]
then $m_i \in \underline{a} M$.
\end{lemma}

\begin{proof}
Induction on $n$. If $a_1 m_1 = 0$ then $m_1 = 0$ since $a_1$ is regular. Now assume that $a_1 m_1 + \cdots + a_r m_r = 0$. Consider this modulo $a_1 M + \cdots + a_{r-1} M$ which implies that $\bar{a}_r \bar{m}_r = 0$ in $M / (a_1 M + \cdots + a_{r-1} M)$. Therefore, since $a_r$ is regular on the quotient,
\[ \bar{m}_r = 0 \]
so
\[ m_r = \sum_{i = 1}^{r-1} a_i n_i \]
for some $n_i \in M$. Therefore,
\[ a_1 m_1 + \cdots + a_{r-1} m_{r-1} + a_r \sum_{i = 1}^{r-1} a_i n_i = 0 \]
which implies that
\[ a_1(m_1 - a_r n_1) + a_2( m_2 - a_r n_2) + \cdots + a_{r-1} (m_{r-1} + a_r n_{r-1}) = 0 \] 
By induction, these terms, $m_i - a_r n_i$ must be inside $a_1 M + \cdots a_{r - 1}M$. Therefore, $m_i \in a_1 M + \cdots a_r M$. 
\end{proof}

\begin{proposition}
Assume that $a_1, \dots, a_r$ is $M$-regular then for any integers $n_1, \dots, n_r$ the sequence $a_1^{n_1}, a_2^{n_2}, \cdots, a_r^{n_r}$ is $M$-regular. 
\end{proposition}

\begin{definition}
Let $a_1, a_2, \dots, a_r \in A$ and let $I = (a_1, \dots, a_r)$ then we say that this sequence is quasi-$M$-regular if the following holds. Let $f(x_1, \dots, x_r) \in M[x_1, \dots, x_R] = M \otimes_A A[x_1, \dots, x_r]$ be a homogeneous degree $n$ polynomial. Then if $f(a_1, \dots, a_r) \in I^{n+1} M$ then $f(x_1, \dots, x_r) \in I M[x_1, \dots, x_n]$.
\end{definition}

\begin{remark}
This property is equivalent to: if $f(x_1, \dots, x_r)$ is homogeneous of degree $n$ in $M[x_1, \dots, x_r]$ then $f(a_1, \dots, a_r) = 0$ implies $f(x_1, \dots, x_r) \in IM[x_1, \dots, x_r]$. 
\end{remark}

\begin{remark}
The map,
\[ M / IM [x_1, \dots, x_r] \to \gr{I}{M} = \bigoplus_{n \ge 0} I^n M / I^{n+1} M \]
given by,
\[ f(x_1, \dots, x_r) \mapsto f(a_1, \dots, a_r) \]
being surjective is equivalent to the sequence being quasi-$M$-regular. 
\end{remark}

\begin{proposition}
Let $a_1, \dots, a_r \in A$ and $M$ and $A$-module. If $a_1, \dots, a_r$ is $M$-regular, then if is quasi-$M$-regular. The converse is true if $M, M/a_1M, \cdots, M / (a_1 M + \cdots + a_r M)$ are separated for the $I$-adic topologiy where $I = (a_1, \cdots, a_r)$. where a module $N$ being separatd for the $I$-adic topology means,
\[ \bigcap_{n \ge 0} I^n \cdot N = (0) \] 
In particular, if $I \subset \rad{A}$ and $N$ is a finitely generated module over a Noetherian ring $A$. 
\end{proposition}

\begin{definition}
Let $I \subset A$ be an ideal and $M$ an $A$-module then $\depth{I}{M}$ is the largest integer $r \ge 0$ such that there exists an $M$-regular sequence in $I$ of length $r$. 
\end{definition}

\begin{theorem}
Let $A$ be Noetherian and $M$ a finitely generated $A$-module and $I \subset A$ such that $IM \neq M$. Let $n \in \Z^+$ then the following are equivalent,
\begin{enumerate}
\item $\Ext{i}{A}{N}{M} = 0$ for all $i < n$ and f.g. $N$ such that $\Supp{A}{N} \subset V(I)$.
\item $\Ext{i}{A}{A/I}{M} = 0$ for all $i < n$.
\item There exists a finitely generated $A$-module $N$ with $\Supp{A}{N} = V(I)$ such that $\Ext{i}{A}{N}{M} = 0$ for all $i < n$.
\item There exists an $M$-regular sequence $a_1, \dots, a_n$ of length $n$ in $I$.
\end{enumerate}
\end{theorem}

\begin{proof}
$(1) \implies (2) \implies (3)$ is clear because $\Supp{A}{A/I} = V(\Ann{A}{A/I}) = V(I)$. Now we show $(3) \implies (4)$ by induction on $n$. For $n = 1$, assume that there exists $N$ with $\Supp{A}{N} = V(I)$ such that $\Homover{A}{N}{M} = 0$. If there does not exist an element of $I$ which is $M$-regular then $I$ is contained in the set of zero divisors of $M$ so,
\[ I \subset \bigcup_{\p \in \Ass{A}{M}} \p \]
Therefore $\exists \p \in \Ass{A}{M}$ such that $\p \supset I$.
Since $\p \in \Ass{A}{M}$ so there is an embedding,
\begin{center}
\begin{tikzcd}
A / \p \arrow[r, hook] & M
\end{tikzcd}
\end{center} 
which implies that we have an embedding,
\begin{center}
\begin{tikzcd}
A_\p / \p A_\p \arrow[r, hook] & M_\p
\end{tikzcd}
\end{center} 
Furthermore, $\p \in V(I) = \Supp{A}{N}$ so $N_\p \neq 0$. Since $N_\p$ is a finitely generated $A_\p$-module and $\p = \rad{A_\p}$ then by Nakayama, $\p N_\p = N_\p \implies N_\p = (0)$. Thus, $N_\p / \p N_\p \neq 0$. Furthermore, let $k = A_\p / \p A_\p$ the residue field, then $N_\p / \p N_\p = N \otimes_A k$ which is a $k$-module (a $k$-vectorspace). Thus, $\Homover{k}{N \otimes_A k}{k} \neq 0$ so we have a nontrivial maps,
\begin{center}
\begin{tikzcd}
N_\p \arrow[r] & N_\p / \p N_\p \arrow[r] & A_\p / \p A_\p \arrow[r] & M_\p
\end{tikzcd}
\end{center}
Therefore, $\Homover{A_\p}{N_\p}{M_\p} \neq 0$ contradicting $\Homover{A}{N}{M} = 0$. 
\bigskip\\
Now for induction, assume $(3) \implies (4)$ for $n$. Suppose that there exists a finitely generated $A$-module $N$ with $\Supp{A}{N} = V(I)$ and $\Ext{i}{A}{N}{M} = 0$ for all $i < n + 1$. The above argument gives an $M$-regular element $a_1 \in I$. Now consider the exact sequence,
\begin{center}
\begin{tikzcd}
0 \arrow[r] & M \arrow[r, "\times a_1"] & M \arrow[r] & \arrow[r] M / a_1 M \arrow[r] & 0
\end{tikzcd}
\end{center} 
and let $M_1 = M / a_1 M$. Applying the functor $\Homover{A}{N}{-}$ we get a long exact sequence containg,
\begin{center}
\begin{tikzcd}
\Ext{i}{A}{N}{M} \arrow[r] & \Ext{i}{A}{N}{M} \arrow[r] & \Ext{i}{A}{N}{M_1} \arrow[r] & \Ext{i+1}{A}{N}{M} 
\end{tikzcd}
\end{center}
However, by assumption, for $i < n$ we have, $\Ext{i}{A}{N}{M} = \Ext{i+1}{A}{N}{M} = 0$ which implies by the exactness of the above sequence that $\Ext{i}{A}{N}{M_1} = 0$. Furthermore, if $I M_1 = M_1$ then $I M + a_1 M = M$ but $a_1 \in I$ so this implies $I M = M$. Thus we have $I M_1 \neq M_1$ so $M_1$ satisfies the induction hypothesis so there exists an $M_1$-regular sequence $a_2, \cdots, a_{n+1}$ in $I$ of length $n$. Therefore, $a_1, a_2, \cdots, a_{n+1}$ is   
an $M$-regular sequence in $I$. We have used $IM \neq M$ to ensure the second property of a regular sequence for any sequence inside $I$. 
\bigskip\\
Given an $M$-regular squence $a_1, \dots, a_{n+1}$ in $I$ and finitely generated $A$-module $N$ such that $\Supp{A}{N} \subset V(I)$ we prove (1). Assume the induction hypothesis  that $(4) \implies (1)$ for $n$. Now consider the exact sequence,
\begin{center}
\begin{tikzcd}
0 \arrow[r] & M \arrow[r, "\times a_i"] & M \arrow[r] & M / a_i M \arrow[r] & 0
\end{tikzcd}
\end{center}
and set $M_1 = M / a_1 M$. Applying the hom functor to the above exact sequence,
\begin{center}
\begin{tikzcd}
\Ext{i-1}{A}{N}{M_1} \arrow[r] & \Ext{i}{A}{N}{M} \arrow[r, "\times a_1"] & \Ext{i}{A}{N}{M} \arrow[r] & \Ext{i}{A}{N}{M_1}
\end{tikzcd}
\end{center}
For $i = 0$ we have,
\begin{center}
\begin{tikzcd}
0 \arrow[r] & \Homover{A}{N}{M} \arrow[r, "\times a_1"] & \Homover{A}{N}{M} \arrow[r] & \Ext{1}{A}{N}{M_1} 
\end{tikzcd}
\end{center}
And therefore multipication by $a_1$ is injective on $\Ext{i}{A}{N}{M}$ for $i = 0$ which will give the $n = 1$ case. For $i > 0$, we need to use the induction hypothesis. The induction hypothesis gives that
\[ \Ext{i}{A}{N}{M_1} = 0 \]
for $i < n$ because $a_2, \dots, a_{n+1}$ is $M_1$-regular. By exactness, the map
\begin{center}
\begin{tikzcd}
\Ext{i}{A}{N}{M} \arrow[r, "\times a_1"] & \Ext{i}{A}{N}{M}
\end{tikzcd}
\end{center}
is an isomorphism for $i < n$ and an injection for $i = n$. 
Since $\supp{A}{N} \subset V(I)$ then we have,
\[ 
I \subset \bigcap_{\p \in \Supp{A}{M}} \p = \bigcap_{\p \supset \Ann{A}{M}} = \sqrt{\Ann{A}{M}} \]
becuase $M$ is finitely generated so $\Supp{A}{M} = V(\Ann{A}{M})$. 
Thus, $a_1 \in I$ implies that $a_1^k$ annihilates $N$ for some $k$. Thus, $a_1^k$ annihilates $\Ext{i}{A}{N}{M}$ because $a_1^k$ anhilates the hom functior. Thus, $\Ext{i}{A}{N}{M} = 0$ for $i < n + 1$ because the map given by multiplication by $a_1$ is injective and has zero image. Thus $(4) \implies (1)$ by induction.
\end{proof}

\begin{corollary}
$\depth{I}{M}$ is the largest $n \in \Z^{+}$ such that $\Ext{i}{A}{A/I}{M} = 0$ for all $i < n$. 
\end{corollary}


\begin{definition}
The \textit{depth} of an $A$-module $M$, denoted $\depth{I}{M}$ is the largest integer $n$ satisfying the eqivalent conditions of the above theorem. If $A$ is a local ring with maximal ideal $\m$ and $M$ an $A$-module then $\depth{}{M} = \depth{\m}{M}$. 
\end{definition}

\begin{lemma}
Let $A$ be a local Noetherian ring and $M,N$ be finitely-generated $A$-modules. Let $k = \depth{}{M}$ and $r = \dim{N}$ then $\Ext{i}{A}{N}{M} = 0$ for all $i < k - r$. 
\end{lemma}

\begin{proof}
We proceed by induction on $r$. For $r = 0$, if $\dim{N} = \dim{(A / \Ann{A}{N})} = 0$ then every prime in $\Supp{A}{N} = V(\Ann{A}{N})$ is maximal. Since $A$ is local it can only contain $\m$. However, $\Ass{A}{N} \subset \Supp{A}{N}$ and $\Ass{A}{N}$ is nonempy since $N \neq (0)$ so $\Ass{A}{N}$ so $\Ass{A}{N} = \{ \m \}$. Then we have,
\[ \Ext{i}{A}{N}{M} = 0 \]
whenever $i < m$ since $\Supp{A}{N} = \{ \m \} = V(\m)$ this is the definition of depth. Now for $r > 0$ we can reduce to the case that $N = A / \p$ with $\dim{A / \p} = r$. Pick $x \in \m \setminus \p$ then consider the exact sequence,
\begin{center}
\begin{tikzcd}
0 \arrow[r] & N \arrow[r, "\times x"] & A / \p \arrow[r] & A / (\p + x A) \arrow[r] & 0 
\end{tikzcd}
\end{center}
From this sequence we get a long exact sequence,
\begin{center}
\begin{tikzcd}
\Ext{i}{A}{A / (\p + x A)}{M} \arrow[r] & \Ext{i}{A}{A / \p}{M} \arrow[r, "\times x"] & \Ext{i}{A}{A / \p}{M} \arrow[r] & \Ext{i + 1}{A}{A / (\p + x A)}{M} 
\end{tikzcd}
\end{center}
$\dim{(A / (\p + x A))} < r$ since $x \notin \p$ so, by induction, $\Ext{j}{A}{A / (\p + x A)}{M} = 0$ whenever $j < k - \dim{(A /(\p + x A ))}$. Now if $i < k  - r$ then $i + 1 < k - \dim{(A /(\p + x A ))}$. Therefore, by exactness,
\[ x \Ext{i}{A}{N}{M}  = \Ext{i}{A}{N}{M} \]
but $x \in \m = \rad{A}$ so, by Nakayama, $\Ext{i}{A}{N}{M} = 0$. Thus the claim holds by induction since we make take a filtration of $N$ with $N_{i+1} / N_i \cong A / \p$.  
\end{proof}

\begin{theorem}
Let $A$ be a local Noetherian ring an $M$ a finitely-generated $A$-module then for any $\p \in \Ass{A}{M}$ we have,
\[ \depth{}{M} \le \dim{(A / \p)} \]
In particular, taking $\p = (0)$ we find $\depth{}{A} \le \dim{A}$.
\end{theorem}

\begin{proof}
Let $\p \in \Ass{A}{M}$ then we have an embedding,
\begin{center}
\begin{tikzcd}
A / \p \arrow[r, hook] & M 
\end{tikzcd}
\end{center}
In particular, $\Homover{A}{A / \p}{M}$ is nonempty. Therefore,
\[ \depth{}{M} - \dim{A / \p} = k - r \le 0 \]
by the lemma because $\Ext{0}{A}{A / \p}{M} = \Homover{A}{A / \p}{M} \neq 0$. 
\end{proof}

\begin{proposition}
Let $A$ be a local Noetherian ring and $M$ a finitely-generated $A$-module and $a_1, \dots, a_r$ an $M$-regular sequence. Then,
\[ \dim{(M / (a_1, \dots, a_r) M)} = \dim{M} - r \]
\end{proposition}

\section{Normal Rings}

\begin{definition}
A \textit{normal domain} is a domain that is integrally closed in its field of fractions. 
\end{definition}

\begin{definition}
Let $K = \Frac{A}$ then $x$ is \text{almost integral} over $K$ if there exists $c \in A \setminus \{0\}$ such that $c x^n \in A$ for all $n$. 
\end{definition}

\begin{lemma}
Integral implies almost integral. If the ring is Noetherian then integral and almost integral are equivalent.
\end{lemma}

\begin{proof}
Suppose that $x \in K$ is integral over $A$ then there exists monic $p \in A[X]$ such that $p(x) = 0$ of degree $n$. Using the polynomial relation, we may express all powers of $x$ as $A$-linear combinations of $1, x, \dots, x^{n-1}$. Take $x = \frac{a}{b}$ for $a, b \in A$ then let $c = b^n$. We have $c x^i = a^i b^{n-i} \in A$ for $i < n$. Therefore, $c x^k \in A$ for all $k$ since $c x^k$ is a linear combination of $c, cx, \dots, cx^{n-1}$ which area all in $A$. Thus, $x$ is almost integral.
\bigskip\\
Assume that $A$ is Noetherian and $x$ is almost integral so there exists $c \in A \setminus \{0\}$ such that $c x^n \in A$ for all $n$. Consider $M = A c^{-1}$ which is a finitely generated $A$-module and thus Noetherian. However, $x^n = \frac{a}{c}$ for some $a \in A$ and thus $x^n \in M$ so we have $A[x] \subset M$. However, all submodules of a Noetherian module are finitely generated so $x$ satisfies a monic polynomial over $A$.   
\end{proof}

\begin{definition}
A ring is \textit{completely normal} if every almost integral element is in $A$.
\end{definition}

\begin{proposition}
If $A$ is completely normal then so are $A[X]$ and $A[[X]]$. If $A$ is Noetherian and normal then $A[X]$ is normal. 
\end{proposition}

\begin{lemma}
A is (completely) normal iff $A_{\p}$ is (completely) normal for each $\p$. 
\end{lemma}

\begin{proposition}
If $A$ is a DVR then $A$ is normal. 
\end{proposition}

\begin{remark}
Let $I \subset A$ be such that,
\[ \bigcap_{n} I^n = (0) \]
for example if $A$ is Noetherian and $I \subset \rad{A}$, then we define,
\[ A^* = \gr{I}{A} = \bigoplus_{n \ge 0} I^n / I^{n + 1} \]
For $a \in A$ there exists $n$ such that $a \in I^n \setminus I^{n+1}$ then $a^* \in I^n / I^{n+1} \subset A^*$ but $a \mapsto a^*$ is not multiplicative. However, if $a* b^* \neq 0$ then $(ab)^* = a^* b^*$.
\end{remark}

\begin{theorem}
Let $I \subset A$ be such that,
\[ \bigcap_{n} I^n = (0) \]
We have,
\begin{enumerate}
\item If $A^*$ is a domain then $A$ is a domain,
\item If $A$ is Noetherian and $I \subset \rad{A}$ then if $A^*$ is a normal domain then $A$ is a normal domain. 
\end{enumerate}
\end{theorem}

\begin{proof}
Let $a, b \neq 0$ then $a^*, a^* \neq 0$ so, assuming that $A^*$ is a domain then $a^* b^* \neq 0$ so $(a b)^* = a^* b^* \neq 0$ and thus $ab \neq 0$. Thus, $A$ is a domain. Furthermore, if we assume that $A^*$ is a normal domain. Take $\frac{a}{b} \in \Frac{A}$ which is integral. First, note that,
\[ \bigcap_{n \ge 0} (I^n + b A) = b A \]
Thus, we will show that $a \in I^n + b A$ for all $n \ge 0$ to prove that $\frac{a}{b} \in A$. Since $a \in A + b A$ we proceed by induction. Assume that $a \in I^{n-1} + b A$ and, in fact, that $a \in I^{n-1}$ since the difference is trivially in $b A$. Consider $a^* \in I^{n-1} / I^n$ if $a \notin I^n$ since if $a \in I^n$ we are done by induction. Since $A$ is Noetherian, $\frac{a}{b}$ is almost integral so there exists $c \in A$ such that $c a^m \in b^m A$ for all $m \ge 0$. Thus $c^* (a^*)^m \in (b^*)^m A$ meaning that $\frac{a^*}{b^*}$ is almost integral. Howevr, $A^*$ is completely normal so $\frac{a^*}{b^*} \in A$. Therefore, there exists $d \in A$ such that $a^* = b^* d^*$ i.e. $a = bd \text{ mod } I^{N+1}$ for some $N$ such that $a \in I^N \setminus I^{N+1}$. Thus, $N = n - 1$ so we have shown that $a = bd \text{ mod } I^{n}$ so $a \in I^n + b A$ which is what we needed to prove. The result then follows by induction.  
\end{proof}

\begin{corollary}
Let $(A, \m, k)$ be a local Noetherian ring then,
\[ A \text{ is regular } \iff \gr{\m}{A} = \bigoplus_{n \ge 0} \m^n / \m^{n+1} \cong k[x_1, \dots, x_d] \]
with $d = \dim{A}$ and, in particular, if $A$ is regular, then it is an integral domain. 
\end{corollary}

\begin{proof}
Let $d = \dim{k}{\m / \m^2}$ where $k = A / \m$ is the residue field. Then $\m = (x_1, \dots, x_d)$ so we may construct the map,
\begin{center}
\begin{tikzcd}
k[x_1, \dots, x_d] \arrow[r, "\phi"] & \gr{\m}{A} = \bigoplus_{ n \ge 0} \m^n / \m^{n+1} 
\end{tikzcd}
\end{center}
where we send homogeneous polynomials $p$ to $p(x_1, \dots, x_d) \: \text{mod} \: \m^{n + 1}$ which is surjective because $\m$ is generated by $x_1, \dots, x_d$. If $\varphi$ is injective then $\ell(A / \m^n) = \ell\left( k[x_1, \dots, x_d] /(x_1, \cdots, x_d)^n \right) = \left( {n + d \choose d } \right)$ is a polynomial of degree $d$ so $\dim{A} = d$. If $\varphi$ is not injective then there exits $Q \in \ker{\phi}$ which is hologeneous of poitive degree. Then,
\[ \ell(A / \m^n) \le \left( {n+ d \choose d} \right) - \left( {n  - q + d \choose d } \right) \]
which has degree $d - 1$ and thus $\dim{A} \le d  - 1$ so $A$ is not regular. Assume that $A$ is regular then $\gr{\m}{A} \cong k[x_1, \dots, x_d]$ is a normal domain then $A$ is a normal domain.
\end{proof}

\begin{corollary}
Let $A$ be a local Noetherian ring then the following are equivalent,
\begin{enumerate}
\item $A$ is a DVR
\item $A$ is a normal domain with $\dim{A} = 1$
\item $A$ is a regular locl ring with $\dim{A} = 1$
\end{enumerate}
\end{corollary}

\begin{definition}
A regular system of parameters is a system of parameters gnerating the maximal ideal. 
\end{definition}

\begin{theorem}
Let $A$ be a regular local ring and let $x_1, \dots, x_r$ be a regular system of parameters so $(x_1, \dots, x_r) = \m$ then,
\begin{enumerate}
\item $A$ is a domain
\item $x_1, \dots, x_d$ is a regular sequence and in particular $A$ is CM
\item Let $\p_{\ell} = (x_1, \dots, x_\ell)$ then $\p_{\ell}$ is a prime ideal of height $\ell$ and $A / \p_{\ell} = A / (x_1, \cdots, x_\ell)$ is regular of dimension $d - \ell$
\item If $\p \in \Spec{A}$ is of hieght $\ell$ and $A / \p$ is regular then there exists a system of parameters $(y_1, \dots, y_{\ell})$ such that $\p = (y_1, \dots, y_{\ell})$. 
\end{enumerate}
\end{theorem}

\begin{proof}
We have already shown that regular ring $\implies$ domain. Since $\gr{\m}{A} \cong k[x_1, \dots, x_d]$ then $x_1, \dots, x_d$ is quasi-regular so $x_1, \dots, x_d$ is a regular sequence since $A$ is Noetherian. This implies that $d \le \depth{}{A} \le d$ so $A$ is CM. Since $x_1, \dots, x_d$ is a regular sequence $d = \dim{A}$ and $\dim{A / (x_1, \dots, x_\ell)} = d - \ell$. Therefore $\height{(x_1, \dots, x_\ell)} = \ell$  Furthermore,
$\bar{x}_{\ell + 1}, \cdots, \bar{x}_d$ in $\bar{\m} = \m / (x_1, \cdots, x_\ell) \subset A / \p_{\ell}$ 
so $\depth{A / (x_1, \dots, x_\ell)} = d - \ell$ so $A / \p_i$ is CM and thus regular and thus a domain so $\p_i$ is prime.
\bigskip\\
Now let $\p \subset A$ be prime and $\bar{\m} = \m / \p$ with $A / \p$ regular and height $\ell$. Then $\bar{\m} / \bar{\m}^2$ has dimension $d - \ell$. Furthermore,
\[ \bar{\m} / \bar{\m}^2 = \m / (\p + \m^2) \]
Furthermore, $(\p + \m^2)/\m^2$ has dimension $\ell$. Thus there exists $y_1, \cdots, y_\ell \in \p$ such that $\bar{y_1}, \cdots, \bar{y}_i$ s a basis of $(\p + \m^2)/\m^2$. We may extend this to a basis of $\m / \m^2$ denoted $\bar{y}_1, \dots, \bar{y}_d$ which is a regular system of parameters. Then $(y_1, \dots, y_{\ell}) \subset \p$ so $\dim{A / \p} = d - \ell$. By the previous argumen $\dim{A / (y_1, \dots, y_{\ell})} = d - \ell$ which implies that $(y_1, \dots, y_\ell) = \p$ since $\q \subset \p$ with $\dim{A / \q} = \dim{A / \p}$ implies $\p = \q$ because we can always augment a chain in $A / \q$ to one in $A / \p$ by adjoing $\p$.  
\end{proof}

\begin{theorem}
Let $A$ be a Noetherian normal domain then any nonzero principal ideal is unmixed (i.e. any $\p \in \Ass{A}{A / a A}$ then $\height{\p} = 1$). 
\end{theorem}

\begin{proof}
Let $a \in A$ be nonzero and take $\p \in \Ass{A}{A / a A}$. Define $\p^{-1} = \{ x \in A \mid x \p \subset A_{\p} \}$. Then $\p \subset \p^{-1} \p \subset A_{\p}$. If $\p = \p^{-1} \p$ then elements in $\p^{-1}$ are are integral over $A_{\p}$ since $A$ is normal. Thus $\p^{-1} = A_{\p}$ which is impossible. Since $\p = \Ann{A}{\bar{b}}$ for some $b \in A$ with $\bar{b}$ its image in $A / a A$. Then $\p = \{ x \in A \mid x b \subset a A \}$ which implies that $\p \cdot \frac{b}{a} \subset A \implies \frac{b}{a} \in \p^{-1}$. Thus $\frac{b}{a} \notin A$ becuase $\bar{b} \neq 0$ in $A / a A$. Therefore, $\p^{-1} \p = A_{\p}$ implies that $\p$ is principal and $A_{\p}$ is a DVR which implies $\height{\p} = 1$, 
\end{proof}

\begin{corollary}
Let $A$ be a Noetherian normal domain then,
\[ A = \bigcap_{\height{\p} = 1} A_{\p} \]
\end{corollary}

\begin{proof}
Define,
\[ \check{A} = \bigcap_{\height{\p} = 1} A_{\p}\] 
Let $x = \frac{b}{a} \in \check{A}$ if $\frac{b}{a} \notin A$ then the exists a prime $\p \in \Ass{A}{A / aA}$ with $\height{\p} = 1$. This contradicts the fact that $\frac{b}{a} \in A_{\p}$.  
\end{proof}

\begin{corollary}
Let $A$ be a Noetherian normal domain with $\dim{A} \le 2$ then $A$ is CM.
\end{corollary}

\begin{proof}
From the theorem we know that any principal ideal is unixed. Then $I = (x_1, x_2)$ with $\height{I} = 2$ so $\p \in \Ass{A}{A / I}$ of $\height{\p} \ge 2$ implies that $\p$ is maximal since $\dim{A} \ge 2$. 
\end{proof}

\begin{theorem}
Let $A$ be a Noetherian ring, the $A$ is normal if and only if 
\begin{enumerate}
\item $\forall \p \in \Spec{A} : \depth{}{A_{\p}} \ge \min(2, \height{\p})$
\item $\forall \p \in \Spec{A}$ if $\height{\p} \le 1$ then $A_{\p}$ is regular
\end{enumerate}
\end{theorem}

\begin{proof}
Suppose that $A$ is normal. If $\p \in \Spec{A}$ then $A_{\p}$ is normal by definition. If $\height{\p} = 0$ then $A_{\p}$ is a field since it is an artinian domain. If $\height{\p} = 1$ then $A_{\p}$ is a DVR and thus regular. In these cases $\depth{}{A_\p} = \dim{A_\p} = \height{\p}$. For $\height{\p} \ge 2$ then we can find $\q \subset \p$ such that $\height{\q} = 2$. Thus $A_{\q}$ has dimension $2$ and thus CM by the previous corollary and thus $\depth{}{A_{\q}} = 2$ so there exists a regular sequence of length $2$ inside $\q$. However, $\q \subset \p$ so this gives a regular sequence in $\p$ of length at least $2$ so $\depth{}{\p} \ge 2$.  
\\
Conversely, we know that $A$ is reduced (excercise). Consider $\p_1, \dots, \p_s$ the minimal primes of $A$ then $\p_1 \cap \cdots \cap \p_s = \nilrad{A} = (0)$. Therefore, we can embedd,
\begin{center}
\begin{tikzcd}
A \arrow[r, hook] & \prod\limits_{i = 1}^s A / \p_i \arrow[r, hook] & \prod\limits_{i = 1}^s K_i
\end{tikzcd}
\end{center}  
where $K_i = \Frac{A / \p_i}$. We need to show that $A$ is normal in the product of $K_i$. Then,
\[ A \cong \prod_{i = 1}^s A / \p_i \]
\end{proof}



\section{Cohen-Macaulay Rings}

\begin{definition}
Let $(A, \m)$ be a Noetherian local ring. An $A$-module $M$ is Cohen-Macaulay if $\dim{M} = \depth{}{M}$. We say that $A$ is Cohen-Macaulay if $\dim{A} = \depth{}{A}$. For any Noetherian $A$ we say that $A$ is Cohen-Macaulay if $A_{\p}$ is Cohen-Macaulay for each $\p \in \Spec{A}$. 
\end{definition}

\begin{definition}
Let $M$ be an $A$-module. Then $\q$ is an embedded prime of $\p$ if $\p \subset \q$ and $\p, \q \in \Ass{A}{M}$. 
\end{definition}

\begin{remark}
$A_{\m}$ is CM for all maximal ideals $\m$ if and only if $A$ is CM.
\end{remark}

\begin{lemma}
Let $a_1, \dots, a_r$ be an $M$-regular sequence then $\dim{M_r} = \dim{M} - r$ where $M_r = M/(a_1, \dots, a_r)M$. 
\end{lemma}

\begin{proof}
For $r = 1$ 
\end{proof}

\begin{theorem}
Let $(A, \m)$ be a Noetherian local ring and $M$ a fintely-generated $A$-module. Then,
\begin{enumerate}
\item If $M$ is Cohen-Macaulay and $\p \in \Ass{A}{M}$ then $\dim{A / \p} = \dim{M}$.
Therefore there are no embedded associated primes.
\item If $a_1, \dots, a_r$ is an $M$-regular sequence then $M_k = M / (a_1, \dots, a_r) M$ is CM iff $M$ is CM.
\item If $M$ is CM then for any $\p \in \Spec{A}$ we have $M_{\p}$ is CM over $A_\p$ and $\depth{\p}{M} = \depth{A_\p}{M_\p}$. 
\end{enumerate}
\end{theorem}

\begin{proof}
By definition, 
\[ \dim{M} = \sup\{ \dim(A / \p) \mid \p \in \Ass{A}{M} \} \] 
\end{proof}

\begin{example}
Let $A = k[x,y]/(x^2, xy)$ is not CW. Let $\p = (x)$ and $\q = (x,y)$ then $\p A_{\q} \subset \q A_{\q}$ but both are associated primes so $\q A_{\q}$ is embedded so $A_{\p}$ is not CW. 
\end{example}

\begin{example}
Let $A = k[x,y,z] / (x)(y,z)$ then $\dim{A} = \dim{A_{\m}} = 2$ with $\m = (x, y, z)$ and $I = (y, z)$. Then $\height{I} = 0$ but $\dim{A / I} = 1$ and the sum is strictly less than $\dim{A} = 2$. 
\end{example}

\begin{theorem}
Let $(A, \m)$ be a Noetherian local Cohen-Macaulay ring. Then for any proper ideal $I \subset A$,
\begin{enumerate}
\item $\height{I} = \depth{I}{A}$
\item $\height{I} + \dim{A / I} = \dim{A}$
\item $A$ is catenary
\item Let $a_1, \dots, a_r \in \m$ be a sequence then the following are equivalent,
\begin{enumerate}
\item $\underline{a}$ is regular
\item $\height{(a_1, \dots, a_i)} = i$ for $i = 1, \dots, r$
\item $\height{(a_1, \dots, a_r)} = r$
\item $a_1, \dots, a_r$ is part of a system of parameters. 
\end{enumerate}
\end{enumerate}
\end{theorem}

\begin{corollary}
Let $(A, \m)$ be a Noetherian local ring then $A$ is CM iff ther exists a system of parameters which is $A$-regular.
\end{corollary}

\begin{proposition}
If $A$ is a regular local ring then $A$ is CM.
\end{proposition}

\begin{proof}
We have $\dim{A} = n$ with $(a_1, \dots, a_n) = \m$ a system of parameters. Since $A$ is a domain $A / a_1$ is a regular local ring then $A / a_1$ is a domain. By induction, $A$ is a CM local ring.
\end{proof}

\begin{proposition}
Let $A$ be a Noetherian ring,
\begin{enumerate}
\item $A$ is CM $\iff A[X]$ is CM.
\item $A$ is CM $\iff \hat{A}$ is CM.
\item $A$ is CM $\implies A$ is universally cantenary.
\end{enumerate}
\end{proposition}

\begin{theorem}
Let $A$ be a Noetherian ring then $A$ is Cohen-Macaulay if and only if for any ideal $I = (a_1, \dots, a_r)$ such that $\height{I} = r$ then $\p \in \Ass{A}{A/I}$ implies $\height{\p} = r$. 
\end{theorem}

\begin{proof}
Suppose that $A$ is CM and $I = (a_1, \dots, a_r)$ such that $\height{I} = r$. Take $\p \in \Ass{A}{A / I}$. Suppose that $\p$ is embedded then $A_{\p}$ is local and $(a_1, \dots, a_r)$ is regular so $A_{\p} / I A_{\p}$ is CM local which implies that there exist no embedded primes. Thus all primes in $\Ass{A}{A / I}$ are minimal so $\height{\p} = \height{I} = r$. 
\bigskip\\
Conversely, let $\p \in \Spec{A}$ have $\height{\p} = r$. 
\end{proof}

\section{Last Lecture}

Let $A$ be a Noetherian ring and $k \ge 0$. Then define two properties,
\begin{definition}
We have $(S_k)$ exactly when $\forall \p \in \Spec{A}$ that $\depth{}{A_{\p}} \ge \inf(k, \height{\p})$ and we have $(R_k)$ exactly when $\forall \p \in \Spec{A}$ such that $\height{\p} \le k$ then $A_{\p}$ is regular.
\end{definition}

\begin{lemma}
$A$ is unmixed (i.e. $\Ass{A}{A}$ is exactly the set of minimal primes) iff $S_1$ holds
\end{lemma}

\begin{proof}
Assume that $A$ is unmixed. If $\p \in \Spec{A}$ with $\height{\p} \ge 1$ then $A_\p$ is unmixed so $\Ass{A}{A_\p} = \{ \q \subsetneq \p \mid \q \text{ is minimal} \}$. Thus there exists,
\[ x \in \p A_{\p} \setminus \bigcup_{\q \in \Ass{A}{A_\p}} \q A_\p \]
is is $A_{\p}$-regular which implies that $\depth{}{A_\p} \ge 1$.
\bigskip\\
Converesely, if $A$ is mixed then there exists $\p \in \Ass{A}{A}$ which is not mixed. Thus $\height{\p} \ge 1$ and $\p A_\p$ is contained in the zero divisors of $A_\p$. But $\p A_\p \in \Ass{A}{A_\p}$ which implies that $\depth{}{A_\p} =0$ so $S_1$ does not hold.  
\end{proof}

\begin{corollary}
$S_k$ holds iff $\forall i \in \{0, \cdots, k - 1\}$ and any $A$-regular sequence $a_1, \dots, a_i$ then $\Ass{A}{A / (a_1, \dots, a_i)}$ is unmixed. For $i = 0$, $A$ is unmixed. In particular, $(S_2)$ holds $\iff$ $A$ is unmixed and $\forall a \in A$ which is $A$-regular then $\Ass{A}{A / a A}$ is unmixed. 
\end{corollary}

\begin{proof}
Suppose that $(a_1, \dots, a_i)$ is $A$-regular then 
\[ \depth{A_\p}{A_\p / (a_1, \dots, a_r)} = \depth{}{A_\p} - i \]
and also 
\[ \dim{A_\p / (a_1, \dots, a_r)} = \dim{A_\p} - i \]
Therefore, $A_\p$ is regular. 
\end{proof}

\begin{corollary}
If $(S_2)$ holds and $A$ is a domain then,
\[ A = \bigcap_{\height{\p} = 1} A_\p \]
\end{corollary}

\begin{proof}
Take,
\[ \frac{\alpha}{\beta} \in \bigcap_{\height{\p}} A_\p \]
Consider $I = \{ x \in A \mid \frac{\alpha}{\beta} x \in A \} = \Ann{A}{\bar{\alpha} \in A / \beta A}$. Then $A / I$ embedds in $A / \beta A$. Thus $\Ass{A}{A/I} \subset \Ass{A}{A / \beta A}$ . If $I \neq A$ then $A / I$ is nonempty so $\Ass{A}{A/I} \neq 0$. Thus take $\q \in \Ass{A}{A/\beta A}$. Because $\Ass{A}{\beta A}$ is unmixed this forces $\height{\q} = 1$. Moreover, $I \subset \q$ which is a contradiction because $\frac{\alpha}{\beta} \in A_{\q}$ which implies that $I_\q = A_\q$ which is impossible because $\q \in \Ass{A}{A / I}$ and thus $(A / I)_{\q} = A_\q / I_\q \neq 0$. To see this, consider the exact sequence,
\begin{center}
\begin{tikzcd}
0 \arrow[r] & I \arrow[r] & A \arrow[r] & A / \beta A
\end{tikzcd}
\end{center}
under the map $y \mapsto y \bar{\alpha} \in A / \beta A$. Then localizing, we get the exact sequence,
\begin{center}
\begin{tikzcd}
0 \arrow[r] & I_\q \arrow[r] & A_\q \arrow[r] & (A / \beta I)_\q
\end{tikzcd}
\end{center}
However, $y \mapsto y \bar{\alpha}$ is the zero map inside $(A / \beta I)_\q$ since $\frac{\alpha}{\beta} \in A_\q$ giving $I_\q = A_\q$. 
\end{proof}


\begin{theorem}
$A$ is normal $\iff$ both $(S_2)$ and $(R_1)$ hold.
\end{theorem}

\begin{proof}
Suppose that $A$ satisfies $(S_2)$ and $(R_1)$. Consider $\Frac{A}$ the \textit{ring} of fractions since $A$ is not necessarily a domain. $\Frac{A} = S^{-1}_0 A$ where $S_0$ is the set of $A$-regular elemetns. Then,
\[ S_0 = A \setminus \bigcup_{i = 1}^s \p_i \]
where $\p_1, \dots, \p_s$ are the minimal primes. Then by $(R_2)$, for all minimal $\p \in \Spec{A}$ we have $A_\p$ is regular (since it is a domain) and thus a field and thus contains no nilpotents so $A$ is reduced. Therefore,
(COMPLETE)
\end{proof}

\begin{lemma}
If $A$ is noetherian domain then any element can be written as the product of irreducibles.
\end{lemma}

\begin{proof}
Let $\Sigma$ be the set of elements of $A$ which cannot be written as a product of irreducibles. $\Sigma$ is a poset under $x \le y \iff (x) \subset (y) \iff y \divides x$. Because any ascending chain of ideals of $A$ must stabilize this means that any chain in $\Sigma$ must achieve a maximum.  Thus, if we assume that $\Sigma$ is nonempty then, by Zorn's Lemma, $\Sigma$ has a maximal element $z \in \Sigma$. Since $z \in \Sigma$ it cannot be irreducible so $z = ab$ with $a, b \notin A^\times$. Then $z \not\divides a$ and $z \not\divides b$ otherwise $a = zk$ and then $z = z kb$ implying that $b \in A^\times$ since $A$ is a domain. Thus, $a, b \le z$ so $a,b \notin \Sigma$. Therefore, $a$ and $b$ are products of irreducibles and thus $z = ab$ is a product of irreducibles contradicting $z \in \Sigma$. Therefore, $\Sigma$ must be empty.   
\end{proof}

\begin{proposition}
Let $A$ be a Noetherian domain. Then $A$ is a UFD if and only if for all $\pi \in A$ irreducible, $\pi A$ is a prime ideal. 
\end{proposition}

\begin{proof}
Assume that $A$ is a UFD. Suppose that $\pi$ is irreducible and $ab \in \pi A$ then $ab = \pi k$. Since $A$ is a UFD then $\pi$ must appear in the factorization of $ab$ and thus in the factorization of either $a$ or $b$. Thus, $a \in \pi A$ or $b \in \pi A$.
\bigskip\\
Conversely, since $A$ is Noetherian any $x$ may be expressed as a product of irreducibles. It suffices to prove that such a decomposition is unique. Suppose that $\pi_1 \cdots \pi_n = \pi_1' \cdots \pi_n'$. Then $\pi_1 \divides \pi_1' \cdots \pi_n'$ so it must divides some $\pi_i'$ since $\pi_1 A$ is a prime ideal. However, $\pi_i'$ is irreducible so $\pi_1$ and $\pi_i'$ differ up to a unit. Prodceeding in this way, we see that each $\pi_i$ corresponds up to a unit to some $\pi_j'$. 
\end{proof}

\begin{remark}
The condition $\pi A$ is a prime ideal for all irreducible $\pi$ is the statment of Euclid's lemma. If $\pi \divides ab$ then $ab \in \pi A$ so $a \in \pi A$ or $a \in \pi A$ and thus $\pi \divides a$ or $\pi \divides b$. 
\end{remark}

\begin{theorem}
Let $A$ be a Noetherian domain, then $A$ is a UFD if and only if every ideal of height one is principal.
\end{theorem}

\begin{proof}
Suppose the condition holds. If $\pi$ is irreducible then let $\pi A \subset \p$ where $\p$ is a minimal prime above $\pi A$. Then $\height{\p} = \height{\pi A} = 1$ because $\pi$ is $A$-regular. Thus, by the condition, $\p$ is principal so $\p = a A$ and thus $\pi = ab$ for some $b \in A$ which implies that $b \in A^\times$ since $a \notin A^\times$ (because $\p$ is proper) and $\pi$ is irreducible. Thus, $\p = \pi A$ so $\pi A$ is a prime ideal proving that $A$ is a UFD. 
\end{proof}

\begin{theorem}
If $A$ is a regular ring then $A$ is a UFD. 
\end{theorem}



\end{document}
