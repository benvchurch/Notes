\documentclass[12pt]{extarticle}
\subimport{../General/}{General_Includes}

% Fields and Polynomials

\newcommand{\ch}[1]{\mathrm{char} \: #1}
\newcommand{\minimal}[2]{\mathrm{Min}(#1;#2)}
\newcommand{\Disc}[1]{\mathrm{Disc}(#1)}
\newcommand{\sgn}[1]{\mathrm{sgn}(#1)}


\begin{document}
\atitle{5}

\section*{Problem 1.}

\begin{enumerate}
\item The subgroup $\left< (1 \: 2 \: 3 \: 4) \right>$ on its own acts trivially on $\{1,2,3,4\}$ because given two numbers $x, y \in \{1, 2, 3, 4\}$ then $(1 \: 2 \: 3 \: 4)^k$ takes $x$ to $y$ where $\mod{k}{y - x}{4}$. However, $D_4$ preserves a square so it cannot map adjacent points to non adjacent points. For example, the pair $(1, 2)$ cannot be sent to $(1, 3)$. Thus, $D_4$ is not doubly transitive. 

\item Consider the standard representation fo $D_4$ on the vectorspace $\C^4 = W_1 \oplus W_2$ where $W_1 = \{(t_1, t_2, t_3, t_3) \mid \sum_{i = 1}^4 t_i = 0 \}$. The character is a class function so we need only compute it on a representative from each equivalence class. $D_4$ has $5$ conjugacy classes, using the notation $r = (1 \: 2 \: 3 \: 4)$ and $f = (1 \: 2) (3 \: 4)$, namely,
\[[e] = \{e\} \quad [r] = \{r, r^{-1}\} \quad [r^2] = \{r^2\} \quad [f] = \{f, r^2 f\} \quad [rf] = \{rf, r^3 f\} \]
The character of the permutation relation is the number of fixed points,  $\chi_{\C[X]}(g) = \# (X^g)$. Thus, $\chi_{\C^4}(e) = 4$ and $\chi_{\C^4}(r) = \chi_{\C^4}(r^2) = 0$ and $\chi_{\C^4}(f) = 0$ and $\chi_{\C^4}(rf) = 2$ because $rf = (1 \: 2 \: 3 \: 4)(1 \: 2)(3 \: 4) = (1 \: 3)$ which fixes $2$ and $4$. Thus, $\chi_{W_2} = \chi_{\C^4} - \chi_{W_1} = \chi_{\C^4} - 1$.    

\item 
\[ \left< \chi_{\C^4}, \chi_{\C^4} \right> = \frac{1}{8} \left( 4^2 + 2 \cdot 0 + 0 + 2 \cdot 0 + 2 \cdot 2^2 \right) = 3\]
Also,
\[ \left< \chi_{W_2}, 1 \right> = \frac{1}{8} \left( (4-1) + 2 \cdot (0 - 1) + (0 - 1) + 2 \cdot(0 - 1) + 2 \cdot(2 - 1)  \right) = 0 \]
Likewise,
\[ \left< \chi_{W_2}, \chi_{W_2} \right> = \frac{1}{8} \left( (4-1)^2 + 2 \cdot (0 - 1)^2 + (0 - 1)^2 + 2 \cdot(0 - 1)^2 + 2 \cdot (2 - 1)^2  \right) = 2 \]
Since this value is not $1$, the representation $W_2$ cannot be irreducible. 
\item Since $\left< \chi_{W_2}, \chi_{W_2} \right> = 2$ we know that $W_2 = V_1 \oplus V_2$ where $V_1$ and $V_2$ must be distinct irreducible representations. Consider the dimension $2$ representation $V$, whose character satisfies $\chi_V(1) = 2$ and $\chi_V(r^2) = -2$ and $\chi_V(g) = 0$ otherwise. Therefore,
\[ \left< \chi_V, \chi_{W_2} \right> = \frac{1}{8} \left( 2 \cdot (4 - 1) + 2 \cdot 0 \cdot (0 - 1) + (-2) \cdot (0 - 1) + 2 \cdot 0 \cdot (0 - 1) + 2 \cdot 0 \cdot (2 - 1) \right) = 1\]
so there is exactly one copy of $V$ in the decomposition of $W_2$. Therefore, $W_2 = V \oplus V_2$ where $\dim{V_2} = 1$ by dimension counting. Therefore, $V_2 \cong \C(\lambda)$ for some homomorphism $\lambda : D_4 \to \C^\times$. We know that $\lambda(f) \in \C^\times$ has order dividing two so $\lambda(f) = (-1)^s$ and $\lambda(rf) = \lambda(fr^3)$ so $\lambda(r) \lambda(f) = \lambda(r)^3 \lambda(f)$ and thus $\lambda(r) = \lambda(r)^3$ so $\lambda(r)$ has order dividing $2$ so $\lambda(r) = (-1)^k$. The character of the representation $\C(\lambda)$ is simply $\lambda$ iself. Calculating the inner product,
\begin{align*}
\left< \chi_\lambda, \chi_{W_2} \right> = \frac{1}{8} \left( 3 + 2 \cdot (-1)^k \cdot (-1) + 1 \cdot (-1) + 2 (-1)^s \cdot (-1) + 2 \cdot (-1)^{k + s} \cdot 1 \right)
\end{align*} 
If $\C(\lambda)$ is to be irreducible and in the expansion of $W_2$ then we need this inner product to be $1$. Therefore,
\[ (-1)^{k + s} - (-1)^k - (-1)^s = 3\]
which forces $k$ and $s$ to be odd. Therefore, $\lambda(r) = \lambda(f) = -1$ and since $D_4 = \left<r, f \right>$ the homomorphism is determined on the entire group. 
  
\end{enumerate}

\section*{Problem 2.}

Let $(V, \rho_V)$ be a $G$-representation. Consider the dual representation $(V^*, \rho_{V^*})$ where the action of $\rho_{V^*}$ is defined as, $\rho_{V^*}(g) \cdot \varphi = \varphi \circ \rho_{V}(g^{-1})$. Thus, $\rho_{V^*}(g) = (\rho_{V}(g^{-1}))^*$. Thus, 
\[\chi_{V_*}(g) = \tr{\rho_{V^*}(g)} = \tr{(\rho_{V}(g^{-1}))^*} = \tr{\rho_{V}(g^{-1})} = \overline{\tr{\rho_{V}(g)}} = \overline{\chi_V(g)}\]
However, we know that two $G$-representations $V$ and $W$ are isomorphic if and only if $\chi_V = \chi_W$. Thus, 
\[ V \cong V^* \iff \chi_V = \chi_{V^*} = \overline{\chi_V}\]
However, $\overline{\chi_V(g)} = \chi_V(g) \iff \chi_V(g) \in \R$. Thus,
\[ V \cong V^* \iff \chi_V = \overline{\chi_V} \iff \Im{\chi_V} \subset \R \]

\section*{Problem 3.}

Let $\lambda : G \to \C^\times$ be a homomorphism. Consider the one-dimensional representation $\C(\lambda)$ inside $\C[G]$. This corresponds to a vector $v \in \C[G]$ such that $\rho_{reg}(g) \cdot v = \lambda(g) v$ for all $g \in G$. We can write any $v \in \C[G]$ as $v = \sum\limits_{h \in G} t_h \: h$ then,
\[\rho_{reg}(g) \cdot v = \sum\limits_{h \in G} t_h \: gh = \sum\limits_{h' \in G} t_{g^{-1}h'} \: h' = \lambda(g) v = \sum\limits_{h \in G} ( \lambda (g) \cdot  t_h ) \: h \]
For these to be equal, the coeficients must be equal since $G$ is a basis of $\C[G]$. Thus, $\lambda(g) \cdot t_h = t_{g^{-1}h}$ for each $h$ and $g$. Therefore, $\lambda(g^{-1}) \cdot t_{e} = t_{g}$ so every constant $t_g$ is determined by the single constant $t_{e} = c$. Then,
\[ v = c \sum_{g \in G} \lambda(g^{-1}) \cdot g \]
In particular, the representation $\C(\lambda)$ inside $\C[G]$ is the span of $\sum\limits_{g \in G} \lambda(g^{-1}) \cdot g$. \bigskip\\
Viewing $\C[G]$ as the space of (finitely supported) functions $f : G \to \C$, the function corresponding to $v$ is the map $f(g) = \lambda(g)^{-1}$. Then, 
\[(\rho_{reg}(g) \cdot f)(h) = f(g^{-1}h) = \lambda(g^{-1} h)^{-1} = \lambda(g^{-1})^{-1} \cdot \lambda(h)^{-1} = \lambda(g) \cdot \lambda(h)^{-1} = \lambda(g) \cdot f(h)\]
Therefore, on the subspaces spanned by $f$, the map $\rho_{reg}(g)$ acts as multiplication by $\lambda(g)$. 
  
\section*{Problem 4.}
Let $\rho_V : G \to \aut{V}$ be a representation and let $\lambda : G \to \C^\times$ be a homomorpism, corresponding to the one-dimensional representation $\C(\lambda)$. 
\begin{enumerate}
\item Define the map $\lambda \otimes \rho_V : G \to \aut{V}$ by,
\[ (\lambda \otimes \rho_V)(g) \cdot v = \lambda(g) \rho_V(g) \cdot v \]
Since $\lambda (g) \neq 0$ the linear map $\lambda(g) \rho_V(g)$ is invertible. Likewise, view $\lambda(g)$ as a linear map on $V$ given by $\lambda(g)(v) = \lambda(g) \cdot v$. Thus, $(\lambda \otimes \rho_V)(g) = \lambda(g) \circ \rho_V(g)$ so, since it is a composition of linear maps, $(\lambda \otimes \rho_V)(g)$ is linear. Finally, take any $g, h \in G$ and consider,
\begin{align*}
(\lambda \otimes \rho_V)(gh) v & = \lambda(gh) \rho_V(gh) v = \lambda(g) \lambda(h) \cdot \rho_V(g) \circ \rho_V(h) v = \lambda(g) \rho_V(g) ( \lambda(h) \rho_V(h) v)
\\
& = (\lambda \otimes \rho_V)(g) \circ (\lambda \otimes \rho_V)(h) v
\end{align*} 
where I have used the fact that $\rho_V(g)$ is linear. Thus, $\lambda \otimes \rho_V$ is a homomorphism to $\aut{V}$ and thus a $G$-representation.

\item Suppose that $W \subset V$ is a $G$-invariant subspace under $\rho_V$. Then for $w \in W$ consider $(\lambda \otimes \rho_V)(g) w = \lambda(g) \cdot \rho_V(g) w$ but $\rho_V(g) w = w' \in W$ and $\lambda(g) w' \in W$ so $(\lambda \otimes \rho_V)(g) w \in W$. Therefore, $W$ is a $G$-invariant subspace under $\lambda \otimes \rho_V$. Likewise let $W$ be a $G$-invariant subspace under $\lambda \otimes W$ and take $w \in W$. Then, consider $(\lambda \otimes \rho_V)(g) w = \lambda(g) \cdot \rho_V(g) w \in W$. Since $W$ is invariant under scaling and $\lambda(g) \neq 0$, take $\lambda(g)^{-1} \cdot (\lambda(g) \cdot \rho_V(g) w) = \rho_V(g) w \in W$. Therefore, $W$ is a $G$-invariant subspace under $\rho_V$. \bigskip\\
We have shown that the $G$-invariant subspaces under these two representations correspond. Therefore, $(V, \rho_V)$ is irreducible $\iff (V, \lambda \otimes \rho_V)$ is irreducible.

\item 
$\chi_{\lambda \otimes \rho} = \tr{\lambda(g) \rho_V(g)} = \lambda(g) \tr{\rho_V(g)} = \lambda(g) \chi_V$ since the trace scales linearly under scalar multiplication. Therefore, $\chi_{\lambda \otimes \rho} = \chi_{V} \iff \lambda(g) \chi_V(g) = \chi_V(g)$ iff either $\lambda(g) = 1$ or $\chi_V(g) = 0$. However, two $G$-representations are isomorphic if and only if their characters agree. Thus, $\rho_V$ and $\lambda \otimes \rho_V$ are isomorphic iff for every $g$ either $g \in \ker{\lambda}$ or $\chi_V(g) = 0$. 

\item The trace of the $S_n$-representation $(W_2, \rho_{W_2})$ where $W_2 = \{(t_1, \dots, t_n) \mid \sum_{i = 1}^n t_i = 0 \}$ is given by $\chi_{W_2}(g) = \chi_{\C^n}(g) - 1 = \#(\{1, \cdots, n\}^g) - 1$. We explicitly calculated that standard character of $S^3$ on the previous homework. $\chi_{st}(e) = 3$ and $\chi_{st}(\sigma) = 0$ and $\chi_{st}(\tau) = 1$. Therefore, $\chi_{W_2}(e) = 2$ and $\chi_{W_2}(\sigma) = -1$ and $\chi_{W_2}(\tau) = 0$. Because the character is a class function, these values determine the character everywhere. In particular, $\chi_{W_2}$ is zero on all the odd permutations. By the above criterion, $\epsilon \otimes \rho$ is therefore isomorphic to $\rho$. However, for $n \ge 4$, since $\chi_{W_2}(g) = \#(\{1, \cdots, n\}^g) - 1$ character of $(1 \: 2)$ is $n - 3 \ge 1$ for $n \ge 4$. However, $(1 \: 2)$ is an odd permutation ($(1 \: 2) \notin \ker{\epsilon}$) and $\chi((1 \: 2)) \neq 0$ so $\epsilon \otimes \rho$ and $\rho$ cannot be isomorphic by the above criterion.  
\end{enumerate}

\section*{Problem 5.}

Let $N \triangleleft G$ and $(V, \rho)$ be a $G/N$-representation of $G$ and let $\pi : G \to G/N$ be the quotient map. Suppose that $W \subset V$ is $G/N$-invariant then $\rho(gN) \cdot W \subset W$ for any $gN \in G/N$. Then, for any $g \in G$ the map $\rho \circ \pi(g) \cdot W = \rho(gN) \cdot W \subset W$. Thus, $W$ is $G$-invariant. Conversely, if $W$ is $G$-invariant, then $\rho \circ \pi(g) \cdot W \subset W$ but $\rho \cdot \pi(g) = \rho(g N)$ so $W$ is also $G/N$-invariant. Therefore, the set of $G/N$-invariant subspaces of $V$ w.r.t the map $\rho$ is equal to the set of $G$-invariant subspaces of $V$ w.r.t the map $\rho \circ \pi$. Therefore,
\[ (V, \rho) \text{ is an irreducible } G/N\text{-representation } \iff (v, \rho \circ \pi) \text{ is an irreducible } G\text{-representation} \] 

\section*{Problem 6.}
Let $H = \{1, (1 \: 2)(3 \: 4), (1 \: 3)(2 \: 4), (1 \: 4)(2 \: 3) \} \triangleleft A_4$
\begin{enumerate}
\item Since $|A_4/H| = |A_4|/|H| = 12/4 = 3$, the quotient $A_4/H$ must be cyclic of order $3$. Furthermore, $\Z/3\Z$ is generated by any nonzero element and since $(1 \: 2 \: 3) \notin H$ we have that $(1 \: 2 \: 3)$ generates $A_4/H$. 

\item Since $|A_4/H| = 3 = \sum\limits_{i = 1}^g d_i^2$ we must have three one-dimensional irreducible representations of $A_4/H$. Each of these representations lift to one-dimensional irreducible representations of $A_4$. These representations of $\Z / 3 \Z$ are generated by homomorphisms $\lambda : \Z / 3 \Z \to \C^\times$ which are uniquely characterized by an integer $k$ such that $\lambda(1) = \zeta_3^k$ since $\lambda(1)$ has order dividing $3$. For these one-dimensional representations $\chi_{\rho} = \lambda$. However, the characters of these lifts are simply given by, $\chi_{\rho \circ \pi} = \chi_\rho \circ \pi$. Thus, the characters of these one-dimensional $A_4$-representations are, $\chi_k(g) = \chi_\rho(g H) = (\zeta_3^k)^m$ where $gH = (1 \: 2 \: 3)^m H$.

\item $(1 \: 3 \: 2) = (1 \: 2 \: 3)^2$ and thus $\chi_1((1 \: 3 \: 2)) = \zeta_3^2 \neq \zeta_3 = \chi_1((1 \: 2 \: 3))$. However, any character is a class function so $(1 \: 2 \: 3)$ and $(1 \: 3 \: 2)$ cannot lie in the same conjugacy class.  

\item The character $\chi_{W_2}$ on $A_4$ is the same as the character for $S_4$ under the standard representation restricted to elements of $A_4$. Thus, $\chi_{A_4} = \chi_{st} - 1$. However, $\chi_{st}$ simply counts the fixed points of the action of a permutation. The only elements with fixed points are three cycles and two cycles, the latter of which are not contained in $A_4$. Therefore, $\chi_{W_2}(e) = 3$ and $\chi_{W_2}((a \: b \: c)) = 1 - 1 = 0$ and $\chi_{W_2}(g) = 0 - 1 = - 1$ otherwise. Since there are eight three-cycles,
\[ \left< \chi_{W_2}, \chi_{W_2} \right> = \frac{1}{12} \left( 3^2 + 8 \cdot 0^2 + 3 \cdot (-1)^2 \right) = \frac{12}{12} = 1 \]
Therefore, $W_2$ is an irreducible $A_4$-representation.    

\item We know that $|A_4| = \sum\limits_{i = 1}^g d_i^2$. However, in part (b) we found three one-dimensional irreducible representations and in part (e) we found a three-dimensional irreducible representation. Furthermore, $1 + 1 + 1 + 3^2 = 12 = |A_4|$ so we have found every irreducible $A_4$-representation up to isomorphism. 
\end{enumerate}


\end{document}