\documentclass[12pt]{extarticle}
\subimport{../General/}{General_Includes}

% Fields and Polynomials

\newcommand{\ch}[1]{\mathrm{char} \: #1}
\newcommand{\minimal}[2]{\mathrm{Min}(#1;#2)}
\newcommand{\Disc}[1]{\mathrm{Disc}(#1)}
\newcommand{\sgn}[1]{\mathrm{sgn}(#1)}
 
\begin{document}
\atitle{1}
 
\section*{Problem 1.}

Let $v_1 = (2, 5)$ and $v_2 = (-1, 3)$ then consider the map $G : \R^2 \to \R^2$ such that $G(e_1) = v_1 = 2 e_1 + 5 e_2$ and $G(e_2) = v_2 = - e_1 + 3 e_2$. Using the expansion $G(e_i) = \sum_{i} C_{ji} e_j$. Therefore, $G$ is represeneted by the matrix, 
\[ 
C = \begin{pmatrix}
2 & -1 \\
5 & 3 \\
\end{pmatrix} 
\]  
therefore, the inverse transformation cooresponds to the matrix,
\[ 
C^{-1} = \frac{1}{11} \begin{pmatrix}
3 & 1 \\
-5 & 2 \\
\end{pmatrix} 
\]  
Now, let $F : \R^2 \to \R^2$ be a map such that $F(v_1) = 2v_1$ and $F(v_2) - v_2$. We can rewrite these equations as $F \circ G(e_1) = 2 v_1 = 4 e_1 + 10 e_2$ and $F \circ G(e_2) = -v_2 =  e_1 - 3 e_2$. Using the above relation, the matrix of $F \circ G$ is,
\[ 
(F \circ G)_m = \begin{pmatrix}
4 & 1 \\
10 & -3 \\
\end{pmatrix} 
\]  
and therefore, the matrix for $F$ is given by,
\[F_m = (F \circ G)_m C^{-1} = \frac{1}{11} 
\begin{pmatrix}
7 & 6 \\
45 & 4 \\
\end{pmatrix} 
\]

\section*{Problem 2.}
Let,
\[ 
A = \begin{pmatrix}
5 & -2 \\
7 & -4 \\
\end{pmatrix} 
\] 
 
\begin{enumerate}
\item $\det{A} = 5 \cdot (-4) - 7 \cdot (-2) = -6$ and $ \tr{A} = 5 - 4 = 1$

\item 
\[ p_{A}(t) = \det{\left( I t - A \right)}  = \det{
\begin{pmatrix}
t - 5 & +2 \\
-7 & t + 4
\end{pmatrix}} = (t - 5)(t + 4) + 14 = t^2 - t - 6\] Therefore, the roots of $p_A$ are,
\[ \lambda_{\pm} = \frac{1 \pm \sqrt{1 + 4 \cdot 6}}{2} = \frac{1 \pm \sqrt{25}}{2} = 3, -2 \quad \text{so} \quad \lambda_1 + \lambda_2 = 1 = \tr{A} \quad \text{and} \quad \lambda_1 \lambda_2 = \frac{1 - 25}{4} = - 6 = \det{A} \] 

\item If $\lambda$ is an eigenvalue of $T$ then $\lambda^n$ is an eigenvalue of $T^n$. This is easily proven by induction. Consider $T^n v = \lambda^n v$ then,
\[T^{n+1} v = T (T^n v) = T (\lambda^n v) = \lambda^n T v = \lambda^{n + 1}\]
Therefore, the eigenvalues of $A^n$ are $\lambda_1^n$ and $\lambda_2^n$ and then the trace is given by,
\[ \tr{A^n} = \lambda_1^n + \lambda_2^n = 3^n + (-2)^n\] 

\end{enumerate}

\section*{Problem 3.}

\begin{enumerate}
\item[(i).] 

$\tr{A} = \lambda_1 + \lambda_2$ and $\tr{A^2} = \lambda_1^2 + \lambda_2^2$. Therefore, $(\tr{A})^2 - \tr{A^2} = 2 \lambda_1 \lambda_2 = 2 \lambda_1 (\tr{A} - \lambda_1)$. Thus, $2 \lambda_1^2 - 2 (\tr{A}) \lambda_1 + \left[ (\tr{A})^2 - \tr{A^2} \right]$. Therfore,
\[ \lambda_1 = \frac{2 \tr{A} \pm \sqrt{4 (\tr{A})^2 - 4 \left[ (\tr{A})^2 - \tr{A^2}\right]}}{4} = \frac{\tr{A} \pm \sqrt{ \tr{A^2} }}{2} \]
Then the other root,
\[ \lambda_2 = \tr{A} - \lambda_1 = \frac{\tr{A} \mp \sqrt{ \tr{A^2} }}{2} \]
Therefore, we can determine the eigenvalues up to ordering. 

\item[(ii).] Let $a = \lambda_1 + \lambda_2$ and $b = \lambda_1 \lambda_2$. Then take $f(t) = t^2 - a t + b$. First, 
\[f(\lambda_1) = \lambda_1^2 - (\lambda_1 + \lambda_2) \lambda_1 + \lambda_1 \lambda_2 = \lambda_1^2 - \lambda_1^2 - \lambda_1 \lambda_2 + \lambda_1 \lambda_2 = 0\]
Similarly,
\[f(\lambda_2) = \lambda_2^2 - (\lambda_1 + \lambda_2) \lambda_2 + \lambda_1 \lambda_2 = \lambda_2^2 - \lambda_2^2 - \lambda_1 \lambda_2 + \lambda_1 \lambda_2 = 0\]
Thus, $\lambda_1$ and $\lambda_2$ are the roots of $f$. Futhermore,
\[ (t - \lambda_1) (t - \lambda_2) = t^2 - (\lambda_1 + \lambda_2) t + \lambda_1 \lambda_2 = t^2 - a t + b = f(t) \]
\end{enumerate}

\section*{Problem 4.}

Let $k$ be a field and $G$ a group. Take $f \in k[G]$ and $\delta_h \in k[G]$ is the indicator function at $h$. Then
\[ (\delta_h * f)(g) = \sum_{h_1 h_2 = g} \delta_h(h_1) f(h_2) = \delta_h(h) f(h_2)\]
such that $h h_2 = g$ i.e. $h_2 = h^{-1} g$. Therefore,
\[ (\delta_h * f)(g) = f(h^{-1} g)\]

\section*{Problem 5.}
\begin{enumerate}
\item[(i).]
Let $V_1$ be a vector space with basis $v_1, \dots, v_n$ and $V_2$ a vector space with basis $w_1, \dots, w_m$. Then, consider the set of vectors, $(v_1, 0), \dots, (v_n, 0), (0, w_1), \dots, (0, w_m)$. Take any vector $(v, w) \in V_1 \oplus V_2$ then $v \in V_1$ 
and $w \in V_2$ so these vectors can be expressed in terms of the respective bases,
\[ v = c_1 v_1 + \dots + c_n v_n \quad \text{and} \quad w = d_1 w_1 + \dots d_m w_m\]
for constants in the common field, $c_i, d_j \in k$. Therefore,
\[(v, w) = (c_1 v_1 + \dots + c_n v_n, d_1 w_1 + \dots d_m w_m) = c_1 (v_1, 0) + \dots + c_n (v_n, 0) + d_1 (0, w_1) + \dots + d_m (0, w_m) \]
so these vectors span $V_1 \oplus V_2$. Futhermore, if there exist constants $c_i, d_j \in k$ such that,
\[ c_1 (v_1, 0) + \dots + c_n (v_n, 0) + d_1 (0, w_1) + \dots + d_m (0, w_m) = (v, w) = 0_{V_1 \oplus V_2} = (0, 0)\]
then we know that,
\[ v = c_1 v_1 + \dots + c_n v_n = 0 \quad \text{and} \quad w = d_1 w_1 + \dots d_m w_m = 0 \]
therefore, by the linear independence of the bases $\{v_1\}$ and $\{w_i\}$ we know that all $c_i = d_j = 0$. Therefore, all the coeficients are forced to be zero so this set of vectors is independent. Therefore, $(v_1, 0), \dots, (v_n, 0), (0, w_1), \dots, (0, w_m)$ form a basis of $V_1 \oplus V_2$. 

\item[(ii).] 
Let $V$ be a vector space and $W \subset V$ a subspace with basis $w_1, \dots, w_\alpha$. This is extended to a basis $w_1, \dots w_\alpha, w_{\alpha + 1}, \dots, w_n$ of $V$.  An arbitrary element of $V/W$ can be written as $v + W$ for some $v \in V$. Therefore, there exist coeficients $c_i \in k$ such that $v = c_1 w_1 + \cdots + c_n w_n$. Thus,
\[ v + W = c_1 w_1 + \dots + c_\alpha w_\alpha + c_{\alpha + 1} w_{\alpha + 1} + \dots + c_n w_n + W = c_{\alpha + 1} (w_{\alpha + 1} + W) + \dots + c_n (w_n + W) \]  
because $w_i \in W$ for $1 \le i \le \alpha$ so $w_i + W = W$. Thus, $w_{\alpha + 1} + W, \dots, w_n + W$ spans $V/W$. Futhermore, suppose that there exist coeficients $c_i \in k$ such that,
\[ c_{\alpha + 1} (w_{\alpha + 1} + W) + \dots + c_n (w_n + W) = c_{\alpha + 1} w_{\alpha + 1} + \dots + c_n w_n + W  = W\]
then the vector $c_{\alpha + 1} w_{\alpha + 1} + \dots + c_n w_n \in W$ so it can be expressed in terms of the basis $w_1, \dots, w_\alpha$. Therefore, there exist coeficients such that,
\[c_{\alpha + 1} w_{\alpha + 1} + \dots + c_n w_n = c_1 w_1 + \dots + c_\alpha w_\alpha   \quad \text{so} \quad c_1 w_1 + \dots + c_\alpha w_\alpha  - \left[ c_{\alpha + 1} w_{\alpha + 1} + \dots + c_n w_n \right] = 0\]
However, $w_1, \dots w_\alpha, w_{\alpha + 1}, \dots, w_n$ is a basis of $V$ so by linear independence, all $c_i = 0$. Therefore, $w_{\alpha + 1} + W, \dots, w_n + W$ is independent and thus a basis of $V/W$.
\end{enumerate}

\section*{Problem 6.}
Let $V$ be a vectorspace, $W \subset V$ a subspace, and $p : V \to W$ a projection such $\Im{p} = W$ and $\forall w \in W : p(w) = w$. Let $p' : V \to V$ be the map $p'(v) = v - p(v)$. 

\begin{enumerate}
\item For $v \in V$, we have $p'(v) = v \iff v - p(v) = v \iff p(v) = 0 \iff v \in \ker{p}$
\item If $v \in \Im{p'}$ then $\exists u \in V$ such that $p'(u) = u - p(u) = v$. Then, $p(v) = p(u) - p(p(u)) = p(u) - p(u) = 0$ so $v \in \ker{p}$. Therefore, $\Im{p'} \subset \ker{p}$. Futhermore, by (a), if $v \in ker{p}$ then $p'(v) = v$ so $v \in \Im{p}$ so $\Im{p'} = \ker{p}$. 
\item Take $v \in \ker{p'}$ then $p'(v) = v - p(v) = 0$ so $p(v) = v$ but $\Im{p} \subset W$ so $v \in W$. Thus, $\ker{p'} \subset W$. Futhermore, if $v \in W$ then $p(v) = v$ so $p'(v) = v - p(v) = 0$ so $v \in \ker{p'}$. Thus, $W \subset \ker{p'}$ so $\ker{p'} = W$. 
\end{enumerate}

\section*{Problem 7.}

Let $V$ be a vector space and let $p, p' : V \to V$ be linear maps such that $p + p' = \id_V$ and $p \circ p' = p' \circ p = 0$. Let $W = \Im{p}$. Take $v \in \ker{p'}$ then $p'(v) = 0$ but $p(v) + p'(v) = v$ so $p(v) = v$ and thus $v \in \Im{p} = W$. Similarly, take $v \in W$ then $\exists u \in V$ such that $p(u) = v$ then $p' \circ p(u) = 0$ so $p'(v) = 0$ and thus $v \in \ker{p'}$. Therefore $W = \ker{p'}$. By an exactly analogous argument with $p$ and $p'$ swapped, we have that $W' = \Im{p'} =\ker{p}$. \bigskip \\
Finally, if $v \in W \cap W'$ then $v \in \ker{p}$ and $v \in \ker{p'}$ so $p(v) + p'(v) = v$ but $p(v) = p'(v) = 0$ so $v = 0$. Thus, $W \cap W' = \{0\}$. Also, because $p + p' = \id_V$, the function $p + p'$ is surjective which implies that $\Im{p} + \Im{p'} = V$ and thus $W + W' = V$. Therefore, $V = W \oplus W'$.   

\end{document}