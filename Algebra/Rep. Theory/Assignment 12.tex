\documentclass[12pt]{extarticle}
\subimport{../General/}{General_Includes}

% Fields and Polynomials

\newcommand{\ch}[1]{\mathrm{char} \: #1}
\newcommand{\minimal}[2]{\mathrm{Min}(#1;#2)}
\newcommand{\Disc}[1]{\mathrm{Disc}(#1)}
\newcommand{\sgn}[1]{\mathrm{sgn}(#1)}

\usepackage{youngtab}

\begin{document}
\atitle{12}

\section*{Problem 1.}

Let $\lambda = (\lambda_1, \dots, \lambda_{\ell}) \vdash n$. Consider the partition $\lambda^\top = (\mu_1, \dots, \mu_m)$. We know that $\mu_i$ is the length downward in the Young diagram $t$ of $\lambda$. However, the number of rows in which a square appears in column $i$ is exactly the number of $\lambda_j$ which are greater than or equal to $i$. However, the rows of $t^\top$ are the columns of $t$ so $\mu_i$ is the number of $\lambda_j$ greater than or equal to $i$.   

\section*{Problem 2.}

\begin{enumerate}
\item Consider the five partitions of $4$,
\[ (4), \: (3 , 1), \: (2, 2), \: (2 , 1, 1), \: (1, 1, 1, 1) \]
We know that $S^(4)$ is the trivial representation, $S^{(3, 1)}$ is the standard irreducible representation and $S^{(1, 1, 1, 1)}$ to the representation $\C(\epsilon)$. 

\item For $\lambda = (2, 2)$ the Young subgroup $S_\lambda$ is the subgroup permuting inside the partition into two equal groups. Thus, $S_{\lambda} \cong S_2 \times S_2$ since $S_\lambda$ acts transitivly inside the partition. Thus, $\#(S_4 / S_{\lambda}) = 4$. Consider the tabloids of type $\lambda$, 
\begin{center}
\young(ij,\bullet\bullet)
\end{center}
Given fixed $i$ and $j$ the bottom row is the missing two elements in $\{1, 2, 3, 4\}$ and is thus determined up to order. Therefore, chooing a canonical order $i < j$ will uniquely determine the tabloid since the rows are determined up to permutation. Call this tabloid $[ij]$. 

\item Consider the tableau, \[
t = \Yvcentermath1\young(12,34)\]
Which has corresponding tabloid $[t] = [12]$. The column stabilizer of this tableau is the subgroup of $S_4$ which preserves $\{1, 3\}$ and $\{2, 4\}$ and is therefore generated by the 2-cycles $(1 \, 3)$ and $(2 \, 4)$. The corresponding polytabloid is given by,
\[ E_{12} = A_t([t]) = \sum_{\sigma \in C_t} \epsilon(\sigma) [\sigma \cdot t] = [12] - [23] - [14] + [34] \] 
Likewise, consider the tableau and associated polytabloids,
\begin{align*}
t & = \Yvcentermath1\young(13,24) \implies E_{13} = A_t([t]) = [13] - [23] - [14] + [24]
\\
t & = \Yvcentermath1\young(14,23) \implies E_{14} = A_t([t]) = [14] - [24] - [13] + [23]
\\
t & = \Yvcentermath1\young(23,14) \implies E_{23} = A_t([t]) = [23] - [13] - [24] + [14]
\\
t & = \Yvcentermath1\young(24,13) \implies E_{24} = A_t([t]) = [24] - [14] - [23] + [13]
\\
t & = \Yvcentermath1\young(34,12) \implies E_{34} = A_t([t]) = [34] - [14] - [23] + [12]
\\
t & = \Yvcentermath1\young(12,43) \implies E_{12}' = A_t([t]) = [12] - [13] - [24] + [34]
\end{align*}
Therefore, $E_{12}' = E_{12} - E_{13}$. 
\item 
First, let $t$ be a tableau and $C_t$ be the column stabilizer. Then we know that $C_{\tau \cdot t} = \tau C_{t} \tau^{-1}$. Therefore, since $\sigma \in \tau C_t \tau^{-1} \iff \tau^{-1} \sigma \tau \in C_t$ we have,
\[ e_{\tau \cdot t} = A_{\tau \cdot t}([\tau \cdot t]) = \sum_{\sigma \in \tau C_t \tau^{-1}} \epsilon(\sigma) [(\sigma \tau) \cdot t] = \sum_{\sigma \in C_t} \epsilon(\tau^{-1} \sigma \tau) [(\tau \sigma) \cdot t] = \tau \cdot A_t([t]) = \tau \cdot e_t \] Consider the tableaux,
\[
\Yvcentermath1\young(ij,k\ell) \quad \text{and} \quad \Yvcentermath1\young(ji,\ell k)
\]
This pair corresponds to swapping the entire columns. Therefore, the two are related by the element $\tau = (i \, j) (k \, \ell)$. By the above argument, their associated polytabloids are related by,
\[ e_{\tau \cdot t} = \tau \cdot E_{ij} = \tau \cdot \left( [ij] - [kj] - [i \ell] + [k \ell] \right) = [ji] - [\ell i] - [j k] + [\ell k] = [ij] - [kj] - [i \ell] + [k \ell]\]
Therefore, these two tableaux have the same associated polytabloids.
Futhermore, consider the tableaux,
\[
\Yvcentermath1\young(ij,k\ell) \quad \text{and} \quad \Yvcentermath1\young(ij,\ell k)
\]
The two are related by the element $\tau = (k \, \ell)$. Thus, their polytabloids are related by,
\[ e_{\tau \cdot t} = \tau \cdot E_{ij} = \tau \cdot \left( [ij] - [kj] - [i \ell] + [k \ell] \right) = [ij] - [\ell j] - [i k] + [k \ell] = E_{ij} - E_{i k} \]
where $E_{ik}$ is given by acting with $\tau = (j \, k)$,
\[ E_{ik} = e_{\tau \cdot t} = \tau \cdot E_{ij} = \tau \cdot \left( [ij] - [kj] - [i \ell] + [k \ell] \right) = [ik] - [jk] - [i \ell] + [j \ell] \] 

\item Consider three standard tableaux,
\[
\Yvcentermath1\young(12,34) \quad \text{and} \quad \Yvcentermath1\young(12,43) \quad \text{and} \quad \Yvcentermath1\young(13,24) 
\]
and call their associated polytabloids $e_1$ and $e_2$ respectively. We know that permuting the elements in the columns corresponds to acting wit an element of the column stabilizer. However, if $\tau \in C_t$ then,
\[ \tau \cdot A_t([t]) = \sum_{\sigma \in C_t} \epsilon(\sigma) [(\tau \sigma) \cdot t] = \epsilon(\tau) \sum_{\sigma' \in C_t} \epsilon(\sigma') [\sigma' \cdot t] = \epsilon(\tau) A_t([t]) \]
Therefore, permuting the elemnts of the columns only changes the sign of a polytabloid. Furthermore, by the previous problem, swapping the columns leaves the polytabloid invariant. Finally, permuting just the bottom row gives a polytabloid wich is the sum of the origional and the polytabloid with one diagonal reversed (up to sign). However, permuting elements in the columns and swapping columns allows for any tableau to be reduced to one of the three standard tableaux. Furthermore, from above, $e_3 = e_1 - e_2$. Thus, $e_1$ and $e_2$ span $S^{\lambda}$. However, we have shown above that $e_1$ and $e_2$ are not multiples of eachother. Thus, $\dim{S^{(2,2)}} = 2$.     
\end{enumerate}

\section*{Problem 3.}
Let $\lambda = (n - 2, 1, 1) \vdash n$. Any tableau of type $\lambda$ is of the form,
\begin{center}
\young(i\bullet\bullet\bullet\cdots\bullet,j,k)
\end{center}


\begin{enumerate}
\item 
The column stabilizer $C_t$ is the subgroup $S_3$ acting on $(i \: j \: k)$. 
Consider the polytabloid, 
\[ A([t]) = E_{ijk} = \sum_{\sigma \in C_t} \epsilon(\sigma) \sigma \cdot [t] = [jk] - [kj] + [ki] + [ij] - [ji] \]
where the positive terms are ones requiring an even number of swaps. 

\item If $\tau \in C_t$ then,
\[ \cdot A_t([\tau \cdot t]) = \sum_{\sigma \in C_t} \epsilon(\sigma) [\sigma \cdot (\tau \cdot t)] = \epsilon(\tau) \sum_{\sigma \in C_t}
\epsilon(\sigma \tau) [(\sigma \tau) \cdot t] = \epsilon(\tau) A_t([t]) \]
Furthermore ,cyclicly permuting the elements $(i \: j \: k)$ is equivalent to actig acting on the tabloid by a 3-cycle which is even. Thus,
\[ E_{ijk} = E_{jki} = E_{kij}\]
Furthermore, swapping any two of $(i \: j \: k)$ is equivalent to acting on the tabloid by a 2-cycle which is odd. Thus,
\[ E_{jik} = - E_{ijk} \]
This means that the Sprecht representation is spanned by the polytabloids $E_{ikj}$ where we can restic to the case $1 \le i < j < k \le n$. 
\item Furthermore,
\begin{align*}
E_{ijk} - E_{ij\ell} + E_{ik \ell} & = [jk] - [kj] + [ki] - [ik] + [ij] - [ji]
\\
& - \left( [j \ell] - [\ell j] + [\ell i] - [i \ell] + [i j] - [ji] \right)
\\
& + [k \ell] - [\ell k] + [\ell i] - [i \ell] + [i k] - [ki] 
\\
& = [k \ell] - [\ell k] + [jk] - [kj] + [\ell j] - [j \ell] = E_{jk \ell} 
\end{align*}
Therefore, we can generate the span of all $E_{ijk}$ with a fixed $i$. Thus, we only need the span of $E_{1jk}$ where $1 < j < k \le n$. Thus, $S^{(n-2, 1, 1)}$ is spanned by $n-1$ choose $2$ elements taking any two $j, k$ of the numbers $2, \cdots, n$. An arbitrary element can be written as,
\[ E_{1jk} = [jk]  [kj] + [k1] - [1k] + [1j] - [j1] \]
However, the element $[1k]$ only appears in the exansion of $E_{1ik}$ and $E_{1ki}$. Summs of these terms over $i$ contain the elements $[1i]$ respectivly. Thus, the only one which can appear in any expansion of $E_{1jk}$ is $E_{1jk}$ because the only elements of the form $[1i]$ which appear are $[1j]$ and $[1k]$ and elements of this form cannot cancel in the sum because only $E_{1ik}$ contains the element $[1i]$ in the expansion. Therefore, $E_{1jk}$ cannot be written in terms of the other $E_{1 j'k'}$ so this spanning set is a basis. Thus,
\[ \dim{S^{(n-2,1,1)}} = {n - 1 \choose 2} \]

\end{enumerate}


\end{document}