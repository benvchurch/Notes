\documentclass[12pt]{extarticle}
\subimport{../General/}{General_Includes}

% Fields and Polynomials

\newcommand{\ch}[1]{\mathrm{char} \: #1}
\newcommand{\minimal}[2]{\mathrm{Min}(#1;#2)}
\newcommand{\Disc}[1]{\mathrm{Disc}(#1)}
\newcommand{\sgn}[1]{\mathrm{sgn}(#1)}


\begin{document}
\atitle{8}

\section*{Problem 1.}

\begin{enumerate}
\item Let $G$ be a finite group with even order. Let $X \subset G$ be the set of elements of order greater than $2$. For each $x \in X$ we know that $x^{-1} \in X$ since $\ord{x} = \ord{x^{-1}}$. However, if $x = x^{-1}$ then $x^2 = e$ so $ord{x} \le 2$ and thus $x \notin X$. Therefore, $X$ is even because $x$ and $x^{-1}$ are not equal and inverses are unque so $X$ splits up into pairs. Therefore, $G - X$ has at least two elements because $\#(G)$ and $\#(X)$ are even so $\#(G - X)$ is even but $e \in G - X$ so there must be at least two elements in $G - X$.

\item Let $\#(G) = 2a$ such that $a > 1$ is odd. Consider the homomorphism $f : G \to S_G \cong S_{2a}$ defined by the action of $G$ on itself by left multiplication. If $g \in G$ has order $2$ then consider the permutation corresponding to the action $g \cdot h = gh$. If $gh = h$ then $g = e$ contradicting $g$ having order $2$. Therefore, $g$ must swap pairs of elements since $g^2 = e$ so $g \cdot (g \cdot h) = h$. Therefore, $f(g)$ is the product of $a$ disjoint $2$-cycles so $f(g)$ is odd since $a$ is odd. Now, consider the subgroup $f^{-1}(A_{2a}) \triangleleft G$ which is normal because $A_{2a} \triangleleft S_{2a}$. However, by part (i) $G$ has at least two elements of order $2$. For each such $g$, we know that $f(g) \notin A_{2a}$ since it is an odd permutation. Thus, $f^{-1}(A_{2a}) \neq G$. Furthermore, for any $g \in G$ we know that $f(g^2) = f(g)^2 \in A_{2a}$ but if $a > 1$ then there must exist nontrival squares in $G$ else the order of every element would be $2$ implying that there are no Sylow $p$-groups for any $p \neq 2$ which cannot be possible since $a > 1$ is odd and therefore must have an odd prime factor. Thus, $f^{-1}(A_{2a})$ contains some nontrivial element so $f^{-1}(A_{2a})$ is a nontrivial proper normal subgroup so $G$ is not simple. 
\end{enumerate}

\section*{Problem 2.}

Let $p$ and $q$ be primes with $p < q$ with $\mod{q}{1}{p}$. Let $G$ be a non-abelian group of order $pq$. We know by Frobenius that the dimension of any irreducible representation is one of $1, p, q, pq$. Furthermore, the dimensions sum,
\[ \sum_{i = 1}^h d_i^2 = pq \]
Therefore, no irreducible representation can have dimension $q$ or $pq$. Let $c_1$ be the number of $1$-dimensional representations of $G$ and $c_p$ be the number of $p$-dimensional representations. Then,
\[ c_1 + c_p p^2 = pq \implies p \divides c_1\]
Therefore, $c_1 = pk$. However, $c_1 = \#(G^{ab})$ which divides $\#(G)$. Thus, $c_1 = p$ or $c_1 = pq$. If $c_1 = pq$ then $c_p = 0$ and thus $G$ is abelian. Otherwise, $c_1 = p$ and $1 + c_p p = q$. Since $G$ is non-abelian, we have $c_1 = p$ and $c_p = \frac{1}{p}(q - 1)$. Furthermore, the number of conjugacy classes is equal to the number of irreducible representations, $h = c_1 + c_p = p + \frac{1}{p} (q - 1)$. 

\section*{Problem 3.}

Let $G$ be a finite group with $\#(G) = p^a$. By Frobenius' theorem, we know that $d_i \divides p^a$ so $d_i = p^{k_i}$ for each irreducible representation. Furthermore, we know that,
\[ \sum_{i = 1}^h d_i^2 = \sum_{i = 1}^h p^{2 k_i} = p^a \]
For any irreducible representation of dimension greater than one, $p \divides d_i$ so if $c_1$ is the number of $1$-dimensional representations then $p \divides c_1$ since, 
\[c_1 = p^a - \sum_{d_i > 1} p^{2 k_i}\]
Therefore, there must be a nontrivial $G$-representation of dimension $1$ and thus a nontrivial homomorphism $\lambda : G \to \C^\times$ but $\C^\times$ is an abelian group so $\ker{\lambda} \supset G'$ but $\ker{\lambda} \neq G$ since $\lambda$ is nontrivial. Thus, $G' \neq G$. Furthermore, either $G' = \{e\}$ and then all comutators vanish so $G$ is abelian or $G' \neq \{e\}$. Thus, if $G$ is nonabelian then $G'$ is a nontrivial proper normal subgroup. Therefore, if $G$ is simple then $G$ must be abelian and thus must have no nontrivial subgroups with implies that the order of every element is either $1$ or $\#(G)$ so $G$ is cyclic of prime order. Clearly, for the converse, any group with $\#(G) = p$ is cyclic because $\ord{x} \divides p$ so $\ord{x} = p$ for nontrivial $x$ and thus $G$ is cyclic of prime order and then simple.    

\section*{Problem 4.}

Let $d$ be an integer non equal to $\pm 1$ which is square-free. Let $K = \Q(\sqrt{d})$. Consider the element $\alpha = r + s \sqrt{d}$. We know that $\alpha$ is a root of the polynomial,
\[ f(x) = (x - \alpha)(x - \bar{\alpha}) = x^2 - (\alpha + \bar{\alpha}) x + \alpha \bar{\alpha} = x^2 - 2 r + r^2 - s^2 d \]
Therefore, if $2r$ and $r^2 - s^2 d$ are integers then $\alpha$ is an algebraic integers. Thus, if $r, s \in \Z$ then $\alpha$ is an algebraic integer. Furthermore, if $\mod{d}{1}{4}$, and $\alpha \in \Z[\frac{1 + \sqrt{d}}{2}]$ then $r = a + \frac{1}{2} b$ and $s = \frac{b}{2}$ for $a, b \in \Z$. Then, $2 r = 2a + b \in \Z$ and $r^2 - s^2 d = a^2 + ab + \frac{b^2 - b^2 d}{4} = a^2 + ab + b^2 \frac{1 - d}{4} \in \Z$ because $\mod{d}{1}{4}$. Therefore, $\Z[\sqrt{d}]$ are all algebraic integers and for $\mod{d}{1}{4}$ the set $\Z[\frac{1 + \sqrt{d}}{2}]$ are all algebraic integers. 
\bigskip\\
Furthermore, we know that $\alpha, \bar{\alpha} \in \C$ are algebraic integers. Therefore, $\alpha + \bar{\alpha} = 2 r$ is an algebraic integer and a rational and thus an integer. Likewise, $\alpha \bar{\alpha} = r^2 - s^2 d$ is an algebraic integer and is rational and thus is an integer. Therefore, $r = \frac{a}{2}$ where $a\in Z$ which implies that $s^2 d = b + \frac{a^2}{4}$ for $b \in \Z$. Thus, $(2s)^2 d = b + a^2$ so $(2s)^2 d \in \Z$ which implies that $2s$ is an integer because $d$ is squarefree so any denominator of $(2s)^2$ cannot divide $d$. Now we will analyize the following cases, $r = n + \frac{1}{2}$ and $s = m + \frac{1}{2}$ for $n,m \in \Z$. Then,
\[ \left( n + \frac{1}{2} \right)^2 - \left(m + \frac{1}{2} \right)^2 d = n^2 + n + \frac{1}{4} - (m^2 - m - \frac{1}{4})d = n^2 + n - m^2 d - m d + \frac{1 - d}{4} \in \Z\]
Therefore $\mod{d}{1}{4}$.
Next case,  $r = n$ and $s = m + \frac{1}{2}$ for $n,m \in \Z$,
\[ \left( n \right)^2 - \left(m + \frac{1}{2} \right)^2 d = n^2 - (m^2 - m - \frac{1}{4})d = n^2 - m^2 d - m d - \frac{d}{4} \in \Z \]
which is impossible since $d$ is squarefree.
Next case,  $r = n + \frac{1}{2}$ and $s = m$ for $n,m \in \Z$,
\[ \left( n + \frac{1}{2} \right)^2 - \left(m \right)^2 d = n^2 + n + \frac{1}{4} - m ^2d = n^2 + n + \frac{1}{4} - m^2 d \in \Z\]
which is clearly impossible. Finally, if $r, s \in \Z$ then $r^2 - s^2 d \in \Z$ is clearly true. Therfore, we have shown that if $\alpha$ is an algebraic integer then $\alpha \in \Z[\sqrt{d}]$ unless $\mod{h}{1}{4}$ in which case $\alpha \in \Z[\frac{1 + \sqrt{d}}{2}]$. 

\section*{Problem 5.}

Consider the field $\Q(\sqrt{d})$. If $\alpha \in \Q(\sqrt{d})$ and $\alpha$ is an algebraic integer then if $d < 0$ then either $|\alpha| > 0$ or $\alpha = 0$. This is because $\bar{\alpha}$ is also an algebraic integer such that $\alpha \bar{\alpha}$ is a rational algebraic integer so $\alpha \bar{\alpha}$ is an integer. Therefore, $|\alpha| = \sqrt{\alpha \bar{\alpha}} \ge 1$ or $\alpha = 0$ since it is the square root of an integer. However, if $d > 0$ this may not be true. For example, consider $\alpha = 1 - \sqrt{2}$ which has absolute value less than one but positive.

\section*{Problem 6.}

\begin{enumerate}
\item Let $G$ be a finite group and let $C_1$ and $C_2$ be two conjugacy classes in $G$. Consider,
\begin{align*}
(f_{C_1} * f_{C_2})(g) & = \sum_{xy = g} f_{C_1}(x) f_{C_2}(y) = \sum_{xy = g} \mathbf{1}_{(x \in C_1 \text{ and } y \in C_2)}
\\
& = \#(\{(x, y) \in G \times G \mid xy = g \} \cap \{ (x, y) \in G \times G \mid x \in C_1 \text{ and } y \in C_2 \} )
\end{align*}
which is the number of $(x, y) \in G \times G$ such that $xy = g$ and $x \in C_1$ and $y \in C_2$. 

\item The group $S_3$ has conjugacy classes,
\[ C_1 = \{1\}, \quad  C_2 = \{(1 \, 2), \, (1 \, 3), \, (2 \, 3) \}, \quad C_3 = \{ (1 \, 2 \, 3), \, (1 \, 3 \, 2) \} \]
Define $f_{ij} = f_{C_i} * f_{C_j}$. Clearly, $f_{ij} = f_{ji}$. Furthermore, if $j = 1$ then $f_{i1}(g)$ is the number of $(x, y) \in G \times G$ such that $xy = g$ and $x \in C_i$ and $y \in C_1$ so $y = 1$ and then $x = g$ so there is exactly one solution if and only if $g \in C_i$ and none otherwise. Thus, 
\[ f_{i1}(g) = \begin{cases}
1 & g \in C_i \\
0 & g \notin C_i
\end{cases} = f_{C_i}\] 
Therefore, we have determined $f_{ij}$ except $f_{22}$, $f_{23}$, and $f_{33}$. Consider, $f_{22}(1) = 3$ because $(a \, b)^2 = e$. Next, the product of two $2$-cycles is an even permutation and thus in $A_3 \cong \Z/3\Z$. Thus, $f_{22}(g) = 1$ if $g \in A_3$ and $f_{22}(g) = 0$ otherwise. \bigskip\\
Next, $f_{33}(1) = 2$ because the three cycles are inverses. Furthermore, the product of any two three cycles is either a three cycle or $1$. However, there is exacly one way to make any thee cycle from products of two of $(1 \, 2 \, 3)$ and $(1 \, 3 \, 2)$. Thus, $f_{33}(g) = 1$ if $g$ is a three cycle and $f_{33}(g) = 0$ if $g$ is not a three cycle or $1$. 
\bigskip\\
Finally, products of three cycles and two cycles are odd permutations and thus are  $2$-cycles. Given a $2$-cycle $g$ there are exactly two ways to make $g$ as a product $x \in C_2$ and $y \in C_3$ as $(a \, b)(a \, b \, c) = (b \, c)$ and $(b \, c)(a \, c \, b) = (c \, b) = (b \, c)$. Therefore, $f_{23}(g) = 2$ if $g$ is a $2$-cycle and $f_{23}(g) = 0$ otherwise.     
\end{enumerate}

\end{document}