\documentclass[12pt]{extarticle}
\subimport{../General/}{General_Includes}

% Fields and Polynomials

\newcommand{\ch}[1]{\mathrm{char} \: #1}
\newcommand{\minimal}[2]{\mathrm{Min}(#1;#2)}
\newcommand{\Disc}[1]{\mathrm{Disc}(#1)}
\newcommand{\sgn}[1]{\mathrm{sgn}(#1)}


\begin{document}
\atitle{4}

\section*{Problem 1.}

\begin{enumerate}
\item Let $V_1$ and $V_2$ be $G$-representations. Consider the vectorspace $(V_1 \oplus V_2)^G$ and take any $(v_1, v_2) \in (V_1 \oplus V_2)^G$. Then,
\[(\rho_1 \oplus \rho_2) (v_1, v_2) = (\rho_1(v_1), \rho_2(v_2)) = (v_1, v_2) \iff \rho_1(v_1) = v_1 \text{ and } \rho_2(v_2) = v_2\]
Thus, $(v_1, v_2) \in (V_1 \oplus V_2)^G \iff (v_1, v_2) \in V_1^G \oplus V_2^G$. Therefore, $(V_1 \oplus V_2)^G = V_1^G \oplus V_2^G$. In particular, since $(\Hom{V}{W})^G = \repHom{G}{V}{W}$ and $\Hom{W}{V_1 \oplus V_2} \cong \Hom{W}{V_1} \oplus \Hom{W}{V_2}$, we have the relationship,
\[ \repHom{G}{W}{V_1 \oplus V_2} \cong \repHom{G}{W}{V_1} \oplus \repHom{G}{W}{V_2}\]  

\item  Let $W$ be an irreducible $G$-representation of a finite group $G$ and let $V$ be another $G$-representation which is totally reducible. Thus, we can write, $V \cong V_1 \oplus \cdots V_l$ where each $V_i$ is an irreducible $G$-representation. Applying the lemma above inductively,
\[ \repHom{G}{W}{V} \cong \repHom{G}{W}{V_1} \oplus \cdots \oplus \repHom{G}{W}{V_k} \]
and therefore, 
\[ \dim{\repHom{G}{W}{V}} = \dim{\repHom{G}{W}{V_1}} + \cdots + \dim{\repHom{G}{W}{V_k}} \]
However, by Schur's lemma, since $W$ and $V_i$ are both irreducible $G$-representations, $\dim{\repHom{G}{W}{V_i}} = 1$ if $W \cong V_i$ and zero otherwise. Therefore,
\[ \dim{\repHom{G}{W}{V}} = \sum_{i = 1}^k \mathbf{1}(W \cong V_i)
= \#\{V_i \text{ isomorphic to } W\} \]
  
\end{enumerate}

\section*{Problem 2.}

\begin{enumerate}
\item[(i)] Let $G = D_3$ with a two-dimensional representation given by,
\[ \rho(r) = 
\begin{pmatrix}
-1/2 & -\sqrt{3}/2 \\
\sqrt{3}/2 & -1/2
\end{pmatrix} 
\quad 
\rho(s) = 
\begin{pmatrix}
1 & 0 \\
0 & -1
\end{pmatrix} 
\]
Thus, $\chi_2(e) = 2$ and $\chi_2(r) = -1$ and $\chi_2(s) = 0$.
The group $D_3$ has three conjugacy classes, $[e] = \{e\}, [r] = \{r, r^{-1}\}, [s] = \{s, rs, r^2 s\}$. Since the character $\chi_2$ is a class function,
\[ \frac{1}{6} \sum_{g \in D_3} |\chi_2(g)|^2 = \frac{1}{6} \left( |\chi_2(e)|^2 + 2 |\chi_2(r)|^2 + 3 |\chi_2(s)|^2 \right) = \frac{1}{6} \left( 2^2 + 2 \right) = 1\] 

\item[(ii)]
Now, consider the standard representation of $S_3$ on $\C^3$. The group $S_3$ is similarly generated by $\sigma = (1 \: 2 \: 3)$ and $\tau = (1\: 2)$. These are represented by the matrices.
\[ \rho_{st}(\sigma) = 
\begin{pmatrix}
0 & 0 & 1 \\
1 & 0 & 0 \\
0 & 1 & 0
\end{pmatrix} 
\quad 
\rho_{st}(\tau) = 
\begin{pmatrix}
0 & 1 & 0 \\
1 & 0 & 0 \\
0 & 0 & 1
\end{pmatrix} 
\]
Since $\chi_{st}$ is a class function, we need only to compute it on representatives of the three conjugacy classes, $[e] = \{e\}$, $[\sigma] = \{ \sigma, \sigma^{-1} \}$, $[\tau] = \{ \tau, \sigma \tau, \sigma^2 \tau \}$. Now, $\chi_{st}(e) = \dim{\C^3} = 3 = \chi_2(e) + 1$. Likewise, $\chi_{st}(\sigma) = 0 = \chi_2(r) + 1$ and $\chi_{st}(\tau) = 1 = \chi_2(s) + 1$. Therefore, because they are both class functions which agree on each conjugacy class, $\chi_{st} = \chi_2 + 1$.   
\end{enumerate}

\section*{Problem 3.}

Let $Q = \{\pm 1, \pm i, \pm j, \pm k\}$ be the quaternion group with $\rho : Q \to \GL{2}{\C}$ given by,
\[ \rho(\pm 1) = \pm \id \quad
\rho(\pm i) = \pm 
\begin{pmatrix}
i & 0 \\
0 & -i
\end{pmatrix} 
\quad 
\rho(\pm j) = \pm 
\begin{pmatrix}
0 & -1 \\
1 & 0
\end{pmatrix} 
\quad
\rho(\pm k) = \pm 
\begin{pmatrix}
0 & -i \\
-i & 0
\end{pmatrix} 
\]

Therefore, $\chi(\pm 1) = \pm 2$ and $\chi(\pm i) = \chi(\pm j) = \chi(\pm k) = 0$. Then, 
\[ \frac{1}{8} \sum_{g \in Q} \chi(g) \overline{\chi(g)} = \frac{1}{8} \left( |\chi(1)|^2 + |\chi(2)|^2 + 0 \right) = \frac{1}{8} \left( 2^2 + 2^2 \right) = 1\]

\section*{Problem 4.}

\begin{enumerate}
\item[(i)]
Let $V_1$ and $V_2$ be two one-dimensional represnetations of $G$ correspondsing to homomorphisms $\lambda_1, \lambda_2 : G \to \C^\times$. Clearly, if $\lambda_1 = \lambda_2$ then $V_1 \cong V_2$ because they correspond to equal representations. Conversely, if $V_1 \cong V_2$ then there exists a $G$-isomorphism $F : V_1 \to V_2$ such that, $F \circ \lambda_1(g) = \lambda_2(g) \circ F$ for all $g \in G$. Then, because $F$ is linear, 
\[F \circ \lambda_1(g) (v) = F(\lambda_1(g) v) = \lambda_1(g) F(v) = \lambda_2(g) \circ F(v)\] for all $g \in G$. Since $F$ is an isomorphism and the representations are nontrivial, there must be a nonzero vector in the image of $F$. Therefore, $\lambda_1(g) = \lambda_2(g)$ for all $g \in G$. 
\item[(ii)]
Let $V$ be a one-dimensional $G$-representation with corresponding homomorphism $\lambda : G \to \C^\times$. Consider the dual representation $(V^*, \rho_{V^*})$ such that $\rho_{V^*}(g) \cdot f = f \circ \rho_V(g)^{-1}= f \circ \rho_V(g^{-1})$. Consider a linear functional $f \in V^*$ and a vector $v \in V$ then, $\rho_{V^*}(g) \cdot f(v) = f \circ \rho_V(g^{-1})(v) = f(\lambda(g^{-1}v) = \lambda(g^{-1}) f(v)$. Therefore, $\rho_{V^*}$ acts on $V^*$ by multiplication by $\lambda(g^{-1})$. Thus, the $G$-representation $\rho_{V^*}$ corresponds to the homomorphism $\lambda^{-1} : G \to \C^\times$ where $\lambda^{-1}(g) = \lambda(g^{-1})$.   

\item[(iii)]
Let $V_1$ denote the one-dimensional representatiob of $\Z/n\Z$ corresponding to the homomorphism $\lambda_1(k) = e^{2 \pi i k / n}$. Suppose that $n > 2$ then, the corresponding representation on the dual space $(V_1)^*$ is given by the homomorphism $\lambda_1^{-1}(k) = e^{2 \pi i (-k)/n} = e^{- 2\pi i k /n}$. For $n > 2$ we have that $\lambda_1^{-1}(1) = e^{- 2 \pi / n} \neq e^{2 \pi i /n}$ else $4 \pi / n \in 2 \pi \Z$ which it cannot be for $n > 2$. Therefore, by part (i), we have that $V_1 \not\cong (V_1)^*$ because the corresponding complex homomorphisms are not equal. For the case $n = 2$, inversion is the identity automorphism on $\Z / 2 \Z$ so $\lambda_1 = \lambda_1^{-1}$ and thus $V_1 \cong (V_1)^*$ because they correspond to equal homomorphisms into $\C^\times$.   
\end{enumerate}

\section*{Problem 5.}
Let $V = \C^n$ be the standard representation of $S_n$ with a homomorphism $\rho_{st} : S_n \to \aut{\C^n}$. We have $S_n$-invariant projection maps, $p : V \to \C$ and $p' : V \to V^G = \vspan {e_1 + \cdots + e_n}$ given by,
\[ p(t_1, \cdots, t_n) = \frac{1}{n} \sum_{i = 1}^n t_i \quad \text{and} \quad p'(v) = \frac{1}{\#(S_n)} \sum_{\sigma \in S_n} \rho_{st}(v) = \frac{1}{n!} \sum_{\sigma \in S_n} \rho_{st}(v) \]
These maps are, in fact, equal under the identification $\C \cong \vspan{e_1 + \cdots + e_n}$. This holds because there are exactly $(n-1)!$ permutations in $S_n$ taking $e_i$ to $e_j$ for any $1 \le i, j \le n$. Now, write $v = t_1 e_1 + \cdots + t_n e_n$ and consider,
\begin{align*}
p'(v) & = \frac{1}{n!} \sum_{\sigma \in S_n} \rho_{st}(t_1 e_1 + \cdots + t_n e_n) = \frac{1}{n!} \sum_{i = 1}^n \sum_{\sigma \in S_n} t_i \: \rho_{st}(e_i)
\\
& = \frac{1}{n!} \sum_{i = 1}^n t_i \sum_{j = 1}^n (n - 1)! \: e_j = \frac{1}{n} \sum_{i = 1}^n t_i \: (e_1 + \cdots + e_n) = p(t_1, \cdots, t_n) (e_1 + \cdots + e_n)
\end{align*}

\section*{Problem 6.}
Let $V$ be a representation of a finite group $G$. Let $\C[G] \cdot v = \vspan{\rho(g) \cdot v \mid g \in G}$ 
\begin{enumerate}
\item Take $w \in \C[G] \cdot v$ then $w = \sum\limits_{g \in G} t_g \rho(g) v$ with coeficients $t_g \in \C$. Then, for any $h \in G$ consider,
\[\rho(h) w = \sum_{g \in G} \rho(h) (t_g  \: \rho(g) v) = \sum_{g \in G} t_g \: \rho(hg) v) = \sum_{g' \in G} t_{h^{-1}g'} \: \rho(g') v) \in \C[G] \cdot v \]
Therefore $\C[G] \cdot v$ is a $G$-invariant subspace of $V$. 

\item Let $V$ be irreducible and take $v \neq 0$. Then, $v \in \C[G] \cdot v$ so $\C[G] \cdot v$ is a nonempty $G$-invariant subspace of $V$. However, since $V$ is irreducible, there is exactly one such subspace, namely $\C[G] \cdot v = V$. 

\item Since $\C[G] \cdot v = \vspan{\rho(g) \cdot v \mid g \in G} = V$ the set $\{\rho(g) \cdot v \mid g \in G\} = G \cdot v$ spans $V$ and therefore must be at least a large as the dimension of the space. Thus, $\#(G \cdot v) \ge \dim{V}$.  

\item Suppose that $V$ is irreducible and $H \le G$ is an abelian subgroup. Since $H$ is abelian, the restricted representation has a common eigenvector $v$ for all $h \in H$. Suppose that $g_1, g_2 \in G$ lie in the same coset $G/H$ then $g_2{-1} g_1 = h$ so $\rho_{g_2^{-1} g_1} v = \rho_{h} v = \lambda(h) v$ and thus $\rho_{g_1} v = \lambda(h) \rho_{g_2} v$. Thus, $\vspan{\rho_{g_1} v} = \vspan{\rho_{g_2} v}$. Thus, each coset produces a one-dimensional span. Since for purposes of calculating spans, each coset can be replaced by a single element, $\C[G] \cdot v = \vspan{\rho(g) v \mid g \in G} =  \vspan{\rho(h) v \mid h H \in G/H}$ and thus, $\dim{\C[G] \cdot v} \le \#(G/H)$. However, $\C[G] \cdot v = V$ so $\dim{V} \le \#(G/H)$.  
\end{enumerate}


\end{document}