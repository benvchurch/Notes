\documentclass[12pt]{extarticle}
\subimport{../General/}{General_Includes}

% Fields and Polynomials

\newcommand{\ch}[1]{\mathrm{char} \: #1}
\newcommand{\minimal}[2]{\mathrm{Min}(#1;#2)}
\newcommand{\Disc}[1]{\mathrm{Disc}(#1)}
\newcommand{\sgn}[1]{\mathrm{sgn}(#1)}


\begin{document}
\atitle{7}

\section*{Problem 1.}

Let $G$ be a finite group and $V$ and $W$ be irreducible representations of $G$ which characters $\chi_V$ and $\chi_W$. 

\begin{enumerate}
\item[(i)]

Let,
\[F_{V,\chi_W} = \sum_{h \in G} \chi_W(h) \rho_V(h)\]
I claim that $F$ is a $G$-morphism,
\[ \rho_V(g)^-1 \circ F_{V,\chi_W} \circ \rho_V(g) = \sum_{h \in G} \chi_W(h) \rho_V(g^{-1}hg) = \sum_{h' \in G} \chi_W(gh'g^{-1}) \rho_V(h') = F_{V, \chi_W} \]
because $\chi_W$ is a class function. Therefore, by Schur's lemma, $F_{V, \chi_W} = t \cdot \id$ because $V$ is irreducible. Taking the trace,
\[ t \cdot \dim{V} = \tr{F_{V, \chi_W}} = \sum_{g \in G} \chi_W(g) \tr{\rho_V(g)} = \sum_{g \in G} \chi_W(g) \chi_V(g) = \#(G) \inner{\chi_W}{\overline{\chi_V}} \]
and thus 
\[F_{V, \chi_W} = \frac{\#(G) \inner{\chi_W}{\overline{\chi_V}}}{\dim{V}} \id \]
Define,
\[ e_W = \frac{\dim{W}}{\#(G)} \]
and then,
\[ F_{V, e_W} = \sum_{g \in G} \frac{\dim{W}}{\#(G)} \chi_W(g) \rho_V(g) =  \frac{\dim{W}}{\#(G)} \cdot \frac{\#(G) \inner{\chi_W}{\overline{\chi_V}}}{\dim{V}} \id = \frac{\dim{W}}{\dim{V}} \inner{\chi_W}{\overline{\chi_V}} \cdot \id \] 
However, if $V \cong W^*$ then $\dim{V} = \dim{W^*} = \dim{W}$ and we know $\inner{\chi_W}{\overline{chi_V}} = 1$ so $F_{V, e_W} = \id$. If $V \not\cong W^*$ then $\inner{\chi_W}{\overline{\chi_V}} = 0$ so $F_{V, e_W} = 0$. Thus,
\[ F_{V, e_W} = \begin{cases}
\id & V \cong W^* \\
0 & V \not\cong W^*
\end{cases}\]

\item[(ii)]

From the above,
\[ F_{V, \overline{\chi_W}} = \sum_{g \in G} \overline{\chi_W}(g) \rho_V(g) = \sum_{h \in G} \chi_W(h) \rho_V(h^{-1}) = \frac{\#(G) \inner{\overline{\chi_W}}{\overline{\chi_V}}}{\dim{V}} \cdot \id \]
Multipying by $\rho_V(g)$,
\[
\sum_{h \in G} \chi_W(h) \rho_V(h^{-1}g) = \frac{\#(G) \inner{\overline{\chi_W}}{\overline{\chi_V}}}{\dim{V}} \cdot \rho_V(g) \]
then taking the trace,
\[
(\chi_W * \chi_V)(g) = \sum_{h \in G} \chi_W(h) \chi_V(h^{-1}g) = \frac{\#(G) \inner{\overline{\chi_W}}{\overline{\chi_V}}}{\dim{V}} \cdot \chi_V(g) = \frac{\#(G) \inner{\chi_W}{\chi_V}}{\dim{V}} \cdot \chi_V(g) \]

\item[(iii)]

Let $V_1, \cdots, V_h$ be the irreducible representations of $G$ up to isomorphism with $dim{V_i} = d_i$. For each $i$, we define $e_i \in L^2(G)$ by,
\[ e_i = e_{V_i^*} = \frac{d_i}{\#(G)} \chi_{V_i^*} = \frac{d_i}{\#(G)} \overline{\chi_{V_i}} \]
Then using the above result,
\[ e_i * e_j = \frac{d_i d_j}{\#(G)^2} \chi_{V_i^*} * \chi_{V_j^*} = \frac{d_i d_j}{\#(G)^2} \cdot \frac{\#(G) \inner{\chi_{V_i}}{\chi_{V_j}}}{\dim{V_i}} \cdot \chi_V = \frac{d_j}{\#(G)} \inner{\chi_{V_i}}{\chi_{V_j}} \cdot \chi_{V_j} \]
If $i \neq j$ then by definition $V_i \not\cong V_j$ so $\inner{\chi_{V_i}}{\chi_{V_j}} = 0$ and thus $e_i * e_j = 0$. Furthermore, if $i = j$ then since $V_i$ is irreducible, $\inner{\chi_{V_i}}{\chi_{V_i}} = 1$. Thus,
\[ e_i * e_i = \frac{d_i}{\#(G)} \chi_{V_i} = e_i \]
so in summary,
\[ e_i * e_j = 
\begin{cases}
e_i & i = j \\
0 & i \neq j
\end{cases}\]
Furthermore, since the regular representation contains every irreducible $G$-representation with multiplicty $d_i$, we know that,
\[\chi_{reg} = d_1 \cdot \chi_{V_1} + \cdots + d_h \cdot \chi_{V_h} = \#(G) \left( e_1 + \cdots + e_h \right) \]
However,
\[\chi_{reg}(g) = 
\begin{cases}
\#(G) & g = e \\
0 & g \neq e 
\end{cases}\]
and therefore, $e_1 + \cdots + e_h = \delta_e$.
\end{enumerate}

\section*{Problem 2.}

\begin{enumerate}
\item[(a)]

Let $G_1$ and $G_2$ be finite abelian groups and $f : G_1 \to G_2$ is a homomorphism. Let $\chi \in \hat{G}_2$ then $\chi$ is a homomorphism so $\chi \circ f$ is a homomorphism since it is the composition of homomorphisms. Therefore, $f^*(\chi) = \chi \circ f \in \hat{G}_1$. Furthermore, let $\chi_1, \chi_2 \in \hat{G}_2$ then for $g \in G_1$ we have, 
\[f^*(\chi_1 \cdot \chi_2)(g) = (\chi_1 \cdot \chi_2) \circ f(g) = \chi_1(f(g)) \chi_2(f(g)) = (f^*(\chi_1)(g)) (f^*(\chi_2)(g)) = (f^*(\chi_1) \cdot f^*(\chi_2))(g) \] 
Therefore,
\[ f^*(\chi_1 \cdot \chi_2) = f^*(\chi_1) \cdot f^*(\chi_2) \]
so $f^*$ is a homomorphism.

\item[(b)]

Let $G_1, G_2, G_3$ be three finite abelian groups. Let $f_1 : G_1 \to G_2$ and $f_2 : G_2 \to G_3$ be homomorphisms. Consider the map $(f_2 \circ f_1)^* : \hat{G}_3 \to \hat{G}_1$,
\[ (f_2 \circ f_1)^*(\chi) = \chi \circ (f_2 \circ f_1) = (\chi \circ f_2) \circ f_1 = (f_2^* (\chi)) \circ f_1 = f_1^*(f_2^*(\chi)) = (f_1^* \circ f_2^*) (\chi)\]
Thus,
\[ (f_2 \circ f_1)^* = f_1^* \circ f_2^* \]
In summary, the map $G \mapsto \hat{G}$ and $f \to f^*$ is a contravariant endofunctor on the category of finite abelian groups. This endofunctor is a special case of the contravariant hom functor $\Hom{-}{\C}$. Explicitly, $\hat{G} = \Hom{G}{\C}$
and given a map $f : G_1 \to G_2$ we have a map $f^* : \Hom{G_2}{\C} \to \Hom{G_1}{\C}$ given by its action of a map $h : G_2 \to \C$ by $f^*(h) = h \circ f : C_1 \to \C$.     

\item[(c)]

Let $G$ be a finite abelian group and let $H \subset G$ be a subgroup with the quotient map $\pi : G \to G/H$. Consider the map $\pi^* : \widehat{G/H} \to \hat{G}$. Suppose that $\pi^*(\chi) = 1$ then $\chi \circ \pi = 1$. Thus, for any $g \in G$ we have $\chi \circ \pi(g) = 1$ so $\chi(g H) = 1$ for any $g$. However $g H$ enumerates every element of $G/H$ so $\chi = 1$. Thus, $\pi^*$ is an injection. Next, we consider $\Im{\pi^*}$. If $\chi = \pi^*(\chi') = \chi' \circ \pi$ then for any $h \in H$ we have $\chi(h) = \chi' \circ \pi(h) = \chi'(e_{G/H}) = 1$. Conversely, if $\chi(h) = 1$ for any $h \in H$ then if $g_1$ and $g_2$ lie in the same coset i.e. $g_1 H = g_2 H$ so $g_1 = g_2 h$ and thus 
\[\chi(g_1) = \chi(g_2) \cdot \chi(h) \implies \chi(g_1) = \chi(g_2) \]
Thus, $\chi$ is constant on cosets so it decends to a map $\chi'$ on the quotient $G/H$ such that $\chi = \chi' \circ \pi$. Therefore, the image $\pi^*$ is equvalent to the set of characters which are trivial on $H$. 

\item[(d)]

Let $\iota : H \to G$ be the inclusion map. Suppose that $\chi \in \ker{\iota^*}$ then $\iota^* (\chi) = \chi \circ \iota = 1$. Thus, for any $h\in H$ we have $\chi \circ \iota(h) = \chi(h) = \iota^*(\chi)(h) = 1$. Conversely, suppose that $\chi(h) = 1$ for every $h \in H$ then $\iota^*(\chi)(h) = \chi \circ \iota(h) = \chi(h) = 1$. Thus, $\iota^*(\chi)$ is the trivial character in $\hat{H}$ so $\chi \in \ker{\iota^*}$. Thus, $\ker{\iota^*}$ is the set of characters which are trivial on $H$ and thus $\ker{\iota^*} = \Im{\pi^*}$. Since $\pi^*$ is injective we know that $\pi^*$ is a bijection onto its image so,
\[ \#(\widehat{G/H}) = \#(\Im{\pi^*}) \]
and therefore,
\[\#(\ker{\iota^*}) = \#(\Im{\pi^*}) = \#(\widehat{G/H}) = \#(G/H) = \#(G)/\#(H) \]
because $\#(\hat{K}) = \#(K)$. 

\item[(e)]

Since $\iota^* : \hat{G} \to \hat{H}$ is a homomorphism, we know that $\Im{\iota^*} \cong \hat{G}/\ker{\iota^*}$ and therefore,
\[ \#(\Im{\iota^*}) = \#(\hat{G}/\ker{\iota^*}) = \#(\hat{G})/\#(\ker{\iota^*}) = \#(G) \frac{\#(H)}{\#(G)} = \#(H) = \#(\hat{H}) \] 
and therefore $\iota$ is a surjection since $\#(\Im{\iota}) = \#(\hat{H})$ but $\Im{\iota} \subset \hat{H}$ so $\Im{\iota} = \hat{H}$. 
\end{enumerate}

\section*{Problem 3.}

\newcommand{\ev}{\mathrm{ev}}
Let $G$ be abelian and define the map $ev : G \to \hat{\hat{G}}$ given by,
\[ \ev(g)(\chi) = \chi(g) \]
Let $g, h \in G$ then,
\[ \ev(gh)(\chi) = \chi(gh) = \chi(g) \cdot \chi(h) = \ev(g)(\chi) \cdot \ev(h)(\chi) = (\ev(g) \cdot \ev(h))(\chi) \]
so $\ev$ is a homomorphism. Furthermore, suppose $\ev(g) = \hat{\hat{e}}$, that is, $\ev(g)(\chi) = 1$ for every character $\chi$. Then, $\chi(g) = 1$ for each character $\chi$. If every character is trivial on $g$ then it is also triival on $\left<g \right>$ by multiplicativity. Therefore, $\chi(g)$ decends to the quotient $G / \left< g \right>$ so $\hat{G} \cong \widehat{G/\left<g\right>}$ which contradicts the fact that 
\[\#(G) = \#(\hat{G}) = \#(\widehat{G/\left<g\right>}) = \#(G/\left<g\right>) = \#(G) / \#(\left< g \right>)\]
unless $\#(\left< g \right>) = 1$. 
Thus, $g = e$ which implies that $\ev$ is injective. However, $\#(G) = \#(\hat{G}) = \#(\hat{\hat{G}})$ so $\ev$ must also be a surjection. Thus, $\ev : G \to \hat{\hat{G}}$ is a isomorphism. 


\end{document}