\documentclass[12pt]{extarticle}
\subimport{../General/}{General_Includes}

% Fields and Polynomials

\newcommand{\ch}[1]{\mathrm{char} \: #1}
\newcommand{\minimal}[2]{\mathrm{Min}(#1;#2)}
\newcommand{\Disc}[1]{\mathrm{Disc}(#1)}
\newcommand{\sgn}[1]{\mathrm{sgn}(#1)}


\begin{document}
\atitle{10}

\section*{Problem 1.}
Let $G$ be a nonabelian group of order $pq$ with $p < q$ and $\mod{q}{1}{p}$. Let $x \in G$ have order $q$ and $y \in G$ such that $y x y^{-1} = x^t$ for $t \in (\Z / q \Z)^\times$ with order $p$. Finally, $H = \left< x \right>$ is the Sylow $q$-subgroup. 
\begin{enumerate}
\item Let $\lambda : H \to \C^\times$ be a homomomorphism. Given a generator $x \in H$ write $\lambda(x) = e^{2 \pi i a / q}$ for some $0 \le a \le q-1$. Consider, 
\[ \lambda \circ i_y(x) = \lambda(y x y^{-1}) = \lambda(x^{t}) = e^{2 \pi i a t / q} = \lambda_{ta}(x) \]
Furthermore,
\[ \lambda \circ i_{y^k}(x) = \lambda \circ (i_y)^{\circ k}(x) = \lambda(x^{t^k}) = e^{2 \pi i a t^k / q} = \lambda_{a t^k}(x) \]
Now, suppose that $a \neq 0$. Take any $z \in G$. Suppose that $\lambda \circ i_z = \lambda$ let $z x z^{-1} \in x^r$. Then, $\lambda \circ i_z(x) = e^{2 \pi i a r / q} = e^{2 \pi i a / q}$ and thus $\mod{ar}{a}{q} \implies \mod{r}{1}{q} \implies x^r = 1$. Therefore $z x z^{-1} = e$ which implies that $z \in H$ since $|C_x|$ must divide $pq$ and cannot be $pq$ since $G$ is nonabelian and $H \subset C_x$. Conversely, if $z \in H$ then since $H$ is abelian, $i_z = \id_{H}$ and thus $\lambda \circ i_z = \lambda$. 

\item Consider $V$ an irreducible $G$-representation of dimension $p$. We know that $\Res{G}{H}{V}$ is the sum of $p$ irreducible $1$-dimensional representations $W_i$ since $H$ is abelian. Givn $\lambda$ define $\lambda_k = \lambda \circ i_{y^k}$ with character $\chi_k$. However, using Frobenius reciprocity,
\[ \inner{\chi_{k}}{\chi_{\Res{G}{H}{V}}} = \inner{\Ind{G}{H}{\chi_k}}{\chi_V} = \inner{\Ind{G}{H}{\chi_\lambda}}{\chi_V} \]
because $\Ind{G}{H}{\chi_{\lambda}}$ is a class function on $G$ and $i_{y^k}$ does not change $G$ conjugacy classes. 
In the above formula $\Ind{G}{H}{\chi_\lambda}$ is the character of the induced representation $\Ind{G}{H}{\C(\lambda)}$. Therefore, if $\lambda$ is a homomorphism corresponding to a $1$-dimensional summand of $\Res{G}{H}{V}$ then so are all $\C(\lambda \circ i_z)$. However, $\lambda_{y^k}$ and $\lambda_{y^{k'}}$ will be distinct if $\nmod{k}{k'}{p}$ since $y$ has multiplicative order $p$. Therefore, $k = 0, 1, \dots, p - 1$ gives all $p$ one dimensional summands of $\Res{G}{H}{V}$. Furthermore, $V \cong \Ind{G}{H}{W_i}$ so there is a correspondence between $p$-dimensional irreducible $G$-representations and homomorphisms $\lambda : H \to \C^\times$ modulo $\lambda \sim \lambda \circ i_{y^k}$. We know that the set of homomorphisms $\lambda : H \to \C^\times$ is,
\[ \hat{H} \cong (\Z / q \Z )^\times \] 
and we are taking the quotient by the subgroup generated by $1 \circ i_{y}$ since $i_{y^k} = (i_y)^{\circ k}$. However, $y$ has multiplicative order $p$ and this so does $i_y$. Therefore, the $p$-dimensional irreducible $G$-representations are in one-to-one correspondence with the set,
\[ (\Z / q \Z)^\times / \left< 1 \circ i_{y} \right> \]
In particular, there are $(q - 1)/p$ of them.  
\end{enumerate}

\section*{Problem 2.}

We need to write $S_3$ and $S_4$ as unions of conjugacy classes.
\[ S_3 = [e] \cup [ (1 \: 2) ] \cup [(1 \: 2 \: 3)] \quad \text{and} \quad S_4 = [e] \cup [(1 \: 2)] \cup [ (1 \: 2 \: 3) ] \cup [(1 \: 2 \: 3 \: 4)] \cup [(1 \: 2) (3 \: 4)] \]
with sizes,
\[ |S_3| = 6 = 1 + 3 + 2 \quad \text{and} \quad |S_4| = 24 = 1 + 6 + 8 + 6 + 3 \]
Note that $\chi_{W_2} = \chi_{\C^3} - 1$ and $\chi_{V_3} = \chi_{\C^4} - 1$. 

\begin{enumerate}
\item The $S_4$-representations $\C$ and $\C(\varepsilon)$ are defined by the acton of the homomorphisms $1, \varepsilon : S_4 \to \C^\times$ which clearly restrict to $1, \varepsilon : S_3 \to \C^\times$ and thus $\Res{S_4}{S_3}{\C} = \C$ and $\Res{S_4}{S_3}{\C(\varepsilon)} = \C(\varepsilon)$.  

\item We will compute the characters,
\[ \inner{\chi_{\Res{S_4}{S_3}{V_3}}}{\chi_{W_2 \oplus \C}}_{S_3} = \frac{1}{6} \left[ 3 \cdot 3 + 3 \cdot 1 \cdot 1 + 2 \cdot 0 \cdot 0 \right]  = 2 \]
However, the sum of the squared multiplicities of $W_2 \oplus \C$ is $2$ and likewise,
\[ \inner{\chi_{\Res{S_4}{S_3}{V_3}}}{\chi_{\Res{S_4}{S_3}{V_3}}}_{S_3} = \frac{1}{6} \left[ 3 \cdot 3 + 3 \cdot 1 \cdot 1 + 2 \cdot 0 \cdot 0 \right]  = 2 \]
so $\Res{S_4}{S_3}{V_3}$ must also have $2$ irreducible components with multiplicity $1$. Thus, each component must be isomorphic so,
\[ \Res{S_4}{S_3}{V_3} \cong W_2 \oplus \C \]
Furthermore, consider,
\[ \inner{\chi_{\Res{S_4}{S_3}{V_3 \otimes \varepsilon}}}{\chi_{W_2 \oplus \C(\epsilon)}}_{S_3} = \frac{1}{6} \left[ 3 \cdot 3 - 3 \cdot (1-2) \cdot 1 + 2 \cdot 0 \cdot 0 \right]  = 2 \]
However, the sum of the multiplicities of $W_2 \oplus \C(\varepsilon)$ is $2$ and likewise,
\[ \inner{\chi_{\Res{S_4}{S_3}{V_3 \otimes \varepsilon}}}{\chi_{\Res{S_4}{S_3}{V_3  \otimes \varepsilon }}}_{S_3} = \frac{1}{6} \left[ 3 \cdot 3 + 3 \cdot (-1) \cdot (-1) + 2 \cdot 0 \cdot 0 \right]  = 2 \] 
so $\Res{S_4}{S_3}{V_3 \otimes \varepsilon}$ must be the sum of two disjoint irreducible representations. Therefore, 
\[ \Res{S_4}{S_3}{V_3 \otimes \varepsilon} \cong W_2 \oplus \C(\varepsilon) \cong (W_2 \oplus \C ) \otimes \varepsilon \]
because $W_2 \otimes \varepsilon \cong W_2$ since $S_3$ has a unique $2$-dimensional irreducible representation.

\item Consider the $S_3$ representation $\Ind{S_4}{S_3}{\C}$. Using Frobenius reciprocity,
\[ \inner{\chi_{\Ind{S_4}{S_3}{\C}}}{\chi_{V_3 \oplus \C}}_{S_4} = \inner{\chi_{\C}}{\chi_{\Res{S_4}{S_3}{V_3 \oplus \C}}}_{S_3} \] 
However, we can calculate,
\[
\inner{\chi_{\C}}{\chi_{\Res{S_4}{S_3}{V_3 \oplus \C}}}_{S_3} = \frac{1}{6} \left[ 1 \cdot 4 + 3 \cdot (1 \cdot 2 ) + 2 \cdot (1 \cdot 1) \right] = 2 \]
Which implies that $V_3 \oplus \C$ appears without multiplicity in the expansion of $\Ind{S_4}{S_3}{\C}$ as a sum of irreducible representations. However, since $\dim{\Ind{S_4}{S_3}{\C}} = [S_4 : S_3] \dim{\C} = 4$, by dimension counting,
\[ \Ind{S_4}{S_3}{\C} \cong V_3 \oplus \C \]
An identical argument shows that,
\[ \inner{\chi_{\Ind{S_4}{S_3}{\C(\varepsilon)}}}{\chi_{(V_3 \oplus \C) \otimes \varepsilon}}_{S_4} = \inner{\chi_{\C(\varepsilon)}}{\chi_{\Res{S_4}{S_3}{(V_3 \oplus \C) \otimes \varepsilon}}}_{S_3} = 2\]
since each term in the inner product is modified only by a multiplication by $-1$ for each $\varepsilon$ factor and thus an overall factor of $1$. Therefore, by dimension counting,
\[ \Ind{S_4}{S_3}{\C(\varepsilon)} \cong (V_3 \oplus \C) \otimes \varepsilon \cong (V_3 \otimes \varepsilon) \oplus \C(\varepsilon) \]  

\item Using Frobenius reciprocity,
\[ \inner{\chi_{\Ind{S_4}{S_3}{W_2}}}{V_3}_{S_4} = \inner{\chi_{{W_2}}}{\chi_{\Res{S_4}{S_3} {V_3}}}_{S_3} = 1\]
since we know that,
\[ \Res{S_4}{S_3}{V_3} \cong W_2 \oplus \C \]
and thus $W_2$ is an irreducible factor of $\Res{S_4}{S_3}{V_3}$ without multiplicity. Therefore, $V_3$ is an irreducible factor of $\Ind{S_4}{S_3}{W_2}$ without multiplicity. Furthermore,  
\[ \inner{\chi_{\Ind{S_4}{S_3}{W_2}}}{V_3 \otimes \varepsilon}_{S_4} = \inner{\chi_{{W_2}}}{\chi_{\Res{S_4}{S_3} {(V_3 \otimes \varepsilon)}}}_{S_3} = 1\]
since 
\[ \Res{S_4}{S_3}{(V_3 \otimes \varepsilon)} \cong W_2 \otimes \C(\varepsilon) \]
and thus $V_3 \otimes \varepsilon$ is also an irreducible factor of $\Ind{S_4}{S_3}{W_2}$ without multiplicity. 
\bigskip\\ 
Using more Frobenius reciprocity,
\[ \inner{\chi_{\Ind{S_4}{S_3}{W_2}}}{\C}_{S_4} = \inner{\chi_{{W_2}}}{\chi_{\Res{S_4}{S_3}{\C}}}_{S_3} = \inner{\chi_{{W_2}}}{\chi_{\C}}_{S_3} = 0\]
and likewise, 
\[ \inner{\chi_{\Ind{S_4}{S_3}{W_2}}}{\C(\varepsilon)}_{S_4} = \inner{\chi_{{W_2}}}{\chi_{\Res{S_4}{S_3}{\C(\varepsilon)}}}_{S_3} = \inner{\chi_{{W_2}}}{\chi_{\C(\varepsilon)}}_{S_3} = 0 \]
by $(a)$ and the fact that $W_2 \not\cong \C$ and $W_2 \not\cong \C(\varepsilon)$.
Therefore, $\C$ and $\C(\varepsilon)$ are not summands of $\Ind{S_4}{S_3}{W_2}$. However, by $\dim{\Ind{S_4}{S_3}{W_2}} = [S_4 : S_3] \dim{W_2} = 8$ so by dimension counting, the fact that no $1$-dimensional irreducible $S_4$-representation is a summand, and the fact that $V_3$ and $V_3 \otimes \varepsilon$ have multiplicity $1$,
\[ \Ind{S_4}{S_3}{W_2} \cong V_3 \oplus (V_3 \otimes \varepsilon) \oplus V\]
where $\dim{V} = 2$ since $\dim{V_3} = \dim{(V_3 \otimes \varepsilon)} = 3$. However, $S_4$ has a unique $2$-dimensional representation up to isomorphism so,
\[ \Ind{S_4}{S_3}{W_2} \cong V_3 \oplus (V_3 \otimes \varepsilon) \oplus V_2\]
\end{enumerate}

\section*{Problem 3.}

We already know that $V = \Ind{G}{H}{W}$ is irreducible and $ \Res{G}{H}{V} = W \oplus W_x$ if and only if $W \not\cong W_x$. It remains to prove that in the case that $W \cong W_x$ that the entire statement (ii) follows. \bigskip\\
Suppose that $W \cong W_x$ then by above $V =  \Res{G}{H}{W}$ is irreducible. Since $[G : H] = 2$ take coset representatives $H$ and $x H$. Because $H$ has index $2$ the subgroup $H$ is normal in $G$ so $z^{-1} g z \in H \iff g \in H$. Therefore, $i_z(g) \in H \iff g \in H$. Now we apply the formula for the character of the induced representation, if $g \notin H$
\[ \chi_{V}(g) = \frac{1}{H} \sum_{z^{-1} g z \in H} \chi_W(z^{-1} g z \in H) = 0 \]
Otherwise, 
\[ \chi_{V}(h) = \chi_{\Res{G}{H}{V}}(h) = \chi_W(h) + \chi_{W_x}(h) \] 
This implies that,
\[ \inner{\chi_V}{\chi_V}_G = \frac{\#(H)}{\#(G)} \inner{\chi_{\Res{G}{H}{V}}}{\chi_{\Res{G}{H}{V}}}_H = \frac{1}{2} \inner{\chi_W + \chi_{W_x}}{\chi_W + \chi_{W_x}} = \frac{1}{2} \cdot 4 = 2\]
where I have used the fact that $W \cong W_x$ are both are irreducible so each of the four inner products of characters is $1$. This implies that $V \cong V_1 \oplus V_2$ where $V_1$ and $V_2$ are irreducible. Because $V = \Ind{G}{H}{W}$ we can apply Frobenius reciprocity to conclude that,
\[ \inner{V}{\chi_{V_i}} = \inner{\Ind{G}{H}{W}}{\chi_{V_i}} = \inner{\chi_{W}}{\chi_{\Res{G}{H}{V_i}}} = 1 \]
which shows that,
\[ \Res{G}{H}{V_i} \cong W \implies \Ind{G}{H}{\Res{G}{H}{V_i}} \cong \Ind{G}{H}{W} \]  
Therefore, we have that,
\[ \Ind{G}{H}{\Res{G}{H}{V_1}} \cong \Ind{G}{H}{W} \cong \Ind{G}{H}{\Res{G}{H}{V_2}} \]
From class we have the formula,
\[ \Ind{G}{H}{\Res{G}{H}{V_1}} \cong V_1 \otimes \C[G/H] = V_1 \otimes (\C \oplus \C(\varepsilon)) = V_1 \oplus (V_1 \otimes \varepsilon) \]
because $G/H \cong \Z/ 2 \Z$ since it has index $2$. The same argument shows that,
\[ \Ind{G}{H}{\Res{G}{H}{V_1}} \cong V_2 \oplus (V_2 \otimes \varepsilon) \]
However, we know that these are isomorphic,
\[ V_1 \otimes (V_1 \otimes \varepsilon) \cong V_2 \otimes (V_2 \otimes \varepsilon) \]
But we have shown that $V$ has no multiplicity so $V_1 \not\cong V_2$ which implies that $V_2 \cong V_1 \otimes \varepsilon$ since it is the only other irreducible factor. This proves that $V \cong V_1 \oplus (V_1 \otimes \varepsilon)$ where $V_1$ is an irreducible $G$-representation and $V_1 \not\cong V_1 \otimes \varepsilon$ and that $W \cong \Res{G}{H}{V_1} \cong \Res{G}{H}{V_1 \otimes \varepsilon} \cong W$. If this is true then $V$ is not irreducible so $W \cong W_x$. Therefore, we have (ii) exactly when $W \cong W_x$ and (i) exactly when $W \not\cong W_x$. 


\end{document}