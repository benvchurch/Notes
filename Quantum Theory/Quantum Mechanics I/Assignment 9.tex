\documentclass[12pt]{extarticle}
\usepackage[utf8]{inputenc}
\usepackage[english]{babel}
\usepackage[a4paper, total={7.25in, 9.5in}]{geometry}
\usepackage{tikz-feynman}
\tikzfeynmanset{compat=1.0.0} 
\usepackage{subcaption}
\usepackage{float}
\floatplacement{figure}{H}
\usepackage{simpler-wick}
\usepackage{mathrsfs}  
\usepackage{dsfont}
\usepackage{relsize}
\usepackage{tikz-cd}
\DeclareMathAlphabet{\mathdutchcal}{U}{dutchcal}{m}{n}

\usepackage{cancel}



\newcommand{\field}{\hat{\Phi}}
\newcommand{\dfield}{\hat{\Phi}^\dagger}
 
\usepackage{amsthm, amssymb, amsmath, centernot}
\usepackage{slashed}
\newcommand{\notimplies}{%
  \mathrel{{\ooalign{\hidewidth$\not\phantom{=}$\hidewidth\cr$\implies$}}}}
 
\renewcommand\qedsymbol{$\square$}
\newcommand{\cont}{$\boxtimes$}
\newcommand{\divides}{\mid}
\newcommand{\ndivides}{\centernot \mid}

\newcommand{\Integers}{\mathbb{Z}}
\newcommand{\Natural}{\mathbb{N}}
\newcommand{\Complex}{\mathbb{C}}
\newcommand{\Zplus}{\mathbb{Z}^{+}}
\newcommand{\Primes}{\mathbb{P}}
\newcommand{\Q}{\mathbb{Q}}
\newcommand{\R}{\mathbb{R}}
\newcommand{\ball}[2]{B_{#1} \! \left(#2 \right)}
\newcommand{\Rplus}{\mathbb{R}^+}
\renewcommand{\Re}[1]{\mathrm{Re}\left[ #1 \right]}
\renewcommand{\Im}[1]{\mathrm{Im}\left[ #1 \right]}
\newcommand{\Op}{\mathcal{O}}

\newcommand{\invI}[2]{#1^{-1} \left( #2 \right)}
\newcommand{\End}[1]{\text{End}\left( A \right)}
\newcommand{\legsym}[2]{\left(\frac{#1}{#2} \right)}
\renewcommand{\mod}[3]{\: #1 \equiv #2 \: \mathrm{mod} \: #3 \:}
\newcommand{\nmod}[3]{\: #1 \centernot \equiv #2 \: mod \: #3 \:}
\newcommand{\ndiv}{\hspace{-4pt}\not \divides \hspace{2pt}}
\newcommand{\finfield}[1]{\mathbb{F}_{#1}}
\newcommand{\finunits}[1]{\mathbb{F}_{#1}^{\times}}
\newcommand{\ord}[1]{\mathrm{ord}\! \left(#1 \right)}
\newcommand{\quadfield}[1]{\Q \small(\sqrt{#1} \small)}
\newcommand{\vspan}[1]{\mathrm{span}\! \left\{#1 \right\}}
\newcommand{\galgroup}[1]{Gal \small(#1 \small)}
\newcommand{\bra}[1]{\left| #1 \right>}
\newcommand{\Oa}{O_\alpha}
\newcommand{\Od}{O_\alpha^{\dagger}}
\newcommand{\Oap}{O_{\alpha '}}
\newcommand{\Odp}{O_{\alpha '}^{\dagger}}
\newcommand{\im}[1]{\mathrm{im} \: #1}
\renewcommand{\ker}[1]{\mathrm{ker} \: #1}
\newcommand{\ket}[1]{\left| #1 \right>}
\renewcommand{\bra}[1]{\left< #1 \right|}
\newcommand{\inner}[2]{\left< #1 | #2 \right>}
\newcommand{\expect}[2]{\left< #1 \right| #2 \left| #1 \right>}
\renewcommand{\d}[1]{ \mathrm{d}#1 \:}
\newcommand{\dn}[2]{ \mathrm{d}^{#1} #2 \:}
\newcommand{\deriv}[2]{\frac{\d{#1}}{\d{#2}}}
\newcommand{\nderiv}[3]{\frac{\dn{#1}{#2}}{\d{#3^{#1}}}}
\newcommand{\pderiv}[2]{\frac{\partial{#1}}{\partial{#2}}}
\newcommand{\fderiv}[2]{\frac{\delta #1}{\delta #2}}
\newcommand{\parsq}[2]{\frac{\partial^2{#1}}{\partial{#2}^2}}
\newcommand{\topo}{\mathcal{T}}
\newcommand{\base}{\mathcal{B}}
\renewcommand{\bf}[1]{\mathbf{#1}}
\renewcommand{\a}{\hat{a}}
\newcommand{\adag}{\hat{a}^\dagger}
\renewcommand{\b}{\hat{b}}
\newcommand{\bdag}{\hat{b}^\dagger}
\renewcommand{\c}{\hat{c}}
\newcommand{\cdag}{\hat{c}^\dagger}
\newcommand{\hamilt}{\hat{H}}
\renewcommand{\L}{\hat{L}}
\newcommand{\Lz}{\hat{L}_z}
\newcommand{\Lsquared}{\hat{L}^2}
\renewcommand{\S}{\hat{S}}
\renewcommand{\empty}{\varnothing}
\newcommand{\J}{\hat{J}}
\newcommand{\lagrange}{\mathcal{L}}
\newcommand{\dfourx}{\mathrm{d}^4x}
\newcommand{\meson}{\phi}
\newcommand{\dpsi}{\psi^\dagger}
\newcommand{\ipic}{\mathrm{int}}
\newcommand{\tr}[1]{\mathrm{tr} \left( #1 \right)}
\newcommand{\C}{\mathbb{C}}
\newcommand{\CP}[1]{\mathbb{CP}^{#1}}
\newcommand{\Vol}[1]{\mathrm{Vol}\left(#1\right)}

\newcommand{\Tr}[1]{\mathrm{Tr}\left( #1 \right)}
\newcommand{\Charge}{\hat{\mathbf{C}}}
\newcommand{\Parity}{\hat{\mathbf{P}}}
\newcommand{\Time}{\hat{\mathbf{T}}}
\newcommand{\Torder}[1]{\mathbf{T}\left[ #1 \right]}
\newcommand{\Norder}[1]{\mathbf{N}\left[ #1 \right]}
\newcommand{\Znorm}{\mathcal{Z}}
\newcommand{\EV}[1]{\left< #1 \right>}
\newcommand{\interact}{\mathrm{int}}
\newcommand{\covD}{\mathcal{D}}
\newcommand{\conj}[1]{\overline{#1}}

\newcommand{\SO}[2]{\mathrm{SO}(#1, #2)}
\newcommand{\SU}[2]{\mathrm{SU}(#1, #2)}

\newcommand{\anticom}[2]{\left\{ #1 , #2 \right\}}


\newcommand{\pathd}[1]{\! \mathdutchcal{D} #1 \:}

\renewcommand{\theenumi}{(\alph{enumi})}


\renewcommand{\theenumi}{(\alph{enumi})}

\newcommand{\atitle}[1]{\title{% 
	\large \textbf{Physics GR8048 Quantum Field Theory II
	\\ Assignment \# #1} \vspace{-2ex}}
\author{Benjamin Church }
\maketitle}

\newcommand{\atitleIII}[1]{\title{% 
	\large \textbf{Physics GR8049 Quantum Field Theory III
	\\ Assignment \# #1} \vspace{-2ex}}
\author{Benjamin Church }
\maketitle}

\theoremstyle{definition}
\newtheorem{theorem}{Theorem}[section]
\newtheorem{definition}{definition}[section]
\newtheorem{lemma}[theorem]{Lemma}
\newtheorem{proposition}[theorem]{Proposition}
\newtheorem{corollary}[theorem]{Corollary}
\newtheorem{example}[theorem]{Example}
\newtheorem{remark}[theorem]{Remark}



\begin{document}
\atitle{9}
 
\section*{Problem 28.}

\subsection*{(a)}

We identify the state with $S_x = + \frac{\hbar}{2}$ as the general state with spin along $\hat{n}$ parametrized by the angles $\theta = \frac{\pi}{2}$ and $\phi = 0$. Thus, $\ket{+ \hat{x}} = \frac{1}{\sqrt{2}} \ket{+ \hat{z}} + \frac{1}{\sqrt{2}} \ket{- \hat{z}}$. The wavefunction multiplying the $+\hat{z}$ is directed upwards and the wavefunction multipying the $-\hat{z}$ is directed downwards. Therefore, $P(+\hat{z}) = \left|\frac{1}{\sqrt{2}}\right|^2 = \frac{1}{2}$ and $P(-\hat{z}) = \left|\frac{1}{\sqrt{2}}\right|^2 = \frac{1}{2}$. 

\subsection*{(b)}

In the upper region, an inhomogeneous magnetic field $\vec{B}(y(t))$ in the $z$-direction is introduced. In the upper region, the Hamiltonian acting on the spin is $\hamilt_u = - \gamma \S_z B_z(y(t))$. Although the Hamiltonian is time dependent, it commutes with itself at all times because $\vec{B}$ has a fixed direction. Therefore, a spin in this region is evolved by the operator, 
\[\hat{U}_u = e^{i \frac{\gamma}{\hbar} \S_z \int B_z(y(t)) \d{t}} = e^{i \frac{\gamma}{\hbar} \S_z I_B} \] 
If we send a election in the $\ket{+ \hat{x}}$ state through the apparatus, the $\ket{-\hat{z}}$ component goes through without any alteration. However, the $\ket{+ \hat{z}}$ component is multiplied by $\hat{U}_u$. Therefore, the final state is,
\[\ket{\psi} = \tfrac{1}{\sqrt{2}} e^{i \frac{\gamma}{\hbar} \S_z I_B} \ket{+ \hat{z}} + \tfrac{1}{\sqrt{2}} \ket{- \hat{z}} = \tfrac{1}{\sqrt{2}} e^{i \frac{\gamma}{2} I_B} \ket{+ \hat{z}} + \tfrac{1}{\sqrt{2}} \ket{- \hat{z}}\]
Now, we calculate the polatization of this state,
\begin{align*}
\left< \S_x \right> & = \tfrac{1}{2} \bra{\psi} \S_{+} + \S_{-} \ket{\psi} = \tfrac{1}{4} (  \bra{+ \hat{z}} e^{-i \frac{\gamma}{2} I_B} + \bra{- \hat{z}}) \left[ \S_{+} + \S_{-} \right] (  e^{i \frac{\gamma}{2} I_B} \ket{+ \hat{z}}  + \ket{- \hat{z}}) \\ 
& = \tfrac{\hbar}{4} (  \bra{+ \hat{z}} e^{-i \frac{\gamma}{2} I_B} + \bra{- \hat{z}}) (  e^{i \frac{\gamma}{2} I_B} \ket{- \hat{z}}  + \ket{+ \hat{z}}) = \tfrac{\hbar}{4} \left( e^{i \frac{\gamma}{2} I_B} + e^{-i \frac{\gamma}{2} I_B} \right) = \tfrac{\hbar}{2} \cos{ \tfrac{\gamma}{2} I_B} \\
\left< \S_y \right> & = \tfrac{1}{2i} \bra{\psi} \S_{+} - \S_{-} \ket{\psi} = \tfrac{1}{4i} (  \bra{+ \hat{z}} e^{-i \frac{\gamma}{2} I_B} + \bra{- \hat{z}}) \left[ \S_{+} - \S_{-} \right] (  e^{i \frac{\gamma}{2} I_B} \ket{+ \hat{z}}  + \ket{- \hat{z}}) \\ 
& = \tfrac{\hbar}{4i} (  \bra{+ \hat{z}} e^{-i \frac{\gamma}{2} I_B} + \bra{- \hat{z}}) ( - e^{i \frac{\gamma}{2} I_B} \ket{- \hat{z}}  + \ket{+ \hat{z}}) = \tfrac{\hbar}{4i} \left( e^{-i \frac{\gamma}{2} I_B} - e^{i \frac{\gamma}{2} I_B} \right) = -\tfrac{\hbar}{2} \sin{ \tfrac{\gamma}{2} I_B} \\
\left< \S_z \right> & = \bra{\psi} \S_{z} \ket{\psi} = \tfrac{1}{2} (\bra{+ \hat{z}} e^{-i \frac{\gamma}{2} I_B} + \bra{- \hat{z}}) \S_{z} (e^{i \frac{\gamma}{2} I_B} \ket{+ \hat{z}}  + \ket{- \hat{z}}) \\ 
& = \tfrac{1}{2} (\bra{+ \hat{z}} e^{-i \frac{\gamma}{2} I_B} + \bra{- \hat{z}}) \tfrac{\hbar}{2} (e^{i \frac{\gamma}{2} I_B} \ket{+ \hat{z}}  - \ket{- \hat{z}}) \tfrac{\hbar}{4} \left( e^{i \frac{\gamma}{2} I_B} e^{-i \frac{\gamma}{2} I_B}  - 1 \right) = 0 \\
\end{align*}
Thus, the polarization vector is,
\[\vec{n} = \frac{\hbar}{2} \left(\cos{\tfrac{\gamma}{2} I_B}, -\sin{\tfrac{\gamma}{2} I_B}, 0 \right)\]

\subsection*{(c)}

Suppose now the inhomogeneous magnetic field $\vec{B}(y(t))$ in the upper region points in the $y$-direction. The Hamiltonian acting on the spin is $\hamilt_u = - \gamma \S_y B_y(y(t))$. Although the Hamiltonian is time dependent, it commutes with itself at all times because $\vec{B}$ has a fixed direction. Therefore, a spin in this region is evolved by the operator, 
\[\hat{U}_u = e^{i \frac{\gamma}{\hbar} \S_y \int B_y(y(t)) \d{t}} = e^{i \frac{\gamma}{\hbar} \S_z I_B} \] 
If we send a election in the $\ket{+ \hat{x}}$ state through the apparatus, the $\ket{-\hat{z}}$ component goes through without any alteration. However, the $\ket{+ \hat{z}}$ component is multiplied by $\hat{U}_u$. Therefore, the upper state is,
\[\ket{\psi_u} = \tfrac{1}{\sqrt{2}} e^{i \frac{\gamma}{\hbar} \S_z I_B} \ket{+ \hat{z}}  = \tfrac{1}{\sqrt{2}} (I \cos{\tfrac{\gamma}{2} I_B} + i \sigma_y \sin{\tfrac{\gamma}{2} I_B}) \ket{+ \hat{z}} = \tfrac{1}{\sqrt{2}} \cos{\tfrac{\gamma}{2} I_B} \ket{+ \hat{z}} - \tfrac{1}{\sqrt{2}} \sin{\tfrac{\gamma}{2} I_B} \ket{- \hat{z}} \]
where $\sigma_y = \tfrac{1}{i \hbar} \left[\S_{+} - \S_{-} \right]$ and thus $\sigma_y \ket{+ \hat{z}} = - \frac{1}{i} \ket{- \hat{z}} = i \ket{- \hat{z}}$. Only the spin up state is deflected back down to recombine with the lower state. Therefore, the recombined state is,
\[\ket{\psi} = \tfrac{1}{\sqrt{2}} \cos{\tfrac{\gamma}{2} I_B} \ket{+\hat{z}} + \tfrac{1}{\sqrt{2}} \ket{-\hat{z}} \] 
This state has reduced total probability due to part of the wavefunction not recombinding. The norm of the state gives the probability for an electron to be found in region C. This is,
\[P = \inner{\psi}{\psi} = \tfrac{1}{2}(1 + \cos^2{\tfrac{\gamma}{2}I_B})\]  

\subsection*{(d)}

This is not a normalized spin state because we have lost some probability to the upper magnet. To find the conditional probability of an electron in region C having a given spin, we must divide by the probability for an electron to reach this point which is, $\frac{1}{2}(1 + \cos^2{\tfrac{\gamma}{2}I_B})$.     
Now, we calculate the polatization of this state,
\begin{align*}
\left< \S_x \right> & = \frac{\hbar}{4} \frac{1}{\frac{1}{2}(1 + \cos^2{\tfrac{\gamma}{2}I_B})}
\begin{pmatrix}
\cos{\tfrac{\gamma}{2}} & 1
\end{pmatrix} 
\begin{pmatrix}
0 & 1 \\
1 & 0
\end{pmatrix} 
\begin{pmatrix}
\cos{\tfrac{\gamma}{2}} \\
1
\end{pmatrix} 
= \frac{\hbar}{4} \frac{2 \cos{\tfrac{\gamma}{2} I_B}}{\frac{1}{2}(1 + \cos^2{\tfrac{\gamma}{2}I_B})} = \frac{\hbar \cos{\tfrac{\gamma}{2} I_B}}{1 + \cos^2{\tfrac{\gamma}{2}I_B}} \\
\left< \S_y \right> & = \frac{\hbar}{4} \frac{1}{\frac{1}{2}(1 + \cos^2{\tfrac{\gamma}{2}I_B})}
\begin{pmatrix}
\cos{\tfrac{\gamma}{2}} & 1
\end{pmatrix} 
\begin{pmatrix}
0 & -i \\
i & 0
\end{pmatrix} 
\begin{pmatrix}
\cos{\tfrac{\gamma}{2}} \\
1
\end{pmatrix} 
= 0 \\
\left< \S_y \right> & = \frac{\hbar}{4} \frac{1}{\frac{1}{2}(1 + \cos^2{\tfrac{\gamma}{2}I_B})}
\begin{pmatrix}
\cos{\tfrac{\gamma}{2}} & 1
\end{pmatrix} 
\begin{pmatrix}
1 & 0 \\
0 & -1
\end{pmatrix} 
\begin{pmatrix}
\cos{\tfrac{\gamma}{2}} \\
1
\end{pmatrix} 
= \frac{\hbar}{4} \frac{\cos^2{\tfrac{\gamma}{2} I_B} - 1}{\frac{1}{2}(1 + \cos^2{\tfrac{\gamma}{2}I_B})} = - \frac{\hbar}{2} \frac{\sin^2{\tfrac{\gamma}{2} I_B}}{1 + \cos^2{\tfrac{\gamma}{2}I_B}}
\end{align*}
Thus, the polarization vector is, 
\[\vec{n} = \frac{\hbar}{2} \left(\frac{2 \cos{\tfrac{\gamma}{2} I_B}}{1 + \cos^2{\tfrac{\gamma}{2}I_B}}, 0, -\frac{\sin^2{\tfrac{\gamma}{2} I_B}}{1 + \cos^2{\tfrac{\gamma}{2}I_B}} \right)\]

\subsection*{(e)}

The light source entangles the state of the electron with the state of the photophraphic plate. Let $\ket{*}$ be the state representing an expossed grain and $\ket{0}$ the unexposed state. These states correspond to distinct values of an observable and are therefore orthogonal. Then, the state $\ket{\psi} = \frac{1}{\sqrt{2}} \ket{+\hat{z}} + \frac{1}{\sqrt{2}} \ket{-\hat{z}}$ splits into the up state going through the upper path and the down state going through the lower path. Thus, the state is transformed to,
\[\ket{\psi} = \tfrac{1}{\sqrt{2}} \hat{U}_u \ket{+\hat{z}} \otimes \ket{0} + \tfrac{1}{\sqrt{2}} \ket{-\hat{z}} \otimes \ket{*} \]   
In the case with $\vec{B}$ in the $z$-direction, $\hat{U}_u$ simply acts to rotate the up state by a phase $e^{-i \tfrac{\gamma}{2} I_B}$. Therefore, we can calculate the polarization vector,
\begin{align*}
\left< \S_x \right> & = \bra{\psi} \S_x \ket{\psi} = \tfrac{1}{2} \bra{+ \hat{z}} e^{i \tfrac{\gamma}{2} I_B} \S_x e^{-i \tfrac{\gamma}{2} I_B} \ket{+\hat{z}} \inner{0}{0} + \tfrac{1}{2} \bra{+ \hat{z}} \S_x \ket{+\hat{z}} \inner{*}{*} = 0\\
\left< \S_y \right> & = \bra{\psi} \S_y \ket{\psi} = \tfrac{1}{2} \bra{+ \hat{z}} e^{i \tfrac{\gamma}{2} I_B} \S_y e^{-i \tfrac{\gamma}{2} I_B} \ket{+\hat{z}} \inner{0}{0} + \tfrac{1}{2} \bra{+ \hat{z}} \S_y \ket{+\hat{z}} \inner{*}{*} = 0\\
\left< \S_z \right> & = \bra{\psi} \S_z \ket{\psi} = \tfrac{1}{2} \bra{+ \hat{z}} e^{i \tfrac{\gamma}{2} I_B} \S_z e^{-i \tfrac{\gamma}{2} I_B} \ket{+\hat{z}} \inner{0}{0} + \tfrac{1}{2} \bra{+ \hat{z}} \S_z \ket{+\hat{z}} \inner{*}{*} = \tfrac{\hbar}{4}(1 - 1) = 0  \\
\end{align*} 
Thus, the polarization vector is, $\vec{n} = \vec{0}$. \\\\
In the case with $\vec{B}$ in the $y$-direction, $\hat{U}_u$ rotates the state about the $y$-axis. However, we must remember to only include the component that rotates into an up state so that is it deflected downwards by the Stern-Gerlach apparatus.
\[\ket{\psi_u} = \tfrac{1}{\sqrt{2}} e^{i \frac{\gamma}{\hbar} \S_z I_B} \ket{+ \hat{z}}  \otimes \ket{0} = \tfrac{1}{\sqrt{2}} \cos{\tfrac{\gamma}{2} I_B} \ket{+ \hat{z}}  \otimes \ket{0} - \tfrac{1}{\sqrt{2}} \sin{\tfrac{\gamma}{2} I_B} \ket{- \hat{z}}  \otimes \ket{0} \]
Therefore, the entire state in region $C$ is,
\[\ket{\psi} = \tfrac{1}{\sqrt{2}} \cos{\tfrac{\gamma}{2} I_B} \ket{+\hat{z}} \otimes \ket{0} + \tfrac{1}{\sqrt{2}} \ket{-\hat{z}} \otimes \ket{*}\] 
As before, this is not a normalized spin state because we have lost some probability to the upper magnet. Thus, we must normalize by the probability for an electron to reach this point which is, $\frac{1}{2}(1 + \cos^2{\tfrac{\gamma}{2}I_B})$.     
We now calculate the polarization vector,
\begin{align*}
\left< \S_x \right> & = \bra{\psi} \S_x \ket{\psi} = \frac{1}{\frac{1}{2}(1 + \cos^2{\tfrac{\gamma}{2}I_B})} \left[ \tfrac{1}{2}\cos^2{\tfrac{\gamma}{2} I_B} \bra{+ \hat{z}} \S_x \ket{+\hat{z}} \inner{0}{0} + \tfrac{1}{2} \bra{+ \hat{z}} \S_x \ket{+\hat{z}} \inner{*}{*} \right] = 0\\
\left< \S_y \right> & = \bra{\psi} \S_y \ket{\psi} = \frac{1}{\frac{1}{2}(1 + \cos^2{\tfrac{\gamma}{2}I_B})} \left[ \tfrac{1}{2}\cos^2{\tfrac{\gamma}{2} I_B} \bra{+ \hat{z}} \S_y \ket{+\hat{z}} \inner{0}{0} + \tfrac{1}{2} \bra{+ \hat{z}} \S_y \ket{+\hat{z}} \inner{*}{*} \right] = 0\\
\left< \S_z \right> & = \bra{\psi} \S_z \ket{\psi} = \frac{1}{\frac{1}{2}(1 + \cos^2{\tfrac{\gamma}{2}I_B})} \left[ \tfrac{1}{2}\cos^2{\tfrac{\gamma}{2} I_B} \bra{+ \hat{z}} \S_z \ket{+\hat{z}} \inner{0}{0} + \tfrac{1}{2} \bra{+ \hat{z}} \S_z \ket{+\hat{z}} \inner{*}{*} \right] \\
& = \frac{\hbar}{4} \frac{\cos^2{\tfrac{\gamma}{2} I_B} - 1}{\frac{1}{2}(1 + \cos^2{\tfrac{\gamma}{2}I_B})} = -\frac{\hbar}{2} \frac{\sin^2{\tfrac{\gamma}{2} I_B}}{1 + \cos^2{\tfrac{\gamma}{2}I_B}}
\end{align*} 
Thus, the polarization vector is, 
\[\vec{n} = \frac{\hbar}{2} \left(0, 0, -\frac{\sin^2{\tfrac{\gamma}{2}I_B}}{1 + \cos^2{\tfrac{\gamma}{2}I_B}}\right)\] 
\section*{Problem 29.}

\subsection*{(a)}

The Hamiltonian of the combined system is,
\[\hamilt = \frac{\vec{p}_e^{\, 2}}{2m_e} + \frac{\vec{p}_p^{\,2}}{2m_p} + V(\vec{r}_e - \vec{r}_p) = \frac{\vec{p}_e^{\, 2}}{2m_e} + \frac{\vec{p}_p^{\,2}}{2m_p} - \frac{e^2}{|\vec{r}_e - \vec{r}_p|}\]

\subsection*{(b)}

We can rewrite the Hamiltonian in terms of the quantities,
\[\vec{r} = \vec{r}_e - \vec{r}_p \quad \text{and} \quad \vec{R} = \frac{1}{m_e + m_p} \left[ m_e \vec{r}_e + m_p \vec{r}_p \right] \]
We can impliment this coordinate transformation via the type two canonical generating function,
\[F_2(q, P) = f_j(q) P_j\] 
which generates the point transformation,
\[Q_i = \pderiv{F_2}{P_i} = f_j(q) \quad \text{and} \quad p_i = \pderiv{F_2}{q_i} = \pderiv{f_j}{q_i} P_j\]
where the lowercase variables are the original coordinates and uppercase variables are the new coordinates. Let,
\[f_1(\vec{r}_e, \vec{r}_p) = \vec{r}_e - \vec{r}_p \quad \text{and} \quad f_2(\vec{r}_e, \vec{r}_p) = \frac{1}{m_e + m_p} \left[ m_e \vec{r}_e + m_p \vec{r}_p \right]\]
Therefore, 
\begin{align*}
\vec{p}_e &= \pderiv{f_1}{\vec{r}_e} \cdot \vec{p}_r + \pderiv{f_2}{\vec{r}_e} \cdot \vec{p}_R = \vec{p}_r + \frac{m_e}{m_e + m_p} \vec{p}_R  \\
\vec{p}_p &= \pderiv{f_1}{\vec{r}_p} \cdot \vec{p}_r + \pderiv{f_2}{\vec{r}_p} \cdot \vec{p}_R = - \vec{p}_r + \frac{m_p}{m_e + m_p} \vec{p}_R 
\end{align*}
Therefore,
\begin{align*}
\hamilt & = \frac{1}{2m_e} \left[ \vec{p}_r + \frac{m_e}{m_e + m_p} \vec{p}_R  \right]^2 + \frac{1}{2m_p} \left[- \vec{p}_r + \frac{m_p}{m_e + m_p} \vec{p}_R \right]^2 - \frac{e^2}{r} \\
& =  \left[ \frac{\vec{p}_r^{\,2}}{2m_e} + \frac{\vec{p}_r \cdot \vec{p}_R}{m_e + m_p} +  \frac{1}{2m_e} \left(\frac{m_e}{m_e + m_p} \right)^2 \vec{p}_R^{\,2} \right] + \left[ \frac{\vec{p}_r^{\,2}}{2m_p} - \frac{\vec{p}_r \cdot \vec{p}_R}{m_e + m_p} +  \frac{1}{2m_p} \left(\frac{m_p}{m_e + m_p} \right)^2 \vec{p}_R^{\,2} \right] - \frac{e^2}{r} \\
& = \frac{\vec{p}_r^{\,2}}{2m_e} + \frac{\vec{p}_r^{\,2}}{2m_p} + \frac{1}{2} \left( \frac{m_e}{(m_e + m_p)^2} + \frac{m_p}{(m_e + m_p)^2} \right)\vec{p}_R^{\,2} - \frac{e^2}{r} = \frac{\vec{p}_r^{\,2}}{2 \mu} + \frac{\vec{p}_R^{\,2}}{2 M} - \frac{e^2}{r}
\end{align*}
where $\frac{1}{\mu} = \frac{1}{m_e} + \frac{1}{m_p}$ and $M = m_e + m_p$. We are justified in commuting momentum variables because all depend only on canonical momenta operators which commute with eachother and no positions operators which do not. We derived this canonical transformation classically but the canonical transformation preserves poisson brackets and thus we hope that it preserves canonical commutators. Therefore using the fact that proton and electron operators commute with eachother, we check the commutation relations of the new operators. Solving for the new momenta,
\begin{align*}
\vec{p}_R & = \vec{p}_e + \vec{p}_p \\
\vec{p}_r & = \frac{m_p}{m_e + m_p} \vec{p}_e  - \frac{m_e}{m_e + m_p} \vec{p}_p
\end{align*}
\begin{align*}
[(r)_i, (p_r)_j] &= \left[(r_e)_i - (r_p)_i, \: \frac{m_p}{m_e + m_p} (p_e)_i - \frac{m_e}{m_e + m_p} (p_p)_i \right] \\ & =  \frac{m_p}{m_e + m_p} [(r_e)_i, (p_e)_j] + \frac{m_e}{m_e + m_p} [(r_p)_i, (p_p)_j] = \frac{m_p}{m_e + m_p} i \hbar \delta_{ij} + \frac{m_e}{m_e + m_p} i \hbar \delta_{ij} = i \hbar \delta_{ij} \\
[(R)_i, (p_r)_j] &= \left[\frac{1}{m_e + m_p} \big( m_e (r_e)_i + m_p (r_p)_i \big), \: \frac{m_p}{m_e + m_p} (p_e)_i - \frac{m_e}{m_e + m_p} (p_p)_i \right] \\ & =  \frac{m_e m_p}{(m_e + m_p)^2} [(r_e)_i, (p_e)_j] - \frac{m_p m_e}{(m_e + m_p)^2} [(r_p)_i, (p_p)_j] = \frac{m_e m_p}{(m_e + m_p)^2} i \hbar \delta_{ij} - \frac{m_p m_e}{(m_e + m_p)^2} i \hbar \delta_{ij} = 0 \\
[(r)_i, (p_R)_j] &= \left[(r_e)_i - (r_p)_i, \: (p_e)_i + (p_p)_i \right] =  [(r_e)_i, (p_e)_j] - [(r_p)_i, (p_p)_j] = i \hbar \delta_{ij} - i \hbar \delta_{ij} = 0 \\
[(R)_i, (p_R)_j] &= \left[\frac{1}{m_e + m_p} \big( m_e (r_e)_i + m_p (r_p)_i \big), \: (p_e)_i + (p_p)_i \right] \\ & = \frac{m_e}{m_e + m_p} [(r_e)_i, (p_e)_j] + \frac{m_p}{m_e + m_p} [(r_p)_i, (p_p)_j] = \frac{m_e}{m_e + m_p} i \hbar \delta_{ij} + \frac{m_p}{m_e + m_p} i \hbar \delta_{ij} = i \hbar \delta_{ij} 
\end{align*} 

Thus,
\[\hamilt = \frac{\vec{p}_R^{\,2}}{2 M} + \frac{\vec{p}_r^{\, 2}}{2\mu} - \frac{e^2}{r}\]
and thus, the time independent Schrodinger Equation can be written as,
\[ E \psi(\vec{r}, \vec{R}) = - \frac{\hbar^2}{2 M} \nabla_R^2 \psi(\vec{r}, \vec{R}) - \frac{\hbar^2}{2 \mu} \nabla_r^2 \psi(\vec{r}, \vec{R}) - \frac{e^2}{r} \psi(\vec{r}, \vec{R})\]
\subsection*{(c)}

This Hamiltonian is seperable in the new variables. Write, $\psi(\vec{r}, \vec{R}) = \psi_r(\vec{r}) \psi_R(\vec{R})$. Let $\psi_R(\vec{R})$ be a momentum eigenstate such that $\vec{p}_R \: \psi_R(\vec{R}) = \vec{p} \: \psi_R(\vec{R})$ where $\vec{p}$ is the eigenvalue of the total momentum of the Hydrogen atom. Now, plugging in, 

\begin{align*}
 E \psi_r(\vec{r}) \psi_R(\vec{R}) & = \frac{\vec{p}_R^{\,2}}{2 M} \psi_r(\vec{r}) \psi_R(\vec{R}) + \frac{\vec{p}_r^{\, 2}}{2 \mu} \psi_r(\vec{r}) \psi_R(\vec{R}) - \frac{e^2}{r} \psi_r(\vec{r}) \psi_R(\vec{R}) \\ 
& = \frac{\vec{p}^{\,2}}{2 M} \: \psi_r(\vec{r}) \psi_R(\vec{R}) + \left[ \frac{\hbar^2}{2 \mu} \nabla_r^2 \psi_r(\vec{r}) - \frac{e^2}{r} \psi_r(\vec{r}) \right] \psi_R(\vec{R}) \\ 
\end{align*} 

Therefore,
\[\left(E - \frac{\vec{p}^{\,2}}{2 M} \right) \psi_r(\vec{r}) =  - \frac{\hbar^2}{2 \mu} \nabla_r^2 \psi_r(\vec{r}) - \frac{e^2}{r} \psi_r(\vec{r})  \]
which corresponds the Schrodinger equation for the single particle fixed potential case with mass $\mu = \frac{m_e m_p}{m_e + m_p}$ and energy eigenvalues shifted by a constant which depends only on the eigenvalues of the $\vec{R}$ coordinates. Therefore, the specturm is,
\[E_n = - \frac{\mu e^4}{2 \hbar^2} \frac{1}{n^2} + \frac{\vec{p}^{\,2}}{2 M} = - \frac{m_e m_p e^4}{2 (m_e + m_p) \hbar^2} \frac{1}{n^2} + \frac{\vec{p}^{\,2}}{2 M} = - \frac{m_e e^4}{2 \hbar^2} \frac{1}{n^2} \frac{1}{1 + \frac{m_e}{m_p}} + \frac{\vec{p}^{\,2}}{2 M} \]

\section*{Problem 30.}

We consider the correction to the electric potential given by the finite size of the proton. Suppose the proton is a sphere with constant electric charge density with radius $r_0$. Then the electric field generated by the proton is,
\[E = \frac{4 \pi Q_{enc}}{4 \pi r^2} = \frac{Q_{enc}}{r^2} = \frac{e}{r^2}
\begin{cases}
\left(\frac{r}{r_0} \right)^3 & r < r_0 \\
1 & r > r_0
\end{cases}\]
because the enclosed charge is proportional to the enclosed volume. Using the charge of the electron, $-e$, outside the sphere, the potential is given by,
\[ V(r) - V(\infty) = - \int_{\infty}^{r} -e E \: \d{r} = e^2 \int_{\infty}^r \frac{1}{r^2} \: \d{r} = - \frac{e^2}{r}\]
and setting $V(\infty) = 0$ we get, for $r \ge r_0$,
\[V(r) = - \frac{e^2}{r}\]
Likewise, the potential energy inside the sphere is given by,
\[V(r) - V(r_0) = - \int_{r_0}^{r} -e E \: \d{r} = e^2 \int_{r_0}^r \frac{r}{r_0^3} \: \d{r} = \frac{e^2}{2r_0} \left[\frac{r^2}{r_0^2} - 1 \right] \]
Therefore, inside the sphere, 
\[ V(r) = \frac{e^2}{2r_0} \left[\frac{r^2}{r_0^2} - 1 \right] + V(r_0) = \frac{e^2}{2r_0} \left[\frac{r^2}{r_0^2} - 1 \right]  - \frac{e^2}{r_0} = \frac{e^2}{2r_0} \left[\frac{r^2}{r_0^2} - 3 \right] \]
Therefore,
\[V(r) = - \frac{e^2}{r_0}
\begin{cases}
\frac{1}{2} \left[3 - \frac{r^2}{r_0^2} \right] & r \le r_0 \\
\frac{r_0}{r} & r \ge r_0
\end{cases}\]
Now, we write the wavefunction as,
\[ \psi_{n \ell m}(r, \theta, \phi) = \left(\tfrac{1}{a_0} \right)^{\frac{3}{2}} R_{n \ell}\left(\tfrac{r}{a_0} \right) Y^{\ell}_m (\theta, \phi) \]
Define $\rho = \frac{r}{a_0}$ and $u = \frac{r}{r_0}$. Then,
\[ \psi_{n \ell m}(r, \theta, \phi) = \left(\tfrac{1}{a_0} \right)^{\frac{3}{2}} R_{n \ell}(\rho) Y^{\ell}_m (\theta, \phi)\]
where the radial function is,
\[R_{n \ell}(\rho) = \sqrt{\left(\frac{2}{n}\right)^3 \frac{(n - \ell - 1)!}{2n (n + \ell)!}} \exp{\left[-\frac{\rho}{n}\right]} \: \left(\frac{2 \rho}{n} \right)^\ell \: L^{2\ell + 1}_{n - \ell - 1}\left(\tfrac{2 \rho}{n} \right) \]
where $L^{2\ell + 1}_{n - \ell - 1}\left(\tfrac{2 \rho}{n} \right)$ are the associated Laguerre polynomials given by the generating function,
\[\sum_{n = 0}^{\infty} t^n L^{\alpha}_{n} \left( x \right) = \frac{1}{(1 - t)^{1 + \alpha}} e^{-\frac{tx}{1 - t}} \]

\subsection*{(a)}
The correction to the Hamiltonian is the corrected potential minus the old potential,
\[ \Delta \hamilt = V(r) - V_0(r) = - \frac{e^2}{r_0}
\begin{cases}
\frac{1}{2} \left[3 - \frac{r^2}{r_0^2} \right] - \frac{r_0}{r} & r \le r_0 \\
0 & r \ge r_0
\end{cases}\]
The correction to the energy from first-order perturbation theory is,
\begin{align*}
\Delta E_{n \ell m} & = \bra{\psi_{n \ell m}} \Delta \hamilt \ket{ \psi_{n \ell m}} = \int_{0}^{\infty} r^2 \d{r} \int_{0}^{\pi} \sin{\theta} \d{\theta} \int_{0}^{2 \pi} \d{\phi} \: \psi_{n \ell m}^{*}(r, \theta, \phi) \: \Delta \hamilt \: \psi_{n \ell m}^{*}(r, \theta, \phi) \\
& = \frac{1}{a_0^3} \int_{0}^{\infty} r^2  R_{n \ell}(\rho) \: \Delta V(r) \: R_{n \ell}(\rho)  \: \d{r} \int_{0}^{\pi} \sin{\theta} \d{\theta} \int_{0}^{2 \pi} \d{\phi} \: Y^{\ell}_m (\theta, \phi)^{*} Y^{\ell}_m (\theta, \phi) \\
& = \int_{0}^{\infty} \rho^2  R_{n \ell}(\rho) \: \Delta V(\rho) \: R_{n \ell}(\rho)  \: \d{\rho} = \frac{e^2}{r_0} \int_{0}^{\frac{r_0}{a_0}} \rho^2  R_{n \ell}(\rho) \: \left( \frac{r_0}{r} - \frac{1}{2} \left[3 - \frac{r^2}{r_0^2} \right] \right) \: R_{n \ell}(\rho)  \: \d{\rho} \\ 
& = \frac{e^2}{r_0} \int_{0}^{u_0} \rho^2  R_{n \ell}(\rho) \: \left( \frac{u_0}{\rho} - \frac{1}{2} \left[3 - \frac{\rho^2}{u_0^2} \right] \right) \: R_{n \ell}(\rho)  \: \d{\rho}
\end{align*}
where $u_0 = \frac{r_0}{a_0}$. Now, we Taylor expand the radial function,
\[ R_{n \ell}(\rho) =  R_{n \ell}(0) +  R_{n \ell}'(0) \rho + O(\rho^2) \quad \text{and thus} \quad R_{n \ell}(\rho)^2 = R_{n \ell}(0)^2 +  2 R_{n \ell}(0) R_{n \ell}'(0) \rho + O(\rho^2)\]
Therefore, 
\begin{align*}
\Delta E_{n \ell m} & = \frac{e^2}{r_0} \int_{0}^{u_0} \rho^2  \left( \frac{u_0}{\rho} - \frac{1}{2} \left[3 - \frac{\rho^2}{u_0^2} \right] \right) \: R_{n \ell}(\rho)^2  \: \d{\rho} \\
& = \frac{e^2}{r_0} \int_{0}^{u_0} \left( u_0 \rho - \frac{1}{2} \left[3 \rho^2 - \frac{\rho^4}{u_0^2} \right] \right) \: R_{n \ell}(\rho)^2  \: \d{\rho} \\
& = \frac{e^2}{r_0} \int_{0}^{u_0} \left[ \left( u_0 \rho - \frac{1}{2} \left[3 \rho^2 - \frac{\rho^4}{u_0^2} \right] \right) R_{n \ell}(0)^2 + \left( u_0 \rho^2 - \frac{1}{2} \left[3 \rho^3 - \frac{\rho^5}{u_0^2} \right] \right)  2 R_{n \ell}(0) R_{n \ell}'(0) + O(u_0^4) \right] \: \d{\rho} \\
& = \frac{e^2}{u_0 a_0} \left[ \left( \tfrac{1}{2} u_0^3 - \tfrac{1}{2} \left[ u_0^3 - \tfrac{1}{5} u_0^3 \right] \right) R_{n \ell}(0)^2 + \left( \tfrac{1}{3} u_0^4 - \tfrac{1}{2} \left[\tfrac{3}{4} u_0^4 - \tfrac{1}{6} u_0^4 \right] \right)  2 R_{n \ell}(0) R_{n \ell}'(0) + O(u_0^5) \right]  \\
& = \frac{e^2}{a_0} \left[ \tfrac{1}{10} R_{n \ell}(0)^2 u_0^2 + \tfrac{1}{12} R_{n \ell}(0) R_{n \ell}'(0) u_0^3 + O(u_0^4) \right]  \\
& = E_0 \left[ \tfrac{1}{5} R_{n \ell}(0)^2 u_0^2 + \tfrac{1}{6} R_{n \ell}(0) R_{n \ell}'(0) u_0^3 + O(u_0^4) \right] 
\end{align*}
where $E_0 = \frac{e^2}{2 a_0} = \frac{m e^2}{2 \hbar^2} = 13.605 \: \mathrm{eV}$ is (negative) the ground state energy. 
Thus, to leading order,
\[ \Delta E_{n \ell m} = \tfrac{1}{5} E_0  R_{n \ell}(0)^2 u_0^2 \]
For the case $\ell = 0$,
\[ R_{n0}(\rho) = \sqrt{\left(\frac{2}{n}\right)^3 \frac{(n - 1)!}{2n (n)!}}\: \exp{\left[-\frac{\rho}{n}\right]} \: L^{1}_{n - 1}\left(\tfrac{2 \rho}{n} \right) = \sqrt{\left(\frac{2}{n}\right)^3 \frac{1}{2 n^2}} \: \exp{\left[-\frac{\rho}{n}\right]} \: L^{1}_{n - 1}\left(\tfrac{2 \rho}{n} \right) \]
Therefore, we consider the constant term of the polynomials, $L^{1}_{n - 1}\left( 0 \right)$. Using the generating functon,
\[ \sum_{n = 0}^{\infty} t^n L^{1}_n(0) = \frac{1}{(1 - t)^2} = \sum_{n = 0}^\infty (n + 1) t^n \]
so we match terms with the Taylor series of $\frac{1}{(1 - t)^2}$ to get, $L^1_{n - 1}(0) = n$. Thus,
\[ R_{n0}(0) = \sqrt{\left(\frac{2}{n}\right)^3 \frac{1}{2 n^2}} \: L^{1}_{n - 1}\left( 0 \right) = \sqrt{\frac{4}{n^3}} \]
Finally, using $r_0 = 10^{-13} \mathrm{cm}$, 
\[ \Delta E_{n 00} = \tfrac{1}{5} E_0  R_{n \ell}(0)^2 u_0^2 = \frac{4}{5} \frac{E_0 u_0^2}{n^3} \approx 3.887 \cdot 10^{-9} \mathrm{eV} \cdot \frac{1}{n^3} \]
\subsection*{(b)}
Both the leading order and next to leading order (orders 2 and 3 in $u_0$) terms in $\Delta E_{n \ell m}$ are proportional to $R_{n \ell}(0)$ which is proportional to a polynomial with lowest order term $\rho^{\ell}$. Therefore, for $\ell > 0$ we have $R_{n \ell}(0) = 0$ and therefore, $\Delta E_{n \ell m} = 0$ to leading and next to leading order. To get non-zero corrections to $\Delta E_{n \ell m}$ for $\ell > 0$ we must go to fourth order in $u_0$ at which point there is a term proportional to $R_{n \ell}'(0)^2 \neq 0$ for $\ell = 1$. For $\ell > 1$ we must go to even higher order in $u_0$ to reach nonzero corrections.    



\end{document}

