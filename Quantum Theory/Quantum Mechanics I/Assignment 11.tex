\documentclass[12pt]{extarticle}
\usepackage[utf8]{inputenc}
\usepackage[english]{babel}
\usepackage[a4paper, total={7.25in, 9.5in}]{geometry}
\usepackage{tikz-feynman}
\tikzfeynmanset{compat=1.0.0} 
\usepackage{subcaption}
\usepackage{float}
\floatplacement{figure}{H}
\usepackage{simpler-wick}
\usepackage{mathrsfs}  
\usepackage{dsfont}
\usepackage{relsize}
\usepackage{tikz-cd}
\DeclareMathAlphabet{\mathdutchcal}{U}{dutchcal}{m}{n}

\usepackage{cancel}



\newcommand{\field}{\hat{\Phi}}
\newcommand{\dfield}{\hat{\Phi}^\dagger}
 
\usepackage{amsthm, amssymb, amsmath, centernot}
\usepackage{slashed}
\newcommand{\notimplies}{%
  \mathrel{{\ooalign{\hidewidth$\not\phantom{=}$\hidewidth\cr$\implies$}}}}
 
\renewcommand\qedsymbol{$\square$}
\newcommand{\cont}{$\boxtimes$}
\newcommand{\divides}{\mid}
\newcommand{\ndivides}{\centernot \mid}

\newcommand{\Integers}{\mathbb{Z}}
\newcommand{\Natural}{\mathbb{N}}
\newcommand{\Complex}{\mathbb{C}}
\newcommand{\Zplus}{\mathbb{Z}^{+}}
\newcommand{\Primes}{\mathbb{P}}
\newcommand{\Q}{\mathbb{Q}}
\newcommand{\R}{\mathbb{R}}
\newcommand{\ball}[2]{B_{#1} \! \left(#2 \right)}
\newcommand{\Rplus}{\mathbb{R}^+}
\renewcommand{\Re}[1]{\mathrm{Re}\left[ #1 \right]}
\renewcommand{\Im}[1]{\mathrm{Im}\left[ #1 \right]}
\newcommand{\Op}{\mathcal{O}}

\newcommand{\invI}[2]{#1^{-1} \left( #2 \right)}
\newcommand{\End}[1]{\text{End}\left( A \right)}
\newcommand{\legsym}[2]{\left(\frac{#1}{#2} \right)}
\renewcommand{\mod}[3]{\: #1 \equiv #2 \: \mathrm{mod} \: #3 \:}
\newcommand{\nmod}[3]{\: #1 \centernot \equiv #2 \: mod \: #3 \:}
\newcommand{\ndiv}{\hspace{-4pt}\not \divides \hspace{2pt}}
\newcommand{\finfield}[1]{\mathbb{F}_{#1}}
\newcommand{\finunits}[1]{\mathbb{F}_{#1}^{\times}}
\newcommand{\ord}[1]{\mathrm{ord}\! \left(#1 \right)}
\newcommand{\quadfield}[1]{\Q \small(\sqrt{#1} \small)}
\newcommand{\vspan}[1]{\mathrm{span}\! \left\{#1 \right\}}
\newcommand{\galgroup}[1]{Gal \small(#1 \small)}
\newcommand{\bra}[1]{\left| #1 \right>}
\newcommand{\Oa}{O_\alpha}
\newcommand{\Od}{O_\alpha^{\dagger}}
\newcommand{\Oap}{O_{\alpha '}}
\newcommand{\Odp}{O_{\alpha '}^{\dagger}}
\newcommand{\im}[1]{\mathrm{im} \: #1}
\renewcommand{\ker}[1]{\mathrm{ker} \: #1}
\newcommand{\ket}[1]{\left| #1 \right>}
\renewcommand{\bra}[1]{\left< #1 \right|}
\newcommand{\inner}[2]{\left< #1 | #2 \right>}
\newcommand{\expect}[2]{\left< #1 \right| #2 \left| #1 \right>}
\renewcommand{\d}[1]{ \mathrm{d}#1 \:}
\newcommand{\dn}[2]{ \mathrm{d}^{#1} #2 \:}
\newcommand{\deriv}[2]{\frac{\d{#1}}{\d{#2}}}
\newcommand{\nderiv}[3]{\frac{\dn{#1}{#2}}{\d{#3^{#1}}}}
\newcommand{\pderiv}[2]{\frac{\partial{#1}}{\partial{#2}}}
\newcommand{\fderiv}[2]{\frac{\delta #1}{\delta #2}}
\newcommand{\parsq}[2]{\frac{\partial^2{#1}}{\partial{#2}^2}}
\newcommand{\topo}{\mathcal{T}}
\newcommand{\base}{\mathcal{B}}
\renewcommand{\bf}[1]{\mathbf{#1}}
\renewcommand{\a}{\hat{a}}
\newcommand{\adag}{\hat{a}^\dagger}
\renewcommand{\b}{\hat{b}}
\newcommand{\bdag}{\hat{b}^\dagger}
\renewcommand{\c}{\hat{c}}
\newcommand{\cdag}{\hat{c}^\dagger}
\newcommand{\hamilt}{\hat{H}}
\renewcommand{\L}{\hat{L}}
\newcommand{\Lz}{\hat{L}_z}
\newcommand{\Lsquared}{\hat{L}^2}
\renewcommand{\S}{\hat{S}}
\renewcommand{\empty}{\varnothing}
\newcommand{\J}{\hat{J}}
\newcommand{\lagrange}{\mathcal{L}}
\newcommand{\dfourx}{\mathrm{d}^4x}
\newcommand{\meson}{\phi}
\newcommand{\dpsi}{\psi^\dagger}
\newcommand{\ipic}{\mathrm{int}}
\newcommand{\tr}[1]{\mathrm{tr} \left( #1 \right)}
\newcommand{\C}{\mathbb{C}}
\newcommand{\CP}[1]{\mathbb{CP}^{#1}}
\newcommand{\Vol}[1]{\mathrm{Vol}\left(#1\right)}

\newcommand{\Tr}[1]{\mathrm{Tr}\left( #1 \right)}
\newcommand{\Charge}{\hat{\mathbf{C}}}
\newcommand{\Parity}{\hat{\mathbf{P}}}
\newcommand{\Time}{\hat{\mathbf{T}}}
\newcommand{\Torder}[1]{\mathbf{T}\left[ #1 \right]}
\newcommand{\Norder}[1]{\mathbf{N}\left[ #1 \right]}
\newcommand{\Znorm}{\mathcal{Z}}
\newcommand{\EV}[1]{\left< #1 \right>}
\newcommand{\interact}{\mathrm{int}}
\newcommand{\covD}{\mathcal{D}}
\newcommand{\conj}[1]{\overline{#1}}

\newcommand{\SO}[2]{\mathrm{SO}(#1, #2)}
\newcommand{\SU}[2]{\mathrm{SU}(#1, #2)}

\newcommand{\anticom}[2]{\left\{ #1 , #2 \right\}}


\newcommand{\pathd}[1]{\! \mathdutchcal{D} #1 \:}

\renewcommand{\theenumi}{(\alph{enumi})}


\renewcommand{\theenumi}{(\alph{enumi})}

\newcommand{\atitle}[1]{\title{% 
	\large \textbf{Physics GR8048 Quantum Field Theory II
	\\ Assignment \# #1} \vspace{-2ex}}
\author{Benjamin Church }
\maketitle}

\newcommand{\atitleIII}[1]{\title{% 
	\large \textbf{Physics GR8049 Quantum Field Theory III
	\\ Assignment \# #1} \vspace{-2ex}}
\author{Benjamin Church }
\maketitle}

\theoremstyle{definition}
\newtheorem{theorem}{Theorem}[section]
\newtheorem{definition}{definition}[section]
\newtheorem{lemma}[theorem]{Lemma}
\newtheorem{proposition}[theorem]{Proposition}
\newtheorem{corollary}[theorem]{Corollary}
\newtheorem{example}[theorem]{Example}
\newtheorem{remark}[theorem]{Remark}



\begin{document}
\atitle{11}
 
\section*{Problem 35.}
The Hamiltonian of a spherically symmetric potential is given by,
\[\hamilt = \frac{\hat{p}^2}{2m} + V(|\vec{r}|) = \frac{\hat{p}_r^2}{2m} + \frac{\L^2}{2mr^2} + V(|\vec{r}|)\]
Also, let $\ket{\psi_\ell}$ be an energy eigenstate with the lowest energy, namely $E_\ell$, for a fixed value of the angular momentum $\ell$. Because $V$ is spherically symmetric, the $\L_z$ and $\hamilt$ commute so we can choose all the eigenvectors of $\hamilt$, in particular $\ket{\psi_\ell}$, to be a simultaneous eigenvectors of $\L_z$ and $\hamilt$. Because $\ket{\psi_\ell}$ is an eigenvector of $\L^2$ by definition, it is an eigenvector of $\L^2$ and $\L_z$ and therefore it factors as $\ket{\psi_\ell} = \ket{\psi_R} \otimes \ket{\ell, m}$ where $\ket{\psi_R}$ only depends on the radial variable. Then, the expectation value of this state is,
\[\bra{\psi_\ell} \hamilt \ket{\psi_\ell} = \bra{\psi_R} \left[ \frac{\hat{p}_r^2}{2m} + \frac{\hbar^2 \ell (\ell + 1) }{2mr^2} + V(|\vec{r}|) \right] \ket{\psi_R}\]
which is exactly the expectation value of the one dimensional state $\ket{\psi_R}$ under the effective Hamiltonian,
\[\hamilt_\ell = \frac{\hat{p}_r^2}{2m} + \frac{\hbar^2 \ell (\ell + 1) }{2mr^2} + V(|\vec{r}|)\]
By assumption, $\ket{\psi_\ell}$ is the simultaneous eigenstate with the lowest energy eigenvalue implying that $\ket{\psi_R}$ is the ground state of $\hamilt_\ell$ with eigenvalue $E_\ell$. Now take $\ell > \ell'$ so that $\frac{\hbar^2 \ell (\ell + 1)}{2mr^2} > \frac{\hbar^2 \ell' (\ell' + 1)}{2mr^2}$ everywhere. Therefore,
\[\bra{\psi_R} \frac{\hbar^2 \ell (\ell + 1)}{2mr^2} \ket{\psi_R} > \bra{\psi_R} \frac{\hbar^2 \ell' (\ell' + 1)}{2mr^2} \ket{\psi_R} \]
Finally, 
\begin{align*}
E_\ell = \bra{\psi_\ell} \hamilt \ket{\psi_\ell} & = \bra{\psi_R} \left[ \frac{\hat{p}_r^2}{2m} + \frac{\hbar^2 \ell (\ell + 1) }{2mr^2} + V(|\vec{r}|) \right] \ket{\psi_R} > \bra{\psi_R} \left[ \frac{\hat{p}_r^2}{2m} + \frac{\hbar^2 \ell' (\ell' + 1) }{2mr^2} + V(|\vec{r}|) \right] \ket{\psi_R} \\ & = \bra{\psi_R} \hamilt_{\ell'} \ket{\psi_R} 
\end{align*} 
however, by the variational principle, this expectation value of any normalized state is greater than or equal to the ground state energy of the Hamiltonian $\hamilt_{\ell'}$ which is $E_{\ell'}$. Therefore,
\[E_\ell >  \bra{\psi_R} \hamilt_{\ell'} \ket{\psi_R}  \ge E_{\ell'}\] 
\section*{Problem 36.}

\subsection*{(a)}
I will make the added assumptions that there exists a point $|c| < x_0$ such that $V(c) < 0$ and $V(x)$ is continuous at $c$. Otherwise, I believe this statement is false. The potential $V(x) = 0$ for $x \neq 0$ and $V(0) = -1$ satisfies the given conditions yet has no bound states so this added criterion is nessecary. \bigskip \\
Take the Hamiltonian,
\[\hamilt = \frac{\hat{p}^2}{2m} + \lambda^2 V(x)\]
and suppose that $V(x)$ satisfies the above conditions and $V(x) = 0$ for $|x| > x_0$ and $V(x) \le 0$ everywhere. By continuity at $c$, take $\epsilon = \tfrac{1}{2}|V(c)| > 0$, then there exists some $\delta > 0$ such that,
\[|x - c| < \delta \implies |V(x) - V(c)| < \tfrac{1}{2} |V(c)| \implies V(x) < \tfrac{1}{2} V(c) < 0 \]
therefore on the inerval $(c - \delta, c + \delta)$ we have $V(x) < \tfrac{1}{2}V(c) = -V_0$. Take $L = \tfrac{1}{2} \delta$ so on the interval $[c - L, c + L]$ we have that $V(x) < - V_0$. Define the potential,
\[\tilde{V}(x) = \begin{cases}
-\lambda ^2 V_0 & x \in [c - L, c + L] \\
0 & x \notin [c - L, c + L] 
\end{cases}\] 
because $V(x) \le 0$, we have that $\lambda^2 V(x) \le \tilde{V}(x)$ for all $x$. However, $\tilde{V}(x)$ is a 1D finite square well and therefore has at least one bound state, $\ket{\psi_0}$ with energy $\tilde{E}_0 < 0$ which I will prove at the end of the problem. Now, 
\[ \tilde{E}_0 = \bra{\psi_0} \hamilt' \ket{\psi_0} = \bra{\psi_0} \left[ \frac{p^2}{2m} + \tilde{V}(x) \right] \ket{\psi_0} \ge \bra{\psi_0} \left[ \frac{p^2}{2m} + \lambda^2 V(x) \right] \ket{\psi_0}  = \bra{\psi_0} \hamilt \ket{\psi_0} \]
However, by the variational principle, the ground state energy is the minimum of this expectation value of any normalized state. Therefore, the ground state energy, $E_0$, satisfies, $E_0 \le \tilde{E_0} < 0$. Therefore, the ground state of $\hamilt$ is bound.  

\subsection*{(b)}
Now consider the 3D spherically symmetric potential well given by the Hamiltonian,
\[\hamilt = \frac{\hat{p}^2}{2m} + V(|\vec{r}|) = \frac{\hat{p}_r^2}{2m} + \frac{\L^2}{2mr^2} + V(|\vec{r}|)\]
where $V(x)$ satisfies $V(x) \le 0$ for all $x$ and $V(x) = 0$ for $|x| > x_0$. I will make use of the added assumption that $V(x)$ is bounded from below. If $v(x)$ is not continous, it is possible for $V(x)$ to be finite everywhere yet unbounded on the interval $[-x_0, x_0]$. If this is the case, then $\lambda^2 V(x)$ may have bound states for every positive $\lambda$. Take $-V_0$ to be a lower bound on $V(x)$. Define the potential, 
\[\tilde{V}(x) = \begin{cases}
-\lambda ^2 V_0 & x \in [-x_0, x_0] \\
0 & x \notin [-x_0, x_0]
\end{cases}\] 
because $V(x) = 0$ for $|x| > x_0$ and $-V_0 < V(x)$ for every $x$ so $\tilde{V}(x) \le \lambda^2 V(x)$ everywhere. Any bound state of the potential $\tilde{V}$ must satisfy the radial equation,
\[ \tilde{E}_0 (r \psi_R) = - \frac{\hbar^2}{2m} \parsq{}{r} (r \psi_R) + \left[ \frac{\hbar^2 \ell (\ell + 1)}{2 m} + \tilde{V}(r) \right] (r \psi_R) \] 
By problem $35$, the ground state must have $\ell = 0$ because otherwise the energy of the minimal $\ell = 0$ state would be lower than it. Therefore, $u(r) = r \psi_R(r)$ satisfies the 1D Schrodinger equation for a finite square well. However, $u(0) = 0$ because otherwise $\psi_R(r)$ diverges like $\frac{1}{r}$ as $r \to 0$ which can only happen if the potential contains a delta function at the origin, since $\nabla^2 \left(\frac{1}{r}\right) = 4 \pi \delta^3(\vec{r})$, which it does not in this case. Since the full 1D potential $\tilde{V}$ (allowing $x < 0$) is symmetric, the bound states are either even or odd. Since $u(0) = 0$, all the eigenstates must be odd. We will show that for a sufficiently small choice of $\lambda$, there are no odd eigenfunctions to this potential and therefore no bound states of the potential $\tilde{V}$. However, if $\lambda^2 \tilde{V}$ had a bound state $\psi$ then,
\[ \bra{\psi} \hamilt' \ket{\psi} = \bra{\psi} \left[ \frac{p^2}{2m} + \tilde{V}(x) \right] \ket{\psi} \le \bra{\psi} \left[ \frac{p^2}{2m} + \lambda^2 V(x) \right] \ket{\psi} = \bra{\psi} \hamilt \ket{\psi} = E_0 < 0 \]
However, the variational principle implies that the ground state of $\hamilt'$ has energy less than $\bra{\psi} \hamilt' \ket{\psi} \le E_0 < 0$ so the ground state would be bound which contradicts out earlier result that $\hamilt'$ does not have any bound states. Therefore, $\hamilt$ cannot have and bound states. \bigskip \\\\
Finally, consider the 1D Schrodinger equation for a particle in a finite square well,
\[ E \psi(x) = - \frac{\hbar^2}{2m} \parsq{}{x} \psi(x) +  V(x) \psi(x) \] 
where the potential is,
\[V(x) = \begin{cases}
-\lambda^2 V_0 & -L \le x \le L \\
0 & |x| > L
\end{cases}\] 
Because the potential is symmetric, i.e. $V(-x) = V(x)$, the wavefunctions of bound states will be either even or odd. Considering the seperate parity cases and using the general solution for a constant potential in each of the three regions, we obtain a general set of trial solutions,
\[ \psi_{+}(x) = \begin{cases}
A e^{\alpha x} & x < -L \\
B \cos{kx} &  -L \le x \le L \\
A e^{-\alpha x} & x > L
\end{cases} \quad \text{and} \quad \psi_{-}(x) = \begin{cases}
-A e^{\alpha x} & x < -L \\
B \sin{kx} &  -L \le x \le L \\
A e^{-\alpha x} & x > L
\end{cases} \] 
where $\alpha = \sqrt{\frac{-2mE}{\hbar^2}}$ and $k = \sqrt{\frac{2m(\lambda^2 V_0 + E)}{\hbar^2}}$. 
First, consider the even parity case. Matching boundary conditions for the wave function and its derivative at $L$ we obtain, 
\[ A e^{-\alpha L} = B \cos{k L} \quad \text{and} \quad -\alpha A e^{-\alpha L} = - k \sin{k L} \]
and the conditions at $-L$ are identical because we have choosen the wavefunction to be symmetric. Dividing these conditions we arrive at,
\[ \alpha = k \tan{kL} \] 
Futhermore, $(\alpha L)^2 + (kL)^2 = \frac{2m L^2}{\hbar^2} \lambda^2 V_0 = r_0$ so $(
\alpha L)$ and $(k L)$ are constrained to lie on a circle about the origin. However, the curve $(k L) \tan{k L}$ passes through the origin into the first quadrant when plotted against $k$. Therefore, for arbitrarily small radii $r_0$, the circle intersects this curve at a point in the first quadrant. Therefore, there exist positive values of $\alpha$ and $k$ which satisfy both equations for arbitrarily small values of $\lambda$ and therefore correspond to a decaying exponential. Thus, a bound state exists for arbitrarily small values of $\lambda$, the statement we needed in part (a). \bigskip \\
Now consider the odd pariy case. Matching boundary conditions for the wave function and its derivative at $L$ we obtain, 
\[ A e^{-\alpha L} = B \sin{k L} \quad \text{and} \quad -\alpha A e^{-\alpha L} = k \cos{k L} \] 
and the conditions at $-L$ are identical because we have choosen the wavefunction to have definite parity. Dividing these conditions we arrive at,
\[ \alpha = - k \cot{kL} \] 
Again, $(\alpha L)^2 + (kL)^2 = \frac{2m L^2}{\hbar^2} \lambda^2 V_0 = r_0$ so $(
\alpha L)$ and $(k L)$ are constrained to lie on a circle about the origin with radius $r_0$. However, $-(k L) \cot{kL} < 0$ for $kL < \frac{\pi}{2}$ but we need $\alpha > 0$ else the exponentials are not decaying so the derived wavefunction will not be normalizable. Thus, there cannot be any solutions if $kL$ is constrained such that $kL < \frac{\pi}{2}$. For instance, if we choose $\lambda < \sqrt{\frac{\hbar^2}{2mL^2} \frac{\pi}{2V_0}}$ then $r_0 = \frac{2m L^2}{\hbar^2} \lambda^2 V_0 < \frac{\pi}{2}$ so $kL < \frac{\pi}{2}$ and therefore there are no odd parity solutions, proving the claim made in part (b). 
 
\section*{Problem 37.}

\subsection*{(a)}
The energy of a linear rigid rotor with moment of inertial $I$ is given by the classical formula,
\[E = \frac{P^2}{2M} + \frac{L^2}{2I} + V(\vec{r}, \vec{R})\]
If we constrain the system to rotate about a fixed center in the $x$-$y$ plane, then $P^2 = 0$ and $L^2 = L_z^2$. Therefore, we choose the quantum mechanical Hamiltonian,
\[\hamilt = \frac{\L_z^2}{2I} + V(\theta)\]
where I choose $\theta$ to be the polar angle in the $x$-$y$ plane measured counterclockwise from the $+\hat{y}$ direction. Now, the interaction of the dipole and the electric field gives a potential energy term,
\begin{align*} V(\theta) & = - \vec{p} \cdot \vec{E} = - (Q\vec{r}_1 - Q \vec{r}_2) \cdot (E \hat{\jmath}) \\ & = - \left(\frac{Qd}{2} \cos{\theta} \hat{\jmath} + \frac{Qd}{2} \sin{\theta} (-\hat{\imath}) + \frac{-Qd}{2} \cos{\theta} (-\hat{\jmath}) + \frac{-Qd}{2} \sin{\theta} \hat{\imath}  \right) \cdot (E \hat{\jmath}) \\
& = - Qd \left( \cos{\theta} \hat{\jmath} - \sin{\theta} \hat{\imath} \right) \cdot (E \hat{\jmath}) = - QdE \cos{\theta}
\end{align*}
Therefore, the full Hamiltonian is given by,
\[\hamilt = \frac{\L_z^2}{2I} - QdE \cos{\theta}\]

\subsection*{(b)}

Let $V(\theta)$ be a perturbation on the Hamiltonian $\hamilt_0 = \frac{\L_z^2}{2I}$. The eigenfunctions of this operator are the spherical harmonics evaluated in the $x$-$y$ plane which are simply the wavefunctions on a circle, 
\[ \psi_m(\theta) = \frac{1}{\sqrt{2\pi}} e^{i m \theta}\]
where $m \in \Z$. These states are eigenvectors of $\L_z$ since $\L_z = \frac{\hbar}{i} \pderiv{}{\theta}$ so $\hbar m$ is the eigenvalue of $m$. The energy spectrum is given by, $\hamilt_0 \ket{\psi_m} = \frac{\L_z^2}{2I} \ket{\psi_m} = \frac{\hbar^2 m^2}{2I} \ket{\psi_m}$ so $E^{(0)}_m = \frac{\hbar^2 m^2}{2 I}$.

\subsection*{(c)}

Treating the term $V(\theta) = - QdE \cos{\theta}$ as a perturbation, we need to calculate the matrix elements, $V_{ab} = \bra{\psi_a} V \ket{\psi_b}$ of $V$ with the eigenvectors of $\hamilt_0$. These are easily done by writing,
\begin{align*}
\bra{\psi_a} V \ket{\psi_b} & = - QdE \int_{0}^{2 \pi} \psi_a(\theta)^* \cos{\theta} \psi_b(\theta) \d{\theta} = -QdE \int_{0}^{2 \pi} \frac{1}{2 \pi} e^{-ia \theta} \tfrac{1}{2} \left[ e^{i \theta} + e^{-i\theta} \right] e^{ib \theta} \d{\theta} \\
& = - \tfrac{1}{2} QdE \int_{0}^{2 \pi} e^{i(b - a + 1) \theta} + e^{i(b - a - 1) \theta} \d{\theta} = - \tfrac{1}{2} QdE \left[ \delta_{a, b + 1} + \delta_{a + 1, b}\right]
\end{align*} 
These matrix elements are very nice because they only connect states with a difference of $1$ in level. In particular, this perturbation does not connect any degenerate states. This because the only degenerate subspaces are given by the span of $\ket{\psi_m}$ and $\ket{\psi_{-m}}$ which have a difference of $2m$ which is even. Furthermore, the diagonal terms are all zero because $V(\theta)$ has odd parity and therefore can only connect states of different parity. Thus every degenerate submatrix of $V_{ab}$ is entirely zero so we do not need to worry about diagonalizing the subspaces in applying first order perturbation theory. Furthermore, because all diagonal terms are zero, the first order energies all vanish: $E^{(1)}_m = 0$.  

\subsection*{(d)}

The degeneracy is not lifted at first order because every first order term is zero. We can calculate the diagonal terms of of the second order correction via,
\begin{align*}
E^{(2)}_m & = \sum_{m' \neq m} \frac{|\bra{\psi_m} V \ket{\psi_{m'}}|^2}{E^{(0)}_m - E^{(0)}_{m'}} = \frac{2I}{\hbar^2} (\tfrac{1}{2} QdE)^2 \left[ \frac{1}{m^2 - (m - 1)^2} + \frac{1}{m^2 - (m + 1)^2} \right] \\ 
& = \frac{Q^2 d^2 E^2 I}{2\hbar^2} \left[ \frac{1}{m^2 - m^2 + 2m - 1} + \frac{1}{m^2 - m^2 - 2m - 1} \right] = \frac{Q^2 d^2 E^2 I}{2\hbar^2} \left[ \frac{1}{2m - 1} - \frac{1}{2m + 1} \right] \\ & = \frac{Q^2 d^2 E^2 I}{2\hbar^2} \left[ \frac{2m + 1}{4m^2 - 1} - \frac{2m - 1}{4m^2 - 1} \right] = \frac{Q^2 d^2 E^2 I}{\hbar^2} \frac{1}{4m^2 - 1}
\end{align*}
However, we may also connect two degenerate terms at second order. For $i > j$ in the same degenerate subspace, the off-diagonal terms are given by,
\begin{align*}
E^{(2)}_{ij} & = \sum_{m \notin W_{deg}} \frac{V_{im} V_{m j}}{E^{(0)}_i - E^{(0)}_{m}} = \frac{2I}{\hbar^2} (\tfrac{1}{2} QdE)^2 \frac{1}{i^2 - (i - 1)^2} \delta_{i, m + 1} \delta_{m, j + 1} = \frac{Q^2 d^2 E^2 I}{2\hbar^2} \frac{1}{2 i - 1} \delta_{i, m + 1} \delta_{m, j + 1} 
\end{align*}
Since $i > j$ the term with $i = m - 1$ cannot appear because $j = m \pm 1$ so then $j = i + 1 \pm 1$ contradicting the assumption $i > j$. This diagonal can only be nonzero if $i = m + 1$ and $m = j + 1$ so $i = j + 2$. However, $i \neq j$ must be in the same degenerate subspace so $i = - j$. Therefore, for nonzero diagonal terms, $i = -i + 2$ so $i = 1$. This term is explicitly, $E^{(2)}_{1,-1} = \frac{Q^2 d^2 E^2 I}{2\hbar^2}$. We now use the fact that $E^{(2)}_{ij}$ is Hermitian. This is easily checked by considering,
\[ (E^{(2)}_{ij})^* = \sum_{m \notin W_{deg}} \frac{V_{im}^* V_{m j}^*}{E^{(0)}_i - E^{(0)}_{m}} = \sum_{m \notin W_{deg}} \frac{V_{mi} V_{jm}}{E^{(0)}_j - E^{(0)}_{m}} = E^{(2)}_{ji}\]
where I can freely swap $E^{(0)}_i$ and $E^{(0)}_j$ because both $i$ and $j$ belong to the same degenerate subspace. Therefore, $E^{(0)}_{-1, 1} = (E^{(2)}_{1,-1})^* =  \frac{Q^2 d^2 E^2 I}{2\hbar^2}$. \bigskip \\
Now, using the above formula for the diagonal terms, consider the submatrix for the degenerate $m = \pm 1$ subspace,
\[ V_1 = \frac{Q^2 d^2 E^2 I}{2\hbar^2} \begin{pmatrix}
2/3 & 1 \\
1 & 2/3 
\end{pmatrix}\]
This is easily diagonalized by the eigenvectors,
\begin{align*}
\ket{1^{+}} & = \frac{1}{\sqrt{2}} \left[ \ket{\psi_1} + \ket{\psi_{-1}} \right] \quad \text{with eigenvalue} \quad E^{(2)}_{1^+} = \frac{5}{3} \frac{Q^2 d^2 E^2 I}{2\hbar^2}   \\ 
\ket{1^{-}} & = \frac{1}{\sqrt{2}} \left[ \ket{\psi_1} - \ket{\psi_{-1}} \right]  \quad \text{with eigenvalue} \quad E^{(2)}_{1^-} = - \frac{1}{3} \frac{Q^2 d^2 E^2 I}{2\hbar^2}
\end{align*}
As show above, the other degenerate subspaces cannot have conecting off-diagonal terms and thus their second order energies are already diagonalized. Thus, their second order energy shift is exactly that diagonal term. Finally, to second order, the energy spectrum for $m \neq \pm 1$ is given by,
\[E_m = E^{(0)}_m + E^{(1)}_m + E^{(2)}_m = \frac{\hbar^2 m^2}{2 I} + \frac{Q^2 d^2 E^2 I}{\hbar^2} \frac{1}{4m^2 - 1} = \frac{\hbar^2}{2I} \left[ m^2 + \frac{Q^2 d^2 E^2 I^2}{\hbar^4} \frac{2}{4 m^2 - 1} \right] \] 
and the $m = \pm 1$ degeneracy is lifted at second order leading to two non-degenerate states, 
\begin{align*}
\ket{1^{+}} & = \frac{1}{\sqrt{2}} \left[ \ket{\psi_1} + \ket{\psi_{-1}} \right] \quad \text{with energy} \quad E_{1^+}  = \frac{\hbar^2}{2 I} + \frac{5}{3} \frac{Q^2 d^2 E^2 I}{2\hbar^2}  = \frac{\hbar^2}{2I} \left[1 + \frac{5}{3} \frac{Q^2 d^2 E^2 I^2}{\hbar^4} \right]  \\ 
\ket{1^{-}} & = \frac{1}{\sqrt{2}} \left[ \ket{\psi_1} - \ket{\psi_{-1}} \right]  \quad \text{with energy} \quad E_{1^-} = \frac{\hbar^2}{2 I} - \frac{1}{3} \frac{Q^2 d^2 E^2 I}{2\hbar^2} = \frac{\hbar^2}{2I} \left[1 - \frac{1}{3} \frac{Q^2 d^2 E^2 I^2}{\hbar^4} \right] 
\end{align*}

\section*{Problem 38.}
Consider the 2D harmonic oscilator with Hamiltonian,
\[\hamilt_0 = \frac{\hat{p}^2_x + \hat{p}^2_y}{2m} + \frac{1}{2} m \omega^2 (\hat{x}^2 + \hat{y}^2) \] 

\subsection*{(a)}
The eigenstates are easily found because this Hamiltonian groups into two 1D harmonic oscilators with no coupling between the two,
\[\hamilt_0 = \left( \frac{\hat{p}^2_x}{2m} + \frac{1}{2} m \omega^2 \hat{x}^2 \right) + \left( \frac{\hat{p}^2_y}{2m} + \frac{1}{2} m \omega^2 \hat{y}^2 \right) = \hbar \omega \left( \adag_x \a_x + \adag_y \a_y + 1\right) \] 
Thus, the eigenstates are given by acting on the ground state $\ket{0}$ with the rasing operators where $\ket{0}$ is the unique state killed by both lowering operators. That is,
\[ \ket{n_x, n_y} = \frac{(\adag_x)^{n_x}}{\sqrt{n_x!}} \frac{(\adag_y)^{n_y}}{\sqrt{n_y!}} \ket{0}\]
with energies,
\[E^{(0)}_{n_x,n_y} = \bra{n_x, n_y} \hamilt_0 \ket{n_x, n_y} = \hbar \omega \left(n_x + n_y + 1 \right)\]

\subsection*{(b)}

The six lowest energy states are arranged in three eigenspaces. The $E = \hbar \omega$ eigenspace is spanned by $\ket{0}$ alone. The $E = 2 \hbar \omega$ is spanned by $\ket{1,0}$ and $\ket{0,1}$. Finally, the $E = 3 \hbar \omega$ is spanned by $\ket{2,0}$, $\ket{1,1}$, and $\ket{0,2}$ collectively giving the six lowest energy independent energy eigenstates. Add the perturbation, $\hat{V} = \lambda \hat{x} \hat{y}$ to the Hamiltonian,
\[\hamilt = \hamilt_0 + \lambda \hat{x} \hat{y} = \frac{\hat{p}^2_x + \hat{p}^2_y}{2m} + \frac{1}{2} m \omega^2 (\hat{x}^2 + 2 \eta \hat{x} \hat{y} + \hat{y}^2)\]
where $\eta = \frac{\lambda}{m \omega^2}$. Now, we calculate the matrix elements, $V_{abcd} = \bra{a,b} \hat{V} \ket{c, d}$,
\begin{align*}
V_{abcd} & = \bra{a,b} \lambda \hat{x} \hat{y} \ket{c,d} = m \omega^2 \eta \bra{a,b} \hat{x} \hat{y} \ket{c,d} = m \omega^2 \eta \bra{a,b} \frac{\hbar}{2 m \omega} (\adag_x + \a_x)(\adag_y + \a_y) \ket{c,d} \\
& =  \tfrac{1}{2} \eta \hbar \omega \bra{a,b} \left[ \adag_x \adag_y + \adag_x \a_y + \a_x \adag_y + \a_x \a_y \right] \ket{c,d} \\ & = \tfrac{1}{2} \eta \hbar \omega \left[ \delta_{a,c + 1} \delta_{b, d + 1} \sqrt{ab} + \delta_{a,c + 1} \delta_{b, d - 1} \sqrt{ad} + \delta_{a, c - 1} \delta_{b, d + 1} \sqrt{bc} + \delta_{a, c - 1} \delta_{b, d - 1} \sqrt{cd} \right]
\end{align*}
Now we consider the submatrices for the first three degenerate subspaces. The $E = \hbar \omega$ subspace only has one independent vector and $V_{abcd}$ vanishes for diagonal terms. Therefore, the energies of this subspace are uneffected to first order. For $E = 2 \hbar \omega$ there are two states to consider. In the basis $\{\ket{1,0}, \ket{0,1}\}$, the submatrix becomes,
\[ V_2 = \tfrac{1}{2} \eta \hbar \omega \begin{pmatrix}
0 & 1 \\
1 & 0 
\end{pmatrix}\]
whose eigenvectors are:
\begin{align*}
\ket{2^+} & = \frac{1}{\sqrt{2}} \left[ \ket{1,0} + \ket{0,1} \right] \quad \text{with eigenvalue} \quad E^{(1)}_{2^+} = \tfrac{1}{2} \eta \hbar \omega \\ 
\ket{2^-} & = \frac{1}{\sqrt{2}} \left[ \ket{1,0} - \ket{0,1} \right] \quad \text{with eigenvalue} \quad E^{(1)}_{2^-} = - \tfrac{1}{2} \eta \hbar \omega
\end{align*}
Finally, in the $E = 3 \hbar \omega$ subspace with the basis, $\{\ket{2,0}, \ket{1,1}, \ket{0,2}\}$ the $V_{abcd}$ submatrix is,
\[ V_3 = \tfrac{1}{2} \eta \hbar \omega \begin{pmatrix}
0 & \sqrt{2} & 0 \\
\sqrt{2} & 0 & \sqrt{2} \\
0 & \sqrt{2} & 0
\end{pmatrix}\]
whose eigenvectors are:
\begin{align*}
\ket{3^+} & = \frac{1}{2\sqrt{2}} \left[ \ket{2,0} + \sqrt{2} \ket{1,1} + \ket{0,2} \right] \quad \text{with eigenvalue} \quad E^{(1)}_{3^+} = \eta \hbar \omega \\ 
\ket{3^0} & = \frac{1}{\sqrt{2}} \left[ \ket{2,0} - \ket{0,2} \right] \quad \text{with eigenvalue} \quad E^{(1)}_{3^0} = 0 \\ 
\ket{3^-} & = \frac{1}{2\sqrt{2}} \left[ \ket{2,0} - \sqrt{2} \ket{1,1} + \ket{0,2} \right] \quad \text{with eigenvalue} \quad E^{(1)}_{3^-} = - \eta \hbar \omega 
\end{align*}

\subsection*{(c)}
For the Hamiltonian, 
\[\hamilt = \frac{\hat{p}^2_x + \hat{p}^2_y}{2m} + \frac{1}{2} m \omega^2 (\hat{x}^2 + 2 \eta \hat{x} \hat{y} + \hat{y}^2)\]
we can change coordinates to diagonalize the quadratic form. We use the coordinate transformation,
\begin{align*}
\hat{q}_1 & = \tfrac{1}{\sqrt{2}} \left( \hat{x} + \hat{y} \right) \\
\hat{q}_2 & = \tfrac{1}{\sqrt{2}}  \left( \hat{x} - \hat{y} \right) \\
\hat{p}_1 & = \tfrac{1}{\sqrt{2}}  \left( \hat{p}_x + \hat{p}_y \right) \\
\hat{p}_2 & = \tfrac{1}{\sqrt{2}}  \left( \hat{p}_x - \hat{p}_y \right) \\
\end{align*}
which satisfy the relations, $[\hat{q}_i, \hat{p}_j] =  i \hbar \delta_{ij}$ and all other pairs commute. Therefore, the given transformation is canonical. We reverse the transformation to write the Hamiltonian in terms of the new coordinates,
\begin{align*}
\hat{x} & = \tfrac{1}{\sqrt{2}}  \left( \hat{q}_1 + \hat{q}_2 \right) \\
\hat{y} & = \tfrac{1}{\sqrt{2}}  \left( \hat{q}_1 - \hat{q}_2 \right) \\
\hat{p}_x & = \tfrac{1}{\sqrt{2}}  \left( \hat{p}_1 + \hat{p}_2 \right)  \\
\hat{p}_y & = \tfrac{1}{\sqrt{2}} \left( \hat{p}_1 - \hat{p}_2 \right) \\
\end{align*}
then the Hamiltonian becomes,
\begin{align*}
\hamilt & = \frac{( \hat{p}_1 + \hat{p}_2)^2 + ( \hat{p}_1 - \hat{p}_2)^2}{4m} + \frac{1}{4} m \omega^2 (( \hat{q}_1 + \hat{q}_2)^2 + 2 \eta ( \hat{q}_1 + \hat{q}_2)( \hat{q}_1 - \hat{q}_2)  + (\hat{q}_1 - \hat{q}_2)^2) \\
& = \frac{\hat{p}^2_1 + \hat{p}^2_2}{2m} + \frac{1}{2} m \omega^2 ((1 + \eta) \hat{q}_1^2 + (1 - \eta) \hat{q}_2^2) = \left( \frac{\hat{p}^2_1}{2m} + \frac{1}{2} m \omega^2 (1 + \eta) \hat{q}_1^2 \right) + \left( \frac{\hat{p}^2_2}{2m} + \frac{1}{2} m \omega^2 (1 - \eta) \hat{q}_2^2 \right) \\ & = \hbar \omega \left( \adag_1 \a_1 \sqrt{1 + \eta} + \adag_2 \a_2 \sqrt{1 - \eta}  + \tfrac{1}{2} \left[ \sqrt{1 + \eta} + \sqrt{1 - \eta} \right] \right)
\end{align*}

Thus, the eigenstates are given by acting on the ground state $\ket{0}$ with the rasing operators where $\ket{0}$ is the unique state killed by both lowering operators. That is,
\[ \ket{n_1, n_2} = \frac{(\adag_1)^{n_1}}{\sqrt{n_1!}} \frac{(\adag_2)^{n_2}}{\sqrt{n_2!}} \ket{0}\]
with energies,
\[E_{n_1,n_2} = \bra{n_1, n_2} \hamilt \ket{n_1, n_2} = \hbar \omega \left(n_1 \sqrt{1 + \eta} + n_2 \sqrt{1 - \eta} + \tfrac{1}{2} \left[ \sqrt{1 + \eta} + \sqrt{1 - \eta} \right] \right)\]
To first order,
\[E_{n_1,n_2} \approx \hbar \omega \left(n_1 (1 + \tfrac{1}{2} \eta) + n_2 (1 - \tfrac{1}{2} \eta) + 1 \right)\]
which agrees with the earlier result. 
\subsection*{(d)}
Now, we add the perturbation, $\hat{V} = \lambda \hat{x}^2 \hat{y}^2$ to the Hamiltonian and, letting $\eta = \frac{\hbar \lambda }{m^2 \omega^3}$, calculate the matrix elements, $V_{abcd} = \bra{a,b} \hat{V} \ket{c, d}$,
\begin{align*}
V_{abcd} & = \bra{a,b} \lambda \hat{x}^2 \hat{y}^2 \ket{c,d} = \lambda \bra{a,b} \left(\frac{\hbar}{2 m \omega} \right)^2 (\adag_x + \a_x)^2(\adag_y + \a_y)^2 \ket{c,d} \\ & = \tfrac{1}{4} \eta \hbar \omega \bra{a,b} (\adag_x + \a_x)^2(\adag_y + \a_y)^2 \ket{c,d}\\
& = \tfrac{1}{4} \eta \hbar \omega \bra{a,b} ((\adag_x)^2 + \adag_x \a_x + \a_x \adag_x + (\a_x)^2)((\adag_y)^2 + \adag_y \a_y + \a_y \adag_y + (\a_y)^2) \ket{c,d}
\end{align*}
In particular, the diagonal terms are,
\begin{align*}
V_{abab} & = \tfrac{1}{4} \eta \hbar \omega \bra{a,b} ((\adag_x)^2 + \adag_x \a_x + \a_x \adag_x + (\a_x)^2)((\adag_y)^2 + \adag_y \a_y + \a_y \adag_y + (\a_y)^2) \ket{a,b} \\
& =  \tfrac{1}{4} \eta \hbar \omega \bra{a,b} (0 + a + a + 1 + 0)(0 + b + b + 1 + 0) \ket{a,b} = \tfrac{1}{4} \eta \hbar \omega (2a + 1)(2b + 1)
\end{align*}
Now we consider the submatrices for the first three degenerate subspaces. For the $E = \hbar \omega$ subspace we need only consider the diagonal $V_{0000}$ term which only connects operators with leading $\adag$ and equal numbers of raising and lowering operators. Only the $\a_x \adag_x \a_y \adag_y$ terms contributes and $\bra{0} \a_x \adag_x \a_y \adag_y \ket{0} = 1$. Therefore, $E^{(1)}_{1} = E^{(1)}_{0,0} = \tfrac{1}{4} \eta \hbar \omega$. \bigskip \\\\
The $E = 2 \hbar \omega$ subspace has two independent vectors and $V_{abcd}$ vanishes for terms which differ by and odd number in $n_x$ or $n_y$ because the operators enter in second order. Therefore, we need only consider diagonal terms. Using the diagonal formula, the energy shifts are,
\[ E^{(1)}_{1,0} = \bra{1, 0} \hat{V} \ket{1, 0} = \tfrac{3}{4} \eta \hbar \omega\]
\[ E^{(1)}_{0,1} = \bra{0, 1} \hat{V} \ket{0, 1} = \tfrac{3}{4} \eta \hbar \omega\]
For the $E = 3 \hbar \omega$ subspace, the diagonal terms are easily calculated from the formula,
\[V_{2020} = \tfrac{5}{4} \eta \hbar \omega \quad V_{1111} = \tfrac{9}{4} \eta \hbar \omega \quad V_{0202} = \tfrac{5}{4} \eta \hbar \omega \]
Because each term in the perturbation is second order in rasing and lowering operators, terms differing by an odd number in ether the $x$ or $y$ quantum numbers have zero connection. Therefore, the only nonzero off-diagonal terms in this subspace are the $V_{2002}$ and $V_{0220}$ terms. Calculating these terms,
\begin{align*}
V_{2002} & = \tfrac{1}{4} \eta \hbar \omega \bra{2, 0} ((\adag_x)^2 + \adag_x \a_x + \a_x \adag_x + (\a_x)^2)((\adag_y)^2 + \adag_y \a_y + \a_y \adag_y + (\a_y)^2) \ket{0, 2} \\
& = \tfrac{1}{4} \eta \hbar \omega \bra{2, 0} (\adag_x)^2 (\a_y)^2 \ket{0, 2} = \tfrac{1}{2} \eta \hbar \omega 
\end{align*}
and similarly,
\begin{align*}
V_{0220} & = \tfrac{1}{4} \eta \hbar \omega \bra{0, 2} ((\adag_x)^2 + \adag_x \a_x + \a_x \adag_x + (\a_x)^2)((\adag_y)^2 + \adag_y \a_y + \a_y \adag_y + (\a_y)^2) \ket{2, 0} \\
& = \tfrac{1}{4} \eta \hbar \omega \bra{0, 2} (\a_x)^2 (\adag_y)^2 \ket{2, 0} = \tfrac{1}{2} \eta \hbar \omega 
\end{align*}

Finally, using the basis, $\{\ket{2,0}, \ket{1,1}, \ket{0,2}\}$ the $V_{abcd}$ submatrix is,
\[ V_3 = \tfrac{1}{4} \eta \hbar \omega \begin{pmatrix}
5 & 0 & 2 \\
0 & 9 & 0 \\
2 & 0 & 5
\end{pmatrix}\]
whose eigenvectors are:
\begin{align*}
\ket{3^+} & = \frac{1}{\sqrt{2}} \left[ \ket{2,0} + \ket{0,2} \right] \quad \text{with eigenvalue} \quad E^{(1)}_{3^+} = \tfrac{9}{4} \eta \hbar \omega \\ 
\ket{3^0} & = \ket{1,1} \quad \text{with eigenvalue} \quad E^{(1)}_{3^0} = E^{(1)}_{1,1} = \tfrac{7}{4} \eta \hbar \omega \\ 
\ket{3^-} & = \frac{1}{\sqrt{2}} \left[ \ket{2,0} - \ket{0,2} \right] \quad \text{with eigenvalue} \quad E^{(1)}_{3^-} = \tfrac{3}{4} \eta \hbar \omega \\ 
\end{align*}


\end{document}

