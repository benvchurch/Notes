\documentclass[12pt]{extarticle}
\usepackage[utf8]{inputenc}
\usepackage[english]{babel}
\usepackage[a4paper, total={7.25in, 9.5in}]{geometry}
\usepackage{tikz-feynman}
\tikzfeynmanset{compat=1.0.0} 
\usepackage{subcaption}
\usepackage{float}
\floatplacement{figure}{H}
\usepackage{simpler-wick}
\usepackage{mathrsfs}  
\usepackage{dsfont}
\usepackage{relsize}
\usepackage{tikz-cd}
\DeclareMathAlphabet{\mathdutchcal}{U}{dutchcal}{m}{n}

\usepackage{cancel}



\newcommand{\field}{\hat{\Phi}}
\newcommand{\dfield}{\hat{\Phi}^\dagger}
 
\usepackage{amsthm, amssymb, amsmath, centernot}
\usepackage{slashed}
\newcommand{\notimplies}{%
  \mathrel{{\ooalign{\hidewidth$\not\phantom{=}$\hidewidth\cr$\implies$}}}}
 
\renewcommand\qedsymbol{$\square$}
\newcommand{\cont}{$\boxtimes$}
\newcommand{\divides}{\mid}
\newcommand{\ndivides}{\centernot \mid}

\newcommand{\Integers}{\mathbb{Z}}
\newcommand{\Natural}{\mathbb{N}}
\newcommand{\Complex}{\mathbb{C}}
\newcommand{\Zplus}{\mathbb{Z}^{+}}
\newcommand{\Primes}{\mathbb{P}}
\newcommand{\Q}{\mathbb{Q}}
\newcommand{\R}{\mathbb{R}}
\newcommand{\ball}[2]{B_{#1} \! \left(#2 \right)}
\newcommand{\Rplus}{\mathbb{R}^+}
\renewcommand{\Re}[1]{\mathrm{Re}\left[ #1 \right]}
\renewcommand{\Im}[1]{\mathrm{Im}\left[ #1 \right]}
\newcommand{\Op}{\mathcal{O}}

\newcommand{\invI}[2]{#1^{-1} \left( #2 \right)}
\newcommand{\End}[1]{\text{End}\left( A \right)}
\newcommand{\legsym}[2]{\left(\frac{#1}{#2} \right)}
\renewcommand{\mod}[3]{\: #1 \equiv #2 \: \mathrm{mod} \: #3 \:}
\newcommand{\nmod}[3]{\: #1 \centernot \equiv #2 \: mod \: #3 \:}
\newcommand{\ndiv}{\hspace{-4pt}\not \divides \hspace{2pt}}
\newcommand{\finfield}[1]{\mathbb{F}_{#1}}
\newcommand{\finunits}[1]{\mathbb{F}_{#1}^{\times}}
\newcommand{\ord}[1]{\mathrm{ord}\! \left(#1 \right)}
\newcommand{\quadfield}[1]{\Q \small(\sqrt{#1} \small)}
\newcommand{\vspan}[1]{\mathrm{span}\! \left\{#1 \right\}}
\newcommand{\galgroup}[1]{Gal \small(#1 \small)}
\newcommand{\bra}[1]{\left| #1 \right>}
\newcommand{\Oa}{O_\alpha}
\newcommand{\Od}{O_\alpha^{\dagger}}
\newcommand{\Oap}{O_{\alpha '}}
\newcommand{\Odp}{O_{\alpha '}^{\dagger}}
\newcommand{\im}[1]{\mathrm{im} \: #1}
\renewcommand{\ker}[1]{\mathrm{ker} \: #1}
\newcommand{\ket}[1]{\left| #1 \right>}
\renewcommand{\bra}[1]{\left< #1 \right|}
\newcommand{\inner}[2]{\left< #1 | #2 \right>}
\newcommand{\expect}[2]{\left< #1 \right| #2 \left| #1 \right>}
\renewcommand{\d}[1]{ \mathrm{d}#1 \:}
\newcommand{\dn}[2]{ \mathrm{d}^{#1} #2 \:}
\newcommand{\deriv}[2]{\frac{\d{#1}}{\d{#2}}}
\newcommand{\nderiv}[3]{\frac{\dn{#1}{#2}}{\d{#3^{#1}}}}
\newcommand{\pderiv}[2]{\frac{\partial{#1}}{\partial{#2}}}
\newcommand{\fderiv}[2]{\frac{\delta #1}{\delta #2}}
\newcommand{\parsq}[2]{\frac{\partial^2{#1}}{\partial{#2}^2}}
\newcommand{\topo}{\mathcal{T}}
\newcommand{\base}{\mathcal{B}}
\renewcommand{\bf}[1]{\mathbf{#1}}
\renewcommand{\a}{\hat{a}}
\newcommand{\adag}{\hat{a}^\dagger}
\renewcommand{\b}{\hat{b}}
\newcommand{\bdag}{\hat{b}^\dagger}
\renewcommand{\c}{\hat{c}}
\newcommand{\cdag}{\hat{c}^\dagger}
\newcommand{\hamilt}{\hat{H}}
\renewcommand{\L}{\hat{L}}
\newcommand{\Lz}{\hat{L}_z}
\newcommand{\Lsquared}{\hat{L}^2}
\renewcommand{\S}{\hat{S}}
\renewcommand{\empty}{\varnothing}
\newcommand{\J}{\hat{J}}
\newcommand{\lagrange}{\mathcal{L}}
\newcommand{\dfourx}{\mathrm{d}^4x}
\newcommand{\meson}{\phi}
\newcommand{\dpsi}{\psi^\dagger}
\newcommand{\ipic}{\mathrm{int}}
\newcommand{\tr}[1]{\mathrm{tr} \left( #1 \right)}
\newcommand{\C}{\mathbb{C}}
\newcommand{\CP}[1]{\mathbb{CP}^{#1}}
\newcommand{\Vol}[1]{\mathrm{Vol}\left(#1\right)}

\newcommand{\Tr}[1]{\mathrm{Tr}\left( #1 \right)}
\newcommand{\Charge}{\hat{\mathbf{C}}}
\newcommand{\Parity}{\hat{\mathbf{P}}}
\newcommand{\Time}{\hat{\mathbf{T}}}
\newcommand{\Torder}[1]{\mathbf{T}\left[ #1 \right]}
\newcommand{\Norder}[1]{\mathbf{N}\left[ #1 \right]}
\newcommand{\Znorm}{\mathcal{Z}}
\newcommand{\EV}[1]{\left< #1 \right>}
\newcommand{\interact}{\mathrm{int}}
\newcommand{\covD}{\mathcal{D}}
\newcommand{\conj}[1]{\overline{#1}}

\newcommand{\SO}[2]{\mathrm{SO}(#1, #2)}
\newcommand{\SU}[2]{\mathrm{SU}(#1, #2)}

\newcommand{\anticom}[2]{\left\{ #1 , #2 \right\}}


\newcommand{\pathd}[1]{\! \mathdutchcal{D} #1 \:}

\renewcommand{\theenumi}{(\alph{enumi})}


\renewcommand{\theenumi}{(\alph{enumi})}

\newcommand{\atitle}[1]{\title{% 
	\large \textbf{Physics GR8048 Quantum Field Theory II
	\\ Assignment \# #1} \vspace{-2ex}}
\author{Benjamin Church }
\maketitle}

\newcommand{\atitleIII}[1]{\title{% 
	\large \textbf{Physics GR8049 Quantum Field Theory III
	\\ Assignment \# #1} \vspace{-2ex}}
\author{Benjamin Church }
\maketitle}

\theoremstyle{definition}
\newtheorem{theorem}{Theorem}[section]
\newtheorem{definition}{definition}[section]
\newtheorem{lemma}[theorem]{Lemma}
\newtheorem{proposition}[theorem]{Proposition}
\newtheorem{corollary}[theorem]{Corollary}
\newtheorem{example}[theorem]{Example}
\newtheorem{remark}[theorem]{Remark}


\newcommand{\pL}{\phi^L_n}
\newcommand{\pR}{\phi^R_n}
\newcommand{\pspm}{\psi_{n}^{\pm}}
\newcommand{\psp}{\psi_{n}^{+}}
\newcommand{\psm}{\psi_{n}^{-}}
\newcommand{\dpsp}{(\psi_{n}^{+})'}
\newcommand{\dpsm}{(\psi_{n}^{-})'}
\newcommand{\ddpsp}{(\psi_{n}^{+})''}
\newcommand{\ddpsm}{(\psi_{n}^{-})''}



\begin{document}
\atitle{8}
 
\section*{Problem 26.}

\subsection*{(a)}
First, we prove that there cannot be degenerate bound states in one dimension. Suppose that $\psi_1$ and $\psi_2$ are bound states with equal energies. Then,
\[ E \psi_1(x) = - \frac{\hbar^2}{2m} \parsq{}{x} \psi_1(x) + V(x) \psi_1(x) \quad \text{and} \quad E \psi_2(x) = - \frac{\hbar^2}{2m} \parsq{}{x} \psi_2(x) + V(x) \psi_2(x) \]
Therefore, we can write,
\[ (E - V(x))\psi_1(x) \psi_2(x) = -\psi_2(x) \frac{\hbar^2}{2m} \parsq{}{x} \psi_1(x) \quad \text{and} \quad (E - V(x))\psi_2(x) \psi_1(x) = - \psi_1(x) \frac{\hbar^2}{2m} \parsq{}{x} \psi_2(x) \]
Subtracting the two equations and multiplying out the constants,
\begin{align*}
\psi''_1(x) \psi_2(x) - \psi_1(x) \psi''_2(x) & = 0 
\end{align*}
We can rewrite this expression by adding and subtracting a term cleverly,
\[ \psi''_1(x) \psi_2(x) - \psi_1(x) \psi''_2(x) = \psi''_1(x) \psi_2(x) + \psi'_1(x) \psi'_2(x) - \psi'_1(x) \psi'_2(x) - \psi_1(x) \psi''_2(x) = 0\]
Therefore, 
\[ \pderiv{}{x} \left[ \psi'_1(x) \psi_2(x) - \psi_1(x) \psi'_2(x) \right] = 0 \]
So we can set the inner term equal to a constant. Thus,
\[\psi'_1(x) \psi_2(x) - \psi_1(x) \psi'_2(x) = C\]
However, these are bound states so the limit of this term as $x \to \pm \infty$ is zero because the wavefunctions vanish at $\pm \infty$. Therefore, the constant $C = 0$ or else this equation cannot be true everywhere. Thus,

\[ \psi'_1(x) \psi_2(x) - \psi_1(x) \psi'_2(x) = 0 \quad \text{so} \quad \psi'_1(x) \psi_2(x) = \psi_1(x) \psi'_2(x)\]
Therefore,
\[\frac{\psi'_1(x)}{\psi_1(x)} = \frac{\psi'_2(x)}{\psi_2(x)}\]
which can be integrated to get,
\[ \ln {\psi_1(x) }  = \ln { \psi_2(x) } + c \]
and therefore,
\[ \psi_1(x) = C \psi_2(x)\]
where $C = e^c$ is an arbitrary constant. Thus we have shown that all energy eigenspaces are one-dimensional. Therefore, there is no degeneracy. Consider a potential with $V(-x) = V(x)$. Now given any bound state wavefunction $\psi(x)$, consider $\tilde{\psi}(x) = \psi(-x)$. Now, $\tilde{\psi}'(x) = - \psi'(-x)$ so $\tilde{\psi}''(x) = \psi''(-x)$. Plugging in $x \to -x$ into the Schrodinger equation,

\[E \psi(-x) = -\frac{\hbar^2}{2m} \left[\parsq{}{x'} \psi(x') \right]_{x' = -x} + V(-x) \psi(-x) = -\frac{\hbar^2}{2m} \psi''(-x) + V(-x) \psi(-x) \]  
Replacing $\psi(-x) = \tilde{\psi}(x)$ and $\tilde{\psi}''(x) = \psi''(-x)$ and $V(-x) = V(x)$ we get,
\[E \tilde{\psi}(x) = -\frac{\hbar^2}{2m} \tilde{\psi}''(x) + V(x) \tilde{\psi}(x) \] 
Therefore, $\tilde{\psi}$ is a bound state with the same energy as $\psi(x)$. By the above proposition, $\tilde{\psi}(x) = \lambda \psi(x)$ for some constant $\lambda$. Thus, $\psi(-x) = \lambda \psi(x)$ so plugging in $-x$ we get, $\psi(x) = \psi(-(-x)) = \lambda \psi(-x) = \lambda^2 \psi(x)$. Therefore, $\lambda^2 = 1$ so $\lambda = \pm 1$.
Finally, we have, $\psi(-x) = \pm \psi(x)$. In fact, every bound state is even or odd. This is actually a stronger statement than we need. Given that $\psi$ and $\tilde{\psi}$ are eigenstates with equal energies, then the combinations \[ \psi_{\pm}(x) =  \frac{1}{\sqrt{2}} \left[ \psi(x) \pm \tilde{\psi}(x) \right] \] are also eigenstates with the same energy. These states have the property that, \[ \psi_{\pm}(-x) = \frac{1}{\sqrt{2}} \left[ \psi(-x) \pm \psi(-(-x)) \right] = \frac{1}{\sqrt{2}} \pm \left[ \psi(x) \pm \psi(-x) \right] = \pm \psi_{\pm}(x)\]
so we can choose even or odd functions in the eigenspace. However, one of these combinations is guaranteed to be zero because we have proven that there cannot be any degeneracy in the bound eigenstates and therefore the function $\psi$ is, in fact, already even or odd. 

\subsection*{(b)}

Suppose the potential is given by a symmetric double well so that we can write $V(x) = V_L(x) + V_R(x)$ with $V_L(x) = V_R(-x)$ and for $x > 0$ we have $V_L(x) = 0$. We approximate the energy eigenstates with the functions,
\[ \pspm(x) = \frac{1}{\sqrt{2}} \left[ \pL(x) \pm \pR(x) \right]\]
where $\pL(x) = \phi^R_n(-x)$ is the normalized real $n^{\mathrm{th}}$ energy eigenstate of the potential $V_L$ (energy eigenstates in 1D can always be choosen to be real functions). Both functions $\pspm$ solve the Schrodinger equation with the same potential so we can apply the same trick used above. Consider the function,

\begin{align*}
\deriv{}{x} \left[ \dpsp \psm - \psp \dpsm \right] & = \ddpsp \psm + \dpsp \dpsm - \dpsp \dpsm - \psm \ddpsm \\ &= \ddpsp \psm - \psp \ddpsm 
\end{align*} 
However, from the Schrodinger equation,
\[ E_{\pm} \pspm(x) = - \frac{\hbar^2}{2m} \parsq{}{x} \pspm(x) + V(x) \pspm(x) \]
Therefore, 
\[ (\pspm)''(x) = - \frac{2m}{\hbar^2} \Big( E_{\pm} - V(x) \Big) \pspm(x)\]
so pluggin in,
\begin{align*}
\deriv{}{x} \left[ \dpsp \psm - \psp \dpsm \right] & = - \frac{2m}{\hbar^2} \Big( E_{+} - V(x) \Big) \psp(x) \psm(x) + \frac{2m}{\hbar^2} \Big( E_{\pm} - V(x) \Big) \psm(x) \psp(x) \\
& = \frac{2m}{\hbar^2} \Big( E_{-} - E_{+} \Big) \psp(x) \psm(x) = \frac{2m}{\hbar^2} \Delta E_n \: \cdot \: \frac{1}{2} \left[ \pL(x) + \pR(x) \right] \left[ \pL(x) - \pR(x) \right]
\end{align*} 
The left hand side is a total derivate so we wish to integrate this quantity. However, integrating over the entire range would yield $0 = 0$ because these functions are bound states and thus have vanishing value and derivative at infinity and the contribution from $(\pL)^2$ and $- (\pR)^2$ will exactly cancel on the left. Therefore we integrate,
\begin{align*}
\int_{-\infty}^{0} \deriv{}{x} \left[ \dpsp \psm - \psp \dpsm \right] \d{x} & = \frac{m \Delta E_n}{\hbar^2} \int_{-\infty}^{0} \left[(\pL(x))^2 - \pL(x) \pR(x) + \pR(x) \pL(x) - (\pR(x))^2 \right] \d{x} \\
\dpsp(0) \psm(0) - \psp(0) \dpsm(0) & \approx \frac{m \Delta E_n}{\hbar^2}  \int_{-\infty}^{0} (\pL(x))^2 \d{x} \approx \frac{m \Delta E_n}{\hbar^2} 
\end{align*}
Where I have ignored the terms proportional to $\pR$ in the region $x < 0$ because it is exponentially small and ignored the probability in the $x > 0$ exponential tail of $\pL$. Finally, because $\psm$ is and odd function, $\psm(0) = 0$ so,
\[- \psp(0) \dpsm(0) = \frac{m \Delta E_n}{\hbar^2} \]
It remains to determine the left hand side using the WKB approximation. The wavefunction $\pL$ solves the Schrodinger equation for a single potential well for which the WKB approximation is,
\[ \pL(x) = 
\begin{cases}
\frac{A}{\sqrt{|p|}} \exp \left[ \frac{1}{\hbar} \int_{x}^{x_1} |p(x)| \: \d{x} \right] & x \le x_1 \\
\frac{2A}{\sqrt{|p|}} \sin \left[ \frac{1}{\hbar} \int_{x_1}^{x} |p(x)| \: \d{x} + \frac{\pi}{4} \right] & x_1 \le x \le x_2 \\
\frac{A}{\sqrt{|p|}} \exp \left[ - \frac{1}{\hbar} \int_{x_2}^{x} |p(x)| \: \d{x} \right] & x \ge x_2 \\
\end{cases}\] 
where $x_1, x_2$ are the classical turning points and $p(x) = \sqrt{2m (E - V(x))}$ and the matching condition from the connection formula at $x = x_2$ imposes the Bohr-Sommerfeld quantization condition: 
\[ \oint p(x) \: \d{x} = 2 \pi \hbar \left(n + \frac{1}{2} \right) \]
Because the wells are separated, we know that $x_1 < x_2 < 0$ for every bound state. Therefore, 
\[ \pL(0) = \frac{A}{\sqrt{|p(0)|}} \exp \left[ - \frac{1}{\hbar} \int_{x_2}^{0} |p(x)| \: \d{x} \right] \quad \text{and} \quad (\pL)'(0) = - \frac{A \sqrt{|p(0)|}}{\hbar} \exp \left[ - \frac{1}{\hbar} \int_{x_2}^{0} |p(x)| \: \d{x} \right]\]
Finally, we determine the coefficient $A$ by normalization. Ignoring the exponentially decaying regions,  
\begin{align*}
\int_{-\infty}^{\infty} (\pL)^2(x) \d{x} & = \int_{x_1}^{x_2} \frac{4 A^2}{|p|} \sin^2 \left[ \frac{1}{\hbar} \int_{x_1}^{x} |p(x)| \: \d{x} + \frac{\pi}{4} \right] \d{x} = 1
\end{align*}
If we assume the sin term is rapidly variying which is necessary to make the WKB approximation in the first place, we can replace $\sin^2$ with its average value of $\frac{1}{2}$. Therefore, 
\[
\int_{-\infty}^{\infty} (\pL)^2(x) \d{x} = \int_{x_1}^{x_2} \frac{2 A^2}{|p|} \d{x} = 1 \quad \text{ thus } \quad A = \left( 2 \int_{x_1}^{x_2} \frac{1}{|p(x)|} \: \d{x} \right)^{-\frac{1}{2}} \]
Now, if we interpret $|p(x)|$ as the classical momentum of the particle, then this integral is, up to a factor of $m$, exactly the classical half period i.e. the time it takes for the particle to traverse the potential from one classical turning point to the  other. Thus, we can write,
\[ 2 \int_{x_1}^{x_2} \frac{1}{|p(x)|} \: \d{x} = \frac{1}{m} \oint \frac{1}{|v(x)|} \: \d{x} = \frac{T_n}{m} \quad \text{ therefore } \quad A = \sqrt{\frac{m}{T_n}}\]
Now, using the definitions of the symmetric and antisymmetric states,
\[ \pspm(x) = \frac{1}{\sqrt{2}} \left[ \pL(x) \pm \pR(x) \right] = \frac{1}{\sqrt{2}} \left[ \pL(x) \pm \pL(-x) \right] \quad \text{and} \quad (\pspm)'(x) = \frac{1}{\sqrt{2}} \left[ (\pL)'(x) \mp (\pL)'(-x) \right] \]
We have, $\psp(0) = \sqrt{2} \pL(0)$ and $\dpsm(0) = \sqrt{2} (\pL)'(0)$ whereas, $\psm(0) = \dpsp(0) = 0$. Therefore,
\[- \psp(0) \dpsm(0) = - 2 \pL(0) (\pL)'(0) = \frac{2A^2}{\hbar} \exp \left[ - \frac{2}{\hbar} \int_{x_2}^{0} |p(x)| \: \d{x} \right] = \frac{2m}{\hbar T_n} \exp \left[ - \frac{2}{\hbar} \int_{x_2}^{0} |p(x)| \: \d{x} \right]\]
Finally,
\[\Delta E_n = \frac{\hbar^2}{m} \left[- \psp(0) \dpsm(0) \right] = \frac{2\hbar}{T_n} \exp \left[ - \frac{2}{\hbar} \int_{x_2}^{0} |p(x)| \: \d{x} \right]\] 
This can be rewritten by using the classical frequency $\omega_n = \frac{2\pi}{T_n}$ and introducing the length $2L$ between the closest to zero negative and positive turning points i.e. $x_2 = -L$. Using these variables,
\[\Delta E_n = E_{-} - E_{+} =  \frac{\hbar \omega_n}{\pi} \exp \left[ - \frac{1}{\hbar} \int_{-L}^{L} |p(x)| \: \d{x} \right]\] 
where I can write the integral from $-L$ to $0$ as half the integral from $-L$ to $L$ because $V$ and therefore $p$ is an even function. This answer depends on two integrals which both include the energy levels of the single well system system which is determined by the quantization condition. 

\newpage

\section*{Problem 27.}

\subsection*{(a)}


Suppose the Hamiltonian for a spin-$\frac{1}{2}$ particle with $\vec{\mu} = \gamma \vec{S}$ is given by, 
\[\hamilt = - \gamma \vec{S} \cdot \vec{B}\]
This Hamiltonian is time independent so the time evolution operator is given by,
\[ \hat{U} = \exp{\left[i \frac{\gamma}{\hbar} \vec{S} \cdot \vec{B} t \right]} \]
We can identify this operator as a rotation about $\hat{B}$ by angle $\theta = - \gamma |B| \: t$. Let $\omega_{e} = \gamma |B|$ so $\theta = \omega_{e} t$. For $j = \frac{1}{2}$ we can expand this matrix explicitly.

\begin{align*}
\hat{U} & = \exp{\left[i \frac{\omega_{e} t }{\hbar} \vec{S} \cdot \hat{B} \right]} = \exp{\left[i \frac{\omega_{e} t }{2} \vec{\sigma} \cdot \hat{B} \right]} = \sum_{n = 0}^{\infty} \frac{1}{n!} \left(\frac{i \omega_{e} t}{2}\right)^n ( \vec{\sigma} \cdot \hat{B} )^n \\
& = I \cos{\left(\tfrac{1}{2} \omega_{e} t \right)} + i \vec{\sigma} \cdot \hat{B} \sin{\left(\tfrac{1}{2} \omega_{e} t \right)}
\end{align*} 
which holds because $(\vec{\sigma} \cdot \hat{B})^2 = I$. Now, we write out the matrix,
\begin{align*}
\hat{U} & = 
\begin{pmatrix}
1 & 0 \\
0 & 1
\end{pmatrix} 
\cos{\left(\tfrac{1}{2} \omega_{e} t \right)} + 
i \left[
\begin{pmatrix}
0 & 1 \\
1 & 0
\end{pmatrix} \hat{B}_x +
\begin{pmatrix}
0 & -i \\
i & 0
\end{pmatrix} \hat{B}_y  +
\begin{pmatrix}
1 & 0 \\
0 & -1
\end{pmatrix} \hat{B}_z \right] \sin{\left(\tfrac{1}{2} \omega_{e} t \right)} \\\\
& = 
\begin{pmatrix}
\cos{\left(\tfrac{1}{2} \omega_{e} t \right)} + i \hat{B}_z \sin{\left(\tfrac{1}{2} \omega_{e} t \right)} & (i \hat{B}_x + \hat{B}_y) \sin{\left(\tfrac{1}{2} \omega_{e} t \right)} \\ \\
(i \hat{B}_x - \hat{B}_y) \sin{\left(\tfrac{1}{2} \omega_{e} t \right)}  & \cos{\left(\tfrac{1}{2} \omega_{e} t \right)} - i \hat{B}_z \sin{\left(\tfrac{1}{2} \omega_{e} t \right)}
\end{pmatrix}
\end{align*}
With the initial state $\ket{\psi} = \ket{\tfrac{1}{2}, \tfrac{1}{2}}$, we get the time evolved state,

\[\ket{\psi(t)} = \hat{U} \ket{\tfrac{1}{2}, \tfrac{1}{2}} = \left[ \cos{\left(\tfrac{1}{2} \omega_{e} t \right)} + i \hat{B}_z \sin{\left(\tfrac{1}{2} \omega_{e} t \right)} \right] \ket{\tfrac{1}{2}, \tfrac{1}{2}} + \left[ (i \hat{B}_x - \hat{B}_y) \sin{\left(\tfrac{1}{2} \omega_{e} t \right)} \right] \ket{\tfrac{1}{2}, - \tfrac{1}{2}}  \] 

Let $\vec{B} = B_0 \hat{z}$ then $\hat{B}_z = 1$ and $\hat{B}_x = \hat{B}_y = 0$ so
\[ \hat{U} = 
\begin{pmatrix}
e^{i \omega_{e} t /2} & 0 \\ \\
0  & e^{-i \omega_{e} t /2}
\end{pmatrix}
\]
Let the initial state be, 
\[\ket{\psi_0} = a_{+} \ket{\tfrac{1}{2}, \tfrac{1}{2}} + a_{-} \ket{\tfrac{1}{2}, - \tfrac{1}{2}} \]
Then the evolved state is,
\begin{align*}
\ket{\psi(t)} = \hat{U} \ket{\psi_0} & = a_{+} e^{i \omega_{e} t /2} \ket{\tfrac{1}{2}, \tfrac{1}{2}} + \a_{-} e^{- i \omega_{e} t /2} \ket{\tfrac{1}{2}, - \tfrac{1}{2}} 
\end{align*}
The state rotates clockwise about the $z$-axis with rate given by the Larmor precession frequency, \[\omega_{e} = \gamma B_0\]


\subsection*{(b)}
Now, we introduce the time-dependent magnetic field,
\[\vec{B} = B_1 \left(\hat{x} \cos{\omega t} - \hat{y} \sin{\omega t} \right) + B_0 \hat{z} \]
Then, the Hamiltonian is,
\[\hamilt = - \gamma B_1 (\hat{S}_x \cos{\omega t} - \hat{S}_y \sin{\omega t}) - \gamma B_0 \hat{S}_z \]
Consider the transformed state,
\[ \ket{\psi'(t)} = e^{-i \omega t \hat{S}_z / \hbar} \ket{\psi(t)}\] 
First, we investigate the time-dependence of this state,
\begin{align*}
i\hbar \pderiv{}{t} \ket{\psi'(t)} & = \omega \hat{S}_z e^{-i \omega t \hat{S}_z / \hbar} \ket{\psi(t)} + e^{-i \omega t \hat{S}_z / \hbar} \hamilt \ket{\psi(t)}  \\ & = \left(\omega \hat{S}_z + e^{-i \omega t \hat{S}_z / \hbar} \hamilt e^{i \omega t \hat{S}_z / \hbar} \right) e^{-i \omega t \hat{S}_z / \hbar} \ket{\psi(t)} \\ & = \left(\omega \hat{S}_z + e^{-i \omega t \hat{S}_z / \hbar} \hamilt e^{i \omega t \hat{S}_z / \hbar} \right) \ket{\psi'(t)} 
\end{align*}

Now, we determine how the conjugated Hamiltonian depends on time,
\begin{align*}
i \hbar \pderiv{}{t} \left[ e^{-i \omega t \hat{S}_z / \hbar} \hamilt e^{i \omega t \hat{S}_z / \hbar} \right] & = e^{-i \omega t \hat{S}_z / \hbar} \left( \omega \hat{S}_z \hamilt - \omega  \hamilt \hat{S}_z + i \hbar \pderiv{}{t} \hamilt \right) e^{i \omega t \hat{S}_z / \hbar} \\ & = e^{-i \omega t \hat{S}_z / \hbar} \left( \omega [\hat{S}_z, \hamilt] + i \hbar \pderiv{}{t} \hamilt \right) e^{i \omega t \hat{S}_z / \hbar}
\end{align*} 

Calculating these terms,
\begin{align*}
\omega [\hat{S_z}, \hamilt] & = - \omega \gamma B_1 \cos{\omega t} [\hat{S}_z, \hat{S}_x] + \omega \gamma B_1 \sin{\omega t} [\hat{S}_z, \hat{S}_y] \\ & = - \omega \gamma B_1 i \hbar \hat{S}_y \cos{\omega t} - \omega \gamma B_1 i \hbar \hat{S}_x \sin{\omega t} = - i \hbar \omega \gamma B_1 (\hat{S}_x \sin{\omega t} + \hat{S}_y \cos{\omega t})
\end{align*}
Likewise,
\[ i \hbar \pderiv{}{t} \hamilt = i \hbar \omega \gamma B_1 \left(\hat{S}_x \sin{\omega t} + \hat{S}_y \cos{\omega t} \right) \]
Therefore, 
\[i \hbar \pderiv{}{t} \left[ e^{-i \omega t \hat{S}_z / \hbar} \hamilt e^{i \omega t \hat{S}_z / \hbar} \right] = 0\] 
so we can evaluate the operator at $t = 0$ and set it equal to that value at all time.
\[\left[ e^{-i \omega t \hat{S}_z / \hbar} \hamilt e^{i \omega t \hat{S}_z / \hbar} \right]_{t = 0} = \hamilt(0) = - \gamma B_1 \hat{S_x} - \gamma B_0 \hat{S}_z\]
Let $\omega_0 = \gamma B_0$ and $\omega_1 = \gamma B_1$. Then plugging into the equation for the time dependence of $\ket{\psi'(t)}$,
\[ i\hbar \pderiv{}{t} \ket{\psi'(t)} = \left( \omega \hat{S}_z - \omega_1 \hat{S}_x - \omega_0 \hat{S}_z \right) \ket{\psi'(t)} = \left( [\omega - \omega_0] \hat{S}_z - \omega_1 \hat{S}_x \right) \ket{\psi'(t)} \] 
The effective Hamiltonian is time-independent so the time evolution of this state is simply found by acting with the exponentiated operator,
\[ \hat{U} = \exp{\left[- \frac{i}{\hbar} \left( [\omega - \omega_0] \hat{S}_z - \omega_1 \hat{S}_x \right) t \right]} = \exp{\left[i \frac{\gamma}{\hbar} \left( [B_0 - B_0 \frac{\omega}{\omega_0} ] \hat{S}_z + B_1 \hat{S}_x \right) t \right]} \]
which can be rewritten as,
\[ \hat{U} = \exp{\left[i \frac{\gamma}{\hbar} \vec{S} \cdot \vec{B}_{\mathrm{eff}} \: t \right]} \]
where the effective magnetic field is,
\[\vec{B}_{\mathrm{eff}} = B_1 \hat{x} + \left(1 - \frac{\omega}{\omega_0} \right) B_0 \hat{z} \]
Therefore, we can use the explict matrix for $\hat{U}$ calculated above using $\omega_{e} = \gamma |B_{\mathrm{eff}}| $. 
Furthermore, since $\ket{\psi'(0)} = \ket{\psi(0)}$ we have $\ket{\psi'(t)} = \hat{U} \ket{\psi(0)}$ and $\ket{\psi(t)} = e^{i \omega t \hat{S}_z / \hbar} \ket{\psi'(t)}$. 
Thus,
\[ \ket{\psi(t)} = e^{i \omega t \hat{S}_z / \hbar} \: \hat{U} \ket{\psi(0)} = e^{i \omega t \hat{S}_z / \hbar} \: \exp{\left[i \frac{\gamma}{\hbar} \vec{S} \cdot \vec{B}_{\mathrm{eff}} \: t \right]} \ket{\psi(0)} \]
Now, we plug into the above formula for the matrix representations of the operators and states,

\[ \ket{\psi(t)} =
\begin{pmatrix}
e^{i \omega t /2} & 0 \\ \\
0  & e^{-i \omega t /2}
\end{pmatrix}
\begin{pmatrix}
\cos{\left(\tfrac{1}{2} \omega_{e} t \right)} + i \hat{n}^{\mathrm{eff}}_z \sin{\left(\tfrac{1}{2} \omega_{e} t \right)} & (i \hat{n}^{\mathrm{eff}}_x + \hat{n}^{\mathrm{eff}}_y) \sin{\left(\tfrac{1}{2} \omega_{e} t \right)} \\ \\
(i \hat{n}^{\mathrm{eff}}_x - \hat{n}^{\mathrm{eff}}_y) \sin{\left(\tfrac{1}{2} \omega_{e} t \right)}  & \cos{\left(\tfrac{1}{2} \omega_{e} t \right)} - i \hat{n}^{\mathrm{eff}}_z \sin{\left(\tfrac{1}{2} \omega_{e} t \right)}
\end{pmatrix}
\begin{pmatrix}
a_{+}  \\\\
a_{-}
\end{pmatrix}
\]
Where $\hat{n}^{\mathrm{eff}} = \frac{\vec{B}_{\mathrm{eff}}}{|B_{\mathrm{eff}}|}$. Explicitly, $|B_{\mathrm{eff}}| = \sqrt{B_1^2 + \left(1 - \frac{\omega}{\omega_0}\right)^2 B_0^2}$ so $\omega_{e} = \sqrt{\omega_1^2 + (\omega_0 - \omega)^2}$ and \[\hat{n}^{\mathrm{eff}} = \frac{\omega_1}{\omega_{e}} \hat{x} + \frac{\omega_0 - \omega}{\omega_{e}} \hat{z}\] 
Therefore, 

\[ \ket{\psi(t)} =
\begin{pmatrix}
\left( \cos{\left(\tfrac{1}{2} \omega_{e} t \right)} + i \frac{\omega_0 - \omega}{\omega_{e}} \sin{\left(\tfrac{1}{2} \omega_{e} t \right)} \right) e^{i \omega t /2} & i \frac{\omega_1}{\omega_{e}} \sin{\left(\tfrac{1}{2} \omega_{e} t \right)} e^{i \omega t /2} \\ \\
i \frac{\omega_1}{\omega_{e}} \sin{\left(\tfrac{1}{2} \omega_{e} t \right)} e^{-i \omega t /2}  & \left( \cos{\left(\tfrac{1}{2} \omega_{e} t \right)} - i \frac{\omega_0 - \omega}{\omega_{e}} \sin{\left(\tfrac{1}{2} \omega_{e} t \right)} \right) e^{-i \omega t /2}
\end{pmatrix}
\begin{pmatrix}
a_{+}  \\\\
a_{-}
\end{pmatrix}
\]
Therefore,
\begin{align*}
\ket{\psi(t)} & = \left[ \left( \cos{\left(\tfrac{1}{2} \omega_{e} t \right)} + i \frac{\omega_0 - \omega}{\omega_{e}} \sin{\left(\tfrac{1}{2} \omega_{e} t \right)} \right) e^{i \omega t /2} a_{+} + i \frac{\omega_1}{\omega_{e}} \sin{\left(\tfrac{1}{2} \omega_{e} t \right)} e^{i \omega t /2} a_{-} \right] \ket{+ \tfrac{1}{2}} \\  
& + 
\left[ i \frac{\omega_1}{\omega_{e}} \sin{\left(\tfrac{1}{2} \omega_{e} t \right)} e^{-i \omega t /2} a_{+} + \left( \cos{\left(\tfrac{1}{2} \omega_{e} t \right)} - i \frac{\omega_0 - \omega}{\omega_{e}} \sin{\left(\tfrac{1}{2} \omega_{e} t \right)} \right) e^{-i \omega t /2} a_{-} \right] \ket{- \tfrac{1}{2}}
\end{align*} 

\subsection*{(c)}

It is helpful to use both the above explict formula for the time evolution of a general state and the operator expression,
\[ \ket{\psi(t)} = e^{i \omega t \hat{S}_z / \hbar} \: \exp{\left[i \frac{\gamma}{\hbar} \vec{S} \cdot \vec{B}_{\mathrm{eff}} \: t \right]} \ket{\psi(0)} \]
which we can write in terms of rotation operators, 
\[\ket{\psi(t)} = \hat{R}_{z}(-\omega t) \hat{R}_{\hat{n}^{\mathrm{eff}}}(-\omega_{e} t) \ket{\psi(0)}\]
In general this motion can be described by considering the rotation operators alone. The first operator makes the spin precess in a cone about the direction $\hat{n}^{\mathrm{eff}}$ clockwise with frequency $\omega_{e}$. The second operator makes the entire cone rotate clockwise about the $z$-axis with frequency $\omega$. Therefore, as $\vec{B}$ rotates, the spin precesses in a cone around a direction close (only dispaced in the $z$-direction) to the current position of $\vec{B}$, so the cone rotates about $\hat{z}$ to track the rotation of $\vec{B}$.   \bigskip \\
Suppose the initial state were $\ket{\psi(0)} = \ket{+ \tfrac{1}{2}}$. Then, the time evolved state is given by,
\begin{align*}
\ket{\psi(t)} & = \left( \cos{\left(\tfrac{1}{2} \omega_{e} t \right)} + i \frac{\omega_0 - \omega}{\omega_{e}} \sin{\left(\tfrac{1}{2} \omega_{e} t \right)} \right) e^{i \omega t /2} \ket{+ \tfrac{1}{2}} +
i \frac{\omega_1}{\omega_{e}} \sin{\left(\tfrac{1}{2} \omega_{e} t \right)} e^{-i \omega t /2} \ket{- \tfrac{1}{2}}
\end{align*} 
Therefore, the probabilities to measure the spin up and down are,
\begin{align*}
P(+\tfrac{1}{2}) &= \bigg| \left( \cos{\left(\tfrac{1}{2} \omega_{e} t \right)} + i \frac{\omega_0 - \omega}{\omega_{e}} \sin{\left(\tfrac{1}{2} \omega_{e} t \right)} \right) e^{i \omega t /2} \bigg|^2 =  \cos^2{\left(\tfrac{1}{2} \omega_{e} t \right)} + \left(\frac{\omega_0 - \omega}{\omega_{e}} \right)^2 \sin^2{\left(\tfrac{1}{2} \omega_{e} t \right)} \\
P(-\tfrac{1}{2}) &= \bigg| i \frac{\omega_1}{\omega_{e}} \sin{\left(\tfrac{1}{2} \omega_{e} t \right)} e^{-i \omega t /2} \bigg|^2 =  \left(\frac{\omega_1}{\omega_{e}} \right)^2 \sin^2{\left(\tfrac{1}{2} \omega_{e} t \right)} 
\end{align*}
Which satisfy conservation of probability because $\omega_1^2 + (\omega_0 - \omega)^2 = \omega_e^2$ and thus, \[P(+\tfrac{1}{2}) + P(-\tfrac{1}{2}) =  \cos^2{\left(\tfrac{1}{2} \omega_{e} t \right)} + \sin^2{\left(\tfrac{1}{2} \omega_{e} t \right)} = 1\]
The maximum probability of finding the state flipped from spin up to spin down occurs when $\frac{1}{2} \omega_e t = n \pi + \frac{\pi}{2}$ so $t = \frac{2\pi}{\omega_e} \left( n + \frac{1}{2} \right)$. This maximum probability is, 
\[\left(\frac{\omega_1}{\omega_{e}} \right)^2 = \frac{\omega_1^2}{\omega_1^2 + (\omega_0 - \omega)^2} = \frac{1}{1 + \frac{1}{\omega_1^2} (\omega_0 - \omega)^2} \]
Therefore, the peak transition probability is maximized at $1$ when $\omega = \gamma B_0$, the Larmor frequency. As $\omega$ is varied, the peak transition probability decreases because the axis of the spin cone gains a component in the $\hat{z}$ direction and therefore cannot rotate a pure up state to a pure down state. \bigskip \\
Now, we will analyze the general time evolution in three cases: $\omega \ll \omega_0$, $\omega \approx \omega_0$ and $\omega \gg \omega_0$. 
\\ \\ \textbf{Case 1. $\omega \ll \omega_0$} \\
\[\vec{B}_{\mathrm{eff}} = B_1 \hat{x} + \left(1 - \frac{\omega}{\omega_0} \right) B_0 \hat{z} \approx B_1 \hat{x} + B_0 \hat{z} = \vec{B}(0)\]
Therefore, $\omega_1 = \gamma |\vec{B}|$ and thus the first rotation operator rotates the state in a cone about $\vec{B}(0)$ clockwise with exactly the Larmor frequency of the total magnetic field and the second operator rotates the entire cone slowly about the $z$-axis clockwise with frequency $\omega$. The spin precesses about $\vec{B}$ identically to the static field case except that the direction of the field and the precessing spin slowly rotates clockwise about $\hat{z}$ with frequency $\omega$. 
\\ \\ \textbf{Case 2. $\omega \approx \omega_0$} \\
\[\vec{B}_{\mathrm{eff}} = B_1 \hat{x} + \left(1 - \frac{\omega}{\omega_0} \right) B_0 \hat{z} \approx B_1 \hat{x} + B_0 \hat{z} = \vec{B}(0) \approx B_1 \hat{x}\]
Therefore, the first rotation operator rotates the state clockwise about the $x$-axis with frequency $\omega_e = \omega_1$. The second operator then rotates the state clockwise about the $z$-axis with frequency $\omega \approx \omega_0$. Thus, assuming that $B_1 \ll B_0$, the state makes a spiral motion about the $z$-axis on the surface of a sphere going from the upper hemisphere to the lower and then spirialling back up to the upper hemisphere.
\\ \\ \textbf{Case 3. $\omega \gg \omega_0$} \\  
\[\vec{B}_{\mathrm{eff}} = B_1 \hat{x} + \left(1 - \frac{\omega}{\omega_0} \right) B_0 \hat{z} \approx B_1 \hat{x} - \frac{\omega}{\gamma} \hat{z}\]
The motion in this case is similar to case 1 except that the spin does not rotate in a cone about $\vec{B}$ but about a completly different vector $\hat{n}^{\mathrm{eff}} = \frac{\vec{B}_{\mathrm{eff}}}{|\vec{B}_{\mathrm{eff}}|}$ which is determined by $\omega$. The state rotates clockwise in a cone about $\hat{n}^{\mathrm{eff}} = \frac{\vec{B}_{\mathrm{eff}}}{|\vec{B}_{\mathrm{eff}}|}$ with frequency $\omega_e$ while the cone as as a whole rotates about the $z$-axis with frequency $\omega$.    
\end{document}

