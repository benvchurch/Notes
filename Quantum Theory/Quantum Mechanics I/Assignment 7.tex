\documentclass[12pt]{extarticle}
\usepackage[utf8]{inputenc}
\usepackage[english]{babel}
\usepackage[a4paper, total={7.25in, 9.5in}]{geometry}
\usepackage{tikz-feynman}
\tikzfeynmanset{compat=1.0.0} 
\usepackage{subcaption}
\usepackage{float}
\floatplacement{figure}{H}
\usepackage{simpler-wick}
\usepackage{mathrsfs}  
\usepackage{dsfont}
\usepackage{relsize}
\usepackage{tikz-cd}
\DeclareMathAlphabet{\mathdutchcal}{U}{dutchcal}{m}{n}

\usepackage{cancel}



\newcommand{\field}{\hat{\Phi}}
\newcommand{\dfield}{\hat{\Phi}^\dagger}
 
\usepackage{amsthm, amssymb, amsmath, centernot}
\usepackage{slashed}
\newcommand{\notimplies}{%
  \mathrel{{\ooalign{\hidewidth$\not\phantom{=}$\hidewidth\cr$\implies$}}}}
 
\renewcommand\qedsymbol{$\square$}
\newcommand{\cont}{$\boxtimes$}
\newcommand{\divides}{\mid}
\newcommand{\ndivides}{\centernot \mid}

\newcommand{\Integers}{\mathbb{Z}}
\newcommand{\Natural}{\mathbb{N}}
\newcommand{\Complex}{\mathbb{C}}
\newcommand{\Zplus}{\mathbb{Z}^{+}}
\newcommand{\Primes}{\mathbb{P}}
\newcommand{\Q}{\mathbb{Q}}
\newcommand{\R}{\mathbb{R}}
\newcommand{\ball}[2]{B_{#1} \! \left(#2 \right)}
\newcommand{\Rplus}{\mathbb{R}^+}
\renewcommand{\Re}[1]{\mathrm{Re}\left[ #1 \right]}
\renewcommand{\Im}[1]{\mathrm{Im}\left[ #1 \right]}
\newcommand{\Op}{\mathcal{O}}

\newcommand{\invI}[2]{#1^{-1} \left( #2 \right)}
\newcommand{\End}[1]{\text{End}\left( A \right)}
\newcommand{\legsym}[2]{\left(\frac{#1}{#2} \right)}
\renewcommand{\mod}[3]{\: #1 \equiv #2 \: \mathrm{mod} \: #3 \:}
\newcommand{\nmod}[3]{\: #1 \centernot \equiv #2 \: mod \: #3 \:}
\newcommand{\ndiv}{\hspace{-4pt}\not \divides \hspace{2pt}}
\newcommand{\finfield}[1]{\mathbb{F}_{#1}}
\newcommand{\finunits}[1]{\mathbb{F}_{#1}^{\times}}
\newcommand{\ord}[1]{\mathrm{ord}\! \left(#1 \right)}
\newcommand{\quadfield}[1]{\Q \small(\sqrt{#1} \small)}
\newcommand{\vspan}[1]{\mathrm{span}\! \left\{#1 \right\}}
\newcommand{\galgroup}[1]{Gal \small(#1 \small)}
\newcommand{\bra}[1]{\left| #1 \right>}
\newcommand{\Oa}{O_\alpha}
\newcommand{\Od}{O_\alpha^{\dagger}}
\newcommand{\Oap}{O_{\alpha '}}
\newcommand{\Odp}{O_{\alpha '}^{\dagger}}
\newcommand{\im}[1]{\mathrm{im} \: #1}
\renewcommand{\ker}[1]{\mathrm{ker} \: #1}
\newcommand{\ket}[1]{\left| #1 \right>}
\renewcommand{\bra}[1]{\left< #1 \right|}
\newcommand{\inner}[2]{\left< #1 | #2 \right>}
\newcommand{\expect}[2]{\left< #1 \right| #2 \left| #1 \right>}
\renewcommand{\d}[1]{ \mathrm{d}#1 \:}
\newcommand{\dn}[2]{ \mathrm{d}^{#1} #2 \:}
\newcommand{\deriv}[2]{\frac{\d{#1}}{\d{#2}}}
\newcommand{\nderiv}[3]{\frac{\dn{#1}{#2}}{\d{#3^{#1}}}}
\newcommand{\pderiv}[2]{\frac{\partial{#1}}{\partial{#2}}}
\newcommand{\fderiv}[2]{\frac{\delta #1}{\delta #2}}
\newcommand{\parsq}[2]{\frac{\partial^2{#1}}{\partial{#2}^2}}
\newcommand{\topo}{\mathcal{T}}
\newcommand{\base}{\mathcal{B}}
\renewcommand{\bf}[1]{\mathbf{#1}}
\renewcommand{\a}{\hat{a}}
\newcommand{\adag}{\hat{a}^\dagger}
\renewcommand{\b}{\hat{b}}
\newcommand{\bdag}{\hat{b}^\dagger}
\renewcommand{\c}{\hat{c}}
\newcommand{\cdag}{\hat{c}^\dagger}
\newcommand{\hamilt}{\hat{H}}
\renewcommand{\L}{\hat{L}}
\newcommand{\Lz}{\hat{L}_z}
\newcommand{\Lsquared}{\hat{L}^2}
\renewcommand{\S}{\hat{S}}
\renewcommand{\empty}{\varnothing}
\newcommand{\J}{\hat{J}}
\newcommand{\lagrange}{\mathcal{L}}
\newcommand{\dfourx}{\mathrm{d}^4x}
\newcommand{\meson}{\phi}
\newcommand{\dpsi}{\psi^\dagger}
\newcommand{\ipic}{\mathrm{int}}
\newcommand{\tr}[1]{\mathrm{tr} \left( #1 \right)}
\newcommand{\C}{\mathbb{C}}
\newcommand{\CP}[1]{\mathbb{CP}^{#1}}
\newcommand{\Vol}[1]{\mathrm{Vol}\left(#1\right)}

\newcommand{\Tr}[1]{\mathrm{Tr}\left( #1 \right)}
\newcommand{\Charge}{\hat{\mathbf{C}}}
\newcommand{\Parity}{\hat{\mathbf{P}}}
\newcommand{\Time}{\hat{\mathbf{T}}}
\newcommand{\Torder}[1]{\mathbf{T}\left[ #1 \right]}
\newcommand{\Norder}[1]{\mathbf{N}\left[ #1 \right]}
\newcommand{\Znorm}{\mathcal{Z}}
\newcommand{\EV}[1]{\left< #1 \right>}
\newcommand{\interact}{\mathrm{int}}
\newcommand{\covD}{\mathcal{D}}
\newcommand{\conj}[1]{\overline{#1}}

\newcommand{\SO}[2]{\mathrm{SO}(#1, #2)}
\newcommand{\SU}[2]{\mathrm{SU}(#1, #2)}

\newcommand{\anticom}[2]{\left\{ #1 , #2 \right\}}


\newcommand{\pathd}[1]{\! \mathdutchcal{D} #1 \:}

\renewcommand{\theenumi}{(\alph{enumi})}


\renewcommand{\theenumi}{(\alph{enumi})}

\newcommand{\atitle}[1]{\title{% 
	\large \textbf{Physics GR8048 Quantum Field Theory II
	\\ Assignment \# #1} \vspace{-2ex}}
\author{Benjamin Church }
\maketitle}

\newcommand{\atitleIII}[1]{\title{% 
	\large \textbf{Physics GR8049 Quantum Field Theory III
	\\ Assignment \# #1} \vspace{-2ex}}
\author{Benjamin Church }
\maketitle}

\theoremstyle{definition}
\newtheorem{theorem}{Theorem}[section]
\newtheorem{definition}{definition}[section]
\newtheorem{lemma}[theorem]{Lemma}
\newtheorem{proposition}[theorem]{Proposition}
\newtheorem{corollary}[theorem]{Corollary}
\newtheorem{example}[theorem]{Example}
\newtheorem{remark}[theorem]{Remark}


\newcommand{\Jn}{\J_{\hat{n}}}


\begin{document}
\atitle{7}
 
\section*{Problem 22.}
Consider a spin-$j$ particle with $j = \tfrac{1}{2}$. A general state can be written in the form, 
\[\ket{\psi} = a_{+} \ket{\tfrac{1}{2}, \tfrac{1}{2}} + a_{-} \ket{\tfrac{1}{2}, -\tfrac{1}{2}}\] 
We first consider the eigenvectors of the operator $\Jn = \vec{J} \cdot \hat{n}$ for some unit vector $\hat{n}$. For $j = \tfrac{1}{2}$, these eigenvectors can be easily found from the matrix representation of $\Jn$. 
\begin{align*}
\Jn = \frac{\hbar}{2} \hat{n}_x 
\begin{pmatrix}
0 & 1 \\
1 & 0
\end{pmatrix}
+ \frac{\hbar}{2} \hat{n}_y
\begin{pmatrix}
0 & -i \\
i & 0
\end{pmatrix}
+ \frac{\hbar}{2} \hat{n}_z
\begin{pmatrix}
1 & 0 \\
0 & -1
\end{pmatrix}
= \frac{\hbar}{2}
\begin{pmatrix}
n_z & n_x - i n_y \\
n_x + i n_y & -n_z
\end{pmatrix}
\end{align*}
We know by rotational symmetry that the eigenvalues of this matrix are $\pm \frac{\hbar}{2}$. Thus, to find the eigenvectors, consider the matrix equations,
\begin{align*}
\frac{2}{\hbar} \left(\Jn - I \frac{\hbar}{2}\right) \ket{\hat{n} +} = 
\begin{pmatrix}
n_z - 1 & n_x - i n_y \\
n_x + i n_y & - n_z - 1
\end{pmatrix}
\begin{pmatrix}
a_{+} \\
a_{-}
\end{pmatrix}
= 0
\end{align*}
Thus, $(n_z - 1) a_{+} + (n_x - i n_y) a_{-} = 0$. Take $a_{+} = \frac{1}{\sqrt{2(1- n_z)}} (n_{x} - i n_{y})$ and $a_{-} = \frac{1}{\sqrt{2(1- n_z)}} (1 - n_z)$ such that $|a_{+}|^2 + |a_{-}|^2 = 1$. These also satisfy the second row because, 
\begin{align*}
(n_x + i n_y)a_{+} - (n_z + 1) a_{-} & = \frac{1}{\sqrt{2 ( 1 - n_z)}} \left[(n_x + i n_y)(n_x - i n_y) - (1 - n_z)(1 + n_z) \right] \\ & = \frac{1}{\sqrt{2 ( 1 - n_z)}} \left[n_x^2 + n_y^2 + n_z^2 - 1 \right] = 0
\end{align*}
Therefore, 
\[ \ket{\hat{n} +} = \frac{n_{x} - i n_{y}}{\sqrt{2(1- n_z)}} \ket{\tfrac{1}{2}, \tfrac{1}{2}} + \frac{1 - n_z}{\sqrt{2(1- n_z)}} \ket{\tfrac{1}{2}, - \tfrac{1}{2}}\]
Similarly for the spin down state, we can consider $\hat{n} \mapsto -\hat{n}$ and look at the spin up state. This state will equal the spin down state in the original direction up to phase. Thus,
\[ \ket{\hat{n} -} = \frac{-n_{x} + i n_{y}}{\sqrt{2(1 + n_z)}} \ket{\tfrac{1}{2}, \tfrac{1}{2}} + \frac{1 + n_z}{\sqrt{2(1 + n_z)}} \ket{\tfrac{1}{2}, - \tfrac{1}{2}}\] 

\subsection*{(a)}
Consider the state $\ket{\psi} = \ket{\tfrac{1}{2}, \tfrac{1}{2}}$ such that $\bra{\psi} \J_3 \ket{\psi} = \frac{\hbar}{2}$ and the operator $\J' = \J_3 \cos{\theta} + \J_2 \sin{\theta}$. The eigenvectors of $\J'$ correspond to $\hat{n} = (0, \sin{\theta}, \cos{\theta})$. Thus, 
\[\ket{\hat{n} + } = \frac{- i \sin{\theta} }{\sqrt{2(1 - \cos{\theta})}} \ket{\tfrac{1}{2}, \tfrac{1}{2}} + \frac{1 - \cos{\theta}}{\sqrt{2(1 - \cos{\theta})}} \ket{\tfrac{1}{2}, - \tfrac{1}{2}} = -i \cos{\left(\tfrac{\theta}{2} \right)}\ket{\tfrac{1}{2}, \tfrac{1}{2}} + \sin{\left( \tfrac{\theta}{2} \right)} \ket{\tfrac{1}{2}, -\tfrac{1}{2}} \]
by half-angle formulae. Therefore, the probability to have spin up with respect to $\J'$ is 
\[|\inner{\hat{n}+}{\tfrac{1}{2}, \tfrac{1}{2}}|^2 = \cos^2{\left(\tfrac{\theta}{2}\right)}\]

\subsection*{(b)}

Since we can always find an eigenstate of $\Jn = \vec{J} \cdot \hat{n}$, consider this state $\ket{\hat{n} + }$. By definition, $\hat{n} \cdot \vec{J} \ket{\hat{n} + } = \frac{\hbar}{2} \ket{\hat{n} + }$ and thus $\bra{\hat{n} +} \Jn \ket{\hat{n} + } = \frac{\hbar}{2}$. Futhermore, 
\begin{align*}
\bra{\hat{n} +} \J_1 \ket{\hat{n} + } & = \frac{\hbar}{2}
\begin{pmatrix}
a_{+}^* & a_{-}^* 
\end{pmatrix}
\begin{pmatrix}
0 & 1 \\
1 & 0
\end{pmatrix}
\begin{pmatrix}
a_{+} \\
a_{-}
\end{pmatrix} = \frac{\hbar}{2} (a_{+}^* a_{-} + a_{-}^* a_{+}) \\ & = \frac{\hbar}{2} \frac{1}{2(1 - n_z)} \Big[(n_x + i n_y)(1 - n_z) + (1 - n_z)(n_x - i n_y) \Big] \\
& = \frac{\hbar}{2} \frac{1}{2(1 - n_z)} \Big[2 n_x(1 - n_z) \Big] = \frac{\hbar}{2} n_x
\end{align*}
Similarly, 
\begin{align*}
\bra{\hat{n} +} \J_2 \ket{\hat{n} + } & = \frac{\hbar}{2}
\begin{pmatrix}
a_{+}^* & a_{-}^* 
\end{pmatrix}
\begin{pmatrix}
0 & -i \\
i & 0
\end{pmatrix}
\begin{pmatrix}
a_{+} \\
a_{-}
\end{pmatrix} = -i \frac{\hbar}{2} (a_{+}^* a_{-} - a_{-}^* a_{+}) \\ & = \frac{\hbar}{2} \frac{-i}{2(1 - n_z)} \Big[(n_x + i n_y)(1 - n_z) - (1 - n_z)(n_x - i n_y) \Big] \\
& = \frac{\hbar}{2} \frac{-i}{2(1 - n_z)} \Big[2i n_y(1 - n_z) \Big] = \frac{\hbar}{2} n_y
\end{align*}
And finally,
\begin{align*}
\bra{\hat{n} +} \J_2 \ket{\hat{n} + } & = \frac{\hbar}{2}
\begin{pmatrix}
a_{+}^* & a_{-}^* 
\end{pmatrix}
\begin{pmatrix}
1 & 0 \\
0 & -1
\end{pmatrix}
\begin{pmatrix}
a_{+} \\
a_{-}
\end{pmatrix} = \frac{\hbar}{2} (a_{+}^* a_{+} - a_{-}^* a_{-}) \\ & = \frac{\hbar}{2} \frac{1}{2(1 - n_z)} \Big[(n_x + i n_y)(n_x - n_y) - (1 - n_z)^2 \Big] \\
& = \frac{\hbar}{2} \frac{1}{2(1 - n_z)} \Big[n_x^2 + n_y^2 - 1 + 2n_z - n_z^2  \Big] = \frac{\hbar}{2} \frac{1}{2(1 - n_z)} \Big[ 2n_z - 2n_z^2  \Big] =  \frac{\hbar}{2} n_z
\end{align*} 
Therefore, 
\[\bra{\hat{n} +} \vec{J} \ket{\hat{n} + } = \frac{\hbar}{2} \hat{n}\]
Thus, writing $\hat{n} = (\sin{\theta}\cos{\phi}, \sin{\theta} \sin{\phi}, \cos{\theta})$ we conclude that the desired state is,
\begin{align*}
\ket{\hat{n} +} & = 
\frac{\sin{\theta}\cos{\phi} - i  \sin{\theta} \sin{\phi}}{\sqrt{2(1-  \cos{\theta})}} \ket{\tfrac{1}{2}, \tfrac{1}{2}} + \frac{1 -  \cos{\theta}}{\sqrt{2(1-  \cos{\theta})}} \ket{\tfrac{1}{2}, - \tfrac{1}{2}} 
\\ & =
\frac{\sin{\theta}}{\sqrt{2(1-  \cos{\theta})}} e^{- i \phi} \ket{\tfrac{1}{2}, \tfrac{1}{2}} + \sqrt{\frac{1 -  \cos{\theta}}{2}} \ket{\tfrac{1}{2}, - \tfrac{1}{2}} 
\\ & = 
\cos{\left(\tfrac{\theta}{2} \right)} e^{- i \phi} \ket{\tfrac{1}{2}, \tfrac{1}{2}} + \sin{\left(\tfrac{\theta}{2} \right)} \ket{\tfrac{1}{2}, - \tfrac{1}{2}} 
\end{align*}
For future convinience and to exhibit the symmety in the components, I will multiply this state by the total phase $e^{i \phi/2}$. Thus, $a_{+} = \cos{\left(\tfrac{\theta}{2} \right)} e^{- i \phi/2}$ and $a_{-} = \sin{\left(\tfrac{\theta}{2} \right)} e^{+ i \phi/2}$ so we use the notation, 

\[\ket{\psi(\theta, \phi)} =  \cos{\left(\tfrac{\theta}{2} \right)} e^{- i \phi/2} \ket{\tfrac{1}{2}, \tfrac{1}{2}} + \sin{\left(\tfrac{\theta}{2} \right)} e^{i \phi/2} \ket{\tfrac{1}{2}, - \tfrac{1}{2}} \]

\subsection*{(c)} 

Suppose the Hamiltonian is given by, 
\[\hamilt = - \frac{g e}{2 m c} \vec{J} \cdot \vec{B}\]
This Hamiltonian is time independent so the time evolution operator is given by,
\[ \hat{U} = \exp{\left[i \frac{g e}{2 m c \hbar} \vec{J} \cdot \vec{B} t \right]} \]
We can identify this operator as a rotation about $\hat{B}$ by angle $\theta = - \frac{g e |B|}{2mc} \: t$. Let $\omega_L = \frac{g e |B|}{2mc}$ so $\theta = \omega_L t$. For $j = \frac{1}{2}$ we can expand this matrix explicitly.

\begin{align*}
\hat{U} & = \exp{\left[i \frac{\omega_L t }{\hbar} \vec{J} \cdot \hat{B} \right]} = \exp{\left[i \frac{\omega_L t }{2} \vec{\sigma} \cdot \hat{B} \right]} = \sum_{n = 0}^{\infty} \frac{1}{n!} \left(\frac{i \omega_L t}{2}\right)^n ( \vec{\sigma} \cdot \hat{B} )^n \\
& = I \cos{\left(\tfrac{1}{2} \omega_L t \right)} + i \vec{\sigma} \cdot \hat{B} \sin{\left(\tfrac{1}{2} \omega_L t \right)}
\end{align*} 
which holds because $(\vec{\sigma} \cdot \hat{B})^2 = I$. Now, we write out the matrix,
\begin{align*}
\hat{U} & = 
\begin{pmatrix}
1 & 0 \\
0 & 1
\end{pmatrix} 
\cos{\left(\tfrac{1}{2} \omega_L t \right)} + 
i \left[
\begin{pmatrix}
0 & 1 \\
1 & 0
\end{pmatrix} \hat{B}_x +
\begin{pmatrix}
0 & -i \\
i & 0
\end{pmatrix} \hat{B}_y  +
\begin{pmatrix}
1 & 0 \\
0 & -1
\end{pmatrix} \hat{B}_z \right] \sin{\left(\tfrac{1}{2} \omega_L t \right)} \\\\
& = 
\begin{pmatrix}
\cos{\left(\tfrac{1}{2} \omega_L t \right)} + i \hat{B}_z \sin{\left(\tfrac{1}{2} \omega_L t \right)} & (i \hat{B}_x + \hat{B}_y) \sin{\left(\tfrac{1}{2} \omega_L t \right)} \\ \\
(i \hat{B}_x - \hat{B}_y) \sin{\left(\tfrac{1}{2} \omega_L t \right)}  & \cos{\left(\tfrac{1}{2} \omega_L t \right)} - i \hat{B}_z \sin{\left(\tfrac{1}{2} \omega_L t \right)}
\end{pmatrix}
\end{align*}
With the initial state $\ket{\psi} = \ket{\tfrac{1}{2}, \tfrac{1}{2}}$, we get the time evolved state,

\[\ket{\psi(t)} = \hat{U} \ket{\tfrac{1}{2}, \tfrac{1}{2}} = \left[ \cos{\left(\tfrac{1}{2} \omega_L t \right)} + i \hat{B}_z \sin{\left(\tfrac{1}{2} \omega_L t \right)} \right] \ket{\tfrac{1}{2}, \tfrac{1}{2}} + \left[ (i \hat{B}_x - \hat{B}_y) \sin{\left(\tfrac{1}{2} \omega_L t \right)} \right] \ket{\tfrac{1}{2}, - \tfrac{1}{2}}  \]

\subsection*{(d)} 

Let $\vec{B} = B \hat{z}$ then $\hat{B}_z = B$ and $\hat{B}_x = \hat{B}_y = 0$ so
\[ \hat{U} = 
\begin{pmatrix}
e^{i \omega_L t /2} & 0 \\ \\
0  & e^{-i \omega_L t /2}
\end{pmatrix}
\]
Let the initial state be, 
\[\ket{\psi_0} = \ket{\psi(\theta, \phi)} =  \cos{\left(\tfrac{\theta}{2} \right)} e^{- i \phi/2} \ket{\tfrac{1}{2}, \tfrac{1}{2}} + \sin{\left(\tfrac{\theta}{2} \right)} e^{i \phi/2} \ket{\tfrac{1}{2}, - \tfrac{1}{2}} \]
Then the evolved state is,
\begin{align*}
\ket{\psi(t)} = \hat{U} \ket{\psi_0} & = \cos{\left(\tfrac{\theta}{2} \right)} e^{- i \phi/2} e^{i \omega_L t /2} \ket{\tfrac{1}{2}, \tfrac{1}{2}} + \sin{\left(\tfrac{\theta}{2} \right)} e^{i \phi/2} e^{- i \omega_L t /2} \ket{\tfrac{1}{2}, - \tfrac{1}{2}} \\ & = 
\cos{\left(\tfrac{\theta}{2} \right)} e^{- i (\phi - \omega_L t) /2} \ket{\tfrac{1}{2}, \tfrac{1}{2}} + \sin{\left(\tfrac{\theta}{2} \right)} e^{i (\phi - \omega_L t)/2}  \ket{\tfrac{1}{2}, - \tfrac{1}{2}} \\ & = \ket{\psi(\theta, \phi - \omega_L t)}
\end{align*}
This is exactly the classical motion. The state rotates clockwise about the $z$-axis with rate given by the Larmor precession frequency $\omega_L = \frac{g e |B|}{2 m c}$


\section*{Problem 23.}

\subsection*{(a)}

Consider any normalized $j = \tfrac{1}{2}$ spin state. Define $\hat{n} = \bra{\psi} \vec{J} \ket{\psi} \frac{2}{\hbar}$.
\begin{align*}
\bra{\psi} \J_1 \ket{\psi} & = \frac{\hbar}{2}
\begin{pmatrix}
a_{+}^* & a_{-}^* 
\end{pmatrix}
\begin{pmatrix}
0 & 1 \\
1 & 0
\end{pmatrix}
\begin{pmatrix}
a_{+} \\
a_{-}
\end{pmatrix} = \frac{\hbar}{2} (a_{+}^* a_{-} + a_{-}^* a_{+}) = \frac{\hbar}{2} n_x
\end{align*}
Similarly, 
\begin{align*}
\bra{\psi} \J_2 \ket{\psi} & = \frac{\hbar}{2}
\begin{pmatrix}
a_{+}^* & a_{-}^* 
\end{pmatrix}
\begin{pmatrix}
0 & -i \\
i & 0
\end{pmatrix}
\begin{pmatrix}
a_{+} \\
a_{-}
\end{pmatrix} = -i \frac{\hbar}{2} (a_{+}^* a_{-} - a_{-}^* a_{+}) = \frac{\hbar}{2} n_y
\end{align*}
And finally,
\begin{align*}
\bra{\psi} \J_2 \ket{\psi} & = \frac{\hbar}{2}
\begin{pmatrix}
a_{+}^* & a_{-}^* 
\end{pmatrix}
\begin{pmatrix}
1 & 0 \\
0 & -1
\end{pmatrix}
\begin{pmatrix}
a_{+} \\
a_{-}
\end{pmatrix} = \frac{\hbar}{2} (a_{+}^* a_{+} - a_{-}^* a_{-}) = \frac{\hbar}{2} n_z
\end{align*}
Now, consider the length of $\hat{n}$,
\begin{align*}
\hat{n}^2 & = n_x^2 + n_y^2 + n_z^2 = (a_{+}^* a_{-} + a_{-}^* a_{+})^2 + (\tfrac{1}{i} \left[a_{+}^* a_{-} - a_{-}^* a_{+}\right])^2 + (a_{+}^* a_{+} - a_{-}^* a_{-})^2 \\ & = \left(2 \mathfrak{Re}\left[a_{+}^* a_{-}\right] \right)^2 + \left(2 \mathfrak{Im}\left[a_{+}^* a_{-}\right] \right)^2 + |a_{+}|^4 - 2 |a_{+}|^2 |a_{-}|^2 + |a_{-}|^4 \\
& = 4 |a_{+}^* a_{-}|^2 + |a_{+}|^4 - 2 |a_{+}|^2 |a_{-}|^2 + |a_{-}|^4 = 4 |a_{+}|^2 |a_{-}|^2 + |a_{+}|^4 - 2 |a_{+}|^2 |a_{-}|^2 + |a_{-}|^4 \\
& = |a_{+}|^4 + 2 |a_{+}|^2 |a_{-}|^2 + |a_{-}|^4 = (|a_{+}|^2 + |a_{-}|^2)^2 = 1
\end{align*}
where the last equality holds by the fact that $\ket{\psi}$ is normalized so $|a_{+}|^2 + |a_{-}|^2 = 1$. 

\subsection*{(b)}

No! Consider the the $j = 1$ and $m = 0$ state. Then, 
\begin{align*}
\bra{1, 0} \J_3 \ket{1, 0} & = 0 \\
\bra{1, 0} \J_2 \ket{1, 0} & = \bra{1, 0} \frac{1}{2i}(\J_{+} - \J_{-}) \ket{1, 0} = \frac{1}{2i} \left( \sqrt{2} \inner{1, 0}{1,1} - \sqrt{2} \inner{1, 0}{1, -1} \right) = 0 \\
\bra{1, 0} \J_1 \ket{1, 0} & = \bra{1, 0} \frac{1}{2}(\J_{+} + \J_{-}) \ket{1, 0} = \frac{1}{2} \left( \sqrt{2} \inner{1, 0}{1,1} + \sqrt{2} \inner{1, 0}{1, -1} \right) = 0 
\end{align*}
Therefore, $\bra{1, 0} \vec{J} \ket{1, 0}$ has zero length and therefore no multiple of it is a unit vector. 

\section*{Problem 24.}

Let $F(x, y, z) = \sum\limits_{i, j, k}^N c_{ijk} x^i y^j z^k$ where the sum runs over values such that $i + j + k = N$. 

\subsection*{(a)}

$\vec{L} = - i \hbar \vec{r} \times \nabla$ and thus, $\L_i = - i \hbar \epsilon_{ijk} r_j \partial_k$. If we act on a monomial with the operator $\L_i$,
\[ \L_1 x^a y^b z^c = - i \hbar (y \partial_z - z \partial_y) \partial_k x^a y^b z^c = - i \hbar (c y x^a y^b z^{c-1} - b z x^a y^{a-1} z^c) = - i \hbar (c x^a y^{b + 1} z^{c - 1} - b x^a y^{b - 1} z^{c + 1}) \]    
If $a + b + c = N$ then the final polynomial will have each term of overall order $N$ because $a + (b + 1) + (c - 1) = a + (b - 1) + (c + 1) = a + b + c = N$. Therefore, $\L_1$ acting on monomials produces another polynomial in this subspace. The other components of $\L$ and thus $\Lsquared$ act similarly to produce homogeneous polynomials from monomials of the same order. Therefore, since a homogeneous polynomial is a sum of monomials of equal order, each component of $\L_i$ acts on each term to generate two terms of the same order. Thus, the overall order of the polynomial is preserved so order $N$ homogeneous polynomials are mapped into other order $N$ homogeneous polynomials. Furthermore, the rotation operator about $\hat{n}$ is given by  
\[R(\hat{n}, \theta) = e^{- \frac{i}{\hbar} \vec{L} \cdot \hat{n} \theta} = I - \frac{i}{\hbar} \vec{L} \cdot \hat{n} \theta + \frac{1}{2} \left(\frac{i}{\hbar} \vec{L} \cdot \hat{n} \theta \right)^2 + \frac{1}{3!} \left(\frac{i}{\hbar} \vec{L} \cdot \hat{n} \theta \right)^3 + \cdots  \]
Therefore, acting with $R(\hat{n}, \theta)$ preserves the order and homogeneity of the polynomials because each term in the series preserves the property.    

\subsection*{(b)}

Consider the function, 
\[\psi_{0,0}(r, \theta, \phi) = r^N Y_{0, 0}(\theta, \phi)\]
Because $\L_i$ only acts on angular functions, we can move the $\L_i$ operators through the $r^N$ to act only on the spherical harmonic. Thus, $\psi_{0,0}$ is a state with $\ell = 0$. Now, $Y_{0,0} = \frac{1}{2\sqrt{\pi}}$ so 
\[\psi_{0,0}(x, y, z) = \frac{1}{2\sqrt{\pi}} (x^2 + y^2 + z^2)^{N/2}\]
If $N$ is even, we can expand this expession as a homogeneous polynomial of order $N$ using the trinomial coeficients.  

\subsection*{(c)}

Consider the functions,
\[\psi_{1, m}(r, \theta, \phi) = r^{N} Y_{1, m}(\theta, \phi)\]
for values $m = +1, 0, -1$. These are $\ell = 1$ multiplet eigenstates of angular momentum because the angular momentum operators commute with the radial distance $r$. We write these functions out explicitly in angular functions, 
\begin{align*}
\psi_{1, 1}(r, \theta, \phi) & = - r^N \frac{1}{2} \sqrt{\frac{3}{2\pi}} \sin{\theta} e^{i \phi} = - r^{N-1} \frac{1}{2} \sqrt{\frac{3}{2\pi}} \: r \sin{\theta} (\cos{\phi} + i \sin{\phi}) \\
\psi_{1, 0}(r, \theta, \phi) & = r^N \frac{1}{2} \sqrt{\frac{3}{\pi}} \cos{\theta} = r^{N-1} \frac{1}{2} \sqrt{\frac{3}{\pi}} \: r \cos{\theta} \\
\psi_{1, -1}(r, \theta, \phi) & = r^N \frac{1}{2} \sqrt{\frac{3}{2\pi}} \sin{\theta} e^{-i \phi} = r^{N-1} \frac{1}{2} \sqrt{\frac{3}{2\pi}} \: r \sin{\theta} (\cos{\phi} - i \sin{\phi}) 
\end{align*}
If $N$ is odd so $N - 1$ is even, we can write these functions as homogeneous polynomials in Cartesian coordinates,
\begin{align*}
F_{1, 1}(x, y, z) & = - \frac{1}{2} \sqrt{\frac{3}{2\pi}}  (x^2 + y^2 + z^2)^{(N-1)/2} \: (x + i y) \\
F_{1, 0}(r, \theta, \phi) &  = \frac{1}{2} \sqrt{\frac{3}{\pi}} (x^2 + y^2 + z^2)^{(N-1)/2} z \\
F_{1, -1}(r, \theta, \phi) & = \frac{1}{2} \sqrt{\frac{3}{2\pi}} (x^2 + y^2 + z^2)^{(N-1)/2} (x - iy) 
\end{align*} 
The function $(x^2 + y^2 + z^2)^{(N-1)/2}$ is a homogeneous polynomial of degree $N-1$ if $N-1$ is an even number i.e. if $N$ is odd. This is multiplied by a homogeneous polynomial of degree $1$ to get in total a homogeneous polynomial of degree $N$. 

\subsection*{(d)}

The spherical harmonics for $m = \ell$ are given by $Y_{l,l} \propto e^{i \ell \phi} \sin^l{\theta}$. Now, consider the function, 
\[\psi_{N,N}(r, \theta, \phi) = r^N e^{i N \phi} \sin^N{\theta} = r^N(\cos{\phi} + i \sin{\phi})^N \sin^N {\theta} = (r \cos{\phi} \sin{\theta} + i \: r \sin{\phi} \cos{\theta} )^N\]
However, $x = r \cos{\phi} \sin{\theta}$ and $y = r \sin{\phi} \sin{\theta}$ therefore, we can write this function as a degree $N$ homogenous polynomial,
\[F_{N,N} = (x + iy)^N = \sum_{n = 0}^N i^k \binom{N}{k} x^{N - k} y^k \]
Since the angular dependence of this function is identical to $Y_{N,N}$, this must be a state with $\ell = N$ and $m = N$. 

\section*{Problem 25.}

\subsection*{(a)}

Consider the Hamiltonian, 
\[\hamilt = \frac{\hat{p}^2}{2m} + \frac{1}{2} m \omega^2 r^2\]
We apply the standard factorization to this Hamiltonian by first splitting it into 1D factors. Specifically, define the lowering operators,
\[\a_i = \sqrt{\frac{m \omega}{2 \hbar}} \left(\hat{r}_i + \frac{i}{m \omega} \hat{p}_i \right)\]
These operators satisfy the commutation relations,
\begin{align*}
[\a_i, \a_j] &= 0 \quad \quad
[\adag_i, \adag_j] = 0  \quad \quad [\a_i, \adag_j] = \delta_{ij}
\end{align*}
Therefore, we can rewrite the Hamiltonian as,
\begin{align*}
\hamilt & = \left(\frac{\hat{p}^2_x}{2m} + \frac{1}{2} m \omega^2 x^2 \right) + \left(\frac{\hat{p}^2_y}{2m} + \frac{1}{2} m \omega^2 y^2 \right) + \left(\frac{\hat{p}^2_z}{2m} + \frac{1}{2} m \omega^2 z^2 \right) \\
& =  \hbar \omega \left(\adag_x \a_x + \frac{1}{2} \right) + \hbar \omega \left(\adag_y \a_y + \frac{1}{2} \right) + \hbar \omega \left(\adag_z \a_z + \frac{1}{2} \right) \\
& = \hbar \omega \left(\adag_x \a_x + \adag_y \a_y + \adag_z \a_z + \frac{3}{2} \right)
\end{align*}
Using the above commutation relations,
\[ [\hamilt, \adag_i] = \hbar \omega \adag_i \quad \quad [\hamilt, \a_i] = - \hbar \omega \a_i\] 
Therefore, we can describe any energy eigenstate as,
\[\ket{n_x, n_y, n_z} = \frac{(\adag_x)^{n_x} (\adag_y)^{n_y} (\adag_z)^{n_z}}{\sqrt{n_x!} \sqrt{n_y!} \sqrt{n_z!}} \ket{0}\]
With energy $E = \hbar \omega \left( n_x + n_y + n_z + \frac{3}{2} \right)$. The degeneracy of a state with energy \[E_N = \hbar \omega (N + \frac{3}{2})\] is given by the number of nonnegative integer solutions to $n_x + n_y + n_z = N$. For each of the $N + 1$ possible values of $n_x$, there are $N + 1 - n_x$ possible values of $n_y$ and, for $n_x$ and $n_y$ given, $n_z$ is fixed. Therefore, the degeneracy of the state $E_N$ is,
\begin{align*}
D_N & = \sum_{n = 0}^N (N + 1) - n = (N + 1)^2 - \sum_{n = 0}^N n = (N + 1)^2 - \frac{N(N + 1)}{2} = N^2 + 2 N + 1 - \frac{1}{2} (N^2 + N) \\ & = \frac{1}{2} (N^2 + 3 N + 2) = \frac{(N + 1)(N + 2)}{2}
\end{align*}   
\subsection*{(b)}
First, notice that $\hat{p}^2$ and $\hat{r}^2$ are dot products of vectors under rotation and therefore commute with every component of $\vec{L}$. Therefore, 
\[ [\hamilt, \L_z] = 0 \quad \quad [\hamilt, \Lsquared] =0 \]
I introduce the right and left circular ladder operators, 
\[\a_L = \frac{1}{\sqrt{2}} (\a_x + i \a_y) \quad \quad \a_R = \frac{1}{\sqrt{2}} (\a_x - i \a_y)\]
These operators have the expected commutation relations: 
\begin{align*}
[\hat{a}_R, \hat{a}^\dagger_R] &= 1 \quad \quad
[\hat{a}_L, \hat{a}^\dagger_L] = 1 \\
[\hat{a}_R, \hat{a}_L] &= 0 \quad \quad
[\hat{a}_R^\dagger, \hat{a}^\dagger_L] = 0 \\
[\hat{a}^\dagger_R, \hat{a}_L] &= 0 \quad \quad
[\hat{a}_R, \hat{a}^\dagger_L] = 0 
\end{align*}
which are simple yet tedious to check. 
Furthermore, these operators commute with $\a_z$ and its adjoint because both $\a_x$ and $\a_y$ do. Now the Hamiltonian can be rewitten using the fact that,
\begin{align*}
\adag_L \a_L + \adag_R \a_R & = \frac{1}{2} \Big( \adag_x \a_x - i \adag_y \a_x i \adag_x \a_y + \adag_y \a_y + \adag_x \a_x + i \adag_y \a_x - i \adag_x \a_y + \adag_y \a_y \Big) \\ & = \frac{1}{2} \Big(2 \adag_x \a_x + 2 \adag_y \a_y \Big) = \adag_x \a_x + \adag_y \a_y
\end{align*}
So therefore, 
\[\hamilt = \hbar \omega \left(\adag_L \a_L + \adag_R \a_R + \adag_z \a_z + \frac{3}{2} \right)\]
Furthermore, we can express the angular momentum operators in terms of these operators by expressing the coordinates and momenta. For the sake of sanity, I will omit these calculations and simply state the results. 
\begin{align*}
\L_x & = \frac{\hbar}{\sqrt{2}} \left[ (\adag_R - \adag_L) \a_z + (\a_R - \a_L) \adag_z \right] \\
\L_y & = \frac{i \hbar}{\sqrt{2}} \left[ (\adag_R + \adag_L) \a_z - (\a_R + \a_L) \adag_z \right] \\
\L_z & = \hbar \left[ \adag_R \a_R - \adag_L \a_L \right] 
\end{align*}
Therefore, 
\begin{align*}
[\L_z, \adag_R] & = \hbar \adag_R \quad \quad [\L_z, \a_R] = - \hbar \a_R \\
[\L_z, \adag_L] & =  - \hbar \adag_L \quad \quad [\L_z, \a_L] = \hbar \a_L
\end{align*} 
so the right and left rasing and lowering operators act as ladder operators for $\L_z$. Now, we can exhibit the angular momentum ladder operators. 
\begin{align*}
\L_{+} & = \L_x + i \L_y = \hbar \sqrt{2} \left[ \adag_R \a_z -\adag_z \a_L \right] \\
\L_{-} & = \L_x - i \L_y = \hbar \sqrt{2} \left[ \adag_z \a_R - \adag_L \a_z \right] \\
\end{align*}
And we will use the identity,
\[\Lsquared = \L_{-} \L_{+} + \L_z^2 + \hbar \L_z\]
Using the commutation relations, 
\[ \L_z (\adag_R)^n \ket{0} = (\adag_R)^n (n \hbar + \L_z)\ket{0} = \hbar n (\adag_R)^n \ket{0}\]
Therefore, this state has $m = n$. Furthermore, 
\begin{align*} \L_{+} (\adag_R)^n \ket{0} = \hbar \sqrt{2} \left[ \adag_R \a_z - \adag_z \a_L \right] (\adag_R)^n \ket{0} = \hbar \sqrt{2} (\adag_R)^n \left[ \adag_R \a_z - \adag_z \a_L \right]  \ket{0} = 0
\end{align*}
where I have used the fact that $[ \a_z, \adag_R] = [\a_L, \adag_R] = 0$. Therefore, this state must have $\ell = m = n$ because it is annihilated by the rasing operator. We can check this condition explicitly by considering the action of $\Lsquared$,
\begin{align*}
\Lsquared (\adag_R)^n \ket{0} & = \left(\L_{-} \L_{+} + \L_{z}^2 + \hbar \L_z\right) (\adag_R)^n \ket{0} = \L_{-} \L_{+} (\adag_R)^n \ket{0} + \L_{z} \left( \L_z + \hbar \right)  (\adag_R)^n \ket{0} \\ & = \hbar^2 \ell(\ell + 1) (\adag_R)^n \ket{0}
\end{align*} 
Thus, $(\adag_R)^n \ket{0}$ is an eigenstate of $\Lsquared$ with eigenvalue $\hbar^2 \ell (\ell + 1) = \hbar^2 n(n + 1)$ and thus the top state in an $\ell = n$ multiplet. By acting with $L_{-}$ we recover every $2 n + 1$ state in the multiplet. Consider the operator,
\[ (\adag_z)^2 + 2 \adag_R \adag_L\]
We will show that this operator commutes with $\Lsquared$ by calculating the following commutators,
\begin{align*}
[\L_{-}, (\adag_z)^2 + 2 \adag_R \adag_L] & = [\L_{-}, (\adag_z)^2] + [\L_{-},  2\adag_R \adag_L] = - \hbar \sqrt{2} \left[\adag_L \a_z, (\adag_z)^2 \right] + \hbar \sqrt{2} \left[ \adag_z \a_R, 2 \adag_R \adag_L \right] \\ 
& =  - \hbar \sqrt{2} \left[ 2 \adag_L \adag_z \right] + \hbar \sqrt{2} \left[2 \adag_z \adag_L \right] = 0 \\
[\L_{+}, (\adag_z)^2 + 2 \adag_R \adag_L] & = [\L_{+}, (\adag_z)^2] + [\L_{+},  \adag_R \adag_L] = \hbar \sqrt{2} \left[\adag_R \a_z, (\adag_z)^2 \right] + \hbar \sqrt{2} \left[ \adag_z \a_L, 2 \adag_R \adag_L \right] \\
& = \hbar \sqrt{2} \left[ 2 \adag_R \adag_z \right] - \hbar \sqrt{2} \left[2 \adag_z \adag_R \right] = 0 \\
[\L_{z}, (\adag_z)^2 + 2 \adag_R \adag_L] & = [\L_{z}, (\adag_z)^2] + [\L_{z}, 2 \adag_R \adag_L] = 2 \hbar [ \adag_R \a_R - \adag_L \a_L, \adag_R \adag_L] = 2 \hbar \left[ \adag_R \adag_L - \adag_L \adag_R \right] = 0
\end{align*}  
Therefore, 
\[ [\Lsquared, (\adag_z)^2 + 2 \adag_R \adag_L] = [\L_{-} \L_{+} + \L_{z}^2 + \hbar \L_{z}, (\adag_z)^2 + 2 \adag_R \adag_L ]  = 0 \]
We have constructed a rasing operator of $\hamilt$ which commutes with $\Lsquared$ and $\L_z$ and therefore preserves the angular momentum state. Consider the states,
\[ \ket{k, \ell} = \left( (\adag_z)^2 + 2 \adag_R \adag_L \right)^k (\adag_R)^\ell \ket{0}\] 
This state has $2k + \ell$ powers acting evenly on $\ket{0}$ so the state has energy raised by $(2k + \ell) \hbar \omega$ above the groundstate. Thus, \[\hamilt \ket{k, \ell} = \hbar \omega \left( 2k + \ell + \frac{3}{2} \right) \ket{k, \ell}\]
so this state has energy level $N = 2k + \ell$. Furthermore, because $\L_z$ and $\Lsquared$ commute with the first operator, we can explitly find the angular momentum eigenvalues of these states. 
\begin{align*}
\L_{z} \ket{k, \ell} & = \L_{z} \left( (\adag_z)^2 + 2 \adag_R \adag_L \right)^k (\adag_R)^\ell \ket{0} = \left( (\adag_z)^2 + 2 \adag_R \adag_L \right)^k \L_{z} (\adag_R)^\ell \ket{0} \\ & = \hbar \ell \left( (\adag_z)^2 + 2 \adag_R \adag_L \right)^k  (\adag_R)^\ell \ket{0} = \hbar \ell \ket{k, \ell} \\
\Lsquared \ket{k, \ell} & = \L_{z} \left( (\adag_z)^2 + 2 \adag_R \adag_L \right)^k (\adag_R)^\ell \ket{0} = \left( (\adag_z)^2 + 2 \adag_R \adag_L \right)^k \Lsquared (\adag_R)^\ell \ket{0} \\ & = \hbar^2 \ell (\ell + 1) \left( (\adag_z)^2 + 2 \adag_R \adag_L \right)^k  (\adag_R)^\ell \ket{0} = \hbar^2 \ell (\ell + 1) \ket{k, \ell} 
\end{align*} 
Therefore, $\ket{k, \ell}$ is a state with energy level $N = 2k + \ell$, total angular momentum $\ell$, and maximum $z$ angular momentum $m = \ell$. Therefore, we get every state in a $\ell$ multiplet at this energy level because $\L_{-}$ commutes with $\hamilt$. Therefore, we have found simultaneous eigenvectors of $\hamilt$, $\L_z$, and $\Lsquared$ which we knew must have been possible from the start because these operators commute. Fixing $N$, there is an angular momentum multiplet for each $\ell$ for which $N = 2 k + \ell$ has nonnegative integer solutions. Thus, $N \ge \ell \ge 0$ and $\mod{N}{\ell}{2}$. For even $N$ there are $N/2 + 1$ possible values of $\ell$ and for odd $N$ there are $(N+1)/2$ possible values. Counting the total degeneracy, the multiplicity of states in each $E_N$ energy eigenspace is for even $N$,
\[ D_N =
\sum_{i = 0}^{N/2} (2 \ell_i + 1) = \sum_{i = 0}^{N/2} (4i + 1) = 4 \frac{(N/2)(N/2+1)}{2} + (N/2 + 1) = (N+1)(N/2 + 1) = \frac{(N + 1)(N + 2)}{2} \]
and for odd $N$,
\begin{align*}
D_N & =
\sum_{i = 1}^{(N + 1)/2} (2 \ell_i + 1) = \sum_{i = 1}^{(N + 1)/2} (2 (2i-1) + 1) = \sum_{i = 1}^{(N + 1)/2} (4i - 1)  = 4 \frac{(N + 1)((N + 1)/2+1)}{4} - (N + 1)/2 \\ & = \frac{(N + 1)^2}{2} + (N + 1) - (N + 1)/2 = \frac{(N + 1)^2 + (N + 1)}{2} = \frac{(N + 1)(N + 2)}{2}
\end{align*} 
However, this is exactly the degeneracy of the energy eigenspace corresponding to energy level $N$ caluculated in part $(a)$ which means that the decomposition in terms of angular momentum multiplets has covered all of the states. Therefore, there is no multiplicity of fixed $\ell$ angular momentum representations within a given energy level. In summary, the states with energy level $N$ have a full $\ell$-multiplet of states for $\ell$ starting at $N$ and decreasing by $2$ i.e. the eigenspace is spanned by angular momentum states for $\ell = N, N - 2, N - 4, \dots, 0$ if $N$ is even which give $N/2 + 1$ full angular momentum multiplets and $\ell = N, N - 2, N - 4, \dots, 1$ if $N$ is odd, giving $(N + 1)/2$ such multiplets.  
\end{document}