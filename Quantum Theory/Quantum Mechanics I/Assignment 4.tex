\documentclass[12pt]{extarticle}
\usepackage[utf8]{inputenc}
\usepackage[english]{babel}
\usepackage[a4paper, total={7.25in, 9.5in}]{geometry}
\usepackage{tikz-feynman}
\tikzfeynmanset{compat=1.0.0} 
\usepackage{subcaption}
\usepackage{float}
\floatplacement{figure}{H}
\usepackage{simpler-wick}
\usepackage{mathrsfs}  
\usepackage{dsfont}
\usepackage{relsize}
\usepackage{tikz-cd}
\DeclareMathAlphabet{\mathdutchcal}{U}{dutchcal}{m}{n}

\usepackage{cancel}



\newcommand{\field}{\hat{\Phi}}
\newcommand{\dfield}{\hat{\Phi}^\dagger}
 
\usepackage{amsthm, amssymb, amsmath, centernot}
\usepackage{slashed}
\newcommand{\notimplies}{%
  \mathrel{{\ooalign{\hidewidth$\not\phantom{=}$\hidewidth\cr$\implies$}}}}
 
\renewcommand\qedsymbol{$\square$}
\newcommand{\cont}{$\boxtimes$}
\newcommand{\divides}{\mid}
\newcommand{\ndivides}{\centernot \mid}

\newcommand{\Integers}{\mathbb{Z}}
\newcommand{\Natural}{\mathbb{N}}
\newcommand{\Complex}{\mathbb{C}}
\newcommand{\Zplus}{\mathbb{Z}^{+}}
\newcommand{\Primes}{\mathbb{P}}
\newcommand{\Q}{\mathbb{Q}}
\newcommand{\R}{\mathbb{R}}
\newcommand{\ball}[2]{B_{#1} \! \left(#2 \right)}
\newcommand{\Rplus}{\mathbb{R}^+}
\renewcommand{\Re}[1]{\mathrm{Re}\left[ #1 \right]}
\renewcommand{\Im}[1]{\mathrm{Im}\left[ #1 \right]}
\newcommand{\Op}{\mathcal{O}}

\newcommand{\invI}[2]{#1^{-1} \left( #2 \right)}
\newcommand{\End}[1]{\text{End}\left( A \right)}
\newcommand{\legsym}[2]{\left(\frac{#1}{#2} \right)}
\renewcommand{\mod}[3]{\: #1 \equiv #2 \: \mathrm{mod} \: #3 \:}
\newcommand{\nmod}[3]{\: #1 \centernot \equiv #2 \: mod \: #3 \:}
\newcommand{\ndiv}{\hspace{-4pt}\not \divides \hspace{2pt}}
\newcommand{\finfield}[1]{\mathbb{F}_{#1}}
\newcommand{\finunits}[1]{\mathbb{F}_{#1}^{\times}}
\newcommand{\ord}[1]{\mathrm{ord}\! \left(#1 \right)}
\newcommand{\quadfield}[1]{\Q \small(\sqrt{#1} \small)}
\newcommand{\vspan}[1]{\mathrm{span}\! \left\{#1 \right\}}
\newcommand{\galgroup}[1]{Gal \small(#1 \small)}
\newcommand{\bra}[1]{\left| #1 \right>}
\newcommand{\Oa}{O_\alpha}
\newcommand{\Od}{O_\alpha^{\dagger}}
\newcommand{\Oap}{O_{\alpha '}}
\newcommand{\Odp}{O_{\alpha '}^{\dagger}}
\newcommand{\im}[1]{\mathrm{im} \: #1}
\renewcommand{\ker}[1]{\mathrm{ker} \: #1}
\newcommand{\ket}[1]{\left| #1 \right>}
\renewcommand{\bra}[1]{\left< #1 \right|}
\newcommand{\inner}[2]{\left< #1 | #2 \right>}
\newcommand{\expect}[2]{\left< #1 \right| #2 \left| #1 \right>}
\renewcommand{\d}[1]{ \mathrm{d}#1 \:}
\newcommand{\dn}[2]{ \mathrm{d}^{#1} #2 \:}
\newcommand{\deriv}[2]{\frac{\d{#1}}{\d{#2}}}
\newcommand{\nderiv}[3]{\frac{\dn{#1}{#2}}{\d{#3^{#1}}}}
\newcommand{\pderiv}[2]{\frac{\partial{#1}}{\partial{#2}}}
\newcommand{\fderiv}[2]{\frac{\delta #1}{\delta #2}}
\newcommand{\parsq}[2]{\frac{\partial^2{#1}}{\partial{#2}^2}}
\newcommand{\topo}{\mathcal{T}}
\newcommand{\base}{\mathcal{B}}
\renewcommand{\bf}[1]{\mathbf{#1}}
\renewcommand{\a}{\hat{a}}
\newcommand{\adag}{\hat{a}^\dagger}
\renewcommand{\b}{\hat{b}}
\newcommand{\bdag}{\hat{b}^\dagger}
\renewcommand{\c}{\hat{c}}
\newcommand{\cdag}{\hat{c}^\dagger}
\newcommand{\hamilt}{\hat{H}}
\renewcommand{\L}{\hat{L}}
\newcommand{\Lz}{\hat{L}_z}
\newcommand{\Lsquared}{\hat{L}^2}
\renewcommand{\S}{\hat{S}}
\renewcommand{\empty}{\varnothing}
\newcommand{\J}{\hat{J}}
\newcommand{\lagrange}{\mathcal{L}}
\newcommand{\dfourx}{\mathrm{d}^4x}
\newcommand{\meson}{\phi}
\newcommand{\dpsi}{\psi^\dagger}
\newcommand{\ipic}{\mathrm{int}}
\newcommand{\tr}[1]{\mathrm{tr} \left( #1 \right)}
\newcommand{\C}{\mathbb{C}}
\newcommand{\CP}[1]{\mathbb{CP}^{#1}}
\newcommand{\Vol}[1]{\mathrm{Vol}\left(#1\right)}

\newcommand{\Tr}[1]{\mathrm{Tr}\left( #1 \right)}
\newcommand{\Charge}{\hat{\mathbf{C}}}
\newcommand{\Parity}{\hat{\mathbf{P}}}
\newcommand{\Time}{\hat{\mathbf{T}}}
\newcommand{\Torder}[1]{\mathbf{T}\left[ #1 \right]}
\newcommand{\Norder}[1]{\mathbf{N}\left[ #1 \right]}
\newcommand{\Znorm}{\mathcal{Z}}
\newcommand{\EV}[1]{\left< #1 \right>}
\newcommand{\interact}{\mathrm{int}}
\newcommand{\covD}{\mathcal{D}}
\newcommand{\conj}[1]{\overline{#1}}

\newcommand{\SO}[2]{\mathrm{SO}(#1, #2)}
\newcommand{\SU}[2]{\mathrm{SU}(#1, #2)}

\newcommand{\anticom}[2]{\left\{ #1 , #2 \right\}}


\newcommand{\pathd}[1]{\! \mathdutchcal{D} #1 \:}

\renewcommand{\theenumi}{(\alph{enumi})}


\renewcommand{\theenumi}{(\alph{enumi})}

\newcommand{\atitle}[1]{\title{% 
	\large \textbf{Physics GR8048 Quantum Field Theory II
	\\ Assignment \# #1} \vspace{-2ex}}
\author{Benjamin Church }
\maketitle}

\newcommand{\atitleIII}[1]{\title{% 
	\large \textbf{Physics GR8049 Quantum Field Theory III
	\\ Assignment \# #1} \vspace{-2ex}}
\author{Benjamin Church }
\maketitle}

\theoremstyle{definition}
\newtheorem{theorem}{Theorem}[section]
\newtheorem{definition}{definition}[section]
\newtheorem{lemma}[theorem]{Lemma}
\newtheorem{proposition}[theorem]{Proposition}
\newtheorem{corollary}[theorem]{Corollary}
\newtheorem{example}[theorem]{Example}
\newtheorem{remark}[theorem]{Remark}


\begin{document}
\atitle{4}

\section*{Problem 12.}

Let the Hamiltonian be given by, 
\[H(\vec{r}, \vec{p}) = \frac{\left(\vec{p} - \frac{p}{c} \vec{A}(\vec{r}) \right)^2}{2 m}\]
\begin{enumerate}
\item Applying Hamilton's Equations:
\begin{align*}
\deriv{H}{r_i} &= \frac{1}{m} \left(p_j - \frac{q}{c} A_j \right) \left(-\frac{q}{c}  \partial_i A_j \right) = -\dot{p}_i \\ \deriv{H}{p_i} &= \frac{1}{m} \left(p_i - \frac{q}{c} A_i \right) = \dot{r}_i
\end{align*}
\item Differentiating,
\[ \ddot{r}_i = \frac{1}{m} \left(\dot{p}_i - \frac{q}{c} \deriv{}{t}A(\vec{r}) \right) = \frac{1}{m} \left(\dot{p}_i - \frac{q}{c} \dot{r}_j \partial_j A_i \right)\]
Now rewriting the first Hamilton equation as $\dot{p}_i = \frac{q}{c} \dot{r}_j \partial_i A_j$ and plugging in,
\[\ddot{r}_i = \frac{1}{m} \left(\frac{q}{c} \dot{r}_j \partial_i A_j - \frac{q}{c} \dot{r}_j \partial_j A_i \right) =  \frac{q}{mc} \dot{r}_i \left( \partial_i A_j - \partial_j A_i\right) = \frac{q}{mc} \dot{r}_j F_{ij}\]
The space-space components of the Faraday tensor are $F_{ij} = \epsilon_{ijk} \left(\nabla \times \vec{A} \right)_k$ so,
\[\ddot{r}_i = \frac{q}{mc} \epsilon_{ijk} \dot{r}_j \left(\nabla \times \vec{A} \right)_k = \frac{q}{mc} \left(\dot{\vec{r}} \times \left(\nabla \times \vec{A} \right) \right)_i \]
Thus, \[m \ddot{\vec{r}} = \frac{q}{c} \: \dot{\vec{r}} \times \vec{B}\]
\end{enumerate}

\section*{Problem 13.}

\begin{enumerate}
\item In cylindrical coordinates, the time independent Schrodinger Equation becomes, 
\begin{align*}
E \psi(\rho, \theta, z) &= -\frac{\hbar^2}{2m} \nabla^2 \psi(\rho, \theta, z) + + V_\rho(\rho)\psi(\rho, \theta, z)  +  V_z(z) \psi(\rho, \theta, z) \\ &= -\frac{\hbar^2}{2m} \left(\frac{1}{\rho} \pderiv{}{\rho}\left(\rho \pderiv{\psi}{\rho} \right) + \frac{1}{\rho^2} \parsq{\psi}{\theta} + \parsq{\psi}{z} \right) + V_\rho(\rho) \psi(\rho, \theta, z)  +  V_z(z) \psi(\rho, \theta, z) 
\end{align*}
Making a seperation of variables, $\psi(\rho, \theta, z) = \psi_\rho(\rho) \psi_\theta(\theta) \psi_z(z)$ we get, 
\[E = -\frac{\hbar^2}{2m} \left(\frac{1}{\rho}  \pderiv{}{\rho}\left(\rho \pderiv{\psi_\rho}{\rho} \right) \frac{1}{\psi_\rho} + \frac{1}{\rho^2} \parsq{\psi_\theta}{\theta} \frac{1}{\psi_\theta} + \parsq{\psi_z}{z} \frac{1}{\psi_z} \right) + V_\rho(\rho) +  V_z(z) \]

This can be partitioned into terms which depend only on disjoint variables,
\[\left(E + \frac{\hbar^2}{2m} \left(\frac{1}{\rho}  \pderiv{}{\rho}\left(\rho \pderiv{\psi_\rho}{\rho} \right) \frac{1}{\psi_\rho} + \parsq{\psi_z}{z} \frac{1}{\psi_z} \right) - V_\rho(\rho) -  V_z(z) \right) \frac{\rho^2}{R^2} = -\frac{\hbar^2}{2mR^2} \parsq{\psi_\theta}{\theta} \frac{1}{\psi_\theta}\]
Both sides must be constant because the LHS does not depend on $\theta$ but the RHS depends on $\theta$ alone. Thus,
\begin{align*}
\mathcal{E} &= -\frac{\hbar^2}{2m R^2} \parsq{\psi_\theta}{\theta} \frac{1}{\psi_\theta} \\ E &= - \frac{\hbar^2}{2m} \left(\frac{1}{\rho}  \pderiv{}{\rho}\left(\rho \pderiv{\psi_\rho}{\rho} \right) \frac{1}{\psi_\rho} + \parsq{\psi_z}{z} \frac{1}{\psi_z} \right) + \mathcal{E} \frac{R^2}{\rho^2} + V_\rho(\rho) + V_z(z)
\end{align*}
The first equation is trivially solved by 
\[ \psi_\theta(\theta) = A \cos{k \theta} + B \sin{k \theta} \quad \text{ with } \quad k = \sqrt{\frac{2 m R^2 \mathcal{E}}{\hbar^2}}\]We can ignore negative $\mathcal{E}$ solutions because rising and falling exponentials cannot meet the periodic boundary conditions. The periodic boundary conditions give: \\ $\psi_\theta(\theta +  2\pi) = \psi_\theta(\theta)$ so $k = n \in \Z$ Thus, \[\mathcal{E} = \frac{\hbar^2 n^2}{2m R^2}\]
Now we make the approximation that $V_\rho$ and $V_z$ tightly bind the particle about $(R, \theta, 0)$ and change much more rapidly than $\frac{\rho^2}{R^2}$. Thus, we make the approximation, \[ E = - \frac{\hbar^2}{2m} \left(\frac{1}{\rho}  \pderiv{}{\rho}\left(\rho \pderiv{\psi_\rho}{\rho} \right) \frac{1}{\psi_\rho} + \parsq{\psi_z}{z} \frac{1}{\psi_z} \right) + \mathcal{E} + V_\rho(\rho) + V_z(z) \]
Also, if the potentials are very tightly binding, then the exicted states in $\rho, z$ coordinates will be on a much higher energy scale than motion about $R$. Let $E_0$ be the ground state energy of, \[ E = - \frac{\hbar^2}{2m} \left(\frac{1}{\rho}  \pderiv{}{\rho}\left(\rho \pderiv{\psi_\rho}{\rho} \right) \frac{1}{\psi_\rho} + \parsq{\psi_z}{z} \frac{1}{\psi_z} \right) + V_\rho(\rho) + V_z(z) \]
At energies above $E_0$ which are small compared to the potentials, \[ E - \mathcal{E}  = - \frac{\hbar^2}{2m} \left(\frac{1}{\rho}  \pderiv{}{\rho}\left(\rho \pderiv{\psi_\rho}{\rho} \right) \frac{1}{\psi_\rho} + \parsq{\psi_z}{z} \frac{1}{\psi_z} \right) + V_\rho(\rho) + V_z(z) \]
can only be in the ground state so \[E = E_0 + \frac{\hbar^2 n^2}{2m R^2}\]
\item In cartesian coordinates, the time independent Schrodinger Equation becomes, 
\begin{align*}
E \psi(x, y, z) &= -\frac{\hbar^2}{2m} \nabla^2 \psi(x, y, z) + V(x, y, z)  \psi(x, y, z) \\ &= -\frac{\hbar^2}{2m} \left(\frac{1}{\rho} \parsq{\psi}{x} + \parsq{\psi}{y} + \parsq{\psi}{z} \right)  + V(x, y, z)  \psi(x, y, z)
\end{align*} 

Making a seperation of variables, $\psi(x, y, z) = \psi_x(x) \psi_y(y) \psi_z(z)$ we get, 
\[E = -\frac{\hbar^2}{2m} \left(\parsq{\psi_x}{x} \frac{1}{\psi_x} + \parsq{\psi_y}{y} \frac{1}{\psi_\theta} + \parsq{\psi_z}{z} \frac{1}{\psi_z} \right) + V(x,y,z) \]
Inside the box, $V(x, y, z) = 0$ so the equation is totally seperated, 

\begin{align*}
k_x^2 &= -\parsq{\psi_x}{x} \frac{1}{\psi_x} \\
k_y^2 &= -\parsq{\psi_y}{y} \frac{1}{\psi_y} \\
k_z^2 &= -\parsq{\psi_z}{z} \frac{1}{\psi_z} \\
E &= \frac{\hbar^2}{2m} (k_x^2 + k_y^2 + k_z^2)
\end{align*} 
Each equation is easily solved by \[\psi_i(r_i) = A_i \cos{k_i r_i} + B_i \sin{k_i r_i}\] but in each coordinate, $\psi_i(0) = \psi_i(L) = 0$ by boundary conditions. We can ignore imaginary $k_i$ solutions because rising and falling exponentials cannot be zero at more than one point which violates the boundary conditions. Thus, $k_i = \frac{n_i \pi}{L}$ for $n \in \Zplus$ and $A_i = 0$. Then, \[E = \frac{\hbar^2 \pi^2}{2m L^2}\left(n_x^2 + n_y^2 + n_z^2 \right) \]

\item Consider a torus embedded in $\R^3$ with the parametrization:
\begin{align*}
x &= (R + r \cos{\phi}) \cos{\theta} \\
y &= (R + r \cos{\phi}) \sin{\theta} \\
z &= r \sin{\phi} 
\end{align*}
We calculate the basis vectors in the surface via:
\begin{align*}
\vec{e}_\theta &= \deriv{\vec{r}}{\theta} = -(R + r \cos{\phi}) \sin{\theta} \: \hat \imath + (R + r \cos{\phi}) \cos{\theta} \: \hat \jmath \\
\vec{e}_\phi &= \deriv{\vec{r}}{\phi} = - r \sin{\phi} \cos{\theta} \: \hat \imath - r \sin{\phi} \sin{\theta} \: \hat \jmath + r \cos{\phi} \: \hat k \\
\end{align*}
Then the metric is calculated from the dot products of basis vectors:
\begin{align*}
g_{\theta \theta} &= \vec{e}_\theta \cdot \vec{e}_\theta = (R + r \cos{\phi})^2 \cdot (\sin^2{\theta} + \cos^2{\theta}) = (R + r \cos{\phi})^2 \\
g_{\theta \phi} &= g_{\phi \theta} = \vec{e}_\theta \cdot \vec{e}_\phi = r(R + r \cos{\phi}) \sin{\theta} \sin{\phi} \cos{\theta} - r(R + r \cos{\phi}) \cos{\theta} \sin{\phi} \sin{\theta} = 0 \\
g_{\phi \phi} &= \vec{e}_\phi \cdot \vec{e}_\phi = r^2 \sin^2{\phi} \cos^2{\theta} + r^2 \sin^2{\phi} \sin^2{\theta} + r^2 \cos^2{\phi} = r^2
\end{align*}
So the metric is, 
\[\bf{g} = 
\begin{pmatrix}
(R + r \cos{\phi})^2 & 0 \\
0 & r^2
\end{pmatrix}
\]
Which is (thank the heavens) diagonal and has determinant, $g = \det{\bf{g}} = r^2(R + r \cos{\phi})^2$. Applying the Voss-Weyl formula, \[ \nabla^2 \psi = \frac{1}{\sqrt{g}} \partial_i \left( \sqrt{g} \: g^{ij} \: \partial_j \psi \right)\] where $g^{ij} = (g^{-1})_{ij}$ we arive at, 
\begin{align*} \nabla^2 \psi &= \frac{1}{r(R + r \cos{\phi})} \left[ \pderiv{}{\theta} \left( r(R + r \cos{\phi}) (R + r \cos{\phi})^{-2} \pderiv{}{\theta} \psi \right) + \pderiv{}{\phi} \left( r(R + r \cos{\phi}) r^{-2} \pderiv{}{\phi} \psi \right) \right] \\ & =
\frac{1}{(R + r \cos{\phi})^2} \parsq{}{\theta} \psi + \frac{1}{r^2 (R + r \cos{\phi})} \pderiv{}{\phi} \left( (R + r \cos{\phi}) \pderiv{}{\phi} \psi \right)
\end{align*}
Thus, on the surface of the torus, the time independent Schrodinger Equation becomes, 
\begin{align*} E \psi(\theta, \phi) &= - \frac{\hbar^2}{2m} \nabla^2 \psi(\theta, \phi) \\ & = - \frac{\hbar^2}{2m} \left[ \frac{1}{(R + r \cos{\phi})^2} \parsq{}{\theta} \psi + \frac{1}{r^2 (R + r \cos{\phi})} \pderiv{}{\phi} \left( (R + r \cos{\phi}) \pderiv{}{\phi} \psi \right) \right]
\end{align*}
Now, let us take $r \ll R$ so that we may drop $r \cos{\phi}$ compared with $R$,
\begin{align*} E \psi(\theta, \phi) &= - \frac{\hbar^2}{2m} \left( \frac{1}{R^2} \parsq{}{\theta} + \frac{1}{r^2} \parsq{}{\phi} \right) \psi(\theta, \phi)
\end{align*}
Next, introduce a seperation of variables, $\psi(\theta, \phi) = \psi_\theta(\theta) \psi_\phi(\phi)$ then,

\begin{align*} E &= - \frac{\hbar^2}{2m} \left( \frac{1}{R^2} \parsq{\psi_\theta}{\theta} \frac{1}{\psi_\theta} + \frac{1}{r^2} \parsq{\psi_\phi}{\phi} \frac{1}{\psi_\phi} \right)
\end{align*}
The two terms contain only disjoint variables so they must both be constants. Let,
\begin{align*} & \parsq{\psi_\theta}{\theta} \frac{1}{\psi_\theta} = -k_\theta^2 \\
& \parsq{\psi_\phi}{\phi} \frac{1}{\psi_\phi} = -k_\phi^2 \\
& E = \frac{\hbar^2}{2m} \left(\frac{k_\theta^2}{R^2} + \frac{k_\phi^2}{r^2} \right)
\end{align*}
These equations are eaily solved by, 
\begin{align*}
\psi_\theta(\theta) &= A_\theta \cos{k_\theta \theta} + B_\theta \sin{k_\theta \theta} \\
\psi_\phi(\phi) &= A_\phi \cos{k_\phi \phi} + B_\phi \sin{k_\phi \phi} 
\end{align*} 
We can ignore negative $k$ solutions because rising and falling exponentials cannot meet the periodic boundary conditions. The periodic boundary conditions give $\psi_\theta(\theta +  2\pi) = \psi_\phi(\phi)$ so $k_\theta = n_\theta \in \N$ and $\psi_\phi(\phi +  2\pi) = \psi_\theta(\theta)$ so $k_\phi = n_\phi \in \N$ Thus, \[E = \frac{\hbar^2}{2m} \left(\frac{n_\theta^2}{R^2} + \frac{n_\phi^2}{r^2} \right)\] 

\end{enumerate}

\section*{Problem 14.}

Let the Hamiltonian be given by, \[\hamilt = \frac{\hat{p}^2}{2m} + \frac{1}{2} m \omega^2 \hat{x}^2 - q E \hat{x}\]

\begin{enumerate}
\item We proceede by completing the square in $\hat{x}$,
\begin{align*}
\hamilt &= \frac{\hat{p}^2}{2m} + \frac{1}{2} m \omega^2 \left( \hat{x}^2 - \frac{2 q E}{m \omega^2} \hat{x} + \left( \frac{q E}{m \omega^2} \right)^2 - \left( \frac{q E}{m \omega^2} \right)^2\right) \\ &=  \frac{\hat{p}^2}{2m} + \frac{1}{2} m \omega^2 \left( \hat{x} - \frac{q E}{m \omega^2} \right)^2 - \left( \frac{q^2 E^2}{2 m \omega^2} \right)
\end{align*}
Introduce rasing and lowering operators,
\begin{align*}
\adag &= \sqrt{\frac{m \omega}{2 \hbar}} \left(\hat{x} - \frac{qE}{m \omega^2} - \frac{i}{m \omega} \hat{p} \right) \\
\a &= \sqrt{\frac{m \omega}{2 \hbar}} \left(\hat{x} - \frac{qE}{m \omega^2} + \frac{i}{m \omega} \hat{p} \right)
\end{align*} 
Then,
\begin{align*}
\adag \a &= \frac{m \omega}{2 \hbar} \left\{ \left(\hat{x} - \frac{qE}{m \omega^2} \right)^2 + \frac{\hat{p}^2}{m^2 \omega^2} + \frac{i}{m \omega} \left[ \left(\hat{x} - \frac{qE}{m \omega^2} \right) \hat{p} - \hat{p} \left(\hat{x} - \frac{qE}{m \omega^2} \right) \right] \right\} \\
& =  \frac{1}{\hbar \omega} \left[ \frac{1}{2} m \omega^2 \left(\hat{x} - \frac{qE}{m \omega^2} \right)^2 + \frac{\hat{p}^2}{2 m} - \frac{1}{2} \hbar \omega \right]
\end{align*} 
because $[\hat{x} - c, \hat{p}] = i \hbar$ for any constant $c$. Pluggin into $\hamilt$,
\begin{align*}
\hamilt &= \hbar \omega \adag \a + \frac{1}{2} \hbar \omega - \left( \frac{q^2 E^2}{2 m \omega^2} \right)
\end{align*}
This is a standard quantum harmonic oscilator shifted by a constant energy. Now,
\begin{align*}
\a \adag &= \frac{m \omega}{2 \hbar} \left\{ \left(\hat{x} - \frac{qE}{m \omega^2} \right)^2 + \frac{\hat{p}^2}{m^2 \omega^2} - \frac{i}{m \omega} \left[ \left(\hat{x} - \frac{qE}{m \omega^2} \right) \hat{p} - \hat{p} \left(\hat{x} - \frac{qE}{m \omega^2} \right) \right] \right\} \\
& =  \frac{1}{\hbar \omega} \left[ \frac{1}{2} m \omega^2 \left(\hat{x} - \frac{qE}{m \omega^2} \right)^2 + \frac{\hat{p}^2}{2 m} + \frac{1}{2} \hbar \omega \right]
\end{align*}
Thus, $[\a, \adag] = 1$ so $[\hamilt, \adag] = \hbar \omega \adag$ and $[\hamilt, \a] = - \hbar \omega \a$. Thus, standard quantum harmonic oscilator results apply. In particular, there exists $\ket{0}$ such that $\a \ket{0} = 0$ and every energy eigenstate is some $\ket{n} = \frac{1}{\sqrt{n}} (\adag)^n \ket{0}$ with $\adag \a \ket{n} = n \ket{n}$ and $\inner{m}{n} = \delta_{mn}$. \\

Now, \[\hamilt \ket{n} = \left[ \hbar \omega \adag \a + \frac{1}{2} \hbar \omega - \frac{q^2E^2}{2m \omega^2} \right] \ket{n} = \left[ \hbar \omega \left(n + \frac{1}{2}\right) - \frac{q^2E^2}{2 m \omega^2}  \right] \ket{n} \]
Thus the energy spectrum is given by
\[E_n = \hbar \omega \left(n + \frac{1}{2}\right) - \frac{q^2E^2}{2m \omega^2} \]
Furthermore, \[\hat{x} = \sqrt{\frac{\hbar}{2 m \omega}} \left( \a + \adag \right) + \frac{qE}{m \omega^2} \]
And, \[\hat{p} = i \sqrt{\frac{m \hbar \omega}{2}} \left( \adag - \a \right)\]

Therefore, 
\begin{align*}
\left< \hat{x} \right> &= \bra{n} \hat{x} \ket{n} = \sqrt{\frac{\hbar}{2 m \omega}} \bra{n} \left( \a + \adag \right) \ket{n}  + \frac{qE}{m \omega^2} \inner{n}{n} \\ &= \sqrt{\frac{\hbar}{2 m \omega}} \left( \sqrt{n+1} \inner{n}{n+1} + \sqrt{n} \inner{n}{n-1} \right) + \frac{qE}{m \omega^2} = + \frac{qE}{m \omega^2} \\
\left< \hat{p} \right> &= \bra{n} \hat{p} \ket{n} = i \sqrt{\frac{m \hbar \omega}{2}} \bra{n} \left( \adag - \a \right) \ket{n} \\ &= i \sqrt{\frac{m \hbar \omega}{2}} \left( \sqrt{n+1} \inner{n}{n+1} - \sqrt{n} \inner{n}{n-1} \right) = 0 \\
\left< \hat{p}^2 \right> &= \bra{n} \hat{p}^2 \ket{n} = - \frac{m \hbar \omega}{2} \bra{n} \left( \adag - \a \right)^2 \ket{n} \\ &= - \frac{m \hbar \omega}{2} \bra{n} \left[ (\adag)^2 -\adag \a - \a \adag + (\a)^2 \right] \ket{n} = \frac{m \hbar \omega}{2} \left(2n + 1 \right)
\end{align*}

\item When the field is switched off, the new Hamiltonian becomes exactly the centered quantum harmonic oscilator. Let $\ket{\psi_0} = \ket{0_{old}}$ and then the new eigenstates are the standard quantum harmonic oscillator states: $\ket{n_{new}}$. Now the probability for the particle to be found in the new ground state is $P_0(t) = |\inner{0_{new}}{\psi(t)}|^2$. However, \[\deriv{}{t} \inner{0_{new}}{\psi(t)} = \bra{0_{new}} \frac{1}{i \hbar} \hamilt \ket{\psi(t)} = \bra{\psi(t)} \frac{-1}{i \hbar} \hamilt \ket{0_{new}}^* = \frac{\omega}{2i} \inner{\psi(t)}{0_{new}}^* = \frac{\omega}{2i} \inner{0_{new}}{\psi(t)} \]

Thus, \[\inner{0_{new}}{\psi(t)} = \inner{0_{new}}{\psi(0)} e^{-i \omega t /2}\]
so, \[P_0(t) = |\inner{0_{new}}{\psi(t)}|^2 = | \inner{0_{new}}{\psi(0)}|^2 = P_0(0) = | \inner{0_{new}}{0_{old}}|^2\]
The old and new ground states are easily found by anihilating them with thier respective lowering operators,
\[\bra{x} \a_{old} \ket{0_{old}} = \sqrt{\frac{m \omega}{2 \hbar}} \left(x - \frac{qE}{m \omega^2} + \frac{\hbar}{m \omega} \pderiv{}{x}  \right) \psi_0^{old}(x) = 0\] 
which gives a normalized wavefunction: \[\psi_0^{old}(x) = \left( \frac{m \omega}{\pi \hbar} \right)^{1/4} \exp{\left[ - \left(\frac{m \omega}{2 \hbar} \right) \left( x - \frac{qE}{m \omega^2} \right)^2  \right] }\]
Similarly, the equation, 
\[\bra{x} \a_{new} \ket{0_{new}} = \sqrt{\frac{m \omega}{2 \hbar}} \left(x + \frac{\hbar}{m \omega} \pderiv{}{x}  \right) \psi_0^{new}(x) = 0\] 
has a normalized solution: \[\psi_0^{new}(x) = \left( \frac{m \omega}{\pi \hbar} \right)^{1/4}  \exp{\left[ - \frac{m \omega x^2}{2 \hbar} \right]}\]
Therefore, 
\begin{align*}
\inner{0_{new}}{0_{old}} &= \int_{-\infty}^{\infty} \psi_0^{new}(x)^* \psi_0^{old}(x) \d{x} \\ & =
\left( \frac{m \omega}{\pi \hbar} \right)^{1/2} \int_{-\infty}^{\infty} \exp{\left\{- \frac{m \omega x^2}{2 \hbar} \left[\left( x - \frac{qE}{m \omega^2} \right)^2 + x^2 \right] \right\}} \d{x}
\end{align*}
Expanding the exponent, 
\begin{align*}
& - \frac{m \omega}{2 \hbar} \left[\left( x - \frac{qE}{m \omega^2} \right)^2 + x^2 \right] = - \frac{m \omega x^2}{2 \hbar} \left( 2 x^2 - \frac{2qE}{m \omega^2} x +  \left( \frac{qE}{m \omega^2} \right)^2 \right) \\ &= - \frac{m \omega}{ \hbar} \left( x^2 - \frac{qE}{m \omega^2} x +  \left( \frac{qE}{2 m \omega^2} \right)^2 + \frac{1}{4} \left( \frac{qE}{m \omega^2} \right)^2 \right) \\ &= - \frac{m \omega}{ \hbar} \left(x - \frac{qE}{2 m \omega^2} \right)^2 - \frac{1}{4} \left( \frac{q^2E^2}{m \hbar \omega^3} \right)
\end{align*}
Then pluggin back into the integral,
\begin{align*}
\inner{0_{new}}{0_{old}} &= 
\left( \frac{m \omega}{\pi \hbar} \right)^{1/2} \int_{-\infty}^{\infty} \exp{\left[ - \frac{m \omega}{ \hbar} \left(x - \frac{qE}{2 m \omega^2} \right)^2 - \frac{1}{4} \left( \frac{q^2E^2}{m \hbar \omega^3} \right) \right]} \d{x} \\ &= \exp{\left[  - \frac{1}{4} \left( \frac{q^2E^2}{m \hbar \omega^3} \right) \right]}
\end{align*} 
Thus, \[P_0(t) = \exp{\left[ - \frac{1}{2} \left( \frac{q^2E^2}{m \hbar \omega^3} \right)\right]}\]
\end{enumerate}

\section*{Problem 15.}

Classically, the Hamiltonian for a particle of mass $m$ constrained on a sphere of radius $r$ is  $H = \frac{p^2}{2m} + mgr(1 - \cos{\theta})$ in shperical coordinates with the $+ \hat{z}$ direction aligned with $\vec{g}$. 
\begin{enumerate}
\item Canonical quantization gives: \[\hamilt = -\frac{\hbar^2}{2m} \nabla^2 + mgr(1 - \cos{\theta})\] where $\nabla^2$ constrained to a sphere in spherical coordinates is:
\[\nabla^2 = \frac{1}{r^2 \sin{\theta}} \pderiv{}{\theta} \left( \sin{\theta} \pderiv{}{\theta} \right) + \frac{1}{r^2 \sin^2{\theta}} \parsq{}{\phi}\] Thus, the time independent Schrodinger Equation becomes:
\[E \psi(\theta, \phi) = -\frac{\hbar^2}{2m}  \left[ \frac{1}{r^2 \sin{\theta}} \pderiv{}{\theta} \left( \sin{\theta} \pderiv{}{\theta} \right) + \frac{1}{r^2 \sin^2{\theta}} \parsq{}{\phi} \right] \psi(\theta, \phi) + mgr(1 - \cos{\theta}) \psi(\theta, \phi) \]
This equation is seperable. It can be written as:
\[ r^2 \sin^2{\theta} \left( E + \frac{\hbar^2}{2m}  \left[ \frac{1}{r^2 \sin{\theta}} \pderiv{}{\theta} \left( \sin{\theta} \pderiv{}{\theta} \right) \right] - mgr(1 - \cos{\theta}) \right) \psi(\theta, \phi) =  - \frac{\hbar^2}{2m} \parsq{}{\phi} \psi(\theta, \phi) \]
Thus if we perform a seperation of variables, $\psi(\theta, \phi) = \psi_\theta(\theta) \psi_\phi(\phi)$ then,
\[ r^2 \sin^2{\theta} \left( E + \frac{\hbar^2}{2m}  \left[ \frac{1}{r^2 \sin{\theta}} \pderiv{}{\theta} \left( \sin{\theta} \pderiv{\psi_\theta}{\theta} \right) \frac{1}{\psi_\theta} \right] - mgr(1 - \cos{\theta}) \right) =  - \frac{\hbar^2}{2m} \parsq{\psi_{\phi}}{\phi} \frac{1}{\psi_{\phi}} \]
Since the two sides of this equation depend only on disjoint variables, both sides must equal a constant. In particular, 
\[\parsq{\psi_{\phi}}{\phi} \frac{1}{\psi_{\phi}} = -k_\phi^2 \]
\[E \psi_\theta(\theta) = -\frac{\hbar^2}{2m}  \left[ \frac{1}{r^2 \sin{\theta}} \pderiv{}{\theta} \left( \sin{\theta} \pderiv{}{\theta} \right) \right] \psi_\theta(\theta) + \frac{\hbar^2 k_\phi^2}{2m r^2 \sin^2{\theta}} \psi_\theta(\theta) + mgr(1 - \cos{\theta}) \psi_\theta(\theta) \]
As before, periodic boundary conditions on $\phi$ require that $k_\phi$ is real with $k_\phi = m_\phi \in \N$. Now, to make the differential equation tractable, we introduce a small angle approximation to second order. The Schrodinger Equation becomes: 
\[ E \psi_\theta(\theta) = -\frac{\hbar^2}{2m}  \left[ \frac{1}{r^2 \theta} \pderiv{}{\theta} \left( \theta \pderiv{}{\theta} \right) \right] \psi_\theta(\theta) + \left[ \frac{\hbar^2 m_\phi^2}{2m r^2 \theta^2} + \frac{1}{2}mgr \theta^2 \right] \psi_\theta(\theta) \]
Let's make this equation a bit less nasty by introducing a characteristic angluar frequency $\omega = \sqrt{\frac{g}{r}}$ and angular scale $\alpha = \sqrt{\frac{\hbar}{m \omega r^2}}$ then write $x = \frac{\theta}{\alpha}$. Now setting $\mathcal{E} = \frac{E}{\hbar \omega}$, we rewrite:
\begin{align*}
\frac{E}{\hbar \omega} \psi_\theta(\theta) &= -\frac{\hbar}{2m \omega r^2}  \left[ \frac{1}{\theta} \pderiv{}{\theta} \left( \theta \pderiv{}{\theta} \right) \right] \psi_\theta(\theta) + \left[ \frac{\hbar m_\phi^2}{2m \omega r^2 \theta^2} + \frac{mgr}{2 \hbar \omega} \theta^2 \right] \psi_\theta(\theta) \\
\frac{E}{\hbar \omega} \psi_\theta(\theta) &= -\frac{\hbar}{2m \omega r^2}  \left[ \frac{1}{\theta} \pderiv{}{\theta} \left( \theta \pderiv{}{\theta} \right) \right] \psi_\theta(\theta) + \left[ \frac{\hbar m_\phi^2}{2m \omega r^2 \theta^2} + \frac{m \omega r^2}{2 \hbar} \theta^2 \right] \psi_\theta(\theta) \\
2\mathcal{E} \psi_\theta(\theta) &= -\left[ \frac{1}{x} \pderiv{}{x} \left( x \pderiv{}{x} \right) \right] \psi_\theta(\theta) + \left[ \frac{m_\phi^2}{x^2} + x^2 \right] \psi_\theta(\theta)
\end{align*}
Now, as $x \rightarrow \infty$, the equation becomes,
\[-\parsq{\psi_\theta}{x} + x^2 \psi_\theta = 0\]
Thus for large $x$, $\psi_\theta(x) \propto e^{-\frac{1}{2} x^2}$ so write $\psi_\theta(x) = u(x) e^{-\frac{1}{2}x^2}$. Plugging in,
\begin{align*}
2\mathcal{E} u e^{-\frac{1}{2}x^2} &= - \frac{1}{x} \pderiv{}{x} \left( x \pderiv{}{x} \right) u(x) e^{-\frac{1}{2}x^2} + \left[ \frac{m_\phi^2}{x^2} + x^2 \right]u(x) e^{-\frac{1}{2}x^2} \\
2\mathcal{E} u e^{-\frac{1}{2}x^2} &= -\frac{1}{x} \pderiv{}{x} \left(x u' e^{-\frac{1}{2}x^2} - x^2 u e^{-\frac{1}{2} x^2} \right) + \left[ \frac{m_\phi^2}{x^2} + x^2 \right]u e^{-\frac{1}{2}x^2} \\
2\mathcal{E} u e^{-\frac{1}{2}x^2} &= \left(-\frac{u'}{x} + 2u - u'' + 2x u' - x^2 u \right) e^{-\frac{1}{2} x^2} + \left[ \frac{m_\phi^2}{x^2} + x^2 \right]u e^{-\frac{1}{2}x^2} \\
2\mathcal{E} u &= -\frac{u'}{x} + 2u - u'' + 2x u' +  \frac{m_\phi^2}{x^2} u \\
\end{align*}
Thus collecting terms of equal power in $x$, 
\[2(\mathcal{E} - 1) u - 2x u' = -\frac{u'}{x} - u'' +  \frac{m_\phi^2}{x^2} u\]
Now, we apply the series expansion $u = \sum\limits_{i = 0}^{\infty} c_i x^i$:
\[\sum\limits_{i = 0}^{\infty} \left[2(\mathcal{E} - 1) - 2i\right]c_i x^i = \sum_{i = 0}^{\infty} \left[-i - i(i-1) + m_\phi^2 \right]c_i x^{i-2} =  \sum_{i = 0}^{\infty} \left[m_\phi^2 - i^2 \right]c_i x^{i-2} \]
Reparametrizing the second sum, 
\[\sum\limits_{i = 0}^{\infty} \left[2(\mathcal{E} - 1) - 2i\right]c_i x^i = \sum_{i = 0}^{\infty} \left[m_\phi^2 - (i+2)^2 \right]c_{i+2} x^{i} + \frac{m_\phi^2}{x^2}c_0 + \frac{m_\phi^2 - 1}{x} c_1 \]
There are no terms on the left which can cancel the $\frac{1}{x}$ and $\frac{1}{x^2}$ divergences so if $m_\phi \neq 0$ then $c_0 = 0$ and if $m_\phi^2 \neq 1$ then $c_1 = 0$. However, all other terms of equal order must cancel so,
\[\frac{c_{i+2}}{c_i} = \frac{2(\mathcal{E} - 1 - i)}{m_\phi^2 - (i+2)^2} \]
If this series never terminates, then for large $i$ the recurrence relation goes as,
\[\frac{c_{i+2}}{c_i} \approx \frac{2(i + 1)}{(i+2)^2} \approx \frac{1}{i/2}\]
and therefore, 
\[c_i \propto \frac{1}{(i/2)!} \quad \text{ so } \quad u = \sum\limits_{i = 0}^{\infty} \frac{x^{2i}}{i!} = e^{x^2}\]
which diverges faster than the other factor so $\psi_\theta \rightarrow \infty$ which violates the small angle approximation let alone normalizability! Thus, the series must terminate at some step $i_{max}$ so the numerator $2(\mathcal{E} - i_{max} - 1) = 0$ so $\mathcal{E} = i_{max} + 1$. Thus, the energy levels are \[E = \hbar \omega (i_{max} + 1) = \hbar \sqrt{\frac{g}{r}} (i_{max} + 1)\] 
where $i_{max} + 1 \in \Zplus$. In particular, the ground state energy is $E_0 = \hbar \sqrt{\frac{g}{r}}$ with a wave function given by: $c_0 = N$ and all other terms are zero since $\mathcal{E} - 1 = 0$ and $c_1 = 0$ so that all odd terms are zero (else they will not terminante). Thus, $m_\phi = 0$ because $c_0 \neq 0$. In summary, $u(x) = N$ and therefore,
\[\psi_0(\theta,\phi) = N e^{-\frac{m \omega}{2\hbar} (r \theta)^2}\]
In general, the $\phi$-wavefunction is given by,
\[\psi_\phi(\phi) \propto e^{i m_\phi \phi}\]

\item We continue the series to higher order terms. Note that for $m_\phi \ge 2$ the denominator will blow up for $i_m = |m_\phi| - 2$ so in that case, we begin the series with $c_{i_m} = 0$, $c_{i_m + 1} = 0$, and $c_{i_m + 2} \neq 0$ so that $c_{i_m + 2} \cdot (m_\phi^2 - (i_m + 2)^2) = c_{i_m} \cdot 2(\mathcal{E} - i_m - 1)$ is satisfied. This means that the series must terminate at $i_{max} \ge  i_m + 2$ with $i_{max}$ and $i_m$ having the same parity (since a zero term series gives $\psi = 0$ and the non-zero terms skip by two) so $\mathcal{E} \ge |m_\phi| + 1$ with $\mod{\mathcal{E}}{(|m_\phi| + 1)}{2}$. \\ \\
Thus, the next energy level corresponds to $\mathcal{E} = 2$ and $m_\phi = -1, +1$ These two cases correspond to series $u = N x $ and a $\phi$-wavefunction $\psi_\phi \propto e^{i m_\phi \phi} = e^{\pm i \phi}$. Thus we have two states with $E = 2 \hbar \sqrt{\frac{g}{r}}$,
\[\psi_{1, \pm 1}(\theta, \phi) = N \sqrt{\frac{m \omega}{\hbar}} e^{\pm i \phi} \: r \theta \: e^{-\frac{m \omega}{2\hbar} (r \theta)^2} \]
The next energy level corresponds to $\mathcal{E} = 3$ and $m_\phi = -2, 0, +2$ These three cases correspond to series $u = Nx^2$ for $m_\phi = \pm 2$ and $N(1 - x^2)$ for $m_\phi = 0$. Thus we have three states with $E = 3 \hbar \sqrt{\frac{g}{r}}$,
\begin{align*}
\psi_{2, 0}(\theta, \phi) &= N \left(1 -  \frac{m \omega}{\hbar} (r \theta)^2 \right) e^{-\frac{m \omega}{2\hbar} (r \theta)^2} \\
\psi_{2, \pm 2}(\theta, \phi) &= N \frac{m \omega}{\hbar} e^{\pm 2 i \phi} \: (r \theta)^2 \: e^{-\frac{m \omega}{2\hbar} (r \theta)^2}
\end{align*}

\item For the small angle approximation to be reasonable for the ground state, we require that the spread of the gaussian be much less than $1 \: \mathrm{rad}$. Thus, \[\frac{\hbar}{m \omega r^2} \ll 1 \quad \text{ thus } \quad \hbar^2 \ll m^2 g r^3\]
\end{enumerate}

Alternatively: if one notices that, in the small angle approximation, this Hamiltonian corresponds to a 2D harmonic oscillator in polar coordinates then the problem can easily be solved by factorization. For the small angle approximation Hamiltonain,
\[\hamilt = -\frac{\hbar^2}{2m}  \left[ \frac{1}{r^2 \theta} \pderiv{}{\theta} \left( \theta \pderiv{}{\theta} \right) + \frac{1}{r^2 \theta} \parsq{}{\phi} \right] + \frac{1}{2} mgr \theta^2 \]
Define right and left circular ladder operators using the scale parameter $\alpha = \sqrt{\frac{\hbar}{m \omega r^2}}$,
\begin{align*}
\hat{a}_R = \frac{e^{-i \phi}}{2} \left[\frac{\theta}{\alpha}  + \alpha \pderiv{}{\theta} - \alpha \frac{i}{\theta} \pderiv{}{\phi}  \right] \\
\hat{a}_L = \frac{e^{i \phi}}{2} \left[\frac{\theta}{\alpha}  + \alpha \pderiv{}{\theta} + \alpha \frac{i}{\theta} \pderiv{}{\phi}  \right] 
\end{align*}
These operators have the expected commutation relations: 
\begin{align*}
[\hat{a}_R, \hat{a}^\dagger_R] &= 1 \quad \quad
[\hat{a}_L, \hat{a}^\dagger_L] = 1 \\
[\hat{a}_R, \hat{a}_L] &= 0 \quad \quad
[\hat{a}_R^\dagger, \hat{a}^\dagger_L] = 0 \\
[\hat{a}^\dagger_R, \hat{a}_L] &= 0 \quad \quad
[\hat{a}_R, \hat{a}^\dagger_L] = 0 
\end{align*}
which are simple yet tedious to check. 



Now consider the combination,
\begin{align*}
\hat{a}_R^\dagger \hat{a}_R + \hat{a}_L^\dagger \hat{a}_L &= \frac{e^{i \phi}}{2} \left[\frac{\theta}{\alpha}  - \alpha \pderiv{}{\theta} - \alpha \frac{i}{\theta} \pderiv{}{\phi}  \right] \frac{e^{-i \phi}}{2} \left[\frac{\theta}{\alpha}  + \alpha \pderiv{}{\theta} - \alpha \frac{i}{\theta} \pderiv{}{\phi}  \right]  \\ & + \frac{e^{-i \phi}}{2} \left[\frac{\theta}{\alpha}  - \alpha \pderiv{}{\theta}  + \alpha \frac{i}{\theta} \pderiv{}{\phi}  \right] \frac{e^{i \phi}}{2} \left[\frac{\theta}{\alpha}  + \alpha \pderiv{}{\theta} + \alpha \frac{i}{\theta} \pderiv{}{\phi}  \right] \\ 
&= \frac{1}{4} \left[z^2 - \pderiv{}{z} z - 1 - i \pderiv{}{\phi} + z \pderiv{}{z} - \parsq{}{z} \right. \\ & \left. - \frac{1}{z} \pderiv{}{z}  - \frac{i}{z} \pderiv{}{\phi} \pderiv{}{x}  - i \pderiv{}{\phi} + i \pderiv{}{z} z \pderiv{}{\phi} + \frac{i}{z^2} \pderiv{}{\phi} - \frac{1}{z^2} \parsq{}{\phi} \right] \\ & + 
\frac{1}{4} \left[z^2 - \pderiv{}{z} z - 1 + i \pderiv{}{\phi} + z \pderiv{}{z} - \parsq{}{z} \right. \\ & \left. - \frac{1}{z} \pderiv{}{z}  + \frac{i}{z} \pderiv{}{\phi} \pderiv{}{x}  + i \pderiv{}{\phi} - i \pderiv{}{z} z \pderiv{}{\phi} - \frac{i}{z^2} \pderiv{}{\phi} - \frac{1}{z^2} \parsq{}{\phi} \right] \\ & = \frac{1}{4} \left[z^2 - 2 - 2i \pderiv{}{\phi} - \frac{1}{z} \pderiv{}{z} - \parsq{}{z} - \frac{1}{z^2} \parsq{}{\phi} \right] \\ & + 
\frac{1}{4} \left[z^2 - 2 + 2i \pderiv{}{\phi} - \frac{1}{z} \pderiv{}{z} - \parsq{}{z} - \frac{1}{z^2} \parsq{}{\phi} \right] \\ & = - \frac{1}{2} \left[\frac{1}{z} \pderiv{}{z} \left( z \pderiv{}{z} \right) + \frac{1}{z^2} \parsq{}{\phi} \right] + \frac{1}{2} z^2 - 1 \\ & = \frac{\hbar}{2m \omega r^2} \left[\frac{1}{\theta} \pderiv{}{\theta} \left( \theta \pderiv{}{\theta} \right) + \frac{1}{\theta^2} \parsq{}{\phi} \right] + \frac{1}{2} \frac{m g r}{\hbar \omega} \theta^2 - 1 = \frac{1}{\hbar \omega} \hamilt - 1 
\end{align*}
Thus,
\[\hamilt = \hbar \omega \left( \hat{a}_R^\dagger \hat{a}_R + \hat{a}_L^\dagger \hat{a}_L + 1 \right)\]
Because $\bra{\psi} \hat{a}^\dagger \hat{a} \ket{\psi} = \inner{\hat{a} \psi}{\hat{a} \psi} \ge 0$ we immedietly see that the ground state is killed by $\hat{a}_L$ and $\hat{a}_R$ and thus has energy $\hbar \omega$. From the above commutation relations, 
\begin{align*}
[\hamilt, \hat{a}^\dagger_R] &= \hbar \omega \hat{a}^\dagger_R \quad \quad
[\hamilt, \hat{a}_R] = - \hbar \omega \hat{a}_R \\
[\hamilt, \hat{a}^\dagger_L] &= \hbar \omega \hat{a}^\dagger_L \quad \quad
[\hamilt, \hat{a}_L] = - \hbar \omega \hat{a}_L
\end{align*}
Therefore, $\hat{a}^\dagger_R$ and $\hat{a}^\dagger_L$ increse the energy of a state by $\hbar \omega$ while $\hat{a}_R$ and $\hat{a}_L$ decrese the energy by $\hbar \omega$. Futhermore, by subtracting the second term in the above derivation, we see that:
\begin{align*}
\hat{a}_R^\dagger \hat{a}_R - \hat{a}_L^\dagger \hat{a}_L &= \frac{1}{4} \left[z^2 - 2 - 2i \pderiv{}{\phi} - \frac{1}{z} \pderiv{}{z} - \parsq{}{z} - \frac{1}{z^2} \parsq{}{\phi} \right] \\ & - 
\frac{1}{4} \left[z^2 - 2 + 2i \pderiv{}{\phi} - \frac{1}{z} \pderiv{}{z} - \parsq{}{z} - \frac{1}{z^2} \parsq{}{\phi} \right] \\ & = -i \pderiv{}{\theta} = \frac{1}{\hbar} \Lz
\end{align*}
Thus, $\Lz = \hbar \left( \hat{a}_R^\dagger \hat{a}_R - \hat{a}_L^\dagger \hat{a}_L \right)$ so from the above commutation relations:
\begin{align*}
[\Lz, \hat{a}^\dagger_R] &= \hbar \hat{a}^\dagger_R \quad \quad
[\Lz, \hat{a}_R] = - \hbar \hat{a}_R \\
[\Lz, \hat{a}^\dagger_L] &= - \hbar \hat{a}^\dagger_L \quad \quad
[\Lz, \hat{a}_L] =  \hbar \hat{a}_L
\end{align*}
Therefore, $\hat{a}^\dagger_R$ acts to add an energy mode with positive $z$-angluar momentum and $\hat{a}^\dagger_L$ acts to add an energy mode with negative $z$-angular momentum. For our seperated eigenstates, 
\[\bra{(\theta, \phi)} \Lz \ket{\psi} = - i \hbar \pderiv{}{\phi} \psi(\theta, \phi) = \hbar m \psi(\theta, \phi) \] 
So $m = N_R - N_L$ and $n = N_R + N_L$ where $N_R$ and $N_L$ are the eigenstates of $\hat{a}^\dagger_R \hat{a}_R$ and $\hat{a}^\dagger_L \hat{a}_L$ respectively.
This explains the spectrum found above because $\mathcal{E} = n + 1 = N_R + N_L + 1 \ge m + 1$ and $n - m = 2 N_L$ so $\mod{n}{m}{2}$.
Acting on the ground state, there are two choices,
\begin{align*}
\ket{\psi_{1, +1}} &= \hat{a}^\dagger_R \ket{0} \\
\ket{\psi_{1, -1}} &= \hat{a}^\dagger_L \ket{0} \\
\end{align*} 
one right circulating mode or one left circulating mode. For some reason we are missing the $m = 0$ state of a $l = 1$ multiplet. This has to do with the fact that we are in 2D and the $m = 0$ state in 3D protrudes perpendicular to the plane defined by $\Lz$. Continuing to the next energy level we get three possibilities (because $[\hat{a}^\dagger_R, \hat{a}^\dagger_L] = 0$) which are:
\begin{align*}
\ket{\psi_{2, +2}} &= \frac{1}{\sqrt{2}} (\hat{a}^\dagger_R)^2 \ket{0} \\
\ket{\psi_{2, -2}} &= \frac{1}{\sqrt{2}} (\hat{a}^\dagger_L)^2 \ket{0} \\
\ket{\psi_{2, 0}} &= \frac{1}{\sqrt{2}} \hat{a}^\dagger_R \hat{a}^\dagger_L \ket{0} 
\end{align*} 
We can also get the explicit wavefunctions from these ladder operators. If both $\hat{a}_R$ and $\hat{a}_L$ kill $\ket{0}$ then so does any linear combination. In particular,
\[\bra{(\theta, \phi)} (e^{i \phi} \hat{a}_R + e^{-i \phi} \hat{a}_L) \ket{0} = \left[ \frac{\theta}{\alpha} + \alpha \pderiv{}{\theta} \right] \psi(\theta, \phi) = 0 \]
therefore,
\[\psi_0(\theta, \phi) = f(\phi) e^{-\frac{1}{2} \left(\frac{\theta}{\alpha} \right)^2}\]
Similarly, 
\[ i \bra{(\theta, \phi)} (e^{i \phi} \hat{a}_R - e^{-i \phi} \hat{a}_L) \ket{0} = \frac{\alpha}{\theta} \pderiv{}{\phi} f(\phi) e^{-\frac{1}{2} \left(\frac{\theta}{\alpha} \right)^2} = 0 \]
and thus $f(\phi)$ is constant so,
\[\psi_0(\theta, \phi) = K e^{-\frac{1}{2} \left(\frac{\theta}{\alpha} \right)^2}\]
Now, acting with the rasing operators, 
\begin{align*}
\psi_{1, +1} &= \hat{a}^\dagger_R \psi_0 = \frac{e^{i \phi}}{2} \left[\frac{\theta}{\alpha}  - \alpha \pderiv{}{\theta} - \alpha \frac{i}{\theta} \pderiv{}{\phi}  \right] K e^{-\frac{1}{2} \left(\frac{\theta}{\alpha} \right)^2}  = \frac{K}{\alpha} e^{i \phi} \: \theta \: e^{-\frac{1}{2} \left(\frac{\theta}{\alpha} \right)^2} \\
\psi_{1, -1} &= \hat{a}^\dagger_L \psi_0 = \frac{e^{-i \phi}}{2} \left[\frac{\theta}{\alpha}  - \alpha \pderiv{}{\theta} + \alpha \frac{i}{\theta} \pderiv{}{\phi}  \right] K e^{-\frac{1}{2} \left(\frac{\theta}{\alpha} \right)^2} = \frac{K}{\alpha} e^{-i \phi} \: \theta \: e^{-\frac{1}{2} \left(\frac{\theta}{\alpha} \right)^2}
\end{align*}
Similarly, we can obtain the second excited states:
\begin{align*}
\psi_{2, +2} &= \hat{a}^\dagger_R \psi_{1, +1} = \frac{e^{i \phi}}{2} \left[\frac{\theta}{\alpha}  - \alpha \pderiv{}{\theta} - \alpha \frac{i}{\theta} \pderiv{}{\phi}  \right] \frac{K}{\alpha} e^{i \phi} \: \theta \: e^{-\frac{1}{2} \left(\frac{\theta}{\alpha} \right)^2}  = \frac{K}{\alpha^2} e^{2 i \phi} \: \theta^2 \: e^{-\frac{1}{2} \left(\frac{\theta}{\alpha} \right)^2} \\
\psi_{2, -2} &= \hat{a}^\dagger_L \psi_{1, -1} = \frac{e^{-i \phi}}{2} \left[\frac{\theta}{\alpha}  - \alpha \pderiv{}{\theta} + \alpha \frac{i}{\theta} \pderiv{}{\phi}  \right]\frac{K}{\alpha} e^{-i \phi} \: \theta \: e^{-\frac{1}{2} \left(\frac{\theta}{\alpha} \right)^2} = \frac{K}{\alpha^2} e^{-2 i \phi} \: \theta^2 \: e^{-\frac{1}{2} \left(\frac{\theta}{\alpha} \right)^2} \\
\psi_{2, 0} &= \hat{a}^\dagger_L \psi_{1, +1} = \frac{e^{-i \phi}}{2} \left[\frac{\theta}{\alpha}  - \alpha \pderiv{}{\theta} + \alpha \frac{i}{\theta} \pderiv{}{\phi}  \right]\frac{K}{\alpha} e^{i \phi} \: \theta \: e^{-\frac{1}{2} \left(\frac{\theta}{\alpha} \right)^2} = K \left( \frac{\theta^2}{\alpha^2} - 1 \right) \: e^{-\frac{1}{2} \left(\frac{\theta}{\alpha} \right)^2}
\end{align*}
\end{document}

