\documentclass[12pt]{extarticle}
\usepackage[utf8]{inputenc}
\usepackage[english]{babel}
\usepackage[a4paper, total={7.25in, 9.5in}]{geometry}
\usepackage{tikz-feynman}
\tikzfeynmanset{compat=1.0.0} 
\usepackage{subcaption}
\usepackage{float}
\floatplacement{figure}{H}
\usepackage{simpler-wick}
\usepackage{mathrsfs}  
\usepackage{dsfont}
\usepackage{relsize}
\usepackage{tikz-cd}
\DeclareMathAlphabet{\mathdutchcal}{U}{dutchcal}{m}{n}

\usepackage{cancel}



\newcommand{\field}{\hat{\Phi}}
\newcommand{\dfield}{\hat{\Phi}^\dagger}
 
\usepackage{amsthm, amssymb, amsmath, centernot}
\usepackage{slashed}
\newcommand{\notimplies}{%
  \mathrel{{\ooalign{\hidewidth$\not\phantom{=}$\hidewidth\cr$\implies$}}}}
 
\renewcommand\qedsymbol{$\square$}
\newcommand{\cont}{$\boxtimes$}
\newcommand{\divides}{\mid}
\newcommand{\ndivides}{\centernot \mid}

\newcommand{\Integers}{\mathbb{Z}}
\newcommand{\Natural}{\mathbb{N}}
\newcommand{\Complex}{\mathbb{C}}
\newcommand{\Zplus}{\mathbb{Z}^{+}}
\newcommand{\Primes}{\mathbb{P}}
\newcommand{\Q}{\mathbb{Q}}
\newcommand{\R}{\mathbb{R}}
\newcommand{\ball}[2]{B_{#1} \! \left(#2 \right)}
\newcommand{\Rplus}{\mathbb{R}^+}
\renewcommand{\Re}[1]{\mathrm{Re}\left[ #1 \right]}
\renewcommand{\Im}[1]{\mathrm{Im}\left[ #1 \right]}
\newcommand{\Op}{\mathcal{O}}

\newcommand{\invI}[2]{#1^{-1} \left( #2 \right)}
\newcommand{\End}[1]{\text{End}\left( A \right)}
\newcommand{\legsym}[2]{\left(\frac{#1}{#2} \right)}
\renewcommand{\mod}[3]{\: #1 \equiv #2 \: \mathrm{mod} \: #3 \:}
\newcommand{\nmod}[3]{\: #1 \centernot \equiv #2 \: mod \: #3 \:}
\newcommand{\ndiv}{\hspace{-4pt}\not \divides \hspace{2pt}}
\newcommand{\finfield}[1]{\mathbb{F}_{#1}}
\newcommand{\finunits}[1]{\mathbb{F}_{#1}^{\times}}
\newcommand{\ord}[1]{\mathrm{ord}\! \left(#1 \right)}
\newcommand{\quadfield}[1]{\Q \small(\sqrt{#1} \small)}
\newcommand{\vspan}[1]{\mathrm{span}\! \left\{#1 \right\}}
\newcommand{\galgroup}[1]{Gal \small(#1 \small)}
\newcommand{\bra}[1]{\left| #1 \right>}
\newcommand{\Oa}{O_\alpha}
\newcommand{\Od}{O_\alpha^{\dagger}}
\newcommand{\Oap}{O_{\alpha '}}
\newcommand{\Odp}{O_{\alpha '}^{\dagger}}
\newcommand{\im}[1]{\mathrm{im} \: #1}
\renewcommand{\ker}[1]{\mathrm{ker} \: #1}
\newcommand{\ket}[1]{\left| #1 \right>}
\renewcommand{\bra}[1]{\left< #1 \right|}
\newcommand{\inner}[2]{\left< #1 | #2 \right>}
\newcommand{\expect}[2]{\left< #1 \right| #2 \left| #1 \right>}
\renewcommand{\d}[1]{ \mathrm{d}#1 \:}
\newcommand{\dn}[2]{ \mathrm{d}^{#1} #2 \:}
\newcommand{\deriv}[2]{\frac{\d{#1}}{\d{#2}}}
\newcommand{\nderiv}[3]{\frac{\dn{#1}{#2}}{\d{#3^{#1}}}}
\newcommand{\pderiv}[2]{\frac{\partial{#1}}{\partial{#2}}}
\newcommand{\fderiv}[2]{\frac{\delta #1}{\delta #2}}
\newcommand{\parsq}[2]{\frac{\partial^2{#1}}{\partial{#2}^2}}
\newcommand{\topo}{\mathcal{T}}
\newcommand{\base}{\mathcal{B}}
\renewcommand{\bf}[1]{\mathbf{#1}}
\renewcommand{\a}{\hat{a}}
\newcommand{\adag}{\hat{a}^\dagger}
\renewcommand{\b}{\hat{b}}
\newcommand{\bdag}{\hat{b}^\dagger}
\renewcommand{\c}{\hat{c}}
\newcommand{\cdag}{\hat{c}^\dagger}
\newcommand{\hamilt}{\hat{H}}
\renewcommand{\L}{\hat{L}}
\newcommand{\Lz}{\hat{L}_z}
\newcommand{\Lsquared}{\hat{L}^2}
\renewcommand{\S}{\hat{S}}
\renewcommand{\empty}{\varnothing}
\newcommand{\J}{\hat{J}}
\newcommand{\lagrange}{\mathcal{L}}
\newcommand{\dfourx}{\mathrm{d}^4x}
\newcommand{\meson}{\phi}
\newcommand{\dpsi}{\psi^\dagger}
\newcommand{\ipic}{\mathrm{int}}
\newcommand{\tr}[1]{\mathrm{tr} \left( #1 \right)}
\newcommand{\C}{\mathbb{C}}
\newcommand{\CP}[1]{\mathbb{CP}^{#1}}
\newcommand{\Vol}[1]{\mathrm{Vol}\left(#1\right)}

\newcommand{\Tr}[1]{\mathrm{Tr}\left( #1 \right)}
\newcommand{\Charge}{\hat{\mathbf{C}}}
\newcommand{\Parity}{\hat{\mathbf{P}}}
\newcommand{\Time}{\hat{\mathbf{T}}}
\newcommand{\Torder}[1]{\mathbf{T}\left[ #1 \right]}
\newcommand{\Norder}[1]{\mathbf{N}\left[ #1 \right]}
\newcommand{\Znorm}{\mathcal{Z}}
\newcommand{\EV}[1]{\left< #1 \right>}
\newcommand{\interact}{\mathrm{int}}
\newcommand{\covD}{\mathcal{D}}
\newcommand{\conj}[1]{\overline{#1}}

\newcommand{\SO}[2]{\mathrm{SO}(#1, #2)}
\newcommand{\SU}[2]{\mathrm{SU}(#1, #2)}

\newcommand{\anticom}[2]{\left\{ #1 , #2 \right\}}


\newcommand{\pathd}[1]{\! \mathdutchcal{D} #1 \:}

\renewcommand{\theenumi}{(\alph{enumi})}


\renewcommand{\theenumi}{(\alph{enumi})}

\newcommand{\atitle}[1]{\title{% 
	\large \textbf{Physics GR8048 Quantum Field Theory II
	\\ Assignment \# #1} \vspace{-2ex}}
\author{Benjamin Church }
\maketitle}

\newcommand{\atitleIII}[1]{\title{% 
	\large \textbf{Physics GR8049 Quantum Field Theory III
	\\ Assignment \# #1} \vspace{-2ex}}
\author{Benjamin Church }
\maketitle}

\theoremstyle{definition}
\newtheorem{theorem}{Theorem}[section]
\newtheorem{definition}{definition}[section]
\newtheorem{lemma}[theorem]{Lemma}
\newtheorem{proposition}[theorem]{Proposition}
\newtheorem{corollary}[theorem]{Corollary}
\newtheorem{example}[theorem]{Example}
\newtheorem{remark}[theorem]{Remark}

\usepackage{cancel}

\begin{document}
\atitle{10}
 
\section*{Problem 32.}
Consider the unperturbed Hamiltonian,
\[\hamilt_0 = \frac{\hat{p}^2}{2m} + \frac{1}{2} m \omega^2 \hat{x}^2 \]
and the perturbation,
\[\hamilt = \hamilt_0 + \hat{V} \quad \text{with} \quad \hat{V} = \lambda  \hat{x}\]
The eigenstates of the unperturbed Hamiltonian are, 
\[\ket{n} = \frac{(\adag)^n}{\sqrt{n}} \ket{0}\]
We can write $\hat{x} = \sqrt{\frac{\hbar}{2 m \omega}} (\adag + \a)$ and then calculate the matrix elements of the operator $\hat{x}$ as follows,
\begin{align*}
\bra{n} \hat{x} \ket{k} &= \sqrt{\frac{\hbar}{2 m \omega}} \bra{n} (\adag + \a) \ket{k} = \sqrt{\frac{\hbar}{2 m \omega}} \bra{n} \left[ \sqrt{k + 1} \ket{k + 1} + \sqrt{k} \ket{k - 1} \right] \\ &= \sqrt{\frac{\hbar}{2 m \omega}} \left[\sqrt{n} \delta_{n, k+1} + \sqrt{k} \delta_{n, k-1} \right]
\end{align*}
by the orthogonality of states with distinct eigenvalues.
Now, we calculate the first-order correction to the energies:
\[ E^{(1)}_n = \bra{n} \hat{V} \ket{n} = \lambda  \bra{n} \hat{x} \ket{n} = 0 \]
Therefore, to first-order, there is no shift in the energies. Now we proceed to second order, 
\begin{align*}
E^{(2)}_n & = \sum_{k \neq n} \frac{|\bra{n} \hat{V} \ket{k}|^2}{E^{(0)}_n - E^{(0)}_k} = \lambda^2 \sum_{k \neq n} \frac{|\bra{n} \hat{x} \ket{k}|^2}{E^{(0)}_n - E^{(0)}_k} = \frac{\lambda^2}{\hbar \omega} \sum_{k \neq n} \frac{|\bra{n} \hat{x} \ket{k}|^2}{n - k} \\
& = \frac{\lambda^2}{2m \omega^2} \sum_{k \neq n} \frac{|\sqrt{n} \delta_{n, k+1} + \sqrt{k} \delta_{n, k-1}|^2}{n - k} = \frac{\lambda^2}{2m\omega^2} \left[\frac{|\sqrt{n}|^2}{n - (n - 1)} + \frac{|\sqrt{n + 1}|^2}{n - (n + 1)} \right] \\
& = \frac{\lambda^2}{2m \omega^2} \left[n - (n + 1) \right] = - \frac{\lambda^2}{2m \omega^2}
\end{align*} 
To second order, the energies are,
\[E_n = E^{(0)}_n + E^{(1)}_n + E^{(2)}_n = \hbar \omega \left(n + \frac{1}{2}\right) - \frac{\lambda^2}{2m \omega^2}\]
Alternatively, consider the exact solutions to the hamiltonian,
\[\hamilt = \hamilt_0 + \hat{V} = \frac{\hat{p}^2}{2m} + \frac{1}{2} m \omega^2 \hat{x}^2 + \lambda \hat{x} = \frac{\hat{p}^2}{2m} + \frac{1}{2} m \omega^2 \left(\hat{x} + \frac{\lambda}{m \omega^2} \right)^2 -  \frac{\lambda^2}{2 m \omega^2} = \hbar \omega \left( \adag_\lambda \a_\lambda + \frac{1}{2} \right) - \frac{\lambda^2}{2 m \omega^2} \] 
where $\a_{\lambda}$ is the lowering operator with shifted $\hat{x}$. However, these operators are different from the standard rasing and lowering operators only by a constant and therefore they satify the same commutation relations. Thus, the spectrum of $\adag_\lambda \a_\lambda$ is identical to the standard QHO. Therefore, the energy eigenstates are,
\[E_n = \hbar \omega \left(n + \frac{1}{2} \right) - \frac{\lambda^2}{2 m \omega^2}\]
which agree exactly with the second order perturbation theory correction. 

\section*{Problem 33.}

Consider the Hamiltonian, $\hamilt = \hamilt_0 + \lambda \hat{V}$ and assume that $\hamilt_0$ has no degeneracy. 
\subsection*{(a)}

Suppose that the energies and eigenstates are expanded in power series. Now, we expand the equation,
\[\hamilt \ket{\psi_n} = E_n \ket{\psi_n}\]
as a power series in $\lambda$. 
\begin{align*}
(\hamilt_0 + \lambda \hat{V}) & \bigg(\ket{\psi^{(0)}_n} + \lambda \ket{\psi^{(1)}_n} + \lambda^2 \ket{\psi^{(2)}_n} + \lambda^3 \ket{\psi^{(3)}_n} + \cdots \bigg) \\ & = \bigg(E^{(0)}_n + \lambda E^{(1)}_n + \lambda^2 E^{(2)}_n + \lambda^3 E^{(3)}_n + \cdots \bigg) \bigg(\ket{\psi^{(0)}_n} + \lambda \ket{\psi^{(1)}_n} + \lambda^2 \ket{\psi^{(2)}_n} + \lambda^3 \ket{\psi^{(3)}_n} + \cdots \bigg)
\end{align*}
Now, we collect the terms of equal order in $\lambda$,
\begin{align*}
\hamilt_0 \ket{\psi^{(0)}_n} &= E^{(0)}_n \ket{\psi^{(0)}_n} \\
\hamilt_0 \ket{\psi^{(1)}_n} + \hat{V} \ket{\psi^{(0)}_n} &= E^{(0)}_n \ket{\psi^{(1)}_n} + E^{(1)}_n \ket{\psi^{(0)}_n}  \\
\hamilt_0 \ket{\psi^{(2)}_n} + \hat{V} \ket{\psi^{(1)}_n} &= E^{(0)}_n \ket{\psi^{(2)}_n} + E^{(1)}_n \ket{\psi^{(1)}_n} + E^{(2)}_n \ket{\psi^{(0)}_n}  \\
\hamilt_0 \ket{\psi^{(3)}_n} + \hat{V} \ket{\psi^{(2)}_n} &= E^{(0)}_n \ket{\psi^{(3)}_n} + E^{(1)}_n \ket{\psi^{(2)}_n} + E^{(2)}_n \ket{\psi^{(1)}_n} + E^{(3)}_n \ket{\psi^{(0)}_n}
\end{align*}
From the zero order terms, we have that the zero order eigenstates are the eigenstates of the unperturbed hamiltonian. Now, define the overlap between the pertrubed states and the origional eigenstates and the matrix elements of $\hat{V}$,
\begin{align*}
S^{(r)}_{nm} = \left< \psi^{(0)}_n \, \Big| \, \psi^{(r)}_{m} \right> \quad \quad V_{nm} = \bra{\psi^{(0)}_n} \hat{V} \ket{\psi^{(0)}_{m}}
\end{align*}
Now, take the operlap of the first-order terms with $\bra{\psi^{(0)}_n}$ i.e.
\begin{align*}
\bra{\psi^{(0)}_n} \hamilt_0 \ket{\psi^{(1)}_n} + \bra{\psi^{(0)}_n} \hat{V} \ket{\psi^{(0)}_n} &= \bra{\psi^{(0)}_n} E^{(0)}_n \ket{\psi^{(1)}_n} + \bra{\psi^{(0)}_n} E^{(1)}_n \ket{\psi^{(0)}_n} \\
\cancel{E^{(0)}_n \inner{\psi^{(0)}_n}{\psi^{(1)}_n}} + \bra{\psi^{(0)}_n} \hat{V} \ket{\psi^{(0)}_n} &= \cancel{E^{(0)}_n  \inner{\psi^{(0)}_n}{\psi^{(1)}_n}} + E^{(1)}_n \inner{\psi^{(0)}_n} {\psi^{(0)}_n} 
\end{align*}
Therefore,
\[E_n^{(1)} = \bra{\psi^{(0)}_n} \hat{V} \ket{\psi^{(0)}_n}\]
Now, take the operlap of the first-order terms with a state with distinct eigenvalue, 
\begin{align*}
\bra{\psi^{(0)}_n} \hamilt_0 \ket{\psi^{(1)}_m} + \bra{\psi^{(0)}_n} \hat{V} \ket{\psi^{(0)}_m} &= \bra{\psi^{(0)}_n} E^{(0)}_m \ket{\psi^{(1)}_m} + \bra{\psi^{(0)}_n} E^{(1)}_m \ket{\psi^{(0)}_m} \\
E^{(0)}_n \inner{\psi^{(0)}_n}{\psi^{(1)}_m} + \bra{\psi^{(0)}_n} \hat{V} \ket{\psi^{(0)}_m} &= E^{(0)}_m  \inner{\psi^{(0)}_n}{\psi^{(1)}_m} + E^{(1)}_m \cancel{\inner{\psi^{(0)}_n} {\psi^{(0)}_m}} 
\end{align*}
Therefore,
\[ S^{(1)}_{nm} = \inner{\psi^{(0)}_n}{\psi^{(1)}_m} = \frac{\bra{\psi^{(0)}_n} \hat{V} \ket{\psi^{(0)}_m}}{E^{(0)}_m - E^{(0)}_n}\]
We normalize the states to first-order, 
\begin{align*}
\left(\bra{\psi^{(0)}_n} + \lambda \bra{\psi^{(1)}_n} \right) \left( \ket{\psi^{(0)}_n} + \lambda \ket{\psi^{(1)}_n} \right) & = \inner{\psi^{(0)}_n}{\psi^{(0)}_n} + \lambda \left[\inner{\psi^{(0)}_n}{\psi^{(1)}_n} + \inner{\psi^{(1)}_n}{\psi^{(0)}_n} \right] + \lambda^2 \inner{\psi^{(1)}_n}{\psi^{(1)}_n} \\
& = 1 + 2\lambda \: \mathfrak{Re}\left[ S_{nn}^{(1)} \right] + \lambda^2 \inner{\psi^{(1)}_n}{\psi^{(1)}_n} = 1
\end{align*}
Therefore taking terms to first-order, $S_{nn}^{(1)} = 0$ since we take the overlap of the perturbed and zero-order states to be real. Now, take the operlap of the second-order terms,
\begin{align*}
\bra{\psi^{(0)}_n} \hamilt_0 \ket{\psi^{(2)}_n} + \bra{\psi^{(0)}_n} \hat{V} \ket{\psi^{(1)}_n} &= \bra{\psi^{(0)}_n} E^{(0)}_n \ket{\psi^{(2)}_n} + \bra{\psi^{(0)}_n} E^{(1)}_n \ket{\psi^{(1)}_n} + \bra{\psi^{(0)}_n} E^{(2)}_n \ket{\psi^{(0)}_n} \\
\cancel{E^{(0)}_n \inner{\psi^{(0)}_n}{\psi^{(2)}_n}} + \bra{\psi^{(0)}_n} \hat{V} \ket{\psi^{(1)}_n} &= \cancel{E^{(0)}_n \inner{\psi^{(0)}_n}{\psi^{(2)}_n}} + E^{(1)}_n \cancel{\inner{\psi^{(0)}_n}{\psi^{(1)}_n}} + E^{(2)}_n \inner{\psi^{(0)}_n}{\psi^{(0)}_n} 
\end{align*}
Therefore,
\begin{align*}
E_n^{(2)} &= \bra{\psi^{(0)}_n} \hat{V} \ket{\psi^{(1)}_n} = \sum_{m = 0}^\infty \bra{\psi^{(0)}_n} \hat{V} \ket{\psi^{(0)}_m} \inner{\psi^{(0)}_m}{\psi^{(1)}_n} = \sum_{m \neq n} \frac{\left|\bra{\psi^{(0)}_n} \hat{V} \ket{\psi^{(0)}_m} \right|^2}{E^{(0)}_n - E^{(0)}_m}
\end{align*}
Now, take the operlap of the first-order terms with a state with distinct eigenvalue, 
\begin{align*}
\bra{\psi^{(0)}_n} \hamilt_0 \ket{\psi^{(2)}_m} + \bra{\psi^{(0)}_n} \hat{V} \ket{\psi^{(1)}_m} &= \bra{\psi^{(0)}_n} E^{(0)}_m \ket{\psi^{(2)}_m} + \bra{\psi^{(0)}_n} E^{(1)}_m \ket{\psi^{(1)}_m} + \bra{\psi^{(0)}_n} E^{(2)}_m \ket{\psi^{(0)}_m} \\
E^{(0)}_n \inner{\psi^{(0)}_n}{\psi^{(2)}_m} + \bra{\psi^{(0)}_n} \hat{V} \ket{\psi^{(1)}_m} &= E^{(0)}_m \inner{\psi^{(0)}_n}{\psi^{(2)}_m} + E^{(1)}_m \inner{\psi^{(0)}_n}{\psi^{(1)}_m} + E^{(2)}_m \cancel{\inner{\psi^{(0)}_n}{\psi^{(0)}_m}} 
\end{align*}
Therefore,
\begin{align*}
S^{(2)}_{nm} &= \frac{\bra{\psi^{(0)}_n} \hat{V} \ket{\psi^{(1)}_m}}{E^{(0)}_m - E^{(0)}_n} - \frac{E^{(1)}_m \inner{\psi^{(0)}_n}{\psi^{(1)}_m}}{E^{(0)}_m - E^{(0)}_n} =   \sum_{r = 0}^\infty \frac{\bra{\psi^{(0)}_n} \hat{V} \ket{\psi^{(0)}_r} \inner{\psi^{(0)}_r}{\psi^{(1)}_m}}{E^{(0)}_m - E^{(0)}_n} - \frac{E^{(1)}_m \inner{\psi^{(0)}_n}{\psi^{(1)}_m}}{E^{(0)}_m - E^{(0)}_n} \\ & = \sum_{r \neq m} \frac{\bra{\psi^{(0)}_n} \hat{V} \ket{\psi^{(0)}_r} \bra{\psi^{(0)}_r} \hat{V} \ket{\psi^{(0)}_m}}{\left(E^{(0)}_m - E^{(0)}_n \right)\left(E^{(0)}_m - E^{(0)}_r \right)} - \frac{\bra{\psi^{(0)}_m} \hat{V} \ket{\psi^{(0)}_m} \bra{\psi^{(0)}_n} \hat{V} \ket{\psi^{(0)}_m}}{\left(E^{(0)}_m - E^{(0)}_n \right)\left(E^{(0)}_m - E^{(0)}_n \right)}
\end{align*}
We normalize the states to second-order, 
\begin{align*}
\left(\bra{\psi^{(0)}_n} + \lambda \bra{\psi^{(1)}_n} + \lambda^2 \bra{\psi^{(2)}_n} \right) & \left( \ket{\psi^{(0)}_n} + \lambda \ket{\psi^{(1)}_n} + \lambda^2 \ket{\psi^{(2)}_n} \right) \\ & = \inner{\psi^{(0)}_n}{\psi^{(0)}_n} + 2\lambda \inner{\psi^{(0)}_n}{\psi^{(1)}_n} + \lambda^2 \left[ \inner{\psi^{(1)}_n}{\psi^{(1)}_n}  + 2 \inner{\psi^{(0)}_n}{\psi^{(2)}_n} \right] + O(\lambda^3) \\
\end{align*}
Were we have taken the operlap between orders to be real. Now, because $\inner{\psi^{(0)}_n}{\psi^{(0)}_n} = 1$ and $\inner{\psi^{(0)}_n}{\psi^{(1)}_n} = 0$ we have,
\[ \inner{\psi^{(1)}_n}{\psi^{(1)}_n} + 2 \inner{\psi^{(0)}_n}{\psi^{(2)}_n} = 0\]
Therefore,
\[S^{(2)}_{nn} = \inner{\psi^{(0)}_n}{\psi^{(2)}_n} = -\tfrac{1}{2} \inner{\psi^{(1)}_n}{\psi^{(1)}_n} = -\frac{1}{2} \sum_{m \neq n} \frac{\left|\bra{\psi^{(0)}_n} \hat{V} \ket{\psi^{(0)}_m} \right|^2}{\left(E^{(0)}_n - E^{(0)}_m \right)^2}\]
Finally, take the operlap of the third-order terms with the unperturbed eigenvectors
\begin{align*}
\bra{\psi^{(0)}_n} \hamilt_0 \ket{\psi^{(3)}_n} + \bra{\psi^{(0)}_n} \hat{V} \ket{\psi^{(2)}_n} &= E^{(0)}_n \inner{\psi^{(0)}_n}{\psi^{(3)}_n} +  E^{(1)}_n \inner{\psi^{(0)}_n}{\psi^{(2)}_n} +  E^{(2)}_n \inner{\psi^{(0)}_n}{\psi^{(1)}_n} + E^{(3)}_n \inner{\psi^{(0)}_n}{\psi^{(0)}_n} \\
\cancel{E^{(0)}_n \inner{\psi^{(0)}_n}{\psi^{(3)}_n}} + \bra{\psi^{(0)}_n} \hat{V} \ket{\psi^{(2)}_n} &= \cancel{E^{(0)}_n \inner{\psi^{(0)}_n}{\psi^{(3)}_n}} +  E^{(1)}_n \inner{\psi^{(0)}_n}{\psi^{(2)}_n} +  E^{(2)}_n \cancel{\inner{\psi^{(0)}_n}{\psi^{(1)}_n}} + E^{(3)}_n \inner{\psi^{(0)}_n}{\psi^{(0)}_n}
\end{align*}
Therefore,
\begin{align*}
E^{(3)}_n & = \bra{\psi^{(0)}_n} \hat{V} \ket{\psi^{(2)}_n} - E^{(1)}_n \inner{\psi^{(0)}_n}{\psi^{(2)}_n} = \sum_{m = 0}^\infty \bra{\psi^{(0)}_n} \hat{V} \ket{\psi^{(0)}_m} \inner{\psi^{(0)}_m}{\psi^{(2)}_n} - E^{(1)}_n \inner{\psi^{(0)}_n}{\psi^{(2)}_n} \\
& = \sum_{m \neq n} \bra{\psi^{(0)}_n} \hat{V} \ket{\psi^{(0)}_m} \inner{\psi^{(0)}_m}{\psi^{(2)}_n} + \cancel{\bra{\psi^{(0)}_n} \hat{V} \ket{\psi^{(0)}_n} \inner{\psi^{(0)}_n}{\psi^{(2)}_n}} - \cancel{E^{(1)}_n \inner{\psi^{(0)}_n}{\psi^{(2)}_n}} \\
& = \sum_{m \neq n} \bra{\psi^{(0)}_n} \hat{V} \ket{\psi^{(0)}_m} \left[ \sum_{r \neq n} \frac{ \bra{\psi^{(0)}_m} \hat{V} \ket{\psi^{(0)}_r} \bra{\psi^{(0)}_r} \hat{V} \ket{\psi^{(0)}_n}}{\left(E^{(0)}_n - E^{(0)}_m \right)\left(E^{(0)}_n - E^{(0)}_r \right)} - \frac{ \bra{\psi^{(0)}_n} \hat{V} \ket{\psi^{(0)}_n} \bra{\psi^{(0)}_m} \hat{V} \ket{\psi^{(0)}_n}}{\left(E^{(0)}_n - E^{(0)}_m \right)\left(E^{(0)}_n - E^{(0)}_m \right)}\right] \\
& = \sum_{m \neq n} \left[ \sum_{r \neq n} \frac{V_{nm} V_{mr} V_{rn}}{\left(E^{(0)}_n - E^{(0)}_m \right)\left(E^{(0)}_n - E^{(0)}_r \right)} - \frac{ V_{nm} V_{nn} V_{mn}}{\left(E^{(0)}_n - E^{(0)}_m \right)\left(E^{(0)}_n - E^{(0)}_m \right)}\right]
\end{align*}
\subsection*{(b)} 
Begin again with the equation,
\[\left(\hamilt_0 + \lambda \hat{V} \right) \ket{\psi_n} = E_n \ket{\psi_n}\]
Now, we take the inner product with the eigenstates of the unperturbed Hamiltonian,
\[ \bra{\phi_n} \hamilt_0 + \lambda \hat{V} \ket{\psi_m} = \bra{\phi_n} E_m \ket{\psi_m}= E_m \inner{\phi_n}{\psi_m}\]
We choose to keep the states $\ket{\psi_n}$ unnormalized such that $\inner{\phi_n}{\psi_n} = 1$. Thus,
\[ E^{(0)}_n \inner{\phi_n}{\psi_m} + \lambda \bra{\phi_n} \hat{V} \ket{\psi_m} = E_m \inner{\phi_n}{\psi_m}\]
For $n = m$,
\[E_n - E^{(0)}_n = \lambda \bra{\phi_n} \hat{V} \ket{\psi_n} = \sum_{m = 0}^{\infty} \lambda \bra{\phi_n} \hat{V} \ket{\phi_m} \inner{\phi_m}{\psi_n}  = \lambda V_{nn} + \lambda \sum_{m \neq n} V_{nm} S_{mn} \]
For $n \neq m$,
\[S_{nm} = \inner{\phi_n}{\psi_m} = \lambda \frac{\bra{\phi_n} \hat{V} \ket{\psi_m}}{E_m - E^{(0)}_n} = \sum_{r = 0}^{\infty} \lambda \frac{\bra{\phi_n} \hat{V} \ket{\phi_r} \inner{\phi_r}{\psi_m}}{E_m - E^{(0)}_n} = \lambda \frac{V_{nm}}{E_m - E^{(0)}_n} + \lambda \sum_{r \neq m}^{\infty}\frac{V_{nr} S_{rm}}{E_m - E^{(0)}_n} \]
These equations are iterated starting from $\ket{\psi_n} = \ket{\phi_n}$ and $E_n = E^{(0)}_n$,
\[E_n = E^{(0)}_n + \lambda \bra{\phi_n} \hat{V} \ket{\phi_n} \quad \text{and} \quad S_{nm} = \lambda \frac{\bra{\phi_n} \hat{V} \ket{\phi_m}}{E^{(0)}_m - E^{(0)}_n}\]
Then using these values, we iterate to second order,
\[E_n = E^{(0)}_n + \lambda \bra{\phi_n} \hat{V} \ket{\phi_n} + \lambda^2 \sum_{m \neq n} \frac{\left| \bra{\phi_n} \hat{V} \ket{\phi_m} \right|^2}{E^{(0)}_n - E^{(0)}_m}\]
\begin{align*}
S_{nm} & = \lambda \frac{V_{nm}}{E^{(0)}_m + \lambda V_{mm} - E^{(0)}_n} + \lambda^2 \sum_{r \neq m} \frac{V_{nr} V_{rm}}{\left(E^{(0)}_m + \lambda V_{mm} - E^{(0)}_n \right)\left(E^{(0)}_m - E^{(0)}_r \right)} \\
& = \lambda \frac{V_{nm}}{E^{(0)}_m - E^{(0)}_n} - \lambda^2 \frac{V_{nm} V_{mm}}{\left(E^{(0)}_m - E^{(0)}_n \right)^2} + \lambda^2 \sum_{r \neq m} \frac{V_{nr} V_{rm}}{\left(E^{(0)}_m - E^{(0)}_n \right)\left(E^{(0)}_m - E^{(0)}_r \right)}
\end{align*}
Finally, we use these values to iterate the energy equation one more time,
\begin{align*}
E_n & = E^{(0)}_n + \lambda V_{nn} + \lambda \sum_{m \neq n} V_{nm} \left[ \lambda \frac{V_{mn}}{E^{(0)}_n - E^{(0)}_m} - \lambda^2 \frac{V_{mn} V_{nn}}{\left(E^{(0)}_n - E^{(0)}_m \right)^2} + \lambda^2 \sum_{r \neq n} \frac{V_{mr} V_{rn}}{\left(E^{(0)}_n - E^{(0)}_m \right)\left(E^{(0)}_n - E^{(0)}_r \right)} \right] \\
& = E^{(0)}_n + \lambda V_{nn} + \lambda^2 \sum_{m \neq n} \frac{V_{nm} V_{mn}}{E^{(0)}_n - E^{(0)}_m} + \lambda^3 \sum_{m \neq n} \left[\sum_{r \neq n} \frac{V_{nm} V_{mr} V_{rn}}{\left(E^{(0)}_n - E^{(0)}_m \right)\left(E^{(0)}_n - E^{(0)}_r \right)} - \frac{V_{nm} V_{mn} V_{nn}}{\left(E^{(0)}_n - E^{(0)}_m \right)^2} \right]
\end{align*}
which agrees with the terms up to third order from the Rayleigh-Schrodinger theory.

\section*{Problem 34.}

For a particle constrained to the surface of a sphere of radius $R$, the momentum is given by, \[\vec{p}^{\,2} = \frac{\L^2}{R^2}\]
Therefore, the Hamiltonian for the particle constrained to the sphere is,
\[\hamilt = \frac{\L^2}{2mR^2} + mgz = \frac{\L^2}{2mR^2} + mg R \cos{\theta} = \frac{\hbar^2}{mR^2} \left[ \frac{\L^2}{2 \hbar^2} + \lambda \cos{\theta} \right] \] 
where $\lambda = m^2 g R^3/\hbar^2 \ll 1$. 

\subsection*{(a)}
For $\lambda = 0$, the eigenstates are exactly the eigenstates of $\L^2$ which are the spherical harmonics,
\[ Y_\ell^m(\theta, \phi) = (-1)^m \sqrt{\frac{2\ell + 1}{4\pi} \frac{(\ell - m)!}{(\ell + m)!}} P^m_\ell(\cos{\theta}) e^{i m \phi} \]
where $P^m_\ell(x)$ are the associated Legendre polynomials. The corresponding energy eigenstates are given by,
\[\hamilt_0 Y_\ell^m = \frac{\L^2}{2mR^2} Y_\ell^m = \frac{\hbar^2 \ell(\ell + 1)}{2 m R^2} Y_\ell^m \]
Therefore, 
\[E^{(0)}_\ell = \frac{\hbar^2 \ell(\ell + 1)}{2 m R^2}\]
The eigenstates corresponding to the two lowest eigenvalues are,
\begin{align*}
E^{(0)}_0 &= 0 \quad \text{with} \quad Y^0_0(\theta, \phi) = \sqrt{\frac{1}{4 \pi}} \\
E^{(0)}_1 &= \frac{\hbar^2}{m R^2} \quad \text{with} \quad
\begin{cases}
Y^1_1(\theta, \phi) = -\frac{1}{2}  \sqrt{\frac{3}{2 \pi}} \sin{\theta} e^{i \phi} \\
Y^0_1(\theta, \phi) = \frac{1}{4} \sqrt{\frac{3}{\pi}} \cos{\theta} \\
Y^{-1}_1(\theta, \phi) = \frac{1}{2}  \sqrt{\frac{3}{2 \pi}} \sin{\theta} e^{-i \phi}
\end{cases}
\end{align*}

\subsection*{(b)}
We view the gravitational potential as a small perturbing term $\hat{V} = \frac{\hbar^2}{mR^2} \cos{\theta}$ so that the total Hamiltonian is equal to,
\[\hamilt = \hamilt_0 + \lambda \hat{V}\] 
To find the corrections to the energies from perturbation theory, we must compute the matrix elements of the term $\hat{V}$. 
\begin{align*}
\bra{\psi^{(0)}_{\ell' m'}} \hat{V} \ket{\psi^{(0)}_{\ell m}} & = \frac{\hbar^2}{mR^2}\int_{0}^{\pi} \! \sin{\theta} \, \d{\theta} \int_{0}^{2 \pi} \! \d{\phi} \: (Y_{\ell'}^{m'}(\theta, \phi))^{*} \cos{\theta} \: Y_\ell^m(\theta, \phi) = (-1)^{m + m'} \frac{\hbar^2 \sqrt{(2 \ell' + 1) (2 \ell + 1)}}{4 \pi mR^2} \\ & \quad \cdot \sqrt{\frac{(\ell' - m')!}{(\ell' + m')!} \frac{(\ell - m)!}{(\ell + m)!}}  \int_{0}^{\pi} \! P^{m'}_{\ell'}(\cos{\theta}) P^m_\ell(\cos{\theta}) \cos{\theta} \sin{\theta} \, \d{\theta} \int_0^{2 \pi} \!\! e^{i (m - m')\phi} \: \d{\phi} 
\end{align*}
However,
\[ \int_0^{2 \pi} \!\! e^{i (m - m')\phi} \: \d{\phi} = 
\begin{cases} 
2 \pi & m = m' \\
0 & m \neq m'
\end{cases}\]
which means we need not worry about degeracy because all the degenerate states produce zero matrix elements and therefore do not enter the summations in the correction terms. Continuing with the non-zero terms,
\begin{align*}
\bra{\psi^{(0)}_{\ell' m}} \hat{V} \ket{\psi^{(0)}_{\ell m}} & = \frac{\hbar^2 \sqrt{(2 \ell' + 1) (2 \ell + 1)}}{2 mR^2} \sqrt{\frac{(\ell' - m)!}{(\ell' + m)!} \frac{(\ell - m)!}{(\ell + m)!}}  \int_{0}^{\pi} \! P^{m}_{\ell'}(\cos{\theta}) P^m_\ell(\cos{\theta}) \cos{\theta} \sin{\theta} \, \d{\theta} \\ 
& = \frac{\hbar^2 \sqrt{(2 \ell' + 1) (2 \ell + 1)}}{2 mR^2} \sqrt{\frac{(\ell' - m)!}{(\ell' + m)!} \frac{(\ell - m)!}{(\ell + m)!}}  \int_{-1}^{1} \! P^{m}_{\ell'}(z) P^m_\ell(z) z \, \d{z} 
\end{align*}
where $z = \cos{\theta}$. Since these polynomials are always odd or even functions, when $\ell = \ell'$ we have the integral of the odd function, $P^{m}_{\ell}(z)^2 \: z $, over a symmetric interval about $0$. Therefore,  
\[E^{(1)}_{\ell m} = \bra{\psi^{(0)}_{\ell m}} \hat{V} \ket{\psi^{(0)}_{\ell m}} = 0\]
To first order, the energies of all eigenstates are unchanged from their unperturbed values. 
\subsection*{(c)}
We will exploit the orthogonality condition of the assoicated Legendre polynomials,
\[ \int_{-1}^{1} \! P^{m}_{\ell'}(z) P^m_\ell(z) \, \d{z} = \frac{2 (\ell + m)!}{(2 \ell + 1)(\ell - m)!} \delta_{\ell' \ell}\]
to find the matrix elements in the important special cases to calculate the second order energy. First, take $\ell' = 0$ and $m = 0$ then $P^0_0(z) = 1$ and $P^0_1(z) = z$ so,
\[\int_{-1}^{1} \! P^{0}_{0}(z) P^0_\ell(z) z \, \d{z} = \int_{-1}^{1} \! P^0_\ell(z) P^0_1(z) \, \d{z} = \frac{2}{3} \delta_{\ell, 1}\]
Therefore,
\[\bra{\psi^{(0)}_{00}} \hat{V} \ket{\psi^{(0)}_{\ell 0}} = \frac{1}{\sqrt{3}}\frac{\hbar^2}{mR^2} \delta_{\ell,1}\] 
so
\[ E^{(2)}_{00} = \sum_{\ell' \neq 0} \frac{\left|\bra{\psi^{(0)}_{00}} \hat{V} \ket{\psi^{(0)}_{\ell'0}} \right|^2}{E^{(0)}_{0} - E^{(0)}_{\ell'}} = - \frac{\left|\bra{\psi^{(0)}_{00}} \hat{V} \ket{\psi^{(0)}_{10}} \right|^2}{E^{(0)}_{1}} = - \frac{\hbar^2}{3mR^2}\]
Therefore, to second order in $\lambda$,
\[E_{00} = E^{(0)}_0 + \lambda E^{(1)}_{00} + \lambda^2 E^{(2)}_{00} = - \lambda^2 \frac{\hbar^2}{mR^2} = - \frac{m^3 g^2 R^4}{3\hbar^2} \]
Now, take $\ell' = 1$ and $m = 0$ then since $P^0_1(z) = z$ and $z^2 = \tfrac{1}{3}\left(2P^0_2(z) + 1 \right)$,
\[\int_{-1}^{1} \! P^{0}_{1}(z) P^0_\ell(z) z \, \d{z} = \int_{-1}^{1} \! P^0_\ell(z) \: \tfrac{1}{3}\left(2P^0_2(z) + 1 \right) \, \d{z} = \frac{2}{3} \frac{4}{10} \delta_{\ell,2} + \frac{2}{3} \delta_{\ell,0}\]
Therefore,
\[\bra{\psi^{(0)}_{10}} \hat{V} \ket{\psi^{(0)}_{\ell 0}} = \frac{\hbar^2}{mR^2} \left[ \frac{2}{\sqrt{15}} \delta_{\ell,2} + \frac{1}{\sqrt{3}} \delta_{\ell,0} \right] \] 
so
\[ E^{(2)}_{10} = \sum_{\ell' \neq 1} \frac{\left|\bra{\psi^{(0)}_{10}} \hat{V} \ket{\psi^{(0)}_{\ell'0}} \right|^2}{E^{(0)}_{1} - E^{(0)}_{\ell'}} = \frac{\left|\bra{\psi^{(0)}_{10}} \hat{V} \ket{\psi^{(0)}_{00}} \right|^2}{E^{(0)}_{1}} + \frac{\left|\bra{\psi^{(0)}_{10}} \hat{V} \ket{\psi^{(0)}_{20}} \right|^2}{E^{(0)}_{1} - E^{(0)}_{2}} = \frac{\hbar^2}{mR^2} \left[\frac{1}{3} - \frac{2}{15} \right] = \frac{ \hbar^2}{5 m R^2} \]
Therefore, to second order in $\lambda$,
\[E_{10} = E^{(0)}_1 + \lambda E^{(1)}_{10} + \lambda^2 E^{(2)}_{10} = \frac{\hbar^2}{m R^2} - \lambda^2 \frac{\hbar^2}{5 mR^2} = \frac{\hbar^2}{m R^2} - \frac{m^3 g^2 R^4}{5 \hbar^2} \]
Now, take $\ell' = 1$ and $m = 1$ then since $P^{1}_1(z)z = \frac{1}{3} P^1_2(z)$,
\[\int_{-1}^{1} \! P^{1}_{1}(z) P^1_\ell(z) z \, \d{z} = \frac{1}{3} \int_{-1}^{1} \! P^1_\ell(z) P^1_2(z)  \, \d{z} = \frac{4}{5} \delta_{\ell,2}\]
Therefore,
\[\bra{\psi^{(0)}_{11}} \hat{V} \ket{\psi^{(0)}_{\ell 1}} = \frac{1}{\sqrt{5}} \frac{\hbar^2}{ mR^2} \delta_{\ell,2} \] 
so
\[ E^{(2)}_{11} = \sum_{\ell' \neq 1} \frac{\left|\bra{\psi^{(0)}_{11}} \hat{V} \ket{\psi^{(0)}_{\ell'1}} \right|^2}{E^{(0)}_{1} - E^{(0)}_{\ell'}} = \frac{\left|\bra{\psi^{(0)}_{11}} \hat{V} \ket{\psi^{(0)}_{21}} \right|^2}{E^{(0)}_{1} - E^{(2)}_2} = - \frac{1}{10} \frac{\hbar^2}{mR^2} \]
Therefore, to second order in $\lambda$,
\[E_{11} = E^{(0)}_1 + \lambda E^{(1)}_{11} + \lambda^2 E^{(2)}_{11} = \frac{\hbar^2}{m R^2} - \lambda^2 \frac{\hbar^2}{10 m R^2} = \frac{\hbar^2}{m R^2} - \frac{m^3 g^2 R^4}{10 \hbar^2} \] 
Because $Y_{\ell}^{-m} = (-1)^m (Y_{\ell}^{m})^{*}$, the theta dependence of these functions is indentical up to sign. Therefore, (remembering that $m = m'$ else the inner product is zero),
\[ \bra{\psi^{(0)}_{1,-1}} \hat{V} \ket{\psi^{(0)}_{\ell, -1}} = \bra{\psi^{(0)}_{11}} \hat{V} \ket{\psi^{(0)}_{\ell 1}} = \frac{1}{\sqrt{5}} \frac{\hbar^2}{mR^2} \delta_{\ell,2}\]
Therefore, because the zeroth-order energy eigenvalues only depend on $\ell$, the $\ell = 1$ and $m = -1$ state has the same energy correction to second order in $\lambda$,
\[E_{1,-1} = E^{(0)}_1 + \lambda E^{(1)}_{1,-1} + \lambda^2 E^{(2)}_{1,-1} = \frac{\hbar^2}{m R^2} - \lambda^2 \frac{\hbar^2}{10 mR^2} = \frac{\hbar^2}{m R^2} - \frac{m^3 g^2 R^4}{10 \hbar^2} \] 
\end{document}

