\documentclass[12pt]{extarticle}
\usepackage[utf8]{inputenc}
\usepackage[english]{babel}
\usepackage[a4paper, total={7.25in, 9.5in}]{geometry}
\usepackage{tikz-feynman}
\tikzfeynmanset{compat=1.0.0} 
\usepackage{subcaption}
\usepackage{float}
\floatplacement{figure}{H}
\usepackage{simpler-wick}
\usepackage{mathrsfs}  
\usepackage{dsfont}
\usepackage{relsize}
\usepackage{tikz-cd}
\DeclareMathAlphabet{\mathdutchcal}{U}{dutchcal}{m}{n}

\usepackage{cancel}



\newcommand{\field}{\hat{\Phi}}
\newcommand{\dfield}{\hat{\Phi}^\dagger}
 
\usepackage{amsthm, amssymb, amsmath, centernot}
\usepackage{slashed}
\newcommand{\notimplies}{%
  \mathrel{{\ooalign{\hidewidth$\not\phantom{=}$\hidewidth\cr$\implies$}}}}
 
\renewcommand\qedsymbol{$\square$}
\newcommand{\cont}{$\boxtimes$}
\newcommand{\divides}{\mid}
\newcommand{\ndivides}{\centernot \mid}

\newcommand{\Integers}{\mathbb{Z}}
\newcommand{\Natural}{\mathbb{N}}
\newcommand{\Complex}{\mathbb{C}}
\newcommand{\Zplus}{\mathbb{Z}^{+}}
\newcommand{\Primes}{\mathbb{P}}
\newcommand{\Q}{\mathbb{Q}}
\newcommand{\R}{\mathbb{R}}
\newcommand{\ball}[2]{B_{#1} \! \left(#2 \right)}
\newcommand{\Rplus}{\mathbb{R}^+}
\renewcommand{\Re}[1]{\mathrm{Re}\left[ #1 \right]}
\renewcommand{\Im}[1]{\mathrm{Im}\left[ #1 \right]}
\newcommand{\Op}{\mathcal{O}}

\newcommand{\invI}[2]{#1^{-1} \left( #2 \right)}
\newcommand{\End}[1]{\text{End}\left( A \right)}
\newcommand{\legsym}[2]{\left(\frac{#1}{#2} \right)}
\renewcommand{\mod}[3]{\: #1 \equiv #2 \: \mathrm{mod} \: #3 \:}
\newcommand{\nmod}[3]{\: #1 \centernot \equiv #2 \: mod \: #3 \:}
\newcommand{\ndiv}{\hspace{-4pt}\not \divides \hspace{2pt}}
\newcommand{\finfield}[1]{\mathbb{F}_{#1}}
\newcommand{\finunits}[1]{\mathbb{F}_{#1}^{\times}}
\newcommand{\ord}[1]{\mathrm{ord}\! \left(#1 \right)}
\newcommand{\quadfield}[1]{\Q \small(\sqrt{#1} \small)}
\newcommand{\vspan}[1]{\mathrm{span}\! \left\{#1 \right\}}
\newcommand{\galgroup}[1]{Gal \small(#1 \small)}
\newcommand{\bra}[1]{\left| #1 \right>}
\newcommand{\Oa}{O_\alpha}
\newcommand{\Od}{O_\alpha^{\dagger}}
\newcommand{\Oap}{O_{\alpha '}}
\newcommand{\Odp}{O_{\alpha '}^{\dagger}}
\newcommand{\im}[1]{\mathrm{im} \: #1}
\renewcommand{\ker}[1]{\mathrm{ker} \: #1}
\newcommand{\ket}[1]{\left| #1 \right>}
\renewcommand{\bra}[1]{\left< #1 \right|}
\newcommand{\inner}[2]{\left< #1 | #2 \right>}
\newcommand{\expect}[2]{\left< #1 \right| #2 \left| #1 \right>}
\renewcommand{\d}[1]{ \mathrm{d}#1 \:}
\newcommand{\dn}[2]{ \mathrm{d}^{#1} #2 \:}
\newcommand{\deriv}[2]{\frac{\d{#1}}{\d{#2}}}
\newcommand{\nderiv}[3]{\frac{\dn{#1}{#2}}{\d{#3^{#1}}}}
\newcommand{\pderiv}[2]{\frac{\partial{#1}}{\partial{#2}}}
\newcommand{\fderiv}[2]{\frac{\delta #1}{\delta #2}}
\newcommand{\parsq}[2]{\frac{\partial^2{#1}}{\partial{#2}^2}}
\newcommand{\topo}{\mathcal{T}}
\newcommand{\base}{\mathcal{B}}
\renewcommand{\bf}[1]{\mathbf{#1}}
\renewcommand{\a}{\hat{a}}
\newcommand{\adag}{\hat{a}^\dagger}
\renewcommand{\b}{\hat{b}}
\newcommand{\bdag}{\hat{b}^\dagger}
\renewcommand{\c}{\hat{c}}
\newcommand{\cdag}{\hat{c}^\dagger}
\newcommand{\hamilt}{\hat{H}}
\renewcommand{\L}{\hat{L}}
\newcommand{\Lz}{\hat{L}_z}
\newcommand{\Lsquared}{\hat{L}^2}
\renewcommand{\S}{\hat{S}}
\renewcommand{\empty}{\varnothing}
\newcommand{\J}{\hat{J}}
\newcommand{\lagrange}{\mathcal{L}}
\newcommand{\dfourx}{\mathrm{d}^4x}
\newcommand{\meson}{\phi}
\newcommand{\dpsi}{\psi^\dagger}
\newcommand{\ipic}{\mathrm{int}}
\newcommand{\tr}[1]{\mathrm{tr} \left( #1 \right)}
\newcommand{\C}{\mathbb{C}}
\newcommand{\CP}[1]{\mathbb{CP}^{#1}}
\newcommand{\Vol}[1]{\mathrm{Vol}\left(#1\right)}

\newcommand{\Tr}[1]{\mathrm{Tr}\left( #1 \right)}
\newcommand{\Charge}{\hat{\mathbf{C}}}
\newcommand{\Parity}{\hat{\mathbf{P}}}
\newcommand{\Time}{\hat{\mathbf{T}}}
\newcommand{\Torder}[1]{\mathbf{T}\left[ #1 \right]}
\newcommand{\Norder}[1]{\mathbf{N}\left[ #1 \right]}
\newcommand{\Znorm}{\mathcal{Z}}
\newcommand{\EV}[1]{\left< #1 \right>}
\newcommand{\interact}{\mathrm{int}}
\newcommand{\covD}{\mathcal{D}}
\newcommand{\conj}[1]{\overline{#1}}

\newcommand{\SO}[2]{\mathrm{SO}(#1, #2)}
\newcommand{\SU}[2]{\mathrm{SU}(#1, #2)}

\newcommand{\anticom}[2]{\left\{ #1 , #2 \right\}}


\newcommand{\pathd}[1]{\! \mathdutchcal{D} #1 \:}

\renewcommand{\theenumi}{(\alph{enumi})}


\renewcommand{\theenumi}{(\alph{enumi})}

\newcommand{\atitle}[1]{\title{% 
	\large \textbf{Physics GR8048 Quantum Field Theory II
	\\ Assignment \# #1} \vspace{-2ex}}
\author{Benjamin Church }
\maketitle}

\newcommand{\atitleIII}[1]{\title{% 
	\large \textbf{Physics GR8049 Quantum Field Theory III
	\\ Assignment \# #1} \vspace{-2ex}}
\author{Benjamin Church }
\maketitle}

\theoremstyle{definition}
\newtheorem{theorem}{Theorem}[section]
\newtheorem{definition}{definition}[section]
\newtheorem{lemma}[theorem]{Lemma}
\newtheorem{proposition}[theorem]{Proposition}
\newtheorem{corollary}[theorem]{Corollary}
\newtheorem{example}[theorem]{Example}
\newtheorem{remark}[theorem]{Remark}


\begin{document}
\atitle{2}
 
\section*{Problem 3.}
Let an operator $O$ on a 3 dimensional vector space be given as 
\[
O =
  \begin{pmatrix}
    0 & 1 & 0 \\
    1 & 0 & 1 \\
    0 & 1 & 0
  \end{pmatrix}
\]

\begin{enumerate}
\item Let $\det{\left( I \lambda - O \right) } = 0$ then $\lambda^3 - 2 \lambda = 0$ so $\lambda = 0, \pm \sqrt{2}$ \\ \\ For $\lambda = 0$, 
\[
  \begin{pmatrix}
    0 & 1 & 0 \\
    1 & 0 & 1 \\
    0 & 1 & 0
  \end{pmatrix}
  \begin{pmatrix}
	a \\
	b \\
	c
  \end{pmatrix}
  = 
  \begin{pmatrix}
	b \\
	a + c \\
	b
  \end{pmatrix}
  =
  \begin{pmatrix}
	0 \\
	0 \\
	0
  \end{pmatrix}
\] 
Therefore, $a = -c$ and $b = 0$ so 
\[ \ket{v_0} = \frac{1}{\sqrt{2}} 
  \begin{pmatrix}
	1 \\
	0 \\
	-1
  \end{pmatrix}
\]

For $\lambda = \sqrt{2}$, 
\[
  \begin{pmatrix}
    -\sqrt{2} & 1 & 0 \\
    1 & -\sqrt{2} & 1 \\
    0 & 1 & -\sqrt{2}
  \end{pmatrix}
  \begin{pmatrix}
	a  \\
	b \\
	c
  \end{pmatrix}
  = 
  \begin{pmatrix}
	-\sqrt{2} a + b \\
	a -\sqrt{2} b + c \\
	b - \sqrt{2} c
  \end{pmatrix}
  =
  \begin{pmatrix}
	0 \\
	0 \\
	0
  \end{pmatrix}
\] 
Therefore, $b = a \sqrt{2}$ and $c = a$ so 
\[ \ket{v_{\sqrt{2}}} = \frac{1}{2} 
  \begin{pmatrix}
	1 \\
	\sqrt{2} \\
	1
  \end{pmatrix}
\]

For $\lambda = -\sqrt{2}$, 
\[
  \begin{pmatrix}
    \sqrt{2} & 1 & 0 \\
    1 & \sqrt{2} & 1 \\
    0 & 1 & \sqrt{2}
  \end{pmatrix}
  \begin{pmatrix}
	a \\
	b \\
	c
  \end{pmatrix}
  = 
  \begin{pmatrix}
	\sqrt{2} a + b \\
	a + \sqrt{2} b + c \\
	b + \sqrt{2} c
  \end{pmatrix}
  =
  \begin{pmatrix}
	0 \\
	0 \\
	0
  \end{pmatrix}
\] 
Therefore, $b = -\sqrt{2} a$ and $c = a$ so 
\[ \ket{v_{-\sqrt{2}}} = \frac{1}{2} 
  \begin{pmatrix}
	1 \\
	-\sqrt{2} \\
	1
  \end{pmatrix}
\]
\newpage
\item $\mathbf{P}\left(\lambda = 0 \right) = \left| \inner{v_0}{\psi} \right|^2$ = $\left( \frac{1}{2}(1 + 0 + 0)  \right)^2 = 
\frac{1}{4}$ \\ \\
$\mathbf{P}\left(\lambda = \sqrt{2} \right) = \left|\inner{v_{\sqrt{2}}}{\psi} \right|^2$ = $\left( \frac{1}{2\sqrt{2}}(1 + \sqrt{2} + 0)  \right)^2 = \frac{1}{8} (3 + 2\sqrt{2})$ \\ \\
$\mathbf{P}\left(\lambda = -\sqrt{2} \right) = \left| \inner{v_{-\sqrt{2}}}{\psi} \right|^2$ = $\left( \frac{1}{2\sqrt{2}}(1 - \sqrt{2} + 0)  \right)^2 = \frac{1}{8} (3 - 2\sqrt{2})$ 

\end{enumerate}

\section*{Problem 4.}

\begin{enumerate}
\item For every $\lambda$, $E(\lambda)$ is an orthogonal projection so $E(\lambda)^2 = E(\lambda)$ and $E(\lambda)^\dagger = E(\lambda)$. Now consider $\bra{\psi} E(\lambda) \ket{\psi} = \bra{\psi} E(\lambda) E(\lambda) \ket{\psi} = \inner{E(\lambda) \psi}{E(\lambda) \psi} \ge 0$ by Hermiticity and positive definiteness. \smallskip

Also, let $\lambda_1 < \lambda_2$ then \begin{align*}
& \bra{\psi} E(\lambda_2) \ket{\psi}  = \bra{\psi} E(\lambda_2) - E(\lambda_1) \ket{\psi} + \bra{\psi} E(\lambda_1) \ket{\psi} \text{ but }  \\ & \bra{\psi} E(\lambda_2) - E(\lambda_1) \ket{\psi} = \bra{\psi} (E(\lambda_2) - E(\lambda_1))^2 \ket{\psi} = |(E(\lambda_2) - E(\lambda_1)) \ket{\psi}|^2 \ge 0 
\end{align*}

Therefore,  $\bra{\psi} E(\lambda_2) \ket{\psi} \ge \bra{\psi} E(\lambda_1) \ket{\psi}$

\item Let 
\begin{align*}
F &= \int_{-\infty}^\infty \lambda \: \mathrm{d} E_F(\lambda) \text{ then } F^2 = \int_{-\infty}^\infty \lambda \: \frac{\mathrm{d} E_F(\lambda)}{\mathrm{d} \lambda} \mathrm{d} \lambda \int_{-\infty}^\infty \lambda' \: \frac{\mathrm{d} E_F(\lambda')}{\mathrm{d} \lambda'} \mathrm{d} \lambda' \\ &= \int \! \! \int_{-\infty}^{\infty} \lambda \lambda' \ket{\lambda} \inner{\lambda}{\lambda'} \bra{\lambda'} \mathrm{d} \lambda \mathrm{d} \lambda' =  \int_{-\infty}^{\infty} \lambda^2 \ket{\lambda} \! \bra{\lambda} \mathrm{d} \lambda = \int_{-\infty}^{\infty} \lambda^2 \frac{\mathrm{d} E_F(\lambda)}{\mathrm{d} \lambda} \mathrm{d} \lambda
\end{align*}

Now, the eigenvectors of $F^2$ are $\xi = \lambda^2$. Then $F^2 = \int_{0}^{\infty} \lambda^2 \frac{\mathrm{d} E_F(\lambda)}{\mathrm{d} \lambda} \mathrm{d} \lambda + \int_{-\infty}^{0} \lambda^2 \frac{\mathrm{d} E_F(\lambda)}{\mathrm{d} \lambda} \mathrm{d} \lambda$. In the first integral, reparametrize by $\lambda = \sqrt{\xi}$ and in the second, $\lambda = -\sqrt{\xi}$. Thus, \smallskip
\begin{align*}
F^2 &= \int_{0}^{\infty} \xi \frac{\mathrm{d} E_F(\sqrt{\xi})}{\mathrm{d} \xi} \frac{\mathrm{d} \xi}{\mathrm{d} \lambda} \frac{\mathrm{d} \lambda}{\mathrm{d} \xi} \mathrm{d} \xi + \int_{\infty}^{0} \xi \frac{\mathrm{d} E_F(-\sqrt{\xi})}{\mathrm{d} \xi} \frac{\mathrm{d} \xi}{\mathrm{d} \lambda} \frac{\mathrm{d} \lambda}{\mathrm{d} \xi} \mathrm{d} \xi \\ &=
\int_{0}^{\infty} \xi \frac{\mathrm{d} E_F(\sqrt{\xi})}{\mathrm{d} \xi}  \mathrm{d} \xi - \int_{0}^{\infty} \xi \frac{\mathrm{d} E_F(-\sqrt{\xi})}{\mathrm{d} \xi}  \mathrm{d} \xi \int_{0}^{\infty} \xi  \left[\frac{\mathrm{d} E_F(\sqrt{\xi})}{\mathrm{d} \xi}  - \frac{\mathrm{d} E_F(-\sqrt{\xi})}{\mathrm{d} \xi} \right] \mathrm{d} \xi \\ &= \int_{0}^{\infty} \xi \frac{\mathrm{d}}{\mathrm{d} \xi} \left[ E(\sqrt{\xi}) - E(-\sqrt{\xi}) \right] \mathrm{d} \xi
\end{align*}

This is a resolution of the identity for $F^2$ if we let $E_{F^2}(\xi) = E_F(\sqrt{\xi}) - E(-\sqrt{\xi})$ for $\xi \ge 0$ and $E_{F^2} = \mathbf{0}$ for $\xi < 0$.  
\end{enumerate}

\section*{Problem 5.} Let both $A$ and $B$ be commuting Hermitian operators with complete spectra: 
\[A \ket{n_A} = a_n\ket{n_A} \text{ and } B \ket{n_B} = b_n \ket{n_B}\]

\begin{enumerate}

\item Suppose that $A$ has a non-degenerate spectrum. Then $AB \ket{n_A} = BA \ket{n_A} = B a_n \ket{n_A}$. Thus, $A(B \ket{n_A}) = a_n (B \ket{n_A})$ so $B \ket{n_A}$ is an eigenvector with of $A$ with eigenvalue $a_n$. Since there is no degeneracy, $B \ket{a_n} = \omega \ket{n_A}$ and therefore, $\ket{n_A}$ is also an eigenvector for $B$ so the basis $\{ \ket{n_A} \}$ consists of eigenvectors of both $A$ and $B$. 

\item let $V_\lambda^A = \{\ket{v} \in \mathcal{H} \mid A \ket{v} = \lambda \ket{v} \}$. For any $\ket{v} \in V_\lambda^A$ take $AB \ket{v} = BA \ket{v} = B \lambda \ket{v}$. Thus, $A(B \ket{v} = \lambda (B \ket{v}$ so $B \ket{v} \in V_\lambda^A$. Therefore, restricting $B$ to the subspace $V_\lambda^A$ which by assumption is finite dimensional, we get a linear map $B : V_\lambda^A \rightarrow V_\lambda^A$ which is Hermitian on finite dimensional spaces. Thus, by the finite dimensional spectral theorem (problem 6), there exists a basis of $V_\lambda^A$ consisting of eigenvectors of $B$ namely, $\{\ket{w^\lambda_1}, \dots , \ket{w^\lambda_{n_\lambda}}\}$. Now since $\vspan{\ket{w^\lambda_1}, \dots , \ket{w^\lambda_{n_\lambda}}} = V_\lambda^A$ then since every $\ket{n_A} \in V_{a_n}^A$ then \[\bigcup\limits_{\lambda \in \{a_n\}} \{\ket{w^\lambda_1}, \dots , \ket{w^\lambda_{n_\lambda}}\}\]
Is a complete set because every $\ket{n_A}$ is contained in its span. However each vector in the set is an eigenvector of $B$ by construction. Also, $\ket{w^\lambda_i} \in V_\lambda^A$ so $A \ket{w^\lambda_i} = \lambda \ket{w^\lambda_i}$ so $\ket{w^\lambda_i}$ is also an eigenvector of $A$. 

\item Since the eigenvectors of $A$ span the entire space, the problem is reduced to diagonalizing $B$ in each eigenspace of $A$. Then these vectors will be simultaneous eigenvectors of $A$ and $B$ and will space each eigenspace and thus span the entire space. Now, for any $\ket{v} \in V_\lambda^A$ take $AB \ket{v} = BA \ket{v} = B \lambda \ket{v}$. Thus, $A(B \ket{v} = \lambda (B \ket{v}$ so $B \ket{v} \in V_\lambda^A$. Since $B|_{V_\lambda^A}$ is self-adjoint, there is a resolution of the identity, \[B|_{V_\lambda^A} = \int_{-\infty}^{\infty} \lambda_B \frac{\mathrm{d} E_B(\lambda_B)}{\mathrm{d} \lambda_B} \mathrm{d} \lambda_B \]
With $E_B(\lambda_B) V_\lambda^A \subset V_\lambda^A$. Then 
\begin{align*}
B|_{V_\lambda^A} \frac{\mathrm{d} E_B(\lambda_B)}{\mathrm{d} \lambda_B} V_\lambda^A &= \int_{-\infty}^{\infty} \lambda_B \ket{\lambda_B'}\! \inner{\lambda_B'}{\lambda_B}\! \bra{\lambda_B} \mathrm{d} \lambda_B' V_\lambda^A \\ &= \lambda_B \ket{\lambda_B}\! \bra{\lambda_B} \mathrm{d} \lambda_B V_\lambda^A = \lambda_B \frac{\mathrm{d} E_B(\lambda_B)}{\mathrm{d} \lambda_B} V_\lambda^A 
\end{align*}
Thus, $\frac{\mathrm{d} E_B(\lambda_B)}{\mathrm{d} \lambda_B} V_\lambda^A$ is an eignvector of $B$. Furthermore, 
\begin{align*}
\int_{-\infty}^{\infty} \frac{\mathrm{d} E_B(\lambda_B)}{\mathrm{d} \lambda_B} V_\lambda^A \mathrm{d} \lambda_B = \int_{-\infty}^{\infty} \mathrm{d} E_B(\lambda_B) V_\lambda^A = (E(\infty) - E(-\infty))V_\lambda^A = V_\lambda^A
\end{align*}
So these eigenvectors of $B$ span the eigenspace of $V_\lambda^A$. Because we are working only in $V_\lambda^A$, these vectors are automatically eigenvectors of $A$ as well. 
\end{enumerate}



\section*{Problem 6.} Let $\dim{\mathcal{H}} = N$ and $O : \mathcal{H} \rightarrow \mathcal{H}$ be hermitian. Then let $S = \{ \ket{\psi} \in \mathcal{H} \mid \inner{\psi}{\psi} = 1 \}$ is an N-sphere and thus is compact in $\mathcal{H}$. Since $O$ is hermitian, it has real expectation values so $\expect{\psi}{O} : S \rightarrow \R$ is a continuous function by the linearity of $O$. $\expect{\psi}{O}$ is a continuous function and $S$ is compact, therefore, $\Im{\expect{\psi}{O}}$ is compact in $\R$ so it is closed and bounded and in particular much achive a minumum value $\expect{\psi_0}{O} \in \R$.

\begin{enumerate}

\item Take normalized $\ket{\delta \psi} \in \left( \vspan{\ket{\psi_0}} \right)^\perp$ and $\epsilon \in \C$ then define: \[\ket{\psi_\epsilon} = \frac{1}{\sqrt{1 + |\epsilon|^2}} \left( \ket{\psi_0} + \epsilon \ket{\delta \psi} \right) \] 

Now calculate: $\inner{\psi_\epsilon}{\psi_\epsilon} = $ \[ \frac{1}{1+|\epsilon|^2} \left(\inner{\psi_0}{\psi_0} + \epsilon \inner{\psi_0}{\delta \psi} + \epsilon^* \inner{\delta \epsilon}{\psi_0} + | \epsilon |^2 \inner{\delta \psi}{\delta \psi} \right) = \frac{1}{1 + | \epsilon |^2} \left(1 + | \epsilon |^2 \right) = 1\]
Because $\inner{\psi_0}{\delta\psi} = 0$ and $\inner{\psi}{\psi} = \inner{\delta \psi}{\delta \psi} = 1$. \\

\item By the minimum property, $\expect{\psi_\epsilon}{O} \ge \expect{\psi_0}{O}$ therefore, 
\[ \frac{1}{1+|\epsilon|^2} \left( \expect{\psi_0)}{O} + \epsilon \bra{\psi_0}O \ket{\delta\psi} + \epsilon^* \bra{\delta\psi}O\ket{\psi_0} + |\epsilon|^2 \expect{\delta \psi}{O}  \right) \ge \expect{\psi_0}{O} \]
To first order in $\epsilon$, \[2 \mathfrak{Re}\left[ \epsilon^* \bra{\delta\psi}O\ket{\psi_0} \right] \ge 0 \] Thus take $\epsilon = - \varepsilon \bra{\delta\psi}O\ket{\psi_0}$ for $\varepsilon \in \Rplus$. Therefore, \[2 \mathfrak{Re}\left[ - \varepsilon \left| \bra{\delta\psi}O\ket{\psi_0} \right|^2 \right] =  - \varepsilon \left| \bra{\delta\psi}O\ket{\psi_0} \right|^2 \ge 0\] Which is a contradiction unless $\bra{\delta\psi}O\ket{\psi_0} = 0$. \\

\item Because $\mathcal{H}$ is finite dimensional, $\mathcal{H} = W \bigoplus W^\perp$ with $W = \left( \vspan{\ket{\psi_0}} \right)^{\perp}$ and also $W^{\perp \perp} = W$ but $\forall \ket{\delta \psi} \in W : \bra{\delta \psi} O \ket{\psi_0} = 0$ therefore, $O\ket{\psi_0} \in W^{\perp} = \vspan{\ket{\psi_0}}$. But if $O\ket{\psi_0} \in \vspan{\ket{\psi_0}}$ then $O\ket{\psi_0} = \lambda \ket{\psi_0}$.

\item Take $W = \left( \vspan{\ket{\psi_0}} \right)^{\perp}$ then for $\ket{\psi} \in W$, $\inner{\psi_0}{O \psi} = \inner{O \psi_0}{\psi} = \lambda^* \inner{\psi_0}{\psi} = 0$. Thus, $\Im{O} |_W \subset W$. Therefore, $O|_W$ is a well defined operator on $W$ which inherits Hermiticity. We can apply the above argument to $W$ since $\dim{W} = N-1$ is finite and produce a new eigenvector $\ket{\psi_1} \in W$ which is perpendicular to the span of $\ket{\psi_0}$.

\item We can therefore prove the finite dimensional spectral theorem by induction on the dimension of $\mathcal{H}$. If $\dim{\mathcal{H}} = 1$ then $O \ket{v} \in \vspan{\ket{v}}$ trivially. Suppose the theorem holds on every space with $\dim{\mathcal{H}} = N$. Let $\dim{\mathcal{H}} = N + 1$. Then since $\dim{W} = N$, $W$ admits a orthonormal basis of eigenvectors of $O|_W$ namely $\{\ket{v_1}, \ket{v_2}, \dots , \ket{v_N} \}$. Then since the eigenvector found above $\ket{\psi_0} \in W^\perp$ then the set $\{\ket{v_1}, \dots, \ket{v_N}, \ket{\psi_0} \}$ is an orthonormal set of eigenvectors which are therefore independent.  

\end{enumerate}

\section*{Problem 7.} Let $\ket{\psi} = \ket{A} + \alpha \ket{B}$ where $\alpha \in \C$ then $\inner{\psi}{\psi} \ge 0$ therefore, \[\inner{A}{A} + \alpha \inner{A}{B} + \alpha^* \inner{B}{A} + \alpha^2 \inner{B}{B} = |B|^2 |\alpha|^2 + 2 \mathfrak{Re} \left[ \inner{A}{B} \alpha \right] + |A|^2 \ge 0 \]
Let $\alpha = \inner{B}{A} r$ for $r \in \R$ then because $\inner{A}{B} \inner{B}{A} \in \R$ \[ |B|^2  \:\left| \inner{A}{B} \right|^2 r^2 + 2  \inner{A}{B} \inner{B}{A} r + |A|^2 \ge 0 \]
The innequality must hold for every $r$ therefore, the discriminant of the quadratic form must be non-positive. Therefore, $4 \left| \inner{A}{B} \right|^4 - 4 |A|^2 |B|^2 \: \left| \inner{A}{B} \right|^2 \ge 0$ Thus, \[ |A||B| \ge \left| \inner{A}{B} \right| \]

\end{document}

