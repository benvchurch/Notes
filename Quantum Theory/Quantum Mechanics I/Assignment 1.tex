\documentclass[12pt]{extarticle}
\usepackage[utf8]{inputenc}
\usepackage[english]{babel}
\usepackage[a4paper, total={7.25in, 9.5in}]{geometry}
\usepackage{tikz-feynman}
\tikzfeynmanset{compat=1.0.0} 
\usepackage{subcaption}
\usepackage{float}
\floatplacement{figure}{H}
\usepackage{simpler-wick}
\usepackage{mathrsfs}  
\usepackage{dsfont}
\usepackage{relsize}
\usepackage{tikz-cd}
\DeclareMathAlphabet{\mathdutchcal}{U}{dutchcal}{m}{n}

\usepackage{cancel}



\newcommand{\field}{\hat{\Phi}}
\newcommand{\dfield}{\hat{\Phi}^\dagger}
 
\usepackage{amsthm, amssymb, amsmath, centernot}
\usepackage{slashed}
\newcommand{\notimplies}{%
  \mathrel{{\ooalign{\hidewidth$\not\phantom{=}$\hidewidth\cr$\implies$}}}}
 
\renewcommand\qedsymbol{$\square$}
\newcommand{\cont}{$\boxtimes$}
\newcommand{\divides}{\mid}
\newcommand{\ndivides}{\centernot \mid}

\newcommand{\Integers}{\mathbb{Z}}
\newcommand{\Natural}{\mathbb{N}}
\newcommand{\Complex}{\mathbb{C}}
\newcommand{\Zplus}{\mathbb{Z}^{+}}
\newcommand{\Primes}{\mathbb{P}}
\newcommand{\Q}{\mathbb{Q}}
\newcommand{\R}{\mathbb{R}}
\newcommand{\ball}[2]{B_{#1} \! \left(#2 \right)}
\newcommand{\Rplus}{\mathbb{R}^+}
\renewcommand{\Re}[1]{\mathrm{Re}\left[ #1 \right]}
\renewcommand{\Im}[1]{\mathrm{Im}\left[ #1 \right]}
\newcommand{\Op}{\mathcal{O}}

\newcommand{\invI}[2]{#1^{-1} \left( #2 \right)}
\newcommand{\End}[1]{\text{End}\left( A \right)}
\newcommand{\legsym}[2]{\left(\frac{#1}{#2} \right)}
\renewcommand{\mod}[3]{\: #1 \equiv #2 \: \mathrm{mod} \: #3 \:}
\newcommand{\nmod}[3]{\: #1 \centernot \equiv #2 \: mod \: #3 \:}
\newcommand{\ndiv}{\hspace{-4pt}\not \divides \hspace{2pt}}
\newcommand{\finfield}[1]{\mathbb{F}_{#1}}
\newcommand{\finunits}[1]{\mathbb{F}_{#1}^{\times}}
\newcommand{\ord}[1]{\mathrm{ord}\! \left(#1 \right)}
\newcommand{\quadfield}[1]{\Q \small(\sqrt{#1} \small)}
\newcommand{\vspan}[1]{\mathrm{span}\! \left\{#1 \right\}}
\newcommand{\galgroup}[1]{Gal \small(#1 \small)}
\newcommand{\bra}[1]{\left| #1 \right>}
\newcommand{\Oa}{O_\alpha}
\newcommand{\Od}{O_\alpha^{\dagger}}
\newcommand{\Oap}{O_{\alpha '}}
\newcommand{\Odp}{O_{\alpha '}^{\dagger}}
\newcommand{\im}[1]{\mathrm{im} \: #1}
\renewcommand{\ker}[1]{\mathrm{ker} \: #1}
\newcommand{\ket}[1]{\left| #1 \right>}
\renewcommand{\bra}[1]{\left< #1 \right|}
\newcommand{\inner}[2]{\left< #1 | #2 \right>}
\newcommand{\expect}[2]{\left< #1 \right| #2 \left| #1 \right>}
\renewcommand{\d}[1]{ \mathrm{d}#1 \:}
\newcommand{\dn}[2]{ \mathrm{d}^{#1} #2 \:}
\newcommand{\deriv}[2]{\frac{\d{#1}}{\d{#2}}}
\newcommand{\nderiv}[3]{\frac{\dn{#1}{#2}}{\d{#3^{#1}}}}
\newcommand{\pderiv}[2]{\frac{\partial{#1}}{\partial{#2}}}
\newcommand{\fderiv}[2]{\frac{\delta #1}{\delta #2}}
\newcommand{\parsq}[2]{\frac{\partial^2{#1}}{\partial{#2}^2}}
\newcommand{\topo}{\mathcal{T}}
\newcommand{\base}{\mathcal{B}}
\renewcommand{\bf}[1]{\mathbf{#1}}
\renewcommand{\a}{\hat{a}}
\newcommand{\adag}{\hat{a}^\dagger}
\renewcommand{\b}{\hat{b}}
\newcommand{\bdag}{\hat{b}^\dagger}
\renewcommand{\c}{\hat{c}}
\newcommand{\cdag}{\hat{c}^\dagger}
\newcommand{\hamilt}{\hat{H}}
\renewcommand{\L}{\hat{L}}
\newcommand{\Lz}{\hat{L}_z}
\newcommand{\Lsquared}{\hat{L}^2}
\renewcommand{\S}{\hat{S}}
\renewcommand{\empty}{\varnothing}
\newcommand{\J}{\hat{J}}
\newcommand{\lagrange}{\mathcal{L}}
\newcommand{\dfourx}{\mathrm{d}^4x}
\newcommand{\meson}{\phi}
\newcommand{\dpsi}{\psi^\dagger}
\newcommand{\ipic}{\mathrm{int}}
\newcommand{\tr}[1]{\mathrm{tr} \left( #1 \right)}
\newcommand{\C}{\mathbb{C}}
\newcommand{\CP}[1]{\mathbb{CP}^{#1}}
\newcommand{\Vol}[1]{\mathrm{Vol}\left(#1\right)}

\newcommand{\Tr}[1]{\mathrm{Tr}\left( #1 \right)}
\newcommand{\Charge}{\hat{\mathbf{C}}}
\newcommand{\Parity}{\hat{\mathbf{P}}}
\newcommand{\Time}{\hat{\mathbf{T}}}
\newcommand{\Torder}[1]{\mathbf{T}\left[ #1 \right]}
\newcommand{\Norder}[1]{\mathbf{N}\left[ #1 \right]}
\newcommand{\Znorm}{\mathcal{Z}}
\newcommand{\EV}[1]{\left< #1 \right>}
\newcommand{\interact}{\mathrm{int}}
\newcommand{\covD}{\mathcal{D}}
\newcommand{\conj}[1]{\overline{#1}}

\newcommand{\SO}[2]{\mathrm{SO}(#1, #2)}
\newcommand{\SU}[2]{\mathrm{SU}(#1, #2)}

\newcommand{\anticom}[2]{\left\{ #1 , #2 \right\}}


\newcommand{\pathd}[1]{\! \mathdutchcal{D} #1 \:}

\renewcommand{\theenumi}{(\alph{enumi})}


\renewcommand{\theenumi}{(\alph{enumi})}

\newcommand{\atitle}[1]{\title{% 
	\large \textbf{Physics GR8048 Quantum Field Theory II
	\\ Assignment \# #1} \vspace{-2ex}}
\author{Benjamin Church }
\maketitle}

\newcommand{\atitleIII}[1]{\title{% 
	\large \textbf{Physics GR8049 Quantum Field Theory III
	\\ Assignment \# #1} \vspace{-2ex}}
\author{Benjamin Church }
\maketitle}

\theoremstyle{definition}
\newtheorem{theorem}{Theorem}[section]
\newtheorem{definition}{definition}[section]
\newtheorem{lemma}[theorem]{Lemma}
\newtheorem{proposition}[theorem]{Proposition}
\newtheorem{corollary}[theorem]{Corollary}
\newtheorem{example}[theorem]{Example}
\newtheorem{remark}[theorem]{Remark}


\begin{document}
\atitle{1}
 
\section*{Problem 1.}
 \[\psi(x, y) = A \sum_{j = -N/2}^{N/2} e^{i \sqrt{x^2 + (y - jd)^2} \frac{p}{\hbar}} \]
\begin{enumerate}
\item $\sqrt{x^2 + (y - jd)^2} = \sqrt{x^2 + y^2 - 2yjd + j^2d^2} = r \sqrt{1 - (2yjd - j^2 d^2)/r^2} =  \\ r - jd \sin{\theta} + O(\frac{1}{r}) \approx r - jd \sin{\theta}$. Thus,  \[\psi(x, y) = A \sum_{j = -N/2}^{N/2} e^{i (r - jd \sin{\theta}) \frac{p}{\hbar}} \]

\item \[\psi(x, y) = A e^{ir \frac{p}{\hbar}} \sum_{j = -N/2}^{N/2} e^{- i jd \sin{\theta} \frac{p}{\hbar}} = A e^{i(r + \frac{N}{2} d \sin{\theta}) \frac{p}{\hbar}}  \frac{1 - e^{-i(N+1)d \sin{\theta} \frac{p}{\hbar}}}{1 - e^{-id \sin{\theta} \frac{p}{\hbar}}}  \]
Therefore, taking the norm square, $ \left| \psi(x,y) \right|^2 $
\[ = \left| A \right|^2 \left | \frac{1 - e^{-i(N+1)d \sin{\theta} \frac{p}{\hbar}}}{1 - e^{-id \sin{\theta} \frac{p}{\hbar}}}  \right|^2 = \left| A \right|^2 \left | \frac{e^{i\frac{N+1}{2}d \sin{\theta} \frac{p}{\hbar}} - e^{-i\frac{N+1}{2}d \sin{\theta} \frac{p}{\hbar}}}{e^{i \frac{1}{2} d \sin{\theta} \frac{p}{\hbar}} - e^{-i \frac{1}{2} d \sin{\theta} \frac{p}{\hbar}}}  \right|^2  = \left| A \right|^2 \frac{\sin^2{\left( \frac{N+1}{2}d \sin{\theta} \frac{p}{\hbar}\right)}}{\sin^2{\left(  \frac{1}{2} d \sin{\theta} \frac{p}{\hbar} \right)}} \] 

Take $D \gg d$ then $r \approx D$ and $\sin{\theta} = \frac{y}{D}$ so \[ \left| \psi(x,y) \right|^2 = \left| A \right|^2 \frac{\sin^2{\left(\frac{y}{D} \frac{(N+1)pd}{2\hbar} \right)}}{\sin^2{\left( \frac{y}{D} \frac{pd}{2\hbar} \right)}}  \] 

\item Let $f(y)$ be the probability amplitude of the incident wave at the screen. Also, $f(y) = 0$ for $|y| > d$ then,

\[ \left| \psi(x,y) \right|^2 = A \int_{-d}^{d} e^{i \sqrt{x^2 + (y-y')^2} \frac{p}{\hbar}} f(y')  \, \mathrm{d} y'  = A \int_{-\infty}^{\infty} e^{i \sqrt{x^2 + (y-y')^2} \frac{p}{\hbar}} f(y')  \, \mathrm{d} y' \] 

because $f(y) = 0$ on the added domain. As before, we approximate: $\sqrt{x^2 + (y-y')^2} \approx r - y' \sin{\theta}$ so $\left| \psi(x,y) \right|^2 =$


\[ \left| A \int_{-\infty}^{\infty} e^{i \frac{p}{\hbar} (r - y' \sin{\theta}}) f(y')  \, \mathrm{d} y' \right|^2 = \left| A e^{i \frac{p}{\hbar} r} \right| \left| \int_{-\infty}^{\infty} e^{- i \frac{p}{\hbar} y' \sin{\theta}} f(y')  \, \mathrm{d} y' \right|^2 = |A|^2 \left| \tilde{f}\left(\frac{p \sin{\theta}}{\hbar}\right) \right|^2  \]

Take $x = D \gg d$ then $r = \sqrt{D^2 + y^2}$ and $\sin{\theta} = \frac{y}{r}$ so $\left| \psi(x,y) \right|^2 = |A|^2 \left| \tilde{f}\left(\frac{p_y}{\hbar}\right) \right|^2$ \\  where $p_y = p \sin{\theta} = p \frac{y}{\sqrt{D^2 + y^2}} \approx p \frac{y}{D}$.
\end{enumerate}

\section*{Problem 2.}

\begin{enumerate}
\item Let $\mathcal{H}$ be finite dimensional and $O \! : \mathcal{H} \rightarrow \mathcal{H}$ be linear. \\ \\
Claim: $\ker{O^\dagger} = \left(\Im{O} \right)^{\perp}$: \\ \\
Proof: if $v \in \ker{O^\dagger}$ then $O^\dagger v = 0$ so $\left< u, O^\dagger v \right> = 0$ for any $u \in \mathcal{H}$. Thus, $\left< Ou, v \right> = 0$ so $v \in \left( \Im{O} \right)^{\perp}$. Likewise, if $v \in \left( \Im{O} \right)^{\perp}$ then for any $u \in \mathcal{H}$, $\left<Ou, v \right> = 0$ so $\left< u, O^\dagger v \right> = 0$. Now take $u = O^\dagger v$ then $\left< O^\dagger v, O^\dagger v \right> = 0$ so $O^\dagger v = 0$ thus $v \in \ker{O^\dagger}$. \\ \\
Because $\mathcal{H}$ is finite dimensional, \[\dim{\mathcal{H}} = \dim{\Im{O}} + \dim{\left( \Im{O} \right)^{\perp}} = \dim{\Im{O}} + \dim{\ker{O^\dagger}}\] By rank-nulty, \[\dim{\mathcal{H}} = \dim{\Im{O}} + \dim{\ker{O}}\] thus,\[\dim{\Im{O}} + \dim{\ker{O}} = \dim{\Im{O}} + \dim{\ker{O^\dagger}}\] Therefore, \[\dim{\ker{O}} = \dim{\ker{O^\dagger}}\]

\item Let $H$ be the hilbert subspace of $L^2 \left( \R \right)$ spaned by the solutions to a one-dimensional quantum harmonic oscillator. Then any state is a sum, $\bra{\psi} =  \sum_{n = 0}^{\infty} c_n \bra{n}$. Now $\hat{a} \bra{n} = \sqrt{n} \bra{n-1} \neq 0$ for $n > 0$ and $\hat{a} \bra{0} = 0$ so $\dim{\ker\hat{a} } = 1$. \\ \\ However, $\hat{a}^\dagger \bra{n} = \sqrt{n+1}\bra{n+1}$ which is always non-zero. Thus, $\dim{\ker{\hat{a}^\dagger}} = 0$.

\item Claim: For any operator $\hat{U}$, $\ker{U^\dagger U} = \ker{U}$. \\ \\
Proof: Trivially, $\ker{U} \subset \ker{U^\dagger U}$. Thus, consider $v \in \ker{U^\dagger U}$ the $U^\dagger U v = 0$ so $\left<v, U^\dagger U v\right> = 0$. However, $\left<v, U^\dagger U v\right> = \left<Uv, Uv\right> = 0$. Thus, $Uv = 0$ so $v \in \ker{u}$. Thus, $\ker{U^\dagger U} \subset \ker{U}$. \\ \\
In particular, $\ker{O_\alpha^\dagger O_\alpha} = \ker{O_\alpha}$ and $\ker{O_\alpha O_\alpha^\dagger} = \ker{O^\dagger}$ \\ \\
Claim: the $\lambda$ eigenspaces $V_\lambda$ of $\Oa \Od$ and $\Od \Oa$ are isomorphic for $\lambda \neq 0$.\\ \\
Proof: I claim that $\frac{1}{\sqrt{\lambda}} \Oa \!: V_\lambda^{\Od \Oa} \rightarrow V_\lambda^{\Oa \Od}$ is an isomorphism. First, if $v \in V_\lambda^{\Od \Oa}$ then $\Od \Oa v = \lambda v$ so $(\Oa \Od) \Oa v = \lambda (\Oa v)$ thus $\Oa v \in V_\lambda^{\Oa \Od}$ so the transformation is well defined. Second, if $v \in V_\lambda^{\Od \Oa}$ then $\Od \Oa v = \lambda v$ therefore, $(\frac{1}{\sqrt{\lambda}} \Od) (\frac{1}{\sqrt{\lambda}} \Oa) v = v$. Similarly, $v \in V_\lambda^{\Oa \Od}$ then $\Oa \Od v = \lambda v$ therefore, $(\frac{1}{\sqrt{\lambda}} \Oa) (\frac{1}{\sqrt{\lambda}} \Od) v = v$ so we have constructed an inverse operator of $\frac{1}{\sqrt{\lambda}} \Oa$, namely, $\frac{1}{\sqrt{\lambda}} \Od$ thus the transformation is a bijection. Since $\Oa$ is linear, the transformation is an isomorphism. \\ \\ 
Because the operators are continuous functions of $\alpha$, the eigenvalues $\lambda(\alpha)$ are also continuous functions. Suppose that at two values of $\alpha$ one of the product operators have different kernels. Without loss of generality, take $\ker{\Oa \Od} \neq \ker{ \Oap \Odp}$. Then take some $\lambda (\alpha)$ to have a root at $\alpha '$. For $\lambda(\alpha) \neq 0$ there is a one-to-one correspondance between the $\lambda(\alpha)$ eigenspaces of $\Oa \Od$ and $\Od \Oa$. In particular, $\lambda(\alpha)^{\Oa \Od} = \lambda(\alpha)^{\Od \Oa}$ when the eigenvalues are non-zero. However, these functions are continuous and since they are equal arbitrarily close to $\alpha '$ they must both be zero for identical values of $\alpha$. Therefore, corresponding eigenvalues and eigenspaces are isomorphic for all $\alpha$ even when $\lambda(\alpha) = 0$. Whenever $\lambda (\alpha) = 0$, an eigenvector of $\Oa \Od$ adds one to the dimension of $\ker{\Oa \Od}$ because $\Oa \Od$ is self-adjoint and therefore the distinct eigenvalues correspond to orthogonal and thus linearly independent spans. Orthognality of eigenvectors with distinct eigenvalues is preserved even when both eigenvalues go to zero because the vectors are continuous and orthogonal so they cannot jump from being orthogonal to dependent. $\dim{\ker{\Od \Oa}} - \dim{\ker{\Oa \Od}}$ is constant. Then $\dim{\ker{\Od \Oa}} = \dim{\ker{\Oa}}$ and $\dim{\ker{\Oa \Od}} = \dim{\ker{\Od}}$ therefore $\mathrm{ind}(\Oa) = \dim{\ker{\Oa}} - \dim{\ker{\Od}}$ is contant.  

\end{enumerate}

\end{document}

