\documentclass[12pt]{article}
\usepackage[english]{babel}
\usepackage[utf8]{inputenc}
\usepackage[english]{babel}
\usepackage[a4paper, total={7.25in, 9.5in}]{geometry}
\usepackage{tikz-feynman}
\tikzfeynmanset{compat=1.0.0} 
\usepackage{subcaption}
\usepackage{float}
\floatplacement{figure}{H}
\usepackage{simpler-wick}
\usepackage{mathrsfs}  
\usepackage{dsfont}
\usepackage{relsize}
\usepackage{tikz-cd}
\DeclareMathAlphabet{\mathdutchcal}{U}{dutchcal}{m}{n}

\usepackage{cancel}



\newcommand{\field}{\hat{\Phi}}
\newcommand{\dfield}{\hat{\Phi}^\dagger}
 
\usepackage{amsthm, amssymb, amsmath, centernot}
\usepackage{slashed}
\newcommand{\notimplies}{%
  \mathrel{{\ooalign{\hidewidth$\not\phantom{=}$\hidewidth\cr$\implies$}}}}
 
\renewcommand\qedsymbol{$\square$}
\newcommand{\cont}{$\boxtimes$}
\newcommand{\divides}{\mid}
\newcommand{\ndivides}{\centernot \mid}

\newcommand{\Integers}{\mathbb{Z}}
\newcommand{\Natural}{\mathbb{N}}
\newcommand{\Complex}{\mathbb{C}}
\newcommand{\Zplus}{\mathbb{Z}^{+}}
\newcommand{\Primes}{\mathbb{P}}
\newcommand{\Q}{\mathbb{Q}}
\newcommand{\R}{\mathbb{R}}
\newcommand{\ball}[2]{B_{#1} \! \left(#2 \right)}
\newcommand{\Rplus}{\mathbb{R}^+}
\renewcommand{\Re}[1]{\mathrm{Re}\left[ #1 \right]}
\renewcommand{\Im}[1]{\mathrm{Im}\left[ #1 \right]}
\newcommand{\Op}{\mathcal{O}}

\newcommand{\invI}[2]{#1^{-1} \left( #2 \right)}
\newcommand{\End}[1]{\text{End}\left( A \right)}
\newcommand{\legsym}[2]{\left(\frac{#1}{#2} \right)}
\renewcommand{\mod}[3]{\: #1 \equiv #2 \: \mathrm{mod} \: #3 \:}
\newcommand{\nmod}[3]{\: #1 \centernot \equiv #2 \: mod \: #3 \:}
\newcommand{\ndiv}{\hspace{-4pt}\not \divides \hspace{2pt}}
\newcommand{\finfield}[1]{\mathbb{F}_{#1}}
\newcommand{\finunits}[1]{\mathbb{F}_{#1}^{\times}}
\newcommand{\ord}[1]{\mathrm{ord}\! \left(#1 \right)}
\newcommand{\quadfield}[1]{\Q \small(\sqrt{#1} \small)}
\newcommand{\vspan}[1]{\mathrm{span}\! \left\{#1 \right\}}
\newcommand{\galgroup}[1]{Gal \small(#1 \small)}
\newcommand{\bra}[1]{\left| #1 \right>}
\newcommand{\Oa}{O_\alpha}
\newcommand{\Od}{O_\alpha^{\dagger}}
\newcommand{\Oap}{O_{\alpha '}}
\newcommand{\Odp}{O_{\alpha '}^{\dagger}}
\newcommand{\im}[1]{\mathrm{im} \: #1}
\renewcommand{\ker}[1]{\mathrm{ker} \: #1}
\newcommand{\ket}[1]{\left| #1 \right>}
\renewcommand{\bra}[1]{\left< #1 \right|}
\newcommand{\inner}[2]{\left< #1 | #2 \right>}
\newcommand{\expect}[2]{\left< #1 \right| #2 \left| #1 \right>}
\renewcommand{\d}[1]{ \mathrm{d}#1 \:}
\newcommand{\dn}[2]{ \mathrm{d}^{#1} #2 \:}
\newcommand{\deriv}[2]{\frac{\d{#1}}{\d{#2}}}
\newcommand{\nderiv}[3]{\frac{\dn{#1}{#2}}{\d{#3^{#1}}}}
\newcommand{\pderiv}[2]{\frac{\partial{#1}}{\partial{#2}}}
\newcommand{\fderiv}[2]{\frac{\delta #1}{\delta #2}}
\newcommand{\parsq}[2]{\frac{\partial^2{#1}}{\partial{#2}^2}}
\newcommand{\topo}{\mathcal{T}}
\newcommand{\base}{\mathcal{B}}
\renewcommand{\bf}[1]{\mathbf{#1}}
\renewcommand{\a}{\hat{a}}
\newcommand{\adag}{\hat{a}^\dagger}
\renewcommand{\b}{\hat{b}}
\newcommand{\bdag}{\hat{b}^\dagger}
\renewcommand{\c}{\hat{c}}
\newcommand{\cdag}{\hat{c}^\dagger}
\newcommand{\hamilt}{\hat{H}}
\renewcommand{\L}{\hat{L}}
\newcommand{\Lz}{\hat{L}_z}
\newcommand{\Lsquared}{\hat{L}^2}
\renewcommand{\S}{\hat{S}}
\renewcommand{\empty}{\varnothing}
\newcommand{\J}{\hat{J}}
\newcommand{\lagrange}{\mathcal{L}}
\newcommand{\dfourx}{\mathrm{d}^4x}
\newcommand{\meson}{\phi}
\newcommand{\dpsi}{\psi^\dagger}
\newcommand{\ipic}{\mathrm{int}}
\newcommand{\tr}[1]{\mathrm{tr} \left( #1 \right)}
\newcommand{\C}{\mathbb{C}}
\newcommand{\CP}[1]{\mathbb{CP}^{#1}}
\newcommand{\Vol}[1]{\mathrm{Vol}\left(#1\right)}

\newcommand{\Tr}[1]{\mathrm{Tr}\left( #1 \right)}
\newcommand{\Charge}{\hat{\mathbf{C}}}
\newcommand{\Parity}{\hat{\mathbf{P}}}
\newcommand{\Time}{\hat{\mathbf{T}}}
\newcommand{\Torder}[1]{\mathbf{T}\left[ #1 \right]}
\newcommand{\Norder}[1]{\mathbf{N}\left[ #1 \right]}
\newcommand{\Znorm}{\mathcal{Z}}
\newcommand{\EV}[1]{\left< #1 \right>}
\newcommand{\interact}{\mathrm{int}}
\newcommand{\covD}{\mathcal{D}}
\newcommand{\conj}[1]{\overline{#1}}

\newcommand{\SO}[2]{\mathrm{SO}(#1, #2)}
\newcommand{\SU}[2]{\mathrm{SU}(#1, #2)}

\newcommand{\anticom}[2]{\left\{ #1 , #2 \right\}}


\newcommand{\pathd}[1]{\! \mathdutchcal{D} #1 \:}

\renewcommand{\theenumi}{(\alph{enumi})}


\renewcommand{\theenumi}{(\alph{enumi})}

\newcommand{\atitle}[1]{\title{% 
	\large \textbf{Physics GR8048 Quantum Field Theory II
	\\ Assignment \# #1} \vspace{-2ex}}
\author{Benjamin Church }
\maketitle}

\newcommand{\atitleIII}[1]{\title{% 
	\large \textbf{Physics GR8049 Quantum Field Theory III
	\\ Assignment \# #1} \vspace{-2ex}}
\author{Benjamin Church }
\maketitle}

\theoremstyle{definition}
\newtheorem{theorem}{Theorem}[section]
\newtheorem{definition}{definition}[section]
\newtheorem{lemma}[theorem]{Lemma}
\newtheorem{proposition}[theorem]{Proposition}
\newtheorem{corollary}[theorem]{Corollary}
\newtheorem{example}[theorem]{Example}
\newtheorem{remark}[theorem]{Remark}

\begin{document}

\atitle{6}

\section{Computing the Fermionic Two-Point Function}

Consider the decomposition of the two-point function,
\begin{align*}
\bra{\Omega} \psi_{\alpha}(x) \bar{\psi}_{\beta}(y) \ket{\Omega} & = \sum_{\lambda} \sum_{s} \int \frac{\dn{3}{\vec{p}}}{(2 \pi)^3} \frac{1}{2 p^0} \bra{\Omega} \psi_{\alpha}(x) \ket{\lambda_{s p}} \bra{\lambda_{s p}} \bar{\psi}_{\beta}(y) \ket{\Omega} 
\\
& = \sum_{s} \int \frac{\dn{3}{\vec{p}}}{(2 \pi)^3} \frac{1}{2 p^0} \bra{\Omega} \psi_{\alpha}(x) \ket{\lambda_{s p}}  \bra{\Omega} \psi_{\beta'}(x) \ket{\lambda_{s p}}^* \gamma^0_{\beta' \beta}
\end{align*}
Where I have made the sum over $\lambda$ implicit. For convinience, I will omit this sum throughout the calculation. 
Therefore, we should investigate the amplitude, 
\[  \bra{\Omega} \psi_{\alpha}(x) \ket{\lambda_{s p}} \]
in more detail. 

\subsection{The Component Amplitude}
Returning to the calculation of the amplitude,
\[  \bra{\Omega} \psi(x) \ket{\lambda_{s p}} = \bra{\Omega} e^{i P_\mu x^\mu} \psi(0) e^{- i P_{\mu} x^\mu}  \ket{\lambda_{sp}} =  \bra{\Omega}  \psi(0)  \ket{\lambda_{sp}} e^{- i p \cdot x} \]
where $p \cdot x = p_\mu x^\mu$. This result follows because $\ket{\Omega}$ is shift invariant and $P_\mu \ket{\lambda_{sp}} = p_\mu \ket{\lambda_{sp}}$. 
Now, note that the states $\ket{\lambda_{sp}}$ are Lorentz boosts of zero momentum states,
\[ \ket{\lambda_{sp}} = U(\vec{\chi}_p) \ket{\lambda_s} \]
Therefore, since the vacuum is Lorentz-invariant,
\[  \bra{\Omega} \psi(0) \ket{\lambda_{s p}} = \bra{\Omega} U(\vec{\chi}_p)^{-1}  \psi(0) U(\vec{\chi}_p) \ket{\lambda_{s}} = \bra{\Omega} e^{\vec{\chi_p} \cdot \vec{S}} \psi(0)  \ket{\lambda_{s}} \]
where we have defined the generators of the matrix algebra for the spin-$\tfrac{1}{2}$ representation as,
\[ S^{\mu \nu} = \tfrac{1}{4} [ \gamma^\mu, \gamma^\nu ] \]
with, in particular, the boost part,
\[ (\vec{S})^i = S^{0i} = \tfrac{1}{4} [ \gamma^0, \gamma^i] = \tfrac{1}{2} \gamma^0 \gamma^i = \frac{1}{2}
\begin{pmatrix}
- \sigma^i & 0 \\
0 & \sigma^i
\end{pmatrix}\]
Therefore, 
\[ (e^{\vec{\chi_p} \cdot \vec{S}})^2 = e^{\gamma^0 \vec{\chi_p} \cdot \vec{\gamma}} = \mathbf{1} + \gamma^0 \vec{\chi_p} \cdot \vec{\gamma} + \frac{1}{2} (\gamma^0 \vec{\chi_p} \cdot \vec{\gamma})^2 + \cdots \]
However,
\[ (\gamma^0 \vec{\chi_p} \cdot \vec{\gamma})^2 = \gamma^0 (\chi_p)_i (\chi_p)_j \gamma^i \gamma^0 \gamma^j = - (\chi_p)_i (\chi_p)_j \gamma^i \gamma^j = - \tfrac{1}{2} (\chi_p)_i (\chi_p)_j \{ \gamma^i, \gamma^j \} = - (\chi_p)_i (\chi_p)_j \eta^{i j} \mathbf{1} = (\chi_p)^2 \mathbf{1} \]
Thus, combining terms in the sequence,
\begin{align*}
(e^{\vec{\chi_p} \cdot \vec{S}})^2 = \mathbf{1} ( 1 + \tfrac{1}{2} \chi_p^2 + \tfrac{1}{4!} \chi_p^4 + \cdots) + \gamma^0 \hat{p} \cdot \vec{\gamma} (\chi_p + \tfrac{1}{3!} \chi_p^3 + \cdots) = \mathbf{1} \cosh{\chi_p} + \gamma^0 \hat{p} \cdot \vec{\gamma} \sinh{\chi_p} 
\end{align*}
However, we know that the boost factor is given by $\gamma = \cosh{\chi_p}$ and $\beta \gamma = \sinh{\chi_p}$ so $E = m_\lambda \cosh{\chi_p}$ and $\vec{p} = m_\lambda \hat{p} \sinh{\chi_p}$.
Therefore, since $(\gamma^0)^2 = \mathbf{1}$ we have,
\begin{align*}
(e^{\vec{\chi_p} \cdot \vec{S}})^2 = \frac{1}{m_\lambda} \left( \mathbf{1} E + \gamma^0 \vec{p} \cdot \vec{\gamma} \right) = \frac{\gamma^0 p_\mu \gamma^\mu}{m_\lambda} = \frac{1}{m_\lambda} \gamma^0 \slashed{p} = \frac{1}{m_\lambda}
\begin{pmatrix}
p_\mu \sigma^\mu & 0 \\
0 & p_\mu \bar{\sigma}^\mu 
\end{pmatrix}
\end{align*}
Finally,
\begin{align*}
e^{\vec{\chi_p} \cdot \vec{S}} = \frac{1}{\sqrt{m_\lambda}} \sqrt{\gamma^0 \slashed{p}} = \frac{1}{\sqrt{m_\lambda}} \sqrt{
\begin{pmatrix}
p_\mu \sigma^\mu & 0 \\
0 & p_\mu \bar{\sigma}^\mu 
\end{pmatrix}}
= \frac{1}{\sqrt{m_\lambda}} 
\begin{pmatrix}
\sqrt{p_\mu \sigma^\mu} & 0 \\
0 & \sqrt{p_\mu \bar{\sigma}^\mu}
\end{pmatrix}
\end{align*}
Putting this together, we see that,
\[ \bra{\Omega} \psi(x) \ket{\lambda_{s p}} = \sqrt{\frac{\gamma^0 \slashed{p}}{m_\lambda}} \bra{\Omega} \psi(0) \ket{\lambda_{s}} e^{ -i p \cdot x} \] 

\subsection{Inner Products of Spinors}
Since parity is going to be a symmetry of the theory, we need to have both left and right handed spinors. We will give the chiral components of the spinor field special names,
\[ \psi(x) = 
\begin{pmatrix}
L(x) \\
R(x)
\end{pmatrix} \]
By the spin-$\tfrac{1}{2}$ transformation properties,
\[ \bra{\Omega} L_{s'}(0) \ket{\lambda_{s}} = C_{\lambda, L} \delta_{s' s} \quad \text{and} \quad \bra{\Omega} R_{s'}(0) \ket{\lambda_{s}} = C_{\lambda, R} \delta_{s' s}\]
However, these matrix element can be related via parity transformation. Since parity commutes with spin operators and, by assumption, with the Hamiltonian, we can choose the rest states $\ket{\lambda_s}$ to be parity eigenstates. This would not work for the states $\ket{\lambda_{sp}}$ because $\parity^{-1} P^i \parity = - P^i$ and thus we cannot simultaneously diagonalize both operators. Take the state $\ket{\lambda_s}$ to have intrinsic parity $\eta_\lambda$. That is,
\[ \parity \ket{\lambda_s} = \eta_\lambda \ket{\lambda_s} \]
Furthermore, the right-handed and left-handed spinors are related by,
\[ \parity^{-1} L(x) \parity = R(\bar{x}) \quad \text{and} \quad \parity^{-1} R(x) \parity = L(\bar{x}) \]
Therefore, since $\ket{\Omega}$ is invariant under parity,
\[ \bra{\Omega} L_{s'}(0) \ket{\lambda_{s}} = \bra{\Omega} \parity^{-1} R_{s'}(0) \parity  \ket{\lambda_{s}} = \eta_\lambda \bra{\Omega} R_{s'}(0) \ket{\lambda_{s}} = \eta_\lambda C_{\lambda, R} \delta_{s's} = C_{\lambda, L} \delta_{s's} \]
Likewise,
\[ \bra{\Omega} R_{s'}(0) \ket{\lambda_{s}} = \bra{\Omega} \parity^{-1} L_{s'}(0) \parity  \ket{\lambda_{s}} = \eta_\lambda \bra{\Omega} L_{s'}(0) \ket{\lambda_{s}} = \eta_\lambda C_{\lambda, L} \delta_{s's} = C_{\lambda, R} \delta_{s's} \]
Therefore, $C_{\lambda, L} = \eta_{\lambda} C_{\lambda, R}$ and $C_{\lambda, R} = \eta_{\lambda} C_{\lambda, L}$. Therefore, $\eta_{\lambda} = \pm 1$. Now, define the field strength,
\[ Z_{\lambda}^{1/2} = \frac{1}{\sqrt{m_\lambda}} C_{\lambda, L} = \frac{\eta_\lambda}{\sqrt{m_\lambda}}  C_{\lambda, R} \]
which, because the $\lambda$ states are relativistically normalized, has the correct units.
Using this definition, we see that the entire inner product becomes,
\[ \bra{\Omega} \psi(0) \ket{\lambda_s} = \bra{\Omega}
\begin{pmatrix}
L(0) \\
R(0)
\end{pmatrix} \ket{\lambda_s} = Z_{\lambda}^{1/2} \sqrt{m_\lambda} 
\begin{pmatrix}
\xi^s \\
\eta_\lambda \xi^s
\end{pmatrix} \]
where $\xi^s$ is the two-component spinor defined by,

\[ \xi^{\uparrow} = 
\begin{pmatrix}
1 \\
0
\end{pmatrix}
\quad \text{and} \quad 
\xi^{\downarrow} = 
\begin{pmatrix}
0 \\
1
\end{pmatrix}\]

Putting it all together, 
\[  \bra{\Omega} \psi(x) \ket{\lambda_{s p}} = Z_{\lambda}^{1/2} \sqrt{ \gamma^0 \slashed{p}}  \begin{pmatrix}
\xi^s \\
\eta_\lambda \xi^s
\end{pmatrix} e^{- i p \cdot x} = Z_{\lambda}^{1/2} 
\begin{pmatrix}
\sqrt{p \cdot \sigma} \xi^s \\
\eta_\lambda \sqrt{p \cdot \bar{\sigma}} \xi^s
\end{pmatrix} 
e^{- i p \cdot x} = Z_{\lambda}^{1/2} u^s_{\eta_\lambda} e^{ -i p \cdot x} \]
where we have found the bispinors,
\[ u^s_{\eta_\lambda} = 
\begin{pmatrix}
\sqrt{p \cdot \sigma} \xi^s \\
\eta_\lambda \sqrt{p \cdot \bar{\sigma}} \xi^s
\end{pmatrix}  \]
If we restrict this to the cases $\eta_\lambda = \pm 1$ then we find,
\[ u^s_{+} = 
\begin{pmatrix}
\sqrt{p \cdot \sigma} \xi^s \\
\sqrt{p \cdot \bar{\sigma}} \xi^s
\end{pmatrix} 
= u^s \quad \text{and} \quad
u^s_{-} = 
\begin{pmatrix}
\sqrt{p \cdot \sigma} \xi^s \\
- \sqrt{p \cdot \bar{\sigma}} \xi^s
\end{pmatrix}
= v'^{s} \]
where $v'^{s}$ is the spinor $v^s$ with the reversed spin. Therefore, the two-point function becomes,
\begin{align*}
\bra{\Omega} \psi_{\alpha}(x) \bar{\psi}_{\beta}(y) \ket{\Omega} &= \sum_{s} \int \frac{\dn{3}{\vec{p}}}{(2 \pi)^3} \frac{1}{2 p^0} \bra{\Omega} \psi_{\alpha}(x) \ket{\lambda_{s p}}  \bra{\Omega} \psi_{\beta'}(x) \ket{\lambda_{s p}}^* \gamma^0_{\beta' \beta}
\\
& = \sum_{s} \int \frac{\dn{3}{\vec{p}}}{(2 \pi)^3} \frac{Z_{\lambda}}{2 p^0} (u^s_{\eta_\lambda})_{\alpha} (u^s_{\eta_\lambda})_{\beta'}^* \gamma^0_{\beta' \beta} \: e^{- i p \cdot (x - y) }
\end{align*}
Where $Z_\lambda \ge 0$ because $Z_\lambda = C_{\lambda, L} C_{\lambda, L}^*$.
Now, we need to compute the outer product,
\begin{align*} 
\sum_{s} (u^s_{\eta_\lambda})_{\alpha} (u^s_{\eta_\lambda})_{\beta'}^* \gamma^0_{\beta' \beta} & = 
\sum_{s} \begin{pmatrix}
\sqrt{p \cdot \sigma} \xi^s \\
\eta_\lambda \sqrt{p \cdot \bar{\sigma}} \xi^s
\end{pmatrix}
\begin{pmatrix}
\eta_\lambda  (\xi^s)^\dagger \sqrt{p \cdot \bar{\sigma}}  &  (\xi^s)^\dagger \sqrt{p \cdot \sigma}
\end{pmatrix}
\\
& = \sum_{s}
\begin{pmatrix}
\eta_\lambda \sqrt{p \cdot \sigma} \xi^s (\xi^s)^\dagger \sqrt{p \cdot \bar{\sigma}} & \sqrt{p \cdot \sigma} \xi^s (\xi^s)^\dagger \sqrt{p \cdot \sigma} 
\\
\sqrt{p \cdot \bar{\sigma}} \xi^s (\xi^s)^\dagger \sqrt{p \cdot \bar{\sigma}} & \eta_\lambda \sqrt{p \cdot \bar{\sigma}} \xi^s (\xi^s)^\dagger \sqrt{p \cdot \sigma} 
\end{pmatrix}
\end{align*}
However,
\[ \sum_{s} \xi^s (\xi^s)^\dagger = \mathbf{1} \]
Therefore,
\begin{align*} 
\sum_{s} (u^s_{\eta_\lambda})_{\alpha} (u^s_{\eta_\lambda})_{\beta'}^* \gamma^0_{\beta' \beta} & =
\begin{pmatrix}
\eta_\lambda \sqrt{p \cdot \sigma} \sqrt{p \cdot \bar{\sigma}} & p \cdot \sigma 
\\
p \cdot \bar{\sigma} & \eta_\lambda \sqrt{p \cdot \bar{\sigma}} \sqrt{p \cdot \sigma} 
\end{pmatrix}
\\
& =
\begin{pmatrix}
\eta_\lambda m_\lambda  & p \cdot \sigma 
\\
p \cdot \bar{\sigma} & \eta_\lambda m_\lambda
\end{pmatrix}
= (\slashed{p} + \eta_\lambda m_\lambda)_{\alpha \beta}
\end{align*}
where I have used the fact that,
\[ \sqrt{p \cdot \sigma} \sqrt{p \cdot \bar{\sigma}} = \sqrt{(p \cdot \sigma)(p \cdot \bar{\sigma})} = \sqrt{p_\mu p_\nu \sigma^\mu \bar{\sigma}^\nu} = \sqrt{p_\mu p_\nu \eta^{\mu \nu}} = m_\lambda \]
Finally, the two-point function becomes,
\begin{align*}
\bra{\Omega} \psi_{\alpha}(x) \bar{\psi}_{\beta}(y) \ket{\Omega} & = \sum_{\lambda} \int \frac{\dn{3}{\vec{p}}}{(2 \pi)^3} \frac{Z_{\lambda}}{2 p^0} (\slashed{p} + \eta_\lambda m_\lambda)_{\alpha \beta} \: e^{- i p \cdot (x - y) }
\end{align*}

\section{Computing the Reversed Two-Point Function}

Consider the decomposition of the two-point function,
\begin{align*}
\bra{\Omega} \bar{\psi}_{\beta}(y) \psi_{\alpha}(x) \ket{\Omega} & = \sum_{\bar{\lambda}} \sum_{s} \int \frac{\dn{3}{\vec{p}}}{(2 \pi)^3} \frac{1}{2 p^0} \bra{\Omega} \bar{\psi}_{\beta}(y) \ket{{\bar{\lambda}}_{s p}} \bra{{\bar{\lambda}}_{s p}} \psi_{\alpha}(x) \ket{\Omega} 
\\
& = \sum_{s} \int \frac{\dn{3}{\vec{p}}}{(2 \pi)^3} \frac{1}{2 p^0} \bra{\Omega} \psi^*_{\beta'}(y) \ket{{\bar{\lambda}}_{s p}} \bra{\Omega} \psi^*_{\alpha}(x) \ket{{\bar{\lambda}}_{s p}}^* \gamma^0_{\beta' \beta}
\end{align*}
As before, I will make the sum over $\bar{\lambda}$ implicit. Therefore, we should investigate the amplitude, 
\[  \bra{\Omega} \psi^*_{\alpha}(x) \ket{{\bar{\lambda}}_{s p}} \]
in more detail. 

\subsection{The Component Amplitude}
Returning to the calculation of the amplitude,
\[  \bra{\Omega} \psi^*(x) \ket{{\bar{\lambda}}_{s p}} = \bra{\Omega} e^{i P_\mu x^\mu} \psi^*(0) e^{- i P_{\mu} x^\mu}  \ket{{\bar{\lambda}}_{ps}} =  \bra{\Omega}  \psi^*(0)  \ket{{\bar{\lambda}}_{sp}} e^{- i p \cdot x} \]
where $p \cdot x = p_\mu x^\mu$. This result follows because $\ket{\Omega}$ is shift invariant and $P_\mu \ket{{\bar{\lambda}}_{ps}} = p_\mu \ket{{\bar{\lambda}}_{ps}}$. 
Now, note that the states $\ket{{\bar{\lambda}}_{sp}}$ are Lorentz boosts of zero momentum states,
\[ \ket{{\bar{\lambda}}_{sp}} = U(\vec{\chi}_p) \ket{{\bar{\lambda}}_s} \]
Therefore, since the vacuum is Lorentz-invariant,
\[  \bra{\Omega} \psi^*(0) \ket{{\bar{\lambda}}_{s p}} = \bra{\Omega} U(\vec{\chi}_p)^{-1}  \psi^*(0) U(\vec{\chi}_p) \ket{{\bar{\lambda}}_{sp}} = \bra{\Omega} e^{\vec{\chi_p} \cdot \vec{S}^*} \psi^*(0)  \ket{{\bar{\lambda}}_{s}} \]
As before, we have that,
\begin{align*}
e^{\vec{\chi_p} \cdot \vec{S}^*} = \frac{1}{\sqrt{m_{\bar{\lambda}}}} \sqrt{\gamma^0 \slashed{p}^*} = \frac{1}{\sqrt{m_{\bar{\lambda}}}} \sqrt{
\begin{pmatrix}
p_\mu \sigma^{*\mu} & 0 \\
0 & p_\mu \bar{\sigma}^{*\mu} 
\end{pmatrix}}
= \frac{1}{\sqrt{m_{\bar{\lambda}}}} 
\begin{pmatrix}
\sqrt{p_\mu \sigma^{*\mu}} & 0 \\
0 & \sqrt{p_\mu \bar{\sigma}^{*\mu}}
\end{pmatrix}
\end{align*}
Putting this together, we see that,
\[ \bra{\Omega} \psi^*(x) \ket{{\bar{\lambda}}_{s p}} = \sqrt{\frac{\gamma^0 \slashed{p}^*}{m_{\bar{\lambda}}}} \bra{\Omega} \psi^*(0) \ket{{\bar{\lambda}}_{s}} e^{ - i p \cdot x} \] 

\subsection{Inner Products of Conjugate Spinors}
Consider the chiral components of the conjugated spinor,
\[ \psi^*(x) = 
\begin{pmatrix}
L^*(x) \\
R^*(x)
\end{pmatrix} \]
We cannot directly use the spin-$\tfrac{1}{2}$ transformation properties because $L^*$ is not a spinor. However, $\epsilon L^*$ is a right handed spinor. This can be checked by considering its transformation properties under rotations,
\[ \epsilon L^* \mapsto \epsilon \left( e^{-\frac{i}{2} \vec{\phi} \cdot \vec{\sigma}} L \right)^* = \epsilon  e^{\frac{i}{2} \vec{\phi} \cdot \vec{\sigma}^*} L^* =  e^{- \frac{i}{2} \vec{\phi} \cdot \vec{\sigma}} \epsilon L^*\]
and under boosts,
\[ \epsilon L^* \mapsto \epsilon \left( e^{-\frac{1}{2} \vec{\chi} \cdot \vec{\sigma}} L \right)^* = \epsilon  e^{-\frac{1}{2} \vec{\chi} \cdot \vec{\sigma}^*} L^* =  e^{\frac{1}{2} \vec{\chi} \cdot \vec{\sigma}} \epsilon L^*\]
Therefore, we apply the spin-$\tfrac{1}{2}$ selection rules to the spinors $\epsilon L^*$ and $\epsilon R^*$,
\[ \bra{\Omega} \epsilon_{s' s''} L^*_{s''}(0) \ket{{\bar{\lambda}}_{s}} = C_{{\bar{\lambda}}, L} \delta_{s' s} \quad \text{and} \quad \bra{\Omega} \epsilon_{s' s''} R^*_{s''}(0) \ket{{\bar{\lambda}}_{s}} = C_{{\bar{\lambda}}, R} \delta_{s' s}\]
Therefore,
\[ \bra{\Omega} L_{s'}^*(0) \ket{{\bar{\lambda}}_{s}} = C_{{\bar{\lambda}}, L} \epsilon_{s s'} \quad \text{and} \quad \bra{\Omega} R_{s'}^*(0) \ket{{\bar{\lambda}}_{s}} = C_{{\bar{\lambda}}, R} \epsilon_{s s'}\]
These matrix element can be related via parity transformation. Since parity commutes with spin operators and, by assumption, with the Hamiltonian, we can choose the rest states $\ket{{\bar{\lambda}}_s}$ to be parity eigenstates. Take the state $\ket{{\bar{\lambda}}_s}$ to have intrinsic parity $\eta_{\bar{\lambda}}$. That is,
\[ \parity \ket{{\bar{\lambda}}_s} = \eta_{\bar{\lambda}} \ket{{\bar{\lambda}}_s} \]
Furthermore, the right-handed and left-handed conjugate spinors are related by,
\[ \parity^{-1} \epsilon L^*(x) \parity = \epsilon R^*(\bar{x}) \quad \text{and} \quad \parity^{-1} \epsilon R^*(x) \parity = \epsilon L^*(\bar{x}) \]
Therefore, since $\ket{\Omega}$ is invariant under parity,
\[ \bra{\Omega} (\epsilon L^*)_{s'} \ket{{\bar{\lambda}}_{s}} = \bra{\Omega} \parity^{-1} (\epsilon R^*)_{s'} \parity  \ket{{\bar{\lambda}}_{s}} = \eta_{\bar{\lambda}} \bra{\Omega} (\epsilon R^*)_{s'} \ket{{\bar{\lambda}}_{s}} = \eta_{\bar{\lambda}} C_{{\bar{\lambda}}, R} \delta_{s's} = C_{{\bar{\lambda}}, L} \delta_{s's} \]
Likewise,
\[ \bra{\Omega} (\epsilon R^*)_{s'} \ket{{\bar{\lambda}}_{s}} = \bra{\Omega} \parity^{-1} (\epsilon L^*)_{s'} \parity  \ket{{\bar{\lambda}}_{s}} = \eta_{\bar{\lambda}} \bra{\Omega} (\epsilon L^*)_{s'}(0) \ket{{\bar{\lambda}}_{s}} = \eta_{\bar{\lambda}} C_{{\bar{\lambda}}, L} \delta_{s's} = C_{{\bar{\lambda}}, R} \delta_{s's} \]
Therefore, $C_{{\bar{\lambda}}, L} = \eta_{{\bar{\lambda}}} C_{{\bar{\lambda}}, R}$ and $C_{{\bar{\lambda}}, R} = \eta_{{\bar{\lambda}}} C_{{\bar{\lambda}}, L}$. As before, our choices require, $\eta_{{\bar{\lambda}}} = \pm 1$. Now, define the field strength,
\[ Z_{{\bar{\lambda}}}^{1/2} = \frac{1}{\sqrt{m_{\bar{\lambda}}}} C_{{\bar{\lambda}}, L} = \frac{\eta_{\bar{\lambda}}}{\sqrt{m_{\bar{\lambda}}}}  C_{{\bar{\lambda}}, R} \]
which, because the ${\bar{\lambda}}$ states are relativistically normalized, has the correct units.
Using this definition, we see that the entire inner product becomes,
\[ \bra{\Omega} \psi^*(0) \ket{{\bar{\lambda}}_s} = \bra{\Omega}
\begin{pmatrix}
L^*(0) \\
R^*(0)
\end{pmatrix} \ket{{\bar{\lambda}}_s} = Z_{{\bar{\lambda}}}^{1/2} \sqrt{m_{\bar{\lambda}}} 
\begin{pmatrix}
\epsilon \xi^s \\
\eta_{\bar{\lambda}} \epsilon \xi^s
\end{pmatrix} \]
Putting it all together, 
\begin{align*}  
\bra{\Omega} \psi^*(x) \ket{{\bar{\lambda}}_{s p}}
& = Z_{{\bar{\lambda}}}^{1/2} \sqrt{ \gamma^0 \slashed{p}^*}  \begin{pmatrix}
\epsilon \xi^s \\
\eta_{\bar{\lambda}} \epsilon \xi^s
\end{pmatrix} e^{- i p \cdot x} 
= Z_{{\bar{\lambda}}}^{1/2} 
\begin{pmatrix}
\sqrt{p \cdot \sigma^*} \: \epsilon \xi^s \\
\eta_{\bar{\lambda}} \sqrt{p \cdot \bar{\sigma}^*} \: \epsilon \xi^s
\end{pmatrix} 
e^{- i p \cdot x} 
\\
&= Z_{{\bar{\lambda}}}^{1/2} 
\begin{pmatrix}
\epsilon \sqrt{p \cdot \bar{\sigma}} \: \xi^s \\
\eta_{\bar{\lambda}} \epsilon \sqrt{p \cdot \sigma} \: \xi^s
\end{pmatrix} 
e^{- i p \cdot x} 
= Z_{{\bar{\lambda}}}^{1/2} (v^s_{\eta_{\bar{\lambda}}})^* e^{ -i p \cdot x} 
\end{align*}
where we have found bispinors,
\[ v^s_{\eta_{\bar{\lambda}}} = 
\begin{pmatrix}
\sqrt{p \cdot \sigma} \: \epsilon \xi^s \\
\eta_{\bar{\lambda}} \sqrt{p \cdot \bar{\sigma}} \: \epsilon \xi^s
\end{pmatrix} = (u^s_{-\eta_{\bar{\lambda}}})^{(c)} = \mathcal{C} \cdot (u^s_{-\eta_{\bar{\lambda}}})^*  \]
If we restrict this to the cases $\eta_{\bar{\lambda}} = \pm 1$ then we find,
\[ v^s_{+} = 
\begin{pmatrix}
\sqrt{p \cdot \sigma} \: \epsilon \xi^s \\
\sqrt{p \cdot \bar{\sigma}} \: \epsilon \xi^s
\end{pmatrix} 
= u'^s \quad \text{and} \quad
v^s_{-} = 
\begin{pmatrix}
\sqrt{p \cdot \sigma} \: \epsilon \xi^s \\
- \sqrt{p \cdot \bar{\sigma}} \: \epsilon \xi^s
\end{pmatrix}
= v^{s} \]
where $u'^{s}$ is the spinor $u^s$ with the reversed spin. Therefore, the two-point function becomes,
\begin{align*}
\bra{\Omega} \bar{\psi}_{\beta}(y) \psi_{\alpha}(x) \ket{\Omega}
& = \sum_{s} \int \frac{\dn{3}{\vec{p}}}{(2 \pi)^3} \frac{1}{2 p^0} \bra{\Omega} \psi^*_{\beta'}(y) \ket{{\bar{\lambda}}_{s p}} \bra{\Omega} \psi^*_{\alpha}(x) \ket{{\bar{\lambda}}_{s p}}^* \gamma^0_{\beta' \beta}
\\
& = \sum_{s} \int \frac{\dn{3}{\vec{p}}}{(2 \pi)^3} \frac{Z_{\bar{\lambda}}}{2 p^0} (v^s_{\eta_{\bar{\lambda}}})_\alpha (v^s_{\eta_{\bar{\lambda}}})^*_\beta \gamma^0_{\beta' \beta} \: e^{i p \cdot (x - y)}
\end{align*}
Where $Z_{\bar{\lambda}} \ge 0$ because $Z_{\bar{\lambda}} = C_{\bar{\lambda}, L} C_{\bar{\lambda}, L}^*$.
Now, we need to compute the outer product,
\begin{align*} 
\sum_{s} (v^s_{\eta_{\bar{\lambda}}})_{\alpha} (v^s_{\eta_{\bar{\lambda}}})_{\beta'}^* \gamma^0_{\beta' \beta} & = 
\sum_{s} \begin{pmatrix}
\sqrt{p \cdot \sigma} \: \epsilon \xi^s \\
\eta_{\bar{\lambda}} \sqrt{p \cdot \bar{\sigma}} \: \epsilon \xi^s
\end{pmatrix}
\begin{pmatrix}
\eta_{\bar{\lambda}}  (\epsilon \xi^s)^\dagger \sqrt{p \cdot \bar{\sigma}}  &  (\epsilon \xi^s)^\dagger \sqrt{p \cdot \sigma}
\end{pmatrix}
\\
& = \sum_{s}
\begin{pmatrix}
\eta_{\bar{\lambda}} \sqrt{p \cdot \sigma} \: \epsilon \xi^s (\epsilon \xi^s)^\dagger \sqrt{p \cdot \bar{\sigma}} & \sqrt{p \cdot \sigma} \: \epsilon \xi^s (\epsilon \xi^s)^\dagger \sqrt{p \cdot \sigma} 
\\
\sqrt{p \cdot \bar{\sigma}} \: \epsilon \xi^s (\epsilon \xi^s)^\dagger \sqrt{p \cdot \bar{\sigma}} & \eta_{\bar{\lambda}} \sqrt{p \cdot \bar{\sigma}} \: \epsilon \xi^s (\epsilon \xi^s)^\dagger \sqrt{p \cdot \sigma} 
\end{pmatrix}
\end{align*}
However,
\[ \sum_{s} \epsilon \xi^s (\epsilon \xi^s)^\dagger = \epsilon \left( \sum_{s} \xi^s (\xi^s)^\dagger \right) \epsilon^{-1} = \mathbf{1} \]
Therefore,
\begin{align*} 
\sum_{s} (v^s_{\eta_{\bar{\lambda}}})_{\alpha} (v^s_{\eta_{\bar{\lambda}}})_{\beta'}^* \gamma^0_{\beta' \beta} & =
\begin{pmatrix}
\eta_{\bar{\lambda}} \sqrt{p \cdot \sigma} \sqrt{p \cdot \bar{\sigma}} & p \cdot \sigma 
\\
p \cdot \bar{\sigma} & \eta_{\bar{\lambda}} \sqrt{p \cdot \bar{\sigma}} \sqrt{p \cdot \sigma} 
\end{pmatrix}
\\
& =
\begin{pmatrix}
\eta_{\bar{\lambda}} m_{\bar{\lambda}}  & p \cdot \sigma 
\\
p \cdot \bar{\sigma} & \eta_{\bar{\lambda}} m_{\bar{\lambda}}
\end{pmatrix}
= (\slashed{p} + \eta_{\bar{\lambda}} m_{\bar{\lambda}})_{\alpha \beta}
\end{align*}
Finally, the reversed two-point function becomes,
\begin{align*}
\bra{\Omega} \bar{\psi}_{\beta}(y) \psi_{\alpha}(x) \ket{\Omega} & = \sum_{\bar{\lambda}} \int \frac{\dn{3}{\vec{p}}}{(2 \pi)^3} \frac{Z_{{\bar{\lambda}}}}{2 p^0} (\slashed{p} + \eta_{\bar{\lambda}} m_{\bar{\lambda}})_{\alpha \beta} \: e^{i p \cdot (x - y) }
\end{align*}

\section{The Anti-Commutation Relations}
Knowing that observable fields must either commute or anti-commute at space-like speration, we want to find out whether we are forced to impose commutation or anti-commutation relations on our spinor fields. Suppose $x$ and $y$ are space-like separated. First, assume that $x$ and $y$ are at equal times. Consider the the reversed two-point function,
\begin{align*}
\bra{\Omega} \bar{\psi}_{\beta}(y) \psi_{\alpha}(x) \ket{\Omega} & = \sum_{\bar{\lambda}} \int \frac{\dn{3}{\vec{p}}}{(2 \pi)^3} \frac{Z_{{\bar{\lambda}}}}{2 p^0} (\slashed{p} + \eta_{\bar{\lambda}} m_{\bar{\lambda}})_{\alpha \beta} \: e^{i \vec{p} \cdot (\vec{x} - \vec{y}) }
\\
& = \sum_{\bar{\lambda}} \int \frac{\dn{3}{\vec{p}}}{(2 \pi)^3} \frac{Z_{{\bar{\lambda}}}}{2 p^0} (\gamma^0 p^0 - \vec{p} \cdot \vec{\gamma} + \eta_{\bar{\lambda}} m_{\bar{\lambda}})_{\alpha \beta} \: e^{i \vec{p} \cdot (\vec{x} - \vec{y})}
\end{align*}
Flipping the direction of the integrated out variable $\vec{p}$ we get,
\begin{align*}
\bra{\Omega} \bar{\psi}_{\beta}(y) \psi_{\alpha}(x) \ket{\Omega} 
& = \sum_{\bar{\lambda}} \int \frac{\dn{3}{\vec{p}}}{(2 \pi)^3} \frac{Z_{{\bar{\lambda}}}}{2 p^0} (2 \gamma^0 p^0 - \gamma^0 p^0 + \vec{p} \cdot \vec{\gamma} + \eta_{\bar{\lambda}} m_{\bar{\lambda}})_{\alpha \beta} \: e^{- i \vec{p} \cdot (\vec{x} - \vec{y})}
\\
& = \sum_{\bar{\lambda}} \left[ \int \frac{\dn{3}{\vec{p}}}{(2 \pi)^3} Z_{\bar{\lambda}} \gamma^0_{\alpha \beta} \: e^{- i \vec{p} \cdot (\vec{x} - \vec{y})} - \int \frac{\dn{3}{\vec{p}}}{(2 \pi)^3} \frac{Z_{{\bar{\lambda}}}}{2 p^0} (\slashed{p} - \eta_{\bar{\lambda}} m_{\bar{\lambda}})_{\alpha \beta} \: e^{- i \vec{p} \cdot (\vec{x} - \vec{y})} \right]
\\
& = \sum_{\bar{\lambda}} Z_{\bar{\lambda}} \gamma^0_{\alpha \beta} \delta^3(\vec{x} - \vec{y}) - \sum_{\bar{\lambda}} \int \frac{\dn{3}{\vec{p}}}{(2 \pi)^3} \frac{Z_{{\bar{\lambda}}}}{2 p^0} (\slashed{p} - \eta_{\bar{\lambda}} m_{\bar{\lambda}})_{\alpha \beta} \: e^{- i \vec{p} \cdot (\vec{x} - \vec{y})}
\end{align*}
Therefore, since $Z_\lambda, Z_{\bar{\lambda}} \ge 0$ then we must have equal time anti-commutation relations. 
Furthermore, compare the above to, 
\begin{align*}
\bra{\Omega} \psi_{\alpha}(x) \bar{\psi}_{\beta}(y) \ket{\Omega} & = \sum_{\bar{\lambda}} \int \frac{\dn{3}{\vec{p}}}{(2 \pi)^3} \frac{Z_{\lambda}}{2 p^0} (\slashed{p} + \eta_\lambda m_\lambda)_{\alpha \beta} \: e^{- i p \cdot (x - y) }
\end{align*}
If these terms are going to actually be minus eachother, there must be a spectral symmetry, $\lambda \leftrightarrow \bar{\lambda}$  which makes $Z_{\lambda} = Z_{\bar{\lambda}}$ and $m_{\lambda} = m_{\bar{\lambda}}$ and finally $\eta_{\bar{\lambda}} = - \eta_{\lambda}$. In that case,
\[ \bra{\Omega} \psi_{\alpha}(x) \bar{\psi}_{\beta}(y) \ket{\Omega} + \bra{\Omega} \bar{\psi}_{\beta}(y) \psi_{\alpha}(x) \ket{\Omega} = \sum_{\lambda} Z_{\lambda} \gamma^0_{\alpha \beta} \delta^3(\vec{x} - \vec{y})\]
which fixes the vacuum expectation of equal-time anti-commutation relations,
\[ \bra{\Omega} \{\psi_{\alpha}(x), \bar{\psi}_{\beta}(y) \} \ket{\Omega} = \sum_{\lambda} Z_{\lambda} \gamma^0_{\alpha \beta} \delta^3(\vec{x} - \vec{y})\]
By swapping which we call the particle and which the anti-particle, we are free to choose $\eta_{\lambda} = 1$ and therefore $\eta_{\bar{\lambda}} = -1$.

\section{The Feynman Propagator}

I claim that the Feynman propagator can be written in the form,
\[ \Delta_F(x - y)_{\alpha \beta} = \sum_{\lambda} \int \frac{\dn{4}{p}}{(2 \pi)^4} \frac{ i (\slashed{p} + m)_{\alpha \beta}}{p^2 - m^2 + i \epsilon} Z_{\lambda} e^{-i p \cdot (x - y)} \]
Throughout the derivation, I will drop the sum over $\lambda$ for notational convenience.
I will prove this formula by integrating over $p^0$,
\begin{align*}
\int \frac{\dn{4}{p}}{(2 \pi)^4} \frac{ i (\slashed{p} + m)_{\alpha \beta}}{p^2 - m^2 + i \epsilon} Z_{\lambda} e^{-i p \cdot (x - y)} & =  \int \frac{i \: \d{p^0}}{2 \pi} e^{-i p^0 (x^0 - y^0)} \int \frac{\dn{3}{\vec{p}}}{(2\pi)^3} Z_{\lambda} \frac{(\slashed{p} + m)_{\alpha \beta}}{(p^0)^2 - E_p^2 + i \epsilon} e^{\vec{p} \cdot (\vec{x} - \vec{y})}
\end{align*}
When $x^0 > y^0$ then we can close the contour below such that $-ip^0 < 0$ and thus $e^{- p^0 (x^0 - y^0)}$ is exponentially small. However, we can write the term,
\[ \frac{\slashed{p} + m}{(p^0)^2 - E_p^2 + i \epsilon} = \frac{\slashed{p} + m}{(p^0 - E_p + i \epsilon)(p^0 + E_p - i \epsilon)}  \]
Therefore, the residue at $p^0 = E_p - i \epsilon$ is,
\[ \frac{\slashed{p} + m}{2 E_p} e^{-i E_p (x^0 - y^0)} \]
where $\slashed{p}$ is on shell. Therefore, by the residue theorem (remembering to use the factor $- 2 \pi i$ for a clockwise contour), 
\begin{align*}
\int \frac{\dn{4}{p}}{(2 \pi)^4} \frac{ i (\slashed{p} + m)_{\alpha \beta}}{p^2 - m^2 + i \epsilon} Z_{\lambda} e^{-i p \cdot (x - y)}  \xrightarrow{x^0 > y^0} & \int \frac{\dn{3}{\vec{p}}}{(2\pi)^3} Z_{\lambda} \frac{(\slashed{p} + m)_{\alpha \beta}}{2 E_p} e^{-i E_p (x^0 - y^0)}  e^{\vec{p} \cdot (\vec{x} - \vec{y})}
\\
& =
\int \frac{\dn{3}{\vec{p}}}{(2\pi)^3} \frac{Z_{\lambda}}{2 E_p} (\slashed{p} + m)_{\alpha \beta} \: e^{- i p \cdot (x - y)} = \bra{\Omega} \psi_{\alpha}(x) \bar{\psi}_{\beta}(y) \ket{\Omega}
\end{align*}
Likewise, for $x^0 < y^0$ we close the contour above such that $-ip^0 > 0$ and thus $e^{-i p^0 (x^0 - y^0)}$ is exponentially small. The residue at $p^0 = - E_p + i \epsilon$ is given by,
\[ \frac{- E_p \gamma^0 - \vec{p} \cdot \vec{\gamma} + m}{-2 E_p} e^{+i E_p (x^0 - y^0)} \]
Therefore, by the residue theorem,
\begin{align*}
\int \frac{\dn{4}{p}}{(2 \pi)^4} \frac{ i (\slashed{p} + m)_{\alpha \beta}}{p^2 - m^2 + i \epsilon} Z_{\lambda} e^{-i p \cdot (x - y)}  \xrightarrow{x^0 < y^0} & \int \frac{\dn{3}{\vec{p}}}{(2\pi)^3} Z_{\lambda} \frac{(- E_p \gamma^0 - \vec{p} \cdot \vec{\gamma} + m)_{\alpha \beta}}{2 E_p} e^{+i E_p (x^0 - y^0)}  e^{\vec{p} \cdot (\vec{x} - \vec{y})}
\\
& = - \int \frac{\dn{3}{\vec{p}}}{(2\pi)^3} Z_{\lambda} \frac{(E_p \gamma^0 - \vec{p} \cdot \vec{\gamma} - m)_{\alpha \beta}}{2 E_p} e^{i E_p (x^0 - y^0)}  e^{- \vec{p} \cdot (\vec{x} - \vec{y})} 
\\
& =
- \int \frac{\dn{3}{\vec{p}}}{(2\pi)^3} \frac{Z_{\lambda}}{2 E_p} (\slashed{p} - m)_{\alpha \beta} \: e^{i p \cdot (x - y)} = - \bra{\Omega} \bar{\psi}_{\beta}(y) \psi_{\alpha}(x) \ket{\Omega}
\end{align*}
where I have changed variables $\vec{p} \to -\vec{p}$. Putting these results together and reintroducing the sum over states $\lambda$ we find,
\[ \sum_{\lambda} \int \frac{\dn{4}{p}}{(2 \pi)^4} \frac{ i (\slashed{p} + m)_{\alpha \beta}}{p^2 - m^2 + i \epsilon} Z_{\lambda} e^{-i p \cdot (x - y)} = \bra{\Omega} \mathbf{T} [\psi_{\alpha}(x) \bar{\psi}_{\beta}(y)] \ket{\Omega} = 
\begin{cases}
\bra{\Omega} \psi_{\alpha}(x) \bar{\psi}_{\beta}(y) \ket{\Omega} & x^0 > y^0 \\
- \bra{\Omega} \bar{\psi}_{\beta}(y) \psi_{\alpha}(x) \ket{\Omega} & x^0 < y^0
\end{cases}\]
which is the Feynman propagator.
\end{document}