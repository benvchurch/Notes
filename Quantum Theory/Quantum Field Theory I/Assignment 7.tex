\documentclass[12pt]{article}
\usepackage[english]{babel}
\usepackage[utf8]{inputenc}
\usepackage[english]{babel}
\usepackage[a4paper, total={7.25in, 9.5in}]{geometry}
\usepackage{tikz-feynman}
\tikzfeynmanset{compat=1.0.0} 
\usepackage{subcaption}
\usepackage{float}
\floatplacement{figure}{H}
\usepackage{simpler-wick}
\usepackage{mathrsfs}  
\usepackage{dsfont}
\usepackage{relsize}
\usepackage{tikz-cd}
\DeclareMathAlphabet{\mathdutchcal}{U}{dutchcal}{m}{n}

\usepackage{cancel}



\newcommand{\field}{\hat{\Phi}}
\newcommand{\dfield}{\hat{\Phi}^\dagger}
 
\usepackage{amsthm, amssymb, amsmath, centernot}
\usepackage{slashed}
\newcommand{\notimplies}{%
  \mathrel{{\ooalign{\hidewidth$\not\phantom{=}$\hidewidth\cr$\implies$}}}}
 
\renewcommand\qedsymbol{$\square$}
\newcommand{\cont}{$\boxtimes$}
\newcommand{\divides}{\mid}
\newcommand{\ndivides}{\centernot \mid}

\newcommand{\Integers}{\mathbb{Z}}
\newcommand{\Natural}{\mathbb{N}}
\newcommand{\Complex}{\mathbb{C}}
\newcommand{\Zplus}{\mathbb{Z}^{+}}
\newcommand{\Primes}{\mathbb{P}}
\newcommand{\Q}{\mathbb{Q}}
\newcommand{\R}{\mathbb{R}}
\newcommand{\ball}[2]{B_{#1} \! \left(#2 \right)}
\newcommand{\Rplus}{\mathbb{R}^+}
\renewcommand{\Re}[1]{\mathrm{Re}\left[ #1 \right]}
\renewcommand{\Im}[1]{\mathrm{Im}\left[ #1 \right]}
\newcommand{\Op}{\mathcal{O}}

\newcommand{\invI}[2]{#1^{-1} \left( #2 \right)}
\newcommand{\End}[1]{\text{End}\left( A \right)}
\newcommand{\legsym}[2]{\left(\frac{#1}{#2} \right)}
\renewcommand{\mod}[3]{\: #1 \equiv #2 \: \mathrm{mod} \: #3 \:}
\newcommand{\nmod}[3]{\: #1 \centernot \equiv #2 \: mod \: #3 \:}
\newcommand{\ndiv}{\hspace{-4pt}\not \divides \hspace{2pt}}
\newcommand{\finfield}[1]{\mathbb{F}_{#1}}
\newcommand{\finunits}[1]{\mathbb{F}_{#1}^{\times}}
\newcommand{\ord}[1]{\mathrm{ord}\! \left(#1 \right)}
\newcommand{\quadfield}[1]{\Q \small(\sqrt{#1} \small)}
\newcommand{\vspan}[1]{\mathrm{span}\! \left\{#1 \right\}}
\newcommand{\galgroup}[1]{Gal \small(#1 \small)}
\newcommand{\bra}[1]{\left| #1 \right>}
\newcommand{\Oa}{O_\alpha}
\newcommand{\Od}{O_\alpha^{\dagger}}
\newcommand{\Oap}{O_{\alpha '}}
\newcommand{\Odp}{O_{\alpha '}^{\dagger}}
\newcommand{\im}[1]{\mathrm{im} \: #1}
\renewcommand{\ker}[1]{\mathrm{ker} \: #1}
\newcommand{\ket}[1]{\left| #1 \right>}
\renewcommand{\bra}[1]{\left< #1 \right|}
\newcommand{\inner}[2]{\left< #1 | #2 \right>}
\newcommand{\expect}[2]{\left< #1 \right| #2 \left| #1 \right>}
\renewcommand{\d}[1]{ \mathrm{d}#1 \:}
\newcommand{\dn}[2]{ \mathrm{d}^{#1} #2 \:}
\newcommand{\deriv}[2]{\frac{\d{#1}}{\d{#2}}}
\newcommand{\nderiv}[3]{\frac{\dn{#1}{#2}}{\d{#3^{#1}}}}
\newcommand{\pderiv}[2]{\frac{\partial{#1}}{\partial{#2}}}
\newcommand{\fderiv}[2]{\frac{\delta #1}{\delta #2}}
\newcommand{\parsq}[2]{\frac{\partial^2{#1}}{\partial{#2}^2}}
\newcommand{\topo}{\mathcal{T}}
\newcommand{\base}{\mathcal{B}}
\renewcommand{\bf}[1]{\mathbf{#1}}
\renewcommand{\a}{\hat{a}}
\newcommand{\adag}{\hat{a}^\dagger}
\renewcommand{\b}{\hat{b}}
\newcommand{\bdag}{\hat{b}^\dagger}
\renewcommand{\c}{\hat{c}}
\newcommand{\cdag}{\hat{c}^\dagger}
\newcommand{\hamilt}{\hat{H}}
\renewcommand{\L}{\hat{L}}
\newcommand{\Lz}{\hat{L}_z}
\newcommand{\Lsquared}{\hat{L}^2}
\renewcommand{\S}{\hat{S}}
\renewcommand{\empty}{\varnothing}
\newcommand{\J}{\hat{J}}
\newcommand{\lagrange}{\mathcal{L}}
\newcommand{\dfourx}{\mathrm{d}^4x}
\newcommand{\meson}{\phi}
\newcommand{\dpsi}{\psi^\dagger}
\newcommand{\ipic}{\mathrm{int}}
\newcommand{\tr}[1]{\mathrm{tr} \left( #1 \right)}
\newcommand{\C}{\mathbb{C}}
\newcommand{\CP}[1]{\mathbb{CP}^{#1}}
\newcommand{\Vol}[1]{\mathrm{Vol}\left(#1\right)}

\newcommand{\Tr}[1]{\mathrm{Tr}\left( #1 \right)}
\newcommand{\Charge}{\hat{\mathbf{C}}}
\newcommand{\Parity}{\hat{\mathbf{P}}}
\newcommand{\Time}{\hat{\mathbf{T}}}
\newcommand{\Torder}[1]{\mathbf{T}\left[ #1 \right]}
\newcommand{\Norder}[1]{\mathbf{N}\left[ #1 \right]}
\newcommand{\Znorm}{\mathcal{Z}}
\newcommand{\EV}[1]{\left< #1 \right>}
\newcommand{\interact}{\mathrm{int}}
\newcommand{\covD}{\mathcal{D}}
\newcommand{\conj}[1]{\overline{#1}}

\newcommand{\SO}[2]{\mathrm{SO}(#1, #2)}
\newcommand{\SU}[2]{\mathrm{SU}(#1, #2)}

\newcommand{\anticom}[2]{\left\{ #1 , #2 \right\}}


\newcommand{\pathd}[1]{\! \mathdutchcal{D} #1 \:}

\renewcommand{\theenumi}{(\alph{enumi})}


\renewcommand{\theenumi}{(\alph{enumi})}

\newcommand{\atitle}[1]{\title{% 
	\large \textbf{Physics GR8048 Quantum Field Theory II
	\\ Assignment \# #1} \vspace{-2ex}}
\author{Benjamin Church }
\maketitle}

\newcommand{\atitleIII}[1]{\title{% 
	\large \textbf{Physics GR8049 Quantum Field Theory III
	\\ Assignment \# #1} \vspace{-2ex}}
\author{Benjamin Church }
\maketitle}

\theoremstyle{definition}
\newtheorem{theorem}{Theorem}[section]
\newtheorem{definition}{definition}[section]
\newtheorem{lemma}[theorem]{Lemma}
\newtheorem{proposition}[theorem]{Proposition}
\newtheorem{corollary}[theorem]{Corollary}
\newtheorem{example}[theorem]{Example}
\newtheorem{remark}[theorem]{Remark}

\begin{document}

\newcommand{\gs}{g_{+}}

\newcommand{\gp}{g_{-}}

\atitle{7}

\section{Background Theory}

We are going to consider a class of parity violating theories of Dirac fermions with a scalar or pseudo-scalar interactions with a Lagrangian,
\[ \lagrange = \bar{\psi} \left( i \slashed{\partial} - m  \right) \psi  + \tfrac{1}{2} \partial_{\mu} \phi \partial^\mu \phi - \tfrac{1}{2} \mu^2 \phi^2 - \gs \phi \bar{\psi} \psi - \gp \phi \bar{\psi} i \gamma^5 \psi \]

The Feynman rules for this theory are,
	\begin{equation*}
	\feynmandiagram[horizontal = b to a, small, baseline = (a.base)] {
	i1 -- [fermion] a -- [fermion] f1,
	a -- [scalar] b
	};
	= - i \Gamma = - i ( \gs + i \gp \gamma^5 ) 
	\hspace{2cm}
	\feynmandiagram[horizontal = i1 to o1, large, inline = (i1),  tree layout] {
	i1 -- [fermion] o1
	}; = \frac{i \left( \slashed{p} + m \right)}{p^2 - m^2 + i \epsilon} 
	\hspace{2cm}
	\feynmandiagram[horizontal = i1 to o1, large, inline = (i1),  tree layout] {
	i1 -- [scalar]  o1
	}; = \frac{i}{p^2 - \mu^2 + i \epsilon} 
	\end{equation*}

I will define some combined coupling constants $g^2 = \gs^2 + \gp^2$ and $\delta = \gs^2 - \gp^2$ to simplify notation.

\subsection{Trace Tricks}

We're going to need some serious trace machinery to simplify the scattering cross sections. 

\begin{remark}
I use the notation $\gamma^5 = i \gamma^0 \gamma^1 \gamma^2 \gamma^3$ which is the negative of Tong's convention. This is equivalent to reversing the chirality of the $L$ and $R$ spinors. In my convention $R$ has chirality $+1$ and $L$ has chirality $-1$ which is the opposite of Tong. This choice manifests as a change of sign in the parity violating coupling constant $\gp$.
\end{remark}
\begin{proposition}
I will make use of these trace tricks,
\begin{enumerate}
\item[1.] The trace of any odd number of gamma matrices is zero.

\item[2.] The trace of $\gamma^5$ times an odd number of gamma matrices is zero.

\item[3.] $ \tr{ \gamma^\mu \gamma^\nu} = 4 \eta^{\mu \nu} $

\item[4.] $ \tr{ \gamma^\mu \gamma^\nu \gamma^5 } = \tr{ \gamma^5 } = 0 $

\item[5.] $ \tr{ \gamma^\mu \gamma^\nu \gamma^\rho \gamma^\sigma } = 4 \left( \eta^{\mu \nu} \eta^{\rho \sigma} - \eta^{\mu \rho} \eta^{\nu \sigma} + \eta^{\mu \sigma} \eta^{\nu \rho} \right) $  

\item[6.] $ \tr{ \gamma^\mu \gamma^\nu \gamma^\rho \gamma^\sigma \gamma^5 } = 4 i \epsilon^{\mu \nu \rho \sigma} $
\end{enumerate}
\end{proposition}

I will uses these properties to derive some important properties of traces of combintations of $\slashed{p}$ and $\Gamma$. 

\begin{proposition}

\item[1] $ \tr{ \Gamma } = 4 \gs $ and $\tr{ \Gamma^2 } = 4 \left( \gs^2 - \gp^2 \right)$ 

\item[2] $ \tr{ \slashed{p} \Gamma } = \tr{ \slashed{p} } = \tr{ \Gamma \slashed{p} \Gamma } = 0 $ 

\item[3] $  \tr{ \slashed{p} \Gamma \slashed{q} \Gamma } = (\gs^2 + \gp^2) \tr{ \slashed{p} \slashed{q} } = 4 (\gs^2 + \gp^2) p \cdot q $

\item[4] $\tr{ (\slashed{p} + \eta_p m) \Gamma (\slashed{q} + \eta_q m) \Gamma } = 4 ( \gs^2 + \gp^2) p \cdot q + 4 \eta_p \eta_q m^2 (\gs^2 - \gp^2) $

\item[5] $\tr{ (\slashed{p} \pm m) \Gamma (\slashed{q} \pm m) \Gamma } = 4 ( \gs^2 + \gp^2) p \cdot q + 4 m^2 (\gs^2 - \gp^2)$ 
\end{proposition}

\begin{proof}

First, $\tr{ \Gamma } = \tr{ \gs + i \gp \gamma^5 } = \tr { \gs } = 4 \gs$ and
 \[\tr{ \Gamma^2 } = \tr{ \gs^2 - \gp^2 + 2 i \gs \gp \gamma^5 } =  4 \left( \gs^2 - \gp^2 \right)\] 

$ \tr{ \slashed{p} } = \tr{ \gamma^\mu p_\mu } = p_\mu \tr{ \gamma^\mu} = 0 $ and $ \tr{ \slashed{p} \Gamma } = p_\mu \tr{ \gamma^\mu (\gs + i \gp \gamma^5 ) } = 0$ and 
\[ \tr{ \Gamma \slashed{p} \Gamma } = p_\mu \tr{ \gamma^\mu \Gamma^2} = p_\mu \tr{ \gamma^\mu (\gs^2 - \gp^2 + 2 i \gs \gp \gamma^5) } = 0 \]
\bigskip\\
\begin{align*} 
\tr{ \slashed{p} \Gamma \slashed{q} \Gamma } & = p_\mu q_\nu \tr{ \gamma^\mu (\gs + i \gp \gamma^5 ) \gamma^\nu ( \gs + i \gp \gamma^5 ) }  = \tr{ \gamma^\mu \gamma^\nu ( \gs - i \gp \gamma^5) (\gs + i \gp \gamma^5) } 
\\
& = p_{\mu} q_{\nu} \tr{ \gamma^\mu \gamma^\nu (\gs^2 + \gp^2) } = p_\mu q_\nu 4 (\gs^2 + \gp^2)\eta^{\mu \nu} = 4 (\gs^2 + \gp^2)  p \cdot q 
\end{align*}
\bigskip\\
\begin{align*}
\tr{ (\slashed{p} + \eta_p m) \Gamma (\slashed{q} + \eta_q m) \Gamma } & = \tr{ \slashed{p} \Gamma \slashed{q} \Gamma } + \tr{ \eta_p m \Gamma \slashed{q} \Gamma } + \tr{ \slashed{p} \Gamma \eta_q m \Gamma } + \tr{ \eta_p \eta_q m^2 \Gamma^2 } 
\\
& = 4 ( \gs^2 + \gp^2) p \cdot q + 4 \eta_p \eta_q m^2 (\gs^2 - \gp^2) 
\end{align*}
since the middle terms are zero. 
\end{proof}

\subsection{Computing the Scattering Amplitudes}

The scattering process $\psi \bar{\psi} \to \psi \bar{\psi}$ is given by two tree-level diagrams,

\begin{figure}
\centering
\begin{minipage}{.5\textwidth}
  \centering
  
\feynmandiagram [vertical'=a to b] {
i1 -- [fermion, momentum' = \( u^{s_1}(p_1) \)] a -- [fermion, momentum' = \( \bar{u}^{s_1'}(p_1') \)] f1,
a -- [scalar] b,
f2 -- [anti fermion, momentum = \( \bar{v}^{s_2}(p_2) \)] b -- [anti fermion, momentum = \( v^{s_2'}(p_2') \)] i2,
};

\end{minipage}%
\begin{minipage}{.5\textwidth}
  \centering
  
\feynmandiagram [horizontal=a to b] {
i1 -- [fermion, momentum' = \( u^{s_1}(p_1) \)] a -- [fermion, rmomentum = \( \bar{v}^{s_2}(p_2) \)] i2,
a -- [scalar] b,
f1 -- [fermion, rmomentum' = \( \bar{u}^{s_1'}(p_1') \)] b -- [fermion, momentum = \( v^{s_2'}(p_2') \)] f2,
};

\end{minipage}
\end{figure}
Which gives an amplitude,

\[ i \mathcal{M} = \left[ - \frac{1}{t - \mu^2} \bigg( \bar{u}^{s_1'}(p_1') \Gamma u^{s_1}(p_1) \bigg) 
\bigg( \bar{v}^{s_2}(p_2) \Gamma v^{s_2'}(p_2') \bigg) 
+  \frac{1}{s - \mu^2} \bigg( \bar{v}^{s_2}(p_2) \Gamma u^{s_1}(p_1) \bigg) 
\bigg( \bar{u}^{s_1'}(p_1') \Gamma v^{s_2'}(p_2') \bigg) \right]
\]
Call the two terms $A$ and $B$. We need to consider the complex conjugate of these terms. First, because $\gamma^5$ is Hermitian,
\[ \Gamma^\dagger = \gs - i \gp \gamma^5 = \gs - i \gp (- \gamma^0 \gamma^5 \gamma^0) = \gamma^0 \left(\gs + i \gp \gamma^5 \right) \gamma^0 = \gamma^0 \Gamma \gamma^0 \]
Therefore, since the term is just a complex number,

\[ \overline{\bar{u}^{s_1'}(p_1') \Gamma u^{s_1}(p_1) } = \bigg( \bar{u}^{s_1'}(p_1') \Gamma u^{s_1}(p_1) \bigg)^\dagger = u^{s_1}(p_1)^\dagger \Gamma^\dagger \gamma^0 u^{s_1'}(p_1') = u^{s_1}(p_1)^\dagger \gamma^0 \Gamma u^{s_1'}(p_1') = \bar{u}^{s_1}(p_1) \Gamma u^{s_1'}(p_1')  \]
Furthermore, each term in large parentheses is a complex number so we can reorder them and to complex conjugate all we need to do is ``flip'' the order (remembering to also flip which one is conjugated). 

\subsection{Computing the Spin-Averaged Cross Section}

For scattering in the center of mass frame, the spin averaged cross section is given by,
\[ \deriv{\sigma}{\Omega} = \frac{ \left< | \mathcal{M} |^2 \right>_{\text{spin}} }{ 64 \pi^2 E_{CM}^2} \]
where the spin averaged probability is give by,
\begin{align*}
 \left< | \mathcal{M} |^2 \right>_{\text{spin}} & = \frac{1}{4} \sum_{s_1, s_2, s_1', s_2'} |\mathcal{M}(s_1, s_2, s_1', s_2') |^2 
\\
& = \frac{1}{(t - \mu^2)^2} \: \frac{1}{4} \sum_{\text{spin}} A \overline{A} - \frac{1}{(t - \mu^2) (s - \mu^2)} \: 2 \mathfrak{Re} \left[ \frac{1}{4} \sum_{\text{spin}}  A \overline{B} \; \right] + \frac{1}{(s - \mu^2)^2} \: \frac{1}{4} \sum_{\text{spin}} B \overline{B}  
\\
 & = \frac{1}{(t - \mu^2)^2} \: P_1 - \frac{1}{(t - \mu^2) (s - \mu^2)} \: 2 \mathfrak{Re} \left[ P_2 \right] + \frac{1}{(s - \mu^2)^2} \: P_3
\end{align*}
\subsubsection{Center of Mass Quantities}

In the center of mass frame, the momenta can all be written in terms of the energy of one particle $E$ with $E_{CM} = 2 E$ and two vectors $\vec{p}$ an $\vec{p}'$ with equal magnitudes. We write,
\begin{align*}
p_1 & = (E, \vec{p}) \\
p_2 & = (E, - \vec{p}) \\
p_1' & = (E, \vec{p'}) \\
p_2' & = (E, - \vec{p'})
\end{align*}
Therefore, the Mandelstam parameters become,
\begin{align*}
s & = (p_1 + p_2)^2 = 4 E^2 = E_{CM}^2 \\
t & = (p_1 - p_1')^2 = - (\vec{p} - \vec{p'}) = - \vec{p}^{\, 2} - \vec{p'}^{\, 2} + 2 \vec{p} \cdot \vec{p'} = - 2 \vec{p}^{\, 2} (1 - \cos{\Theta}) \\
u & = (p_1 - p_2')^2 = - (\vec{p} + \vec{p'}) = - \vec{p}^{\, 2} - \vec{p'}^{\, 2} - 2 \vec{p} \cdot \vec{p'} = - 2 \vec{p}^{\, 2} (1 + \cos{\Theta})
\end{align*}
Furthermore, we will need the inner products of these momenta,
\begin{align*}
p_1^2 & = p_2^2 = p_1'^2 = p_2'^2 = m^2 \\
p_1 \cdot p_2 & = p_1' \cdot p_2' = E^2 + \vec{p}^{\, 2} = 2 E^2 - m^2 \\
p_1 \cdot p_1'& = p_2 \cdot p_2' = E^2 - \vec{p} \cdot \vec{p'} = E^2 - \vec{p}^{\, 2} \cos{\Theta} = E^2 (1 - \cos{\Theta}) + m^2 \cos{\Theta}  \\
p_1 \cdot p_2'& = p_2 \cdot p_1' = E^2 + \vec{p} \cdot \vec{p'} = E^2 + \vec{p}^{\, 2} \cos{\Theta} = E^2 (1 + \cos{\Theta}) - m^2 \cos{\Theta} 
\end{align*}

\subsubsection{Calculating $P_1$}

\begin{align*}
\frac{1}{4} \sum_{\text{spin}} A \overline{A} & = \frac{1}{4} \sum_{\text{spin}} \bigg( \bar{u}^{s_1'}(p_1') \Gamma u^{s_1}(p_1) \bar{u}^{s_1}(p_1) \Gamma u^{s_1'}(p_1')  \bigg) \bigg( \bar{v}^{s_2}(p_2) \Gamma v^{s_2'}(p_2') \bar{v}^{s_2'}(p_2') \Gamma v^{s_2}(p_2) \bigg)
\\
& = \frac{1}{4} \sum_{\text{spin}} \tr{ u^{s_1'}(p_1') \bar{u}^{s_1'}(p_1') \Gamma u^{s_1}(p_1) \bar{u}^{s_1}(p_1) \Gamma } \cdot \tr{ v^{s_2}(p_2) \bar{v}^{s_2}(p_2) \Gamma v^{s_2'}(p_2') \bar{v}^{s_2'}(p_2') \Gamma }
\\
& = \frac{1}{4} \tr{ \left( \slashed{p}_1' + m \right) \Gamma \left( \slashed{p}_1 + m \right) \Gamma } \cdot \tr{ \left( \slashed{p}_2 - m \right) \Gamma \left( \slashed{p}'_2 - m \right) \Gamma }
\\
& = 4 \bigg( ( \gs^2 + \gp^2) p_1 \cdot p_1' + m^2 (\gs^2 - \gp^2) \bigg) \cdot \bigg( ( \gs^2 + \gp^2) p_2 \cdot p_2' +  m^2 (\gs^2 - \gp^2) \bigg)
\\
& = 4 \bigg( g^2 E^2 (1 - \cos{\Theta}) + g^2 m^2 \cos{\Theta} + m^2 \delta \bigg)^2
\end{align*}

\subsubsection{Calculating $P_2$}

\begin{align*}
\frac{1}{4} \sum_{\text{spin}}  A \overline{B} & = 
\bigg( \bar{u}^{s_1'}(p_1') \Gamma u^{s_1}(p_1) \bar{u}^{s_1}(p_1) \Gamma v^{s_2}(p_2) \bar{v}^{s_2}(p_2) \Gamma v^{s_2'}(p_2') \bar{v}^{s_2'}(p_2') \Gamma  u^{s_1'}(p_1')  \bigg)
\\
& = \frac{1}{4} \sum_{\text{spin}} 
\tr{ u^{s_1'}(p_1') \bar{u}^{s_1'}(p_1') \Gamma u^{s_1}(p_1) \bar{u}^{s_1}(p_1) \Gamma v^{s_2}(p_2) \bar{v}^{s_2}(p_2) \Gamma v^{s_2'}(p_2') \bar{v}^{s_2'}(p_2') \Gamma  }
\\
& = \frac{1}{4}
\tr{ \left( \slashed{p}'_1 + m \right) \Gamma \left( \slashed{p}_1 + m \right) \Gamma \left( \slashed{p}_2 - m \right) \Gamma \left( \slashed{p}'_2 - m \right) \Gamma  }
\\
& = 4 \Big( E^2 g^2 - \gp^2 m^2 \Big) \Big( E^2 g^2 + m^2 \delta - g^2 (E^2 - m^2) \cos{\Theta} \Big)
\end{align*}
The last line follows from Mathematica working explicitly with gamma matrices in the chiral basis.

\subsubsection{Calculating $P_3$}

\begin{align*}
\frac{1}{4} \sum_{\text{spin}} B \overline{B} & = \frac{1}{4} \sum_{\text{spin}} \bigg( \bar{v}^{s_2}(p_2) \Gamma u^{s_1}(p_1) \bar{u}^{s_1}(p_1) \Gamma v^{s_2}(p_2) \bigg) 
\bigg( \bar{u}^{s_1'}(p_1') \Gamma v^{s_2'}(p_2') \bar{v}^{s_2'}(p_2') \Gamma  u^{s_1'}(p_1')  \bigg)
\\
& = \frac{1}{4} \sum_{\text{spin}} \tr{ v^{s_2}(p_2) \bar{v}^{s_2}(p_2) \Gamma u^{s_1}(p_1) \bar{u}^{s_1}(p_1) \Gamma } \cdot \tr{ u^{s_1'}(p_1') \bar{u}^{s_1'}(p_1') \Gamma v^{s_2'}(p_2') \bar{v}^{s_2'}(p_2') \Gamma   }
\\
& = \frac{1}{4} \tr{ \left( \slashed{p}_2 - m \right) \Gamma \left( \slashed{p}_1 + m \right) \Gamma } \cdot \tr{ \left( \slashed{p}_1' + m \right) \Gamma \left( \slashed{p}'_2 - m \right) \Gamma }
\\
& = 4 \bigg( ( \gs^2 + \gp^2) p_1 \cdot p_2 - m^2 (\gs^2 - \gp^2) \bigg) \cdot \bigg( ( \gs^2 + \gp^2) p_1' \cdot p_2' -  m^2 (\gs^2 - \gp^2) \bigg)
\\
& = 4 \bigg( 2 E^2 g^2  -2 m^2  \gp^2  \bigg)^2
\end{align*}

\subsubsection{The Full Cross Section}

Putting everything together,

\begin{align*}
\deriv{\sigma}{\Omega} & = \frac{ 1 }{ 16 \pi^2 E_{CM}^2} \left[ \frac{\bigg( g^2 E^2 (1 - \cos{\Theta}) + g^2 m^2 \cos{\Theta} + m^2 \delta \bigg)^2}{\Big( 2 (E^2 - m^2) (1 - \cos{\Theta}) + \mu^2 \Big)^2} 
\right. 
\\ & \left. 
\quad \quad \quad + 2 \frac{\Big( E^2 g^2 - \gp^2 m^2 \Big) \Big( E^2 g^2 + m^2 \delta - g^2 (E^2 - m^2) \cos{\Theta} \Big)}{\Big( 2 (E^2 - m^2) (1 - \cos{\Theta}) + \mu^2 \Big) \Big( 4 E^2 - \mu^2 \Big) } + \frac{\bigg( 2 E^2 g^2  -2 m^2  \gp^2  \bigg)^2}{ \Big( 4 E^2 - \mu^2 \Big)^2} \right]
\end{align*}

\section{Fitting the Data}

Given our expression for the differential scattering cross section, we need to compute the number of expected events scattering into a specified solid angle. Since our detector only measures the cosine of the scattering angle, the angle between the scattered momentum and the beam axis. The solid angle associated to an angular interval $\d{\Theta}$ is $\d{\Omega} = 2 \pi \sin{\Theta} \: \d{\Theta} = 2 \pi \: \d{ (\cos{\Theta}) }$. 
Furthermore, we know the integrated lumenosity of the particle beam,
\[ L = \frac{1}{\sigma} \deriv{N}{t} \implies L_{\text{int}} = \int L \d{t} = \frac{N}{\sigma} \]
is set at $L_{\text{int}} = 1 \: \mathrm{fb}^{-1}$. Therefore, considering the fraction of events which scatter into a solid angle as the ratio between the differential cross section and the total integrated cross section, we arrive at,
\[ \deriv{\sigma}{\Omega} = \frac{1}{L_{\text{int}}} \deriv{N}{\Omega} \]
Therefore, the formula we will used to compare with experimental data becomes,
\begin{align*}
\frac{1}{L_{\text{int}}} \frac{\d{N( \cos{\Theta} )}}{\d{(\cos{\Theta})}} & = \frac{1}{8 \pi E_{CM}^2}  \left[ \frac{\bigg( g^2 E^2 (1 - \cos{\Theta}) + g^2 m^2 \cos{\Theta} + m^2 \delta \bigg)^2}{\Big( 2 (E^2 - m^2) (1 - \cos{\Theta}) + \mu^2 \Big)^2} 
\right. 
\\ & \left. 
\quad \quad \quad + 2 \frac{\Big( E^2 g^2 - \gp^2 m^2 \Big) \Big( E^2 g^2 + m^2 \delta - g^2 (E^2 - m^2) \cos{\Theta} \Big)}{\Big( 2 (E^2 - m^2) (1 - \cos{\Theta}) + \mu^2 \Big) \Big( 4 E^2 - \mu^2 \Big) } + \frac{\bigg( 2 E^2 g^2  -2 m^2  \gp^2  \bigg)^2}{ \Big( 4 E^2 - \mu^2 \Big)^2} \right]
\end{align*}
This formula is written in natural units. However,
\[ \frac{\hbar^2 c^2}{\mathrm{Gev}^2} = 3.894 \cdot 10^{11} \, \mathrm{fb} \]
Which implies that the formula should have a factor of $\hbar^2 c^2$ in the numerator. Explicitly, in physical units,
\begin{align*}
\frac{\d{N( \cos{\Theta} )}}{\d{(\cos{\Theta})}} & = 3.8734 \cdot 10^{9} \left( \frac{L_{\text{int}}}{1 \, \text{fb}^{-1}} \right) \left( \frac{1 \, \mathrm{GeV}}{E} \right)^2 \left[ \frac{\bigg( g^2 E^2 (1 - \cos{\Theta}) + g^2 m^2 \cos{\Theta} + m^2 \delta \bigg)^2}{\Big( 2 (E^2 - m^2) (1 - \cos{\Theta}) + \mu^2 \Big)^2} 
\right. 
\\ & \left. 
\quad \quad \quad + 2 \frac{\Big( E^2 g^2 - \gp^2 m^2 \Big) \Big( E^2 g^2 + m^2 \delta - g^2 (E^2 - m^2) \cos{\Theta} \Big)}{\Big( 2 (E^2 - m^2) (1 - \cos{\Theta}) + \mu^2 \Big) \Big( 4 E^2 - \mu^2 \Big) } + \frac{\bigg( 2 E^2 g^2  -2 m^2  \gp^2  \bigg)^2}{ \Big( 4 E^2 - \mu^2 \Big)^2} \right]
\end{align*}
The integral of this expression over angles gives the theoretical prediction for the total number of events observed,
\[ N(E) = \int_{-1}^{1}  \frac{\d{N( \cos{\Theta} )}}{\d{(\cos{\Theta})}} \d{(\cos{\Theta})} \]
In the given experimental data, the mass of the fnucleon is $m = 0.94 \mathrm{GeV}$ and the total beam luminosity for each run is $L_{\text{int}} = 1 \, \text{fb}^{-1}$.  
By comparing this theoretical result to the acutal number of events recorded at each particular energy gives a fit for the parameters, 
\begin{align*}
\mu & = 11.586 \pm 0.019 \: \mathrm{GeV}
\\
\gs & = 0.35501 \pm 5.5 \cdot 10^{-4} 
\\
\gp & = 0.11758 \pm 2.3 \cdot 10^{-3}
\end{align*}
where the statistical error in total bin counts is taken to be $\sqrt{N}$ if $N$ is the observed number of events in a bin. A more percise fit comes from comparing the number of events in angular bins to the differential cross section integrated over a small solid angle. I choose to take $20$ angular bins each with an width in $w = \cos{\Theta}$ space of $\d{w} = \d{(\cos{\Theta})} = 0.1$. Comparing these bin counts to the theoretical differential cross section integrated across the bin width gives a best fit for the interaction parameters,
\begin{align*}
\mu & = 11.378 \pm 0.018 \: \mathrm{GeV}
\\
\gs & = 0.34587 \pm 5.3 \cdot 10^{-4} 
\\
\gp & = 0.11953 \pm 2.2 \cdot 10^{-3}
\end{align*}
These parameters were initialized in the fitting software at the best fit values calculated from the total cross section as listed above.
These fits were performed using the Mathematica script ``NonlinearModelFit'' using the optional method ``NMinimize''. Weights were calculated as the reciprocal of the experimental variance $1/\Delta y^2$ where the standard deviation of a bin count measurement $N$ is assumed to be the standard derivation of the mean over individual events with independent random noise and thus proportional to $\sigma \propto \sqrt{N}$.       
\end{document}
