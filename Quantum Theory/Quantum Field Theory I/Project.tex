\documentclass[12pt]{article}
\usepackage[utf8]{inputenc}
\usepackage[english]{babel}
\usepackage[a4paper, total={7.25in, 9.5in}]{geometry}
\usepackage{tikz-feynman}
\tikzfeynmanset{compat=1.0.0} 
\usepackage{subcaption}
\usepackage{float}
\floatplacement{figure}{H}
\usepackage{simplewick}
 
\newcommand{\field}{\hat{\Phi}}
\newcommand{\dfield}{\hat{\Phi}^\dagger}
 
\usepackage{amsthm, amssymb, amsmath, centernot}
\usepackage{slashed}
\newcommand{\notimplies}{%
  \mathrel{{\ooalign{\hidewidth$\not\phantom{=}$\hidewidth\cr$\implies$}}}}
 
\renewcommand\qedsymbol{$\square$}
\newcommand{\cont}{$\boxtimes$}
\newcommand{\divides}{\mid}
\newcommand{\ndivides}{\centernot \mid}
\newcommand{\Z}{\mathbb{Z}}
\newcommand{\N}{\mathbb{N}}
\newcommand{\C}{\mathbb{C}}
\newcommand{\Zplus}{\mathbb{Z}^{+}}
\newcommand{\Primes}{\mathbb{P}}
\newcommand{\ball}[2]{B_{#1} \! \left(#2 \right)}
\newcommand{\Q}{\mathbb{Q}}
\newcommand{\R}{\mathbb{R}}
\newcommand{\Rplus}{\mathbb{R}^+}
\newcommand{\invI}[2]{#1^{-1} \left( #2 \right)}
\newcommand{\End}[1]{\text{End}\left( A \right)}
\newcommand{\legsym}[2]{\left(\frac{#1}{#2} \right)}
\renewcommand{\mod}[3]{\: #1 \equiv #2 \: \mathrm{mod} \: #3 \:}
\newcommand{\nmod}[3]{\: #1 \centernot \equiv #2 \: mod \: #3 \:}
\newcommand{\ndiv}{\hspace{-4pt}\not \divides \hspace{2pt}}
\newcommand{\finfield}[1]{\mathbb{F}_{#1}}
\newcommand{\finunits}[1]{\mathbb{F}_{#1}^{\times}}
\newcommand{\ord}[1]{\mathrm{ord}\! \left(#1 \right)}
\newcommand{\quadfield}[1]{\Q \small(\sqrt{#1} \small)}
\newcommand{\vspan}[1]{\mathrm{span}\! \left\{#1 \right\}}
\newcommand{\galgroup}[1]{Gal \small(#1 \small)}
\newcommand{\bra}[1]{\left| #1 \right>}
\newcommand{\Oa}{O_\alpha}
\newcommand{\Od}{O_\alpha^{\dagger}}
\newcommand{\Oap}{O_{\alpha '}}
\newcommand{\Odp}{O_{\alpha '}^{\dagger}}
\renewcommand{\Im}[1]{\mathrm{Im} \: #1}
\newcommand{\ket}[1]{\left| #1 \right>}
\renewcommand{\bra}[1]{\left< #1 \right|}
\newcommand{\inner}[2]{\left< #1 | #2 \right>}
\newcommand{\expect}[2]{\left< #1 \right| #2 \left| #1 \right>}
\renewcommand{\d}[1]{ \mathrm{d}#1 \:}
\newcommand{\dn}[2]{ \mathrm{d}^{#1} #2 \:}
\newcommand{\deriv}[2]{\frac{\d{#1}}{\d{#2}}}
\newcommand{\pderiv}[2]{\frac{\partial{#1}}{\partial{#2}}}
\newcommand{\parsq}[2]{\frac{\partial^2{#1}}{\partial{#2}^2}}
\newcommand{\topo}{\mathcal{T}}
\newcommand{\base}{\mathcal{B}}
\renewcommand{\bf}[1]{\mathbf{#1}}
\renewcommand{\a}{\hat{a}}
\newcommand{\adag}{\hat{a}^\dagger}
\renewcommand{\b}{\hat{b}}
\newcommand{\bdag}{\hat{b}^\dagger}
\newcommand{\cdag}{\hat{c}^\dagger}
\newcommand{\hamilt}{\hat{H}}
\renewcommand{\L}{\hat{L}}
\newcommand{\Lz}{\hat{L}_z}
\newcommand{\Lsquared}{\hat{L}^2}
\renewcommand{\S}{\hat{S}}
\renewcommand{\empty}{\varnothing}
\newcommand{\J}{\hat{J}}
\newcommand{\lagrange}{\mathcal{L}}
\newcommand{\dfourx}{\mathrm{d}^4x}
\newcommand{\meson}{\phi}
\newcommand{\dpsi}{\psi^\dagger}
\newcommand{\ipic}{\mathrm{int}}
\newcommand{\parity}{\mathbf{P}}
\newcommand{\conj}{\mathbf{C}}
\newcommand{\tr}[1]{\mathrm{Tr} \left( #1 \right)}

\renewcommand{\theenumi}{(\alph{enumi})}

\newcommand{\atitle}[1]{\title{% 
	\large \textbf{Physics GR6047 Quantum Field Theory I
	\\ Assignment \# #1} \vspace{-2ex}}
\author{Benjamin Church }
\maketitle}

 
\newtheorem{theorem}{Theorem}[section]
\newtheorem{lemma}[theorem]{Lemma}
\newtheorem{proposition}[theorem]{Proposition}
\newtheorem{corollary}[theorem]{Corollary}
\newtheorem{remark}[theorem]{Remark}
 
\usepackage{natbib}
\bibliographystyle{apa}
 

\begin{document}
\newcommand{\MeV}{\: \mathrm{MeV}}
\newcommand{\GeV}{\: \mathrm{GeV}}

\title{% 
	\large \textbf{Physics GR6047 Quantum Field Theory I
	\\
	Project:
	\\ On The Existence of Bound States of Massless Fields \vspace{-2ex}}
\author{Benjamin Church }
}
\maketitle

\section{Introduction}
The study of bound states at all in quantum field theory is a notoriously difficult problem. There are no know general methods for determining the bound state spectra of a QFT. Furthermore, the vast majority of what is know about bound states in quantum field theories considers only the nonrelativistic or semi-classical limits in which the conditions for a bound state to appear reduce to solving the Schr\"{o}dinger equation or a first-order relativistic correction to it\footnote{Peskin Chapter 5, page 148.} \footnote{ Landau, Lifshitz - Quantum Electrodynamics Chapter XII, The Relativistic Equation For Bound States, page 552}. However, such results are inapplicable to hypothetical bound states of massless or nearly massless fields. Such configurations are necessarily highly relativistic systems and therefore standard nonrelativistic approximations are of little use. Furthermore, such a system would necessarily have a binding energy comparable or much greater than the masses of all, or at least one, of the fields involved. Therefore, we should expect significant mixing between two-particle and multi-particle bound states which introduces further complications for the standard methods. In this report, we will provide approximate criteria for the existence of bound states consisting entirely of massless fields and bound states between a massless particle and a highly massive partner. We will use a mixture of arguments based on the known spectral resolution of the exact propagator and a generalized wavefunction condition derived from the Bethe-Saltpeter equations for the existence of bound states. Specifically, I will consider the derivation given in Landau, Lifshitz Quantum Electrodynamics\footnote{ Landau, Lifshitz - Quantum Electrodynamics Chapter XII, page 557} of the time-independent Schr\"{o}diner equation from the Bethe-Saltpeter condition for scattering resonances.      

\subsection{The Case of Exclusively Massless Fields}

The existence of bound states consisting exclusively of massless fields is a somewhat open question. As of yet, there are no experimental detection of any bound states of massless particles and most such predicted bound states are hypothetical. A prominent example of theoretical bound states of massless particles are glueballs, bound states of massless gluons. Glueballs have never been experimentally detected although some calculations expect their masses to be within the ranges accessible by modern particle accelerators. Furthermore, their theoretical properties are extremely uncertain and the QCD physics underlying their existence is still somewhat hypothetical. However, QCD offers an important phenomenon lacking in other quantum field theories such as QED which most likely does not produce photonic bound states. QCD is a confining theory which becomes strongly coupled at low energy. I will claim that such properties is necessary for any bound states of massless particles to exist.   
\par
A remarkable theoretical prediction of bound states of massless particles comes from the seminal paper of Gross and Neveu who considered a $D = 1 + 1$ dimensional system which exhibits spontaneous symmetry breaking in the large $N$ limit and turns out to be exactly solvable \citep{Gross}. The Gross-Neveu model consists of $N$ massless fermionic fields mediated by Yukawa-like highly massive bosons in the $M \to \infty$ limit. The Lagrangian in question is given by,
\begin{equation}
\lagrange_{G-N} = \bar{\psi}_A (i \slashed{\partial}) \psi_A + \tfrac{1}{2} \partial^\mu \phi \partial_\mu \phi - \tfrac{1}{2} M^2 \phi^2 - g M \bar{\psi} \psi \phi
\end{equation}  
For the case of massless fermions, at low temperature the vacuum becomes unstable to decay into $\phi \bar{\psi}$ pairs which give $\psi$ a nonzero expectation value in the true vacuum. This instability is signaled by a tachyonic ($M^2 < 0$) bound state appearing in the spectrum of $\psi \bar{\psi}$ scattering. Such a tachyonic bound state is characteristic of the concave down shape around $\phi = 0$ of the ``wine-bottle'' shaped effective potential. Since this model has been solved exactly, we indeed see that bound states of massless particles are indeed possible but, at least in this case, their existence by signal an instability in the vacuum as the massless fields pair up to form a condensate. Furthermore, when expanded around the true vacuum, the new $\sigma$ particle pair acquires an effective mass and thus the phenomenology of massless fields constituting bound states is here a residual effect of expanding about the false vacuum. 
\par
I claim that such a phenomenon is not unique to the Gross-Neveu model but is actually a more general phenomenon. First, a true stable bound state of massless particles must have less energy that its constituents asymptotically separated to arbitrarily great distances. However, if the constituent particles are massless then such an asymptotic separated state may have arbitrarily low energy. Therefore, if the bound state has any finite binding energy then the total energy of the composite particle must be negative. Therefore, the vacuum is unstable to decay into such composite particles and their anti-particles. Therefore, the existence of a true bound state of massless particles necessitates a vacuum instability. Furthermore, we will see in our analysis of the Bethe-Saltpeter equation that any scattering resonance of massless fields requires a confining theory such as QCD which produces glueballs and massless Gross-Neveu which at low energies forces the massless $\psi$ field to condense into pairs. 

\subsection{Scattering Resonances between Massless and Massive Fields}

Recent efforts have calculated exact bound state spectra in the weakly coupled scalar Yukawa theory using the Feshback-Villars formulation \citep{Villars}. For the Yukawa theory mediated my massless mesons, the exact spectrum of two-particle bound states of scalar fields with masses $m_1$ and $m_2$ is calculated to be,
\begin{equation}
E_{\pm n} = \sqrt{m_1^2 + m_2^2 \pm 2 m_1 m_2 \sqrt{1 - \frac{\alpha}{n^2}}} 
\end{equation}  
\citep{two-body-coulomb}. Such results are confirmed using multiple methods including searching for poles in the eikonal approximation to the S-matrix, solving the Bethe-Saltpeter equation, and a Fokker action formulation \citep{spinor-scalar}. Furthermore, the bound state spectrum of a Yukawa-type interaction between a massive scalar field and a massive Dirac spinor field has been worked out explicitly \citep{spinor-scalar} again using the Feshbach-Villars formalism. These results give a similar spectrum to the scalar-scalar bound states with a modification to the principal qua tum number based on the energy and angular momentum of the bound state. In both cases, the bound state spectrum is poorly behaved in the limit as either $m_1 \to 0$ or $m_2 \to 0$. In either case, the energies of all bound states degenerate to the mass of the other particle i.e. in the limit $m_1 \to 0$ we have $E_{\pm n} \to m_2$. This implies that weakly-coupled Yukawa theories cannot give bound states or resonances for a massless particle scattering off a massive target. I have not been able to determine if this is a general phenomenon or an artifact of the scalar Yukawa model.  

\section{The Bethe-Saltpeter Equation}

The Bethe-Saltpeter equation gives, in principal, a general method for computing the bound state spectrum on any QFT \citep{Bethe}. However, there are monumental difficulties in solving the equation in any generalities or without nonrelativistic assumptions. Here, I present a derivation of the Bethe-Saltpeter equation and a generalized reduction to a wavefunction equation. 

\subsection{Derivation}

We define the compact two-point vertex which contains only diagrams which cannot be separated by cutting only two $\psi$ propagators. Then, the total interaction vertex is given by the infinite sum of ``ladder'' diagrams,
\begin{equation}
\feynmandiagram[horizontal=a to b, inline=(a.base), layered layout] {
i1 -- [fermion] a [blob],
a -- [anti fermion] i2,
f1 -- [anti fermion] a -- [fermion] f2,
};
\quad + \quad 
\feynmandiagram[horizontal=a to b, inline=(a.base)] {
i1 -- [fermion] a [blob],
a -- [anti fermion] i2,
b [blob] -- [anti fermion, half left] a -- [fermion, half left] b,
f1 -- [anti fermion] b -- [fermion] f2,
};
\quad + \quad 
\feynmandiagram[horizontal=a to b, inline=(a.base)] {
i1 -- [fermion] a [blob],
a -- [anti fermion] i2,
b [blob] -- [anti fermion, half left] a -- [fermion, half left] b,
c [blob] -- [anti fermion, half left] b -- [fermion, half left] c,
f1 -- [anti fermion] c -- [fermion] f2
};
\quad + \quad 
\feynmandiagram[horizontal=a to b, inline=(a.base)] {
i1 -- [fermion] a [blob],
a -- [anti fermion] i2,
b [blob] -- [anti fermion, half left] a -- [fermion, half left] b,
c [blob] -- [anti fermion, half left] b -- [fermion, half left] c,
d [blob] -- [anti fermion, half left] c -- [fermion, half left] d,
f1 -- [anti fermion] d -- [fermion] f2
};
\quad + \cdots
\end{equation}
Therefore, we can write the total four-point vertex as,
\begin{equation}
\feynmandiagram[horizontal=a to b, inline=(a.base), layered layout] {
i1 -- [fermion] a [blob],
a -- [anti fermion] i2,
f1 -- [anti fermion] a -- [fermion] f2,
};
\quad = \quad
\feynmandiagram[horizontal=a to b, inline=(a.base), layered layout] {
i1 -- [fermion] a [blob],
a -- [anti fermion] i2,
f1 -- [anti fermion] a -- [fermion] f2,
};
\quad + \quad 
\feynmandiagram[horizontal=a to b, inline=(a.base)] {
i1 -- [fermion] a [blob],
a -- [anti fermion] i2,
b [blob] -- [anti fermion, half left] a -- [fermion, half left] b,
f1 -- [anti fermion] b -- [fermion] f2,
};
\end{equation}
Therefore, the total scattering amplitude $\mathcal{M}$ can be written in terms of the compact interaction amplitude $\Gamma$ as,
\begin{align} 
i \mathcal{M}(p_1, p_2, p_1', p_2') & = i \Gamma(p_1, p_2, p_1', p_2') 
\\
& \quad + \int \frac{\dn{D}{k}}{(2 \pi)^D} G_{\psi_1}(p_1 + k) G_{\psi_2}(p_2 - k) \Gamma(p_1, p_2, p_1 + k, p_2 - k) \mathcal{M}(p_1 + k, p_2 - k, p_1', p_2') 
\end{align}
where $G_{\psi}$ is the exact $\psi$ propagator. We want to search for poles of $\mathcal{M}$ which correspond to bound states or resonances. In particular, if we refine the search to only consider resonances corresponding to composite particles or excited states rather than the creation of dressed particles explicitly in the Lagrangian then we can assume that such a pole only appears in the infinite ladder and is not a pole of compact vertex amplitude $\Gamma$. In the vicinity of such a pole, $\mathcal{M} \gg \Gamma$ so we search for solutions to the equation, 
\begin{equation}
i \mathcal{M}(p_1, p_2, p_1', p_2') = \int \frac{\dn{D}{k}}{(2 \pi)^D} G_{\psi_1}(p_1 + k) G_{\psi_2}(p_2 - k) \Gamma(p_1, p_2, p_1 + k, p_2 - k) \mathcal{M}(p_1 + k, p_2 - k, p_1', p_2') 
\end{equation}
Now we define the auxiliary function,
\begin{equation} 
\chi(P, q) = G_{\psi_1}(\tfrac{m_1}{M} P + q) G_{\psi_2}(\tfrac{m_2}{M} P - q) \mathcal{M}(\tfrac{m_1}{M} P + q, \tfrac{m_2}{M} P - q, p_1', p_2') 
\end{equation}
where we can identify, $P = p_1 + p_2$ and $q = \tfrac{1}{M}(m_2 p_1 - m_1 p_2)$ and $M = m_1 + m_2$. Therefore, in this notation,
\begin{equation} 
i \chi(P, q) = G_{\psi_1}(\tfrac{m_1}{M} P + q) G_{\psi_2}(\tfrac{m_2}{M} P - q) \int \frac{\dn{D}{k}}{(2\pi)^D} \chi(P, q + k) \Gamma(p_1, p_2, p_1 + k, p_2 - k) 
\end{equation}
It turns out that integrals of $\chi$ will take on physical interpretation as momentum-space wavefunctions.  

\subsection{An Approximate Condition for Resonances} 

So far the discussion has been exact. However, to get anywhere with the Bethe-Saltpeter equation we need to make some approximations. First, we make the approximation that the compact vertex amplitude is, under the Born approximation, given by a Fourier-transformed potential function,
\begin{equation}
\Gamma(p_1, p_2, p_1', p_2') = \tilde{U}(P, \vec{q}) 
\end{equation}
where $\tilde{U}(P, \vec{q})$ at a given total 4-momentum only depends on the vector part of $q = p_1' - p_1 = k$. This assumption restricts our analysis from detecting bound states in strongly coupled confining theories which are poorly described by such a potential function. This approximation is important because the total 4-momentum, 
\begin{equation}
P = p_1 + k + p_2 - k = p_1 + p_2
\end{equation} 
and thus $\tilde{U}(P, \vec{k})$ does not depend on $k^0$. The Bethe-Saltpeter equation becomes,
\begin{equation}
\chi(P, q) = i G_{\psi_1}(\tfrac{m_1}{M} P + q) G_{\psi_2}(\tfrac{m_2}{M} P - q) \int \frac{\dn{D}{k}}{(2\pi)^D} \chi(P, q + k) \tilde{U}(P, \vec{k}) 
\end{equation}
In this approximation scheme, $P$ becomes a parameter. Our goal is to determine for which values of $P$ this equation has pole solutions.  
Now, I define the wavefunction,
\begin{equation}
\psi_P(\vec{p}) = \int_{-\infty}^{\infty} \d{q^0} \chi(P, q) 
\end{equation}
Therefore, 
\begin{equation}
i \int \d{q^0} \chi(P, q) = \int_{-\infty}^{\infty} \d{q^0}  G_{\psi_2}(\tfrac{m_1}{M} P + q) G_{\psi_2}(\tfrac{m_2}{M} P - q) \int \frac{\d{\vec{k}}}{(2\pi)^{D}} \left[ \int \d{k^0} \chi(P, q + k) \right] \tilde{U}(P, \vec{k}) 
\end{equation}
However, we can reparametrize the integral of $\chi$ over $k^0$ to absorb the factor of $q^0$ and thus the left integral factors to not depend on $q^0$. Thus, this expression becomes,
\begin{equation}
i \psi_P(\vec{p}) = \left[ \int_{-\infty}^{\infty} \d{q^0} G_{\psi_1}(\tfrac{m_2}{M} P + q) G_{\psi_2}(\tfrac{m_1}{M} P - q) \right] \cdot \int \frac{\d{\vec{k}}}{(2\pi)^{D}} \psi_P(\vec{q} + \vec{k}) \tilde{U}(P, \vec{k}) 
\end{equation}

\subsubsection{Computing the Propagator Contour Integral}

We need to consider the integral, 
\begin{equation}
K(P, q) = \frac{1}{2 \pi i} \int_{-\infty}^{\infty} \d{q^0} G_{\psi_1}(\tfrac{m_1}{M} P + q) G_{\psi_2}(\tfrac{m_2}{M} P - q)
\end{equation}
Using the K\"{a}llen-Lehmann spectral representation we know that the exact propagator can be written as,
\begin{equation}
G_\psi(x-y) = \bra{\Omega} \mathrm{T} \psi(x) \psi^\dagger(y) \ket{\Omega} = \int \frac{\d{m^2}}{2 \pi} \rho(m^2) \int \frac{\dn{D}{p}}{(2\pi)^D} \frac{i}{p^2 - m^2 + i \epsilon} e^{-i p \cdot (x-y)}   
\end{equation} 
so applying the Fourier transform,
\begin{equation}
G_\psi(p) = \int \frac{\d{m^2}}{2 \pi} \rho(m^2) \frac{i}{p^2 - m^2 + i \epsilon} 
\end{equation}
As a first approximation, we suppose that the momentum $p$ remains close to on shell at the physical mass of the $\psi$ particle. Furthermore, we use the field strength renormalization scheme such that,
\begin{align*}
 G_{\psi}(p) = \frac{i}{p^2 - m_\psi^2 + i \epsilon} 
\end{align*}
This is a nontrivial approximation because we are ignoring the contribution of the multiparticle continuum and bound state poles which clearly are important when searching for bound states. Moreover, bound states of massless fields are necessarily highly relativistic and therefore we would expect there to be a good deal of mixing between the one-particle and multi-particle states in the internal propagators. For now, using this approximation, we can calculate the integral. For simplicity, we restrict to the center of mass frame such that $P = (E, 0)$. Then,
\begin{subequations}
\begin{align}
K(P, q) & = \frac{1}{2 \pi i} \int_{-\infty}^{\infty} \d{q^0} G_{\psi_1}(\tfrac{m_2}{M} P + q) G_{\psi_2}(\tfrac{m_1}{M} P - q)
\\
& = -\frac{1}{2 \pi i} \int_{-\infty}^{\infty} \d{q^0} \frac{1}{(\tfrac{m_1}{M} E + q^0)^2 - \vec{q}^{\, 2} - m_1^2 + i \epsilon} \cdot \frac{1}{(\tfrac{m_2}{M} E - q^0)^2 - \vec{q}^{\, 2} - m_2^2 + i \epsilon}
\end{align}
\end{subequations}
We will perform this integral via contour integration by closing the integration along the real axis with a contour in the upper half plane. To simplify the ensuing calculation, I will introduce parameters,
\begin{equation}
Q_1^2 = \vec{q}^{\, 2} + m_1^2 \quad \quad Q_2^2 = \vec{q}^{\, 2} + m_2^2 \quad \quad E_1 = \frac{m_1}{M} E \quad \quad E_2 = \frac{m_2}{M} E
\end{equation}
Therefore, introducing partial fractions,
\begin{subequations}
\begin{align}
K(P, q) & = -\frac{1}{2 \pi i} \int_{-\infty}^{\infty} \d{q^0} \frac{1}{(E_1 + q^0)^2 - Q_1^2 + i \epsilon} \cdot \frac{1}{(E_2 - q^0)^2 - Q_2^2 + i \epsilon}
\\
& = -\frac{1}{2 \pi i} \int_{-\infty}^{\infty} \d{q^0} \frac{1}{2 Q_1} \left[ \frac{1}{q^0 - (-E_1 + Q_1 - i \epsilon )} - \frac{1}{q^0 - (-E_1 - Q_1 + i \epsilon)} \right]
\\
& \quad \quad \quad \quad \quad \quad \cdot \frac{1}{2 Q_2} \left[ \frac{1}{q^0 - (E_2 + Q_2 - i \epsilon)} - \frac{1}{q^0 - (E_2 - Q_2 + i \epsilon)} \right]
\\
& = \frac{1}{4 Q_1 Q_2} \left[ \left( \frac{1}{E - Q_1 - Q_2} - \frac{1}{E + Q_1 - Q_2} \right) + \left( \frac{1}{-E - Q_1 - Q_2} - \frac{1}{-E - Q_1 + Q_2} \right) \right] 
\\
& = \frac{1}{4 Q_1 Q_2} \left[ \frac{1}{E - Q_1 - Q_2} +  \frac{1}{-E - Q_1 - Q_2}  \right] 
\\
& = \frac{2(Q_1 + Q_2)}{4 Q_1 Q_2} \cdot \frac{1}{E^2 - (Q_1 + Q_2)^2} = \frac{1}{2 Q_{\mathrm{red}}} \cdot \frac{1}{E^2 - (Q_1 + Q_2)^2} 
\end{align}
\end{subequations}
Where I have defined the reduced energy parameter,
\begin{equation}
Q_{\mathrm{red}} = \frac{Q_1 Q_2}{Q_1 + Q_2} 
\end{equation}

\subsection{The Momentum Space Condition}

Using this explicit form of the propagator integral the resonance condition becomes,
\begin{equation}
[ E^2 - (Q_1 + Q_2)^2 ] \psi_P(\vec{q}) = \frac{1}{2 Q_{\mathrm{red}}} \int \frac{\d{\vec{k}}}{(2\pi)^{D-1}} \psi_P(\vec{q} + \vec{k}) \tilde{U}(P, \vec{k}) 
\end{equation}
The left hand size of this equation is a convolution of the momentum wavefunction with the Fourier transformed potential. First, let us consider the dimensions of the potential. I identified $\tilde{U}$ with a scattering amplitude which is dimensionless in $D = 4$. However, for $\tilde{U}$ to be the Fourier transform of an honest-to-god squared potential energy it ought to have mass dimension $-1$ (in $D = 4$) so that when Fourier transformed it has mass dimension $+2$. Therefore, for dimensional consistency, we define the ``true'' potential,
\begin{equation}
U(P, \vec{k}) = \frac{1}{2Q_{\mathrm{red}}} \tilde{U}(P, \vec{k}) 
\end{equation}
This factor is exactly the ratio of the relativistic normalization factors,
\begin{equation}
\frac{2 E_1 2 E_2}{2E_{\mathrm{tot}}}
\end{equation} 
going from the independent two particle normalization to the normalization of a single particle with the prescribed total energy. Since $\tilde{U} = \Gamma$ is a scattering amplitude defined using such two-particle relativistic normalization, it makes sense that this factor appears to shift this quantity into the conventions of a single particle. Our final expression becomes,
\begin{equation}
[ E^2 - (Q_1 + Q_2)^2 ] \psi_P(\vec{q}) = \int \frac{\d{\vec{k}}}{(2\pi)^{D-1}} \psi_P(\vec{q} + \vec{k}) U(P, \vec{k}) = (\psi_P * U_P) (\vec{q})
\end{equation}
 

\subsection{A Klein-Gordon Condition for the $m \to 0$ Limit}
Consider, $Q_i^2 = \vec{q}^{\, 2} + m_i^2$. In the limit $m_1 \to 0$ and $m_2 \to 0$ we have $Q_i = |\vec{q}|$. Therefore,
\begin{equation}
(Q_1 + Q_2)^2 = (2 |\vec{q}|)^2 = 4 \vec{q}^{\, 2}
\end{equation}
Therefore, in momentum space, the bound state condition becomes,
\begin{equation}
[E^2 - 4 \vec{q}^{\, 2}] \psi_P(\vec{q}) = (U_P * \psi_P)(\vec{q}) 
\end{equation}
Applying the Fourier transform the convolution becomes a product and multiplication by the momentum becomes the differential momentum operator,
\begin{equation}
[E^2 - 4 \hat{q}^2 ] \psi_P(\vec{x}) = U_P(\vec{x}) \psi_P(\vec{x})
\end{equation}
This is exactly the massless Klein-Gordon equation\footnote{As compared with the Klein-Gordon equation there is an extra factor of $2$ in the momentum operator reflecting the fact that this is the relative momentum rather than the momentum of a single particle.} with a source term $U_P(\vec{x})$. 
\bigskip\\
I claim that this equation has no solutions for asymptotically zero potentials. Suppose that $U_P \to 0$ as $|\vec{x}| \to 0$. In this case, the $|\vec{x}| \to 0$ limit of this differential equation becomes,
\begin{equation}
E^2 \psi_P(\vec{x}) = \hat{q}^2 \psi_P(\vec{x})
\end{equation}
However, $E^2 > 0$ and thus $\psi_P$ has a positive expectation value for $\hat{q}^2$ which means that $\psi_P$ must be oscillatory out to arbitrarily large distances rather than exponentially suppressed. Therefore, the only solutions are scattering states with an energy continuum rather than discrete bound state energy levels. These do not give resonances in the total scattering amplitude but instead correspond to run-of-the-mill continuous scattering. 
\bigskip\\
Therefore, for theories without confinement at low energy there cannot exist bound states of two massless particles. This confirms the argument due to vacuum decay but also implies that there cannot be unstable resonances in the scattering of massless particles which do not appear in the compact diagrams i.e. which are not due to the creation of an (approximately) on-shell fundamental particle of a field appearing explicitly the Lagrangian. A consequence of this result is that given a theory with only massless fields, no mass terms can appear when applying coordinate transformations or introducing auxiliary fields unless there exists confinement at zero energy or an instability of the vacuum.     

\subsection{The Condition for Resonant Scattering of Massless Fields from Massive Targets}

We now consider a different limit of our general momentum space bound state condition. If we consider the limit in which one particle is massless and the other is highly massive in the nonrelativistic regime, $m_1^2 \ll \vec{q}^{\, 2} \ll m_2^2$ then $Q_1 = |\vec{q}|$ and $Q_2 \approx m_2 + \frac{\vec{q}^{\, 2}}{2 m_2}$. Therefore,
\begin{equation}
(Q_1 + Q_2)^2 = \left( |\vec{q}| + m_2 + \frac{\vec{q}^{\, 2}}{2 m_2} \right)^2 = m_2^2 + 2\vec{q}^{\, 2} + 2m_2 |\vec{q}| + \frac{|\vec{q}|^3}{m_2} + \frac{\vec{q}^{\, 4}}{4 m_2^2}  \approx m_2^2 + 2 m_2 |\vec{q}|
\end{equation}
so the momentum space condition reduces to,
\begin{equation}
\left[E^2 - m_2^2 - 2 m_2 |\vec{q}| \right] \psi_P(\vec{q}) = (U_P * \psi_P)(\vec{q})
\end{equation}
The absolute value precludes solving this equation by Fourier transforming back to position space. We might attempt to look for solutions by using trial potential functions. A particularly convenient potential is a delta function in $D = 4$ dimensions,
\begin{equation}
U_P(\vec{x}) = -C_P \delta^3(\vec{x}) \implies U_P(\vec{q}) = -C_P
\end{equation}
Therefore, 
\begin{equation}
(U_P * \psi_P)(\vec{q}) = -\int \frac{\d{\vec{k}}}{(2\pi)^3} C_P \psi_P(\vec{k} + \vec{q}) = -C_P \psi_P(\vec{x} = 0)
\end{equation}
This leads to an immediate solution for $\psi_P$ in momentum space,
\begin{equation}
\psi_P(\vec{q}) = \frac{\psi_P(\vec{x} = 0) C_p}{m_2^2 - E^2 + 2 m_2 |\vec{q}|}
\end{equation}
However, this form is inconsistent because the inverse Fourier transform of this functional form at $\vec{x} = 0$ diverges because the integral over $D = 3$ of $1/(1+r)$ diverges and thus cannot equal the value $\psi_P(\vec{x} = 0)$ used to define the momentum space wavefunction. Therefore, there are no bound state solutions for a delta function interaction in $D = 4$ dimensions. However, this analysis does not preclude the possibility of massless-massive bound states arising from some more complex potential function. As of yet, I have been unable to make general statements regarding solutions to the high mass ($m_2$) limit momentum space condition,
\begin{equation}
\left[E^2 - m_2^2 - 2 m_2 |\vec{q}| \right] \psi_P(\vec{q}) = (U_P * \psi_P)(\vec{q})
\end{equation}

\nocite{*}

\bibliography{mybib}


\end{document}