\documentclass{article}
\usepackage[english]{babel}
\usepackage[utf8]{inputenc}
\usepackage[english]{babel}
\usepackage[a4paper, total={7.25in, 9.5in}]{geometry}
\usepackage{tikz-feynman}
\tikzfeynmanset{compat=1.0.0} 
\usepackage{subcaption}
\usepackage{float}
\floatplacement{figure}{H}
\usepackage{simpler-wick}
\usepackage{mathrsfs}  
\usepackage{dsfont}
\usepackage{relsize}
\usepackage{tikz-cd}
\DeclareMathAlphabet{\mathdutchcal}{U}{dutchcal}{m}{n}

\usepackage{cancel}



\newcommand{\field}{\hat{\Phi}}
\newcommand{\dfield}{\hat{\Phi}^\dagger}
 
\usepackage{amsthm, amssymb, amsmath, centernot}
\usepackage{slashed}
\newcommand{\notimplies}{%
  \mathrel{{\ooalign{\hidewidth$\not\phantom{=}$\hidewidth\cr$\implies$}}}}
 
\renewcommand\qedsymbol{$\square$}
\newcommand{\cont}{$\boxtimes$}
\newcommand{\divides}{\mid}
\newcommand{\ndivides}{\centernot \mid}

\newcommand{\Integers}{\mathbb{Z}}
\newcommand{\Natural}{\mathbb{N}}
\newcommand{\Complex}{\mathbb{C}}
\newcommand{\Zplus}{\mathbb{Z}^{+}}
\newcommand{\Primes}{\mathbb{P}}
\newcommand{\Q}{\mathbb{Q}}
\newcommand{\R}{\mathbb{R}}
\newcommand{\ball}[2]{B_{#1} \! \left(#2 \right)}
\newcommand{\Rplus}{\mathbb{R}^+}
\renewcommand{\Re}[1]{\mathrm{Re}\left[ #1 \right]}
\renewcommand{\Im}[1]{\mathrm{Im}\left[ #1 \right]}
\newcommand{\Op}{\mathcal{O}}

\newcommand{\invI}[2]{#1^{-1} \left( #2 \right)}
\newcommand{\End}[1]{\text{End}\left( A \right)}
\newcommand{\legsym}[2]{\left(\frac{#1}{#2} \right)}
\renewcommand{\mod}[3]{\: #1 \equiv #2 \: \mathrm{mod} \: #3 \:}
\newcommand{\nmod}[3]{\: #1 \centernot \equiv #2 \: mod \: #3 \:}
\newcommand{\ndiv}{\hspace{-4pt}\not \divides \hspace{2pt}}
\newcommand{\finfield}[1]{\mathbb{F}_{#1}}
\newcommand{\finunits}[1]{\mathbb{F}_{#1}^{\times}}
\newcommand{\ord}[1]{\mathrm{ord}\! \left(#1 \right)}
\newcommand{\quadfield}[1]{\Q \small(\sqrt{#1} \small)}
\newcommand{\vspan}[1]{\mathrm{span}\! \left\{#1 \right\}}
\newcommand{\galgroup}[1]{Gal \small(#1 \small)}
\newcommand{\bra}[1]{\left| #1 \right>}
\newcommand{\Oa}{O_\alpha}
\newcommand{\Od}{O_\alpha^{\dagger}}
\newcommand{\Oap}{O_{\alpha '}}
\newcommand{\Odp}{O_{\alpha '}^{\dagger}}
\newcommand{\im}[1]{\mathrm{im} \: #1}
\renewcommand{\ker}[1]{\mathrm{ker} \: #1}
\newcommand{\ket}[1]{\left| #1 \right>}
\renewcommand{\bra}[1]{\left< #1 \right|}
\newcommand{\inner}[2]{\left< #1 | #2 \right>}
\newcommand{\expect}[2]{\left< #1 \right| #2 \left| #1 \right>}
\renewcommand{\d}[1]{ \mathrm{d}#1 \:}
\newcommand{\dn}[2]{ \mathrm{d}^{#1} #2 \:}
\newcommand{\deriv}[2]{\frac{\d{#1}}{\d{#2}}}
\newcommand{\nderiv}[3]{\frac{\dn{#1}{#2}}{\d{#3^{#1}}}}
\newcommand{\pderiv}[2]{\frac{\partial{#1}}{\partial{#2}}}
\newcommand{\fderiv}[2]{\frac{\delta #1}{\delta #2}}
\newcommand{\parsq}[2]{\frac{\partial^2{#1}}{\partial{#2}^2}}
\newcommand{\topo}{\mathcal{T}}
\newcommand{\base}{\mathcal{B}}
\renewcommand{\bf}[1]{\mathbf{#1}}
\renewcommand{\a}{\hat{a}}
\newcommand{\adag}{\hat{a}^\dagger}
\renewcommand{\b}{\hat{b}}
\newcommand{\bdag}{\hat{b}^\dagger}
\renewcommand{\c}{\hat{c}}
\newcommand{\cdag}{\hat{c}^\dagger}
\newcommand{\hamilt}{\hat{H}}
\renewcommand{\L}{\hat{L}}
\newcommand{\Lz}{\hat{L}_z}
\newcommand{\Lsquared}{\hat{L}^2}
\renewcommand{\S}{\hat{S}}
\renewcommand{\empty}{\varnothing}
\newcommand{\J}{\hat{J}}
\newcommand{\lagrange}{\mathcal{L}}
\newcommand{\dfourx}{\mathrm{d}^4x}
\newcommand{\meson}{\phi}
\newcommand{\dpsi}{\psi^\dagger}
\newcommand{\ipic}{\mathrm{int}}
\newcommand{\tr}[1]{\mathrm{tr} \left( #1 \right)}
\newcommand{\C}{\mathbb{C}}
\newcommand{\CP}[1]{\mathbb{CP}^{#1}}
\newcommand{\Vol}[1]{\mathrm{Vol}\left(#1\right)}

\newcommand{\Tr}[1]{\mathrm{Tr}\left( #1 \right)}
\newcommand{\Charge}{\hat{\mathbf{C}}}
\newcommand{\Parity}{\hat{\mathbf{P}}}
\newcommand{\Time}{\hat{\mathbf{T}}}
\newcommand{\Torder}[1]{\mathbf{T}\left[ #1 \right]}
\newcommand{\Norder}[1]{\mathbf{N}\left[ #1 \right]}
\newcommand{\Znorm}{\mathcal{Z}}
\newcommand{\EV}[1]{\left< #1 \right>}
\newcommand{\interact}{\mathrm{int}}
\newcommand{\covD}{\mathcal{D}}
\newcommand{\conj}[1]{\overline{#1}}

\newcommand{\SO}[2]{\mathrm{SO}(#1, #2)}
\newcommand{\SU}[2]{\mathrm{SU}(#1, #2)}

\newcommand{\anticom}[2]{\left\{ #1 , #2 \right\}}


\newcommand{\pathd}[1]{\! \mathdutchcal{D} #1 \:}

\renewcommand{\theenumi}{(\alph{enumi})}


\renewcommand{\theenumi}{(\alph{enumi})}

\newcommand{\atitle}[1]{\title{% 
	\large \textbf{Physics GR8048 Quantum Field Theory II
	\\ Assignment \# #1} \vspace{-2ex}}
\author{Benjamin Church }
\maketitle}

\newcommand{\atitleIII}[1]{\title{% 
	\large \textbf{Physics GR8049 Quantum Field Theory III
	\\ Assignment \# #1} \vspace{-2ex}}
\author{Benjamin Church }
\maketitle}

\theoremstyle{definition}
\newtheorem{theorem}{Theorem}[section]
\newtheorem{definition}{definition}[section]
\newtheorem{lemma}[theorem]{Lemma}
\newtheorem{proposition}[theorem]{Proposition}
\newtheorem{corollary}[theorem]{Corollary}
\newtheorem{example}[theorem]{Example}
\newtheorem{remark}[theorem]{Remark}

\begin{document}

\subsection{Center of Mass Frame Calculations}

Although we have computed a valid expression for the scattering cross section, it is not much use without further simplifications. All plots (not directly referencing these ratios) will be for the case $M = 2.1 m$ and $g = 0.1 M$. First, notice that the reduced vertex function only appears with on-shell momentum arguments. Therefore, we can simplyfy,
\[ - i \tilde{\Gamma}_{on-shell}(p_1, p_2) = \frac{i g^2}{16 \pi^2} \int_0^1 \frac{\d{x} \d{y} \d{z} \delta(x + y + z - 1) [(y - z)^2 M^2 - (y p_1 - zp_2)^2]}{[(y p_1 - z p_2)^2 + x m^2][(y - z)^2 M^2  + x m^2]} \]
The next simplification comes from the facts of two particle scattering in the center of mass frame. In this reference frame, we can write our four momentum fourvectors as,
\[ p_1 = (E, \vec{p}) \quad p_2 = (E, -\vec{p}) \quad p_1' = (E, \vec{p'}) \quad p_2' = (E, -\vec{p'}) \]
With this parametrization, the Mandelstam paramters become,
\[ s = (p_1 + p_2)^2 = 4 E^2 \quad t = (p_1 - p_1')^2 = -2 \vec{p}^{\, 2} (1 - \cos{\Theta}) \quad u = (p_1 - p_2')^2 = -2 \vec{p}^{\, 2} (1 + \cos{\Theta}) \]
For notational convinience, let $\alpha = 4 \vec{p}^{\, 2}$.
Now we need to simplify the eight integrals,
\begin{align*} 
- i \tilde{\Gamma}_{on-shell}(p_1, p_2) &= \frac{i g^2}{16 \pi^2} \int_0^1 \frac{\d{x} \d{y} \d{z} \delta(x + y + z - 1) [(y - z)^2 M^2 - (y - z)^2 E^2 + (y + z)^2 \alpha]}{[(y - z)^2 E^2 - (y + z)^2 \alpha/4 + x m^2][(y - z)^2 M^2  + x m^2]} 
\\
& = \frac{i g^2}{16 \pi^2} \int_0^1 \frac{\d{x} \d{y} \d{z} \delta(x + y + z - 1) [ (y + z)^2 \alpha/4 - (y - z)^2 \alpha/4 ]}{[(y - z)^2 E^2 - (y + z)^2 \alpha/4 + x m^2][(y - z)^2 M^2  + x m^2]}
\\
& = \frac{i g^2}{16 \pi^2} \int_0^1 \frac{[yz \alpha] \: \: \d{x} \d{y} \d{z} \delta(x + y + z - 1) }{[(y - z)^2 M^2 - yz \alpha + x m^2][(y - z)^2 M^2  + x m^2]}
\end{align*}
We call this function $- i \tilde{\Gamma}(\alpha)$ which is plotted below.
(INSERT PLOT)
Similarly, $-i \tilde{\Gamma}(p_1', p_2') = -i \tilde{\Gamma}(\alpha)$ since the integral is expressed only in terms of rotationally invariant quantities and $(p_1', p_2')$ is a rotation of $(p_1, p_2)$ through the scattering angle. Next, we need to calculate, 
\begin{align*} 
- i \tilde{\Gamma}_{on-shell}(p_1, p_1') &= \frac{i g^2}{16 \pi^2} \int_0^1 \frac{\d{x} \d{y} \d{z} \delta(x + y + z - 1) [(y - z)^2 (M^2 -  E^2) + (y^2 + z^2 - 2 yz \cos{\Theta}) \alpha/4 ]}{[(y - z)^2 (M^2 + \alpha/4) - (y^2 + z^2 - 2 yz \cos{\Theta}) \alpha/4 + x m^2][(y - z)^2 M^2  + x m^2]} 
\\
& = \frac{i g^2}{16 \pi^2} \int_0^1 \frac{\d{x} \d{y} \d{z} \delta(x + y + z - 1) [ - 2(yz + yz \cos{\Theta}) \alpha/4 ]}{[(y - z)^2 M^2 -2yz(1 + \cos{\Theta})\alpha/4 + x m^2][(y - z)^2 M^2  + x m^2]}
\\
& = \frac{i g^2}{16 \pi^2} \int_0^1 \frac{[yz u] \: \: \d{x} \d{y} \d{z} \delta(x + y + z - 1) }{[(y - z)^2M^2 - yz u + x m^2][(y - z)^2 M^2  + x m^2]}
\end{align*}
This is the same function $-i \tilde{\Gamma}(u)$ simply with $\alpha$ replaced with $u$. Similarly, $-i \tilde{\Gamma}(p_2, p_2') = -i \tilde{\Gamma}(u)$ because this integral is invariant under partiy inversion and $(p_2, p_2')$ is a inversion through the center of mass of $(p_1, p_1')$. \bigskip\\
Now, we need to calculate the box diagram integrals,
\begin{align*}  
i\mathcal{M}_B(p_1, p_2, p_1', p_2') &= \frac{ig^4}{16 \pi^2} \int_{0}^{1} \frac{\d{x} \d{y} \d{z} \d{w} \delta(x + y + z + w - 1) }{[(y p_1 + z (p_1 + p_2) + w p_1' )^2 - z (p_1 + p_2)^2 + (x + z)m^2 - i \epsilon]} 
\\
& = \frac{ig^4}{16 \pi^2} \int_{0}^{1} \frac{\d{x} \d{y} \d{z} \d{w} \delta(x + y + z + w - 1) }{[(y + 2 z + w)^2 (M^2 + \alpha/4) - \alpha/4(y^2 + w^2 + 2 yw \cos{\Theta}) - z s + (x + z)m^2 - i \epsilon]} 
\\
& = \frac{ig^4}{16 \pi^2} \int_{0}^{1} \frac{\d{x} \d{y} \d{z} \d{w} \delta(x + y + z + w - 1) }{[M^2 (y + w)^2 - yw t + (y + w) s + (x + z)m^2 - i \epsilon]^2} 
\end{align*}
similarly,
\begin{align*}  
i\mathcal{M}_B(p_1, -p_1', -p_2, p_2') & = \frac{ig^4}{16 \pi^2} \int_{0}^{1} \frac{\d{x} \d{y} \d{z} \d{w} \delta(x + y + z + w - 1) }{[M^2 (y + w)^2 - yw s + (y + w)z t + (z^2 - z) t + (x + z)m^2 - i \epsilon]^2} 
\\
i\mathcal{M}_B(p_1, p_2, p_2', p_1') & = \frac{ig^4}{16 \pi^2} \int_{0}^{1} \frac{\d{x} \d{y} \d{z} \d{w} \delta(x + y + z + w - 1) }{[M^2 (y + w)^2 - yw u + (y + w)z s + (z^2 - z) s + (x + z)m^2 - i \epsilon]^2} 
\\
i\mathcal{M}_B(p_1, p_2, p_2', p_1') & = \frac{ig^4}{4 \pi^2} \int_{0}^{1} \frac{\d{x} \d{y} \d{z} \d{w} \delta(x + y + z + w - 1) }{[M^2 (y + w)^2 - yw t + (y + w)z u + (z^2 - z) u + (x + z)m^2 - i \epsilon]^2} 
\end{align*}
Therefore, these three integrals are all expressible in terms of a single function,
\[ B(a,b) = -\frac{g^2}{16 \pi^2} \int_{0}^{1} \frac{\d{x} \d{y} \d{z} \d{w} \delta(x + y + z + w - 1) }{[M^2 (y + w)^2 - yw a + (y + w)z b + (z^2 - z) b + (x + z)m^2 - i \epsilon]} \]
In terms of this function,
\begin{align*}
(-ig)^2 B(t, s) &= \mathcal{M}_B(p_1, p_2, p_1', p_2') \\
(-ig)^2 B(s, t) &= \mathcal{M}_B(p_1, -p_1', -p_2, p_2') \\
(-ig)^2 B(u, s) &= \mathcal{M}_B(p_1, p_2, p_2', p_1') \\
(-ig)^2 B(t, u) &= \mathcal{M}_B(p_1, -p_2', p_1', -p_2) 
\end{align*}
Therefore, 
\[\mathcal{M}_{box} = (-ig)^2[B(t, s) + B(s, t) + B(u, s) + B(t, u)]\]
The $B$ integral must be computed numerically.
(ADD FIGURE OF B) 
Now we have all the tools to evaluate the explict scattering cross section in the center of mass reference frame. In terms of these new functions,
\begin{align*}
\deriv{\sigma}{\Omega} 
&= \frac{g^4}{64 \pi^2 s} \left( \left[ \frac{2\tilde{\Gamma}(\alpha)}{s - m^2} + \frac{2\tilde{\Gamma}(u)}{t - m^2} - \frac{\mathrm{Re}[\Sigma(t)]}{(t - m^2)^2} - \frac{\mathrm{Re}[\Sigma(s)]}{(s - m^2)^2} \right]^2 \right.
\\ 
& \left.
\quad \quad \quad \quad \quad \quad 
+ \left[ \frac{1}{t - m^2} + \frac{1}{s - m^2} + B(t, s) + B(s, t) + B(u, s) + B(t, u) + \frac{\mathrm{Re}[\Sigma(t)]}{(t - m^2)^2} + \frac{\mathrm{Re}[\Sigma(s)]}{(s - m^2)^2} \right]^2 \right) 
\end{align*}

\subsection{Explicit Evaluation of the Scattering Cross Section}





\subsection{The Scattering Cross Section and Field Strength Renormalization}

Finally, the glorious total scattering amplitude is given by the sum of all these contributions. In the center of mass reference frame, remembering the factor of $Z_\psi^{1/2}$ which appear due to the LSZ reduction formula expression for the $S$ matrix, the differential scattering cross section is given by,
\begin{align*}
\deriv{\sigma}{\Omega} & = \frac{|Z^2_{\psi} \mathcal{M}|^2}{64 \pi^2 E_{CM}^2}
\\
& = \frac{Z^4_{\psi}}{64 \pi^2 E_{CM}^2} \left| \mathcal{M}_a + \mathcal{M}_b + \mathcal{M}_c + \mathcal{M}_d + \mathcal{M}_e + \mathcal{M}_f + \mathcal{M}_{box} + \mathcal{M}_{tree} \right|^2 
\end{align*}
Because of our incomplete renormalization prescription, this scattering cross section is carrying around awkward factors of $Z_\psi^{1/2}$. Since the entire expression is proportional to $g^4$ we might consider rescaling the coupling constant $g$ to $g' = Z_\psi^{1/2} g$ such that there is no explict dependence on $Z_\psi$ out front. This is not quite right, as we will more explicitly see, because it does not account for the terms in $\mathcal{M}$ which depend on higher powers of $g$. We also see hints that our coupling constant might be wrong by seriously considering our vertex renormalization condition. We imposed that $-i \Gamma(p_1, p_2) = -ig$, the physical couplingn constant, at a specific renormalization point. However, if one calculates diagrams contributing to the vertex function, the LSZ reduction formula\footnote{While these diagrams cannot conserve momentum on shell and thus cannot be interpreted as scattering amplitudes, the pole structure of their associated correlation functions should align with the LSZ calculation and would produce the same overal factor when included as a subdiagram of a larger physically possible scattering process.} implies that the amplitude should contain a factor of $Z_\psi^2 Z_\phi^{1/2}$. So we are actually renormalizing our physical coupling constant at $- i Z_\psi Z_\phi^{1/2} g$ meaning that $g$ is not the truly physical coupling paramter of the theory. \bigskip\\
To fix this issue once and for all, let us return to the unrenormalized Lagrangian,
\[ \lagrange = \lagrange_0 + \lagrange_{\ipic} \quad \quad \lagrange_0 = \tfrac{1}{2} \partial_\mu \phi_0 \partial^\mu \phi_0 - \tfrac{1}{2} m_0^2 \phi^2 + \partial_\mu \dpsi_0 \partial^\mu \psi_0 - M_0^2 \dpsi_0 \psi_0 \quad \quad \lagrange_{\ipic} = -g_0 \dpsi \psi \meson \]  
where zeros indicate the bare paramters. The first step is to shift the the constants to more physical values (which I will label with a one) by putting the difference into counterterms,
\[ \lagrange = \tfrac{1}{2} \partial_\mu \phi_0 \partial^\mu \phi_0 - \tfrac{1}{2} m_1^2 \phi^2 + \partial_\mu \dpsi_0 \partial^\mu \psi_0 - M_1^2 \dpsi \psi - g_1 \dpsi_0 \psi_0 \meson_0 - c_0 - c_1 \phi_0 - c_2 \dpsi_0 \psi_0 -\tfrac{1}{2} c_3 \phi_0^2 - c_4 \dpsi_0 \psi_0 \phi_0 \]  
However, we have yet to fix the issue of resuides in the dressed propagator. To do this, rescale the fields, $\phi_0 = Z^{1/2}_\phi \phi$ and $\psi = Z^{1/2}_\psi \psi$ which contributes a factor of $Z^{-1/2}$ for each field in a correlation function and thus fixes the pole structure of the exact dressed propagator that would appear in the LSZ reduction formula as,
\[ \tilde{G}^{(2)}(p) = \frac{i}{p^2 - m^2 + i \epsilon} \] 
for a stable particle and for an unstable particle, the two point function becomes,
\[ \tilde{G}^{(2)}(p) = \frac{i}{p^2 - m^2 + i m \Gamma} \] 
The Lagrangian becomes,
\begin{align*}
\lagrange & = \tfrac{1}{2} Z_\phi \partial_\mu \phi \partial^\mu \phi - Z_\phi \tfrac{1}{2} m_1^2 \phi^2 + Z_\psi \partial_\mu \dpsi \partial^\mu \psi - Z_\psi M_1^2 \dpsi \psi - Z_\psi Z_\phi^{1/2} g_1 \dpsi \psi \meson 
\\
& - c_0 - c_1 Z_\phi^{1/2} \phi - c_2 Z_\psi \dpsi \psi - Z_\phi \tfrac{1}{2} c_3 \phi^2 - Z_\psi Z_\phi^{1/2} c_4 \dpsi \psi \phi
\end{align*} 
which explains why $Z_\psi Z_\phi^{1/2} g_1$ seemed to be the more physical quantity in the partially renormalized theory. However, if we are going to expand this Lagrangian in perturbation theory, we want our noninteracting part to look like the free theories we have perviously considered. To do this, we can offload the difference into counterterms, 
\[ \lagrange = \tfrac{1}{2} \partial_\mu \phi \partial^\mu \phi -  \tfrac{1}{2} m^2 \phi^2 + \partial_\mu \dpsi \partial^\mu \psi - M_1^2 \dpsi \psi - g \dpsi \psi \meson - c_0 - c_1 \phi - c_2 \dpsi \psi - \tfrac{1}{2} c_3 \phi^2 - c_4 \dpsi \psi \phi - c_5 \tfrac{1}{2} \partial_\mu \phi \partial^\mu \phi - c_6 \partial_\mu \dpsi \partial^\mu \psi \] 
where $c_2 = M_0^2 Z_\psi - M^2$ and $c_3 = \tfrac{1}{2} (m_0^2 Z_\phi - m^2)$ and $c_5 = Z_\phi - 1$ and $c_6 = Z_\psi - 1$. 
Now the free Lagrangian is written solely in terms of physical measurable quantities. 
Putting the counterterms into the interacting part of the Lagrangian, the Feynman rules (specifically vertex rules) are only slightly altered, 
	\begin{equation*}
	\feynmandiagram[horizontal = b to a, small, inline = (a)] {
	i1 -- [fermion] a -- [fermion] f1,
	a -- [scalar] b
	}; = - i g
	\hspace{2cm}
	\feynmandiagram[horizontal = i1 to o1, small, inline = (c),  tree layout] {
	i1 -- [fermion] c [crossed dot] -- [fermion] o1
	}; = - i (c_2 - p^2 c_6)
	\hspace{2cm}
	\feynmandiagram[horizontal = i1 to o1, small, inline = (c),  tree layout] {
	i1 -- [scalar] c [crossed dot] -- [scalar] o1
	}; = - i (c_3 - p^2 c_5)
	\end{equation*}
The only place in our derivations where this matters is in the calulation of the one-particle irreducible terms $-i \Sigma(p^2)$. We now have stricter renormalization conditions, $\mathrm{Re}[\Sigma(p^2 = m^2)] = 0$ and $\deriv{\Sigma}{p^2} \bigg|_{p^2 = m^2} = 0$ to retain the form of the dressed propagator. This extra condition fixes the counterterms $c_5$ and $c_6$. Matching the counterterms, the new field strength normalized one-particle irreducible amplitude is,
\[ - i \Sigma(p^2) = - i \Sigma^{(old)}(p^2) + i (p^2 - m^2) \deriv{\Sigma^{(old)}}{p^2} \] 
Beyond that, calculations in this renormalization prescription are identical except for the fact that vertex renormalization now has a more physical meaning. To get single vertex amplitudes to work out correctly in the nonrenormalized theory, we can fix $g_1$ by the condition $g = Z_\psi Z_\phi^{1/2} g_1$ where $g$ is taken to be the physical coupling constant and zero momentum. Now, we will consider the physically measurable differential cross section as calculated in these two different renormalization prescriptions. In terms of the old nonrenormalized (not field strength renormalized) variables,
\begin{align*}
\deriv{\sigma^{(old)}}{\Omega} & = \frac{|Z^2_{\psi} \mathcal{M}|^2}{64 \pi^2 E_{CM}^2}
\\
& = \frac{Z^4_{\psi}}{64 \pi^2 E_{CM}^2} \left| \mathcal{M}_a + \mathcal{M}_b + \mathcal{M}_c + \mathcal{M}_d + \mathcal{M}_e + \mathcal{M}_f + i\mathcal{M}_{box} + \mathcal{M}_{tree} \right|^2 
\\
& = \frac{g_1^4 Z^4_{\psi}}{64 \pi^2 E_{CM}^2} \left|  \frac{ [-i \tilde{\Gamma}(p_1, p_2)] i}{s  - m^2} + \frac{i}{t - m^2} \cdot\left[ -i \Sigma^{(old)}(t) \right] \cdot \frac{i}{t - m^2} + \frac{[ -i \tilde{\Gamma}(p_1', p_2')]i}{s - m^2} 
+  \frac{[-i \tilde{\Gamma}(p_1, p_1')]i}{t - m^2} \right.
\\ 
& \left.
\quad \quad \quad \quad \quad \quad 
+ \frac{i}{s - m^2} \left[- i \Sigma^{(old)}(s) \right] \frac{i}{s - m^2} + \frac{[-i \tilde{\Gamma}(p_2, p_2')]i}{t - m^2} + i\mathcal{M}_{box}/(-ig)^2 +  \left[\frac{i}{t - m^2 } + \frac{i}{s - m^2 }\right] \right|^2 
\\
& = \frac{g_1^4 Z^4_{\psi}}{64 \pi^2 E_{CM}^2} \left|  \frac{ \tilde{\Gamma}(p_1, p_2)}{s  - m^2} + \frac{i \Sigma^{(old)}(t)}{(t - m^2)^2} + \frac{\tilde{\Gamma}(p_1', p_2')}{s - m^2} 
+  \frac{\tilde{\Gamma}(p_1, p_1')}{t - m^2} \right.
\\ 
& \left.
\quad \quad \quad \quad \quad \quad 
+ \frac{i \Sigma^{(old)}(s)}{(s - m^2)^2} + \frac{\tilde{\Gamma}(p_2, p_2')}{t - m^2} + i\mathcal{M}_{box}/(-ig)^2 + \left[\frac{i}{t - m^2 } + \frac{i}{s - m^2 }\right] \right|^2 
\end{align*} 
Since $Z = 1 + O(g^2)$ to the level of perturbation theory we are working, $\Sigma$, $\tilde{\Gamma}$, and $i\mathcal{M}_{box}/(-ig)^2$ which are all quantities of order $g_1^2$ are unchanged by the substitution $g_1 \to g$. The shift in the scattering amplitude due to changing these terms is order $g^6$ (since the entire amplitude is multiplied by a factor of $g^2$) which is above the order of our pertubation theory calculations. Therefore, substituting $g = Z_\psi Z_\phi^{1/2} g_1$,
\begin{align*}
\deriv{\sigma^{(old)}}{\Omega} &= \frac{g^4 Z_\phi^{-2}}{64 \pi^2 E_{CM}^2} \left|  \frac{ \tilde{\Gamma}(p_1, p_2)}{s  - m^2} + \frac{i \Sigma^{(old)}(t)}{(t - m^2)^2} + \frac{\tilde{\Gamma}(p_1', p_2')}{s - m^2} 
+  \frac{\tilde{\Gamma}(p_1, p_1')}{t - m^2} \right.
\\ 
& \left.
\quad \quad \quad \quad \quad \quad 
+ \frac{i \Sigma^{(old)}(s)}{(s - m^2)^2} + \frac{\tilde{\Gamma}(p_2, p_2')}{t - m^2} + i\mathcal{M}_{box}/(-ig)^2 + \left[\frac{i}{t - m^2 } + \frac{i}{s - m^2 }\right] \right|^2 
\end{align*}  
Now, I will apply the factor of $Z_\phi^{-2}$ to each term. However, since we are only working to order $g^4$, any term of order above $g$ will be unchanged by the multiplication of $Z_\phi = 1 + O(g^2)$. Therefore, only the tree-level terms are affected,
\begin{align*}
\deriv{\sigma^{(old)}}{\Omega} &= \frac{g^4 Z_\phi^{-2}}{64 \pi^2 E_{CM}^2} \left|  \frac{ \tilde{\Gamma}(p_1, p_2)}{s  - m^2} + \frac{i \Sigma^{(old)}(t)}{(t - m^2)^2} + \frac{\tilde{\Gamma}(p_1', p_2')}{s - m^2} 
+  \frac{\tilde{\Gamma}(p_1, p_1')}{t - m^2} \right.
\\ 
& \left.
\quad \quad \quad \quad \quad \quad 
+ \frac{i \Sigma^{(old)}(s)}{(s - m^2)^2} + \frac{\tilde{\Gamma}(p_2, p_2')}{t - m^2} + i\mathcal{M}_{box}/(-ig)^2 + Z_\phi^{-1} \left[\frac{i}{t - m^2 } + \frac{i}{s - m^2 }\right] \right|^2 
\end{align*}  
Using the formula for $Z_\phi^{-1} = 1 - \deriv{\Sigma^{(old)}}{p^2} \bigg|_{p^2 = m^2}$ we get,
\begin{align*}
\deriv{\sigma^{(old)}}{\Omega} &= \frac{g^4}{64 \pi^2 E_{CM}^2} \left|  \frac{ \tilde{\Gamma}(p_1, p_2)}{s  - m^2} + \frac{i \Sigma^{(old)}}{(t - m^2)^2} + \frac{\tilde{\Gamma}(p_1', p_2')}{s - m^2} 
+  \frac{\tilde{\Gamma}(p_1, p_1')}{t - m^2} \right.
\\ 
& \left.
\quad \quad \quad \quad \quad \quad 
+ \frac{i \Sigma^{(old)}(s)}{(s - m^2)^2} + \frac{\tilde{\Gamma}(p_2, p_2')}{t - m^2} + i\mathcal{M}_{box}/(-ig)^2 + \left( 1 - \deriv{\Sigma^{(old)}}{p^2} \right) \left[\frac{i}{t - m^2 } + \frac{i}{s - m^2 }\right] \right|^2  
\\
&= \frac{g^4}{64 \pi^2 E_{CM}^2} \left|  \frac{ \tilde{\Gamma}(p_1, p_2)}{s  - m^2} + \frac{i \Sigma^{(old)}(t) - i(t - m^2) \deriv{\Sigma^{(old)}}{p^2}}{(t - m^2)^2} + \frac{\tilde{\Gamma}(p_1', p_2')}{s - m^2} 
+  \frac{\tilde{\Gamma}(p_1, p_1')}{t - m^2} \right.
\\ 
& \left.
\quad \quad \quad \quad \quad \quad 
+ \frac{i \Sigma^{(old)}(s) - i(s - m^2) \deriv{\Sigma^{(old)}}{p^2} }{(s - m^2)^2} + \frac{\tilde{\Gamma}(p_2, p_2')}{t - m^2} + i\mathcal{M}_{box}/(-ig)^2 + \left[\frac{i}{t - m^2 } + \frac{i}{s - m^2 }\right] \right|^2
\\
&= \frac{g^4}{64 \pi^2 E_{CM}^2} \left|  \frac{ \tilde{\Gamma}(p_1, p_2)}{s  - m^2} + \frac{i \Sigma(t)}{(t - m^2)^2} + \frac{\tilde{\Gamma}(p_1', p_2')}{s - m^2} 
+  \frac{\tilde{\Gamma}(p_1, p_1')}{t - m^2} \right.
\\ 
& \left.
\quad \quad \quad \quad \quad \quad 
+ \frac{i \Sigma(s)}{(s - m^2)^2} + \frac{\tilde{\Gamma}(p_2, p_2')}{t - m^2} + i\mathcal{M}_{box}/(-ig)^2 + \left[\frac{i}{t - m^2 } + \frac{i}{s - m^2 }\right] \right|^2    
\end{align*}  
which is the scattering cross section in the form given by the fully renormalized theory. The fact that both theories predict the same physically measurable scattering cross section is comforting and checks that we performed our field strength renormalization prescription correctly. Now it is finally time to explicitly compute this cross section.
\end{document}