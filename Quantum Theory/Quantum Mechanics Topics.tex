\documentclass[12pt]{extarticle}
\usepackage[utf8]{inputenc}
\usepackage[english]{babel}
\usepackage[a4paper, total={6in, 9in}]{geometry}
\usepackage{tikz-cd}
 
\usepackage{amsthm, amssymb, amsmath, centernot}
\usepackage{mathrsfs} 

\newcommand{\notimplies}{%
  \mathrel{{\ooalign{\hidewidth$\not\phantom{=}$\hidewidth\cr$\implies$}}}}
 
\renewcommand\qedsymbol{$\square$}
\newcommand{\cont}{$\boxtimes$}
\newcommand{\divides}{\mid}
\newcommand{\ndivides}{\centernot \mid}
\newcommand{\Z}{\mathbb{Z}}
\newcommand{\R}{\mathbb{R}}
\newcommand{\N}{\mathbb{N}}
\newcommand{\Zplus}{\mathbb{Z}^{+}}
\newcommand{\Primes}{\mathbb{P}}
\newcommand{\colim}[1]{\mathrm{colim}(#1)}
\newcommand{\Ob}[1]{\mathrm{Ob}(#1)}
\newcommand{\cat}[1]{\mathcal{#1}}
\newcommand{\id}{\mathrm{id}}
\newcommand{\Hom}[2]{\mathrm{Hom}\left( #1, #2 \right)}
\newcommand{\catHom}[3]{\mathrm{Hom}_{#1}\left( #2, #3 \right)}
\newcommand{\End}[1]{\mathrm{End}\left(#1\right)}
\newcommand{\Top}{\mathbf{Top}}
\newcommand{\pTop}{\mathbf{Top}_{\bullet}}
\newcommand{\Set}{\mathbf{Set}}
\newcommand{\pSet}{\mathbf{Set}_\bullet}
\newcommand{\hTop}{\mathbf{hTop}}
\newcommand{\phTop}{\mathbf{hTop}_{\bullet}}
\renewcommand{\Im}[1]{\mathrm{Im}(#1)}
\newcommand{\homspace}[2]{\left< #1, #2 \right>}
\newcommand{\rp}{\mathbb{RP}}
\newcommand{\coker}[1]{\mathrm{coker}\: #1}

\renewcommand{\d}[1]{ \mathrm{d}#1 \:}
\newcommand{\dn}[2]{ \mathrm{d}^{#1} #2 \:}
\newcommand{\deriv}[2]{\frac{\d{#1}}{\d{#2}}}
\newcommand{\nderiv}[3]{\frac{\dn{#1}{#2}}{\d{#3^{#1}}}}
\newcommand{\pderiv}[2]{\frac{\partial{#1}}{\partial{#2}}}
\newcommand{\fderiv}[2]{\frac{\delta #1}{\delta #2}}

\theoremstyle{definition}
\newtheorem{theorem}{Theorem}[section]
\newtheorem{lemma}[theorem]{Lemma}
\newtheorem{proposition}[theorem]{Proposition}
\newtheorem{example}[theorem]{Example}
\newtheorem{corollary}[theorem]{Corollary}
\newtheorem{remark}{Remark}

\newenvironment{definition}[1][Definition:]{\begin{trivlist}
\item[\hskip \labelsep {\bfseries #1}]}{\end{trivlist}}


\newenvironment{lproof}{\begin{proof} \renewcommand{\qedsymbol}{}}{\end{proof}}
\renewcommand{\mod}[3]{\: #1 \equiv #2 \: mod \: #3 \:}
\newcommand{\nmod}[3]{\: #1 \centernot \equiv #2 \: mod \: #3 \:}
\newcommand{\ndiv}{\hspace{-4pt}\not \divides \hspace{2pt}}
\newcommand{\gen}[1]{\langle #1 \rangle}
\newcommand{\hook}{\hookrightarrow}
\newcommand{\Tor}[4]{\mathrm{Tor}^{#1}_{#2} \left( #3, #4 \right)}
\newcommand{\Ext}[4]{\mathrm{Ext}^{#1}_{#2} \left( #3, #4 \right)}

\tikzset{
    labl/.style={anchor=south, rotate=90, inner sep=.5mm}
}

\renewcommand{\bf}[1]{\mathbf{#1}}
\newcommand{\res}{\mathrm{res}}
\newcommand{\F}{\mathcal{F}}
\newcommand{\G}{\mathcal{G}}
\renewcommand{\O}{\mathcal{O}}
\newcommand{\m}{\mathfrak{m}}

\newcommand{\GL}[1]{\mathrm{GL}\left(#1\right)}
\newcommand{\SL}[1]{\mathrm{SL}\left(#1\right)}
\newcommand{\PGL}[1]{\mathrm{PGL}\left(#1\right)}
\newcommand{\PSL}[1]{\mathrm{PSL}\left(#1\right)}

\newcommand{\Orth}[1]{\mathrm{O}\left(#1\right)}
\newcommand{\U}[1]{\mathrm{U}\left(#1\right)}
\newcommand{\SO}[1]{\mathrm{SO}\left(#1\right)}
\newcommand{\SU}[1]{\mathrm{SU}\left(#1\right)}
\newcommand{\g}{\mathfrak{g}}
\newcommand{\h}{\mathfrak{h}}
\newcommand{\gl}[1]{\mathfrak{gl}\left(#1\right)}
\newcommand{\Lie}[1]{\mathrm{Lie}\left(#1 \right)}
\newcommand{\Aut}[1]{\mathrm{Aut}\left(#1 \right)}

\newcommand{\C}{\mathbb{C}}
\renewcommand{\H}{\hat{H}}
\newcommand{\Hil}{\mathcal{H}}
\renewcommand{\P}{\mathbb{P}}
\newcommand{\PAU}[1]{\mathrm{PAU}\left( #1 \right)}
\newcommand{\CP}{\mathbb{CP}}
\newcommand{\Cl}{\mathrm{C} \ell}

\newcommand{\inner}[2]{\left< #1 \middle| #2 \right>}
\newcommand{\ket}[1]{\left| #1 \right>}
\newcommand{\bra}[1]{\left< #1 \right|}
\newcommand{\sign}[1]{\mathrm{sign}\left( #1 \right)}

\begin{document}

\section{Misc}

\subsection{Feynman's Rotation Trick (III.19 Eq. 19.32 and Eq. 19.33)}

Feynman claims, based on his argument that the electron can only be found along an axis if it has zero angular momentum about that axis, that the angular dependence of the wavefunction is proportional to,
\[ \bra{\ell, 0} R_y(\theta) R_z(\phi) \ket{\ell, m} \]
in other words the spherical harmonics satisfy,
\[ Y^\ell_m(\theta, \phi) \propto \bra{\ell, 0} R_y(\theta) R_z(\phi) \ket{\ell, m} \]
meaning the entire wavefunction is,
\[ \psi_{\ell, m}(r, \theta, \phi) = \bra{\ell, 0} R_y(\theta) R_z(\phi) \ket{\ell, m} F_\ell(r) \]
where $F_\ell(r)$ is the radial profile of the electron in the $\ket{\ell, 0}$ state along the $z$-axis. Recall that Feynman's rotation operators $R$ are \textit{passitive rotation operators} meaning they rotate the coordinate system with respect to which measurements are performed rather than rotating the quantum system. As a consequence, $R_{\text{Fey}} = R_{\text{std}}^\dagger$. In this section, we prove this result.
\bigskip\\
Here I will use rotation operators in the usual sense so don't be confused when they show up with a dagger!
By definition,
\[ Y^\ell_m(\theta, \phi) = \inner{\theta, \phi}{\ell, m} \]
Using the definition of spherical coordinates, 
\[ \ket{\theta, \phi} = R_z(\phi) R_y(\theta) \ket{N} \]
where $\ket{N}$ is the north pole. Therefore,
\[ Y^\ell_m(\theta, \phi) = \bra{N} R_y(\theta)^\dagger R_z(\phi)^\dagger \ket{\ell, m} = \sum_{\ell', m'} \inner{N}{\ell',m'} \bra{\ell',m'} R_y(\theta)^\dagger R_z(\phi)^\dagger \ket{\ell, m} \]
However, by Feynman's argument $\inner{N}{\ell',m'} = a_\ell \delta_{m',0}$ which can be quickly checked. Therefore, 
\[ Y^\ell_m(\theta, \phi) = a_\ell \bra{\ell, 0} R_y(\theta)^\dagger R_z(\phi)^\dagger \ket{\ell, m} \]
proving the claim.

\section{Quantum Mechanincs In 1D}

\section{Quantum Erasers}

\section{Spherical Harmonics}

\subsection{Feynman's Trick}

Consider the Spherical Harmonic,
\[ Y^\ell_m(\theta, \phi) = \inner{\theta, \phi}{\ell, m} \]
Now denote the state with total angular momentum $\ell(\ell + 1)$ and component $m$ along the axis defined by $(\theta, \phi)$ as $\ket{\ell, m, \theta, \phi}$. That is,
\[ \ket{\ell, m, \theta, \phi} = R(\theta, \phi) \ket{\ell, m} \]
 These states with fixed $(\theta, \phi)$ and $\ell$ varying over $m$ form a complete set of states for the fixed $\ell$-multiplet subspace. thus,
\[ Y^\ell_m(\theta, \phi) = \inner{\theta, \phi}{\ell, m} = \sum_{m' = - \ell}^\ell \inner{\theta, \phi}{\ell, m', \theta, \phi} \inner{\ell, m', \theta, \phi}{\ell, m} \]
However, it is easily shown that,
\[ \inner{\theta, \phi}{\ell, m', \theta, \phi}  = \bra{+z} R(\theta, \phi)^\dagger R(\theta, \phi) \ket{\ell, m'} = \inner{+z}{\ell, m'} = Y^\ell_{m'}(0,0) \]
is zero unless $m = 0$. Thus we find,
\[ Y^{\ell}_m(\theta, \phi) = Y^\ell_{0}(0,0) \inner{\ell, 0, \theta, \phi}{\ell, m} = Y^\ell_{0}(0,0) \bra{\ell, 0} R(\theta, \phi)^\dagger \ket{\ell, m} \]
which agrees with Feynman's formula. 

\section{Multi-Electron Atoms}

\subsection{The Born-Oppenheimer Approximation}

\subsection{Exchange Energies}

First consider two electrons in a central nuclear potential with Hamiltonian,
\[ \H = \H_1 + \H_2 + \frac{e^2}{|r_1 - r_2|} = \frac{\hat{p}_1^2}{2 m_e} + \frac{\hat{p}_2^2}{2 m_e} - \frac{Z e^2}{r_1} - \frac{Z e^2}{r_2} + \frac{e^2}{|r_1 - r_2|} \]
where $\H_1 = \H_{e} \otimes I_2$ and $\H_2 = I_1 \otimes \H_e$ and,
\[ \H_e = \frac{\hat{p}^2}{2 m_e} - \frac{Z e^2}{r} \]
Consider two electrons in states $\ket{\psi}$ and $\ket{\phi}$ given by the slater determinant,
\[ \ket{\Psi} = \frac{1}{\sqrt{2}} \left[ \ket{\psi} \otimes \ket{\phi} - \ket{\phi} \otimes \ket{\psi} \right] \] 
Note we may require $\inner{\psi}{\psi} = \inner{\phi}{\phi} =1$ and $\inner{\psi}{\phi} = 0$ without loss of generality since any parallel term is removed in the antisymmetrization. Now consider,
\begin{align*}
\bra{\Psi} \H \ket{\Psi} & = \frac{1}{2} \left[ \bra{\psi} \H_e \ket{\psi} + \bra{\phi} \H_e \ket{\phi} + \bra{\psi} \otimes \bra{\phi} \frac{e^2}{|r_1 - r_2|} \ket{\psi} \otimes \ket{\phi} \right]
\\
& \: - \frac{1}{2} \left[ \bra{\psi} \H_e \ket{\phi} \inner{\phi}{\psi} + \inner{\psi}{\phi} \bra{\phi} \H_e \ket{\psi}  + \bra{\psi} \otimes \bra{\phi} \frac{e^2}{|r_1 - r_2|} \ket{\phi} \otimes \ket{\psi} \right]
\\
& \: - \frac{1}{2} \left[ \bra{\phi} \H_e \ket{\psi} \inner{\psi}{\phi} + \inner{\phi}{\psi} \bra{\psi} \H_e \ket{\phi} + \bra{\phi} \otimes \bra{\psi} \frac{e^2}{|r_1 - r_2|} \ket{\psi} \otimes \ket{\phi} \right]
\\
& \: + \frac{1}{2} \left[ \bra{\phi} \H_e \ket{\phi} + \bra{\psi} \H_e \ket{\psi} + \bra{\phi} \otimes \bra{\psi} \frac{e^2}{|r_1 - r_2|} \ket{\phi} \otimes \ket{\psi} \right]
\end{align*}
We may combine these matrix ellements to find,
\begin{align*}
\bra{\Psi} \H \ket{\Psi} & = \bra{\psi} \H_e \ket{\psi} + \bra{\phi} \H_e \ket{\phi}
 + \bra{\psi} \otimes \bra{\phi} \frac{e^2}{|r_1 - r_2|} \ket{\psi} \otimes \ket{\phi} 
\\
& - \bra{\phi} \otimes \bra{\phi} \frac{e^2}{|r_1 - r_2|} \ket{\phi} \otimes \ket{\phi}
\end{align*}
This is the energy we expect for the first electron in state $\ket{\psi}$ and the second in state $\ket{\phi}$ except for the last term, the exchange energy,
\begin{align*}
E_{\text{ex}} & = - \bra{\phi} \otimes \bra{\phi} \frac{e^2}{|r_1 - r_2|} \ket{\phi} \otimes \ket{\phi} = - \int \dn{3}{x} \dn{3}{y} \phi^*(x) \psi^*(y) \frac{e^2}{|\vec{x} - \vec{y}|} \psi(x) \phi(y)
\\
& -  \int \dn{3}{x} \dn{3}{y} \frac{e^2}{|\vec{x} - \vec{y}|} \psi(x) \phi^*(x) \psi^*(y) \phi(y)
\end{align*}

\subsection{Hartee-Fock Theory}

The wavefunction of an arbitrary $n$-electron state must be totally antisymmetric. Given an orthonormal set of $n$ single-electron states the closest mult-electron state which satisfies the proper fermionic condition is given by a Slater determinant,
\[ \Psi(x_1, \cdots, x_n) = \frac{1}{\sqrt{n!}} \begin{vmatrix}
\psi_1(x_1) & \psi_1 (x_2) & \cdots & \psi_1(x_n)
\\
\psi_2(x_1) & \psi_2 (x_2) & \cdots & \psi_2(x_n)
\\
\vdots & \vdots & \ddots & \vdots
\\
\psi_n(x_1) & \psi_n (x_2) & \cdots & \psi_n(x_n)
\end{vmatrix} \]
Where $\psi_i$ is the $i$-th spin-orbital i.e. the variable $x_i$ lives in the product space of position and spin degrees of freedom. Explicity, $x = (\vec{r}, s)$ and $\psi_i(x) = (\bra{\vec{r}} \otimes \bra{s}) \ket{\psi_i}$. Recall that this multi-electron state is only properly normalized if we require that,
\[ \inner{\psi_i}{\psi_j} = \delta_{ij} \]
Our first task is to compute the energy of such a Slater determinant. Recall that, under the Born-Oppenhiemer approximation, the multi-electron Hamiltonian is, 
\[ \H = \sum_{i = 1}^n \left[ \frac{\vec{p}^{\, 2}_i}{2 m_e} - \sum_{j = 1}^N \frac{Z_j e^2}{|\vec{r}_i - \vec{R}_j|} \right] + \frac{1}{2} \sum_{i \neq j} \frac{e^2}{|\vec{r}_i - \vec{r}_j|} = \sum_{i = 1}^n \H_S^{(i)} + \frac{1}{2} \sum_{i \neq j} \H_R^{(ij)} \]
where $\H_S$, the single particle Hamiltonian acts on the product space as a product, and $\H_R$ only mixes pairs of spaces. Thus we have,
\begin{align*}
E_\Psi & =  \bra{\Psi} \H \ket{\Psi} = \sum_{i = 1}^n \bra{\Psi} \H_S^{(i)} \ket{\Psi} + \frac{1}{2} \sum_{i \neq j} \bra{\Psi} \H_R^{(ij)} \ket{\Psi}
\\
& = \frac{1}{n!} \sum_{\sigma, \tau \in S_n} \sum_{i = 1}^n (-1)^{\sign{\sigma} + \sign{\tau}} \bra{\psi_{\sigma(1)}} \otimes \cdots \otimes \bra{\psi_{\sigma(n)}} \H_S^{(i)} \ket{\psi_{\tau(1)}} \otimes \cdots \otimes \ket{\psi_{\tau(n)}}
\\
& + \frac{1}{2 n!} \sum_{\sigma, \tau \in S_n} \sum_{i \neq j} (-1)^{\sign{\sigma} + \sign{\tau}} \bra{\psi_{\sigma(1)}} \otimes \cdots \otimes \bra{\psi_{\sigma(n)}} \H_R^{(ij)} \ket{\psi_{\tau(1)}} \otimes \cdots \otimes \ket{\psi_{\tau(n)}}
\end{align*}
Now if $\sigma \neq \tau$ then there are at least two values which differ between them (since $\sigma$ and $\tau$ are permuations). Therefore, the first inner product is zero unless $\sigma = \tau$. Thus, the single particle Hamiltonian gives a contribution,
\[ \frac{1}{n!} \sum_{\sigma \in S_n} \sum_{i = 1}^n \bra{\psi_{\sigma(1)}} \otimes \cdots \otimes \bra{\psi_{\sigma(n)}} \H_S^{(i)} \ket{\psi_{\sigma(1)}} \otimes \cdots \otimes \ket{\psi_{\sigma(n)}} = \frac{1}{n!}  \sum_{\sigma \in S_n} \sum_{i = 1}^n h_{\sigma(i)} = \sum_{i = 1}^n h_i \]
where
\[ h_i = \bra{\psi_i} \H_S \ket{\psi_i} \]
since the sum over such values is invariant under permutation. Now consider the two-electron repulsion terms,
\begin{align*}
\frac{1}{n!} \sum_{\sigma, \tau \in S_n} \sum_{i \neq j} (-1)^{\sign{\sigma} + \sign{\tau}} \bra{\psi_{\sigma(1)}} \otimes \cdots \otimes \bra{\psi_{\sigma(n)}} \H_R^{(ij)} \ket{\psi_{\tau(1)}} \otimes \cdots \otimes \ket{\psi_{\tau(n)}}
\end{align*}
Since all but two of the single-particle states must correspond for the inner product to be nonzero, the only non-equal option is that $\sigma$ and $\tau$ differ by a single transposition, that is, $\sigma = \tau \circ (a \: b)$. Thus, the repulsion terms become,
\begin{align*}
& \frac{1}{n!} \sum_{\sigma \in S_n} \sum_{i \neq j}  \bra{\psi_{\sigma(1)}} \otimes \cdots \otimes \bra{\psi_{\sigma(n)}} \H_R^{(ij)} \ket{\psi_{\sigma(1)}} \otimes \cdots \otimes \ket{\psi_{\sigma(n)}}
\\
& -
\frac{1}{n!} \sum_{\sigma \in S_n} \sum_{a < b} \sum_{i \neq j} \bra{\psi_{\sigma(1)}} \otimes \cdots \otimes \bra{\psi_{\sigma(n)}} \H_R^{(ij)} \ket{\psi_{\sigma(1)}} \otimes \cdots \otimes \ket{\psi_{\sigma(n)}}
\\
& \frac{1}{n!} \sum_{\sigma \in S_n} \sum_{i \neq j}  J_{\sigma(i) \sigma(j)} - \frac{1}{n!} \sum_{\sigma \in S_n} \sum_{i \neq j} K_{\sigma(i) \sigma(j)}
\\
& = \sum_{i \neq j} [ J_{ij} - K_{ij} ]
\end{align*} 
where,
\begin{align*}
J_{ij} & = \bra{\psi_i} \otimes \bra{\psi_j} H_R \ket{\psi_i} \otimes \ket{\psi_j} = \int \dn{3}{x_1} \dn{3}{x_2} \psi_i^*(x_1) \psi_j^*(x_2) \frac{e^2}{|\vec{r}_1 - \vec{r}_2|} \psi_i(x_1) \psi_j(x_2)
\\
K_{ij} & = \bra{\psi_i} \otimes \bra{\psi_j} H_R \ket{\psi_j} \otimes \ket{\psi_i} = \int \dn{3}{x_1} \dn{3}{x_2} \psi_i^*(x_1) \psi_j^*(x_2) \frac{e^2}{|\vec{r}_1 - \vec{r}_2|} \psi_j(x_1) \psi_i(x_2)
\end{align*}
Note that if $\ket{\psi_i}$ and $\ket{\psi_j}$ are product states in the spin degrees of freedom then $K_{ij}$ is zero for opposite spin states. Thus, finally, the energy becomes,
\[ E_\Psi =  \bra{\Psi} \H \ket{\Psi} = \sum_{i = 1}^n h_i + \frac{1}{2} \sum_{i,j = 1}^n [J_{ij} - K_{ij}] \]
Note that we may include the $i = j$ cross terms noting that $K_{ii} = J_{ii}$. 
Now we use the variational principal with respect to the states $\ket{\psi_1}$ subject to the constraint,
\[ \inner{\psi_1}{\psi_j} = \delta_{ij} \]
Therefore we introduce Largange multipliers and set the variation to zero,
\[ \frac{\delta}{\delta \psi_i^*(x)} \left[ \bra{\Psi} \H \ket{\Psi} - \sum_{i,j = 1}^n (\inner{\psi_i}{\psi_j} - \delta_{ij} ) \varepsilon_{ij} \right]  = 0 \]
Therefore, 
\[ \left( \H_S + \sum_{j = 1}^n [\hat{J}_j - \hat{K}_j] \right) \psi_i(x) = \sum_{j = 1}^n \varepsilon_{ij} \psi_j(x) \]
where,
\begin{align*}
\hat{J}_j \psi(x) & = \left[ \int \dn{3}{x_j} \psi_j^*(x_j) \frac{e^2}{|\vec{r} - \vec{r}_j |} \psi_j(x_j) \right] \psi(x)
\\
\hat{K}_j \psi(x) & = \left[ \int \dn{3}{x_j} \psi_j^*(x_j) \frac{e^2}{|\vec{r} - \vec{r}_j |} \psi(x_j) \right] \psi_j(x)
\end{align*}
Thus we have derived the Fock operator,
\[ \hat{F} = \H_S + \sum_{j = 1}^n [ \hat{J}_j - \hat{K}_j] \]
It remains to solve this pseudo-eigenvalue problem with the nonlinear operators. First, let $U_{ij}$ be an $n \times n$ special unitary matrix. Consider the transformation,
\[ \ket{\psi'_i} = \sum_{j = 1}^n U_{ij} \ket{\psi_j} \]
which preserves the inner product,
\[ \inner{\psi'_i}{\psi'_j} = \sum_{k,\ell = 1}^n \inner{\psi_k}{\psi_\ell} U_{ik}^* U_{j \ell} = \sum_{k,\ell = 1}^n \delta_{k \ell} U_{ik}^* U_{j \ell} = \sum_{k = 1}^n \inner{\psi_k}{\psi_\ell} U_{jk} (U^\dagger)_{ki} = \delta_{ij} \]
Now consider the matrix,
\begin{align*}
\begin{pmatrix}
\psi_1'(x_1) & \psi_1'(x_2) & \cdots & \psi_1'(x_n)
\\
\psi_2'(x_1) & \psi_2'(x_2) & \cdots & \psi_2'(x_n)
\\
\vdots & \vdots & \ddots & \vdots
\\
\psi_n'(x_1) & \psi_n'(x_2) & \cdots & \psi_n'(x_n)
\end{pmatrix}
= 
\begin{pmatrix}
U_{11} & U_{12} & \cdots & U_{1n}
\\
U_{21} & U_{22} & \cdots & U_{2n}
\\
\vdots & \vdots & \ddots & \vdots
\\
U_{n1} & U_{n2} & \cdots & U_{nn}
\end{pmatrix}
\begin{pmatrix}
\psi_1(x_1) & \psi_1 (x_2) & \cdots & \psi_1(x_n)
\\
\psi_2(x_1) & \psi_2 (x_2) & \cdots & \psi_2(x_n)
\\
\vdots & \vdots & \ddots & \vdots
\\
\psi_n(x_1) & \psi_n (x_2) & \cdots & \psi_n(x_n)
\end{pmatrix}
\end{align*}
Since $\det{U} = 1$ thus we have shown that the determinant of this matrix remains unchanged and thus $\ket{\Psi'} = \ket{\Psi}$. Therefore, transforming a solution set of single-electron wave functions via $U$ will not change of the physical wavefunction and thus still represent an energy minimum. Since $\varepsilon_{ij}$ is a symmetric matrix, it can be diagonalized via a special unitary matrix $U$. Therefore, we can apply such a transformation to make the representing single-electron wavefunction eigenvectors of the matrix $\epsilon_{ij}$ to reduce the Fock equation to the eigenvalue form,
\[ \hat{F} \psi_i = \varepsilon_{i} \psi_i \] 

\section{Time Independent Perturbation Theory}

\subsection{Non-Degenerate Perturbation Theory}

\subsection{First-Order}

\subsubsection{Second-Order Non-Degenerate Perturbation Theory}

\subsubsection{Higher-Order Non-Degenerate Perturbation Theory}

\section{Degenerate Perturbation Theory}

\subsection{First-Order Degenerate Perturbation Theory}

\subsubsection{Second-Order Degenerate Perturbation Theory}


\section{Symmetries and Conservation}

\section{Wigner's Theorem}

\subsection{Quantum Dynamics}

\subsubsection{Unitary Time Evolution}

\subsubsection{The Heisenberg Picture}

\subsubsection{The Interaction Picture}

\subsection{The Adjoint Representation}

(DEGENERACY) (WINGER THEOREM)

\section{Representation Theory}

\subsection{Tensor Operators}

\subsection{The Wigner-Eckart Theorem}

\section{Angular Momentum}

\section{The Radial Equation}



\section{The Harmonic Oscillator}

\subsection{The Explicit Series Solution}

\subsection{The Operator Method}

\subsection{Coherent States}

\subsection{Squeezed States}

\subsection{Multidimensional Harmoic Oscilators}

(SYMMETRY SU(N))

\subsection{Coupled Oscillators}




\section{The Hydrogen Atom}

\subsection{The Non-Relativistic Solution}

\subsubsection{The Explicit Series Solution}

\subsubsection{The Operator Method}

\subsection{Hidden Symmetry of the Coulomb Problem}

(RUNGE LENZ)

\subsection{Relativistic Corrections}

\subsubsection{Relativistic Momentum}

\subsubsection{Spin-Orbit Coupling}

\subsubsection{The Darwin Term}

\subsubsection{The First-Order Relativistic Correction}

\subsection{The Hyperfine Structure}

\subsubsection{Nuclear Spin and Applied Magnetic Fields}

\subsection{Atoms in Applied Fields}

\subsubsection{The Stark Effect}

\subsubsection{The (Anamalous) Zeeman Effect}

\subsubsection{The Lande $g$ Factor}



\section{Two State Systems}

\subsection{Time-Independent Solution}

\subsection{The Amonia Mazer}

\subsection{Nuclear Magnetic Resonance}


\section{Time-Dependent Perturbation Theory}

\subsection{The General Case}

\subsubsection{First-Order Time-Dependent Perturbation Theory}

\subsubsection{Higher-Order Time-Dependent Perturbation Theory}

\subsection{Harmonic Perturbations}

\subsubsection{First-Order Time-Dependent Perturbation Theory}

\subsubsection{Second-Order Time-Dependent Perturbation Theory}

\subsubsection{The Exact Result}

\section{The Atomic Theory of Radiation}

\subsection{Stimulated Emission and Absoption}

\subsection{Spontaneous Emission}

\section{The Dirac Equation}

\subsection{Introduction}

\subsection{Spinor Transformations}

\subsection{The Non-Relativistic Limit}

\subsubsection{First Relativistic Corrections}

\subsection{Solution for the Hydrogen Atom}

\section{The Dirac Field}


\end{document}


