\documentclass[12pt]{article}
\usepackage[english]{babel}
\usepackage[utf8]{inputenc}
\usepackage[english]{babel}
\usepackage[a4paper, total={7.25in, 9.5in}]{geometry}
\usepackage{tikz-feynman}
\tikzfeynmanset{compat=1.0.0} 
\usepackage{subcaption}
\usepackage{float}
\floatplacement{figure}{H}
\usepackage{simpler-wick}
\usepackage{mathrsfs}  
\usepackage{dsfont}
\usepackage{relsize}
\usepackage{tikz-cd}
\DeclareMathAlphabet{\mathdutchcal}{U}{dutchcal}{m}{n}

\usepackage{cancel}



\newcommand{\field}{\hat{\Phi}}
\newcommand{\dfield}{\hat{\Phi}^\dagger}
 
\usepackage{amsthm, amssymb, amsmath, centernot}
\usepackage{slashed}
\newcommand{\notimplies}{%
  \mathrel{{\ooalign{\hidewidth$\not\phantom{=}$\hidewidth\cr$\implies$}}}}
 
\renewcommand\qedsymbol{$\square$}
\newcommand{\cont}{$\boxtimes$}
\newcommand{\divides}{\mid}
\newcommand{\ndivides}{\centernot \mid}

\newcommand{\Integers}{\mathbb{Z}}
\newcommand{\Natural}{\mathbb{N}}
\newcommand{\Complex}{\mathbb{C}}
\newcommand{\Zplus}{\mathbb{Z}^{+}}
\newcommand{\Primes}{\mathbb{P}}
\newcommand{\Q}{\mathbb{Q}}
\newcommand{\R}{\mathbb{R}}
\newcommand{\ball}[2]{B_{#1} \! \left(#2 \right)}
\newcommand{\Rplus}{\mathbb{R}^+}
\renewcommand{\Re}[1]{\mathrm{Re}\left[ #1 \right]}
\renewcommand{\Im}[1]{\mathrm{Im}\left[ #1 \right]}
\newcommand{\Op}{\mathcal{O}}

\newcommand{\invI}[2]{#1^{-1} \left( #2 \right)}
\newcommand{\End}[1]{\text{End}\left( A \right)}
\newcommand{\legsym}[2]{\left(\frac{#1}{#2} \right)}
\renewcommand{\mod}[3]{\: #1 \equiv #2 \: \mathrm{mod} \: #3 \:}
\newcommand{\nmod}[3]{\: #1 \centernot \equiv #2 \: mod \: #3 \:}
\newcommand{\ndiv}{\hspace{-4pt}\not \divides \hspace{2pt}}
\newcommand{\finfield}[1]{\mathbb{F}_{#1}}
\newcommand{\finunits}[1]{\mathbb{F}_{#1}^{\times}}
\newcommand{\ord}[1]{\mathrm{ord}\! \left(#1 \right)}
\newcommand{\quadfield}[1]{\Q \small(\sqrt{#1} \small)}
\newcommand{\vspan}[1]{\mathrm{span}\! \left\{#1 \right\}}
\newcommand{\galgroup}[1]{Gal \small(#1 \small)}
\newcommand{\bra}[1]{\left| #1 \right>}
\newcommand{\Oa}{O_\alpha}
\newcommand{\Od}{O_\alpha^{\dagger}}
\newcommand{\Oap}{O_{\alpha '}}
\newcommand{\Odp}{O_{\alpha '}^{\dagger}}
\newcommand{\im}[1]{\mathrm{im} \: #1}
\renewcommand{\ker}[1]{\mathrm{ker} \: #1}
\newcommand{\ket}[1]{\left| #1 \right>}
\renewcommand{\bra}[1]{\left< #1 \right|}
\newcommand{\inner}[2]{\left< #1 | #2 \right>}
\newcommand{\expect}[2]{\left< #1 \right| #2 \left| #1 \right>}
\renewcommand{\d}[1]{ \mathrm{d}#1 \:}
\newcommand{\dn}[2]{ \mathrm{d}^{#1} #2 \:}
\newcommand{\deriv}[2]{\frac{\d{#1}}{\d{#2}}}
\newcommand{\nderiv}[3]{\frac{\dn{#1}{#2}}{\d{#3^{#1}}}}
\newcommand{\pderiv}[2]{\frac{\partial{#1}}{\partial{#2}}}
\newcommand{\fderiv}[2]{\frac{\delta #1}{\delta #2}}
\newcommand{\parsq}[2]{\frac{\partial^2{#1}}{\partial{#2}^2}}
\newcommand{\topo}{\mathcal{T}}
\newcommand{\base}{\mathcal{B}}
\renewcommand{\bf}[1]{\mathbf{#1}}
\renewcommand{\a}{\hat{a}}
\newcommand{\adag}{\hat{a}^\dagger}
\renewcommand{\b}{\hat{b}}
\newcommand{\bdag}{\hat{b}^\dagger}
\renewcommand{\c}{\hat{c}}
\newcommand{\cdag}{\hat{c}^\dagger}
\newcommand{\hamilt}{\hat{H}}
\renewcommand{\L}{\hat{L}}
\newcommand{\Lz}{\hat{L}_z}
\newcommand{\Lsquared}{\hat{L}^2}
\renewcommand{\S}{\hat{S}}
\renewcommand{\empty}{\varnothing}
\newcommand{\J}{\hat{J}}
\newcommand{\lagrange}{\mathcal{L}}
\newcommand{\dfourx}{\mathrm{d}^4x}
\newcommand{\meson}{\phi}
\newcommand{\dpsi}{\psi^\dagger}
\newcommand{\ipic}{\mathrm{int}}
\newcommand{\tr}[1]{\mathrm{tr} \left( #1 \right)}
\newcommand{\C}{\mathbb{C}}
\newcommand{\CP}[1]{\mathbb{CP}^{#1}}
\newcommand{\Vol}[1]{\mathrm{Vol}\left(#1\right)}

\newcommand{\Tr}[1]{\mathrm{Tr}\left( #1 \right)}
\newcommand{\Charge}{\hat{\mathbf{C}}}
\newcommand{\Parity}{\hat{\mathbf{P}}}
\newcommand{\Time}{\hat{\mathbf{T}}}
\newcommand{\Torder}[1]{\mathbf{T}\left[ #1 \right]}
\newcommand{\Norder}[1]{\mathbf{N}\left[ #1 \right]}
\newcommand{\Znorm}{\mathcal{Z}}
\newcommand{\EV}[1]{\left< #1 \right>}
\newcommand{\interact}{\mathrm{int}}
\newcommand{\covD}{\mathcal{D}}
\newcommand{\conj}[1]{\overline{#1}}

\newcommand{\SO}[2]{\mathrm{SO}(#1, #2)}
\newcommand{\SU}[2]{\mathrm{SU}(#1, #2)}

\newcommand{\anticom}[2]{\left\{ #1 , #2 \right\}}


\newcommand{\pathd}[1]{\! \mathdutchcal{D} #1 \:}

\renewcommand{\theenumi}{(\alph{enumi})}


\renewcommand{\theenumi}{(\alph{enumi})}

\newcommand{\atitle}[1]{\title{% 
	\large \textbf{Physics GR8048 Quantum Field Theory II
	\\ Assignment \# #1} \vspace{-2ex}}
\author{Benjamin Church }
\maketitle}

\newcommand{\atitleIII}[1]{\title{% 
	\large \textbf{Physics GR8049 Quantum Field Theory III
	\\ Assignment \# #1} \vspace{-2ex}}
\author{Benjamin Church }
\maketitle}

\theoremstyle{definition}
\newtheorem{theorem}{Theorem}[section]
\newtheorem{definition}{definition}[section]
\newtheorem{lemma}[theorem]{Lemma}
\newtheorem{proposition}[theorem]{Proposition}
\newtheorem{corollary}[theorem]{Corollary}
\newtheorem{example}[theorem]{Example}
\newtheorem{remark}[theorem]{Remark}

\begin{document}

\atitle{3}

\section{Renormalization of the Quantum Effective Potential}

The quantum effective potential $V_q$ of a $\phi^4$-theory with tree-level potential $V_0(\phi) = \lambda + \tfrac{1}{2} m^2 \phi^2 + \tfrac{1}{24} g \phi^4$ takes the form $V_q(\phi) = V_0(\phi) + \hbar V_1(\phi) + O(\hbar^2)$ where,
\[ V_1(\phi) = \lim_{\Lambda \to \infty} \left[ c_{\lambda, \Lambda} + \tfrac{1}{2} c_{m,\Lambda} \phi^2 + \tfrac{1}{24} c_{g, \Lambda} \phi^4 + \frac{1}{2} \int_{k < \Lambda} \frac{\dn{4}{k_E}}{(2\pi)^4} \log{\left( \frac{k_E^2 + \tilde{m}^2}{\Lambda_0^2} \right)} \right] \]
where
\[ \tilde{m}^2 = m^2 + \tfrac{1}{2} g \phi^2 \]
These counterterms must be determined by renormalization to fix the renormalized couplings $\lambda, m, g$. 

\subsection{Comparison of Regularization Schemes}

\subsubsection*{(a)}

First we regularize the integral by imposing a hard cutoff $k < \Lambda$ on the $4$-dimensional integral. Using Mathematica, we find,
\begin{align*}
I_a = \int_{k < \Lambda} \frac{\dn{4}{k_E}}{(2\pi)^4} \log{\left( \frac{k_E^2 + \tilde{m}^2}{\Lambda_0^2} \right)} & = \frac{\Omega_3}{(2 \pi)^4} \int_{0}^{\Lambda} {k_E^3 \d{k_E}} \log{\left( \frac{k_E^2 + \tilde{m}^2}{\Lambda_0^2} \right)}
\\
& = \frac{\Omega_3}{(2 \pi)^4} \frac{1}{8} \left[ 2 \tilde{m}^2 \Lambda^2 - \Lambda^4 + 2 \tilde{m}^4 \log{\left( \frac{\tilde{m}^2}{\Lambda^2 + \tilde{m}^2} \right)} + 2 \Lambda^4 \log{\left( \frac{ \Lambda^2 + \tilde{m}^2 }{\Lambda_0^2} \right)} \right] 
\end{align*}
Recall that $\Omega_3 = 2 \pi^2$ and expand this integral in the large $\Lambda \to \infty$ limit,
\[ I_a = \frac{1}{64 \pi^2} \left[ \left( 2 \tilde{m}^2 \Lambda^2 - \Lambda^4 \right) + 2 \Lambda^4 \log{ \left( \frac{\Lambda^2}{\Lambda_0^2} \right) } + 2 \tilde{m}^4 \log{\left( \frac{\tilde{m}^2}{\Lambda^2} \right)} \right] \]
Which I can rewrite as,
\[ I_a = \frac{1}{64 \pi^2} \left[ \left( 2 \tilde{m}^2 \Lambda^2 - \Lambda^4 + 2 \tilde{m}^4 \log{\left( \frac{\Lambda_0^2}{\Lambda^2} \right)}  + 2 \Lambda^4 \log{ \left( \frac{\Lambda^2}{\Lambda_0^2} \right) } \right) + 2 \tilde{m}^4 \log{\left( \frac{\tilde{m}^2}{\Lambda_0^2} \right)} \right] \]
Now consider the terms in parentheses. Notice the the first two terms $2 \tilde{m}^2 \Lambda^2 - \Lambda^4$ only depend quadratically on $\phi$ since $\tilde{m}^2 = m^2 + \tfrac{1}{2} g \phi^2$. Thus, these two terms can be absorbed into the first two counter-terms for $\lambda$ and $m$. Since the third term is a constant depending only on $\Lambda$, it can be absorbed into the counter-term $c_{\lambda, \Lambda}$. Furthermore, the final term has $\phi$ dependence only through the factor,
\[ \tilde{m}^4 = \left( m^2 + \tfrac{1}{2} g \phi^2 \right)^2 \]
which has linear, quadratic, and quartic terms only. Therefore, the second terms can be absorbed into the $\lambda, m, g$ counter-terms.
Thus, the entire term in parentheses is a quadratic polynomial in $\phi$ which can be absorbed into the counter-terms. 

\subsubsection*{(b)}

Consider the identity,
\[ \log{(k_E^2 + \tilde{m}^2)} = - \lim_{\Lambda \to 0} \left[ \int_{\Lambda^{-2}}^\infty \frac{\d{t}}{t} e^{-(k_E^2 + \tilde{m}^2) t} + c_{\Lambda} \right] \]
Therefore, up to a $\Lambda$-dependent but $\phi$-independent constant,
\[ I_b = \int \frac{\dn{4}{k_E}}{(2\pi)^4} \log{\left( \frac{k_E^2 + \tilde{m}^2}{\Lambda_0^2} \right)} = - \int_{\Lambda^{-2}}^{\infty} \frac{\d{t}}{t} \int \frac{\dn{4}{k_E}}{(2\pi)^4} e^{-(k_E^2 + \tilde{m}^2)t} = - \int_{\Lambda^{-2}}^{\infty} \frac{\d{t}}{t} \frac{1}{(4 \pi t)^2} e^{- t \tilde{m}^2} \]
From Mathematica we get,
\begin{align*}
I_b & = \frac{1}{32 \pi^2} \left[ 2 \tilde{m}^2 \Lambda^2 + (\gamma - \tfrac{3}{2}) \tilde{m}^4 + \tilde{m}^4 \log{\left( \frac{\tilde{m}^2}{\Lambda^2} \right)}   \right] 
\\
& = \frac{1}{32 \pi^2} \left[ \left( 2 \tilde{m}^2 \Lambda^2 + (\gamma - \tfrac{3}{2}) \tilde{m}^4 + \tilde{m}^4 \log{\left( \frac{\Lambda_0^2}{\Lambda^2} \right)} \right) + \tilde{m}^4 \log{\left( \frac{\tilde{m}^2}{\Lambda_0^2} \right)}  \right] 
\end{align*}
However, the term in parentheses is a quadratic polynomial in $\phi$ which can be absorbed into the counter-terms leaving an identical $\Lambda$-independent term to part (a). 

\subsubsection*{(c)}

Next, we consider dimensional regularization of the integral
\begin{align*}
\int \frac{\dn{4}{k_E}}{(2\pi)^4} \log{\left( \frac{k_E^2 + \tilde{m}^2}{\Lambda_0^2} \right)} = - \int \frac{\dn{4}{k_E}}{(2\pi)^4} \int_0^{\infty} \frac{\d{t}}{t} e^{-(k_E^2 + \tilde{m}^2)t}  
\end{align*}
up to a $\Lambda$-dependent constant. This gives,
\begin{align*}
I_c & = - \mu^{4-d} \int \frac{\dn{d}{k_E}}{(2\pi)^d} \int_0^{\infty} \frac{\d{t}}{t} e^{-(k_E^2 + \tilde{m}^2)t}  = - \int_0^{\infty} \frac{\d{t}}{t} \frac{\mu^{4 - d} \Omega_d}{(2 \pi)^d} \int_0^{\infty} k_E^{d-1} \d{k_E} e^{-(k_E^2 + \tilde{m}^2)t} 
\end{align*}
However,
\[ \int_0^{\infty} k_E^{d-1} \d{k_E} e^{-k_E^2 t} = \frac{\Gamma{\left( \tfrac{d}{2} \right)}}{2 t^{\frac{d}{2}}}  \]
and furthermore,
\[ \Omega_d = \frac{2 \pi^{\frac{d}{2}}}{\Gamma\left( \tfrac{d}{2} \right)} \]
Therefore,
\[ I_c = - \int_0^\infty \frac{\d{t}}{t} \frac{\mu^{4 - d}}{(4 \pi t)^{\frac{d}{2}}} e^{-\tilde{m}^2 t} \]
Using Mathematica, we can integrate this expression in the domain $d < 0$ and analytically continue to get,
\[ I_c = - \mu^{4 - d} \left( \frac{\tilde{m}^2}{4 \pi} \right)^{\frac{d}{2}} \Gamma{\left(-\tfrac{d}{2} \right)} \]
Now we expand $d = 4 - \delta$ in the limit $\delta \to 0$ to get,
\begin{align*}
I_c = - \mu^{\delta} \left( \frac{\tilde{m}^2}{4 \pi} \right)^{2 - \tfrac{1}{2} \delta} \Gamma{\left(\tfrac{1}{2} \delta - 2 \right)} 
=
 - \left( \frac{\tilde{m}^2}{4 \pi} \right)^2 \left( \frac{\tilde{m}^2}{4 \pi \mu^2} \right)^{-\tfrac{\delta}{2}} \Gamma{\left(\tfrac{1}{2} \delta - 2 \right)}
\end{align*}
Expanding the Gamma function we find,
\[ \Gamma{\left(\tfrac{1}{2} \delta - 2 \right)} = \frac{1}{\delta} + \left( \frac{3}{4} - \frac{\gamma}{2} \right) + O(d) \]
Thus, to order $\delta$ we have,  
\begin{align*}
\left( \frac{\tilde{m}^2}{4 \pi \mu^2} \right)^{-\tfrac{1}{2} \delta} \Gamma{\left(\tfrac{1}{2} \delta - 2 \right)}
&  \xrightarrow{\delta \to 0} \frac{1}{\delta} \left( \frac{\tilde{m}^2}{4 \pi \mu^2} \right)^{-\tfrac{\delta}{2}}  - \left( \frac{\tilde{m}^2}{4 \pi \mu^2} \right)^{-\tfrac{\delta}{2}} \left( \frac{3}{4} - \frac{\gamma}{2} \right) + O(d)
\\
&  
\xrightarrow{\delta \to 0} - \frac{1}{2} \cdot \frac{2}{\delta} \left( 1 - \left( \frac{\tilde{m}^2}{4 \pi \mu^2} \right)^{-\tfrac{\delta}{2}} \right) + \frac{1}{\delta} + \left( \frac{3}{4} - \frac{\gamma}{2} \right) + O(d)
\\
&  
\xrightarrow{\delta \to 0} - \frac{1}{2} \log{\left( \frac{\tilde{m}^2}{4 \pi \mu^2} \right)} + \frac{1}{\delta} + \left( \frac{3}{4} - \frac{\gamma}{2} \right) + O(d)
\end{align*}
Therefore,
\begin{align*}
I_c & = - \left( \frac{\tilde{m}^2}{4 \pi} \right)^2 \left[ \frac{1}{\delta} + \left( \frac{3}{4} - \frac{\gamma}{2} \right) + O(d) \right] + \frac{1}{2} \left( \frac{\tilde{m}^2}{4 \pi} \right)^2 \log{\left( \frac{\tilde{m}^2}{4 \pi \mu^2} \right)^2}
\\
& = - \left( \frac{\tilde{m}^2}{4 \pi} \right)^2 \left[ \frac{1}{\delta} + \left( \frac{3}{4} - \frac{\gamma}{2} \right) + \frac{1}{2} \log{\left( \frac{\Lambda_0^2}{4 \pi \mu^2} \right)} + O(d) \right] + \frac{1}{32 \pi^2} \tilde{m}^4 \log{\left( \frac{\tilde{m}^2}{\Lambda_0^2} \right)^2}
\end{align*}
The term in brackets is only dependent on the regularization scales and not on $\phi$. Therefore, the first term is a quadratic polynomial in $\phi$ which can be absorbed into the counter-terms leaving the second $\Lambda$-independent term as in the previous cases. 

\subsection{Renormalization in the Small-$\phi$ Regime}

Taking any of the above integrals, $I_a$ for concreteness, notice that the entire first part can be absorbed into the counter-terms leaving,
\[ I = \frac{1}{32 \pi^2} \tilde{m}^4 \log{\left( \frac{\tilde{m}^2}{\Lambda_0^2}  \right)} = \frac{1}{32 \pi^2} \left(m^2 + \tfrac{1}{2} g \phi^2 \right)^2 \log{\left( \frac{m^2 + \tfrac{1}{2} g \phi^2 }{\Lambda_0^2}  \right)} \]
Therefore,
\[ V_1(\phi) = \lim_{\Lambda \to \infty} \left[ c_{\lambda, \Lambda} + \tfrac{1}{2} c_{m,\Lambda} \phi^2 + \tfrac{1}{24} c_{g, \Lambda} \phi^4   \right] + \frac{1}{64 \pi^2} \left(m^2 + \tfrac{1}{2} g \phi^2 \right)^2 \log{\left( \frac{m^2 + \tfrac{1}{2} g \phi^2 }{\Lambda_0^2}  \right)} \]
These counter-terms are subject to the renormalization prescription that $V_1(\phi) = 0$ to $O(\phi^4)$ so that, at the one-loop level, $V_q(\phi) = V_0(\phi)$ for small $\phi$. 
Expanding the logarithmic term,
\begin{align*}
\left(m^2 + \tfrac{1}{2}  g \phi^2 \right)^2 & \log{\left( \frac{m^2 + \tfrac{1}{2} g \phi^2 }{\Lambda_0^2}  \right)} 
\\
& = m^4 \log{\left( \frac{m^2}{\Lambda_0^2} \right)} + \left( \frac{1}{2} m^2g +  m^2g \log{\left( \frac{m^2}{\Lambda_0^2} \right)} \right) \phi^2 + \left( \frac{3 g^2}{8} + \frac{1}{4} g^2 \log{\left( \frac{m^2}{\Lambda_0^2} \right)} \right) \phi^4 + O(\phi^6)  
\end{align*}
we find the terms which must be canceled by counter-terms for the renormalization prescription to hold. Therefore,
\begin{align*}
\tilde{c}_\lambda & = - \frac{1}{64 \pi^2} m^4 \log{\left( \frac{m^2}{\Lambda_0^2} \right)}
\\
\tilde{c}_{m} & = - \frac{1}{64 \pi^2} \left( \frac{1}{2} m^2g +  m^2g \log{\left( \frac{m^2}{\Lambda_0^2} \right)} \right)
\\
\tilde{c}_g & = - \frac{1}{64 \pi^2} \left( \frac{3 g^2}{8} + \frac{1}{4} g^2 \log{\left( \frac{m^2}{\Lambda_0^2} \right)} \right)
\end{align*}
where $\tilde{c}$ are the counter-terms after subtracting the UV divergent quantities and taking the $\Lambda \to \infty$ limit. Therefore,
\begin{align*}
V_q(\phi) & = \left\{ \lambda - \frac{\hbar m^4}{64 \pi^2} \log{\left( \frac{m^2}{\Lambda_0^2} \right)} \right\} + \frac{1}{2} m^2 \left\{ 1 - \frac{\hbar g}{64 \pi^2} \left(1 +  2 \log{\left( \frac{m^2}{\Lambda_0^2} \right)} \right) \right\} \phi^2 
\\
& + \frac{1}{24} g \left\{ 1 - \frac{24 \hbar g}{64 \pi^2} \left( \frac{3}{8}  + \frac{1}{4} \log{\left( \frac{m^2}{\Lambda_0^2} \right)} \right) \right\} \phi^4 + \frac{\hbar}{64 \pi^2} \left(m^2 + \tfrac{1}{2}  g \phi^2 \right)^2 \log{\left( \frac{m^2 + \tfrac{1}{2} g \phi^2 }{\Lambda_0^2}  \right)} 
\end{align*}
In the limit $m \to 0$ the logarithmic terms are unbounded below and thus we cannot use this form for the effective potential. For simplicity, we can choose the reference scale to be the physical mass parameter $\Lambda_0 = m$ which greatly simplifies the result to,
\begin{align*}
V_q(\phi) & = \lambda + \frac{1}{2} m^2 \left\{ 1 - \frac{\hbar g}{64 \pi^2} \right\} \phi^2 + \frac{1}{24} g  \left\{ 1 - \frac{9\hbar g}{64 \pi^2} \right\} \phi^4 + \frac{\hbar}{64 \pi^2} \left(m^2 + \tfrac{1}{2}  g \phi^2 \right)^2 \log{\left( 1 + \frac{1}{2} g \left( \frac{\phi}{m} \right)^2 \right)} 
\end{align*}
In the limit $m \to 0$ the logarithm diverges for any nonzero value of the classical field. Therefore we still cannot use this renormalization prescription in the $m \to 0$ case. 

\subsection{Higher-Loop Contributions}

To analyze the contribution of higher-loop terms to the effective action, we need to be a bit more systematic in our treatment. We can write the effective action in terms of a one-particle-irreducible restricted path-integral,
\[ \exp{\left[ i\Gamma[\Phi] / \hbar \right] } = \int_{1PI} \pathd{\phi} \exp{\left[ i S[\Phi + \phi] / \hbar \right]} \]
where $\Phi$ denotes the ``classical'' or ``background'' field and $\phi$ the dynamical quantum fluctuations. We can write the action as,
\begin{align*}
S[\Phi + \phi] & = \int \dn{4}{x} \lagrange(\Phi + \phi) = \int \dn{4}{x} \left[  \sum_{n = 0}^{\infty} \frac{1}{n!} \frac{\delta^n \lagrange}{(\delta \phi)^n} \phi^n \right]
\\
& = \int \dn{4}{x} \left[  \lagrange(\Phi) + \frac{\delta \lagrange}{\delta \phi} \phi + \frac{1}{2} \frac{\delta^2 \lagrange}{\delta \phi \delta \phi} \phi^2 + \frac{1}{3!} \frac{\delta^3 \lagrange}{\delta \phi \delta \phi \delta \phi} \phi^3 + \frac{1}{4!} \frac{\delta^4 \lagrange}{\delta \phi \delta \phi \delta \phi \delta \phi} \phi^4 \right]
\end{align*} 
I will denote these coefficients by $\lagrange^{(n)}(\Phi)$ and their integrals by,
\[ S_n[\Phi; \phi] = \int \dn{4}{x} \frac{1}{n!} \lagrange^{(n)}(\Phi) \phi^n \]
Now we expand,
\begin{align*}
\exp{\left[ i\Gamma[\Phi] / \hbar \right] } & = \int_{1PI} \pathd{\phi} \exp{\left[ \frac{i}{\hbar} \int \dn{4}{x} \left(  \lagrange(\Phi) + \lagrange^{(1)}(\Phi) \phi + \frac{1}{2} \lagrange^{(2)}(\Phi) \phi^2 + \frac{1}{3!} \lagrange^{(3)}(\Phi) \phi^3 + \frac{1}{4!} \lagrange^{(4)}(\Phi) \phi^4 \right) \right]}
\\
& = \exp{\left[ i S_0[\Phi] / \hbar \right]} \int_{1PI} \pathd{\phi} \exp{\left[ i S_2[\Phi ; \phi] / \hbar \right]} \exp{\left[ \frac{i}{\hbar} S_1[\Phi ; \phi] + \frac{i}{\hbar} \sum_{n = 3}^{\infty} S_n[\Phi ; \phi] \right]}
\end{align*} 
Therefore, combining the $\phi$-independent terms,
\begin{align*}
\exp{\left[ \frac{i}{\hbar} \left( \Gamma[\Phi] - S_0[\Phi] \right) \right]} = \int_{1PI} \pathd{\phi} \exp{\left[ \frac{i}{\hbar} \int \dn{4}{x} \frac{1}{2} \lagrange^{(2)}(\Phi) \phi^2 \right]} \exp{\left[ \frac{i}{\hbar} \int \dn{4}{x} \left( \lagrange^{(1)}(\Phi) \phi + \sum_{n = 3}^{\infty} \frac{1}{n!} \lagrange^{(n)}(\Phi) \phi^n \right) \right]}
\end{align*}
We can drop the linear term in $\phi$,
\[ \frac{i}{\hbar} \int \dn{4}{x}  \lagrange^{(1)}(\Phi) \phi \]
because there are no $1PI$ diagrams with any one-valent vertices. Alternatively, we can see that this term gives the expectation value of the field which is fixed to the classical value $\Phi$ when plugged into a standard path-integral. Therefore, its exponential is exactly the term canceled when we go from the full path-integral computing $e^{i W[J]}$ to computing $e^{i \Gamma[\Phi]}$ by the Legendre transformation. Therefore, we have,
\begin{align*}
\exp{\left[ \frac{i}{\hbar} \left( \Gamma[\Phi] - S_0[\Phi] \right) \right]} = \int_{1PI} \pathd{\phi} \exp{\left[ \frac{i}{\hbar} \int \dn{4}{x} \frac{1}{2} \lagrange^{(2)}(\Phi) \phi^2 \right]} \cdot \frac{\int_{1PI} \pathd{\phi} e^{i S_2[\Phi ; \phi] / \hbar} \exp{\left[ \sum_{n = 3}^{\infty} S_n[\Phi ; \phi] \right]}}{\int_{1PI} \pathd{\phi} e^{i S_2[\Phi ; \phi] / \hbar}}
\end{align*}
However, since all two-valent only vacuum diagrams are automatically $1PI$ we can evaluate the first term in the usual fashion,
\[ \int_{1PI} \pathd{\phi} \exp{\left[ \frac{i}{\hbar} \int \dn{4}{x} \frac{1}{2} \lagrange^{(2)}(\Phi) \phi^2 \right]} = \det{\left[ - \hbar^{-1} \lagrange^{(2)}(\Phi) \right]^{-1/2}} \]
Now taking the logarithm of our expression for the quantum effective action allows us to restrict our $1PI$ integrals to only connected diagrams i.e. $1PIC$ diagrams since disconnected diagrams exponentiate. Therefore,
\[ i \Gamma[\Phi] = i S_0[\Phi] - \frac{1}{2} \hbar \log{\det{\left[ - \hbar^{-1} \lagrange^{(2)}(\Phi) \right]}} + \hbar \cdot \frac{\int_{1PIC} \pathd{\phi} e^{i S_2[\Phi ; \phi] / \hbar} \exp{\left[ \sum_{n = 3}^{\infty} S_n[\Phi ; \phi] \right]}}{\int_{1PIC} \pathd{\phi} e^{i S_2[\Phi ; \phi] / \hbar}} \]
where the last term should be interpreted as the sum of all properly normalized\footnote{meaning that these terms can be calculated by expanding the exponential and applying Wick's theorem with the standard normalized two-point function} one-particle-irreducible connected diagrams with interaction vertices from the $\frac{1}{n!} \lagrange^{(n)}(\Phi) \phi^3$ terms for $n \ge 3$ and propagators given by the ``free'' term $\frac{1}{2} \lagrange^{(2)}(\Phi) \phi^2$. Since all vacuum diagrams with all vertices at least three-valent have at minimum two loops, by loop counting, each diagram in the sum denoted by the last term enters with a factor of $\hbar^{L-1}$ and thus each has at least a factor of $\hbar$. Therefore, we have isolated the zero-loop, one-loop, and higher-loop contributions to the quantum effective potential in these three terms,
\[ i \Gamma[\Phi] = i S_0[\Phi] - \frac{1}{2} \hbar \, \Tr{\log{\left[ -  \lagrange^{(2)}(\Phi) \right]}} + \hbar^2 \cdot \frac{\int_{1PIC} \pathd{\phi} e^{i S_2[\Phi ; \phi] / \hbar} \exp{\left[ \sum_{n = 3}^{\infty} S_n[\Phi ; \phi] \right]}}{\hbar \int_{1PIC} \pathd{\phi} e^{i S_2[\Phi ; \phi] / \hbar}} \]
We have already taken into account the first two terms which give the tree-level and one-loop quantum effective potential calculated earlier since these are exactly the classical action density and the one-loop functional determinant. The higher-loop effects come from $1PIC$ diagrams with multiple-loops. Consider ``bubble chain'' diagrams attached by $\phi^4$-interaction vertices. Since,
\[ \lagrange^{(4)}(\Phi) = - g \]
only the standard four-valent coupling is present, that is, the four-valent coupling is independent of the field expectation value. Thus, an $L$-loop bubble chain gives a contribution\footnote{Note that I now switch back to the earlier convention of denoting the classical field by $\phi$.},
\[ iI_L(\phi) = \frac{i}{g} (-i \hbar g)^L \left[ \int \frac{\dn{4}{k}}{(2 \pi)^4} \frac{i}{k^2 - \tilde{m}^2(\phi) + i \epsilon} \right]^2 \cdot \left[ \int \frac{\dn{4}{k}}{(2 \pi)^4} \frac{i}{k^2 - \tilde{m}^2(\phi) + i \epsilon} \cdot \frac{i}{k^2 - \tilde{m}^2(\phi) + i \epsilon} \right]^{L-2}  \]
where the first term is the integration over the circles which cap the end of the bubble chain and the $L-2$ other terms are integrals over the internal loop structure of the chain. Performing a Wick rotation, 
\[ iI_L(\phi) = -\frac{i}{g} (-\hbar g)^L \left[ \int \frac{\dn{4}{k_E}}{(2 \pi)^4} \frac{1}{k^2_E + \tilde{m}^2(\phi) - i \epsilon} \right]^2 \cdot \left[ \int \frac{\dn{4}{k_E}}{(2 \pi)^4} \frac{1}{(k^2 + \tilde{m}^2(\phi) - i \epsilon)^2}  \right]^{L-2}  \]
We expect the first integral to diverge as $\Lambda^2$ and the second as $\log{\Lambda}$ on dimensional grounds. 
However, these integrals are familiar are can easily be computed from the work we did earlier. By differentiating the renormalized value $I$ we find that, up to quadratic terms in $\phi$ and $\Lambda$ dependent terms which can both be canceled by counter-terms,
\begin{align*}
\int \frac{\dn{4}{k_E}}{(2 \pi)^4} \frac{1}{k^2_E + \tilde{m}^2} & = \frac{1}{32 \pi^2} \tilde{m}^2 \log{\left( \frac{\tilde{m}^2}{\Lambda_0^2} \right)}
\\
\int \frac{\dn{4}{k_E}}{(2 \pi)^4} \frac{1}{(k^2_E + \tilde{m}^2)^2} & = \frac{1}{16 \pi^2} \log{\left( \frac{\tilde{m}^2}{\Lambda_0^2} \right)}
\end{align*}
where $\Lambda_0$ is an arbitrary reference scale which is determined by shifting the values of the counter-terms. For convenience, I will set $\Lambda_0 = m$ in this case. Therefore,
\[ iI_L(\phi) = -\frac{i \tilde{m}^4(\phi)}{4 g} \left[ -\frac{\hbar g}{16 \pi^2} \log{\left( \frac{\tilde{m}^2(\phi)}{m^2} \right)} \right]^L \] 
However, for large enough $\phi$ the term $\tilde{m}^2(\phi) = m^2 + \tfrac{1}{2} g \phi^2$ will be large enough compared to $m^2$ to make the term in brackets greater than $1$. When this happens, higher-loop contributions will diverge so the perturbation series will break down for sufficiently large $\phi$. 

\subsection{Renormalization in the Large-$\phi$ Regime}

We want to examine the behavior of $V_q(\phi)$ in the large $\phi$ regime. Because the coefficients diverge logarithmically, we cannot directly expand asymptotically in the limit $\phi \to \infty$ and compare to the tree-level quartic polynomial divergence. Instead, we choose a reference scale $M \gg m$ and expand in $\phi$ about $M$ to get,
\begin{align*}
\left(m^2 + \tfrac{1}{2}  g \phi^2 \right)^2 & \log{\left( \frac{m^2 + \tfrac{1}{2} g \phi^2 }{\Lambda_0^2}  \right)} 
\\
& = \left( \frac{3}{2} m^4 + m^4 \log{\left( \frac{g \phi^2}{2 \Lambda_0^2} \right)} \right) + \frac{1}{2} m^2 \left( g + 2 g \log{\left( \frac{g \phi^2}{2 \Lambda_0^2} \right)} \right) \phi^2 + \frac{1}{4} g^2 \log{\left( \frac{g \phi^2}{2 \Lambda_0^2} \right)} \phi^4 + O([m/\phi]^2)
\end{align*}
However, assuming that $\phi \sim M$ we can replace $\phi$ in the argument of the logarithms with $M$ such that we have an ordinary polynomial expansion. We need to impose the condition that $V_q(\phi) = V_0(\phi)$ up to quartic order for $\phi \sim M$. Thus, we fix out counter-terms such that $V_1(M) = 0$ which sets,
\begin{align*}
\tilde{c}_{\lambda} & = - \frac{1}{64 \pi^2} \left( \frac{3}{2} m^4 + m^4 \log{\left( \frac{g M^2}{2 \Lambda_0^2} \right)} \right)
\\
\tilde{c}_m & = - \frac{1}{64\pi^2} \cdot \frac{1}{2} m^2 \left( g + 2 g \log{\left( \frac{g M^2}{2 \Lambda_0^2} \right)} \right)
\\
\tilde{c}_g & = - \frac{1}{64 \pi^2} \cdot \frac{1}{4} g^2 \log{\left( \frac{g M^2}{2 \Lambda_0^2} \right)}
\end{align*}
Therefore, the effective potential becomes,
\begin{align*}
V_q(\phi) & = \left\{ \lambda - \frac{\hbar}{64 \pi^2} \left( \frac{3}{2} m^4 + m^4 \log{\left( \frac{g M^2}{2 \Lambda_0^2} \right)} \right) \right\} + \frac{1}{2} m^2 \left\{ 1 - \frac{\hbar}{64\pi^2} \left( g + 2 g \log{\left( \frac{g M^2}{2 \Lambda_0^2} \right)} \right) \right\} \phi^2 
\\
& + \frac{1}{24} g \left\{1 -  \frac{6 \hbar g}{64 \pi^2} \log{\left( \frac{g M^2}{2 \Lambda_0^2} \right)} \right\} \phi^4 
+ \frac{\hbar}{64 \pi^2} \left(m^2 + \tfrac{1}{2}  g \phi^2 \right)^2  \log{\left( \frac{m^2 + \tfrac{1}{2} g \phi^2 }{\Lambda_0^2}  \right)} 
\end{align*}
If we choose the convenient reference-scale $\Lambda_0^2 = \tfrac{1}{2} g M^2$ then the effective potential can neatly be written as,
\begin{align*}
V_q(\phi) = \lambda - \frac{3 \hbar}{128 \pi^2} m^4 + \frac{1}{2} m^2 \left\{1 - \frac{\hbar g}{64 \pi^2} \right\} \phi^2 + \frac{1}{24} g \phi^4 + \frac{\hbar}{64 \pi^2} \left(m^2 + \tfrac{1}{2} g \phi^2 \right)^2 \log{\left( \frac{2 m^2 g^{-1} + \phi^2}{M^2} \right)} 
\end{align*}
This expression is well-defined in the limit $m \to 0$ since the mass no longer appears in the numerator of the logarithms. If we expand the logarithm again for $\phi \sim M \gg m$ we have,
\begin{align*}
V_q(\phi) & = \left\{ \lambda + \frac{\hbar}{64 \pi^2}  m^4 \log{\left( \frac{\phi^2}{M^2} \right)} \right\} + \frac{1}{2} m^2 \left\{ 1 + \frac{\hbar g}{32\pi^2} \log{\left( \frac{\phi^2}{M^2} \right)} \right\} \phi^2 
\\
& + \frac{1}{24} g \left\{1 + \frac{6 \hbar g}{64 \pi^2} \log{\left( \frac{\phi^2}{M^2} \right)} \right\} \phi^4 + O([m/\phi]^2)
\end{align*}

\subsection{Higher-Loop Corrections in The Large-$\phi$ Regime}

We take the limiting case $m \to 0$ and the renormalization prescription that $V_q(\phi)$ is equal to tree-level for large $\phi \sim M$. 
In the new renormalization prescription, the procedure and argument are nearly identical. However, when we choose the correct reference scale for the logarithm and throw the rest away into counter-terms we must take not $\Lambda_0 = m$ but $\Lambda_0^2 = \tfrac{1}{2} g M^2$ such that the logarithm terms,
\[ \log{\left( \frac{\tilde{m}^2(\phi)}{\Lambda_0^2} \right)} = \log{\left( \frac{\phi^2}{M^2} \right) } \]
vanish at the renormalization scale so that the higher-loop corrections do not contribute to the tree-level result at this scale and thus the quantum effective potential is equal to tree-level up to fourth-order at scales near $M$ except for logarithmic running of the couplings. Taking this prescription, an identical argument gives,
\[ iI_L(\phi) = -\frac{i \tilde{m}^4(\phi)}{4 g} \left[ -\frac{\hbar g}{16 \pi^2} \log{\left( \frac{\tilde{m}^2(\phi)}{\tfrac{1}{2} g M^2} \right)} \right]^L = - \frac{i g \phi^4}{16} \left[ -\frac{\hbar g}{16 \pi^2} \log{\left( \frac{\phi^2}{M^2} \right)} \right]^L  \] 
However, since the scheme is valid for $\phi$ near the scale $M$ we see that the logarithm is always small and therefore higher-loop corrections do not blow up. 

\subsection{The $\beta$-Function in The Large-$\phi$ Regime}


We require that $V_q$ is independent of the renormalization scale $M$. Therefore, $M \partial_M V_q(\phi) = 0$ which must hold for each power in $\phi$,
\begin{align*}
V_q(\phi) & = \left\{ \lambda - \frac{\hbar}{64 \pi^2} \left( \frac{3}{2} m^4 + m^4 \log{\left( \frac{g M^2}{2 \Lambda_0^2} \right)} \right) \right\} + \frac{1}{2} m^2 \left\{ 1 - \frac{\hbar}{64\pi^2} \left( g + 2 g \log{\left( \frac{g M^2}{2 \Lambda_0^2} \right)} \right) \right\} \phi^2 
\\
& + \frac{1}{24} g \left\{1 -  \frac{6 \hbar g}{64 \pi^2} \log{\left( \frac{g M^2}{2 \Lambda_0^2} \right)} \right\} \phi^4 
+ \frac{\hbar}{64 \pi^2} \left(m^2 + \tfrac{1}{2}  g \phi^2 \right)^2  \log{\left( \frac{m^2 + \tfrac{1}{2} g \phi^2 }{\Lambda_0^2}  \right)} 
\end{align*}
Therefore, each term in curly braces must individually be independent of $M$. Thus,
\begin{align*}
M \partial_M \lambda & = M \partial_M \frac{\hbar}{64 \pi^2} \left( \frac{3}{2} m^4 + m^4 \log{\left( \frac{g M^2}{2 \Lambda_0^2} \right)} \right) = \frac{\hbar m^4}{32 \pi^2} 
\\
M \partial_M m^2 & = M \partial_M \frac{\hbar m^2}{64\pi^2} \left( g + 2 g \log{\left( \frac{g M^2}{2 \Lambda_0^2} \right)} \right) =  \frac{\hbar m^2 g}{16 \pi^2}
\\
M \partial_M g & = M \partial_M \frac{6 \hbar g^2}{64 \pi^2} \log{\left( \frac{g M^2}{2 \Lambda_0^2} \right)} = \frac{3\hbar g^2}{16 \pi^2} 
\end{align*}
We may safely drop derivatives with respect to the couplings on the right-hand side of the equation because these are proportional to $\hbar$ giving an over factor of $\hbar^2$ to such term which we ignore at the one-loop level in perturbation theory. The equation determining the running of the effective coupling $g$ can be rewritten in the form,
\[ \deriv{g}{\log{M}} = \hbar \beta(g) \quad \text{where} \quad \beta(g) = \frac{3}{16\pi^2} g^2 \]
We can easily integrate this equation via,
\[ \int_{g_0}^g \frac{\d{g}}{g^2} = \frac{3\hbar}{16 \pi^2} \int_{\log{M_0}}^{\log{M}} \d{\log{M}} \]
and therefore,
\[ \frac{1}{g_0} - \frac{1}{g} = \frac{3 \hbar}{16\pi^2} \log{\left( \frac{M}{M_0} \right)} \]
which implies that,
\[ g(M) = \frac{g_0}{1 - \frac{3 \hbar g_0}{16\pi^2} \log{\left( \frac{M}{M_0} \right)}} \]
Therefore, there exists an energy scale $M_*$ at which the coupling diverges. The coupling pole happens for,
\[ M_* = M_0 \exp{\left( \frac{16 \pi^2}{3 \hbar g_0} \right)} \]
We can rewrite the solution for $g(M)$ in terms of this ``dynamically generated'' scale. Replacing $M_0$,
\[ g(M) = \frac{16\pi^2}{3 \hbar \log{\left( \frac{M_*}{M} \right)}} \]

\subsection{The Large Logarithm Problem is Vanquished}

The previous explicit form for the running of the coupling constant $g$ derived from the $\beta$-function is exact at the one-loop i.e. $O(\hbar)$ level. The large logarithms appearing in the higher-loop corrections do not diverge at higher-order in this renormalization scheme even though such large logarithms appear when the solution to the $\beta$-function is expanded in a power series. To see why this is true, we need to carefully examine the renormalization prescription used in determining the $\beta$-function. We take the coupling to be renormalized at the scale $M$ i.e. $g = g(M)$ and then consider a small change in the energy scale and compute the change in the effective coefficients. Therefore, in each small step we expand and renormalize the quantum effective action at $g = g(M)$ for the current scale $M$. Therefore, the logarithmic terms have ratios of the current renormalization scale to the next step up in scale which is of order the change in $\log{M}$ which is taken to be small when deriving the flow equation. Therefore, higher-order loops enter with powers of this logarithmic change of scale which goes to zero under higher powers. Therefore, the higher-loop diagrams are suppressed both by powers of $\hbar$ and by powers of $\log{(M'/M)} = \d{\log{M}}$. 

\subsection{Low-Energy Limits}

Using the exact one-loop solution to the $\beta$-function equation in terms of the dynamically generated scale we have, 
\[ 
g(M) = \frac{16\pi^2}{3 \hbar \log{\left( \frac{M_*}{M} \right)}} 
\]
which remains valid for $m = 0$. Taking $M \to 0$ we find that the denominator diverges and therefore $g(M) \to 0$. Thus, we predict that the theory becomes free in the low-energy limit. However, at any finite order in $\hbar$, the expansion of $g$ is a power series in logarithms of ratios of $M$ to some energy scale. In the limit $g \to 0$ these terms badly diverge. We ought to trust the result from the $\beta$-function that $g(M) \to 0$ as $M \to 0$. For any finite value of $M$ we know the $\beta$-function accurately captures the running of $g$. Since the $\beta$ function is always positive, we must have that $g$ strictly decreases as $M \to 0$ thus we do not expect $g$ to diverge. Rather the prediction $g(M) \to 0$ fits the finite behavior of the $\beta$-function and matches the intuitive idea that the $\phi^4$-potential is flat and thus uncoupled near the classical value of $\phi = 0$.  

\subsection{Classical and Quantum Scale-Invariance}

Classically, the massless $\phi^4$-theory exhibits scale invariance because taking $x \mapsto \lambda x$ and $\phi(x) \mapsto \phi_{\lambda}(x) = \lambda \phi(\lambda x)$ we have,
\[ \lagrange[\phi(x)] = \tfrac{1}{2} \partial_\mu \phi \partial^\mu \phi - \tfrac{1}{4!} g \phi^4 \mapsto \tfrac{1}{2} \partial_\mu \phi_{\lambda} \partial^\mu \phi_{\lambda} - \tfrac{1}{4!} g \phi_{\lambda}^4 = \tfrac{1}{2} \lambda^{4} \partial_\mu \phi \partial^\mu \phi - \tfrac{1}{4!} \lambda^{4} g \phi^4 = \lambda^4 \lagrange[\phi(\lambda x)] \]
However, this scale-invariance is \textit{not} preserved in the quantum theory because the $\beta$-function is nonzero and thus there is running of the coupling constants as the scale changed. Furthermore, although the classical theory has no dimensionful parameters to set a characteristic scale, the quantum theory does, namely the dynamically generated scale $M_{*}$. Although scale-invariance is a linearly-realized symmetry it does not need to be preserved into the quantum theory because it does not preserve path-integral measures since it introduces a scale factor in the measure. Therefore, the analysis of Weinberg 16.4 about linearly realized symmetries of the classical and quantum effective action does not apply to scaling symmetry since, although it is linearly realized, scale-invariance does not leave the measure invariant.  

\section{Spontaneous Symmetry Breaking}

Consider the theory,
\[ \lagrange = \tfrac{1}{2} \partial_\mu \phi \partial^\mu \phi - \lambda + \tfrac{1}{2} \kappa \phi^2 - \tfrac{1}{24} g \phi^4 \]
in which we write,
\[ V_0(\phi) = \lambda - \tfrac{1}{2} \kappa \phi^2 + \tfrac{1}{24} g \phi^4 \]
Classically, this theory exhibits spontaneous symmetry breaking of the $\phi \mapsto - \phi$ symmetry. The classical symmetry breaking ground states are at,
\[ \phi = \pm \sqrt{\frac{6\kappa^2}{g}} \]

\subsection{Symmetry Breaking in the Quantum Theory}

We need to compute the quantum effective potential to one-loop for this theory. We see that,
\[ V''_0(\phi) = - \kappa^2 + \tfrac{1}{2} g \phi^2 \]
and therefore,
\[ \lagrange^{(2)}(\phi) = - \partial^2 + \kappa^2 - \tfrac{1}{2} g \phi^2 \]  
so the one-loop correction to the quantum effective potential becomes,
\[ V_1(\phi) = -\tfrac{i}{2} \log{\det{\left( \partial^2 - \kappa^2 + \tfrac{1}{2} g \phi^2 \right)}} \]
We have already calculated such logarithmic determinants up to infinite constants which may be renormalized away. We found,
\[ -\tfrac{i}{2} \log{\det{\left(\partial^2 + \tilde{m}^2 \right)}} = \frac{1}{2} \int \frac{\dn{4}{k_E}}{(2\pi)^4} \log{\left( \frac{k_E^2 + \tilde{m}^2}{\Lambda_0^2} \right)}  = \frac{1}{64 \pi^2} \tilde{m}^4 \log{\left( \frac{\tilde{m}^2}{\Lambda_0^2}  \right)}  \]
therefore the one-loop correction to the quantum effective potential becomes,
\begin{align*}
V_1(\phi) & = \tilde{c}_{\lambda} + \tilde{c}_m \phi^2 + \tilde{c}_g \phi^4 + \frac{1}{64 \pi^2} \left(- \kappa^2 + \tfrac{1}{2} g \phi^2 \right)^2 \log{\left( \frac{- \kappa^2 + \tfrac{1}{2} g \phi^2}{\Lambda_0^2} \right)} 
\end{align*}
We need to renormalize the quantum effective potential at some scale $M \sim \sqrt{\frac{6 \kappa^2}{g}}$. 
I choose the prescription that $V_q(\phi) = V_0(\phi)$ expanded to quartic-order at $M$. A convenient choice of reference scale is $\Lambda_0^2 = 2 \kappa^2$. Expanding $V_1$ to quartic order we find,
\begin{align*}
V_1(\phi) & = \tilde{c}_{\lambda} + \tilde{c}_m \phi^2 + \tilde{c}_g \phi^4 + \frac{1}{64 \pi^2} \left[ \left( \frac{3 \kappa^4}{2} + \kappa^4 \log{\left( \frac{g \phi^2}{2\Lambda_0^2} \right)} \right)
 -\frac{1}{2} g \kappa^2 \left( 1 + \log{\left( \frac{g \phi^2}{2 \Lambda_0^2} \right)} \right) 
 + \frac{1}{4} g^2 \log{\left( \frac{g \phi^2}{2 \Lambda_0^2} \right)} \right]
\end{align*}
Setting the reference scale $\Lambda_0^2 = \tfrac{1}{2} g M^2$ and renormalizing to match the quadratic terms at $\phi \sim M$ we get,
\begin{align*}
\tilde{c}_{\lambda} & = - \frac{1}{64\pi^2} \cdot  \frac{3 \kappa^4}{2} 
\\
\tilde{c}_{m} & = \frac{1}{64\pi^2} \cdot \frac{1}{2} g \kappa^2
\\
\tilde{c}_g & = 0
\end{align*}
Therefore, the renormalized quantum effective potential becomes,
\begin{align*}
V_q(\phi) = \left\{ \lambda - \frac{3 \hbar \kappa^4}{128 \pi^2} \right\} - \frac{1}{2} \kappa^2 \left\{ 1 - \frac{\hbar g}{64 \pi^2} \right\} \phi^2 + \frac{1}{24} g \phi^4 + \frac{\hbar}{64 \pi^2} \left(- \kappa^2 + \tfrac{1}{2} g \phi^2 \right)^2 \log{\left( \frac{- \phi_c^2 + \phi^2}{M^2} \right)} 
\end{align*}
where $\phi_c^2 = \frac{2 \kappa^2}{g}$ is the critical field value at which the effective classical mass goes to zero i.e. it is the inflection point of $V_0$. 
Now we expand the polynomial multiplying the logarithm,
\begin{align*}
V_q(\phi) & = \left\{ \lambda - \frac{\hbar \kappa^4}{64\pi^2} \left( \frac{3}{2} - \log{\left( \frac{\phi^2 - \phi_c^2}{M^2} \right)} \right) \right\} 
- \frac{1}{2} \kappa^2 \left\{ 1 - \frac{\hbar g}{64 \pi^2} \left( 1 - 2 \log{\left( \frac{\phi^2 - \phi_c^2}{M^2} \right)} \right) \right\} \phi^2 
\\
& \quad \quad \quad + \frac{1}{24} g \phi^4 \left\{1 + \frac{6 \hbar g}{64 \pi^2} \log{\left( \frac{\phi^2 - \phi_c^2}{M^2} \right)} \right\}
\end{align*}
Notice that for $M^2 \sim \sqrt{\frac{6 \kappa^2}{g}} = 3 \phi_c^2$ and $\phi \sim M$ we have $\phi^2 - \phi_c^2 \sim 2 \phi_c^2$ so the logarithm is approximately,
\[  \log{\left( \frac{\phi^2 - \phi_c^2}{M^2} \right)} \approx \log{ \left( \frac{2 \phi_c^2}{3 \phi_c^2} \right)} = \log{2} - \log{3} \approx -0.4 \]
Therefore, assuming that $g$ is small at the renormalization scale $M$, the couplings in the quantum effective potential are shifted only slightly from their classical counterparts near the minimum of the classical potential. Thus, the quantum effective potential at the one-loop level still retains a minimum at approximately,
\[ \phi^2 = \frac{6 \kappa^2}{g} \cdot \frac{\left\{ 1 - \frac{\hbar g}{64 \pi^2} \left( 1 - 2 \log{\left(\frac{2}{3} \right)} \right) \right\}}{\left\{1 + \frac{6 \hbar g}{64 \pi^2} \log{\left( \frac{2}{3} \right)} \right\}} \]
which is slightly shifted from the classical result because of the loop corrections. Therefore, the theory retains spontaneous symmetry breaking at the quantum level. 

\subsection{Imaginary Pathologies in the Quantum Effective Potential}

We can immediately see from the argument of the logarithms that when $\phi^2 < \phi_c^2$ that the quantum effective potential becomes complex violating its interpretation as an energy density. This is due to non-perturbative effects in the quantum effective action for small values of $\phi$. The computed quantum effective action is non-convex because it has minima at $\phi \neq 0$. However, the effective potential must be convex because it measures the minimum energy configuration which a given field expectation value. However, since the ground state is degenerate, a mixing between the two ground states via superposition will have the same minimal energy but can have an expectation value anywhere between the values of the minima. Therefore, any apparent convexity in the quantum effective potential can be removed by taking superpositions of states on either side of lower energy. However, this imposed convexity is a fundamentally non-perturbative effect. The mixing between degenerate ground states will not appear at any level of perturbation theory. This is why expanding the quantum effective action in this non-perturbative region gives pathological results at finite orders of perturbation theory. 

\subsection{Apparent Symmetry Breaking in the Massless $\phi^4$-Theory}

Taking the limit $\kappa \to 0$ we find the one-loop quantum effective potential of the massless $\phi^4$-theory,
\begin{align*}
V_q(\phi) & = \lambda + \frac{1}{24} g \phi^4 \left\{1 + \frac{6 \hbar g}{64 \pi^2} \log{\left( \frac{\phi^2}{M^2} \right)} \right\}
\end{align*}
The extrema of this effective potential are found by solving for the zeros of the derivative,
\[ \deriv{}{\phi} V_q(\phi) = \frac{1}{6} g \phi^3 
\left\{1 + \frac{6 \hbar g}{64 \pi^2} \log{\left( \frac{\phi^2}{M^2} \right)} \right\} + \frac{1}{24} g \phi^4 \cdot \frac{3 \hbar g}{16 \pi^2} \frac{1}{\phi} = 0 \]
Therefore,
\[ 1 + \frac{6 \hbar g}{64 \pi^2} \log{\left( \frac{\phi^2}{M^2} \right)} + \frac{3 \hbar g}{64 \pi^2} = 0 \]
Which implies that,
\[ \phi_{\text{min}} = \pm M \exp{\left( -\frac{1}{4} - \frac{16 \pi^2}{3 \hbar g} \right)} \]
These values are minima of the one-loop quantum effective potential. Therefore, although classically there is no spontaneous symmetry breaking for a massless $\phi^4$-interaction, there is symmetry breaking at the quantum level. 

\section{Goldstone Bosons}

Consider a complex scalar $\psi$ with action,
\[ S[\psi] = \int \dn{4}{x} \left( \partial_\mu \psi^\dagger \partial^\mu \psi - V_0(\psi) \right) \]
where we take,
\[ V_0(\psi) = - \kappa^2 \psi^\dagger \psi + \tfrac{1}{4} g (\psi^\dagger \psi)^2 \]
This theory has $U(1)$ symmetry $\psi \mapsto e^{i\alpha} \psi$. 

\subsection{Classical Scalar Boson Production via Symmetry Breaking}

We expand the $\psi$ field about the classical scalar vacuum expectation value $\EV{\psi} = v$ in the form,
\[ \psi(x) = v + \frac{1}{\sqrt{2}} \left[ \rho(x) + i \sigma(x) \right] \]
where $\rho$ and $\sigma$ are real scalar fields and $v = \frac{1}{\sqrt{2}} \tilde{v}$. Plugging into the action,
\begin{align*}
\lagrange & = \tfrac{1}{2} \partial_\mu \left( \rho - i \sigma \right) \partial^\mu \left( \rho + i \sigma \right) + \tfrac{1}{2} \kappa^2 \left( \tilde{v} +  \rho - i \sigma \right) \left( \tilde{v} + \rho + i \sigma \right) - \tfrac{1}{16} g \left[ \left( \tilde{v} +  \rho - i \sigma \right) \left( \tilde{v} + \rho + i \sigma \right) \right]^2 
\\
& = \tfrac{1}{2} \partial_\mu \rho \partial^\mu \rho + \tfrac{1}{2} \partial_\mu \sigma \partial^\mu \sigma + \tfrac{1}{2} \kappa^2 (\tilde{v}^2 + 2 \tilde{v} \rho + \rho^2 + \sigma^2) - \tfrac{1}{16} g (\tilde{v}^2 + 2 \tilde{v} \rho + \rho^2 + \sigma^2)^2
\\
& = \tfrac{1}{2} \partial_\mu \rho \partial^\mu \rho + \tfrac{1}{2} \partial_\mu \sigma \partial^\mu \sigma + \tfrac{1}{2} \kappa^2 (\tilde{v}^2 + 2 \tilde{v} \rho + \rho^2 + \sigma^2)
\\
& \quad \quad - \tfrac{1}{16} g[\tilde{v}^4 + 4 \tilde{v}^3 \rho + 4\tilde{v}^2 \rho^2 + 2 \tilde{v}^2( \rho^2 + \sigma^2) + 4 \tilde{v} \rho (\rho^2 + \sigma^2) + (\rho^2 + \sigma^2)^2]
\end{align*}
However, if we take $v$ to be the classical equilibrium value of $v = \EV{\phi} = \sqrt{\frac{2 \kappa^2}{g}}$ then we have,
\begin{align*}
\lagrange = \tfrac{1}{2} \partial_\mu \rho \partial^\mu \rho + \tfrac{1}{2} \partial_\mu \sigma \partial^\mu \sigma + \frac{\kappa^4}{g} - \frac{1}{16} g[4\tilde{v}^2 \rho^2 + 4 \tilde{v} \rho (\rho^2 + \sigma^2) + (\rho^2 + \sigma^2)^2]
\end{align*}
which we rewrite as,
\[ \lagrange = \tfrac{1}{2} \partial_\mu \rho \partial^\mu \rho + \tfrac{1}{2} \partial_\mu \sigma \partial^\mu \sigma + \kappa^4 g^{-1} - \kappa^2 \rho^2 - \tfrac{1}{4} g \tilde{v} \rho (\rho^2 + \sigma^2) - \tfrac{1}{16} g (\rho^2 + \sigma^2)^2 \]
Therefore, we have the classical action in these new variables of the form,
\begin{align*}
S[\rho, \sigma] & = \int \dn{4}{x} \left[ \tfrac{1}{2} \partial_\mu \rho \partial^\mu \rho + \tfrac{1}{2} \partial_\mu \sigma \partial^\mu \sigma + \kappa^4 g^{-1} - \kappa^2 \rho^2 - \tfrac{1}{4} g \tilde{v} \rho (\rho^2 + \sigma^2) - \tfrac{1}{16} g (\rho^2 + \sigma^2)^2 \right] 
\end{align*}
which is the action for two real scalar fields where $\rho$ has mass $2 \kappa^2$ and $\sigma$ is mass-less with cubic and quartic interactions. 
\bigskip\\
The potential can be visualized as a circularly symmetric trough with a maximum in the center. When we break the symmetry by choosing a point along the circle of minima, we can decompose fluctuations about the minimum in the direction along the circular trough and perpendicular to it. In the direction around the circle of the trough the potential is constant so fluctuations in this direction are massless modes. However, perpendicular to the trough, the potential has positive second derivative so fluctuations in this direction correspond to a mode with positive effective mass.

\subsection{Scalar Bosons at One-Loop Level}

The one-loop correction to the quantum effective potential is given by the logarithmic determinant of the quadratic terms of the Lagrangian\footnote{We have not set $v$ to its classical value $v = \sqrt{\frac{2 \kappa}{g}}$ because the quantum effects with perturb this equilibrium position.} The Hessian of $V_0$ with respect to the fields $\rho$ and $\sigma$ is,
\begin{align*}
m_{ij} = \frac{\partial^2 V_0}{\partial \phi^i \partial \phi^j} = 
\begin{pmatrix}
- \kappa^2 + \tfrac{1}{4} g \left( 3 \tilde{v}^2 + 6 \rho \tilde{v} + 3 \rho^2 + \sigma^2 \right) & \tfrac{1}{2} g \sigma (\tilde{v} + \rho) 
\\
\tfrac{1}{2} g \sigma (\tilde{v} + \rho)  & - \kappa^2 + \tfrac{1}{4} g \left( \tilde{v}^2 + 2 \rho \tilde{v} + \rho^2 + 3 \sigma^2 \right)  
\end{pmatrix} 
\end{align*}
Therefore, the second-order Lagrangian has the form,
\begin{align*}
\frac{\delta^2 \lagrange}{\delta \phi^i \delta \phi^j} = 
\begin{pmatrix}
- \partial^2 + \kappa^2 - \tfrac{1}{4} g \left( 3 \tilde{v}^2 + 6 \rho \tilde{v} + 3 \rho^2 + \sigma^2 \right) & -\tfrac{1}{2} g \sigma (\tilde{v} + \rho) 
\\
- \tfrac{1}{2} g \sigma (\tilde{v} + \rho)  & - \partial^2 + \kappa^2 - \tfrac{1}{4} g \left( \tilde{v}^2 + 2 \rho \tilde{v} + \rho^2 + 3 \sigma^2 \right)  
\end{pmatrix} 
\end{align*}
We must diagonalize this matrix to find the mass eigenvalues which appear in the one-loop correction to the effective potential given by,
\[ V_1(\phi) = - \frac{i}{2} \log{\det{\left[ - \frac{\delta^2 \lagrange}{\delta \phi^i \delta \phi^j} \right]}} = - \frac{i}{2} \Tr{\log{\left[ - \frac{\delta^2 \lagrange}{\delta \phi^i \delta \phi^j} \right]}}
\]
The mass eigenvalues are,
\[ m_1^2 = \tfrac{1}{4} \left( - 4 \kappa^2 + g \rho^2 + g \rho^2 + 2 g \rho \tilde{v} + g \tilde{v}^2 \right) \quad \text{and} \quad m_2^2 = \tfrac{1}{4} \left( - 4 \kappa^2 + 3 g \rho^2 + 3 g \sigma^2 + 6 g \tilde{v} \rho + 3 g \tilde{v}^2  \right) \]
so the one-loop correction to the quantum effective potential becomes,
\begin{align*}
 V_1(\rho, \sigma) = & -\tfrac{i}{2} \log{\det{\left( \partial^2 + \tfrac{1}{4} \left( - 4 \kappa^2 + g \rho^2 + g \sigma^2 + 2 g \rho \tilde{v} + g \tilde{v}^2 \right)\right)}} 
\\
& - \tfrac{i}{2} \log{\det{\left(\partial^2  +\tfrac{1}{4} \left( - 4 \kappa^2 + 3 g \rho^2 + 3 g \sigma^2 + 6 g \tilde{v} \rho + 3 g \tilde{v}^2  \right) \right)}}
\end{align*}
We have already calculated such logarithmic determinants up to infinite constants which may be renormalized away. We found,
\[ -\tfrac{i}{2} \log{\det{\left(\partial^2 + \tilde{m}^2 \right)}} = \frac{1}{2} \int \frac{\dn{4}{k_E}}{(2\pi)^4} \log{\left( \frac{k_E^2 + \tilde{m}^2}{\Lambda_0^2} \right)}  = \frac{1}{64 \pi^2} \tilde{m}^4 \log{\left( \frac{\tilde{m}^2}{\Lambda_0^2}  \right)}  \]
therefore the one-loop correction to the quantum effective action becomes,
\begin{align*}
V_1(\rho, \sigma) & = \frac{1}{1024 \pi^2} \left[ \left( - 4 \kappa^2 + g \rho^2 + g \sigma^2 + 2 g \rho \tilde{v} + g \tilde{v}^2 \right)^2 \log{\left( \frac{\tfrac{1}{4} \left( - 4 \kappa^2 + g \rho^2 + g \sigma^2 + 2 g \rho \tilde{v} + g \tilde{v}^2 \right)}{\Lambda_0^2} \right)} 
\right. 
\\
& \quad \quad \quad  \left. + \left( - 4 \kappa^2 + 3 g \rho^2 + 3 g \sigma^2 + 6 g \tilde{v} \rho + 3 g \tilde{v}^2  \right)^2 \log{\left( \frac{\tfrac{1}{4} \left( - 4 \kappa^2 + 3 g \rho^2 + 3 g \sigma^2 + 6 g \tilde{v} \rho + 3 g \tilde{v}^2  \right)}{\Lambda_0^2} \right)} \right]
\end{align*}
The counter-terms have been employed in some variant of minimal subtraction to cancel the divergences in computing the functional determinants but not to fix a particularly chosen renormalization prescription with some physical interpretation. Therefore, the above couplings are not the physical coupling but will express the physical coupling when we expand the quantum effective potential in a power series. This is inconsequential to the current analysis since we will not be computing physical amplitudes. To simplify the above result I choose the reference scale,
\[ \Lambda_0 = \tfrac{1}{4} g \tilde{v}^2 - \kappa^2 \]
At the one-loop level of perturbation theory, we can calculate the derivatives of the full quantum effective potential about $\sigma = \rho = 0$ to determine the equilibrium field $v$ and determine the quantum corrections to the masses. 
\begin{align*}
V_q(\rho, \sigma) = V_0(\rho, \sigma) + \hbar V_1(\rho, \sigma) 
\end{align*}
We find that,
\[ \pderiv{V_q}{\rho} = \tilde{v}  \left\{ (\tfrac{1}{4} g \tilde{v}^2 - \kappa^2) + \frac{\hbar g}{128 \pi^2}  \left[ (\tfrac{1}{4} g \tilde{v}^2 - \kappa^2) + 3 (3 g \tilde{v}^2 - 4 \kappa^2) \left( 1 + 2 \log{\left( \frac{3 g \tilde{v}^2 - 4 \kappa^2}{g \tilde{v}^2 - 4 \kappa^2} \right)} \right) \right] \right\} \]
This expression has a zero for a value close to the classical vacuum expectation. Furthermore, we want to find the effective mass of the $\sigma$ mode at the one-loop level of perturbation theory. The effective mass is given by,
\begin{align*}
\frac{\partial^2 V_q}{\partial^2 \sigma} = \left\{ (\tfrac{1}{4} g \tilde{v}^2 - \kappa^2) + \frac{\hbar g}{128 \pi^2}  \left[ (\tfrac{1}{4} g \tilde{v}^2 - \kappa^2) + 3 (3 g \tilde{v}^2 - 4 \kappa^2) \left( 1 + 2 \log{\left( \frac{3 g \tilde{v}^2 - 4 \kappa^2}{g \tilde{v}^2 - 4 \kappa^2} \right)} \right) \right] \right\} 
\end{align*}  
which is exactly the expression we found above and thus has the same roots. Therefore, $\sigma$ is exactly massless to the one-loop level when expanded about the new minimum of the potential after corrected by loop effects. This miraculous ``coincidence'' is a manifestation of the maintained symmetry of the theory.

\subsection{Numbers of Goldstone Bosons}

Suppose we have some set of fields $\phi^a$ which form a vector space $V$ of field configurations and a symmetry Lie group $G$ acting on $V$. Take an action,
\[ S = \int \dn{4}{x} \left[ \tfrac{1}{2} \partial_\mu \phi^a \partial^\mu \phi_a - V_0(\phi) \right] \]
and consider its expansion about a minimum $\phi_0$ and let $\sigma^a = \phi^a - \phi^a_0$. Then we can write,
\[ V_0(\phi) = V_0(\phi_0) + \left( \pderiv{V_0}{\phi^a} \right)_{\phi_0} \sigma^a + \frac{1}{2} \left( \frac{\partial^2 V_0}{\partial \phi^a \partial \phi^b} \right)_{\phi_0} \sigma^a \sigma^b + \tilde{V}_{\phi_0}(\sigma) \]
where $\tilde{V}_{\phi_0}(\sigma)$ depends on $\sigma$ at third-order and above. However, $\phi_0$ is a minimum of the potential so,
\[ \left( \pderiv{V_0}{\phi^a} \right)_{\phi_0} = 0 \]
Therefore, the potential takes the form,
\[ V_0(\phi) = V_0(\phi_0) + \frac{1}{2} \left( \frac{\partial^2 V_0}{\partial \phi^a \partial \phi^b} \right)_{\phi_0} \sigma^a \sigma^b + \tilde{V}_{\phi_0}(\sigma) \]
which can be written as,
\[ V_0(\phi) = V_0(\phi_0) +  \frac{1}{2} m_{ab} \sigma^a \sigma^b + \tilde{V}_{\phi_0}(\sigma) \]
where,
\[ m_{ab} = \left( \frac{\partial^2 V_0}{\partial \phi^a \partial \phi^b} \right)_{\phi_0} \]
is the effective mass matrix. A massless $\phi$-mode i.e. a Goldstone boson is exactly an eigenvector with zero eigenvalue. Furthermore, we have the symmetry group $G$ acting on $V$ by $g \cdot \phi$. Since the theory is invariant under the action of $G$ the potential must be invariant,
\[ V_0(g \cdot \phi) = V_0 \implies \deriv{}{\lambda} V_0(g(\lambda) \cdot \phi) = 0 \]  
Therefore,
\[ \left(\deriv{g^a_c}{\lambda} \cdot \phi^c \right) \left( \pderiv{V_0}{\phi^a} \right) = 0 \]
Denote,
\[ 
\Delta(\phi)^a = \left(\deriv{g^a_c}{\lambda} \cdot \phi^c \right) 
\]
Then we differentiate again by $\phi^b$ and set $\phi = \phi_0$,
\[ \left( \pderiv{\Delta^a}{\phi^b} \right)_{\phi_0} \left( \pderiv{V_0}{\phi^a} \right)_{\phi_0} + \Delta^a(\phi_0) \left( \frac{\partial^2 V_0}{\partial \phi^a \partial \phi^b} \right)_{\phi_0} = 0 \]
However, since $\phi_0$ is a minimum,
\[ \left( \pderiv{V_0}{\phi^a} \right)_{\phi_0} = 0 \]
and therefore,
\[ \Delta^a(\phi_0) \left( \frac{\partial^2 V_0}{\partial \phi^a \partial \phi^b} \right)_{\phi_0} = 0 \]
Thus, $\Delta(\phi_0)$ is an eigenvector of $m_{ab}$ with eigenvalue zero exactly if $\Delta(\phi_0) \neq 0$ i.e. it is a symmetry of the potential which is not a symmetry of the symmetry breaking prescription $\phi = \phi_0$. Therefore, the number of Goldstone bosons is the number of \textit{broken} symmetries which is the dimension of $G$ minus the dimension of the stabilizer of $\phi_0$. 
\bigskip\\
Consider $\phi \in \C^N$ which is a multiplet of $N$ complex scalar fields with the fundamental action of $U(N)$ on $\C^N$ under which the potential $V_0(\phi) = g(||\phi||^2 - v^2)^2$ is invariant. We know that $\dim{U(N)} = N^2$ and the stabilizer is the symmetry group of the subspace perpendicular to the chosen symmetry breaking field $\phi_0$ (such that $||\phi_0||^2 = v^2$) which is $U(N-1)$. Therefore the number of Goldstone bosons is the number of broken symmetries i.e.
\[ N_G = \dim{U(N)} - \dim{U(N-1)} = N^2 - (N-1)^2 = 2 N - 1 \]
Likewise, consider an $N \times N$ real scalar multiplet $\Phi$ which transforms in the adjoint representation of $U(N)$. The potential $V_0(\Phi) = (\Tr{\Phi^2} - v^2)^2$ is invariant under this action. We need to find the number of broken symmetries which is equal to the dimension of the orbit of the symmetry broken equilibrium point which I can choose to be,
\[ 
\Phi_0
= 
\begin{pmatrix}
1 & 0 & \cdots & 0
\\
0 & 0 & \cdots & 0
\\
\vdots & \vdots & \ddots & \vdots
\\
0 & 0 & \cdots & 0
\end{pmatrix}
\]
We need to find the orbit of $\Phi_0 \mapsto U \Phi_0 U^{-1}$ where $U$ is unitary. 
I choose to write $U$ in block form as.
\[  U =
\begin{pmatrix}
\alpha & v_1^\dagger
\\
v_2 & A 
\end{pmatrix}
\]
where $v_1, v_2 \in \C^{N-1}$ and $A \in M_{N-1 \times N-1}$ is an $N-1$ by $N-1$ matrix. Since $U$ is unitary we must have,
\[ U^\dagger U =
\begin{pmatrix}
\alpha^* & v_2^\dagger 
\\
v_1 & A^\dagger
\end{pmatrix}
\begin{pmatrix}
\alpha & v_1^\dagger 
\\
v_2 & A
\end{pmatrix}
= 
\begin{pmatrix}
|\alpha|^2 + v_2^\dagger v_2 & \alpha^* v_1^\dagger + v_2^\dagger A 
\\
\alpha v_1 + A^\dagger v_2 & v_1 v_1^\dagger + A^\dagger A
\end{pmatrix} 
= I \]
Therefore,
\[ |\alpha|^2 + v_2^\dagger v_2 = 1 \quad \text{and} \quad \alpha^* v_1^\dagger + v_2^\dagger A = \alpha v_1 + A^\dagger v_2 = 0 \quad \text{and} \quad v_1 v_1^\dagger + A^\dagger A = I_{N-1} \]
Now consider,
\begin{align*}
U \Phi_0 U^{-1} & =
\begin{pmatrix}
\alpha & v_1^\dagger 
\\
v_2 & A
\end{pmatrix}
\begin{pmatrix}
1 & 0_{N-1}
\\
0_{N-1} & 0_{N-1 \times N-1}
\end{pmatrix}
\begin{pmatrix}
\alpha^* & v_2^\dagger 
\\
v_1 & A^\dagger
\end{pmatrix}
= 
\begin{pmatrix}
|\alpha|^2 & \alpha^* v_2^\dagger
\\
\alpha v_2 & v_2 v_2^\dagger 
\end{pmatrix}
\end{align*} 
This matrix is determined entirely by $\alpha \in \C$ and $v_2 \in \C^{N-1}$. Furthermore, $\alpha$ and $v_2$ satisfy the relation,
\[ |\alpha|^2 + v_2^\dagger v_2 = 1 \]
This constraint is equivalent to the sum of the square magnitudes being fixed to $1$. Therefore, $U \Phi_0 U^{-1}$ is determined exactly by a single vector lying on the unit sphere $S^{2N-1} \subset \C^{N}$. Thus, the orbit has dimension $2N - 1$ which is therefore equal to the number of Goldstone bosons.   
\end{document}