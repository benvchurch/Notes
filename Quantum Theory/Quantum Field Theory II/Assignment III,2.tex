\documentclass[12pt]{article}
\usepackage[english]{babel}
\usepackage[utf8]{inputenc}
\usepackage[english]{babel}
\usepackage[a4paper, total={7.25in, 9.5in}]{geometry}
\usepackage{tikz-feynman}
\tikzfeynmanset{compat=1.0.0} 
\usepackage{subcaption}
\usepackage{float}
\floatplacement{figure}{H}
\usepackage{simpler-wick}
\usepackage{mathrsfs}  
\usepackage{dsfont}
\usepackage{relsize}
\usepackage{tikz-cd}
\DeclareMathAlphabet{\mathdutchcal}{U}{dutchcal}{m}{n}

\usepackage{cancel}



\newcommand{\field}{\hat{\Phi}}
\newcommand{\dfield}{\hat{\Phi}^\dagger}
 
\usepackage{amsthm, amssymb, amsmath, centernot}
\usepackage{slashed}
\newcommand{\notimplies}{%
  \mathrel{{\ooalign{\hidewidth$\not\phantom{=}$\hidewidth\cr$\implies$}}}}
 
\renewcommand\qedsymbol{$\square$}
\newcommand{\cont}{$\boxtimes$}
\newcommand{\divides}{\mid}
\newcommand{\ndivides}{\centernot \mid}

\newcommand{\Integers}{\mathbb{Z}}
\newcommand{\Natural}{\mathbb{N}}
\newcommand{\Complex}{\mathbb{C}}
\newcommand{\Zplus}{\mathbb{Z}^{+}}
\newcommand{\Primes}{\mathbb{P}}
\newcommand{\Q}{\mathbb{Q}}
\newcommand{\R}{\mathbb{R}}
\newcommand{\ball}[2]{B_{#1} \! \left(#2 \right)}
\newcommand{\Rplus}{\mathbb{R}^+}
\renewcommand{\Re}[1]{\mathrm{Re}\left[ #1 \right]}
\renewcommand{\Im}[1]{\mathrm{Im}\left[ #1 \right]}
\newcommand{\Op}{\mathcal{O}}

\newcommand{\invI}[2]{#1^{-1} \left( #2 \right)}
\newcommand{\End}[1]{\text{End}\left( A \right)}
\newcommand{\legsym}[2]{\left(\frac{#1}{#2} \right)}
\renewcommand{\mod}[3]{\: #1 \equiv #2 \: \mathrm{mod} \: #3 \:}
\newcommand{\nmod}[3]{\: #1 \centernot \equiv #2 \: mod \: #3 \:}
\newcommand{\ndiv}{\hspace{-4pt}\not \divides \hspace{2pt}}
\newcommand{\finfield}[1]{\mathbb{F}_{#1}}
\newcommand{\finunits}[1]{\mathbb{F}_{#1}^{\times}}
\newcommand{\ord}[1]{\mathrm{ord}\! \left(#1 \right)}
\newcommand{\quadfield}[1]{\Q \small(\sqrt{#1} \small)}
\newcommand{\vspan}[1]{\mathrm{span}\! \left\{#1 \right\}}
\newcommand{\galgroup}[1]{Gal \small(#1 \small)}
\newcommand{\bra}[1]{\left| #1 \right>}
\newcommand{\Oa}{O_\alpha}
\newcommand{\Od}{O_\alpha^{\dagger}}
\newcommand{\Oap}{O_{\alpha '}}
\newcommand{\Odp}{O_{\alpha '}^{\dagger}}
\newcommand{\im}[1]{\mathrm{im} \: #1}
\renewcommand{\ker}[1]{\mathrm{ker} \: #1}
\newcommand{\ket}[1]{\left| #1 \right>}
\renewcommand{\bra}[1]{\left< #1 \right|}
\newcommand{\inner}[2]{\left< #1 | #2 \right>}
\newcommand{\expect}[2]{\left< #1 \right| #2 \left| #1 \right>}
\renewcommand{\d}[1]{ \mathrm{d}#1 \:}
\newcommand{\dn}[2]{ \mathrm{d}^{#1} #2 \:}
\newcommand{\deriv}[2]{\frac{\d{#1}}{\d{#2}}}
\newcommand{\nderiv}[3]{\frac{\dn{#1}{#2}}{\d{#3^{#1}}}}
\newcommand{\pderiv}[2]{\frac{\partial{#1}}{\partial{#2}}}
\newcommand{\fderiv}[2]{\frac{\delta #1}{\delta #2}}
\newcommand{\parsq}[2]{\frac{\partial^2{#1}}{\partial{#2}^2}}
\newcommand{\topo}{\mathcal{T}}
\newcommand{\base}{\mathcal{B}}
\renewcommand{\bf}[1]{\mathbf{#1}}
\renewcommand{\a}{\hat{a}}
\newcommand{\adag}{\hat{a}^\dagger}
\renewcommand{\b}{\hat{b}}
\newcommand{\bdag}{\hat{b}^\dagger}
\renewcommand{\c}{\hat{c}}
\newcommand{\cdag}{\hat{c}^\dagger}
\newcommand{\hamilt}{\hat{H}}
\renewcommand{\L}{\hat{L}}
\newcommand{\Lz}{\hat{L}_z}
\newcommand{\Lsquared}{\hat{L}^2}
\renewcommand{\S}{\hat{S}}
\renewcommand{\empty}{\varnothing}
\newcommand{\J}{\hat{J}}
\newcommand{\lagrange}{\mathcal{L}}
\newcommand{\dfourx}{\mathrm{d}^4x}
\newcommand{\meson}{\phi}
\newcommand{\dpsi}{\psi^\dagger}
\newcommand{\ipic}{\mathrm{int}}
\newcommand{\tr}[1]{\mathrm{tr} \left( #1 \right)}
\newcommand{\C}{\mathbb{C}}
\newcommand{\CP}[1]{\mathbb{CP}^{#1}}
\newcommand{\Vol}[1]{\mathrm{Vol}\left(#1\right)}

\newcommand{\Tr}[1]{\mathrm{Tr}\left( #1 \right)}
\newcommand{\Charge}{\hat{\mathbf{C}}}
\newcommand{\Parity}{\hat{\mathbf{P}}}
\newcommand{\Time}{\hat{\mathbf{T}}}
\newcommand{\Torder}[1]{\mathbf{T}\left[ #1 \right]}
\newcommand{\Norder}[1]{\mathbf{N}\left[ #1 \right]}
\newcommand{\Znorm}{\mathcal{Z}}
\newcommand{\EV}[1]{\left< #1 \right>}
\newcommand{\interact}{\mathrm{int}}
\newcommand{\covD}{\mathcal{D}}
\newcommand{\conj}[1]{\overline{#1}}

\newcommand{\SO}[2]{\mathrm{SO}(#1, #2)}
\newcommand{\SU}[2]{\mathrm{SU}(#1, #2)}

\newcommand{\anticom}[2]{\left\{ #1 , #2 \right\}}


\newcommand{\pathd}[1]{\! \mathdutchcal{D} #1 \:}

\renewcommand{\theenumi}{(\alph{enumi})}


\renewcommand{\theenumi}{(\alph{enumi})}

\newcommand{\atitle}[1]{\title{% 
	\large \textbf{Physics GR8048 Quantum Field Theory II
	\\ Assignment \# #1} \vspace{-2ex}}
\author{Benjamin Church }
\maketitle}

\newcommand{\atitleIII}[1]{\title{% 
	\large \textbf{Physics GR8049 Quantum Field Theory III
	\\ Assignment \# #1} \vspace{-2ex}}
\author{Benjamin Church }
\maketitle}

\theoremstyle{definition}
\newtheorem{theorem}{Theorem}[section]
\newtheorem{definition}{definition}[section]
\newtheorem{lemma}[theorem]{Lemma}
\newtheorem{proposition}[theorem]{Proposition}
\newtheorem{corollary}[theorem]{Corollary}
\newtheorem{example}[theorem]{Example}
\newtheorem{remark}[theorem]{Remark}

\begin{document}

\atitleIII{1}

\section{Unitarity Bound}

\subsection{}

Consider the conformal algebra with generators $D$, $K_\mu$, $P_\mu$, and $M_{\mu \nu}$ satisfying the standard (Euclidean) commutation relations of which I only write down the relevant ones,
\begin{align*}
[ D, P_\mu] & = P_\mu 
\\
[ D, K_\mu] & = - K_\mu
\\
[ K_\mu, P_\nu] & = 2 \left( \delta_{\mu \nu} D -  M_{\mu \nu} \right)
\\
[M_{\mu \nu}, P_\rho] & =  
\delta_{\nu \rho} P_\mu  - \delta_{\mu \rho} P_\nu 
\\
[M_{\mu \nu}, K_{\rho}] & =  \delta_{\nu \rho} K_\mu  - \delta_{\mu \rho} K_\nu  
\end{align*}
Let $\ket{\Delta} = \Op_\Delta(0) \ket{\Omega}$ be a spinless primary state corresponding to the spinles primary operator $\Op_\Delta(x)$ with scaling dimension $\Delta$ i.e. $M_{\mu \nu} \ket{\Delta} = 0$ and $K_\mu \ket{\Delta} = 0$. This state has scaling dimension $\Delta$ i.e. $D \ket{\Delta} = \Delta \ket{\Delta}$. Then consider the combination,
\begin{align*}
\bra{\Delta} K_\mu K_{\nu} P_{\rho} P_{\sigma} \ket{\Delta} & = \bra{\Delta} K_\mu [ K_{\nu}, P_{\rho}] P_{\sigma} \ket{\Delta} + \bra{\Delta} K_\mu  P_{\rho} K_{\nu} P_{\sigma} \ket{\Delta}
\\
& = \bra{\Delta} K_\mu 2 (\delta_{\nu \rho} D - M_{\nu \rho}) P_{\sigma} \ket{\Delta} + \bra{\Delta} K_\mu  P_{\rho} [K_{\nu}, P_{\sigma}]\ket{\Delta} + \bra{\Delta} K_\mu  P_{\rho}  P_{\sigma}  K_{\nu}\ket{\Delta}
\end{align*}
Where I have used the fact that $K_\nu \ket{\Delta} = 0$ since $\ket{\Delta}$ is primary.
Then we have,
\begin{align*}
\bra{\Delta} K_\mu K_{\nu} P_{\rho} P_{\sigma} \ket{\Delta} 
& = \bra{\Delta} K_\mu 2 (\delta_{\nu \rho} D - M_{\nu \rho}) P_{\sigma} \ket{\Delta} + \bra{\Delta} K_\mu  P_{\rho} 2 (\delta_{\mu \sigma} D - M_{\mu \sigma} ) \ket{\Delta} 
\\
& = \bra{\Delta} K_\mu 2 [\delta_{\nu \rho} D - M_{\nu \rho}, P_{\sigma}] \ket{\Delta} + \bra{\Delta} K_\mu P_{\sigma} 2 (\delta_{\nu \rho} D - M_{\nu \rho})  \ket{\Delta} + \bra{\Delta} K_\mu  P_{\rho} 2 \delta_{\mu \sigma} \Delta \ket{\Delta} 
\\
& = 2 \bra{\Delta} K_\mu (\delta_{\nu \rho} P_\sigma - (\delta_{\rho \sigma} P_\nu - \delta_{\nu \sigma} P_\rho))  \ket{\Delta}  + 2 (\delta_{\mu \sigma} + \delta_{\nu \sigma}) \Delta \bra{\Delta} K_\mu  P_{\rho}  \ket{\Delta} 
\end{align*}
Therefore, taking traces,
\begin{align*}
\bra{\Delta} K_\mu K^\mu P_{\rho} P^{\rho} \ket{\Delta} 
& = 2 \bra{\Delta} K_\mu (\delta^{\mu \rho} P_\rho - (d \delta^{\mu \nu} P_\nu - \delta^{\mu \rho} P_\rho))  \ket{\Delta}  + 4 \delta^{\mu \rho} \Delta \bra{\Delta} K_\mu  P_{\rho}  \ket{\Delta}
\\
& = 2 (2 - d + 2 \Delta) \bra{\Delta} K^\mu  P_{\mu}  \ket{\Delta}
\end{align*}
However, in radial quantization we have the relation $P_\mu^\dagger = K_\mu$ and thus,
\[ | P_\mu P^\mu \ket{\Delta}|^2 = \bra{\Delta} P^\dagger_\mu (P^\mu)^\dagger P_\rho P^\rho \ker{\Delta} = \bra{\Delta} K_\mu K^\mu P_\rho P^\rho \ker{\Delta} \ge 0 \]
Furthermore,
\[ \bra{\Delta} K^\mu P_\mu \ket{\Delta} = \delta^{\mu \rho} \bra{\Delta} P^\dagger_\rho P_\mu \ket{\Delta} = \sum_{\mu} |P_\mu \ket{\Delta}|^2 \ge 0 \]
Since both sides of the derived expression must be positive and $\bra{\Delta} K_\mu  P_{\rho}  \ket{\Delta} \ge 0$ we find that,
\[ 2 - d + 2 \Delta \ge 0 \implies \Delta \ge \frac{d - 2}{2} \]
giving the unitarity bound in arbitrary dimensions. 

\subsection{}

Using the scalar equations of motion, we find that, for a scalar of mass $m$, the primary scaling dimension must satisfy,
\[ \Delta(\Delta - d) = m^2 \]
which has solutions,
\[ \Delta_{\pm} = \frac{d}{2} \pm \sqrt{m^2 + \left( \frac{d}{2} \right)^2} \] 
Since the modes have asymptotic behavior $r^{-\Delta}$ basic conditions on normalizability imply that $\Delta > 0$. The range in which normalization issue do not arise for either choice of $\Delta$ occurs when both $\Delta_{+}$ and $\Delta_{-}$ are positive. Therefore, we must have,
\[ \frac{d}{2} > \sqrt{m^2 + \left( \frac{d}{2} \right)^2} \implies m^2 < 0 \]
However, the scaling dimension $\Delta$ cannot become complex since this will violate the Hermiticity of $D$ whose eigenvalues are $\Delta$. Therefore, we are limited by the minimum value of the quadratic $\Delta (\Delta - d)$ over the reals which occurs at $\Delta = \frac{d}{2}$ and gives,
\[ m^2 \ge - \left( \frac{d}{2} \right)^2 \]
Therefore, the consistent range of masses for two choices of boundary conditions is,
\begin{align*}
- \left( \frac{d}{2} \right)^2 \le m^2 \le 0 
\end{align*}

\subsection{}

Suppose we choose to quantize both modes with ``standard'' and ``alternate'' boundary conditions simultaneously. Then we have a mode expansion containing both,
\[ \phi_{+}(t,x) = \frac{e^{- i \Delta_{+} t}}{(1 + r^2)^{\Delta_{+}/2}} \quad \text{and} \quad \phi_{-}(t,x) = \frac{e^{- i \Delta_{-} t}}{(1 + r^2)^{\Delta_{-}/2}} \]
However, we quantize with respect to the Klien-Gordon inner product,
\[ (\phi_1, \phi_2) = - i \int \sqrt{-g} g^{tt} \dn{3}{x} \left( \phi_1^* \partial_t \phi_2 - \phi_2 \partial_t \phi_1^* \right) \]
However, applying this inner product to the plus and minus modes we find,
\begin{align*} 
(\phi_+, \phi_-) 
& = - i \int \sqrt{-g} g^{tt} \dn{3}{x} \left( \phi_{+}^* \partial_t \phi_{-} - \phi_{-} \partial_t \phi_{+}^* \right)
\\
& = - \int \sqrt{-g} g^{tt} \dn{3}{x} (1 + r^2)^{-\frac{\Delta_{+} + \Delta_{-}}{2}} \left( e^{i (\Delta_{+} - \Delta_{-} ) t} \Delta_{-} + e^{i (\Delta_{+} - \Delta_{-} ) t} \Delta_{+} \right)
\\
& = - \left(  \Delta_{-} +  \Delta_{+} \right) e^{i (\Delta_{+} - \Delta_{-} ) t} \int \sqrt{-g} g^{tt} \dn{3}{x} (1 + r^2)^{-\frac{\Delta_{+} + \Delta_{-}}{2}} 
\end{align*}
which is both nonzero and time-dependent. Since $\phi_{+}$ and $\phi_{-}$ are supposed to be energy modes with different energy eigenvalues, namely $\Delta_{+}$ and $\Delta_{-}$ respectively, if $\hat{H}$ is Hermitian then they must be orthogonal. However, since they are not in fact orthogonal, this quantization prescription violates the Hermiticity of $\hat{H}$ and thus the unitarity of the time evolution operator $e^{- i \hat{H} t}$. More directly, unitary time evolution implies that inner products are invariant under time translation and therefore cannot be time-dependent as we observed the inner product $(\phi_+, \phi_-)$ to be. 

\section{Boundary Correlators of EAdS and Conformal Symmetry}

\subsection{}

The boundary two-point function has the expression,
\[ \EV{\Op(\vec{u}) \Op(\vec{u}')} = \frac{c_\Delta}{|\vec{u} - \vec{u}'|^{2 \Delta}} \]
where the constant $c_{\Delta}$ is determined by matching the propagator in Minkowski space in the short distance limit to be,
\[ c_\Delta = \frac{\Gamma(\Delta)}{2 \pi^{d / 2} \Gamma(\Delta - 1 - \tfrac{d}{2})} \]
We have show that there is a unitarity bound,
\[ \Delta \ge \frac{d - 2}{2} \]
When $\Delta$ crosses this threshold, the argument of the denominator Gamma function becomes negative. Furthermore, $\Gamma(x)$ as $x$ crosses zero diverges to $+ \infty$ and then comes back from $-\infty$ as $x$ becomes negative. Thus $\Gamma(x)$ crosses from being positive to negative as its argument does. This is more clearly summarized by taylor expanding,
\[ \frac{1}{\Gamma(x)} = x + O(x^2) \]
Thus, we have, about $\epsilon = 0$ where $\epsilon = \Delta - \frac{d-2}{2}$ we have,
\[ c_\Delta = \frac{\Gamma\left( \tfrac{d-2}{2} \right)}{2 \pi^{d/2}} \epsilon + O(\epsilon^2) \]
Thus as we lower $\Delta$ past the unitarity bound $\frac{d-2}{2}$ (i.e. take $\epsilon$ negative) the coefficient $c_\Delta$ changes sign (passing though zero) from positive to negative and, since all other terms remain well-defined in this transition, the two-point function must also become negative. Intreguingly, exactly at the unitarity bound $\Delta = \frac{d-2}{2}$ the constant $c_{\Delta}$ vanishes and thus,
\[ \EV{\Op(\vec{u}) \Op(\vec{u}')} = 0 \]

\subsection{}

First, consider EAdS in Euclidean global coordinates,
\[ X^0 = \sqrt{1 + r^2} \cosh{t} \quad \quad X^d = \sqrt{1 + r^2} \sinh{t} \quad \quad X^i = x^i \]
in which the metric becomes,
\[ \d{s^2} = (1 + r^2) \d{t^2} + \frac{\d{r^2}}{1 + r^2} + r^2 \d{\Omega^2} \]
In Euclidean EAdS the length inner product becomes,
\[ P = X^I Y_I = - X^0 Y^0 + X^I Y^I = -\sqrt{(1 + r^2)(1 + r'^2)}  \cosh{(t - t')} + r r' \cos{\eta} \]
where $\eta$ is the angle between $x^i$ and $y^i$ on the boundary sphere.  
We have shown that the two-point function in the large-distance limit becomes,
\[ G_\Delta \to c_\Delta (-2 p)^{-\Delta} \]
Therefore, setting $r = r'$ and taking the limit $r \to \infty$ gives,
\[ P = r^2 \left( -\cosh{(t - t')} + \cos{\eta} \right) \]
Thus, define the boundary operator via,
\begin{align*}
\Op(t, \Omega) = \lim_{r \to \infty} r^{\Delta} \phi(r, t, \Omega)
\end{align*}
such that the boundary two-point function becomes,
\begin{align*}
\EV{\Op(t, \Omega) \Op(t', \Omega')} = \lim_{r \to \infty} r^{2\Delta} \EV{ \phi(r, t, \Omega) \phi(r, t', \Omega')} & = \lim_{r \to \infty} \frac{r^{2\Delta} c_\Delta}{2^{\Delta} r^{2 \Delta} (\cosh{(t - t')} - \cos{\eta})^{\Delta}}
\\
& = \frac{c_\Delta}{2^\Delta (\cosh{(t - t')} - \cos{\eta})^{\Delta}} 
\end{align*}
Second, consider EAdS in Euclidean hyperbolic ball coordinates,
\[ X^I = \frac{2y^I}{1 - |y|^2} \quad \quad \quad X^0 = \frac{1 + |y|^2}{1 - |y|^2}  \]
where I restrict,
\[ |y|^2 = \sum_{I = 1}^{d + 1} (y^I)^2 < 1 \]
In these coordinates, the ambiant coordinate dot product becomes,
\begin{align*}
P = X^I X_I = -\frac{1 + |y|^2}{1 - |y|^2} \frac{1 + |y'|^2}{1 - |y'|^2}  + \frac{4 y^I y'_I}{(1 - |y|^2) (1 - |y'|^2)} 
\end{align*}
In the case that $|y| = |y'|$ (preparing to take the limits of both to the boundary). Then we have,
\begin{align*}
P = - \left( \frac{1 + |y|^2}{1 - |y|^2} \right)^2 + \frac{4 y^I y'_I}{(1 - |y'|^2)} = - \frac{1 +  2|y|^2 - 4 y^I y'_I + |y|^4}{(1 - |y|^2)^2} = - 1 + \frac{4 (y^I y'_I - |y|^2)}{(1 - |y|^2)^2}
\end{align*}
Thus, in the limit $|y|^2 \to 1$ we find,
\[ P = \frac{4 (y^I y'_I - |y|^2)}{(1 - |y|^2)^2} \]
Thus, define the boundary operator via,
\begin{align*}
\Op(y^I) = \lim_{|y| \to 1} (1 - |y|^2)^{-\Delta} \phi(y^I)
\end{align*}
such that the boundary two-point function becomes,
\begin{align*}
\EV{\Op(y^I) \Op(y'^I)} = \lim_{|y| \to 1} (1 - |y|^2)^{-2 \Delta} \EV{ \phi(y^I) \phi(y'^I)} & = \lim_{|y| \to 1} (1 - |y|^2)^{-2 \Delta} \frac{(1 - |y|^2)^{2 \Delta} c_\Delta}{8^{\Delta} (1 - y^I y'^I)^{\Delta}}
\\
& = \frac{c_\Delta}{8^\Delta (1 - y^I y'_I)^{\Delta}}
\end{align*}

\subsection{}

In Euclidean Poincare coodinates, consider a change of coordinates $z = \lambda \tilde{z}$ and $\vec{u} = \lambda \vec{\tilde{u}}$ with $\lambda > 0$ some constant. Then the bulk metric becomes,
\[ \d{s^2} = \frac{\d{\vec{u}^{\, 2}} + \d{z^2}}{z^2} = \frac{\d{\vec{\tilde{u}}^{\, 2}} + \d{\tilde{z}^2}}{\tilde{z}^2} \]
and thus this scaling is an isometry. In these new coordinates, the boundary metric and operators are defined via,
\newcommand{\vt}[1]{\vec{\tilde{#1}}}
\newcommand{\vts}[1]{\vec{\tilde{#1}}^{\, 2}}
\begin{align*}
\d{\tilde{\sigma}^2} & = \lim_{\tilde{z} \to 0} \tilde{z}^2 \d{s^2}|_{\tilde{z}} = \d{\vec{\tilde{u}}^{\, 2}}
\\
\tilde{\Op}(\vts{u}) & = \lim_{\tilde{z} \to 0} \tilde{z}^{-\Delta} \tilde{\phi}(\tilde{z}, \vt{u}) = \lim_{\tilde{z} \to 0} (z / \lambda)^{-\Delta} \phi(z, u) = \lambda^\Delta \Op(\vec{u}) = \lambda^\Delta \Op(\lambda \vt{u})
\end{align*}
Now consider the two-point function expressed in terms of these new boundary operators in the new coordinates.
\begin{align*}
\EV{\tilde{\Op}(\vt{u}) \tilde{\Op}(\vt{u}')} = \EV{\lambda^\Delta \Op(\lambda \vt{u}) \lambda^{\Delta} \Op(\lambda \vt{u}')} = \lambda^{2 \Delta} \EV{\Op(\lambda \vt{u})  \Op(\lambda \vt{u}')} = \frac{\lambda^{2 \Delta} c_\Delta}{|\lambda^2 \vt{v} - \lambda^2 \vt{v}'|^2} = \frac{c_\Delta}{|\vt{u} - \vt{u}'|^2}
\end{align*}
Therefore, the two-point function of the new boundary operators in the new boundary coordinates takes exactly the same form as the old operators in the old coordinates.

\subsection{}

In Poincare coordinates, consider a coordinate transformation $z = Z(\tilde{z}, \tilde{u})$ and $u^i = U^i(\tilde{z}, \tilde{u})$ such that we preserve the metric,
\[ \d{s^2} = \frac{(\d{u^2} + \d{z^2})}{z^2} = \frac{(\d{\tilde{u}^2} + \d{\tilde{z}^2})}{\tilde{z}^2} \]
We care about the metric near the boundary so we taylor expand about $\tilde{z} 0$. Since such isometries must preserve the boundary, they map $z = 0$ to $\tilde{z} = 0$. Therefore we may expand,
\[ z = (\partial_{\tilde{z}} Z(\tilde{z}, \tilde{u}) |_{\tilde{z} = 0}) \: \tilde{z} + O(\tilde{z}^2) \]
Let $\gamma(\tilde{u}) = \partial_{\tilde{z}} Z(\tilde{z}, \tilde{u}) |_{\tilde{z} = 0}$. Then plugging in with $\d{\tilde{z}} = 0$ we find,
\[ \d{s^2} = \frac{\d{u^2} + \d{\gamma(\tilde{u})^2 \tilde{z}^2} + O(\tilde{z})}{\gamma(\tilde{u})^2 \tilde{z}^2 + O(\tilde{z}^3)} = \frac{\d{u^2} + \d{\gamma(\tilde{u})^2} \tilde{z}^2}{\gamma(\tilde{u})^2 \tilde{z}^2} + O(\tilde{z}^{-1}) = \frac{\d{u^2}}{\gamma(\tilde{u})^2 \tilde{z}^2} + O(1)  \] 
Furthermore, using the fact that this transformation preserves the metric we find (recalling that we set $\d{\tilde{z}} = 0$),
\[ \d{s^2} = \frac{\d{\tilde{u}^2}}{\tilde{z}^2} = \frac{\d{u^2}}{\gamma(\tilde{u})^2 \tilde{z}^2} + O(1)  \]
and thus,
\[ \gamma(\tilde{u})^2 \d{\tilde{u}^2} = \d{u^2} + O(\tilde{z}^2) \]
and therefore, in the limit $z \to 0$ we find,
\[ \gamma(\tilde{u})^2 \d{\tilde{u}^2} = \d{u^2} \]
or equivalently,
\[ \d{\tilde{\sigma}^2} = \lim_{\tilde{z} \to 0} \tilde{z}^2 \d{s^2} = \lim_{\tilde{z} \to 0} \d{\tilde{u}^2} = \lim_{\tilde{z} \to 0} \left[ \frac{\d{u^2}}{\gamma(\tilde{u})^2} + O(\tilde{z}^2) \right] = \frac{\d{u^2}}{\gamma(\tilde{u})^2} = \frac{\d{\sigma^2}}{\gamma(\tilde{u}^2)} \] 
Therefore the conformal factor is 
\[ \rho(\tilde{u}) = \gamma(\tilde{u}) = (\partial_{\tilde{z}} Z(\tilde{z}, \tilde{u}) |_{\tilde{z} = 0}) \]
Furthermore, the boundary operators can be written as,
\begin{align*}
\Op(u) & = \lim_{z \to 0} z^{-\Delta} \phi(z, u) = \lim_{z \to 0} (z / \tilde{z})^{-\Delta} \tilde{z}^{-\Delta} \tilde{\phi}(\tilde{z}, \tilde{u}) = \lim_{z \to 0} (Z(\tilde{z}, \tilde{u}) / \tilde{z})^{-\Delta} \lim_{z \to 0} \tilde{z}^{-\Delta} \tilde{\phi}(\tilde{z}, \tilde{u}) 
\\
& = (\partial_{\tilde{z}} Z(\tilde{z}, \tilde{u})|_{\tilde{z} = 0})^{-\Delta} \tilde{\Op}(\tilde{u}) = \gamma(\tilde{u})^{- \Delta} \tilde{\Op}(\tilde{u})
\end{align*}
Therefore,
\[ \tilde{\Op}(\tilde{u}) = \gamma(\tilde{u})^{\Delta} \Op(U(\tilde{z}, \tilde{u})) \]

\subsection{}

Consider the coordinate transformation from Euclidean Poincare coordinates to Euclidean global coordinates given by,
\[ t = \log{\sqrt{\vec{u}^{\, 2} + z^2}} \xrightarrow{z \to 0} \log{|u|} \quad \quad \quad x^i = \frac{u^i}{z} \]
Now consider the tranformation between boundary operators,
\begin{align*}
\Op_{\text{cyl}}(t, \Omega) = \lim_{r \to \infty} r^{\Delta} \phi(r, t, \Omega) = \lim_{z \to 0} \frac{|u|^\Delta}{z^\Delta} \phi(z, u) = |u|^{\Delta} \lim_{z \to 0} z^{-\Delta} \phi(z, u) = |u|^{\Delta} \Op_{\text{pl}}(u)
\end{align*} 
Using the inverse transformation,
\begin{align*}
z = \frac{e^{t}}{\sqrt{1 + r^2}} \quad \quad \vec{u} = \frac{\vec{x} e^t}{\sqrt{1 + r^2}} \xrightarrow{r \to \infty} \vec{\Omega} e^t 
\end{align*}
we can also go the opposite direction to find,
\begin{align*}
\Op_{\text{pl}}(u) = \lim_{z \to 0} z^{-\Delta} \phi(z, u) = \lim_{r \to \infty} e^{-t \Delta} (1 + r^2)^{\Delta / 2} \phi(r, t, \Omega) = e^{-t \Delta} \lim_{r \to \infty}  r^{\Delta} \phi(r, t, \Omega) = e^{- t \Delta} \Op_{\text{cyl}}(t, \Omega)
\end{align*}
Furthermore, these are compatible because $e^t = |u|$. Now consider the conformal two-point function,
\begin{align*}
\EV{\Op_{\text{cyl}}(t, \Omega) \Op_{\text{cyl}}(t', \Omega')} & = (|u| |u'|)^{\Delta} \EV{\Op_{\text{pl}}(u) \Op_{\text{pl}}(u')} = \frac{c_\Delta (|u| |u'|)^{\Delta}}{|\vec{u} - \vec{u}'|^{2\Delta}}
\\
& = \frac{c_\Delta e^{(t + t') \Delta}}{|\vec{\Omega} e^t - \vec{\Omega}' e^{t'}|^{2\Delta}} = \frac{c_{\Delta}}{\left|\vec{\Omega} e^{\tfrac{1}{2}(t - t') \Delta} - \vec{\Omega}' e^{\tfrac{1}{2}(t' - t) \Delta} \right|^{2 \Delta}}
\end{align*}
However,
\begin{align*}
\left|\vec{\Omega} e^{\tfrac{1}{2}(t - t') \Delta} - \vec{\Omega}' e^{\tfrac{1}{2}(t' - t) \Delta} \right|^2 & = |\vec{\Omega}|^2 e^{(t - t')\Delta} + |\vec{\Omega}'|^2 e^{(t' - t)\Delta} - 2 \vec{\Omega} \cdot \vec{\Omega}'
\\
& = (e^{(t - t') \Delta} + e^{(t' - t)}) - 2 \cos{\eta} = 2 \left( \cosh{(t - t')} - \cos{\eta} \right)
\end{align*}
Therefore,
\[ \EV{\Op_{\text{cyl}}(t, \Omega) \Op_{\text{cyl}}(t', \Omega')} = \frac{c_\Delta}{2^{\Delta} \left( \cosh{(t - t')} - \cos{\eta} \right)^\Delta} \]
which is exactly the two-point function we computed earlier in Euclidean global coordinates. Likewise, we may check this result using the forward transformation,
\begin{align*}
\EV{\Op_{\text{pl}}(u) \Op_{\text{pl}}(u')} & = (|u| |u'|)^{-\Delta} \EV{\Op_{\text{cyl}}(t, \Omega) \Op_{\text{cyl}}(t', \Omega')} = \frac{c_\Delta (|u| |u'|)^{-\Delta}}{2^{\Delta}  \left( \cosh{(t - t')} - \cos{\eta} \right)^\Delta}
\\
& = \frac{c_\Delta (|u| |u'|)^{-\Delta}}{2^{\Delta}  \left( \cosh{(\log{|u|} - \log{|u'|})} - (|u| |u'|)^{-1} \vec{u} \cdot \vec{u}' \right)^\Delta}
\\
& = \frac{c_\Delta (|u| |u'|)^{-\Delta}}{ \left( \frac{|u|}{|u'|} + \frac{|u'|}{|u|} - 2 (|u| |u'|)^{-1} \vec{u} \cdot \vec{u}' \right)^\Delta} = \frac{c_\Delta}{ \left( |u|^2 + |u'|^2 - 2  \vec{u} \cdot \vec{u}' \right)^\Delta}
\\
& = \frac{c_{\Delta}}{|\vec{u} - \vec{u}'|^{2 \Delta}} 
\end{align*}
which is exactly the conformal two-point function in Poincare coordinates. 

\subsection{}

Consider the Eulcidean generators $M_{ij}$, $P_i$, $K_i$ and $D$ which are given in Poincare coordinates as,
\begin{align*}
M_{ij} & = - i (u^i \partial_{u^j} - u^j \partial_{u^i}) 
\\
D & = - i ( u^i \partial_{u^i} + z \partial_z) 
\\
P_i & = - i \partial_{u^i}
\\
K_i & = - i(u^2 + z^2) \partial_{u^i} + 2 i u^i (u^j \partial_{u^j} + z \partial_z) 
\end{align*}
Such transformations act on the conformal boundary operators under a transformation $\delta \phi = i \epsilon G \phi$ via,
\begin{align*}
\delta \Op = \lim_{z \to 0} z^{-\Delta} \delta \phi = i \epsilon \lim_{z \to 0} z^{-\Delta} G \phi = i \epsilon \left( G \lim z^{-\Delta} \phi - \lim_{z \to 0} [G, z^{-\Delta}] \phi \right) = \mathcal{G} \Op
\end{align*}
Therefore, first we need to compute the commutators,
\begin{align*}
[M_{ij}, z^{-\Delta}] & = 0
\\
[D, z^{-\Delta}] & = i \Delta z^{-\Delta}
\\
[P_i, z^{-\Delta}] & = 0
\\
[K_i, z^{-\Delta}] & = - 2 i u^i \Delta z^{-\Delta}
\end{align*}
Therefore, we can write,
\[ [G, z^{-\Delta}] = H e^{-\Delta} \]
for some function $H$ simplifying,
\[ \delta \Op = i \epsilon \left( G \lim_{z \to 0} z^{-\Delta} \phi - \lim_{z \to 0} H z^{-\Delta} \phi \right) = i \epsilon \left( G \Op - H \Op \right) \]
Therefore, plugging in for the given generators,
\begin{align*}
M_{ij} & \implies \delta \Op = \epsilon \left( u^i \partial_{u^j} - u^j \partial_{u^i} \right) \Op(u)
\\
D & \implies \delta \Op = \epsilon \left( u^i \partial_{u^i} + \Delta \right) \Op
\\
P_i & \implies \delta \Op = \epsilon \partial_{u^i} \Op
\\
K_i & \implies \delta \Op = \epsilon \left( u^2 \partial_{u^i} - 2 u^i u^j \partial_{u^j} - 2 u^i \Delta  \right) \Op
\end{align*}


\end{document}