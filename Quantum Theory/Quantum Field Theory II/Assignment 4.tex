\documentclass[12pt]{article}
\usepackage[english]{babel}
\usepackage[utf8]{inputenc}
\usepackage[english]{babel}
\usepackage[a4paper, total={7.25in, 9.5in}]{geometry}
\usepackage{tikz-feynman}
\tikzfeynmanset{compat=1.0.0} 
\usepackage{subcaption}
\usepackage{float}
\floatplacement{figure}{H}
\usepackage{simpler-wick}
\usepackage{mathrsfs}  
\usepackage{dsfont}
\usepackage{relsize}
\usepackage{tikz-cd}
\DeclareMathAlphabet{\mathdutchcal}{U}{dutchcal}{m}{n}

\usepackage{cancel}



\newcommand{\field}{\hat{\Phi}}
\newcommand{\dfield}{\hat{\Phi}^\dagger}
 
\usepackage{amsthm, amssymb, amsmath, centernot}
\usepackage{slashed}
\newcommand{\notimplies}{%
  \mathrel{{\ooalign{\hidewidth$\not\phantom{=}$\hidewidth\cr$\implies$}}}}
 
\renewcommand\qedsymbol{$\square$}
\newcommand{\cont}{$\boxtimes$}
\newcommand{\divides}{\mid}
\newcommand{\ndivides}{\centernot \mid}

\newcommand{\Integers}{\mathbb{Z}}
\newcommand{\Natural}{\mathbb{N}}
\newcommand{\Complex}{\mathbb{C}}
\newcommand{\Zplus}{\mathbb{Z}^{+}}
\newcommand{\Primes}{\mathbb{P}}
\newcommand{\Q}{\mathbb{Q}}
\newcommand{\R}{\mathbb{R}}
\newcommand{\ball}[2]{B_{#1} \! \left(#2 \right)}
\newcommand{\Rplus}{\mathbb{R}^+}
\renewcommand{\Re}[1]{\mathrm{Re}\left[ #1 \right]}
\renewcommand{\Im}[1]{\mathrm{Im}\left[ #1 \right]}
\newcommand{\Op}{\mathcal{O}}

\newcommand{\invI}[2]{#1^{-1} \left( #2 \right)}
\newcommand{\End}[1]{\text{End}\left( A \right)}
\newcommand{\legsym}[2]{\left(\frac{#1}{#2} \right)}
\renewcommand{\mod}[3]{\: #1 \equiv #2 \: \mathrm{mod} \: #3 \:}
\newcommand{\nmod}[3]{\: #1 \centernot \equiv #2 \: mod \: #3 \:}
\newcommand{\ndiv}{\hspace{-4pt}\not \divides \hspace{2pt}}
\newcommand{\finfield}[1]{\mathbb{F}_{#1}}
\newcommand{\finunits}[1]{\mathbb{F}_{#1}^{\times}}
\newcommand{\ord}[1]{\mathrm{ord}\! \left(#1 \right)}
\newcommand{\quadfield}[1]{\Q \small(\sqrt{#1} \small)}
\newcommand{\vspan}[1]{\mathrm{span}\! \left\{#1 \right\}}
\newcommand{\galgroup}[1]{Gal \small(#1 \small)}
\newcommand{\bra}[1]{\left| #1 \right>}
\newcommand{\Oa}{O_\alpha}
\newcommand{\Od}{O_\alpha^{\dagger}}
\newcommand{\Oap}{O_{\alpha '}}
\newcommand{\Odp}{O_{\alpha '}^{\dagger}}
\newcommand{\im}[1]{\mathrm{im} \: #1}
\renewcommand{\ker}[1]{\mathrm{ker} \: #1}
\newcommand{\ket}[1]{\left| #1 \right>}
\renewcommand{\bra}[1]{\left< #1 \right|}
\newcommand{\inner}[2]{\left< #1 | #2 \right>}
\newcommand{\expect}[2]{\left< #1 \right| #2 \left| #1 \right>}
\renewcommand{\d}[1]{ \mathrm{d}#1 \:}
\newcommand{\dn}[2]{ \mathrm{d}^{#1} #2 \:}
\newcommand{\deriv}[2]{\frac{\d{#1}}{\d{#2}}}
\newcommand{\nderiv}[3]{\frac{\dn{#1}{#2}}{\d{#3^{#1}}}}
\newcommand{\pderiv}[2]{\frac{\partial{#1}}{\partial{#2}}}
\newcommand{\fderiv}[2]{\frac{\delta #1}{\delta #2}}
\newcommand{\parsq}[2]{\frac{\partial^2{#1}}{\partial{#2}^2}}
\newcommand{\topo}{\mathcal{T}}
\newcommand{\base}{\mathcal{B}}
\renewcommand{\bf}[1]{\mathbf{#1}}
\renewcommand{\a}{\hat{a}}
\newcommand{\adag}{\hat{a}^\dagger}
\renewcommand{\b}{\hat{b}}
\newcommand{\bdag}{\hat{b}^\dagger}
\renewcommand{\c}{\hat{c}}
\newcommand{\cdag}{\hat{c}^\dagger}
\newcommand{\hamilt}{\hat{H}}
\renewcommand{\L}{\hat{L}}
\newcommand{\Lz}{\hat{L}_z}
\newcommand{\Lsquared}{\hat{L}^2}
\renewcommand{\S}{\hat{S}}
\renewcommand{\empty}{\varnothing}
\newcommand{\J}{\hat{J}}
\newcommand{\lagrange}{\mathcal{L}}
\newcommand{\dfourx}{\mathrm{d}^4x}
\newcommand{\meson}{\phi}
\newcommand{\dpsi}{\psi^\dagger}
\newcommand{\ipic}{\mathrm{int}}
\newcommand{\tr}[1]{\mathrm{tr} \left( #1 \right)}
\newcommand{\C}{\mathbb{C}}
\newcommand{\CP}[1]{\mathbb{CP}^{#1}}
\newcommand{\Vol}[1]{\mathrm{Vol}\left(#1\right)}

\newcommand{\Tr}[1]{\mathrm{Tr}\left( #1 \right)}
\newcommand{\Charge}{\hat{\mathbf{C}}}
\newcommand{\Parity}{\hat{\mathbf{P}}}
\newcommand{\Time}{\hat{\mathbf{T}}}
\newcommand{\Torder}[1]{\mathbf{T}\left[ #1 \right]}
\newcommand{\Norder}[1]{\mathbf{N}\left[ #1 \right]}
\newcommand{\Znorm}{\mathcal{Z}}
\newcommand{\EV}[1]{\left< #1 \right>}
\newcommand{\interact}{\mathrm{int}}
\newcommand{\covD}{\mathcal{D}}
\newcommand{\conj}[1]{\overline{#1}}

\newcommand{\SO}[2]{\mathrm{SO}(#1, #2)}
\newcommand{\SU}[2]{\mathrm{SU}(#1, #2)}

\newcommand{\anticom}[2]{\left\{ #1 , #2 \right\}}


\newcommand{\pathd}[1]{\! \mathdutchcal{D} #1 \:}

\renewcommand{\theenumi}{(\alph{enumi})}


\renewcommand{\theenumi}{(\alph{enumi})}

\newcommand{\atitle}[1]{\title{% 
	\large \textbf{Physics GR8048 Quantum Field Theory II
	\\ Assignment \# #1} \vspace{-2ex}}
\author{Benjamin Church }
\maketitle}

\newcommand{\atitleIII}[1]{\title{% 
	\large \textbf{Physics GR8049 Quantum Field Theory III
	\\ Assignment \# #1} \vspace{-2ex}}
\author{Benjamin Church }
\maketitle}

\theoremstyle{definition}
\newtheorem{theorem}{Theorem}[section]
\newtheorem{definition}{definition}[section]
\newtheorem{lemma}[theorem]{Lemma}
\newtheorem{proposition}[theorem]{Proposition}
\newtheorem{corollary}[theorem]{Corollary}
\newtheorem{example}[theorem]{Example}
\newtheorem{remark}[theorem]{Remark}

\begin{document}

\atitle{4}

\section{UV Divergences in $\mathcal{N} = 4$, $\mathrm{SU}(N)$ Super Yang-Mills Theory}

Consider the $\mathcal{N} = 4$ $\mathrm{SU}(N)$ Super Yang-Mills Theory in $D = 3 + 1$ dimensions which is given by the Lagrangian,
\begin{align*}
\lagrange & = 2 \mathrm{tr}\left\{
- \frac{1}{4} F_{\mu \nu} F^{\mu \nu}  
+ \frac{\theta g^2}{64 \pi^2} \epsilon^{\mu \nu \rho \sigma} F_{\mu \nu} F_{\rho \sigma} + i \bar{\lambda}^A \bar{\sigma}^\mu \covD_\mu \lambda^A + \frac{1}{2} \covD^\mu X^I \covD_\mu X^I  
\right. \\
& \left. \quad \quad \quad \quad + \frac{g^2}{4} [X^I, X^J]^2 + g c^I_AB [X^I, \lambda^A] \varepsilon \lambda^B + g c^I_{AB} [X^I, \bar{\lambda}^A] \varepsilon \bar{\lambda}^B \right\}
\end{align*}
with field content,
\begin{itemize}
\item One $SU(N)$ gauge field $A_\mu = A^a_\mu t^a$.
\item Six adjoint scalar fields $X^I = X^{Ia} t^a$.
\item Four adjoint left-handed Weyl spinor fields $\lambda^{A}_{\alpha} = \lambda^{Aa}_{\alpha} t^a$.
\end{itemize}
We take the convention that the generators of the $SU(N)$ lie algebra satisfy $[t^a, t^b] = i f^{abc} t^c$ and $\tr{t^a t^b} = \tfrac{1}{2} \delta^{ab}$. The $SU(N)$ covariant derivatives are defined as,
\[ \covD_\mu X^I = \partial_\mu X^I - ig [A_\mu, X^I] \quad \text{and} \quad \covD_\mu \lambda^A = \partial_\mu \lambda^A - ig [A_\mu, \lambda^A] \]
Our goal is to compute the divergence of the $F^2 = F_{\mu\nu}^a F^{a \mu \nu}$ term in the one-loop quantum effective action for this super Yang-Mills theory. We need to investigate the quantum-effective action $\Gamma[A, X, \lambda]$ calculated for constant background fields $A_{\mu}^a$ and $X^I = 0$ and $\lambda^A = 0$ since we are only interested in pure gauge-field effective action terms. The one-loop quantum effective action is computed by taking the functional logarithmic determinant over the quadratic part of the Lagrangian expanded about the above constant field profiles. From there we will be able to extract the $O(A^4)$ terms. Any term which is higher than second-order in $X^I$ or $\lambda^A$ will not contribute to the quadratic part of the Lagrangian expanded at $X^I = 0$ and $\lambda^A = 0$. Therefore, we need only consider the first line of the above Lagrangian. Furthermore, the kinetic terms are quadratic in matter fields which implies that the only nonzero contributions to the quadratic Lagrangian must differentiate twice with respect to that matter field. That is, the quadratic term mixing matter fields with gauge fields due to covariant derivatives vanish because they are proportional to the background matter fields which vanish. Thus, 
\[ \frac{\delta^2 \lagrange}{\delta A_\mu \delta X^I} = \frac{\delta^2 \lagrange}{\delta A_\mu \delta \lambda^A} = \frac{\delta^2 \lagrange}{\delta X^I \delta \lambda^A} = 0 \]
Therefore, the quadratic part of the Lagrangian is block-diagonal so we can write,
\[ -\frac{1}{2} \log{\det{\left[ - \lagrange^{(2)}(A) \right]}} = - \frac{1}{2} \log{\det{\left[ - \frac{\delta^2 \lagrange}{\delta A_\mu \delta A_\nu} \right]}} - \frac{1}{2} \log{\det{\left[ - \frac{\delta^2 \lagrange}{\delta X^I \delta X^J} \right]} } + \log{\det{\left[ - \frac{\delta^2 \lagrange}{\delta \bar{\lambda}^A \delta \lambda^B} \right]} }
\]
where the last term is given a relative exponent of $-2$ because it is the quadratic operator for  a fermioninc field which, when integrated over as a Gaussian term, gives the determinant with an exponent $+1$ as opposed to the bosonic $-\frac{1}{2}$. We need to compute each of these terms. 

\subsection{Gauge Field Contribution}

Consider the terms which depend on the gauge field alone. First,
\[ - 2 \tr{ \frac{1}{4} F_{\mu \nu} F^{\mu \nu}} = -\frac{1}{2} F_{\mu\nu}^a F^{b \mu \nu} \tr{t^a t^b} = - \frac{1}{4} F_{\mu \nu}^a F^{a \mu \nu}  \]
which is the usual Yang-Mills kinetic term. 
Second, using total anti-symmetry,
\[ \epsilon^{\mu \nu \rho \sigma} F_{\mu \nu} F_{\rho \sigma} = 4 \epsilon^{\mu \nu \rho \sigma} \partial_\mu A_{\nu} \partial_\rho A_\sigma \]
We can write this term as a total derivative,
\[  \partial_\mu \left( \epsilon^{\mu \nu \rho \sigma} A_{\nu} \partial_\rho A_\sigma \right) =  \epsilon^{\mu \nu \rho \sigma} \partial_\mu A_{\nu} \partial_\rho A_\sigma + \epsilon^{\mu \nu \rho \sigma} A_{\nu} \partial_\mu \partial_\rho A_\sigma = \epsilon^{\mu \nu \rho \sigma} \partial_\mu A_{\nu} \partial_\rho A_\sigma = \tfrac{1}{4} \epsilon^{\mu \nu \rho \sigma} F_{\mu \nu} F_{\rho \sigma} \]
in which the term $\epsilon^{\mu \nu \rho \sigma} A_{\nu} \partial_\mu \partial_\rho A_\sigma$ gives zero because $\epsilon^{\mu \nu rho \sigma}$ is anti-symmetric in $\mu \leftrightarrow \rho$ and $\partial_\mu \partial_\rho$ is symmetric in $\mu \leftrightarrow \rho$. Therefore, $\epsilon^{\mu \nu \rho \sigma} F_{\mu \nu} F_{\rho \sigma}$ gives zero contribution to the action since it is a total derivative and therefore integrates to zero. We should be very careful, however, with this brazen use of the divergence theorem or, equivalently, integration by parts. The integration by parts is valid when the boundary term $\epsilon^{\mu \nu \rho \sigma} A_\nu \partial_\rho A_\sigma$ vanishes at infinity. For a state with topological charge, this will not vanish which explains the convention of referring to,
\[ \frac{\theta g^2}{64 \pi^2} \epsilon^{\mu \nu \rho \sigma} F_{\mu \nu} F_{\rho \sigma} \]
as a topological term. However, in the case at hand, we are expanding perturbatively about a constant gauge field which clearly has zero boundary term because it is constant at infinity. Therefore, we are justified in throwing away this term from the action. Equivalently, using integration by parts, we can write,
\begin{align*}
2 \tr{ \frac{\theta g^2}{64 \pi^2} \epsilon^{\mu \nu \rho \sigma} F_{\mu \nu} F_{\rho \sigma} } & = 8 \cdot \frac{\theta g^2}{64 \pi^2} \epsilon^{\mu \nu \rho \sigma} \partial_\mu A^a_{\nu} \partial_\rho A^b_\sigma \tr{t^a t^b} = - 4 \cdot \frac{\theta g^2}{64 \pi^2} \epsilon^{\mu \nu \rho \sigma} A^a_{\nu} \partial_\mu \partial_\rho A^b_\sigma 
\end{align*}
Therefore, 
\[ \frac{\delta^2}{\delta A_\mu \delta A_\nu} 2 \tr{ \frac{\theta g^2}{64 \pi^2} \epsilon^{\mu \nu \rho \sigma} F_{\mu \nu} F_{\rho \sigma} } =  \frac{\theta g^2}{8 \pi^2} \epsilon^{\alpha \beta \mu \nu} \partial_{\alpha} \partial_{\beta} + \frac{\theta g^2}{8 \pi^2} \epsilon^{\alpha \beta \nu \mu} \partial_{\alpha} \partial_{\beta} = 0 \]
which gives zero because $\epsilon^{\alpha \beta \mu \nu}$ is anti-symmetric in $\alpha \leftrightarrow \beta$ and $\partial_\alpha \partial_\beta$ is symmetric in $\alpha \leftrightarrow \beta$ and also the anti-symmetry $\mu \leftrightarrow \nu$.
\bigskip\\
Therefore, only the kinetic term,
\[ - \frac{1}{4} F_{\mu \nu}^a F^{a \mu \nu}  \]
and the ghost fields from the gauge fixing action contribute to the gauge field part of the one-loop quantum effective action. Weinberg, in chapter 17.5, computes the $O(A^4)$ correction due to these contributions to be,
\[ \Gamma^{(1)}_A[A]_{A^4} = \left[ \frac{11}{12}  C(G) \right] \cdot \mathcal{I} g^2  F_{\mu\nu}^a F^{a \mu \nu} \]
where $C(G) = N$ is the Casimir of the gauge group $G = SU(N)$ and $\mathcal{I}$ is the UV and IR divergent integral,
\[ \mathcal{I} = -i \int \dn{4}{x} \int \frac{\dn{4}{k}}{(2 \pi)^4} \frac{1}{(k_\mu k^\mu - i \epsilon)^2} =  \int \dn{4}{x} \int \frac{\dn{4}{k_E}}{(2 \pi)^4} \frac{1}{(k_E^2 + i \epsilon)^2}   \]
Which, in Weinberg's notation, is therefore equal to,
\[ \mathcal{I} = \frac{-i \mathscr{J}}{(2 \pi)^4} \int \dn{4}{x} \quad \text{where} \quad \mathscr{J} = \int \dn{4}{k} \frac{1}{(k_\mu k^\mu - i \epsilon)^2} \]
 
\subsection{Scalar Field Contribution}

Consider the scalar kinetic term,
\begin{align*}
\lagrange^{(2)}_X & = 2 \tr{ \frac{1}{2} \covD_\mu X^I \covD^\mu X^I }
\\
& = \tr{ \left( \partial_\mu t^a X^{Ia} - i g [ A_\mu^c t^c, X^{Ia} t^a] \right) \left( \partial^\mu t^b X^{Ib} - i g [ A^{\mu c'} t^{c'}, X^{Ib} t^b] \right) }
\\
& = \tr{ \left( \partial_\mu t^a X^{Ia} + g f^{cad} t^d A_\mu^c X^{Ia} \right) \left( \partial^\mu t^b X^{Ib} +  g f^{c'be} t^e A^{\mu c'} X^{I b}  \right) }
\\
& = \partial_\mu X^{Ia} \partial^\mu X^{Ib} \tr{t^a t^b} + \partial_\mu X^{I a} g f^{cbe} A^{\mu c} X^{Ib} \tr{t^a t^e} + g f^{cad} A^c_\mu X^{Ia} \partial^\mu X^{Ib} \tr{t^d t^b} 
\\
& \quad + g^2 f^{cad} f^{c'be} A_\mu^c A^{\mu c'}  X^{Ia} X^{I b} \tr{t^d t^e}
\\
& = \frac{1}{2} \bigg\{ \partial_\mu X^{Ia} \partial^\mu X^{Ia} + \partial_\mu X^{I a} g f^{cba} A^{\mu c} X^{Ib}  + g f^{cab} A^c_\mu X^{Ia} \partial^\mu X^{Ib}  + g^2 f^{cad} f^{c'bd} A_\mu^c A^{\mu c'}  X^{Ia} X^{I b} \bigg\}
\end{align*}
Now we integrate by parts noticing that we may pass derivatives through the gauge fields $A$ because, as background fields, they are constant and the fluctuations above the background do not contribute at quadratic order because they appear multiplied by quadratic combinations of $X^I$ fields. Thus,
\[ \lagrange^{(2)}_X = X^{Ia} \bigg\{ - \partial^2 \delta^{ab} -  g f^{cba} A^{\mu c} \partial_\mu  + g f^{cab} A^c_\mu \partial^\mu  + g^2 f^{cad} f^{c'bd} A_\mu^c A^{\mu c'}  \bigg\} X^{Ib} \]
Since we conventionally take $f^{cab} = -f^{cda}$ the middle terms are equal and combine to give,
\[ \lagrange^{(2)}_X = X^{Ia} \bigg\{ - \partial^2 \delta^{ab} + 2 g f^{cab} A^c_\mu \partial^\mu  + g^2 f^{cad} f^{c'bd} A_\mu^c A^{\mu c'}  \bigg\} X^{Ib} \] 
It is now our task to compute,
\[ \Gamma^{(1)}_X[A] = \frac{i}{2} \log{\det{\left[ - \frac{\partial^2 \lagrange^{(2)}_X(A)}{\partial X^{Ia} \partial X^{J b}} \right]}} \]
which requires performing a Wick rotation of the above quantity. Plugging in Euclidean variables $A_0 = i A_0^E$ and $\partial_0 = i \partial_0^E$ and remembering that Lorentzian inner products are sign reversed Euclidean inner products because $X_\mu Y^{\mu} = X_0 Y_0 - \vec{X} \cdot \vec{Y} = - X^0_E Y^0_E - \vec{X} \cdot \vec{Y} = - X_E \cdot Y_E$ we get,
\[ \lagrange^{(2)}_X = X^{Ia} \bigg\{ \partial_E^2 \delta^{ab} - 2 g f^{cab} A^{c}_E \cdot \partial_E - g^2 f^{cad} f^{c'bd} A^c_E \cdot A^{c'}_E \bigg\} X^{Ib} \] 
And therefore, dropping the integral over all space for brevity, the one-loop quantum effective action becomes,
\begin{align*}
\Gamma^{(1)}_X[A] & = \frac{i}{2} \Tr{\log{ \delta_{IJ} \Big( - \partial_E^2 \delta^{ab} + 2 g f^{cab} A^{c}_E \cdot \partial_E + g^2 f^{cad} f^{c'bd} A^c_E \cdot A^{c'}_E \Big) }}
\\
& = - \frac{6}{2} \int \frac{\dn{4}{k_E}}{(2 \pi)^4} \tr{\log{ \Big( k_E^2 \delta^{ab} + 2i g f^{cab} A^{c}_E \cdot k_E + g^2 f^{cad} f^{c'bd} A^c_E \cdot A^{c'}_E \Big) }}
\end{align*}
where the factor of $6$ comes from taking a partial trace over the $I,J$ identity matrix and represents the contribution from the six different real scalar fields. Now we need to introduce some additional notation to simplify this computation. 
\newcommand{\A}{\mathcal{A}}
\newcommand{\M}{\mathcal{M}}
Let $\A^{ab}_{\mu}$ denote the matrix $- i g f^{cab} (A^c_E)_{\mu}$. Then the argument can be written as,
\[ \M = k_E^2 I - 2 \A_\mu k_E^\mu + \A_\mu \A^\mu = \M_0 + \M_1 + \M_2 \]
Now we can use the identity (17.4.20) proved in Weinberg chapter 7.4,
\[ [\tr{\log{\M}}]_{A^4} = \tr{- \tfrac{1}{2} [\M_0^{-1} \M_2]^2 + [\M_0^{-1} \M_1]^2 \M_0^{-1} \M_2 - \tfrac{1}{4} [\M_0^{-1} \M_1]^4 } \] 
which, plugging in, gives,
\[ [\tr{\log{\M}}]_{A^4} = \tr{ -\frac{1}{2} \frac{(\A_\mu \A^\mu)^2}{k_E^4} + 4 \frac{(\A_\mu k_E^\mu)^2 \A_\nu \A^\nu}{k_E^6} - 4 \frac{(\A_\mu k_E^\mu)^4}{k_E^8} }  \]
Therefore, 
\begin{align*}
\Gamma^{(1)}_X[A]_{A^4} & = - 3 \int \frac{\dn{4}{k_E}}{(2 \pi)^4} \tr{ -\frac{1}{2} \frac{(\A_\mu \A^\mu)^2}{k_E^4} + 4 \frac{(\A_\mu k_E^\mu)^2 \A_\nu \A^\nu}{k_E^6} - 4 \frac{(\A_\mu k_E^\mu)^4}{k_E^8} } 
\end{align*}
However, by the symmetry properties of the integral, we may replace $k_E^\mu k_E^\nu$ in the integral by $\frac{1}{4} \eta^{\mu \nu} k_E^2$ and furthermore we may replace $k_E^{\mu} k_E^{\nu} k_E^{\rho} k_E^{\sigma}$ inside the integral by $\frac{1}{24} k_E^4 (\eta^{\mu \nu} \eta^{\rho \sigma} + \eta^{\mu \rho} \eta^{\nu \sigma} + \eta^{\mu \sigma} \eta^{\nu \rho})$. 
Therefore, under the integral sign, when multiplied only by spherically symmetric functions of $k_E$ we can replace,
\[ (\A_\mu k_E^\mu)^2 = k_E^\mu k_E^\nu \A_\mu \A_\nu \to \tfrac{1}{4} k_E^2 \eta^{\mu \nu} \A_\mu \A_\nu = \tfrac{1}{4} k_E^2 \A_\mu \A^\mu \]
and likewise,
\begin{align*}
(\A_\mu k_E^\mu)^4 = k_E^{\mu} k_E^{\nu} k_E^{\rho} k_E^{\sigma} \A_\mu \A_\nu \A_\rho \A_\sigma & \to \tfrac{1}{24} \k_E^4 (\eta^{\mu \nu} \eta^{\rho \sigma} + \eta^{\mu \rho} \eta^{\nu \sigma} + \eta^{\mu \sigma} \eta^{\nu \rho})\A_\mu A_\nu \A_\rho \A_\sigma
\\
& = \tfrac{1}{24} k_E^4 ( \A_\mu \A^\mu \A_\nu \A^\nu + \A_\mu \A_\nu \A^\mu \A^\nu + \A_\mu \A_\nu \A^{\nu} \A^{\mu} )
\end{align*}
Plugging in these terms, we find,
\begin{align*}
\Gamma^{(1)}_X[A]_{A^4} & = - 3 \int \frac{\dn{4}{k_E}}{(2 \pi)^4} \frac{1}{k_E^4} \tr{ - \frac{1}{2} \A_\mu \A^\mu \A_\nu \A^\nu + \A_\mu \A^\mu \A_\nu \A^\nu - \frac{1}{6} ( \A_\mu \A^\mu \A_\nu \A^\nu + \A_\mu \A_\nu \A^\mu \A^\nu + \A_\mu \A_\nu \A^{\nu} \A^{\mu} ) } 
\end{align*}
Using the cyclic symmetry of the trace,
\[ \tr{\A_\mu \A_\nu \A^{\nu} \A^{\mu}} = \tr{\A^{\mu} \A_\mu \A_\nu \A^{\nu}} = \tr{\A_\mu \A^\mu \A_\nu \A^\nu} \]
Combining like terms, and recalling the $i \epsilon$ prescription in use,
\begin{align*}
\Gamma^{(1)}_X[A]_{A^4} & = - 3 \int \frac{\dn{4}{k_E}}{(2 \pi)^4} \frac{1}{k_E^4} \cdot \frac{1}{6} \tr{ \A_\mu \A^\mu \A_\nu \A^\nu  - \A_\mu \A_\nu \A^\mu \A^\nu} = - \frac{1}{2} \mathcal{I} \: \tr{ \A_\mu \A^\mu \A_\nu \A^\nu  - \A_\mu \A_\nu \A^\mu \A^\nu}
\end{align*}
At last, using the identify, 
\[ \tr{ \A_\mu \A^\mu \A_\nu \A^\nu  - \A_\mu \A_\nu \A^\mu \A^\nu} = \frac{1}{2} g^2 C(G) F_{\mu \nu}^a F^{a \mu \nu} \]
derived in Weinberg chapter 17, we get,
\[ \Gamma^{(1)}_X[A]_{A^4} = \left[ - \frac{1}{4} C(G) \right] \mathcal{I} g^2 F_{\mu \nu}^a F^{a \mu \nu} \]

\subsection{Weyl-Spinor Field Contribution}

Consider,
\begin{align*}
\lagrange^{(2)}_{\lambda}(A) & = 2 \tr{ i \bar{\lambda}^A \bar{\sigma}^\mu \covD_\mu \lambda^A} = 2 \tr{ i \bar{\lambda}^A \bar{\sigma}^\mu ( \partial_\mu \lambda^A - i g [A_\mu, \lambda^A])  }
\\
& = 2 \tr{ i \bar{\lambda}^{Aa} t^a \bar{\sigma}^\mu ( \partial_\mu \lambda^{Ab} t^b - i g [t^c, t^b] A_\mu^c \lambda^{A b})  }
\\
& = 2 \tr{ i \bar{\lambda}^{Aa} t^a \bar{\sigma}^\mu ( \partial_\mu \lambda^{Ab} t^b + g f^{cbd} t^d A_\mu^c \lambda^{A b})  }
\\
& = 2 i \bar{\lambda}^{Aa} \bar{\sigma}^\mu   ( \partial_\mu \tr{ t^a t^b} + g f^{cbd} \tr{t^a t^d} A_\mu^c )  \lambda^{A b} = i \bar{\lambda}^{Aa} \bar{\sigma}^\mu   ( \partial_\mu \delta^{ab} + g f^{cba} A_\mu^c )  \lambda^{A b}
\end{align*} 
Therefore, the contribution to the one-loop quantum effective action is,
\[ \Gamma_\lambda[A] = - i \log{\det{(- \delta_{AB} i \bar{\sigma}^\mu (\delta^{ab} \partial^\mu + g f^{cba} A_\mu^c) )}} \]
This functional determinant is nearly identical to the one computed in Weinberg for Dirac spinors. However, there are exactly two differences. First, we have Weyl spinors as opposed to Dirac spinors. Since a massless Dirac spinor is just two uncoupled Weyl spinors this means our computation must be exactly one half of the corresponding calculation in Weinberg. Furthermore, the matrix $t^c$ in Weinberg is replaced here by $- i g f^{c\cdot \cdot}$ in matrix form because the Weinberg considered fermions in the fundamental representation while we are considering fermions in the adjoint representation. Therefore, we may use the result computed in Weinberg chapter 17 (17.5.30) for the $O(A^4)$ term in the one-loop quantum effective action generated by fermions remembering to replace all $t^c$ by $ -i g f^{c \cdot \cdot}$ and multiply by $\tfrac{1}{2}$ to select a single handed Weyl component. Therefore,
\[ \Gamma_\lambda[A]_{A^4} = \tr{\delta_{AB}} \cdot \tfrac{1}{2} \cdot \left[ - \tfrac{1}{3} \mathcal{I} g^2 F^a_{\mu \nu} F^{b \mu \nu} \tr{i f^{a \cdot \cdot} i f^{b \cdot \cdot}} \right] = \left[ - \frac{2}{3} C(G)  \right] \mathcal{I} g^2 F^a_{\mu \nu} F^{a \mu \nu} \]
where $\tr{\delta_{AB}} = 4$ gives the contribution from the four different Weyl scalar fields. Furthermore, I have used, $\tr{if^{a \cdot \cdot} i f^{b \cdot \cdot}} = - f^{ars} f^{bsr} = f^{ars} f^{brs} = C(G) \delta^{ab}$ which is again the Casimir of the gauge group $G = SU(N)$. 

\subsection{The Quantum Effective Action}

Putting these contributions together,
\[ \Gamma[A]_{A^4} = \Gamma_A[A]_{A^4} + \Gamma_X[A]_{A^4} + \Gamma_{\lambda}[A]_{A^4} = \left[ \frac{11}{12} - \frac{1}{4} - \frac{2}{3} \right] C(G) \mathcal{I} g^2 F^a_{\mu \nu} F^{a \mu \nu} = 0 \]
These cancellations show that there is no running of the gauge field coupling with energy scale. Therefore, the theory is scale-invariant. We ought to have predicted this since $\mathcal{N} = 4$ $\mathrm{SU}(N)$ Super Yang-Mills Theory is a confromal theory, putting the ``C'' in AdS-CFT correspondence! 
 
\section{Renormalizability of Massive Yang-Mills Theory}

Consider the Lagrangian for massive abelian $U(1)$ Yang-Mills theory with a quartic interaction,
\[ \lagrange = - \frac{1}{4} F_{\mu \nu} F^{\mu \nu} + \frac{1}{2} m^2 A_\mu A^{\mu} - \frac{1}{4} g (A_\mu A^\mu)^2 \]
We want to compute the one-loop quantum effective action of this theory. To do so, we need to quadratic part of this Lagrangian expanded about a constant $A_{\mu}$ field.
We can write, by anti-symmetry,
\[ \frac{1}{2} F_{\mu \nu} F^{\mu \nu} = \frac{1}{2} (\partial_\mu A_\nu - \partial_\nu A_\mu) F^{\mu \nu} = \partial_\mu A_\nu F^{\mu \nu} = \partial_\mu A_\nu \partial^\mu A^\nu - \partial_\mu A_{\nu} \partial^\nu A^\mu  \]
Integrating my parts,
\[  F_{\mu \nu} F^{\mu \nu} = - A_\nu \partial^2 A^\nu + A_\nu \partial_\mu \partial^\nu A^\mu = - A_\mu(\partial^2 \eta^{\mu \nu} - \partial^\mu \partial^\nu) A_\nu \]
Now consider the expansion,
\[ \lagrange = \frac{1}{2} A_\mu (\partial^2 \eta^{\mu \nu} - \partial^\mu \partial^\nu) A_\nu + \frac{1}{2} m^2 A_\mu \eta^{\mu \nu} A_{\nu} - \frac{1}{4} g \eta^{\alpha \beta} \eta^{\gamma \delta} A_\mu A_{\alpha} A_{\beta} A_{\gamma} A_{\delta} \]
which implies that,
\[ \frac{\delta^2 \lagrange}{\delta A_\mu \delta A_{\nu}} = \partial^2 \eta^{\mu \nu} - \partial^\mu \partial^\nu + m^2 \eta^{\mu \nu} - g \eta^{\mu \nu} \eta^{\gamma \delta} A_{\gamma} A_{\delta} - 2 g \eta^{\mu \beta} \eta^{\nu \delta} A_{\beta} A_{\delta} \]
Which, after contracting indices gives,
\[ \frac{\delta^2 \lagrange}{\delta A_\mu \delta A_{\nu}} = \partial^2 \eta^{\mu \nu} - \partial^\mu \partial^\nu + \eta^{\mu \nu} (m^2 - g A_{\alpha} A^{\alpha}) - 2 g A^{\mu} A^{\nu} \]
To compute the one-loop correction to the quantum effective action given by,
\[ - \Gamma^{(1)}[A] = - \frac{i}{2} \log{\det{\left[ -  \frac{\delta^2 \lagrange}{\delta A_\mu \delta A_{\nu}} \right]}} \]
we need to Wick rotate the above expression. Which involves replacing $A_0 = i A_0^{E}$ and $\partial_0 = i \partial_0^E$. Furthermore, $\lagrange = - \lagrange_E$. Since the derivative of the Wick rotated Lagrangian with respect to Wick rotated fields gives a relative minus sign in the time direction\footnote{The spatial directions become positive due to overall minus sign $\lagrange = - \lagrange_E$ and the relative minus sign from the $A$ fields then flips the time direction to positive as well.} due to the factors of $i$ in $A_0 = i A_0^E$ we replace $\eta^{\mu \nu} \to \delta^{\mu \nu}$. Finally, recall that $X_\mu X^{\mu} = X_0^2 - \vec{X}^{\, 2} = - (X^0_E)^2 - \vec{X}^{\, 2} = - X_E^2$ for any vector under Wick rotation. Therefore,
\[ \frac{\delta^2 \lagrange_E}{\delta A^E_\mu \delta A^E_\nu} = - \delta^{\mu \nu} \partial_E^2 + \partial^\mu \partial^\nu + \delta^{\mu \nu} (m^2 + g A_E^2) + 2 g A^\mu_E A^{\nu}_E \]  
At last, the one-loop quantum effective action is equal to,
\begin{align*}
- \Gamma^{(1)}[A] = - \frac{i}{2} \log{\det{\left[ \frac{\delta^2 \lagrange_E}{\delta A^E_\mu \delta A^E_{\nu}} \right]}} = \frac{1}{2} \int \dn{4}{x} \int \frac{\dn{4}{k_E}}{(2 \pi)^4} \tr{\log{( \delta^{\mu \nu} k_E^2 - k_E^\mu k_E^\nu + \delta^{\mu \nu} (m^2 + g A_E^2) + 2 g A^\mu_E A^{\nu}_E)}}
\end{align*}
I will now use the following identity which I derive in the appendix,
\[ \det{(\lambda I + \bf{x} \bf{x}^\top + \bf{y} \bf{y}^\top)} = \lambda^{N - 1} \left( \lambda + |\bf{x}|^2 + |\bf{y}|^2 + \lambda^{-1} |\bf{x}|^2 |\bf{y}|^2 - \lambda^{-1} (\bf{x} \cdot \bf{y})^2 \right) \]
where $I$ is the $N \times N$ identity matrix and $\bf{x}, \bf{y} \in \R^N$ are two vectors. Using this formula, we can write,
\begin{align*}
& \det{([k_E^2 + m^2 + g A_E^2] I - k_E k_E^\top + 2 g A_E A_E^\top)}
\\
& \quad \quad  = (k_E^2 + m^2 + g A_E^2)^3 \cdot \left( m^2 + 3 g A_E^2 - \frac{2 g k_E^2 A_E^2}{k_E^2 + m^2 + g A_E^2} + \frac{2g (k_E \cdot A_E)^2}{k_E^2 + m^2 + g A_E^2}  \right)  
\end{align*}
Let $\tilde{m}^2 = m^2 + g A_E^2$ and $\tilde{m}3^2 = m^2 + 3 g A_E^2$ then,
\begin{align*}
\det{([k_E^2 + m^2 + g A_E^2] I - k_E k_E^\top + 2 g A_E A_E^\top)} & = (k_E^2 + \tilde{m}^2)^2 \cdot \left( k_E^2 \tilde{m}_3^2 + \tilde{m}^2 \tilde{m}_3^2 - 2 g k_E^2 A_E^2 + 2g (k_E \cdot A_E)^2 \right)  
\\
& = (k_E^2 + \tilde{m}^2)^2 \tilde{m}^2 \left( k_E^2 + \tilde{m}_3^2 + 2 g (k_E \cdot A_E)^2 \tilde{m}^{-2} \right)
\end{align*}
Thus,
\begin{align*}
- \Gamma^{(1)}[A] & = \frac{1}{2} \int \dn{4}{x} \int \frac{\dn{4}{k_E}}{(2 \pi)^4} \left[ 2 \log{(k_E^2 + \tilde{m}^2)} + \log{\tilde{m}^2} + \log{(k_E^2 + \tilde{m}_3^2 + 2 g (k_E \cdot A_E)^2 \tilde{m}^{-2})} \right]
\end{align*}
We understand the first two integrals very well. Thus, we need to examine more carefully the third integral,
\begin{align*}
I_3 & = \int \frac{\dn{4}{k_E}}{(2 \pi)^4} \log{(k_E^2 + \tilde{m}_3^2 + 2 g (k_E \cdot A_E)^2 \tilde{m}^{-2})} 
\\
& = \frac{4 \pi}{(2 \pi)^4} \int_0^{\Lambda} \int_0^{\pi} \log{(k_E^2 + \tilde{m}_3^2 + 2 g k_E^2 A_E^2 \tilde{m}^{-2} \cos^2{\theta})}  k_E^3 \sin^2{\theta} \d{\theta} \d{k_E} 
\end{align*}
If we expand at large $A$ then $\tilde{m}^2 \approx g A_E^2$ and thus,
\[ I_3 = \frac{4 \pi}{(2 \pi)^4} \int_0^{\Lambda} \int_0^{\pi} \log{(k_E^2 + \tilde{m}_3^2 + 2 g k_E^2 \cos^2{\theta})}  k_E^3 \sin^2{\theta} \d{\theta} \d{k_E}  \]
The third derivative of this integral with respect to $\tilde{m}_3^2$ is then equal to,
\[ \nderiv{3}{I_3}{(\tilde{m}^2)} = \frac{8 \pi}{(2 \pi)^4} \int_0^{\Lambda} \int_0^\pi \frac{k_E^3}{(k_E^2 + \tilde{m}_3^2 + 2 g k_E^2 \cos^2{\theta})^6} \sin^2{\theta} \d{\theta} \d{k_E} \]
which is convergent because the integrand goes as $k_E^{-3}$ for large $k_E$. Therefore, $I_3$ can only have divergences in terms proportional to $\tilde{m}_3^2$ and $\tilde{m}_3^4$ and therefore all divergences are quartic polynomials in $A$ which are presumably renormalizable. Furthermore, we have seen before that the term,
\[ I_1 = \int \frac{\dn{4}{k_E}}{(2 \pi)^4} \log{(k_E^2 + \tilde{m}^2)} \]
also only contains divergences in terms up to quartic order in $A$ which are renormalizable. This leaves a final term,
\begin{align*}
I_2 = \int \frac{\dn{4}{k_E}}{(2 \pi)^4} \log{\tilde{m}^2} = \frac{\tfrac{1}{2} \pi^2}{(2 \pi)^4} \Lambda^4 \log{(m^2 + g A_E^2)} = - \frac{\Lambda^4}{32 \pi^2} \log(m^2) \sum_{n = 1}^{\infty} \frac{1}{n} \left( - \frac{g A_E^2}{m^2} \right)^n
\end{align*} 
This term is highly problematic because it has $\Lambda$ divergences at every order in $A^2$. This shows that the one-loop quantum effective action,
\[ - \Gamma^{(1)}[A] = I_1 + \tfrac{1}{2} I_3 - \frac{\Lambda^4}{64 \pi^2} \log(m^2) \sum_{n = 1}^{\infty} \frac{1}{n} \left( \frac{g A_\mu A^\mu}{m^2} \right)^n \]
is non-renormalizable because there are divergences at all orders of $A_\mu A^\mu$ which we have shown cannot be canceled by renormalizable terms $I_1$ and $I_3$ since these only produce divergence in degree four polynomials in $A$ which can be absorbed into counterterms of the form which appear at tree-level. The issue with this theory can be seen in the form of the $A$-field propagator for the free massive Yang-Mills theory. The free theory has quadratic operator,
\[ \mathcal{O}_{\mu \nu} = \frac{\delta^2 \lagrange}{\delta A_\mu \delta A_{\nu}} = \partial^2 \eta^{\mu \nu} - \partial^\mu \partial^\nu + m^2 \eta^{\mu \nu} \] 
The Schwinger–Dyson equation implies that the propagator must satisfy,
\[ \mathcal{O}_{\mu \alpha} D^{\alpha \nu}(x - y) = i \delta_{\mu}^{\nu} \delta(x - y) \]
This fixes the form to be,
\[ D^{\mu \nu} = \int \frac{\dn{4}{k}}{(2 \pi)^4} \frac{ - i }{k^2 - m^2 + i \epsilon} \left[ \eta^{\mu \nu} - \frac{ k^\mu k^\nu }{m^2} \right] e^{ - k(x - y)} \]
However, in momentum space this propagator has unexpected behavior. If we take,
\[ \frac{1}{k^2} \tilde{D}_{\mu \nu}(k) k^\mu k^\nu = \frac{-i}{k^2 - m^2 + i \epsilon} \left[ 1 - \frac{k^2}{m^2} \right] = \frac{i}{m^2} \]
Therefore, there exist directions along which the propagator does not decrease with increasing $k$ but rather gives a constant factor of $i m^{-2}$. This means that inverse powers of the mass can appear in the divergent parts of loop integrals. This destroys the dimensional analysis argument which predicts that the divergent part of the quantum effective action can be written as,
\[ \Gamma_{ \infty } [ A ] \sim \sum _ { n _ { 0 } n _ { 1 } n _ { 2 } n _ { 3 } n _ { 4 } } f _ { n _ { 0 } n _ { 1 } n _ { 2 } n _ { 3 } n _ { 4 } } ( g ) ( \log ( \Lambda / m ) ) ^ { n _ { 0 } } \Lambda ^ { n _ { 1 } } m ^ { n _ { 2 } } \partial ^ { n _ { 3 } } A ^ { n _ { 4 } } \]
where $n_1, n_2, n_3, n_4$ are all nonnegative. This fails because, as we have shown, this theory allows for terms derived from loop integrals with negative powers of the mass i.e. $n_2$ can be negative which may cancel arbitrarily large powers of fields i.e. $n_4$ is now unbounded above as we have seen occurs explicitly in the one-loop quantum effective action. 

\section{Unification Scale of $\mathrm{SU}(2)$ and $\mathrm{SU}(3)$ in the Standard Model}

Consider the $\beta$-function for $G = SU(N)$ Yang-Mills theory with $n_s$ complex scalars transforming in the $G$-representation $R_s$ and $n_f$ Weyl fermions transforming in the $G$-representation $R_f$,
\[ \beta(\alpha) = - \left( \frac{11}{3} C_2(G) - \frac{n_s}{3} T(R_s) - \frac{2}{3} T(R_f) \right) \frac{\alpha^2}{2 \pi} \]
where $C_2(G) = N$ is the Casimir of $SU(N)$ and $T(R)$ is the invariant $\tr{t^a t^b} = T(R) \delta^{ab}$. If $R$ is a fundamental representation then, canonically, $T(R) = \frac{1}{2}$. For the strong force with $G = SU(2)$, in the standard model, there are six quarks which are Dirac spinors (Weyl spinor doublets) transforming in the fundamental representation and there are no color charged scalars. Therefore, for the strong force, $n_f = 2 \cdot 6$ and $n_s = 0$ so we have,
\[ \beta_s(\alpha_s) = - \left( \frac{11}{3} \cdot 3 - \frac{0}{3} - \frac{2}{3} \cdot 2 \cdot 6 \right) \frac{\alpha_s^2}{2 \pi} = - \frac{7 \alpha_s^2}{2 \pi} \]
Likewise, the $G = SU(2)$ gauge field couples to only left-handed Weyl spinor isospin doublets. These are left-handed up-down, strange-charm, or top-bottom quarks and lepton-neutrino doublets. Furthermore, each quark has three color states which do not mix, thus acting as distinct modes, under the $SU(2)$ action. Furthermore, there is a single complex scalar $SU(2)$-doublet, the Higgs field. Therefore, for the $SU(2)$ gauge field, $n_s = 1$ and $n_f = 3 \cdot 3 + 3 = 12$ for the three quark pairs, each with three colors, and the three lepton-neutrino pairs. Plugging in, we get,
\[ \beta_w(\alpha_w) = - \left( \frac{11}{3} \cdot 2 - \frac{1}{3} - \frac{2}{3} \cdot 12 \right) \frac{\alpha_w^2}{2 \pi} = - \frac{19}{6} \cdot \frac{\alpha_w^2}{2 \pi} \]   
Now, since,
\[ \beta(\alpha) = \deriv{\alpha}{\log{M}} \]
we can write,
\[ \alpha^{-2} \beta(\alpha) = \frac{1}{\alpha^2} \deriv{\alpha}{\log{M}} = - \deriv{\alpha^{-1}}{\log{M}} \]
Therefore,
\[ \deriv{}{\log{M}} \left[ \alpha_w^{-1} - \alpha_s^{-1} \right] = \alpha_s^{-2} \beta_s(\alpha_s) - \alpha_w^{-2} \beta_w(\alpha_w) = - \frac{7}{2 \pi} + \frac{19}{12 \pi} = - \frac{23}{12 \pi}  \] 
Thus, integrating,
\[ \left[ \alpha_w^{-1}(M_{*}) -  \alpha_s^{-1}(M_{*}) \right] - \left[ \alpha_w^{-1}(M_Z) -  \alpha_s^{-1}(M_Z) \right] = - \frac{23}{12 \pi} \log{\left( \frac{M_*}{M_Z} \right)} \] 
Now, at the scale $M_Z = 91 \: \mathrm{GeV}$ we have $\alpha_w^{-1}(M_Z) = 30$ and $\alpha_s^{-1}(M_Z) = 8.5$. If we search for the unification scale $M_*$ at which $\alpha_w(M_*) = \alpha_s(M_*)$ then we find,
\[ \log{\left( \frac{M_*}{M_Z} \right)} = \frac{12 \pi}{23} \left[ \alpha_w^{-1}(M_Z) -  \alpha_s^{-1}(M_Z) \right] = 35.2405 \]
Therefore,
\[ M_* = M_Z e^{35.2405} \approx 1.83 \cdot 10^{17} \: \mathrm{GeV} \approx 29.4 \: \mathrm{MJ}  \]
which is rather high energy!

\section{Appendix}

Consider the determinant,
\[ \det{(\lambda I + \bf{x} \bf{x}^\top + \bf{y} \bf{y}^\top)} \]
where $I$ is the $N \times N$ identity matrix and $\bf{x}, \bf{y} \in \R^N$ are two vectors.
We can always find a rotation matrix $R$ such that $\bf{x} = R e_1 |\bf{x}|$ where $e_1$ is the first standard  basis vector. Then,
\[ R^{\top} (\lambda I + \bf{x} \bf{x}^\top + \bf{y} \bf{y}^\top) R = \lambda I + |\bf{x}|^2 e_1 e_1^\top + R^{\top} \bf{y} \bf{y}^{\top} R \]
Define, $D = R^{\top} (\lambda I + \bf{x} \bf{x}^\top)R$ which is the diagonal matrix,
\[ D = 
\begin{pmatrix}
\lambda + |\bf{x}|^2 & 0 & 0 & \cdots & 0
\\
0 & \lambda & 0 & \cdots & 0
\\
0 & 0 & \lambda & \cdots & 0
\\
\vdots & \vdots & \vdots & \ddots & \vdots
\\
0 & 0 & 0 & \cdots & \lambda 
\end{pmatrix}\]
Furthermore, 
\[ D^{-1} = \frac{1}{\lambda} \left( I + e_1 e_1^{\top} \left( \frac{\lambda}{\lambda + |\bf{x}|^2} - 1 \right) \right) = \frac{1}{\lambda}  \left( I - e_1 e_1^\top \frac{|\bf{x}|^2}{\lambda + |\bf{x}|^2} \right) \]
Therefore,
\[ \det{(\lambda I + \bf{x} \bf{x}^\top + \bf{y} \bf{y}^\top)} = \det{R^{\top} (\lambda I + \bf{x} \bf{x}^\top + \bf{y} \bf{y}^\top) R} = \det{(D + \tilde{\bf{y}} \tilde{\bf{y}}^\top)} \]
where $\tilde{\bf{y}} = R^{\top} \bf{y}$. Then,
\[  \det{(\lambda I + \bf{x} \bf{x}^\top + \bf{y} \bf{y}^\top)} = \det{D} \cdot \det{(I + D^{-\frac{1}{2}} \tilde{\bf{y}} \tilde{\bf{y}}^\top D^{- \frac{1}{2}})} = \det{D} \cdot \det{(1 + \bf{z} \bf{z}^{\top})} \]
where $\bf{z} = D^{-\frac{1}{2}} \tilde{\bf{y}}$. Now, there exists a rotation matrix $\tilde{R}$ such that $\bf{z} = \tilde{R} e_1 |\bf{z}|$. 
Therefore,
\[ \det{(I + \bf{z} \bf{z}^\top)} = \det{\tilde{R}^{\top} (I + \bf{z} \bf{z}^\top) \tilde{R}} = \det{(I + |\bf{z}|^2 e_1 e_1^{\top})} = 1 + |\bf{z}|^2 \]
because it is a diagonal matrix with eigenvalues $1$ and $1 + |\bf{z}|^2$. Furthermore,
\[ 1 + |\bf{z}|^2 = 1 + \bf{z}^\top \bf{z} = 1 + \tilde{\bf{y}}^\top D^{-1} \tilde{\bf{y}} = \bf{y}^\top R D^{-1} R^{\top} \bf{y} = 1 + \bf{y}^\top (\lambda I + \bf{x} \bf{x}^\top )^{-1} \bf{y} \]
We can compute,
\[ (\lambda I + \bf{x} \bf{x}^\top )^{-1} = R D^{-1} R^{\top} = \frac{1}{\lambda} R \left( I - e_1 e_1^\top \frac{|\bf{x}|^2}{\lambda + |\bf{x}|^2} \right) R^{\top} 
= \frac{1}{\lambda} \left( I - \frac{\bf{x} \bf{x}^\top}{\lambda + |\bf{x}|^2} \right) \]
Finally,
\begin{align*}
\det{(\lambda I + \bf{x} \bf{x}^\top + \bf{y} \bf{y}^\top)} & = \det{D} \cdot \left(1 + \bf{y}^\top \frac{1}{\lambda} \left( I - \frac{\bf{x} \bf{x}^\top}{\lambda + |\bf{x}|^2} \right)  \bf{y} \right) = \det{D} \cdot \left(1 + \lambda^{-1} |\bf{y}|^2 - \lambda^{-1} \frac{(\bf{x} \cdot \bf{y})^2}{\lambda + |\bf{x}|^2} \right)
\\
& = \lambda^{N-1} (\lambda + |\bf{x}|^2) \cdot \left(1 + \lambda^{-1} |\bf{y}|^2 - \lambda^{-1} \frac{(\bf{x} \cdot \bf{y})^2}{\lambda + |\bf{x}|^2} \right)
\\
& = \lambda^{N - 1} \left( \lambda + |\bf{x}|^2 + |\bf{y}|^2 + \lambda^{-1} |\bf{x}|^2 |\bf{y}|^2 - \lambda^{-1} (\bf{x} \cdot \bf{y})^2 \right)
\end{align*}
\end{document}