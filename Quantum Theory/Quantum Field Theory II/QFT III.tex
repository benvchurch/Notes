\documentclass[12pt]{extarticle}
\usepackage[utf8]{inputenc}
\usepackage[english]{babel}



\usepackage[utf8]{inputenc}
\usepackage[english]{babel}
\usepackage[a4paper, total={7.25in, 9.5in}]{geometry}
\usepackage{tikz-feynman}
\tikzfeynmanset{compat=1.0.0} 
\usepackage{subcaption}
\usepackage{float}
\floatplacement{figure}{H}
\usepackage{simpler-wick}
\usepackage{mathrsfs}  
\usepackage{dsfont}
\usepackage{relsize}
\usepackage{tikz-cd}
\DeclareMathAlphabet{\mathdutchcal}{U}{dutchcal}{m}{n}

\usepackage{cancel}



\newcommand{\field}{\hat{\Phi}}
\newcommand{\dfield}{\hat{\Phi}^\dagger}
 
\usepackage{amsthm, amssymb, amsmath, centernot}
\usepackage{slashed}
\newcommand{\notimplies}{%
  \mathrel{{\ooalign{\hidewidth$\not\phantom{=}$\hidewidth\cr$\implies$}}}}
 
\renewcommand\qedsymbol{$\square$}
\newcommand{\cont}{$\boxtimes$}
\newcommand{\divides}{\mid}
\newcommand{\ndivides}{\centernot \mid}

\newcommand{\Integers}{\mathbb{Z}}
\newcommand{\Natural}{\mathbb{N}}
\newcommand{\Complex}{\mathbb{C}}
\newcommand{\Zplus}{\mathbb{Z}^{+}}
\newcommand{\Primes}{\mathbb{P}}
\newcommand{\Q}{\mathbb{Q}}
\newcommand{\R}{\mathbb{R}}
\newcommand{\ball}[2]{B_{#1} \! \left(#2 \right)}
\newcommand{\Rplus}{\mathbb{R}^+}
\renewcommand{\Re}[1]{\mathrm{Re}\left[ #1 \right]}
\renewcommand{\Im}[1]{\mathrm{Im}\left[ #1 \right]}
\newcommand{\Op}{\mathcal{O}}

\newcommand{\invI}[2]{#1^{-1} \left( #2 \right)}
\newcommand{\End}[1]{\text{End}\left( A \right)}
\newcommand{\legsym}[2]{\left(\frac{#1}{#2} \right)}
\renewcommand{\mod}[3]{\: #1 \equiv #2 \: \mathrm{mod} \: #3 \:}
\newcommand{\nmod}[3]{\: #1 \centernot \equiv #2 \: mod \: #3 \:}
\newcommand{\ndiv}{\hspace{-4pt}\not \divides \hspace{2pt}}
\newcommand{\finfield}[1]{\mathbb{F}_{#1}}
\newcommand{\finunits}[1]{\mathbb{F}_{#1}^{\times}}
\newcommand{\ord}[1]{\mathrm{ord}\! \left(#1 \right)}
\newcommand{\quadfield}[1]{\Q \small(\sqrt{#1} \small)}
\newcommand{\vspan}[1]{\mathrm{span}\! \left\{#1 \right\}}
\newcommand{\galgroup}[1]{Gal \small(#1 \small)}
\newcommand{\bra}[1]{\left| #1 \right>}
\newcommand{\Oa}{O_\alpha}
\newcommand{\Od}{O_\alpha^{\dagger}}
\newcommand{\Oap}{O_{\alpha '}}
\newcommand{\Odp}{O_{\alpha '}^{\dagger}}
\newcommand{\im}[1]{\mathrm{im} \: #1}
\renewcommand{\ker}[1]{\mathrm{ker} \: #1}
\newcommand{\ket}[1]{\left| #1 \right>}
\renewcommand{\bra}[1]{\left< #1 \right|}
\newcommand{\inner}[2]{\left< #1 | #2 \right>}
\newcommand{\expect}[2]{\left< #1 \right| #2 \left| #1 \right>}
\renewcommand{\d}[1]{ \mathrm{d}#1 \:}
\newcommand{\dn}[2]{ \mathrm{d}^{#1} #2 \:}
\newcommand{\deriv}[2]{\frac{\d{#1}}{\d{#2}}}
\newcommand{\nderiv}[3]{\frac{\dn{#1}{#2}}{\d{#3^{#1}}}}
\newcommand{\pderiv}[2]{\frac{\partial{#1}}{\partial{#2}}}
\newcommand{\fderiv}[2]{\frac{\delta #1}{\delta #2}}
\newcommand{\parsq}[2]{\frac{\partial^2{#1}}{\partial{#2}^2}}
\newcommand{\topo}{\mathcal{T}}
\newcommand{\base}{\mathcal{B}}
\renewcommand{\bf}[1]{\mathbf{#1}}
\renewcommand{\a}{\hat{a}}
\newcommand{\adag}{\hat{a}^\dagger}
\renewcommand{\b}{\hat{b}}
\newcommand{\bdag}{\hat{b}^\dagger}
\renewcommand{\c}{\hat{c}}
\newcommand{\cdag}{\hat{c}^\dagger}
\newcommand{\hamilt}{\hat{H}}
\renewcommand{\L}{\hat{L}}
\newcommand{\Lz}{\hat{L}_z}
\newcommand{\Lsquared}{\hat{L}^2}
\renewcommand{\S}{\hat{S}}
\renewcommand{\empty}{\varnothing}
\newcommand{\J}{\hat{J}}
\newcommand{\lagrange}{\mathcal{L}}
\newcommand{\dfourx}{\mathrm{d}^4x}
\newcommand{\meson}{\phi}
\newcommand{\dpsi}{\psi^\dagger}
\newcommand{\ipic}{\mathrm{int}}
\newcommand{\tr}[1]{\mathrm{tr} \left( #1 \right)}
\newcommand{\C}{\mathbb{C}}
\newcommand{\CP}[1]{\mathbb{CP}^{#1}}
\newcommand{\Vol}[1]{\mathrm{Vol}\left(#1\right)}

\newcommand{\Tr}[1]{\mathrm{Tr}\left( #1 \right)}
\newcommand{\Charge}{\hat{\mathbf{C}}}
\newcommand{\Parity}{\hat{\mathbf{P}}}
\newcommand{\Time}{\hat{\mathbf{T}}}
\newcommand{\Torder}[1]{\mathbf{T}\left[ #1 \right]}
\newcommand{\Norder}[1]{\mathbf{N}\left[ #1 \right]}
\newcommand{\Znorm}{\mathcal{Z}}
\newcommand{\EV}[1]{\left< #1 \right>}
\newcommand{\interact}{\mathrm{int}}
\newcommand{\covD}{\mathcal{D}}
\newcommand{\conj}[1]{\overline{#1}}

\newcommand{\SO}[2]{\mathrm{SO}(#1, #2)}
\newcommand{\SU}[2]{\mathrm{SU}(#1, #2)}

\newcommand{\anticom}[2]{\left\{ #1 , #2 \right\}}


\newcommand{\pathd}[1]{\! \mathdutchcal{D} #1 \:}

\renewcommand{\theenumi}{(\alph{enumi})}


\renewcommand{\theenumi}{(\alph{enumi})}

\newcommand{\atitle}[1]{\title{% 
	\large \textbf{Physics GR8048 Quantum Field Theory II
	\\ Assignment \# #1} \vspace{-2ex}}
\author{Benjamin Church }
\maketitle}

\newcommand{\atitleIII}[1]{\title{% 
	\large \textbf{Physics GR8049 Quantum Field Theory III
	\\ Assignment \# #1} \vspace{-2ex}}
\author{Benjamin Church }
\maketitle}

\theoremstyle{definition}
\newtheorem{theorem}{Theorem}[section]
\newtheorem{definition}{definition}[section]
\newtheorem{lemma}[theorem]{Lemma}
\newtheorem{proposition}[theorem]{Proposition}
\newtheorem{corollary}[theorem]{Corollary}
\newtheorem{example}[theorem]{Example}
\newtheorem{remark}[theorem]{Remark}

 
\newcommand{\dS}[1]{\mathrm{dS}_{#1}}
\newcommand{\AdS}[1]{\mathrm{AdS}_{#1}}

\begin{document}
\section{Conformal Group}

We will consider the groups $\SO{2}{d}$ and $\SO{1}{d+1}$ which are the conformal groups for Lorentzian and Euclidean space-times respectively. We have the symmetry groups,
\begin{enumerate}
\item $\AdS{d+1} \iff \SO{2}{d}$
\item $\dS{d+1} \iff \SO{1}{d+1}$
\item $d$-dimensional CFT $\iff \SO{2}{d}$
\item ``conformal Hilbert spaces'' on $\R^d \iff \SO{1}{d+1}$
\end{enumerate} 
We define $\AdS{d+1}$ by the subspace of $\R^{d+1}$ given by the quadratic form,
\[ -(X^0)^2 - (X^{d})^2 + (X^1)^2 + \cdots + (X^{d-1})^2 = - \ell^2 \]
where the metric on $\AdS{d+1}$ is inherited from the following metric on $\R^d$,
\[ \d{s^2} = - (\d{X^0})^2 - (\d{X^d})^2 + (\d{X^1})^2 + \cdots + (\d{X^{d-1}})^2 \]
We define $\eta_{MN}$ to be the metric tensor (with two time-like directions) above such that,
\[ \d{s^2} = \eta_{MN} \d{X^M} \d{X^N} \]
Now $\SO{2}{d}$ is the group of linear transformatins preserving this quadratic form i.e. isometries of this metric. The generators of these symmetries are given by,
\[ L_{MN} = - i (X_M \partial_{X^N} - X_N \partial_{X^M}) \]
where $X_M = \eta_{MN} X^N$. These generators satisfy,
\[ [L_{MN}, L_{RS}] = -i \left( \eta_{NR} L_{MS} + \eta_{SN} L_{RM} + \eta_{MS} L_{NR} + \eta_{RM} L_{SN} \right) \]
 
\subsection{Coordinates on $\AdS{d+1}$}

We may put global coordinates on $\AdS{d+1}$ via,
\begin{align*}
X^0 & = \sqrt{r^2 + 1} \cos{t} 
\\
X^d & = \sqrt{r^2 + 1} \sin{t} 
\\
X^i & = x^i
\end{align*} 
where $r = \vec{x}^{\, 2}$.
In these coordinates, the metric becomes,
\[ \d{s^2} = - (r^2 + 1) \d{t^2} + \frac{\d{r^2}}{r^2 + 1} + r^2 \d{\Omega} \]
Now consider a heavy particle moving in this space with $r \ll 1$ (with the convention $\ell = 1$) and $|\dot{x}| \ll 1$. Then the action of this particle becomes,
\begin{align*}
S & = - m \int \d{t} \sqrt{- g_{\mu \nu} \dot{x}^\mu \dot{x}^\nu} 
\\
& = - m \int \d{t} \sqrt{(r^2 + 1) - \frac{1}{r^2 + 1} \dot{r}^2 - r^2 \dot{\Omega}^2}
\\
& = - m \int \d{t} \sqrt{1 + r^2} \sqrt{1 - \frac{1}{(r^2 + 1)^2} \dot{r}^2 - \frac{r^2}{r^2 + 1} \dot{\Omega}^2}
\\
& = - m \int \d{t} \left( 1 + \tfrac{1}{2} r^2 \right) \left( 1 - \tfrac{1}{2} \dot{\vec{x}}^{\, 2} \right)
\\
& = \int \d{t} m \left[ \tfrac{1}{2} \dot{\vec{x}}^{\, 2} - \tfrac{1}{2} \vec{x}^{\, 2} - m  \right]
\end{align*}
which is a gravitational harmonic oscillator with $\omega = 1$. If we quantize this action, we find a ground state energy of $E_0 = m + \frac{3}{2}$.  

\subsection{Exact Quantum Mechanics for Particle in $\AdS{4}$}

We give specific names to the generators of the conformal group,
\begin{align*}
H & = L_{04} = i \pderiv{}{t} 
\\
J_i & = \tfrac{1}{2} \epsilon_{ijk} L_{jk} = - i \epsilon_{ikj} x^j \partial_{x^k}
\\
L_i^{\pm} & = L_{i0} \mp i L_{i4} = - i e^{\mp i t}  \left( \sqrt{1 + r^2} \pderiv{}{x^i} \mp i \frac{x^i}{\sqrt{1 + r^2}} \pderiv{}{t} \right)
\end{align*}
These operators satisfy the commutation relations,
\begin{align*}
[H, J_i] = 0 \quad \quad [J_i, J_j] = i \epsilon_{ijk} J_k \quad \quad [H, L_i^{\pm}] = \pm L_i^{\pm} 
\end{align*}
and furthermore,
\begin{align*}
[Ji, L_j^{\pm}] = i \epsilon_{ijk} L_k^{\pm} \quad \quad [L_i^{+}, L_j^{+}] = [L_i^{-}, L_j^{-}] = 0 
\end{align*}
and finally,
\[ [L_i^{-}, L_j^{+}] = 2 \delta_{ij} H - 2 i \epsilon_{ijk} J_k \]
Now supose we have an energy eigenstatre $H \ket{\psi} = E \ket{\psi}$. Then we find,
\begin{align*}
H L_i^{+} \ket{\psi} = ( [ H, L_i^{+}] + L_i^{+} H) \ket{\psi} = (E + 1) L_i^{+} \ket{\psi} 
\end{align*}
We define the ground state by $L_i^{-} \ket{\psi_0} = 0$ and then $H \ket{\psi_0} = E_0 \ket{\psi_0}$. Therefore, we have the states,
\[ \ket{\psi_{n_1, n_2, n_3}} \propto (L_1^+)^{n_1} (L_2^+)^{n_2} (L_3^+)^{n_3} \ket{\psi_0} \]
which has energy,
\[ E_{n_1, n_2, n_3} = E_0 + n_1 + n_2 + n_3 \]
However, we have yet to compute the ground state energy $E_0$. Consider a scalar field with action,
\[ S = - \int \dn{4}{x} \sqrt{-g} \left[ \tfrac{1}{2} g^{\mu \nu} \partial_\mu \phi \partial_\nu \phi + \tfrac{1}{2} m^2 \phi^2 \right] \]
which gives equations of motion,
\[ - \frac{1}{r^2 + 1} \partial_t^2 \phi + \frac{1}{r^2} ( \partial_r (r^2(r^2 + 1) \partial_r \phi)) + \frac{1}{r^2} \nabla_{S^2}^2 \phi = m^2 \phi \]
This is horrendous to solve. However, we know that the ground state satisfies, $L_i^{-} \ket{\psi_0} = 0$. Let $\phi_0 \propto e^{- i \omega_0 t} f(x)$ and thus,
\[ \left( \sqrt{1 + r^2} \pderiv{}{x^i} + \frac{x^i \omega_0}{\sqrt{1 + r^2}} \right) f(x) = 0 \]
Therefore, 
\[ \pderiv{}{x^i} f = - \frac{\omega_0 x^i}{1 + r^2} f \]
which implies that,
\[ \phi_0 = \frac{C e^{- i \omega_0 t}}{(1 + r^2)^{\frac{1}{2} \omega_0}} \]
Plugging in, this given wavefunction satisfies the equations of the motion exactly when $m^2 = \omega_0 ( \omega_0 - 3)$. Thus, we find two solutions for the ground state frequency,
\[ \omega_0 = \frac{3}{2} \pm \sqrt{m^2 + \left( \frac{3}{2} \right)^2} \]
In the high-mass limit,
\[ E_0 = m + \frac{3}{2} + \frac{1}{2 m} \left( \frac{3}{2} \right)^2 + \cdots \]
which, to order $\frac{1}{m}$ reproduces the nonrelativistic limit. 

\subsection{Alternative Coordinates on $\AdS{4}$}

\subsubsection{Poincare Coodinates}

Consider the Poincare coordinate system $\mu = 0,1,2$ with coordinates $u^0, u^1, u^2, z$ and $\eta_{\mu \nu} = \mathrm{diag}(-1,1,1)$. These parametrize the surface via,
\[ X^\mu = \frac{u^\mu}{z} \quad \quad X^3  = \frac{z^2 + u^\mu u_\mu - 1}{2 z} \quad \quad X^4 = \frac{z^2 + u^\mu u_\mu + 1}{2 z} \]
In these new coordinates, the metric becomes,
\[ \d{s^2} = \frac{1}{z^2} \left( \d{u^\mu} \d{u_\mu} + \d{z^2} \right) = \frac{1}{z^2} \left( -\d{u_0^2} + \d{u_1^2} + \d{u_2^2} + \d{z^2} \right) \]
We have manefestly a Poincare $(1, 2)$ symmetry group from transformations of the boundary and a $\SO{1}{2}$ scaling symmetry $z \mapsto \lambda z$ and $u^\mu \mapsto \lambda u^\mu$. There must be a set of extra special conformal transformation to have a complete set of generators. These generators are $M_{\mu \nu}$ and $P_\mu$ corresponding to the Poincare algebra, and $D$ corresponding to scaling, and $K_\mu$ corresponding to the special conformal transformation. These take the form,
\begin{align*}
M_{\mu \nu} & = L_{\mu \nu} = - i \left( u_\mu \partial_{u^\nu} - u_{\nu} \partial_{u^\mu} \right) 
\\
P_\mu & = L_{\mu 4} + L_{\mu 3} = - i \partial_{u^\mu} 
\\
K_\mu & = L_{\mu 4} - L_{\mu 3} = - i \left( (u^2 + z^2) \partial_\mu - 2 u_\mu (u^\nu \partial_\nu + z \partial_z) \right) 
\\
D & = L_{04} = i (u^\nu \partial_\nu + z \partial_z )   
\end{align*}

\subsubsection{Penrse Coordinates}

Take $r = \tan{\rho}$ for $0 \le \rho < \frac{\pi}{2}$. Then the metric becomes,
\[ \d{s^2} = \frac{1}{\cos^2{\rho}} \left( - \d{t^2} + \d{\rho^2} + \sin^2{\rho} \: \d{\Omega_2^2} \right) \]
In these coordinates, light rays traveling radially follow $45^\circ$ lines in a Penrose diagram. 

\subsubsection{Action of $\SO{2}{3}$ on Boundary of $\AdS{4}$}

\section{Feb. 6}

We have the Lorentzian path integral,
\[  \EV{\phi(x) \phi(y)} = \int \pathd{\psi} e^{i S[\phi]} \phi(x) \phi(y) \]
and,
\[ \bra{\phi_1} \Torder{\hat{\phi}(x) \hat{\phi}(y)} \ket{\phi_2} = \int \pathd{\phi}|_{\phi_i}^{\phi_f} \Psi_1[\phi_g]^* \Phi_2[\phi_i] e^{i S[\phi]} \phi(x) \phi(y) \]
which implies that with proper $i\epsilon$ prescription,
\[ \bra{\Omega} \Torder{\hat{\phi}(x) \hat{\phi}(y)} \ket{\Omega} = \int \pathd{\phi} e^{i S[\phi]} |_{i \epsilon} \phi(x) \phi(y)  \]
which (due to the $i\epsilon$ prescription dictating the direction contours can be closed without hitting poles) is equivalent to the Wick rotation $t_E = i t$ which is the deformation $t_E = e^{i \theta} t$ for $\theta$ going $\epsilon \to \tfrac{\pi}{2}$. 

\subsection{Euclidean $\AdS{d+1}$} 

In Global $\AdS{d+1}$ coordinates,
\[ \d{s^2} = - (1 + r^2) \d{t^2} + \frac{\d{r^2}}{1 + r^2} + r^2 \d{\Omega^2} \]
Under the Wick rotation $t_E = i t$ we find,
\[ \d{s^2} = (1 + r^2) \d{t_E^2} + \frac{\d{r^2}}{1 + r^2} + r^2 \d{\Omega^2} \]
In Penrose coordinates, this metric becomes,
\[ \d{s^2} = \frac{1}{\cos^2{\rho}} \left( \d{t_E^2} + \d{\rho^2} + \sin^2{\rho} \d{\Omega^2} \right) \]
The conformal boundary metric as $r \to \infty$ i.e. $\rho \to \pi/2$ gives,
\[ \d{\sigma^2} = \lim_{\rho \to \frac{\pi}{2}} \frac{\d{s^2}}{\cos^2{\rho}} = \d{t_E^2} + \d{\Omega_{d-1}^2} \]
which is the metric of a cylinder. Hyperbilic ball coordinates,
\[ \d{s_E^2} = \frac{4(\d{y_1^2} + \cdots + \d{y_{d+1}^2})}{(1 - |y|^2)^2} \] 
where $y$ is bounded by $|y| < 1$. 

\section{Feb. 8}

In $\AdS{4}$ we have,
\[ - X_0^2 + X_1^2 + \cdots + X_n^2 = - 1\]
The propagator is given by $G = (-\nabla^2 + m^2)^{-1}$. By general symmetry arguments $G(x, y) = G(L(x,y))$  where $L(x, y)$ is the geodesic length from $x$ to $y$. Then,
\[ \cosh{L(x,y)} = - X \cdot Y = X_m Y^m \]
Therefore, we can write $G(x, y) = G(P)$ with $P = X \cdot Y$. Then the equation $(- \nabla^2 + m^2) G = 0$ for $x \neq y$ becomes in embedding space,
\[ \left[ (1 - p^2) \frac{\partial^2}{\partial p^2}  4 p \pderiv{p} + m^2 \right] G = 0 \]
The solutions are linear combinations of $G_{\Delta_+}$ and $G_{\Delta_{-}}$ where $\Delta_{\pm}$ are the solutions to $\Delta(\Delta - d) = m^2$. These solutions can be expressed in terms of hypergeometic functions,
\[ G_{\Delta} = C_\Delta (-2 p)^{-\Delta} {}_2F_1\left(\tfrac{\Delta}{2}, \tfrac{\Delta+1}{2}, \tfrac{\Delta-1}{2}, \tfrac{1}{p^2} \right) \]
Take $y$ fixed and take $x$ to the boundary i.e. $p \to - \infty$. In this limit we find,
\[ G_{\Delta} \sim (-p)^{-\Delta} \sim (\cosh{L})^{-\Delta} \sim e^{- \Delta \cdot L} \]
Therefore, we should ignore solutions with $\Delta < 0$. Furthermore, in the $L \to 0$ limit we must have 
\[ G_{\Delta} \to G_{\text{FLAT}} = (-\nabla_{\R^4}^2 + m^2)^{-1} \xrightarrow{L << m^{-1}} \frac{1}{4 \pi^2 L^2} \]
However, we can compute the $L \to 0$ limit explictly form the functional form to find,
\[ C_{\Delta} = \frac{\Gamma(\Delta)}{2 \pi^{3/2} \Gamma(\Delta - \tfrac{1}{2})} \]

\subsection{Poincare Patch}

In Poincare coordinates,
\[ \d{s^2} = \frac{\d{\vec{u}^{\, 2}} + \d{z^2}}{z^2} \]
In these coordinates,
\[ - P = \cosh{L} = - X^m X_m = \frac{(\vec{u} - \vec{u}')^2 +  z^2 + z'^2}{2 z z'} = 1 + \frac{(\vec{u} - \vec{u}')^2 + (z - z')^2}{2 z z'} \]
If $z = z'$ then,
\[ \cosh{L} = 1 + \frac{(\vec{u} - \vec{u}')^2}{2 z^2} \]
Thus for small distances,
\[ L \approx \frac{|\vec{u} - \vec{u}'|^2}{z} \]
However, in the large distance limit $L \to \infty$ then $\cosh{L} \to \tfrac{1}{2} e^L$. Then we find
\[ e^L \approx \frac{(\vec{u} - \vec{u}')^2}{z^2} \]
so we have 
\[ L \approx 2 \log{\frac{|\vec{u} - \vec{u}'|^2}{z}} \]
thus the geodesic length only grows logarithmically. 

\subsection{Boundary Operators}

Therefore we can compute using our functional form,
\[ \lim_{z, z' \to 0} G((z, \vec{u}), (z', \vec{u}')) = C_\Delta (- 2 p)^{-\Delta} \to C_\Delta \left( \frac{|\vec{u} - \vec{u}'|^2}{z z'} \right)^{-\Delta} = C_\Delta \frac{z^\Delta z'^\Delta}{|\vec{u} - \vec{u}'|^{2 \Delta}} \]
And thus,
\[ \lim_{z, z' \to 0} \EV{\phi(z, \vec{u}) \phi(z', \vec{u}')} = C_\Delta \frac{z^\Delta z'^\Delta}{|\vec{u} - \vec{u}'|^{2\Delta}} \]
\begin{definition}
The boundary operator is defined as,
\[ \Op(\vec{u}) = \lim_{z \to 0} z^{-\Delta} \phi(z, \vec{u}) \]
And thus we find,
\[ \EV{\Op(\vec{u}) \Op(\vec{u}')} =  \frac{C_\Delta}{|\vec{u} - \vec{u}'|^{2\Delta}} \]
recovering the general form of a CFT 2-point function.
\end{definition}

\subsection{Local CFT on the Boundary}

We want the CFT on the boundary to have local operators such as $T_{ab}$ which are conserved currents due to the conformal symmetries of the boundary. 

\subsection{Black Hole Thermodynamics in $\AdS{4}$}

Suppose we have some CFT on the boundary $\R^{1,2}$. Suppose we have some action $S[\phi]$ then we may compute the thermodynamics by a Wick rotation $S[\phi] \to S_{E}[\phi]$ sending 
\[ \d{s^2} = - \d{t^2} + \d{x^2} + \d{y^2} \mapsto \d{s^2} = \d{t_E^2} + \d{x^2} + \d{y^2} \]
under the identification $\tau \sim \tau + \beta$. We consider the theory constrained to be a ``box'' $S^2$ which is the boundary in Euclidean Global coordinates where the spatial part of the boundary lives on $S^2$. Thus, we take the metric,
\[ \d{s^2} = \d{\tau^2} + \d{\Omega_2^2} \]
We consider the bulk theory,
\[ Z(\beta) = \int \pathd{g} \pathd{\phi} e^{-I_E[g, \phi]} \] 
where $I_E[g, \phi]$ is the Euclidean gravitational action. At tree level the leading contribution to the partition function comes from the local minima,
\[ Z(\beta) \approx \sum_{\text{min} I_E[g_*] } e^{- I_E[g_*, \phi] } \]
For now, we assume that at these minima we may set $\phi = 0$ i.e. the matter fields are unimportant. Equivalently, we could consider gravity-only theory. Thus we need to find the minima of the Euclidean Einstein-Hilbert action,
\[ I_E[g] = \frac{1}{16 \pi G} \int \sqrt{g} (R + 2 \Lambda) \]
subject to the boundary conditions that the space is asymtotically $\AdS{4}$ i.e. as $r \to 0$ we have $g \to g_{\text{global AdS}}$ which is the metric,
\[ \d{s^2} = (1 + r^2) \d{\tau^2} + \frac{\d{r^2}}{1 + r^2} + r^2 \d{\Omega^2} \]
Therefore, we may write our boundary conditions as,
\[ \d{s^2} \to r^2 \d{\tau^2} + \frac{\d{r^2}}{r^2} + r^2 \d{\Omega^2} \]
in the limit $r \to \infty$. Then we idenfity the boundary metric $\d{\sigma^2} = \d{\tau^2} + \d{\Omega^2}$ with $\tau \sim \tau + \beta$. The minima of $I_E$ are solutions to Einsteins equations. Global EAdS gives one minimum. However, Schwarchild solutions in AdS also give solutions to Einstein's equations. The AdS Schwarchild solution is,
\begin{align*}
\d{s^2} = \left( 1 + r^2 - \frac{A}{r} \right) \d{\tau^2} + \frac{d{r^2}}{1 + r^2 - \frac{A}{r}} + r^2 \d{\Omega^2} 
\end{align*} 
The Schwarchild radius of the horizon appears at $r_+$ such that $1 + r_+^2 - A r_+^{-1} = 0$ and thus $A = r_+(1 + r_+^2)$. However, the entire family of solutions is not allowed since we need to match our boundary conditions. For the metric to be a minimum we need the solution to be smooth. Consider the general metric form,
\[ \d{s^2} = V(r) \d{\tau^2} + \frac{\d{r^2}}{V(r)} + r^2 \d{\Omega^2} \]
Define $R^2 = V(r)$ then we have $R \d{R} = \tfrac{1}{2} V'(r) \d{r}$. Thus,
\[ \frac{\d{r^2}}{V(r)} = \frac{\d{R^2}}{\left( \tfrac{1}{2} V'(r) \right)^2} \]
Therefore, as $R \to 0$ i.e. as $r \to r_+$ the metric takes the form,
\[ \d{s^2} = R^2 \d{\tau^2} + \frac{4 \d{R^2}}{V'(r_+)^2} + r_+^2 \d{\Omega^2} + O(R^3) \]
Or, rewriting,
\[ \d{s^2} = \frac{4}{V'(r_+)^2} \left[ \d{R^2} + R^2 \d{\phi^2} \right] + r_+^2 \d{\Omega^2} + O(R^3) \]
where we have set,
\[ \varphi = \frac{V'(r_+)}{2} \tau \]
we require that $\varphi$ be a polar coordinate and thus have period $2 \pi$.
Thus, For this to not have any conical singularities, we must have, 
\[ \beta = \frac{4 \pi}{V'(r_+)} \]

\subsubsection{Flat Space}

In flat space, we have the ordinary Schwarchild metric,
\[ \d{s^2} = \left( 1 - \frac{r_+}{r} \right) \d{\tau^2} + \frac{\d{r^2}}{1 - \frac{r_+}{r}} + r^2 \d{\Omega^2} \]
Therefore,
\[ V'(r_+) = \frac{r_+}{r_+^2} = \frac{1}{r_+} \]
which gives a temperature,
\[ T = \frac{V'(r_+)}{4 \pi} = \frac{1}{4 \pi r_+} \]
Using the relation $r_+ = 2 M G$ we find a temperature,
\[ T = \frac{1}{8 \pi G M} \]
Given this temperature, we use the entropy relation,
\[ \d{S} = \frac{1}{T} \d{M} = 4 \pi r_+ \cdot \frac{1}{2 G} \d{r_+} = \frac{1}{4 G} \d{(4 \pi r_+^2)} = \frac{1}{4 G} \d{A} \]
Setting the entropy to zero at zero mass we find the entropy of the black hole to be,
\[ S = \frac{A}{4 G} \]

\subsubsection{Back to AdS}

The metric of a AdS Black Hole is,
\[ \d{s^2} = V(r) \d{\tau^2} + \frac{\d{r^2}}{V(r)} + r^2 \d{\Omega^2} \]
where,
\[ V(r) = 1 + r^2 - \frac{A}{r} \]
and $r_+$ is defined such that $V(r_+) = 0$ and $r_+$ is the largest such root. Thus, $A = r_+(r_+^2 + 1)$. We have showed that,
\[ T_H = \frac{1}{\beta} = \frac{V'(r_+)}{4 \pi} = \frac{1}{4 \pi} \left( 2 r+ r_+ + \frac{A}{r_+^2} \right) = \frac{1}{4 \pi} \left( 2 r_+ + \frac{r_+(r_+^2 + 1)}{r_+^2} \right) = \frac{1}{4 \pi} \left( 3 r_+ + \frac{1}{r_+} \right) \]
If the black hole is much smaller than $1$, i.e. the AdS horizon, then the $r_+^{-1}$ term dominates which reproduces the Schwarchild case. However, for $r_+ \gg 1$ we find a temperature growth with radius. Now we return to computing the parition function, 
\[ Z = \sum_{g^*} e^{- I_E[g^*]} \]  


\section{Feb. 22}

\subsection{Examples From String Theory}

Type IIB string theory on $\AdS{5} \times S^5$ gives $\mathcal{N} = 4$ $\mathrm{SU}(N)$ super-Yang-Mills theory in $3+1$ dimensions. In the near horizon limit of a Black Brane we have,
\[ \d{s^2} = Q^2 \left( \frac{-\d{t^2} + \d{x_1^2} + \d{x_2^2} + \d{x-3^2} + \d{z^2}}{z^2} + \d{\Omega^2} \right) \]
Semiclassical gravity approximation reliable if $ \ell_{\text{AdS}} \gg \ell_{\text{Planck}}^{5D}$ and thus loop corrections are small. Furthermore we need $\ell_{\text{AdS}} \gg \ell_{S}$ where $\ell_{S}$ is the string scale with $T_S = \ell_S^{-2}$ the string tension.
\bigskip\\
Quantizing string actions gives the particle spectrum. We consider scattering and interactions via worldsheet interactions,
\[ \int \pathd{X} e^{- \int_{\Sigma} \d{A}} = \int \d{\tau} \int \pathd{X} e^{- \int \d{t} \d{s} \left[ (\partial_t X)^2 + (\partial_x Y)^2 \right]} \]
which is a generalization of the world-line formulation of QFT scattering via,
\begin{align*}
D(x, y) & = \int \frac{\dn{4}{p}}{(2 \pi)^4} \frac{1}{p^2 + m^2} e^{i p \cdot (x - y)}
\\
& = \int_0^\infty \d{\tau} \int_p e^{-\tau(p^2 + m^2)} e^{i p \cdot (x - y)} 
\\
& = \int_0^\infty \d{\tau} \bra{y} e^{-\tau(\hat{p}^2 + m^2)} \ket{x} 
\\
& = \int_0^\infty \d{\tau} \int \pathd{X} \Big|_x^y e^{-\int_0^\tau \d{t} (\tfrac{1}{4} \dot{X}^2 + m^2)} 
\end{align*}

\subsection{Worldline}

The worldline metric is $\d{s^2} = g \d{t^2}$. 
Diffeomorphism invariant worldline action,
\[ S = \frac{1}{2} \int \sqrt{g} \d{t} \left( g^{-1} \partial_t X^\mu \partial_t X_\mu + m^2 \right) \]
Writing $g = e(t)^2$ which is dynamical we find,
\[ S = \int \d{t} \left( \frac{1}{2 e} \dot{X}^2 + \frac{1}{2} e m^2 \right) \]
Because the theory is one-dimensional, there is no dynamical Einstein-Hilbert action for $e$. Varying $e$ in the above action gives,
\[ - \frac{1}{e^2} \dot{X}^2 + m^2 = 0 \implies e = \frac{\sqrt{\dot{X}^2}}{m} \]
Therefore, we find,
\[ S = \int \d{t} m \sqrt{\dot{X}^2} \]
which gives the standard action for a free particle. However, this theory has gauge invariance under diffeomorphism invariance. Therefore, the path integral,
\[ \int \pathd{X} \pathd{e} e^{-S[X, e]} \]
requires Gauge fixing $\dot{e} = 0$ we find that we need to integrate over the Gauge fixing parameter $\tau$ which gives the length of the path. 

\subsection{Polyakov Action}

We seek a diffeomorphism invariant action. The worldsheet is parametrized by coordinates $u^i$ for $i = 1,2$ with a worldsheet metric,
\[ \d{s^2} = h_{ij} \d{u^i} \d{u^j} \]
Then we have the Polyakov Action,
\[ S = \int T_S \dn{2}{u} \sqrt{h} \left( h^{ij} \partial_i X^\mu(u) \partial_j X_\mu(u) \right) + \frac{\phi_S}{4 \pi} \int \dn{2}{u} \sqrt{h} R \] 
where $X^\mu(u)$ are the embedding coordinates of the worldsheet parametrized by $u$ and $T_S$ is the string tension with units of energy per unit length. The second term is the Einstein-Hilbert action where $\phi_S$ is a dimensionless constant. However, in two dimensions, this second terms is actually a topological invariant so there is no contribution from the additional gravity term. Note that there is no mass terms leading to an additional Gauge invariance $h_{ij}(u) \to \lambda(u) h_{ij}(u)$ known as Weyl invariance.  

\subsection{Quantum Strings}


\subsubsection{Particles}

Again consider the particle case, $\dot{X}^2 = -(\dot{X}_0)^2 + (\dot{X}^i)^2$ with action,
\[ S = \int \d{\tau} \left( \frac{1}{2 e} \dot{X}^2 - \frac{e}{2} m^2 \right) \]
This action may be rewritten in phase space.
\[ S = \int \d{\tau} \left( p \dot{X} - e(p^2 + m^2) \right) \]
Then the Hamilton equatons of motion are,
\[ \dot{X} = e p \quad \quad \dot{p} = 0 \quad \quad p^2 + m^2 = 0 \]
There is an analog of the mode expansion giving,
\[ X^\mu(\tau) = p^\mu \tau + x^\mu \quad \quad p^0 = \sqrt{p_i^2 + m^2} \]
where we Gauge fix to $e = 1$. This choice leaves a residual gauge symmetry given by $\tau$ translation. This shifts $x^\mu \\mapsto x^\mu + p^\mu \tau$ use this to put $x^0 = 0$. The physical degrees of freedom $(x^i, p^i)$

\subsubsection{Strings}

The action is,
\[ S = \frac{1}{2} \int \dn{2}{\tau} \sqrt{-h} \left( h^{\alpha \beta} \partial_\alpha X \partial_\beta X \right) \]
We have the classical constraint,
\[ \frac{\delta S}{\delta h_{\alpha \beta}} = 0 \implies T_{\alpha \beta} = 0 \]
where,
\[ T_{\alpha \beta} = \partial_\alpha X \partial_\beta X - \tfrac{1}{2} h_{\alpha \beta} (\partial X)^2 \]
Imposing the gauge fixing condition that $h_{\alpha \beta} = \eta_{\alpha \beta}$, the action becomes,
\[ S = \frac{1}{2} \int \d{\tau} \d{\sigma} \left( (\partial_\tau X)^2 - (\partial_\sigma X)^2 \right) \]
The (almost physical degrees of freedom are $D - 2$ scalar fields $i = 1, \dots, D-2$ we have creation operators,
\[ [\a_k^i, (\a_k^j)^\dagger] = \delta^{ij} \delta_{k k'}\]
with $k \in \mathbb{Z}$ because the momentum is constrained to a circle. There is also the center of mass momentum $p^\mu$ subject to the global constraints,
\[ H = 0 \quad \quad L = 0 \]
where,
\[ H = \tfrac{1}{2} p^2 + \sum_{i,k} N_{i,k} |k| + E_0  = 0 \quad \quad K = \sum_{i,k} N_{i,k} \cdot k = 0 \] 
and $E_0$ is the Casimir energy on the worldsheet. 
The second condition gives,
\[ \sum_{k > 0} N_{i,k} \cdot |k| = \sum_{k < 0} N_{i,k} \cdot |k| \]
refered to as the level matching condition which may be written as $E_+ = E_-$. 

\subsubsection{Spectrum}

From the condition $H = 0$ we get $p_0 = \sqrt{p_i^2 + 2 (E_{+} + E_{-} + E_0)}$ which implies that $M^2 = 2 (E_{+} + E_{-} + E_0)$ is the physical mass of the wordsheet. In principle,
\begin{align*}
E_0 = \sum \text{(zero point energies)} = \sum_{k,i} \tfrac{1}{2} |k| = (D - 2) \cdot 2 \cdot \tfrac{1}{2} \sum_{i = 1}^{\infty} k = (D - 2) \sum_{i = 1}^\infty k 
\end{align*}
We apply the regularization scheme,
\[ \sum_{i = 1}^\infty k \mapsto \sum_{i = 1}^\infty k e^{- \epsilon k} = - \pderiv{}{\epsilon} \frac{1}{1 - e^{-\epsilon}} = \frac{1}{\epsilon^2} - \frac{1}{12} \]
The first term can be canceled by renormalization of the cosmological constant. Thus we find,
\[ E_0 = - \frac{D - 2}{12} \]
The ground state of the string $\ket{0, \vec{p}}$ which is annihilated by all operators $\a^i_k$ has $E_{+} = E_{-} = 0$ and thus has energy,
\[ M^2 = - \frac{D - 2}{6} < 0 \]
which indicates a tachyonic instability for $D > 2$. The first excited state takes the form, $\ket{ij : \vec{p}} = (\a^i_{1})^\dagger (\a^j_{-1})^\dagger \ket{0, \vec{p}}$ to satisfy the level matching condition. The mass of this state becomes, 
\[ \tfrac{1}{2} M^2 = - \frac{D - 2}{12} + 2 \]
The most general first excited state of momentum $\vec{p}$ is,
\[ \ket{\psi} = \sum_{ij} \psi_{ij} \ket{ij, \vec{p}} \]
We split $\psi_{ij}$ into a symmetric traceless part, a trace part, and an antisymmetric part. These give a representation of $\SO{D - 2}$. To preserve Lorentz symmetry, we need that this embedds into a unitary representation of the $D$-dimensional Poincare group. In $D = 4$, we have $\SO(2)$ isometry with basis $\ket{1,1}, \ket{1,2}, \ket{2,1}, \ket{2,2}$. A more convienient basis is,
\[ \ket{1}, \ket{2} \mapsto \ket{\pm} = \tfrac{1}{\sqrt{2}} \left( \ket{1} \pm i \ket{2} \right) \] 
Then $J_{12} \ket{+} = \ket{+}$ and $J_{12} \ket{-} = - \ket{-}$. Then we have a total basis,
\begin{align*}
J_{12} \ket{++} & = + 2 \ket{++} \\
J_{12} \ket{++} & = 0 \\
J_{12} \ket{++} & = 0 \\
J_{12} \ket{++} & = - 2 \ket{++} 
\end{align*} 
However, for this excited state $M^2 > 0$ but a massive particle cannot have such a representation becausing it is missing the spin $\pm 1$ states. In general there will be missing vector particles so this algebra never fits a massive representation of the Lorentz group but it does match a massless representation. Therefore to preserve Poincare invariance, we must have $M^2 = 0$ which implies $D = 26$ by the above formula. 

\section{March. 1}

After fixing the worldsheet metic,
\[ \d{s^2} = - \d{\tau^2} + \d{\sigma^2} \]
there is residual gauge symmetry given by diffeomorphisms and Weyl rescalings which preserve the above metric. That is, $(\tau, \sigma) \to (\tilde{\tau}, \tilde{\sigma})$ such that,
\[ \d{s^2} = - \d{\tau^2} + \d{\sigma^2} = \rho(\tilde{\tau}, \tilde{\sigma}) (- \d{\tilde{\tau}^2} + \d{\tilde{\sigma}^2} \]
which are exactly the conformal transformations in two dimensions. Define the light-conce coordinates, $\tau^{\pm} = \tau \pm \sigma$ in whcich the metric becomes,
\[ \d{s^2} = - \d{\tau^+} \d{\tau^-} \]
Therefore, if we define new coordinates via the special functional form, $\tau^{-}(\tilde{\tau}^+)$ and $\tau^{-}(\tilde{\tau}^{+})$ then implies that,
\[ \d{s^2} = - \left( \pderiv{\tau^+}{\tilde{\tau}^+} \pderiv{\tau^-}{\tilde{\tau}^-} \right) \d{\tilde{\tau}^+} \d{\tilde{\tau}^-} \]
is automatically conformal. Therefore, the two-dimensional conformal algebra is infinite dimensional and in correspondence with holomorphic transfromations. We use this residual symmetry to choose the lightcone gauge,
In the $D$-dimensional target space we can also choose lightcone coordinates,
\[ X^{\pm} = \tfrac{1}{\sqrt{2}} (x^0 \pm x^1) \]
in which the embedding space metric becomes,
\[ \d{s_D^2} = - 2 \d{x^+} \d{x^-} + \sum_{i = 2}^{D-1} \d{x_i^2} \] 
Therefore $g^{(D)}_{+-} = g^{(D)}_{-+} = -1$. In these coodinates, the worldsheet equation of motion becomes,
\[ \pderiv{}{\tau^+} \partial{}{\tau^-} X^\mu = 0 \implies X^\mu = X_L^\mu(\tau^+) + X_R^\mu(\tau^-) \]
The constraint $T_{\alpha \beta} = 0$ imposes $(\partial_{+} X)^2 = (\partial_{-} X)^2 = 0$. Then we write,
\begin{align*}
X_L^\mu(\tau^+) & = \tfrac{1}{2} x^\mu + \tfrac{1}{2} p^\mu \tau^{+} + \sum_{n \neq 0} \alpha_n^\mu e^{i n \tau^+}
\\
X_R^\mu(\tau^-) & = \tfrac{1}{2} x^\mu + \tfrac{1}{2} p^\mu \tau^{-} + \sum_{n \neq 0} \tilde{\alpha}_n^\mu e^{i n \tau^-}  
\end{align*}
Now choose the $\tau$ coordinate via,
\[ \tau = \frac{X^+ - x^+}{p^+} \]
such that we remove the $X^+$ oscillations to get,
\begin{align*}
X_L^\mu(\tau^+) & = \tfrac{1}{2} x^\mu + \tfrac{1}{2} p^\mu \tau^{+} 
\\
X_R^\mu(\tau^-) & = \tfrac{1}{2} x^\mu + \tfrac{1}{2} p^\mu \tau^{-} 
\end{align*}
Furthermore, we have the constraint,
\[ - 2 \partial_+ X^+ \partial_+ X^- + (\partial_+ X^i)^2 = 0 \]
and $\partial_+ X^+ = p^+$ giving,
\[ \partial_+ X^- = \frac{1}{p^+} (\partial_+ X^i)^2 \quad \text{and} \quad \partial_- X^- = \frac{1}{p^+} (\partial_- X^i)^2 \]
Therefore, this removes two dynamical degrees of freedom. 
\bigskip\\
The sates in the Fock space can be written in the form,
\[ (\a_{k_1}^{i_1})^\dagger \cdots (\a_{k_n}^{i_n})^\dagger \ket{p} \]
The level maching condition gives the constraint,
\[ K = 0 \implies \sum_{k < 0} N_k |k| = \sum_{k > 0} N_k |k| \]
and the energy constraint gives,
\[ H = 0 \implies M^2 / M_S^2 = E_0 + \sum_{k} N_k |k| = - \frac{D - 2}{12} + \sum_{k} N_k |k| \] 
Setting $D = 26$, we have $\ket{p}$ with mass $M^2 = - 2 M_S^2$ and massless sates,
\[ (a_{+1}^i)^\dagger (a_{-1}^j)^\dagger \ket{p} \]
and then massive states with multiple excitations. 

\section{Supersymmetric String Theory}
\newcommand{\Z}{\mathbb{Z}}

For the bosonic $X^\mu$ we seperate the left-moving and right-moving modes $\a^i_k$ and $\tilde{\a}^i_k$ which have energy $k$ and momentum $\mp k$. For fermions on the worldsheet cooresponding to the field $\psi^\mu$ we also seperate the left-moving and right-moving modes to $\c^i_k$ and $\tilde{c}^i_k$ with energy $k$ and mometum $\mp k$. For Bosons the boundary condition $X^\mu(\sigma + 2 \pi) = X^\mu(\sigma)$ fixes $k \in \Zplus$. For fermions we can either take periodic or anti-periodic boundary conditions which give $k \in \Z_{\ge 0}$ or $k \in \tfrac{1}{2} + \Z_{\ge 0}$. 
\bigskip\\
The constraint $K = 0$ give a level maching condition and the constrait $H = 0$ fixes the mass of the string state to,
\[ M^2 = M_S^2 \left( E_0 + E_\text{osc} + \tilde{E}_{\text{osc}} \right) \] 
For fermionic modes, the Casimir or vacuum energy is,
\[ E_0^{\text{ferm. left.}} = - \sum_{k} \tfrac{1}{2} k e^{- \epsilon k} \]
In the periodic case, 
\[ E_0^{\text{ferm. left.}} = - \frac{1}{2 \epsilon^2} + \frac{1}{24} \]
and in the antiperiodic case,
\[ E_0^{\text{ferm. left.}}  = - \frac{1}{2 \epsilon^2} - \frac{1}{48} \]
We may choose the boundary conditions of the left and right movers independently giving four sectors:
\begin{enumerate}
\item[PP] $E_0^{\text{tot}} = - \frac{1}{\epsilon^2} + \frac{1}{12} + \frac{1}{\epsilon^2}  - \frac{1}{12} = 0$
\item[PA] $E_0^{\text{tot}} = - \frac{1}{\epsilon^2} + \frac{1}{48} + \frac{1}{\epsilon^2}  - \frac{1}{12} = - \frac{1}{16}$
\item[AP] $E_0^{\text{tot}} = - \frac{1}{\epsilon^2} + \frac{1}{48} + \frac{1}{\epsilon^2}  - \frac{1}{12} = - \frac{1}{16}$
\item[AA] $E_0^{\text{tot}} = - \frac{1}{\epsilon^2} - \frac{1}{24} + \frac{1}{\epsilon^2}  - \frac{1}{12} = - \frac{1}{8}$
\end{enumerate}

\section{Closed String Scattering}

At tree level for bosonic strings, have Mandelstam variables,
\begin{align*}
s & = - (p_1 + p_2)^2 / M_s^2
\\
s & = - (p_1 + p_3)^2 / M_s^2
\\
s & = - (p_1 + p_4)^2 / M_s^2
\end{align*}
This gives a scattering amplitude for,
\[ \mathcal{A} \sim \delta_{\sum_i p_i} \frac{\Gamma\left( - \tfrac{s}{2} - 1 \right) \Gamma\left( - \tfrac{t}{2} - 1 \right) \Gamma\left( - \tfrac{u}{2} - 1 \right)}{\Gamma\left( - \tfrac{s+t}{2} - 1 \right)\Gamma\left( - \tfrac{s+u}{2} - 1 \right)\Gamma\left( - \tfrac{u+t}{2} - 1 \right)} \]
This has poles at $s = -2,0,2,4,6, \cdots$ which corresponds to massive states,
\[ \frac{M^2}{M_s^2} = -2,0,2,4,6, \cdots \]
As opposed to QFT, in string theory at a fixed scattering angle,
\[ \mathcal{A} \sim e^{-f(\theta) s} \]
therefore the scattering amplitudes fall off exponentially for large $s$ i.e. energies greater than the string scale. 

\subsection{Loops}


\section{Open Strings}

An open string has boundaries and thus we must impose boundary conditions. Consider the string world-sheet integral,
\[ S = \int \d{\sigma} \d{\tau} \tfrac{1}{2} [(\partial_\tau X)^2 - (\partial_\sigma X)^2 ] \]
Conventionally we take the boundaries at $\sigma = 0$ and $\sigma = \pi$. Due to the time-translation invariance of $\tau$ there is a conserved current,
\[ j^\tau = \tfrac{1}{2} [ (\partial_\tau X)^2 + (\partial_\sigma X)^2 ] \quad \quad j^\sigma = - \partial_\tau X \partial_\sigma X \]
which satisfies,
\[ \partial_\tau j^\tau + \partial_\sigma j^\sigma = 0 \]
For energy to not flow off the string we require $j^\sigma(\tau, \sigma = 0) = 0$ and $j^\sigma(\tau, \sigma = \pi) = 0$. Therefore, we require $(\partial_\tau X) (\partial_\sigma X) |_{\sigma = 0, \pi} = 0$. This gives two choices of boundary conditions, Dirichlet: $\partial_\tau X = 0$, and Neumann: $\partial_\sigma X = 0$. 

\begin{example}[D3-BRANE in 10D]
Three spatial directions and time along brane $\mu = 0,1,2,3$ for which have Neumann boundary conditions and six transversal dirctons $\mu = 4,5,6,7,8,9$ which have Dirichlet boundary conditions. Then we may expand for $\mu = 0,1,2,3$,
\[ X^\mu = x^\mu + p^\mu \tau + \sum_n [a_n^\mu e^{-i n \tau} \cos{n \sigma} + (a^\mu_n)^\dagger e^{in \tau} \cos{n \sigma} ] \]
and for $\mu = 4,5,6,7,8,9$,
\[ X^\mu = \sum_n [a_n^\mu e^{-i n \tau} \sin{n \sigma} + (a^\mu_n)^\dagger e^{in \tau} \sin{n \sigma} ] \]
because the momentum must vanish since $\partial_\tau X^\mu = 0$. 
\bigskip\\
Now we go into light-cone coordinates $\tau^{\pm} = \tau \pm \sigma$. Which gives equations of motion, $\partial_{+} \partial_{-} X = 0$ meaning that we can decompose,
\begin{align*}
X(\sigma, \tau) & = X_L(\sigma + \tau) + X_R(\sigma - \tau) 
\\
& = \tfrac{1}{2} x^\mu + \tfrac{1}{2} p^\mu (\sigma + \tau) + \sum_n [a^\mu_n e^{-i n (\tau + \sigma)} + (a^\mu_n)^\dagger e^{in(\tau + \sigma)} ]
\\
& +  \tfrac{1}{2} x^\mu + \tfrac{1}{2} p^\mu (\sigma - \tau) + \sum_n [ \tilde{a}^\mu_n e^{-i n (\tau - \sigma)} + (\tilde{a}^\mu_n)^\dagger e^{in(\tau - \sigma)} ]
\end{align*}
Therefore, in the Neumann directions we take $\tilde{a}^\mu_n = + a^\mu_n$ and in the Dirichlet directions $\tilde{a}^\mu_n = - a^\mu_n$.
\end{example} 

\subsection{Fermions}

Consider the two-dimensional Dirac algebra $\{ \gamma^\alpha, \gamma^\beta \} = 2 \eta^{\alpha \beta}$ with $\gamma^0 = \sigma^1$ and $\gamma^2 = i \sigma^2$. Then the action for a Fermionic string is,
\[ S = \int \d{\tau} \d{\sigma} i \bar{\psi}^\mu \gamma^\alpha \partial_\alpha \psi_\mu \] 
Explicitly,
\begin{align*}
S = \int \d{\tau} \d{\sigma} \left[ \psi_1^* (\partial_\tau - \partial_\sigma) \psi_1 + \psi_2^*(\partial_\tau + \partial_\sigma) \psi_2 \right]
\end{align*}
Therfore, the equations of motion give $(\partial_\tau - \partial_\sigma) \psi_1 = 0$ and $(\partial_\tau + \partial_\sigma) \psi_2 = 0$ and therefore $\phi_1$ is left-moving and $\phi_2$ is right-moving. Then we may write down the mode expansion,
\begin{align*}
\psi^\mu_L & = \sum_n [b^\mu_n e^{i n (\tau + \sigma)} + (c^\mu_n)^\dagger e^{- in(\tau + \sigma)}]  
\\
\psi^\mu_R & = \sum_n [\tilde{c}^\mu_n e^{i n (\tau - \sigma)} + (\tilde{b}^\mu_n)^\dagger e^{- in(\tau - \sigma)}]  
\end{align*}
Here we have more fermionic degrees of freedom because there are complex fields as opposed to the real scalar fields. Therefore, we impose the Majorana condition $c^\mu_n = b^\mu_n$ and $\tilde{c}^\mu_n = \tilde{b}_n^\mu$. In the fermionic sector, the energy current is,
\[ j^\alpha = i \bar{\psi} \gamma^\alpha \psi \]
which must be conserved at the boundary. Therefore, we find that,
\[ j^\sigma|_{\sigma = 0,\pi} = 0 \implies \psi_1^* \psi_2 - \psi_2^* \psi_2 = 0 \]
This again gives two choices, $\tilde{b}^\mu_N = + b^\mu_n$ or $\tilde{b}^\mu_n = - b^\mu_n$. 
\bigskip\\
Then we get the spectrum. 

\section{Supersymmetric Quantum Mechanincs}

Simple example 1D quantum mechanics,
\[ \Psi = \begin{pmatrix}
\psi_1(x)
\\
\psi_2(x)
\end{pmatrix} \]
Consider the particle confined harmonically to a plane with a constant magnetic field such that the Hamiltonian is,
\[ \hamilt = \begin{pmatrix}
- \tfrac{1}{2} \partial_x^2 + \tfrac{1}{2} x^2 - B & 0
\\
0 & -\tfrac{1}{2} \partial_x^2 + \tfrac{1}{2} x^2 + B 
\end{pmatrix} \]
The spectrum of this theory is given by,
\begin{align*}
\hamilt \ket{n \uparrow} & = (n + \tfrac{1}{2} - B) \ket{n \uparrow} \\
\hamilt \ket{n \downarrow} & = (n + \tfrac{1}{2} +B) \ket{n \downarrow} 
\end{align*}
Consider the special case in which $B = \tfrac{1}{2}$. Then we have,
\begin{align*}
\hamilt \ket{n \uparrow} & = n \ket{n \uparrow} \\
\hamilt \ket{n \downarrow} & = (n+1) \ket{n \downarrow} 
\end{align*}
so all levels are double degenerate except for the ground state $\ket{0 \uparrow}$. This degeneracy implies a symmetry which is given by a Hermtian charge $Q^\dagger = Q$ which commutes with the Hamiltonian $[Q, H] = 0$. Define creation and anihilation operators,
\[ \hat{a} = \frac{1}{\sqrt{2}} (x + \partial_x) \quad \quad \hat{a}^\dagger = \frac{1}{\sqrt{2}} (x - \partial_x) \]
in which we can write,
\[ H = \begin{pmatrix}
a^\dagger a & 0 
\\
0 & a^\dagger a + 1 
\end{pmatrix} \]
Consider,
\[ Q_1 = 
\begin{pmatrix}
0 & a^\dagger 
\\
a & 0 
\end{pmatrix} \quad \quad Q_2 = 
\begin{pmatrix}
0 & - i a^\dagger
\\
i a & 0
\end{pmatrix} \]
Then we have,
\[ Q_1^2 = Q_2^2 = 
\begin{pmatrix}
a^\dagger a & 0 
\\
0 & a a^\dagger
\end{pmatrix} = H \]
Therefore is is clear that $[Q_1, H] = [Q_2, H] = 0$. Furthermore, 
\[ \{ Q_1, Q_2 \} = 0 \]
Therefore we have an anticommutator algebra,
\[ \{ Q_i, Q_j \} = 2 \delta_{ij} H \]
This is called a supersymmetry. Since $Q^\dagger = Q$ then $Q^2 = H \ge 0$. Therefore, to find zero energy states,
\[ H \ket{\psi} = 0 \iff Q_i \ket{\psi} = 0 \]
Now define,
\[ Q = 
\begin{pmatrix}
0 & 0 
\\
a & 0
\end{pmatrix}
\quad \quad Q^\dagger = 
\begin{pmatrix}
0 & a^\dagger
\\
0 & 0 
\end{pmatrix} \]
Therefore, $Q_1 = Q + Q^\dagger$ and $Q_2 = i (Q - Q^\dagger)$. In terms of these new operators we have,
\[ Q_i \ket{\psi} = 0 \iff Q \ket{\psi} = 0 \text{ and } Q^\dagger \ket{\psi} = 0 \]
Furthermore,
\[ H = \tfrac{1}{2} \{ Q, Q^\dagger \} \quad \quad Q^2 = 0 \quad \quad (Q^\dagger)^2 = 0 \]
Discrete ``spin'' degree of freedom in QM is equivalent to a Fermionic operator,
\begin{align*}
c \ket{\downarrow} & = \ket{\uparrow}
\\
c \ket{\downarrow} & = \ket{\uparrow}
\end{align*}
Therefore,
\[ c = 
\begin{pmatrix}
0 & 1
\\
0 & 0 
\end{pmatrix} 
\quad \quad \quad c^\dagger 
= 
\begin{pmatrix}
0 & 0
\\
1 & 0
\end{pmatrix} \]
and these satisfy $c^2 = (c^\dagger)^2 = 0$ and $\{ c, c^\dagger \} = 1$. These are reminiscent of the above commutation relations. In fact,
\[ Q = c^\dagger \otimes a \quad \quad \quad Q^\dagger = c \otimes a^\dagger \]
Define the Fermion number operator $F = c^\dagger c$ which we use to label,
\[ c \ket{\uparrow} = 0 \quad \quad \ket{\uparrow} = \ket{0} \quad \quad \ket{1} = c^\dagger \ket{0} = \ket{\downarrow} \]
Then we find that,
\[ Q a^\dagger \ket{0} = c^\dagger a a^\dagger \ket{0} = c^\dagger \ket{0} = \ket{1} \]
so $Q$ swaps a boson for a fermion. 

\subsection{Generalizations}

Given operators $Q, Q^\dagger$ with $Q^2 = (Q^\dagger)^2 = 0$ we want,
\[ H = \{ Q, Q^\dagger \} \]
Take,
\[ Q = c^\dagger (\partial_x + h'(x)) / \sqrt{2} \quad \quad Q^\dagger = c( -\partial_x + h'(x)) / \sqrt{2} \]
then define the Hamiltonian,
\[ H = \{ Q, Q^\dagger \} = \frac{1}{2} \begin{pmatrix}
- \partial_x^2 + h'(x)^2 - h''(x) & 0 
\\
0 & - \partial_x^2 + h'(x)^2 + h''(x) 
\end{pmatrix} \]
Therefore we have a class of models for supersymmetric quantum systems if we take $V(x) = \tfrac{1}{2} h'(x)^2$ and $B(x) = \tfrac{1}{2} h''(x)$. Similarly, if we define,
\[ Q_1 = Q + Q^\dagger \quad \quad Q_2 = i(Q - Q^\dagger) \]
then we find,
\[ Q_1^2 = Q_2^2 = H \]
To find the ground state, we use the fact that,
\[ H \ket{\psi} = 0 \iff Q \ket{\psi} = 0 \text{ and } Q^\dagger \ket{\psi} \]
Let,
\[ \ket{\psi} = \psi_0(x) \ket{0} + \psi_1(x) c^\dagger \ket{1} \]
Then,
\[ Q \ket{\psi} = (\partial_x + h'(x)) \psi_0 c^\dagger \ket{0} = 0 \]
implies that,
\[ \psi_0(x) = A_0 e^{- h(x)} \]
and 
\[ Q^\dagger \ket{\psi} = (-\partial_x + h'(x)) \psi_1(x) \ket{0} = 0\]
implies that,
\[ \psi_1(x) = A_1 e^{h(x)} \]
At most one of these two is nonzero giving the unique supersymmetric ground state. In greater generality, we may define,
\[ Q = \sum_I c_I^\dagger \left( \pderiv{}{X^I} + \pderiv{h(X)}{X^I} \right) \]
Then we define the Witten index,
\[ \Omega = \# \{ \text{bosonic SUSY GS} \} - \# \{ \text{fermionic SUSY GS} \} \]
We can write this more formally as,
\[ \Omega = \Tr{(-1)^F e^{-\beta H}} = \int \pathd{} e^{-S_E}  \] 
where $e^{-\beta H}$ is added for convergence but due to the symmetry between excited fermionic and bosonic states all excited contributions cancel. The path integral is taken with periodic boundary conditions for both fermions and bosons. 

\section{March 29}

\renewcommand{\H}{\mathcal{H}}
\newcommand{\id}{\mathrm{id}}

\begin{definition}
A supersymmetric quantum theory is a $(\Z / 2 \Z)$-graded Hibert space $\H$ via the grading $(-1)^F$ where $F$ is the Fermion number and operators $Q$ and $Q$ such that $Q^2 = (Q^\dagger) = 0$ and the Hamiltonian can be written as,
\[ \hamilt = \tfrac{1}{2} \{ Q, Q^\dagger \} \] 
We decompose $\H$ with respect to the grading,
\[ \H = \H_B \oplus \H_F \]
Furthermore, $(-1)^F Q =  -Q(-1)^F$.
\end{definition}

\begin{remark}
This implies that $\hamilt \ge 0$ by positivity of the norm. Furthermore,
\[ \hamilt H \ket{\psi} = 0 \iff Q \ket{\psi} = 0 \text{ and } Q^\dagger \ket{\psi} = 0 \]
Furthermore,
\[ [Q, H] = [Q^\dagger, H] = [(-1)^F, H] = 0 \]
Therefre, there is a one-to-one correspondence between fermionic and bosonic nonground states via the operator,
\[ \tilde{Q} = \frac{Q + Q^\dagger}{\sqrt{2 H}} \]
Then $\tilde{Q}^2 = \id_{\H_{+}} \oplus 0_{\H_0}$ where,
\[ \H = \H_+ \oplus \H_0 \]
is the space of nonzero energy states and the space of zero energy ground states. 
\end{remark}

\begin{definition}
The Witten index is,
\[ \Omega = n_B^{(0)} - n_F^{(0)} = \Tr{(-1)^F e^{- \beta \hamilt}} = \int \pathd{X} \pathd{\psi} e^{-S_E[\psi, X]} \]
where the integral is given periodic boundary conditions, $X^I(\beta) = X^I(0)$ and $\psi^A(\beta) = \phi^A(0)$.  
\end{definition}

\begin{remark}
Recall that in computing the ordinary thermal trace,
\[ Z = \Tr{e^{-\beta \hamilt}} = \int \pathd{X} \pathd{\psi} e^{-S_E[\psi, X]} \]
the integral is given boundary conditions,
\[ X^I(\beta) = X^I(0) \quad \quad \quad \psi^A(\beta) = \psi^A(0) \]
which are \textit{anti}-periodic in the fermionic variables. To see this consider,
\begin{align*}
\int \pathd{\psi} e^{-S_E[\psi]} \bar{\psi}(\beta) \psi(t) & = \Tr{\hat{\bar{\psi}} e^{-(\beta - t) \hamilt} \hat{\psi} e^{-t \hamilt}}
\\
& = \Tr{e^{-(\beta - t) \hamilt} \hat{\psi} e^{-t \hamilt} \hat{\bar{\psi}}}
\\
& = \int \pathd{\psi} e^{-S_E[\psi]} \psi(t) \bar{\psi}(0)
\\
& = - \int \pathd{\psi} e^{-S_E[\psi]} \bar{\psi}(0) \psi(t) 
\end{align*}
Therefore we must have,
\[ \psi(\beta) = - \psi(\beta) \]
\end{remark}

\begin{example}
For a 1D system,
\[ Q = \bar{\psi} \left( \partial_x + h'(x) \right) \]
where $\{ \psi, \bar{\psi} \} = 1$ and $\psi^2 = \bar{\psi}^2 = 0$. Then,
\[ \hamilt = \tfrac{1}{2} \{ Q, Q^\dagger \} = -\tfrac{1}{2} \partial_x^2 + \tfrac{1}{2} h'(x)^2 + \tfrac{1}{2} h''(x) [\bar{\psi}, \psi] \]
We can write any state as,
\[ \Phi(x, \bar{\psi}) = \phi_B(x) + \phi_F(x) \bar{\psi} \]
To find the ground state, we impose the conditions,
\[ Q \Phi_0(x, \bar{\psi}) = 0 \text{ and } Q^\dagger \Phi_0(x, \bar{\psi}) = 0 \implies \Phi_0 = A_B e^{-h(x)} + A_F e^{h(x)} \bar{\psi} \]
Now consider the cases,
\begin{enumerate}
\item $\lim\limits_{|x| \to \infty} h(x) = + \infty \implies \Phi_0 = e^{-h(x)} \implies \Omega = + 1$

\item $\lim\limits_{|x| \to \infty} h(x) = - \infty \implies \Phi_0 = e^{h(x)} \bar{\psi} \implies \Omega = - 1$

\item $\lim\limits_{x \to \infty} h(x) \neq \pm\infty \implies \H_0 = \{ 0 \} \implies \Omega = 0$
\end{enumerate}
\end{example}

\begin{theorem}
The Witten index is invariant under continuous deformations under which the first excited state does not become gapless.
\end{theorem}

\begin{example}
Take $h(x) = \tfrac{1}{4} a x^4 + \tfrac{1}{3} b x^3 + \tfrac{1}{2} c x^2$ then $V(x) = h'(x)^2 = x^2 (a x^2 + b x + c)^2$. For $a > 0$ we have $\Omega = + 1$ and for $a < 0$ we have $\Omega = -1$. However, for $a = 0$ the theory becomes gapless. 
\end{example}

\begin{theorem}
We can compute $\delta \Omega = 0$ semiclassically i.e. $h(x) \to \lambda h(x)$ for $\lambda \to \infty$. Consider $\hamilt$ near each local extremum of $h$,
\begin{align*}
h(x) & = \lambda \left( \frac{x^2}{2} + a x^3 + \cdots \right)
\\
\hamilt & = - \tfrac{1}{2} \partial_x^2 + \tfrac{1}{2} \lambda^2 (x + a x^2 + \cdots +)^2 + \lambda(1 + a x + \cdots ) [\bar{\psi}, \psi]
\end{align*}
Rescaling $x = \lambda^{-\frac{1}{2}} y$ then,
\begin{align*}
\hamilt = \lambda \left( - \tfrac{1}{2} \partial_y^2 + \tfrac{1}{2} (y + \lambda^{-\frac{1}{2}} a y^2 + \cdots )^2 + \tfrac{1}{2} (1 + \lambda^{-\frac{1}{2}} a y + \cdots) [\bar{\psi}, \psi] \right) 
\end{align*}
Therefore, in the $\lambda \to \infty$ limit we recover the Harmonic potential. Therefore, about each extremum of $h$ we get semiclassically,
\[ \Omega = \sum_{h'(x_*) = 0} \mathrm{sign}(h''(x_*)) \] 
\end{theorem}

\subsection{Extension to Multiple Variables}

Given a supersymmetric quantum theory:
\[ Q = \bar{\psi}^I \left( \pderiv{}{X^I} + \partial_I h(x) \right) \quad \quad \quad Q^\dagger = \psi^I \left( - \pderiv{}{X^I} + \partial_I h(x) \right) \]
Then,
\[ \hamilt = \tfrac{1}{2} \{ Q, Q^\dagger \} = \sum_{I} \left( - \tfrac{1}{2} \partial_I^2 + \tfrac{1}{2} (\partial_I h)^2 \right) + \tfrac{1}{2} \partial_I \partial_J h [\bar{\psi}^I, \psi^J] \]
In the semiclassical limit we may consider the ground states. Near a critical point $x_*$ such that $\partial_I h(x_*) = 0$ we may diagonalize the Hessian,
\[ h(y) = h(x_*) + sum_{I = 1}^N \omega_I \tfrac{1}{2} y_I^2 + O(y^3) \]
and therefore,
\[ V(y) = \sum_{I = 1}^N \tfrac{1}{2} \omega_I^2 y_I^2 + O(y^3) \]
Therefore, we have a semiclassical SUSY ground state associated with the critical point $x_*$,
\[ \Psi(X, \bar{\psi}) = \left( \prod_{\omega_I > 0} e^{- \tfrac{1}{2} |\omega_I| X_I^2} \right) \cdot \left( \prod_{\omega_I < 0} e^{- \tfrac{1}{2} |\omega_I| X_I^2} \bar{\psi}_I \right) \]
which has fermion number $F = | \{ I \mid \omega_I < 0 \} |$ which is the Morse index of the critical point. Therefore, 
\[ \Omega = \sum_{x_* : \nabla h(x_*) = 0} (-1)^F(x_*) = \sum_{x_* : \nabla h(x_*) = 0} \mathrm{sign}(\det{\partial^2 h(x_*)}) \]

\subsection{Curved Spaces}

\newcommand{\cod}[1]{\mathrm{d}^\dagger{#1}}

Let the manifold $M$ have metric $\d{s^2} = g_{IJ} \d{x^I} \d{x^J}$ we may define a supersymmetric quantum theory,
\[ Q = \bar{\psi}^I \left( \nabla_I  + \partial_I h \right) \quad \quad \quad Q^\dagger = \psi^I \left( - \nabla_I + \partial_I h \right) \]
where the fermionic degrees of freedom satisfy,
\[ \{ \psi^I, \bar{\psi}^J \} = g^{IJ} \] 
Then we find,
\[ H = \tfrac{1}{2} \{ Q, Q^\dagger \} = - \tfrac{1}{2} g^{IJ} \nabla_I \nabla_J + \tfrac{1}{2} (\partial_I h)^2 + \tfrac{1}{2} \nabla_I \nabla_J h [\bar{\psi}^I, \psi^J] - \tfrac{1}{2} R_{IJKL} \psi^I \bar{\psi}^J \psi^K \bar{\psi}^L \]
For the case $h = 0$ we have a free particle on $M$ with spin degrees of freedom which couple to the curvature tensor. There is a correspondence,
\[ \bar{\psi}^I = \d{x^I} \wedge \quad \text{and} \quad \psi^I = g^{IJ} \pderiv{}{\bar{\psi}^J} = \iota_{\pderiv{}{X^I}} \quad \text{and} \quad Q = \d{x^I} \pderiv{}{X^I} = \d{} \quad \text{and} \quad Q^\dagger = \d{}^\dagger \]
and finally,
\[ \hamilt = \tfrac{1}{2} \left( \cod{\d{}} + \d{\cod{}} \right) \]
Then the Betti numbers $b^p$ are the number of SUSY ground states with $F = p$ equal to the number of harmonic differential forms of $\deg{p}$ equal to thenumber of homology $p$-cycles. In the case $h = 0$ we find,
\[ \Omega = \sum_{F = 0}^\infty (-1)^F \#\{ \text{SUSY groud states with Fermion number F} \} = \sum_{F = 0}^\infty (-1)^F b_F = \chi(M) \]

\section{April 3}

Now consider the potential $h = 0$ operator,
\[ Q = \d{x^I} \pderiv{}{x^I} \]
and the $h \neq 0$ deformed version,
\[ Q_h = \d{x^I} \left( \pderiv{}{x^I} + \pderiv{h}{x^I} \right) = e^{-h} Q e^h = \d{} + \d{h} \]
However, $\ket{\psi} \mapsto e^h \ket{\psi}$ is not unitary and so does not generically preserve the spectrum. However, in SUSY,
\begin{align*}
H \ket{\psi} & = 0 \iff Q \ket{\psi} = 0 \text{  and  } Q^\dagger \ket{\psi} = 0 
\\
H_h \ket{\psi} & = 0 \iff Q_h \ket{\psi} = 0 \text{  and  } Q_h^\dagger \ket{\psi} = 0
\end{align*} 
Say $Q \ket{\psi_0} = 0$ and $Q^\dagger \ket{\psi_0} = 0$. We define $\ket{\psi_0'} = e^{-h} \ket{\psi_0}$ then,
\[ Q_h \ket{\psi_0'} = e^{-h} Q \ket{\phi} = 0 \quad \text{and} \quad Q_h^\dagger \ket{\psi_0'} = e^h Q^\dagger e^{-h} e^{-h} \ket{\psi_0} \neq 0 \]
so the Naieve construction indeed does not preserve the spectrum. 

\subsection{$Q$-cohomology}

\begin{definition}
We call a state SUSY iff $Q \ket{\psi} = Q^\dagger \ket{\psi} = 0$. This is equivalent to $\hamilt \ket{\psi} = 0$ i.e. the state being a zero energy groud state or SUSY ground state.
\end{definition}

\begin{proposition}
$\H_{\text{SUSY}} \cong H^*(Q)$. 
\end{proposition}

\begin{proof}
Clearly any SUSY state is $Q$-closed so there is a map $\H_{\text{SUSY}} \to Z(Q) \to H^*(Q)$. The map is injective because if $\ket{\psi} = 0$ in $H^*(Q)$ then $\ket{\psi} = Q \ket{\chi}$. But if $\ket{\psi}$ is SUSY then $Q^\dagger Q \ket{\chi} = 0$ which implies that $Q \ket{\chi} = 0$ so $\ket{\psi} = 0$. Thus it suffices to show the map is surjective i.e. that all $Q$-closed non-SUSY states are $Q$-exact. We may consider an energy eigenstate $\ket{\psi}$ with $\hamilt \ket{\psi} = E \ket{\psi}$ such that $E > 0$. Then,
\[ (Q Q^\dagger + Q^\dagger Q) \ket{\psi} = 2 E \ket{\psi} \]
But $\ket{\psi}$ is $Q$-closed so $Q \ket{\psi} = 0$.
Therefore,
\[ \ket{\psi} = \frac{Q Q^\dagger}{2 E} \ket{\psi} \]
and thus $\ket{\psi}$ is $Q$-exact. All non-SUSY states are spanned by these positive energy eigenstates and thus are sent to zero under $Z(Q) \to H(Q)$.  
\end{proof}

\begin{proposition}
The mapping $Q \mapsto e^{-h} Q e^h = Q_h$ induces an isomorphism on cohmology. Therefore, $\H_{\text{SUSY}}^0 = \H_{\text{SUSY}}^h$.
\end{proposition}

\begin{proof}
Let $\ket{\psi'} = e^{-h} \ket{\psi}$. Then,
\[ Q \ket{\psi} = 0 \iff Q_h \ket{\phi'} = 0 \]
and furthermore,
\[ \ket{\psi} = Q \ket{\chi} \iff \ket{\psi'} = Q_h \ket{\chi'} \]
Equivalently, the map $\lambda : \ket{\psi} \mapsto e^{-h} \ket{\psi}$ gives an isomorphism of complexes graded by Fermion number,
\begin{center}
\begin{tikzcd}
0 \arrow[r] & \H^0 \arrow[r, "Q"] \arrow[d, "\lambda"] & \H^1 \arrow[r, "Q"] \arrow[d, "\lambda"] & \arrow[r, "Q"] \H^2 \arrow[r, "Q"] \arrow[d, "\lambda"] & \H^3 \arrow[r] \arrow[d, "\lambda"] & \cdots
\\
0 \arrow[r] & \H^0 \arrow[r, "Q_h"] & \H^1 \arrow[r, "Q_h"] & \arrow[r] \H^2 \arrow[r, "Q_h"] & \H^3 \arrow[r] & \cdots
\end{tikzcd}
\end{center}
because $Q_h \lambda \ket{\psi} = e^{-h} Q e^h e^{-h} \ket{\psi} = e^{-h} Q \ket{\psi} = \lambda Q \ket{\psi}$. 
\end{proof}

\begin{corollary}
Semiclassically the number of SUSY states of $F = p$ is the number of critical points of Morse index $p$ i.e. the number of negative eigenvalues of the Hessian $\partial_I \partial_J h|_{x_*}$. This is bounded below by $b^p(M)$. 
\end{corollary}

We may change from a Hamiltonian framework to a Lagrangian one in Euclidean signature via,
\[ \lagrange = \tfrac{1}{2} g_{IJ} \dot{X}^I \dot{X}^J + \tfrac{1}{2} g^{IJ} \partial_I h \partial_J h + g_{IJ} \bar{\partial}^I D_\tau \psi^J - \tfrac{1}{2} R_{IJKL} \psi^I \bar{\psi}^J \psi^K \bar{\psi}^L + \nabla_I \nabla_J h \bar{\psi}^I \psi^J \]
where,
\[ \nabla_I V_J = \partial_I V_J - \Gamma^K_{IJ} V_K \]
and
\[ D_\tau \psi^I = \partial_\tau \psi^I + \Gamma^I_{JK} \dot{X}^J \psi^K \]
A SUSY transformation can be written as,
\[ \delta_\epsilon(W) = [\epsilon Q, W] + [\bar{\epsilon} Q^\dagger, W] \]
where $\epsilon$ is a Grassmann parameter. Then,
\begin{align*}
\delta X^I & = \epsilon \bar{\psi}^I - \bar{\epsilon} \psi^I
\\
\delta \psi^I & = \epsilon \left( - \dot{X}^I - \Gamma^I_{JK} bar{\psi}^J \psi^K + g^{IJ} \partial_J h \right) 
\\
\delta \bar{\psi}^I & = \cdots 
\end{align*}
We want to compute the path integral,
\[ \Omega = \Tr{(-1)^F e^{-\beta F}} = \int \pathd{X} \pathd{\psi} e^{-S_E[X, \psi]} \]
in the limit $\beta \to 0$. As a warmup we compute the path integral for a free bosonic particle of a circle and no fermions. We have,
\[ \Tr{e^{-\beta H}} = \int \pathd{X}|_{X(\beta) = X(0)} \: e^{-\tfrac{1}{2} \int_0^\beta \dot{X}^2 \d{\tau} } \]
Where $X : S^1 \to S^1$ is a map from the time circle of period $\beta$ to the space circle of period $L$. We have classical solutions $X(\tau) = X_0 + v \tau$ but $X(\beta) = X_0 + n L$ for $n \in \Z$ implies that $v = nL/\beta$. Thus,
\[ S_{\text{cl}}(n) = \frac{1}{2} \int_0^\beta \d{\tau} \left( \frac{n L}{\beta} \right)^2 = \frac{n^2 L^2}{2\beta^2} \]
Then we may write $X(\tau) = X_{\text{cl}}(\tau) + y(\tau)$ where $y(\beta) = y(0)$ strictly periodic i.e. not on the circle. Thus having isolated the topologically distinct sectors,
\[ \Tr{e^{-\beta H}} = \int\d{X_0} \sum_{n} e^{- \frac{n^2 L^2}{2 \beta}} \int \pathd{y} e^{- \int_0^\beta \tfrac{1}{2} \dot{y}^2 \d{\tau}} \]
Then 
\[  \int \pathd{y} e^{- \int_0^\beta \tfrac{1}{2} \dot{y}^2 \d{\tau}} \propto \det'(-\partial_\tau^2)^{-\frac{1}{2}} \]
where we have removed the zero mode by integrating over $X_0$. In the limit $L \to \infty$ we are computing,
\begin{align*}
\Tr{e^{-\beta H}} = \int \d{x} \d{p} \bra{x} e^{-\beta H} \ket{p} \inner{p}{x} = \int \d{x} \d{p} e^{-\frac{1}{2} \beta p^2} |\inner{x}{p}|^2 = \int \frac{\d{x}\d{p}}{2 \pi} e^{-\frac{1}{2} \beta p^2} 
\end{align*}
Therefore,
\[ \Tr{e^{-\beta H}} = \frac{L}{\sqrt{2 \pi \beta}} \sum_n e^{-\frac{n^2 L^2}{2 \beta}} \]
Furthermore, we can canonically compute this trace by computing the energy eigenvalues,
\[ \Tr{e^{-\beta H}} = \sum_m e^{-\frac{1}{2} \beta \left( \frac{2 \pi}{L} \right) m^2} \]
since the alowed momenta are $p = \frac{2 \pi}{L} m$ for $m \in \Z$. 

\section{SUSY QM from D-BRANES}

In general we have $M,N = 0, \dots, q$ and $\mu, \nu = 0, \cdots, 3$ and $m,n = 1, \dots, 6$. Then we have Type IIB string theory is SYSY on a 10D manifold,
\[ M_{10} = \R^{1,3} \times M_6 \]
Then the 4D effectie theory = some 4D supergravity theory. Now,
\[ \d{s^2} = - \d{x_0^2} + \d{x_1^2} + \d{x_2^2} + \d{x_3^2} + R_1^2 \d{y_1^2} + \cdots R_6^2 \d{y_6^2} \]
with periodicity $y_i \cong y_i + 2 \pi$. There are 10D RR fields $C^{(0)}(X,Y)$ and $C^{(2)}_{MN}(X,Y)$ and $C^({4)}_{MNRS}(X,Y)$. Then,
\[ F_M^{(1)} = \partial_M C^{(0)} \quad \quad F^{(3)}_{MNR} = 3 \partial_{[M} C^{(2)}_{NR]} \quad \quad F^{(5)}_{MNRST} = 5 \partial_{[M} C^{(4)}_{NRST]} \]
Therefore,
\[ S_{10D} \supset \int \sqrt{G} \left( G^{MN} F^{(1)}_M F^{(1)}_N + F^{(3) MNR} F^{(3)}_{MNR} + F^{(5) MNRST} F^{(5)}_{MNRST}  \right) \] 
where we have the self-dual condition,
\[ F^{(5)}_{MNRST} = \epsilon_{MNRSTUVWXY} F^{(5) UVWXY} \]
We may write,
\[ C^{(0)}(X<Y) = \sum_{n_1, \dots, n_6} C^{(0)}_{n_1, \cdots, n_6}(x) e^{i(n_1 y_1 + \cdots + n_6 y_6)} \]
which gives,
\[ \int \dn{4}{x} \dn{6}{y} \left| \partial_\mu C^{(0)}_{n_1, \dots, n_6}(x) \right|^2 - \left( \frac{n_1^2}{R_1^2} + \cdots + \frac{n_6^2}{R_6^2} \right) | C_{n_1, \dots, n_6} |^2 \]
At low energy we only keep $n_1 = n_2 =\cdots = n_6 = 0$. Therefore we get a singel massless scalar $C^{(0)}(x)$. 


\section{April 17: BPS States}

In the Lagrangian framework SUSY QM takes the form,
\begin{align*}
\delta x^I & = \epsilon \bar{\psi}^I - \bar{\epsilon} \psi^I
\\
\delta \phi^I & = \epsilon (- \dot{x}^I + \partial_I h) 
\\
\delta \bar{\psi}^I & = \epsilon (\dot{x}^I + \partial_I h)
\end{align*}
with the action,
\[ S = \int \d{t} \left[ \tfrac{1}{2} \dot{x}^2 + \tfrac{1}{2} (\partial_I h)^2 + \bar{\psi}^I \partial_t \psi^I + \partial_I \partial_J \bar{\psi}^I \psi^J \right] \]
Consider only the bosonic part of the action,
\[ S_B = \int \d{t} \left[ \tfrac{1}{2} (\dot{x}^I + s \partial_I h)^2  - s \dot{x}^I \partial_I h \right] \]
However, 
\[ - s \dot{x}^I \partial_I h = - s \deriv{}{t} h(x) \]
is a total derivative and thus,
\[ S_B = - s [h(t_f) - h(t_f)] + \int \d{t} \tfrac{1}{2} (\dot{x}^I + s \partial_I h)^2 \]
Since the functional is always nonegative we have a bound,
\[ \int \d{t} \tfrac{1}{2} (\dot{x}^I + s \partial_I h)^2 \ge s[h(t_f) - h(t_f)] \]
Furthermore, since the term,
\[ \int \d{t} \tfrac{1}{2} (\dot{x}^I + s \partial_I h)^2 \ge 0 \]
is always nonegative then we find that,
\[ S_B \ge - s [h(t_f) - h(t_f)]  \]
Since we can choose $s = \pm 1$ this implies that,
\[ S_B \ge |h(t_f) - h(t_i)| \]
This bound is saturated when $\dot{x}^I = - s \partial_I h$ which is the gradient flow of $h$ which has fixed points at critical points of $h$. The solution with $s = +1$ corresponds to $\delta \bar{\psi}^I = 0$ and with $s = -1$ corresponds to $\delta \psi = 0$. Therefore, if we set $\dot{x} = - \partial h$ and $\psi = \bar{\psi} = 0$ then $\delta_\epsilon = 0$ so BPS states are annihilated by half of the SUSY. 
\bigskip\\
Consider a 2D SUSY model with supercharges $\{ Q_\alpha, Q_\beta \} = \gamma^\mu_{\alpha \beta} p_\mu$ and action,
\[ S_B = \int \d{t} \d{x} \tfrac{1}{2} \left[ (\partial_t \phi^I)^2 - (\partial_x \phi^I)^2 - (\partial_I h)^2 \right] \]
We may consider the conditions $\partial_t \phi = 0$ and $\partial_x \phi^I = \pm \partial_I h$. Then the energy functional becomes,
\[ E = \int_{-\infty}^\infty \d{x} \tfrac{1}{2} \left( (\partial_x \phi)^2 + (\partial_\phi h)^2 \right] \ge | \phi(+ \infty) - \phi(-\infty) | \] 

\section{Quantum BPS Bound}

We want to consider a 4D field theory with SUSY. However, $Q^2 = H$ violates Lorentz invariance because $H$ mixes with the momentum under boosts. The simiplest extension of this algebra which is consistent with Lorentz invariance is,
\[ \{ Q_\alpha, Q_\beta \} = 2 \gamma_{\alpha \beta}^\mu P_\mu \]
and the supercharges $Q_\alpha$ are Majorana i.e. $Q_\alpha^\dagger = Q_\alpha$. Therefore, there are four real SUSYs. This is the minimal number of SUSY which we denote $\mathcal{N} = 1$. Now we compactify $4D \to 2D$ on $S^1 \times S^1$ which produces an effective $2D$ theory. We still have the commutaton relations $\{ Q_\alpha, Q_\beta \} = 2 \gamma^\mu_{\alpha \beta} P_\mu$ but now $\mu = 0,1$ corresponds to the 2D spacetime while $\mu = 2,3$ are ``internal'' indicies. Thus we write the commutator as,
\[ \{ Q_\alpha, Q_\beta \} = 2 \gamma^\mu_{\alpha \beta} P_m + 2 (\gamma^2_{\alpha \beta} + \gamma^3_{\alpha \beta} P_3) \]
This has double the minimal number of SUSY so it is denoted $\mathcal{N} = 2$. There are charges $P_2$ and $P_3$ which commute with spacetime symmetries. 
\bigskip\\
The basic picture for extended SUSY is $Q_1^2 = H + P_{\text{int}}$ and $Q_2^2 = H - P_{\text{int}}$. 
and $Q_1^\dagger = Q_1$ and $Q_2^\dagger = Q_2$. 
Then the BPS bound can be derived via,
\[ Q_1^2 \ge 0 \implies H + P_{\text{int}} \ge 0 \]
and,
\[ Q_2^2 \ge 0 \implies H - P_{\text{int}} \ge 0 \]
therefore,
\[ H \ge |P_{\text{int}}| \]
This bound simply follows from the algebra and thus is exact quantum mechanically.
If the theory arises from a compactification, this bound is simply the statement that $H$ is greater than the momentum in the compactified directions.  

\section{$\mathcal{N} = 2$ 4D SUGRA}

Consider the supersymmetric theory with supersymmetry generators satifying,
\[ \{ Q_\alpha^A, Q_\beta^B \} = 2 \delta^{AB} \gamma^\mu_{\alpha \beta} P_\mu - 2 i \epsilon^{AB} (\Re{Z} \delta_{\alpha \beta} + \Im{Z} \gamma^5_{\alpha \beta}) \]
where $Z$ is a complex central charge.
The BPS bound: $M \ge |Z|$ with $M^2 = P^\mu P_\mu$. For BPS states this bound is saturated corresponding to half of the SUSY unborken. 
\bigskip\\
The spectrum of $\mathcal{N} = 2$ SUGRA spilits up into multiplets,
\begin{enumerate}
\item Gravity multiplet: graviton + photon + fermion superpartners
\item Vector multiplet: photon + complex scalar + fermion superpartners
\item Hyper multiplet: quaternionic scalar + fermion superpartner
\end{enumerate}
With $8 \pi G = 1$ the action for gravity coupled to vector multiplets is, in its most general form,
\[ S = \frac{1}{2} \int \sqrt{g} \left[ R - G_{a\bar{b}} (\phi, \bar{\phi}) \partial_\mu \phi^a \partial^\mu \bar{\phi}^a - \mathcal{F}_I \wedge \mathcal{G}^I - V(\varphi, \bar{\varphi}) \right] \]
with indicies $I = 0,\dots,n$ and $a = 1, \dots, n$
where the field strenght tensor is,
\[ F^I_{\mu \nu} = \partial_\mu A^I_\nu - \partial_\nu A_\mu^I \]
The constraint of $\mathcal{N} = 2$ SUSY implies that all couplings are determined by a single holomorphic section of the canonical bundle over the moduli space $F[X^0, \dots, X^n]$ called the prepotential. The metric $G_{a \bar{b}}$ must be K\"{a}hler meaning that there exists a K\"{a}hler potential $K(\phi, \bar{\phi})$ s.t. $G_{a \bar{a}} = \partial_{\phi^a} \partial_{\bar{\phi}^b} K(\phi, \bar{\phi})$. 

\section{BPS Black Hole Solutions in $\mathcal{N} = 2$ SUGRA}

\subsection{Einstein-Maxwell}


We have the Einstein-Maxwell action,
\[ S = \frac{1}{16 \pi} \int \dn{4}{x} \sqrt{-g} R + \frac{1}{4} \int \dn{4}{x} \sqrt{-g} F_{\mu \nu} F^{\mu \nu} \]
where $G_N = 1$. We have extremal R.N. black hole solutions with $M = Q$. In isotropic coordinates,
\[ \d{s^2} = - \frac{1}{H(r)} \d{t^2} + H(r)^2 \d{\vec{x}^2} \]
where $r = |\vec{x}|^2$ and $H(r) = 1 + \frac{Q}{r}$. We need to comute the entropy of such a solution. The horizon occurs at $r = 0$ which coincides with the curvature singularity. In the near horizon limit, $H(r) = \frac{Q}{r}$ so the metric transformed to spherical coordinates becomes,
\[ \d{s^2} = - \frac{r^2}{Q^2} \d{t^2} + \frac{Q^2}{r^2} \left( \d{r^2} + r^2 \d{\Omega^2} \right) = \left( - \frac{r^2}{Q^2} \d{t^2} + \frac{Q^2}{r^2} \d{r^2} \right) + Q^2 \d{\Omega^2} \]
so the spacetime is $\AdS{2} \times S^2$. To see why the first factor is the metric of $\AdS{2}$ consider the change of variables $r = \frac{Q^2}{z}$ which gives,
\[ \left( - \frac{r^2}{Q^2} \d{t^2} + \frac{Q^2}{r^2} \d{r^2} \right) = Q^2 \left( \frac{- \d{t^2} + \d{z^2}}{z^2} \right) \]
the standard metric on $\AdS{2}$. Then the area is,
\[ A = 4 \pi Q^2 \]
so the entropy is,
\[ S_{BH} = \frac{A}{4} = \pi Q^2 \] 
 
\subsection{$\mathcal{N} = 2$ SUGRA}

There are $\phi^A$ complex scalars with $A = 1, \dots, n$ with $n = b_2(CY)$. We have an action,
\begin{align*}
S = \frac{1}{16 \pi} \int \sqrt{-g} \left[ R - G_{AB}(\phi) \partial_\mu \phi^A \partial^\mu \bar{\phi}^B + \tfrac{1}{4} h_{IJ}(\phi) F_{\mu \nu}^I F^{J \mu \nu} + \tfrac{1}{4} \tilde{h}_{IJ}(\phi) F^I_{\mu \nu} \tilde{F}^{J \mu \nu} \right]
\end{align*} 
We search for spherically symmetric solutions. Consider a metric Ansatz,
\[ \d{s^2} = - e^{2 U} \d{t^2} + e^{-2 U} \d{\vec{x}^2} \]
for some function $U(\tau)$ with $\tau = \frac{1}{r}$. Then we may consider a spherically symmetric field,
\[ F^I = P^I \sin{\theta} \d{\theta} \wedge \d{\phi} \]
Assume that the scalar fields $\phi$ only depend radially. Then $\d{F^I} = 0$ and $\d{* F^I} = 0$ solving the classical equations of motion. Frist recall that the prepotential may be written in the form,
\[ F(X) = - \frac{D_{ABC} X^A X^B X^C}{6 X^0} \]
where $D_{ABC}$ are the tripple intersection numbers from the chomology intersection from on our manifold. 
Now from the action we may comute the energy of such a configuration,
\[ E = \frac{1}{2} \int_0^{\tau_h} \d{\tau} \left( \dot{U}^2 + G_{A \bar{B}} \dot{\psi}^A \dot{\bar{\phi}}^B + e^{2 U} V(\phi) \right) \]
where,
\[ V(\phi) = |Z(\phi)|^2 + 4 G^{A \bar{B}} \partial_A |Z(\phi)| \partial_{\bar{B}} |Z(\phi)| \]
and likewise,
\[ Z(\phi) = \left[ \frac{4}{3} D_{ABC} \Im{\phi^A} \Im{\phi^B} \Im{\phi^C} \right]^{-1/2} \cdot \left( \tfrac{1}{6} P^0 D_{ABC} \phi^A \phi^B \phi^C - \tfrac{1}{2} D_{ABC} P^A \phi^B \phi^C + Q_A \phi^A + Q_0 \right)  \]
where $Q$ is the electric charge and $P$ is the magnetic charge. We can complete the square in the energy expression,
\begin{align*}
E = \frac{1}{2} \int_0^{\tau_h} \d{\tau} \left[ \left( \dot{U} + e^U |Z| \right)^2 + || \dot{\phi}^A + 2 e^U G^{A \bar{B}} \partial_{\bar{B}} |Z| ||^2 - \deriv{}{\tau} \left( 	e^U Z \right) \right]
\end{align*}
Therefore,
\[ E = \frac{1}{2} \int_0^{\tau_h} \d{\tau} \left[ \left( \dot{U} + e^U |Z| \right)^2 + || \dot{\phi}^A + 2 e^U G^{A \bar{B}} \partial_{\bar{B}} |Z| ||^2  \right] + |Z|_{\tau = 0} \]
Therefore we find a BPS bound,
\[ E \ge |Z|_{\tau = 0} \]
The state is BPS when this bound is saturated iff we satisfy the first-order flow equations $\dot{U} = 0 e^U |Z|$ and $\dot{\phi}^A = - 2 e^U G^{A \bar{B}} \partial_{\bar{B}} |Z|$. This is a gradient flow of $|Z|$ since,
\begin{align*}
\deriv{}{\tau} |Z| & = \dot{\psi}^A \partial_A |Z| + \dot{\bar{\phi}}^B \partial_{\bar{B}} |Z| 
\\
& = - 2 e^U G^{A \bar{B}} \partial_{\bar{B}} |Z| \partial_A |Z| - 2 e^U G^{A \bar{B}} \partial_A |Z| \partial_{\bar{B}} |Z|
\\
& = - 4 e^U || \nabla |Z| ||^2 
\end{align*}
Therefore, the evolution of $Z$ is by gradient diffusion. If there exists a minium of $Z$ at $\phi_*$ then as $\tau \to \infty$ we find $\phi \to \phi_*$. At this attractor point $\dot{U} = - e^U |Z_*|$ and $\dot{\phi}^A = 0$ where $Z_* = Z(\phi_*) = \inf_{\phi} Z(\phi)$. Therefore, in the $\tau \to \infty$ limit i.e. the near-horizon limit,
\[ e^{-U} \to |Z_*| \tau  \]
which implies that, the metric takes the form,
\[ \d{s^2} = - \frac{|Z_*|^2}{r^2} \d{t^2} + \frac{r^2}{|Z_*|^2} \d{\vec{x}^2} \]
which has exactly the same form as the extremal R.N. black hole. Thus the near-horizon geometry is $\AdS{2} \times S^2$ where the $S^2$ has area $A = 4 \pi |Z_*|$. Thus the Bekenstein-Hawking entropy is,
\[ S_{BH} = \pi |Z_*|^2 \]

\section{FINAL LECTURE}

Consider IIA / CY3 string theory which is $\mathcal{N} = 2$ SUGRA. We have shown that the BPS BH entropy is $S = \pi |Z_*|^2 = \min_\phi \pi |Z(Q, P, \phi)|^2$ where,
\[ Z(Q,P, \phi) = (\tfrac{4}{3} D_{ABC} \phi^A_2 \phi^B_2 \phi^C_2)^{-\frac{1}{2}} ( \tfrac{1}{6} P^0 D_{ABC} \phi^A \phi^B \phi^B - \tfrac{1}{2} P^A D_{ABC} \phi^B \phi^C + Q_A \phi^A + Q_0) \]
and $\phi_2^A = \Im{\phi^A}$. Here, $|Z|$ is the mass of the SUSY D-brane in homology class $H_{\text{even}}(N)$ given by $(P^0, P^A, Q_A, Q_0)$ 
\[ B + i J = \phi^A \Sigma_A \implies B = 0 \iff \Re{\phi^A} = 0 \implies \Phi^A = i J^A \]
The coefficients $D_{ABC}$ are the tripple intersection forms,
\[ D_{ABC} = \# (\Sigma_A \cap \Sigma_B \cap \Sigma_C) = \int_M \Sigma_A \wedge \Sigma_B \wedge \Sigma_C \]
Thus,
\begin{align*}
\tfrac{4}{3} D_{ABC} \phi^A_2 \phi^B_2 \phi^C_2 & = \tfrac{4}{3} D_{ABC} J^A J^B J^C 
\\
& = \tfrac{4}{3} \int_M J^A \Sigma_A \wedge J^B \Sigma_B \wedge J^C \Sigma_C 
\\
& = \tfrac{4}{3} \int_M J \wedge J \wedge J
\\
& = 8 \int_M \tfrac{1}{6} J \wedge J \wedge J
\\
& = 8 \cdot \Vol{M}
\end{align*}
\bigskip\\
For pure $D4$ states $P^A \neq 0$ others all zero i.e.the brane wraps $\Sigma = P^A \Sigma_A$. Then,
\begin{align*}
|Z| & = \frac{1}{\sqrt{8 \Vol{M}}} | \tfrac{1}{2} P^A D_{ABC} J^B J^C |
\\
& = \frac{1}{\sqrt{8 \Vol{M}}} \int_M \tfrac{1}{2} \Sigma \wedge J \wedge J = \frac{1}{\sqrt{8 \Vol{M}}} \Vol{\Sigma} 
\end{align*}
because when $\Sigma$ is holomorphic then,
\[ \int_\Sigma \tfrac{1}{2} J \wedge J = \Vol{\Sigma} \]

\subsection{Born-Infeld Action for Dp-brane}

In pur metric background $(B = 0)$ we have the action,
\[ S_{BI} = T_p \int \dn{p+1}{x} \sqrt{-h} \]
If the brane is at rest in $\R^{1,3}$ then,
\[ S_{BI} = T_p \int \d{t} \int_{C_p} \dn{p}{x} \sqrt{-h} = T_p \Vol{C_p} \int \d{t} \]
where $T_p \propto \frac{m_s^{p+1}}{g_s}$.
In 10D,
\[ S_{IIA} = \frac{m_s^8}{g_s^2} \int \dn{10}{x} \sqrt{-g} R + \cdots = \frac{m_s^8}{g_s^2} \Vol{M} \int \dn{4}{x} \sqrt{-g_4} R_4 + \cdots \sim \frac{1}{G_N} \sim M_{p,q}^2 \]
Then,
\[ M_{4,P} = \frac{m_s^4}{g_s} \sqrt{\Vol{M}} = \frac{m_s}{g_s} \sqrt{ \Vol{M} m_s^6} \]
\begin{example}
D4-D0 and then $P^0 = 0$ and $Q_A = 0$ gives an entrpy,
\[ S = 2\pi \sqrt{\tfrac{1}{6} D_{ABC} P^A P^B P^C Q_0} \]
where $\Sigma_4 = P^A \Sigma_A$ and,
\[ D_{ABC} P^A P^B P^C = \# (\Sigma_4 \cap \Sigma_4 \cap \Sigma_4) = \int_M \Sigma_4 \wedge \Sigma_4 \wedge \Sigma_4 \]
What is the microscopic origin on this entropy?
\end{example}


\end{document}


