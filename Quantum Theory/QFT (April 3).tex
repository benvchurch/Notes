\documentclass[12pt]{extarticle}
\usepackage[utf8]{inputenc}
\usepackage[english]{babel}

\usepackage[a4paper, total={7.25in, 9.5in}]{geometry}
\usepackage{tikz-feynman}
\tikzfeynmanset{compat=1.0.0} 
\usepackage{subcaption}
\usepackage{float}
\floatplacement{figure}{H}
\newcommand{\field}{\hat{\Phi}}
\newcommand{\dfield}{\hat{\Phi}^\dagger}
\usepackage{simplewick}
 
\usepackage{amsthm, amssymb, amsmath, centernot}

\newcommand{\notimplies}{%
  \mathrel{{\ooalign{\hidewidth$\not\phantom{=}$\hidewidth\cr$\implies$}}}}
 
\renewcommand\qedsymbol{$\square$}
\newcommand{\cont}{$\boxtimes$}
\newcommand{\divides}{\mid}
\newcommand{\ndivides}{\centernot \mid}
\newcommand{\Z}{\mathbb{Z}}
\newcommand{\N}{\mathbb{N}}
\newcommand{\C}{\mathbb{C}}
\newcommand{\Zplus}{\mathbb{Z}^{+}}
\newcommand{\Primes}{\mathbb{P}}
\newcommand{\ball}[2]{B_{#1} \! \left(#2 \right)}
\newcommand{\Q}{\mathbb{Q}}
\newcommand{\R}{\mathbb{R}}
\newcommand{\Rplus}{\mathbb{R}^+}
\newcommand{\invI}[2]{#1^{-1} \left( #2 \right)}
\newcommand{\End}[1]{\text{End}\left( A \right)}
\newcommand{\legsym}[2]{\left(\frac{#1}{#2} \right)}
\renewcommand{\mod}[3]{\: #1 \equiv #2 \: \mathrm{mod} \: #3 \:}
\newcommand{\nmod}[3]{\: #1 \centernot \equiv #2 \: mod \: #3 \:}
\newcommand{\ndiv}{\hspace{-4pt}\not \divides \hspace{2pt}}
\newcommand{\finfield}[1]{\mathbb{F}_{#1}}
\newcommand{\finunits}[1]{\mathbb{F}_{#1}^{\times}}
\newcommand{\ord}[1]{\mathrm{ord}\! \left(#1 \right)}
\newcommand{\quadfield}[1]{\Q \small(\sqrt{#1} \small)}
\newcommand{\vspan}[1]{\mathrm{span}\! \left\{#1 \right\}}
\newcommand{\galgroup}[1]{Gal \small(#1 \small)}
\newcommand{\bra}[1]{\left| #1 \right>}
\newcommand{\Oa}{O_\alpha}
\newcommand{\Od}{O_\alpha^{\dagger}}
\newcommand{\Oap}{O_{\alpha '}}
\newcommand{\Odp}{O_{\alpha '}^{\dagger}}
\renewcommand{\Im}[1]{\mathrm{Im} \: #1}
\newcommand{\ket}[1]{\left| #1 \right>}
\renewcommand{\bra}[1]{\left< #1 \right|}
\newcommand{\inner}[2]{\left< #1 | #2 \right>}
\newcommand{\expect}[2]{\left< #1 \right| #2 \left| #1 \right>}
\renewcommand{\d}[1]{\! \mathrm{d}#1 \:}
\newcommand{\dn}[2]{\! \mathrm{d}^{#1} #2 \:}
\newcommand{\deriv}[2]{\frac{\d{#1}}{\d{#2}}}
\newcommand{\pderiv}[2]{\frac{\partial{#1}}{\partial{#2}}}
\newcommand{\parsq}[2]{\frac{\partial^2{#1}}{\partial{#2}^2}}
\newcommand{\topo}{\mathcal{T}}
\newcommand{\base}{\mathcal{B}}
\renewcommand{\bf}[1]{\mathbf{#1}}
\renewcommand{\a}{\hat{a}}
\newcommand{\adag}{\hat{a}^\dagger}
\renewcommand{\b}{\hat{b}}
\newcommand{\bdag}{\hat{b}^\dagger}
\renewcommand{\c}{\hat{c}}
\newcommand{\cdag}{\hat{c}^\dagger}
\newcommand{\hamilt}{\hat{H}}
\renewcommand{\L}{\hat{L}}
\newcommand{\Lz}{\hat{L}_z}
\newcommand{\Lsquared}{\hat{L}^2}
\renewcommand{\S}{\hat{S}}
\renewcommand{\empty}{\varnothing}
\newcommand{\J}{\hat{J}}
\newcommand{\lagrange}{\mathcal{L}}
\newcommand{\dfourx}{\mathrm{d}^4x}
\newcommand{\meson}{\phi}
\newcommand{\dpsi}{\psi^\dagger}
\newcommand{\ipic}{\mathrm{int}}
\usepackage{slashed}
\newcommand{\parity}{\mathbf{P}}
\newcommand{\Tr}[1]{\mathrm{Tr}\left( #1 \right)}

\renewcommand{\theenumi}{(\alph{enumi})}

\newcommand{\atitle}[1]{\title{% 
	\large \textbf{Physics GR6047 Quantum Field Theory I
	\\ Assignment \# #1} \vspace{-2ex}}
\author{Benjamin Church }
\maketitle}

 
\theoremstyle{definition}
\newtheorem{theorem}{Theorem}[section]
\newtheorem{lemma}[theorem]{Lemma}
\newtheorem{proposition}[theorem]{Proposition}
\newtheorem{corollary}[theorem]{Corollary}
\newtheorem{example}[theorem]{Example}
\newtheorem{remark}[theorem]{Remark}
 




\begin{document}

\section{Perturbative Expansion of Scattering Amplitudes for Spinor Fields}

\begin{example}[Fermionic Yukawa Theory]
\[ \lagrange = \lagrange_0 + \lagrange_{\ipic} \]
where we have the generic free theory for a real scalar and a fermionic field,
\[ \lagrange_0 = \tfrac{1}{2} \partial_\mu \phi \partial^\mu \phi - \tfrac{1}{2} \mu^2 \phi^2 + \bar{\psi} (i \slashed{\partial} - m)\psi \]
and the most general renormalizable interacting Lagragian in $D = 3 + 1$ dimensions,
\[ \lagrange_{\ipic} = - \left( \lambda_0 + \lambda_1 \phi + \tfrac{1}{3!} \lambda_3^2 \phi^3 + \tfrac{1}{4!} \lambda_4 \phi^4 + g_y \phi \bar{\psi} \psi + g_s \phi \bar{\psi}   i \gamma_5 \psi \right) \]
If we also require that parity be a symmetry of the theory,
\[ \mathbf{P}^{-1} \phi(x) \mathbf{P} = \pm \phi(\bar{x}) \quad \text{and} \quad \mathbf{P}^{-1} \psi(x) \mathbf{P} = \gamma^0 \psi(\bar{x}) \]
If $\phi$ is a scalar (parity $+1$) then the interaction terms must have the form,
\[ \lagrange_{\ipic} = -\left( \lambda_0 + \lambda_1 \phi + \tfrac{1}{3!} \lambda_3^2 \phi^3 + \tfrac{1}{4!} \lambda_4 \phi^4 + g_y \phi \bar{\psi} \psi \right) \]
On the other hand, if $\phi$ is a pseudo-scalar (parity $-1$) then the interaction terms must have the form,  
and the most general renormalizable interacting Lagragian in $D = 3 + 1$ dimensions,
\[ \lagrange_{\ipic} = - \left( \lambda_0  + \tfrac{1}{4!} \lambda_4 \phi^4 + g_s \phi \bar{\psi}  i \gamma_5 \psi \right) \]
such that all the terms are even under parity. 
\end{example}

\begin{proposition}
\[  \parity^{-1} \bar{\psi} \psi(x) \parity = \bar{\psi} \psi(\bar{x}) \quad \text{and} \quad \parity^{-1} \bar{\psi} i \gamma^5 \psi (x) \parity = - \bar{\psi} i \gamma^5 \psi (\bar{x}) \]
\end{proposition}

\begin{proof}
First consider,
\[ \parity^{-1} \bar{\psi}(x) \parity = \parity^{-1} \psi^{\dagger}(x) \gamma^0 \parity = (\parity^{-1} \psi(x) \parity)^\dagger \gamma^0 = ( \psi^\dagger(\bar{x}) \gamma^0) \gamma^0  = \bar{\psi}(\bar{x}) \gamma^0 \]
Therefore, 
\[ \parity^{-1} \bar{\psi}(x) \psi(x) \parity = \parity^{-1} \bar{\psi}(x) \parity \parity^{-1} \psi(x) \parity = \bar{\psi}(\bar{x}) (\gamma^0)^2 \psi(\bar{x}) = \bar{\psi}(\bar{x}) \psi(\bar{x}) \]
Likewise, using the anticommutation relations,
\[ \parity^{-1} \bar{\psi}(x) i \gamma^5 \psi(x) \parity = \parity^{-1} \bar{\psi}(x) \parity i \gamma^5 \parity^{-1} \psi(x) \parity = \bar{\psi}(\bar{x}) \gamma^0 i \gamma^5 \gamma^0 \psi(\bar{x}) = \bar{\psi}(\bar{x}) (-  [\gamma^0]^2 i \gamma^5 ) \psi(\bar{x}) = - \bar{\psi}(\bar{x}) i \gamma^5 \psi (\bar{x})  \]
which shows that the combination,
\[ \parity^{-1} \phi \bar{\psi}(x) i \gamma^5 \psi(x) \parity = \phi \bar{\psi}(x) i \gamma^5 \psi(x) \]
is invariant under parity for a pseudo-scalar field $\phi$. 
\end{proof}

\section{Interaction Picture at Tree Level}

Let $\ket{p} = \adag_p \ket{0}$ with commutation relations $[\a_p, \adag_p] = (2 \pi)^3 (2 p^0) \delta^3(\vec{p} - \vec{p}')$. Now, we use the mode expansions,
\[ \phi(x) = \int \frac{\d{^3p}}{(2\pi)^3} \frac{1}{2 p^0} \left( \a_p e^{- i p x} + \adag_p e^{i p x} \right) \]
for the pseudo-scalar and for the fermion fields, 
\[ \psi(x) = \sum_{s} \int \frac{\d{^3p}}{(2\pi)^3} \frac{1}{2 p^0} \left( \b^s_p u^s_p e^{- i p x} + (\c^s_p)^\dagger v^s_p e^{i p x} \right) \]
and likewise,
\[ \bar{\psi}(x) = \sum_{s} \int \frac{\d{^3p}}{(2\pi)^3} \frac{1}{2 p^0} \left( (\b^s_p)^\dagger \bar{u^s_p} e^{i p x} + \c^s_p \bar{v^s_p} e^{-i p x} \right) \]
Then we can write the interaction time evolution operator,
\[ U_I = \mathbf{T} e^{-i \int \d{^4 x} \mathcal{H}_I(x)} \]
where the interaction Hamiltonian density is,
\[ \mathcal{H}_I(x) = \tfrac{1}{4!} \lambda \phi(x)^4 + g \phi(x) \bar{\psi}(x) i \gamma^5 \psi(x) \]
Define the incomming and outgoing states as,
\[ \ket{i} = (\b_{p_1}^{s_1})^\dagger (\b_{p_2}^{s_2})^\dagger \ket{0} \quad \text{and} \quad \ket{f} = (\b_{p_1'}^{s_1'})^\dagger (\b_{p_2'}^{s_2'})^\dagger \ket{0} \]
To tree-level, we can expand the scattering matrix as,
\[ \bra{f} U_I \ker{i} = \inner{f}{i} + \tfrac{1}{2} g^2 \int \dn{4}{x_1} \d{4}{x_2} \bra{0} \b_{p_2'}^{s_2'} \b_{p_1'}^{s_1'} \mathbf{T} [ \phi(x_1) \bar{\psi}(x_1) \gamma^5 \psi(x_1) \phi(x_2) \bar{\psi}(x_2) \gamma^5 \psi(x_2)  ] (\b_{p_1}^{s_1})^\dagger (\b_{p_2}^{s_2})^\dagger \ket{0} \]
By Wick's theorem, the only term in the time ordered product that contributes is, 
\[ \Delta_{\phi}(x_1 - x_2) :  \bar{\psi}(x_1) \gamma^5 \psi(x_1) \bar{\psi}(x_2) \gamma^5 \psi(x_2) : \]
Therefore, adding two minus signs due to moving $\phi$ fields past eachother,
\[ \mathcal{A} = \frac{1}{2} g^2 \int \dn{4}{x_1} \dn{4}{x_2} \Delta_\phi(x_1 - x_2) \gamma^5_{\alpha_1 \beta_1} \gamma^5_{\alpha_2 \beta_2} \bra{0} \b_{p_2'}^{s_2'} \b_{p_1'}^{s_1'}  \bar{\psi}_{\alpha_1}^{cr}(x_1) \bar{\psi}_{\alpha_2}^{cr}(x_2) \psi_{\beta_2}^{an}(x_2) \psi_{\beta_1}^{an}(x_1)  (\b_{p_1}^{s_1})^\dagger (\b_{p_2}^{s_2})^\dagger \ket{0} \]


\section{April 5}
Writing the mode expansions,
\[ \phi(x) = \int \frac{\dn{3}{\vec{p}}}{(2\pi)^3} \left( \a_p e^{- i p x} + \adag_p e^{i p x} \right) \]
We call the plane wave solutions, $w_p (x) = e^{- i p x}$ and thus,
\[ \phi(x) = \int \frac{\dn{3}{\vec{p}}}{(2\pi)^3} \frac{1}{2 p^0} \left( \a_p w_p(x) + \adag_p w_p(x)^* \right) \]
Similarly for Fermions,
\[ \psi(x) = \sum_{s} \int \frac{\dn{3}{\vec{p}}}{(2\pi)^3} \frac{1}{2 p^0} \left( \b_p^s u^s_p e^{-i p x} + (\c_p^s)^\dagger v_p^s e^{i p x} \right) \]

Use condensed notation,
\[ \phi_X = \sum_i \a_i x_i(x) + \adag_i w_i(x)^* \]
and 
\[ \psi_X = \sum_{i} \b_i u_i(x) + \cdag_i v_i(x) \]
and 
\[ \bar{\psi}_X = \sum_{i} \bdag_i \bar{u}_i (x) + \c_i \bar{v}_i (x) \]

\subsection{External Contractions}
We have the contractions,
$$
\contraction{}{\psi_{\alpha}(x)}{}{(\b_p^s)^\dagger}
\psi_{\alpha}(x) (\b_p^s)^\dagger = \bra{0} \psi_\alpha(x) (\b_p^s)^\dagger \ket{0} = (u_p^s)_\alpha e^{ - i p x}
$$
and in condensed notation,
$$
\contraction{}{\psi_X}{}{\bdag_1}
\psi_X \bdag_1 = u_1(x)
$$
and likewise,
$$
\contraction{}{\b_1}{}{\bar{\psi}_X}
\b_1 \bar{\psi}_X = \bra{0} \b_1 \bar{\psi}_X \ket{0} = \bar{u}_1(X)
$$
and,
$$
\contraction{}{\c_1}{}{\psi_X}
\c_1 \psi_X = v_1(x)
$$
and finally,
$$
\contraction{}{\bar{\psi}_X}{}{\cdag_1}
\bar{\psi}_X \cdag_1 = \bar{v}_1(x)
$$

\subsection{Internal Contractions,}
$$
\contraction{}{\phi_X}{}{\phi_Y}
\phi_X \phi_Y = \bra{0} \mathbf{T} \phi_X \phi_Y \ket{0} = \Delta_{X Y} = \int \frac{\dn{4}{p}}{(2 \pi)^4} \frac{i}{p^2 - \mu^2 + i \epsilon} e^{- i p \cdot (x - y)}
$$
and similarly,
$$
\contraction{}{-\bar{\psi}_Y}{}{\psi_X}
- \bar{\psi}_Y \psi_X = 
\contraction{}{\psi_X}{}{\bar{\psi}_Y}
\psi_X \psi_Y = \bra{0} \mathbf{T} \psi_X \bar{\psi}_Y \ket{0} = S_{X Y} = \int \frac{\dn{4}{p}}{(2 \pi)^4} \frac{i(\slashed{p} + m)}{p^2 - m^2 + i \epsilon} e^{- i p \cdot (x - y)}
$$

\subsection{S-Matrix Contribution at Tree Level}
For the process $\psi \psi \to \psi \psi$ in the scalar Yukawa theory, we get,
\[ i \mathcal{M} = (2\pi)^4 \delta^4(p_{\mathrm{out}} - p_{\mathrm{in}}) (-ig)^2 \left[ \frac{i}{(p_1 - p_1')^2 - m^2} (\bar{u}^{s_1'}_{p_1'} \cdot u_{p_1}^{s_1}) (\bar{u}_{p_2'}^{s_2'} \cdot u_{p_2}^{s_2}) -  \frac{i}{(p_1 - p_2')^2 - m^2} (\bar{u}^{s_1'}_{p_1'} \cdot u_{p_2}^{s_2}) (\bar{u}_{p_2'}^{s_2'} \cdot u_{p_1}^{s_1}) \right]
\]

Similarly, for the process $\psi \psi \to \psi \psi$ in the pseudo-scalar Yukawa theory,
we have to take $-ig \mapsto g$ and insert $\gamma^5$ matrices between the endpoints connected to the matrices. Therefore,
\[ i \mathcal{M} = (2\pi)^4 \delta^4(p_{\mathrm{out}} - p_{\mathrm{in}}) (g)^2 \left[ \frac{i}{(p_1 - p_1')^2 - m^2} (\bar{u}^{s_1'}_{p_1'} \gamma^5 u_{p_1}^{s_1}) (\bar{u}_{p_2'}^{s_2'} \gamma^5 u_{p_2}^{s_2}) -  \frac{i}{(p_1 - p_2')^2 - m^2} (\bar{u}^{s_1'}_{p_1'} \gamma^5 u_{p_2}^{s_2}) (\bar{u}_{p_2'}^{s_2'} \gamma^5 u_{p_1}^{s_1}) \right]
\]

\section{Higgs Model}

The Higgs field is a doublet of complex scalars,
\[ \Phi = \begin{pmatrix}
\phi^1 \\
\phi^2 
\end{pmatrix}\]
In the model, we have doublets of left-handed spinors,
\[ \begin{pmatrix}
L^1 \\
L^2
\end{pmatrix}\]
and singlets of right handed spinors $R$. The free Lagrangian is written as,
\[ \lagrange_0 = (L^1)^\dagger i \bar{\sigma}^\mu \partial_\mu L^1 + (L^2)^\dagger i \bar{\sigma}^\mu \partial_\mu L^2 + R^\dagger i \sigma^\mu \partial_\mu R + \partial_\mu (\phi^1)^\dagger \partial^\mu \phi^1 + \partial_\mu (\phi^2)^\dagger \partial^\mu \phi^2 \]
We add an interaction Lagrangian,
\[ \lagrange_{\ipic} = - \lambda \left( \phi^1 (L^1)^\dagger + \phi^2 (L^2)^\dagger \right) R + h.c. + g \left( (\phi^1)^* \phi^1 + (\phi^2)^* \phi^2 - v^2 \right)^2 \]
This theory has $\mathrm{SU}(2) \times \mathrm{U}(1)$ symmetry via the action,
\[ U = e^{-i \frac{1}{2} \vec{\alpha} \cdot \vec{\sigma}} \quad 
\begin{pmatrix}
\phi^1 \\
\phi^2 
\end{pmatrix}
\mapsto U
\begin{pmatrix}
\phi^1 \\
\phi^2 
\end{pmatrix} \]
for the $\mathrm{SU}(2)$ symmetry and the $\mathrm{U}(1)$ acts with charges $q_R = 1$ and $q_L = \frac{1}{2}$ and $q_{\phi} = -\frac{1}{2}$.
Now we pick a vacuum,
\[ 
\begin{pmatrix}
\phi^1 \\
\phi^2 
\end{pmatrix}
=
\begin{pmatrix}
v \\
0 
\end{pmatrix}\]
which breaks the total $\mathrm{SU}(1)$ symmetry. 

\section{Quantum Electrodynamics}

Classically, we have the action,

\[ S = - \frac{1}{4} \int \dn{4}{x} F_{\mu \nu} F^{\mu \nu} \]
where the strength tensor is,
\[ F_{\mu \nu} = \partial_\mu A_\nu - \partial_\nu A_\mu \]
We define,
\[ \vec{E} = - \nabla \phi - \pderiv{}{t} \vec{A} \]
and therefore,
\[ E^i = - \partial_i A_0 + \partial_0 A_i = F_{0 i} \]
Furthermore,
\[ \vec{B} = - (F_{23}, F_{31}, F_{12}) \]
Therefore, we can write,
\[ S = \int \dn{4}{x} \left[ - \frac{1}{2} F_{0i} F^{0i} - \frac{1}{4} F_{ij} F^{ij} \right] = \int \dn{4}{x} \tfrac{1}{2} \left[E^2 - B^2 \right] \]
Varying this action with respect to the fields $A^\mu$ we get the equation of motion,
\[ \partial_\mu F^{\mu \nu} = 0 \]
which give Gauss' law and Ampere's law. Furthermore, the Bianchi identity,
\[ \partial_\alpha F_{\beta \gamma} + \partial_\beta F_{\gamma \alpha} + \partial_{\gamma} F_{\alpha \beta} = 0 \]
gives Gauss' law for magnetism and Faraday's law. 

\begin{remark}
We should notice the following,
\begin{enumerate}
\item $A_0$ is not dynamical. $\partial_t A_0$ does not appear in the Lagrangian.

\item $A_0$ is determined by $\vec{A}(x)$ and the equation $\nabla \cdot \vec{E} = 0$. Using this equation,
\[ \nabla \cdot (\partial_0 \vec{A} + \nabla A_0) = 0 \implies \nabla^2 A_0 = - \nabla \cdot (\partial_0 \vec{A}) \] 

\item If $A_\mu$ is a solution to the equations of motion then $A_\mu' = A_\mu + \partial_\mu \lambda$ for an arbitrary $\lambda(x)$ is also a solution. This is because, $F'_{\mu \nu} = F_{\mu \nu}$.  
\end{enumerate}
\end{remark}

For cannonical quantization, we need to choose a gauge condition to lift the gauge freedom. There are many possible choices of gauge fixing,
\begin{enumerate}
\item Coulomb Gauge: $\nabla \cdot \vec{A} = 0 \implies A_0 = 0$ in vacuum. However, not Lorentz invariant.

\item Lorenz Gauge: $\partial_\mu A^\mu = 0$. Does not completely fix the gauge up to gauge a transformation such that $\partial_\mu \partial^\mu \lambda = 0$ 
\end{enumerate}

\subsection{Quantization}
The conjugate momentum becomes,
\[ \Pi^i = \pderiv{\lagrange}{(\partial_0 A_i)} = F^{0 i} = E^i = \partial_0 A_i - \partial_i A_0 \]
and we have,
\[ \Pi^0  = \pderiv{\lagrange}{(\partial_0 A_0)} = 0 \]
which reflects the fact that $A_0$ is not dynamical.
These conjugate variables give rise to a Hamiltonian,
\[ H = \int \dn{3}{x} \Pi^i \partial_0 A_i - \lagrange = \int \dn{3}{x} \left[ \tfrac{1}{2} \Pi^2 + \tfrac{1}{2} B^2 - A_0 \nabla \cdot \vec{\Pi} \right] \]
\subsubsection{Mode Expansion}
We take the Coulomb gauge. Therefore,
\[ \partial_\mu \partial^\mu \vec{A} = 0 \]
so we have a solution,
\[ \vec{A} = \vec{e}_p e^{- i p x} \]
and the gauge condition forces,
\[ \vec{p} \cdot \vec{e}_p = 0 \]
Using these quantities, we can write a mode expansion,
\[ \vec{A}(x) = \int \frac{\dn{3}{\vec{p}}}{(2 \pi )^3} \frac{1}{\sqrt{2 p^0}} \left( \a_{\vec{p}}^r \: \vec{e}^{\, r}_{\vec{p}} e^{-i p x} + (\a^r_{\vec{p}})^\dagger \: (\vec{e}^{\, r}_{\vec{p}})^* e^{i p x} \right) \]
with (nonrelativistic) commutation relations,
\[ [ \a^r_{\vec{p}}, \a^{r'}_{\vec{p'}} ] = (2 \pi)^3 \delta^3( \vec{p} - \vec{p'}) \]
These give equal time commutation relations,
\[ [A_i(\vec{x}), \Pi^j(\vec{y}) ] = \int \frac{ \dn{3}{\vec{p}}}{(2 \pi )^3} \: i \left( \delta_{ij} - \frac{p_i p_j}{\vec{p}^{\, 2}}  \right) e^{i \vec{p} \cdot (\vec{x} - \vec{y})} \]

\subsection{Over-Quantization}

We want to work in manifestly Gauge invariant notation without fixing the Coulomb gauge condition. We would want to write commutation relations,
\[ [ A_\mu(t, \vec{x}), \Pi_\nu(t, \vec{y}) ] = i \eta_{\mu \nu} \delta^3(\vec{x} - \vec{y}) \]
However, this cannot hold since $\Pi_0  = 0$. This will not work unless we change our Lagrangian. We choose a new Lagrangian,
\[ \lagrange = - \frac{1}{4} F_{\mu \nu} F^{\mu \nu} - \frac{\alpha}{2} (\partial_\mu A^\mu )^2 \]
Under the Lorenz gauge condition, $\partial_\mu A^\mu = 0$ we recover our original Lagrangian. 
Now we find that,
\[ \Pi^0 = \pderiv{\lagrange}{(\partial_0 A_0)} = - \alpha \partial_\mu A^\mu \]
The equations of motion from this action become,
\begin{align*}
\delta S & = - \int (\partial_\mu \delta A_\nu) F^{\mu \nu} + \alpha (\partial_\nu \delta A^\nu ) (\partial_\mu A^\mu) 
\\
& = \int \partial_\mu F^{\mu \nu} \delta A_\nu + \alpha \partial^\nu (\partial_\mu A^\mu) \delta A_\nu
\\
& = \int \left( \partial_\mu F^{\mu \nu} + \alpha \partial^\nu \partial_\mu A^\mu \right) \delta_\nu = 0
\end{align*}
Therefore, the equations of motion become,
\[ \partial_\mu \partial^\mu A^\nu + (\alpha - 1) \partial^\nu \partial_\mu A^\mu = 0 \]
If we choose $\alpha = 0$ this equation is vastly simplified to,
\[ \partial_\mu \partial^\mu A^\nu = 0 \]
Therefore, $A$ satisfies a Klien-Gordon equation so it has a mode expansion,
\[ A_\mu(x) = \int \frac{\dn{3}{k}}{(2 \pi )^3} \frac{1}{\sqrt{2 k}} \sum_{\lambda = 0}^3 \left( \epsilon_\mu(\vec{k}, \lambda) \a(k, \lambda) e^{- i k x} + h.c. \right) \]

\subsubsection{Polarization Vectors}
The time-like polarization,
\[ \epsilon^\mu(k, 0) = 
\begin{pmatrix}
1 \\
0 \\
0 \\
0
\end{pmatrix} \]
and the space-like polarization,
\[ \epsilon^\mu(k, 3) = 
\begin{pmatrix}
1 \\
\hat{k}
\end{pmatrix} \]
We have that,
\[ \epsilon^\mu(k, \lambda) \epsilon_\mu(k, \lambda') = \eta^{\lambda \lambda'} \]
and also,
\[ \epsilon^\mu(k, \lambda) \epsilon^\nu(k, \lambda') \eta_{\lambda \lambda'} = \eta^{\mu \nu }\]
Imposing the condition that,
\[ [ A_\mu(t, \vec{x}), \Pi_\nu(t, \vec{y}) ] = i \eta_{\mu \nu} \delta^3(\vec{x} - \vec{y}) \] 
gives the worrying conditions,
\[ [\a(k, \lambda), \adag(k', \lambda') ] = - (2 \pi )^4 \delta^4(k - k') \delta_{\lambda \lambda'} \]


\subsubsection{Constraints} 
We need to constrain the Lagrangian such that $\partial_\mu A^\mu = 0$. However, we cannot do this directly. 
We have some options for constraints,
\begin{enumerate}
\item[1.] 

\item[2.] Let $\partial_\mu A^\mu$ kills the physical states,
\[ \partial_\mu \hat{A}^\mu \ket{\psi_{\text{phys}}} = 0 \]
This does not work because, $\Pi_0 \ket{\psi_{\text{phys}}} = 0$ so,
\[ \bra{\psi_{\text{phys}}} [ A_0(t, \vec{x}, \Pi_0(t, \vec{y} ] \ket{\psi_{\text{phys}}} = i \delta^3(\vec{x} - \vec{y}) \]
but it is zero by the constraints. 

\item[3.] Take the expectation values,
\[ \bra{\psi_{\text{phys}}} \partial_\mu \hat{A}^\mu \ket{\psi_{\text{phys}}}  = 0 \]
\end{enumerate}

\subsection{The Action for Quantum Electrodynamics}

\subsubsection{The Gauge Covariant Derivative}

Define $\mathcal{D}_\mu = \partial_\mu + iq A_\mu$ 

\begin{lemma}
Under a local gauge transformation given by $\psi(x) \mapsto e^{i q \lambda(x) } \psi(x)$ and $A_\mu \mapsto A_\mu - \partial_\mu \lambda$ the gauge covariant derivative transforms as, 
$\mathcal{D}_\mu \psi(x) \to e^{i q \lambda(x) } \psi(x)$
\end{lemma}

\begin{proof}

\end{proof}

Therefore, the term $\bar{\psi} \mathcal{D}_\mu \psi$ is gauge invariant. Therefore the entire action,
\[ \lagrange_{QED} = - \frac{1}{4} F_{\mu \nu} F^{\mu \nu} + \bar{\psi} (i \gamma^\mu \mathcal{D}_\mu - m ) \psi \]
We can expand this action,
\[ \lagrange_{QED} = - \frac{1}{4} F_{\mu \nu} F^{\mu \nu} + \bar{\psi} (i \gamma^\mu \partial_\mu + i q \gamma^\mu A_\mu) - m ) \psi = - \frac{1}{4} F_{\mu \nu} F^{\mu \nu} + \bar{\psi} (i \gamma^\mu \partial_\mu - m ) \psi -  q \bar{\psi} \gamma^\mu A_\mu \psi \]
Therefore, the QED interaction term is,
\[ \lagrange_{\text{int}} = - q \bar{\psi} \gamma^\mu A_\mu \psi \]

\subsubsection{Scalar QED}
The action of a complex scalar field,
\[ \lagrange_{\text{scalar}} = \partial_\mu \psi^\dagger \partial^\mu \psi - m^2 \psi^\dagger \psi \] 
has global $U(1)$ symmetry. We can upgrade this to a local $U(1)$ symmetry by replacing these derivatives with covariant derivatives. 
\[ \lagrange_{\text{scalar}-QED} = - \frac{1}{4} F_{\mu \nu} F^{\mu \nu} + (\mathcal{D}_\mu \psi)^\dagger (\mathcal{D}^\mu \psi) - m^2 \psi^\dagger \psi \]
This action is gauge invariant under a global $U(1)$ symmetry. Expanding this action,
\begin{align*}
\lagrange_{\text{scalar}-QED} & = - \frac{1}{4} F_{\mu \nu} F^{\mu \nu} + (\partial_\mu - i q A_\mu) \psi^\dagger (\partial^\mu + i q A^\mu) \psi - m^2 \psi^\dagger \psi
\\
& = - \frac{1}{4} F_{\mu \nu} F^{\mu \nu} + \partial_\mu \psi^\dagger \partial^\mu \psi + i q A_\mu ( \partial^\mu \psi - \partial^\mu \psi^\dagger) + q^2 A_\mu A^\mu \psi^\dagger \psi - m^2 \psi^\dagger \psi 
\end{align*}
Therefore, in scalar QED the interaction term is,
\[ \lagrange_{\text{int}} = i q A_\mu ( \partial^\mu \psi - \partial^\mu \psi^\dagger) + q^2 A_\mu A^\mu \psi^\dagger \psi \]
If we consider the free Lagrangian alone,
\[ \lagrange_{\text{free}} = \partial_\mu \psi^\dagger \partial^\mu \psi - m^2 \psi^\dagger \psi \]
we get a conserved Noether current,
\[ j^\mu = i \left( \psi^\dagger \partial^\mu \psi - \psi \partial^\mu \psi^\dagger \right) \]
One might try to couple the vector potential to this current $j^\mu$. However, the Lagrangian,
\[ \lagrange_{1} = \partial_\mu \psi^\dagger \partial^\mu \psi - m^2 \psi^\dagger \psi - e A_\mu i \left( \psi^\dagger \partial^\mu \psi - \psi \partial^\mu \psi^\dagger \right) \]
is not Gauge invariant. This might not be a problem but what is certianlly a problem is that this Lagrangian is not consistent. We know that,
\[ \partial_\mu F^{\mu \nu} = j^\nu \implies \partial_{\nu} \partial_{\mu} F^{\mu \nu} = \partial_{\nu} j^{\nu} = 0 \]
However, $j^\mu$ is no longer the Noether current because we have introdced new derivative terms in $\psi$. Thus, $j^{\mu}$ will not in general be conserved. We can fix this by considering the Noether current of global $U(1)$ symmetry in this new Lagrangian. This current becomes,
\[ j_{\text{true}}^\mu = i \left( \psi^\dagger \partial^\mu \psi - \psi \partial^\mu \psi^\dagger \right) + e A^\mu \psi^\dagger \psi \]
If we couple this true Noether current to the electromagnetic field we get the Lagrangian,
\begin{align*} 
\lagrange_{true} & = \partial_\mu \psi^\dagger \partial^\mu \psi - m^2 \psi^\dagger \psi - e A_\mu i \left( \psi^\dagger \partial^\mu \psi - \psi \partial^\mu \psi^\dagger \right) - e^2 A_mu A^\mu \psi^\dagger \psi
\\
& = (\mathcal{D}_\mu \psi)^\dagger \mathcal{D}^\mu \psi - m^2 \psi^\dagger \psi
\end{align*}
Which is manifestly gauge invariant. Furthermore, we have not introduced any new derivatives so the true Noether current remains the same when this new term is added to the Lagrangian. Thus, this process terminates since the current is still conserved which means that the equations of motion are consistent. 

\section{Non-Abelian Gauge Theory}

Suppose there exists a Lie Group $G$ acting on the field variable $\psi$ via $g \cdot \psi$ such that the Lagrangian is invariant under global $G$-symmetry. That is, the transformation $\psi \mapsto g \cdot \psi$ takes $\lagrange \to \lagrange$. Now let $g : X \to G$ be a continuous map from our space-time manifold to the Lie Group. We want to upgrade our global $G$-symmetry to local $G$-symmetry which is invariant under maps $\psi \mapsto g(x) \cdot \psi$. Consider a four-vector field of $G$-elements defined to transforms as,
\[ A_\mu \mapsto g(x) A_{\mu} g^{-1}(x) + i (\partial_\mu g(x)) g^{-1}(x) \]
We can define the covariant derivative as,
\[ \mathcal{D}_\mu \psi = \partial_\mu \psi + i  A_\mu \cdot \psi \]

\begin{lemma}
Under the tranformation $\psi \mapsto g(x) \cdot \psi$ the covariant derivative transforms as,
\[ \mathcal{D}_\mu \psi \mapsto g(x) \cdot \mathcal{D}_\mu \psi\]
\end{lemma}

\begin{proof}
\begin{align*}
\mathcal{D}_\mu' \psi' & = \partial_\mu g(x) \cdot \psi + i A_\mu' g(x) \cdot \psi 
\\
& = g(x) \cdot \partial_\mu \psi + (\partial_\mu g(x)) \cdot \psi + i g(x) A_\mu g^{-1}(x) g(x) \cdot \psi - (\partial_\mu g(x))  g^{-1}(x) g(x) \cdot \psi   
\\
& = g(x) \cdot \partial_\mu \psi + (\partial_\mu g(x)) \cdot \psi + i g(x) A_\mu \cdot \psi - (\partial_\mu g(x)) \cdot \psi   
\\
& = g(x) \cdot \left(\partial_\mu \psi \psi + i A_\mu \psi \right) 
\\
& = g(x) \cdot \mathcal{D}_\mu \psi 
\end{align*}
\end{proof}

Now that we have good covariant derivates we can construct locally gauge invariant Lagrangians. Furthermore, we can define the field strength tensor as the curvature associated with the covariant derivative,
\[ F_{\mu \nu} = - i [ \mathcal{D}_\mu, \mathcal{D}_\nu ]  =\partial_\mu A_{\nu} \partial_{\nu} A_\mu - [ A_{\mu}, A_{\nu}] \]
which transforms as,
\[ F_{\mu \nu} \mapsto g F_{\mu \nu} g^{-1} \]
The simpliest (most relevant) gauge invariant Lorentz invariant scalar we can form from $A_\mu$ and its derivatives is,
\[ \Tr{F_{\mu \nu} F^{\mu \nu}} \]
Using these objects, we can write down the Yang-Mills action, the simpliest gauge invariant and Lorentz invariant interacting Lagrangian for a non-abelian gauge theory,
\[ \lagrange_{YM} = - \tfrac{1}{4} \Tr{F_{\mu \nu} F^{\mu \nu}} + \left( \mathcal{D}_\mu \psi \right)^\dagger \mathcal{D}^\mu \psi - m^2 \psi^\dagger \psi \]
where the $\psi^\dagger$ transforms as $\psi^\dagger \mapsto \psi^\dagger \cdot g^{-1}$. This is equivalent to the requirement that the action of $G$ on $\psi$ is an isometry i.e. preserves the inner product $\psi^{\dagger} \psi$.

\section{The Magnetic Moment}

In the nonrelativistic limit when scattering an election off a backgroud electromagnetic field there is an effective potentia,
\[ V(\vec{x})) = e A^0(\vec{x}) - g \frac{e}{2m} \vec{S} \cdot \vec{B} \]
where $g = 2$ at tree level. However, when loops are included in the scattering before being integrated out, the g-factor can be written as $g = 2 + 2 a$ where $a$ is known as the anomalous magnetic moment. We want ot calculate the interaction between an electron and a backgroud ``classical'' field which we model in quantum field theory as scattering off a very heaving particle, 

  



\end{document}


