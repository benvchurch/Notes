\documentclass[12pt]{article}
\usepackage{import}
\import{./}{AlgGeoCommands}

\begin{document}

\newcommand{\g}[2]{\mathfrak{g}^{#1}_{#2}}

\section{Basic Definitions and Examples}

\subsection{Genera}

\subsection{Riemann-Roch}

\subsection{Riemann–Hurwitz}

\section{Hyperelliptic Curves}

\begin{defn}
A curve $C$ is \textit{hyperelliptic} if there exists a degree two map $f : C \to \P^1$. 
\end{defn}

\begin{lemma}
A curve $C$ is hyperelliptic iff $\Omega_C^1$ is not very ample.
\end{lemma}

\begin{proof}
(DO THIS)
\end{proof}

\begin{prop}
Plane curves with $g > 1$ cannot be hyperelliptic.
\end{prop}

\begin{proof}
Let $\iota : C \embed \P^2$ be a plane curve. Then $\Omega^1_C = \iota^* \struct{\P^2}(d - 3)$ where $d$ is the degree of $C$. Since $g > 1$ we must have $d > 3$ and thus $\struct{\P^2}(d - 3)$ is very ample defining the Veronese embedding $v : \P^2 \to \P^N$ s.t. $\struct{\P^2}(d - 3) = v^* \struct{\P^N}(1)$. Then $v \circ \iota : C \to \P^N$ is an embedding such that $(v \circ \iota)^* \struct{\P^N}(1) = \Omega^1_C$. Thus $\Omega^1_C$ is very ample so $C$ cannot be hyperelliptic.  
\end{proof}

\begin{lemma}
Let $C$ have a $\g{1}{2}$ then $C$ is either hyperelliptic or rational.
\end{lemma}

\begin{proof}
Let $D$ be a $\g{1}{2}$ then $|D|$ defines a rational map $C \rat \P^1$ of degree two. Suppose $P$ were a basepoint of $|D|$ then $\dim |D - P| = 1$ which implies that $C$ is rational because there is a rational degree one map $C \rat \P^1$. 
\end{proof}

\begin{prop}
Any genus $2$ curve is hyperelliptic. 
\end{prop}

\begin{proof}
Consider the canonical divisor $K_X$ which has $\deg{K_X} = 2g - 2 = 2$ and $\dim{|K_X|} = g - 1 = 1$ and thus gives a $\g{1}{2}$. 
\end{proof}

\section{Tangent Space}

\newcommand{\Bl}[2]{\mathrm{Bl}_{#1} \left( #2 \right)}

\begin{defn}
Let $X$ be a scheme and $x \in X$ a point. Then we define:
\begin{enumerate}
\item the geometric tangent space $T_x X = \Spec{\Sym{\kappa(x)}{\m_x / \m_x^2}}$
\item the projectiveized tangent space $\P(T_x X) = \Proj{\Sym{\kappa(x)}{\m_x / \m_x^2}}$
\item the geometric tangent cone $C_x X = \Spec{\gr{\m_x}{\stalk{X}{x}}}$ where,
\[ \gr{\m_x}{\stalk{X}{x}} = \bigoplus_{n = 0}^\infty \m_x^n / \m_x^{n+1} \]
\item the projectiveized tangent cone $\P(C_x X) = \Proj{\gr{\m_x}{\stalk{X}{x}}}$.
\end{enumerate}
\end{defn}

\begin{rmk}
In particular, blowing up $X$ at the sheaf of ideals $\I_x$ (defined as the subsheaf of $\struct{X}$ where evaluation in $\kappa(x)$ gives zero) gives the following,
\[ \tilde{X} =  \rProj{X}{\bigoplus_{n = 0}^\infty \I_x^n} \]
Choose an affine open neighborhood $x \in \Spec{A} = U \subset X$ then we see $\I_x |_U = \wt{\p} \subset A$ is the prime corresponding to $x \in \Spec{A}$ and $\m_x = \p A_\p$. Therefore, restricting $\pi : \tilde{X} \to X$ over $U$ gives,
\[ \Proj{\bigoplus_{n = 0}^\infty \p^n} \to \Spec{A} \]
and,
\[ \Bl{\p}{A} = \bigoplus_{n = 0}^\infty \p^n \]
is the blowup algebra which is a graded $A$-algebra. Consider the fiber over $x$,
\[ \Proj{\Bl{\p}{A} / \p \Bl{\p}{A}} \to \Spec{\kappa(x)} \]
where we see,
\[ \Bl{\p}{A} / \p \Bl{\p}{A} = \bigoplus_{n = 0}^\infty \p^n = \gr{\p}{A} \]
and therefore $\tilde{X}_x \to \Spec{\kappa(x)}$ is $\Proj{\gr{\p}{A}} \to \Spec{\kappa(x)}$.
In particular, the tangent cone is the fiber over $x$ in the blowup. 
\end{rmk}

\begin{rmk}
The exact same construction shows that given a ring $A$ and ideal $I \subset A$ the blowup $\Proj{\Bl{I}{A}} \to \Spec{A}$ where,
\[ \Bl{I}{A} = \bigoplus_{n = 0}^\infty I^n \]
is the blowup algebra, has fiber over the closed subscheme $V(I)$ equal to,
\[ \Proj{\gr{I}{A}} \to \Spec{A/I} \]
which is the projectized tangent cone of $I$. 
\end{rmk}

\begin{rmk}
We can generalize this further. For a sheaf of ideals $\I \subset \struct{X}$ we can form the blowup,
\[ \tilde{X} = \rProj{X}{\bigoplus_{n = 0}^\infty \I^n} \to X \]
Restricting to the closed subscheme $Z = V(\I) \subset X$ we find,
\[ \rProj{Z}{\bigoplus_{n = 0}^\infty \I^n / \I^{n+1}} \to Z \]
but notice that the graded algebra,
\[ (\struct{X} / \I) \otimes_{\struct{X}} \bigoplus_{n = 0}^\infty \I^n =  \bigoplus_{n = 0}^\infty \I^n / \I^{n+1} = \bigoplus_{n = 0} (\I / \I^2)^{\otimes n} / K = \Sym{\struct{Z}}{\I / \I^2} \]
and $\C_{Z/X} = \I / \I^2$ is the conormal bundle (sheaf) so we find a pullback diagram,
\begin{center}
\begin{tikzcd}
\P(\C_{Z/X}) \arrow[d] \arrow[r] & \tilde{X} \arrow[d, "\pi"]
\\
Z \arrow[r] & X 
\end{tikzcd}
\end{center}
and thus $\tilde{X} \to X$ is a projective bundle over $Z$ and an isomorphism over $X \setminus Z$. We call $\P(\C_{Z/X})$ the projectiveized tangent cone of $Z$.
\end{rmk}

\begin{prop}
If $x \in X$ is a regular point then $C_x X = T_x X$. 
\end{prop}

\begin{proof}
When $\stalk{X}{x}$ is regular then $\gr{\m_x}{\stalk{X}{x}} \cong \kappa(x)[x_1, \dots, x_r]$ where $x_1, \dots, x_r \in \m_x$ are a basis of $\m_x / \m_x^2$ as a $\kappa(x)$-vectorspace. Therefore, $\gr{\m_x}{\stalk{X}{x}} = \Sym{\kappa(x)}{\m_x / \m_x^2}$ as graded rings and thus $C_x X = T_x X$ as well as the projective versons.
\end{proof}

\begin{prop}
Let $X$ be finite type over $k$ and $x \in X$ be a closed point. There is a canonical map,
\[ \widehat{T_x X} \leftarrow \Spec{\widehat{\stalk{X}{x}}} \to \Spec{\stalk{X}{x}} \to X \]
which is an isomorphism exactly when $x \in X$ is regular.
\end{prop}

\begin{proof}
By the Cohen structure theorem $\widehat{\stalk{X}{x}} = k[[x_1, \dots, x_r]] / I$ where $x_1, \dots, x_r \in \m_x$ are a basis of $\m_x / \m_x^2$ and $I = (0)$ exactly when $x \in X$ is regular proving that the canonical map $T_x X \to \Spec{\widehat{\stalk{X}{x}}}$ is 
\end{proof}

\section{Formal Schemes}

\begin{defn}
Let $A$ be a ring and $I \subset A$ an ideal. Then the completion of $A$ along $I$ is,
\[ \hat{A} = \varprojlim_{n} A / I^n \]
Furthermore, for any $A$-module $M$ we can complete $M$ along $I$ to get,
\[ \hat{M} = \varprojlim_{n} M / I^n M = \varprojlim_{n}( M \otimes_A A / I^n) = M \otimes_A \hat{A} \]
\end{defn}

\begin{prop}
Let $A$ be a ring and $I \subset A$ an ideal and $M$ an $A$-module. Then $\hat{M}$ satisfies the following universal property. Any map $\varphi : M \to N$ to a complete $A$-module $N$ facrors uniquely as $M \to \hat{M} \xrightarrow{\tilde{\varphi}} N$. 
\end{prop}

\begin{proof}
The kernel of $M \to N/I^n N$ contains $I^n M$ and thus factors as $M \to M / I^n M \to N / I^n N$. Taking inverse limits gives $M \to \hat{M} \to N$. Uniqueness follows from the fact that a map $\hat{M} \to M$ is determined completely by $\hat{M} \to M / I^n M \to N / I^n N$. 
\end{proof}

\begin{lemma}
Let $A$ be a ring and $I \subset A$ an ideal. Then the units of $\hat{A}$ are exactly those elements which map to units under $\hat{A} \to A / I$.
\end{lemma}

\begin{proof}
Suppose that $u \in \hat{A}$ is a unit. Then clearly its image under $\hat{A} \to A / I$ is a unit. Conversely, suppose that $u \mapsto u_1 \in A / I$ is a unit. Then there exists $v_1 \in A/I$ s.t. $u_1 v_1 = 1$ so lifting $v_1$ we get $u_2 \tilde{v}_1 = 1 + r$ for $r \in I$ so we can take $r u_2 \tilde{v}_1 = r + r^2 = r$ and thus $u_2 (\tilde{v}_1 - r \tilde{v}_1) = 1$. Write $v_2 = \tilde{v}_1 - r \tilde{v}_1$ and we lift to see $u_3 \tilde{v}_2 = 1 + r'$ for $r' \in I^2$ etc giving by induction an element $v \in \hat{A}$ such that $u v = 1$ in each $A / I^n$ and thus in $\hat{A}$.
\end{proof}

\begin{lemma}
Let $\m \subset A$ be a maximal ideal. Then $\hat{A} = \widehat{A_\m}$ is local.
\end{lemma}

\begin{proof}
Consider,
\[ \widehat{A_\m} = \varprojlim_{n} (A_\m / \m^n A_\m) = \varprojlim_{n} (A / \m^n)_\m \]
However, since $A / \m^n$ is local with maximal ideal $\m$ we see that $(A / \m^n)_\m = A / \m^n$ and thus,
\[ \widehat{A_\m} = \varprojlim_{n} (A / \m^n)_\m = \varprojlim_{n} A / \m^n = \hat{A} \]
\end{proof}

\begin{rmk}
Localization does not, in general, behave nicely with completion. For example, let $A = \Z_p[x]$ and $\p = (x)$. Then $\hat{A_{\p}} = \widehat{\Q_p[x]_{(x)}} = \Q_p[[x]]$. However, $\hat{A} = \Z_p[[x]]$ and $\hat{A}_{\hat{\p}} = \Z_p[[x]]_{(x)}$ which is a proper subring of $\Q_p[[x]]$ because it does not contain $1 + p^{-1} x + p^{-2} x^2 + \cdots$ and is this, in particular, not complete.
\end{rmk}

\begin{lemma}
Suppose that $(A, \m)$ is a local ring. Then $\Spec{\hat{A}} \to \Spec{A}$ is a homeomorphism.
\end{lemma}

(THIS IS TOTALLY FALSE IMPLIES A IS UNABRANCH AT LEAST)

\begin{proof}
The units in $\hat{A}$ are everything except the preimage of zero under $\hat{A} \to A / \m = \kappa$. Therefore $\hat{\m} = \m \hat{A}$ is the unique maximal ideal of $\hat{A}$ making $A$ local. I claim that $\p \mapsto \p \hat{A}$ is an inverse to $\Spec{\hat{A}} \to \Spec{A}$. (DO THIS!!)
\end{proof}

\begin{example}
Consider $X = \Spec{k[x, y]/(y^2 - x^2(x-1))} \subset \A^2_k$. Take $p = (x, y)$. We know $X$ is connected and thus $\stalk{X}{p} = A_\p$ has a unique minimal prime. However, 
\[ \widehat{\stalk{X}{p}} = \hat{A} = k[[x, y]] / (y^2 - x^2(x + 1)) \cong k[[x, y]]/(x^2 - y^2) = k[[x, y]]/((y - x)(x + y)) \cong k[[u]] \times k[[v]] \]
which has two minimal primes (branches). 
\end{example}


\section{Multiplicity of a Point}


\begin{defn}
Let $X$ be a curve and $x \in X$ a point. Then the multiplicity $m(x)$ is defined as:
\[ m(x) = \lim_{n \to \infty} \dim_{\kappa(x)} \left( \m_x^n / \m_x^{n+1} \right) \]
\end{defn}

\end{document}