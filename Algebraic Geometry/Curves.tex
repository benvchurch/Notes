\documentclass[12pt]{article}
\usepackage{import}
\import{./}{AlgGeoCommands}

\begin{document}

\newcommand{\g}[2]{\mathfrak{g}^{#1}_{#2}}

\section{Basic Definitions and Examples}

\subsection{Genera}

\subsection{Riemann-Roch}

\subsection{Riemann–Hurwitz}

\section{Hyperelliptic Curves}

\begin{defn}
A curve $C$ is \textit{hyperelliptic} if there exists a degree two map $f : C \to \P^1$. 
\end{defn}

\begin{lemma}
A curve $C$ is hyperelliptic iff $\Omega_C^1$ is not very ample.
\end{lemma}

\begin{proof}
(DO THIS)
\end{proof}

\begin{prop}
Plane curves with $g > 1$ cannot be hyperelliptic.
\end{prop}

\begin{proof}
Let $\iota : C \embed \P^2$ be a plane curve. Then $\Omega^1_C = \iota^* \struct{\P^2}(d - 3)$ where $d$ is the degree of $C$. Since $g > 1$ we must have $d > 3$ and thus $\struct{\P^2}(d - 3)$ is very ample defining the Veronese embedding $v : \P^2 \to \P^N$ s.t. $\struct{\P^2}(d - 3) = v^* \struct{\P^N}(1)$. Then $v \circ \iota : C \to \P^N$ is an embedding such that $(v \circ \iota)^* \struct{\P^N}(1) = \Omega^1_C$. Thus $\Omega^1_C$ is very ample so $C$ cannot be hyperelliptic.  
\end{proof}

\begin{lemma}
Let $C$ have a $\g{1}{2}$ then $C$ is either hyperelliptic or rational.
\end{lemma}

\begin{proof}
Let $D$ be a $\g{1}{2}$ then $|D|$ defines a rational map $C \rat \P^1$ of degree two. Suppose $P$ were a basepoint of $|D|$ then $\dim |D - P| = 1$ which implies that $C$ is rational because there is a rational degree one map $C \rat \P^1$. 
\end{proof}

\begin{prop}
Any genus $2$ curve is hyperelliptic. 
\end{prop}

\begin{proof}
Consider the canonical divisor $K_X$ which has $\deg{K_X} = 2g - 2 = 2$ and $\dim{|K_X|} = g - 1 = 1$ and thus gives a $\g{1}{2}$. 
\end{proof}

\end{document}