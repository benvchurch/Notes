\documentclass[12pt]{article}
\usepackage{hyperref}
\hypersetup{
    colorlinks=true,
    linkcolor=blue,
    filecolor=magenta,      
    urlcolor=cyan,
}
 
\urlstyle{same}
\usepackage{import}
\import{./}{AlgGeoCommands}


\AtBeginDocument{\renewcommand{\d}{\mathrm{d}}}
\newcommand{\ch}[1]{\mathrm{char}\left( #1 \right)}
 

\begin{document}

\tableofcontents

\newpage

\section{I Varieties}

\subsection{Section 1}

\subsubsection{1.1 DO THIS}

\begin{enumerate}
\item Let $Y$ be the plane curve $y = x^2$. Let $A(Y)$ be the affine coodinate ring 
\[ A(Y) = k[x,y]/(y - x^2) \cong k[x] \]
 via the map $y \mapsto x^2$.

\item Let $Z$ be the plane curve $xy = 1$. Considre the affine coodinate ring $A(Y) = k[x,y]/(xy - 1)$. Consider a map $k[x,y]/(xy - 1) \to k[t]$ then $x, y$ map to units but $(k[t])^\times = k^\times$ and thus the map is not surjective. Therefore there cannot be such an isomorphism.

\item Let $f$ be any irreducible quadratic polynomial $f \in k[x,y]$ and let $W$ be the conic defined by $f$. Then write,
\[ f(x,y) = a_0 + a_{1,0} x + a_{0,1} y + a_{1,1} xy + a_{2,0} x^2 + a_{0,2} y^2 \]
where not all $a_{1,1}, a_{2,0}, a_{0,2}$ are zero. Let's do the characteristic not equal to two case first. When $a_{2,0} \neq 0$ we can write,
\[ f(x,y) = a_{2,0} (x - ay - b)^2 + a_{0,2} (y - a' x - b')^2 + a_0' \] 

\end{enumerate}

\subsubsection{1.2}

Let $Y \subset \A^3$ be the set $Y = \{(t, t^2, t^3) \mid t \in k \}$. Clearly, $Y \subset Z = Z(f_1, f_2, f_3)$ where $f_1 = x^2 - y$ and $f_2 = y^3 - z^2$ and $f_3 = z - x^3$. Furthermore, for any $p \in Z$ we know that $y = x^2$ and $z = x^3$ so $p = (x, x^2, x^3) \in Y$ and thus $Y = Z$. Clearly, $\dim{Y} = 1$ because it is is infinite and the image of $\A^1 \to \A^3$. Then,
\[ I(Y) = (y - x^2, z - x^3, y^3 - z^2) \]
Now consider,
\[ A(Y) = k[x,y,z]/I(Y) = k[x] \]
because $y \mapsto x^2$ and $z \mapsto x^3$. 

\subsubsection{1.3}

Let $Y$ be the algebraic set in $\A^3$ defined by the two polynomials $f_1 = x^2  - yz$ and $f_2 = xz - x$. Then $Y = Z(I)$ where $I = (x^2 - yz, xz - x)$. We need to find the minimal primes over $I$. Clearly $(x,y) \supset I$ and $(x,z) \supset I$ and $(z - 1, y - x^2) \supset I$. These are prime ideals and they are minimal because $I$ has height two. Furthermore,
\[ (x, y) \cap (x,z) \cap (z-1, y - x^2) = I \]
so $I$ has three irreducible components.

\subsubsection{1.5}

Let $B$ be a $k$-algebra. It is clear that if $B = A(Y)$ for some affine algebraic set then $B = A(Y) = k[x_1, \dots, x_n]/I(Y)$ is finitely generated and moreover $I$ is radical so $B$ is reduced.
\bigskip\\
Now suppose that $B$ is a reduced finite type $k$-algebra. Then there is a surjection $k[x_1, \dots, x_n] \onto B$ whose kernel is some ideal $I$. Therefore, $B \cong k[x_1, \dots, x_n]/I$. Since $B$ is reduced we see that $I$ is radical and thus $I = I(Z(I))$ and therefore $B = A(Z(I))$.

\subsubsection{1.7 (IN MY NOTES SOMEWHERE PRETTY OBVIOUS)}

\subsubsection{1.10}

\begin{enumerate}
\item Let $Y \subset X$ then choose a maximal chain of closed irreducibles,
\[ Z_0 \subsetneq Z_1 \subsetneq \cdots \subsetneq Z_n \]
inside $Y$ where $n = \dim{Y}$. Then taking closures in $X$ we see that,
\[ \overline{Z}_0 \subsetneq \overline{Z}_1 \subsetneq \cdots \subsetneq \overline{Z}_n \]
is also a chain of closed irreducibles. Furthermore, the inclusions are strict because $\overline{Z}_i \cap Y = Z_i$ and therefore if $\overline{Z}_i = \overline{Z}_{i+1}$ then $Z_i = Z_{i+1}$ which is false. Thus, $\dim{X} \ge n$.

\item Let $X$ be a topological space covered by a family of open subsets $\{ U_i \}$. By the previous part,
\[ \sup \dim{U_i} \le \dim{X} \]
Now choose a maximal chain of closed irreducibles,
\[ Z_0 \subsetneq Z_1 \subsetneq \cdots \subsetneq Z_n \]
in $X$. There is some $U_i$ such that $Z_0 \cap U_s$ is nonempty. Then I claim that $Z_i \cap U_s$ gives such a chain. It is clear that $Z_i \cap U_s$ is closed and irreducible now if $Z_i \cap U_s = Z_{i+1} \cap U_s$ then $U_s^C$ and $Z_i$ cover $Z_{i+1}$ but $Z_{i+1}$ is irreducible so $U_s^C \cap Z_{i+1} = \empty$ which is impossible because $Z_0 \subset Z_{i+1}$ so this must be a chain. Thus, $\dim{U_s} \ge \dim{X}$ proving the proposition.

\item Let $X = \Spec{\Z_p}$ then the point $(p) \in \Spec{\Z_p}$ is closed so $(0) \in \Spec{\Z_p}$ is open and also dense since this is an integral scheme (so all opens are dense). However, $U = \{ (0) \}$ clearly has dimension zero but $\dim{X} = 1$ since we have a chain $(0) \subsetneq (p)$. 

\item Let $Y$ be a closed subset of an irreducible finite-dimensional topological space $X$. Suppose that $\dim{Y} = \dim{X}$. If $Y \subsetneq X$ then any maximal chain in $Y$ can be augmented to give a longer chain by adding on $X$ (since closed sets in $Y$ are closed in $X$ since $Y \subset X$ is closed and irreducibility is not relative). Thus $\dim{Y} < \dim{X}$.

\item (EXAMPLE HERE!)
\end{enumerate}

\subsubsection{1.11}

Let $Y \subset \A^3$ be the curve given by $(t^3, t^4, t^5)$. Consider the ideal,
\[ I = (x^4 - y^3, x^5 - z^3, y^5 - z^4, xz - y^2, yz - x^3, x^2 y - z^2) = (xz - y^2, yz - x^3, x^2 y - z^2) \]
It is clear that $Y \subset Z(I)$. For any $p \in Z(I)$ we choose $t \in k$ such that $t^3 = x$ (we can do this since $k$ is algebraically closed). Then $y^3 = x^4 = t^{12}$ so we can change $t$ by a third root of unity such that $y = t^4$. Then $z^4 = y^5 = t^{20}$ so we can choose $z = t^5$ (WHY) and thus $Z(I) \subset Y$. Therefore $Y = Z(I)$. For dimension reasons ($\dim{Y} = 1$) we see that $\height{I} = 2$. We need to show that $I$ cannot have two generators. Then $I/I^2$ would have two generators as a $A/I$-module where $A = k[x,y,z]$. Then consider $\m = (x,y,z) \subset A$ then $I/I^2 \otimes_A A / \m$ would have two generators as a $A / \m$-module which is a field. However,
\[ M = I/I^2 \otimes_A A / \m = I/\m I \]
Suppose that $x^4 - y^3, x^5 - z^3, y^5 - z^4$ are dependent in $M$ then,
\[ \alpha (xz - y^2) + \beta (yz - x^3) + \gamma (x^2 y - z^3) \in \m I \]
However, every term in $\m  I$ has degree at least $3$ and thus $\alpha = \beta = 0$ because they cannot cancel eachother. Furthermore, there is no $z^3$ in any term of an element of $\m I$ and thus $\gamma = 0$. Thus $\dim{M} = 3$ contradicting the fact that it has two generators.

\subsubsection{1.12}

Consider $f = x^2(x-1)^2 + y^2 \in \RR[x,y]$ then $f$ is irreducible in $\RR[x,y]$ because of unique factorization in $\CC[x,y]$ we have,
\[ f = (x(x-1) + i y)(x(x-1) - iy) \]
but neither factor is in $\RR[x,y]$ and thus $f$ cannot factor. Furthermore, $Z(f)$ is the union of two points $(0,0)$ and $(1,0)$ and thus cannot be irreducible (it's not even connected!).  

\section{II Schemes}

\section{III Cohomology}

\subsection{Section 2}

\subsubsection{2.1 DO!!}

\begin{enumerate}
\item Let $X = \A^1_k$ be the affine line over an infinite field $k$ and $P, Q \in X$ be distinct points. Ex. II.1.19 gives an exact sequence,
\begin{center}
\begin{tikzcd}
0 \arrow[r] & \Z_U \arrow[r] & \Z \arrow[r] & \Z_Y \arrow[r] & 0
\end{tikzcd}
\end{center}
where $Y = \{ P, Q \}$ and $U = X \setminus Y$. Then $\Z_Y = \iota_P \Z \oplus \iota_Q \Z$. Taking the cohomology sequence,
\begin{center}
\begin{tikzcd}
0 \arrow[r] & \Gamma(X, \Z_U) \arrow[r] & \Gamma(X, \Z) \arrow[r] & \Gamma(X, \Z_Y) \arrow[r] & H^1(X, \Z_U) 
\end{tikzcd}
\end{center}
However, $\Gamma(X, \Z) = \Z$ because $X$ is connected and $\Gamma(X, \Z_Y) = \Z \oplus \Z$ because $P, Q \in X$. Therefore, $\Gamma(X, \Z) \to \Gamma(X, \Z_Y)$ cannot be surjective so we must have $H^1(X, \Z_U) \neq 0$.

\item Let $Y \subset X = \A^n_k$ be the union of $n + 1$ hyperplanes in general position and let $U = X \setminus Y$. 
\end{enumerate}

\subsubsection{2.2 DO!!}

\subsubsection{2.5 DO!!}

\subsubsection{2.6 DO!!}


Let $X$ be a noetherian topological space and let $\{ \I_\alpha \}_{\alpha \in A}$ be a directed system of injective sheaves of abelian groups on $X$. 
\bigskip\\
I claim that $\I$ is injective if and only if for every open $U \subset X$ and subsheaf $\F \subset \Z_U$ and map $f : \F \to \I$ there exists an extension to $\Z_U \to \I$. Given this property consider an injection $\cA \embed \cB$ of sheaves and a map $f : \cA \to \I$. Then for every local section $s \in \cB(U)$ we take the map $\underline{\Z}_U \to \cB|_U$ such that,
\begin{center}
\begin{tikzcd}
\mathcal{R} \arrow[d] \arrow[dd, bend right] \arrow[r, hook] & \underline{\Z}_U \arrow[d] \arrow[ddl, bend left, dashed]
\\
\cA|_U \arrow[r, hook] \arrow[d] & \cB|_U
\\
\I|_U
\end{tikzcd}
\end{center}
where $\mathcal{R}$ is the preimage of $\cA$ under $\underline{Z}_U \to \cB|_U$. 


Clearly, if $\I$ is injective the above property holds.

\begin{lemma}
If $X$ is a noetherian space then every subsheaf of $\underline{\Z}$ is finite type.
\end{lemma}

\begin{proof}
Let $\F \subset \underline{\Z}$ be a subsheaf. For each $x \in X$ we see that $\F_x \subset \Z$ and thus $\F_x = (n_x)$ for some $n_x \in \Z$. Thus there exists some open $U_x$ containing $x$ such that $n_x \in \F(U_x)$. Now if $y \in U_x$ then $n_x \in \F_y$ so $n_y \divides n_x$. Because $\Z$ is noetherian, there is some $x_0 \in U_x$ such that $n_{x_0}$ is minimal and thus $n_y$ is constant for $y \in V_{x} = U_{x_0} \cap U_x$. Therefore, consider $\Z|_V \to \F|_V$ by sedning $1 \mapsto n_{x_0}$ which is an isomorphism on stalks and thus is an isomorphism. Therefore, inside any open $U$ there is a smaller (nonempty) open $V \subset U$ on which $\F|_V$ is finite type. Therefore, if $\F|_{X \setminus V}$ is finite type 
\end{proof}

\subsubsection{2.7 DO!!}

Let $X = S^1$ be the circle with its usual topology. Write $S^1 = U_1 \cup U_2$ for a pair of arcs such that $U_{12} = U_1 \cap U_2$ is the union of two contractible spaces. Consider the Godement resolution,
\begin{center}
\begin{tikzcd}
0 \arrow[r] & \Z_X \arrow[r] & \prod_{x \in S^1} \Z_x \arrow[r] & 0
\end{tikzcd}
\end{center}

\subsection{Section 3}

\subsubsection{3.1}

Let $X$ be a noetherian scheme. If $X = \Spec{A}$ is affine then $X_\red = \Spec{A_\red}$ is clearly affine. Conversely, suppose that $X_\red = \Spec{A}$ is affine. There is a closed immersion $X_\red \embed X$ which sheaf of ideals $\sN$ which is coherent since $X$ is noetherian. Therefore, since $\sN$ is the sheaf of nilpotents as an ideal $\sN^{n+1} = 0$ for some $n$ because locally $\sN |_{\Spec{B}} = \wt{\nilrad{B}}$ which is finitely generated because $B$ is Noetherian. Therefore, for any quasi-coherent sheaf $\F$ there is a filtration,
\[ \F \supset \sN \cdot \F \supset \sN^2 \cdot \F \supset \cdots \supset \sN^n \cdot \F \supset \sN^{n+1} \cdot \F = 0 \]
let $\F_i = \sN^i \cdot \F$ then $\G_i = \F_i / \F_{i+1}$ satisfies $\sN \cdot \G_i = 0$. Since $\iota : X_\red \to X$ is a closed immersion $\iota_*$ induces an equivalence of categories between quasi-coherent $\struct{X_\red}$-modules and quasi-coherent $\struct{X}$-modules killed by $\sN$. Thus $\G_i = \iota_* \G_i'$ where $\G_i'$ is a $\struct{X_\red}$-module. Then $H^q(X, \G_i) = H^q(X, \iota_* \G_i') = H^q(X_\red, \G_i') = 0$ for $q > 0$ because $\G_i'$ is a quasi-coherent $\struct{X_\red}$-module and $X_\red$ is affine. Clearly $H^q(X, \F_{n+1}) = 0$. Now assume that $H^q(X, \F_{i+1}) = 0$ for $q > 0$. Using the exact sequence,
\begin{center}
\begin{tikzcd}
0 \arrow[r] & \F_{i+1} \arrow[r] & \F_{i} \arrow[r] & \G_i \arrow[r] & 0
\end{tikzcd}
\end{center}
we apply cohomology to find,
\begin{center}
\begin{tikzcd}
H^q(X, \G_i) \arrow[r] & H^{q+1}(X, \F_{i+1}) \arrow[r] & H^{q+1}(X, \F_{i}) \arrow[r] & H^{q+1}(X, \G_i) 
\end{tikzcd}
\end{center}
and thus $H^{q+1}(X, \F_{i+1}) \iso H^{q+1}(X, \F_{i})$ is an isomorphism for $q > 0$ and $H^1(X, \F_{i+1}) \onto H^1(X, \F_i)$ is a surjection. Therefore, $H^q(X, \F_i) = 0$ for $q > 0$ because $H^q(X, \F_{i+1}) \onto H^q(X, \F_{i})$ and $H^q(X, \F_{i+1}) = 0$ for $q > 0$. Thus $X$ is affine by Serre's criterion.

\subsubsection{3.2}

Let $X$ be a reduced noetherian scheme. Suppose that $X = \Spec{A}$ is affine. Then the irreducible components of $X$ are $\Spec{A/\p_i}$ for the minimal primes $\p_i \subset A$ which are affine. 
\bigskip\\
Conversely, suppose that each irreducible component $Y \subset X$ is affine. Since $X$ is Noetherian there are finitely many irreducible components $Y_i \subset X$. For any coherent sheaf of ideals $\I$ which corresponds to some closed subscheme $Z \subset X$ we want to show that $H^1(X, \I) = 0$. To do so, we proceed by descending induction on the number of irreducible components of $X$ contained in the support of $Z$. If $Z$ contains every component then $\I = (0)$ because $X$ is reduced and thus $H^1(X, \I) = 0$. Now, let $Y$ be an irreducible component not contained in $Z$ and consider the exact sequence,
\begin{center}
\begin{tikzcd}
0 \arrow[r] & \I_{Z \cup Y} \arrow[r] & \I_Z \arrow[r] & (\iota_{Y})_* \I_{Z \cap Y} \arrow[r] & 0
\end{tikzcd}
\end{center}
Because $Y_1$ is affine, $H^1(X, (\iota_{Y})_* \I_{Z \cap Y}) = H^1(Y, \I_{Z \cap Y}) = 0$ and thus the long exact sequence gives a surjection $H^1(X, \I_{Z \cup Y}) \onto H^1(X, \I_Z)$. However, $Z \cup Y$ contains more irreducible components of $X$ than $Z$ since $Y \not\subset Z$ so by the induction hypothesis $H^1(X, \I_{Z \cup Y}) = 0$. Therefore $H^1(X, \I_Z) = 0$ proving the result by induction. Since $H^1(X, \I) = 0$ for every coherent sheaf of ideals $\I$, we conclude that $X$ is affine by Serre's criterion. 
\bigskip\\
Here I give an alternative proof. Because $X$ is Noetherian, there are finitely many irreducible components $Z_i$. We proceed by induction on the number of irreducible components so assume the theorem for $r$ components and let $X$ have irreducible components $Z_1, \dots, Z_{r+1}$. 
If there is only one irreducible component then because $X$ is reduced, $X = Z$ and thus the statement is trivial. Now proceed by induction. Take any coherent $\struct{X}$-module $\F$ and consider the exact sequence,
\begin{center}
\begin{tikzcd}
0 \arrow[r] & \I_{Z} \cdot \F \arrow[r] & \F \arrow[r] & \F / \I_{Z} \F \arrow[r] & 0
\end{tikzcd}
\end{center}
where $Z \subset X$ is an irreducible component. By Lemma \ref{support_component_sheaf_ideal}, $\Supp{\struct{X}}{\I_Z \otimes \F} \subset X' = Z_1 \cup \cdots \cup Z_r$ where $Z_1, \dots, Z_r \subset X$ are the irreducible components besides $Z$ so $X'$ has $r$ components and $\I_{Z} \cdot \F$ is the pushforward of a $\struct{X'}$-module $\F'$ (possibly with nonreduced structure). In particular, $X'$ has the same $Z_1, \dots, Z_r$ irreducible components as $X$ (except for $Z$) and thus each is affine. Likewise, $\G = \F / \I_Z \F$ is anhilated by $\I_Z$ and thus $\F / \I_Z \F = \iota_* \iota^* \G$. 
Then taking the cohomology sequence,
\begin{center}
\begin{tikzcd}
H^q(X', \F') \arrow[r] & H^q(X, \F) \arrow[r] & H^q(Z, \G) 
\end{tikzcd}
\end{center}
By assumption, $Z$ is ample and $X'$ has $r$ irreducible componets all of which are affine so (perhaps after reducing $X'$) $X'$ by the induction hypothesis $X'$ is affine. Since $\F'$ and $\G$ are coherent we get vanishing $H^q(X', \F') = 0$ and $H^q(Z, \G) = 0$ for all $q > 0$.
Therefore, the exact sequence gives that $H^q(X, \F \otimes \L^{\otimes n}) = 0$ for all $q > 0$ proving that $X$ is affine by Serre's criterion. Thus the result holds for any number of irreducible components by induction.

\begin{lemma} \label{support_component_sheaf_ideal}
Let $X$ be a reduced scheme with finitely many irreducible components $Z_1, \dots, Z_r$ corresponding to quasi-coherent sheaves of ideals $\I_{Z_i}$. Then,
\[ X \setminus Z_i \subset \Supp{\struct{X}}{\I_{Z_{i}}} \subset \bigcup_{j \neq i} Z_j \]
\end{lemma}

\begin{proof}
If $x \notin Z$ then we know that $(\I_Z)_x = \stalk{X}{x}$ because $(\struct{X}/\I_Z)_x = 0$ proving the first inclusion.Notice that $\I_{Z_1} \cdots \I_{Z_{r+1}} \subset \I_{X} = (0)$ because $X$ is reduced. Therefore, if $x \in X \setminus \bigcup_{j \neq i} Z_j$ then $(\I_{Z_j})_x = \stalk{X}{x}$ for each $j \neq i$ and thus we must have $(\I_{Z_i})_x = 0$ for the relation to hold proving the complement of the second inclusion.
\end{proof}

\subsubsection{3.3}

Let $A$ be a noetherian ring and $\a \subset A$ an ideal. Let $X = \Spec{A}$ and $Y = V(\a)$.  

\begin{enumerate}
\item We know $\Gamma_\a(M) = \Gamma_Y(X, \wt{M})$ from (II.5.6) and therefore since $\wt{-}$ is exact and $\Gamma_Y(X, -)$ is left exact this shows that $\Gamma_\a(-)$ is left exact. Explicitly, let $\varphi : M \to N$ be a morphism of $A$-modules then $m \in \ker{(\varphi : \Gamma_\a(M) \to \Gamma_\a(N))}$ iff $\varphi(m) = 0$ and $\a^n m = 0$ for some $n > 0$ iff $m \in \Gamma_\a(\ker{\varphi})$. We denote the right dertived functors of $\Gamma_\a(-)$ by $H_\a^i(-)$.

\item Because $\Gamma_\a(-) = \Gamma_Y(X, \wt{-})$ and $\wt{-}$ takes injective modules to flasque sheaves since $A$ is noetherian and thus $H_\a^i(-) = R^i \Gamma_\a(-) = R^i \Gamma_Y(X, -)(\wt{-}) = H_Y^i(X, \wt{-})$ where the last equality follows from (3.6) showing that cohomology of quasi-coherent modules on noetherian schemes is computed as the derived functors of $\Gamma_Y$ on the category of coherent sheaves. 
\bigskip\\
Alternatively, because $\wt{-}$ is exact, the functors $H^q_Y(X, \wt{-})$ form a $\delta$-functor on $\Mod{A}$. Furthermore, $\Mod{A}$ has enough injectives and $\wt{I}$ is flasque since $A$ is noetherian so $H^q_Y(X, \wt{I}) = 0$ and thus $H^q_Y(X, \wt{-})$ is effacable so they form a universal $\delta$-functor. Furthermore, since $H^0_Y(X, \wt{-}) = \Gamma_Y(X, \wt{-}) = \Gamma_\a(-)$ we get a natural isomorphism $H^q_Y(X, \wt{-}) = R^q \Gamma_\a(-) = H^q_\a(-)$.
\bigskip\\
Alternatively, we can show this explicitly by induction and dimension shifitng. Let $M$ be an $A$-module and $M \embed I$ an embedding into an injective $A$-module. Then we find an exact sequence,
\begin{center}
\begin{tikzcd}
0 \arrow[r] & M \arrow[r] & I \arrow[r] & K \arrow[r] & 0
\end{tikzcd}
\end{center}
The long exact sequence gives,
\begin{center}
\begin{tikzcd}
0 \arrow[r] & \Gamma_\a(M) \arrow[r] & \Gamma_\a(I) \arrow[r] & \Gamma_\a(K) \arrow[r] & H^1_\a(M) \arrow[r] & 0
\\
& H^q_\a(I) \arrow[r] & H^q_\a(K) \arrow[r] & H^{q+1}_\a(M) \arrow[r] & H^{q+1}_\a(I)
\end{tikzcd}
\end{center}
and thus $H^q_\a(K) \iso H^{q+1}_\a(M)$ for $q > 0$. Furthermore, applying the exact functor $\wt{-}$ we get an exact sequence,
\begin{center}
\begin{tikzcd}
0 \arrow[r] & \wt{M} \arrow[r] & \wt{I} \arrow[r] & \wt{K} \arrow[r] & 0
\end{tikzcd}
\end{center}
which gives a long exact sequence of cohomology with supports,
\begin{center}
\begin{tikzcd}
0 \arrow[r] & \Gamma_Y(X, \wt{M}) \arrow[r] & \Gamma_Y(X, \wt{I}) \arrow[r] & \Gamma_Y(X, \wt{K}) \arrow[r] & H^1_\a(M) \arrow[r] & 0
\\
& H^q_Y(X, \wt{I}) \arrow[r] & H^q_Y(X, \wt{K}) \arrow[r] & H^{q+1}_Y(X, \wt{M}) \arrow[r] & H^{q+1}_Y(X, \wt{I})
\end{tikzcd}
\end{center}
using that $\wt{I}$ is flasque so its higher cohomology vanishes we see $H^q_Y(X, \wt{K}) \iso H^{q+1}_Y(X, \wt{M})$ for $q > 0$. Since $\Gamma_Y(X, \wt{-}) = \Gamma_\a(-)$ the cokernel sequences imply that $H^1_\a(M) = H^1_Y(X, \wt{M})$ for any $M$ proving our base case.
Now we assume for induction that $H^q_\a(-) = H^q_Y(X, \wt{-})$ for $q > 0$. Then we see,
\[ H^{q+1}_\a(M) = H^q_\a(K) = H^q_Y(X, \wt{K}) = H^{q+1}_Y(X, \wt{M}) \]
proving that $H^q_\a(M) = H^q_Y(X, \wt{M})$ for all $q \ge 0$ and all $M$ by induction.

\item First consider the case $i = 0$. For any $A$-module $M$, if $m \in \Gamma_\a(M)$ then $\a^n m = 0$ for some $m > 0$ so $m \in \Gamma_\a(\Gamma_\a(M))$ and $\Gamma_\a(N) \subset N$ for any $N$ meaning that $\Gamma_\a(\Gamma_\a(M)) = \Gamma_\a(M)$. Now, note that if $M$ has the property that $\Gamma_\a(M) = M$ and $\varphi : M \onto N$ then $\Gamma_\a(N) = N$ because for any $x \in N$ we can lift to some $m \in M$ and $\a^n m = 0$ for some $n > 0$ and thus $\a^n x = \a^n \varphi(m)= \varphi(\a^n x) = 0$. Therefore $\Gamma_\a(N) = N$. Now we proceed by induction and dimension shifting. Embed $M \embed I$ into an injective $A$-module $I$ giving an exact sequence,
\begin{center}
\begin{tikzcd}
0 \arrow[r] & M \arrow[r] & I \arrow[r] & K \arrow[r] & 0
\end{tikzcd}
\end{center}
The long exact sequence gives for any $q \ge 0$,
\begin{center}
\begin{tikzcd}
H^q_\a(I) \arrow[r] & H^q_\a(K) \arrow[r] & H^{q+1}_\a(M) \arrow[r] & H^{q+1}_\a(I)
\end{tikzcd}
\end{center}
but $H^{q+1}_\a(I) = 0$ since $I$ is injective and thus $H^q_\a(K) \onto H^{q+1}_\a(M)$. Therefore, if $\Gamma_\a(H^q_\a(K)) = H^q_\a(K)$ for any $A$-module $K$ then we see that $\Gamma_\a(H^{q+1}_\a(M)) = H^{q+1}_\a(M)$ so by induction $\Gamma_\a(H^{q}_\a(M)) = H^q_\a(M)$ for any $q \ge 0$ and any $A$-module $M$.
\end{enumerate}

\subsubsection{3.4 DO!!}

Let $A$ be a noetherian ring, $\a \subset A$ an ideal, and $M$ an $A$-module. 

\begin{enumerate}
\item 
If $M$ has an $M$-regular sequence $x_1 \in \a$ of length $1$ meaning $M \xrightarrow{x_1} M$ is injective and $M / x_1 M \neq 0$. Suppose that $m \in \Gamma_\a(M)$ then $\a^n m = 0$ so in particular $x_1^n m = 0$ but $M \xrightarrow{x_1} M$ is injective and so $M \xrightarrow{x_1^n} M$ is also injective showing that $m = 0$ so $\Gamma_\a(M) = 0$. 
\bigskip\\
Now let $M$ be finitely generated and assume that there does not exist a $M$-regular sequence in $\a$ then $\a$ is contained in the set of zero divisors on $M$ which is the union of the finitely many associated primes of $M$ since $M$ is finitely generated. By prime avoidance, $\a$ is contained in some associated prime $\p = \Ann{A}{m}$ meaning that $\a m = 0$ so $m \in \Gamma_\a(M)$ is nonzero and thus $\Gamma_\a(M) \neq 0$.

\item Let $M$ be finitely generated. We want to show that for any $A$-module $M$ and $n \ge 0$ the following are equivalent,
\begin{enumerate}
\item there exists a $M$-regular sequence in $\a$ of length $n$
\item $H^i_\a(M) = 0$ for all $i < n$
\end{enumerate}
We have shown this for $n = 1$. Now assume the equivalence for $n$. First, suppose there is a length $n$ regular sequence $x_1, \dots, x_{n+1} \in \a$ then,
\begin{center}
\begin{tikzcd}
0 \arrow[r] & M \arrow[r, "x_1"] & M \arrow[r] & M / x_1 M \arrow[r] & 0
\end{tikzcd}
\end{center}
and $M / x_1 M$ has a regular sequence in $\a$ of length $n$. Applying the long exact sequence,
\begin{center}
\begin{tikzcd}
H^i_\a(M / x_1 M) \arrow[r] & H^{i+1}_\a(M) \arrow[r, "x_1"] & H^{i+1}_\a(M) \arrow[r] & H^{i+1}_\a(M / x_1 M)
\end{tikzcd}
\end{center}
By the induction hypothesis $H^i_\a(M/x_1M) = 0$ for $i < n$ so the map $H^{i + 1}_\a(M) \xrightarrow{x_1} H^{i + 1}_\a(M)$ is injective. However, $\Gamma_\a(H^{i+1}_\a(M)) = H^{i+1}_\a(M)$ so for any $m \in H^{i+1}_\a(M)$ there is a $k > 0$ such that $\a^k m = 0$ and thus $x_1^k \cdot m = 0$ so $m = 0$ by injectivity. Therefore $H^{i}_\a(M) = 0$ for any $i < n + 1$ proving the second condition by induction.
\bigskip\\
Now suppose that $H^i_\a(M) = 0$ for $i < n + 1$. Since $\Gamma_\a(M) = 0$ we know there exists an $M$-regular element $x_1 \in \a$ such that the sequence,
\begin{center}
\begin{tikzcd}
0 \arrow[r] & M \arrow[r, "x_1"] & M \arrow[r] & M / x_1 M \arrow[r] & 0
\end{tikzcd}
\end{center}
is exact. Applying the long exact sequence we get,
\begin{center}
\begin{tikzcd}
H^{i}_\a(M) \arrow[r, "x_1"] & H^{i}_\a(M) \arrow[r] & H^{i}_\a(M / x_1 M) \arrow[r] & H^{i+1}_\a(M)
\end{tikzcd}
\end{center}
By the hypothesis we see $H^i_\a(M) = 0$ and $H^{i+1}_\a(M) = 0$ for $i < n$ meaning that $H^i_\a(M/x_1 M) = 0$ for $i < n$ so by the induction hypothesis $M / x_1 M$ has a regular sequence $x_2, \dots, x_{n+1} \in \a$ of length $n$. Therefore, $x_1, \dots, x_n$ is an $M$-regular sequence in $\a$ of length $n+1$. 
\end{enumerate}
\noindent
Therefore we can define $\depth{\a}{M} = \min \{ n \in \Z \mid H^n_\a(M) \neq 0 \}$. Then every $M$-regular sequence in $\a$ may be extended to a maximal sequence and all such maximal sequences have length $n$.

\subsubsection{3.5 CHECK!!}

Let $X$ be a noetherian scheme and $x \in X$ a closed point. We want to show the following are equivalent:
\begin{enumerate}
\item $\depth{\m_x}{\stalk{X}{x}} \ge 2$
\item if $U$ is any open neighborhood of $x$ then $\Gamma(U, \struct{X}) \to \Gamma(U \setminus \{ x \}, \struct{X})$ is an isomorphism.
\end{enumerate}
Let $Y = \{ x \} \subset U$ is closed and let $U^\times = U \setminus Y$ the punctured neighborhood. Applying the excision sequence (III.2.3 (e)) for cohomology with supports,
\begin{center}
\begin{tikzcd}
0 \arrow[r] & H^0_Y(U, \struct{U}) \arrow[r] & H^0(U, \struct{U}) \arrow[r] & H^0(U^\times, \struct{U^\times}) \arrow[r] & H^1_Y(U, \struct{U})
\end{tikzcd}
\end{center}
so we need to show that $H^i_Y(U, \struct{U}) = 0$ for $i = 0, 1$ in order to show that $H^0(U, \struct{U}) \iso H^0(U^\times, \struct{U})$ is an isomorphism. Let $V = \Spec{A}$ be an affine open neighborhood of $x = \p \in \Spec{A}$ then $Y = V(\p)$.  Applying excision for cohomology with supports (III.2.3 (f)),
\[ H^i_Y(U, \struct{U}) \cong H^i_Y(V, \struct{V}) = \varinjlim_{x \in V} H^i_Y(V, \struct{V}) = \varinjlim_{f \in A \setminus \p} H^i_{\p}(A_f) = H^i_{\p}(A_\p) = H^i_{\m_x}(\stalk{X}{x}) \]
Therefore, if $\depth{\m_x}{\stalk{X}{x}} \ge 2$ then $H^i_Y(U, \struct{U}) = H^i_{\m_x}(\stalk{X}{x}) = 0$ for $i < 2$ proving the required statement.
\bigskip\\
Conversely suppose that $\Gamma(U, \struct{X}) \to \Gamma(U \setminus \{ x \}, \struct{X})$ is an isomorphism for any open neighborhood. In paricular, choose $U = \Spec{A}$ to be an affine open neighborhood of $x = \p \in \Spec{A}$. Applying the excision sequence (III.2.3 (e)) for cohomology with supports,
\begin{center}
\begin{tikzcd}
0 \arrow[r] & H^0_Y(U, \struct{U}) \arrow[r] & H^0(U, \struct{U}) \arrow[r] & H^0(U^\times, \struct{U^\times}) \arrow[r] & H^1_Y(U, \struct{U}) \arrow[r] & H^1(U, \struct{U})
\end{tikzcd}
\end{center}
but $H^0(U, \struct{U}) \to H^0(U^\times, \struct{U^\times})$ is an isomorphism and $U$ is affine so $H^1(U, \struct{U}) = 0$ and thus $H^i_Y(U, \struct{U}) = 0$ for $i = 0,1$. Applying excision for cohomology with supports (III.2.3 (f)),
\[ H^i_Y(U, \struct{U}) \cong \varinjlim_{x \in V} H^i_Y(V, \struct{V}) = \varinjlim_{f \in A \setminus \p} H^i_{\p}(A_f) = H^i_{\p}(A_\p) = H^i_{\m_x}(\stalk{X}{x}) \]
Therefore, $H^i_{\m_x}(\stalk{X}{x}) = H^i_Y(U, \struct{U}) = 0$ for $i < 2$ proving that $\depth{\m_x}{\stalk{X}{x}} = \depth{\p}{A_\p} \ge 2$


\subsubsection{3.6 CHECK!!}

Let $X$ be a noetherian scheme and choose a finite cover $U_i = \Spec{A_i}$ of noetherian affine opens.

\begin{enumerate}
\item Let $\F$ be a quasi-coherent $\struct{X}$-module. Then $\F|_{U_i} = \wt{M_i}$ for some $A_i$-module $M_i$. Embed $M_i \embed I_i$ where $I_i$ is an injective $A_i$-module. Let $j_i : U_i \embed X$ be the open inclusion and define,
\[ \G = \bigoplus_{i = 1}^n (j_i)_*(\wt{I}_i) \]
The natural map $\F \to \G$ is injective because for any $x \in X$ there is some $i$ such that $x \in U_i$ and $\F_x \to \G_x$ is $(M_i)_x \embed (I_i)_x$ in the $i$-component which is injective. Since $X$ is Noetherian $j$ is quasi-compact and quasi-separated ($U$ is retrocompact) so $f_*(\wt{I}_i)$ is quasi-coherent and the finite sum of quasi-coherent modules is quasi-coherent so $\G$ is quasi-coherent. 
\bigskip
\\
Furthermore, $(j_i)_*$ is right adjoint to $(j_i)^* = (j_i)^{-1}$ which is exact because $j_i$ is an open immersion. Therefore, $j_i$ preserves injective quasi-coherent modules. However, since $I_i$ is injective and there is an equivalence of categories between $A_i$-modules and quasi-coherent $\struct{U_i}$-modules we see that $\wt{I}_i$ is injective in the category of quasi-coherent $\struct{U_i}$-modules. Therefore, $f_*(\wt{I}_i)$ is injective in the category of quasi-coherent $\struct{X}$-modules. Furthermore, the direct sum of injectives is injective so $\G$ is injective in $\QCoh{X}$ proving that $\QCoh{X}$ has enough injectives. (CHECK!!)


Furthermore, let $\M \embed \sN$ be an injection of quasi-coherent $\struct{X}$-modules and suppose there is a map $\M \to \G$. Then locally $\M|_{U_i} = \wt{M_i}$ and $\sN|_{U_i} = \wt{N_i}$ and there is an injection $M_i \embed N_i$ 

\item Let $\I \in \QCoh{X}$ be injective and $U \subset X$ an open where $j : U \to X$ is the inclusion which is quasi-compact and quasi-separated since $X$ is noetherian. Let $\M, \sN \in \QCoh{U}$ be quasi-coherent $\struct{U}$-modules with an injection $\M \embed \sN$ and given a map $\M \to \I|_U$. Then $\iota_* \M \embed \iota_* \sN$ is injective and both are quasi-coherent $\struct{X}$-modules (since $U$ is retrocompact). By quotienting $\M \subset \sN$ by the kernel of $\M \to \I|_U$ we can reduce to the case that $\M \to \I|_U$ is injective. Now view $\M \subset \I|_U$ as a submodule. Then by (II.5.15) there exists a quasi-coherent $\struct{X}$-submodule $\M' \subset \I$ such that $\M|_U = \M$ and a quasi-coherent $\struct{X}$-module $\sN'$ such that $\M' \subset \sN'$ and $\sN'|_U = \sN$. Thus we have a diagram,
\begin{center}
\begin{tikzcd}
\M' \arrow[r, hook] \arrow[rd, hook] & \sN' \arrow[d, dashed]
\\
& \I 
\end{tikzcd}
\end{center}
restricting to $U$ we get a diagram,
\begin{center}
\begin{tikzcd}
\M \arrow[r, hook] \arrow[rd, hook] & \sN \arrow[d, dashed]
\\
& \I|_U
\end{tikzcd}
\end{center}
and therefore $\I|_U$ is injective. In particular, $\I|_{U_i} = \wt{I}_i$ where $I_i$ is a quasi-coherent $A_i$-module since $\I|_{U_i}$ is an injective quasi-coherent $\struct{U_i}$-module and the category of quasi-coherent $\struct{U_i}$-modules is equivalent to the category of $A_i$-modules. By (3.4) $\wt{I}_i$ is flasque.
\bigskip\\
To show that $\I$ is flasque, it suffices to show that $\res : \I(X) \to \I(U)$ is surjective. Consider the filtration,
\[ \tilde{U}_i = U \cup \bigcup_{j = 1}^i U_i \]
with $\tilde{U}_0 = U$ and $\tilde{U}_n = X$.
Take a section $s_0 \in \I(U) = \I(\tilde{U}_0)$. For induction, let $s_i \in \I(\tilde{U}_i)$ be a section over $\tilde{U}_i$ such that $s_i |_{U} = s_0$. Since $\I|_{U_{i+1}} = \wt{I}_{i+1}$ is flasque,
\[ \res : \I(U_{i+1}) \to \I(\tilde{U}_i \cap U_{i+1}) \]
is surjective and thus we can lift to $s_i' \in \I(U_{i+1})$ such that $s_i' |_{\tilde{U}_i \cap U_{i+1}} = s_i |_{\tilde{U} \cap U_{i+1}}$ therefore we can glue to get a section $s_{i+1} \in \I(\tilde{U}_{i+1})$ such that $s_{i+1}|_{\tilde{U}_i} = s_i$ and $s_{i+1}|_{U_{i+1}} = s_i'$ and $s_{i+1}|_U = s_{i}|_U = s_0$. Thus, by induction, we get a section $s_n \in \I(X)$ such that $s |_U = s_0$ so $\I$ is flasque.

\item Let $\iota : \QCoh{X} \embed \Sh{(X)}$ be the inclusion of categories from quasi-coherent $\struct{X}$-modules to abelian sheaves on $X$. Then there is a diagram of functors,
\begin{center}
\begin{tikzcd}
\QCoh{X} \arrow[dr, "\Gamma'"'] \arrow[rr, "\iota"] & & \Sh{(X)} \arrow[dl, "\Gamma"]
\\
& \Ab
\end{tikzcd}
\end{center}
Then since $\iota$ takes injectives to flasques which are $\Gamma$-acyclic, there is a Grothendieck spectral sequence $E^{p,q}_2 = R^p \Gamma \circ R^q \iota \implies R^{p+q} \Gamma'$ but $R^p \Gamma = H^p(X, -)$ and $R^0 \iota = \iota$ and $R^q \iota = 0$ for $q > 0$ because $\iota$ is exact. Therefore, $H^p(X, -) = R^p \Gamma'(X, -)$.
\bigskip\\
Alternatively, we compute the derived functors of $\Gamma'$ on $\QCoh{X}$ applied to $\F$ by taking an injective resolution in $\QCoh{X}$,
\begin{center}
\begin{tikzcd}
0 \arrow[r] & \F \arrow[r] & \I^0 \arrow[r] & \I^2 \arrow[r] & \cdots 
\end{tikzcd}
\end{center}
then applying $\iota$ gives a flasque resolution of $\iota(\F)$ in $\Sh{(X)}$ because $\iota$ is exact. Therefore,
\[ H^p(X, \iota(\F)) = H^p(\Gamma(X, \iota(\I^\bullet))) = H^p(\Gamma(X, \I^\bullet)) \]
so we can compute abelian sheaf cohomology of $\iota(\F)$ (i.e. of $\F$ viewed in $\Sh{(X)}$) via taking injective resolutions in $\QCoh{X}$. 
\end{enumerate}

\subsubsection{3.7 DO!!}

Let $A$ be a noetherian ring, $X = \Spec{A}$, $\a \subset A$ an ideal, and let $U \subset X$ be the open $X \setminus V(\a)$. 
\begin{enumerate}
\item Let $M$ be an $A$-module. Because $A$ is Noetherian, $\a = (f_1, \dots, f_r)$ is finitely generated. Consider the map $\varphi_n : \Hom{A}{\a^n}{M} \to \Gamma(U, \wt{M})$ sending $\psi : \a^n \to M$ to the section $s \in \Gamma(U, \wt{M})$ such that $s|_{D(f_i)} = \psi_{f_i}(1)$ where $\psi_{f_i} : \a^n_{f_i} \to M_{f_i}$ maps $1 = f_i^n/f_i^n$ to $\psi_{f_1}(1)$. Suppose that $\varphi_n(\phi) = 0$. Then $\psi_{f_i} = 0$ for each $i$

\item Let $I$ be an injective $A$-module. Then for any open $U \subset X$ the complement $X \setminus U$ is closed and thus $X \setminus U = V(\a)$ for some ideal $\a$. Then consider,
\[ \Gamma(U, \wt{I}) = \varinjlim_{n} \Hom{A}{\a^n}{I} \]
and the map $\Gamma(X, \wt{I}) \to \Gamma(U, \wt{I})$ is given by,
\[ I \to \varinjlim_{n} \Hom{A}{\a^n}{I} \]
defined by $\Hom{A}{A}{I} \to \Hom{A}{\a^n}{I}$ from $\a^n \embed A$. However, since $I$ is injective the map $I = \Hom{A}{A}{I} \to \Hom{A}{\a^n}{I}$ is surjective meaning that $\Gamma(X, \wt{I}) \to \Gamma(U, \wt{I})$ is surjective so $\wt{I}$ is flasque.
\end{enumerate}

\subsubsection{3.8}

Let $A = k[x_0, x_1, x_2, \dots ]$ with relations $x_0^n x_n = 0$ for each $n$. Now let $I$ be an injective $A$-module and $A \embed I$ an injective map. Consider the map $I \to I_{x_0}$. If we assume this is surjective then $\frac{1}{x_0}$ must have a preimage $m \in I$. Therefore, $m = \frac{1}{x_0}$ so there exists some $n$ such that $x_0^n(x_0 m - 1) = 0$ in $I$. Then $x_{n+1} x_0^n(x_0 m - 1) = 0$ but $x_{n+1} x_0^{n+1} = 0$ and therefore $x_{n+1} x_0^n = 0$ in $I$ contracting the fact that $A \embed I$ is injective. Therefore $I \to I_{x_0}$ cannot be surjective.

\subsection{4}

\subsubsection{4.8}

\newcommand{\cd}[1]{\mathrm{cd}\left( #1 \right)}

Let $X$ be a noetherian separated scheme. Define the cohomological dimension $\cd{X}$ of $X$ as the minimal integer $n$ such that $H^i(X, \F) = 0$ for all quasi-coherent sheaves $\F$ and all $i  > n$. 

\begin{enumerate}
\item To show we can replace quasi-coherent with coherent in the definition, it suffices to show that fixing $i$ if $H^i(X, \F) = 0$ for all coherent sheaves $\F$ then $H^i(X, \G) = 0$ for all quasi-coherent sheaves $\G$. However, by (Ex. II.5.15(e)) we can write any quasi-coherent sheaf $\G$ as a direct limit over coherent subsheaves,
\[ \G = \varinjlim \F_\alpha \]
and then by III.2.9 we have,
\[ H^q(X, \G) = H^q(X, \varinjlim \F_\alpha) = \varinjlim H^q(X, \F_\alpha) = 0 \]

\item Let $X$ be quasi-projective over a field $k$ so there is an ample line bundle $\L$ on $X$. Clearly for any finite locally free $\struct{X}$-module $\E$ we know $H^i(X, \E) = 0$ for all $i > \cd{X}$. Therefore, it suffices to assume $H^i(X, \E)$ for all finite locally free $\E$ and all $i > n$ and conclude that $n \ge \cd{X}$. We need to show that for each coherent sheaf $\F$ that $H^i(X, \F) = 0$ for $i > n$. We proceed by descending induction on $i$. For $i > \cd{X}$ this is obvious. Now assume for $i$ and use the ampleness of $\L$ to choose a surjection from a finite locally free module $\E$ which is a sum of twists of $\L$. Extending to an exact sequence,
\begin{center}
\begin{tikzcd}
0 \arrow[r] & \G \arrow[r] & \E \arrow[r] & \F \arrow[r] & 0
\end{tikzcd}
\end{center}
Therefore, we get a long exact sequence,
\begin{center}
\begin{tikzcd}
H^i(X, \E) \arrow[r] & H^i(X, \F) \arrow[r] & H^{i+1}(X, \G) \arrow[r] & H^{i+1}(X, \E)
\end{tikzcd}
\end{center}
For $i > n$ we have $H^i(X, \E) = H^{i+1}(X, \E) = 0$ and thus $H^i(X, \F) \iso H^{i+1}(X, \G)$ and by the induction hypothesis $H^{i+1}(X, \G) = 0$ so $H^i(X, \F) = 0$ and thus by induction $n \ge \cd{X}$.
 
\item Suppose that $X$ has a covering by $r+1$ affine open subsets $\U = \{ U_i \}$. On a Noetherian separated scheme, Cech cohomology on affine covers computes derived cohomology for quasi-coherent sheaves and thus,
\[ H^i(X, \F) = \check{H}^i(\U, \F) = H^i(\check{C}^\bullet(\U, \F)) \]
However, for $i > r$ we have $\check{C}^i(\U, \F) = 0$ because there are only $r+1$ values for the $i+1$ indices and repetition is not allowed. Therefore, for $i > r$ we find $H^i(X, \F) = 0$ for all quasi-coherent sheaves and thus $\cd{X} \le r$.

\item Let $X$ be quasi-projective over dimension $r$ over a field $k$. We need to show that $X$ has a cover by $\dim{X} + 1$ affine open subsets. Given this, by (c) we immediately see that $\cd{X} \le \dim{X}$. 
\bigskip\\
Now we prove the claim by induction on $r = \dim{X}$. We can take the projective closure of $X$ under an immersion $j : X \to \P^n$ to reduce to the case that $X$ is projective. This suffices because an affine open cover of $\overline{X}$ intersects to an affine open cover of $X$ because $\overline{X}$ is separated. First, projective schemes of dimension $0$ are affine since they are a finite discrete set of (possibly nonreduced) points and thus lie in the complement of a suitable hyperplane not passing through the finitely many points. Given a projective scheme $X \subset \P^n_k$ of dimension $r+1$ take a general hyperplane section $X \cap H \subset \P^{n-1}_k$ such that $\dim{X \cap H} = r$. Then by induction, $X \cap H$ can be covered by $r+1$ affine opens $U_0, \dots, U_{r}$ which are the complements of hyperplane sections in $H$. Thus, these extend to opens $U'_0, \dots, U'_r$ of $X$ which are the complements of hyperplane sections in $\P^n_k$ because we can always choose a hyperplane intersecting $H$ at a given hyperplane of $H$. These cover $X \cap H$ and $U_{r+1} = X \cap (\P^n \setminus H)$ is affine because $X \embed \P^n_k$ is affine and the complement of a hyperplane is affine. Thus $U_0', \dots, U_r', U_{r+1}$ is an affine open cover of $X$ proving the claim by induction.

\item Suppose that $Y$ is the set-theoretic intersection of hypersurfaces $H_1, \dots, H_r$ of codimension $r$ in $X = \P^n_k$. Then $U_i = X \setminus H_i$ are affine opens and because $Y = H_1 \cap \cdots \cap H_r$ set-theoretically we have $U_1 \cup \cdots \cup U_r = X \setminus Y$. Therefore, pulling back to $X \setminus Y$ the open cover $U_1, \dots, U_r$ is affine (because $X$ is separated) and therefore $\cd{X \setminus Y} \le r-1$.
\bigskip\\
Notice this argument works in the more general situation that $X$ is a quasi-projective scheme, $Y \subset X$ is a set-theoretic complete intersection $D_1 \cap \cdots \cap D_r$ for ample divisors $D_i \subset X$ then $\cd{X \setminus Y} \le r-1$. This is because $U_i = X \setminus D_i$ is an affine open and,
\[ U_1 \cup \cdots U_r = X \setminus (D_1 \cap \cdots D_r) = X \setminus Y \]
since $Y = D_1 \cap \cdots \cap D_r$ set-theoretically. Then $U_1, \dots, U_r$ forms an affine open cover of $X \setminus Y$ showing that $\cd{X \setminus Y} \le r-1$. 

\begin{rmk}
For a projective scheme $X$ the complement of an ample divisor $D$ is always affine. This is because we can find an embedding $X \embed \P^n$ such that $D = X \cap H$ set-theoretically and thus $X \setminus D = X \cap (\P^n \setminus H)$ is ample since $X \embed \P^n$ is affine. However, if $X$ is merely quasi-projective this may not be true because $j : X \embed \P^n$ may not be affine so the pullback $X \cap (\P^n \setminus H)$ need not be affine. This happens when the inclusion $j : X \embed \overline{X}$ into the projective closure is not an affine map. For example, let $X = \A^2 \setminus \{ (0,0) \}$. Then $\struct{X}$ is ample but the divisor $V(1 + x) = \A^2 \setminus \{ x = 1 \text{ or } (x,y) = (0,0) \}$ is not affine. This is because $j : X \to \overline{X} = \P^2$ is not affine. 
\end{rmk}
\end{enumerate}

\subsubsection{4.9}

Let $X = \Spec{k[x_1,x_2,x_3,x_4]}$ be affine four-space over $k$. Let $Y = Y_1 \cup Y_2$ where $Y_1 = V(x_1, x_2)$ and $Y_2 = V(x_3, x_4)$. If we suppose that $Y$ is a set theoretic complete intersection of dimension $2$ in $X$ then $\cd{X \setminus Y} \le 1$ by the extended version of Ex. III.4.8(e). Let $U = X \setminus Y$. To reach a contradiction we will show that $H^2(U, \struct{U}) \neq 0$.
\bigskip\\
Consider the cohomology with supports sequence,
\begin{center}
\begin{tikzcd}
H^2(X, \struct{X}) \arrow[r] & H^2(U, \struct{U}) \arrow[r] & H^3_Y(X, \struct{X}) \arrow[r] & H^3(X, \struct{X})
\end{tikzcd}
\end{center}
Since $H^q(X, \struct{X}) = 0$ for $q > 0$ there is an isomorphism $H^2(U, \struct{U}) \iso H^3_Y(X, \struct{X})$ so it suffices to show that $H^3_Y(X, \struct{X}) \neq 0$. Furthermore, by Mayer-Vietoris for cohomology with supports,
\begin{center}
\begin{tikzcd}
H^3_{Y_1}(X, \struct{X}) \oplus H^3_{Y_2}(X, \struct{X}) \arrow[r] & H^3_Y(X, \struct{X}) \arrow[r] & H^4_{Y_1 \cap Y_2}(X, \struct{X}) \arrow[r] & H^4_{Y_1}(X, \struct{X}) \oplus H^4_{Y_2}(X, \struct{X})
\end{tikzcd}
\end{center}
Furthermore, consider the cohomology with supports sequences,
\begin{center}
\begin{tikzcd}
H^q_{Y_i}(X, \struct{X}) \arrow[r] & H^q(X, \struct{X}) \arrow[r] & H^q(X \setminus Y_i, \struct{X}) \arrow[r] & H^{q+1}_{Y_i}(X, \struct{X}) \arrow[r] & H^{q+1}(X, \struct{X})
\end{tikzcd}
\end{center}
But $H^q(X, \struct{X}) = 0$ for $q > 0$ and $H^q(X \setminus Y_i, \struct{X}) = 0$ for $q > 1$ because $Y_i$ is the complete intersection of $V(x_1) \cap V(x_2)$ (or $V(x_3) \cap V(x_4)$) so $\cd{X \setminus Y_i} \le 1$. Therefore, $H^q_{Y_i}(X, \struct{X}) = 0$ for $q > 2$. Thus, returning to the Mayer-Vietoris sequence,
\begin{center}
\begin{tikzcd}
0 \arrow[r] & H^3_Y(X, \struct{X}) \arrow[r] & H^4_{Y_1 \cap Y_2}(X, \struct{X}) \arrow[r] & 0
\end{tikzcd}
\end{center}
gives an isomorphism $H^3_Y(X, \struct{X}) \iso H^4_{Y_1 \cap Y_2}(X, \struct{X})$ so it suffices to show that $H^4_{Y_1 \cap Y_2}(X, \struct{X}) \neq 0$. Applying the cohomology with supports in $P = Y_1 \cap Y_2$ sequence,
\begin{center}
\begin{tikzcd}
H^3(X, \struct{X}) \arrow[r] & H^3(X \setminus P, \struct{X}) \arrow[r] & H^4_P(X, \struct{X}) \arrow[r] & H^4(X, \struct{X}) 
\end{tikzcd}
\end{center}
using that $H^q(X, \struct{X}) = 0$ for $q > 0$ we get an isomorphism $H^3(X \setminus P, \struct{X}) \iso H^4_P(X, \struct{X})$ so, in total we have, 
\[ H^2(U, \struct{U}) \iso H^3_Y(X, \struct{X}) \iso H^4_{Y_1 \cap Y_2}(X, \struct{X}) \iso H^3(X \setminus P, \struct{X}) \]
and it suffices to show that $H^3(X \setminus P, \struct{X}) \neq 0$.
\bigskip\\
Now we take the cover $U_i = D(x_i)$ of $X \setminus P$ and consider the Cech complex begining in degree $3$,
\begin{center}
\begin{tikzcd}
\bigoplus\limits_{i = 1}^4 k[x_1,x_2,x_3,x_4]_{x_1 \cdots \hat{x_i} \cdots x_4} \arrow[r] & k[x_1^{\pm 1}, x_2^{\pm 1}, x_3^{\pm 1}, x_4^{\pm 1}] \arrow[r] & 0
\end{tikzcd}
\end{center}
where the map is the alternating sum. Notice that $x_1^{i_1} x_2^{i_2} x_3^{i_3} x_4^{i_4}$ cannot be in the image if all $i_j < 0$ because each term in the image comes from a ring with not every $x_i$ inverted. Therefore this is not surjective so $H^3(X \setminus P, \struct{X}) \neq 0$ proving that $H^2(U, \struct{U}) \neq 0$ so $Y$ cannot be a set-theoretic complete intersection. 

\subsubsection{4.10}

\subsection{5}

\subsubsection{5.2}

\begin{enumerate}
\item Let $X$ be a projective scheme over $k$ and $\struct{X}(1)$ be a very ample invertible sheaf on $X$ over $k$. Let $\F$ be a coherent $\struct{X}$-module. We will prove that $P(n) = \chi(\F(n))$ is a rational polynomial by induction on $\dim{\Supp{\struct{X}}{\F}}$. First, notice that under the embedding $\iota : X \embed \P^r_k$ associated to $\struct{X}(1)$ we have,
\begin{align*}
H^q(X, \F(n)) & = H^q(X, \F \otimes \struct{X}(n)) = H^q(X, \F \otimes \iota^* \struct{\P}(n)) = H^q(\P^r_k, \iota_* (\F \otimes \iota^* \struct{\P}(n)) 
\\
& = H^q(\P^r_k, \iota_* \F \otimes \struct{\P}(n)) = H^q(\P^r_k, (\iota_* \F)(n))
\end{align*}
using the projection formula and thus $\chi(X, \F(n)) = \chi(\P^r_k, \iota_* \F(n))$ and $\iota_* \F$ is a coherent sheaf on $\P^r_k$ with the same support (under the embedding $\iota : X \embed \P^r_k$). Thus we reduce to the case of coherent sheaves on $X = \P^r_k$.
\bigskip\\
Consider the base case $\dim{\Supp{\struct{X}}{\F}} = 0$ then the support is a discrete set of points and thus $\F(n) \cong \F$ so $\chi(\F(n))$ is a constant integrer and thus $P_\F \in \Q[z]$. 
\bigskip\\
Now proceed by induction. We want to choose a section $\ell \in \Gamma(\P^r_k, \struct{\P}(1))$ such that $\F(-1) \xrightarrow{\cdot \ell} \F$ is injective. To check that $\F(-1) \to \F$ is injective it suffices to on the stalks at the associated points $x \in \Ass{\struct{X}}{\F}$ of which there are finitely many (since $\F$ is coherent and $\P^r_k$ is Noetherian). Thus we may choose such an $\ell \in \Gamma(\P^r_k, \struct{\P}(1))$ by ensuring that $\ell_x \notin \m_x$ for $x \in \Ass{\struct{X}}{\F}$ then $\F_x \to \F_x$ via multiplication by $\ell_x$ is an isomorphism because $\stalk{X}{x}$ is local and $\F_x \to \F_x$ becomes an isomorphism after tensoring by $\kappa(x)$ since the image $\ell(x) \in \kappa(x)$ is nonzero. Therefore, we get an exact sequence,
\begin{center}
\begin{tikzcd}
0 \arrow[r] & \F(-1) \arrow[r] & \F \arrow[r] & \F \otimes \struct{H} \arrow[r] & 0
\end{tikzcd}
\end{center}
where $H = V(\ell)$ is a hyperplane and $\coker{(\F(-1) \to \F)} = \F \otimes \struct{H}$ via right exactness of $\F \otimes -$. Notice, if we only ensured that $\ell$ not vanish at the generic points of the componetns of $\Supp{\struct{X}}{\F}$ then $\F(-1) \to \F$ would have a nonzero kernel but one with strictly smaller dimensional support. Indeed, let $\G = \F \otimes \struct{H}$, then from the previous calculation, we see that $\G_x = 0$ for $x \in \Ass{\struct{X}}{\F}$ and $\Supp{\struct{X}}{\G} \subset \Supp{\struct{X}}{\F}$ so we must have, 
\[ \dim \Supp{\struct{X}}{\G} \le \Supp{\struct{X}}{\F} - 1 \]
In fact, we have equality because $s|_Z$ is a regular section of $\struct{Z}(1)$ where $Z = \Supp{\struct{X}}{\F}$ and thus $Z \cap H \subset Z$ is Cartier so the equality follows from Krull. Anyway, from the exact sequence twisted by $\struct{\P}(n)$,
\[ \chi(\F(n)) - \chi(\F(n-1)) = \chi(\G(n)) \]
However, by the induction hypothesis $P_\G(n) = \chi(\G(n))$ for a polynomial $P_\G \in \Q[z]$ and therefore since $P_\F(n) - P_\F(n-1) = P_\G(n)$ is a polynomial it implies that $P_\F \in \Q[z]$ proving the claim by induction.

\item Let $S = k[x_0, \dots, x_r]$. Recall that for a graded $S$-module $M$ we define the Hilbert function $\varphi_M(n) = \dim_k M_n$ and the Hilbert polynomial $P_M \in \Q[z]$ is the unique polynomial agreeing with $\varphi_M$ for $n \gg 1$. Now let $M = \Gamma_*(\F)$ so $M_n = H^0(\P^r_k, \F(n))$. For $n \gg 0$ we know that $\chi(\F(n)) = H^0(\P^r_k, \F(n))$ by vanishing of cohomology. Therefore $P_\F(n) = \varphi_M(n)$ for $n \gg 0$ and $P_\F \in \Q[z]$ proving that $P_\F = P_M$ by uniqueness. 
\end{enumerate}

\subsubsection{5.3}

Let $X$ be a projective scheme of dimension $r$ over a field $k$. The \textit{arithmetic genus} of $X$ is defined by,
\[ p_a(X) = (-1)^r \left( \chi(\struct{X}) - 1 \right) \]
Note that being projective is equivalent to being quasi-projective and proper so $\chi$ is defined for any coherent $\struct{X}$-module so, in particular, for $\struct{X}$ itself. 
 
\begin{enumerate}
\item Let $X$ be a projective integral scheme over an algebraically closed field $k$. By Lemma \ref{projective_scheme_proper} the scheme $X$ is proper over $k$ so by Lemma \ref{global_sections_proper_scheme}, $\struct{X}(X) = H^0(X, \struct{X})$ is a finite and thus algebraic extension of $k$. Since $k$ is algebraically closed, $\struct{X}(X) = k$
 and thus \[ \dim_k H^0(X, \struct{X}) = 1 \]
Therefore,
\begin{align*}
p_a(X) & = (-1)^{r + 1} + (-1)^r \sum_{i = 0}^r (-1)^i \dim_k H^i(X, \struct{X})
\\
& = (-1)^{r + 1} + (-1)^r + (-1)^r \sum_{i = 1}^r (-1)^i \dim_k H^i(X, \struct{X})
\\
& = \sum_{i = 1}^r (-1)^{i + r} \dim_k H^i(X, \struct{X}) = \sum_{i = 0}^{r-1} (-1)^i \dim_k H^{r - i}(X, \struct{X})
\end{align*}
In particular, when $X$ is a projective curve,
\[ p_a(X) = \dim_k H^1(X, \struct{X}) \]

\item In section I, we defined $p_a(Y) := (-1)^r (P_Y(0) - 1)$ where $P_Y$ is the Hilbert polynomial of the embedding $\iota : Y \embed \P^N_k$. However, in the previous exercise we showed that $P_Y(n)$ agrees with $\chi(\struct{Y}(n))$ where $\struct{Y}(n) = \iota^* \struct{\P^N_k}(n)$ and therefore $P_Y(0) = \chi(\struct{Y})$ so the two definitions agree.

\item We want to show that $p_a$ is a birational invariant for nonsingular projective curves over an algebraically closed field $k$. This is simply because each birational class of curves has a single nonsingular projective model (MAYBE GIVE A BETTER PROOF?).
\bigskip\\
In particular, a degree $3$ plane curve has $p_a(X) = 1$ and thus cannot be birational to $\P^1$.
\end{enumerate}

\subsubsection{5.4}

Let $X$ be a projective scheme over a field $k$ and let $\struct{X}(1)$ be a very ample line bundle on $X$. Consider the map,
\[ P : K(X) \to \Q[z] \]
sending the class of the coherent sheaf $\F$ to its Hilbert polynomial: $[\F] \mapsto P_\F$ where $P_\F(n) := \chi(\F(n))$ is the Hilbert polynomial. This is well-defined because given an exact sequence,
\begin{center}
\begin{tikzcd}
0 \arrow[r] & \F_1 \arrow[r] & \F_2 \arrow[r] & \F_3 \arrow[r] & 0
\end{tikzcd}
\end{center}
of coherent sheaves, then $[\F_2] = [\F_1] + [\F_3]$ but we also know $P_{F_2} = P_{\F_1} + P_{\F_2}$ and therefore $P([\F_2]) = P([\F_1] + [\F_3])$. Furthermore, this map is unique for the condition that $P([\F]) = P_\F$ since $K(X)$ is generated by these classes.
\bigskip\\
Now let $X = \P^r_k$ and let $L_i \subset \P^r_k$ be a linear space of dimension $i$ for each $i = 0,1, \dots, r$. Then notice,
\[ \chi(\struct{L_i}(n)) = {n + i \choose i} = \tfrac{1}{i!} (n + i)(n + i - 1) \cdots (n + 1) \]
We want to show that,
\begin{enumerate}
\item $K(X)$ is free abelian generated by $[\struct{L_i}]$ for $i = 0,1,\dots,r$
\item the map $P : K(X) \to \Q[z]$ is injective.
\end{enumerate}
First notice that (a) $\implies$ (b) because the polynomials $P_{L_i}$ are $\Q$-linearly independent. To show this, suppose that,
\[ \sum_{i = 0}^r a_i P_{L_i} = 0 \]
Since the leading order term $n^r$ only appears in $P_{L_r}$ so we must have $a_r = 0$ and thus,
\[ \sum_{i = 0}^{r-1} a_i P_{L_i} = 0 \]
reducing to the $r - 1$ case proving the linear independence by induction. 
\bigskip\\
Now we prove (a) and (b) by induction on $r$. The caes $r = 0$ is trivial because the Grothendieck group of finite $k$-modules is clearly free abelian on one generator $[k]$. Now for $X = \P^{r+1}_k$ consider a hyperplane $H \subset X$ so $H \cong \P^r_k$ and we may take $L_r = H$. In fact, we may take a flag on linear spaces,
\[ L_0 \subsetneq L_1 \subsetneq \cdots \subsetneq L_r = H \subsetneq L_{r+1} = X \] 
so that $\struct{L_i}$ have support contained in $H$.
Let $U = X \setminus H \cong \A^{r+1}_k$. Now by Exercise (II.6.10c) there is an exact sequence,
\begin{center}
\begin{tikzcd}
K(H) \arrow[r] & K(X) \arrow[r] & K(U) \arrow[r] & 0
\end{tikzcd}
\end{center}
Where the map $K(H) \to K(X)$ sends $[\F] \mapsto [\iota_* \F]$. Notice that $P_{\iota_* \F}(n) = \chi(X, \iota_* \F(n)) = \chi(H, \F(n)) = P_\F(n)$ because $\iota^* \struct{X}(1) = \struct{H}(1)$ and $H^q(X, \iota_* \F) = H^q(H, \F)$ and using the projection formula. Therefore, there is a commutative diagram,
\begin{center}
\begin{tikzcd}
K(H) \arrow[rd, "P"] \arrow[r, "\iota_*"] & K(X) \arrow[d, "P"]
\\
& \Q[z]
\end{tikzcd}
\end{center}
However, by the induction hypothesis, $P : K(H) \to \Q(z)$ is injective and therefore $K(H) \to K(X)$ is injective. Furthermore, $K(U) \cong \Z \cdot [\struct{U}]$ because $U \cong \A^{r+1}_k$ and thus every finite module has a finite free resolution by Hilbert's theorem on syzygies\footnote{The fact that $U$ is regular and affine is not enough as this only shows there is a finite locally free resolution but we need aditionally that on affine space finite projective modules are free.} and thus $K(U)$ is generated by $[\struct{U}]$. Since $\Z$ is projective, the sequence splits giving,
\[ K(X) = K(H) \oplus K(U) \]
Furthermore, because we assumed the linear spaces $L_i$ form a flag inside $H$ for $i \le r$ we see that $K(H)$ is a free abelian group generated by $[\struct{L_i}]$ for $i = 0,1, \dots, r$ by the induction hypothesis. Additionally, the coherent sheaves $\struct{L_i}$ have support inside $H$ and thus map to zero under $K(X) \to K(U)$ whereas $[\struct{L_{r+1}}] = [\struct{X}] \mapsto [\struct{U}]$ which is the generator and therefore we can choose a section $K(U) \to K(X)$ via $[\struct{U}] \to [\struct{X}]$. Thus, from the splitting $K(X) = K(H) \oplus K(U)$ we see that $K(X)$ is a free $\Z$-module generated by $[\struct{L_i}]$ for $i = 0,1, \dots, r, r+1$ proving (a) and thus also (b) for $r + 1$ and thus for all $r$ by induction.

\subsubsection{5.5}

Let $X = \P^r_k$ and $Y \subset X$ be a closed subscheme of dimension $q \ge 1$ which is a complete intersection. We want to prove the following,
\begin{enumerate}
\item for all $n \in \Z$ the natural map,
\[ H^0(X, \struct{X}(n)) \onto H^0(Y, \struct{Y}(n)) \]
is surjective

\item $Y$ is connected

\item $H^i(Y, \struct{Y}(n)) = 0$ for $0 < i < q$ and all $n \in \Z$

\item $p_a(Y) = \dim_k H^q(Y, \struct{Y})$
\end{enumerate}
First (a) $\implies$ (b) because $H^0(X, \struct{X}) \onto H^0(Y, \struct{Y})$ is thus one dimensional so $Y$ is connected. Furthermore, (a) and (c) $\implies$ (d) because $\dim_k H^0(Y, \struct{Y}) = 1$ and $H^i(Y, \struct{Y}) = 0$ for $0 < i < q$ and therefore,
\[ p_a(Y) = (-1)^{q} (\chi(\struct{Y}) - 1) = \sum_{i = 1}^q (-1)^{q - i} \dim_k H^i(Y, \struct{Y}) = \dim_k H^q(Y, \struct{Y}) \]
Thus it suffices to prove (a) and (c).
\bigskip\\
We proceed by descending induction on $q$. For $q = r$ we consider the case $Y = X$ for which (a) is obvious and we know $H^i(X, \struct{X}) = 0$ for $0 < i < r$. Now assume (a) and (c) for dimension $q + 1$. Let $Y$ be a complete intersection of dimension $q$ then $Y$ is the intersection of a hypersurface of degree $d$ and a complete intersection $W$ of dimension $q + 1$. Therefore, $Y \subset W$ is a closed subscheme cut out by a section of $\struct{W}(d)$ so there is an exact sequence,
\begin{center}
\begin{tikzcd}
0 \arrow[r] & \struct{W}(n-d) \arrow[r] & \struct{W}(n) \arrow[r] & \struct{Y}(n) \arrow[r] & 0
\end{tikzcd}
\end{center}
Therefore, we get an exact sequence,
\begin{center}
\begin{tikzcd}
H^0(W, \struct{W}(n)) \arrow[r] & H^0(Y, \struct{Y}(n)) \arrow[r] & H^1(W, \struct{W}(n-d))
\end{tikzcd}
\end{center}
However, by assumption (c) of the induction hypothesis $H^1(W, \struct{W}(n-d)) = 0$ because $1 < q + 1$ so $H^0(W, \struct{W}(n)) \onto H^0(Y, \struct{Y}(n))$ is surjective. By assumption (a), the map $H^0(X, \struct{X}(n)) \onto H^0(W, \struct{W}(n))$ is surjective and therefore,
\[ H^0(X, \struct{X}(n)) \onto H^0(W, \struct{W}(n)) \onto H^0(Y, \struct{Y}(n)) \]
is surjective. Furthermore, the long exact sequence contains,
\begin{center}
\begin{tikzcd}
H^i(W, \struct{W}(n)) \arrow[r] & H^i(Y, \struct{Y}(n)) \arrow[r] & H^{i+1}(W, \struct{W}(n-d))
\end{tikzcd}
\end{center}
By assumption (c), when $i > 0$ and $i+1 < q+1$ we know that $H^i(W, \struct{W}(n)) = H^{i+1}(W, \struct{W}(n)) = 0$ and therefore $H^i(Y, \struct{Y}(n)) = 0$ for $0 < i < q$. This proves (a) and (c) by induction for all complete intersections of dimension $q \ge 1$.  

\subsubsection{5.6 DO!!}

Let $Q$ be the nonsingular quadric surface $xy = zw$ in $X = \P^3_k$ over a field $k$. Since $\Pic{Q} = \Z \oplus \Z$ so effective Cartier divisors correspond to nonzero sections of $\struct{Q}(a,b)$ so divisors on $Q$ are bigraded in degree $(a,b)$.

\begin{enumerate}
\item

\item 

\item 

\item 
\end{enumerate}

\subsubsection{5.7 DO!!}

Let $X, Y, Z$ be proper schemes over a noetherian ring $A$ and $\L$ and invertible sheaf.

\begin{enumerate}
\item If $\L$ is ample on $X$ and $\iota Z \embed X$ is a closed embedding then consider $\iota^* \L$. For any coherent $\struct{Z}$-module $\F$ consider $\F \otimes \iota^* \L^{\otimes n}$. We know that,
\[ H^0(Z, \F \otimes \iota^* \L^{\otimes n}) = H^0(X, \iota_* (\F \otimes \iota^* L^{\otimes n})) \]
but by the projection formula,
\[ \iota_* (\F \otimes \iota^* L^{\otimes n}) = \iota_* \F \otimes \L^{\otimes n} \]
which is generated by global sections for $n \gg 0$ because $\iota_* \F$ is coherent and $\L$ is ample. Therefore, we get a surjection,
\[ \bigoplus_{i \in I} \struct{X} \onto \iota_* \F \otimes \L^{\otimes n} \]
and pulling back gives a surjection,
\[ \bigoplus_{i \in I} \struct{Z} \onto \F \otimes \iota^* \L^{\otimes n} \]
so $\F \otimes \iota^* \L^{\otimes n}$ is globally generated for $n \gg 0$ and thus $\iota^* \L$ is ample.

\item If $\L$ is ample on $X$ then $\L \otimes \struct{X_{\red}}$ is ample on $X_{\red}$ by (a) using the closed immersion $X_{\red} \embed X$. Conversely suppose that $\L \otimes \struct{X_{\red}}$ is ample on $X_{\red}$. To show that $\L$ is ample, it suffices to show that for each coherent sheaf $\F$ there exists a constant $n_\F$ such that for all $n \ge n_\F$ and $q > 0$ that $H^q(X, \F \otimes \L^{\otimes n}) = 0$. Consider the filtration,
\[ \F \supset \sN \cdot \F \supset \sN^2 \cdot \F \supset \cdots \supset \sN^n \cdot \F \supset \sN^{n+1} \cdot \F = 0 \]
let $\F_i = \sN^i \cdot \F$ then $\G_i = \F_i / \F_{i+1}$ satisfies $\sN \cdot \G_i = 0$. Since $\iota : X_\red \to X$ is a closed immersion $\iota_*$ induces an equivalence of categories between quasi-coherent $\struct{X_\red}$-modules and quasi-coherent $\struct{X}$-modules killed by $\sN$. Thus $\G_i = \iota_* \G_i'$ where $\G_i'$ is a $\struct{X_\red}$-module. The twisted exact sequence,
\begin{center}
\begin{tikzcd}
0 \arrow[r] & \F_{i+1} \otimes \L^{\otimes n} \arrow[r] & \F_i \otimes \L^{\otimes n} \arrow[r] & \G_i \otimes \L^{\otimes n} \arrow[r] & 0
\end{tikzcd}
\end{center}
gives an exact sequence,
\begin{center}
\begin{tikzcd}
H^q(X, \F_{i+1} \otimes \L^{\otimes n}) \arrow[r] & H^q(X, \F_i \otimes \L^{\otimes n}) \arrow[r] & H^q(X, \G_i \otimes \L^{\otimes n})
\end{tikzcd}
\end{center}
Using the projection formula, $\G_i \otimes \L^{\otimes n} = \iota_* \G_i' \otimes \L^n = \iota_* (\G_i' \otimes (\iota^* \L)^{\otimes n})$ and thus,
\[ H^q(X, \G_i \otimes \L^{\otimes n}) = H^q(X_{\red}, \G_i' \otimes (\L \otimes \struct{X_\red})^{\otimes n}) \]
which vanishes for $q > 0$ and $n \ge n_{\G_i'}$. Because $\F_{n+1} = 0$ vanishing holds for $i = n+1$. Thus we proceed by descending induction by assuming that $H^q(X, \F_{i+1} \otimes \L^{\otimes n}) = 0$ for $q > 0$ and $n \ge n_{i+1}$. Then if $n \ge n_i = \max\{(n_i, n_{\G_i})\}$ and $q > 0$ we see that $H^q(X, \F_i \otimes \L^{\otimes n})$ from the exact sequence. Thus, by induction, vanishing holds for $\F = \F_0$ and $n \ge n_0$ meaning that $\L$ is ample on $X$.

\item If $\L$ is ample on $X$ then any irreducible component $Z \embed X$ is included via a closed immersion and thus $\L|_Z$ is ample on $Z$. 
\bigskip\\
Conversely, suppose that $X$ is reduced and $\L|_Z$ is ample for each irreducible component $Z \subset X$. Because $X$ is Noetherian, there are finitely many irreducible components $Z_i$. We proceed by induction on the number of irreducible components so assume the theorem for $r$ components and let $X$ have irreducible components $Z_1, \dots, Z_{r+1}$. 
If there is only one irreducible component then because $X$ is reduced $X = Z$ and thus the statement is trivial. Now proceed by induction. Take any coherent $\struct{X}$-module $\F$ and consider the exact sequence,
\begin{center}
\begin{tikzcd}
0 \arrow[r] & \I_{Z} \cdot \F \arrow[r] & \F \arrow[r] & \F / \I_{Z} \F \arrow[r] & 0
\end{tikzcd}
\end{center}
where $Z \subset X$ is an irreducible component. By Lemma \ref{support_component_sheaf_ideal}, 
\[ \Supp{\struct{X}}{\I_Z \otimes \F} \subset X' = Z_1 \cup \cdots \cup Z_r \]
where $Z_1, \dots, Z_r \subset X$ are the irreducible components besides $Z$ so $X'$ has $r$ components and $\I_{Z} \cdot \F$ is the pushforward of a $\struct{X'}$-module $\F'$ (possibly with nonreduced structure but ampleness is preserved under reduction). Likewise, $\G = \F / \I_Z \F$ is anhilated by $\I_Z$ and thus $\F / \I_Z \F = \iota_* \iota^* \G$. Twisting by $\L^{\otimes n}$ and applying the projection formula gives an exact sequence,
\begin{center}
\begin{tikzcd}
0 \arrow[r] & j_* (\F' \otimes \L^{\otimes n}|_{X'}) \arrow[r] & \F \otimes \L^{\otimes n} \arrow[r] & \iota_* (\G \otimes \L^{\otimes n}|_Z) \arrow[r] & 0
\end{tikzcd}
\end{center}
Then taking the cohomology sequence,
\begin{center}
\begin{tikzcd}
H^q(X', \F' \otimes \L|_{X'}^{\otimes n}) \arrow[r] & H^q(X, \F \otimes \L^{\otimes n}) \arrow[r] & H^q(Z, \G \otimes \L|_Z^{\otimes n}) 
\end{tikzcd}
\end{center}
By assumption, $\L|_Z$ is ample and $\L|_{X'}$ is ample when restricted to the $r$ irreducible components of $X'$ so (perhaps after reducing $X'$) by the induction hypothesis $\L|_{X'}$ is ample. Since $\F'$ and $\G$ are coherent there exist integers $n_0'$ and $n_Z$ such that for all $q > 0$,
\[ n \ge n_0' \implies H^q(X', \F' \otimes \L|_{X'}^{\otimes n}) = 0 \quad \text{ and } \quad n \ge n_Z \implies H^q(Z, \G \otimes \L|_{Z}^{\otimes n}) = 0 \]
Therefore, for $n \ge n_0 = \max\{n_0', n_Z\}$ and $q > 0$ the exact sequence gives that $H^q(X, \F \otimes \L^{\otimes n}) = 0$ proving that $\L$ is ample on $X$. Thus the result holds for any number of irreducible components by induction.

\item First, let $f : X \to Y$ be a finite morphism and $\L$ ample on $Y$. Then I claim that $f^* \L$ is ample on $X$. Let $\F$ be any coherent $\struct{X}$-module then by the projection formula $f_* (\F \otimes f^* \L^{\otimes n}) = f_* \F \otimes \L^{\otimes n}$. Furthermore, $f$ is affine so $f_*$ preserves cohomology showing that,
\[ H^q(X, \F \otimes f^* \L^{\otimes n}) = H^q(Y, f_*(\F \otimes f^* \L^{\otimes n})) = H^q(Y, f_* \F \otimes \L^{\otimes n}) \]
Because $\F$ is coherent and $f : X \to Y$ is proper then $f_* \F$ is coherent so there exists an integer $n_{f_* \F}$ such that for all $n \ge n_{f_* \F}$ and $q > 0$ we have,
\[ H^q(X, \F \otimes f^* \L^{\otimes n}) = H^q(Y, f_* \F \otimes \L^{\otimes n}) = 0 \]
and therefore $f^* \L$ is ample on $X$. 
\bigskip\\
Now suppose that $f : X \to Y$ is finite and surjective and $f^* \L$ is ample. We now will show that $\L$ is ample by by Noetherian induction on $Y$. By (b) and (c) $\L$ is ample iff $\L|_{Y_{\red}}$ is ample iff $\L|_{Z}$ is ample for each irreducible component $Z \subset Y_\red$. Let $\cP$ be the property of closed subsets $Z \subset Y$ that $\L|_Z$ is ample. Then if $Y$ has $\cP$ meaning $\L|_{Y_\red}$ is ample then $\L$ is ample proving the claim. Thus, towards Noetherian induction, it suffices to show that if $Z \subset Y$ is a closed subset such that every proper closed subset $C \subsetneq Z$ has $\cP$ then $Z$ has $\cP$. Notice if $Z$ is reducible this is automatic because $\L|_Z$ is ample iff $\L|_Z$ restricted to irreducible component is ample by (c) thus we need only consider the case that $Z$ is irreducible.
\bigskip\\
Base changing by $Z \embed Y$ we get a finite surjective map $X_Z \to Z$ where $X_Z \embed X$ is a closed immersion so $(f^* \L)|_{X_Z}$ is ample. Since $X_Z \to Z$ is surjective, some $\xi \in X_Z$ must hit the generic point $\eta \in Z$. Give $W = \overline{\{ \xi \}}$ the reduced subscheme structure then composing with the closed immersion $W \embed X_Z$ gives a finite map $f' : W \to Z$ which is dominant because $\xi \mapsto \eta$ and thus surjective since $f' : W \to Z$ is closed. Since $(f')^* \L = (f^* \L)|_W$ is ample using the closed immerison $W \embed X$ and both $W$ and $Z$ are integral we have reduced to the integral case.
\bigskip\\
We will show that $\L|_Z$ is ample by using Serre's criterion. For any coherent $\struct{Z}$-module $\F$, by Ex. III.4.2(b) there is a coherent $\struct{W}$-module $\G$ and a morphism $\beta : f_* \G \to \F^{\oplus r}$ which is an isomorphism at the generic point $\eta \in Z$. Extend to an exact sequence,
\begin{center}
\begin{tikzcd}
0 \arrow[r] & \ker{\beta} \arrow[r] & f_* \G \arrow[r, "\beta"] & \F^{\oplus r} \arrow[r] & \coker{\beta} \arrow[r] & 0
\end{tikzcd}
\end{center} 
Taking the stalk at $\eta$ gives an exact sequence,
\begin{center}
\begin{tikzcd}
0 \arrow[r] & (\ker{\beta})_\eta \arrow[r] & (f_* \G)_\eta \arrow[r, "\beta"] & \F^{\oplus r}_\eta \arrow[r] & (\coker{\beta})_\eta \arrow[r] & 0
\end{tikzcd}
\end{center} 
but $\beta$ is an isomorphism at $\eta$ so $(\ker{\beta})_\eta = (\coker{\beta})_\eta = 0$ and thus their supports are proper closed subsets $C_1$ and $C_2$ of $Z$. In particular, $\ker{\beta}$ and $\coker{\beta}$ are extensions of coherent sheaves on $C_1$ and $C_2$ (with possibly nonreduced structure) but by the induction hypothesis $\L|_{(C_i)_\red}$ is ample and thus $\L|_{C_i}$ is ample. Since $\ker{\beta}$ and $\coker{\beta}$ are coherent there exists $n_0'$ such that for $n \ge n_0'$ and $q > 0$,
\[ H^q(X, \ker{\beta} \otimes \L^{\otimes n}) = H^q(X, \iota_* \iota^* \ker{\beta} \otimes \L|_{C_1}^{\otimes n}) = H^q(C_1, \iota^* \ker{\beta} \otimes \L|_{C_1}^{\otimes n}) = 0 \]
and likewise $H^q(X, \coker{\beta} \otimes \L^{\otimes n}) = 0$. Now split the exact sequence into short exact sequences,
\begin{center}
\begin{tikzcd}
0 \arrow[r] & \ker{\beta} \arrow[r] & f_* \G \arrow[r] & \sC \arrow[r] & 0
\\
0 \arrow[r] & \sC \arrow[r] & \F^{\oplus r} \arrow[r] & \coker{\beta} \arrow[r] & 0
\end{tikzcd}
\end{center}
and consider the long exact sequences after twisting,
\begin{center}
\begin{tikzcd}[column sep = small]
H^q(Z, \ker{\beta} \otimes \L^{\otimes n}) \arrow[r] & H^q(Z, f_* \G \otimes \L^{\otimes n}) \arrow[r] & H^q(Z, \sC \otimes \L^{\otimes n}) \arrow[r] & H^{q+1}(Z, \ker{\beta} \otimes \L^{\otimes n})
\\
H^q(Z, \sC \otimes \L^{\otimes n}) \arrow[r] & H^q(Z, \F \otimes \L^{\otimes n})^{\oplus r} \arrow[r] & H^q(Z, \coker{\beta} \otimes \L^{\otimes n}) \arrow[r] & H^{q+1}(Z, \sC \otimes \L^{\otimes n})
\end{tikzcd}
\end{center}
giving $H^q(Z, f_* \G \otimes \L^{\otimes n}) \iso H^q(Z, \sC \otimes \L^{\otimes n})$ and $H^q(Z, \sC \otimes \L^{\otimes n}) \onto H^q(Z, \F \otimes \L^{\otimes n})^{\oplus r}$ for $q > 0$ and $n \ge n_0'$  by the vanishing of cohomology for $\ker{\beta}$ and $\coker{\beta}$. Furthermore, using that $f$ is affine and the projection formula,
\[ H^q(Z, f_* \G \otimes \L^{\otimes n}) = H^q(Z, f_* (\G \otimes f^* \L^{\otimes n})) = H^q(W, \G \otimes f^* \L^{\otimes n}) \]
By assumption, $f^* \L$ is ample so because $\G$ is coherent there exists an integer $n_1$ such that for $n \ge n_1$ and $q > 0$ we have $H^q(Z, \G \otimes f^* \L^{\otimes n}) = 0$. Thus, the exact sequence shows that $H^q(Z, \F \otimes \L^{\otimes n}) = 0$ for $q > 0$ and $n \ge n_0 = \max\{n_0', n_1\}$ proving that $\L$ is affine by Serre's criterion and thus showing that $Z$ satisfies $\cP$.
\end{enumerate}

\subsubsection{5.8 DO!!}

We prove that one-dimensional proper schemes $X$ over an algebraically closed field $k$ are projective.

\begin{enumerate}
\item Let $X$ be irreducible and nonsingular. Then $X$ is a nonsingular complete curve over $k$ and thus projective by II.6.7. 

\item Let $X$ be integral and $\nu : \tilde{X} \to X$ be its normalization. 

\item

\item
\end{enumerate}

\subsubsection{5.9 DO!!}

\subsubsection{5.10}

Let $X$ be a projective scheme over a noetherian ring $A$. First, notice that if $\F \onto \G$ is a surjection of coherent sheaves then we may extend to an exact sequence,
\begin{center}
\begin{tikzcd}
0 \arrow[r] & \K \arrow[r] & \F \arrow[r] & \G \arrow[r] & 0
\end{tikzcd}
\end{center} 
Twisting by $\struct{X}(n)$ and taking the long exact sequence gives,
\begin{center}
\begin{tikzcd}
0 \arrow[r] & \Gamma(X, \K(n)) \arrow[r] & \Gamma(X, \F(n)) \arrow[r] & \Gamma(X, \G(n)) \arrow[r] & H^1(X, \K(n))
\end{tikzcd}
\end{center}
Since $\K$ is coherent, there exists a $n_\K$ such that for all $n \ge n_\K$ we have $H^1(X, \K(n)) = 0$ and thus $\Gamma(X, \F(n)) \onto \Gamma(X, \G(n))$ is surjective.
\bigskip\\
Now, we will prove the proposition by induction on $r$. The cases $r = 0,1,2$ are trivial. Now suppose the result holds for $r$ and let 
\begin{center}
\begin{tikzcd}
\F_1 \arrow[r] & \F_2 \arrow[r] & \cdots \arrow[r] & \F_r \arrow[r] & \F_{r+1}
\end{tikzcd}
\end{center}
be an exact sequence of coherent sheaves on $X$.
Then we can split this into sequences,
\begin{center}
\begin{tikzcd}
\F_1 \arrow[r] & \F_2 \arrow[r] & \cdots \arrow[r] & \F_{r-1} \arrow[r] & \K_r \arrow[r] & 0
\\
0 \arrow[r] & \K \arrow[r] & \F_r \arrow[r] & \sC \arrow[r] & 0
\end{tikzcd}
\end{center}
for subsheaves $\K \subset \F_r$ and $\sC \subset \F_{r+1}$. By the induction hypothesis there is an integer $n_1$ such that for all $n \ge n_1$ we have,
\begin{center}
\begin{tikzcd}
\Gamma(X, \F_1(n)) \arrow[r] & \Gamma(X, \F_2(n)) \arrow[r] & \cdots \arrow[r] & \Gamma(X, \F_{r-1}(n)) \arrow[r] & \Gamma(X, \K(n)) 
\end{tikzcd}
\end{center}
and from the long exact sequence of the twist of the second short exact sequence,
\begin{center}
\begin{tikzcd}
0 \arrow[r] & \Gamma(X, \K(n)) \arrow[r] & \Gamma(X, \F_r(n)) \arrow[r] & \Gamma(X, \sC(n)) \arrow[r] & H^1(X, \K(n))
\end{tikzcd}
\end{center}
and because $\K$ is coherent for $n \ge n_2$ we have $H^1(X, \K(n)) = 0$ and thus the sequence
\begin{center}
\begin{tikzcd}
0 \arrow[r] & \Gamma(X, \K(n)) \arrow[r] & \Gamma(X, \F_r(n)) \arrow[r] & \Gamma(X, \sC(n)) \arrow[r] & 0
\end{tikzcd}
\end{center}
is exact. 
Furthermore, for $n \ge n_3$ we know that $\Gamma(X, \F_{r-1}(n)) \onto \Gamma(X, \K(n))$ is surjecitve. Lastly, $\Gamma(X, \sC(n)) \embed \Gamma(X, \F_{r+1}(n))$ is injective because $\Gamma$ is right exact. Thus, for $n \ge n_0 = \max{(n_1, n_2, n_3)}$, we can patch these together to get a long exact sequence
\begin{center}
\begin{tikzcd}
\Gamma(X, \F_1(n)) \arrow[r] & \Gamma(X, \F_2(n)) \arrow[r] & \cdots \arrow[r] & \Gamma(X, \F_r(n)) \arrow[r] & \Gamma(X, \F_{r+1}(n))
\end{tikzcd}
\end{center}
proving the claim by induction.


\section{Appendix}

\subsection{A Intersection Theory}

\subsubsection{6.7}

\newcommand{\td}{\mathrm{td}}
\renewcommand{\ch}{\mathrm{ch}}

Let $X$ be a nonsingular projective $3$-fold with Chern classes $c_1, c_2, c_3$. Then we apply Grothendieck-Riemann-Roch,
\[ \ch(f_! \E) = f_* (\mathrm{ch}(\E) \cdot \td(\T_X)) \]
to the morphism $f : X \to \Spec{k}$. To give,
\[ \chi(\E) = \deg{(\ch(\L) \cdot \td(\T_X))_n} \]
where pushing forward onto a point selects the dimension zero (i.e. codimension $3$) part and takes degrees.
Thus it suffices to compute the Todd class,
\[ \td(\T_X) = 1 + \tfrac{1}{2} c_1(\T_X) + \tfrac{1}{12} (c_1(\T_X)^2 + c_2(\T_X)) + \tfrac{1}{24} c_1(\T_X) c_2(\T_X) \]
and by definition $c_i(\T_X) = c_i$. For a line bundle $\L$ with $c(\L) = 1 + D \in A^*(X)$ for some divisor $D$ we have,
\[ \ch(\L) = 1 + D + \tfrac{1}{2} D \cdot D + \tfrac{1}{6} D \cdot D \cdot D \]
and thus we find,
\begin{align*}
(\ch(\L) \cdot \td(\T_X))_n & = \tfrac{1}{24} c_1 c_2 + D \cdot \tfrac{1}{12} (c_1^2 + c_2) + \tfrac{1}{2} D^2 \cdot \tfrac{1}{2} c_1 + \tfrac{1}{6} D^3 
\\
& = \tfrac{1}{12} D \cdot (D + c_1) \cdot (2 D + c_1) + \tfrac{1}{12} D \cdot c_2 + \tfrac{1}{24} c_1 c_2
\end{align*}
For $D = 0$ we find, $\chi(\struct{X}) = \tfrac{1}{24} c_1 c_2$
and therefore $p_a(X) = 1 - \chi(\struct{X}) = 1 - \tfrac{1}{24} c_1 c_2$. Furthermore, $c_1 = -K_X$ and therefore,
\[ \chi(\L) = \tfrac{1}{12} D \cdot (D - K_X) \cdot (2D - K_X) + \tfrac{1}{12} D \cdot c_2 + 1 - p_a \]

\subsubsection{6.8}

Let $\E$ be a locally free sheaf of rank $2$ on $X = \P^3$. Hirzburch Riemann-Roch shows that,
\[ \chi(\E) = \deg{(\ch{(\E)} \cdot \td{(\T_X)})_n} \]
First notice,
\[ \ch{(\E)} = 2 + c_1(\E) + \tfrac{1}{2} (c_1(\E)^2 - 2 c_2(\E)) + \tfrac{1}{6} (c_1(\E)^3 - 3 c_1(\E) c_2(\E)) \]
Then we compute,
\begin{align*}
(\ch{(\E)} \cdot \td{(\T_X)})_n = \tfrac{2}{24} c_1 c_2 + c_1(\E) \cdot \tfrac{1}{12}(c_1^2 + c_2) + \tfrac{1}{2} (c_1(\E)^2 - 2 c_2(\E)) \cdot \tfrac{1}{2} c_1 + \tfrac{1}{6} (c_1(\E)^3 - 3 c_1(\E) c_2(\E))
\end{align*}
From the Euler sequence on $X = \P^n_k$,
\begin{center}
\begin{tikzcd}
0 \arrow[r] & \struct{X} \arrow[r] & \struct{X}(1)^{\oplus n+1} \arrow[r] & \T_X \arrow[r] & 0
\end{tikzcd}
\end{center}
we see that $c(\T_X) = (1 + c_1(\struct{X}))^{n+1} = (1 + H)^{n+1}$ where $H \in A^1(X)$ is the hyperplane class. For the case $n = 3$,
\[ c(\T_X) = 1 + 4 H + 6 H^2 + 4 H^3 \]
Therefore,
\[ (\ch{(\E)} \cdot \td{(\T_X)})_n = 2 H^3 + \tfrac{11}{6} c_1(\E) \cdot H^2 + (c_1(\E)^2 - 2 c_2(\E)) \cdot H + \tfrac{1}{6} (c_1(\E)^3 - 3 c_1(\E) c_2(\E)) \]
Now, because $A(X) = \Z[H]/(H^4)$ we must have $c_i(\E) = d_i H$ for integers $d_i$. Thus,
\[ (\ch{(\E)} \cdot \td{(\T_X)})_n = [2 + \tfrac{11}{6} d_1 + (d_1^2 - 2 d_2) + \tfrac{1}{6} (d_1^3 - 3 d_1 d_2)] H^3 \]
Therefore, since $\int_X H^3 = \deg{H^3} = 1$ we find,
\[ \chi(\E) = 2 + \tfrac{1}{6}(d_1^3 + 11 d_1) + d_1^2 - 2 d_2 - \tfrac{1}{2} d_1 d_2 \]
Notice that $n^3 \equiv n \mod 6$ and thus $d_1^3 + 11 d_1 \equiv d_1^3 - d_1 \equiv 0 \mod 6$ so $\tfrac{1}{6} (d_1^3 + 11 d_1)$ is an integer. Furthermore, $2 + d_1^2 - 2 d_2$ is obviously an integer. Since $\chi(\E)$ is an integer this implies that $d_1 d_2$ is divisible by $2$ that is $d_1 d_2 \equiv 0 \mod 2$.

\subsubsection{6.9 DO!!}

Let $\iota : X \embed \P^4_k$ be a smooth surface of degree $d$. Consider the normal sequence,
\begin{center}
\begin{tikzcd}
0 \arrow[r] & \T_X \arrow[r] & \iota^* \T_{\P^4} \arrow[r] & \sN_{X/\P^4} \arrow[r] & 0
\end{tikzcd}
\end{center}
Applying Chern classes we find that,
\[ c(\T_X) \cdot c(\sN_{X/\P^4}) = c(\iota^* \T_\P^4) \]
From the Euler sequence,
\[ c(\T_\P^4) = (1 + H)^5 \]
where $H$ is the hyperplane class. Therefore in $A^*(X)$,
\[ c(\iota^* \T_{\P^4}) = \iota^* c(\T_{\P^4}) = (1 + \iota^* H)^5  = 1 + 5 \iota^* H + 10 (\iota^* H)^2 \]
However, $(\iota^* H)^2 = \iota^* H^2$ is the class of $d$ points on $X$. Now expand,
\[ (1 + c_1 + c_2)(1 + c_1(\sN) + c_2(\sN)) = 1 + (c_1 + c_1(\sN)) + (c_1 c_1(\sN) + c_2 + c_2(\sN)) \]
Therefore, matching terms,
\begin{align*}
c_1 + c_1(\sN) & = 5 \iota^* H
\\
c_1 c_1(\sN) + c_2 + c_2(\sN) &= 10 (\iota^* H)^2
\end{align*}
and plugging in gives,
\[ c_2(\sN) + c_1 \cdot (5 \iota^* H - c_1) + c_2 = 10 (\iota^* H)^2 \]
Therefore, 
\[ c_2(\sN) = 10 (\iota^* H)^2 + c_1^2 - c_2 - 5 c_1 \cdot \iota^* H \]
Finally, taking degrees, and using $K_X = - c_1$ and $c_2 = - K_X^2 + 12(p_a(X) + 1)$ we find,
\[ \deg{(c_2(\sN)} = 10 d + 2  K_X^2 - 12(p_a(X) + 1) + 5 K_X \cdot \iota^* H \]
Finally, $X = d H^2$ in $A^*(\P^4_k)$ so we know that $\deg{X \cdot X} = d^2$ and furthermore we have $\iota_* c_2(\sN) = X \cdot X$ and thus $\deg{c_2(\sN)} = d^2$ giving a relation,
\[ 10 d - d^2 + 2  K_X^2 - 12(p_a(X) + 1) + 5 K_X \cdot \iota^* H = 0 \]

\begin{enumerate}
\item 

\item Let $X \subset \P^4_k$ be a K3 surface. Then by definition $K_X = 0$ and $h^1(X, \struct{X}) = 0$ so, using Serre duality $h^2(X, \struct{X}) = h^0(X, \struct{X})$ since $\omega_X = \struct{X}$, we find $p_a(X) = 1$. Therefore,
\[ 10 d - d^2 = 24 \]
meaning that $d^2 - 10 d + 24 = (d - 4)(d - 6) = 0$ and thus $d = 4$ or $d = 6$.

\item Let $X \subset \P^4_k$ be an abelian surface. Then $K_X = 0$ and $c_1 = c_2 = 0$ so $p_a = -1$. Therefore,
\[ 10 d - d^2 = 0 \]
which implies that $d = 10$.

\item 
\end{enumerate}

\subsubsection{6.10}

Suppose that $X$ is an abelian $3$-fold with an embedding $\iota : X \embed \P^5$. Then consider the normal sequence,
\begin{center}
\begin{tikzcd}
0 \arrow[r] & \T_X \arrow[r] & \iota^* \T_{\P^5} \arrow[r] & \sN_{X/\P^5} \arrow[r] & 0
\end{tikzcd}
\end{center}
Therefore,
\[ c(\T_X ) c(\sN_{X/\P^5}) = c(\iota^* \T_{\P^5}) \]
However, $\T_X$ is trivial so $c(\T_X) = 1$ and therefore,
\[ c(\sN_{X/\P^5}) = \iota^* c(\T_{\P^5}) \]
From the Euler sequence,
\[ c(\T_{\P^5}) = (1 + H)^6 \]
In particular we find,
\[ c_3(\sN_{X/\P^5}) = {6 \choose 3 } \iota^* H^3 = 20 \iota^* H^3 \]
which is nonzero because $\iota_* c_3(\sN_{X/\P^5}) = 20 \iota_* \iota^* H^3 = 20 X \cdot H^3 = 20 d H^5$ where $d$ is the degree of $X$ in $\P^5$ and thus $\deg{c_3(\sN_{X/\P^5})} = 20 d$. However, $\sN_{X/\P^5}$ is a vector bundle of rank $\codim{X, \P^5} = 2$ and must have $c_3(\sN_{X/\P^5}) = 0$ leading to a contradiction. Thus $\T_X$ cannot be trivial so $X$ cannot be an abelian surface.

\subsection{B Transcendental Methods}

\newcommand{\h}{\mathfrak{h}}
\renewcommand{\C}{\mathbb{C}}

\subsubsection{6.1}


Consider the open unit disk $D^\circ \subset \C$. Let $X$ be a scheme of finite type over $\C$ such that $X_h \cong D^\circ$. Thus we must have $\dim{X} = 1$ and $\pi_1^{\et}(X) = 0$. Therefore, because curves of positive genus always admit \etale covers, we must have $X \cong \A^1$ or $X \cong \P^1$ (open subschemes of $\A^1$ involve removing finitely many points and thus are not simply connected). Clearly $\P^1$ cannot work because it is compact. Therefore we must have $X \cong \A^1$ in which case $X_h \cong \C$. However, I claim that $D^\circ$ is not biholomorphic to $\C$. To see this, notice that $D^\circ$ is biholomorphic to $\h = \{ z \in \C \mid \Im{z} > 0 \}$ via the map,
\[ z \mapsto i \cdot \frac{1 + z}{1 - z} \] 
Furthermore, there are no nonconstant maps $f : \C \to \h$ because then $\exp{(if)} : \C \to \C$ is bounded because $|e^{if}| = e^{- \Im{f}} \le 1$ and therefore constant by Liouville's theorem. Thus we cannot have a biholomorphic map $f : D^\circ \to \C$ showing that no such $X$ exists. 

\subsubsection{6.2}

Let $z_1, z_2, \dots \in \C$ be an infinite sequence with $|z_n| \to \infty$ as $n \to \infty$. Let $\I \subset \struct{\C}$ be the sheaf of ideals of holomorphic functions vanishing at all $z_n$. First we need to show that $\I$ is nonzero. Using the hypothesis that $|z_n| \to \infty$, the Weierstrass factorization theorem (or equivalently the solvability of the second cousins problem on a complex manifold with $\Pic{X} = 0$ using that the points $z_i$ are isolated and thus taking $f_i = z - z_i$ on a small disk about $z_i$) implies that there exists an entire function $f$ with a simple pole at each $z_i$. Thus $f \in \Gamma(\C, \I)$ so $\I \neq 0$. In particular, $V(\I) = \{ z_i \mid i \in \N \}$ is an infinite set and $V(\I) \neq \C$.
\bigskip\\
Now let $X = \A^1_\C$. Coherent sheaves of ideals $\J \subset \struct{X}$ correspond to Zariski closed subsets $Z \subset \A^1_\C$ which are finite (unless $\J = 0$) and thus $\J_h$ cannot correspond to $\I$ as sheaves of ideals because $\I$ cuts out an infinite subset. Explicilty, $\J = \wt{(p)}$ for some $p \in \C[z]$ because $\C[z]$ is a PID and $f$ has finitely many roots. Then $\J_h = (p) \cdot \struct{\C}$ which cannot equal $\I$ because $p \in \J_h$ viewed as a holomorphic function which has finitely many roots but every section of $\I$ (of which at least one exits) vanishes at all $z_i$ of which there are infinitely many. 
\bigskip\\
However, any section $s \in \Gamma(X, \I)$ is an entire function vanishing at the $z_i$ and thus $\frac{s}{f}$ is entire. Therefore $\I = (f) \cdot \struct{\C}$ which implies that $\I \cong \struct{\C} = (\struct{X})_h$ as coherent sheaves.

\begin{rmk}
To apply sovability of the second cousins problem, we need that the set of points $\{ z_i \}$ is discrete. Here we show that $\{ z_i \}$ being discrete is the same as $|z_n| \to \infty$. First, if $|z_n| \to \infty$ is it clear that $\{ z_i \}$ is discrete since all but finitely many have $|z_i| > M$ for each $M$ so $\{ z_i \} \cap D_M$ is finite and thus discrete because $\C$ is Hausdorff. Conversely, if $\{ z_i \}$ is discrete, then for each compact $\overline{D_M}$ we have $\{ z_i \} \cap \overline{D_M}$ is compact and discrete and thus finite. Therefore $|z_i| > M$ for all but finitely many $z_i$ for each $M > 0$ meaning that there is some $n_M$ such that $n \ge n_M \implies |z_n| > M$ implying that $|z_n| \to \infty$. 
\end{rmk}

\subsubsection{6.3 DO!!}


\subsubsection{6.4 DO!!}

\subsubsection{6.5 DO!!}

\subsubsection{6.6 DO!!}


\subsection{C Weil Conjectures}

\end{document}