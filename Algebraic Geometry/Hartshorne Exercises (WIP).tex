\documentclass[12pt]{article}
\usepackage{hyperref}
\hypersetup{
    colorlinks=true,
    linkcolor=blue,
    filecolor=magenta,      
    urlcolor=cyan,
}
 
\urlstyle{same}
\usepackage{import}
\import{./}{AlgGeoCommands}

\begin{document}

\tableofcontents

\newpage

\section{II Schemes}


\subsection{7}

\subsubsection{7.1}

Let $(X, \struct{X})$ be a locally ringed space and $f : \L \to \I$ a surjective map of invertible sheaves on $X$. Then for each $x \in X$ the map $f_x : \L_x \to \I_x$ is a surjective map of free rank one $\stalk{X}{x}$-modules. Then we have get a diagram of $\stalk{X}{x}$-module morphisms,
\begin{center}
\begin{tikzcd}
\L_x \arrow[r, "f_x", two heads] \arrow[d, "\sim"] & \I_x \arrow[d, "\sim"]
\\
\stalk{X}{x} \arrow[r, dashed, two heads] & \stalk{X}{x}
\end{tikzcd}
\end{center}
Therefore, it suffices to prove that if a $\stalk{X}{x}$-module map $\stalk{X}{x} \to \stalk{X}{x}$ is surjective then it is injective. Such a map satisfies $f(a) = a \cdot f(1)$ and since $f$ is surjective we must have $a \cdot f(1) = 1$ for some $a$ and therefore $f(1)$ is a unit so $f(a) = a \cdot f(1)$ is injective.
\bigskip\\
The fact that $\stalk{X}{x}$ is local in not necessary. In fact, we have the following more general fact.

\begin{theorem}
Let $(X, \struct{X})$ be a ringed space and $f : \F \to \K$ be a surjective map of finite locally free $\struct{X}$-modules of equal rank then $f : \F \to \K$ is an isomorphism. 
\end{theorem}

\begin{proof}
For each $x \in X$ the map $f_x : \F_x \to \K_x$ is surjective and since $\F$ and $\K$ are finite locally free both of rank $n$ we have a diagram,
\begin{center}
\begin{tikzcd}
\F_x \arrow[r, "f", two heads] \arrow[d, "\sim"] & \K_x \arrow[d, "\sim"]
\\
\stalk{X}{x}^{\oplus n} \arrow[r, dashed] & \stalk{X}{x}^{\oplus n}
\end{tikzcd}
\end{center}
Now since $\stalk{X}{x}^{\oplus n}$ as a finitely generated $\stalk{X}{x}$-module,  we use that any surjective endomorphism of a finitely generated module (Noetherian is not necessary) is injective to conclude that $f_x : \F_x \to \K_x$ is injective and thus $f : \F \to \K$ is an isomorphism. 
\end{proof}

\subsubsection{7.2}

Let $X$ be a scheme over a field $k$ and $\L$ an invertible sheaf on $X$. Let $\{ s_0, \dots, s_n \}$ and $\{ t_0, \dots, t_m \}$ be two sets of sections of $\L$ which generate the same subscpace $V \subset \Gamma(X, \L)$ and which generate the sjeaf $\L$ at every point. Suppose WLOG $n \le m$. Let $\varphi : X \to \P^n_k$ and $\psi : X \to \P^m_k$ be the corresponding morphisms. There is a matrix $a_{ij} \in k$ such that $t_j = s_i a_{ij}$. Then $a_{ij}$ defines a rational map $\P^m_k \rat \P^n_k$ which is projection from the linear subspace $L$ defined by $\ker{(a_{ij})}$ composed with an automorphism $\P^n_k \to \P^n_k$. This is because any surjective linear transformation $T : V \onto W$ can be factored as $T : V \onto V/\ker{T} \iso W$. Finally, becasue $t_j = s_i a_{ij}$ the following diagram commutes,
\begin{center}
\begin{tikzcd}
X \arrow[r, "\psi"] \arrow[dr, "\varphi"'] & \P^m_k \arrow[d, dashed]
\\
& \P^n_k
\end{tikzcd}
\end{center}

\subsubsection{7.3 DO!!}

Let $\varphi : \P^n_k \to \P^m_k$ be a morphism. Then $\varphi^* \struct{\P^m_k}(1) = \struct{\P^m_k}(n)$ for $n = \deg{\varphi}$ which is a nonegative integer. We put explicit coordinates on these spaces: $\P^n_k = \Proj{k[x_0, \dots, x_n]}$ and $\P^m_k = \Proj{k[y_0, \dots, y_m]}$. Suppose that $n = 0$ then $\varphi^* y_i = a_i \in k$ and thus $\varphi(\P^n_k) = [a_0 : \cdots : a_m]$ is a single point. Otherwise $n > 0$ and the image is not a single point (else clearly $\varphi^* \struct{\P^m_k}(1) = \struct{\P^n_k}$ since every line bundle over a point is trivial). 
\bigskip\\
Now if $n > 0$ we require that $\varphi^* y_i$ globally generate $\struct{\P^n_k}(n)$ 

(FINISH THIS!!!)

\subsubsection{7.4}

\begin{enumerate}
\item Let $X$ be finite type over a Noetherian ring $A$. Suppose $\L$ is an ample line bundle on $X$. Then we know that for some $n > 0$ the line bundle $\L^{\otimes n}$ is very ample. Therefore, there must be an immersion $i : X \to \P^r_A$ for some $r > 0$. Now the immersion $i$ can be factored as,
\begin{center}
\begin{tikzcd}
X \arrow[r, "j", hook] & U \arrow[r, "q", hook] & \P^r_A
\end{tikzcd}
\end{center} 
where $j : X \to U$ is a closed immersion and $q : U \to \P^r_A$ is an open immersion. Since $\P^r_A$ is a separated scheme over $A$ then $U \embed \P^r_A$ must be separated over $A$. Now we apply the following lemma. 

\begin{remark}
Hartshorne is wrong about the definition of an immersion which he defines to be a morphsism giving a homeomorphism onto an open subscheme of a closed subscheme and thus an open immersion followed by a closed immersion. This is backwards, an immersion is a closed immersion followed by an open immersion. The two definitions are incompatible (see Tag 01QW) and Hartshorne's definition is not even stable under composition so it is a bad notion.  
\end{remark}

\begin{lemma}
Let $f : X \to Y$ be a closed immersion of schemes over $S$. If $Y$ is separated then $X$ is separated.
\end{lemma}

\begin{proof}
Consider the compositions,
\begin{center}
\begin{tikzcd}
X \arrow[rd, "f", bend right] \arrow[r, "\Delta_X"] & X \times_S X \arrow[r, "f \times f"] & Y \times_S Y
\\
& Y \arrow[ru, "\Delta_Y", bend right]
\end{tikzcd}
\end{center}
Since $\Delta_Y$ and $f$ are both closed immersions, by separatedness of $Y$ and hypothesis respectily, the composition $\Delta_Y \circ f = (f \times f) \circ \Delta_X$ is a closed immersion. Furthermore, $f \times f : X \times_S X \to Y \times_S Y$ is a closed immersion which implies that $\Delta_X : X \to X \times_S X$ must also be a closed immersion since it must have surjective sheaf map if the composition does and since $f \times f$ and $(f \times f) \circ \Delta_X$ are homeomorphisms onto a closed image then for any closed $Z \subset X$ the image $\Delta_X(Z)$ must be closed because it maps to a closed set under $f \times f$ which is a homeomorphism onto its closed immage.
\bigskip\\
A better proof is to show that closed immersions are separated and composition of separated morphisms are separated so $X \to Y \to S$ gives $X \to S$ separated and thus $X$ is separated as a scheme over $S$.
\end{proof}

\item Consider the $X$ the affine line with a doubled point and let $U$ and $V$ the the open affine copies of $\A^1_k$ and $W$ the single glued open $U \setminus \{ P_1 \} = V \setminus \{ P_2 \}$. Since $X$ is not separated we cannot apply Weil divisors. However, $X$ is integral so the map $\CaCl{X} \to \Pic{X}$ is an isomorphism so we need only consider Cariter divisors. 
\bigskip\\
First we compute Cartier divisors on $\A^1_k$. We know that $k[x]$ is a UFD and thus its class group is trivial so $\CaCl{\A^1_k} = 0$. Now any Cartier diviosr $D \in \Gamma(X, \K^\times_X / \struct{X}^\times)$ its restriction to $U$ and $V$ must be principal i.e. rational functions $f, g \in K(\A^1_k) = k(x)$. Furthermore, on the overlap, we must have $f/g \in \struct{X}^\times(W)$ but $\struct{X}^\times(W) = k[x]_{(x)}$ then $f / g = a x^n$ for $a \in k^\times$ and $n \in \Z$. Then quotienting by principal divisors $k(x)$ we can set $g = 1$ and quoting by units on $U$ we can set $a = 1$ so we find $f = x^n$. This gives $\CaCl{X} = \Z$. 
\bigskip\\
Now we construct the line bundles defined by the Cartier divisor $D_n$ which is defined by $D|_U = x^n$ and $D|_V = 1$. Then, $\L_n = \L(D_n)$ satisfies $\L_n|_U = x^{-n} \struct{X}|_U$ and $\L_n|_V = \struct{X}|_V$.     
Now the diagram,
\begin{center}
\begin{tikzcd}
& \L_n(X) \arrow[ld] \arrow[rd]
\\
\L_n(U) \arrow[dr] & & \L_n(V) \arrow[dl]
\\
& \L_n(W)
\end{tikzcd}
\end{center}
is cartesian. So we have,
\begin{center}
\begin{tikzcd}
& \L_n(X) \arrow[ld] \arrow[rd]
\\
x^{-n} k[x] \arrow[dr] & & k[x] \arrow[dl]
\\
& k[x]_{(x)}
\end{tikzcd}
\end{center}
However, in $k[x]_{(x)}$ the intersection $x^{-n} k[x] \cap k[x] = k[x]_{\ge -n}$ meaning terms of at least degree $-n$. However, since $f$ and $g$ satisfy $f/g = x^n$ and therefore for effective divisors corresponding to the vanishing of sections $s \in \L_n$ we have $f,g \in k[x]$. Therefore, if $n > 0$ we see that $f$ vanishes at $x = 0$ and if $n < 0$ we see that $g$ vanishes at $x = 0$. Therefore, if $n > 0$ then $\L_n$ has base locus containing the first origin and if $n < 0$ the base locus of $\L_n$ contains the second origin. Therefore, the only line bundle on $X$ which is generated by global sections is $\L_0 = \struct{X}$. Therefore, there cannot be any ample line bundles on $X$ since any line bundle is of the form $\L_m$ but $\L_m^{\otimes n} = \L_{mn}$ is not globally generated $m \neq 0$ and furthermore, 
\[ \F \otimes_{\struct{X}} \L_0^{\otimes n} = \F \otimes_{\struct{X}} \struct{X}^{\otimes n} = \F \]
is not generated by global sections for $\F = \L_1$ so $\L_0 = \struct{X}$ is not ample. 
\end{enumerate}

\subsubsection{7.5}


Let $X$ be a Noetherian scheme and $\L$ and $\M$ be line bundles.

\begin{enumerate}
\item Suppose that $\L$ is ample and $\M$ is generated by global sections. For any coherent $\struct{X}$-module $\F$ there is some $n(\F)$ such that for all $n \ge n(\F)$ we have,
\[ \F \otimes_{\struct{X}} \L^{\otimes n} \]
is generated by global sections. By $\M$ is generated by global sections and thus so is $\M^{\otimes n}$ so,
\[  \F \otimes_{\struct{X}} (\L^{\otimes n} \otimes_{\struct{X}} \M^{\otimes n}) \]
is generated by global sections. Since this holds for any $n \ge n(\F)$ the sheaf $\L \otimes_{\struct{X}} \M$ is ample. (See Tag 01AO)

\item Let $\M$ be a line bundle. Since $\L$ is ample for sufficiently large $n$ the sheaf $\L^n \otimes_{\struct{X}} \M$ is generated by global sections and thus by the previous part,
\[ \L \otimes_{\struct{X}} (\L^{\otimes n} \otimes_{\struct{X}} \M) = \L^{\otimes n+1} \otimes_{\struct{X}} \M \]
is ample.
\item Now let $\L$ and $\M$ be ample. There must exist $n$ such that,
\[ \M \otimes_{\struct{X}} \L^{\otimes n} \]
is generated by global sections and thus, since $\M$ is ample, by (a) we have,
\[ (\M \otimes_{\struct{X}} \L^{\otimes n}) \otimes_{\struct{X}} \M^{\otimes n - 1} =  \L^{\otimes n} \otimes_{\struct{X}} \M^{\otimes n} = (\L \otimes_{\struct{X}} \M)^{\otimes n} \]
is ample. This implies that $\L \otimes_{\struct{X}} \M$ is ample. 

\item Now let $X$ be finite type over a noetherian ring $A$. Suppose that $\L$ is very ample and $\M$ is generated by global sections. Since $\L$ is very ample there must be an immersion $i : X \to \P^n_A$ such that $\L = i^* \struct{\P^n_A}(1)$. Furthermore a choice of sections generating $\M$ defines a map $j : X \to \P_A^m$ such that $\M = j^* \struct{\P^m_A}(1)$. Now consider the product under the Segre embedding,
\begin{center}
\begin{tikzcd}
X \arrow[r, "\Delta"] & X \times_A X \arrow[r, "i \times j"] & \P^n_A \times_A \P^m_A \arrow[r] & \P^{N}_A
\end{tikzcd}
\end{center} 
Thus it suffices to prove that $q : X \to \P^N_A$ is an immersion and,
\[ q^* \struct{\P^N_A}(1) = \L \otimes_{\struct{X}} \M \]
which implies that $\L \otimes_{\struct{X}} \M$ is very ample. 
\bigskip\\
Under the Segre embedding $s : \P_A^n \times_A \P^m_A \to \P_A^N$ we have,
\[ s^* \struct{\P^N_A}(1) = p_1^* \struct{\P^n_A}(1) \otimes_{\struct{}} p_2^* \struct{\P^m_A}(1) \]
Now, consider,
\begin{align*}
(i, j)^* s^* \struct{\P^N_A}(1) & = (i, j)^* [p_1^* \struct{\P^n_A}(1) \otimes_{\struct{}} p_2^* \struct{\P^m_A}(1)] 
\\
& = [(i, j)^* p_1^* \struct{\P^n_A}(1)] \otimes_{\struct{X}} [(i, j)^*  p_2^* \struct{\P^m_A}(1)] 
\\
& = [p_1 \circ (i, j)]^* \struct{\P^n_A}(1)] \otimes_{\struct{X}} [p_2 \circ (i, j)]^* \struct{\P^m_A}(1)
\\
& = i^* \struct{\P^n_A}(1) \otimes_{\struct{X}} j^* \struct{\P^m_A}(1)
\\
& = \L \otimes_{\struct{X}} \M 
\end{align*}
Now we need to show that $s \circ (i \times j) \circ \Delta$ is an embedding. Using the lemma, $(i, j) = (i \times j) \circ \Delta$ is an embedding since $i : X \to \P^n_A$ is an embedding. Furthermore, $s : \P^n_A \times_A \P^m_A \to \P^N_A$ is an embedding so $s \circ (i, j) \circ \Delta$ is an embedding. 

\item Let $X$ be finite type over a noetherian ring $A$ and $\L$ an ample sheaf on $X$. We know there exists some $n_0 > 0$ such that $\L^{\otimes n_0}$ is very ample and, as a consequence, generated by global sections. Furthermore, for $n \ge n_1$ we know $\L^{\otimes n}$ is generated by global sections since $\L$ is ample. Now, for any $n \ge n_0 + n_1$ the sheaf,
\[ \L^{\otimes n} = \L^{\otimes n_0} \otimes_{\struct{X}} \L^{\otimes (n - n_0)} \] 
is ample by the previous result since $\L^{\otimes n_0}$ is ample and $\L^{\otimes (n - n_0)}$ is generated by global sections because $n - n_0 \ge n_1$.   
\end{enumerate}

\begin{lemma}
Tensor product of sheaves commutes with pullback.
\end{lemma}

\begin{proof}
Let $f : X \to Y$ be a morphism and $\F$ and $\G$ be $\struct{Y}$-modules on $Y$ and $\K$ a $\struct{X}$-module on $X$ then,
\begin{align*}
\Hom{\struct{X}}{f^* \F \otimes_{\struct{X}} f^* \G}{\K} & = \Hom{\struct{X}}{f^* \F}{\shHom{\struct{X}}{f^* \G}{\K}}
\\
& = \Hom{\struct{Y}}{\F}{f_* \shHom{\struct{X}}{f^* \G}{\K}}
\\
& = \Hom{\struct{Y}}{\F}{\shHom{\struct{X}}{\G}{f_* \K}}
\\
& = \Hom{\struct{Y}}{\F \otimes_{\struct{Y}} \G}{f_* \K}
\\
& = \Hom{\struct{X}}{f^* (\F \otimes_{\struct{Y}} \G)}{\K}
\end{align*}
Therefore, by Yoneda, $f^* \F \otimes_{\struct{X}} f^* \G \cong f^* (\F \otimes_{\struct{Y}} \G)$. 
\end{proof}

\begin{lemma} \label{product_of_immersion}
Let $f : X \to Y$ be an immersion and $g : X \to Z$ is any morphism all over $S$ then $X \to Y \times_S Z$ is an immersion. 
\end{lemma}

\begin{proof}
The map $X \to Y \times_S Z$ can be factored into the graph morphism,
\[ \Gamma_g = (\id_X, g) : X \to X \times_S Z \]
and the product $f \times \id_Z : X \times_S Z \to Y \times_S Z$,
\begin{center}
\begin{tikzcd}
X \arrow[r, "\Gamma_g"] & X \times_S Z \arrow[r, "f \times \id"] & Y \times_S Y
\end{tikzcd}
\end{center}
It suffices to show that both maps are immersions. Since $f$ and $\id_Z$ are immersion then $f \times \id_Z : X \times_S Z \to Y \times_S Z$ is an immersion. Furthermore, the morphism $\Gamma_g$ can be obtained via a base extension of $\Delta : Z \to Z \times_S Z$ along the map $X \times_S Z \xrightarrow{g \times \id_Z} Z \times_S Z$ since,
\begin{center}
\begin{tikzcd}
X \arrow[r, "\Gamma_g"] \arrow[d, "g"] & X \times_S Z \arrow[d, "g \times \id_Z"]
\\
Z \arrow[r, "\Delta"] & Z \times_S Z
\end{tikzcd}
\end{center}
is cartesian because $(X \times_S Z) \times_{Z \times_S Z} Z = X$. (BE EXPLICIT) Since immersions are stable under base change, the morphism $\Gamma_g : X \to X \times_S Z$ is an immersion. Thus $(f, g) =  (f \times \id) \circ \Gamma_g$ is a composition of immersions and thus an immersion. 
\end{proof}

\begin{corollary}
If with the above data $f : X \to Y$ is a closed immersion and $Z$ is separated then $X \to Y \times_S Z$ is a closed immersion.
\end{corollary}

\begin{proof}
The above proof holds equally for closed immersions since they are stable under products and base extensions and composition. However, the map $\Delta : Z \to Z \times_S Z$ must be a closed immersion for the base extension to be a closed immersion so we must assume $Z$ is separated. 
\end{proof}

\subsubsection{7.6}

Let $X$ be a nonsingular projective varitety over an algebraically closed field $k$ (I do not think nonsingular or algebraically closed are necessary). Let $D$ be a divisor and $\L$ the associated line bundle. We consider $f(n) = \dim |n D| = \dim_k \Gamma(X, \L^n) - 1$.

\begin{enumerate}
\item Let $\L$ be very ample and $\iota : X \embed \P^n_k$ the corresponding embedding into projective space. Then $\L = \iota^* \struct{\P^n_k}(1)$ and therefore, by the projection formula,
\[ \iota_* \L^n = \iota_* \struct{X}(n) \]
Therefore, $\dim_k \Gamma(X, \L^n) = \dim_k \Gamma(\P^n_k, \iota_* \struct{X}(n)) = P_X(n)$ for sufficiently large $n$ where $P_X(n)$ is the Hilbert polynomial. Therefore, for $n \gg 0$,
\[ f(n) = P_X(n) - 1 \]

\item Let $D$ be torsion in $\Pic{X}$ of order $r$. Then we know $r D = 0$ or equivalently, $\L^{\otimes r} \cong \struct{X}$. Therefore, it is clear that,
\[ \dim |rkD| = \dim_k \Gamma(X, (\L^{\otimes r})^{\otimes k}) - 1 = \Gamma(X, \struct{X}) - 1 = 0 \]
and therefore $\dim |n D| = 0$ if $r \divides n$. Otherwise, for $n \neq rk$ I claim that $\Gamma(X, \L^{\otimes n}) = 0$. Indeed, suppose we have a global section $s : \struct{X} \to \L^{\otimes n}$ then $s^{\otimes r} : \struct{X} \to \L^{\otimes rn}$ must be constant because $\L^{\otimes rn} \cong \struct{X}$ i.e. $\div{(s^{\otimes r})} = 0$. Therefore, locally, $s^r_x \in k \subset \stalk{X}{x}$ so suppose that $s^r_x = 0$ then because $X$ is integral $s_x = 0$ and thus $s = 0$ because global sections $\struct{X} \to \struct{X}$ are constant. Otherwise, $s$ must be nonvanishing i.e. $s_x \in \kappa(x)$ is nonzero, because $s^r_x \in \kappa(x)$ is nonzero. Thus, $s_x : \stalk{X}{x} \to \stalk{X}{x}$ is an endomorphism such that $s_x \otimes \id : \kappa(x) \to \kappa(x)$ is an isomorphism so $s_x$ is an isomorphism showing that $s : \struct{X} \to \L^{\otimes n}$ is an isomorphism contradicting the fact that $n$ is not divisible by the order of $\L$. Therefore, if $r \ndivides n$ then $\Gamma(X, \L^{\otimes r}) = 0$ and thus $f(n) = \dim{|nD|} = -1$. 
\end{enumerate}


\subsubsection{7.7 DO!}
This does not work in characteristic equal to $2$.
\bigskip\\
Let $X = \P^2_k = \Proj{k[x,y,z]}$ and let $|D|$ be the complete linear system of all divisors of degree $2$ on $X$. Then $D$ corresponds to the invertible sheaf $\struct{X}(2)$ whose space of global sections has a basis,
\[ x^2, y^2, z^2, xy, xz, yz \]

\begin{enumerate}
\item To show that $|D|$ defined a closed embedding $\P^2 \to \P^5$ we can work locally on the affine charts of $\P^5 = \Proj{k[T_0, \dots, T_4, T_5]}$. We have the maps,
\begin{align*}
k[\tfrac{T_1}{T_0}, \dots, \tfrac{T_5}{T_0}] \to (k[x,y,z])_{(x^2)} = k[\tfrac{y}{x}, \tfrac{z}{x}] \quad \text{via} \quad \tfrac{T_1}{T_0} \mapsto \tfrac{y^2}{x^2} \quad \tfrac{T_2}{T_0} \mapsto \tfrac{z^2}{x^2} \quad \tfrac{T_3}{T_0} \mapsto \tfrac{xy}{x^2} \quad \tfrac{T_4}{T_0} \mapsto \tfrac{xz}{x^2} \quad \tfrac{T_5}{T_0} \mapsto \tfrac{yz}{x^2}
\end{align*}
which is clearly surjective because $\tfrac{T_3}{T_0} \mapsto \tfrac{y}{x}$ and $\tfrac{T_4}{T_0} \mapsto \tfrac{z}{x}$. The other charts are similar proving that this is a closed immersion.

\item Now we consider the subsystem $\mathfrak{d}$ spanned by,
\[ x^2, y^2, z^2, y (z - x), (x - y) z \]
To show that $\mathfrak{d}$ defined a closed embedding $\P^2 \to \P^4$ we can work locally on the affine charts of $\P^4 = \Proj{k[T_0, \dots, T_4]}$. We have the maps,
\begin{align*}
k[\tfrac{T_1}{T_0}, \dots, \tfrac{T_4}{T_0}] & \to (k[x,y,z])_{(x^2)} = k[\tfrac{y}{x}, \tfrac{z}{x}] \quad \text{via} \quad \tfrac{T_1}{T_0} \mapsto \tfrac{y^2}{x^2} \quad \tfrac{T_2}{T_0} \mapsto \tfrac{z^2}{x^2} \quad \tfrac{T_3}{T_0} \mapsto \tfrac{y (z - x)}{x^2} \quad \tfrac{T_4}{T_0} \mapsto \tfrac{(x - y)z}{x^2} 
\\
k[\tfrac{T_0}{T_1}, \dots, \tfrac{T_4}{T_1}] & \to (k[x,y,z])_{(y^2)} = k[\tfrac{x}{y}, \tfrac{z}{y}] \quad \text{via} \quad \tfrac{T_0}{T_1} \mapsto \tfrac{x^2}{y^2} \quad \tfrac{T_2}{T_1} \mapsto \tfrac{z^2}{y^2} \quad \tfrac{T_3}{T_1} \mapsto \tfrac{y (z - x)}{y^2} \quad \tfrac{T_4}{T_1} \mapsto \tfrac{(x - y)z}{y^2} 
\\
k[\tfrac{T_0}{T_2}, \dots, \tfrac{T_4}{T_2}] & \to (k[x,y,z])_{(z^2)} = k[\tfrac{x}{z}, \tfrac{y}{z}] \quad \text{via} \quad \tfrac{T_0}{T_2} \mapsto \tfrac{x^2}{z^2} \quad \tfrac{T_1}{T_2} \mapsto \tfrac{y^2}{z^2} \quad \tfrac{T_3}{T_2} \mapsto \tfrac{y (z - x)}{z^2} \quad \tfrac{T_4}{T_2} \mapsto \tfrac{(x - y)z}{z^2}  
\\
k[\tfrac{T_0}{T_3}, \dots, \tfrac{T_4}{T_3}] & \to (k[x,y,z])_{(y(z-x))}  \quad \text{via} \quad \tfrac{T_0}{T_3} \mapsto \tfrac{x^2}{y(z-x)} \quad \tfrac{T_1}{T_3} \mapsto \tfrac{y^2}{y(z-x)} \quad \tfrac{T_2}{T_3} \mapsto \tfrac{z^2}{y(z-x)} \quad \tfrac{T_4}{T_3} \mapsto \tfrac{(x - y)z}{y(z - x)} 
\\
k[\tfrac{T_0}{T_4}, \dots, \tfrac{T_3}{T_4}] & \to (k[x,y,z])_{((x-y)z)} \quad \text{via} \quad \tfrac{T_0}{T_4} \mapsto \tfrac{x^2}{(x-y)z} \quad \tfrac{T_1}{T_4} \mapsto \tfrac{y^2}{(x-y)z} \quad \tfrac{T_2}{T_4} \mapsto \tfrac{z^2}{(x-y)z} \quad \tfrac{T_3}{T_4} \mapsto \tfrac{y(z - x)}{(x - y)z} 
\end{align*}
The first is surjective because,
\[ \tfrac{T_3}{T_0} + \tfrac{T_4}{T_0} \mapsto \tfrac{z - y}{x} \quad \text{and} \quad \tfrac{T_4}{T_0} - \tfrac{1}{2} \left[ \left( \tfrac{T_3}{T_0} + \tfrac{T_4}{T_0} \right)^2 - \tfrac{T_1}{T_0} - \tfrac{T_2}{T_0} \right] \mapsto \tfrac{z}{x} \]
The second is surjective because,
\[ \tfrac{T_3}{T_1} \mapsto \tfrac{z-x}{y} \quad \text{and} \quad - \tfrac{T_4}{T_0} - \tfrac{1}{2} \left[ \left( \tfrac{T_3}{T_1} \right)^2 - \tfrac{T_0}{T_1} - \tfrac{T_2}{T_1} \right] \mapsto \tfrac{z}{y} \]
The third is surjective because,
\[ \tfrac{T_4}{T_2} \mapsto \tfrac{x-y}{z} \quad \text{and} \quad \tfrac{T_3}{T_2} - \tfrac{1}{2} \left[ \left( \tfrac{T_4}{T_2} \right)^2 - \tfrac{T_0}{T_2} - \tfrac{T_1}{T_2} \right] \mapsto \tfrac{y}{z} \]
The fourth is surjective because,
\[ \tfrac{T_2}{T_3} - \tfrac{T_0}{T_3} = \tfrac{x + z}{y} \quad \tfrac{T_1}{T_3} \mapsto \tfrac{y}{z - x} \quad \tfrac{T_0}{T_3} + \tfrac{T_2}{T_3} - 2 \tfrac{T_4}{T_3} = \tfrac{z - x}{y} + \tfrac{2 z}{z - x} \quad \tfrac{1}{2} \left( 1 - \tfrac{T_0}{T_3} \tfrac{T_1}{T_3} + \tfrac{T_1}{T_3} \tfrac{T_2}{T_3} \right) \mapsto \tfrac{z}{z - x} \]
and finally because,
\[ \tfrac{x}{x - z} = 1 - \tfrac{z}{z - x} \] 
The fifth is surjective because,
\[ \tfrac{T_0}{T_4} - \tfrac{T_1}{T_4} = \tfrac{x+y}{z} \quad \tfrac{T_2}{T_4} \mapsto \tfrac{z}{x - y} \quad \tfrac{T_0}{T_4} + \tfrac{T_1}{T_4} + 2 \tfrac{T_3}{T_4} \mapsto \tfrac{x - y}{z} + \tfrac{2 y}{x - y} \quad \tfrac{1}{2} \left( 1 + \tfrac{T_0}{T_4} \tfrac{T_2}{T_4} - \tfrac{T_2}{T_4} \tfrac{T_1}{T_4} \right) \mapsto \tfrac{x}{x - y}  \]
and finally because,
\[ \tfrac{y}{x - y} = \tfrac{x}{x - y} - 1 \]
Therefore this map $\P^2 \to \P^4$ is a closed immersion. 

\item (THIS IS THE INTERESTING ONE!!!) 

Now let $\mathfrak{d} \subset |D|$ be the linear system of all conics passing through a fixed point $P$. We need to show that $\mathfrak{d}$ gives an immersion of $U = \P^2 \setminus \{ P \}$ into $\P^4$ and if we blow up at $P$ to get $\pi : \wt{X} \to X$ then this map extends to give a closed immersion $\wt{X} \embed \P^4$ such that $\wt{X}$ has degree $3$ and that lines in $X$ through $P$ are transformed into lines in $\wt{X}$ which do not meet. Furthermore, $\wt{X}$ is exactly the union of these lines and thus is ruled.
\bigskip\\
By applying an automorphism of $\P^2$ we can assume that $P = [0 : 0 : 1]$. Therefore, the linear system $\mathfrak{d}$ is spanned by,
\[ x^2, y^2, xy, xz, yz \]
It is clear that the only basepoint of $\mathfrak{d}$ is $P$ because $xz$ and $yz$ only both vanish at $P$. Now we will show that $U \to \P^4$ is a locally closed immersion. On affine charts of $\P^4 = \Proj{k[T_0, \dots, T_4]}$ we have maps,
\begin{align*}
k[\tfrac{T_1}{T_0}, \dots, \tfrac{T_4}{T_0}] & \to (k[x,y,z])_{(x^2)} = k[\tfrac{y}{x}, \tfrac{z}{x}] \quad \text{via} \quad \tfrac{T_1}{T_0} \mapsto \tfrac{y^2}{x^2} \quad \tfrac{T_2}{T_0} \mapsto \tfrac{xy}{x^2} \quad \tfrac{T_3}{T_0} \mapsto \tfrac{xz}{x^2} \quad \tfrac{T_4}{T_0} \mapsto \tfrac{yz}{x^2} 
\\
k[\tfrac{T_0}{T_1}, \dots, \tfrac{T_4}{T_1}] & \to (k[x,y,z])_{(y^2)} = k[\tfrac{x}{y}, \tfrac{z}{y}] \quad \text{via} \quad \tfrac{T_0}{T_1} \mapsto \tfrac{x^2}{y^2} \quad \tfrac{T_2}{T_1} \mapsto \tfrac{xy}{y^2} \quad \tfrac{T_3}{T_1} \mapsto \tfrac{xz}{y^2} \quad \tfrac{T_4}{T_1} \mapsto \tfrac{yz}{y^2} 
\\
k[\tfrac{T_0}{T_2}, \dots, \tfrac{T_4}{T_2}] & \to (k[x,y,z])_{(xy)} \quad \text{via} \quad \tfrac{T_0}{T_2} \mapsto \tfrac{x^2}{xy} \quad \tfrac{T_1}{T_2} \mapsto \tfrac{y^2}{xy} \quad \tfrac{T_3}{T_2} \mapsto \tfrac{xz}{xy} \quad \tfrac{T_4}{T_2} \mapsto \tfrac{yz}{xy}  
\\
k[\tfrac{T_0}{T_3}, \dots, \tfrac{T_4}{T_3}] & \to (k[x,y,z])_{(xz)}  \quad \text{via} \quad \tfrac{T_0}{T_3} \mapsto \tfrac{x^2}{xz} \quad \tfrac{T_1}{T_3} \mapsto \tfrac{y^2}{xz} \quad \tfrac{T_2}{T_3} \mapsto \tfrac{xy}{xz} \quad \tfrac{T_4}{T_3} \mapsto \tfrac{yz}{xz} 
\\
k[\tfrac{T_0}{T_4}, \dots, \tfrac{T_3}{T_4}] & \to (k[x,y,z])_{(yz)} \quad \text{via} \quad \tfrac{T_0}{T_4} \mapsto \tfrac{x^2}{yz} \quad \tfrac{T_1}{T_4} \mapsto \tfrac{y^2}{yz} \quad \tfrac{T_2}{T_4} \mapsto \tfrac{xy}{yz} \quad \tfrac{T_3}{T_4} \mapsto \tfrac{xz}{yz} 
\end{align*}
The first map is surjective because,
\[ \tfrac{T_2}{T_0} \mapsto \tfrac{y}{x} \quad \tfrac{T_3}{T_0} \mapsto \tfrac{z}{x} \]
The second map is surjective because,
\[ \tfrac{T_2}{T_1} \mapsto \tfrac{x}{y} \quad \tfrac{T_4}{T_1} \mapsto \tfrac{z}{y} \]
The third map is surjective because,
\[ \tfrac{T_0}{T_2} \mapsto \tfrac{x}{y} \quad \tfrac{T_1}{T_2} \mapsto \tfrac{y}{x} \quad \tfrac{T_3}{T_2} \mapsto \tfrac{z}{y} \quad \tfrac{T_4}{T_2} \mapsto \tfrac{z}{x} \]
The fourth map is not surjective because $\frac{z}{x}$ is not in the image of,
\[ \tfrac{T_0}{T_3} \mapsto \tfrac{x}{z} \quad \tfrac{T_1}{T_3} \mapsto \tfrac{y^2}{xz} \quad \tfrac{T_2}{T_3} \mapsto \tfrac{y}{z} \quad \tfrac{T_4}{T_3} \mapsto \tfrac{y}{x} \]
However, $\tfrac{T_0}{T_3}$ maps to the unit $\frac{x}{z}$ and after inverting $\tfrac{T_0}{T_3}$ this map becomes surjective. 
The fifth map is also not surjective because $\frac{z}{y}$ is not in the image of,
\[ \tfrac{T_0}{T_4} \mapsto \tfrac{x^2}{yz} \quad \tfrac{T_1}{T_4} \mapsto \tfrac{y}{z} \quad \tfrac{T_2}{T_4} \mapsto \tfrac{x}{z} \quad \tfrac{T_3}{T_4} \mapsto \tfrac{x}{y} \]
However, $\tfrac{T_1}{T_4}$ maps to the unit $\frac{y}{z}$ and after inverting $\tfrac{T_1}{T_4}$ this map becomes surjective. Therefore we see that $U \to \P^4$ is a locally closed immersion.
\bigskip\\
Now we consider the blowup $\pi : \wt{X} \to X$ of $P \in X$. The sections, DDOOOOO
\end{enumerate}

\subsubsection{7.8}

Let $X$ be a Noetherian scheme and $\E$ a coherent locally free sheaf on $X$ and $\pi : \P(\E) \to X$. We want to show that sections $\sigma : X \to \P(\E)$ of $\pi : \P(\E) \to X$ and quotient invertible sheaves $\E \to \L \to 0$. This follows immediately from Proposition 7.12.

\subsubsection{7.9 CHECK THIS!!}

Let $X$ be a regular noetherian scheme, and $\E$ a locally free coherent sheaf of rank $r \ge 2$ on $X$. Consider $\pi : \P_X(\E) \to X$.
\bigskip\\
We also need to assume that $X$ is connected, and thus integral, otherwise we could take a collection of points and the formula fails. 

\begin{enumerate}
\item There is a map $\Phi : \Pic{X} \oplus \Z \to \Pic{\P(\E)}$ via $(\L, n) \mapsto \pi^* \L \otimes \struct{\P(\E)}(n)$. First, we show that $\Phi$ is injective. If $\pi^* \L \otimes \struct{\P(\E)}(n)$ is trivial, by restricting to a fiber $\P^{r-1} \to \Spec{\kappa(x)}$ we see that $n = 0$ so it suffices to show that $\L \cong \struct{X}$. Because $\pi_* \struct{\P(\E)} = \struct{X}$ and $\pi^* \L$ is locally free we know that $\L \to \pi_* \pi^* \L$ is an isomorphism by the projection formula. Thus, if $\pi^* \L \cong \struct{\P(\E)}$ then,
\[ \L \to \pi_* \struct{\P(\E)} = \struct{X} \]
is an isomorphism showing that $\Phi$ is injective for any scheme $X$ (DO I NEED ANY HYPOTHESES)
\bigskip\\
Now we prove surjectivity. First, we consider the case that $\E$ is trivial so that $\P_X(\E) = \P^{r-1} \times X$. Since $X$ is regular (and hence normal and locally factorial etc) we get a diagram,
\begin{center}
\begin{tikzcd}
\Pic{\P^{r-1} \times X} \arrow[r, "\sim"] & \Cl{\P^{r-1} \times X}
\\
\Pic{X} \oplus \Z \arrow[u] \arrow[r, "\sim"] & \Cl{X} \oplus \Z \arrow[u, "\sim"]
\end{tikzcd}
\end{center}
where the map $\Cl{X} \oplus \Z \to \Cl{\P^{r-1} \times X}$ is an isomorphism by [II, Ex. 6.1]. Therefore, we conclude for the case that $\E$ is trivial.
\bigskip\\
For the general case, cover $X$ by opens $U_\alpha \subset X$ over which $\E$ is trivial. Then for $\L \in \Pic{\P_X(\E)}$ we get line bundles $\L|_{U_\alpha}$. Since $\E|_U \cong \struct{U}^{\oplus r}$ then $\pi^{-1}(U) \cong \P^{r-1} \times U$ so by the previous part there are isomorphisms,
\[ \varphi_\alpha : \L|_{\pi^{-1}(U_\alpha)} \iso \pi^* \M_\alpha \ot \struct{\P}(n_\alpha) \]
for some $\M_\alpha \in \Pic{U_\alpha}$ and $n_\alpha \in \Z$. However, since the $n_\alpha$ are uniquely determined by $\L$ (given by the projection $\Pic{\P^{r-1}_X} \to \Z$) they must agree on the overlaps so because $X$ is connected we see that $n_\alpha = n$ is constant. Since $\struct{\P(\E)}(1)$ restricts to $\struct{\P}(1)$ over each $U_\alpha$ then we can assume that $n = 0$ by tensoring $\L$ by $\struct{\P(\E)}(-n)$. Now I claim that the line bundles $\M_\alpha \in \Pic{U_\alpha}$ glue to a line bundle $\M$. 
\bigskip\\
Indeed, we get isomorphisms $\varphi_\alpha \circ \varphi_\beta^{-1} : \pi^* \M_\beta|_{U_\alpha \cap U_\beta} \to \pi^* \M_\alpha |_{U_\alpha \cap U_\beta}$ satisfying the cocycle condition. Applying the following lemma gives gluing data $\psi_{\alpha \beta}$ such that $\pi^* \psi_{\alpha \beta} = \varphi_\alpha \circ \varphi_\beta^{-1}$.  This gluing data determines a line bundle $\M \in \Pic{X}$ equipped with isomorphism $\psi_\alpha : \M|_{U_\alpha} \iso \M_\alpha$ compatible with $\psi_{\alpha \beta}$ on overlaps. Since $\L$ is the unique line bundle along with the data of isomorphisms $\varphi_\alpha : \L|_{U_\alpha} \iso \pi^* \M_\alpha$ compatible with the gluing data $\varphi_\alpha \circ \varphi_\beta^{-1}$ but $\pi^* \M$ along with $\pi^* \psi_\alpha : \pi^* \M |_{U_\alpha} \iso \pi^* \M_\alpha$ is also compatible with $\pi^* \psi_{\alpha \beta} = \varphi_\alpha \circ \varphi_\beta^{-1}$ showing that $\L \cong \pi^* \M$ by uniqueness. Therefore $\Pic{X} \oplus \Z \to \Pic{\P_X(\E)}$ is surjective.

\begin{lemma}
Let $\pi : X \to S$ be a morphism with $\pi_* \struct{X} = \struct{S}$. Let $\E_1$ and $\E_2$ be locally free $\struct{X}$-modules. Then,
\[ \Hom{\struct{S}}{\E_1}{\E_2} \iso \Hom{\struct{X}}{\pi^* \E_1}{\pi^* \E_2} \]
is an isomorphism. 
\end{lemma}

\begin{proof}
Because $\E_1$ and $\E_2$ are locally free $\shHom{\struct{S}}{\E_1}{\E_2}$ is locally free and,
\[ \pi^* \shHom{\struct{S}}{\E_1}{\E_2} = \shHom{\struct{X}}{\pi^*\E_1}{\pi^* \E_2} \]
Therefore, by the projection formula,
\[ \pi_* \pi^* \shHom{\struct{S}}{\E_1}{\E_2} = \shHom{\struct{S}}{\E_1}{\E_2} \ot_{\struct{S}} \pi_* \struct{X}) = \shHom{\struct{S}}{\E_1}{\E_2} \]
which implies that,
\begin{align*}
\Hom{\struct{X}}{\pi^* \E_1}{\pi^* \E_2} & = H^0(X, \shHom{\struct{X}}{\pi^*\E_1}{\pi^* \E_2}) = H^0(X, \pi^* \shHom{\struct{S}}{\E_1}{\E_2}) 
\\
& = H^0(S, \pi_* \pi^* \shHom{\struct{S}}{\E_1}{\E_2}) = H^0(S, \shHom{\struct{S}}{\E_1}{\E_2}) = \Hom{\struct{S}}{\E_1}{\E_2} 
\end{align*}
\end{proof}

\begin{cor}
Let $\pi : \P_X(\E) \to X$ be a projective bundle for a locally free $\struct{X}$-module $\E$ and let $\L_1, \L_2 \in \Pic{X}$ be line bundles on $X$. Then,
\[ \Hom{\struct{X}}{\L_1}{\L_2} \iso \Hom{\struct{\P(\E)}}{\pi^* \L_1}{\pi^* \L_2} \]
is an isomorphism. 
\end{cor}

\begin{proof}
This follows immediately from above given that $\pi_* \struct{\P(\E)} = \struct{X}$ [II, Prop. 7.11].
\end{proof}

Alternatively, the isomorphism $\varphi_\alpha : \L|_{\pi^{-1}(U_\alpha)} \iso \pi^* \M_\alpha$ prove that,
\[ (\pi_* \L)|_{U_\alpha} = \pi_* (\L|_{\pi^{-1}(U_\alpha)}) \cong \pi_* \pi^* \M_\alpha = \M_\alpha \]
because $\pi_* \struct{\P} = \struct{X}$ using the projection formla. Therefore $\pi_* \L$ is a line bundle. Then the natural map $\pi^* \pi_* \L \to \L$ is an isomorphism because locally it is given by $\pi^* \pi_* \pi^* \M_\alpha \to \pi^* \M_\alpha$ which is an isomorphism because $\pi_* \pi^* \M_\alpha = \M_\alpha$ by the adjunction formula. 


\item Let $\E$ and $\E'$ be locally free coherent sheaves on $X$. Then $\P(\E)$ represents the functor $F_\E$ taking a scheme $f : T \to X$ to isomorphism classes of rank one quotients of $f^* \E$. By Yoneda, isomorphisms $\P(\E) \iso \P(\E')$ correspond to natural isomorphisms $\eta : F_{\E} \to F_{\E'}$. Suppose that $\E' \cong \E \otimes \L$ for a line bundle $\L$. Then $\eta_\L : F_\E \to F_{\E'}$ via,
\[ (f^* \E \to \sQ \to 0) \mapsto (f^* \E' \to \sQ \otimes \L \to 0) \] 
is a natural isomorphism because $\L$ is invertible. Thus $\varphi_\L : \P(\E) \iso \P(\E')$ is an isomorphism. Conversely, a map $\varphi : \P(\E) \to \P(\E')$ over $X$ is equivalent to the data of an exact sequence,
\[ \pi^* \E' \to \varphi^*\struct{\P(\E')}(1) \to 0 \]
If $\varphi$ is an isomorphism then $\varphi^* : \Pic{\P(\E)} \to \Pic{\P(\E')}$ is an isomorphism preserving $\Pic{X}$. Using the decomposition $\Pic{\P(\E)} = \Pic{X} \oplus \Z$, the map 
\[ \Z \to \Pic{\P(\E')} \to \Pic{\P(\E)} \to \Z \]
must be the identity (sections must pull back so $1 \mapsto -1$ is impossible). Thus, by part (a) there is some $\L \in \Pic{X}$ such that,
\[ \varphi^* \struct{\P(\E')}(1) \cong \pi^* \L \otimes \struct{\P(\E)}(1) \]
Likewise, by the same argument,
\[ (\varphi^{-1})^* \struct{\P(\E)}(1) \cong \pi'^* \L' \otimes \struct{\P(\E')}(1) \]
Composing these maps,
\[ \struct{\P(\E)}(1) = \varphi^* (\varphi^{-1})^* \struct{\P(E)}(1) = \pi^* \L \otimes \pi^* \L' \otimes \struct{\P(E)}(1) \]
Since the map $\Pic{X} \to \Pic{\P(\E)}$ is injective we see that $\L \otimes \L' \cong \struct{X}$ so $\L' \cong \L^\vee$. Furthermore, $(\varphi^{-1})^* = \varphi_*$ and $\pi' \circ \varphi = \pi$. Recall that $\E = \pi_* \struct{\P(\E)}(1)$. Thus, using the projection formula,
\[ \E = \pi_* \struct{\P(\E)}(1) = \pi'_* \varphi_* \struct{\P(\E)}(1) \cong \pi'_* (\pi'^* \L' \otimes \struct{\P(\E')}(1)) = \L' \otimes \E' \]
but $\L' \cong \L^{\vee}$ and therefore,
\[ \E' \cong \E \otimes \L \]
\end{enumerate}

\subsubsection{7.10 DO!}

Let $X$ be a noetherian scheme. 

\begin{enumerate}
\item We define a projective $n$-space bundle on $X$ as a scheme $\pi : P \to X$ such that $P$ is locally isomorphic to $U \times \P^n \to U$ for $U \subset X$ open and the transition automorphims on $\P^n_A$ over affine open $\Spec{A}$ are given by $A$-linear automorphisms of the homogeneous coordinate ring $A[x_0, \dots, x_n]$.

\item If $\E$ is a locally free sheaf of rank $n+1$ then consider $\P(\E) \to X$. Then there is an affine cover $\{ U_i \}$ of $X$ such that $\E|_{U_i} \cong \struct{U_i}^{n+1}$.

\item Let $\pi : P \to X$ be a projective space bundle over a regular noetherian scheme $X$. It suffices to consider the case that $X$ is connected otherwise $P$ is a disjoint union of projective bundles over each connected component of $X$ (since $X$ is noetherian there are finitely many connected components which are thus open and closed). Since $X$ is regular it is thus integral. Then $\pi$ is separated and projective so it is closed. Then $P$ is regular (look at the open cover) and connected because $\pi$ is closed with connected fibers and hence integral. Assuming further that $X$ is separated, $P$ is separated. Choose an affine open $U \subset X$ such that $\pi^{-1}(U) \cong U \times \P^n$ and let $\L_0 \in \Pic{\pi^{-1}(U)}$ be the line bundle $\struct{}(1)$ on $U \times \P^n$. Because $X$ is regular, $P$ is regular and hence locally factorial. Therefore, there is a diagram,
\begin{center}
\begin{tikzcd}
\Pic{P} \arrow[r] \arrow[d, "\sim"] & \Pic{\pi^{-1}(U)} \arrow[d, "\sim"]
\\
\Cl{P} \arrow[r] & \Cl{\pi^{-1}(U)}
\end{tikzcd}
\end{center}
and therefore $\Pic{P} \to \Pic{\pi^{-1}(U)}$ is surjective so there exists a line bundle $\L \in \Pic{P}$ extending $\L_0$. Let $\E = \pi_* \L$. I claim that $\E$ is a vector bundle. For each open $U \subset X$ such that $\pi^{-1}(U) \cong U \times \P^n$ we see that,
\[ \Cl{\pi^{-1}(U)} = \Pic{U} \oplus \Z \]
and therefore $\L |_{\pi^{-1}(U)} = \pi^* \M \otimes \struct{}(n)$ for $\L' \in \Pic{U}$. However, because $P$ is integral every pair of nonempty open sets intersect so  the number $n$ is constant via restricting to the overlaps. Thus $n = 1$ by construction. Then, (by flat base change if you need a fany way to justify it) 
\[ \E|_U = \pi_* (\L|_{\pi^{-1}(U)}) = \pi_* (\pi^* \M \otimes \struct{}(1)) = \M \otimes \pi_* \struct{}(1) = \M \otimes_{\Gamma(U, \struct{U})} \Gamma(U \times \P^n, \struct{}(1)) \cong \M^{\oplus {n+1}} \] 
by the projection formula. Therefore $\E$ is locally free of rank $n + 1$. Furthermore, the natural map $\eta : \pi^* \pi_* \L \to L$ is surjective because locally on $\pi^{-1}(U)$ it is given by $\pi^* \M^{\oplus n + 1} \onto \pi^* \M \otimes \struct{}(1)$ defined by the $n+1$ sections of $\struct{}(1)$ on $U \times \P^n$ globally generating $\struct{}(1)$. Then $\eta : \pi^* \E \onto \L$ defines a morphism $f : P \to \P_X(\E)$. I claim that $f$ is an isomorphism of projective bundles over $X$. This follows immediately from a local calculation. Over an an affine open $U \subset X$ trivializing $\E$ and $\pi$ we have $f : U \times \P^n \to \P_U(\E_U) \cong U \times \P^n$ defined by, up to a choice of basis of $\M$ i.e. a linear automorphism over $U$, the line bundle $\struct{}$ and its standard global sections which is an isomorphism. (DO THIS BETTER)



(ALTERNATIVE STRATEGY SPENCER)


The $\struct{}(1)$ defined on each $U \times \P^n$ have gluing data induced by the linear automorphism for the overlaps. These define a global line bundle $\L$ from the cocycle condition for the automorphisms from the fact that $P$ is actually a scheme. Then $\pi_* \L$ is clearly locally free since it is locally $\pi_* \struct{}(1) \cong \struct{U}^{\oplus n + 1}$. The line bundle $\L$ has a surjection $\pi^* \pi_* \L \onto \L$ because locally this is $\struct{\P^n} \ot \Gamma(\P^n, \struct{\P^n}(1)) \onto \struct{\P^n}(1)$ which is surjective. Therefore, $\pi^* \E \onto \L$ for $\E = \pi_* \L$ giving an $X$-morphism $P \to \P_X(\E)$. This is an isomorphism because locally it is given by $\P^n \times U \to \P^n \times U$ defined by a linear change of coordinates (DO THIS FOR REAL). 

\item 
\end{enumerate}

\subsubsection{7.11 DO!}

Let $X$ be a Noetherian scheme. 

\begin{enumerate}
\item If $\I$ is any coherent ideal sheaf then consider the ideal sheaf $\I^d$. Let $\wt{X} \to X$ JUST APPLY EX II 5.13

\item Just locally trivialize this then they are isomorphic algebras 

\item NOT SURE !!
\end{enumerate}


\subsubsection{7.12 DO!}

Let $X$ be a Noetherian scheme and $Y,Z \subset X$ closed subschemes not contained in eachother. Let $\tilde{X} \to X$ be the blowing up of $X$ along $Y \cap Z$ i.e. $\I_Y + \I_Z$. Consider the strict transforms $\tilde{Y}$ and $\tilde{Z}$ defined by the blowups of $Y$ and $Z$ at the pullback ideal of $\I_Y + \I_Z$.
\bigskip\\
Notice that,
\[ \iota_Z^{-1} (\I_Y + \I_Z) \cdot \struct{Z} = \iota_Z^{-1} \I_Y \cdot \struct{Z} \]
and likewise for $Y$ because $\I_Z$ kills $\struct{Z}$ by definition. We need to consider, the sheaves of graded algebras corresponding to the ideals $\I_Y + \I_Z$ and $(\I_Y + \I_Z) / \I_Y = \I_Z \cdot \struct{Y}$ and $(\I_Y + \I_Z) / \I_Z = \I_Z \cdot \struct{Z}$. The subschemes $\wt{Y}$ and $\wt{Z}$ are cut out in term of the proj construction by the kernels of the maps,
\[ \J_Y = \ker{ \left( \bigoplus_{n \ge 0} (\I_Y + \I_Z)^n \to \bigoplus_{n \ge 0} (\I_Y \cdot \struct{Z})^n \right)} \]
and likewise, 
\[ \J_Z = \ker{ \left( \bigoplus_{n \ge 0} (\I_Y + \I_Z)^n \to \bigoplus_{n \ge 0} (\I_Z \cdot \struct{Y})^n \right)} \]
However, every section generated by $\I_Y \subset \I_Y + \I_Z$ is in the kernel $\J_Z$ and likewise every section generated by $\I_Z \subset \I_Y + \I_Z$ is in the kernel $\J_Y$. Since $\I_Y + \I_Z$ generates the positive-degree part of the subalgebra we see that $\J_Y + \J_Z$ contains the irrelevant ideal and thus $\wt{Y} \cap \wt{Z}$ which is defined by $\J_Y + \J_Z$ is the sense of Proj must be empty. (DO THIS BETTER!!!!)

\begin{rmk}
Blowing up does not respect arbitrary pullbacks. Indeed, if it did there would not be a difference between strict transform and the total transform (preimage). However relative Proj does commute with base change so why doesn't blowup? To see why consider the blowup with center $Z \subset X$ and pulling back along $\iota : Z \embed X$. Then the ideal pulls back to zero on $Z$ so the strict transform is empty. However, the fiber over $Z$ is the exceptional divisor. Now,
\[ \iota^* \rProj{X}{\bigoplus_{n \ge 0} \I^n} = \rProj{Z}{\iota^* \bigoplus_{n \ge 0} \I^n } \]
but the issue is that the pullback of the Rees algebra is not equal to the Rees algebra of the pullback ideal so it is the formation of this algebra not the Proj construction where preservation under pullbacks fails. Indeed, in our example,
\[ \iota^* \bigoplus_{n \ge 0} \I^n = \bigoplus_{n \ge 0} \I^n / \I^{n+1} \]
because $\I^n \ot_{\struct{X}} \struct{Z} = \I^n \ot_{\struct{X}} \struct{X} / \I = \I^n / \I^{n+1}$ which is the graded algebra for the tangent cone meaning that the exceptional fiber is the projectived tangent cone bundle. However, $\iota^{-1} \I \cdot \struct{Z} = 0$ because $\I$ kills $\struct{Z}$ by definition. We see that $\iota^* \I$ is nonzero but the map $\iota^* \I \to \iota^* \struct{X} = \struct{Z}$ is zero. 
\end{rmk}

\begin{rmk}
We really need to blow up at $\I_Y + \I_Z$ for this to work not some other ideal cutting out a closed subscheme supported on $Y \cap Z$. For example, let $X = \A^2$ and $Y,Z$ be curves intersecting tangentially at the origin. Let $\I$ be the ideal sheaf of the origin then $\wt{Y}$ and $\wt{Z}$ still intersect (although possibly transversally now). 
\end{rmk}

\begin{rmk}
In general it is false that the strict transform of an intersection is the intersections of the strict transforms. However, if $X$ is a notherian scheme, $\I$ a coherent sheaf of ideals and $\pi : \wt{X} \to X$ the blowup of $\I$. Given morphisms $f_i : Y_i \to X$ then there is a canonical morphism
\[ \wt{Y_1 \times_X Y_2} \to \wt{Y_1} \times_{\wt{X}} \wt{Y_2} \]
where $\wt{Y_i}$ is the blowup of $Y_i$ at $f_i^{-1} \I \cdot \struct{Y_i}$ equiped with the unique morphism $\wt{Y_i} \to \wt{X}$. By the functoriality of [II, Cor. 7.15] we get a natural map $\wt{Y_1 \times_X Y_2} \to \wt{Y_1} \times_{\wt{X}} \wt{Y_2}$ from the maps $\wt{Y_1 \times_X Y_2} \to \wt{Y_i}$ over $Y_1 \times_X Y_2 \to Y_i$. 
\end{rmk}

\subsubsection{7.13 DO!!}

Let $k$ be an algebraically closed field of characteristic not equal to $2$. Let $C \subset \P^2_k$ be the nodal cubic curve $y^2 z = x^2 + x^2 z$. Let $P_0 = [0:0:1]$ be the singular point then $C \setminus P_0 \cong \Gm$. For each $a \in \Gm(k) = k^\times$ there is a translation map $\Gm \to \Gm$ via $t \mapsto a t$ which induces an automorphism of $C$ denoted by $\varphi_a$. We glue the schemes $C \times (\P^1 \setminus \{ 0 \})$ and $C \times (\P^1 \setminus \{ \infty \})$ along the mutual open $C \times (\P^1 \setminus \{ 0, \infty \})$ by the isomorphism $\varphi : (P, u) \mapsto (\varphi_u(P), u)$ for $P \in C$ and $u \in \Gm = \P^1 \setminus \{ 0, \infty \}$. This gluing produces a scheme $X$ with a natural map $\pi : X \to \P^1$ because the projection maps to the second factor on each open are compatible with the automorphism $\phi$.


\begin{enumerate}
\item To show that $\pi : X \to \P^1$ is proper, it suffices to show that $\pi$ is locally projective on the base. Notice that over the standard affine cover $\P^1 \setminus \{ 0 \}, \P^1 \setminus \{ \infty \} \subset \P^1$ the map $\pi : X \to \P^1$ is isomorphic to $\pi|_{\P^1 \setminus \{ 0 \}} : C \times (\P^1 \setminus \{ 0 \}) \to \P^1 \setminus \{ 0 \}$ and $\pi|_{\P^1 \setminus \{ \infty \}} : C \times (\P^1 \setminus \{ \infty \}) \to \P^1 \setminus \{ \infty \}$ which are both projective because $C$ is projective and base change. Therefore $\pi$ is locally projective on the base. 

\item 

\item 

\item Elements of $\Pic{X}$ are given by a tripple $(\L_1, \L_2, \psi)$ where $\L_1, \L_2$ are line bundles on the two open patches $U_1, U_2$ isomorphism to $C \times \A^1$ respectively and $\psi : \L_1|_{U_1 \cap U_2} \iso \L_2|_{U_1 \cap U_2}$ is the glueing map. However, $U_1 \cap U_2 \cong C \times \Gm$ so these line bundles are classified by a pair of tuples $(t_1, n_1), (t_2, n_2) \in \Pic{C \times \A^1}$ which restrict to the same line bundle on $U_1 \cap U_2$. By the previous part $(t_1, n_1) \mapsto (t_1, 0, n_1) \in \Pic{C \times \Gm}$. However, the other patch is idenfied via the automorphism $\phi$ and therefore $(t_2, n_2) \mapsto (t_2, n_2, n_2)$. Thus for $(t_1, 0, n_1) = (t_2, n_2, n_2)$ we must have $t_1 = t_2$ and $n_1 = n_2 = 0$. Therefore, $\L_1$ and $\L_2$ are abstractly isomorphic as line bundles on $C \times \A^1$ which are the pullback of some $\L \in C$ so we get that $\Pic{X}$ is described as pairs $(\L, \psi)$. Moreover, $\Pic{X} \to \Pic{C \times \{ 0 \}} \cong \Pic{C}$ sends $(\L, \psi) \mapsto \L$ but also we showed that $\L$ is of type $(t, 0)$ and thus $\deg{\L} = 0$ on $C$ proving the first claim.
\bigskip\\
Now suppose that $X$ is projective over $k$. Then there must exist a very ample line bundle $\M = (\L, \psi)$ on $X$. Therefore, $\M$ is generated by global sections so $\L$, its pullback to $C$ (its image under the map $\Pic{X} \to \Pic{C}$), is also generated by global sections. However, $\deg{\L} = 0$ so if $\L$ is generated by global sections then $\L \cong \struct{C}$ but this is not very ample. Therefore, $X$ cannot be projective over $k$ and thus $\pi : X \to \P^1$ cannot be projective otherwise the composition $X \to \P^1 \to \Spec{k}$ would be projective as well. 
\end{enumerate}

\subsubsection{7.14 (CHECK!)}

\begin{enumerate}
\item Let $X = \P^1$ and $\E = \struct{X} \oplus \struct{X}(-n)$ for $n > 0$. Then $\pi : \P(\E) \to X$ is the Hirzebruch surface $H_n$ over $\P^1$. We know there is an isomorphism $\pi_* \struct{\P(\E)}(1) \cong \E$ and therefore,
\[ H^0(\P(\E), \struct{\P(\E)}(1)) = H^0(X, \pi_* \struct{\P(\E)}(1)) = H^0(X, \E) = 0 \]
because $\struct{X}(-n)$ has no nontrivial global sections on $X$. Therefore, $\struct{\P(\E)}(1)$ cannot be very ample relative to $X$ since it does not have enough global sections and $\P(\E) \to X$ has positive dimensional fibers.

\item Let $f : X \to Y$ be a finite type morphism, $\L$ an ample invertible sheaf on $X$, and $\sA$ a sheaf of graded $\struct{X}$-algebras satisfying ($\dagger$). Let $P = \rProj{X}{\sA}$ and $\pi : P \to X$ and $\struct{P}(1)$ the relative antitautological bundle. By Prop. 7.10, for some $n$ the invertible sheaf $\struct{P}(1) \otimes \pi^* \L^{\otimes n}$ is very ample relative to $X$. Furthermore, since $\L$ is ample, for sufficiently large $m$ we know $\L^{\otimes m}$ is very ample relative to $f : X \to Y$ (CHECK THIS!! I THINK I NEED $Y$ TO BE NOETHERIAN FOR THIS). 

Therefore, by [II, Ex. 5.12(b)] we know that,
\[ (\struct{P}(1) \otimes \pi^* \L^{\otimes n}) \otimes \pi^* \L^{\otimes m} = \struct{P}(1) \otimes \pi^* \L^{\otimes (n + m)} \]
is very ample relative to $P \to Y$. Furthermore, since $\L$ is ample we know that for $\ell > \ell_0$ the bundle $\L^{\otimes \ell}$ is generated by global sections and therefore by [II, Ex. 7.5(d)] the invertible sheaf,
\[ (\struct{P}(1) \otimes \pi^* \L^{\otimes n + m}) \otimes \pi^* \L^{\otimes \ell} = \struct{P}(1) \otimes \pi^* \L^{\otimes (n + m + \ell)} \]
is very ample relative to $P \to Y$. Thus for $k > k_0 = n + m + \ell_0$ we have shown that, $\struct{P}(1) \otimes \pi^* \L^{\otimes k}$ is very ample relative to $P \to Y$.
\end{enumerate}

\section{III Cohomology}

\subsection{Section 2}

\subsubsection{2.1 DO!!}

\begin{enumerate}
\item Let $X = \A^1_k$ be the affine line over an infinite field $k$ and $P, Q \in X$ be distinct points. There is an exact sequence [II, Ex. 1.19],
\begin{center}
\begin{tikzcd}
0 \arrow[r] & \Z_U \arrow[r] & \Z_X \arrow[r] & \Z_Y \arrow[r] & 0
\end{tikzcd}
\end{center}
where $Y = \{ P, Q \}$ and $U = X \setminus Y$. Then $\Z_Y = \iota_P \Z \oplus \iota_Q \Z$. Taking the cohomology sequence,
\begin{center}
\begin{tikzcd}
0 \arrow[r] & \Gamma(X, \Z_U) \arrow[r] & \Gamma(X, \Z) \arrow[r] & \Gamma(X, \Z_Y) \arrow[r] & H^1(X, \Z_U) 
\end{tikzcd}
\end{center}
However, $\Gamma(X, \Z) = \Z$ because $X$ is connected and $\Gamma(X, \Z_Y) = \Z \oplus \Z$ because $P, Q \in X$. Therefore, $\Gamma(X, \Z) \to \Gamma(X, \Z_Y)$ cannot be surjective so we must have $H^1(X, \Z_U) \neq 0$.

\item Let $Y \subset X = \A^n_k$ be the union of $n + 1$ hyperplanes in general position and let $U = X \setminus Y$. 
There is an exact sequence [II, Ex. 1.19],
\begin{center}
\begin{tikzcd}
0 \arrow[r] & \Z_U \arrow[r] & \Z_X \arrow[r] & \Z_Y \arrow[r] & 0
\end{tikzcd}
\end{center}
Let $H_i$ be the hyperplanes then,
\[ Y = \bigcup_{i = 1}^{n+1} H_i \]
Then there is a Mayer-Vietoris resolution of $\Z_Y$,
\begin{center}
\begin{tikzcd}
0 \arrow[r] & \Z_Y \arrow[r] & \bigoplus\limits_{i} \Z_{H_i} \arrow[r] & \bigoplus\limits_{i < j}  \Z_{H_i \cap H_j} \arrow[r] & \cdots \arrow[r] & \Z_{H_1 \cap \cdots \cap H_{n+1}} \arrow[r] & 0
\end{tikzcd}
\end{center}
However, each intersection \[ V_I = \bigcap_{i \in I} V_i = V_{i_0} \cap \cdots \cap V_{i_k} \]
is a linear affine subspace of codimension $|I|$ and hence irreducible. Therefore each $\Z_{V_I}$ is flasque because every nonempty open subset of an irreducible space is connected. Hence this gives a flasque resolution of $\Z_Y$. Therefore, the cohomology $H^q(X, \Z_Y) = H^q(Y, \Z_Y)$ (because $\Z_Y$ on $X$ is the pushforward along $\iota : Y \embed X$ and $\iota_*$ is exact because it preserves stalks and therefore $R^q \iota_* = 0$ for $q > 0$ so by the Leray spectral sequence $\iota_*$ preserves cohomology) is computed by the complex,
\begin{center}
\begin{tikzcd}
0 \arrow[r] & \bigoplus\limits_i \Z \arrow[r] & \bigoplus\limits_{i < j} \Z \arrow[r] & \cdots \arrow[r] & \bigoplus_{i_1 < \cdots < i_n} \Z \arrow[r] & 0 \arrow[r] & 0
\end{tikzcd}
\end{center}
because the intersection of $n+1$ hyperplanes in general position is empty. Therefore,
\[ H^{n-1}(X, \Z_Y) = \Z^{\oplus (n+1)} / \im{\left( \bigoplus_{i_1 < \dots < i_{n-1}} \Z \to \bigoplus_{i_1 < \dots < i_n} \Z \right)} \]
Notice that the sum map,
\[ \bigoplus_{i_1 < \dots < i_n} \Z \to \Z \]
sends the image to zero and therefore descends to a surjective map,
\[ H^{n-1}(X, \Z_Y) \onto \Z \]
proving that $H^{n-1}(X, \Z_Y) \neq 0$. Applying cohomology to the first exact sequence we get,
\begin{center}
\begin{tikzcd}
H^{n-1}(X, \Z_X) \arrow[r] & H^{n-1}(X, \Z_Y) \arrow[r] & H^n(X, \Z_U) \arrow[r] & H^n(X, \Z_X) 
\end{tikzcd}
\end{center}
but $\Z_X$ is flasque so $H^q(X, \Z_X) = 0$ for $q > 0$. Therefore we get an isomorphism,
\[ H^{n-1}(X, \Z_Y) \iso H^n(X, \Z_U) \]
and therefore there is a surjective map $H^n(X, \Z_U) \onto \Z$ proving that $H^n(X, \Z_U) \neq 0$.
\end{enumerate}

\subsubsection{2.2}

Let $X = \P^1_k$ with $k$ an algebraically closed field. Consider the exact sequence,
\begin{center}
\begin{tikzcd}
0 \arrow[r] & \struct{X} \arrow[r] & \K \arrow[r] & \K / \struct{X} \arrow[r] & 0 
\end{tikzcd}
\end{center}
Because $X$ is integral $\K$ is (litterally in the presheaf sense) constant and therefore flasque. Thus to show the above is a flasque resolution it suffices to show that $\K / \struct{X}$ is flasque. We can describe the sections $\Gamma(U, \K / \struct{X})$ explicitly as $\{ (U_i, f_i) \}$ where $\{ U_i \}$ form an open cover of $U$ and $f_i \in K(X)$ and on $U_i \cap U_j$ we have $f_i - f_j \in \struct{X}(U_i \cap U_j)$ where the $f_i$ are taken modulo $\struct{X}(U_i)$ and the tuples up to refinement. In fact, I claim that $K(X) \to \Gamma(U, \K/\struct{X})$ is surjective which would in particular show that $\K / \struct{X}$ is flasque.
\bigskip\\
The claim is exactly the first cousin's problem which we can solve explicitly on $\P^1$. Given $\{ (U_i, f_i) \}$ we need to find rational function $g \in K(X)$ such that $g - f_i \in \struct{X}(U_i)$ for each $i$. Since $X$ is noetherian, we can find a finite subcover $\{ (V_i, f_i) \}$ and it is clear that, taking tuple of intersections, these two elements refine to the same element and thus are equal. For each $P \in \P^1$ let $d_P$ be the order of the pole and $r_P$ the residue of any $f_i$ for $P \in V_i$. Notice that if $P \in V_j$ then $f_i - f_j$ have no poles on $V_i \cap V_j$ so $\ord_{P}(f_i) = \ord_P{f_j}$ and $r_P(f_i) = r_P(f_j)$ showing that $d_P$ and $r_P$ are well-defined. Furthermore, there is a finite set $\{ V_i \}$ and each $f_i$ has finitely many poles so $S = \{ P \in X \mid d_P > 0 \}$ is finite. Therefore we can define,
\[ g = \sum_{P \in S} \frac{r_P}{(x - P)^{d_P}} \]
Note that when $P = \infty$ we replace $(x - P)^{-1}$ by $x$. Then $g - f_i$ does not have any poles on $U_i$ because $S \cap U_i$ are exactly the poles of $f_i$ on which $f_i$ has poles of order $d_P$ with residue $r_P$ so $g - f_i$ has no poles. Thus $g - f_i \in \struct{X}(U_i)$ for each $i$ proving the claim.
\bigskip\\
Therefore, we can compute $H^i(X, \struct{X})$ as the cohomology of the exact sequence,
\begin{center}
\begin{tikzcd}
K(X) \arrow[r] & \Gamma(X, \K / \struct{X})
\end{tikzcd}
\end{center}
We see (as is clear a priori) that $H^0(X, \struct{X}) = \Gamma(X, \struct{X})$ and $H^1(X, \struct{X}) = 0$ because we showed, in particular, that $K(X) \onto \Gamma(X, \K/\struct{X})$ is surjective. Finally, $H^i(X, \struct{X}) = 0$ for $i > 1$ because the sequence is zero degrees greater than $1$.
\bigskip\\
This proof relied heavily on $\P^1$ having, for each point $P$, a rational function with a pole of order $1$ exactly at $P$. This is false for curves of higher genus.
\bigskip\\
An alternative solution uses [Ex. 2.1.21] which shows that
\[ \K /\struct{X} \iso \bigoplus_{x \in X} (\iota_x)_* (K / \stalk{X}{x}) \]
which is clearly flasque and also that,
\[ K(X) \to \Gamma(X, \K/\struct{X}) \]
is surjective which were the only two ingredients we needed.

\subsubsection{2.5 CHECK}

Let $X$ be a Zariski space. Let $P \in X$ be a closed point. Let $X_P \subset X$ be the subset of points $Q \in X$ with $P \in \overline{\{Q\}}$ meaning $Q \leadsto P$. Let $j : X_P \to X$ be the inclusion. Let $\F$ be any sheaf on $X$ and $\F|_P = j^* \F$. Notice that if $U$ is open with $P \in U$ then if $P \in \overline{ \{ Q \} }$ then if $Q \notin U$ then $\overline{ \{ Q \} } \subset U^C$ but $P \in U$ so we see that $Q \in U$. Thus if $P \in U$ then $U \cap X_P = X_P$ so the only open of $X_P$ containing $P$ is $X_P$. 
\bigskip\\
First we show that $\Gamma_P(X, \F) \cong \Gamma_P(X_P, \F|_P)$ natrually in $\F$. By excision, for any open $U \ni P$,
\[ \Gamma_P(X, \F) = \Gamma_P(U, \F|_U) \]
We compute $\Gamma_P(X_P, \F|_P)$. By definition, $\F|_P$ is the sheafification of,
\[ V \mapsto \varinjlim_{U \supset V} \F(U) \]
Notice that for any presheaf $\G$ on $X_P$ because $X_P$ is the only open neighborhood of $P$ we have $\G_P = \G(X_P)$ and thus $\G \to \G^+$ gives an isomorphism $\G(X_P) \iso \G^+(X_P)$ since it is an isomorphim on stalks $\G_P \iso \G^+_P$. In particular, 
\[ \varinjlim_{U \ni P} \F(U) \iso \Gamma(X_P, \F|_P) \]
is an isomorphism. Then consider the diagram,
\begin{center}
\begin{tikzcd}
\varinjlim\limits_{U \ni P} \Gamma_P(U, \F) \arrow[d, hook] \arrow[r] & \Gamma_P(X_P, \F|_P) \arrow[d, hook]
\\
\varinjlim\limits_{U \ni P} \Gamma(U, \F) \arrow[r, "\sim"] & \Gamma(X_P, \F|_P)
\end{tikzcd}
\end{center}
Therefore the top map is injective. Since the bottom map is surjective it suffices to prove that the unique class maping to a section $f \in \Gamma_P(X_P, \F|_P)$ with support $P$ also has support $P$. Indeed, suppose that $g \in \F(U)$ mapps to $f$ then $\supp{}{g}$ is closed and $X$ is noetherian so we can write,
\[ \supp{}{g} = Z_1 \cup \cdots Z_n \]
for closed irreducible subsets. If $P \notin Z_i$ then $U' = U \setminus Z_i$ is also an open neighborhood of $P$ and $g|_{U'}$ also maps to $f$ so because there are finitely many components we may assume that $P \in Z_i$ for each $i$. Furthermore because pullback preserves stalks, $\supp{}{f} \cap X_P = \supp{}{f} = \{ P \}$ and therefore $\supp{}{f}$ does not contain the generic point of any irreducible component containiing $P$ other than $\{ P \}$. Therefore, $\supp{}{g} = \{ P \}$ so the map is surjective. Therefore, by excision\footnote{We do not really need to envoke excision because if $g \in \F(U)$ is supported at a closed point $\{ P \}$ then it glues with $0$ on $X \setminus \{ P \}$ to give a global section supported at $P$.},
\[ \Gamma_P(X, \F) \iso \varinjlim_{U \ni P} \F(U) \iso \Gamma_P(X_P, \F|_P) \]
This gives a natural isomorphism $\Gamma_P(X,-) \iso \Gamma_P(X_P, (-)|_P)$. Then taking the derived functors $H^q(X, -) = R^q \Gamma_P(X, -)$ and $H^q(X, (-)|_P) = R^q \Gamma_P(X_P, (-)|_P)$ using that $(-)|_P$ is exact we get an isomorphism of $\delta$-functors,
\[ H^q(X, \F) \iso H^q(X_P, \F|_P) \]
natural in $\F$.

\subsubsection{2.6 ASK DANIEL ABOUT THIS}

\newcommand{\cR}{\mathcal{R}}
\newcommand{\sR}{\mathscr{R}}

\begin{defn}
For any open $U \subset X$ let $\underline{\Z}_U = j_!(\underline{\Z}|_U)$ where $j : U \embed X$ is the inclusion. We say a sheaf $\cR$ is \textit{shreak finitely-generated} if there is a finite collection $\{ U_i \}$ of open sets an a surjection of abelian sheaves,
\[ \bigoplus_{i = 1}^n \underline{\Z}_{U_i} \onto \cR \]
\end{defn}

\noindent
Let $X$ be a noetherian topological space and let $\{ \I_\alpha \}_{\alpha \in A}$ be a directed system of injective sheaves of abelian groups on $X$. 
\bigskip\\
I claim that $\I$ is injective if and only if for every open $U \subset X$ and subsheaf $\F \subset \Z_U$ and map $f : \F \to \I$ there exists an extension to $\Z_U \to \I$. 
Clearly, if $\I$ is injective the above property holds. Now, given this property we want to prove that $\I$ is injective. Consider an injection $\cA \embed \cB$ of sheaves and a map $f : \cA \to \I$. Let $\cP$ be the poset of subsheaves $\cA \subset \cA' \subset \cB$ such that there exists an extension $f' : \cA' \to \I$. We want to show that $\cB \in \cP$. It is clear that every chain in $\cP$ has a maximum via the union since the map to $\I$ is defined at each finite stage. Therefore, by Zorn's lemma there exists a maximal element $\cA' \in \cP$. Suppose that $\cA_M$ is a proper subsheaf of $\cB$. Then there exists a local section $s \in \cB(U)$ not in $\cA_M(U)$. Take the associated map $\underline{\Z}_U \to j_!(\cB|_U) \to \cB$ and consider the diaram,
\begin{center}
\begin{tikzcd}
\cR \arrow[d] \arrow[r, hook] & \underline{\Z}_U \arrow[d] \arrow[ddr, bend left, dashed]
\\
\cA_M \arrow[r, hook] \arrow[rrd, bend right] & \cB
\\
& & \I
\end{tikzcd}
\end{center}
where $\cR$ is the preimage of $\cA$ under $\underline{Z}|_U \to \cB|_U$ and $\underline{\Z}|_U \to \I|_U$ exists by assumption. 
and therefore let $\cA_M'$ be the pushout,
\begin{center}
\begin{tikzcd}
\cR \arrow[d] \arrow[r, hook] & \underline{\Z}_U \arrow[d] \arrow[ddr, bend left]
\\
\cA_M \arrow[r, hook] \arrow[rrd, bend right] & \cA_M' \arrow[rd, dashed]
\\
& & \I
\end{tikzcd}
\end{center}
so we see that $\cA_M' \in \cP$ but $\cA_M \subsetneq \cA_M'$ because on $U$ we have $1 \in \Z_U(U)$ maps to $s \in \cA''(U) \subset \cB(U)$ but $s \notin \cA'(U)$ contradicting the maximality of $\cA_M$. Therefore $\cA_M = \cB$ so $\cB \in \cP$ proving the claim. 


\begin{lemma}
If $X$ is a noetherian space then every subsheaf of $\underline{\Z}_U$ is shreak finitely-generated.
\end{lemma}

\begin{proof}
Since $\underline{\Z}_U = j_!(\Z|_U) \embed \underline{\Z}$ it suffices to prove the claim for subsheaves $\cR \subset \underline{\Z}$ of $\underline{\Z}$. For each $x \in U$ we know that $\cR_x \subset (\underline{\Z}_U)_x = \Z$ is a submodule and thus $\cR_x = (n_x)$ for some $n_x \in \Z$. Therefore, there exists an open $V_x \subset U$ and a section $s_x \in \cR(V_x)$ restricting to $n_x$ and thus generating the stalk.  
\end{proof}

\noindent
Now we prove the main result. Let $\{ \I_\alpha \}_{\alpha \in A}$ be a directed system of injective sheaves of abelian groups on $X$. By above, to show that,
\[ \F = \varinjlim_{\alpha} \I_\alpha \]
is injective it suffices to show that for every subsheaf $\cR \subset \underline{\Z}_U$ any map $j : \cR \to \F$ extends to $\underline{\Z}_U \to \F$. By the lemma, $\cR$ admits a surjection,
\[ \bigoplus_{i = 1}^n \underline{\Z}_{U_i} \onto \cR \]
and the maps $\underline{\Z}_{U_i} \to \F$ are given by sections $s_i \in \F(U_i)$ which factor through some $\I_{\alpha_i}(U_i) \to \F$ and therefore through some $\I_{\alpha}(U) \to \F$ because the system is directed and there are finitely many $s_i$. Thus 

\subsubsection{2.7 (CHECK OTHER PEOPLES SOLUTIONS)}

Let $X = S^1$ be the circle with its usual topology. Consider the exact sequence
\begin{center}
\begin{tikzcd}
0 \arrow[r] & \underline{\Z}_X \arrow[r] & \prod\limits_{x \in X} \iota_{x *}(\Q) \arrow[r] & \sQ \arrow[r] & 0
\end{tikzcd}
\end{center}
where the sheafm
\[ \I = \prod\limits_{x \in X} \iota_{x *}(\RR) \]
is injective because $\Q$ is an injective $\Z$-module and products and  pushforwards of injective sheaves are injective.
Then we get an exact sequence,
\begin{center}
\begin{tikzcd}
0 \arrow[r] & \Z \arrow[r] & \prod\limits_{x \in X} \RR \arrow[r] & H^0(X, \sQ) \arrow[r] & H^1(X, \underline{\Z}_X) \arrow[r] & 0
\end{tikzcd}
\end{center}
where $H^1(X, \I) = 0$ because $\I$ is injective. Therefore, it suffices to compute $H^0(X, \sQ)$. Global sections are given by tuples $\{(U_i, s_i)\}$ where $s_i \in \I(U_i)$ can be described as a (not necessarily continuous) function $s_i : U \to \RR$ such that $s_i - s_j \in \Z$ on the overlap. We can refine the cover to be a (finite by compactness of $S^1$) sequence of intervals with $1$-ply (meaning that only consecutive intervals overlap) and therefore going around the circle choosing representatives we see that,
\[ H^0(X, \sQ) \cong \{ f : \RR \to \RR \mid f(x + 1) = f(x) + n \text{ for } n \in \Z \} / \Z \]
Therefore,
\[ H^1(X, \underline{\Z}_X) = \coker{(H^0(X, \I) \to H^0(X, \sQ))} \cong \Z \]
given by sending $f \mapsto n$.
\bigskip\\
Likewise, let $\sR$ be the sheaf of continuous real functions on $X$. We embedd this into the sheaf of all real functions,
\[ \I = \prod\limits_{x \in S^1} \iota_{x *}(\RR) \]
which is injective to get an exact sequence,
\begin{center}
\begin{tikzcd}
0 \arrow[r] & \sR \arrow[r] & \I \arrow[r] & \sQ \arrow[r] & 0
\end{tikzcd}
\end{center}
Taking cohomology gives an exact sequence,
\begin{center}
\begin{tikzcd}
0 \arrow[r] & H^0(X, \sR) \arrow[r] & \prod\limits_{x \in X} \RR \arrow[r] & H^0(X, \sQ) \arrow[r] & H^1(X, \sR) \arrow[r] & 0
\end{tikzcd}
\end{center}
where $H^1(X, \I) = 0$ because $\I$ is injective. Therefore, to show that $H^1(X, \sR) = 0$ it suffices to show that every global section of $H^0(X, \sQ)$ is the class of some real function on $S^1$.  Global sections are given by tuples $\{(U_i, s_i)\}$ where $s_i \in \I(U_i)$ and $t_{ij} = s_i |_{U_i \cap U_j} - s_j |_{U_i \cap U_j} \in \sR(U_i \cap U_j)$ which automatically satisfy,
\[ t_{ij} |_{U_i \cap U_j \cap U_j} + t_{jk} |_{U_i \cap U_j \cap U_k} - t_{ik} |_{U_i \cap U_j \cap U_k} = 0 \]
Choose a partition of unity $\{ \varphi_i \}$ subordinate to $\{ U_i \}$. Since $\supp{}{\varphi_k} \subset U_k$ we see that $t_{ik} \varphi_k$ glues with $0$ on $\supp{}{\varphi_k}^C$ to give a global section $w_i$. Furthermore,
\[ t_{ij} = \sum_k v_{ij} \varphi_k |_{U_i \cap U_j} = \sum_k (t_{ik} - t_{jk}) \varphi_k |_{U_i \cap U_j} \]
since it is supported inside $U_i \cap U_j \cap U_k$. 
Then we have,
\[ (s_i - w_i)|_{U_i \cap U_j} - (s_j - w_j)|_{U_i \cap U_j)} = t_{ij} - \sum_k (t_{ik} - t_{jk}) \varphi_k |_{U_i \cap U_j} = 0 \]
and therefore the sections $s_i - w_i$ glue to give a global section $s : X \to \RR$. Furthermore, because the $w_i$ are continuous, $s$ maps to the class of $\{(U_i, s_i)\}$ since $w_i = s_i - s|_{U_i}$ is continuous. Therefore, $H^1(X, \sR) = 0$.


\subsection{Section 3}

\subsubsection{3.1}

Let $X$ be a noetherian scheme. If $X = \Spec{A}$ is affine then $X_\red = \Spec{A_\red}$ is clearly affine. Conversely, suppose that $X_\red = \Spec{A}$ is affine. There is a closed immersion $X_\red \embed X$ which sheaf of ideals $\sN$ which is coherent since $X$ is noetherian. Therefore, since $\sN$ is the sheaf of nilpotents as an ideal $\sN^{n+1} = 0$ for some $n$ because locally $\sN |_{\Spec{B}} = \wt{\nilrad{B}}$ which is finitely generated because $B$ is Noetherian. Therefore, for any quasi-coherent sheaf $\F$ there is a filtration,
\[ \F \supset \sN \cdot \F \supset \sN^2 \cdot \F \supset \cdots \supset \sN^n \cdot \F \supset \sN^{n+1} \cdot \F = 0 \]
let $\F_i = \sN^i \cdot \F$ then $\G_i = \F_i / \F_{i+1}$ satisfies $\sN \cdot \G_i = 0$. Since $\iota : X_\red \to X$ is a closed immersion $\iota_*$ induces an equivalence of categories between quasi-coherent $\struct{X_\red}$-modules and quasi-coherent $\struct{X}$-modules killed by $\sN$. Thus $\G_i = \iota_* \G_i'$ where $\G_i'$ is a $\struct{X_\red}$-module. Then $H^q(X, \G_i) = H^q(X, \iota_* \G_i') = H^q(X_\red, \G_i') = 0$ for $q > 0$ because $\G_i'$ is a quasi-coherent $\struct{X_\red}$-module and $X_\red$ is affine. Clearly $H^q(X, \F_{n+1}) = 0$. Now assume that $H^q(X, \F_{i+1}) = 0$ for $q > 0$. Using the exact sequence,
\begin{center}
\begin{tikzcd}
0 \arrow[r] & \F_{i+1} \arrow[r] & \F_{i} \arrow[r] & \G_i \arrow[r] & 0
\end{tikzcd}
\end{center}
we apply cohomology to find,
\begin{center}
\begin{tikzcd}
H^q(X, \G_i) \arrow[r] & H^{q+1}(X, \F_{i+1}) \arrow[r] & H^{q+1}(X, \F_{i}) \arrow[r] & H^{q+1}(X, \G_i) 
\end{tikzcd}
\end{center}
and thus $H^{q+1}(X, \F_{i+1}) \iso H^{q+1}(X, \F_{i})$ is an isomorphism for $q > 0$ and $H^1(X, \F_{i+1}) \onto H^1(X, \F_i)$ is a surjection. Therefore, $H^q(X, \F_i) = 0$ for $q > 0$ because $H^q(X, \F_{i+1}) \onto H^q(X, \F_{i})$ and $H^q(X, \F_{i+1}) = 0$ for $q > 0$. Thus $X$ is affine by Serre's criterion.

\subsubsection{3.2}

Let $X$ be a reduced noetherian scheme. Suppose that $X = \Spec{A}$ is affine. Then the irreducible components of $X$ are $\Spec{A/\p_i}$ for the minimal primes $\p_i \subset A$ which are affine. 
\bigskip\\
Conversely, suppose that each irreducible component $Y \subset X$ is affine. Since $X$ is Noetherian there are finitely many irreducible components $Y_i \subset X$. For any coherent sheaf of ideals $\I$ which corresponds to some closed subscheme $Z \subset X$ we want to show that $H^1(X, \I) = 0$. To do so, we proceed by descending induction on the number of irreducible components of $X$ contained in the support of $Z$. If $Z$ contains every component then $\I = (0)$ because $X$ is reduced and thus $H^1(X, \I) = 0$. Now, let $Y$ be an irreducible component not contained in $Z$ and consider the exact sequence,
\begin{center}
\begin{tikzcd}
0 \arrow[r] & \I_{Z \cup Y} \arrow[r] & \I_Z \arrow[r] & (\iota_{Y})_* \I_{Z \cap Y} \arrow[r] & 0
\end{tikzcd}
\end{center}
Because $Y_1$ is affine, $H^1(X, (\iota_{Y})_* \I_{Z \cap Y}) = H^1(Y, \I_{Z \cap Y}) = 0$ and thus the long exact sequence gives a surjection $H^1(X, \I_{Z \cup Y}) \onto H^1(X, \I_Z)$. However, $Z \cup Y$ contains more irreducible components of $X$ than $Z$ since $Y \not\subset Z$ so by the induction hypothesis $H^1(X, \I_{Z \cup Y}) = 0$. Therefore $H^1(X, \I_Z) = 0$ proving the result by induction. Since $H^1(X, \I) = 0$ for every coherent sheaf of ideals $\I$, we conclude that $X$ is affine by Serre's criterion. 
\bigskip\\
Here I give an alternative proof. Because $X$ is Noetherian, there are finitely many irreducible components $Z_i$. We proceed by induction on the number of irreducible components so assume the theorem for $r$ components and let $X$ have irreducible components $Z_1, \dots, Z_{r+1}$. 
If there is only one irreducible component then because $X$ is reduced, $X = Z$ and thus the statement is trivial. Now proceed by induction. Take any coherent $\struct{X}$-module $\F$ and consider the exact sequence,
\begin{center}
\begin{tikzcd}
0 \arrow[r] & \I_{Z} \cdot \F \arrow[r] & \F \arrow[r] & \F / \I_{Z} \F \arrow[r] & 0
\end{tikzcd}
\end{center}
where $Z \subset X$ is an irreducible component. By Lemma \ref{support_component_sheaf_ideal}, $\Supp{\struct{X}}{\I_Z \otimes \F} \subset X' = Z_1 \cup \cdots \cup Z_r$ where $Z_1, \dots, Z_r \subset X$ are the irreducible components besides $Z$ so $X'$ has $r$ components and $\I_{Z} \cdot \F$ is the pushforward of a $\struct{X'}$-module $\F'$ (possibly with nonreduced structure). In particular, $X'$ has the same $Z_1, \dots, Z_r$ irreducible components as $X$ (except for $Z$) and thus each is affine. Likewise, $\G = \F / \I_Z \F$ is anhilated by $\I_Z$ and thus $\F / \I_Z \F = \iota_* \iota^* \G$. 
Then taking the cohomology sequence,
\begin{center}
\begin{tikzcd}
H^q(X', \F') \arrow[r] & H^q(X, \F) \arrow[r] & H^q(Z, \G) 
\end{tikzcd}
\end{center}
By assumption, $Z$ is ample and $X'$ has $r$ irreducible componets all of which are affine so (perhaps after reducing $X'$) $X'$ by the induction hypothesis $X'$ is affine. Since $\F'$ and $\G$ are coherent we get vanishing $H^q(X', \F') = 0$ and $H^q(Z, \G) = 0$ for all $q > 0$.
Therefore, the exact sequence gives that $H^q(X, \F \otimes \L^{\otimes n}) = 0$ for all $q > 0$ proving that $X$ is affine by Serre's criterion. Thus the result holds for any number of irreducible components by induction.

\begin{lemma} \label{support_component_sheaf_ideal}
Let $X$ be a reduced scheme with finitely many irreducible components $Z_1, \dots, Z_r$ corresponding to quasi-coherent sheaves of ideals $\I_{Z_i}$. Then,
\[ X \setminus Z_i \subset \Supp{\struct{X}}{\I_{Z_{i}}} \subset \bigcup_{j \neq i} Z_j \]
\end{lemma}

\begin{proof}
If $x \notin Z$ then we know that $(\I_Z)_x = \stalk{X}{x}$ because $(\struct{X}/\I_Z)_x = 0$ proving the first inclusion.Notice that $\I_{Z_1} \cdots \I_{Z_{r+1}} \subset \I_{X} = (0)$ because $X$ is reduced. Therefore, if $x \in X \setminus \bigcup_{j \neq i} Z_j$ then $(\I_{Z_j})_x = \stalk{X}{x}$ for each $j \neq i$ and thus we must have $(\I_{Z_i})_x = 0$ for the relation to hold proving the complement of the second inclusion.
\end{proof}

\subsubsection{3.3}

Let $A$ be a noetherian ring and $\a \subset A$ an ideal. Let $X = \Spec{A}$ and $Y = V(\a)$.  

\begin{enumerate}
\item We know $\Gamma_\a(M) = \Gamma_Y(X, \wt{M})$ from (II.5.6) and therefore since $\wt{-}$ is exact and $\Gamma_Y(X, -)$ is left exact this shows that $\Gamma_\a(-)$ is left exact. Explicitly, let $\varphi : M \to N$ be a morphism of $A$-modules then $m \in \ker{(\varphi : \Gamma_\a(M) \to \Gamma_\a(N))}$ iff $\varphi(m) = 0$ and $\a^n m = 0$ for some $n > 0$ iff $m \in \Gamma_\a(\ker{\varphi})$. We denote the right dertived functors of $\Gamma_\a(-)$ by $H_\a^i(-)$.

\item Because $\Gamma_\a(-) = \Gamma_Y(X, \wt{-})$ and $\wt{-}$ takes injective modules to flasque sheaves since $A$ is noetherian and thus $H_\a^i(-) = R^i \Gamma_\a(-) = R^i \Gamma_Y(X, -)(\wt{-}) = H_Y^i(X, \wt{-})$ where the last equality follows from (3.6) showing that cohomology of quasi-coherent modules on noetherian schemes is computed as the derived functors of $\Gamma_Y$ on the category of coherent sheaves. 
\bigskip\\
Alternatively, because $\wt{-}$ is exact, the functors $H^q_Y(X, \wt{-})$ form a $\delta$-functor on $\Mod{A}$. Furthermore, $\Mod{A}$ has enough injectives and $\wt{I}$ is flasque since $A$ is noetherian so $H^q_Y(X, \wt{I}) = 0$ and thus $H^q_Y(X, \wt{-})$ is effacable so they form a universal $\delta$-functor. Furthermore, since $H^0_Y(X, \wt{-}) = \Gamma_Y(X, \wt{-}) = \Gamma_\a(-)$ we get a natural isomorphism $H^q_Y(X, \wt{-}) = R^q \Gamma_\a(-) = H^q_\a(-)$.
\bigskip\\
Alternatively, we can show this explicitly by induction and dimension shifitng. Let $M$ be an $A$-module and $M \embed I$ an embedding into an injective $A$-module. Then we find an exact sequence,
\begin{center}
\begin{tikzcd}
0 \arrow[r] & M \arrow[r] & I \arrow[r] & K \arrow[r] & 0
\end{tikzcd}
\end{center}
The long exact sequence gives,
\begin{center}
\begin{tikzcd}
0 \arrow[r] & \Gamma_\a(M) \arrow[r] & \Gamma_\a(I) \arrow[r] & \Gamma_\a(K) \arrow[r] & H^1_\a(M) \arrow[r] & 0
\\
& H^q_\a(I) \arrow[r] & H^q_\a(K) \arrow[r] & H^{q+1}_\a(M) \arrow[r] & H^{q+1}_\a(I)
\end{tikzcd}
\end{center}
and thus $H^q_\a(K) \iso H^{q+1}_\a(M)$ for $q > 0$. Furthermore, applying the exact functor $\wt{-}$ we get an exact sequence,
\begin{center}
\begin{tikzcd}
0 \arrow[r] & \wt{M} \arrow[r] & \wt{I} \arrow[r] & \wt{K} \arrow[r] & 0
\end{tikzcd}
\end{center}
which gives a long exact sequence of cohomology with supports,
\begin{center}
\begin{tikzcd}
0 \arrow[r] & \Gamma_Y(X, \wt{M}) \arrow[r] & \Gamma_Y(X, \wt{I}) \arrow[r] & \Gamma_Y(X, \wt{K}) \arrow[r] & H^1_\a(M) \arrow[r] & 0
\\
& H^q_Y(X, \wt{I}) \arrow[r] & H^q_Y(X, \wt{K}) \arrow[r] & H^{q+1}_Y(X, \wt{M}) \arrow[r] & H^{q+1}_Y(X, \wt{I})
\end{tikzcd}
\end{center}
using that $\wt{I}$ is flasque so its higher cohomology vanishes we see $H^q_Y(X, \wt{K}) \iso H^{q+1}_Y(X, \wt{M})$ for $q > 0$. Since $\Gamma_Y(X, \wt{-}) = \Gamma_\a(-)$ the cokernel sequences imply that $H^1_\a(M) = H^1_Y(X, \wt{M})$ for any $M$ proving our base case.
Now we assume for induction that $H^q_\a(-) = H^q_Y(X, \wt{-})$ for $q > 0$. Then we see,
\[ H^{q+1}_\a(M) = H^q_\a(K) = H^q_Y(X, \wt{K}) = H^{q+1}_Y(X, \wt{M}) \]
proving that $H^q_\a(M) = H^q_Y(X, \wt{M})$ for all $q \ge 0$ and all $M$ by induction.

\item First consider the case $i = 0$. For any $A$-module $M$, if $m \in \Gamma_\a(M)$ then $\a^n m = 0$ for some $m > 0$ so $m \in \Gamma_\a(\Gamma_\a(M))$ and $\Gamma_\a(N) \subset N$ for any $N$ meaning that $\Gamma_\a(\Gamma_\a(M)) = \Gamma_\a(M)$. Now, note that if $M$ has the property that $\Gamma_\a(M) = M$ and $\varphi : M \onto N$ then $\Gamma_\a(N) = N$ because for any $x \in N$ we can lift to some $m \in M$ and $\a^n m = 0$ for some $n > 0$ and thus $\a^n x = \a^n \varphi(m)= \varphi(\a^n x) = 0$. Therefore $\Gamma_\a(N) = N$. Now we proceed by induction and dimension shifting. Embed $M \embed I$ into an injective $A$-module $I$ giving an exact sequence,
\begin{center}
\begin{tikzcd}
0 \arrow[r] & M \arrow[r] & I \arrow[r] & K \arrow[r] & 0
\end{tikzcd}
\end{center}
The long exact sequence gives for any $q \ge 0$,
\begin{center}
\begin{tikzcd}
H^q_\a(I) \arrow[r] & H^q_\a(K) \arrow[r] & H^{q+1}_\a(M) \arrow[r] & H^{q+1}_\a(I)
\end{tikzcd}
\end{center}
but $H^{q+1}_\a(I) = 0$ since $I$ is injective and thus $H^q_\a(K) \onto H^{q+1}_\a(M)$. Therefore, if $\Gamma_\a(H^q_\a(K)) = H^q_\a(K)$ for any $A$-module $K$ then we see that $\Gamma_\a(H^{q+1}_\a(M)) = H^{q+1}_\a(M)$ so by induction $\Gamma_\a(H^{q}_\a(M)) = H^q_\a(M)$ for any $q \ge 0$ and any $A$-module $M$.
\end{enumerate}

\subsubsection{3.4 (CHECK!!)}

Let $A$ be a noetherian ring, $\a \subset A$ an ideal, and $M$ an $A$-module. 

\begin{enumerate}
\item 
If $M$ has an $M$-regular sequence $x_1 \in \a$ of length $1$ meaning $M \xrightarrow{x_1} M$ is injective and $M / x_1 M \neq 0$. Suppose that $m \in \Gamma_\a(M)$ then $\a^n m = 0$ so in particular $x_1^n m = 0$ but $M \xrightarrow{x_1} M$ is injective and so $M \xrightarrow{x_1^n} M$ is also injective showing that $m = 0$ so $\Gamma_\a(M) = 0$. 
\bigskip\\
Now let $M$ be finitely generated and assume that there does not exist a $M$-regular sequence in $\a$ then $\a$ is contained in the set of zero divisors on $M$ which is the union of the finitely many associated primes of $M$ since $M$ is finitely generated. By prime avoidance, $\a$ is contained in some associated prime $\p = \Ann{A}{m}$ meaning that $\a m = 0$ so $m \in \Gamma_\a(M)$ is nonzero and thus $\Gamma_\a(M) \neq 0$.

\item Let $M$ be finitely generated. We want to show that for any $A$-module $M$ and $n \ge 0$ the following are equivalent,
\begin{enumerate}
\item there exists a $M$-regular sequence in $\a$ of length $n$
\item $H^i_\a(M) = 0$ for all $i < n$
\end{enumerate}
We have shown this for $n = 1$. Now assume the equivalence for $n$. First, suppose there is a length $n$ regular sequence $x_1, \dots, x_{n+1} \in \a$ then,
\begin{center}
\begin{tikzcd}
0 \arrow[r] & M \arrow[r, "x_1"] & M \arrow[r] & M / x_1 M \arrow[r] & 0
\end{tikzcd}
\end{center}
and $M / x_1 M$ has a regular sequence in $\a$ of length $n$. Applying the long exact sequence,
\begin{center}
\begin{tikzcd}
H^i_\a(M / x_1 M) \arrow[r] & H^{i+1}_\a(M) \arrow[r, "x_1"] & H^{i+1}_\a(M) \arrow[r] & H^{i+1}_\a(M / x_1 M)
\end{tikzcd}
\end{center}
By the induction hypothesis $H^i_\a(M/x_1M) = 0$ for $i < n$ so the map $H^{i + 1}_\a(M) \xrightarrow{x_1} H^{i + 1}_\a(M)$ is injective. However, $\Gamma_\a(H^{i+1}_\a(M)) = H^{i+1}_\a(M)$ so for any $m \in H^{i+1}_\a(M)$ there is a $k > 0$ such that $\a^k m = 0$ and thus $x_1^k \cdot m = 0$ so $m = 0$ by injectivity. Therefore $H^{i}_\a(M) = 0$ for any $i < n + 1$ proving the second condition by induction.
\bigskip\\
Now suppose that $H^i_\a(M) = 0$ for $i < n + 1$. Since $\Gamma_\a(M) = 0$ we know there exists an $M$-regular element $x_1 \in \a$ such that the sequence,
\begin{center}
\begin{tikzcd}
0 \arrow[r] & M \arrow[r, "x_1"] & M \arrow[r] & M / x_1 M \arrow[r] & 0
\end{tikzcd}
\end{center}
is exact. Applying the long exact sequence we get,
\begin{center}
\begin{tikzcd}
H^{i}_\a(M) \arrow[r, "x_1"] & H^{i}_\a(M) \arrow[r] & H^{i}_\a(M / x_1 M) \arrow[r] & H^{i+1}_\a(M)
\end{tikzcd}
\end{center}
By the hypothesis we see $H^i_\a(M) = 0$ and $H^{i+1}_\a(M) = 0$ for $i < n$ meaning that $H^i_\a(M/x_1 M) = 0$ for $i < n$ so by the induction hypothesis $M / x_1 M$ has a regular sequence $x_2, \dots, x_{n+1} \in \a$ of length $n$. Therefore, $x_1, \dots, x_n$ is an $M$-regular sequence in $\a$ of length $n+1$. 
\end{enumerate}
\noindent
Therefore we can define $\depth{\a}{M} = \min \{ n \in \Z \mid H^n_\a(M) \neq 0 \}$. Then every $M$-regular sequence in $\a$ may be extended to a maximal sequence and all such maximal sequences have length $n$.

\subsubsection{3.5 CHECK!!}

Let $X$ be a noetherian scheme and $x \in X$ a closed point. We want to show the following are equivalent:
\begin{enumerate}
\item $\depth{\m_x}{\stalk{X}{x}} \ge 2$
\item if $U$ is any open neighborhood of $x$ then $\Gamma(U, \struct{X}) \to \Gamma(U \setminus \{ x \}, \struct{X})$ is an isomorphism.
\end{enumerate}
Let $Y = \{ x \} \subset U$ is closed and let $U^\times = U \setminus Y$ the punctured neighborhood. Applying the excision sequence (III.2.3 (e)) for cohomology with supports,
\begin{center}
\begin{tikzcd}
0 \arrow[r] & H^0_Y(U, \struct{U}) \arrow[r] & H^0(U, \struct{U}) \arrow[r] & H^0(U^\times, \struct{U^\times}) \arrow[r] & H^1_Y(U, \struct{U})
\end{tikzcd}
\end{center}
so we need to show that $H^i_Y(U, \struct{U}) = 0$ for $i = 0, 1$ in order to show that $H^0(U, \struct{U}) \iso H^0(U^\times, \struct{U})$ is an isomorphism. Let $V = \Spec{A}$ be an affine open neighborhood of $x = \p \in \Spec{A}$ then $Y = V(\p)$.  Applying excision for cohomology with supports (III.2.3 (f)),
\[ H^i_Y(U, \struct{U}) \cong H^i_Y(V, \struct{V}) = \varinjlim_{x \in V} H^i_Y(V, \struct{V}) = \varinjlim_{f \in A \setminus \p} H^i_{\p}(A_f) = H^i_{\p}(A_\p) = H^i_{\m_x}(\stalk{X}{x}) \]
Therefore, if $\depth{\m_x}{\stalk{X}{x}} \ge 2$ then $H^i_Y(U, \struct{U}) = H^i_{\m_x}(\stalk{X}{x}) = 0$ for $i < 2$ proving the required statement.
\bigskip\\
Conversely suppose that $\Gamma(U, \struct{X}) \to \Gamma(U \setminus \{ x \}, \struct{X})$ is an isomorphism for any open neighborhood. In paricular, choose $U = \Spec{A}$ to be an affine open neighborhood of $x = \p \in \Spec{A}$. Applying the excision sequence (III.2.3 (e)) for cohomology with supports,
\begin{center}
\begin{tikzcd}
0 \arrow[r] & H^0_Y(U, \struct{U}) \arrow[r] & H^0(U, \struct{U}) \arrow[r] & H^0(U^\times, \struct{U^\times}) \arrow[r] & H^1_Y(U, \struct{U}) \arrow[r] & H^1(U, \struct{U})
\end{tikzcd}
\end{center}
but $H^0(U, \struct{U}) \to H^0(U^\times, \struct{U^\times})$ is an isomorphism and $U$ is affine so $H^1(U, \struct{U}) = 0$ and thus $H^i_Y(U, \struct{U}) = 0$ for $i = 0,1$. Applying excision for cohomology with supports (III.2.3 (f)),
\[ H^i_Y(U, \struct{U}) \cong \varinjlim_{x \in V} H^i_Y(V, \struct{V}) = \varinjlim_{f \in A \setminus \p} H^i_{\p}(A_f) = H^i_{\p}(A_\p) = H^i_{\m_x}(\stalk{X}{x}) \]
Therefore, $H^i_{\m_x}(\stalk{X}{x}) = H^i_Y(U, \struct{U}) = 0$ for $i < 2$ proving that $\depth{\m_x}{\stalk{X}{x}} = \depth{\p}{A_\p} \ge 2$


\subsubsection{3.6 CHECK!!}

Let $X$ be a noetherian scheme and choose a finite cover $U_i = \Spec{A_i}$ of noetherian affine opens.

\begin{enumerate}
\item Let $\F$ be a quasi-coherent $\struct{X}$-module. Then $\F|_{U_i} = \wt{M_i}$ for some $A_i$-module $M_i$. Embed $M_i \embed I_i$ where $I_i$ is an injective $A_i$-module. Let $j_i : U_i \embed X$ be the open inclusion and define,
\[ \G = \bigoplus_{i = 1}^n (j_i)_*(\wt{I}_i) \]
The natural map $\F \to \G$ is injective because for any $x \in X$ there is some $i$ such that $x \in U_i$ and $\F_x \to \G_x$ is $(M_i)_x \embed (I_i)_x$ in the $i$-component which is injective. Since $X$ is Noetherian $j$ is quasi-compact and quasi-separated ($U$ is retrocompact) so $f_*(\wt{I}_i)$ is quasi-coherent and the finite sum of quasi-coherent modules is quasi-coherent so $\G$ is quasi-coherent. 
\bigskip
\\
Furthermore, $(j_i)_*$ is right adjoint to $(j_i)^* = (j_i)^{-1}$ which is exact because $j_i$ is an open immersion. Therefore, $j_i$ preserves injective quasi-coherent modules. However, since $I_i$ is injective and there is an equivalence of categories between $A_i$-modules and quasi-coherent $\struct{U_i}$-modules we see that $\wt{I}_i$ is injective in the category of quasi-coherent $\struct{U_i}$-modules. Therefore, $f_*(\wt{I}_i)$ is injective in the category of quasi-coherent $\struct{X}$-modules. Furthermore, the direct sum of injectives is injective so $\G$ is injective in $\QCoh{X}$ proving that $\QCoh{X}$ has enough injectives. (CHECK!!)


Furthermore, let $\M \embed \sN$ be an injection of quasi-coherent $\struct{X}$-modules and suppose there is a map $\M \to \G$. Then locally $\M|_{U_i} = \wt{M_i}$ and $\sN|_{U_i} = \wt{N_i}$ and there is an injection $M_i \embed N_i$ 

\item Let $\I \in \QCoh{X}$ be injective and $U \subset X$ an open where $j : U \to X$ is the inclusion which is quasi-compact and quasi-separated since $X$ is noetherian. Let $\M, \sN \in \QCoh{U}$ be quasi-coherent $\struct{U}$-modules with an injection $\M \embed \sN$ and given a map $\M \to \I|_U$. Then $\iota_* \M \embed \iota_* \sN$ is injective and both are quasi-coherent $\struct{X}$-modules (since $U$ is retrocompact). By quotienting $\M \subset \sN$ by the kernel of $\M \to \I|_U$ we can reduce to the case that $\M \to \I|_U$ is injective. Now view $\M \subset \I|_U$ as a submodule. Then by (II.5.15) there exists a quasi-coherent $\struct{X}$-submodule $\M' \subset \I$ such that $\M|_U = \M$ and a quasi-coherent $\struct{X}$-module $\sN'$ such that $\M' \subset \sN'$ and $\sN'|_U = \sN$. Thus we have a diagram,
\begin{center}
\begin{tikzcd}
\M' \arrow[r, hook] \arrow[rd, hook] & \sN' \arrow[d, dashed]
\\
& \I 
\end{tikzcd}
\end{center}
restricting to $U$ we get a diagram,
\begin{center}
\begin{tikzcd}
\M \arrow[r, hook] \arrow[rd, hook] & \sN \arrow[d, dashed]
\\
& \I|_U
\end{tikzcd}
\end{center}
and therefore $\I|_U$ is injective. In particular, $\I|_{U_i} = \wt{I}_i$ where $I_i$ is a quasi-coherent $A_i$-module since $\I|_{U_i}$ is an injective quasi-coherent $\struct{U_i}$-module and the category of quasi-coherent $\struct{U_i}$-modules is equivalent to the category of $A_i$-modules. By (3.4) $\wt{I}_i$ is flasque.
\bigskip\\
To show that $\I$ is flasque, it suffices to show that $\res : \I(X) \to \I(U)$ is surjective. Consider the filtration,
\[ \tilde{U}_i = U \cup \bigcup_{j = 1}^i U_i \]
with $\tilde{U}_0 = U$ and $\tilde{U}_n = X$.
Take a section $s_0 \in \I(U) = \I(\tilde{U}_0)$. For induction, let $s_i \in \I(\tilde{U}_i)$ be a section over $\tilde{U}_i$ such that $s_i |_{U} = s_0$. Since $\I|_{U_{i+1}} = \wt{I}_{i+1}$ is flasque,
\[ \res : \I(U_{i+1}) \to \I(\tilde{U}_i \cap U_{i+1}) \]
is surjective and thus we can lift to $s_i' \in \I(U_{i+1})$ such that $s_i' |_{\tilde{U}_i \cap U_{i+1}} = s_i |_{\tilde{U} \cap U_{i+1}}$ therefore we can glue to get a section $s_{i+1} \in \I(\tilde{U}_{i+1})$ such that $s_{i+1}|_{\tilde{U}_i} = s_i$ and $s_{i+1}|_{U_{i+1}} = s_i'$ and $s_{i+1}|_U = s_{i}|_U = s_0$. Thus, by induction, we get a section $s_n \in \I(X)$ such that $s |_U = s_0$ so $\I$ is flasque.

\item Let $\iota : \QCoh{X} \embed \Sh{(X)}$ be the inclusion of categories from quasi-coherent $\struct{X}$-modules to abelian sheaves on $X$. Then there is a diagram of functors,
\begin{center}
\begin{tikzcd}
\QCoh{X} \arrow[dr, "\Gamma'"'] \arrow[rr, "\iota"] & & \Sh{(X)} \arrow[dl, "\Gamma"]
\\
& \Ab
\end{tikzcd}
\end{center}
Then since $\iota$ takes injectives to flasques which are $\Gamma$-acyclic, there is a Grothendieck spectral sequence $E^{p,q}_2 = R^p \Gamma \circ R^q \iota \implies R^{p+q} \Gamma'$ but $R^p \Gamma = H^p(X, -)$ and $R^0 \iota = \iota$ and $R^q \iota = 0$ for $q > 0$ because $\iota$ is exact. Therefore, $H^p(X, -) = R^p \Gamma'(X, -)$.
\bigskip\\
Alternatively, we compute the derived functors of $\Gamma'$ on $\QCoh{X}$ applied to $\F$ by taking an injective resolution in $\QCoh{X}$,
\begin{center}
\begin{tikzcd}
0 \arrow[r] & \F \arrow[r] & \I^0 \arrow[r] & \I^2 \arrow[r] & \cdots 
\end{tikzcd}
\end{center}
then applying $\iota$ gives a flasque resolution of $\iota(\F)$ in $\Sh{(X)}$ because $\iota$ is exact. Therefore,
\[ H^p(X, \iota(\F)) = H^p(\Gamma(X, \iota(\I^\bullet))) = H^p(\Gamma(X, \I^\bullet)) \]
so we can compute abelian sheaf cohomology of $\iota(\F)$ (i.e. of $\F$ viewed in $\Sh{(X)}$) via taking injective resolutions in $\QCoh{X}$. 
\end{enumerate}

\subsubsection{3.7 DO!! (UGH!)}

Let $A$ be a noetherian ring, $X = \Spec{A}$, $\a \subset A$ an ideal, and let $U \subset X$ be the open $X \setminus V(\a)$. 
\begin{enumerate}
\item Let $M$ be an $A$-module. The maps in question $\psi_n : \Hom{A}{\a^n}{M} \to \Gamma(U, \wt{M})$ are defined via,
\[ \Hom{A}{\a^n}{M} = \Hom{A}{\wt{a}^n}{\wt{M}} \to \Hom{A}{\wt{\a}^n|_U}{\wt{M}|_U} = \Gamma(U, \wt{M}) \]
because $\a|_U = \struct{U}$ since $U = X \setminus V(\a)$. We can explicitly describe this map as follows. The map $\psi_n$ sends $\varphi : \a^n \to M$ to the section $s \in \Gamma(U, \wt{M})$ defined over the principal open $D(f) \subset U$ for $f \in \a$ by,
\[ s|_{D(f)} = \varphi_{f}(1) = \varphi_{f}(f^n/f^n) = \varphi(f)^n / f^n \] 
Now we consider,
\[ \psi : \varinjlim_n \Hom{A}{\a^n}{M} \to \Gamma(U, \wt{M}) \]
First we prove injectivity of $\psi$. Because $A$ is Noetherian, $a = (f_1, \dots, f_r)$ is finitely generated. Suppose that $\varphi_n(\psi) = 0$. Then $\psi(f_i^n) = 0$ in $M_{f_i}$ for each $i$ and therefore for some $m_i$ we have $f_i^{m_i} \psi(f_i^n) = 0$ in $M$ for all $i$. Since $r$ is finite we can choose a largest $m$ then $\psi(f_i^{n+m}) = f_i^m \psi(f_i^n) = 0$ in $M$ for all $i$. Therefore, for $n' > r(n + m)$ this implies that $\psi|_{\a^{n'}} = 0$ and therefore $\psi$ is zero in the colimit.
\bigskip\\ 
Now we consider surjectivity. 

(HOW THE HELL DOES THIS WORK)



\item Let $I$ be an injective $A$-module. Then for any open $U \subset X$ the complement $X \setminus U$ is closed and thus $X \setminus U = V(\a)$ for some ideal $\a$. Then consider,
\[ \Gamma(U, \wt{I}) = \varinjlim_{n} \Hom{A}{\a^n}{I} \]
and the map $\Gamma(X, \wt{I}) \to \Gamma(U, \wt{I})$ is given by,
\[ I \to \varinjlim_{n} \Hom{A}{\a^n}{I} \]
defined by $\Hom{A}{A}{I} \to \Hom{A}{\a^n}{I}$ from $\a^n \embed A$. However, since $I$ is injective the map $I = \Hom{A}{A}{I} \to \Hom{A}{\a^n}{I}$ is surjective meaning that $\Gamma(X, \wt{I}) \to \Gamma(U, \wt{I})$ is surjective so $\wt{I}$ is flasque.
\end{enumerate}

\subsubsection{3.8}

Let $A = k[x_0, x_1, x_2, \dots ]$ with relations $x_0^n x_n = 0$ for each $n$. Now let $I$ be an injective $A$-module and $A \embed I$ an injective map. Consider the map $I \to I_{x_0}$. If we assume this is surjective then $\frac{1}{x_0}$ must have a preimage $m \in I$. Therefore, $m = \frac{1}{x_0}$ so there exists some $n$ such that $x_0^n(x_0 m - 1) = 0$ in $I$. Then $x_{n+1} x_0^n(x_0 m - 1) = 0$ but $x_{n+1} x_0^{n+1} = 0$ and therefore $x_{n+1} x_0^n = 0$ in $I$ contracting the fact that $A \embed I$ is injective. Therefore $I \to I_{x_0}$ cannot be surjective.



\subsection{9}

\subsubsection{9.1}

Let $f : X \to Y$ be a flat finite type morphism of noetherian schemes. Let $U \subset X$ be open.
\bigskip\\
First, I claim that $f(U)$ is stable under generalization. Because $\Spec{\stalk{X}{x}} \subset U$, it suffices to show that the local map $\Spec{\stalk{X}{x}} \to \Spec{\stalk{Y}{f(x)}}$ is surjective. Thus we need to show that if $\varphi : A \to B$ is a flat local finite type ring map then $\Spec{B} \to \Spec{A}$ is surjective.
\bigskip\\
Indeed, $\varphi$ is faithfully flat. If $M$ is an $A$-module such that $M \otimes_A B = 0$ then for every finitely generated submodule $M' \subset M$ we have $M' \otimes_A B \subset M \otimes_A B = 0$ (injective by flatness). Consider the injection of fields $\kappa_A \embed \kappa_B$. Since $M ' \otimes_A \kappa_A$ is a flat $\kappa_A$-module ($\kappa_A$ is a field) we get an injection,
\[ M' \otimes_A \kappa_A \embed M' \otimes_A \kappa_B = (M' \otimes_A B) \otimes_B \kappa_B = 0 \]
and therefore $M' \otimes_A \kappa_A = 0$ and thus $M' = 0$ by Nakayama. Therefore $M = 0$ so $\varphi$ is fathfully flat. Now for any $\p \in \Spec{A}$ we know that $A_\p / \p A_\p \neq 0$ so $A_\p / \p A_\p \otimes_A B \neq 0$ by faithful flatness and therefore $\Spec{A_\p / \p A_\p \otimes_A B}$ is nonempty proving that the fiber over $\p$ is nonempty so $\Spec{B} \to \Spec{A}$ is surjective.
\bigskip\\
Thus the image $f(U)$ is stable under generalization. However, because $f$ is locally of finite presentation, the image of constructible sets is constructible. We conclude that $f(U)$ is open as follows. Let $C$ be a constructible set that is stable under generalization. Then,
\[ C = \bigcup_{i = 1}^n U_i \cap Z_i \] 
for open $U_i$ and closed $Z_i$. If $x \in C$ then $x \in U_i \cap Z_i$ for some $Z_i$. By stability under generalization, the generic point $\xi \in C$ of some irreducible component $Z \subset Y$ lies in $C$ and thus in some $Z_i$ meaning that $Z_i = Z$. Then,
\[ x \in U_i \cap Z \subset C \]
but if $U_i \subset Z$ then $U_i$ is an open neighborhood of $x$ and otherwise we can shrink $U_i$ such that it is contained in a single irreducible component by subtracting the other components (of which there are finitely many and they are all closed). Thus $C$ is open.

 
\subsubsection{9.2}

Let $X_1 \subset \P^3_k$ be the twisted cubic defined parametrically as,
\[ [s : t] \mapsto [t^3 : t^2 s : t s^2 : s^3] \]
Let $P = [0 : 0 : 1 : 0]$ then the projection of $X$ away from $P$ will turn out to be a cuspidal cubic plane curve. 
First we need to find equations for $X$. Write $\P^3_k = \Proj{k[x,y,z,w]}$ and the projection is defined by $[x:y:z:w] \mapsto [x:y:w]$. We only are worrying about the behavior near the cusp so we work in the affine patch $\{ w \neq 0 \}$ which means also that $s \neq 0$ so we get a parametric equation in $\A^3$,
\[ t \mapsto (t^3, t^2, t) \]
Now we modify this equation giving the schemes $X_a$ via,
\[ t \mapsto (t^3, t^2, at) \]
We need to compactify $\overline{X} \to \A^1$ over the point $a = 0$ by producting an ideal $I \subset k[x,y,z,a]$ such that the quotient is flat over $k[a]$ and has the correct fibers. To do this, first notice that for $a \neq 0$
\[ x = (z/a)^3 \quad y = (z/a)^2 \]
and therefore,
\[ a^3 x = z^3 \quad a^2 y = z^2 \] 
However the ideal,
\[ I' = (a^3 x - z^3, a^2 y - z^2) \]
is not the correct ideal because $a^2 (ax - yz) = 0$ and therefore $k[x,y,z,a]/I'$ is not a torsion-free $k[a]$-module.
We want an ideal $I$ containing $I'$ so that $k[x,y,z,a]/I$ is flat over $k[a]$. By the slicing criterion for flatness (and that $k[a]/(a)$ is a field) and the fact that flat modules are torsion-free, it is equivalent that $a$ should be a non-zero divisor on $k[x,y,z,a]/I$.
 Therefore, we need the ideal,
\[ I = (I' : a^\infty) = \bigcup_{n \ge 0} \{ f \in k[x,y,z,a] \mid a^n f \in I_0 \} \]
There is an algorithm using Gr\"{o}bner bases to produce generators for this ideal but I am too lazy to do it myself. Sage tells me this ideal is,
\[ I = (y^2 - x^3, ax - yz, y^2 a - xz, x a^3 - z^3, y a^2 - z^2) \]
Therefore, taking the fiber $X_0$ over $a = 0$ corresponds to,
\[ (k[x, y, z, a]/I) \ot_{k[a]} k[a]/(a) = k[x,y,z]/I_0 \]
where $I_0$ is the ideal setting $a = 0$,
\[ I_0 = (y^2 - x^3, yz, xz, z^3, z^2) = (y^2 - x^3, yz, xz, z^2) = (y^2 - x^3) + (z) \cdot (x,y,z) \]
This is the ideal for a cuspidal cubic curve with an embedded point at the cusp pointing in the $z$-direction given by the nilpotent element $z$. To see why $\p = (x,y,z)$ is actually an embedded point, notice that the maximal ideal of $\stalk{X_0}{\p}$ is generated by $x,y,z$ of which every element is a zero divisor because $xz = yz = z^2 0$ and therefore $\p$ is an associated point of $X_0$ but not a minimal prime because $(z)$ is prime. However, at every other point $\p \in X$ of this scheme either $x \notin \p$ or $y \notin \p$ so localizing away from $\p$ kills $z$ so $\stalk{X_0}{\p}$ is reduced. 


\subsubsection{9.3 IS THIS DONE??}

\begin{enumerate}
\item Because $f : X \to Y$ is finite and surjective it is quasi-finite with nonempty fibers and thus every fiber has dimension zero. Therefore flatness of $f$ follows by miracle flatness.
\item Let,
\[ B = k[x,y,z,w]/(z,w) \cap (x+z,y+w) = k[x,y,z,w]/(z(x+z), z(y+w), w(x+z), w(y+w)) \]
and 
\[ A = k[x,y] \]
\begin{rmk}
Notice that intersection of ideals and tensor product does not play nicely. From the explicit description,
\[ B \otimes_{A} A / \m_0 = k[z,w]/(z^2, zw, y^2) \]
while it might seem like the ideal for setting $x=y=0$ should be $(z,w) \cap (z,w) = (z,w)$.
\end{rmk}
\noindent
Then consider $f : \Spec{B} \to \Spec{A}$. I claim that $f$ is not flat at the origin. Then we apply the following lemma.

\begin{lemma}
If $f : A \to B$ is a flat morphism of rings and $I \subset A$ is an ideal. Then the quotient map $f : A/I \to B/IB$ is flat. 
\end{lemma}

\begin{proof}
For any $A/I$-module $M$ we have,
\[ M \otimes_{A/I} B/IB = M \otimes_{A/I} A/I \otimes_A B = M_A \otimes_A B \]
but restriction $M \mapsto M_A$ is exact and $- \otimes_A B$ is exact so $f : A/I \to B/IB$ is flat.
\end{proof}
\noindent
We apply this to $I = (y) \subset A$ then we see that,
\[ k[x] \to k[x,z,w]/(z(x+z), zw, wx, wy) \]
should be flat. However, $xw = 0$ in this ring so the map makes the target not a torsion-free module and therefore it cannot be flat. Therefore $f$ is not flat. We can give an alternative argument as follows. Notice that,
\[ A \cong k[x,y] \times_{k} k[z,w] = \{ (f,g) \in k[x,y] \times k[z,w] \mid f(0,0) = g(0,0) \} \]
Therefore, there is an exact sequence of $B'$-modules,
\begin{center}
\begin{tikzcd}
0 \arrow[r] & A \arrow[r] & B^{\oplus 2} \arrow[r] & k \arrow[r] & 0
\end{tikzcd}
\end{center}
From the long exact sequence for $\Tor{i}{B}{k}{-}$ to get an exact sequence,
\begin{center}
\begin{tikzcd}
\Tor{2}{B}{k}{B^{\oplus 2}} \arrow[r] & \Tor{2}{B}{k}{k} \arrow[r] & \Tor{1}{B}{k}{A} \arrow[r] & \Tor{1}{B}{k}{B^{\oplus 2}} 
\end{tikzcd}
\end{center}
but $B^{\oplus 2}$ is free and thus flat and therefore,
\[ \Tor{1}{B}{k}{A} \cong \Tor{2}{B}{k}{k} \cong k \]
Then the Kozul complex $K_\bullet$ is exact giving a free resolution,
\begin{center}
\begin{tikzcd}
0 \arrow[r] & B \arrow[r] & B^{\oplus 2} \arrow[r] & B \arrow[r] & k \arrow[r] & 0
\end{tikzcd}
\end{center}
which implies that,
\[ \Tor{2}{B}{k}{k} = H_2(K_\bullet \otimes_B k) = k \]
Therefore, $A$ is not a flat $B$-module.

\item Now consider,
\[ A = k[x,y,z,w]/(z^2, zw, w^2, xz-yw) \quad \text{ and } \quad B = k[x,y] \]
and $f : X \to Y$ where $X = \Spec{A}$ and $Y = \Spec{B}$ via $B \to A$. Notice that $A_{\red} = A/(z,w) = k[x,y] \cong B$ so $X_{\red} \to Y$ is an isomorphism. By the same argument as before, if $f$ were flat we would conclude that,
\[ k[x] \to k[x,z,w]/(z^2, zw, w^2, xz) \]
is flat but $x$ is a zero divisor on the right so it cannot be flat because it is not torsion-free. We just need to show that $X$ has no embedded points. We consider the open cover $D(x), D(y)$ of the complement of the origin. Then,
\begin{align*}
D(x) & \cong k[x,y,w]/(w^2)
\\
D(y) & \cong k[x,y,z](z^2)
\end{align*}
have no embedded points. Therefore, we just need to consider the origin. I claim that $x \in \m_0 A_{\m_0}$ is not a zero divisor and thus the origin is not an associated point. This is because $x \in (A_{\m_0})_{\red} \cong k[x,y]_{\m_0}$ is not a zero divisor (CHECK THIS!!)
\end{enumerate}


\subsubsection{9.4 DO THIS!! MOTHER FUCKER}



\subsubsection{9.5 TALK TO DANIEL!!}

Before we get started, I want to make some remarks on the ideals of fibers of a family of closed subschemes of $\P^n$. 
\bigskip\\
Let $A$ be a noetherian domain\footnote{We need $A$ to be a noetherian domain to conclude flatness and freeness properties from constancy of numerical functions in [III, Theorem 9.9] and Grauert [III, Theorem 12.9].} and $X \subset \P^n_A$ be a closed subscheme cut out by a saturated ideal $I \subset S$ with $S = A[x_0, \dots, x_n]$. Then there the following problem relies on the relationship between the following ideals over $t \in \Spec{A}$. Let $\I = \wt{I}$ be the sheaf of ideals. The ideal $I_t$ is the homogeneous ideal of $X_t \subset \P^n_t$ which I take to mean the saturated ideal,
\[ I_t = \Gamma_*(\P^n_t, \I_t) \]
where $\I_t = \im{(\iota_t^* \I \to \struct{X_t})}$ is the ideal of $\iota_t : X_t \subset \P^n_t$. Therefore, we have an exact sequence,
\begin{center}
\begin{tikzcd}
0 \arrow[r] & \I_t \arrow[r] & \struct{\P^n_t} \arrow[r] & (\iota_t)_* \struct{X_t} \arrow[r] & 0
\end{tikzcd}
\end{center}
and therefore an exact sequence,
\begin{center}
\begin{tikzcd}
0 \arrow[r] & (I_t)_d \arrow[r] & (S_t)_d \arrow[r] & H^0(X_t, \struct{X_t}(d)) \arrow[r] & H^1(\P^n_t, \I_t(d))
\end{tikzcd}
\end{center}
which implies that,
\[ H^0(X_t, \struct{X_t}(d)) = (S_t/I_t)_d \]
for $d \gg 0$. In fact, when $X_t$ is projectively normal, we have seen that [II, Ex. 5.14] the equality holds for all $d \ge 0$.
\bigskip\\
Furthermore, we have the ideal $I_{\kappa(t)} = \im{(I \ot_A \kappa(t) \to S_t)}$ where $S_t = S \ot_A \kappa(t) = \kappa(t)[x_0, \dots, x_n]$. Notice that this is the image of the map,
\[ \Gamma_*(\P^n_A, \I) \ot_A \kappa(t) \to \Gamma_*(\P^n_A, \struct{\P^n_A}) \ot_A \kappa(t) \]
However, given a map $f : X \to Y$ there is a natural map $(R^q f_* (-))_y \ot \kappa(y) \to H^q(X_y, (-)_y)$ giving a diagram (because in our case $Y = \Spec{A}$ is affine so $R^q f(-) = \wt{H^q(X, -)}$),
\begin{center}
\begin{tikzcd}
\Gamma_*(\P^n_A, \I) \ot_A \kappa(t) \arrow[d] \arrow[r] & \Gamma_*(\P^n_A, \struct{\P^n_A}) \ot_A \kappa(t) \arrow[d, equals]
\\
\Gamma_*(\P^n_t, \iota_t^* \I) \arrow[r] & \Gamma_*(\P^n_t, \struct{\P^n_t})
\end{tikzcd}
\end{center}
Now we assume that $X \to \Spec{A}$ is flat. Therefore, $\iota_* \struct{X}$ is a flat $A$-module so the pullback sequence is exact so $\iota_t^* \I \iso \I_t$ is an isomorphism. Therefore, the above diagram becomes,
\begin{center}
\begin{tikzcd}
I \ot_A \kappa(t) \arrow[d] \arrow[r] & S_t \arrow[d, equals]
\\
I_t \arrow[r, hook] & S_t
\end{tikzcd}
\end{center}
where the bottom map is injective by left exactness of $\Gamma_*$. Therefore we see that $I_{\kappa(t)} = \im{(I \ot_A \kappa(t) \to S_t)}$ is contained in $I_t$. Furthermore, by [III, Theorem 9.9] the Hilbert polynomial $P_t$ is constant. Then for $d \gg 0$,
\[ P_t(d) = \dim_{\kappa(t)} H^0(X_t, \struct{X_t}(d)) = \dim_{\kappa(t)} (S_t / I_t)_d \]
is constant and because $(S_t)_d$ is constant in $t$ for any $d$ we see that,
\[ h^0(X_t, \I_t(d)) = \dim_{\kappa(t)} H^0(X_t, \I_t(d)) = \dim_{\kappa(t)} (I_t)_d \]
is constant for $d \gg 0$. Because of the exact sequence of sheaves,
\begin{center}
\begin{tikzcd}
0 \arrow[r] & \I(d) \arrow[r] & \struct{\P^n_A}(d) \arrow[r] & \iota_* \struct{X}(d) \arrow[r] & 0
\end{tikzcd}
\end{center}
we see by [III, Prop. 9.1A(e)] that $\I(d)$ is $A$-flat. Then by Grauert [III Theorem 12.9], for all $d \gg 0$ we conclude that $I_d$ is $A$-locally free and,
\[ (I_{\kappa(t)})_d = I_d \ot_A \kappa(t) \iso (I_t)_d \]
is an isomorphism. Therefore, $I_{\kappa(t)}$ and $I_t$ have the same saturation (see the lemma) so $I_t = (I_{\kappa(t)})^{\text{sat}}$.
\bigskip\\
Now suppose the natural map on the left is an isomorphism then the top map is also injective. Therefore, from the exact sequence of $A$-modules,
\begin{center}
\begin{tikzcd}
\Tor{A}{1}{S_d}{\kappa(t)} \arrow[r] & \Tor{A}{1}{(S/I)_d}{\kappa(t)} \arrow[r] & I_d \ot_A \kappa(t) \arrow[r] & S_t \arrow[r] & (S_t/I_{\kappa(t)})_d \arrow[r] & 0
\end{tikzcd}
\end{center} 
but $S_d$ is a finite free $A$-module so $\Tor{A}{1}{S_d}{\kappa(t)} = 0$ and we showed that the map $I \ot_A \kappa(t) \embed S_t$ is injective showing that $\Tor{A}{1}{(S/I)_d}{\kappa(t)} = 0$. Therefore, by \chref{https://stacks.math.columbia.edu/tag/00MK}{Tag 00MF} we see that $(S/I)_d$ is flat over $A$ and therefore,
\[ S/I = \bigoplus_{d \ge 0} (S/I)_d \]
is flat over $A$. In this case we see that $I_{\kappa(t)} = I_t$ and $S/I$ is flat over $A$.
\bigskip\\
In particular, if the family is very flat meaning that $\dim_{\kappa(t)} (S_t/I_t)_d$ is constant in $t$ for all $d$ then because $\dim_{\kappa(t)} (S_t)_d$ is automatically constant in $t$ we see that,
\[ h^0(X_t, \I_t(d)) = \dim_{\kappa(t)} (I_t)_d \]
is constant in $t$ for all $d$ and therefore by Grauert [III Theorem 12.9], we see that
\[ (I_{\kappa(t)})_d = I_d \ot_A \kappa(t) \iso (I_t)_d \]
for all $d$ so we are in the above case and $I_d$ for all $d$, and hence also $I$, are $A$-locally free. We collect the following results.

\begin{prop}
Let $X \subset \P^n_A$ be a closed subscheme flat over $\Spec{A}$. For all $t \in \Spec{A}$ the ideal $I_t$ is the saturation of $I_{\kappa(t)}$ so $I_{\kappa(t)} \subset I_t$ with equality in $d \gg 0$. Additionally, the following are equivalent,
\begin{enumerate}
\item $X$ is very flat over $A$
\item $I \ot_A \kappa(t) \to I_t$ is an isomorphism  
\item $I_{\kappa(t)} = I_t$ for all $t$ (equivalently $I_{\kappa(t)}$ is saturated) and $S/I$ is $A$-flat
\end{enumerate}
Moreover, if $X_t$ is projectively normal then,
\[ (S_t/I_t)_d \iso H^0(X_t, \struct{X_t}(d)) \]
are isomorphims. If $X$ is projectively normal then,
\[ (S/I)_d \iso H^0(X, \struct{X}(d)) \]
are isomorphims.
\end{prop}

\begin{proof}
We have shown (a) $\implies$ (b) $\implies$ (c) so it suffices to show that (c) $\implies$ (a). 
\bigskip\\
First, since $(S/I)_d$ is $A$-flat we see that $(S/I) \ot_A \kappa(t) = S_{t} / I_{\kappa(t)}$ has constant dimension (since $A$ is connected). Since $I_t = I_{\kappa(t)}$ we conclude that,
\[ \dim_{\kappa(t)} (S_t/I_t)_d = \dim_{\kappa(t)} (S_t/I_{\kappa(t)})_d = \dim_{\kappa(t)} (S/I)_d \ot_A \kappa(t) \]
is constant in $t$ for all $d$ proving that $X$ is very flat over $A$. 
\end{proof}

\begin{rmk}
Notice that the constancy of the rank of $(S/I)_d$ implies that $(S/I)_d$ is finite locally-free so we could alternatively assume this rather than $I$ being locally-free.
\end{rmk}
\noindent
Of course we could prove the same theorem for any noetherian integral base scheme $T$.

\begin{prop}
Let $T$ be a noetherian integral scheme and $X \subset \P^n_T$ be a closed subscheme flat over $T$. Then for each $t \in T$,
\[ J_t = \bigoplus_{d \ge 0} (\pi_* \I(d))_t \ot \kappa(t) \to \Gamma_*(X_t, \I_t) = I_t \]
with the larger ideal being the saturation of the smaller so we have equality for $d \gg 0$. Additionally, the following are equivalent,
\begin{enumerate}
\item $X$ is very flat over $T$
\item $J_t = (\pi_* \I(*)) \ot_{\stalk{T}{t}} \kappa(t) \to I_t$ is an isomorphism 
\item $J_t = I_t$ for all $t$ (equivalently $J_t$ is saturated) and $\pi_* \struct{X}$ is a flat $\struct{T}$-module.
\end{enumerate}
Moreover, if each $X_t$ is projectively normal then,
\[ (S_t/I_t)_d \iso H^0(X_t, \struct{X_t}(d)) \]
are isomorphims. 
\end{prop}

\begin{proof}
We use the same argument as above. Specifically, look at the naturality square,
\begin{center}
\begin{tikzcd}
(\pi_* \I(d))_t \ot \kappa(t) \arrow[r] \arrow[d] & (\pi_* \struct{\P^n_T}(d))_t \ot \kappa(t) \arrow[d, equals]
\\
\Gamma(\P^n_t, \I_t(d)) \arrow[r] & \Gamma(\P^n_t, \struct{\P^n_t}(d))
\end{tikzcd}
\end{center}
Then constancy of $\dim_{\kappa(t)} I_t = h^0(t, \I_t(d))$ in sufficiently large degree from constancy of the Hilbert polynomial gives equality of the ideals $J_t = I_t$ for $d \gg 0$. Then if $X$ is very flat we have this constancy for all $d \ge 0$ and thus $J_t = I_t$ and from semicontinuity these pushforward sheaves are flat. The rest of the proof goes identically.
\end{proof}

\begin{lemma}
Let $I,J \subset S$ be homogeneous ideals with $I_d = J_d$ for $d \gg 0$. Then $I^{\text{sat}} = J^{\text{sat}}$.
\end{lemma}

\begin{proof}
It suffices to consider homogeneous elements. Let $f \in I^{\text{sat}}$ be homogeneous of degree $d$ then there is $n$ so that for each $i$ we have $x_i^n f \in I$. However, then $x_i^{n+m} f \in I_{n+d+m} = J_{n+d+m}$ for large enough $m$ so $f \in J^{\text{sat}}$.  Therefore $I^{\text{sat}} \subset J^{\text{sat}}$ and by symmetry $J^{\text{sat}} \subset I^{\text{sat}}$ proving the claim. 
\end{proof}

\newcommand{\sS}{\mathscr{S}}

\begin{enumerate}
\item We can deform three points in $\P^3$ in general position into a line and when they become colinear more functions vanish so the Hilbert function jumps. Indeed, consider the closed subscheme,
\[ \{ [-1 : 0 : 1] \} \cup \{ [1 : 0 : 1] \} \cup \{ [0 : a : 1] \} \]
This is clearly a flat family of schemes. For $a \neq 0$ the ideal $I_a = \ker{(S_a \to k^3)}$ where $S_a = k[x_0, x_1, x_3]$ and the map is evaluation at the three points. Therefore, because the three points are not colinear, a linear functional can take any value on $k^3$ so,
\[ \dim_{k} (S_a/I_a)_d = 
\begin{cases}
1 & d = 0
\\
3 & d = 1
\\
3 & d > 1
\end{cases} \]
However, for $a = 0$ the points are colinear so the the value of a linear functional is determined on by its values on any pair and thus the image is a two-dimensional subspace of $k^3$ so,
\[ \dim_{k} (S_a/I_a)_d = 
\begin{cases}
1 & d = 0
\\
2 & d = 1
\\
3 & d > 1
\end{cases} \]
Therefore, $X_a$ has constant Hilbert polynomial but not constant Hilbert function for $d = 1$. At the end of the question we will see this implies that the family of cones cannot be flat. 
\bigskip\\
However, we would like an example where the fibers are smooth varities.
Looking at part (d) we want to choose a family of non-projectively normal varities. We choose to deform a twisted quadric rational curve. To make the Hilbert function (not Hilbert polynomial!) jump, we want to deform it into a subspace. However, there is no smooth degree four rational curve in $\P^2$ so the subspace we need to deform into is $\P^3 \subset \P^4$. Therefore, we should choose the rational normal curve of degree $4$ and deform it into the twisted quadric rational curve in $\P^3$. Consider,
\[ \P^1_{\A^1} \to \P^4_{\A^1} \]
via $[s:t] \mapsto [s^4 : s^3 t : a s^2 t^2 : s t^3 : t^4]$. Let $X_a \subset \P^4$ be the image of this map. For $a = 1$ we know $\P^1 \embed \P^4$ is a rational normal curve so,
\[ H^0(\P^4, \struct{\P^4}(d)) \onto H^0(X_1, \struct{X_1}(d)) \]
is surjective for all $d$. Furthermore, $\P^1 \to \P^4$ has degree $4$ so we have,
\[ (S_1/I_1)_d =  H^0(X_1, \struct{X_1}(d)) = H^0(\P^1, \struct{\P^1}(4d)) \]
so we get,
\[ \dim_{k}(S_1/I_1)_d = \dim_{k} H^0(\P^1, \struct{\P^1}(4d)) = 4d + 1 \]
However, $X_0 \subset \P^4$ is twisted quadric rational curve $\P^1 \embed \P^3 \subset \P^4$ which is not projectively normal. Since the ideal $I_0$ is the ideal of relations for the functions $s^4, s^3 t, s t^3, t^4$ meaning the kernel of the map,
\[ k[x_0, x_1, x_3, x_4] \to k[s,t] \]
sending $x_0 \mapsto s^4, x_1 \mapsto s^3 t, x_3 \mapsto s t^3, x_4 \mapsto t^4$ and therefore the quotient is isomorphic to its image,
\[ S_0/I_0 = k[x_0, x_1, x_3, x_4]/I' \cong k[s^4, s^3 t, s t^3, t^4] \] 
where we give $s,t$ degree $\frac{1}{4}$ (i.e. view it as a subring of $(k[s,t])^{(4)}$) to make this a graded isomorphism. We see that the ideals $I_a$ are homogeneous and prime since they are the kernel of a map of graded domains (the quotient of such a kernel is a subring of a domain and hence a domain) and therefore saturated (if $\p$ is a prime ideal not containing the irrelevant ideal and $x_i^{n} f \in \p$ then either $x_i \in \p$ for each $i$ or $f \in \p$). Therefore,
\[ \dim_{k} (S_0 / I_0)_d = \dim_k (k[s^4, s^3 t, s t^3, t^4])_d = 
\begin{cases}
1 & d = 0
\\
4 & d = 1
\\
4 d + 1 & d > 1
\end{cases} \]
so indeed we see that $S_d \to H^0(X_0, \struct{X_0}(d))$ is not surjective exactly for $d = 1$. Notice we can make the same argument for $X_1$ since,
\[ S_1 / I_1 \cong k[s^4, s^3 t, s^2 t^2, s t^3, t^4] = (k[s,t])^{(4)} \]
and therefore we immediately see that,
\[ \dim_k (S_1 / I_1)_d = (k[s,t])^{(4)}_d = 4d + 1 \]
Notice that the Hilbert polynomials agree because these formulae are equal for $d \gg 0$. In fact, it is easy to check that $X_a$ for $a \neq 0$ has the same Hilbert function as $X_1$ (they are related by a diagonal $\mathrm{PGL}_5$ transformation) and therefore the family is flat by [III, Thm. 9.9]. However, we see that the Hilbert function jumps meaning this is not a very flat family. This implies that the cones $\{ C(X_a) \}$ cannot form a flat family in $\P^5$ because their Hilbert polynomials are not independent of $a$. Because the ideal for $C(X_a)$ is $I_a[x_{n+1}] \subset S_a[x_{n+1}]$ se find, for $d \gg 0$,
\begin{align*}
P(X_t)(d) & = \dim_{k} (S_a[x_{n+1}]/I_a[x_{n+1}])_d = \dim_{\kappa(t)} ((S_t/I_t)[x_{n+1}])_d
\\
& = \sum_{i = 0}^d \dim_{\kappa(t)} (S_t/I_t)_{i} = 
\begin{cases}
(2 d + 1)(d + 1) & a \neq 0
\\
(2 d + 1)(d + 1) - 1 & a = 0
\end{cases}
\end{align*}
is not constant in $a$ and therefore $\{ C(X_a) \}$ cannot be a flat family of schemes.

\item This is not a question ... 

\item Suppose that $X \subset \P^n_T$ is a very flat family over $T$ where $T$ is a noetherian intergral scheme.  Let $S_t = \kappa(t)[x_0, \dots, x_n]$ and $I_t \subset S_t$ the saturated ideal cutting out $X_t \subset \P^n_{\kappa(t)}$. 
From [II, Ex. 5.14(b)] we see that for $d \gg 0$ that,
\[ P_t(d) = \dim_{\kappa(t)} H^0(X_t, \struct{X_t}(d)) = \dim_{\kappa(t)} (S_t / I_t)_d \]
thus is constant in $t \in T$ for $d \gg 0$ where $P_t$ is the Hilbert polynomial of $X_t$ which is therefore constant in $t \in T$ proving that $X \to T$ is flat by [III, Thm. 9.9].
\bigskip\\
Notice that, if there exists a closed subscheme $X' \subset \P^{n+1}_T$ whose fibers are $X'_t \cong C(X_t)$ then because the ideal for $C(X_t)$ is $I_t[x_{n+1}] \subset S_t[x_{n+1}]$ and thus,
\[ \dim_{\kappa(t)} (S_t[x_{n+1}]/I_t[x_{n+1}])_d = \dim_{\kappa(t)} ((S_t/I_t)[x_{n+1}])_d = \sum_{i = 0}^d \dim_{\kappa(t)} (S_t/I_t)_{i} \]
is constant in $t \in T$ for all $d$ because the numbers,
\[ \dim_{\kappa(t)} (S_t/I_t)_{d} \]
are constant in $t \in T$. Therefore the family $\{ C(X_t) \}$ given by $X' \to T$ is very flat and therefore flat. 
\bigskip\\
Therefore, the entire difficulty is in constructing an algebraic family $X' \to T$ meaning a closed subscheme $X' \subset \P^{n+1}_T$ whose fibers are the cones. We proceed by the natural construction and show that it works in the nice cases we are considering. Let,
\[ \J = \bigoplus_{n \ge 0} \pi_* \I_X(n) \embed \bigoplus_{n \ge 0} \pi_* \struct{\P^n_T}(n) = \struct{T}[x_0, \dots, x_n] = \sS \] 
In terms of previous notation $\J_t \ot \kappa(t) = J_t$ and $I_t = \Gamma_*(\P^n_t, \I_{X_t})$ is its saturation. Then define,
\[ C(X) = \rProj{T}{\sS[x_{n+1}]/\J[x_{n+1}]} \embed \rProj{T}{\sS[x_{n+1}]} = \P^{n+1}_T \]
the relative version of the projective cone. However, we will not have $C(X)_t = C(X_t)$ in general. Using that $\sS_t \ot \kappa(t) = S_t$ and $\J_t \ot \kappa(t) = J_t$ we have,
\[ C(X_t) = \Proj{S_t[x_{n+1}]/I_t[x_{n+1}]} \embed \Proj{S_t[x_{n+1}]/J_t[x_{n+1}]} = C(X)_t \]
because $J_t \subset I_t$ but this closed immersion is only an equality if $J_t = I_t$. However, by the prior discussion, when $X \to T$ is very flat we have $J_t = I_t$ for all $t \in T$ and therefore this proves that $\{ C(X_t) \}$ is realized as an algebraic family by $C(X) \to T$.

\item Let $\{ X_{(t)} \}$ is an algebraic family of projectively normal varieties in $\P^n_k$ parametrized by a nonsingular curve $T$ over an algebraically closed field $k$. 
\bigskip\\
This means there is a closed subscheme $X \subset \P^n_T$ with $f : X \to T$ surjective such that,
\begin{enumerate}
\item[(1)] $X_t$ is irreducible of dimension $\dim{X} - \dim{T}$
\item[(2)] if $\xi \in X_t$ is the generic point then $\m_t \stalk{X_t}{\xi} = \m_\xi$ 
\end{enumerate}
and so that $X_{(t)} = (X_t)_{\red}$. Since each $X_{(t)}$ is normal, by [III, Theorem 9.11] we have that $X \to T$ is flat and $X_{(t)} = X_t$ so $X_t$ is integral and $\{ X_t \}$ forms a flat family of (in this case projectively normal) varities. 
\bigskip\\
Defining $C(X)_{(t)} := (C(X)_t)_{\red}$ I claim that $C(X_{(t)}) = C(X)_{(t)}$ and that $C(X)$ thus realizes $\{ C(X_{(t)}) \}$ as an algebraic family of varieties.

Because $X_{t}$ is projectively normal, $S_t/I_t$ is a normal domain and hence $C(X_t)$ is normal and integral because the affine cone $\Spec{S_t/I_t}$ is normal and integral and $C(X_t)$ is the union of the affine cone and $X_t$. 
Now $C(X_t) \embed C(X)_t$ is actually a homeomorphism because localizing at any point besides $\m = (x_0, \dots, x_n)$ there is some $x_i$ which is a unit and $(S_t/J_t)_{x_i} = (S_t/I_t)_{x_i}$ because $J_t \subset I_t$ and $I_t$ is its saturation so for each $f \in I_t$ there is some $n$ so that $x_i^n f \in J_t$. Therefore, on the affine opens $D_+(x_i)$ for $0 \le i \le n$ we see that $C(X_t) \cap D_+(x_i) = C(X)_t \cap D_+(X_i)$. There is a unique point $\m \in \P^n_t$ not covered by these opens at which the local rings may differ but both include a point at the origin and thus $C(X_t) \embed C(X)_t$ is a closed continuous bijection and thus a homeomorphism. Therefore, the sheaf of ideals cutting out $C(X_t)$ is nilpotent and $C(X_t)$ is reduced so $C(X_t) = (C(X)_t)_{\red} = C(X)_{(t)}$. Furthermore, $C(X) \to T$ is surjective because $X \embed C(X) \to T$ is surjective. Moreover,
\[ \dim C(X_t) = \dim{X_t} + 1 = n + 1 - \dim{T} \]
and $C(X_t)$ is integral so it is irreducible and the condition that $\m_t \stalk{X_t}{\xi} = \m_{\xi} = (0)$ is automatic because $\stalk{T}{t} \to \stalk{X_t}{\xi}$ is a local map. Therefore, we see indeed that $C(X) \to T$ realizes $\{ C(X_{(t)}) \}$ as a algebraic family of varities. 
\bigskip\\
Therefore, because $X_{(t)} \subset \P^n_t$ are all projectively normal we have $C(X_{(t)})$ normal and thus $\{ C(X_{(t)}) \}$ is an algebraic family of normal varities. Then by [III, Theorem 9.11] we see that $\{ C(X_{(t)}) \}$ is a flat family of schemes because $T$ is a nonsingular curve meaning that $C(X) \to T$ is flat and, 
\[ C(X_t) = C(X_{(t)}) = C(X)_{(t)} = C(X)_t \] Therefore, the Hilbert polynomial of $C(X_t)$ is constant but for $d \gg 0$ we have,
\[ P_t(d) = \dim_{\kappa(t)} (S_t[x_{n+1}]/I_t[x_{n+1}])_d = \dim_{\kappa(t)} ((S_t/I_t)[x_{n+1}])_d = \sum_{i = 0}^d \dim_{\kappa(t)} (S_t/I_t)_{i} \]
Furthermore, because $(I_t)_d = H^0(\P^n_t, \I_t(d))$ and $\I(d)$ is a coherent $\struct{\P^n_T}$-module flat over $T$ and therefore by semicontinuity [III, Theorem 12.8],
\[ h^0(y, \I_X) = \dim_{\kappa(t)} H^i(\P^n_t, \I_t(d)) = \dim_{\kappa(t)} (I_t)_d \]
is upper semicontinuous on $T$. Since $\dim_{\kappa(t)} S_t$ is manifestly constant we see that,
\[ \dim_{\kappa(t)} (S_t/I_t)_d \]
are all lower semi-continuous on $T$. However, for $d \gg 0$ the sum of these functions is constant and therefore each must be constant proving that $X \to T$ is very flat.  
\end{enumerate}

\subsubsection{9.6 DO THIS!!}

Let $Y \subset \P^n$ be a nonsingular variety of dimension $\ge 2$ over an algebraically closed field $k$. Let $H \subset \P^n$ be a hyperplane (so $H \cong \P^{n-1}$) which does not contain $Y$ and such that the scheme theoretic intersection $Y' =  Y \cap H$ is nonsingular. I want to show that $Y$ is a complete intersection in $\P^n$ if and only if $Y'$ is a complete intersection in $\P^{n-1}$. Let $S = k[x_0, \dots, x_n]$ and $I = I(Y)$.
\bigskip\\
Since $Y$ is irreducible it has pure dimension $d$ thus codimension $r = n - d$. Suppose that $Y$ is a complete intersection. Then, by definition, $I$ can be generated by $r$ homogeneous (see Lemma \ref{controlled_at_cone_point})) elements. Then clearly $Y'$ is cut out be the image of $I$ in $k[x_0, \dots, x_n]/(h) \cong k[y_0, \dots, y_{n-1}]$ (although this may not be the saturated ideal) which still has $r$ generators but $\codim{Y', H} = (n-1) - (d-1) = n - d = r$ and thus $Y'$ is a complete intersection.
\bigskip\\
Now suppose that $Y' \subset H$ is a complete intersection. Since $Y'$ is normal, we see that the affine cone $C(Y') \subset \Spec{S}$ is normal by [Ex. II.8.4(b)]. Now consider the inclusion of affine cones $C(Y') \embed C(Y)$ where $C(Y) =\Spec{S/I(Y)}$. Let $h \in S$ be the equation of the hyperplane. Because $Y' = Y \cap H$ we see that $h \in S(Y)$ cuts out $C(Y') \setminus \{ 0 \}$ in $C(Y) \setminus \{ 0 \}$. Let $\p \subset S(Y)$ be the prime defining $C(Y') \subset C(Y)$. Then let $A$ be the local ring at the origin. Because $h$ set theoretically cuts out $C(Y') \subset C(Y)$ since away from the origin it scheme-theoretically defines $C(Y') \setminus \{ 0 \}$ but $h$ vanishes at the origin and indeed the origin is in $C(Y')$. Therefore, the ring $A/hA$ has a unique minimal prime corresponding to the correct reduced subscheme structure on $C(Y')$. Furthermore, at the generic point $\xi \in C(Y') \subset C(Y)$ we see that $\m_\xi = (h)$ because on the open $C(Y) \setminus \{ 0 \}$ we know that $Y'$ is cut out by $h$. Therefore, $\p A_\p = h A_\p$ because this is exactly isomorphic to $\m_{\xi}$ since $\p \subset A$ is the prime corresponding to $\xi$. Finally, $A / \p$ is the local ring at the origin of $C(Y')$ which is normal because $Y'$ is projectively normal. Thus by [Lemma 9.12] we see that $\p = h A$ and $A$ is normal proving that $C(Y)$ is normal (it is normal everywhere else because $Y$ is nonsingular so $C(Y) \setminus \{ 0 \}$ is also nonsingular because it is locally isomorphic to $Y \times \A^1$). This proves that $S(Y') = S(Y)/(h)$ globally (Lemma \ref{controlled_at_cone_point}) and therefore $I_{\P^n}(Y') = I_{\P^n}(Y) + (h)$ and likewise $I_{\P^{n-1}}(Y') = I_{\P^n}(Y')/(h) = I_{\P^n}(Y)/(h)$. 
\bigskip\\
We know that $I_{\P^{n-1}}(Y') = (\bar{f}_1, \dots, \bar{f}_r)$ and thus we can write $I_{\P^n}(Y') = (f_1, \dots, f_r, h)$ with $f_i \in I_{\P^n}(Y)$ because $\bar{f}_i \in I_{\P^n}/(h)$ we can choose a lift $f_i = \bar{f}_i + h g_i \in I_{\P^n}(Y) \subset I_{\P^n}(Y')$. Let $J = (f_1, \dots, f_r)$

\subsubsection{9.7 CHECK}

Let $Y \subset X$ be a closed subscheme, where $X$ is a scheme of finite type over a field $k$. Let $D = k[t]/(t^2)$ be the ring of dual numbers. We are looking for infinitessimal deformations of $Y$ as a closed subscheme of $X$ meaning closed subschemes $Y' \subset X \times_k D$ which are flat over $D$ equiped with an isomorphism $Y' \times_D k \iso Y$. 
\bigskip\\
First we consider the affine situation with $X = \Spec{A}$ and $Y = \Spec{A/I}$ for an ideal $I \subset A$. Then let $A' = A[t]/(t^2) = A \otimes_k D$. We are searching for $Y' = \Spec{A'/I'}$ for an ideal $I' \subset A'$. Notice that flatness of $Y'$ over $D$ is equivalent $A'/I'$ being flat as a $D$-module. Flatness is equivalent to the condition that for every ideal $J \subset D$ the natural map,
\[ J \otimes_D A'/I' \to J \cdot (A'/I') \]
is an isomorphism. However, $D$ only has one nontrivial ideal $(t) \cong k$ and thus flatness of $A'/I'$ is equivalent to the exactness of,
\begin{center}
\begin{tikzcd}
0 \arrow[r] & A / I \arrow[r, "t"] & A'/I' \arrow[r] & A/I \arrow[r] & 0
\end{tikzcd}
\end{center}
comming from the sequence,
\begin{center}
\begin{tikzcd}
0 \arrow[r] & A'/I' \otimes_D (t) \arrow[r] & A'/I' \otimes_D D \arrow[r] & A'/I' \otimes_D k \arrow[r] & 0
\end{tikzcd}
\end{center}
and the identification $A'/I' \otimes_D k \cong A/I$ and the isomorphism $(t) \cong k$ and the fact that exactness of this sequence on the left is equivalent to $A/I \to t (A'/I')$ being an isomorphism. 
\bigskip\\
Now consider the diagram,
\begin{center}
\begin{tikzcd}
& 0 \arrow[d] & 0 \arrow[d] & 0 \arrow[d] &
\\
0 \arrow[r] & I \arrow[r, "t"] \arrow[d] & I' \arrow[r] \arrow[d] & I \arrow[r] \arrow[d] & 0
\\
0 \arrow[r] & A \arrow[r, "t"] \arrow[d] & A' \arrow[r] \arrow[d] & A \arrow[r] \arrow[d] & 0
\\
0 \arrow[r] & A/I \arrow[r, "t"] \arrow[d] & A'/I' \arrow[r] \arrow[d] & A/I \arrow[r] \arrow[d] & 0
\\
& 0 & 0 & 0
\end{tikzcd}
\end{center}
where the columns are clearly exact and the bottom two rows are exact by flatness of $A'/I'$ and $A'$ as $D$-modules so the top row is exact by the nine lemma. Notice that the extension $I' \onto I$ such that $A' / I'$ is a flat $D$-module completely determines this diagram. I claim that such extensions $I' \onto I$ are in bijective correspondence with $\Hom{A}{I}{A/I}$ with the zero map corresponding to the split extension $I' = I \ot_k D$ and thus $A' / I' = (A/I) \ot_k D$ which is the trivial deformation. 
\bigskip\\
To see this, notice that the extension,
\begin{center}
\begin{tikzcd}
0 \arrow[r] & k \arrow[r] & D \arrow[r] & k \arrow[r] & 0
\end{tikzcd}
\end{center}
is split meaning there is a section $\sigma : k \to D$ of $k$-modules. Therefore, $A' \cong A \oplus t A$ as $A$-modules though the section $\sigma : A \to A'$. Therefore, given $x \in I$ consider a lift $x' \in I'$ and write $x' = a + t b$ for $a,b \in A$. Then $a$ is determined by $a = \sigma(x)$ but $b$ is well-defined only up to $\ker{(I' \to I)} = t I$ so $\bar{b}$ defines an element of $A / I$ giving a well-defined map $\varphi : I \to A/I$ of $A$-modules. Furthermore, I claim that any such map determines the extension $I' \onto I$. Indeed, 
\[ I' = \{ a + t b \mid a \in I \text{ and } b \in A \text{ such that } \varphi(a) = \bar{b} \text{ in } A / I \} \]
Then the map $I' \onto I$ given by $a + t b \mapsto a$ makes $I'$ fit into the exact sequence,
\begin{center}
\begin{tikzcd}
0 \arrow[r] & I \arrow[r, "t"] & I' \arrow[r] & I \arrow[r] & 0
\end{tikzcd}
\end{center}
because if $a = 0$ then $\bar{b} = 0$ meaning $b \in I$. Then the map $I' \onto I$ gives an isomorphism $I' \ot_D k \iso I$. Therefore, we get a diagram of the above form where the top two rows are exact and therefore the bottom row is exact by the nine lemma and hence $A'/I'$ is flat over $D$. Furthermore, these operations give a bijection between $I' \onto I$ fitting into such a diagram up to isomorphism (compatible with the entire diagram) and maps $\varphi : I \to A / I$ of $A$-modules.
\bigskip\\
These exact arguments globalize (or equivalently the bijection is natural and therefore glues) to give equivalences,
\[ \{ \text{deformations of } Y \subset X \} \leftrightarrow \Hom{\struct{X}}{\I_Y}{\struct{Y}} = \Hom{\struct{X}}{\I_Y / \I_Y^2}{\struct{Y}} = H^0(X, \sN_{Y/X}) \]
where $\I$ is the ideal sheaf of $Y \subset X$ and $\sN_{Y/X} = \shHom{\struct{Y}}{\I_Y/\I_Y^2}{\struct{Y}}$.

\subsubsection{9.8}

\newcommand{\Def}[1]{\mathrm{Def}\left( #1 \right)}

Let $A$ be a finitely generated $k$-algebra. We find a presentation $q : P \onto A$ from a polynomial ring $P$ and $J = \ker{(P \onto A)}$ fitting into an exact sequence,
\begin{center}
\begin{tikzcd}
0 \arrow[r] & J \arrow[r] & P \arrow[r] & A \arrow[r] & 0
\end{tikzcd}
\end{center}
This produces an exact sequence,
\begin{center}
\begin{tikzcd}
J / J^2 \arrow[r] & \Omega_{P/k} \ot_P A \arrow[r] & \Omega_{A/k} \arrow[r] & 0
\end{tikzcd}
\end{center}
then apply the functor $\Hom{A}{-}{A}$ and let $T^1(A)$ be the cokernel,
\begin{center}
\begin{tikzcd}
\Hom{A}{\Omega_{P/k} \ot_P A}{A} \arrow[r] & \Hom{A}{J/J^2}{A} \arrow[r] & T^1(A) \arrow[r] & 0
\end{tikzcd}
\end{center}
We want to show that $T^1(A)$ classifies deformations of $A$ over $D$ up to isomorphism.
\bigskip\\
Let $A'$ be a deformation of $A$ over $k$ meaning a flat $D$-module with a morphism $A' \onto A$ giving an isomorphism $A' \ot_D k \iso A$. Flatness implies that the sequence,
\begin{center}
\begin{tikzcd}
0 \arrow[r] & A \arrow[r, "t"] & A' \arrow[r] & A \arrow[r] & 0
\end{tikzcd}
\end{center}
induced by tensoring,
\begin{center}
\begin{tikzcd}
0 \arrow[r] & k \arrow[r] & D \arrow[r] & k \arrow[r] & 0
\end{tikzcd}
\end{center}
by $A$ is exact. Because $P$ is a free algebra we can lift to get a morphism $q' : P' \to A'$ and therefore a morphism of exact sequences,
\begin{center}
\begin{tikzcd}
0 \arrow[r] & P \arrow[d, two heads] \arrow[r, "t"] & P' \arrow[d] \arrow[r] & P \arrow[d, two heads] \arrow[r] & 0
\\
0 \arrow[r] & A \arrow[r, "t"] & A' \arrow[r] & A \arrow[r] & 0
\end{tikzcd}
\end{center}
and therefore $P' \onto A'$ is surjective. Thus every deformation of $A$ is a deformation \textit{as a closed subscheme} of $P$. However, deformations can be equivalent abstractly without being equivalent as deformations in the ambiant space. 
\bigskip\\
Let $\Def{A}$ be the pointed set of deformations. By the previous problem, there is a surjection,
\[ \Hom{A}{J/J^2}{A} \onto \Def{A} \]
of pointed sets. Therefore, it suffices to show that two deformations are isomorphic if their corresponding maps $\varphi_1, \varphi_2 : \Hom{A}{J/J^2}{A}$ satisfy,
\[ \varphi_1 - \varphi_2 \in K = \im{\left(\Hom{A}{\Omega_{P/k} \ot_A}{A} \to \Hom{A}{J/J^2}{A} \right)} \]
because then $\Def{A}$ is isomorphic to the cokernel,
\[ T^1(A) = \coker{\left(\Hom{A}{\Omega_{P/k} \ot_A}{A} \to \Hom{A}{J/J^2}{A} \right)} \]
A morphism of deformations $\psi : A_1' \to A_2'$ is a morphism of $D$-algebras such that the diagram,
\begin{center}
\begin{tikzcd}
0 \arrow[r] & A \arrow[d, equals] \arrow[r, "t"] & A'_1 \arrow[d, "\psi"] \arrow[r] & A \arrow[d, equals] \arrow[r] & 0
\\
0 \arrow[r] & A \arrow[r, "t"] & A'_2 \arrow[r] & A \arrow[r] & 0
\end{tikzcd}
\end{center}
commutes. In particular, this shows that $\psi$ is an isomorphism. Therefore, it suffices to show that $\varphi_1 - \varphi_2 \in K$ if and only if there exists a morphism of deformations $\psi : A_1' \to A_2'$. 
\bigskip\\
Consider the diagram,
\begin{center}
\begin{tikzcd}
0 \arrow[d] & 0 \arrow[d]
\\
J'_1 \arrow[d] \arrow[r, "h", dashed] \arrow[r, dashed] & A \arrow[d, "t"]
\\
P' \arrow[r, "q'_2", two heads] \arrow[d] & A'_2 \arrow[d]
\\
A'_1 \arrow[d] \arrow[r] & A \arrow[d]
\\
0 & 0
\end{tikzcd}
\end{center}
First notice that $h(J'^2_1) = 0$ because $P' \to A'_2$ is a ring map and $t^2 = 0$ so it is naturally an $A'_1$-module map $J'_1/J'^2_1 \to A$. Furthermore, since $A$ is a $k$-module the map $h$ is determined by its reduction $\bar{h} : J / J^2 \to A$. I claim that $\bar{h} = \varphi_1 - \varphi_2$. Indeed, for any $x \in J$ the map $\varphi_i$ is defined by lifting to $x_i' \in J'_i$ and decomposing $x_i' = a_i + t b_i$ using the trivial extension structure of $P' \onto P$. Then $\varphi_i(x) = \bar{b_i} \in A$. However, $a_i = \sigma(x)$ so $a_1 = a_2$ and therefore $x_1' - x_2' = t(b_1 - b_2)$. Likewise $\bar{h}$ is defined by lifting $x$ to $x_1' \in J_1'$ mapping it to $A_2'$. However, $x_1' - t(b_1 - b_2) = a + t b_2 \in J'_2$ and therefore maps to zero proving that $\bar{h}(x) = b_1 - b_2$ as well. 
\bigskip\\
There exists a morphism $\psi : A_1' \to A_2'$ of $D$-modules making the diagram commute (and thus also the previous morphism of exact sequences because it is a $D$-morphism and thus preserves the $t$ action) if and only if there is a map $\tilde{\psi} : P' \to A_2'$ killing $J'_1$. However, the difference $\theta = q_2' - \tilde{\psi}$ is a $D$-derivation $\theta : P' \to A$ and conversely any map $q_2' - \theta$ is a $D$-algebra map fitting into the diagram. Therefore, the existence of $\tilde{\psi}$ and hence of $\psi$ is equivalent to the existence of a derivation $\theta$ such that $\theta|_{J_1'} = h$ because then $q_2' - \theta$ kills $J_1'$ and thus descends to a ring map $A_1' \to A_2'$. From restriction along $J_1' \to P'$ we get,
\begin{center}
\begin{tikzcd}
\Der{D}{P'}{A} \arrow[d, equals] \arrow[r] & \Hom{D}{J_1'}{A} \arrow[d, equals]
\\
\Der{k}{P}{A} \arrow[d, equals] \arrow[r] & \Hom{k}{J/J^2}{A} \arrow[d, equals]
\\
\Hom{k}{\Omega_{P/k}}{A} \arrow[r] & \Hom{k}{J/J^2}{A}
\end{tikzcd}
\end{center}
and therefore $h$ is the image of $\Der{D}{P'}{A} \to \Hom{D}{J_1'}{A}$ if and only if $\bar{h} = \varphi_1 - \varphi_2$ is in the image of $\Hom{k}{\Omega_{P/k}}{A} \to \Hom{k}{J/J^2}{A}$ proving the claim.

\subsubsection{9.9}

Consider the ring,
\[ B = k[x,y,z,w]/(x,y) \cap (z,w) = k[x,y,z,w]/(xz,xw,yz,yw) \]
Let $P = k[x,y,z,w]$ and $J = \ker{(P \onto A)} = (xz,xw,yz,yw)$. The map $J/J^2 \to \Omega_{P/k} \otimes_P A$ given by $j \mapsto \d{j}$ induces,
\[ \Hom{A}{\Omega_{P/k} \otimes_A}{A} \to \Hom{A}{J/J^2}{A} \]
We need to show that for any map $\varphi : \Hom{A}{J/J^2}{A}$ we can lift it to elements $(a_x, a_y, a_z, a_w) \in A$ such that $\psi(\d{v}) = a_v$ for each variable $v$ lifts via $\varphi = \psi \circ (j \mapsto \d{j})$ because, 
\[ \Omega_{P/k} = A \d{x} \oplus A \d{y} \oplus A \d{z} \oplus A \d{w} \]
Choose a presenation,
\begin{center}
\begin{tikzcd}
P^{\oplus 4} \arrow[r] & P^{\oplus 4} \arrow[r] & J \arrow[r] & 0
\end{tikzcd}
\end{center}
where the second map is $(xz, xw, yz, yw)$ and first map is given by the matrix,
\[ \begin{pmatrix}
w & - z & 0 & 0
\\
0 & 0 & w & - z
\\
y & 0 & -x & 0
\\
0 & y & 0 & -x
\end{pmatrix} \]
Tensoring this by $A = P/J$ we get a presentation of $J/J^2$,
\begin{center}
\begin{tikzcd}
A^{\oplus 4} \arrow[r] & A^{\oplus 4} \arrow[r] & J/J^2 \arrow[r] & 0
\end{tikzcd}
\end{center}
with the same matrices. Therefore, an $A$-module map $\varphi : J/J^2 \to A$ induces a map $A^{\oplus 4} \to A$ composing with the matrix to zero. Let the images be $a_i \in A$. Then, 
\begin{align*}
w a_1 - z a_2 &= 0
\\
w a_3 - z a_4 &= 0
\\
y a_1 - x a_3 &= 0
\\
y a_2 - x a_4 &= 0
\end{align*}
Therefore, $w a_1 \in (z)$ and $y a_1 \in (x)$ so $a_1 \in (x, z)$ so we can write $a_1 = z a_x + x a_z$. Likewise, $z a_4 \in (w)$ and $x a_4 \in (y)$ so $a_4 \in (y,w)$ so we can write $a_4 = y a_w + w a_y$. Now, $w a_3 = zw a_y$ and $x a_3 = yx a_z$ so $a_3 = z a_y + y a_z$. Likewise, $y a_2 = xy a_w$ and $z a_2 = w z a_x$ so $a_2 = w a_x + x a_w$. Therefore, if we define $\psi : \Omega_{P/k} \to A$ by $\psi(\d{v}) = a_v$ for each variable then,
\begin{align*}
\varphi(xz) = a_1 & = z a_x + x a_z = \psi(z \d{x} + x \d{z}) = \psi(\d{(xz)})
\\
\varphi(xw) = a_2 & = w a_x + x a_w = \psi(w \d{x} + x \d{w}) = \psi(\d{(xw)})
\\
\varphi(yz) = a_3 & = z a_y + y a_z = \psi(z \d{y} + y \d{w}) = \psi(\d{(yz)})
\\
\varphi(yw) = a_4 & = y a_w + w a_y = \psi(y \d{w} + w \d{y}) = \psi(\d{(yw)})
\end{align*}
and thus $\varphi = \psi \circ (j \mapsto \d{j})$ so the map,
\[ \Hom{A}{\Omega_{P/k} \otimes_A P}{A} \to \Hom{A}{J/J^2}{A} \]
is surjective and thus $T^1(A) = 0$ so $A$ is rigid.

\subsubsection{9.10 DO THIS!!}

\begin{enumerate}
\item Let $X = \P^1_k$ then,
\[ H^1(X, \T_X) = H^1(\P^1_k, \struct{\P^1_k}(2)) = 0 \]
and therefore $X$ has no nontrivial infinitessimal deformations and thus is rigid.

\item We are looking for a proper flat map $f : X \to \A^2$ over an algebraically closed field $k$ (I will make like even harder by restricting to $k = \CC$) such that $X_o \cong \P^2_k$ such that the family is not locally trivial at the origin $o$ menaing there is no open $U \subset \A^2$ containing $o$ with $X_U \cong U \times \P^1$. 
\bigskip\\
Because $\A^2$ is irreducible, such an open $U$ always contains the generic point $\xi \in \A^2$ so it suffices to find such a family where the generic fiber $X_{\xi} \not\cong \P^1_{\kappa(\xi)}$. 
\bigskip\\
Let us first make some remarks about the nature of such a solution. Because the singular locus is closed and $f$ is proper, its image is closed and does not contain $o$ so there is an open neighbrohood of $o$ on which $f$ is smooth so it poses no real restriction to consider only smooth $f$. Furthermore, if we assume that $f$ is projective (which I hope it is or I'll never find it!) then because $f$ is flat its fibers have constant dimension, arithmetic genus, and degree as subschemes of $\P^n$. Thus, all the fibers must be smooth genus $0$ curves which we know (over any field) are conics in $\P^2$ so it seems reasonable to search for families inside $\P^2_{\A^2}$. But we're over an algebraically closed field so all the fibers should be isomorphic to $\P^1$ right? But remember $\kappa(\xi) = k(s,t)$ is not algebraically closed it is only a $C_2$ field and a conic has degree $d = 2$ in $N = 3$ variables so $d^2 \ge N$ and therefore the conic may not have a point and thus may not be isomorphic to $\P^1_{\kappa(\xi)}$. This is, I assume, why Hartshorne chose $\A^2$ rather than $\A^1$ for the base since if $\kappa(\xi)$ were $C_1$ then we would be out of luck because $d < N$ does hold.
\bigskip\\
Therefore we should look for conics over $K	 = k(s,t)$ with no rational points. My first guess is the conic,
\[ s x^2 + t y^2 + z^2 = 0 \]
or replace $s$ by $1 + s$ and $t$ by $1 + t$ if you want to have your nonsingular fiber at the origin as asked for in the problem but it will be more convenient to work with it in this form. By clearing denominators, I can assume that any solution actually lies in $k[s,t]$. Because when $s = 0$ we get $t y^2 = z^2$ we must have $y = z = 0$ and thus $y,z \in (s)$ so we write $y = s y'$ and $z = s z'$. Then plugging in,
\[ x^2 + st y'^2 + s z'^2 = 0 \]
Now for $s = 0$ we see that $x = 0$ so $x \in (s)$ and therefore $x = s x'$ so,
\[ s x'^2 + t y'^2 + z'^2 = 0 \]
and therefore we get a reduced solution. If we assume that $(x,y,z)$ have minimal degree we arrive at a contradiction showing that no such solution can exist by descent. Therefore, we should consider the flat\footnote{Because these are regular schemes and the fibers all have dimension $1$ it is flat by miracle flatness. Note that $X$ is regular because it is covered by the affine schemes $V((1 + s) + (1 + t) y^2 + z^2)$ and $V((1 + s) x^2 + (1 + t) + z^2)$ and $V((1 + s) x^2 + (1 + t) y^2 + 1)$ which are regular. The first two are immediate because the derivative of $s$ or $t$ never vanish. For the third, if the derivative of $s$ and $t$ vanish then $z^2 = y^2 = 0$ so this is not a point on the hypersurface.} projective family,
\[ \Proj{\CC[s,t][x,y,z]/((1 + s) x^2 + (1 + t) y^2 + z^2)} \to \Spec{\CC[s,t]} \]
the fiber over $o$ is $\Proj{\CC[x,y,z]/(x^2 + y^2 + z^2)} \cong \P^1$ is a conic in $\P^2$ with a rational point $[1 : i : 0]$ and therefore is isomorphic to $\P^1$. However, we have shown that the generic fiber is not isomorphic to $\P^1$ and therefore the family is not locally trivial at any point on the base.

(I DONT UNDERSTAND HOW DANIEL CONCLUDES THAT IT IS REGULAR?) 
\end{enumerate}

\subsubsection{9.11 DO THIS!!}

Let $Y \subset \P^n_k$ be a nonsingular curve of degree $d$ over an algebraically closed field $k$.

SKETCH!!
There should exist a flat family of schemes $\wt{Y} \subset \P^n \times \A^1_k$ over $\A^1_k$ such that $\wt{Y}_1 \cong Y$ and $(\wt{Y}_0)_{\red} \subset \P^{n-1}_k \subset \P^n_k$ (to prove this I need something like it doesnt contain a line). The Hilbert polynomial is constant so $Y$ and $\wt{Y}_0$ have the same degree $d$ and the same $p_a$. Then taking the reduction increases $p_a$ and keeps $d$ fixed (proved in my notes) so we see that,
\[ p_a(Y) = p_a(\wt{Y}_0) \ge p_a((\wt{Y}_0)_\red) = \tfrac{1}{2} (d-1)(d-2) \]
where the last formula is true because $(\wt{Y}_0)_{\red}$ is a reduced pure codimension $1$ subscheme of $\P^2_k$ and hence an effective Cartier divisor of degree $d$ (in a UFD height one primes are principal so a radical ideal $I$ of height $1$ is the intersection the finitely many minimial primes which are height one primes and the intersection of two principal ideals in a UFD is principal)

\section{Appendix}

\subsection{A Intersection Theory}

\subsubsection{6.7}

\newcommand{\td}{\mathrm{td}}
\renewcommand{\ch}{\mathrm{ch}}

Let $X$ be a nonsingular projective $3$-fold with Chern classes $c_1, c_2, c_3$. Then we apply Grothendieck-Riemann-Roch,
\[ \ch(f_! \E) = f_* (\mathrm{ch}(\E) \cdot \td(\T_X)) \]
to the morphism $f : X \to \Spec{k}$. To give,
\[ \chi(\E) = \deg{(\ch(\L) \cdot \td(\T_X))_n} \]
where pushing forward onto a point selects the dimension zero (i.e. codimension $3$) part and takes degrees.
Thus it suffices to compute the Todd class,
\[ \td(\T_X) = 1 + \tfrac{1}{2} c_1(\T_X) + \tfrac{1}{12} (c_1(\T_X)^2 + c_2(\T_X)) + \tfrac{1}{24} c_1(\T_X) c_2(\T_X) \]
and by definition $c_i(\T_X) = c_i$. For a line bundle $\L$ with $c(\L) = 1 + D \in A^*(X)$ for some divisor $D$ we have,
\[ \ch(\L) = 1 + D + \tfrac{1}{2} D \cdot D + \tfrac{1}{6} D \cdot D \cdot D \]
and thus we find,
\begin{align*}
(\ch(\L) \cdot \td(\T_X))_n & = \tfrac{1}{24} c_1 c_2 + D \cdot \tfrac{1}{12} (c_1^2 + c_2) + \tfrac{1}{2} D^2 \cdot \tfrac{1}{2} c_1 + \tfrac{1}{6} D^3 
\\
& = \tfrac{1}{12} D \cdot (D + c_1) \cdot (2 D + c_1) + \tfrac{1}{12} D \cdot c_2 + \tfrac{1}{24} c_1 c_2
\end{align*}
For $D = 0$ we find, $\chi(\struct{X}) = \tfrac{1}{24} c_1 c_2$
and therefore $p_a(X) = 1 - \chi(\struct{X}) = 1 - \tfrac{1}{24} c_1 c_2$. Furthermore, $c_1 = -K_X$ and therefore,
\[ \chi(\L) = \tfrac{1}{12} D \cdot (D - K_X) \cdot (2D - K_X) + \tfrac{1}{12} D \cdot c_2 + 1 - p_a \]

\subsubsection{6.8}

Let $\E$ be a locally free sheaf of rank $2$ on $X = \P^3$. Hirzburch Riemann-Roch shows that,
\[ \chi(\E) = \deg{(\ch{(\E)} \cdot \td{(\T_X)})_n} \]
First notice,
\[ \ch{(\E)} = 2 + c_1(\E) + \tfrac{1}{2} (c_1(\E)^2 - 2 c_2(\E)) + \tfrac{1}{6} (c_1(\E)^3 - 3 c_1(\E) c_2(\E)) \]
Then we compute,
\begin{align*}
(\ch{(\E)} \cdot \td{(\T_X)})_n = \tfrac{2}{24} c_1 c_2 + c_1(\E) \cdot \tfrac{1}{12}(c_1^2 + c_2) + \tfrac{1}{2} (c_1(\E)^2 - 2 c_2(\E)) \cdot \tfrac{1}{2} c_1 + \tfrac{1}{6} (c_1(\E)^3 - 3 c_1(\E) c_2(\E))
\end{align*}
From the Euler sequence on $X = \P^n_k$,
\begin{center}
\begin{tikzcd}
0 \arrow[r] & \struct{X} \arrow[r] & \struct{X}(1)^{\oplus n+1} \arrow[r] & \T_X \arrow[r] & 0
\end{tikzcd}
\end{center}
we see that $c(\T_X) = (1 + c_1(\struct{X}))^{n+1} = (1 + H)^{n+1}$ where $H \in A^1(X)$ is the hyperplane class. For the case $n = 3$,
\[ c(\T_X) = 1 + 4 H + 6 H^2 + 4 H^3 \]
Therefore,
\[ (\ch{(\E)} \cdot \td{(\T_X)})_n = 2 H^3 + \tfrac{11}{6} c_1(\E) \cdot H^2 + (c_1(\E)^2 - 2 c_2(\E)) \cdot H + \tfrac{1}{6} (c_1(\E)^3 - 3 c_1(\E) c_2(\E)) \]
Now, because $A(X) = \Z[H]/(H^4)$ we must have $c_i(\E) = d_i H$ for integers $d_i$. Thus,
\[ (\ch{(\E)} \cdot \td{(\T_X)})_n = [2 + \tfrac{11}{6} d_1 + (d_1^2 - 2 d_2) + \tfrac{1}{6} (d_1^3 - 3 d_1 d_2)] H^3 \]
Therefore, since $\int_X H^3 = \deg{H^3} = 1$ we find,
\[ \chi(\E) = 2 + \tfrac{1}{6}(d_1^3 + 11 d_1) + d_1^2 - 2 d_2 - \tfrac{1}{2} d_1 d_2 \]
Notice that $n^3 \equiv n \mod 6$ and thus $d_1^3 + 11 d_1 \equiv d_1^3 - d_1 \equiv 0 \mod 6$ so $\tfrac{1}{6} (d_1^3 + 11 d_1)$ is an integer. Furthermore, $2 + d_1^2 - 2 d_2$ is obviously an integer. Since $\chi(\E)$ is an integer this implies that $d_1 d_2$ is divisible by $2$ that is $d_1 d_2 \equiv 0 \mod 2$.

\subsubsection{6.9 DO!!}

Let $\iota : X \embed \P^4_k$ be a smooth surface of degree $d$. Consider the normal sequence,
\begin{center}
\begin{tikzcd}
0 \arrow[r] & \T_X \arrow[r] & \iota^* \T_{\P^4} \arrow[r] & \sN_{X/\P^4} \arrow[r] & 0
\end{tikzcd}
\end{center}
Applying Chern classes we find that,
\[ c(\T_X) \cdot c(\sN_{X/\P^4}) = c(\iota^* \T_\P^4) \]
From the Euler sequence,
\[ c(\T_\P^4) = (1 + H)^5 \]
where $H$ is the hyperplane class. Therefore in $A^*(X)$,
\[ c(\iota^* \T_{\P^4}) = \iota^* c(\T_{\P^4}) = (1 + \iota^* H)^5  = 1 + 5 \iota^* H + 10 (\iota^* H)^2 \]
However, $(\iota^* H)^2 = \iota^* H^2$ is the class of $d$ points on $X$. Now expand,
\[ (1 + c_1 + c_2)(1 + c_1(\sN) + c_2(\sN)) = 1 + (c_1 + c_1(\sN)) + (c_1 c_1(\sN) + c_2 + c_2(\sN)) \]
Therefore, matching terms,
\begin{align*}
c_1 + c_1(\sN) & = 5 \iota^* H
\\
c_1 c_1(\sN) + c_2 + c_2(\sN) &= 10 (\iota^* H)^2
\end{align*}
and plugging in gives,
\[ c_2(\sN) + c_1 \cdot (5 \iota^* H - c_1) + c_2 = 10 (\iota^* H)^2 \]
Therefore, 
\[ c_2(\sN) = 10 (\iota^* H)^2 + c_1^2 - c_2 - 5 c_1 \cdot \iota^* H \]
Finally, taking degrees, and using $K_X = - c_1$ and $c_2 = - K_X^2 + 12(p_a(X) + 1)$ we find,
\[ \deg{(c_2(\sN)} = 10 d + 2  K_X^2 - 12(p_a(X) + 1) + 5 K_X \cdot \iota^* H \]
Finally, $X = d H^2$ in $A^*(\P^4_k)$ so we know that $\deg{X \cdot X} = d^2$ and furthermore we have $\iota_* c_2(\sN) = X \cdot X$ and thus $\deg{c_2(\sN)} = d^2$ giving a relation,
\[ 10 d - d^2 + 2  K_X^2 - 12(p_a(X) + 1) + 5 K_X \cdot \iota^* H = 0 \]

\begin{enumerate}
\item 

\item Let $X \subset \P^4_k$ be a K3 surface. Then by definition $K_X = 0$ and $h^1(X, \struct{X}) = 0$ so, using Serre duality $h^2(X, \struct{X}) = h^0(X, \struct{X})$ since $\omega_X = \struct{X}$, we find $p_a(X) = 1$. Therefore,
\[ 10 d - d^2 = 24 \]
meaning that $d^2 - 10 d + 24 = (d - 4)(d - 6) = 0$ and thus $d = 4$ or $d = 6$.

\item Let $X \subset \P^4_k$ be an abelian surface. Then $K_X = 0$ and $c_1 = c_2 = 0$ so $p_a = -1$. Therefore,
\[ 10 d - d^2 = 0 \]
which implies that $d = 10$.

\item 
\end{enumerate}

\subsubsection{6.10}

Suppose that $X$ is an abelian $3$-fold with an embedding $\iota : X \embed \P^5$. Then consider the normal sequence,
\begin{center}
\begin{tikzcd}
0 \arrow[r] & \T_X \arrow[r] & \iota^* \T_{\P^5} \arrow[r] & \sN_{X/\P^5} \arrow[r] & 0
\end{tikzcd}
\end{center}
Therefore,
\[ c(\T_X ) c(\sN_{X/\P^5}) = c(\iota^* \T_{\P^5}) \]
However, $\T_X$ is trivial so $c(\T_X) = 1$ and therefore,
\[ c(\sN_{X/\P^5}) = \iota^* c(\T_{\P^5}) \]
From the Euler sequence,
\[ c(\T_{\P^5}) = (1 + H)^6 \]
In particular we find,
\[ c_3(\sN_{X/\P^5}) = {6 \choose 3 } \iota^* H^3 = 20 \iota^* H^3 \]
which is nonzero because $\iota_* c_3(\sN_{X/\P^5}) = 20 \iota_* \iota^* H^3 = 20 X \cdot H^3 = 20 d H^5$ where $d$ is the degree of $X$ in $\P^5$ and thus $\deg{c_3(\sN_{X/\P^5})} = 20 d$. However, $\sN_{X/\P^5}$ is a vector bundle of rank $\codim{X, \P^5} = 2$ and must have $c_3(\sN_{X/\P^5}) = 0$ leading to a contradiction. Thus $\T_X$ cannot be trivial so $X$ cannot be an abelian surface.

\subsection{B Transcendental Methods}

\newcommand{\h}{\mathfrak{h}}
\renewcommand{\C}{\mathbb{C}}

\subsubsection{6.1}


Consider the open unit disk $D^\circ \subset \C$. Let $X$ be a scheme of finite type over $\C$ such that $X_h \cong D^\circ$. Thus we must have $\dim{X} = 1$ and $\pi_1^{\et}(X) = 0$. Therefore, because curves of positive genus always admit \etale covers, we must have $X \cong \A^1$ or $X \cong \P^1$ (open subschemes of $\A^1$ involve removing finitely many points and thus are not simply connected). Clearly $\P^1$ cannot work because it is compact. Therefore we must have $X \cong \A^1$ in which case $X_h \cong \C$. However, I claim that $D^\circ$ is not biholomorphic to $\C$. To see this, notice that $D^\circ$ is biholomorphic to $\h = \{ z \in \C \mid \Im{z} > 0 \}$ via the map,
\[ z \mapsto i \cdot \frac{1 + z}{1 - z} \] 
Furthermore, there are no nonconstant maps $f : \C \to \h$ because then $\exp{(if)} : \C \to \C$ is bounded because $|e^{if}| = e^{- \Im{f}} \le 1$ and therefore constant by Liouville's theorem. Thus we cannot have a biholomorphic map $f : D^\circ \to \C$ showing that no such $X$ exists. 

\subsubsection{6.2}

Let $z_1, z_2, \dots \in \C$ be an infinite sequence with $|z_n| \to \infty$ as $n \to \infty$. Let $\I \subset \struct{\C}$ be the sheaf of ideals of holomorphic functions vanishing at all $z_n$. First we need to show that $\I$ is nonzero. Using the hypothesis that $|z_n| \to \infty$, the Weierstrass factorization theorem (or equivalently the solvability of the second cousins problem on a complex manifold with $\Pic{X} = 0$ using that the points $z_i$ are isolated and thus taking $f_i = z - z_i$ on a small disk about $z_i$) implies that there exists an entire function $f$ with a simple pole at each $z_i$. Thus $f \in \Gamma(\C, \I)$ so $\I \neq 0$. In particular, $V(\I) = \{ z_i \mid i \in \N \}$ is an infinite set and $V(\I) \neq \C$.
\bigskip\\
Now let $X = \A^1_\C$. Coherent sheaves of ideals $\J \subset \struct{X}$ correspond to Zariski closed subsets $Z \subset \A^1_\C$ which are finite (unless $\J = 0$) and thus $\J_h$ cannot correspond to $\I$ as sheaves of ideals because $\I$ cuts out an infinite subset. Explicilty, $\J = \wt{(p)}$ for some $p \in \C[z]$ because $\C[z]$ is a PID and $f$ has finitely many roots. Then $\J_h = (p) \cdot \struct{\C}$ which cannot equal $\I$ because $p \in \J_h$ viewed as a holomorphic function which has finitely many roots but every section of $\I$ (of which at least one exits) vanishes at all $z_i$ of which there are infinitely many. 
\bigskip\\
However, any section $s \in \Gamma(X, \I)$ is an entire function vanishing at the $z_i$ and thus $\frac{s}{f}$ is entire. Therefore $\I = (f) \cdot \struct{\C}$ which implies that $\I \cong \struct{\C} = (\struct{X})_h$ as coherent sheaves.

\begin{rmk}
To apply sovability of the second cousins problem, we need that the set of points $\{ z_i \}$ is discrete. Here we show that $\{ z_i \}$ being discrete is the same as $|z_n| \to \infty$. First, if $|z_n| \to \infty$ is it clear that $\{ z_i \}$ is discrete since all but finitely many have $|z_i| > M$ for each $M$ so $\{ z_i \} \cap D_M$ is finite and thus discrete because $\C$ is Hausdorff. Conversely, if $\{ z_i \}$ is discrete, then for each compact $\overline{D_M}$ we have $\{ z_i \} \cap \overline{D_M}$ is compact and discrete and thus finite. Therefore $|z_i| > M$ for all but finitely many $z_i$ for each $M > 0$ meaning that there is some $n_M$ such that $n \ge n_M \implies |z_n| > M$ implying that $|z_n| \to \infty$. 
\end{rmk}

\subsubsection{6.3 DO!!}


\subsubsection{6.4 DO!!}

\subsubsection{6.5 DO!!}

\subsubsection{6.6 DO!!}


\subsection{C Weil Conjectures}

\subsubsection{5.1}

Write,
\[ X = \bigcup_{\alpha} X_\alpha \]
as a disjoint union of locally closed subschemes. Then,
\[ N_r = \sum N_r(X_\alpha) \]
where if $X$ is finite type this sum will be finite meaning that only finitely many $N_r(X_\alpha)$ can be zero as $\alpha$ varies. Therefore,
\[ Z(X, t) = \exp{ \left( \sum_{r = 1}^\infty N_r \frac{t^r}{r} \right)} = \prod_{\alpha} \exp{ \left( \sum_{r = 1}^\infty N_r(X_\alpha) \frac{t^r}{r} \right)} = \prod_{\alpha} Z(X_\alpha, t) \]

\subsubsection{5.2}

Let $X = \P^n_k$ for $k = \FF_q$. Then,
\[ N_r(X) = \frac{q^{n+1} - 1}{q - 1} = 1 + \cdots + q^{n} \]
Therefore,
\[ \sum_{r = 1}^\infty N_r \frac{t^r}{r} = \sum_{i = 1}^n \sum_{r = 1} \frac{(q^i t)^r}{r} = - \sum_{i = 1}^n \log{(1 - q^i t)} \]
which shows that,
\[ Z(\P^n_k, t) = \exp{ \left( \sum_{r = 1}^\infty N_r \frac{t^r}{r} \right) } = \prod_{i = 1}^n \exp{\left(\log{(1 - q^i t)}\right)}^{-1} = \prod_{i = 1}^n \frac{1}{1 - q^i t} = \frac{1}{(1 - t)(1 - q t) \cdots (1 - q^n t)} \] 

\subsubsection{5.3}

Let $X$ be a scheme of finite type over $\FF_q$ and $\A^1$ be the affine line. Then $(X \times \A^1)(\FF_{q^r}) = X(\FF_{q^r}) \times \A^1(\FF_{q^r})$ and therefore,
\[ N_r(X \times \A^1) = N_r(X) q^r \]
which implies that,
\[ Z(X \times \A^1, t) = \exp{ \left( \sum_{r = 1}^\infty N_r(X \times \A^1) \frac{t^r}{r} \right) } = \exp{ \left( \sum_{r = 1}^\infty N_r(X) \frac{(qt)^r}{r} \right) } = Z(X, qt) \]
Then the Weil conjectures for $\P^n_k$ are clear.

\subsubsection{5.4}

For a scheme $X$ of finite type over $\Spec{\Z}$ we define,
\[ \zeta_X(s) = \prod_{x \in X^{(0)}} (1 - N(x)^{-s})^{-1} \]
where $X^{(0)}$ is the set of closed points of $X$ and $N(x) = \# \kappa(x)$. Then if $X$ is finite type over $\FF_q$ we see that, $N(x) = q^{\deg{x}}$ and therefore,
\[ \zeta_X(s) = \prod_{x \in X^{(0)}} (1 - q^{-s \deg{x}})^{-1} \]
so we let $t = q^{-s}$ and then,
\[ \zeta_X(s) = \prod_{x \in X^{(0)}} (1 - t^{\deg{x}})^{-1} \]
Therefore,
\[ \deriv{}{t} \log{\zeta_X(s)} = \sum_{x \in X^{(0)}} (\deg{x}) \frac{t^{\deg{x} - 1}}{1 - t^{\deg{x}}} = \sum_{x \in X^{(0)}} \sum_{r = 1}^\infty (\deg{x}) t^{r \deg{x}-1} = \sum_{r = 1}^\infty N_r t^{r - 1} = \deriv{}{t} \log{Z(X, t)} \]
because each $x \in X^{(0)}$ contributes exactly $\deg{x}$ points to $N_r$ if $\deg{x}$ divides $r$ and zero otherwise. Furthermore, the constant term of $\zeta_X(s)$ and $Z(X, t)$ are both $1$ and therefore,
\[ \zeta_X(s) = Z(X, q^{-s}) \]

\subsubsection{5.5 DO!!}

Let $X$ be a curve of genus $g$ over $k$. 

\subsubsection{5.6 DO!!}

\subsubsection{5.7 DO!!}

\end{document}