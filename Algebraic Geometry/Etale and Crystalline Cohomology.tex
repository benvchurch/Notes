\documentclass[12pt]{article}
\usepackage{import}
\import{./}{AlgGeoCommands}
\renewcommand{\U}{\mathfrak{U}}

\begin{document}

\section{The pro-\etale Topology}

\subsection{Problems with the \etale Topology}

Perhaps the most obviously disquieting fact about the \etale topology is that,
\[ H_{\et}^i(X, \Q_\ell) = \varprojlim_{n} H^i(X, \Z / \ell^n \Z) \otimes_{\Z_\ell} \Q_\ell \]
cannot be defined as a derived functor (sheaf cohomology) of a natural representing object $\Q_\ell$ in the \etale topos. This is the main problem we wish to fix.
\bigskip\\
Another problem arises as follows. The site of finite \etale covers is isomorphic to the site of finite continous $\pi_1^{\et}(X)$-sets and thus sheaves on the finite \etale site are equivalent to continuous $\pi_1^{\et}(X)$-modules. Since lcc sheaves are exactly those represented by finite \etale covers we conclude that these are equivalent to continuous $\pi_1^{\et}(X)$-modules. This transfers over to the case of lisse $\Z_\ell$-sheaves which are limits of $\Z/\ell^n\Z$-modules with a $\pi_1^{\et}(X)$-action and thus $\Z_\ell$-modules with a $\pi_1^{\et}(X)$-action. However, the problem arrises when we pass to $\Q_\ell$-sheaves. The correspondence between $\Q_\ell$-local systems and continuous $\Q_\ell$-representations of $\pi_1^{\et}(X)$ fails for non-normal schemes $X$. For example, let $X$ be $\P^1$ with $0$ and $\infty$ glued and take the $\Q_\ell$-local system $\F$ on $X$ given by identifying the fibers over $0$ and $\infty$ by the automorphism $\ell \in \Q_\ell^\times = \GL{1}{\Q_\ell}$ giving $\F_0 \to \F_{\infty}$ by $\Q_\ell \xrightarrow{\times \ell} \Q_\ell$. However, this cannot corespond to any continuous representation $\pi_1^{\et}(X) \to \GL{1}{\Q_\ell}$ sending $1 \in \pi_1^{\et}(X) = \hat{\Z}$ to $\ell$ because then the image of $\hat{\Z} \to \GL{1}{\Q_\ell}$ is invariant under multiplication by $\ell^{-1}$ and thus not bounded but $\hat{\Z}$ is compact so its image is compact giving a contradiciton. Thus suggests that the ``real'' $\pi_1(X)$ should be $\Z$ such that we do indeed get a representation $\Z \to \GL{1}{\Q_\ell}$.

\subsection{Preliminaries}

\newcommand{\proet}{\text{pro\'{e}t}}

\begin{defn}
A map $f : X \to Y$ of schemes is \textit{weakly \etale} if $f$ is flat and $\Delta_f : X \to X \times_Y X$ is flat.
\end{defn}

\begin{prop}
An \etale map (in fact any flat and unramified morphism) is weakly-\etale.
\end{prop}

\begin{proof}
An \etale map $f : X \to Y$ is flat and unramified so $\Delta_f : X \to X \times_Y X$ is an open immersion and thus flat.
\end{proof}

\begin{defn}
The pro-\etale site $X_{\proet}$ is the site of weakly-\etale $X$-schemes, with covers given by fpqc covers.
\end{defn}

\begin{thm}
Let $f : A \to B$ be a map of rings. 
\begin{enumerate}
\item $f$ is \etale if and only if $f$ is weakly-\etale and finitely presented.
\item If $f$ is ind-\etale (a filtered colimit of \etale $A$-algebras) then $f$ is weakly \etale.
\item If $f$ is weakly-\etale, then there exists a faithfully flat ind-\etale $f : B \to C$ such that $g \circ f$ is ind-\etale.
\end{enumerate}
\end{thm}

\begin{proof}
The diagonal $\delta : B \otimes_A B \to B$ is always surjective (a closed immersion) so if $\delta$ is flat then it is open and thus an open immersion so $A \to B$ is unramified but if $A \to B$ is unramified then $\delta$ is an open immersion and thus flat. Furthermore, \etale is equivalent to flat an unramified and finitely presented so we conclude the first part.
\bigskip\\
Because tensor products commute with colimits an ind-limit of flat $A$-algebras is flat. 
\end{proof}

\begin{rmk}
This shows that for a ring $A$, the ind-\etale $A$-algebras are cofinal for the weakly-\etale $A$-algebras and thus we can work with ind-\etale covers. Therefore $\Spec{A}_{\proet}$ has a basis of ind-\etale covers which is why its called the pro-\etale topology. 
\end{rmk}

\subsection{The pro-\etale Fundamental Group}

\newcommand{\Loc}{\mathrm{Loc}}
\newcommand{\Cov}{\mathrm{Cov}}

\begin{theorem}
Let $X$ be a connected scheme whose underlying topological space is locally noetherian. Then the following categories are equivalent,
\begin{enumerate}
\item The category $\Loc_X$ of locally constant sheaves on $X_{\proet}$.
\item The category $\Cov_X$ of \etale $X$-schemes satisfying the aluative criterion o properness. 
\end{enumerate}
\end{theorem}

\begin{defn}
A topological group $G$ is call a Noohi group if $G$ is complete, and admits a basis of open neighborhoods of $1$ given by open subgroups.
\end{defn}

\begin{defn}
Let $x$ be a geometric point of $X$. Equiping $\Cov_X$ (or equivalently $\Loc_X$) with the fiber functor $F_x$ gives a Galois theory giving rise to a Noohi group $\pi_1^{\proet}(X, x) = \Aut{F_x}$.
\end{defn}

\begin{prop}
There is an equivalence of categories between $\Q_\ell$-local systems on $X$ and finite dimensional continuous representations of $\pi_1^{\proet}$.
\end{prop}

\begin{rmk}
The above is also true with $\Q_\ell$ replaced by $\overline{\Q}_\ell$.
\end{rmk}

\begin{prop}
The pro-finite completion of $\pi_1^{\proet}(X, x)$ is $\pi_1^{\et}(X, x)$ and the pro-discrete completion of $\pi_1^{\proet}(X, x)$ is $\pi_1^{\text{SGA3}}(X, x)$. 
\end{prop}

\end{document}