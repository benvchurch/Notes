\documentclass[12pt]{article}
\usepackage{import}
\import{./}{AlgGeoCommands}

\begin{document}

\section{The Chow Group}

\subsection{Flat Pullback}


\subsection{Proper Pushforward}

\section{Introduction to Intersection}

Let $X$ be an integral scheme proper over $S = \Spec{R}$ of dimension $2$ with $R$ noetherian. Given integral closed subschemes $C, D \subset X$ we want to make sense of the intersection $C \frown D$. For simplicitly, suppose that $C, D$ are prime Cartier divisors. Then we would want the intersection multiplicity at $x \in X$ to be,
\[ \iota_x(C, D) = \ell_{\stalk{X}{x}} \left( \stalk{X}{x}/(f, g) \right) \]
where $f, g$ are the local equations cutting out $C$ and $D$ i.e. there is an affine open neightborhood $x \in U = \Spec{A}$ with $C \cap U = V(f)$ and $D \cap U = V(g)$. I claim these intersection multiplicities piece together to give a meaningful cycle in $A_0(X)$. 

\begin{defn}
Let $X$ be a noetherian scheme and $Z_i \subset X$ its irreducible components. Then the fundamental class of $X$ is,
\[ [X] := \sum_{i = 1}^n m_i [Z_i] \in A_*(X) \]
where the multiplicities are,
\[ m_i = \ell_{\stalk{X}{\xi_i}}\left( \stalk{X}{\xi_i} \right) \] 
where $\xi_i$ is the generic point of $Z_i$.
\end{defn}

\begin{example}
Let $X = \Spec{k[x,y]/(xy^2)}$. Then the irreducible components are $Z_1 = V(x)$ and $Z_2 = V(y)$ with generic points $\xi_1 = (x)$ and $\xi_2 = (y)$. However, notice that,
\[ \stalk{X}{\xi_1} = k(y) \]
is a field so $m_1 = 1$ but,
\[ \stalk{X}{\xi_2} = k(x)[y]/(y^2) \]
has length $2$ over itself with submodule $(0) \subset (y) \subset \stalk{X}{\xi_2}$ reflecting the doubling of the $x$-axis. Therefore,
\[ [X] = [Z_1] + 2 [Z_2] \]
\end{example}

\begin{example}
Suppose that $X$ is noetherian dimension zero scheme. Then $X = \Spec{A}$ for an artinian ring $A$. Then,
\[ [X] = \sum_{\m \subset A}  \ell_{A_\m} \left( A_\m \right) \cdot [V(\m)] \]
and we see that,
\[ \deg{[X]} = \sum_{\m \subset A} \ell_{A_\m} \left( A_\m \right) = \ell_{A} \left( A \right) \]
\end{example}


\begin{rmk}
When $C$ and $D$ intersect properly, i.e. $\dim{(C \cap D)} = 0$, we might define the intersection class of $C, D \subset X$ as follows. Let $(C \cap D) \subset X$ be the scheme theoretic intersection,
\begin{center}
\begin{tikzcd}
C \cap D \arrow[dr, phantom, "\usebox\pullback" , very near start, color=black] \arrow[d, hook] \arrow[r, hook] & D \arrow[d, hook]
\\
C \arrow[r, hook] & X
\end{tikzcd}
\end{center}
then define $C \frown D = \iota_* [C \cap D]$ where $\iota : C \cap D \to X$ is the inclusion. If we take a point $x \in C \cap D$ and a sufficiently small open neighborhood $U = \Spec{A}$ then notice $C \cap D \cap U = \Spec{A / (f, g)}$ which is artinian so,
\begin{align*}
[C \cap D \cap U] & = \sum_{\m \in V(f, g)} \ell_{A_\m / (f, g)} \left( A_\m / (f, g) \right) \cdot [\m] = \sum_{\m \in V(f, g)} \ell_{A_\m} \left( A_\m /(f, g) \right) \cdot [\m] 
\\
& = \sum_{x \in C \cap D \cap U} \ell_{\stalk{X}{x}} \left( \stalk{X}{x} / (f, g) \right) \cdot [x] 
\end{align*}
which agrees with our definition of the intersection multiplicity.
\end{rmk}

\begin{prop}
Suppose that $C, D \subset X$ are prime Cartier divisors intersecting properly. Then,
\[ C \frown D = (\iota_C)_* [c_1( \iota^*_C \struct{X}(D))] = \iota_* [C \cap D] \]
where $\iota_C : C \to X$ and $\iota : C \cap D \to X$ are the inclusions.
\end{prop}

\begin{proof}
Since $C$ is a curve, to compute $c_1$ of the line bundle $\L = \iota^*_C \struct{X}(D)$ we need a nonzero section. The effective divisor $D$ corresponds to a section $s_D \in \Gamma(X, \struct{X}(D))$ which pulls back to $s = \iota_C^* s_D$. Since the intersection is proper, $s$ is not identically zero and therefore, 
\[ c_1(\L) = \sum_{p \in C} \ord_{p} \left( s / s_\L \right) \cdot [p] \]
where $s_\L$ is a local trivializing section of $\L$. Choose a sufficiently small affine open $U \subset X$ with $p \in U$ trivializing $\struct{X}(D)$ then $\struct{X}(D)|_U = g^{-1} \struct{U}$ and $s_D$ corresponds to $1$. Then we can take $s_\L = g^{-1}$ and $s_D = 1$ which gives,
\[ \ord_p(s / s_\L) = \ord_p(g) = \ell_{\stalk{C}{p}} \left( \stalk{C}{p}/(g) \right) = \ell_{\stalk{X}{p}} \left( \stalk{X}{p} / (f, g) \right) \]
Therefore,
\[ c_1(\L) = \sum_{p \in C} \ord_{p} \left( s / s_\L \right) \cdot [p] = [C \cap D]_C \]
and pushing forward by $\iota_C$ gives the desired result.
\end{proof}

\begin{rmk}
This result proves the symmetry,
\[ (\iota_C)_* [c_1( \iota_C^* \struct{X}(D) )] = (\iota_D)_* [c_1( \iota_D^* \struct{X}(C) )] \]
\end{rmk}

\begin{rmk}
Notice that even when $C$ and $D$ do not intersect properly the quantity,
\[ C \frown D = (\iota_C)_* [c_1(\iota_C^* \struct{X}(D))] \]
is well-defined. In particular, the self-intersection equals,
\[ C^2 := C \frown C = (\iota_C)_* [c_1 (\iota_C^* \struct{X}(C))] = (\iota_C)_* [c_1(\sN_{C/X})] \]
where $\sN_{C/X} = \iota_C^* \struct{X}(C) = \struct{C} \otimes \struct{X}(C) = \struct{X} / \I \otimes \I^\vee = (\I / \I^2)^\vee = \C_{C/X}^\vee$ is the normal bundle. In paricular,
\[ \deg{C^2} = \deg{\sN_{C/X}} \] 
\end{rmk}


\begin{rmk}
There is a more general formula due to Serre for the intersection multiplicity. Suppose that $C, D \subset X$ are closed subschemes. Let $Z \subset C \cap D$ be an irreducible component with generic point $\xi \in Z$. Then the multiplicity of $Z$ in $C \cap D$ is defined to be,
\[ \iota(Z; C, D) := \sum_{i = 0}^\infty (-1)^i \ell_{\stalk{X}{\xi}} \left( \Tor{\stalk{X}{\xi}}{i}{\stalk{X}{\xi}/I}{\stalk{X}{\xi}/J} \right) \]
where $I$ and $J$ are the ideals defining $C$ and $D$ in $\stalk{X}{\xi}$ and then the intersection cycle is,
\[ C \frown D = \sum_{Z \subset C \cap D} \iota(Z; C, D) \]
Notice that when $C$ and $D$ are prime Cartier divisors we get $I = (f)$ and $J = (g)$ and thus,
\[ \Tor{\stalk{X}{\xi}}{i}{\stalk{X}{\xi}/I}{\stalk{X}{\xi}/J}
= \begin{cases}
\stalk{X}{\xi}/(f,g) & i = 0
\\
\ker{(\stalk{X}{\xi}/(g) \xrightarrow{f} \stalk{X}{\xi}/(g))} & i = 1
\\
0 & i > 1
\end{cases} \]
because $f$ is a nonzerodivisor since we assumed that $C$ is Cartier. Furthermore, since the intersection is proper, we cannot have $f \in (g)$ and since $(g)$ is prime ($Z$ is a prime Cartier divisor) the map $\stalk{X}{\xi}/(g) \xrightarrow{f} \stalk{X}{\xi}/(g)$ is injective. Therefore,
\[ \Tor{\stalk{X}{\xi}}{i}{\stalk{X}{\xi}/I}{\stalk{X}{\xi}/J}
= \begin{cases}
\stalk{X}{\xi}/(f,g) & i = 0
\\
0 & i > 0
\end{cases} \]
giving,
\[ \iota(Z; C, D) = \ell_{\stalk{X}{\xi}} \left( \stalk{X}{\xi}/(f, g) \right) \]
which agrees with our previous formula.
\end{rmk} 

\subsection{Adjunction}

Given a smooth subvariety $Z \subset X$ of a smooth variety $X$, we know
\[ \omega_Z = \omega_X |_Z \otimes \bigwedge^{\mathrm{top}} \sN_{Z|X} \]
Therefore, taking chern classes,
\[ c_1(\omega_Z) = \iota^* c_1(\omega_X) + c_1(\sN_{Z|X}) \]
and thus we find,
\[ K_Z = K_X |_Z + c_1(\sN_{Z|X}) \]

\subsubsection{Divisors}

In particular, if $Z = V(D)$ for some divisor $D$ then if $Z$ is smooth,
\[ \sN_{Z | X} = \iota^* \struct{X}(D) \]
and therefore,
\[ \omega_Z = \iota^* (\omega_X \otimes \struct{X}(D)) = \omega_X |_Z \otimes \struct{D}(D) \]
meaning,
\[ K_Z = (K_X + D)|_Z \]
In fact, even when $Z$ is not smooth we can compute,
\[ \omega_Z = \iota_* \shExt{1}{\struct{X}}{\iota_* \struct{Z}}{\omega_X} \]
However, there is a locally-free resolution,
\begin{center}
\begin{tikzcd}
0 \arrow[r] & \struct{X}(-D) \arrow[r] & \struct{X} \arrow[r] & \iota_* \struct{Z} \arrow[r] & 0
\end{tikzcd}
\end{center}
and then we get an exact sequence,
\begin{center}
\begin{tikzcd}
0 \arrow[r] & \shHom{\struct{X}}{\iota_* \struct{Z}}{\omega_X} \arrow[r] & \omega_X \arrow[r] & \omega_X(D) \arrow[r] & \iota_* \omega_Z \arrow[r] & 0
\end{tikzcd}
\end{center}
Therefore, $\iota_* \omega_Z$ is the cokernel of the map $\omega_X \to \omega_X(D)$ defined by $\id_{\omega_X} \otimes s_D$ where $s_D : \struct{X} \to \struct{X}(D)$ is the canonical section corresponding to the inclusion $\struct{X}(-D) \to \struct{X}$. Therefore,
\[ \iota_* \omega_Z = \omega_X \otimes \iota_* \struct{Z} \otimes \struct{X}(D) = \omega_X(D) \otimes \iota_* \struct{Z} \]
because $\omega_X$ is locally free.
By the projection formula, 
\[ \iota_* \iota^* \omega_X(D) = \iota_* (\struct{Z} \otimes \iota^* \omega_X(D)) = \iota_* \struct{Z} \otimes \omega_X(D) \]
and therefore,
\[ \omega_Z = \iota^* \omega_X(D) \]

\subsection{Adjunction for Surfaces}

Let $X$ be a smooth surface which is a complete intersection in $P = \P^{c+2}$. Then $X \subset \P^{c+2}$ is cut out by $r$ equations $f_1, \dots, f_r$ of degrees $d_1, \dots, d_c$. Because $\dim{X} = 2$ these form a regular sequence meaning the Kozul complex is exact,
\begin{center}
\begin{tikzcd}
0 \arrow[r] & \struct{P}(-\sum\limits_{i = 1}^c d_i) \arrow[r] & \bigoplus\limits_{i = 1}^c \struct{P}(-\sum\limits_{j \neq i} d_j) \arrow[r] & \cdots \arrow[r] & \bigoplus\limits_{i = 1}^c \struct{P}(-d_i) \arrow[r] & \struct{P} \arrow[r] & \struct{X} \arrow[r] & 0
\end{tikzcd}
\end{center}
which gives a locally free resolution of $\struct{X}$. Therefore, we can compute,
\[ \iota_* \omega_X = \shExt{c}{\struct{P}}{\struct{X}}{\omega_P} \]
using this resolution via,
\[ \iota_* \omega_X = H^c(\shHom{\struct{P}}{K_\bullet}{\omega_P}) = \coker{ \left( \bigoplus\limits_{i = 1}^c \omega_P( \sum_{j \neq i} d_j) \xrightarrow{f_1, \dots, f_c} \omega_P(\sum_{i = 1}^c d_i) \right)} = \omega_P(d_1 + \cdots + d_c) \otimes_{\struct{P}} \struct{X} \]
Therefore, we find that,
\[ \omega_X = \omega_P(d_1 + \cdots + d_c) |_X = \struct{X}(d - c - 3) \]
where $d = d_1 + \cdots + d_c$ is the total degree. Therefore, when $X$ is smooth we find,
\[ K_X = (d - c - 3) H \]
where $H = \iota^* c_1(\struct{P}(1))$ is the hyperplane class of the embedding $X \embed P$. 
\bigskip\\
Now consider a divisor $C \subset X$. By the adjunction formula,
\[ \omega_C = \omega_X(C)|_C \]
Therefore, by Riemann-Roch for singular curves,
\[ 2 p_a - 2 = \deg{\omega_C} = \deg{\omega_X(C)|_C} \]
However, we have shown,
\[ D \frown (K_X + C) = c_1(\omega_X(C)|_C) \]
and therefore,
\[ D \cdot (K_X + C) = \deg{[D \frown (K_X + C)]} = \deg{\omega_X(C)|_C} \]
so we find that,
\[ 2 p_a - 2 = C \cdot (K_X + C) \]
However, 
\[ C \cdot K_X = \deg{\omega_X |_C} = \deg{\struct{P}(d - c - 3)|_C} = (d - c - 3) \deg{\struct{P}(1)|_C} \]
we define $\deg{C} = \deg{\struct{P}(1)|_C}$ which implies that,
\[ C \cdot K_X = (d - c - 3) \deg{C} \]
Alternatively, we can use adjunction,
\[ C \cdot K_X = C \cdot (d - c - 3) H = (d - c - 3) C \cdot H \]
and we write $\deg{C} = C \cdot H$ for the intersection of $C$ with a generic hyperplane. Therefore,
\[ 2 p_a - 2 = C^2 + (d - c - 3) \deg{C} \]
In particular, consider the case of $(-1)$-curves i.e. rational curves with $C^2 = -1$. Then we find,
\[ \deg{C} = \frac{1}{3 + c - d} \]
Therefore, we can only have $(-1)$-curves when $d = c + 2$ i.e. when $\omega_X = \struct{X}(-1)$. In this case, $\deg{C} = 1$ meaning that the $(-1)$-curves are lines in $P$. Furthermore, lines in $P$ have $\deg{C} = 1$ and $p_a = 0$ meaning that for a line $L \subset X$ it has self-intersection,
\[ L^2 = c + 1 - d \]

\section{General Intersection Theory}

\section{Chern Classes}

\end{document}