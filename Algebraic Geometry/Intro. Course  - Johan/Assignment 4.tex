\documentclass[12pt]{article}
\usepackage{import}
\import{./}{Includes}

\begin{document}

\atitle{4}

\renewcommand{\P}{\mathbb{P}}

\section{Problem a}
Let $K / k$ be a finitely generated (as a $k$-algebra) extension of fields. Consider the variety over $k$ given by $X = \Spec{K}$. The inclusion map $k \to K$ induces a morphism of schemes $\Spec{K} \to \Spec{k}$ making $X$ a scheme over $k$. We may compute the function field via $k(X) = \stalk{X}{x} / \m_x = \stalk{X}{x} = K$.   Furthermore, $\Spec{K}$ is a one-point space and thus, I claim, a variety over $k$. Since $\Spec{K}$ is a sheaf of fields (and the zero ring over the empty set) over one point it is clearly reduced and irreducible. Therefore, it suffices to show that $\Spec{K}$ is separated of finite type over $k$. Since we assumed that $K$ is a finitely generated $k$-algebra, the map $k \to K$ is finite type and thus $\Spec{K} \to \Spec{k}$ is a finite type morphism by definition (the morphism is clearly quasi-compact since all subsets are quasi-compact). Finally, the diagonal morphism $X \xrightarrow{\Delta} X \times X$ corresponds to the morphism of $k$-algebras $K \to K \otimes_k K$ given by sending $x \mapsto x \otimes_k x$. We need to show that $\Delta$ is a closed immersion. Since every nonempty open subset of $X$ is affine (there is only one) $\Delta$ is clearly affine and the corresponding map $K \otimes_k K \to K$ on the whole affine open is surjective. Furthermore, since $K \otimes_k K \to K$ is surjective, the preimage of the maximal ideal $(0)$ (i.e. the kernel of the multiplication map) is maximal and thus closed in $\Spec{K \otimes_k K} = X \times X$. Therefore, the image of $X$ is closed we have shown that the diagonal morphism $X \xrightarrow{\Delta} X \times X$ is a closed immersion proving that $\Spec{K}$ is indeed separated and thus a variety over $k$. 
\bigskip\\
Now let $K / k$ be a finitely generated (as fields) extension of fields. Then there exist $t_1, \dots, t_n \in K$ such that $k(t_1, \dots, t_n) = K$ take the domain $A = k[t_1, \dots, t_n] \subset K$ and consider the affine scheme $X = \Spec{A}$ over $k$. I claim that $X$ is a variety. Since $X$ is affine it is separated and since its coordinate ring $A$ is a domain $X$ is integral. Finally, the $k$-algebra morphism $k \to k[t_1, \dots, t_n]$ is clearly of finite type so the corresponding morphism of schemes $X \to \Spec{k}$ makes $X$ a scheme of finite type over $k$. Thus $X$ is a variety. Furthermore, the function field of $X$ is,
\[ k(X) = k(\Spec{A}) = \Frac{A} = k(t_1, \dots, t_n) = K \]

\section{Problem b}

Suppose that $A \subset B$ is an extension of domains such that $B$ is a finitely generated $A$-algebra and such that the inclusion map $\iota : A \to B$ induce and isomorphism $\iota : \Frac{A} \xrightarrow{\sim} \Frac{B}$. Since $B$ is finitely generated, there exists a surjective map $A[x_1, \dots, x_n] \to B$. Let $e_i$ be the image of $x_i$ under this surjective map. Since the map $\Frac{A} \to \Frac{B}$ is a surjection and the inclusion $B \to \Frac{B}$ is an injection (since $B$ is a domain) we know that any element $b \in B$ can be written as a fraction $b = \frac{a}{d}$ for $a,d \in A$. Furthermore, we know that any element $d \in A$ can be generated by the elements $f_1, \dots, f_n$. So we may take $a_i d_i \in A$ such that $\frac{a_i}{d_i} = e_i$. Now take $f = d_1 \cdots d_n$. I claim that $A_f = B_f$. Clearly, $A_f \subset B_f$ so it suffices to show that any element $\frac{b}{f^k} \in B_f$ is contained in $A_f$. Since $e_i$ generate $B$ it furthermore suffices to show that $e_i \in A_f$. However, this is clear because $e_i = \frac{a_i}{b_i}$ and,
\[ b_i^{-1} = \frac{\prod\limits_{j \neq i} b_i}{f} \]
is an element of $A_f$. Therefore $A_f$ contains the generators of $B_f$ so $A_f = B_f$. Finally, $f = d_1 \cdots d_n$ is nonzero because $A$ is a domain. 

\section{Problem c}

Let $X$ and $Y$ be varieties over $k$ such that $k(X) \cong k(Y)$ as $k$-algebras i.e. the varieties $X$ and $Y$ are birational over $k$. We may restrict our attention to fixed affine opens of $X$ and $Y$ or equivalently to the case that $X = \Spec{A}$ and $Y = \Spec{B}$ are affine. Since $X$ any $Y$ are varieties over the field $k$, the rings $A$ and $B$ are finitely generated $k$-algebra domains. Since we may compute the function field on any affine open, $k(X) \cong \Frac{A}$ and $k(Y) \cong \Frac{B}$ so we have $\Frac{A} \cong \Frac{B}$. Now consider,
\begin{center}
\begin{tikzcd}
A \arrow[d, hook] & B \arrow[d, hook]
\\
\Frac{A} \arrow[r, "\sim"] & \Frac{B}
\end{tikzcd}
\end{center}
so $A$ injects into $\Frac{B}$. Let $C$ be the subring of $\Frac{B}$ generated by the image of $A$ and $B$ which is also a finitely generated $k$-algebra by Lemma \ref{compositum_fin_gen} and a domain since $C \subset \Frac{B}$. The inclusion $A \subset C$ makes $C$ a finitely-generated $A'$-algebra (since both are finitely generated $k$-algebras) and since $B \subset C \subset \Frac{B}$,
\[ \Frac{A} \cong \Frac{B} = \Frac{C}  \] Therefore, by the previous problem, there exists $g \in B$ such that $B_g \cong C_g$. Since $C$ is a domain the map $C \to C_g$ is injective. These maps give an inclusion,
\begin{center}
\begin{tikzcd}
A \arrow[r, hook] & C \arrow[r, hook] & C_g \arrow[r, "\sim"] & B_g 
\end{tikzcd}
\end{center}
Furthermore $\Frac{A} \cong \Frac{B_g}$ and $B_g = B[g^{-1}]$ is a finitely-generated $A$-algebra domain so there exists $f \in A$ such that $A_f \cong (B_g)_f = B_{f'g}$ where $f \in A$ has image in $B_g \subset \Frac{B}$ and thus has denominator a power of $g$ which we multiply out to get $f'$. This isomorphism gives us an isomorphism of affine opens,  
\begin{center}
\begin{tikzcd}
\Spec{A} & \Spec{B}
\\
\Spec{A_f} \arrow[u, hook] \arrow[r, "\sim"] & \Spec{B_{f'g}} \arrow[u, hook]
\end{tikzcd}
\end{center}
Proving the proposition. Note that all ring maps produces are actually maps of $k$-algebras and thus the induced morphisms of schemes are indeed morphisms of schemes over $\Spec{k}$ as required for the induced isomorphism of affine opens to be an isomorphism as varieties over $k$.  

\section{Problem d}

\begin{remark}
I read this problem incorrectly and thought it was maps $X \to \A^1_k$ rather than $\A^1_k \to X$ so I kept both solutions but present the relevant one first.
\end{remark}

Let $k$ be an algebraically complete field. Take the affine surface,
\[ X = \Spec{k[x, y, z]/(xyz - 1)} \]
and let $A = k[x, y, z]/(xyz - 1)$ such that $X = \Spec{A}$. The polynomial $xyz - 1$ is irreducible so $(xyz - 1)$ is prime and has height $1$ (because $xyz - 1$ is minimal over zero. Therefore, since $k[x, y, z]$ is a f.g. $k$-algebra domain,
\[ \dim{A} = \dim{k[x, y, z]} - \height{(xyz - 1)} = 2 \]
Therefore, $\Spec{A}$ is an affine variety of dimension two and thus a surface. 
Consider the maps,
\[ \Homover{\Sch(k)}{\A_k^1}{\Spec{A}} = \Homover{k-\text{alg}}{k[t]}{A} \]
Therefore, we need to consider all $k$-algebra morphisms $A  \to k[t]$ which is equivalent to a $k$-algebra map $k[x, y, z] \to k[t]$ such that the ideal $(xyz - 1)$ maps to zero. Such a map takes,
\begin{align*}
x & \mapsto f
\\
y & \mapsto g
\\
z & \mapsto h
\end{align*}
for polynomials $f,g,h \in k[t]$. However, we must have $xyz \mapsto 1$ so $fgh = 1$ which implies that $\deg{(fgh)} = 1$ and thus $f,g,h \in k^\times$ are units. Any prime $\p \subset k[t]$ cannot contain any units but must contain zero, $\p \cap k = (0)$. Thus, under any map $\phi : A \to k[t]$ which necessarily maps $A$ inside $k \subset k[t]$, we have $\phi^{-1}(\p) = \phi^{-1}(0) = \ker{\phi} = (x - f, y - g, z - h)$ such that $fgh = 1$ and $f,g,h \in k^\times$. This is a fixed closed point of $X = \Spec{A}$. Therefore, each point of $\A^1_k$ (i.e. primes of $k[t]$) maps to some given fixed closed point of $X$ so all morphisms of $k$-schemes $\A^1_k \to X$ is constant.    
\bigskip\\
For the opposite problem, take $\P^2_k = \Proj{k[x_0, x_1, x_2]}$. The scheme $\P^2_k$ is a surface over $k$ (see Lemma \ref{projective_space_is_variety}.) Furthermore $\mathbb{A}^1_k = \Spec{[t]}$ is an affine scheme over $k$ and thus we have the natural equivalence,
\[ \Homover{\mathbf{Sch}(k)}{\P^2_k}{\mathbb{A}^1_k} \cong \Homover{k-\mathbf{alg}}{k[t]}{\Gamma(\P^2_k, \struct{\P^2_k})} \]
Therefore, we need only consider the $k$-algebra maps $k[t] \to \Gamma(\P^2_k, \struct{\P^2_k})$. By Lemma \ref{projective_sections}, $\Gamma(\P^2_k, \struct{\P^2_k}) \cong k$. Therefore, we need to consider all $k$-algebra maps $k[t] \to k$. However, because these maps must preserve the $k$-algebra structure such a map is uniquely determined by the image of $t$. Let $\ev_x : k[t] \to k$ be the unique $k$-algebra map sending $t \mapsto x$ for $x \in k$. Now by the correspondence, these maps induce all morphisms of schemes $\P^2_k \to \A^1_k$ over $k$. Consider the point $p \in \P^2_k$ then the preimage of the map $\Gamma(\P^2_k, \struct{\P^2_k}) \to \stalk{\P^2_k}{p}$ must take the unique maximal ideal $\m_p$ to the unique prime (also maximal) ideal $(0)$ of $\Gamma(\P^2_k, \struct{\P^2_k}) \cong k$. Therefore, $\tilde{\ev}_x(p) = \ev_x^{-1}(\m_p) = \ev_x^{-1}(0) = (t - x) \in \A^1_k$. Therefore, the map $\tilde{\ev}_x : \P^2_k \to \A^1_k$ is the constant map sending $\P^2_k$ to the closed point corresponding to $x \in k$. Therefore any choice of map $\P^2_k \to \A^1_k$ of schemes over $k$ is constant. 

\section{Problem e}

Let $k$ be a field. We need to construct a surjective map $\A^1_k \to \P^1_k$. We will first consider the problem under base change $k \to \bar{k}$ to the algebraic closure. In terms of concrete classical varieties we can easily produce a morphism $\A^1_{\bar{k}} \to \P^1_{\bar{k}}$ via $s \mapsto [s : s^2 - 1]$. I claim that this map is surjective on the closed points which take the form $(x - s) \in \A^1_{\bar{k}}$ and $(x s - y t) \in \P^1_{\bar{k}}$ respectively. In the affine patch $\A^1_{\bar{k}} \subset \P^1_{\bar{k}}$ given by $[t : 1]$ we can find $s$ such that $[s : s^2 - 1] = [t : 1]$ since $s = (s^2 - 1) t$ has a solution in $s$ for each $t$ over an algebraically closed field. The ``point at infinity'' $[1 : 0]$ is the image of $s = \pm 1$. Since a morphism of varieties is determined on its closed points (and is a surjection when it surjects on closed points) we have constructed a surjection $\A^1_{\bar{k}} \to \P^1_{\bar{k}}$. Furthermore, we consider the base change,
\begin{center}
\begin{tikzcd}[row sep = large, column sep = large]
\A^1_k \arrow[r, dashed, two heads] & \P^1_k
\\
\A^1_{\bar{k}} \arrow[r, two heads] \arrow[u, two heads] & \P^1_{\bar{k}} \arrow[u] \arrow[u, two heads]
\end{tikzcd}
\end{center}
We need to show that this morphism descends to a map $\A^1_k \to \P^1_k$ making this square commute. However, the Galois action $\Gal{\bar{k}/k}$ on the schemes $\A^1_{\bar{k}}$ and $\P^1_{\bar{k}}$ commute with the constructed map since it is given by polynomials in the ground field. Therefore, pulling back an element of $\A^1_k$ to $\A^1_{\bar{k}}$ and applying the mapping to $\P^1_k$ is well-defined since all such pullbacks are permuted by the Galois action and are thus mapped to conjugates under $\A^1_{\bar{k}} \to \P^1_{\bar{k}}$. Finally, the descended map is also surjective because the square commutes and all other maps are surjective. 

\section{Lemmata}

\begin{lemma} \label{projective_sections}
Let $A$ be a ring. Then, $\Gamma(\mathbb{P}^n_A, \struct{\P^n_A}) \cong A$
\end{lemma}

\begin{proof}
$\P^n_A$ is covered by affine opens $D_+(x_i) \cong \Spec{(A[x_0, \dots, x_n]_{x_i})_0}$ where we must take the degree zero part. Therefore, by the sheaf property, the sequence,
\begin{center}
\begin{tikzcd}
0 \arrow[r] & \struct{\P^n_A}(\P^n_A) \arrow[r] & \prod\limits_{i = 0}^n \struct{\P^n_A}(D_+(x_i)) \arrow[r] & \prod\limits_{i,j} \struct{\P^n_A}(D_+(x_i) \cap D_+(x_j))
\end{tikzcd}
\end{center}
is exact. Thus $\struct{\P^n_A}{(\P^n_A)}$ is the kernel of the second map. Consider an arbitrary element $z = \left( \frac{s_1}{x_1^{r_1}}, \dots, \frac{s_n}{x_n^{r_n}} \right)$ where $s_i$ is homogeneous of degree $r_i$. Suppose $z$ is in the kernel then, in the $i,j$-entry, $z$ maps to, 
\[ \frac{s_i}{x_i^{r_i}} - \frac{s_j}{x_j^{r_j}} = 0 \]
which implies that $x_j^{r_j} s_i = x_i^{r_i} s_j$. However, since $x_i \neq x_j$ are irreducible we must have $x_i^{r_i} \divides s_i$ and $x_j^{r_j} \divides s_j$ i.e. $s_i = u_i x_i^{r_i}$ and $s_j = u_j x_j^{r_j}$. However both fractions are supposed to have degree zero so their quotient $u_i$ and $u_j$ must have degree zero meaning that $u_i, u_j \in A$. Furthermore, 
\[ \frac{s_i}{x_i^{r_i}} - \frac{s_j}{x_j^{r_j}} = 0 \implies u_i = u_j \]
Therefore $z = (u, \dots, u)$ so the kernel is isomorphic to $A$. 
\end{proof}

\begin{lemma} \label{projective_space_is_variety}
Let $k$ be an algebraically closed field then
the scheme $\P^n_k = \Proj{k[x_0, \dots, x_n]}$ is a variety over $k$ of dimension $n$.
\end{lemma}

\begin{proof}
Since $k[x_0, \dots, x_n]$ is a finitely generated $k$-algebra and $\Gamma(\P^n_k, \struct{\P^n_k}) = k$ then the correspondence
\[ \Homover{\Sch}{\P^n_k}{\Spec{k}} = \Homover{\Ring}{k}{\Gamma(\mathbb{P}^n_k, \struct{\P^n_k})} = \Homover{\Ring}{k}{k} \]
gives a canonical morphism $\P^n_k \to \Spec{k}$ (corresponding to $\id_k$) of finite type (since each affine open corresponds to a finitely generated $k$-algebra). Furthermore the affine open cover $D_+(x_i) = \Spec{k[x_0, \dots, x_n]_{(x_i)}}$ are integral domains so $\P^n_k$ is integral. Finally, $\P^n_k = \Proj{k[x_0, \dots, x_n]}$ is generically separated for any graded ring.  
\end{proof}

\begin{lemma} \label{compositum_fin_gen}
Let $A, B \subset D$ be finitely generated $k$-algebras. Then the subring $C \subset D$ generated by $A$ and $B$ is a finitely generated $k$-algebra. 
\end{lemma}

\begin{proof}
Since $A$ and $B$ are finitely generated $k$-algebras there are surjective maps $k[x_1, \dots, x_m] \to A$ and $k[x_1, \dots, x_m] \to B$. Then consider the maps,
\begin{center}
\begin{tikzcd}
k[x_1, \dots, x_{n + m}] \arrow[r, "\sim"] & k[x_1, \dots, x_m] \otimes_k k[x_1 \dots, x_n] \arrow[r, two heads] & A \otimes_k B \arrow[r, two heads] & C
\end{tikzcd}
\end{center}
each of which surjects. The last map $A \otimes_k B \to C$ is given by multiplication $a \otimes_k b \mapsto ab$ which clearly surjects onto $C$ since it is generated by all products of $A$ and $B$. 
\end{proof}

\end{document}
