\documentclass[12pt]{article}
\usepackage{import}
\import{./}{Includes}

\begin{document}

\atitle{1}

\section{027A}

\subsection{Exercise 103.6.3.}

\subsection{Exercise 103.6.5.}

Let $A$ be a ring and consider $X = \Spec{A}$. Consider an open cover $\mathcal{U}$ of $X$ so we have,
\[ \bigcup_{U \in \mathcal{U}} U = X \]
The sets $D(f)$ form a base of the topology on $X$. Since $\mathcal{U}$ is a cover of $X$, for any $x \in X$ there exists $U \in \mathcal{U}$ such that $x \in U$ and then by the base property $\exists f \in A$ such that $x \in D(f) \subset U$. Therefore there exists a set $S \subset A$ such that,
\[ \bigcup_{f \in S} D(f) = X \iff \bigcap_{f \in S} V(f) = \varnothing \]
and each $D(f)$ is contained in some $U \in \mathcal{U}$. However,
\[ \bigcap_{f \in S} V(f) = V(S) \]
Since no primes lie above the ideal $(S)$ then $(S) = A$ otherwise $A / (S)$ would be nonzero and thus contain a maximal ideal which would be a prime above $(S)$. Thus, $1 \in S$ so there exist a finite set $\{f_1, \dots, f_n \} \subset S$ and coefficients $a_1, \dots, a_n$ such that,
\[ a_1 f_1 + \cdots + a_n f_n = 1 \] 
Therefore, $V(\{f_1, \dots, f_n \}) = V(f_1) \cap \cdots \cap V(f_n) = \varnothing$ since no prime contains $1$. This is equivalent to,
\[ \bigcup_{i = 1}^n D(f_i) = X \]
which implies that a finite subset of $\mathcal{U}$ covers $X$ since each $D(f_i)$ is contained in some $U \in \mathcal{U}$.  

\subsection{Exercise 103.6.7.}

Let $A$ be a ring and consider $X = \Spec{A}$. Let $\p, \q \subset A$ be distinct prime ideals. By definition, if $\p \neq \q$ then one must contain an element that the other does not. WLOG, let $f \in \p$ but $f \notin \q$. Then $\q \in D(f)$ but $\p \notin D(f)$. Therefore, $D(f)$ is an open neighborhood of $\q$ in $X$ which does not contain $\p$ so $\p$ and $\q$ are topologically distinguishable and thus $\Spec{A}$ is a $T_0$ space. 

\section{028P}

Consider the one-point space $X$ with the following sheaf $\struct{X}$ defined by $\struct{X}(X) = \Z_p$ where $\Z_p$ are the $p$-adic integers for some prime $p$ (this may be replaced with any positive dimension local ring). At the unique point $x \in X = \{x\}$ there is only one open, namely $X$, containing $x$ so 
\[ \stalk{X}{x} = \varinjlim\limits_{x \in U} \struct{X} (U) = \struct{X}(X) = \Z_p \]
which is local so $(X, \struct{X})$ is a locally ringed space. If $(X, \struct{X})$ were a scheme then it must be covered by affine schemes but the one-point space can only be covered by itself so we must have $(X, \struct{X})$ itself be an affine scheme. However, if $(X, \struct{X}) \cong \Spec{A}$ then $\Gamma(X, \struct{X}) \cong \Gamma(\Spec{A}, \struct{\Spec{A}}) = A$ which implies that $A \cong \struct{X}(X) = \Z_p$. Furthermore, $\Spec{\Z_p}$ is a two-point space because $\dim{\Z_p} = 1$ and thus $X$ is not even homeomorphic to $\Spec{Z_\p}$ as topological spaces let alone as locally ringed spaces. Thus, $(X, \struct{X})$ is not a scheme.  

\section{028Q}

Let $X$ be a scheme with underlying space having two points. Since $X$ must be covered by affine schemes, the only possibility for $X$ to be non-affine is if each of the one-point sets contained in $X = \{x, y\}$ are affine schemes individually and are open. This puts the discrete topology on $X$ since all points are open. Thus we have $\struct{X}(\{x\}) = A$ and $\struct{X}(\{y\}) = B$ where $A$ and $B$ are local rings (they are stalks of a scheme) such that $\Spec{A} \cong (\{x\}, \struct{X}|_{\{x\}})$ and $\Spec{B} \cong (\{y\}, \struct{X}|_{\{y\}})$ which implies that $A$ and $B$ are artinian since they must be local rings of dimension $0$. There are restiction maps $\struct{X}(X) \to A$ and $\struct{X}(X) \to B$ and since $\{x\}$ and $\{y\}$ is an open cover of $X$, the sheaf property implies that any pair of sections $(a, b) \in A \times B$ lifts uniquely to $\struct{X}(X)$ which implies that $\struct{X}(X) \cong A \times B$ since it satisfies the universal property of the product (a pair of maps $Z \to A$ and $Z \to B$ lift uniquely to $\struct{X}(X)$  by the sheaf property and the fact that uniqueness of the lift plus the projections being ring maps implies that the lift is a ring map). Furthermore, 
\[ \Spec{A \times B} \cong \Spec{A} \coprod \Spec{B} = (\{x\}, \struct{X}|_{\{x\}}) \coprod (\{y\}, \struct{X}|_{\{y\}}) = X \]
Thus, $X \cong \Spec{A \times B}$ is affine.    

\section{028R}

Let $X = \{x, y, z\}$ have the toplogy defined by declaring $\varnothing$, $\{x\}$, $\{x,y\}$, $\{x,z\}$, $\{x,y,z\}$ to be the open sets. These open sets form the inclusion category (suppressing identity morphisms and initial objects),
\begin{center}
\begin{tikzcd}
& X
\\
\{x, y\} \arrow[ur] & & \{x,z\} \arrow[ul]
\\
& \{x\} \arrow[ul] \arrow[ur] 
\end{tikzcd}
\end{center}
 Next, we define the sheaf, $\struct{X}$ via,
\[ \begin{tikzcd}
& X
\\
\{x, y\} \arrow[ur] & & \{x,z\} \arrow[ul]
\\
& \{x\} \arrow[ul] \arrow[ur] 
\end{tikzcd} 
\mathlarger{\overset{\struct{X}}{\implies}}
\begin{tikzcd}
& \Z_p \arrow[dl, "\id"'] \arrow[dr, "\id"]
\\
\Z_p \arrow[dr] & & \Z_p \arrow[dl]
\\
& \Q_p
\end{tikzcd} \]
Then $X$ is a locally ringed space because the stalk at each point is $\Z_p$ or $\Q_p$ which are local. Furthermore, $(\{x,y\}, \struct{X}|_{\{x,y\}}) \cong \Spec{\Z_p}$ and $(\{x,z\}, \struct{X}|_{\{x,z\}}) \cong \Spec{\Z_p}$ is a cover of $X$ by affine schemes. Therefore, $X$ is a scheme. However, if $X$ were affine with $X \cong \Spec{A}$ then $\Gamma(X, \struct{X}) \cong \Gamma(\Spec{A}, \struct{\Spec{A}}) = A$ and thus $A \cong \struct{X}(X) \cong \Z_p$. However, $\Spec{\Z_p}$ which has two points so it cannot be isomorphic to $X$. Thus $X$ is a scheme but not an affine scheme.

\section{028W} 

Let $X$ be a scheme and a field $K$ with a morphism $\Spec{K} \to X$ \textit{as ringed spaces}. We need to examine the data required to define such a map. Since $\Spec{K}$ has one point $(0_K)$ the structure sheaf is defined simply via $\struct{\Spec{K}}(0_K) = K$. This morphism is a pair $(f, f^\#)$ with $f$ an inclusion of the one-point space into $X$ at  the point $f(0_K) = x_0 \in X$. Furthermore, the map $f^\# : \struct{X} \to f_* \struct{\Spec{K}}$ is a collection of maps, 
\[ f^{\#} : \struct{X}(U) \to \struct{\Spec{K}}(f^{-1}(U)) =  
\begin{cases}
K & x_0 \in U
\\
0 & x_0 \notin U
\end{cases} \]
We need to find a case in which the induced map on stalks is \textit{not} local. In particular, if the map,
\[ f^\#_{x_0} : \stalk{X}{x_0} \to K \]
is nontrivial on $\m_x$ then the map will not be local since $K$ has maximal ideal $(0)$ and then $(f, f^\#)$ will not be a map of schemes. To accomplish this, let $X = \Spec{\Z}$ and $K = \Q$ and send $f(0_\Q) = (p)$ for any nonzero prime $p \in \Z$ and take the maps $f^{\#} : \Z_f \to \Q$ defined on standrd opens $D(f)$ to be inclusions if $f \notin (p)$ and zero maps otherwise. Then the stalk map $f^{\#}_p : \Z_{(p)} \to \Q$ is the inclusion since it must be the inclusion on each localization $\Z_f$ building up the stalk. Thus the maximal ideal $p \Z_{(p)}$ does not map to zero in $\Q$ so this is not a morphism of schemes. We could immediately tell this is the case because $\Spec{Z}$ is terminal in the category of affine schemes so there is a unique morphism of schemes $\Spec{K} \to \Spec{Z}$ but this map sends the ideal $(0_K)$ to $(0_\Z)$ so it is not the map we defined.  

\end{document}
