\documentclass[12pt]{article}
\usepackage{import}
\import{./}{Includes}

\begin{document}

\atitle{5}


\section{Problem 1}

Let $k$ be a field and $X$ be a scheme over $\Spec{k}$. The map $f : X \to \Spec{k}$ gives a map on stalks $f^\# : k \to \stalk{X}{x} \to k(x)$ for each point $x \in X$. By Lemma \ref{functor_of_points}, a morphism $\Spec{k} \to X$ is determined exactly by specifying a point $x \in X$ and an inclusion $k(x) \to k$. However, a $k$-rational point is a morphism $\Spec{k} \to X$ \textit{as $k$-schemes} so the diagram,
\begin{center}
\begin{tikzcd}
\Spec{k} \arrow[rd, "\id"] \arrow[rr, "p"] & & X \arrow[ld, "f"]
\\
& \Spec{k}
\end{tikzcd}
\end{center} 
is required to commute. This implies that the induced map on stalks is required to commute,
\begin{center}
\begin{tikzcd}
k  & & k(x) \arrow[ll, "p^\#"']
\\
& k \arrow[ul, "\id"] \arrow[ur, "f^\#"']
\end{tikzcd}
\end{center} 
The commutativity of this diagram shows that $f^\# : k \to k(x)$ must be an isomorphism. Thus given a $k$-rational point $x$ we have shown that $f^\# : k \to k(x)$ is an isomorphism. Furthermore for any point $x \in X$ if $f^\# : k \to k(x)$ is an isomorphism then its inverse $p^\# : k(x) \to k$ clearly makes the diagram above commute and thus, by the Lemma, induces a morphism of $k$-schemes $\Spec{k} \to X$ so $x$ is $k$-rational. 
\bigskip\\
To show that all such points are closed, I will prove the stronger fact that if $k(x)$ is a finite extension of $k$ then $x$ is a closed point. Assume $k(x)$ is a finite extension of $k$. On each affine open $x \in U$, the corresponding prime $\p$ gives a domain $A / \p$ and thus inclusions
\begin{center}
\begin{tikzcd}
k \arrow[r, hook] & A / \p \arrow[r, hook] & S_\p^{-1} (A / \p) = A_\p / \p A_\p = k(p)
\end{tikzcd}
\end{center}
showing that $A / \p$ is a finite-dimensional $k$-algebra domain and thus a field. Therefore $\p$ is maximal and thus closed in $U$. Therefore we have shown that $x$ is closed in every affine open neighborhood. Therefore there exists a closed $C \subset X$ such that $C \cap U = \{ x \}$ and thus \[ U^C \cup \{ x \} = (U \setminus \{ x \})^C = (C^C \cap U)^C = C \cup U^C \]
is closed. Now let $\{ U_\alpha \}$ be an affine cover of $X$. If $x \in U_\alpha$ then we have shown that $U_\alpha^C \cup \{ x \}$ is closed otherwise $x \in U_\alpha^C$ so $U_\alpha^C \cup \{ x \}$ is closed. Therefore, using the fact that $U_\alpha$ cover $X$, the set
\[ \bigcap_{\alpha} U^C_\alpha \cup \{ x \} = \left( \bigcap_\alpha U_\alpha \right) \cup \{ x \} = \varnothing \cup \{ x \} = \{ x \} \]   
is closed. 

\section{0CYH}

\subsection{}
\renewcommand{\C}{\mathbb{C}}

Consider the ring $R = \C$ and the category $\Mod{\C}$ of $\C$-vectorspaces. Denote by $\Homover{\overline{\C}}{V}{W}$ the $\C$-vectorspace of $\C$-\textit{anti}-linear functions i.e. functions $\varphi : V \to W$ such that $\varphi(\lambda v) = \bar{\lambda} \varphi(v)$ for $\lambda \in \C$. This space is a $\C$-vector space under standard addition and multiplication because $\lambda \varphi$ is still anti-linear.
\bigskip\\
Define the contravariant functor $F : \Mod{\C} \to \Mod{\C}$ given by $F(V) = \Homover{\overline{\C}}{V}{\C}$ what I might call the \textit{anti}-dual space. For maps $f : V \to W$ and $f \in \Homover{\overline{\C}}{W}{\C}$ take $F(f) : \varphi \mapsto \varphi \circ f$. Then, since $f$ is $\C$-linear and $\varphi$ is $\C$-anti-linear, 
\[ \varphi \circ f(\lambda v) = \varphi(\lambda f(v)) = \bar{\lambda} \varphi \circ f(v) \]
so $\varphi \circ f \in \Homover{\overline{\C}}{V}{\C}$. Clearly, $F$ is additive since function composition commutes with addition. Finally, consider, $F(\lambda f)(\varphi) = \varphi \circ (\lambda f) = \bar{\lambda} (\varphi \circ f)$. However, the map $(\lambda \cdot F(f))(\varphi) = \lambda (\varphi \circ f)$ is not equal, so $F$ is \textit{not} $\C$-linear but rather $\C$-\textit{anti}-linear.

\renewcommand{\C}{\mathcal{C}}

\subsection{}

Let $R$ be a commutative ring and $N$ and $R$-module. Consider the functor $F : \Mod{R} \to \Mod{R}$ given by $F(M) = M \otimes_R N$. We know that $F$ is left-adjoint to the internal hom functor $\Homover{R}{N}{-}$ i.e. there is a natural isomorphism,
\[ \Homover{R}{M \otimes_R N}{K} \cong \Homover{R}{M}{\Homover{R}{N}{K}} \]
Therefore, by general abstract nonsense (see Lemma \ref{abstract_nonsense}) $F$ preserves all colimits. In particular $F$ preserves cokernels and therefore is right-exact and $F$ preserved all coproducts and thus all direct sums in the category $\Mod{R}$. Finally, take a map $f : M \to M'$ and consider $F(rf) = (rf) \otimes \id_N : M \otimes_R N \to M' \otimes_R N$. However, 
\[ ((rf) \otimes \id_N)(m \otimes n) = (rf(m)) \otimes n = r(f(m) \otimes n) = r (f \otimes \id_N)(m \otimes n) \]
and therefore $F(rf) = r F(f)$ so $F$ is $R$-linear.  

\subsection{}

Let $F : \Mod{R} \to \Mod{R}$ be a $R$-linear, right-exact functor preserving all direct sums. First we will consider the action of $F$ on free modules. Let $I$ be some index set and take,
\[ P = \bigoplus_{i \in I} R \]
Because $F$ preserves arbitrary direct sums (coproducts) we have,
\[ F(P) = F\left( \bigoplus_{i \in I} R  \right) = \bigoplus_{i \in I} F(R) = \bigoplus_{i \in I} (R \otimes_R F(R)) = \left( \bigoplus_{i \in I_1} R \right) \otimes_R F(R) = P \otimes_R F(R) \]
where we have used the fact that tensor product commutes with arbitrary direct sums. We now need to show that these functors are \textit{naturally} equivalent on free objects. Let $\eta_P : F(P) \to P \otimes_R F(P)$ be the isomorphism constructed above. Let,
\[ P_1 = \bigoplus_{i \in I_1} R \quad \quad P_2 = \bigoplus_{i \in I_2} R \]
then consider a map $f : P_1 \to P_2$. Since $P_1$ is free this map is equivalent to a sequence of maps $f_i : R \to P_2$ for $i \in I_1$. Using the explicit construction of the coproduct in the category $\Mod{R}$ we have maps,
\begin{center}
\begin{tikzcd}
R \arrow[r, hook, "\iota_i"] & \bigoplus\limits_{i \in I_1} R \arrow[r, "f"] & \bigoplus\limits_{r \in I_2} R \arrow[r, hook] & \prod\limits_{i \in I_2} R \arrow[r, two heads, "\pi_j"] & R
\end{tikzcd}
\end{center}
notate the composition by $f_{ij} : R \to R$ and $f_i = f \circ \iota_i : R \to P_2$. Since $f_{ij}$ is an $R$-module map it is uniquely determined by $f_{ij}(1) = r_{ij} \in R$. We may intrinsically define the projection maps $\pi_j : R^{I_2} \to R$ via the universal property applied to the maps $\id_R : R_i \to R_i$ on factor $i$ and the zero map on all other factors. Therefore, because $F$ preserves the universal property of the coproduct it preserves these projection and inclusion maps. 
Now consider the diagram,
\begin{center}
\begin{tikzcd}[column sep = huge, row sep = huge]
F(P_1) \arrow[d, "\sim"]  \arrow[r, "F(f)"] & F(P_2) \arrow[d, "\sim"] 
\\ 
\bigoplus\limits_{i \in I_1} F(R) \arrow[d, "\eta_1"] \arrow[r, "\oplus F(f_i)"] & \bigoplus\limits_{i \in I_2} F(R) \arrow[d, "\eta_2"]
\\
P_1 \otimes_R F(R) \arrow[r, "f \otimes \id_{F(R)}"] & P_2 \otimes_R F(R)  
\end{tikzcd}
\end{center}   
The upper square commutes giving a natural isomorphism because $F$ preserves arbitrary direct sums. We must show that the lower square commutes. The maps $\eta_P$ take a sequence 
\[ (a_i) \in \bigoplus_{i \in I} F(R) \]
 under the isomorphisms,
\begin{center}
\begin{tikzcd}
\bigoplus\limits_{i \in I}  F(R) \arrow[r] & \bigoplus\limits_{i \in I} (R \otimes_R F(R)) \arrow[r] & \left( \bigoplus\limits_{r \in I} R \right) \otimes_R F(R)
\end{tikzcd}
\end{center}
to
\[ (a_i) \mapsto (1 \otimes a_i) \mapsto \sum_{i \in I} \delta_i \otimes a_i \]
where I have defined the sequence $\delta_i = \iota_i(1) \in P = R^I$. Then,
\begin{align*}
(f \otimes \id_{F(R)}) \circ \eta_1((a_i)) = (f \otimes \id_{F(R)})\left( \sum_{i \in I_1} \delta_i \otimes a_i \right) = \sum_{i \in I_1} f(\delta_i) \otimes a_i = \sum_{i \in I_1} f_i(1) \otimes a_i
\end{align*}
Because, $f(\delta_i) = f \circ \iota_i(1) = f_i(1)$.
\bigskip\\
Next, consider,
\begin{align*}
\eta_2 \circ \oplus F(f)((a_i)) = \eta_2 \left( \sum_{i \in I_1} F(f_i)(a_i) \right) = \sum_{i \in I_1} \eta_2(F(f_i)(a_i)) = \sum_{i \in I_1} \sum_{j \in I_2} \delta_j \otimes F(\pi_j) \circ F(f_i)(a_i)
\end{align*} 
because projecting a sequence in $F(R)^{I_2}$ to its components uses the map $F(\pi_2)$ which lifts $\id_{F(R)} : F(R) \to F(R)$ exactly on factor $i$ and zero elsewhere.
Therefore,
\[ F(\pi_j) \circ F(f_i) = F(\pi_j \circ f_i) = F(f_{ij}) \]
However, $f_{ij} = r_{ij} \id_{R}$ so, using the fact that $F$ is an $R$-linear functor, we find that,
\[ F(f_{ij}) = F(r_{ij} \id_{R}) = r_{ij} F(\id_R) = r_{ij} \id_{F(R)} \]
Therefore, 
\[ \eta_2 \circ \oplus F(f)((a_i)) = \sum_{i \in I_1} \sum_{j \in I_2} \delta_j \otimes r_{ij} a_i = \sum_{i \in I_1} \sum_{j \in I_2} r_{ij} \delta_j \otimes a_i \]
Furthermore, summing over the support of $f_i(1)$ we find,
\[ f_i(1) = \sum_{j \in I_2} \iota_j \circ \pi_j \circ f_i(1) = \sum_{j \in I_2} \iota_j \circ f_{ij}(1) = \sum_{j \in I_2} \iota_j(r_{ij}) = \sum_{j \in I_2} r_{ij} \iota_j(1)  = \sum_{j \in I_2} r_{ij} \delta_j  \]
Finally,
\[ \eta_2 \circ \oplus F(f)((a_i)) = \sum_{i \in I_1} \sum_{j \in I_2} r_{ij} \delta_j \otimes a_i = \sum_{i \in I_1} f_i(1) \otimes a_i = (f \otimes \id_{F(R)}) \circ \eta_1((a_i)) \]
which proves that these isomorphisms are natural. 
\bigskip\\
To prove the proposition, take $M \in \Mod{R}$ and take the first two terms of any free resolution of $M$,
\begin{center}
\begin{tikzcd}
P_1 \arrow[r, "f"] & P_0 \arrow[r] & M \arrow[r] & 0
\end{tikzcd}
\end{center}
Now applying both the functor $F$ and the functor $(-) \otimes_R F(R)$ to this sequence and using the natural isomorphism defined above gives a commutative diagram,
\begin{center}
\begin{tikzcd}[column sep = large, row sep = large]
F(P_1) \arrow[d, "\eta_{P_1}"] \arrow[r, "F(f)"] & F(P_0) \arrow[d, "\eta_{P_2}"] \arrow[r] & F(M) \arrow[d, dashed] \arrow[r] & 0 
\\
P_1 \otimes_R F(R) \arrow[r, "f \otimes \id_{F(R)}"] & P_0 \otimes_R F(R) \arrow[r] & M \otimes_R F(R) \arrow[r] & 0
\end{tikzcd}
\end{center}
with exact rows because by $F$ and $(-) \otimes_R F(R)$ are right-exact i.e. preserve cokernels. Therefore, because the downward maps are isomorphisms then the induced map $F(M) \to M \otimes_R F(R)$ is an isomorphism. Because a morphism $M \to N$ lifts to a morphism of free resolutions over $M$ and $N$ this constructed isomorphism is natural. Therefore $F \cong (-) \otimes_R F(R)$.  

\subsection{}

Let $I$ be some infinite index set and consider the functor $F : \Mod{R} \to \Mod{R}$ given by $F(X) = \Homover{R}{R^{I}}{X}$ where 
\[ R^I = \bigoplus_{i \in I} R \]
I claim that $F$ is $R$-linear, right-exact but does not preserve arbitrary direct sums and thus cannot be tensor product by any fixed module (since that functor does preserve direct sums). First, take $r \in R$ and $f : A \to B$ then for $\varphi : R^I \to A$ we have $F(rf) : F(A) \to F(B)$ takes $F(rf) : \varphi \mapsto rf \circ \varphi = r(f \circ \varphi) = r F(f)$. Furthermore, from the definition of an abelian category, the hom functor is additive. Next, $R^I$ is a free $R$-module and therefore a projective which is equivalent to the functor $F(-) = \Homover{R}{R^I}{-}$ being exact (and in particular right-exact). Finally, consider,
\[ F\left( \bigoplus_{i \in I} R \right) = F(R^I) = \Homover{R}{R^I}{R^I} \]
We have the map $\id_{R^I} \in \Homover{R}{R^I}{R^I}$. However, I claim that,
\[ \id_{R^I} \notin \bigoplus_{i \in I} \Homover{R}{R^I}{R} \]
because $\id_{R^I}$ is nonzero projected onto each factor $p_i : R^I \to R$ and since $I$ is infinite this cannot be an element of the direct sum which only contains sequences with finite support. Therefore,
\[ F\left( \bigoplus_{i \in I} R \right) = \Homover{R}{R^I}{R^I} \neq \bigoplus_{i \in I} \Homover{R}{R^I}{R} = \bigoplus_{i \in I} F(R) \]
so $F$ does not preserve arbitrary coproducts and thus cannot be the tensor product functor with any fixed module. 


\section{Lemmas}

\begin{lemma} \label{functor_of_points}
Let $X$ be a scheme and $K$ a field. A morphism $\Spec{K} \to X$ is the same as specifying a point $p \in X$ and an inclusion $\iota : k(p) \to K$ where $k(p) = \stalk{X}{x} / \m_x$ is the residue field at $x$.
\end{lemma}


\begin{proof}
Let $(f, f^\#) : \Spec{K} \to X$ be a morphism. Then take the image $\{p\} = f((0))$. Furthermore, we have a sheaf map,
\[ f^\# : \struct{X}(U) \to \struct{\Spec{K}}(f^{-1}(U)) 
= \begin{cases}
K & p \in U
\\
0 & p \notin U 
\end{cases} \] 
Consider the commutative diagram,
\begin{center}
\begin{tikzcd}
\struct{X}(U) \arrow[d] \arrow[r, "f^\#"] & \struct{\Spec{K}}(f^{-1}(U)) \arrow[d]
\\
\stalk{X}{x} \arrow[r, "f^\#_x"] & \stalk{\Spec{K}}{(0)}
\end{tikzcd}
\end{center}
On opens $U$ with $p \notin U$ clearly the map $f^\# : \struct{X}(U) \to \struct{\Spec{K}}(f^{-1}(U))$ is the zero map. Otherwise, the map $\struct{\Spec{K}}(f^{-1}(U)) \to \stalk{\Spec{K}}{(0)}$ is the identity. Therefore, the above diagram determines $f^\# = f^\#_x \circ \mathrm{res}_{U, x}$ uniquely from the stalk map 
\[f^\#_x : \stalk{X}{x} \to \stalk{\Spec{K}}{(0)} = K \]
Furthermore, $f^\#_x$ must be a local so $f^\#_x(\m_x) = (0)$ since $(0)$ is maximal in $K$. Therefore, this map factors through $k(p) = \stalk{X}{x} / \m_x$. Therefore, $f^\#$ is determined from the map $k(p) \to K$ (which is an inclusion) via the canonical composition,
\begin{center}
\begin{tikzcd}
\struct{X}(U) \arrow[r] & \stalk{X}{x} \arrow[r] & \stalk{X}{x} / \m_x \arrow[r, "f^\#_x"] & K
\end{tikzcd}
\end{center} 
\end{proof}

\begin{lemma} \label{abstract_nonsense}
Let $F : \C \to \D$ and $G : \D \to \C$ be functors and left $F$ be left adjoint to $G$ that is $F \dashv G$. Then $F$ preserves all colimits and $G$ preserves all limits.
\end{lemma}


\begin{proof}
Let $\I$ be a fixed category and $J : \mathcal{I} \to \C$ some diagram. Let $\Delta : \C \to \C^{\I}$ be the constant functor (taking $A \in \C$ to the constant functor with image $A$). Then I claim that $\colim : \C^{\I} \to \C$ is left adjoint to $\Delta : \C \to \C^{\I}$. Therefore, for any $X \in \D$, consider,
\begin{align*}
\Homover{\D}{F(\colim{J})}{X} \cong \Homover{\C}{\colim{J}}{G(X)} \cong \Homover{\C^{\I}}{J}{\Delta \circ G(X)}
\end{align*} 
Any natural transformation $\eta : J \to \Delta \circ G(X)$ is a set of maps $\eta_A : J(A) \to G(X)$ for each $A \in J$ such that,
\begin{center}
\begin{tikzcd}[column sep = large, row sep = large]
J(A) \arrow[d, "\eta_A"'] \arrow[r, "J(f)"] & J(B) \arrow[d, "\eta_B"]
\\
G(X) \arrow[r, "\id_{G(X)}"'] & G(X) 
\end{tikzcd}
\end{center}
However, the natural equivalence,
\[ \Homover{\D}{F(X)}{Y} \cong \Homover{\C}{X}{G(Y)} \]
gives an equivalent natural transformation $\eta' : F \circ J \to \Delta(X)$. Therefore we have shown that,
\[ \Homover{\C^{\I}}{J}{\Delta \circ G(X)} \cong \Homover{\D^{\I}}{F \circ J}{\Delta(X)} \]
Therefore, we have,
\[ \Homover{\D}{F(\colim{J})}{X} \cong \Homover{\D^{\I}}{F \circ J}{\Delta(X)} \cong \Homover{\D}{\colim(F \circ J)}{X} \]
Furthermore, by the injectivity of the Yoneda embedding there is a natural equivalence $F(\colim J) \cong \colim{(F \circ J)}$ so $F$ is cocontinuous. The case for right adjoints is exactly dual.  
\end{proof}


\end{document}
