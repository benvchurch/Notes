\documentclass[12pt]{article}
\usepackage{import}
\import{./}{Includes}

\begin{document}

\atitle{10}

\section{Tag: 0294}

\renewcommand{\C}{\mathbb{C}}

Consider the morphism of affine schemes,
\[ \pi : X = \Spec{\C[x, t, 1/(xt)]} \to S = \Spec{\C[t]} \]
induced by the inclusion $\tilde{\pi} : \C[t] \to \C[x, t, 1/xt]$.
First any map $\sigma : S \to X$ such that $\pi \circ \sigma = \id_S$ is equivalent to a ring map, $\tilde{\sigma} : \C[x, t, 1/xt] \to \C[t]$ such that $\tilde{\sigma} \circ \tilde{\pi} = \id$. However, $t$ is not invertible in $\C[t]$ but $x t$ is invertible in $\C[x,t, 1/xy]$ and thus would map to an invertible element of $\C[t]$ yet the condition $\tilde{\sigma} \circ \tilde{\pi} = \id$ forces $\tilde{\sigma} : t \mapsto t$ which is a contradiction. Thus there cannot exist such a map $\sigma : S \to X$.
\bigskip\\
Now consider the open set $U \subset S$ given by the standard open $D(t) \subset \Spec{\C[t]}$ and thus $U = \Spec{\C[t,t^{-1}]}$. Now consider the $\C$-linear map $\tilde{\sigma} : \C[x, t, 1/xt] \to \C[t, t^{-1}]$ given by $x \mapsto t$ and $t \mapsto t$. Then $1/xt \mapsto 1/t^2$ is well-defined. Furthermore, $\tilde{\sigma} \circ \tilde{\pi} = \iota$ where $\iota : \C[t] \to \C[t, t^{-1}]$ is the inclusion corresponding the open embedding $U \subset S$ via,
\[ \C[t] \to \C[x,t, 1/(xt)] \to \C[t, t^{-1}] \]
sends $t \mapsto t^{-1}$. Thus by the antiequivalence of rings and affine schemes we have a map $\sigma : U \to X$ such that $\pi \circ \sigma = \id_U$ in the sense of the embedding $U \to S$. 

\section{Tag: 0295}

Consider the morphism of affine schemes,
\[ \pi : X = \Spec{\C[x, t]/(x^2 + t)} \to S = \Spec{\C[t]} \]
induced by the inclusion $\tilde{\pi} : \C[t] \to \C[x, t]/(x^2 + t)$. Let $U \subset S$ be an nonempty open subscheme and $\sigma : U \to X$ a section such that $\pi \circ \sigma = \iota_U$ where $\iota_U : U \to S$ is the inclusion. Because $S$ is affine any nonepty open contains a standard open $D(f) \subset U$ for some $f \in \C[t]$ and then restriction we have a map $\sigma : D(f) \to X$ such that $\pi \circ \sigma = \iota_{D(f)}$ where $\iota_{D(f)} : D(f) \to S$ is the inclusion induced by the map $\tilde{\iota} : \C[t] \to \C[t]_{f}$. Becuse these schemse are affine, such are equivalent to a ring map $\tilde{\sigma} : \C[x, t]/(x^2 + t) \to \C[t]_{f}$ such that $\tilde{\sigma} \circ \tilde{\pi} = \tilde{\iota}$ which implies that $\tilde{\sigma} : t \mapsto t$. Let $\tilde{\sigma} : x \mapsto g$ then $g^2 + t = 0$ in $\C[t]_f$. Then write,
\[ g = \frac{a}{f^n} \in \C[t]_f \quad \quad a \in \C[t] \]
and since $\C[t]_f$ is a domain we must have, $a^2 + f^{2n} t = 0$ in $\C[t]$. This is impossible because the degree of $a^2$ is even but the degree of $f^{2n} t$ is odd. Therefore there cannot exist such a section.  

\section{Tag: 0299}

Consider the scheme 
\[ X = \Spec{\Z[x, \frac{1}{x(x-1)(2x - 1)}} = \Spec{\frac{\Z[x,y]}{\big( x(x - 1)(2x - 1)y - 1 \Big)}} \]
Choose the closed subscheme $Z \subset X$ by taking the ideal,
\[ I = \Big( x^2 - 3x + 1, y^2 - 18 y + 1,  x(x - 1)(2x - 1)y - 1 \Big) \supset \big( x(x - 1)(2x - 1)y - 1 \Big) \]
Then take,
\[ Z = \Spec{\Z[x,y]/I} \subset X \]
It is clear that $\Z[x,y]/I$ is finite as a $\Z$-module because, $x$ and $y$ both satisfy monic (single variable) polynomials i.e. their solutions are algebraic integers $\alpha, \beta$ making,
\[ \Z[x,y]/(x^2 - 3x + 1, y^2 - 18 y + 1) = \Z[\alpha, \beta] \]
a $\Z$-module of finite type. I claim that $I$ is not the unit ideal because there exist simultaneous solutions to the polynomials in $\overline{\Q}$. In particular, let,
\[ \alpha = \tfrac{1}{2} (3 - \sqrt{5}) \quad \quad \beta = 9 + 4 \sqrt{5} \]
which solve all three polynomials. Then since $1 \notin I$ we have maximal ideals $(p, I) \in V(I) = \Spec{\Z[x,y]/I} = Z$ for each prime $p \in \Z$ which map down to $(p) \subset \Z$ under the map $Z \to \Spec{\Z}$. Therefore, the map $Z \to \Spec{\Z}$ is surjective. 



\section{Tag: 02EO}

A famous theorem of Ernst S. Selmer says that the cubic,
\[ 3 x^3 + 4 y^3 + 5 z^3 = 0 \]
has nontrivial solutions over $\mathbb{F}_p$ for each prime $p \in \Z$ (in fact also for prime powers) but has non nontrivial rational solutions. We will use this fact to construct an example. 
\bigskip\\
Consider the $\Z$-algebra,
\[ R = \Z[x,y,z,x',y',z'] / (3 x^3 + 4 y^3 + 5 z^3, x x' + y y' + z z' - 1) \]
First, I claim that there is no ring map $R \to \Q$ which is equivalent to a $\Z$-algebra map because we send $1 \mapsto 1$. This is equivalent $\Z$-algebra map $\Z[x, y, z, x', y', z'] \to \Q$ which sends the polynomials $3 x^3 + 4 y^3 + 5 z^3$ and $x x' + y y' + z z' - 1$ to zero and thus is determined by the image of the variables $x, y, z, x', y', z'$. However, there are no nontrivial rational solutions to $3 x^3 + 4 y^3 + 5 z^3 = 0$ which implies that $x \mapsto 0$ and $y \mapsto 0$ and $z \mapsto 0$ under such a map. Thus $x x' + y y' + z z' - 1 \mapsto -1$ and thus no ring map $\Z[x, y, z, x', y', z'] \to \Q$ can factor through the quotient. 
\bigskip\\
Now for each $p \in \Z$ prime there exists $(a_x, a_y, a_z) \in \mathbb{F}_p^3$ not all zero such that $(x, y, z) \mapsto (a_x, b_y, c_z)$ sends $3 x^3 + 4 y^3 + z^3$ to zero. Choose one $v \in \{x, y, z\}$ such that $v \mapsto a_v$ with $a_v \neq 0$. Then send $v' \mapsto a_v^{-1}$. For all other variables $w \in \{ x, y, z \}$ not equal to $v$ send $w' \mapsto 0$. Thus the polynomial, $xx' + yy' + zz' - 1 \mapsto 0$ because all products are zero except for exactly one pair which multiplies to $1$. This gives a ring map $\Z[x, y, z, x', y', z'] \to \mathbb{F}$ sending the ideal $I = (3 x^3 + 4 y^3 + 5 z^3, x x' + y y' + z z' - 1)$ to zero and thus a map $R \to \mathbb{F}_p$. However, any ring map $R \to \mathbb{F}_p$ is surjective because $1$ generates $\mathbb{F}_p$ as an abelian group. Therefore the surjective map $R \to \mathbb{F}_p$ factors through the quotient by the kernel $\m \subset R$ to give an isomorphism $R / \m \cong \mathbb{F}_p$. Since $\mathbb{F}_p$ is a field, this implies that the kernel $\m$ is a maximal ideal completing the proof. 



\end{document}
