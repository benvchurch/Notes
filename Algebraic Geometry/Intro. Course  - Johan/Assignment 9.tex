\documentclass[12pt]{article}
\usepackage{import}
\import{./}{Includes}

\begin{document}

\atitle{9}

\section{Problem 1}

Let $X$ be a scheme over a field $k$ and $x \in X$ have residue field $k$ in the sense that the map $X \to \Spec{k}$ induces the identity at the stalk $\stalk{\Spec{k}}{(0)} \to \stalk{X}{x} \to k(x)$.  
\bigskip\\
Let $U \subset X$ be any affine open neighborhood $U = \Spec{A}$ and $x \in U$ corresponds to $\p \subset A$ then $k(x) = k(\p) = A_\p/\p A_\p = (A / \p)_\p$. Furthermore, the map $\Spec{A} \to \Spec{k}$ makes $A$ a $k$-algebra compatibly with the isomorphism $k(x) = k$ i.e. the diagram commutes,
\begin{center}
\begin{tikzcd}
A \arrow[rr] & &  k(x)
\\
& k \arrow[lu] \arrow[ur]
\end{tikzcd}
\end{center}  
We may factor this map via,
\begin{center}
\begin{tikzcd}
k \arrow[r, hook] & A \arrow[r] & A / \p \arrow[r, hook] & (A /\p)_\p \arrow[r, "\sim"] & k(x) 
\end{tikzcd}
\end{center}
which composes the the identity. Because $A / \p$ is a domain, the map $A / \p \hookrightarrow (A / \p)_\p$ is injective. Therefore, the tower of inclusions collapses showing $A / \p = k(x) = k$ which implies that $\p$ is maximal since $k$ is a field. Thus $\p \in \Spec{A}$ is a closed point. Therefore, $x \in U$ is closed for each affine open neighborhood. 
Therefore there exists a closed $C \subset X$ such that $C \cap U = \{ x \}$ and thus \[ U^C \cup \{ x \} = (U \setminus \{ x \})^C = (C^C \cap U)^C = C \cup U^C \]
is closed. Now let $\{ U_\alpha \}$ be an affine cover of $X$. If $x \in U_\alpha$ then we have shown that $U_\alpha^C \cup \{ x \}$ is closed otherwise $x \in U_\alpha^C$ so $U_\alpha^C \cup \{ x \}$ is closed. Therefore, using the fact that $U_\alpha$ cover $X$, the set
\[ \bigcap_{\alpha} U^C_\alpha \cup \{ x \} = \left( \bigcap_\alpha U_\alpha \right) \cup \{ x \} = \varnothing \cup \{ x \} = \{ x \} \]   
is closed.  

\section{Tag: 029E}

Let $f : X \to S$ be a morphism of schemes. Let $x \in X$ be a point and $s = f(x)$. Note that $\Spec{k(x)[\epsilon]} = \{ (\epsilon) \}$ and $\epsilon^2 = 0$. Consider the commutative diagram,
\begin{center}
\begin{tikzcd}
\Spec{k(x)} \arrow[r] \arrow[rr, bend left] \arrow[dr] & \Spec{k(x)[\epsilon]} \arrow[d] \arrow[r, dashed, "q"] & X \arrow[d, "f"]
\\
& \Spec{k(s)} \arrow[r] & S
\end{tikzcd}
\end{center}
where $\Spec{k(x)} \to \Spec{k(x)[\epsilon]}$ is induced by the quotient map $k(x)[\epsilon] \to k(x)[\epsilon]/(\epsilon) = k(x)$ and $\Spec{k(x)[\epsilon]} \to \Spec{k(s)}$ is induced by the inclusion $k(s) \to k(x)[\epsilon]$ and the maps $\Spec{k(x)} \to X$ and $\Spec{k(s)} \to S$ are the canonical maps inducing the identity at the residue field. 
\bigskip\\
Given a morphism $q : \Spec{k(x)[\epsilon]} \to X$ making the diagram commute we may consider the corresponding maps at stalks,
\begin{center}
\begin{tikzcd}
k(x) \arrow[from=r] \arrow[from=rr, bend right] \arrow[from=dr] & k(x)[\epsilon] \arrow[from=d] \arrow[from=r, "q^\#"'] & \stalk{X}{x} \arrow[from=d, "f^\#"']
\\
& k(s) \arrow[from=r] & \stalk{S}{s}
\end{tikzcd}
\end{center}
Consider the restriction $q^\# : \m_x \to (\epsilon) \subset k(x)[\epsilon]$ since this map is local its image lies in $(\epsilon)$ the maximal ideal of $k(x)[\epsilon]$. Then $q^\#(\m_x^2) \subset (\epsilon^2) = 0$ and thus $\m_x^2 \subset \ker{q^\#}$. Furthermore, by the commutativity of the diagram, the map $\stalk{S}{s} \to \stalk{X}{x} \xrightarrow{q^\#} k(x)[\epsilon]$ factors through $k(s)$ and thus $q^\#(\m_s \stalk{X}{x}) = 0$ so $\m_s \stalk{X}{x} \subset \ker{q^\#}$. Thus we may factor,
\begin{center}
\begin{tikzcd}[row sep = small, column sep = small]
\m_x \arrow[rr, "q^\#"] \arrow[rd] & & (\epsilon) \cong k(x)
\\
& \frac{\m_x}{\m_x^2 + \m_s \stalk{X}{x}} \arrow[ru]
\end{tikzcd}
\end{center} 
Furthermore, $\stalk{X}{x} \to k(x)[\epsilon] \to k(x)$ is the identity so the induced map,
\[ \frac{\m_x}{\m_x^2 + \m_s \stalk{X}{x}}  \to (\epsilon) \]
is $k$-linear.
\bigskip\\
Conversely, suppose that $k(x) = k(s)$. Given the diagram, the doted morphism is uniquely determined on the underlying topological spaces since it must send the unique point of $\Spec{k(x)[\epsilon]}$ to $x$. Therefore it suffices to show that a local stalk map $q^\# : \stalk{X}{x} \to k(x)[\epsilon]$ is uniquely determined by a $k(x)$-linear map,
\[ z : \frac{\m_x}{\m_x^2 + \m_s \stalk{X}{x}} \to k(x) \]
First, note that since $\stalk{S}{s} \to \stalk{X}{x}$ is local we have maps,
\begin{center}
\begin{tikzcd}
\stalk{S}{s} / \m_s \arrow[r] & \stalk{X}{x} / \m_s \stalk{X}{x} \arrow[r] & \stalk{X}{x} / \m_x
\end{tikzcd}
\end{center}
whose composition gives the natural map $k(s) \to k(x)$ which we assume to be an isomorphism. Denote $k(s) = k(x) = k$ then the above maps give $\stalk{X}{x} / \m_s \stalk{X}{x}$ a natural $k$-algebra structure. The projection map (defined since $\m_x^2 + \m_s \stalk{X}{x} \subset \m_x$),
\[ \frac{(\stalk{X}{x} / \m_s \stalk{X}{x})}{\m_x^2} \to \stalk{X}{x} / \m_x = k \]
has kernel $\m_x /(\m_x^2 + \m_s \stalk{X}{x})$ giving a canonical decomposition as $k$-modules,
\[ \frac{(\stalk{X}{x} / \m_s \stalk{X}{x})}{\m_x^2} = \frac{\stalk{X}{x}}{\m_x^2 + \m_s \stalk{X}{x}} = k \oplus \frac{\m_x}{\m_x^2 + \m_s \stalk{X}{x}} \]
Therefore, we get a map $q : \stalk{X}{x} \to k(X)[\epsilon]$ via,
\[ \stalk{X}{x} \to \frac{\stalk{X}{x}}{\m_x^2 + \m_s \stalk{X}{x}} \to k \oplus \frac{\m_x}{\m_x^2 + \m_s \stalk{X}{x}} \xrightarrow{\id \oplus \epsilon z} k(x)[\epsilon] \]
where the last map sends $(a, m) \mapsto a + z(m) \epsilon$. I claim that this map makes the diagram commute and is unique. First, it is clear that restructing $q$ to $\m_x$ recovers the map $z$ with image embdded as $k(x) \epsilon \subset k(x)[\epsilon]$. Next, the diagram commutes because the map sends $\stalk{X}{x} \to k(x)$ under projection to the first factor exactly by the quotient $\pi : \stalk{X}{x} \to k(x)$ since,
\[ a \mapsto [a] \mapsto \pi(a) \oplus [a'] \mapsto \pi(a) \]
for some $a' \in \ker{(\stalk{X}{x} / \m_s \stalk{X}{x} \to k)}$. Furthermore, $\stalk{S}{s} \to k(s) \to k(x)[\epsilon]$ is exactly given by $\stalk{S}{s} \to \stalk{S}{s} / \m_s \to \stalk{X}{x} / \m_x = k(x) \subset k(x)[\epsilon]$ which is just $\pi \circ f^\#$. Since the diagram commutes, it suffices to show that such a construction will recover the origional map $q^\# : \stalk{X}{x} \to k[\epsilon]$. The difference $\tilde{q} = q - q^\#$ is a map $\stalk{X}{s} \to k[\epsilon]$ which factors through,
\[ \frac{\stalk{X}{x}}{\m_x^2 + \m_s \stalk{X}{x}} = k \oplus \frac{\m_x}{\m_x^2 + \m_s \stalk{X}{x}} \]
but is zero on each factor because $q$ and $q^\#$ agree on $\stalk{X}{x} \to k$ and on $\m_x / (\m_x^2 + \m_s \stalk{X}{x})$ by construction. Thus $\tilde{q} = 0$ since it factors though the zero map on each factor of the quoitent. Therefore, $q = q^\#$ proving the result. 

\section{Tag: 029G}

Let $K$ be a field then consider the diagram of schemes,
\begin{center}
\begin{tikzcd}[row sep = huge]
\Spec{K} \arrow[r] \arrow[d] & \Spec{K[\epsilon_1]} \arrow[d]
\\
\Spec{K[\epsilon_2]} \arrow[r] & \Spec{K[\epsilon_1, \epsilon_2]/(\epsilon_1 \epsilon_2)}
\end{tikzcd}
\end{center}
we are asked to show that this diagram is a pushout in the category of schemes. Let $X$ be any scheme and consider a commutative diagram,
\begin{center}
\begin{tikzcd}[row sep = huge]
\Spec{K} \arrow[r] \arrow[d] & \Spec{K[\epsilon_1]} \arrow[d] \arrow[rdd, bend left, "f"]
\\
\Spec{K[\epsilon_2]} \arrow[r] \arrow[rrd, bend right, "g"'] & \Spec{K[\epsilon_1, \epsilon_2]/(\epsilon_1 \epsilon_2)} \arrow[rd, dashed]
\\
& & X
\end{tikzcd}
\end{center}
Each affine scheme has one point so a map $\Spec{K[\epsilon_1, \epsilon_2]/(\epsilon_1 \epsilon_2)} \to X$ is given by choosing a point $x \in X$ and map $\stalk{X}{x} \to K[\epsilon_1, \epsilon_2]/(\epsilon_1 \epsilon_2)$. We chose the point $x \in X$ as the image of $f$ which equals the image of $g$. The sheaf maps (which on a one point space are equivalent to the maps on the stalk) must satisfy the diagram,
\begin{center}
\begin{tikzcd}[row sep = huge]
K \arrow[from=r] \arrow[from=d] & K[\epsilon_1] \arrow[from=d, "\pi_1"'] \arrow[from=rdd, bend right, "f^\#"']
\\
K[\epsilon_2] \arrow[from=r, "\pi_2"] \arrow[from=rrd, bend left, "g^\#"] & K[\epsilon_1, \epsilon_2]/(\epsilon_1 \epsilon_2) \arrow[from=rd, dashed]
\\
& & \stalk{X}{x}
\end{tikzcd}
\end{center}
However, $K[\epsilon_1, \epsilon_2] /(\epsilon_1 \epsilon_2) = K[\epsilon_1] \times_K K[\epsilon_2]$ is the pullback in the category of rings and thus there exists a unique map $\stalk{X}{x} \to K[\epsilon_1, \epsilon_2] /(\epsilon_1 \epsilon_2)$ making the diagram commute. Since the topological part is fixed this is equivalent to a giving a unqiue morphism of schemes $\Spec{K[\epsilon_1, \epsilon_2] /(\epsilon_1 \epsilon_2)} \to X$ such that the first diagram commutes. This proof works because  $K[\epsilon_1] \times_K K[\epsilon_2]$ is the pullback in the category of rings making (by the antiequivalence of the Spec functor) the origional diagram a pushout in the category of affine schemes. However, any morphism $\Spec{k[\epsilon_i]} \to X$ factors through an open immersion of some affine patch because the image is a single point which must lie in some affine open. Therefore, this pushout diagram in the category of affine schemes is a pushout in the category of schemes.  


\end{document}
