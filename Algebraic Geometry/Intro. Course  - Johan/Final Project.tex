\documentclass[12pt]{article}
\usepackage{import}
\import{./}{Includes}

\begin{document}

\title{% 
	\large \textbf{Math GR6262 Algebraic Geometry
	\\ Final Project: 
	\\ Group Schemes and Vector Bundles} \vspace{-2ex}}
\author{Benjamin Church }
\maketitle

\section{Basic Definitions and Examples}

\begin{definition}
Let $\C$ be a category with all finite products (including the empty product which is the terminal object $1$). Then a group object is a tuple $(G, m, e, i)$ where $G \in \C$ is an object and $m : G \times G \to G$, $e : 1 \to G$, and $i : G \to G$ are morphisms such that the diagrams commute,
\begin{center}
\begin{tikzcd}[row sep = huge]
G \times G \times G \arrow[r, "\id \times m"] \arrow[d, "m \times \id"'] & G \times G \arrow[d, "m"]
\\
G \times G \arrow[r, "m"'] & G 
\end{tikzcd}
\end{center}
giving associativity,
\begin{center}
\begin{tikzcd}[row sep = huge]
G \arrow[r, "\id \times e"] \arrow[d, "e \times \id"'] \arrow[rd, "\id"] & G \times G \arrow[d, "m"]
\\
G \times G \arrow[r, "m"'] & G
\end{tikzcd}
\end{center}
giving identity,
\begin{center}
\begin{tikzcd}[row sep = huge]
G \arrow[r, "(\id \times i) \circ \Delta"] \arrow[d, "(i \times \id) \circ \Delta"'] \arrow[rd, "e"] & G \times G \arrow[d, "m"]
\\
G \times G \arrow[r, "m"'] & G
\end{tikzcd}
\end{center}
giving inverses. A morphism of group objects $G$ to $G'$ is a morphism $f : G \to G'$ such that the diagram commutes,
\begin{center}
\begin{tikzcd}[row sep = huge]
G \times G \arrow[r, "m"] \arrow[d, "f \times f"'] & G \arrow[d, "f"]
\\
G' \times G' \arrow[r, "m'"'] & G'
\end{tikzcd}
\end{center}
\end{definition}

\begin{definition}
Let $\C$ be a category with finite products and $G$ a group object in $\C$. Then for $X \in \C$ an action of $G$ on $X$ is a morphism $\rho : G \times X \to X$ such that the following diagrams commute,
\begin{center}
\begin{tikzcd}[row sep = huge]
G \times G \times X \arrow[r, "m \times \id"] \arrow[d, "\id \times \rho"'] &  G \times X \arrow[d, "\rho"] 
\\
G \times X \arrow[r, "\rho"'] & X 
\end{tikzcd}
\end{center}
and
\begin{center}
\begin{tikzcd}[row sep = huge]
X \arrow[r, "e \times \id"] \arrow[rd, "\id"] & G \times X \arrow[d, "\rho"] 
\\
& X
\end{tikzcd}
\end{center}
In this case we call $X$ a $G$-object. A morphism of $G$-objects is a morphism $f : X \to Y$ which is a $G$-intertwiner i.e. the following diagram commutes,
\begin{center}
\begin{tikzcd}[row sep = huge]
G \times X \arrow[r, "\rho_X"]  \arrow[d, "\id \times f"'] & X \arrow[d, "f"]
\\
G \times Y \arrow[r, "\rho_Y"'] & Y
\end{tikzcd}
\end{center}
\end{definition}

\begin{definition}
Let $S$ be a scheme. A group scheme over $S$ is a group object in the category of schemes over $S$. If a group scheme $G$ acts on a scheme $X$ then we say $X$ is a $G$-scheme.
\end{definition}

\begin{example}
The additive group scheme $\Ga$ is the scheme $\Spec{\Z[x]}$ with operation,
\begin{align*}
\Ga \times \Ga & \to \Ga
\\
\Spec{\Z[x] \otimes \Z[x]} & \to \Spec{\Z[x]}
\\
\Z[x] \otimes \Z[x] & \leftarrow \Z[x]
\\
x \otimes 1 + 1 \otimes x & \mapsfrom x 
\end{align*}
We should check that this is actually a group scheme. The identity is the natural map induced by the quotient $\Z[x] \to \Z$ and inverses are given by $\Z[x] \to \Z[x]$ sending $x \mapsto -x$. Then the following diagram commutes,
\begin{center}
\begin{tikzcd}[row sep = huge]
\Z[x] \otimes \Z[x] \otimes \Z[x] \arrow[from = r, "\id \otimes m"'] \arrow[from = d, "m \otimes \id"] & \Z[x] \otimes \Z[x]
\arrow[from = d, "m"']
\\
\Z[x] \otimes \Z[x] \arrow[from = r, "m"] & \Z[x]
\end{tikzcd}
\end{center}
because under the two directions,
\begin{align*}
x & \mapsto (x \otimes 1 + 1 \otimes x) \mapsto (x \otimes 1 \otimes 1 + 1 \otimes (x \otimes 1 + 1 \otimes x)) = x \otimes 1 \otimes 1 + 1 \otimes x \otimes 1 + 1 \otimes 1 \otimes x
\\
x & \mapsto (x \otimes 1 + 1 \otimes x) \mapsto ((x \otimes 1 + 1 \otimes x) \otimes 1 + 1 \otimes 1 \otimes x) = x \otimes 1 \otimes 1 + 1 \otimes x \otimes 1 + 1 \otimes 1 \otimes x
\end{align*}
Furthermore, the diagram commutes,
\begin{center}
\begin{tikzcd}[row sep = huge]
\Z[x] \arrow[from = r, "\Delta \circ (\id \otimes e)"'] \arrow[from = d, "\Delta \circ (e \otimes \id)"] & \Z[x] \otimes \Z[x] \arrow[from = d, "m"']
\\
\Z[x] \otimes \Z[x] \arrow[from = r, "m"] & \Z[x] \arrow[ul, "\id"]
\end{tikzcd}
\end{center}
because under the two directions,
\begin{align*}
x \mapsto (x \otimes 1 + 1 \otimes x) \mapsto \Delta(x \otimes 1) = x
\\
x \mapsto (x \otimes 1 + 1 \otimes x) \mapsto \Delta(1 \otimes x) = x
\end{align*}
Finally, the diagram commutes,
\begin{center}
\begin{tikzcd}[row sep = huge]
\Z[x] \arrow[from = r, "\Delta \circ (\id \otimes i)"'] \arrow[from = d, "\Delta \circ (i \otimes \id)"] & \Z[x] \otimes \Z[x] \arrow[from = d, "m"']
\\
\Z[x] \otimes \Z[x] \arrow[from = r, "m"] & \Z[x] \arrow[ul, "e"]
\end{tikzcd}
\end{center}
because under the two directions,
\begin{align*}
x & \mapsto (x \otimes 1 + 1 \otimes x) \mapsto \Delta(x \otimes 1 - 1 \otimes x) = 0
\\
x & \mapsto (x \otimes 1 + 1 \otimes x) \mapsto \Delta(-x \otimes 1 + 1 \otimes x) = 0
\end{align*}
\end{example}

\begin{example}
The multiplicative group scheme $\Gm$ is the scheme $\Spec{\Z[x, x^{-1}]}$ with multiplication
\begin{align*}
\Gm \times \Gm & \to \Gm
\\
\Spec{\Z[x, x^{-1}] \otimes \Z[x, x^{-1}]} & \to \Spec{\Z[x, x^{-1}]}
\\
\Z[x, x^{-1}] \otimes \Z[x, x^{-1}] & \leftarrow \Z[x, x^{-1}]
\\
x \otimes x & \mapsfrom x 
\end{align*}
and inverse induced by the map $\Z[x, x^{-1}] \to \Z[x, x^{-1}]$ sending $x \mapsto x^{-1}$. 
\end{example}

\begin{example}
There is an action $\Gm^k$ on $\A^n_k$ via the ring map,
\begin{align*}
\Gm^k \times \A^n_k & \to \A^n_k
\\
k[z, z^{-1}] \otimes k[x_1, \dots, x_n] & \leftarrow k[x_1, \dots, x_n] 
\\
z \otimes x_i & \mapsfrom x
\end{align*} 
This is the scaling action $\lambda \cdot (a_1, \dots, a_n)  = (\lambda a_1, \dots, \lambda a_n)$.
\end{example}

\begin{lemma}
The base change of a group scheme is a group scheme.
\end{lemma}

\begin{proof}
Base change is a limit which commutes with limits (in particular finite products). It is clear that any functor preserving products preserves group objects.
\end{proof}

\begin{lemma}
If $G$ is a group scheme over $S$ and $X$ is a scheme over $S$ then the $X$-points of $G$ i.e. the set $G(X) = \Homover{S}{X}{G}$ is naturally a group.
\end{lemma}

\begin{proof}
The functor $\Homover{S}{X}{-} : \mathbf{Sch}_S \to \mathbf{Set}$ is continuous, thus preserves products, and thus preserves group objects. Therefore, $\Homover{S}{X}{G}$ is a group object in $\mathbf{Set}$ which is a group. 
\end{proof}

\begin{definition}
The additive and multiplicative group schemes in the category of schemes over $S$ are $\Ga^S = \Ga \times S$ and $\Gm^S = \Gm \times S$ respectively. 
\end{definition}

\begin{example}
Let $k$ be an algebraically closed field and consider the group schemes $\Ga = \Spec{k[x]}$ and $\Gm = \Spec{k[x, x^{-1}]}$ over $\Spec{k}$. Then, as abelian groups, there are bijections, 
\begin{align*}
\Ga & \to k 
\\
(x - \mu) & \mapsto \mu
\\
\Gm & \to k^\times
\\
(x - \mu) & \mapsto \mu
\end{align*} 
(since $(x) \notin \Spec{k[x, x^{-1}]} = D(x) \subset \Spec{k[x]}$). I claim these maps are isomorphisms. 
\end{example}

\newcommand{\sGL}[1]{\mathbb{GL}_{#1}}

\begin{definition}
\[ \sGL{n} = \Spec{\Z[\{x_{ij}  \mid 1 \le i,j \le n \}]_{(\det{(x_{ij})})}} \]
with multiplication defined via,
\begin{align*}
\sGL{n} \times \sGL{n} & \to \sGL{n}
\\
\Spec{\Z[x_{ij}]_{(\det{(x_{ij})})} \otimes \Z[x_{ij}]_{(\det{(x_{ij})})}} & \to \Spec{\Z[x_{ij}]_{(\det{(x_{ij})})}}
\\
\Z[x_{ij}]_{(\det{(x_{ij})})} \otimes \Z[x_{ij}]_{(\det{(x_{ij})})} & \leftarrow \Z[x_{ij}]_{(\det{(x_{ij})})}
\\
\sum_k x_{ik} \otimes x_{kj} & \mapsfrom x_{ij} 
\end{align*}
\end{definition}


\begin{remark}
In the case $n=1$ we have $\GL{n}{\Z} = \Spec{\Z[x]_{(x)}} = \Spec{\Z[x,x^{-1}]} = \Gm$.
\end{remark}

\begin{example}
There is a defining action of $\sGL{n}$ on $\A^n$ defined by,
\begin{align*}
\sGL{n} \times \A^{n} & \to \A^{n}
\\
\Spec{\Z[x_{ij}]_{(\det{(x_{ij})})} \otimes \Z[y_1, \dots, y_n]} & \to \Spec{\Z[y_1, \dots, y_n]}
\\
\Z[x_{ij}]_{(\det{(x_{ij})})} \otimes \Z[y_1, \dots, y_n] & \leftarrow \Z[y_1, \dots, y_n]
\\
\sum_k x_{ik} \otimes y_{k} & \mapsfrom y_{i} 
\end{align*}
\end{example}


\begin{lemma}
Let $X$ be an $S$ scheme. Then the group schemes $\Gm$ and $\Ga$ have $X$-points,
\begin{align*}
\Homover{S}{X}{\Ga^S} & = \Gamma(X, \struct{X})
\\
\Homover{S}{X}{\Gm^S} & = \Gamma(X, \struct{X}^\times)
\\
\Homover{S}{X}{\sGL{n}^S} & = \GL{n}{\Gamma(X, \struct{X})}
\end{align*}
\end{lemma}

\begin{proof}
\begin{align*}
\Homover{S}{X}{\Ga^S} & = \Homover{S}{X}{S} \times \Hom{}{X}{\Ga} = \Hom{}{X}{\Ga} 
\\
& = \Hom{}{\Z[x]}{\Gamma(X, \struct{X})} = \Gamma(X, \struct{X})
\end{align*}
since any ring map $\Z[x] \to R$ is determined uniquely by the image of $x$. Similarly,
\begin{align*}
\Homover{S}{X}{\Gm^S} & = \Hom{}{X}{\Gm} 
\\
& = \Hom{}{\Z[x, x^{-1}]}{\Gamma(X, \struct{X})} = \Gamma(X, \struct{X}^\times)
\end{align*}
since any ring map $\Z[x, x^{-1}] \to R$ is determined uniquely by the image of $x \in R^\times$.
\begin{align*}
\Homover{S}{X}{\sGL{n}^S} & = \Hom{}{X}{\sGL{n}} 
\\ 
& = \Hom{}{\Z[x_{ij}]_{(\det{(x_{ij})})}}{\Gamma(X, \struct{X})} = \GL{n}{\Gamma(X, \struct{X})} 
\end{align*}
since a ring map $\Z[x_{ij}]_{(\det{(x_{ij})})} \to R$ is exactly determined by a matrix of elements $a_{ij}$ which are the images of $x_{ij}$ such that the determinant polynomial $\det{(x_{ij})}$ is mapped to a unit: $\det{(a_{ij})} \in R^\times$.
\end{proof}

\begin{remark}
In particular, let $S = \Spec{k}$ then by the lemma, the geometric points of these group schemes are,
\begin{align*}
\Homover{S}{S}{\Ga^S} & = \Gamma(S, \struct{S}) = k
\\
\Homover{S}{S}{\Gm^S} & = \Gamma(S, \struct{S}^\times) = k^\times
\end{align*}
which, in the case $k = \bar{k}$ correspond to the closed points as we computed before.
\end{remark}

\section{Vector Bundles on Schemes}

\newcommand{\cA}{\mathcal{A}}

\begin{remark}
Given a scheme $S$ and a quasi-coherent sheaf of $\struct{S}$-algebras $\cA$ Recall the relative spectrum, $\rSpec{S}{\cA}$.
The relative spectrum over $S$ may be characterized as representing the functor,
\[ F : \mathbf{Sch}^{\text{op}} \to \mathbf{Set} \]
defined by sending a scheme $T$ to the set of pairs $(f, g)$ of morphisms $f : T \to S$ and $\struct{T}$-algebra morphisms $g : f^* \cA \to \struct{T}$. The universal element $\xi \in F(\rSpec{S}{\cA})$ is thus a pair of canonical maps,
\[ \pi : \rSpec{S}{\cA} \to S \text{ and (by adjunction) } g : \cA \to \pi_* \struct{\rSpec{S}{\cA}} \]
It turns out that when $\cA$ is a quasi-coherent $\struct{S}$-algebra then $g : \cA \to \pi_* \struct{\rSpec{S}{\cA}}$ is an isomorphism of $\struct{S}$-algebras (Tag 01LX). The explicit isomorphism,
\[ \eta_X : \Hom{X}{\rSpec{S}{\mathcal{A}}} \to F(X) \]
is given by sending $s : X \to \rSpec{S}{\mathcal{A}}$ to $F(s)(\xi) = (\pi \circ s, g \circ \pi_* s^\#)$.
\end{remark}

\begin{definition}
Let $X$ be a scheme. A \textit{vector bundle} over $X$ is an affine morphism $\pi : V \to X$ such that $\pi_* \struct{V}$ is a graded $\struct{X}$-algebra,
\[ \pi_* \struct{V} = \bigoplus_{n \ge 0} \mathcal{E}_n \]
such that $\mathcal{E}_0 = \struct{X}$ and the natural maps,
\begin{center}
\begin{tikzcd}
\mathrm{Sym}^n_{\struct{X}}(\mathcal{E}_1) \arrow[r] & \mathcal{E}_n
\end{tikzcd}
\end{center}
are isomorphisms for all $n \ne 0$. 
\bigskip\\
Given a morphism of schemes $g : X \to Y$ a \textit{bundle map} $f : V_X \to V_Y$  of vector bundles $V_X$ over $X$ and $V_Y$ over $Y$ is a commutative diagram of schemes,
\begin{center}
\begin{tikzcd}[row sep = large, column sep = large]
V_X \arrow[d, "\pi_X"'] \arrow[r, "f"] & V_Y \arrow[d, "\pi_Y"]
\\
X \arrow[r, "g"'] & Y
\end{tikzcd}
\end{center}
such that the induced sheaf map $(\pi_Y)_* \struct{V_Y} \to g_* (\pi_X)_* \struct{V_X}$ is a map of \textit{graded} sheaves. In particular, if we take the map $\id_X : X \to X$ then a morphism of vector bundles over $X$ is a morphism $f : V_1 \to V_2$ such that $\pi_2 \circ f = \pi_1$ and $(\pi_2)_* \struct{V_2} \to (\pi_1)_* \struct{V_1}$ is a morphism of graded sheaves.
\end{definition}

\begin{remark}
We show how to explicitly construct this induced morphism. The map of schemes gives $f^\# : \struct{V_Y} \to f_* \struct{V_Y}$. Then apply the functor $(\pi_Y)_*$ which gives a morphism, $(\pi_Y)_* f^\# :  (\pi_Y)_* \struct{V_Y} \to (\pi_Y)_* f_* \struct{V_Y}$ however, $\pi_Y \circ f = g \circ \pi_X$ giving the desired morphism,
\[ (\pi_Y)_* f^\# :  (\pi_Y)_* \struct{V_Y} \to g_* (\pi_X)_* \struct{V_Y} \]
\end{remark}

\begin{remark}
Vector bundles are important because we can associate them to (quasi)coherent sheaves which will give our most important examples. 
\end{remark}

\newcommand{\V}{\mathbf{V}}

\begin{definition}
Let $X$ be be a scheme and $\F$ a quasi-coherent sheaf of $\struct{X}$-modules. Then the associated vector bundle $\V(\F)$ over $X$ is the scheme over $X$ with structure morphism,
\[ \pi : \rSpec{X}{\shSym{\struct{X}}{\F}} \to X \]
Then by definition,
\[ \pi_* \struct{V(\F)} = \shSym{\struct{X}}{\F} = \bigoplus_{n \ge 0} \mathrm{Sym}^n_{\struct{X}}(\F) \]
which makes $\pi_* \struct{V(\F)}$ a graded $\struct{X}$-algebra where we may recover $\F$ in degree $1$.
\end{definition}

\begin{theorem}
There is an anti-equivalence between the category of quasi-coherent $\struct{X}$-modules and the category of vector bundles over $X$.
\end{theorem}

\begin{proof}
(Sketch) We have shown that given a quasi-coherent sheaf $\F$ we can construct a vector bundle $V(\F)$ and that $(\pi_* V(\F))_1 = \F$ so the functor $V \to (\pi_* \struct{V})_1$ recovers the original sheaf. I claim that the functors $\rSpec{X}{\shSym{\struct{X}}{-}}$ and $V \to (\pi_* \struct{V})_1$ give this anti-equivalence. We should check that the above construction can reproduce any vector bundle over $X$. Given such a vector bundle $\pi : V \to X$, we know that $\pi_* \struct{V}$ is a graded $\struct{X}$-algebra such that we have graded isomorphisms,
\[ \shSym{\struct{X}}{\mathcal{E}_1} \to \pi_* \struct{V} = \bigoplus_{n \ge 0} \mathcal{E}_n \]
By Tag 01LY in the stacks project, since $\pi : V \to X$ is an affine morphism and thus quasi-compact and separated there is a canonical morphism,
\begin{center}
\begin{tikzcd}
V \arrow[r] & \rSpec{X}{\pi_* \struct{V}} = \rSpec{X}{\shSym{\struct{X}}{\mathcal{E}_1}} = \rSpec{X}{\shSym{\struct{X}}{(\pi_* \struct{V})_1}} 
\end{tikzcd}
\end{center}
Lastly, this first map is an isomorphism because $\pi : V \to X$ is affine (Tag 01S8). To see this take any affine open $U \subset X$ then we know the canonical map $V \to \rSpec{X}{\pi_* \struct{V}}$ restricts to,
\[ \pi^{-1}(U) \to \Spec{\Gamma(\pi^{-1}(U), \struct{V})} \]
However, $\pi$ is affine so $\pi^{-1}(U) \subset V$ is affine open meaning that,
\[ \pi^{-1}(U) = \Spec{\Gamma(\pi^{-1}(U), \struct{V})} \]
and the canonical map is the identity because it is, by definition, induced by the identity ring map on $\Gamma(\pi^{-1}(U), \struct{V})$. Thus we have found,
\[ V \cong \rSpec{X}{\pi_* \struct{V}} = \rSpec{X}{\shSym{\struct{X}}{(\pi_* \struct{V})_1}}  \]
 We should also show that these functors are fully faithful but I will leave the proof here.
\end{proof}

\begin{example}
Let $X = \A^n_R$ over some ring $R$. Then,
\[ \shSym{\struct{X}}{\struct{X}} = \mathrm{Sym}_{R[x_1, \dots, x_n]}\left(R[x_1, \dots, x_n]\right)^{\widetilde{}} = R[x_1, \dots, x_n, x_{n+1}]^{\widetilde{}} \]
\[ \V(\struct{X}) = \rSpec{X}{R[x_1, \cdots, x_n, x_{n+1}]^{\widetilde{}} \: } = \Spec{R[x_1, \cdots, x_n, x_{n+1}]} = \A^{n+1}_R \]
with the projection $\pi : \A^{n+1}_R \to \A^n_R$ induced by the embedding $R[x_1, \dots, x_n] \to R[x_1, \dots, x_n, x_{n+1}]$. This recovers nicely the picture of $\A^{n+1}$ as a line bundle over $\A^n$ whose sections are exactly regular functions on $\A^n$. 
\end{example}

\begin{lemma}
Let $X$ be a scheme and $\F$ be a quasi-coherent $\struct{X}$-module. Take $\pi : \V(\F) \to X$ its associated vector bundle. Then there is a canonical correspondence between sections $s : X \to \V(\F)$ (such that $\pi \circ s = \id_X$) and global sections of the dual sheaf $\F^\vee$. That is,
\[ \Homover{X}{X}{\V(\F)} = \Gamma(X, \F^\vee) = \Homover{\struct{X}}{\F}{\struct{X}} \]
\end{lemma}

\begin{proof}
The associated vector bundle is constructed as,
\[ \V(\F) = \rSpec{X}{\shSym{\struct{X}}{\F}} \]
and recall that the relative spectrum represents the functor $F$ defined at the beginning of the section. Denote $\mathcal{A} = \shSym{\struct{X}}{\F}$. Sections $s : X \to \V(\F)$ correspond to pairs $(f : X \to X, g : \mathcal{A} \to f_* \struct{X})$ where we require $f = \id_X$ since $f = \pi \circ s = \id_X$ because the corresponding map is a section. Therefore, sections $s : X \to \V(\F)$ correspond conically to $\struct{X}$-algebra maps $g : \shSym{\struct{X}}{\F} \to \struct{X}$. However, such a map of algebras is uniquely determined by its action in degree $1$ i.e. by a morphism $\F \to \struct{X}$ of $\struct{X}$-modules which is exactly a global section of the dual sheaf $\F^\vee = \shHomover{\struct{X}}{\F}{\struct{X}}$.
\end{proof}

\begin{definition}
Let $\pi : V \to Y$ be a vector bundle and $f : X \to Y$ a morphism of schemes. The \textit{pullback bundle} along $f$, denoted $f^* V$, is the bundle over $X$ given by base change $\pi_X : V \times_Y X \to X$ which is the pullback in the diagram,
\begin{center}
\begin{tikzcd}
V \times_Y X \arrow[r] \arrow[d, "\pi_X"] & V \arrow[d,"\pi"] 
\\
X \arrow[r, "f"] & Y
\end{tikzcd}
\end{center}
\end{definition}

\begin{lemma}
The pullback bundle is a vector bundle and the map $f^* V \to V$ is a bundle map.
\end{lemma}

\begin{proof}
We will explicitly demonstrate this for the case of interest by the following.
\end{proof}

\begin{lemma}
Let $Y$ be a scheme and $\mathcal{A}$ be a quasi-coherent $\struct{Y}$-module. Given a morphism of schemes $f : X \to Y$, the relative spectrum base changes as,
\[ X \times_Y \rSpec{Y}{\mathcal{A}} = \rSpec{X}{f^* \mathcal{A}} \]
\end{lemma}

\begin{proof}
A pair $(a : T \to X, g : a^* f^* \cA \to \struct{T})$ is canonically the same as a pair $(f \circ a : T \to Y, g : (f \circ a)^* \cA \to \struct{T})$ i.e. a pair $(a' : T \to Y : (a')^* : \cA \to \struct{T})$ such that $a'$ factors through $f : X \to Y$ as $a' = f \circ a$. By the representation, such a pair can be identified with a map $\tilde{a} : T \to \rSpec{Y}{\cA}$ such that the map $a' = \pi \circ \tilde{a}$ factors through $f : X \to Y$ i.e. $a' = \pi \circ \tilde{a} = f \circ a$ for some $a : T \to X$. By the universal property, such maps are canonically identified with maps $T \to X \times_Y \rSpec{Y}{\cA}$. Therefore, $X \times_Y \rSpec{Y}{\cA}$ represents the functor $F$ for the pair $(X, f^* \cA)$ so by Yoneda,
\[ X \times_Y \rSpec{Y}{\mathcal{A}} = \rSpec{X}{f^* \mathcal{A}} \]
since these schemes both represent the same functor $F$.
\end{proof}

\begin{lemma}
Let $f : X \to Y$ be a morphism of schemes and $\F$ a quasi-coherent $\struct{Y}$-module. The pullback bundle of the associated vector bundle is the associated vector bundle of the pullback sheaf,
\[ f^* \V(\F) \cong \V(f^* \F) \]
\end{lemma}

\begin{proof}
\begin{align*}
f^* \V(\F) & = X \times_Y \rSpec{Y}{\shSym{\struct{Y}}{\F}} = \rSpec{X}{f^* \shSym{\struct{Y}}{\F}} 
\\
& = \rSpec{X}{\shSym{\struct{X}}{f^* \F}} = \V(f^* \F)
\end{align*}
\end{proof}

\begin{example}
Let $X = \P^n_k = \Proj{k[X_0, \cdots, X_n]}$ and consider the invertable sheaf $\struct{X}(-1)$ on $X$. This is known as the tautological bundle or rather its associated vector bundle $\V(\struct{X}(-1))$ is the tautological bundle. Topologically, it is the line bundle whose fiber above each point in $\P^n_k$ is the line in $\A^{n+1}_k$ it corresponds to. Furthermore, using our formula, the sections of the tautological bundle are exactly,
\[ H^0(X, \struct{X}(-1)^\vee) = H^0(X, \struct{X}(1)) = k[X_0, \cdots, X_n]_{(0)} \]
These sections $X_i$ correspond to the coordinates on $\A_k^{n+1}$.
\end{example}


\section{Group Schemes Acting on Sheaves}

\begin{remark}
It is easy to define an equivariant group scheme action in the category of vector bundles over a scheme. Our strategy to figure out how to act a group scheme on a quasi-coherent sheaf equivariantly is to use the anti-equivalence of quasi-coherent sheaves and vector bundles.
\end{remark}

\begin{definition}
Let $\F$ be a quasi-coherent sheaf of $\struct{X}$-modules and a group scheme $G$ act on $X$. Then an $G$ action on $\F$ is the same as a $G$-equivariant action on the associated vector bundle $\pi : \mathbf{V}(\F) \to X$
such that $\pi$ is a morphism of $G$-schemes,
\begin{center}
\begin{tikzcd}[column sep = large, row sep = large]
G \times \V(\F)  \arrow[d, "\id \times \pi"'] \arrow[r, "\rho_V"] & \V(\F) \arrow[d, "\pi"] 
\\
G \times X \arrow[r, "\rho"'] & X 
\end{tikzcd}
\end{center}
and $\rho_V$ is a morphism of vector bundles i.e. a bundle map over $\rho$.
\end{definition}

\begin{remark}
We will now unwind this definition to recover a purely sheaf-theoretic notion of a $G$-equivariant sheaf action.
\end{remark}

\begin{proof}
Let $p : G \times X \to X$ be the projection. Note that, canonically,
\[ G \times \V(\F) \cong (G \times X) \times_X \V(\F) = p^* \V(\F) \]
Furthermore, we have a diagram,
\begin{center}
\begin{tikzcd}[column sep = large, row sep = large]
G \times \V(\F)  \arrow[rrd, bend left, "\rho_V"] \arrow[ddr, bend right, "\id \times \pi"'] \arrow[rd, dashed, "\varphi"]
\\
& \rho^* \V(\F)  \arrow[d, "\rho^* \pi"'] \arrow[r, "\rho_V"] & \V(\F) \arrow[d, "\pi"] 
\\
& G \times X \arrow[r, "\rho"'] & X 
\end{tikzcd}
\end{center}
commutes. This gives a bundle map $\varphi : G \times \V(\F) \to \rho^* \V(\F)$. Therefore we have a morphism $\varphi : p^* \V(\F) \to \rho^* \V(\F)$ of vector bundles over $G \times X$ and thus, by the lemma, a morphism $\varphi : \V(p^* \F) \to \V(\rho^* \F)$. By the anti-equivalence of vector bundles and quasi-coherent sheaves, this is the same as giving a morphism $\varphi : \rho^* \F \to p^* \F$ of quasi-coherent sheaves on $G \times X$, this morphism will be the defining feature of a $G$-sheaf. Next, we will investigate what restrictions may be placed on such a morphism.
\bigskip\\
The map $\rho : G \times \V(\F) \to \V(\F)$ is an action and thus additionally must satisfy,
\begin{center}
\begin{tikzcd}[row sep = huge]
G \times G \times \V(\F) \arrow[r, "m \times \id"] \arrow[d, "\id \times \rho_V"'] &  G \times \V(\F) \arrow[d, "\rho_V"] 
\\
G \times \V(\F) \arrow[r, "\rho_V"'] & \V(\F)
\end{tikzcd}
\end{center}
The corresponding diagram for the $G$-action on $X$ lets us consider the pullbacks of vector bundles on $G \times X$ over the maps $m \times \id_X$ and $\id \times \rho$. We have a morphism $\varphi : p^* \V(\F) \to \rho^* \V(\F)$ of vector bundles over $G \times X$. Applying the pullback functors we get morphisms,
\begin{align*}
(m \times \id_X)^* \varphi : (m \times \id_X)^* p^* \V(\F) & \to (m \times \id_X)^* \rho^* \V(\F) 
\\
(\id \times \rho)^* \varphi : (\id \times \rho)^* p^* \V(\F) & \to (\id \times \rho)^* \rho^* \V(\F)
\end{align*}
Note that $\rho \circ (\id \times \rho) = \rho \circ (m \times \id_X)$ by commutativity of the diagram and thus $(m \times \id_X)^* \rho^* \V(\F) = (\id \times \rho)^* \rho^* \V(\F)$. Denote this bundle over $G \times G \times X$ as $P$. Also, $p \circ (m \times \id_X) = p \circ p_{23}$ the projection $G \times G \times X \to X$ and $p \circ (\id \times \rho) = \rho \circ p_{23}$ the map $G \times G \times X \to X$ via $(g, h, x) \mapsto (h, x) \mapsto h \cdot x$. Then pulling back the bundle map $\varphi : p^* \V(\F) \to \rho^* \V(\F)$ along $p_{23} : G \times G \times X \to G \times X$ gives a morphism,
\[ p_{23}^* \varphi : p_{23}^* p^* \V(\F) \to p_{23}^* \rho^* \V(\F) \]
of vector bundles over $G \times G \times X$ between the two domains of the previous maps.
We need to be careful because there are two inequivalent bundle maps $P \to \rho^* \V(\F)$ since $P$ is realized as the pullback under two distinct maps. However, if we apply the bundle map down to $f_{\rho^*} : \rho^* \V(\F) \to \V(\F)$ these become equal. Now we will apply the pullback lemma (see below) to show that maps between double pullbacks are uniquely determined by bundle maps to $\V(\F)$ over the corresponding map $G \times G \times X \to X$. Thus, the commutative diagram above implies that the composition of bundle maps to $\V(\F)$ are equal and thus the corresponding pullbacks are also equal,
\[ (\id \times \rho)^* \varphi \circ p_{23}^* \varphi = (m \times \id_X)^* \varphi \]
Via the anti-equivalence between quasi-coherent sheaves and vector-bundles we find that $\varphi$ must satisfy the commutative diagram of quasi-coherent $\struct{G \times G \times X}$-modules,
\begin{center}
\begin{tikzcd}[row sep = huge]
(m \times \id_X)^* p^* \F \arrow[from=d, "(m \times \id_X)^* \varphi"] \arrow[from=r, "p_{23}^* \varphi"'] & (\id \times \rho)^* \rho^* \F \arrow[from=d, "(\id \times \rho)^* \varphi"']
\\
(m \times \id_X)^* \F \arrow[r, equals] & (\id \times \rho)^* \rho^* \F
\end{tikzcd}
\end{center}
Furthermore,
\begin{center}
\begin{tikzcd}[row sep = huge]
\V(\F) \arrow[r, "e \times \id"] \arrow[rd, "\id"] & G \times \V(\F) \arrow[d, "\rho_V"] 
\\
& \V(\F)
\end{tikzcd}
\end{center}
This says we may factor the identity map as,
\begin{center}
\begin{tikzcd}[row sep = huge]
\V(\F) \arrow[d, "\pi"] \arrow[r, "e \times \id_V"] & p^* \V(\F) \arrow[d, "p^* \pi"] \arrow[r, "\varphi"] & \rho^* \V(\F) \arrow[d, "\rho^* \pi"] \arrow[r, "\rho_V"] & \V(\F)  \arrow[d, "\pi"]
\\
X \arrow[r, "e \times \id_X"] & G \times X \arrow[r, "\id"] & G \times X \arrow[r, "\rho"] & X 
\end{tikzcd}
\end{center}
meaning that $\V(\F) \to \rho^* \V(\F)$ is the pullback over $e \times \id_X : X \to G \times X$ so $\id : \V(\F) \to \V(\F)$ is the unique map which projects to $\pi : \V(\F) \to X$ and $\varphi \circ (e \times \id_V) : \V(\F) \to \rho^* \V(\F)$. Therefore, applying the pullback functor on vector bundles, $(e \times \id_X)^* \varphi : \V(\F) \to \V(\F)$ is the identity. Note that,
\[ (e \times \id_X)^* p^* \V(\F) = (e \times \id_X)^* \rho^* \V(\F) = \V(\F) \]
because $\rho \circ (e \times \id_X) = p \circ (e \times \id_X) = \id_X$. Thus applying the anti-equivalence we find the condition $(e \times \id_X)^* \varphi : \F \to \F$ is the identity morphism of $\struct{X}$-modules. 
\end{proof}

\begin{remark}
This derivation leads us to the following definition.
\end{remark}

\begin{definition}
Let $\F$ be a quasi-coherent sheaf of $\struct{X}$-modules and a group scheme $G$ act on $X$. Then an $G$ action on $\F$ making $\F$ a $G$-equivariant sheaf on $X$ is a morphism $\varphi : \rho^* \F \to p^* \F$ of $\struct{G \times X}$-modules which satisfies the following coherence conditions. The diagram,
\begin{center}
\begin{tikzcd}[row sep = huge]
(m \times \id_X)^* p^* \F \arrow[from=d, "(m \times \id_X)^* \varphi"] \arrow[from=r, "p_{23}^* \varphi"'] & (\id \times \rho)^* \rho^* \F \arrow[from=d, "(\id \times \rho)^* \varphi"']
\\
(m \times \id_X)^* \F \arrow[r, equals] & (\id \times \rho)^* \rho^* \F
\end{tikzcd}
\end{center}
commutes in the category of $\struct{G \times G \times X}$-modules
and $(e \times \id_X)^* \varphi : \F \to \F$ is the identity map of $\struct{X}$-modules.
\end{definition}

\begin{lemma}[Pullback]
Given two Cartesian squares,
\begin{center}
\begin{tikzcd}
A \arrow[dr, phantom, "\ulcorner", very near start] \arrow[d] \arrow[r] & B \arrow[dr, phantom, "\ulcorner", very near start]\arrow[d] \arrow[r] & C \arrow[d]
\\
A' \arrow[r] & B' \arrow[r] & C'
\end{tikzcd}
\end{center} 
the outer rectangle is Cartesian as well.
\end{lemma}

\begin{example}
For any group scheme action $G$ on $X$ the structure sheaf $\struct{X}$ is always $G$-equivariant with a trivial action because under $\rho : G \times X \to X$ we can pull back,
\[ \rho^* \struct{X} = \rho^{-1} \struct{X} \otimes_{\rho^{-1} \struct{X}} \struct{G \times X} = \struct{G \times X} = p^* \struct{X} \]
\end{example}

\begin{theorem}
Let $G$ be a group scheme and $X$ a $G$-scheme. Let $\F$ be a quasi-coherent $G$-equivariant sheaf on $X$. Then there is a $G$-action on global sections making $\Gamma(X, \F^\vee)$ a $G$-module.
\end{theorem}

\begin{proof}
Consider a section $s : X \to \V(\F)$ of the vector bundle $\pi : \V(\F) \to X$ associated to the sheaf $\F$. For fixed $g \in G$ we consider the map $\iota_g$ defined by $x \mapsto (g, g^{-1} \cdot x)$. (This may map be defined as follows. The maps $\id : X \to X$ and $X \to \{ g^{-1} \} \subset G$ define $x \mapsto (g^{-1}, x)$ applying $\rho$ gives $x \mapsto g^{-1} x$. Pair this with the constant map $X \to \{ g \} \subset G$). Consider the diagram,
\begin{center}
\begin{tikzcd}[row sep = huge, column sep = huge]
G \times \V(\F) \arrow[d, "\id \times \pi"'] \arrow[r, "\rho_V"] \arrow[d] & \V(F) \arrow[d, "\pi"]
\\
G \times X \arrow[r, "\rho"] \arrow[u, bend right, "\id \times s"'] & X \arrow[l, bend left, "\iota_g"] 
\end{tikzcd}
\end{center}
Now define $g \cdot s = \rho_V \circ (\id \times s) \circ \iota_g$. I claim that $g \cdot s$ is a section of the bundle $\pi : \V(\F) \to X$. To see this, 
\[ \pi \circ (g \cdot s) = \pi \circ \rho_V \circ (\id \times s) \circ \iota_g = \rho \circ (\id \times \pi) \circ (\id \times s) \circ \iota_g = \rho \circ \iota_g = \id_X \]
The coherence conditions then imply that this is an action. This gives a $G$-action on the dual $\Gamma(X, \F^\vee)$. It is instructive to rephrase this action. We have seen how an equivariant action on a vector bundle induces an morphism of the two pullback bundles. The morphism $\varphi : p^* \V(\F) \to \rho^* \V(\F)$ of bundles over $G \times X$ induces a map on their sections $\varphi : \Gamma(X,  p^*\V(\F)) \to \Gamma(X, \rho^* \V(\F))$ 
\end{proof}

\begin{proposition}
In particular, if work in the category of schemes over a field $k$ then we can form a dual $G$-action on $\F$ sections (rather than $\pi : X \to \V(\F)$ sections which are $\F^\vee$ sections) giving $\Gamma(X, \F)$ a $G$-representation structure over $k$. 
\end{proposition}

\begin{proof}
Recall that we have a morphism of $\struct{G \times X}$-modules $\varphi : \rho^* \F \to p^* \F$. Furthermore, the action $\rho : G \times X \to X$ defines the pullback functor,
\[ \rho^* : \QCoh{(\struct{X})} \to \QCoh{(\struct{G \times X})} \]
Applying this to a $\struct{Y}$-module morphism $s : \struct{Y} \to \F$ gives $\rho^* s : \struct{G \times X} \to \rho^* \F$ (note for $f : X \to Y$ that $f^* \struct{Y} = f^{-1} \struct{Y} \otimes_{f^{-1} \struct{Y}} \struct{X} = \struct{X}$). Since $\struct{X}$-module maps $\struct{X} \to \F$ are exactly global sections $\Gamma(X, \F)$ we have constructed the pullback map on sections $\rho^* :  \Gamma(X, \F) \to \Gamma(G \times X, \rho^* \F)$. Composing gives a morphism,
\begin{center}
\begin{tikzcd}
\Gamma(X, \F) \arrow[r, "\rho^*"] & \Gamma(G \times X, \rho^* \F) \arrow[r, "\varphi"] & \Gamma(G \times X, p^* X)
\end{tikzcd}
\end{center}
Since we are working in the category of schemes over $k$, we may now apply the K\"{u}nneth formula,
\begin{center}
\begin{tikzcd}
H^0(G \times X, p^* \F) = H^0(G \times X, p_1^* \struct{G} \otimes_{\struct{G \times X}} p_2^* \F) = H^0(G, \struct{G}) \otimes_k H^0(X, \F)
\end{tikzcd}
\end{center}
Therefore, we have a map,
\[ \Gamma(X, \F) \to \Gamma(G, \struct{G}) \otimes_k \Gamma(X, \F) \]
Since $\Gamma(G, \struct{G}) \cong \Homover{k}{G}{\A^1_k}$ the above map gives an \textit{algebraic action} on the $k$-vectorspace $\Gamma(X, \F)$. 
The coherence of the action follows from the coherence conditions on $\varphi$. 
\end{proof}

\end{document}