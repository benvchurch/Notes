\documentclass[12pt]{article}
\usepackage{import}
\import{./}{Includes}


\begin{document}

\section{Schemes}

\section{More Schemes}

\section{Seperated Schemes}

\begin{lemma}
A topological space $X$ is Hausdorff iff $X \xrightarrow{\Delta} X \times X$ is closed. 
\end{lemma}

\begin{definition}
A scheme is \textit{seperated} iff $X \xrightarrow{\Delta} X \times X$ is a closed immersion. 
\end{definition}

\begin{definition}
A morphism of schemes $f : X \to Y$ is \textit{affine} if and only if for any $V \subset Y$ affine open then $f^{-1}(V) \subset X$ is affine open. 
\end{definition}

\begin{definition}
A morphism $f : X \to Y$ is separated if the diagonal morphism $\Delta_{X/Y} : X \to X \times_Y X$ is a closed immersion. Thus $X$ is separated if $X \to \Spec{Z}$ is separated. 
\end{definition}

\begin{proposition}
Let $f : X \to Y$ and $g : Y \to Z$ be morphisms of schemes. If $g \circ f$ is seperated then $f$ is seperated.
\end{proposition}

\begin{proof}
Assume that $g \circ f : X \to Z$ is separated. The diagonal morphism $\Delta_{X/Z}$ factors though $X \to X \times_Y X \to X \times_Z Y$ then $X \to X \times_Y X$ is a closed immersion of schemes.
\end{proof}

\begin{proposition}
Any affine scheme is separated. 
\end{proposition}

\begin{proof}
Consider the map $\Spec{A} \to \Spec{Z}$. The diagonal morphism $\Delta : \Spec{A} \to \Spec{A} \times \Spec{A} = \Spec{A \otimes A}$ is given by the ring map $A \otimes A \to  A$ by $a \otimes b \to ab$. This ring map is clearly surjective and thus $\Delta$ is a closed immersion. 
\end{proof}

\begin{definition}
Let $S$ be a graded ring then $\Proj{S}$ as a set is the set of homogeneous prime ideals which do not contain $S_+$. Then $\Proj{S}$ is a scheme with structure sheaf $\struct{\Proj{S}}$ such that for any homogenous $f \in S_+$ we have $\struct{\Proj{S}}(D_+(f)) = \Spec{S_{(f)}}$ where $S_{(f)}$ is the zero-degree part of $S_f$ and $D_+(f) = \{ \p \in \Proj{S} \mid f \notin \p \}$. 
\end{definition}

\begin{proposition}
Let $S$ be a graded ring then $\Proj{S}$ is separated. 
\end{proposition}

\begin{proof}
Consider the canonical morphism $\Proj{S} \to \Spec{Z}$. Given any pair of standard opens $D_+(f)$ and $D_+(g)$ we have affine $D_+(f) \cap D_+(g) = D_+(fg)$. Now to show that the diagonal map
\[ \Proj{S} \to \Proj{S} \times \Proj{S} \]
is a closed immersion it suffices to prove that for these affine opens that $u \cap V = \Delta^{-1}(U \times V) \to U \times V$ is a closed immersion. Since these are affine we can check that the ring map $S_{(f)} \otimes S_{(g)} \to S_{(fg)}$ is surjective. For any $s = h/(f^n g^m) \in S_{(fg)}$ we can rewrite $s$ as,
\[ s = h/f^{(n' + n') \deg{s}} \cdot f^{m' \deg{g}} / g^{m' \deg{f}} \]
by multiplying $h$ by a suitable $f^ig^j$ such that $n = n' \deg{g}$ and $m = m' \deg{f}$. Then $s$ is in this image. 
\end{proof}

\begin{definition}
Let $\struct{\Spec{A}}$ be the unique sheaf of rings of $\Spec{A}$ such that,
\[ \struct{\Spec{A}}(D(f)) = A_f \]
and thus $\struct{\Spec{A}}(A) = \struct{\Spec{A}}(D(1)) = A$. Furthermore,
\[ (\struct{\Spec{A}})_{\p} = A_{\p} \]
which is local. 
\end{definition}

\begin{proposition}
Let $A$ and $B$ be rings then,
\[ \Homover{\Sch}{\Spec{B}, \struct{\Spec{B}}}{\Spec{A}, \struct{\Spec{A}}} = \Homover{\Ring}{A}{B} \]
given by sending $(f, f^\#)$ to $f^\#$ on global sections. 
\end{proposition}

\begin{definition}
A morphism of schemes $\iota : Z \to X$ is a \textit{closed immersion} iff $\iota$ is affine and for every affine open $U \subset X$, the map,
\[ \struct{X}(U) \to \struct{Z}(\iota^{-1}(U)) \]
is surjective. 
\end{definition}

\begin{remark}
Any closed subset of $\Spec{A}$ is of the form $V(I) = \{ \p \in \Spec{A} \mid \p \supset I \}$ where $I \subset A$ is an ideal. 
\end{remark}

\begin{remark}
If $\varphi : A \to B$ is a ring map then there is a map $\Spec{\varphi} : \Spec{B} \to \Spec{A}$ given by $\p \mapsto \varphi^{-1}(\p)$ which is continuous with respect to the Zariski topology via, $\Spec{\varphi}^{-1}(D(f)) = D(\varphi(f))$. Furthermore, $\Spec{\varphi}$ is canoncially a morphism of affine schemes via,
\[ \Spec{\varphi}^\# : \struct{\Spec{A}}(D(f)) \to \struct{\Spec{B}}(D(\varphi(A)) \]
given by sending $A_f$ to $B_{\varphi(f)}$ by $a/f^n \mapsto \varphi(a) / \varphi(f)^n$. 
\end{remark}

\begin{proposition}
A surjective ring map $A \to B$ with kernel $I$ induces a homeomorphism $\Spec{B} \to \Spec{A/I} = V(I) \subset \Spec{A}$. If $f \in A$ then $\Spec{A \to A_f}$ is a homeomorphism $D(f) = \Spec{A_f} \to \Spec{A}$
(CHECK THIS)
\end{proposition}


\begin{proposition}
Let $X$ be a scheme and $K$ a field. A morphism $\Spec{K} \to X$ is the same as specifying a point $p \in X$ and an inclusion $\iota : k(p) \to K$ where $k(p) = \stalk{X}{x} / \m_x$ is the residue field at $x$.
\end{proposition}

\begin{proof}
Let $(f, f^\#) : \Spec{K} \to X$ be a morphism. Then take the image $\{p\} = f((0))$. Furthermore, we have a sheaf map,
\[ f^\# : \struct{X}(U) \to \struct{\Spec{K}}(f^{-1}(U)) 
= \begin{cases}
K & p \in U
\\
0 & p \notin U 
\end{cases} \] 
Consider the commutative diagram,
\begin{center}
\begin{tikzcd}
\struct{X}(U) \arrow[d] \arrow[r, "f^\#"] & \struct{\Spec{K}}(f^{-1}(U)) \arrow[d]
\\
\stalk{X}{x} \arrow[r, "f^\#_x"] & \stalk{\Spec{K}}{(0)}
\end{tikzcd}
\end{center}
On opens $U$ with $p \notin U$ clearly the map $f^\# : \struct{X}(U) \to \struct{\Spec{K}}(f^{-1}(U))$ is the zero map. Otherwise, the map $\struct{\Spec{K}}(f^{-1}(U)) \to \stalk{\Spec{K}}{(0)}$ is the identity. Therefore, the above diagram determines $f^\# = f^\#_x \circ \mathrm{res}_{U, x}$ uniquely from the stalk map 
\[f^\#_x : \stalk{X}{x} \to \stalk{\Spec{K}}{(0)} = K \]
Furthermore, $f^\#_x$ must be a local so $f^\#_x(\m_x) = (0)$ since $(0)$ is maximal in $K$. Therefore, this map factors through $k(p) = \stalk{X}{x} / \m_x$. Therefore, $f^\#$ is determined from the map $k(p) \to K$ (which is an inclusion) via the canonical composition,
\begin{center}
\begin{tikzcd}
\struct{X}(U) \arrow[r] & \stalk{X}{x} \arrow[r] & \stalk{X}{x} / \m_x \arrow[r, "f^\#_x"] & K
\end{tikzcd}
\end{center} 
\end{proof}

\subsection{Proper Morphism}

\begin{definition}
We say a continuous map $f : X \to Y$ of topological spaces is \textit{proper} iff for each $K \subset Y$ quasi-compact then $f^{-1}(K)$ is quasi-compact.
\end{definition}

\begin{definition}
A continuous map $f : X \to Y$ of topological spaces is \textit{universally closed} iff for each $Z$ the map $f \times \id_Z : X \times Z \to Y \times Z$ is closed. 
\end{definition}

\begin{definition}
Let $P$ be a property of morphisms in $\Top$. A morphism $f : X \to Y$ in $\Top$ is \textit{universally $P$} if for every $f : Z \to Y$ under base change,
\begin{center}
\begin{tikzcd}[column sep = huge, row sep = huge]
Z \times_Y X \arrow[d, "f'"] \arrow[r, "g'"] & X \arrow[d, "f"]
\\
Z \arrow[r, "g"] & Y
\end{tikzcd}
\end{center} 
then $f'$ has property $P$.
\end{definition}

\begin{theorem}
Let $f : X \to Y$ be a continuous map of toplogical spaces then TFAE:
\begin{enumerate}
\item $f$ is proper and closed
\item $f$ is universally closed
\item $f$ is closed and $\forall y \in Y : f^{-1}(y)$ is quasi-compact. 
\end{enumerate}
\end{theorem}



\begin{definition}
A morphisms of schemes $f : X \to Y$ is proper iff it is
\begin{enumerate}
\item finite type.
\item seperated.
\item universally closed.
\end{enumerate}
\end{definition}

\begin{example}
Let $k$ be a field then $\A^1_k \to \Spec{k}$ is closed but \underline{not} universally closed. 
\end{example}

\begin{proof}
Consider the diagram,
\begin{center}
\begin{tikzcd}
V(xy - 1) = \Spec{k[x,y]/(xy - 1)} \arrow[d] \arrow[r, hook] & \A^2_k \arrow[r] \arrow[d] & \A^1_k \arrow[d]
\\
\A^1_k \setminus \{(x)\} \arrow[r, hook] & \A^1_k \arrow[r] & \Spec{k}
\end{tikzcd}
\end{center}
Since $\A^1_k$ is not closed, the left map is not closed. 
\end{proof} 

\section{Classical Algebraic Varieties}

\begin{definition}
An affine variety over a field $k$ is a subset $V(I) \subset k^n$ with the Zariski topology where $I \subset k[x_1, \dots, x_n]$ is an ideal. An affine variety is a locally ringed space with the sheaf of regular functions. A regular function on open $U \subset V$ is a map $f : U \to k$ such that $f$ is locally a regular (demoninator not vanishing) quotient of polynomials. 
\end{definition}

\begin{definition}
A morphism of classical affine varieties $V$ and $W$ is a continuous map $\varphi : V \to W$ such that for all open $U \subset V$ and $U' \subset W$ such that $\varphi(U) \subset U'$ and a regular function $f' : W' \to k$ then $f' \circ \varphi|_{U}$ is a regular function on $U$
\end{definition}

\begin{definition} 
A (classical) algebraic variety is a locally ringed space which is locally isomorphic to affine varieties.  
\end{definition}

\begin{lemma}
Let $V \subset k^n$ be an affine variety. Then any (global) regular function $f : V \to k$ is a polynomial.
\end{lemma}

\begin{proof}
Let $V \subset k^n$ be an affine variety and $f : V \to k$ be a regular function. Define,
\[ I = \{ g \in k[x_1, \dots, x_n] \mid g|_V \cdot f \in k[x_1, \dots, x_n] \} \]
First, I claim that $I$ is an ideal. Take $g \in I$ and $h \in k[x_1, \dots, x_n]$ then $hg \in I$ since if $g|_V \cdot f = e|_V$ for some $e \in k[x_1, \dots, x_n]$ then $(hg)|_V \cdot f = (h e)|_V$ so $hg \in I$. Similarly, if $g_1, g2 \in I$ then $g_i|_V \cdot f = e_i |_V$ for some $e_i \in k[x_1, \dots, x_n]$ and so $(g_1 + g_2)|_V \cdot f = (e_1 + e_2)|_V$ and thus $g_1 + g_2 \in I$. \bigskip\\
Now I claim that $I$ is not contained in any maximal ideal. Let $\m \subset k[x_1, \dots, x_n]$ be a maximal ideal. Then by Hilbert's Nullstellensatz, 
\[ \m = (x_1 - a_1, \dots, x_n - a_n) = \{g \in k[x_1, \dots, x_n] \mid g(p) = 0\} \]
for some $p = (a_1, \dots, a_n)$. Suppose that $p \notin V$. Since $V$ is Zariski closed, there exists $g \in k[x_1, \dots, x_n]$ such that $g(p) \neq 0$ and $g|_V = 0$. Then $g|_V \cdot f = 0$ so $g \in I$ but $g \notin \m$. Otherwise, $p \in V$ and thus by definition, there exists an open $p \in U \subset V$ and polynomials $a,b \in k[x_1, \dots, x_n]$ such that $b(p') \neq 0$ for all $p' \in U$ and $f|_V = \left( \frac{a}{b} \right)|_U$. Therefore, we have,
\[ b|_U \cdot f|_U = a|_U \implies b|_V \cdot f = a|_V \]
since $U$ is dense in $V$ and $b|_V \cdot f - a$ is continuous. 
\end{proof}

\begin{corollary}
Let $V \subset k^n$ be a classical affine variety and $g : V \to k$ be a regular function. Then,
\[ \struct{V}(\{p \in V \mid g(p) \neq 0\}) = \struct{V}(V)_g = \struct{V}(V)[g^{-1}] \]
\end{corollary}

\begin{corollary}
$\struct{V}(V) = k[x_1, \dots, x_n] / I$
where $I = \{f \in k[x_1, \dots, x_n] \mid f|_V = 0 \}$.
\end{corollary}

\begin{corollary}
$\Hom{}{V}{W} = \Homover{k-\text{alg}}{\struct{V}(V)}{\struct{W}(W)}$
\end{corollary}

\begin{theorem}
Let $k$ be an algebraically closed field. There is a fully faithfull functor from the category of classical varieties over $k$ to the category of schemes over $k$ whose image lies in the subcategory of integral (reduced and irreducible) seperated schemes of finite type over $k$. 
\end{theorem}


\begin{proof}
Let $k$ be an algebraically closed field. The functor $v : \Var(k) \to \Sch(k)$ is constructed by taking the underlying space $X$ and sending it to $v(X)$ the set of irreducible closed subsets with the topology generated by closed sets $v(C)$ for each closed set $C \subset X$. Furthermore, there is a continuout map $\alpha_X : X \to v(X)$ given by $\alpha_X(p) = \overline{\{ p \}}$. 
\end{proof}


\section{Sheaves of Modules}

\begin{definition}
Let $A$ be a ring and $X = \Spec{A}$. For every $A$-module $M$ there is a unique sheaf of $\struct{X}$-modules $\widetilde{M}$ such that,
\[ \widetilde{M}(X) = M \quad \quad \widetilde{M}(D(f)) = M_f \]
The stalks are $(\widetilde{M})_\p = M_\p = M \otimes_A A_\p$. 
\end{definition}

\begin{proposition}
For any $\struct{X}$-module $\mathcal{F}$ we have,
\[ \Homover{\struct{X}}{\widetilde{M}}{\mathcal{F}} = \Homover{\struct{X}(X)}{M}{\mathcal{F}(X)} \]
given by sending $\varphi$ to its action on global sections. 
\end{proposition}

\begin{proof}
Given a map $M \xrightarrow{\psi} \mathcal{F}(X)$ we get a diagram,
\begin{center}
\begin{tikzcd}
\widetilde{M}(X) \arrow[r, "\sim"] & M \arrow[r, "\psi"] \arrow[d] & \mathcal{F}(X) \arrow[d, "\mathrm{res}"]
\\
\widetilde{M}(D(f)) \arrow[r, "\sim"] & M_f \arrow[r, dashed] & \mathcal{F}(D(f))
\end{tikzcd}
\end{center}
We get a map $M_f \to \mathcal{F}(D(f))$ because $f$ becomes invertible in $\mathcal{F}(D(f))$ because $\mathcal{F}(D(f))$  is an $A_f$-module. 
\end{proof}

\begin{corollary}
The functor $M \mapsto \widetilde{M}$ is a fully faithful functor from the category of $A$-modules to the category of $\struct{X}$-modules.  
\end{corollary}

\begin{definition}
Let $(X, \struct{X})$ be a scheme. An $\struct{X}$-module $\mathcal{F}$ is \textit{quasi-coherent} iff for every affine open $U \subset X$ with $U \cong \Spec{A}$ we have $\mathcal{F}|_U \cong \widetilde{M}$ for some $A$-module $M$. 
\end{definition}

\begin{definition}
Let $(f, f^\#) : (X, \struct{X}) \to (Y, \struct{Y})$ is a map of ringed spaces. If $\mathcal{F}$ is an $\struct{X}$-module then $f_* \mathcal{F}$ is the $\struct{Y}$-module with values,
\[ (f_* \mathcal{F})(V) = \mathcal{F}(f^{-1}(V)) \]
as a $\struct{Y}(V)$-module via the map,
\[ f^\# : \struct{Y}(V) \to \struct{X}(f^{-1}(V)) \]
Then $f^*$ is defined to be the left adjoint of $f_*$. This adjoint exists because we can define,
\[ f^* \mathcal{G} = f^{-1} \mathcal{G} \otimes_{f^{-1} \struct{Y}} \struct{X} \]
where $\struct{X}$ is an $f^{-1} \struct{Y}$-module under the map $f^\# : f^{-1} \struct{Y} \to \struct{X}$ and $\mathcal{G}$ is a $\struct{Y}$-module and pullback is additive so $f^{-1} \mathcal{G}$ is a $f^{-1} \struct{Y}$-module. 
This functor has stalks,
\[ (f^* \mathcal{G})_x = \mathcal{G}_{f(x)} \otimes_{\stalk{Y}{f(x)}} \stalk{X}{x} \]
\end{definition}

\begin{lemma}
Let $f : \Spec{B} \to \Spec{A}$ be given by $\varphi : A \to B$ then,
\begin{enumerate}
\item $f_*(\widetilde{N}) = \widetilde{N}_A$ for $N$ a $B$-module
\item $f^*(\widetilde{M}) = \widetilde{M \otimes_A B}$ for $M$ an $A$-module.
\end{enumerate}
\end{lemma}

\begin{proof}
Let $X = \Spec{A}$ and $Y = \Spec{B}$. By adjunction,
\begin{align*}
\Homover{\struct{Y}}{f^* \widetilde{M}}{\mathcal{G}} & = \Homover{\struct{X}}{\widetilde{M}}{f_* \mathcal{G}} = \Homover{A}{M}{(f_* \mathcal{G})(X)}
\\
& = \Homover{B}{M \otimes_A B}{\mathcal{G}(Y)} = \Homover{\struct{X}}{\widetilde{M \otimes_A B}}{\mathcal{G}}
\end{align*}
Because $(f_* \mathcal{G})(X) = \mathcal{G}(f^{-1}(X)) = \mathcal{G}(Y)$. However, 
\[ \Homover{B}{M \otimes_A B}{C} = \Homover{A}{M}{\Homover{B}{B}{C}} = \Homover{A}{M}{C} \]
\end{proof}

\begin{proposition}
For any $A$-module $M$ the sheaf $\widetilde{M}$ is quasi-coherent.
\end{proposition}

\begin{proof}
If $U \subset \Spec{A}$ is affine open say $U = \Spec{B}$ then,
\[ \widetilde{M}|_U = \widetilde{M \otimes_A B} \]
\end{proof}

\section{Picard Group}

Let $(X, \struct{X})$ be a locally ringed space. The Picard group is isomorphism classes of invertible $\struct{X}$-modules with addition given by tensor product. 

\begin{proposition}
Any finite locally free $\struct{X}$-modules on a scheme $X$ is quasi-coherent.
\end{proposition}

\begin{proof}
Suppose that  $\mathcal{E}$ is locally free. We have to show that $\forall x \in X \exists U \subset X$ affine open such that $\mathcal |_U$ is of the form $\tilde{M}$ for some $\struct{X}(U)$-module. However, we know that $\exists x \in W \subset X$ open s.t for some $n$,
\[ \mathcal{E}|_W \cong \struct{W}^{\oplus n} \] 
We may assume that $W = U$ is affine since affine opens form  a basis of the toplogy. Then,
\[ \mathcal{E}|_W \cong \struct{U}^{\oplus n} = \widetilde{\struct{X}(U)^{\oplus n}} \]
\end{proof}

\begin{remark}
We further showed that the modules that $\mathcal{E}$ locally looks like (as a quasi-coherent) sheaf are finitely generated.
\end{remark}

\begin{definition}
An $\struct{X}$-module $\mathcal{F}$ is \textit{of finite type} iff $\forall x \in X \exists x \in U \subset X$ is open s.t. $\mathcal{F}|_U$ is generated by finitely many sections i.e. for some $n$ there exists $s_1, \dots, s_n \in \mathcal{F}(U)$ such that the map, $\struct{U}^{\oplus n} \to \mathcal{F}|_U $ given by $(f_1, \dots, f_n) \to \sum f_i s_i$ is surjective as a map of sheaves. 
\end{definition}

\begin{lemma}
The tilde functor is exact because $(\widetilde{M})_\p = M_\p$ and localization is exact. 
\end{lemma}

\begin{lemma}
Let $\mathcal{F} = \widetilde{M}$ be a quasi-coherent module on $\Spec{A}$ then TFAE,
\begin{enumerate}
\item $\mathcal{F}$ is of finite type (as $\struct{\Spec{A}}$-module)
\item $M$ is of finite type (as $A$-module)
\end{enumerate}
\end{lemma}

\begin{proof}
Suppose that $M$ is f.g. then $M = m_1 A + \cdots + m_n A$ However, $\mathcal{F}(\Spec{A}) = \widetilde{M}(\Spec{A}) = M$ so $m_1, \dots, m_n$ are global sections. I claim that these generate $\mathcal{F}$. The map,
\begin{center}
\begin{tikzcd}
A^{\oplus n} \arrow[r, two heads] & M
\end{tikzcd}
\end{center}
is surjective. Applying the functor,
\begin{center}
\begin{tikzcd}
\struct{\Spec{A}}^{\oplus n} = \widetilde{A^{\oplus n}} \arrow[r, two heads] & \widetilde{M}
\end{tikzcd}
\end{center}
Since tilde is an exact functor, this map remains a surjection. 
\end{proof}

\begin{proposition}

\end{proposition}

\begin{proof}
Let $\mathcal{L} \in \Pic{\Spec{A}}$ then $\mathcal{L} = \widetilde{M}$ and now know that $M$ is finite since $\mathcal{L}$ is rank one and quasi-coherent. and $M_\p \cong A_\p$ for all $\p \in \Spec{A}$. 
\end{proof}

\section{Feb 8}

Ask about why the pullback can be checked locally but not the pushforward in Prop. 5.8. Does it have to do with how we check the map on stalks which we know for the pullback but not for pushforward?

\begin{lemma}
Let $f : X \to Y$ be a morphism of schemes. The pullback of a quasi-coherent module is quasi-coherent.
\end{lemma}

\begin{proof} 
Suppose that $\mathcal{G}$ is a quasi-coherent $\struct{Y}$-module. Pick $V \subset Y$ affine open $f(x) \i V$ and affine open $U \subset X$ such that $x \in U$ and $f(U) \subset V$ (since it is a basis for the topology and $f$ is continuout). Consider the diagram,
\begin{center}
\begin{tikzcd}
U \arrow[r, hook] \arrow[d, "f|_U"] & X \arrow[d, "f"]
\\
V \arrow[r, hook] & Y
\end{tikzcd}
\end{center} 
Thus,
\[ f^* \mathcal{G}|_U = (f_U)^* \mathcal{G}|_V \]
which reduces the case to a morphism of affine schemes. 
\end{proof}

\begin{definition}
A morphism of schemes $f : X \to Y$ is \textit{quasi-compact} iff for all $V \subset Y$ quasi-compact open then $f^{-1}(V)$ is quasi-compact open. 
\end{definition}

\begin{lemma}
Let $f : X \to Y$ be a quasi-compact separated morphism of schemes. The pushforward of a quasi-coherent module is quasi-coherent.
\end{lemma}

\begin{proof}
The questio is local on $Y$ hence we may assume that $Y$ is an affine scheme. Choose an affine open cover $U_i$ of $X$. Then $Y$ is quasi-compact so its preimage is quasi-compact by the assumption that $f$ is quasi-compact. Hence we may take $X$ to be quasi-compact and thus we can finite a finite subcover $U_i$. Take a quasi-coherent sheaf of $\struct{X}$-modules $\mathcal{F}$. Denote $f_i : U_i \to Y$ the restriction of $f$ to $U_i$ which is a morphism of affine schemes. Applying the sheaf property we find,
\begin{center}
\begin{tikzcd}
0 \arrow[r] & \mathcal{F}(f^{-1}(V)) \arrow[r] & \prod_{i = 1}^n \mathcal{F}(U_i \cap f^{-1}(V)) \arrow[r] & \prod_{i,j} \mathcal{F}(U_i \cap U_j \cap f^{-1}(V))
\\
0 \arrow[r] & (f_*\mathcal{F})(V) \arrow[r] & \prod_{i = 1}^n ((f_i)_* \mathcal{F})(V) \arrow[r] & \prod_{i,j} ((f_{ij})_* \mathcal{F})(V)
\end{tikzcd}
\end{center}
The second terms is quasi-coherent because it is the finite product of quasi-coherent sheaves since $(f_i)_*$ is a morphism of affine schemes. If $U_i \cap U_j$ if affine for all $i$ and $j$ then we see that $f_* \mathcal{F}$ is a kernel of a map of quasi-coherent modules and thus quasi-coherent. Seperatedness gives exactly this property. 
\end{proof}

\begin{lemma}
Kernels, cokernels, images, and coimages of quasi-coherent modules are quasi-coherent.
\end{lemma}

\begin{proof}
On an affine path $X = \Spec{A}$ any morphism of quasi-coherent modues has the form $\tilde{\varphi} : \tilde{M} \to \tilde{N}$ for $\varphi : M \to N$and thus $\ker{\tilde{\varphi}} = \wt{\ker{\varphi}}$. 
\end{proof}

\begin{definition}
A morphism of schemes $f : X \to Y$ is called seperated iff the diagonal map $\Delta : X \to X \times_Y X$ is a closed immersion.  
\end{definition}

\begin{lemma}
If $f : X \to Y$ is seperated then for any affine open $U, U' \subset X$ with $f(U) \cup f(U')$ contained in an affine open $V \subset Y$ then $U \cap U'$ is affine. 
\end{lemma}

\begin{proof}
Then $U \times_V U' \subset X \times_Y X$ is open. However, $U \times_V X = U \times_Y U'$. Furthermore, $\Delta^{-1}(U \times_Y U') = U \cap U'$ because $\Delta(x) \in U \times_Y U'$ exactly when $x \in U$ and $x \in U'$ since $\Delta = \id \times_Y \id$.
\begin{center}
\begin{tikzcd}
& X \arrow[r, "\Delta"] & X \times_Y X & 
\\
U \cap U' \arrow[r, "\sim"] & \Delta^{-1}(U \times_V U') \arrow[r] \arrow[u] & U  \times_V U' \arrow[r, "\sim"] \arrow[u, hook] & U \times_Y U'
\end{tikzcd}
\end{center}
Because $\Delta$ is a closed immersion then $\Delta$ is affine so $U \cap U'$ is affine. 
\end{proof}

\begin{proposition}
Let $A_n = k[x,y]/(1 - y(x - t_1) \cdots (x - t_n))$ then if $n \neq m$ then $A_n \not\cong A_m$. 
\end{proposition}

\begin{proof}

\end{proof}

\begin{proposition}
If $X = \Spec{A}$ is any smooth affine curve over an algebracially closed $k$ with $\Frac{A} \cong k(z)$ then $A \cong A_n$ for some $n$ and some $t_1, \dots, t_n \in k$.
\end{proposition}

\begin{proof}

\end{proof}

\section{Feb 14}

\subsection{Number Theory}

Let $K$ be a number field with ring of integers $\ints{K}$ then why is $\Pic{\ints{K}}$ related to $\Cl{\ints{K}}$. Since $\ints{K}$ is a Dedekind domain we know that $(\ints{K})_\p$ at each prime is a DVR. Therefore if $I$ is a fractional ideal then, $I_\p \cong \alpha (\ints{K})_\p$ since $(\ints{K})_\p$ is a PID. Furthermore, the principal ideal are eactly the globally rank $1$ modules. Thus, $\Cl{\ints{K}} \cong \Pic{\ints{K}}$. Furthermore, this group is the free abelian group on prime $\p \subset \ints{K}$ modulo principal divisors i.e. the prime valuations of $\alpha \in K^\times$.

\subsection{Integral Schemes}

\begin{lemma}
Let $\F$ be a sheaf on $X$ and $U \subset X$ an open set. Then the product of the natural maps,
\[ \F(U) \to \prod_{x \in X} \F_x \]
is injective.
\end{lemma}

\begin{proof}
If $f \in \F(U)$ restricts to $f_x = 0$ in each stalk $\F_x$ then there exists an open set $x \in V_x$ such that $f |_{V_x} = 0$. Therefore, $f$ and $0$ have equal restrictions to the open cover $\{ V_x \mid x \in U \}$ so by the sheaf property $f = 0$. 
\end{proof}

\begin{definition}
A scheme is \textit{reduced} if for each open $U \subset X$ the ring $\struct{X}(U)$ is reduced.
\end{definition}

\begin{proposition}
The following are equivalent,
\begin{enumerate}
\item $X$ is reduced
\item for each $x \in X$ the stalk $\stalk{X}{x}$ is reduced
\item for each affine open $U \subset X$ the ring $\struct{X}(U)$ is reduced
\item $X$ has an affine open cover by spectra of reduced rings
\end{enumerate}
\end{proposition}

\begin{proof}
Let $X$ be a reduced schemes. Take $x \in X$ and consider the stalk,
\[ \stalk{X}{x} = \varinjlim_{x \in U} \struct{X}(U) \]
Each $\struct{X}(U)$ is a reduced ring so if $f \in \stalk{X}{x}$ satisfies $f^n = 0$ then on each open neighborhood of $x$ we have $f = 0$ and thus $f = 0$i in $\stalk{X}{x}$. Conversely, if all stalks are reduced then for any open set $U \subset X$ conisder an element $f \in \struct{X}(U)$. If $f^n = 0$ then $f^n = 0$ in each stalk $\stalk{X}{x}$ at $x \in U$ which implies $f = 0$ since $\stalk{X}{x}$ is reduced. Thus $f = 0$ in $\struct{X}(U)$ so $X$ is reduced. 
\bigskip\\
If $X$ is reduced then $\struct{X}(U)$ is reduced for all open sets and thus, in particular, all affine opens which implies that if $\{U_i\}$ is an affine open cover of $X$ with $U_i = \Spec{A_i}$ then $\struct{X}(U_i) = \struct{\Spec{A_i}}(U_i) = A_i$ is a reduced ring.
\bigskip\\
Assume that $\{ U_i \}$ is an affine open cover of $X$ with $U_i = \Spec{A_i}$ where $A_i$ is a reduced ring. At a point $x \in U_i$ corresponding to a prime ideal $\p \subset A_i$, consider the stalk,
\[ \stalk{X}{x} = \stalk{\Spec{A_i}}{\p} = (A_i)_\p \]
Since $A_i$ is reduced, then $(A_i)_\p$ is reduced which implies that $X$ has reduced stalks and thus is reduced. 
\end{proof}

\begin{definition}
A scheme is \textit{integral} if for each open $U \subset X$ the ring $\struct{X}(U)$ is a domain.
\end{definition}

\begin{lemma}
If $X$ is integral then $X$ is irreducible.
\end{lemma}

\begin{proof}
If $X$ is not irreduclbe then there exist non-empty disoint open sets $U_1, U_2 \subset X$ which implies that $\struct{X}(U_1 \cup U_2) = \struct{X}(U_1) \times \struct{X}(U_2)$ which cannot be an integral domain. Thus $X$ is irreducible.
\end{proof}

\begin{proposition}
A scheme $X$ is integral iff for every affine open $U \subset X$ with $U = \Spec{A}$ then $A$ is a domain. 
\end{proposition}

\begin{proof}
First, $X$ is an irreducible scheme and thus has a unique generic point $\xi \in X$. Then for any affine open $U \subset X$ with $U = \Spec{A}$ then $\stalk{X}{\xi} = \Frac{A}$ the localization at the unique generic point of $\Spec{A}$ namely $(0)$ which is prime since $A$ is a domain. Thus, $\stalk{X}{\xi}$ is a field and $\struct{X}(U) \to \stalk{X}{\xi}$ is injective. For any open $U$ take an affine open cover $U_i$. Now consider $f \in \struct{X}(U)$ in the kernel of $\struct{X}(U) \to \stalk{X}{\xi}$. Then $f|_{U_i}$ is in the kernel of $\struct{X}(U_i) \to \stalk{X}{\xi}$ which is an injective map so $f|_{U_i} = 0$. Since $\{ U_i \}$ is an open cover of $U$, then $f = 0$ so the map $\struct{X}(U) \to \stalk{X}{\xi}$ is injective. Since $\stalk{X}{\xi}$ is a field and $\struct{X}(U)$ embedds inside it then $\struct{X}(U)$ is a domain Thus $X$ is integral. 
\end{proof}


\begin{proposition}
A scheme $X$ is integral if and only if it is reduced and irreducible.
\end{proposition}


\begin{proof}
Let $X$ be an integral scheme then clearly $X$ is reduced (each domain is reduced). We have already shown that $X$ is irreducible.
\bigskip\\
Conversely, suppose that $X$ is reduced and irreducible. Then for each affine open $U \subset X$ with $U = \Spec{A}$, we have $\struct{X}(U) = A$ is reduced so $\nilrad{A} = (0)$. Furthermore, since $X$ is irreducible so is $U = \Spec{A}$ which implies that $\nilrad{A}$ is prime. Therefore, $(0)$ is prime in $A$ so $A$ is a domain. Therefore, $X$ is integral. 
\bigskip\\
Alternatively, let $U \subset X$ be an arbitrary open set. Take $f, g \in \struct{X}(U)$ such that $fg = 0$. Consider the sets,
\[ A = \{ x \in X \mid f_x \in \m_x \} \quad \quad B = \{ x \in X \mid g_x \in \m_x \} \]
Since $f_x g_x = (fg)_x = 0$ then $f_x \in \m_x$ or $g_x \in \m_x$. Therefore, $X = A \cup B$. Furthermore, $A$ and $B$ are closed but $X$ is irreducible so $A = X$ or $B = X$. Without loss of generality, take $A = X$. Therefore, $f$ is nilpotent on any affine subset so $f$ is zero. Thus $X$ is integral. (DO THESE EXERCISES)
\end{proof}



\subsection{Abstract Varieties}

\begin{definition}
A scheme is reduced if every ring of sections is reduced i.e. has no nilpotents. A scheme is integral if every ring of sections is an integral domain (maybe besides the zero ring). 
\end{definition}

\begin{definition}
Let $k$ be a field, a \textit{variety} over $k$ is a seperated scheme of finite type over $k$ which is irreducible (as a topological space) and reduced. 
\end{definition}

\begin{remark}
We refer to \textit{abstract varieties} simply as \textit{varieties} from here on. Otherwise we will used the term \textit{classical varieties} to specify the non-scheme object. 
\end{remark}


\begin{definition}
If $X$ is a variety then its \textit{function field} is the residue field at the generic point. Equivalently. For any $U = \Spec{A} \subset X$ nonempty open take $\Frac{A}$. We notate this as $k(X)$.  
\end{definition}

\begin{remark}
These definitions are the same since $(0) \in \Spec{A}$ is the generic point then $\stalk{X}{(0)} = \Frac{A}$ and thus clearly its residue field it itself because it is a field (so its maximal ideal is $(0)$). 
\end{remark}

\begin{definition}
If $X$ is a scheme then $\dim{X}$ is the Krull dimension of the topological space $X$ i.e. the supremum of the length of chains of irreducible closed subsets. 
\end{definition}

\begin{proposition}
Let $X$ be a variety over $k$ then $\dim{X} = \trdeg{k}{k(X)}$. 
\end{proposition}

\begin{proposition}
If $X$ is sober with an open cover $U_i$ then $\dim{X} = \sup{\dim{U_i}}$. 
\end{proposition}

\begin{definition}
A variety $X$ is rational over $k$ iff $k(X)$ is a purely transcendental extension of $k$. 
\end{definition}

\begin{proposition}
Let $k$ be algebraically closed and $X$ a variety over $k$. Then $p \in X$ is closed iff $\kappa(p) = k$.
\end{proposition}

\begin{proof}
Let $p \in V \subset X$ be an affine open with $V \cong \Spec{A}$. Then $\stalk{X}{p} = \stalk{V}{p} = A_{\p}$ where $\p \in \Spec{A}$ is the prime corresponding to $p$. Now the residue field at $p$ is, 
\[ \stalk{X}{p} / \m_p = A_{\p} / \p A_\p = S_\p^{-1} ( A / \p ) \]
by the exactness of localizations. Furthermore, since $X$ is a scheme of finite type over $k$ there is a map $\Spec{A} \to X \to \Spec{k}$ giving a map $k \to A$ making $A$ a finitely generated $k$-algebra. Now $p$ is closed iff $\p$ is maximal i.e. $A / \p$ is a field so $S_\p$ is invertible in $A / \p$ making $\kappa(p) \cong A / \p$. Therefore $k(p)$ is a finitely generated $k$-algebra and a field extension of $k$. Therefore, by Hilbert's Nullstellensatz, $\kappa(p)$ is a finite algebraic extension of $k$. In particular, since we assume that $k$ is algebraically closed, $k = k(p)$. Conversely if $\kappa(p) = k$, since $k$ is a field mapping nontrivially into a ring and $A / \p$ is a domain, we have injections,
\begin{center}
\begin{tikzcd}
k \arrow[r, hook] & A / \p \arrow[r, hook] & S_\p^{-1} (A / \p) = k(p)
\end{tikzcd}
\end{center}
Thus, if $k(p) = k$ as $k$-algebras then the tower of inclusions collapses to $k = A / \p = k(p)$ since the unique $k$-algebra map $k \to k(p)$ is surjective. Therefore, $A / \p = \kappa(p)$ is a field and thus $\p$ is maximal and thus $p$ is closed in ever $U$ and thus closed in $X$.   
\end{proof}


\begin{proposition}
Let $f : X \to Y$ be a morphism of schemes over $k$. If $\kappa(x) = k$ then $\kappa(f(x)) = k$.
\end{proposition}

\begin{proof}
Consider the morphisms of schemes,
\begin{center}
\begin{tikzcd}
X \arrow[rr, "f"] \arrow[rd] & & Y \arrow[dl]
\\
& \Spec{k}
\end{tikzcd}
\end{center}
which, on stalks, induces morphisms of rings, 
\begin{center}
\begin{tikzcd}
\stalk{Y}{f(x)} \arrow[rr, "f^\#"] & & \stalk{X}{x} 
\\
& k \arrow[lu] \arrow[ru]  
\end{tikzcd}
\end{center}
making $f^\# : \stalk{Y}{f(x)} \to \stalk{X}{x}$ a morphism of $k$-algebras. Furthermore, this map is local and thus factoring through a morphism $f^\# : \stalk{Y}{f(x)} / \m_{f(x)} \to \stalk{X}{x} / \m_x$ of $k$-algebras. Therefore, we have an inclusion of fields,
\begin{center}
\begin{tikzcd}
k \arrow[r, hook] & \kappa(f(x)) \arrow[r, hook] & \kappa(x)
\end{tikzcd}
\end{center} 
which implies that if $\kappa(x) = k$ as $k$-algebras then the unique map $k \to \kappa(x)$ must be surjective which factors injectivly though $k \to \kappa(f(x)) \to \kappa(x)$ implying that the tower collapses to give $k = \kappa(f(x)) = \kappa(x)$ as $k$-algebras. 
\end{proof}

\begin{corollary}
Let $f : X \to Y$ be a morphism of varieties over an algebraically closed field. Then the image of a closed point is a closed point. 
\end{corollary}


\subsection{Curves} 

Need to consider nonsingular curves. Take $k$ algebraically closed and $X = \spec{A}$ is an affine curve.


\subsection{Smoothness}

\begin{definition}
Let $X$ be an affine variety then,
\[ X = \Spec{A} \quad \quad A = k[x_1, \dots, x_n] / (f_1, \dot, f_n) \]
such that $(f_1, \dots, f_n)$ is prime. Let $d = \dim{X}$. The singular locus of $X$ is the closed subset of $X$ cut out by the $(n - d) \times (n - d)$ minors of the Jacobian $\frac{\partial f_i}{\partial x_j}$.   
\end{definition}


\begin{example}
Let $X = V(y^2 - x^3)$ then the jacobian is $(2 y, 3x^2)$. Thus, $\Sing{X} = V(y^2 - x^3, 2y, 3 x^2) = \{ (x,y) \} $.
\end{example}

\begin{proposition}
Let $X$ be a variety over an algebraically closed field $k$. If $x \in X$ is in the smooth locus $X \setminus \Sing{X}$ iff $\stalk{X}{x}$ is a regular local ring. 
\end{proposition}

\begin{theorem}
Let $X$ be a curve over an algebraically closed field $k$. If $x \in X$ is a closed poin then TFAE,
\begin{enumerate}
\item $x$ is a smooth point
\item $\stalk{X}{x}$ is regular
\item $\stalk{X}{x}$ is regular of dimension $1$
\item $\m_x$ is principal
\item $\dim_{k(X)} (\m_x / \m_x^2) = 1$
\item $\stalk{X}{x}$ is a DVR
\item $\stalk{X}{x}$ is a normal domain
\item $\stalk{X}{x}$ has finite global dimension
\item $k(X)$ has finite projective dimension. 
\end{enumerate}
\end{theorem}



\begin{definition}
Let $X$ be a curve. $\Div{X}$ is the free abelian group on closed points of $X$.
\end{definition}

\begin{proposition}
If $X$ is a smooth curve over an algebraically closed $k$. Then define $k(X)^\times \to \Div{X}$ given by,
\[ f \mapsto \div{f} = \sum_{x \in X} v_x(f) \cdot [x] \]
where $v_x$ is the discrete valuation on $k(X) = \Frac{\stalk{X}{x}}$ using the fact that $\stalk{X}{x}$ is a DVR. Then, $\Cl{X}$ is the cokernel of this map.  
\end{proposition}

\section{Feb. 19}

\begin{lemma}
Any finite dimensional $k$-algebra domain is a field.
\end{lemma}

\begin{proof}
We can embed $A \subset \Frac{A}$ which is a finite field extension of $k$. The map $k \to A$ makes $A$ a finite dimensional $k$-vectorspace. For any nonzero $a \in A$ the $k$-linear map $T_a : A \to A$ given by $x \mapsto ax$ is injective since $A$ is a domain. Therefore, by rank-nullty, $\dim_{K}{A} = \dim{\ker{T_a}} + \dim{\Im{T_a}}$ but $\ker{T_a} = 0$ and thus $T_a$ is surjective. Therefore there exists $x \in A$ such that $ax = 1$. Therefore $A$ is a field. 
\end{proof}

\begin{lemma}
Let $k$ be a field and $\varphi : A \to B$ a morphism of $k$-algebras of finite type then $\m \subset B$ maximal implies that $\varphi^{-1}(\m) \subset A$ is maximal.
\end{lemma}

\begin{proof}
$B / \m$ is a finitely-generated $k$-algebra and a field extension of $k$. By Hilbert's Nullstellensatz $B / \m$ is a finite extension of $k$ so the inclusion $A / \varphi^{-1}(\m) \subset B / \m$ implies that the domain $A / \varphi^{-1}(\m)$ is a finite-dimensional $k$-algebra. Therefore $A / \varphi^{-1}(\m)$ is a field so $\varphi^{-1}(\m)$ is maximal.  
\end{proof}

\begin{lemma}
Let $k$ be a field and $\varphi : A \to B$ a morphism of $k$-algebras of finite type then the induced map $\varphi^* : \Spec{B} \to \Spec{A}$ is surjective iff the induced map (see previous lemma) $\varphi^* : \mSpec{B} \to \mSpec{A}$ is surjective. 
\end{lemma}

\begin{proof}
Let $\m \in \Spec{A}$ be closed and $\m \in \varphi^*(\p)$. Then $\p \subset \m'$ for some maximal ideal $\m' \subset A$. Then $\varphi^*(\m') \supset \varphi^*(\p) = \m$ so $\varphi^*(\m') = \m$ so $\m$ is in the image of a maximal ideal. 
\bigskip\\
Conversely, suppose that $\varphi^* : \mSpec{B} \to \mSpec{A}$ is surjective. Take any point $\p \in \Spec{A}$ then $Z = \overline{\{ \p \}}$ is an irreducible closed subset of $\mSpec{A}$. 
\end{proof}

\begin{proposition}
Let $X$ be a scheme locally of finite type over a field $k$ then $x \in X$ is closed if and only if $yy\kappa(x)$ is a finite extension of $k$. 
\end{proposition}

\begin{proof}
Let $x \in U \subset X$ be an affine open with $U \cong \Spec{A}$. Then $\stalk{X}{x} = \stalk{U}{x} = A_{\p}$ where $\p \in \Spec{A}$ is the prime corresponding to $x$. Now the residue field at $x$ is, 
\[ \stalk{X}{x} / \m_x = A_{\p} / \p A_\p = S_\p^{-1} ( A / \p ) \]
by the exactness of localizations. Furthermore, since $X$ is a scheme of finite type over $k$ there is a map $\Spec{A} \to X \to \Spec{k}$ giving a map $k \to A$ making $A$ a finitely generated $k$-algebra. 
\bigskip\\
If $x \in X$ is closed then $\p$ is closed in $U$ and thus maximal i.e. $A / \p$ is a field. In this case, $S_\p$ is invertible in $A / \p$ making $\kappa(p) \cong A / \p$. Therefore $\kappa(p)$ is a finitely generated $k$-algebra and a field extension of $k$. Therefore, by Hilbert's Nullstellensatz, $\kappa(p)$ is a finite algebraic extension of $k$.
\bigskip\\
Conversely if $\kappa(x)$ is a finite extension of $k$ then on each affine open $x \in U$ the corresponding prime $\p$ gives a domain $A / \p$ and thus inclusions
\begin{center}
\begin{tikzcd}
k \arrow[r, hook] & A / \p \arrow[r, hook] & S_\p^{-1} (A / \p) = \kappa(p)
\end{tikzcd}
\end{center}
showing that $A / \p$ is a finite-dimensional $k$-vectorspace domain and thus, by the lemma, a field. Therefore $\p$ is maximal and thus closed in $U$. Therefore we have shown that $x$ is closed in every affine open neighborhood. Therefore there exists a closed $C \subset X$ such that $C \cap U = \{ x \}$ and thus \[ U^C \cup \{ x \} = (U \setminus \{ x \})^C = (C^C \cap U)^C = C \cup U^C \]
is closed. Now let $\{ U_\alpha \}$ be an affine cover of $X$. If $x \in U_\alpha$ then we have shown that $U_\alpha^C \cup \{ x \}$ is closed otherwise $x \in U_\alpha^C$ so $U_\alpha^C \cup \{ x \}$ is closed. Therefore, using the fact that $U_\alpha$ cover $X$, the set
\[ \bigcap_{\alpha} U^C_\alpha \cup \{ x \} = \left( \bigcap_\alpha U_\alpha \right) \cup \{ x \} = \varnothing \cup \{ x \} = \{ x \} \]   
is closed. 
\end{proof}


\begin{corollary}
If $X$ is a scheme of finite type over an algebraically closed field $k$ then $x \in X$ is closed $\iff \kappa(x) = k$. 
\end{corollary}

\begin{proposition}
If $X$ is a scheme of finite type over $k$ then any closed point in an affine open is a closed point of $X$. 
\end{proposition}

\begin{proof}
Suppose that there exists some affine open neighborhood $x \in U \subset X$ such that $x = \m$ is closed in $U$ i.e. $x$ corresponds to some maximal prime $\m$. Then $\stalk{X}{x} = \stalk{U}{\m} = \kappa(x)$ is a finite extension of $k$ and therefore $x$ is closed.  
\end{proof}


\begin{proposition}
Let $f : X \to Y$ be a morphism of schemes. Then $\kappa(f(x)) \embed \kappa(x)$.
\end{proposition}

\begin{proof}
The ring map $f^\# : \stalk{Y}{f(x)} \to \stalk{X}{x}$ is local and thus factoring through a morphism $f^\# : \stalk{Y}{f(x)} / \m_{f(x)} \to \stalk{X}{x} / \m_x$. Therefore, we have a ring map of fields which is automatically an embedding,
\begin{center}
\begin{tikzcd}
\kappa(f(x)) \arrow[r, hook] & \kappa(x)
\end{tikzcd}
\end{center}  
\end{proof}

\begin{proposition}
Let $f : X \to Y$ be a morphism of schemes over $k$. If $\kappa(x)$ is a finite extension of $k$ then $\kappa(f(x))$ is a finite extension of $k$. 
\end{proposition}

\begin{proof}
Consider the morphisms of schemes,
\begin{center}
\begin{tikzcd}
X \arrow[rr, "f"] \arrow[rd] & & Y \arrow[dl]
\\
& \Spec{k}
\end{tikzcd}
\end{center}
which, on stalks, induces morphisms of rings, 
\begin{center}
\begin{tikzcd}
\stalk{Y}{f(x)} \arrow[rr, "f^\#"] & & \stalk{X}{x} 
\\
& k \arrow[lu] \arrow[ru]  
\end{tikzcd}
\end{center}
making $f^\# : \stalk{Y}{f(x)} \to \stalk{X}{x}$ a morphism of $k$-algebras. Furthermore, this map is local and thus factoring through a morphism $f^\# : \stalk{Y}{f(x)} / \m_{f(x)} \to \stalk{X}{x} / \m_x$ of $k$-algebras. Therefore, we have an inclusion of fields,
\begin{center}
\begin{tikzcd}
k \arrow[r, hook] & \kappa(f(x)) \arrow[r, hook] & \kappa(x)
\end{tikzcd}
\end{center} 
which implies that $\kappa(f(x))$ includes into $\kappa(x)$ as a morphism of $k$-algebras so if $\kappa(x)$ is a finite extension of $k$ then $\kappa(f(x))$ is a finite extension of $k$. 
\end{proof}

\begin{corollary}
Let $f : X \to Y$ be a morphism of locally finite type schemes over $k$. Then the image of a closed point under $f$ is a closed point.
\end{corollary}

\begin{proof}
A point $x \in X$ is closed iff $\kappa(x)$ is a finite extension of $k$ and likewise for $y \in Y$. 
Then we get an inclusion of fields,
\begin{center}
\begin{tikzcd}
k \arrow[r, hook] & \kappa(f(x)) \arrow[r, hook] & \kappa(x)
\end{tikzcd}
\end{center}  
However, $k$ is obviously Noetherian so if $x \in X$ is closed then $\kappa(x)$ is a finite $k$-module so $\kappa(f(x)) \embed \kappa(x)$ is a finite $k$-module and thus is finite. Therefore, $f(x)$ is closed. 
\end{proof}

\begin{remark}
This is not true for arbitrary schemes even for nice (flat, finite type) maps. For example, $\Spec{\Q} \to \Spec{\Z_{(p)}}$ takes the closed point $(0) \in \Spec{\Q}$ to the generic point $(0) \in \Spec{\Z_{(p)}}$ which is not closed. 
\end{remark}

\begin{corollary}
Let $f : X \to Y$ be a morphism of locally finite type schemes over $k$. Then, $\kappa(x) = k \implies \kappa(f(x)) = k$.
\end{corollary}

\begin{proof}
If $\kappa(x) = k$ as $k$-algebras then the unique map $k \to \kappa(x)$ must be surjective which factors injectivly though,
\begin{center}
\begin{tikzcd}
k \arrow[r, hook] & \kappa(f(x)) \arrow[r, hook] & \kappa(x)
\end{tikzcd}
\end{center}  
implying that the tower collapses to give $k = \kappa(f(x)) = \kappa(x)$ as $k$-algebras.
\end{proof}


\begin{proposition}
If $X$ is a scheme of finite type over $k$ then the map $\delta : X \to \Z$ given by $x \mapsto \trdeg{k}{\kappa(x)}$ satisfies,
\begin{enumerate}
\item $\dim{\overline{\{ x \}}} = \delta(x)$
\item $x$ is closed $\iff \delta(x) = 0$
\item if $x \leadsto y$ and $x \neq y$ then $\delta(x) > \delta(y)$
\item if $x \leadsto y$ and $x \neq y$ but there is no $x \leadsto z \leadsto y$ with $z \neq x$ and $z \neq y$ the $\delta(x) = \delta(y) + 1$. 
\end{enumerate}
\end{proposition}

\begin{example}
Take $A \to B$ with $B = A[x_1, \dots, x_n] / (f_1, \dots, f_c)$. Define the element,
\begin{align*}
g = \det{\left( \frac{\partial f_i}{\partial x_j} \bigg|_{1 \le i,j \le c} \right)}
\end{align*}
If $g$ maps to an invertible element of $B$, then $A \to B$ is a smooth ring map. 
\end{example}

\section{Misc}

\begin{definition}
A topological space is $T_0$ if for each pair of distinct points there is a neighborhood of one that does not contain. 
\end{definition}

\begin{proposition}
All schemes are $T_0$. 
\end{proposition}

\begin{proof}
Let $X$ be a scheme and $x, y \in X$ distinct points. If $x$ and $y$ lie in different affine opens then this is an open seperation. If $x, y$ lie in the same affine open $U = \Spec{A}$ then they correspond to distinct prime ideals $\p, \q \subset A$. Since $\p \neq \q$ there exists some element of one that is not in the other. Without loss of generality suppose that there is some $f \in \p$ with $f \notin \q$. Thus, $\q \in D(f)$ and $\p \notin D(f)$ so $x$ and $y$ are seperated by some open $D(f) \subset U \subset X$.  
\end{proof}

\begin{definition}
A \textit{generic point} $\xi \in Z$ of a closed irreducible set $Z$ is such that $\overline{\{ \xi \} } = Z$. 
\end{definition}

\begin{proposition}
Let $X$ be a toplogical space and $\xi \in X$ then $\overline{\{ \xi \} }$ is a closed irreducible set with generic point $\xi$. 
\end{proposition}

\begin{proof}
Clearly, $\{ \xi \}$ is closed. Suppose that $\overline{\{ \xi \}} \subset Z_1 \cup Z_2$ then $\xi \subset Z_1$ or $\xi \subset Z_2$ and thus $\overline{\{ \xi \} } \subset Z_1$ or $\overline{\{ \xi \} } \subset Z_2$ so $\overline{\{ \xi \}}$ is irreducible. Clearly, $\xi$ is a generic point of $\overline{\{ \xi \}}$. 
\end{proof}

\begin{definition}
A topological space is \textit{sober} if every irreducible closed set has a unique generic point. 
\end{definition}

\begin{proposition}
Any Hausdorff space is sober.
\end{proposition}

\begin{proof}
Let $Z$ be irreducible and closed. Suppose that $Z$ has more than one point. Take distinct $x, y \in Z$ and, using the Hausdorff property, open sets $x \in U$ and $y \in V$ such that $U \cap V = \varnothing$. Now consider $Z_1 = Z \cap U^C$ and $Z_2 = Z \cap V^C$ which are closed in $Z$ proper because $x \notin Z_1$ and $y \notin Z_2$. Furthermore, $Z_1 \cup Z_2 = Z \cap (U^C \cup V^C) = Z \cap (U \cap V)^C = Z$ so $Z$ cannot be irreducble. Thus, the only irreducible sets are points which clearly have a unique generic point because all points in a $T_2$ space are closed.
\end{proof}

\begin{lemma}
Any prime $\p \in \Spec{A}$ in an affine scheme satisfies $\overline{\{\p\}} = V(\p)$. 
\end{lemma}

\begin{proof}
Any closed set in $\Spec{A}$ is of the form $V(I)$ for some ideal $I \subset A$. Consider the closed sets $\p \in V(I)$ containing $\p$ which correspond to $\p \supset I$. Clearly, $\p \in V(\p)$ and if $\p \in V(I)$ then $V(\p) \subset V(I)$ since $\p \supset I$. Therefore $V(\p)$ is the closure of $\p$.
\end{proof}

\begin{lemma}
Every closed irreducible set of an affine scheme $\Spec{A}$ is of the form $V(\p)$ for some prime $\p \subset A$. 
\end{lemma}

\begin{proof}
First, all closed subsets of $\Spec{A}$ are of the form $V(I)$. First, if $I = \p$ is prime and $V(\p) \subset V(I_1) \cup V(I_2) = V(I_1 I_2)$ then $\p \supset I_1 I_2$. However, since $\p$ is prime we have either $\p \supset I_1$ or $\p \subset I_2$ so $V(\p) \subset V(I_1)$ or $V(\p) \subset V(I_2)$ proving that $V(\p)$ is irreducible. Conversely, if $V(I)$ is irreducible then take $x, y \in A$ such that $xy \in \sqrt{I}$ and thus,
\[ \sqrt{(xy)} \subset \sqrt{I} \implies V(I) \subset V((xy)) = V((x)) \cup V((y)) \]
Since $V(I)$ is irreducible we must have either $V(I) \subset V((x))$ or $V((y)) \subset V(I)$ which implies that $\sqrt{(x)} \subset \sqrt{I}$ or $\sqrt{(y)} \subset \sqrt{I}$. Therefore, $x \in \sqrt{I}$ or $y \in \sqrt{I}$ so $\sqrt{I}$ is prime and $V(I) = V(\sqrt{I})$.  
\end{proof}

\begin{proposition}
Any scheme is sober. 
\end{proposition}

\begin{proof}
First consider the affine case $X = \Spec{A}$. Any irreducible closed set in $X$ is of the form $V(\p)$ for some prime $\p \subset A$. Thus $\overline{\{ \p \}} = V(\p)$ is the unique generic point. Now let $X$ be any scheme and $Z \subset X$ a closed irreducible subset. $X$ has a cover by affine opens so take some affine open $U$ which intersects $Z$. Since $U$ is an affine scheme and $U \cap Z$ is a closed irreducible subet of $U$ there exists a unique generic point $\xi \in U \cap Z$. Because $Z$ is closed in $X$ we then have $Z \cap U \subset \overline{\{\xi\}} \subset Z$. However, $Z \cap U$ is open in $Z$ and $\overline{\{\xi\}}$ is closed in $Z$, an irreduvible, which implies that either $U \cap Z$ is empty (which is false by assumption) or $\overline{\{\xi\}} = Z$. Thus $Z$ has a generic point $\xi$. Suppose that $\xi, \xi' \in Z$ were both generic points then both must be limit points of each other and thus have exactly the same open neighborhoods contradicting the fact that $Z \subset X$ is $T_0$. 
\end{proof}

\section{Feb. 21}

\begin{theorem}
Let $k$ be an algebraically closed field and $X$ a nonsingular curve over $k$. Then there is a canonical map $c_1 : \Pic{X} \to \Cl{X}$ called the first Chern class from the Picard group to the Weil divisor class group which is an isomorphism of abelian groups. 
\end{theorem}

\begin{definition}
Let $X$ be a variety and $L \in \Pic{X}$ then a \textit{meromorphic section} or \textit{rational section} of $L$ is an element $s \in L_\eta$ which is the stalk at the generic point of $X$. Note that $L_\eta$ is a $1$-dimensional $k(X) = \stalk{X}{\eta}$ vectorspace. 
\end{definition}



\begin{definition}
If $s, s'$ are two nonzero rational sections of $L$ then $s' = s f$ for some unique $f \in k(X)^\times$. Therefore define,
\[ \div_L(s) = \sum_{p \in X} \ord_{L, p}(s) [p] \]
Now $\stalk{X}{p} \subset \Frac{\stalk{X}{p}} = k(X) = \stalk{X}{\eta}$. In general if $x \leadsto x'$ then there is a map $\stalk{X}{x} \to \stalk{X}{x'}$. Therefore, $L_p$ is a (noncaononically) free rank $1$ $\stalk{X}{p}$-module. Take $L_p = \stalk{X}{p} \cdot e_p$ where $e_p \in L_p$ is a basis element. Now, 
\[ L_\eta = L_p \otimes_{\stalk{X}{p}} \stalk{X}{\eta} = L_p \otimes_{\stalk{X}{p}} k(X) \]
Therefore we can write $s = f_p e_p$ for $f_p \in k(X)^\times$ unique. Then define $\ord_{L,p}(s) = \ord_p(f_p)$. 
\end{definition}

\begin{lemma}
IF $s'$ is a second nonzero rational section of $L$ then,
\[ \div_L(s') = \div_X(f') + \div_L(s) \]
where $f \in k(X)^\times$ is unique such that $s' = f s$. 
\end{lemma}

\begin{proof}
Let $L$ be a invertible $\struct{X}$ module. Then set $c_1(L)$ to be te divisor class of $\div_L(s)$ where $s$ is a nonzero meromorphic section of $L$. Consider $L \otimes L'$ and $s \in L_\eta$ and $s' \in L'_{\eta}$. Then $s \otimes s' \in L \otimes L'$ is a meromorphic section. Computing its divisor gives $\div{L}(s) + \div{L'}(s')$. 
\end{proof}



\begin{example}
Let $X = \P^1_k$. Let $o \in X$ be $[0 : 1]$ let $I$ be the ideal sheaf of $o \subset \struct{X}(X)$. 
\end{example}

\begin{lemma}
If $I \subset \struct{X}$ is the ideal sheaf of a closed point $p \in X$ then $c_1(I) = - [p]$. Therefore, $c_1$ is surjective. 
\end{lemma}

\begin{lemma}

\end{lemma}

\begin{proof}
Suppose that $c_1(L) = 0$ in $\Cl{X}$. We know that $\div_L(s) = \div(f)$ for some $f \in k(X)^\times$ for any meromorphic section $s \in L_\eta$. Then $\div_L(f^{-1} s) = 0$ which implies that $f^{-1} s$ is nonvanishing at closed points. Thus $f^{-1} s = u_p e_p$ where $e_p \in L_p$ and $u_p \in \stalk{X}{p}^\times$ at each closed point $p$. Therefore, $f^{-1} s$ is a section of $L$ on some open neighborhood of each closed point of $p$. Therfore by gluing, $f^{-1} s \in \Gamma(X, L)$ and nonvanishing so $L \cong \struct{X}$. 
\end{proof}

\begin{example}
For $k = \bar{k}$ we have $\Pic{\P^1_k} \cong \Z$. We can compute $\Cl{\P_k^1}$. We have $k(\P^1_k) = k(x)$ thus any rational function can be written as,
\[ f = u \frac{(x - \alpha_1) \cdots (x - \alpha_r)}{(t - \beta_1) \cdots (t - \beta_s)} \]
therefore,
\[ \div{f} = \sum_{i = 1}^r [\alpha_i] - \sum_{i = 1}^s [\beta_s] + (s - r) [\infty] \]
Define the map,
\[ \deg : \Div{X} \to \Z \quad \quad D = \sum_{i = 1}^r n_i [p_i] \mapsto \sum_{i = 1}^r n_i \]
Therefore, for any $f \in k(X)$ we know $\deg{\div{f}} = 0$ so this map descends to $\deg : \Cl{X} \to \Z$. This map is clearly surjective. Furthermore, since,
\[ [\alpha] - [\beta] = \div{\left( \frac{x - \alpha}{x - \beta} \right)} \quad \quad [\alpha] - [\inf] = \div{(x - \alpha)} \]
this map is also injective. 
\end{example}

\begin{definition}
A variety $X$ over $k$ is \textit{projective} if there exists some $n$ and a closed immersion $X \to \P^n_k$.
\end{definition}

\begin{proposition}
Let $X$ be a smooth projective curve. The the degree of the divisor of a nonzero rational function is zero. 
\end{proposition}

\section{Feb. 26}

\begin{definition}
An $\struct{X}$-module is finite type if it is locally finitely generated. For coherent modules this is equivalent to being $\widetilde{M}$ for some finite type module on each affine open.
\end{definition}

\begin{proposition}
Let $X$ be a reduced scheme and $\mathcal{F}$ a finite type quasi-coherent $\struct{X}$-module. If there exists $r$ such that $\forall x \in X : \dim_{k(x)} \F_x / \m_x \F_x = r$ then $\F$ is finite locally free of rank $r$.  
\end{proposition}

\begin{proof}
Let $x \in X$ and $s_1, \dots, s_r \in \F_x$ which map to a basis of,
\[ \F_x / \m_x \F_x = \F_x \otimes_{\stalk{X}{x}} k(x) \]
the fiber of $\F$ at $x$. There exists a ``small'' open $U \subset X$ s.t. $s_1, \dots, x_r \in \F(U)$. By a Nakayama type argument, the fact that $\F$ has finite type and that $s_1, \dots, s_n$ generate $\F_x / \m_x \F_x$ implies that $s_1, \dots, s_n$ generated $\F |_U$ after shrinking $U$. Therefore we have a surjective morphism of sheaves,
\begin{center}
\begin{tikzcd}
\struct{U}^{\otimes r} \arrow[rr, two heads] & & \F |_U 
\end{tikzcd}
\end{center} 
To show that this is injective, say $(f_1, \dots, f_r) \mapsto 0$ then, since $U$ is reduced, if $f_i \neq 0$ fome $i$ then the image of $f_i$ in the residue field at some point $x' \in U$ is nonzero. Then the map,
\begin{center}
\begin{tikzcd}
k(x')^{\otimes r} \arrow[rr, two heads] & & \F_{x'} / \m_{x'} \F_{x'} 
\end{tikzcd}
\end{center}
has some kernel. This contradicts the fact that dimension of the fiber is required to be constant. 
\end{proof}

\begin{lemma}
A ring $A$ is reduced iff,
\[ A \subset \prod_{\p} k(\p) \]
\end{lemma}


\section{Cohomology}

Let $\mathcal{A}$ and $\mathcal{B}$ be abelian categories and $\F : \mathcal{A} \to \mathcal{B}$ be an additive functor. 

\begin{theorem}
$\Ab(X)$ and $\Mod{\struct{X}}$ have enough injectives. 
\end{theorem}

\begin{lemma}
If $\mathcal{A}$ has enough injectives then for each short exact sequence,
\begin{center}
\begin{tikzcd}
0 \arrow[r] & M_1 \arrow[r] & M_2 \arrow[r] & M_3 \arrow[r] & 0
\end{tikzcd}
\end{center}
There is a short exact sequence of injective resolutions such that,
\begin{center}
\begin{tikzcd}
& 0 \arrow[d] & 0 \arrow[d] & 0 \arrow[d]
\\
0 \arrow[r] & M_1 \arrow[d] \arrow[r] & M_2 \arrow[r] \arrow[d] & M_3 \arrow[r] \arrow[d] & 0
\\
0 \arrow[r] & \bf{I}_1 \arrow[r] & \bf{I}_2 \arrow[r] & \bf{I}_3 \arrow[r] & 0
\end{tikzcd}
\end{center}
commutes. Furthermore, each row of injectives is split because there are exact sequences of injectives. 
\end{lemma}

\begin{corollary}
Applying the additive functor $F$ to the above short exact sequence of injective resolution, we get a short exact sequence,
\begin{center}
\begin{tikzcd}
0 \arrow[r] & F(\bf{I}_1) \arrow[r] & F(\bf{I}_2) \arrow[r] & F(\bf{I}_3) \arrow[r] & 0
\end{tikzcd}
\end{center} 
of chain complexes which remains exact after applying $F$ because additive functors preseve split exactness. 
\end{corollary}

\begin{definition}
If $X$ us a topological space, an abelian sheaf $\F$ is called \textit{flasque} or \textit{flabby} iff $\forall U \subset V \subset X$ open then  $\mathcal{F}(V) \to \mathcal{F}(U)$ is a surjection. 
\end{definition}

\begin{lemma}
If $\F$ is flasque then $\F$ is $\Gamma(X, - )$-acylic. 
\end{lemma}

\subsection{Godement Resolution}

Suppose $A_x$ is an abelian group for each $x \in X$ then consider
\[ U \mapsto \prod_{x \in U} A_x \]
This is a flasque sheaf equivalent to,
\[ \prod_{x \in X} (\iota_x)_*(A_x) \]
where $\iota_x : * \to X$ is the inclusion at $x$. The Godement resolution is accquired by mapping $\F$ into this construction where $A_x = \F_x$ i.e. we have,\[ \F(U) \to \prod_{x \in U} \F_x \]

\begin{theorem}[Existence of Injectives]
For each $x \in X$ choose $\F_x \hookrightarrow I_x$ into some injective abelian group. Then $(\iota_x)_*(I_x)$ is an injective object of $\Ab(X)$. Finally, products of injectives are injectives so,
\begin{center}
\begin{tikzcd}
\F \arrow[r] & \prod\limits_{x \in X} (\iota_x)_*(I_x) 
\end{tikzcd}
\end{center}
is an injective map into an injective. 
\end{theorem}

\begin{proposition}
If $f : X \to Y$ is a continuous map of topological space and $\mathcal{I}$ is an injective sheaf on $X$ then $f_* \mathcal{I}$ is an injective sheaf on $Y$.
\end{proposition}

\begin{proof}
Consider the diagram,
\begin{center}
\begin{tikzcd}
\F \arrow[d, hook] \arrow[r] & f_* \mathcal{I} 
\\
\G \arrow[ru, dashed]
\end{tikzcd}
\end{center}
the dashed arrow exists by adjuction to the diagram,
\begin{center}
\begin{tikzcd}
f^{-1} \F \arrow[d, hook] \arrow[r] & \mathcal{I} 
\\
f^{-1} \G \arrow[ru, dashed]
\end{tikzcd}
\end{center}
and the dahsed arrow exists because $f^{-1}$ is exact and $\mathcal{I}$ is injective. 
\end{proof}


\begin{theorem}
If $\I$ is an injective of $\Ab(X)$ then for any open $U \subset X$ then the restriction $\I |_U$ is also.
\end{theorem}

\begin{proof}
Use the fact that $(-)|_U$ is a right-adjoint to $j_{!}$ which is exact. 
\end{proof}

\begin{corollary}
If we have an injective resolution,
\begin{center}
\begin{tikzcd}
0 \arrow[r] & \F \arrow[r] & \bf{\I}
\end{tikzcd}
\end{center}
then
\begin{center}
\begin{tikzcd}
0 \arrow[r] & \F |_U \arrow[r] & \bf{\I} |_U
\end{tikzcd}
\end{center}
is also an injective resolution. This gives a map,
\begin{center}
\begin{tikzcd}
H^q(X, \F) = H^q(\Gamma(X, \mathbf{I})) \arrow[r, "\res_U"] & H^q(\Gamma(U, \mathbf{I}|_U)) = H^q(U, \F |_U) 
\end{tikzcd}
\end{center}
via the maps $\res_{X,U} : \I(X) \to \I(U)$.
\end{corollary}

\begin{theorem}[Locality of Cohomology] 
Given $\xi \in H^q(X, \F)$ then $q > 0$ then there exists an open cover $\mathcal{U}$ such that $\xi |_{U} = 0$ for each $U \in \mathcal{U}$. 
\end{theorem}

\begin{proof}
The cohomology class $\xi$ is represented by some $\sigma \in \Gamma(X, \I^q)$ which is a chochain i.e. $\d{\sigma} = 0$ in the complex,
\begin{center}
\begin{tikzcd}[row sep = large, column sep = large]
\I^{q-1}(X) \arrow[r, "d"] \arrow[d, "\res"] & \I^q(X) \arrow[r, "d"] \arrow[d, "\res"] & \I^{q + 1}(X) \arrow[d, "\res"]
\\
\I^{q-1}(U) \arrow[r, "d"] & \I^q(U) \arrow[r, "d"] & \I^{q+1}(U)
\end{tikzcd}
\end{center}
Since $\F \to \bf{\I}$ is a resolution it is an exact sequence of sheaves and therefore since $\sigma \in \ker{d^q}$ then $\sigma$ in in the image of $d^{d-1}$ as sheaves so locally it is in the image as abelian groups.
\end{proof}

\begin{definition}
Let $\mathcal{U}$ be an open cover $\{U_i\}_{i \in I}$ with $I$ totally corded of $X$ then define the Cech complex $ \check{\mathcal{C}}(\mathcal{U}, \F)$
given by,
\begin{center}
\begin{tikzcd}
\prod\limits_{i_0 \in I} \F(U_{i_0}) \arrow[r] & \prod\limits_{i_0 < i_1} \F(U_{i_0} \cap U_{i_1}) \arrow[r] & \prod\limits_{i_0 < i_1 < i_2} \F(U_{i_0} \cap U_{i_1} \cap U_{i_2}) \arrow[r] & \cdots 
\end{tikzcd}
\end{center}
with the Cech boundary map given by,
\[ d(s)_{i_0, \dots, i_p} = \sum_{j = 0}^{p + 1} (-1)^j \]

\end{definition}

\begin{theorem}
\[ H^1(X, \F) = \check{H}^1(X, \F) \]
\end{theorem}


\section{Feb. 5}

\begin{definition}
Let $X$ be a Noetherian scheme then a $\struct{X}$-module is \textit{coherent} if it is quasi-coherent and locally of finite type. 
\end{definition}

\begin{theorem}
If $X$ is a proper variety over a field $k$ and $\F$ is a coherent $\struct{X}$-module then $H^r(X, \F)$ is a finite dimensional $k$ vectorspace for each $r$. 
\end{theorem}

\begin{lemma}
Any proper birational morphism is surjective.
\end{lemma}

\begin{proof}

\end{proof}

\begin{lemma}[Chow] 
If $X$ is a proper variety over $k$ then there exists a projective variety $X'$ over $k$ and a proper birational morphism $X' \to X$.
\end{lemma}

\begin{remark}
First we prove finiteness for ``projective'' varieties then use Chow's lemma to deduce the theorem for proper varieties. 
\end{remark}

\begin{lemma}
If $f : X \to Y$ is an isomorphism then $f^*$ and $f_*$ are inverse functors between $\Mod{\struct{X}}$ and $\Mod{\struct{Y}}$. 
\end{lemma}

\begin{proof}

\end{proof}

\begin{lemma}
The support of a coherent $\struct{X}$-module is closed. 
\end{lemma}

\begin{proof}
Let $\F$ be a coherent $\struct{X}$-module. There exists an open affine cover $U_i = \Spec{A_i}$ on which $\F \cong \tilde{M_i}$ for some finitely generated $A_i$-module $M_i$. Then,
\[ \Supp{A_i}{M_i} = V(\Ann{A_i}{M_i}) \]
is a closed set. Therefore, 
\[ \Supp{\struct{X}}{\F} \cap U_i = \{ \p \subset A \mid \F_\p \cong M_\p \neq 0 \} = \Supp{A_i}{M_i} \]
is closed in $U_i$. Thus, $\Supp{\struct{X}}{\F}^C \cap U_i = V_i \cap U_i$ where $V_i$ is some open set. Then,
\begin{align*}
\Supp{\struct{X}}{\F}^C = \bigcup_{i \in \mathcal{I}} \Supp{\struct{X}}{\F}^C \cap U_i = \bigcup_{i = 1}^n V_i \cap U_i 
\end{align*}
is open in $X$ so $\Supp{\struct{X}}{\F}$ is closed.
\end{proof}

\begin{lemma}
Proper morphisms from projective varieties are projective. 
\end{lemma}

\begin{definition}
Let $f : X \to Y$ be a morphism of schemes. Then $R^q f_*$ are the right-derived functors of $f_* : \Mod{\struct{X}} \to \Mod{\struct{Y}}$. 
\end{definition}


\begin{proposition}
If $\F$ is an $\struct{X}$-module and $\F_{\ab}$ denotes the underlying sheaf of abelian groups, then,
\[ (R^q f_* (\F))_\ab = R^q f_*(\F_\ab) \]
for all $q \ge 0$ and $H^q(X, \F)_\ab = H^q(X, \F_\ab)$. 
\end{proposition}

\begin{theorem}
The sheaf $R^q f_* \F$ is the sheaf associated to the presheaf,
\[ V \mapsto H^q(f^{-1}(V), \F|_{f^{-1}(V)} \]
\end{theorem}

\begin{proof}

\end{proof}

\begin{lemma}
If $X$ is an affine scheme and $\F$ quasi-coherent on $X$ then, $H^q(X, \F) = 0$ for $q > 0$. If $f : X \to Y$ is an affine morphism, then $R^q f_* \F = 0$ for $q > 0$. 
\end{lemma}

\begin{proof}
USE ABOVE
\end{proof}


\begin{theorem}[Part I]
Let $f : X \to S$ be a locally projective morphism where $S$ is a Noetherian scheme then $R^q f_*$ takes coherent $\struct{X}$-modules to coherent $\struct{S}$-modules. 
\end{theorem}


\begin{definition}
A morphism $f : X \to Y$ is \textit{locally projective} if there exists an affine cover $U_i = \Spec{A_i}$ of $Y$ such that $f|_{f^{-1}(U_i)} : f^{-1}(U_i) \to U_i$ factors though a closed immersion into $\P_{A_i}^n$ for some $n$. 
\end{definition}

\begin{proof}
We may assume that $S = \Spec{A}$ is affine since the morphism is locally projective. In this case, we may assume that $X$ is a closed subscheme of $\P^n_S$ for some $n$. We have the diagram,
\begin{center}
\begin{tikzcd}
X \arrow[rd, "f"] \arrow[rr, "\iota"] & &  \P_A^n \arrow[dl, "p"]
\\
& S = \Spec{A} 
\end{tikzcd}
\end{center}
Relative Leray spectral sequence gives,
\[ E_2^{kl} = R^k p_* R^l \iota_* \F \implies R^{k + l} f_* \F \]
Since $\iota$ is a closed immersion, it is affine and thus, by the lemma, $R^l \iota_* \F$ is zero unless $l = 0$. Therefore,
\[ R^q f_*(\F) = R^q p_* (\iota_* \F) \]
which is coherent since $\iota$ is a closed immersion. Thus we have to show that if $\F$ is a coherent module on $\P_A^n$ then $R^q(\P_A^n \to \Spec{A})_* \F$ is coherent for all $q \ge 0$. Also, $H^q(\P^n_A, \F)$ is a finite $A$-module.
\begin{proposition}
\[ R^q(\P_A^n \to \Spec{A})_* \F = \widetilde{H^q(\P_A^n, \F)} \]
\end{proposition} 
Given the above, we simply need to prove that $H^q(\P^n_A, \F)$ is a finite $A$-module for any coherent $\F$ on $\P^n_A$. 
\end{proof}

\begin{example}
Let's try to compute $H^q(\P_A^1, \struct{\P_A^1}(n))$. We may write,
\[ \P_A^1 = \Spec{A[x]} \coprod_{\Spec{A[x, x^{-1}]}} \Spec{A[x^{-1}]} = U \coprod_{U \cap V} V \]
Then $\struct{\P_A^1}(n)$ has a local trivialization $e$ on $U$ and $f$ on $V$ i.e. $struct{\P_A^1}(n)|_U = \struct{\P_A^1} |_U \cdot e$ and on the overlap $e = x^{-n} f$. Now applying Mayer-Vietoris gives,
\begin{center}
\begin{tikzcd}
0 \arrow[r] & H^0(\P_A^1, \struct{\P_A^1}(n)) \arrow[r] A[x] \cdot e \oplus A[x^{-1}] \cdot f \arrow[r] & A[x, x^{-1}] \cdot \arrow[r] & H^1(\P_A^1, \struct{\P_A^1}(n)) \arrow[r] & 0 
\end{tikzcd}
\end{center}
because the higher cohomology of an affine open vanishes. 
The map sends $P(x) \oplus Q(x^{-1} \mapsto x^{-1} P(x) = Q(x^{-1})$.  
\end{example}

\begin{remark}
Locality of cohomology says that if $V \subset S$ is open then,
\[ (R^q f_* \F)|_V = R^q(f|_{f^{-1}(V)})_* (\F|_{f^{-1}(V)}) \]
\end{remark}

\begin{lemma}
If $\iota : X \to Y$ is a closed immersion then $\iota$ is affine and $\iota_*$ takes coherent $\struct{X}$-modules to coherent $\struct{Y}$-modules. 
\end{lemma}

\begin{proposition}
Let $\F$ be quasi-coherent then,
\begin{enumerate}
\item $H^q(\P_A^n, \F) = 0$ for $q > n$ 
\item If $\F$ is coherent on $\P_A^n$ then there exists a surjection,
\begin{center}
\begin{tikzcd}
\struct{\P_A^n}(m_1) \oplus \cdots \oplus \struct{\P_A^n}(m_r) \arrow[r] & \F
\end{tikzcd}
\end{center}
for some integers $m_1, \dots, m_r$. 
\end{enumerate}
\end{proposition}

\begin{theorem}
Let $\F$ be a cohernet sheaf on $\P_A^n$ then $H^q(\P_A^n, \F)$ is a finite $A$-module.
\end{theorem}

\begin{proof}
There exists an exact sequence of sheaves,
\begin{center}
\begin{tikzcd}
0 \arrow[r] & \G \arrow[r] & \struct{m_1} \oplus \cdots \oplus \struct{m_r} \arrow[r] & \F \arrow[r] & 0
\end{tikzcd}
\end{center}
The long exact sequence of cohomology gives,
\begin{center}
\begin{tikzcd}
H^q(\P_A^n, \struct{\P_A^n}(m_1) \oplus \cdots \oplus \struct{\P_A^n}(m_r)) \arrow[r] & H^q(\P_A^n, \F) \arrow[r] & H^{q+1}(\P_A^n, \G)
\end{tikzcd}
\end{center}
The first term is a finite $A$-module by direct computation and $H^{q+1}(\P_A^n, \G)$ is a finite $A$-module by descending induction on $q$. The base case holds because $H^q(\P_A^n, \F)$ automatically vanishes for $q > n$. 
\end{proof}

\section{March 7}


\begin{theorem}
Let $X$ be a locally ringed space then,
\[ \Pic{X} = H^1(X, \struct{X}^\times) = \varinjlim_{\mathfrak{U}} \check{H}(\mathfrak{U}, \struct{X}^\times) \]
\end{theorem}

\begin{proof}
Decompose,
\[ \Pic{X} = \bigcup_{\mathfrak{U}} \Pic{\mathfrak{U}, X} \quad \text{where} \quad \Pic{\mathfrak{U}, X} = \{ \mathcal{L} \in \Pic{X} \mid \forall U \in \mathfrak{U} : \mathcal{L}|_U \cong \struct{U} \} \]
\end{proof}

\begin{definition}
Let $X$ be a variety over $k$ and $\mathcal{L}$ an invertible sheaf on $X$. Then $\mathcal{L}$ is \textit{very ample} iff there is a locally closed immersion,
\begin{center}
\begin{tikzcd}
X \arrow[r, hook] & \P^n_k
\end{tikzcd}
\end{center}
such that $\mathcal{L} \cong \iota^* \struct{\P^n_k}(1)$. We say that \textit{ample} iff there exists some $n > 0$ such that $\mathcal{L}^{\otimes n}$ is very ample. 
\end{definition}

\begin{theorem}[Serre]
Let $X$ be a proper variety and $\mathcal{L}$ invertible on $X$ then TFAE,
\begin{enumerate}
\item $\L$ is ample
\item for any any coherent $\F$ on $X$ and any $q \ge 0$ there exists $n_0$ such that $\forall n \ge n_0$,  
\[ H^q(X, \F \otimes \L^{\otimes n}) = 0 \]
\end{enumerate}
\end{theorem}

\begin{theorem}
Let $X$ be a locally contractible then,
\[ H^q_{\text{sing}}(X ; \Z) \cong H^q(X, \underline{Z}) \]
\end{theorem}

\subsection{Cech Cohomology and Cohomology}

\begin{lemma}
Let $X$ be a topological space and $\F$ an abelian sheaf, If for all open covers $\mathfrak{U}$ of an open $U \subset X$ then we have $\check{H}(\mathfrak{U}, \F) = 0$ for $i > 0$, then,
\[ H^q(U, \F) = 0 \]
for any open $U \subset X$. 
\end{lemma}

\begin{lemma}
If $\mathcal{B}$ is a basis for $X$ s.t. $U, U' \in \mathcal{B}$ then $U \cap U' \in \mathcal{B}$ and any any 
\end{lemma}

\begin{lemma}
If $X$ is an affine scheme and $\mathfrak{U}$ is an affine open cover and $\F$ is quasi-coherent, then $\check{H}^q(\mathfrak{U}, \F) = 0$ for all $q > 0$.
\end{lemma}

\begin{proof}
In the case where $X = \Spec{A}$ and $\F = \widetilde{M}$ and $f_1, \dots, f_n \in A$ generate the unit ideal, i.e., $D(f_i)$ cover $X$. Then the Cech complex $C^\bullet$ for this cover is,
\begin{center}
\begin{tikzcd}
\prod\limits_{i_0} M_{f_{i_0}} \arrow[r] & \prod\limits_{i_0 < i_1} M_{f_{i_0} f_{i_1}} \arrow[r] & \cdots \arrow[r] & M_{f_1 \cdots f_n} 
\end{tikzcd}
\end{center} 
We need to show that $C^\bullet$ is exact in positive degree. It is a complex of $A$-modules so it suffices to show exactness of the localization at each prime $\p \subset A$. Thus we get,
\begin{center}
\begin{tikzcd}
\left( \prod\limits_{i_0} M_{f_{i_0}} \right)_\p \arrow[r] & \left( \prod\limits_{i_0 < i_1} M_{f_{i_0} f_{i_1}} \right)_\p \arrow[r] & \cdots \arrow[r] & \left( M_{f_1 \cdots f_n} \right)_\p
\end{tikzcd}
\end{center}
which gives,
\begin{center}
\begin{tikzcd}
\prod\limits_{i_0} (M_\p)_{f_{i_0}} \arrow[r] & \prod\limits_{i_0 < i_1} (M_\p)_{f_{i_0} f_{i_1}} \arrow[r] & \cdots \arrow[r] &  (M_\p)_{f_1 \cdots f_n} 
\end{tikzcd}
\end{center}
This is the Cech complex for $\widetilde{M_\p}$ on $\Spec{A_\p}$ with the respect to the open covering,
\[ \Spec{A_\p} = \bigcup_{i = 1}^n D(f_{i_0}) \]
However, the maximal ideal must lie in some open which is then forced to be the entire space now use the next lemma.
\end{proof}

\begin{lemma}
If $\mathfrak{U}$ is an open cover of $X$ such that some $U \in \mathfrak{U}$ is $X$ itself then $\check{H}^q(\mathfrak{U}, -) = 0$ for $q \ge 0$.
\end{lemma}

\begin{proof}
I claim that the extended Cech complex,
\begin{center}
\begin{tikzcd}
0 \arrow[r] & \F(X) \arrow[r] & \prod_{i_0} \F(U_{i_0}) \arrow[r] & \prod_{i_0 < i_1} \F(U_{i_0, i_1}) \arrow[r] & \cdots
\end{tikzcd}
\end{center}
is homotopy equivalent to zero. The homotopy is given by projecting onto the open $X$. 
\end{proof}

\section{}

\begin{definition}
Let $X$ be a projective curve over an algebraically closed field $k$. Then the \textit{genus} of $X$ is,
\[ g = \dim_k H^1(X, \struct{X}) \]
\end{definition}

\renewcommand{\C}{\mathbb{C}}

\begin{example}
When $k$ is not algebraically closed, this may not work. For example, take $X = \P^1_{\Q(i)}$ which is both a variety over $\Q$ and over $\Q(i)$. Consider the base change,
\begin{center}
\begin{tikzcd}
\P^1_{\Q(i)} \times_{\Spec{\Q}} \Spec{\C} \arrow[d] \arrow[r] & \P^1_{\Q(i)} \arrow[d]
\\
\Spec{\Q(i)} \times_{\Spec{\Q}} \Spec{\C} \arrow[d] \arrow[r] & \Spec{\Q(i)} \arrow[d]
\\
\Spec{\C} \arrow[r] & \Spec{\Q}
\end{tikzcd}
\end{center}
However, $\Q(i) \otimes_\Q \C = \C \times \C$ and thus we have,
\begin{center}
\begin{tikzcd}
\P^1_{\C} \coprod \P^1_{\C} \arrow[d] \arrow[r] & \P^1_{\Q(i)} \arrow[d]
\\
\Spec{\C} \coprod \Spec{\C} \arrow[d] \arrow[r] & \Spec{\Q(i)} \arrow[d]
\\
\Spec{\C} \arrow[r] & \Spec{\Q}
\end{tikzcd}
\end{center}
Therefore, under base change this scheme is no longer a variety. Furthermore, $\P^1_{\Q(i)}$ is a projective curve over $\Spec{\Q}$. To see this, consider the closed immersion $\Spec{\Q(i)} \hookrightarrow \P^1_\Q$ via the ring map $\Q[t] \to \Q(i)$ giving the map $\Spec{\Q(i)} \to \A^1_\Q$ into an affine open of $\P^1_{\Q(i)}$. Then we find a closed immersion,
\begin{center}
\begin{tikzcd}
\P^1_{\Q(i)} = \P^1_\Q \times_{\Spec{\Q}} \Spec{\Q(i)} \arrow[r, hook] & P^1_\Q \times_{\Spec{\Q}} P^1_\Q \arrow[r, hook, "segre"] & \P^3_\Q
\end{tikzcd}
\end{center}
so $\P^1_{\Q(i)}$ is a projective curve. However, we have the problem in this case that $\P^1_{\Q(i)}$ will have different $\Q$-dimension and $\Q(i)$-dimension in its cohomology so its genus is not well-defined in the previous sense. 
\end{example}

\renewcommand{\C}{\mathcal{C}}

\begin{definition}
If $X$ is a projective curve over $k$ and $H^0(X, \struct{X}) = k$ then we define the \textit{genus} of $X$ to be,
\[ g = \dim_k H^1(X, \struct{X}) \]  
\end{definition}

\begin{lemma}
If $f : X \to Y$ ar varieties and $X$ is projective, then any $f : X \to Y$ is closed. 
\end{lemma}

\begin{proof}

\end{proof}

\begin{proposition}
Any closed subset $\P^n_k$ is cout out by a collection of homogeneous polynomials. 
\end{proposition}

\begin{example}
The genus of $\P^1_k$ is zero. Consider its map to the curve, $X_0 X_1 X_2 = X_1^3 + X_2^3$ inside $\P^2_k$. 
\end{example}

\begin{definition}
Let $A$ be a graded ring and $M$ a graded module then $M(e)$ is the graded module with $M(e)_n = M_{n + e}$. 
\end{definition}

\begin{proposition}
Let $A$ be a graded ring. There is an exact functor from the category of graded $A$-modules to the category of quasi-coherent $\struct{\Proj{A}}$-modules via $M \mapsto \widetilde{M}$ satisfying,
\[ \Gamma(D_+(f), \widetilde{M}) = (M_f)_0 \]
\end{proposition}

\section{March 14}

\begin{proposition}[Kunneth]
Let $X$ and $Y$ be finite-type seperated schemes over $k$ and $\F$ and $\G$ quasi-coherent on $X$ and $Y$ then given the diagram,
\begin{center}
\begin{tikzcd}
& X \times_{\Spec{k}} Y \arrow[ld, "p"] \arrow[rd, "q"']
\\
X & & Y
\end{tikzcd}
\end{center}
Then we have,
\[ H^n(X \times_{\Spec{k}} Y, p^* \F \otimes_{\struct{X \times Y}} q^* \G) = \bigoplus_{i + j = n} H^i(X, \F) \otimes_k H^j(X, \G) \]
\end{proposition}

\begin{example}
Consider, $X = \P^1 \times \P^1$ and the sheaf $\struct{X}(a,b) = p_1^* \struct{\P^1}(a) \otimes_{\struct{X}} p_2^* \struct{\P^1}(b)$. Then we find,
\[ H^n(X, \struct{X}(a,b)) = \bigoplus_{i + j = n} H^i(\P^1, \struct{\P^1}(a)) \otimes_k H^j(\P^1, \struct{\P^1}(b)) \]
However, $H^i(\P^1, \struct{\P^1}(a))$ is only nonzero for $i = 0, 1$ 
\end{example}

\begin{proposition}
If $C \subset \P^1 \times \P^1$ is a close subscheme and also a curve then its ideal sheaf $\J$ is invertivle and hence of the form $\struct{X}(a,b)$ for some $a,b \in \Z$. Then we have a short exact sequence of sheaves,
\begin{center}
\begin{tikzcd}
0 \arrow[r] & \struct{X}(a, b) \arrow[r] & \struct{X} \arrow[r] & \iota^* \struct{C} \arrow[r] & 0
\end{tikzcd}
\end{center}
which gives the long exact sequence,


From which we find $H^1(C, \struct{C}) = k$ and $\dim_k H^1(C, \struct{C}) = \dim_k H^2(X, \struct{X}(a,b)) = (a + 1)(b + 1)$. Such a curve is called a $(-a, -b)$ curve. Such a curve can be expressed as 
\end{proposition}

\begin{example}
Let $Q = 0$ be a quadric and $K = 0$ a cubic in $\P^3$ and their intersectio is $C$. Suppose we know that $C$ has dimension $1$. Then $Q$ and $K$ form a regular sequence in $P = k[X_0, X_1, X_2, X_3]$. Then the Koszul complex on $K$ an $Q$ is exact. Then we find,
\begin{center}
\begin{tikzcd}
0 \arrow[r] & P(-5) \arrow[r] & P(-2) \otimes P(-3) \arrow[r] & P/(Q, K) \arrow[r] & 0
\end{tikzcd}
\end{center}
Then appling the functor sending these graded modules to coherent modules we find an exact sequence of sheaves,
\begin{center}
\begin{tikzcd}
0 \arrow[r] & \struct{\P^3}(-5) \arrow[r] & \struct{\P^3}(-2) \otimes \struct{\P^3}(-3) \arrow[r] & \struct{\P^3} \arrow[r] & \struct{C} \arrow[r] & 0 
\end{tikzcd}
\end{center}
Looking at the dimensions of the cohomology groups we have,
\end{example}

When $0 -> F -> G -> H -> 0$ is exact sequence of sheaves don't always get a long exact sequence of cech cohomology since applying sections is not exact. If $H^1(F|_U) = 0$ however, then taking sections should be exact so you should get a cech LES. Is this true? Does this ever happen in paractice (it does it F is flasque but then we know cech(F) vanishes also)? It is useful? 

When $0 -> F -> G -> H -> 0$ is exact sequence of sheaves we almost get a short exact sequence of cech complexes just not surjective since taking sections is only left-exact. Is there a spectral sequence or something which corrects the sequence of cech complexes by $H^1(F)$ to get something meaningful out of it. 

Prove Chevallay's theorem. Having trouble with the excercises. 


\section{March 28}

\begin{proposition}
Let $X$ be a variety over $k$ then $\Homover{\Spec{k}}{\Spec{\C[x]/(x^2)}}{X}$ is naturally identified with ``sections of the tangent bundle of X'' i.e. pairs $(x, \theta)$ where $x$ is a closed point of $X$ and $\theta$ is a tangent vector at $x$.
\end{proposition}

\begin{proof}
Denote $k[\epsilon] = k[x]/(x^2)$ then the quotient map,
\begin{center}
\begin{tikzcd}
k[\epsilon] \arrow[r, two heads] & k[\epsilon] / (\epsilon) = k
\end{tikzcd}
\end{center}
gives a closed immersion,
\begin{center}
\begin{tikzcd}
\Spec{k} \arrow[r, hook] & \Spec{k[\epsilon]}
\end{tikzcd}
\end{center}
\end{proof}

\begin{definition}
Let $R \to A$ be a ring map and $M$ an $A$-module. Then an $R$-\textit{derivation} $D : A \to M$ is an $R$-linear map satisfying the Leibniz rule,
\[ D(ab) = a D(b) + b D(a) \] 
\end{definition}

\begin{lemma}
There is a universal $R$-derivation $\d{} : A \to \Omega^1_{A/R}$ i.e. for any $A$-module $M$ we have,
\[ \Homover{A}{\Omega_{A/R}^1}{M} = \{ D : A \to M \mid R-\text{derivation} \} \]
via,
\[ \varphi \mapsto D = \varphi \circ d \]
\end{lemma}

\begin{proof}
Consider $F$ the free $A$-module on $S = \{ \d{a} \mid a \in A \}$ then $\Omega_{A/R}^1$ is the quotient of $F$ over the relations,
\begin{enumerate}
\item $\d{(ra)} = r \d{a}$ for $r \in R$
\item $\d{(a_1 + a_2)} = \d{a_1} + \d{a_2}$
\item $\d{(a_1 a_2)} = a_1 \d{a_2} + \d{a_1} a_2$
\end{enumerate}
\end{proof}

\begin{example}
Consider $\Omega^1_{R[x_1, \dots, x_n] / R}$. Then we have, $R$-derivations,
\[ \frac{\partial}{\partial x_i} : R[x_1, \dots, x_n] \to R[x_1, \dots, x_n] \]
Therefore, this map must factor through $\d{} : R[x_1, \dots, x_n] \to \Omega_{R[x_1, \dots, x_n] / R}$. We have,
\begin{center}
\begin{tikzcd}
R[x_1, \dots, x_n] \arrow[rr, "\partial_i"] \arrow[rd, "\d{}"']  & & R[x_1, \dots, x_n]
\\
& \Omega^1_{R[x_1, \dots, x_n] / R} \arrow[ru, "\psi_i"']
\end{tikzcd}
\end{center}
Therefore, 
\[ \psi_i(\d{x_j}) = \frac{\partial}{\partial x_i} (x_j) = \delta_{ij} \]
This implies that $\d{x_1}, \dots, \d{x_n}$ in $\Omega^1_{R[x_1, \dots, x_n] / R}$ must be $R[x_1, \dots, x_n]$-linearly independent. Furthermore, if $x \in \Omega^1_{R[x_1, \dots, x_n] / R}$ such that $\psi_i(x) = 0$ for each $i$ then $D = \varphi \circ \d{} = 0$ because 
\end{example}

\begin{lemma}
Let $R$ be a ring and $A = R[x_1, \dots, x_n] / I$ with $I  (f_1, \dots, f_m)$ then there is a presentation,
\begin{center}
\begin{tikzcd}
I / I^2 \arrow[r] & \bigoplus A \d{x_i} \arrow[r] & \Omega_{A / R} \arrow[r] & 0 
\end{tikzcd}
\end{center} 
under the map,
\[ f \mapsto \bigoplus_{i = 1}^n \frac{\partial f}{\partial x_i} \d{x_i} \] 
\end{lemma}

\begin{remark}
If $R = k$ is a field and $\m \subset A$ is a maximal ideal with residue field $k$ then there exists a canonical isomorphism,
\[ \Omega^1_{A /k } \otimes_A k(\m) \cong \m / \m^2 \]  
\end{remark}

\begin{definition}
Let $M$ be an $A$-module then,
\[ \Lambda^*(M) = T^*(M)/ \left< m \otimes m \mid m \in M \right> \]
\end{definition}

\begin{definition}
\[ \Omega^p_{A/R} = \Lambda^p (\Omega_{A / R}) \]
\end{definition}


\begin{theorem}
The complex $(\Omega^\bullet_{A / R}, \d{})$ is a differential graded algebra over $A$ where,
\[ \d{ (a \d{a_1} \wedge \cdots \wedge \d{a_p})} = \d{a} \wedge \d{a_1} \wedge \cdots \wedge \d{a_n} \]
Furthermore, the de-Rham cohomology of $A/R$ is the cohomology of this complex,
\[ H^p_{\text{dR}}(A/R) = H^p(\Omega^\bullet_{A/R}) \]
\end{theorem}

\begin{example}
Consider $R = k$ and $A = k[x_1, \dots, x_n]$. For characteristic zero,
\[ H^i_{\dR}(k[x_1, \dots, x_n] / k) = 
\begin{cases}
k & i = 0
\\
0 & i > 0
\end{cases} \] 
\end{example}


\begin{example}
Let $\Gm = \Spec{k[x, x^{-1}]}$ with $k$ characteristic zero. Then,
\[ H^1_{\dR}(k[x, x^{-1}] / k) = k \frac{\d{x}}{x} \]
\end{example}


\section{Closed Subschemes and Divisors}

\begin{definition}
A closed subscheme $Z$ of a scheme $X$ is an equivalence class of closed immersions $\iota : Z \hookrightarrow X$. 
\end{definition}

\begin{theorem}
There is a correspondence between closed subschemes $Z$ of $X$ and sheaves of ideals $\J \subset \struct{X}$ i.e. injections of quasi-coherent $\struct{X}$-modules up to isomorphism,
\begin{center}
\begin{tikzcd}
0 \arrow[r] & \J \arrow[r] & \struct{X} 
\end{tikzcd}
\end{center}
Via the correspondence: given $\iota : Z \to X$ the map of sheaves $\iota^\# : \struct{X} \to \iota_* \struct{Z}$ is surjective and thus fits into an exact sequence,
\begin{center}
\begin{tikzcd}
0 \arrow[r] & \J \arrow[r] & \struct{X} \arrow[r] & \iota_* \struct{Z} \arrow[r] & 0
\end{tikzcd}
\end{center}
Conversely, given a sheaf of ideals $\J \subset \struct{X}$ then let $Z = (\Supp{\struct{X}}{\struct{X} / \J}, \struct{X} / \J)$.
\end{theorem}

\begin{proof}
First, given a closed subscheme $\iota : Z \to X$ we need to check that the corresponding ideal sheaf $\J$ generates $Z$. Since closed immersions are separated and quasi-compact then $\iota_* \struct{Z}$ is a quasi-coherent $\struct{X}$-module which implies that $\J$ is also quasi-coherent. In this case there is an isomorphism $\iota_* \struct{Z} \cong  \struct{X} / \J$. Note that $\iota(Z)$ is closed and thus if $x \notin \iota(Z)$ then any open neighborhood of $x$ contains some $U \subset X \setminus \iota(Z)$ open neighborhood of $x$ on which $(\iota_* \struct{Z})(U) = \struct{Z}(f^{-1}(U)) = \struct{Z}(\varnothing) = 0$. Thus if $x \notin \iota(Z)$ then $(\iota_* \struct{Z})_x = 0$ Furthermore, if $\iota(z) \in \iota(Z)$ then because $\iota$ is a homeomorphism onto its image, every open neighborhood of $z$ is of the form $\iota^{-1}(U)$ for some open $U \subset X$ and thus,
\[ (\iota_* \struct{Z})_{\iota(z)} = \varinjlim_{\iota(z) \in U} \struct{Z}(\iota^{-1}(U)) = \varinjlim_{z \in V} \struct{Z}(V) = \stalk{Z}{z}  \] 
In particular, if $\iota(z) \in \iota(Z)$ then $(\iota_* \struct{Z})_{\iota(z)} = \stalk{Z}{z} \neq 0$. Therefore we have shown that,
\[ x \in \iota(Z) \iff (\iota_* \struct{Z})_x \neq 0 \iff x \in \Supp{\struct{X}}{\struct{X} / \J} \]
Thus let $Z' = (\Supp{\struct{X}}{\struct{X} / \J}, \struct{X} / \J)$ then there is an isomorphism $\iota : Z \to Z'$ which has $\iota^\# : \struct{X} / \J \to \iota_* \struct{X}$ which makes the diagram commute,
\begin{center}
\begin{tikzcd}
0 \arrow[r] & Z \arrow[d] \arrow[r] & X \arrow[d, "\id_X"]
\\
0 \arrow[r] & Z' \arrow[r] & X
\end{tikzcd}
\end{center}
Now, given a quasi-coherent sheaf of ideals $\I \subset \struct{X}$ then we must show that $Z = (\Supp{\struct{X}}{\struct{X} / \J}, \struct{X} / \J)$ is a closed subscheme. This is a local property so take an affine open $U \subset X$ on which $U = \Spec{A}$ and $\J = \widetilde{\mathfrak{a}}$ for some ideal $\mathfrak{a} \subset A$. Then the map $\iota : Z \to X$ is given locally by the ring map $A \to A / \mathfrak{a}$ which gives a closed immersion. Finally, it is clear that the sheaf of ideals corresponding to this $Z$ is exactly $\J$ since it is the kernel of the map $\struct{X} \to \struct{X} / \J$.  
\end{proof}

\begin{definition}
Let $X$ be a scheme then \textit{a locally principal closed subscheme} of $S$ is a closed subsheme $Z \subset X$ such that the sheaf of ideals $\I_Z$ is locally generated by a single element. 
\end{definition}

\begin{definition}
An \textit{effective Carier divisor} on $X$ is a closed subscheme $D \subset X$ whose ideal sheaf $\I_D \subset \struct{X}$ is an invertible $\struct{X}$-module. 
\end{definition}

\begin{definition}
Let $X$ be a scheme and $D \subset X$ a closed subscheme then the following are equivalent,
\begin{enumerate}
\item $D$ is an effective Cartier divisor on $X$
\item for each $x \in D$ there exists an affine open neighborhood $x \in U \subset X$ with $U = \Spec{A}$ such that $U \cap D = \Spec{A / (f)}$ for $f \in A$ a nonzerodivisor. 
\end{enumerate}
\end{definition}

\begin{proof}
Assume that $D$ is an effective Cartier divisor then for each $x \in X$ there exists an affine open $x \in U \subset X$ such that $\I_D |_U \cong \struct{X} |_U$. Since $\I_D$ is quasi-coherent, we may further shrink $U$ such that $\I_D |_U = \widetilde{\a}$ for some ideal of $A$ where $U = \Spec{A}$. The isomorphism $A \to \a$ is uniquely determined by the image of $1 \in \a \subset A$ say $1 \mapsto f$ then $\a = (f)$. Therefore, $\I_D |_U = \widetilde{(f)}$ meaning that locally $D \cap U = \Supp{A}{A / (f)} = \Spec{A / (f)}$. Furthermore, suppose that $\exists g \in A$ such that $fg = 0$. Consider the preimage $\tilde{g} \mapsto g$ under the isomorphism $A \to \tilde{\a}$ and thus $\tilde{g} = 1 \tilde{g} \mapsto fg = 0$ so $\tilde{g}$ is in the kernel of the map so $g = 0$ implying that $f$ cannot be a zero divisor.
\bigskip\\
Conversely, we have $U \cap D = \Spec{A / (f)}$ then locally the map $D \to X$ is given by the ring map $A \to A / (f)$ so $\I_D |_U = \widetilde{(f)}$. Since $f$ is a non-zero divisor, the map $f : A \to (f)$ is an isomorphism proving that $\I_D$ is an invertible sheaf since $\struct{X}|_U = \widetilde{A}$. 
\end{proof}

\begin{definition}
Let $X$ be a scheme. Given effective Carteir divisors $D_1$ and $D_2$ on $X$ we set $D = D_1 + D_2$ to be the closed subscheme of $X$ corresponding o the quasi-coherent sheaf of ideals $\I_{D_1} \cdot \I_{D_2} \subset \struct{X}$. 
\end{definition}

\begin{proposition}
The sum of effective Cartier divisors is an effective Cartier divisor.
\end{proposition}

\begin{proof}
The product of non-zero divisors is a non-zero divisor and thus the product of these ideals is locally invertible.
\end{proof}

\begin{definition}
Let $X$ be a scheme and $D \subset X$ an effective Cartier divisor with an ideal sheaf $\I_D$. Recall that $\I_D$ is an invertible $\struct{X}$-module so we may define,
\begin{enumerate}
\item The invertible sheaf $\struct{X}(D)$ associated to $D$ is defined by,
\[ \struct{X}(D) = \shHomover{\struct{X}}{\I_D}{\struct{X}} = \I_D^{\otimes - 1} \]
\item The canonical section, $1_D \in \struct{X}(D)$ is the inclusion morphism $\I_D \to \struct{X}$. 
\item We write $\struct{X}(-D) = \struct{X}(D)^{\otimes -1} = \I_D$.
\item Given a second effective Cartier divisor $D' \subset X$ we define,
\[ \struct{X}(D - D') = \struct{X}(D) \otimes_{\struct{X}} \struct{X}(-D') \]
\end{enumerate}
\end{definition}

\begin{lemma}
Let $X$ be a scheme and $D, C \subset X$ be effective Cartier divisors with $C \subset D$ and let $D' = D + C$. Then there exists a short exact sequence of $\struct{X}$-modules,
\begin{center}
\begin{tikzcd}
0 \arrow[r] & \struct{X}(-D) |_C \arrow[r] & \struct{D'} \arrow[r] & \struct{D} \arrow[r] & 0
\end{tikzcd}
\end{center}
\end{lemma}

\begin{proof}
Let $\I$ be the ideal sheaf if $D \to D'$. Then there is a short exact sequence,
\begin{center}
\begin{tikzcd}
0 \arrow[r] & \I \arrow[r] & \struct{D'} \arrow[r] & \struct{D} \arrow[r] & 0
\end{tikzcd}
\end{center}
Now I claim that $\struct{X}(-D) |_C = \I_D |_C = \I$. 
\end{proof}

\begin{lemma}
Let $X$ be a scheme and $D_1, D_2 \subset X$ be effective Cartier divisors on $X$. Let $D = D_1 + D_2$. Then there is a unique isomorphism,
\[ \struct{X}(D_1) \otimes_{\struct{X}} \struct{X}(D_2) \to \struct{X}(D) \]
which maps $1_{D_1} \otimes 1_{D_2} \to 1_D$.
\end{lemma}

\begin{proof}
By definition $\I_D = \I_{D_1} \cdot \I_{D_2}$. Consider the map,
\[ \shHomover{\struct{X}}{\I_{D_1}}{\struct{X}} \otimes_{\struct{X}} \shHomover{\struct{X}}{\I_{D_2}}{\struct{X}} \to \shHomover{\struct{X}}{\I_{D}}{\struct{X}} \]
via $f_1 \otimes f_2 \mapsto f_1 \cdot f_2$. Clearly, this map sends $1_{D_1} \otimes 1_{D_2}$ to $1_D$. Thus, it is sufficient to prove that this map is the unique isomorphism. Because these sheaves are invertible, on stalks, this map becomes the isomorphism,
\[ \Homover{\stalk{X}{x}}{(f_1)}{\stalk{X}{x}} \otimes_{\stalk{X}{x}} \Homover{\stalk{X}{x}}{(f_2)}{\stalk{X}{X}} \to \Homover{\stalk{X}{x}}{(f_1 f_2)}{\stalk{X}{x}} \]
This is unique because each side is abstractly isomorphic to $\struct{X}{x}$ and the map abstractly the identity since it sends $(f_1 \mapsto 1) \otimes (f_2 \mapsto 1) \mapsto (f_1 f_2 \mapsto 1)$. 
\end{proof}

\begin{definition}
Let $X$ be a locally ringed space and $\L$ an invertible sheaf on $X$. A global section $s \in \Gamma(X, \L)$ is called a regular section in the map $\struct{X} \to \L$ via $f \mapsto fs$ is injective.
\end{definition}

\begin{lemma}
Let $X$ be a locally ringed space and $f \in \Gamma(X, \struct{X})$. Then the following are equivalent,
\begin{enumerate}
\item $f$ is a regular section 
\item for any $x \in X$ the image $f \in \struct{X}{x}$ is a non-zero divisor.
\end{enumerate}
If $X$ is a scheme there are also equivalent to,
\begin{enumerate}
\item for any affine open $\Spec{A} = U \subset X$ the image $f \in A$ is a non-zero divisor
\end{enumerate}
\end{lemma}

\begin{proof}
The sheaf map $\struct{X} \to \struct{X}$ given by $f \mapsto fs$ is injective iff on stalks $\stalk{X}{x} \to \stalk{X}{x}$ is injective i.e. $f \in \stalk{X}{x}$ is a non-zero divisor. 
\bigskip\\
Now let $X$ be a scheme. If on each affine open $\Spec{A} = U \subset X$ the image $f \in A$ is a non zero divisor then, since affine opens form a base for the topology on $X$, then $f_x \in \stalk{X}{x}$ is a non zero divisor since otherwise it would be a zero divisor on some open neighborhood containing an affine open. Conversely, if $f |_U$ is a zero divisor on some affine open then for each $x \in U$ the image $f_x \in \stalk{X}{x}$ is a zero divisor.
\end{proof}

\begin{remark}
Let $\L$ be an invertible $\struct{X}$-module and $s \in \Gamma(X, \L)$ is a global section. We may realize $s$ as an $\struct{X}$-module map $s : \struct{X} \to \L$. Its dual then gives a map $s : \L^{\otimes - 1} \to \struct{X}$. 
\end{remark}

\begin{definition}
Let $X$ be a scheme and $\L$ an invertible sheaf on $X$. Let $s \in \Gamma(X, \L)$ be a global section. The \textit{zero scheme} of $s$ is the closed subscheme $Z(s) \subset X$ defined by the quasi-coherent sheaf of ideals $\I \subset \struct{X}$ defined by $s : \L^{\otimes -1} \to \struct{X}$. 
\end{definition}

\begin{remark}
Let $f : X \to Y$ be a morphism of locally ringed spaces and $\F$ a sheaf of $\struct{X}$-modules. A global section $s \in \Gamma(Y, \F)$ can be realized as a morphism $s : \struct{Y} \to \F$. Applying the functor $f^*$ gives a morphism $f^* s : f^* \struct{Y} \to f^* \F$ which is equivalent to a section $f^* s : \struct{X} \to f^* \F$ since $f^* \struct{Y} = \struct{X}$. 
\end{remark}

\begin{lemma}
Let $X$ be a scheme and $\L$ an invertible sheaf on $X$ and $s \in \Gamma(X, \L)$ a global section. Then,
\begin{enumerate}
\item Consider the closed immersions $\iota : Z \to X$ such that $\iota^* s \in \Gamma(Z, \iota^* \L)$ is zero, ordered by inclusion. The zero scheme $Z(s)$ is the maximal element of this order set.
\item The zero scheme $Z(s)$ is a locally principal closed subscheme.
\item $Z(s)$ is an effective Cartier divisor iff $s$ is a regular section of $\L$.
\end{enumerate}
\end{lemma}

\begin{proof}
Suppose that $\iota : Z \to X$ is a closed subscheme such that $\iota^* s \in \Gamma(Z, \iota^* \L)$ is zero. It suffices to show that $\I_{Z(s)} \subset \I_{Z}$. 
Consider any morphism $f : \L \to \struct{X}$ of $\struct{X}$-modules. Then $\iota^* f : \iota^* \L \to \struct{Z}$ takes $\iota^* s \mapsto \iota^* f(s)$ which is zero since $\iota^* s = 0$ thus $f(s) \mapsto 0$ under $\struct{X} \to \iota_* \struct{Z}$ so $\I_{Z(s)} \subset \I_{Z}$.
\bigskip\\
Since $\L$ is invertible, there is an affine open cover such that $\L |_U \cong \struct{X} |_U$ on each open $\Spec{A} = U \subset X$. Thus, $\L|_U = \widetilde{M}$ for some $A$-module $M$ such that $M \cong A$ as $A$-modules i.e. $M$ is free of rank $1$. Then consider the map $s : \struct{X} \to \L$ which restricts to the map $s |_U : A \to M$ given by $a \mapsto a s|_U$ whose dual is $s |_U : \L^{\otimes - 1} \to \struct{X}$ takes $(f : M \to A) \mapsto f(s|_U)$. Since $M$ is free of rank $1$ we may write $s|_U = s_A m$ for $s_A \in A$ and $m \in M$ the basis element. Then every $A$-module map $f : M \to A$ is determined by the image of $m \mapsto f(m)$ so $f(s|_U) = s_A f(m)$. In particular, there exists an isomorphism $M \to A$ which has $f(m) = 1$ so $\Homover{A}{M}{A} \cong A$ via $f \mapsto f(m)$ so $\Im{s|_U} = \{s_A f(m) \mid f \in \Homover{A}{M}{A} \} = (s_A) \subset A$. Thus the sheaf of ideals of $Z(s)$ is locally generated by a single element.
\bigskip\\
Furthermore, $s \in \Gamma(X, \L)$ is a regular section iff $s|_U$ is regular for each affine open $U$ i.e. the map $a \mapsto a s_A$ is injective meaning $A \cong (s_A)$. Thus, since locally the sheaf of ideals of $Z(s)$ is $(s_A)$, the section $s$ is regular iff $Z(s)$ is an effective Cartier divisor. 
\end{proof}

\begin{theorem}
Let $X$ be a scheme. 
\begin{enumerate}
\item If $D \subset X$ is an effective Cartier divisor then the canocnical section $1_D$ of $\struct{X}(D)$ is regular.
\item Conversely, if $s$ is a regular section of the invertible sheaf $\L$ then there exists a unique effective Cartier divisor $D = Z(s) \subset X$ and a unique isomorphism $\struct{X}(D) \to \L$ sending $1_D \mapsto s$. 
\end{enumerate}
The construction $D \mapsto (\struct{X}(D), 1_D)$ and $(\L, s) \mapsto Z(s)$ are inverse giving a bijective correspondence between effective Cartier divisors on $X$ and isomorphism classes of pairs $(\L, s)$ where $\L$ is an invertible sheaf of $\struct{X}$-modules and $s \in \Gamma(X, \L)$ is a regular global section. 
\end{theorem}

\begin{proof}
Let $D \subset X$ be an effective Cartier divisor and consider the canonical section $1_D$ of $\struct{X}(D) = \shHomover{\struct{X}}{\I_D}{\struct{X}}$. Consider the map $\struct{X} \to \struct{X}(D)$ given by $f \mapsto f \cdot 1_D$. On stalks, we know that the ideal $(\I_D)_x \cong \stalk{X}{x}$ so $(\I_D)_x = (f)$ where $f \in \stalk{X}{x}$ is the preimage of $1$. Given any section $g \in \stalk{X}{x}$ if $g_x (1_D)_x = 0$ then $g \cdot f = 0$ meaning that either $g_x = 0$ or $f$ is a zero divisor. However, since $\I_D$ is invertible, $f$ cannot be invertible (its isomorphic to $1$ so it cannot be a zero divisor) thus $g_x = 0$. Therefore this map $1_D : \struct{X} \to \struct{X}(D)$ is injective at the stalks and therefore injective.
\bigskip\\
Now suppose that $\L$ is an invertible sheaf and $s \in \Gamma(X, \L)$ a regular secton. Consider $D = Z(s) \subset X$. Since $s$ is regular, we have shown that $Z(s)$ is an effective Cartier divisor. Furthermore, $\I_D = \Im{s : \L^{\otimes - 1} \to \struct{X}} = \L^{\otimes - 1}$ where $s$ is regular so this is injective. Thus, $\struct{X}(D) = \I_D^{\otimes - 1} = \L$. Finally, given an effective Cartier divisor we know that $(\struct{X}(D), 1_D)$ is an invertible sheaf with a regular section. Consider $Z(s)$ which is the closed subscheme uniquely defined by the sheaf of ideals given by the image of $1_D : \struct{X}(D)^{\otimes -1} \to \struct{X}$ which is exactly the inclusion map $\I_D \to \struct{X}$ since $\struct{X}(D) = \I_D^{\otimes -1}$. Therefore, we find that $Z(s) \cong Z$. 
\end{proof}


\section{GAGA}

\renewcommand{\C}{\mathbb{C}}

\begin{theorem}
There is a unique functor from schemes of finite type over $\C$ to topological spaces given by $X \mapsto X^\an$ where $X^\an = (X(\C), \T)$ where $X(\C) = \Homover{\C}{\Spec{\C}}{X}$ are the $\C$-rational points and $\T$ is the analytic topology such that,
\begin{enumerate}
\item $(\A^n_\C)^\an = \C^n$ with its standard topology
\item closed immersions are mapped to closed embeddings
\item open immersions are mapped to open embeddings
\end{enumerate}
\end{theorem}

\begin{remark}
These requirements already fix the topology on $X(\C)$ because if $X$ is finite type over $\C$ then in particular $X$ is quasi-compact. Thus, we write,
\[ X = \bigcup_{i = 1}^n U_i \]
for $U_i$ affine open. However, $U_i$ is finite type over $\C$ and thus $U_i = \Spec{A}$ with $A$ a finitely generated $\C$-algebra and thus $A = \C[x_1, \dots, x_{k_i}] / I$ which implies that $U_i \hookrightarrow \A_\C^{k_i}$ is a closed immersion. Then,
\[ X(\C) = \bigcup_{i = 1}^n U_i(\C) \quad \quad U_i(\C) \hookrightarrow \C^{k_i} \]
\end{remark}

\begin{remark}
This will rely on the fact that for $f \in \C[x_1, \dots, x_n]$ the map $f : \C^n \to \C$ is continuous in the standard topology.   
\end{remark}

\begin{remark}
We chose to take $k = \C$ here but any topological field will work. However, if $k$ is not algebraically closed then $X(k)$ will not be all closed points of $X$.
\end{remark}

\begin{lemma}
Let $X$ be an affine scheme of finite type over $\C$ then,
\begin{enumerate}
\item if $X \to \A^n_\C$ is a closed immersion then $X(\C) \subset \C^n$ is closed (in the standard topology).
\item If $X \hookrightarrow A^n_\C$ and $X \hookrightarrow A^m_\C$ are two closed immersions then the induced topology on $X(\C)$ is the same. 
\end{enumerate}
\end{lemma}

\begin{proof}
Let $X \hookrightarrow \A^n_\C$ be a closed immersion then $X = \Spec{\C[x_1, \dots, x_n / I}$ then,
\[ X(\C) = \{ \vec{x} \in \C^n \mid \forall f \in I : f(\vec{x}) = 0 \} = \bigcap_{f \in I} f^{-1}(0) \]
which is closed because $f : \C^n \to \C$ is continuous. Now suppose that $X \hookrightarrow A^n_\C$ and $X \hookrightarrow A^m_\C$ are two closed immersions then there is a closed immersion $X \hookrightarrow A_\C^n \times_\C \A_\C^m$. Then we have closed embeddings,
\begin{center}
\begin{tikzcd}
& \C^{n + m} \arrow[dd]
\\
X(\C) \arrow[ru, hook] \arrow[rd, hook] 
\\
& \C^n
\end{tikzcd}
\end{center}
which gives that the induced topologies are the same. 
\end{proof}

\begin{theorem}
Given a functor from the category of affine schemes of finite type over $k$ to topological spaces which respects closed and open imersions then the functor can be extended to all schemes of finite type over $k$. 
\end{theorem}

\begin{definition}
Let $X$ be a scheme of finite type over $\C$ then we define a singular cohomology theory for $X$ via,
\[ H^*_{\text{Betti}}(X) = H^*_{\text{sing}}(X^\an, \Z) = H^*(X^\an, \underline{\Z}) \]
\end{definition}

\begin{lemma}
If $X$ is finite type over $\C$ then 
\begin{center}
\begin{enumerate}
\item $X$ is separated $\iff$ $X(\C)$ is Hausdorff 
\item $X$ is proper $\iff$ $X$ is compact Hausdorff. 
\end{enumerate}
\end{center}
\end{lemma}

\begin{theorem}[Grothendieck]
If $X$ is a smooth variety over $\C$ then 
\[ H^*_{\dR}(X/\C) = H^*_{\text{Betti}}(X) \otimes_\Z \C \] 
\end{theorem}

\begin{theorem}
Suppose that $X_0$ is a smooth variety over $\Q$ and $X = X_0 \times_{\Spec{\Q}} \Spec{X}$ such that $X$ is a smooth variety. Then,
\[ H^*_{\dR}(X_0 / \Q) \otimes_\Q \C = H_{\dR}^*(X/\C) = H^*_{\text{sing}}(X^\an, \Q) \otimes_\Q \C \] 
The complex numbers giving this map of $Q$-vector spaces gives the \textit{periods}.
\end{theorem}

\renewcommand{\C}{\mathcal{C}}

\section{April 3}

$\Delta : C \to C \times C$ is an effecitve Cartier divisor i.e. the ideal sheaf $\I$ is locally generated by a single nonzero element ie.e $\I$ is an invertible $\struct{C\times C}$-module. Then,
\[ \struct{C \times C}(\Delta) := \I^{\otimes -1} = \shHomover{\struct{C \times C}}{I}{\struct{C \times C}} \]

\begin{remark}
If $g(C) > 0$ then $\struct{C \times C}(\Delta)$ is not in $\Pic{C} \oplus \Pic{C}$. 
\end{remark}

\begin{example}
If $C$ is a smooth projective curve of positive genus over $k$ (algebraically closed) then consider,
\begin{center}
\begin{tikzcd}
& C
\\
C \times c_0 \arrow[r] & C \times C \arrow[d] \arrow[r] & C
\\
& C
\end{tikzcd}
\end{center}
\end{example}

\begin{theorem}
If $X$ is a variety then
\[ \Pic{X \times \P^1} = \Pic{X} \oplus \Z \]
\end{theorem}

\begin{lemma}
If $C$ is a smooth projective curve over algebraically closed $k$ then,
\[ \dim_k \Gamma(C, \struct{C}(c_0)) = \begin{cases}
1 & g(C) > 0
\\
2 & g(C) = 0
\end{cases} \]
\end{lemma}


\begin{definition}
Let $f : X \to Y$ be a noncostant morphism of smooth projective curves over $k$ algebraically closed. There is a map,
\[ f^* : \Div{Y} \to \Div{X} \]
via,
\[ f^*(\sum n_i [y_i]) = \sum n_i f^*[y_i] \]
where,
\[ f^*[y] = \sum_{x \mapsto y} e_x \cdot [x] \]
where $e_x$ is the   fication index of $\stalk{Y}{y} \to \stalk{X}{x}$ meaning the power $\varpi_y \mapsto u \cdot \varpi^e_x$ where $\varpi$ is the unifromizer of the DVR. 
\end{definition}

\begin{proposition}
\[ \sum_{x \mapsto y} e_x = [k(X) : k(Y)] \]
\end{proposition}

\begin{definition}
The degree of $f$ is $[k(X) : k(Y)]$ such that,
\begin{center}
\begin{tikzcd}
\Div{Y} \arrow[d, "\deg"'] \arrow[r, "f^*"] & \Div{X} \arrow[d, "\deg"]
\\
\Z \arrow[r, "\deg(f)"] & \Z 
\end{tikzcd}
\end{center}
commutes.
\end{definition}

\begin{proposition}
$f^* : \Div{Y} \to \Div{X}$ preserves rational equivalence i.e. $f^* \div_Y(g) = \div_X(g \circ f)$. where $g \circ f \in k(X)$ and $g \in k(Y) \xrightarrow{f} k(X)$. 
\end{proposition}

\begin{remark}
If $g \in k(X)$ is nonconstant then $\div_X(g) = g^*([0] - [\infty])$ viewing $g : X \to \P_k^1$. Then $\deg{\div_X(g)} = \deg{g} - \deg{g} = 0$. 
\end{remark}

\section{Sheaves of Derivations}

\begin{definition}
Let $X$ be a scheme over a field $k$ then,
\[ H^*_{\dR}(X/k) = H^*(X, \Omega^\bullet_{X/k}) \]
where $\Omega^\bullet_{X/k}$ is a complex of sheaves of $k$-modules. The objects of the complex of $\struct{X}$-modules but the maps are only $k$-linear. In algebraic geometry, the de-Rham complex is not exact. Therefore, we need to take an injective resolution,
\begin{center}
\begin{tikzcd}
0 \arrow[r] & \struct{X} \arrow[d, "\alpha^0"] \arrow[r] & \Omega^1_{X/k} \arrow[d, "\alpha^1"] \arrow[r] & \Omega^2_{X/k}  \arrow[d, "\alpha^2"] \arrow[r] & \Omega^3_{X/k}  \arrow[d, "\alpha^3"] \arrow[r] & \cdots
\\
0 \arrow[r] & \I^{0} \arrow[r] & \I^1 \arrow[r] & \I^2 \arrow[r] & \I^3 \arrow[r] & \cdots  
\end{tikzcd}
\end{center}
such that $H^n(\alpha^\bullet) : H^n(\F^\bullet) \to H^n(\I^\bullet)$ is an isomorphism for each $n$ i.e. $\alpha^\bullet$ is a quasi-isomorphism. Then,
\[ H^i(X, \Omega^\bullet_{X/k}) = H^i(\Gamma(X, \I^\bullet)) \]
\end{definition}

\begin{remark}
There is a Hodge to de-Rham spectral sequence,
\[ E^{p,q}_1 = H^q(X, \Omega_{X/k}^p) \implies H_{\dR}^{p+q}(X/k) \]
\end{remark}

\begin{example}
Consider the scheme $X = \P^1_k$ then $\Omega^1_{X/k} = \struct{X}(-2)$. Now $\I, \J$.
\end{example}

\section{April 16}

\begin{lemma}
If $k$ is a field and $F \supset F' \supset k$ is a tower of fields such that $F$ and $F'$ are finitely generated extensions such that 
\[ \trdeg{k}{F} = \trdeg{k}{F'} \]
then $F / F'$ is finite. 
\end{lemma}

\begin{theorem}
Let $F : \mathcal{A} \to \mathcal{B}$ be a left exact functor between abelian categories and $\mathcal{I}^\bullet \to \mathcal{J}^\bullet$ a quasi-isomorphism between bounded below complexes of injectve objects of $\mathcal{A}$. Then $F(\mathcal{I}^\bullet) \to F(\mathcal{J}^\bullet)$ is a quasi-isomorphism.
\end{theorem}

\begin{theorem}
There is and exact sequence of complexes,
\begin{center}
\begin{tikzcd}
\mathcal{I}^\bullet \arrow[r, "\alpha^\bullet"] & \mathcal{J}^\bullet 
\end{tikzcd}
\end{center}
Then $F(C(\alpha)^\bullet) = C(F(\alpha^\bullet))$. Therefore, it suffices to show that $H^n(F(C(\alpha)^\bullet)) = 0$ given tat $H^n(C(\alpha)^\bullet) = 0$. Thus we only need to show that if $\mathcal{I}$ is a bounded below acyclic complex of injective objects in $\mathcal{A}$ then $H^n(F(\mathcal{I}^\bullet)) = 0$. However, since it is a complex of injectives the complex splits. Since $F$ is additive, the complex remains split after applying $F$ and thus remains acylic.  
\end{theorem}

\subsection{Normalization}

\begin{definition}
An integral scheme $X$ is \textit{normal} iff for all affine opens $U = \Spec{A}$ the ring $A$ is a normal domain i.e. $A$ is integrally closed in $\Frac{A}$. 
\end{definition}

\begin{proposition}
For any variety $X$ there is a unique birationl finite morphism $\nu : X^\nu \to X$ such that $X^\nu$ is normal. Additionally, if $U = \Spec{A}$ is an affine open then $\nu^{-1}(U) = \Spec{A^\nu}$ where $A \subset A^\nu \subset \Frac{A}$ is the integral closure of $A$ in $\Frac{A}$. 
\end{proposition}

\begin{proposition}
If $A$ is a finite type domain over a field then $A^\nu / A$ is finite. 
\end{proposition}

\begin{proposition}
If $X \to Y$ is finite and $Y$ is projective then $X$ is projective.
\end{proposition}

\section{Serre Duality}

\begin{remark}
Let $X$ be a projective variety over a field $k$ of dimension $d$.
\end{remark}

\begin{theorem}
$H^i(X, \F) = 0$ for $i > $ and any abelian sheaf $\F$. 
\end{theorem}

\begin{proof}
Grothendieck proved this for any Noetherian topological space of dimension $d$. For $\F$ quasi-coherent $\struct{X}$-module this also follows from the fact that $X$ has an open ocer of $d+1$ affines. 
\end{proof}

\begin{theorem}
The functor $H^d(X -)$ is right exact.
\end{theorem}

\begin{proof}
Let,
\begin{center}
\begin{tikzcd}
0 \arrow[r] & \F_1 \arrow[r] & \F_2 \arrow[r] & \F_3 \arrow[r] & 0
\end{tikzcd}
\end{center}
be an exact sequence of sheaves on $X$ then consider the long exact cohomology sequence
\begin{center}
\begin{tikzcd}
H^d(X, \F_1) \arrow[r] & H^d(X, \F_2) \arrow[r] & H^d(X, \F_3) \arrow[r] & H^{d+1}(X, \F_1) 
\end{tikzcd}
\end{center}
However $H^{d+1}(X, \F_1) = 0$ by Grothendieck's theorem.
\end{proof}

\begin{theorem}
The functors $H^q(X, -)$ commute with direct sums. 
\end{theorem}


\begin{theorem}
$\QCoh(\struct{X})$ is a Grothendieck abelian category meaning that filtered colimits are exact.
\end{theorem}

\begin{theorem}
If $\mathcal{A}$ is a Grothendieck abelian category and $F : \mathcal{A} \to \Ab$ is contravariant, right exact, and transforms direct sums into products. Then $F$ is representable i.e. there exists $A_{\text{univ}} \in \mathcal{A}$ such that, naturally,
\[ F(A) = \Homover{\mathcal{A}}{A}{A_{\text{univ}}} \]
The identification maps $\varphi : A \to A_{\text{univ}}$ to $\xi = F(\varphi)(\xi_{\text{univ}})$ where $\xi_{\text{univ}} \in F(A_{\text{univ}}$ corresponds to $\id_{A_{\text{univ}}} \in \Homover{\mathcal{A}}{A_{\text{univ}}}{A_{\text{univ}}}$.
\end{theorem}

Now apply to the functor $F : \QCoh(\struct{X}) \to \Ab$ defined by 
\[ \F \mapsto H^d(X, \F)^\vee = \Homover{k}{H^d(X,\F)}{k} \]
Conclusion, there exists $\omega_X \in \QCoh(\struct{X})$ and a linea map $t : H^d(X, \omega_X) \to k$ such that for any quasi-coherent $\F$ we have,
\[ \Homover{\struct{X}}{\F}{\omega_X} \xrightarrow{\sim} H^d(X, \F)^\vee \]
via $\varphi \mapsto t \circ H^d(X, \varphi)$. If $\F$ is coherent then $\dim_k H^d(X, \F) < \infty$ so we get a perfect pairing,
\[ \Homover{\struct{X}}{\F}{\omega_X} \times H^d(X, \F) \to k \]
sending $(\varphi, \eta) \mapsto t(\varphi \smile \eta) = t(\varphi(\eta))$.
For example,
\[ H^0(X, \omega_X) \times H^d(X, \struct{X}) \to k \]
is a perfect pairing. 

\begin{theorem}
Let $X$ be a smooth projective curve over $k = \bar{k}$ then,
\[ \Homover{\struct{X}}{\mathcal{L}}{\mathcal{N}} = H^0(X, \mathcal{N} \otimes_{\struct{X}} \mathcal{L}^{\times - 1}) \]
If $\mathcal{L}$ has degree < 0 (viewed as an element of $\Cl{X}$) then $H^0(X, \mathcal{L}) = 0$.
\end{theorem}

\begin{theorem}
The pair $(\struct{X}(-n-1), t)$ on $\P^n_k$ is the dualizing pair and moreover for any $\F$ coherent you get a perfect paring,
\[ \Ext{i}{\struct{\P^n_k}}{\F}{\struct{\P^n_k}(-n-1)} \times H^{n-i}(\P^n_k, \F) \to k \]
for $i = 0, \dots, n$. If $\F$ is finite locally free, then this becomes,
\[ H^i(\P_k^n, \F^\vee(-n-1)) \times H^{n-i}(\P^n_k, \F) \to k \]
is a perfect pairing.
\end{theorem}

\section{April 23}

\begin{theorem}
Let $X$ be a smooth projective curve over $k$, algebraically closed and $\mathcal{L} \in \Pic{X}$ then,
\[ \chi(X, \mathcal{L}) = \dim_k H^0(X, \mathcal{L}) - \dim_k H^1(X, \mathcal{L}) = \deg \mathcal{L} + \chi(X, \struct{X}) \]
where $\chi(X, \struct{X}) = 1 - g$. 
\end{theorem}

\begin{proof}
We can write $\mathcal{L} = \struct{X}(D)$ for some divisor,
\[ D = \sum_{x \in X} n_x [x] \]
It suffices to show that the theorem holds for $\struct{X}(D)$ for some divisor $D$ iff it holds for $\struct{X}(D + [p])$. Consider the short exact sequence,
\begin{center}
\begin{tikzcd}
0 \arrow[r] & \struct{X}(D) \arrow[r] & \struct{X}(D + p) \arrow[r] & (\iota_p)_* \underline{\stalk{X}{p}} \arrow[r] & 0
\end{tikzcd}
\end{center}
Then we are done by induction because $\chi(X, (\iota_p)_* \underline{\stalk{X}{p}}) = 1$.
\end{proof}

\begin{lemma}
By Coherent duality,
\[ \dim_k H^1(X, \mathcal{L}) = \dim_k \Homover{\struct{X}}{\mathcal{L}}{\omega_X} = \dim_k H^0(X, \mathcal{L}^{\otimes -1} \otimes \omega_X) \]
\end{lemma}

\begin{theorem}[Riemann-Roch]
Let $X$ be a smooth projective curve over $k$, algebraically closed and $\mathcal{L} \in \Pic{X}$ then,
\[ \dim_k H^0(X, \mathcal{L}) = \dim_k H^0(X, \mathcal{L}^{\otimes -1} \otimes \omega_X) - \dim_k H^0(X, \omega_X) + 1 \]
\end{theorem}

\begin{proposition}
$\deg{(\omega_X)} = 2g - 2$
\end{proposition}

\begin{proof}
By Riemann-Roch, we have,
\[ \chi(X, \omega_X) = \deg{(\omega_X)} + 1 - g \]
However, 
\[ \dim_k H^0(X, \omega_X) = \dim_k H^1(X, \struct{X}) = g \]
and
\[ \dim_k H^1(X, \omega_X) = \dim_k H^1(X, \omega_X^{\otimes -1} \otimes \omega_X) = \dim_k H^0(X, \struct{X}) = 1  \]
Therefore,
\[ \chi(X, \omega_X) = g - 1\]
\end{proof}

\section{April 26}

\begin{theorem}
For a smooth projective curve $X$, we have $\omega_X = \Omega^1_X$.
\end{theorem}

\begin{theorem}
If $g : T \to W$ is a morphism of $S$-schemes then there is a cononical map $g^* \Omega_{W/S}^1 \to \Omega^1_{T/S}$. The corresponding diagram of rings,
\begin{center}
\begin{tikzcd}
B \arrow[from = rr, "\varphi"'] & & C 
\\
& A \arrow[ru] \arrow[lu]
\end{tikzcd}
\end{center}
gives the map,
\begin{align*}
\Omega_{C/A} \otimes_C B & \to \Omega_{B/A}
\\
(b_1 \d{b_2}) \otimes c & \mapsto c \varphi(b_1) \d{\varphi(b_2)} 
\end{align*}
\end{theorem}

\begin{theorem}[Riemann-Hurewitz]
If $f : X \to Y$ is a nonconstant morphism of smooth projective curves  over an algebraically closed field of characteristic zero then,
\[ 2 g_X - 2 = \deg{(f)} \cdot (2 g_Y - 2) + b \]
where $b$ is the total ramification,
\[ b = \sum_{x \in X(k)} (e_x - 1) \]
\end{theorem}

\begin{proof}
By RR $\deg{\omega_X} = 2 g_X - 2$ and $\deg{f^* \omega_Y} = \deg{(f)} \cdot (2 g_Y - 2)$. 
There is a morphism $\d{f} : f^* \Omega_Y^1 \to \Omega_X^1$ of invertible $\struct{X}$-modules. In characteristic zero $\d{f} \neq 0$ if $f$ is noncontant. To see this, consider the map of stalks at the generic point,
\[ (\Omega_{k(Y) / k} \otimes_{k(Y)} k(X) \to \Omega_{k(X) / k} \]
In characteristic zero these extensions are seperable so this is nonzero. 
\bigskip\\
If follows that $\Omega_X^1 \cong (f^* \Omega_Y)(R)$ where the ramification divisor is,
\[ R = \sum_{\d{f_x} = 0} v_x(\d{f}) [x] \]
Then we have,
\[ 2 g_X - 2 = \deg{(f)} \cdot (2 g_Y - 2) + \deg{(R)} \]
Take a closed point $x \in X$ with image $y \in Y$. Consider,
\[ f^\#_x : \stalk{Y}{y} \to \stalk{X}{x} \]
which is local map of DVRs because $X$ and $Y$ are smooth of dimension $1$. Now let $\varpi_y \in \stalk{Y}{y}$ and $\varpi_x \in \stalk{X}{x}$ be uniformizers then $f^\#_x(\varpi_y) = u \cdot \varpi_x^{e_x}$ with $u \in \stalk{X}{x}^\times$. Furthermore,
\[ \Omega_{X,x} = \stalk{X}{x} \d{\varpi_x} \]
using the fact that $\kappa(x) = k$. Then the local generator $f^* \d{\varpi_y}$ of $f^* \Omega_Y$ maps to,
\begin{align*}
\d{(u \varpi_x^{e_X})} = \varpi_x^{e_x} \d{u} + e_x \varpi_x^{e_x - 1} u \d{\varpi_x} = \left( \varpi_x^{e_x} \d{u} / \d{\varpi_x} + e_x \varpi_x^{e_x - 1} u \right) \d{\varpi_x} 
\end{align*}
Therefore, 
\[ \deg{(R)} \ge \sum_{x \in X(k)} (e_x - 1) \]
which equality in characteristic zero or tame in charactertistic $p$. 
\end{proof}


\begin{lemma}
If $K / L / k$ is an extension of fields with $K / L$ is algebraic seperable then,
\[ \Omega^1_{L / k} \otimes_L K = \Omega^1_{K / k} \]  
\end{lemma}

\begin{definition}
A morphism of smooth projective curves $f : X \to Y$ is \textit{unramified} if $e_x = 1$ for each $x \in X$.
\end{definition}

\begin{example}
$\pi_1(\P^1_\C) = \{1\}$ meaning there is no unramieifed nontrivial finite cover $f : X \to \P^1_\C$ because RH would give,
\[ 2 g_X - 2 = -2 \deg{(g)} < - 2 \]
but the genus cannot be negative. 
\end{example}

\section{Morphisms To Projective Space}

\begin{remark}
Let $S$ be a graded ring then recall that,
\[ X = \Proj{S} = \{ \p \in \Spec{S} \mid \p \text{ is homogeneous and } S_+ \not \subset \p \} \]
with closed sets,
\[ V(\a) = \{ \p \in \Proj{S} \mid \a \subset \p \} \]
and principal open sets,
\[ D(f) = \{ p \in \Proj{S} \mid f \notin \p \} \]
for homogeneous $f \in S_+$.
Furthermore the principal opens form an open cover of $\Proj{S}$ since,
\[ (D(f), \struct{X}|_{D(f)}) = \Spec{S_{(f)}} \]
where $S_{(f)}$ is the degree zero part of $S_f$.  
\bigskip\\
Given a graded $S$-module $M$ we can produce a $\struct{S}$-module $\widetilde{M}$ which satisfies the property $\widetilde{M} |_{D(f)} = \widetilde{(M_{(f)})}$ in the affine scheme sense. In particular, for $M = S$ we have $\struct{X} = \widetilde{S}$.
\bigskip\\
Now, twisting $S(n)_k = S_{k + n}$ we get the twisitng sheafs $\struct{X}(n) = \widetilde{S(n)}$ and for any $\struct{X}$-module $\F$ we have $\F(n) = \F \otimes_{\struct{X}} \struct{X}(n)$.
\end{remark}

\begin{proposition}
Let $S$ be generated by $S_1$ as a $S_0$-algebra. Then, $\struct{X}(n)$ is invertible and $\struct{X}(n + m) = \struct{X}(n) \otimes_{\struct{X}} \struct{X}(m)$. 
\end{proposition}

\begin{proof}
Take $f \in S_1$. We know that $\struct{X}(n)|_{D(f)} = \widetilde{S(n)_{(f)}}$ on $\Spec{S_{(f)}}$. However, $S(n)_{(f)}$ are the degree $n$ elements in $S_f$ so we have an isomorphism $S_(f) \to S(n)_{(f)}$ via $s \mapsto f^n s$ so $S(n)_{(f)}$ is a free rank one $S_{(f)}$-module. Furthermore, since $S_1$ generates $S$ as a $S_0$-algebra then $D(f)$ for $f \in S_1$ cover $X$ and thus $\struct{X}(n)$ is a line bundle.
\bigskip\\
The second fact follows from $\widetilde{M}(n) = \widetilde{M(n)}$ which follows from the fact that $M \otimes_S S(n) = M(n)$ and $\widetilde{M \otimes_S N} = \widetilde{M} \otimes_{\struct{X}} \widetilde{N}$. 
\end{proof}

\begin{definition}
We have $\P^N_A = \Proj{A[x_0, \dots, x_N]}$ for a ring $A$.
\end{definition}

\begin{definition}
Given a twisiting sheaf $\struct{X}(1)$ on a scheme $X$ (usually via a projective embedding $X \embed \P^n_A$) we define the graded ring,
\[ \Gamma_*(X, \F) = \bigoplus_{n \ge 0} \Gamma(X, \F(n)) \]
\end{definition}


\begin{proposition}
On $X = \P^N_A$ we have $\Gamma_*(X, \struct{X}) = A[x_0, \dots, x_N]$. 
Furthermore,
\[ \Gamma_*(X, \F(m)) = \bigoplus_{n \ge 0} \Gamma(X, \F(m + n)) = \Gamma_*(X, \F)(m) \]
which implies that $\Gamma_*(X, \struct{X}(n)) = A[x_0, \dots, x_N](n)$.  
\end{proposition}

\begin{proposition}
On $P^N_A$ the invertible sheaf $\struct{\P}(1)$ is generated by its global sections $x_0, \dots, x_N \in \Gamma(\P^N_A, \struct{\P}(1))$. Thus, for any morphism $\varphi : X \to \P^N_A$ the sheaf $\L = \varphi^* \struct{X}(1)$ is invertible and is generated by global sections $s_i = \varphi^*(x_i)$. 
\end{proposition}

\begin{definition}
If $\varphi$ is an immersion, we call the resulting line bundle $\L = \varphi^* \struct{\P}(1)$ \textit{very ample} on $X$ over $\Spec{A}$.
\end{definition}

\begin{theorem}
Let $X$ be a scheme over $A$ and $\L$ be an invertible sheaf on $X$ with global sections $s_0, \dots, s_N \in \Gamma(X, \L)$ which generate $\L$. Then there exists a morphism $\varphi : X \to \P^N_A$ such that $\L = \varphi^* \struct{\P}(1)$ and $s_i = \varphi^*(x_i)$. 
\end{theorem}

\begin{proof}
Since $\L$ is invertible on sufficiently small opens $\L |_U \cong \struct{X} |_U$ and thus $\L_x$ is local for each $x \in X$ which maximal ideal $\m_x \L_x = \m_x \otimes_{\stalk{X}{x}} \L_x$. 
Consider the open sets,
\[ D(s_i) = \{ x \in X \mid (s_i)_x \notin \m_x \L_x \} \]
Then there is a ring map $A[x_0, \dots, x_N]_{(x_i)} \to \Gamma(D(s_i), \struct{X})$ via,
\[ \frac{x_j}{x_i} \mapsto \frac{s_j}{s_i} \]
This is defined because $s_i \in \Gamma(D(s_i), \L)$ is invertible and since $\L$ is locally free of rank $1$ the quotient can be viewed as a well-defined section of $\struct{X}$ via the map $\struct{X} \to \L$ by $f \mapsto f s$ giving $s_j = f_j s$ and $s_i = f_i s$ and $f_i \in \Gamma(D(s_i), \struct{X})^\times$ so,
\[ \frac{s_j}{s_i} = \frac{f_j s}{f_i s} = \frac{f_j}{f_i} \in \Gamma(D(s_i), \struct{X}) \]
This ring map defines a morphism $D(s_i) \to \Spec{A[x_0, \dots, x_N]_{(x_i)}} = U_i \subset \P^N_A$. These morphisms glue to give the required map $X \to \P^N_A$. 
\end{proof}

\begin{definition}
A line bundle $\L$ on $X$ is \textit{ample} if for any coherent sheaf $\F$ there exists a positive integer $n(\F)$ such that $\F \otimes_{\struct{X}} \L^{\otimes n}$ is generated by global sections for all $n \ge n(\F)$.
\end{definition}

\begin{proposition}
Let $\L$ be a line bundle on $X$ then the following are equivalent,
\begin{enumerate}
\item $\L$ is ample
\item $\L^{\otimes m}$ is ample for \textit{all} positive $m$
\item $\L^{\otimes m}$ is ample for \textit{some} positive $m$
\end{enumerate}
\end{proposition}

\begin{proof}
If $\L$ is ample then $\F \otimes_{\struct{X}} (\L^{\otimes m})^{\otimes n} = \F \otimes_{\struct{X}} \L^{\otimes n m}$ is generated by global sections for any $nm \ge n(\F)$. 
\bigskip\\
It suffices to show that if $\L^{\otimes m}$ is ample for some $m$ then $\L$ is ample. Given a coherent sheaf $\F$ consider the coherent sheaf $(\F \otimes_{\struct{X}} \L^k)$ and then take $n(\F, k)$ such that,
\[ (\F \otimes_{\struct{X}} \L^k) \otimes_{\struct{X}} (\L^{\otimes m})^n = \F \otimes_{\struct{X}} \L^{\otimes (nm + k)} \]
is generated by global sections for $n \ge n(\F)$. Take $n(\F) = m \cdot \max\{ n(\F, k) + 1 \mid k \in \{ 0, 1, \dots, m - 1 \}$. Then for $n \ge n(\F)$ we can write, by the division algorthim $n = q m + k$ for $0 \le k < m$ and $m q \ge m \cdot n(\F, k)$ so $q \ge n(\F, k)$ which implies that,
\[ \F \otimes_{\struct{X}} \L^n = (\F \otimes_{\struct{X}} \L^k) \otimes_{\struct{X}} (L^{\otimes m})^{\otimes q} \]
is generated by global sections so $\L$ is ample. 
\end{proof}

\begin{theorem}
Let $X$ be finite type over a Noetherian $A$. A line bundle $\L$ on $X$ is ample iff there exists a positive integer $n$ such that $\L^{\otimes n}$ is very ample for $\Spec{A}$. 
\end{theorem}

\begin{theorem}[Serre]
Let $A$ be Noetherian and $X$ proper over $A$. Let $\L$ be an invertible sheaf on $X$. Then the following are equivalent,
\begin{enumerate}
\item $\L$ is ample
\item for each coherent sheaf $\F$ on $X$ there exists an integer $n(\F)$ such that for any $n \ge n(\F)$ and $i > 0$ we have $H^i(X, \F \otimes_{\struct{X}} \L^{\otimes n}) = 0$.
\end{enumerate}
\end{theorem}

\subsection{Linear Systems}

\section{Induced Subscheme Structure}

\begin{definition}
Given a closed subset $V \subset \Spec{A}$ we define the ideal,
\[ I_V = \bigcap_{\p \in V} \p \]
Then clearly $V(I_V) = V$ and $I_V$ is a radical ideal so if $V(\a) = V$ then $\a \subset I$.  
\end{definition}

\begin{definition}
Let $X$ be a scheme and $Z \subset X$ a closed subset. We define the reduced induced subscheme structure on $Z$ via (DO)
\end{definition}

\end{document}
