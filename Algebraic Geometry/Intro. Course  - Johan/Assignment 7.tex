\documentclass[12pt]{article}
\usepackage{import}
\import{./}{Includes}

\begin{document}

\atitle{7}

\section{Exercise 103.42.2}

Let $M$ be an $A$-module and $I = (f_1, \cdots, f_t)$ a finitely generated ideal. Now take $U = V(I)^C = D(f_1) \cup \cdots \cup D(f_t)$. Therefore, $D(f_i)$ gives an affine open cover of $U$ so we may apply Cech cohomology to compute sheaf cohomology of quasi-coherent sheaves. The Cech complex $\check{C}^\bullet(\{ D(f_i) \}, \widetilde{M})$ is,
\begin{center}
\begin{tikzcd}[column sep = small]
0 \arrow[r] & \prod\limits_{i = 1}^t \widetilde{M}(D(f_i)) \arrow[r] &  \prod\limits_{i < j}^t \widetilde{M}(D(f_i) \cap D(f_j)) \arrow[r] & \cdots \arrow[r] & \widetilde{M}(D(f_1) \cap \cdots \cap D(f_t)) \arrow[r] & 0
\end{tikzcd}
\end{center}
which is equal to,
\begin{center}
\begin{tikzcd}[column sep = small]
0 \arrow[r] & \prod\limits_{i = 1}^t \widetilde{M}(D(f_i)) \arrow[r] &  \prod\limits_{i < j}^t \widetilde{M}(D(f_i f_j)) \arrow[r] & \cdots \arrow[r] & \widetilde{M}(D(f_1 \cdots f_t)) \arrow[r] & 0
\end{tikzcd}
\end{center}
and therefore, using the defining property $\widetilde{M}(D(f)) = M_f$ the Cech complex becomes,
\begin{center}
\begin{tikzcd}
0 \arrow[r] & \prod\limits_{i = 1}^t M_{f_i} \arrow[r] &  \prod\limits_{i < j}^t M_{f_i f_j} \arrow[r] & \cdots \arrow[r] & M_{f_1 \cdots f_t} \arrow[r] & 0
\end{tikzcd}
\end{center}
whose cohomology gives the Cech cohomology and thus the sheaf cohomology,
\[ \check{H}(\{ D(f_i) \}, \widetilde{M}) \cong H(U, \widetilde{M}) \]
which agree in this case because $U$ is a separated scheme (see Tag 0BDX and Tag 01XD). To see why $U$ is separated apply Lemma \ref{subset_separated}, using the fact that $U$ is a subscheme of the affine scheme $\Spec{A}$ which is automatically separated. 

\section{Exercise 103.42.3}

It will be convenient to label variables as,
\[ \A^d_k = \Spec{k[x_0, \cdots, x_{d-1}]} \]
and $n = d-1$ to line up with the definitions in projective space. Consider the projection morphism $\pi : \A^{n+1}_k \setminus \{ (x_1, \dots, x_n) \} \to \P^{n}_k$ and let $U = \A^d_k \setminus \{ (x_1, \dots, x_n) \}$ and $X = \P^n_k$. The schemes $D_{+}(X_i)$ for each variable $X_i$ constitute an affine open cover of $\P^{n}_k$. Furthermore, $\pi^{-1}(D_{+}(X_i)) = D(x_{i}) \subset k[x_1, \dots, x_d]$. Therefore, $\pi$ is an affine morphism and $\struct{U}$ is a quasi-coherent $\struct{U}$-module so we have shown that,
\[ H^q(\P^{n}_k, \pi_* \struct{U}) = H^q(U, \struct{U}) \] 
Furthermore, denote $S = k[x_0, \cdots, x_n]$, then,
\begin{align*}
\pi_* \struct{U} |_{D_{+}(X_i)} & = \struct{U} |_{D(x_{i})} = \struct{\A^{n+1}_k} |_{D(x_{i})} = \widetilde{S_{x_{i}}}  = \bigoplus_{k \in \Z} \widetilde{\left( S_{x_i} \right)_k} = \bigoplus_{k \in \Z} \widetilde{\left( S(k)_{x_i} \right)_{0}} = \bigoplus_{k \in \Z} \struct{X}(k)|_{D_+(X_i)}
\end{align*}
Thus, because the sheaves agree on an open affine cover, we can identify,
\[ \pi_* \struct{U} = \bigoplus_{k \in \Z} \struct{X}(k) \]
Hartshorne has computed the cohomology of the sum of twists (Hartshorne III.5, Theorem 5.1) to be,
\[  H^q \left(X, \bigoplus_{n \in k} \struct{X}(k) \right) = 
\begin{cases}
k[X_0, \cdots, X_n] & q = 0
\\
0 & 0 < q < n
\\
\frac{1}{X_0 \cdots X_n} k[X_0^{-1}, \dots, X_n^{-1}] & q = n 
\end{cases} \]
Reverting to our initial notation and using the isomorphism $H^q(X, \pi_* \struct{U}) = H^q(U, \struct{U})$ we arrive at,
\[ H^q(U, \struct{U}) = 
\begin{cases}
k[x_1, \cdots, x_d] & q = 0
\\
0 & 0 < q < n
\\
\frac{1}{x_1 \cdots x_d} k[x_1^{-1}, \dots, x_d^{-1}] & q = d-1 
\end{cases} \]

\section{Exercise 103.42.4}

Let $k$ be a field and $Y = \P^1_k \times \P^1_k$. Let $\pi_i : Y \to \P^1_k$ be the projection maps. Now consider the invertable sheaves on $Y$,
\[ \struct{Y}(a,b) = \pi_1^* \struct{\P^1_k}(a) \otimes_{\struct{Y}} \pi_2^* \struct{\P^1_k}(b) \]
The K\"{u}nneth formula allows us to compute the cohomology of such sheaves via,
\[ H^n(Y, \struct{Y}(a,b)) = \bigoplus_{p + q = n} H^p(\P^1_k, \struct{\P^1_k}(a)) \otimes_k H^q(\P^1_k, \struct{\P^1_k}(b)) \]
However, we have computed the cohomology of the twists previously,
\[ H^p(\P^1_k, \struct{\P^1_k}(a)) = 
\begin{cases}
(k[X_0, X_1])_a & p = 0
\\
\left( \frac{1}{X_0 X_1} k[X_0^{-1}, X_1^{-1}] \right)_{a} & p = 1
\\
0 & p \neq 0,1
\end{cases} \]
Consider the case $a,b > 0$ then we have,
\[ H^n(Y, \struct{Y}(a,b)) = H^0(\P^1_k, \struct{\P^1_k}(a)) \otimes_k H^0(\P^1_k, \struct{\P^1_k}(b)) = (k[X_0, X_1])_a \otimes (k[Y_0, Y_1])_b \]
which is exactly the ring of bigraded polynomials of bidegree $(a, b)$. 
An injective $\struct{Y}$-module map $\struct{Y} \to \struct{Y}(a,b)$ is defined by $1 \mapsto F$ for some regular section $F \in H^0(Y, \struct{Y}(a,b))$ which is some $(a,b)$-bigraded polynomial. Then $\struct{Y}(a,b)$ and $F$ define an effective Cartier divisor $X \subset Y$ given by the vanishing of $F$ whose inverse ideal sheaf is $\struct{Y}(a,b)$ (see Tag 01X0) which is a locally principally closed subscheme here of codimension $1$. Since $\dim_k{Y} = 2$ we have $\dim_k{X} = 1$. Furthermore, there are closed immersions,
\begin{center}
\begin{tikzcd}
X \arrow[r, hook] & \P^1_k \times \P^1_k \arrow[r, hook] & \P^3_k
\end{tikzcd}
\end{center}
given by the Segre embedding showing that $X$ is a projective scheme over $k$. Since $\struct{Y}(a,b)$ is the inverse of the sheaf of ideals defining $X \subset Y$ there exists an exact sequence of $\struct{Y}$-modules,
\begin{center}
\begin{tikzcd}
0 \arrow[r] & \struct{Y}(-a,-b) \arrow[r] & \struct{Y} \arrow[r] & \iota_* \struct{X} \arrow[r] & 0
\end{tikzcd}
\end{center}
Since $\iota : X \to Y$ is a closed immersion and thus affine, we may identify $H^n(Y, \iota_* \struct{X}) = H^n(X, \struct{X})$, taking the long exact sequence of cohomology, we find,
\begin{center}
\begin{tikzcd}[column sep = small]
0 \arrow[r] & H^0(Y, \struct{Y}(-a,-b)) \arrow[r] & H^0(Y, \struct{Y}) \arrow[draw=none]{d}[name=Z, shape=coordinate]{}  \arrow[r] & H^0(X, \struct{X}) 
\arrow[dll,
rounded corners, crossing over,
to path={ -- ([xshift=2ex]\tikztostart.east)
|- (Z) [near end]\tikztonodes
-| ([xshift=-2ex]\tikztotarget.west)
-- (\tikztotarget)}]
\\ 
& H^1(Y, \struct{Y}(-a,-b)) \arrow[r] & H^1(Y, \struct{Y}) \arrow[draw=none]{d}[name=K, shape=coordinate]{}  \arrow[r] & H^1(X, \struct{X}) \arrow[dll,
rounded corners, crossing over,
to path={ -- ([xshift=2ex]\tikztostart.east)
|- (K) [near end]\tikztonodes
-| ([xshift=-2ex]\tikztotarget.west)
-- (\tikztotarget)}]
\\ 
& H^2(Y, \struct{Y}(-a,-b)) \arrow[r] & H^2(Y, \struct{Y})  \arrow[r] & H^2(X, \struct{X}) \arrow[r] & \cdots
\end{tikzcd}
\end{center}
Now $H^0(Y, \struct{Y}) = k$ and $H^p(Y, \struct{Y}) = 0$ for $p > 0$ and in the case $a, b > 0$ we have $H^0(Y, \struct{Y}(-a, -b)) = 0$ since there is no negative graded part of $k[X_0, X_1]$ and likewise,  \begin{align*}
H^1(Y, \struct{Y}(-a,-b)) = & H^0(\P^1_k, \struct{X}(-a)) \otimes_k H^1(\P^1_k, \struct{X}(-b)) 
\\
\oplus & H^1(\P^1_k, \struct{X}(-a)) \otimes_k H^0(\P^1_k, \struct{X}(-b)) = 0 
\end{align*}
since one of the factors is zero in both cases. Plugging into the long exact sequence gives exact sequences,
\begin{center}
\begin{tikzcd}
0 \arrow[r] & H^0(Y, \struct{Y}) \arrow[r] & H^0(X, \struct{X}) \arrow[r] & 0
\\
0 \arrow[r] & H^1(X, \struct{X}) \arrow[r] & H^1(Y, \struct{Y}(-a, -b)) \arrow[r] & 0
\end{tikzcd}
\end{center}
which are thus isomorphisms. Finally,
\begin{align*}
\dim_k H^2(Y, \struct{Y}(-a,-b)) & = \dim_k \left( H^1(\P^1_k, \struct{X}(-a)) \otimes_k H^1(\P^1_k, \struct{X}(-b)) \right)  
\\
& = \dim_k \left( \frac{1}{X_0 X_1} k[X_0^{-1}, X_1^{-1}] \right)_{-a} \cdot \dim_k \left( \frac{1}{Y_0 Y_1} k[Y_0^{-1}, Y_1^{-1}] \right)_{-b} \\
& = (-a + 1)(-b + 1) = (a - 1)(b - 1)
\end{align*}
and thus,
\begin{align*}
H^0(X, \struct{X}) & = H^0(Y, \struct{Y}) = k 
\\
H^1(X, \struct{X}) & = H^2(Y, \struct{Y}(-a,-b)) \implies \dim_k H^1(X, \struct{X}) = (a - 1)(b - 1)
\end{align*}
For example, take the $(11, 11)$-bigraded polynomial,
\[ F = X_0^5 X_1^6 Y_0^6 Y_1^5 + X_0^6 X_1^5 Y_0^5 Y_1^6 \in H^0(X, \struct{X}(11, 11)) \]
Then the curve $X = V(F) \subset \P^1_k \times \P^1_k$ defined by the vanishing of $F$ has,
\begin{align*}
H^0(X, \struct{X}) & = k
\\
\dim_k H^1(X, \struct{X}) & = (11 - 1)(11 - 1) = 100 
\end{align*}
and $X$ is a projective scheme over $k$ of dimension $1$. 

\section{Exercise 103.42.6}

Let $X$ be a locally ringed space. Notate by $\struct{X}^\times$, the sheaf of abelian groups given by $U \mapsto \struct{X}(U)^\times$. Now let $\L$ be an invertable sheaf on $X$ meaning that there exists an open cover $\mathfrak{U}$ such that for each $U \in \mathfrak{U}$ we have isomorphisms $\varphi_U : \struct{X}|_U \to \L|_U$. Therefore, on the overlaps we have isomorphism,
\[ \varphi_{ij} = \varphi_{U_i}^{-1}|_{U_i \cap U_j} \circ \varphi_{U_j} |_{U_i \cap U_j} : \struct{X} |_{U_i \cap U_j} \to \struct{X} |_{U_i \cap U_j} \]
which, as $\struct{X}|_{U_i \cap U_j}$-module maps are determined uniquely by $e_{ij} \in \struct{X}(U_i \cap U_j)^\times$ which is a unit because the map it defines is an isomorphism. Thus, $e = (e_{ij})_{ij}$ is an element of the second Cech complex group, $\check{C}^2(\mathfrak{U}, \struct{X}^\times)$. Consider the Cech complex,
\begin{center}
\begin{tikzcd}
0 \arrow[r] & \prod\limits_{i_0} \struct{X}^\times(U_{i_0}) \arrow[r] & \prod\limits_{i_0 < i_1}  \struct{X}^\times(U_{i_0} \cap U_{i_1}) \arrow[r] & \prod\limits_{i_0 < i_1 < i_2} \struct{X}^\times(U_{i_0} \cap U_{i_1} \cap U_{i_2}) 
\end{tikzcd}
\end{center}
Furthermore, on triple overlaps,
\begin{align*}
\varphi_{ij}|_{ijk} \circ \varphi_{jk}|_{ijk} & = \varphi_{U_i}^{-1}|_{U_{ijk}} \circ \varphi_{U_j} |_{U_{ijk}} \circ \varphi_{U_j}^{-1}|_{U_{ijk}} \circ \varphi_{U_k} |_{U_{ijk}} 
\\
& = \varphi_{U_i}^{-1}|_{U_{ijk}} \circ \varphi_{U_k} |_{U_i \cap U_j \cap U_k} = \varphi_{ik} |_{ijk} 
\end{align*}
which clearly implies that $e_{ij} |_{U_{ijk}} \cdot e_{jk} |_{U_{ijk}} = e_{ik} |_{U_{ijk}}$. However, the Cech differential map $\mathrm{d} : \check{C}^1(\mathfrak{U}, \struct{X}^\times) \to \check{C}^2(\mathfrak{U}, \struct{X}^\times)$ acts via,
\[ (\d{\alpha})_{ijk} = \alpha_{jk} |_{U_{ijk}} \cdot \alpha_{ik}^{-1} |_{U_{ijk}} \cdot \alpha_{ij} |_{U_{ijk}} \]
Therefore, by the overlap identity,
\[ (\d{e})_{ijk} = e_{jk}|_{U_{ijk}} \cdot e_{ik}|_{U_{ijk}}^{-1} \cdot e_{ij} |_{U_{ijk}} = 1 \]
Thus $e$ is in the kernel of the Cech differential $\mathrm{d} : \check{C}^1(\mathfrak{U}, \struct{X}^\times) \to \check{C}^2(\mathfrak{U}, \struct{X}^\times)$ and thus $e$ represents a Cech cohomology class $[e] \in \check{H}^1(\mathfrak{U}, \struct{X}^\times)$. Furthermore, if $\tilde{\varphi}_{U_i} : \struct{X} |_{U_i} \to \L |_{U_i}$ is another choice of locally trivializing isomorphisms then denote $\tilde{e}_{ij} \in \struct{X}^\times(U_i \cap U_j)$ for the element determining the isomorphisms,
\[ \tilde{\varphi}_{ij} = \tilde{\varphi}_{U_i}^{-1} |_{U_{ij}} \circ \tilde{\varphi}_{U_j} |_{U_{ij}} : \struct{X} |_{U_i \cap U_j} \to \struct{X} |_{U_i \cap U_j} \]
Then we may consider the isomorphisms $t_i = \tilde{\varphi}_{U_i}^{-1} \circ \varphi_{U_i} : \struct{X} |_{U_i} \to \struct{X} |_{U_i}$ which are defined by an element $f_i \in \struct{X}^\times(U_i)$. Then we find that,
\begin{align*}
\tilde{\varphi}_{ij} & = \tilde{\varphi}_{U_i}^{-1} |_{U_{ij}} \circ \tilde{\varphi}_{U_j} |_{U_{ij}} = \tilde{\varphi}_{U_i}^{-1} |_{U_{ij}} \circ \varphi_{U_i} |_{U_{ijk}} \circ \varphi_{U_i}^{-1} |_{U_{ijk}} \circ \varphi_{U_j} |_{U_{ijk}} \circ  \varphi_{U_j} |_{U_{ijk}}^{-1} \circ \tilde{\varphi}_{U_j} |_{U_{ij}} 
\\
& = t_i |_{U_{ij}} \circ \varphi_{ij} \circ t_j^{-1} |_{U_{ij}}
\end{align*}
This shows that the elements must satisfy, $\tilde{e}_{ij} \cdot e_{ij}^{-1} = t_i |_{U_{ij}} \cdot t_j^{-1} |_{U_{ij}}$. Furthermore, the Cech differential map $\mathrm{d} : \check{C}^0(\mathfrak{U}, \struct{X}^\times) \to \check{C}^1(\mathfrak{U}, \struct{X}^\times)$ acts via,
\[ (\d{\alpha})_{ij} = \alpha_{i} |_{U_{ij}} \cdot \alpha_{j}^{-1} |_{U_{ij}}  \]
Therefore, let $f = (f_i)_i$ then $\d{f} = \tilde{e} \cdot e^{-1}$ which implies that $[\tilde{e}] = [e]$ in $\check{H}^1(\mathfrak{U}, \struct{X}^\times)$ so the cohomology class $[e]$ associated to the invertable sheaf $\L$ is well-defined. The map $\L \mapsto [e]$ is well-defined for sheaves which are locally trivialized on $\mathfrak{U}$. Therefore we get a well-defined map,
\[ \Pic{X} \to \check{H}^1(X, \struct{X}^\times) = \varinjlim_{\mathfrak{U}} \check{H}(\mathfrak{U}, \struct{X}^\times) \]
via decomposing,
\[ \Pic{X} = \bigcup_{\mathfrak{U}} \Pic{\mathfrak{U}, X} \quad \text{where} \quad \Pic{\mathfrak{U}, X} = \{ \mathcal{L} \in \Pic{X} \mid \forall U \in \mathfrak{U} : \mathcal{L}|_U \cong \struct{U} \} \]
and mapping,
\[ \Pic{\mathfrak{U}, X} \to \check{H}^1(\mathfrak{U}, \struct{X}^\times) \to \varinjlim_{\mathfrak{U}} \check{H}(\mathfrak{U}, \struct{X}^\times) = \check{H}^1(X, \struct{X}^\times) \]
using the constructed map. This map is an homomorphism because given invertable sheaves $\L_1$ and $\L_2$ and isomorphisms $\varphi^r_{U_i} : \struct{X} |_{U_i} \to \L_r$ corresponding to cohomology classes $[e^r]$ then there is a natural map,
\[ \varphi^1_{U_i} \otimes \varphi^2_{U_i} \struct{X}|_{U_i} \to \L_1 |_{U_i} \otimes_{\struct{X} |_{U_i}} \L_2 |_{U_i} \]
which therefore gives overlap maps,
\[ \varphi_{ij}^\otimes = ((\varphi^1_{U_i})^{-1} \circ \varphi^1_{U_j}) \otimes ((\varphi^2_{U_i})^{-1} \circ \varphi^2_{U_j}) = \varphi_{ij}^1 \otimes \varphi_{ij}^2 \]
and thus, $\varphi_{ij}^\otimes(1) = e_{ij}^1 \otimes e^2_{ij} \mapsto e_{ij}^1 e_{ij}^2$ under the natural identification,
\[ \struct{X}(U_{ij}) \otimes_{\struct{X}(U_{ij})} \struct{X}(U_{ij}) \to \struct{X}(U_{ij}) \]
Therefore, the invertable sheaf $\L_1 \otimes_{\struct{X}} \L_2$ maps to the cohomology class $[e^1 e^2] = [e^1] [e^2]$ so this map is a homomorphism. 
\bigskip\\
I claim that this map is, in fact, an isomorphism. Let $\L$ be an invertable sheaf represented by the cohomology class $[e] = [1]$ then we know that $e_{ij} = t_i |_{U_{ij}} \cdot t_j^{-1} |_{U_{ij}}$ for some set of invertable sections $t_i$. Therefore, modify the isomorphism $\varphi_{U_i} : \struct{X}|_{U_i} \to \L |_{U_i}$ which gave rise to this cohomology representative via $\tilde{\varphi}_{U_i} = t_i \varphi_{U_i}$ which are still isomorphism because $t_i \in \struct{X}(U_i)^\times$ is invertable. Therefore, 
\[ \tilde{\varphi}_{U_i}^{-1}|_{U_{ij}} \circ \tilde{\varphi}_{U_j}|_{U_{ij}} = (t_i |_{U_{ij}}^{-1} \cdot t_j |_{U_{ij}}) \varphi_{U_i}^{-1}|_{U_{ij}} \circ \varphi_{U_j}|_{U_{ij}} = \id_{\struct{X}(U_{ij})} \]
this map takes $1 \mapsto (t_i |_{U_{ij}}^{-1} \cdot t_j |_{U_{ij}}) e_{ij} = 1$ so as a morphism of $\struct{X}|_{U_{ij}}$-modules is the identity map. Thus $\tilde{\varphi}_{U_i} |_{U_{ij}} = \tilde{\varphi}_{U_j} |_{U_{ij}}$, so the isomorphisms $\tilde{\varphi}_{U_i} \in \shHom{\struct{X}|_{U_i}}{\L|_{U_i}}$ glue since they agree on this open cover to a global isomorphism $\tilde{\varphi} : \struct{X} \to \L$ so $\L$ is a trivial invertable sheaf. Thus $\Pic{X} \to \check{H}^1(X, \struct{X})$ is injective. It remains to prove that it is surjective. Given any cohomology class $[e] \in \check{H}^1(X, \struct{X}^\times)$ we may construct an invertable sheaf as follows. Define $\L$ via,
\[ \L(V) = \{ f_i \in \struct{X}(U_i \cap V) \mid f_i |_{U_{ij} \cap V} \cdot e_{ij} |_{U_{ij} \cap V} = f_j |_{U_{ij} \cap V} \} \]
It is clear that this is an invertable sheaf if $e_{ij}$ satisfies the transition property given by its Cech differential vanishing and that $\L \mapsto [e]$. 
\bigskip\\
Finally, we use the general fact that $H^1(X, \F) = \check{H}^1(X, \F)$ to conclude that,
\[ \Pic{X} \cong H^1(X, \struct{X}^\times) \]

\section{Exercise 103.42.7}

On a previous homework assignment we showed that the affine variety,
\[ X   = \Spec{k[x,y] / (y^2 - f(x))} \]
where $k$ is a field and $f(x) = (x - t_1) \cdots (x - t_n)$ for $n \ge 3$ and odd admits nontrivial invertable sheaves so that $\Pic{X}$ is nontrivial. Thus $H^1(X, \struct{X}^\times) = \Pic{X}$ which implies that $H^1(X, -)$ is not the zero functor even though $X$ is affine. 

\newpage

\section{Lemmas}

\begin{lemma} \label{subset_separated}
Let $X$ be a separated scheme and $Z \to X$ an injection then $Z$ is separated.
\end{lemma}

\begin{proof}
Consider the map,
\begin{center}
\begin{tikzcd}
Z \arrow[r, hook] & X \arrow[r] & \Spec{\Z}
\end{tikzcd}
\end{center}
The second map is separated by definition. The first map is separated because it is an injection (see Tag 0DVA). Since the composition of separated maps is separated, then $Z$ is a separated scheme. 
\end{proof}
\end{document}
