\documentclass[12pt]{article}
\usepackage{import}
\import{./}{Includes}

\begin{document}

\section{Jan. 30 Tag 0280}

(DO EIGHT EXERCISES)

\section{Feb. 6}

\subsection{Tag 028O}

(DO FOUR EXERCISES)

\subsection{Tag 02A1}

\begin{exr}
Let $n \ge 1$ be an integer. Find a surjective morphism $X \to \P^n_{k}$ where $X$ is affine.
\end{exr}
\noindent\\
A ``dumb'' solution is as follows. Let $U_i = D_+(T_i)$ be the standard open cover of $\P^n_k$. Then there is a surjective map,
\[ \coprod_{i = 0}^n U_i \onto \P^n_k \]
However, it would be more interesting to give an example where $X$ is an (irreducible) $k$-variety. 
\bigskip\\
(DO THIS)

\section{Feb. 13}

\subsection{Tag 029Q}

\begin{exr}
Let $X$ be a scheme and $x, x' \in X$. Let $\F$ be a quasi-coherent sheaf of $\struct{X}$-modules. If $x$ is a specialization of $x'$ and $\F_{x'} \neq 0$ then show that $\F_x \neq 0$.
\end{exr}
\noindent\\
The terminology means that $x \in \overline{x'}$. Then for any open $x \in U$ we must have $x' \in U$ else $U^C$ would be a closed containing $x'$ not containing $x$. Now, choose some affine open $x \in U = \Spec{A}$ such that $\F |_U = \wt{M}$ and primes $\p = x$ and $\q = x'$ so $\p \supset \q$. Then $\F_{x} = M_{\p}$ and $\F_{x'} = M_{\q} \neq 0$. However, since $\p \supset \q$ then $M_{\q} = (M_\p)_\q$ so $\F_{x} = M_\p \neq 0$.

\subsection{Tag 069T}

\begin{exr}
Show that the composition of affine morphisms is affine.
\end{exr}
\noindent\\
I think this is trivial. If $f : X \to Y$ and $g : Y \to Z$ are affine and $U \subset Z$ is affine open then $(g \circ f)^{-1}(U) = f^{-1}(g^{-1}(U))$ but $g^{-1}(U)$ is affine open since $g$ is affine and thus $f^{-1}(g^{-1}(U))$ is affine open since $f$ is affine so $g \circ f$ is affine. 

\subsection{Tag 028Z}

\begin{exr}
Find a morphism of integral schemes $f : X \to Y$ such that $f^\# : \stalk{Y}{f(x)} \to \stalk{X}{x}$ are surjective for all $x \in X$ but $f$ is not a closed immersion. 
\end{exr}
\noindent\\
Consider the map $\Z \embed \Q$ which gives $\Spec{\Q} \to \Spec{\Z}$. Then, the stalk map at the unique point $(0) \in \Spec{\Q}$ is $\Z_{(0)} \to \Q_{(0)}$ which is simply the identity $\Q \to \Q$ and thus surjective. However, $\Z \to \Q$ is not surjective so $\Spec{\Q} \to \Spec{\Z}$ is not a closed immersion. 
\bigskip\\
Notice that if $\stalk{Y}{y} \to (f_* \struct{X})_{y}$ were always surjective then the map of sheaves $\struct{Y} \to f_* \struct{X}$ would necessarily be surjective. However, the map in question is $\stalk{Y}{f(x)} \to (f_* \struct{X})_{f(x)} \to \stalk{X}{x}$. Note, in our example, that we have the map of sheaves $\wt{\Z} \to \wt{\Q}$ which on stalks at $(p) \in \Spec{\Z}$ gives $\Z_{(p)} \to \Q$ which is not surjective but $\Z_{(0)} \to \Q$ is at the point in the image.
\bigskip\\
Consider $f : \A^1_k \to \A^1_k$ sending every point to zero, i.e. the map $k[x] \to k[x]$ sending $x \mapsto 0$. Then $f_* \struct{X} = \wt{M}$ where $M = k[x]$ as an abelian group but has multiplication defined by $k[x] \to  k[x]$ sending $x \mapsto 0$. Then the map $\stalk{Y}{(x)} \to M_{(x)}$ is $k[x]_{(x)} \to k[x]$ sending $x \mapsto 0$. Then $\stalk{Y}{(x)} \to \stalk{X}{\p}$ is $k[x]_{(x)} \to k[x]_{\p}$ sending $x \mapsto 0$. This gives an example where $(f_* \struct{X})_{f(x)}$ and $\stalk{X}{x}$ differ.

\subsection{Tag 0293}

(DO TWO EXERCISES)

\section{Feb. 20}

\subsection{Tag 02A3}

\begin{exr}
Given examples of $R$ such that on $\Proj{R}$,
\begin{enumerate}
\item $\struct{X}(1)$ is not an invertible $\struct{X}$-module.
\item $\struct{X}(1)$ is invertible but the natural map $\struct{X}(1) \otimes_{\struct{X}} \struct{X}(1) \to \struct{X}(2)$ is not an isomorphism.
\end{enumerate}
\end{exr}
\noindent
Consider the ring $R = k[x_0, x_1]$ graded with $x_i$ in degree $2$. 

\subsection{Tag 02AL}

(VERY LONG)

\subsection{Tag 0DT4}

\begin{exr}
Give an example of a scheme $X$ and a nontrivial invertible $\struct{X}$-module $\L$ such that $H^0(X, \L)$ and $H^0(X, \L^{\otimes -1})$ are nonzero.  
\end{exr}
\noindent\\
I will give two examples. First, consider the number field $K = \Q(\sqrt{-5})$ with ring of integers $\struct{K} = \Z[\sqrt{-5}]$. Then the class group $\Cl{\struct{K}} = \Pic{\struct{K}} = \Z / 2 \Z$ with generator $J = (2, 1 + \sqrt{-5})$. Thus, $J = J^{\otimes - 1}$ and $H^0(\Spec{\struct{K}}, \wt{J}) = J$ is nontrivial and its dual is the same module. 
\bigskip\\

(NO, NOT POSSIBLE FOR SMOOTH CURVES PROVE THIS!!!!, FIND EXAMPLE)
For the second example, let $(X, O)$ be an elliptic curve over $k = \overline{k}$. Riemann-Roch on $X$ applied to a line bundle $\L$ with $\deg{\L} = 0$ shows that,
\[ \dim_k H^0(X, \L) - \dim_k H^0(X, \omega_X \otimes_{\struct{X}} \L^\vee) = 0 \]
but on an elliptic curve $\omega_X = \struct{X}$ so we find,
\[ \dim_k H^0(X, \L) = \dim_k H^0(X, \L^{\otimes -1}) \]
Therefore, it suffices to find a line bundle on $X$ with $\deg{\L} = 0$ and $H^0(X, \L) \neq 0$. 
(FINISH THIS)




The isomorphism $X \to \mathrm{Pic}^0(X)$ shows that, for points, $P,Q \in X$, we have,
\[ [P] + [Q] \sim [P + Q] + [O] \]
and thus,
\[ [P] + [Q] - [P + Q] - [O] \sim 0 \]

\subsection{Tag 029U}

\begin{exr}
Let $X = \Spec{R}$ be an affine scheme.
\begin{enumerate}
\item Let $f \in R$ and $\G$ be a quasi-coherent sheaf of $\struct{D(f)}$-modules on the open subscheme $D(f)$. Show that $\G = \F |_{D(f)}$ for some coherent sheaf of $\struct{X}$-modules $\F$.
\item Let $I \subset R$ be an ideal. Let $\iota : Z \embed X$ be the closed subscheme of $X$ corresponding to $I$. Let $\G$ be a quasi-coherent sheaf of $\struct{Z}$-modules on the closed subscheme $Z$. Show that $\G = \iota^* \F$ for some quasi-coherent sheaf of $\struct{X}$-modules $\F$. (Why is this silly?).
\item Assume that $R$ is Noetherian. Let $f \in R$ and $\G$ be a coherent sheaf of $\struct{D(f)}$-modules on the open subscheme $D(f)$. Show that $\G = \F |_{D(f)}$ for some coherent sheaf of $\struct{X}$-modules $\F$.
\end{enumerate}
\end{exr}
\noindent\\
For the first, we know $\G = \wt{M}$ for some $R_f$-module $M$ since $D(f) = \Spec{R_f}$ is affine. Then consider $M$ as a $A$-module and take $\F = \wt{M}$ then $\F |_{D(f)} = \wt{M \otimes_R R_f} = \wt{M}$ since $M$ is an $R_f$-module.
\bigskip\\
Any quasi-coherent sheaf $\G$ on $Z = \Spec{R/I}$ is $\wt{M}$ for some $A/I$-module $M$. Then consider $M$ as a $A$-module giving a quasi-coherent sheaf $\F = \wt{M}$ on $\Spec{A}$. Then $\iota^* \F = \wt{M \otimes_A A/I} = \wt{M}$ since $M$ is an $A/I$-module. This is ``silly'' because in general for a closed immersion $\iota : Z \embed X$ we have $\iota^* \iota_* \G = \G$. 
\bigskip\\
Now let $R$ be Noetherian. Any coherent $\struct{D(f)}$-module is of the form $\G = \wt{M}$ for a finitely generated $R_f$-module $M$. We need a coherent sheaf $\F = \wt{N}$ for some finitely generated $R$-module $N$ such that $N \otimes_R R_f = M$.  

\subsection{Tag 0D8V}

\begin{exr}
Let $A$ be a ring and $\P^n_A = \Proj{A[T_0, \dots, T_n]}$ be projective space over $A$. Let $\A^{n+1}_A = \Spec{A[T_0, \dots, T_n]}$ and let,
\[ U = \bigcup_{i = 0}^n D(T_i) \subset \A_A^{n+1} \]
be the complement of the closed point $(T_0, \dots, T_n) \in \A_A^{n+1}$. Construct an affine surjective morphism,
\[ f : U \to \P^n_A \]
such that $f_* \struct{U} = \bigoplus_{d \in \Z} \struct{\P^n_A}(d)$. More generally, show that for any graded $A[T_0, \dots, T_n]$-module $M$ we have,
\[ f_*(\wt{M}|_U) = \bigoplus_{d \in \Z} \wt{M(d)} \]
where on the left we have the quasi-coherent sheaf $\wt{M}$ associated to $M$ on $\A^{n+1}_A$ and on the right we have the quasi-coherent sheaves $\wt{M(d)}$ associated to the graded module $M(d)$. 
\end{exr}
\noindent
Consider the line bundle $\struct{U}$ on $U$ and choose sections $s_i = T_i$ which generate $\struct{U}$ globally since these do not all vanish on $U$. These sections define the map $U \to \P^n_A$ which satisfies $f^{-1}(D_{+}(T_i)) = D(T_i)$ and thus is affine. On the affine opens $D_+(T_i)$ this is given by the ring map,
\[ A[T_0/T_i, \dots, T_n/T_i] \to A[T_0, \dots, T_n, T_i^{-1}] \]
which is not surjective because this is not a closed immersion but it is injective so the associated map,
\[ \Spec{A[T_0, \dots, T_n, T_i^{-1}]} \to \Spec{A[T_0/T_i, \dots, T_n/T_i]} \]
is dominant and, in fact, surjective because (SHOW)
\bigskip\\
Now, suppose that $M$ is an $A[T_0, \dots, T_n]$-module. We can check the equality of sheaves on the affine cover $D_+(T_i)$. The map $f : D(T_i) \to D_{+}(T_i)$ takes $\wt{M}|_{D(T_i)} = \wt{M_{T_i}} \mapsto \wt{M_{T_i}}$ viewing  $M_{T_i}$ as a $A[T_0/T_i, \dots, T_n/T_i]$-module. Since $M$ is a graded $A[T_0, \dots, T_n]$-module we can wrtie,
\[ M_{T_i} = \bigoplus_{d \in \Z} (M_{T_i})_d = \bigoplus_{d \in \Z} (M(d))_{(T_i)} \]
since the degree $d$ part of $M_{T_i}$ is the degree zero part of $(M(d))_{T_i}$
Therefore,
\[ f_*(\wt{M}|_{D(T_i)}) = \bigoplus_{d \in \Z} \wt{M(d)} |_{D_+(T_i)} \]  
and thus, globally,
\[ f_*(\wt{M}|_U) = \bigoplus_{d \in \Z} \wt{M(d)} \]
Alternatively, we can use the fact that for any quasi-coherent $\struct{\P^n_A}$-module $\F$ we have $\wt{\Gamma_*(\F)} \xrightarrow{\sim} \F$ and since $f : U \to \P^n_A$ is quasi-compact and quasi-separated then $f_*(\wt{M}|_U)$ is quasi-coherent so,
\[ f_*(\wt{M}|_U) = \wt{\Gamma_*(f_*(\wt{M}|_U))} = \bigoplus_{d \in \Z} \wt{\Gamma(\P^n_A, f_*(\wt{M}|_U)(d))} = \bigoplus_{d \in \Z} \wt{M(d)} \]

\section{Feb. 27}

\subsection{Tag 0D8Q}

\newcommand{\R}{\mathbb{R}}

\begin{exr}
Let $X = \R$ with the usual Euclidean topology. Using only formal $\delta$-functor properties of cohomology show that there exists a sheaf $\F$ on $X$ with nonzero $H^1(X, \F)$.
\end{exr}

Let $p, q \in \R$ be two distinct points and consider the inclusions $\iota_x : \{ x \} \to \R$ at the point $x$. Then let $\F = \underline{\Z}$ be the constant sheaf $\Z$ over $\R$ and $\G = (\iota_p)_*(\underline{\Z}) \oplus (\iota_p)_*(\underline{\Z})$ be the sum of skyscraper sheaves at $p$ and at $q$. Then there is a surjection of sheaves $\F \to \G$ determined on the open cover $U = (-\infty, \tfrac{1}{2}(p + q))$ and $V = (\tfrac{1}{2}(p + q), \infty)$ by the identity map $\Z \to \Z$ on both since each open set contains exactly one of $p$ and $q$. This map is a surjection because $f_p : \F_p \to \G_p$ and $f_q : \F_q \to \G_q$ are both the identity maps $\Z \to \Z$ and $f_x : \F_x \to \G_x$ is the zero map for $x \neq p,q$. Taking the kernel of this map gives an exact sequence of sheaves,
\begin{center}
\begin{tikzcd}
0 \arrow[r] & \mathcal{K} \arrow[r] & \F \arrow[r] & \G \arrow[r] & 0
\end{tikzcd}
\end{center}
which gives a long exact sequence of cohomology,
\begin{center}
\begin{tikzcd}
0 \arrow[r] & \Gamma(\R, \mathcal{K}) \arrow[r] & \Gamma(\R, \F) \arrow[r] & \Gamma(\R, \G) \arrow[r] & H^1(\R, \mathcal{K}) \arrow[r] & \cdots
\end{tikzcd}
\end{center}
However, $\Gamma(\R, \F) = \Z$ and $\Gamma(\R, \G) = \Gamma(\R, (\iota_p)_*(\underline{\Z})) \oplus \Gamma(\R, (\iota_q)_*(\underline{\Z})) = \Z \oplus \Z$. In particular, there is an exact sequence,
\begin{center}
\begin{tikzcd}
\Z \arrow[r] & \Z \oplus \Z \arrow[r] & H^1(\R, \mathcal{K}) 
\end{tikzcd}
\end{center}
Since there does not exist a surjection $\Z \to \Z \oplus \Z$ the map $\Gamma(\R, \G) \to H^1(X, \mathcal{K})$ cannot be the zero map so $H^1(\R, \mathcal{K})$ must be nontrivial. 


\subsection{Tag 0D8R}
(See Assigment 6)

\subsection{Tag 0D8S}

\begin{exr}
Show that if $X$ has two or fewer points then $H^i(X, \F) = 0$ for all $i > 0$ and any $\F$ on $X$. What about if $X$ has three points.
\end{exr}
(DO THIS!!!!)

\subsection{Tag 0D8T}

\begin{exr}
Let $X = \Spec{R}$ for a local ring $R$. Then show that $H^i(X, \F) = 0$ for all $i > 0$ and any abelian sheaf $\F$.
\end{exr}
\noindent\\
Consider the maximal ideal $\m \in \Spec{R}$. Note that if $f \notin \m$ then $f \in R^\times$ so the only open set containing $\m$ is $X$. Thus,
\[ \F_{\m} = \varinjlim_{\m \in U} \F(U) = \F(X) \]
This means that given an exact sequence of abelian sheaves,
\begin{center}
\begin{tikzcd}
0 \arrow[r] & \F \arrow[r] & \G \arrow[r] & \H \arrow[r] & 0
\end{tikzcd}
\end{center}
it must be exact on the stalks. In particular,
\begin{center}
\begin{tikzcd}
0 \arrow[r] & \F_\m \arrow[r] & \G_\m \arrow[r] & \H_\m \arrow[r] & 0
\end{tikzcd}
\end{center}
is exact but this is just,
\begin{center}
\begin{tikzcd}
0 \arrow[r] & \F(X) \arrow[r] & \G(X) \arrow[r] & \H(X) \arrow[r] & 0
\end{tikzcd}
\end{center}
showing that $\Gamma(X, -)$ is an exact functor so $H^i(X, -) = 0$ for $i > 0$. 

\subsection{Tag 0DAK}

\begin{exr}
Let $A$ be a ring and $I = (f_1, \dots, f_t)$ a finitely generated ideal of $A$. Let $U \subset \Spec{A}$ be $V(I)^C$. For any $A$-module $M$ find a complex of $A$-modules (in terms of $A, f_1, \dots, f_n, M$) whose cohomology groups compute $H^n(U, \wt{M})$. 
\end{exr}
\noindent\\
The affine opens $D(f_i) \subset \Spec{A}$ cover $U$ since,
\[ \bigcup D(f_i) = D((f_1, \dots, f_n)) = D(I) \]
Furthermore, $D(f_i) \cap D(f_j) = D(f_i f_j)$ is affine so this gives a cover whose Cech cohomology computes the cohomology of quasi-coherent modules, in particular $\wt{M}$. Then, the ordered Cech complex gives,
\begin{center}
\begin{tikzcd}
0 \arrow[r] & \prod\limits_{i_0} M_{f_{i_0}} \arrow[r] & \prod\limits_{i_0 < i_1} M_{f_{i_0} f_{i_1}} \arrow[r] & \prod\limits_{i_0 < i_1 < i_2} M_{f_{i_0} f_{i_1} f_{i_2}} \arrow[r] & \cdots \arrow[r] & M_{f_1 \cdots f_t} \arrow[r] & 0
\end{tikzcd}
\end{center}
The cohomology of this complex computes $\check{H}^i(\mathfrak{U}, \wt{M}) = H^i(U, \wt{M})$. 

\section{March 5}

\subsection{Tag 0DB9}
(KUNNETH)

\subsection{Tag 0DBA}
(TWISTS ON $\P^1 \times \P^1$)

\subsection{Tag 0DCF}

Let $P_\F(t) = \chi(X, \F(t))$ be the Hilbert polynomial for $X = \P^n_k$ and $\F(t) = \F \otimes_{\struct{X}} \struct{X}(t)$. 

\begin{enumerate}
\item For $\F(-d)$ we have,
\[ P_{\F(-d)} = \chi(X, \F(t - d)) = P_{\F}(t - d) \]
\item If $\F = \F_1 \oplus \F_2$,
\[ P_\F(t) = \chi(X, \F(t)) = \chi(X, \F_1(t) \oplus \F_2(t)) = \chi(X, \F_1(t)) + \chi(X, \F_2(t)) = P_{\F_1}(t) + P_{\F_2}(t) \]
\item Let,
\begin{center}
\begin{tikzcd}
0 \arrow[r] & \F_1 \arrow[r] & \F_2 \arrow[r] & \F_3 \arrow[r] & 0
\end{tikzcd}
\end{center}
be an exact sequence of coherent sheaves. Then twising by $\struct{X}(t)$ gives an exact sequence,
\begin{center}
\begin{tikzcd}
0 \arrow[r] & \F_1(t) \arrow[r] & \F_2(t) \arrow[r] & \F_3(t) \arrow[r] & 0
\end{tikzcd}
\end{center}
Then we get,
\[ \chi(X, \F_2(t)) = \chi(X, \F_1(t)) + \chi(X, \F_3(t)) \]
and thus,
\[ P_{\F_2} = P_{\F_1} + P_{\F_2} \]
and in particular,
\[ P_{\F_1} = P_{\F_2} - P_{\F_3} \]
\end{enumerate}


\subsection{Tag 0DCG}

For $X = \P^n_k$ and $\F$ a coherent $\struct{X}$-module. Then we define the Hilbert polynomial,
\[ P_\F(t) = \chi(X, \F(t)) \] Note that,
\[ \chi(X, \struct{X}(t)) = { n + t \choose n } + (-1)^n { - t - 1 \choose n } \]
In particular, in the case $n = 1$ we have, 
\[ \forall t \in \Z : \chi(X, \struct{X}(t)) = t + 1 \]
Now we can take, $X = \P^1_k$ then take $\F = \struct{X}(-102)$ which has,
\[ P_{\F}(t) = \chi(X, \F(t)) = t - 101 \]
In general, on $X = \P^n_k$ the pushforward of this sheaf on any line will have $P(t) = t - 101$.


\subsection{Tag 0DCH}

Let $X = \P^2_k$ and consider,
\[ \F = \struct{X}(-3) \oplus \struct{H}(-2)^{\oplus 2} \]
for some hyperplane $H \subset X$. Then we get,
\begin{align*}
P_{\F}(t) & = \chi(X, \F(t)) = \chi(X, \struct{X}(t - 2)) = 2 \chi(X, \struct{H}(t - 2)) 
\\
& = { t - 2 \choose 2} + 2 { t - 1 \choose 1 } = \tfrac{1}{2}(t - 1) (t - 2) = \tfrac{1}{2} t^2 - \tfrac{3}{2} t + 1 + 2 (t - 1) 
\\
& = \tfrac{1}{2} t^2 + \tfrac{1}{2} t - 1
\end{align*} 

\section{March 12}

\subsection{Tag 0DD1}

\begin{exr}
Let $k$ be a field and $X = \P_k^3$. Let $L \subset X$ be a line and $P \subset X$ a plane defined as closed subschemes cut out by 1, resp., 2 linear equations. Compute,
\[ \Ext{i}{\struct{X}}{\struct{L}}{\struct{P}} \]
\end{exr}
\noindent\\


\subsection{Tag 0EEN}

Let $k$ be a field and $X = \P^1_k$. Let $\E$ be a finite locally free $\struct{X}$-module and $h^i(X, \E(d)) = \dim_k H^i(X, \E(d))$ where $\E(d) = \E \otimes_{\struct{X}} \struct{X}(d)$. Then for $\P^1_k$ we can always decompose,
\[ \E = \bigoplus_{i = 1}^n \struct{X}(q_i) \]
and then,
\[ H^p(X, \E(d)) = \bigoplus_i H^p(X, \struct{X}(q_i + d)) \]
Therefore,
\[ h^0(X, \E(d)) = \sum_{i = 1}^n \dim_k H^0(X, \struct{X}(q_i + d)) = \sum_{i = 1}^n
\begin{cases}
q_i + d + 1 & q_i \ge - d
\\
0 & q_i < -d
\end{cases} \]
and, by Serre duality,
\[ h^1(X, \E(d)) = \sum_{i = 1}^n \dim_k H^1(X, \struct{X}(q_i + d)) = \dim_k H^0(X, \struct{X}(- q_i - d - 2)) = \sum_{i = 1}^n
\begin{cases}
-q_i - d - 1 & q_i \le -d - 2
\\
0 & q_i > -d - 2
\end{cases} \]
Thus, we get some explicit formulae,
\begin{align*}
h^0(X, \E) & = \sum_{q_i \ge 0} (q_i + 1) 
\\
h^0(X, \E(1)) & = \sum_{q_i \ge -1} (q_i + 2)
\end{align*}
Notice that $h^0(X, \E) \le h^0(X, \E(1))$ showing that:
\begin{center}
(1) there does not exist $\E$ with $h^0(X, \E) = 5$ and $h^0(X, \E(1)) = 4$.
\end{center}
\noindent\\
Furthermore, we have explicit formulae,
\begin{align*}
h^1(X, \E) & = - \sum_{q_i \le -2} (q_i + 1) 
\\
h^1(X, \E(1)) & =  - \sum_{q_i \le -3} (q_i + 2)
\end{align*}
Notice that $h^1(X, \E) \ge h^1(X, \E(1))$ showing that:
\begin{center}
(1) there does not exist $\E$ with $h^1(X, \E) = 4$ and $h^1(X, \E(1)) = 5$.
\end{center}
Now we consider $\E$ satisfying the following conditions,
\begin{align*}
h^0(X, \E) & = 1 \quad h^1(X, \E) = 1
\\
h^0(X, \E(1)) & = 2 \quad h^1(X, \E(1)) = 0
\\
h^0(X, \E(2)) & = 4 \quad h^1(X, \E(2)) = a
\end{align*}
Choose an ordering $q_0 \ge q_1 \ge q_2 \ge \cdots \ge q_n$. The condition $h^0(X, \E) = 1$ implies that $q_0 = 0$ and $q_1 < 0$. Then $h^1(X, \E) = 1$ implies that $q_n = -2$ and $q_{n-1} > -2$. Therefore, we have,
\[ \E = \struct{X} \oplus \struct{X}(-2) \oplus \struct{X}(-1)^{\oplus s} \]
Therefore, $h^0(X, \E(1)) = 2 + s$ and $h^1(X, \E(1)) = 0$ but since $h^0(X, \E(1)) = 2$ so we find $s = 0$ and thus,
\[ \E = \struct{X} \oplus \struct{X}(-2) \]
Thus we get $h^0(X, \E(2)) = 3 + 1 = 4$ compatible with our requirement. Finally, we compute,
\[ h^1(X, \E(2)) = h^1(X, \struct{X}(2) \oplus \struct{X}) = 0 \]
so we must have $a = 0$. Note that we could have more easily done this by noting that,
\[ h^1(X, \E(2)) \le h^1(X, \E(1)) \]
in general but we require $h^1(X, \E(1)) = 0$ so $h^1(X, \E(2)) = 0$ giving $a = 0$. 


\subsection{Tag 0DT3}

\begin{exr}
Let $k$ be an algebraically closed field and $X$ a reduced, projective scheme over $k$ all of whose irreducible components all have dimension 1. Let $\omega_{X/k}$ be the relative dualizing module. Show that if $\dim_k H^1(X, \omega_{X/k}) > 1$ then $X$ is disconnected.
\end{exr}

(DO THIS!!)

\subsection{Tag 0EEP}

\begin{exr}
Let $X$ be a topological space which is the union $X = Y \cup Z$ of two closed subsets $Y$ and $Z$ whose intersection is $W = Y \cap Z$. Denote $i : Y \to X$ and $j : Z \to X$ and $k : W \to X$ the inclusion maps. 
\begin{enumerate}
\item Show that there is a short exact sequence of sheaves,
\begin{center}
\begin{tikzcd}
0 \arrow[r] & \underline{\Z}_X \arrow[r] & i_* \underline{\Z}_Y \oplus j_* \underline{\Z}_Z \arrow[r] & k_* \underline{\Z}_W \arrow[r] & 0
\end{tikzcd}
\end{center}
\item What can you conclude about the cohomology of $X, Y, Z, W$ with $\Z$-coefficients. 
\end{enumerate}
\end{exr}

(DO THIS)

\subsection{Tag 0EEQ}

\begin{exr}
Let $k$ be a field and $A = k[x_1, x_2, x_3, \dots]$ and $\m = (x_1, x_2, x_3, \dots)$. Let $X = \Spec{A}$ and $U = X \setminus \{ \m \}$. Compute $H^i(U, \struct{U})$.
\end{exr}
\noindent\\
Consider the Cech complex for the cover $U_i = D(x_i)$,
\begin{center}
\begin{tikzcd}
0 \arrow[r] & \prod\limits_{i_0} A_{x_i} \arrow[r] & \prod\limits_{i_0 < i_1} A_{x_{i_0} x_{i_1}} \arrow[r] & \prod\limits_{i_0 < i_1 < i_2} A_{x_{i_0} x_{i_1} x_{i_2}} \arrow[r] & \cdots
\end{tikzcd}
\end{center}
(FINSIH THIS!!)
\end{document}
