\documentclass[12pt]{article}
\usepackage{import}
\import{../}{AlgGeoCommands}


\begin{document}

\section{Mar. 30}

\begin{rmk}
Here is a reason that a $-1$-category should be the initial (empty) category or the terminal category (single object and single morphism). We want the Hom spaces of a $0$-category (a set) to be $-1$-categories but these are singletons or empty. Therefore, we can say that a $-1$-sheaf ($-2$-stack) should be a function because its a category fibered over $X$ in $-1$-categories so just a single element over open compatibly with restriction. 
\end{rmk}

Given a (pre)sheaf on a topological space $X$ we can ``glue the fibers together'' to get a fibered category over $\mathrm{Open}(U)$ where the fibers are the sets $\F(U)$ over some open. The morphisms in this category are exactly $f|_U \to f$ where the morphism represents the restriction of the function $f$ over $U \embed V$. To make this a sheaf we need axioms involving the topology. 
\bigskip\\
We do the same thing for stacks.

\begin{defn}
A 2-presheaf over a topological space $X$ is a functor $f : \C \to \mathrm{Open}(X)$ such that
\begin{enumerate}
\item pullbacks exist
\item every morphism in $\C$ is a pullback
\end{enumerate}
\end{defn}

\begin{rmk}
Some exercises:
\begin{enumerate}
\item the fibers of a 2-presheaf are groupoids. 
\end{enumerate}
\end{rmk}

\newcommand{\Isom}[2]{\mathrm{Isom}\left(#1, #2\right)}

\begin{defn}
Let $\C$ be a 2-presheaf then $\C$ is a stack if
\begin{enumerate}
\item for $a, b \in \C(U)$ the functor $\Isom{a}{b}$ is a sheaf on $U$
\item objects glue. 
\end{enumerate}
\end{defn}

\begin{defn}
A category of geometric spaces is a category $\G$ such that there is a distinguished class of ``open immersions'' which is 
\begin{enumerate}
\item closed under composition
\item local in nature
\item preserved by pullbacks (fibered products)
\end{enumerate}
\end{defn}

\newcommand{\PSh}{\mathrm{PSh}}

\begin{prop}[Yonega]
Consider the category $\PSh_\C$ which is the contravariant functors from $\C$ to Set. Then $X \mapsto h^X = \Hom{\C}{-}{X}$ gives a fully faithfully embedding $\C \embed \PSh_\C$.
\end{prop}

\section{Geometry on PSh}

What is a vectorbundle on a presheaf? If we are going to give it geometry we should know an answer to this question. 
\bigskip\\
Example, the Hodge bundle. For any $S \to \M_g$ there is a family of curves $\pi : \C \to S$ and thus we get the Hodge bundle $\pi_* \Omega_{\C/S}$ which is a rank $g$ vector bundle on $S$. These vector bundles are compatible (by cohomology and base change) with pullbacks $S' \to S \to \M_g$. 
\bigskip\\
We call this data a vector bundle on $\M_g$.

\begin{defn}
A vector bundle $\E$ on $\F \in \PSh_\G$ is a vector bundle $\E(S)$ on each $S \in \G$ along with isomorphisms (DO THIS)
\end{defn}

\begin{exercise}
Let $\G$ be the category of open balls in $\C^n$ and holomorphic maps between them. Then $\mathrm{Man}_{\mathbb{C}} \to \PSh_\G$ is a fully faithful embedding.
\end{exercise}

\subsection{Fiber Products}

$\G$ may not have fiber products because. For example if $\G$ is the category of smooth manifolds and smooth maps then fiber products of non submersions is not a smooth manifold.
\bigskip\\
However, $\PSh_\G$ does have fiber products. Indeed we construct fiber products point-wise. 

\begin{exercise}
Any fiber product in $\G$ agrees with the corresponding fiber product in $\PSh_\G$ (the Yoneda embedding preserves fiber products). 
\end{exercise}

The Yoneda functor preserves fiber products basically by definition because 
\[ h^{A \times_B C}(X) = \Hom{\G}{X}{A \times_B C} = \Hom{\G}{X}{A} \times_{\Hom{\G}{X}{B}} \Hom{\G}{X}{C} \]

\begin{defn}
A morphism $f : F \to G$ in $\PSh_\G$ is representable when for any map $S \to G$ from $S \in \G$ then $F \times_G S$ is representable. 
\end{defn}

\begin{rmk}
If $\G$ has fiber products then every morphism between representable functors is representable.
\end{rmk}

\begin{exercise}
Representable morphisms are preserved by base change.
\end{exercise}

\begin{defn}
Given a property $\cP$ of morphisms in $\G$. Then we say a representable morphism $f : F \to G$ in $\PSh_\G$ has property $\cP$ if for every $S \to G$ with $S \in \G$ the morphism $F \times_G S \to S$ (which is a $\G$-morphism) has property $\cP$.
\end{defn}

\begin{rmk}
For this to make sense, we need $\cP$ to be a property preserved under base change so that $X \to Y$ has property $\cP$ if and only if $X_{Y'} \to Y'$ has property $\cP$. 
\end{rmk}

\begin{defn}
We can now define an open cover in $\PSh_\G$. A representable morphism is open in the above sense. 
\end{defn}

\section{April 4}

\begin{rmk}
Notice that every representable presheaf on $\G$ is a sheaf when restricted to each object $X \in \G$. 
\end{rmk}

\begin{defn}
A presheaf $F \in \PSh_\G$ is a \textit{sheaf} if for each $X \in \G$ the presheaf $X|_{\G}$ (restriction to the open subsets of $\G$) is a sheaf. 
\end{defn}

\begin{rmk}
This will be a sheaf for the topology on $\G$ induced by open embeddings. 
\end{rmk}

\begin{defn}
Let $\G$ be a category (not necessarily with fiber products). A topology on $\G$ is a connection of morphisms $\G^\circ \subset \G$ (the ``open immersions'') satisfying the following properties:
\begin{enumerate}
\item and isomorphism $f : X \to Y$ is in $\G^\circ$ (for example $\id_X$ because $\G^\circ$ is a subcategory)
\item openness is preserved under composition ($\G$ is a subcategory)
\item pullbacks of morphisms in $\G^\circ$ by morphisms in $\G$ exist and are in $\G^\circ$.
\item the fiber product of $U_1 \to X$ and $U_2 \to X$ gives $U_1 \times_X U_2 \to X$ is open (this is implied by composition and preservation under fiber products).
\end{enumerate}
Along with the data of distinguished collections of morphisms in $\G^\circ$ called covering families such that
\begin{enumerate}
\item every isomorphism $f : X \to Y$ is a covering family
\item given a covering on $Y$ and a morphism $f : X \to Y$ then the base change is a cover of $X$
\item a cover of a cover is a cover meaning if $\{ X_\alpha \to X \}$ is a covering family and $\{ X_{\beta \alpha} \to X_\alpha \}$ are covering families then $\{ X_{\beta \alpha} \to X \}$ is a covering family. 
\end{enumerate}
\end{defn}

\begin{defn}
The category of sheaves $\Sh_\G \subset \PSh_\G$ is the full subcategory of objects ``determined locally on covers'' i.e. satisfying the usual sheaf axiom. 
\end{defn}

\begin{exercise}
$\Sh_\G$ has all fiber products and they agree with fiber products in $\G$ (when they exist) and an in $\PSh_\G$ under the fully faithfully embeddings,
\[ \G \embed \Sh_\G \embed \PSh_\G \]
\end{exercise}

\begin{defn}
If $X \in \G$ define $\G_X$ the slice category of morphisms $f : Y \to X$.
\end{defn}

\begin{defn}
A sheaf $F \in \Sh_\G$ is \textit{locally representable} if there is an open cover by representable sheaves. Explicitly there are representable sheaves and representable morphisms $U_i \to F$ such that for every such diagram,
\begin{center}
\begin{tikzcd}
X_i \pullback \arrow[d] \arrow[r] & X \arrow[d]
\\
U_i \arrow[r] & F
\end{tikzcd}
\end{center}
for $X \in \G$ we have $\{ X_i \to X \}$ is a covering family in $\G$.  
\end{defn}

\begin{rmk}
Applying this construction we get:
\begin{enumerate}
\item for affine schemes and open immersions get all schemes
\item for varieties and open immersions get pre-varieties (no separatedness or quasi-compactness)
\item for open balls in $\C^n$ with open holomorphic embeddings get pre-manifolds (no Hausdorffness or second countability).
\end{enumerate}
\end{rmk}


\section{April 6}


\newcommand{\LRep}{\mathrm{LRep}}
\newcommand{\AffSch}{\mathrm{AffSch}}

Reminder about $\G$: maybe it doesn't have fiber products (e.g. manifolds). We require our topology is \textit{subcanonical} meaning for any $Y \in \G$ the functor $h^Y$ is a sheaf. 
\bigskip\\
Our category $\G$ is often a subcategory of locally ringed spaces. In most cases we can recover the sheaf of rings via maps to a ring object $\A^1 \in \G$. 

\begin{exercise}
$\LRep_{\AffSch} \subset \Sch$. 
\end{exercise}

\begin{thm}
If $\G$ contains all fiber products (and a terminal object) then every $M \in \LRep_\G$ has $\Delta : M \to M \times M$ representable.
\end{thm} 

\begin{rmk}
In the example $\G = \AffSch$ we see that $\A^2$ with the doubled origin is not in $\LRep_{\G}$ because its diagonal is not affine and hence not representable.
\end{rmk}

\begin{lemma}
If $\G$ has products and fiber products. Then all maps $X \to F$ for $X \in \G$ and $F \in \PSh_\G$ are representable if and only if $\Delta : F \to F \times F$ is representable.
\end{lemma}

\begin{proof}
First, assume that $\Delta$ is representable. Then,
\begin{center}
\begin{tikzcd}
X \times_F Y \arrow[r] \arrow[d] \pullback & Y \arrow[d]
\\
X \arrow[r] & F
\end{tikzcd}
\end{center}
The following diagram is also cartesian,
\begin{center}
\begin{tikzcd}
X \times_F Y \arrow[d] \arrow[r] \pullback & X \times Y \arrow[d]
\\
F \arrow[r] & F \times F 
\end{tikzcd}
\end{center}
and $X \times Y \in \G$ so $X \times_F Y \in \G$ proving the claim. Next, suppose that $X \to F$ is always representable. For $U \in \G$ and $U \to F \times F$ we want to show that $U \times_{F \times F} F \in \G$. 
\begin{center}
\begin{tikzcd}
U \times_{F \times F} F \pullback \arrow[r] \arrow[d] & U \arrow[d]
\\
U \times_F U \arrow[d] \arrow[r] \pullback & U \times U \arrow[d]
\\
F \arrow[r] & F \times F
\end{tikzcd}
\end{center}
We see that $U \times_F U$ is representable because $U \to F$ is representable and thus the top square is all in $\G$ and hence because $\G$ has fiber products we conclude.
\end{proof}

\begin{rmk}
Given a diagram,
\begin{center}
\begin{tikzcd}
\mathrm{Isom} \pullback \arrow[r] \arrow[d] & U \arrow[d] 
\\
F \arrow[r] & F \times F
\end{tikzcd}
\end{center}
then the pullback will classify isomorphisms between the objects over $U$ represented by $F$ under the two maps $U \to F$. 
\end{rmk}

\begin{proof}[Proof of Theorem]
We need to show that $F \to F \times F$ is representable. By the lemma, it is equivalent to ask if for every diagram with $X, Y \in \G$ we have,
\begin{center}
\begin{tikzcd}
X \times_F Y \arrow[r] \arrow[d] & Y \arrow[d]
\\
X \arrow[r] & F 
\end{tikzcd}
\end{center}
we want $X \times_F Y$ is representable. Choose a cover $U_i \to F$ by representable $U_i$. Then we pullback to get
\end{proof}

\subsection{Complex Analytic Spaces}

\subsection{Sheafification}

\section{April 8}

Given the data $U_i \in \G$ along with opens immersions $U_{ij} \embed U_i$ and $U_{ij} \to U_j$ such that $U_{ij} \times_{U_i} U_{ik} = U_{kj} \times_{U_j} U_{ij}$ MAKE THIS WORK!!!
\bigskip\\
Then we get that,
\[ M^-(X) = \Hom{\G}{X}{\coprod U_i} / \Hom{\G}{X}{\coprod U_{ij}} \]
automatically $M^-(X) \in \PSh^+_{\G}$. This cannot be the right presheaf however, for example $\id_M$ doesn't make sense because we are not stratifying $X$. To do this we exactly take the sheafification. Therefore we define $M = (M^-)^+$. 
\bigskip\\
The claim is that $U_i \to M$ is open and make $M$ be locally representable. 

\begin{rmk}
Suppose that $\Delta : M \to M \times M$ is representable. Then for any cover $U_i \to M$ is 
\end{rmk}

\section{April ---}

\section{April 13}

Question: given $\pi : U \to X$ in $\G$ does $F$ satisfy the sheaf condition for $\pi$ meaning is,
\begin{center}
\begin{tikzcd}
F(X) \arrow[r] & F(U) \arrow[r, shift left] \arrow[r, shift right] & F(U \times_X U)
\end{tikzcd}
\end{center}

\begin{theorem}
If there is $\sigma : X \to U$ such that $\pi \circ \sigma = \id$ then any presheaf $F$ satisfies the sheaf condition for $\pi : U \to X$. 
\end{theorem}

\begin{proof}
We get maps $\sigma_i : U \to U \times_X U$ given by $(\sigma \circ \pi, \id)$ and $(\id, \sigma \circ \pi)$. Now we get $s^* \pi^* = \id$ so we see that $\pi^*$ is injective giving the first part. For gluing, given $s \in F(U)$ such that $\pi_1^* s = \pi_2^* s$. Then take $t = \sigma^* s$ and we need to show that $\pi^* t = s$.  Now $\pi_1 \circ \sigma_2 = \sigma \circ \pi$ and thus,
\[ \pi^* \sigma^* s = \sigma_2^* \pi_1^* s = \sigma_2^* \pi_2^* s = s \]
because $\pi_1^* s = \pi_2^* s$ and $\pi_2 \circ \sigma_2 = \id$.
\end{proof}

\begin{rmk}
Here is an alternative proof. We rewrite the above sequence via the Yoneda lemma as,
\begin{center}
\begin{tikzcd}
\Hom{}{h^X}{F} \arrow[r] & \Hom{}{h^U}{F} \arrow[r, shift left] \arrow[r, shift right] & \Hom{}{h^{U \times_X U}}{F} 
\end{tikzcd}
\end{center}
since $\Hom{}{-}{F}$ takes colimits to limits it suffices to show that,
\[ h^{U \times_X U} \rightrightarrows h^U \to h^X \]
is a coequalizer in the category of presheaves. However, limits and colimits are computed pointwise in the category of presheaves so we just need to show that
\begin{center}
\begin{tikzcd}
\Hom{}{T}{U \times_X U} \arrow[r, shift left] \arrow[r, shift right] & \Hom{}{T}{U} \arrow[r] & \Hom{}{T}{X}
\end{tikzcd}
\end{center}
is a coequalizer in Set. Since the map to $\Hom{}{T}{X}$ coequalizes and the section $\sigma : U \to X$ shows that $\Hom{}{T}{U} \to \Hom{}{T}{X}$ is sujective. Finally, given two maps $\alpha, \beta : T \to U$ then $\pi \circ \alpha = \pi \circ \beta$ if and only if $\alpha, \beta$ arise as the projections of a morphism $(\alpha, \beta) : T \to U \times_X U$ so we conclude.
\end{rmk}

\begin{cor}
Suppose we have a diagram,
\begin{center}
\begin{tikzcd}
V \arrow[rd] \arrow[r] & U \arrow[d]
\\
& X
\end{tikzcd}
\end{center}
Suppose that,
\begin{enumerate}
\item $F$ satisfies the sheaf condition for $V \to X$

\item $F(U) \to F(V \times_X U)$ is injective 
\end{enumerate}
then $F$ satisfies the sheaf condition for $U \to X$.
\end{cor}

\begin{proof}
Consider,
\begin{center}
\begin{tikzcd}
V \times_X U \arrow[d] \arrow[r] & U \arrow[d]
\\
V \arrow[r] \arrow[ru] & X
\end{tikzcd}
\end{center}
We know automatically that $F$ satisfies the sheaf condition for $V \times_X U \to V$ because there is a section $V \to V \times_X U$. By assumption $F$ is a sheaf for $V \to X$ so $F$ is a sheaf for $V \times_X U \to X$. Using that $F(V \times_X U) \to F(U)$ is injective, we reduce to the following lemma. 
\end{proof}

\begin{lemma}
Consider maps $V \to U \to X$ and let $F$ be a presheaf. If $F$ satisfies the sheaf condition for $V \to U \to X$ and $\pi^* : F(U) \to F(V)$ is injective then $F$ satisfies the sheaf condition for $U \to X$.
\end{lemma}

\begin{proof}
Consider the diagram,
\begin{center}
\begin{tikzcd}
0 \arrow[r] & F(X) \arrow[r] & F(V) \arrow[r, shift left] \arrow[r, shift right] & F(V \times_X V) 
\\
0 \arrow[r] & F(X) \arrow[u, equals] \arrow[r] & F(U) \arrow[u] \arrow[r, shift left] \arrow[u] \arrow[r, shift right] & F(U \times_X U) \arrow[u]
\end{tikzcd}
\end{center}
where the top row is an equalizer. Then $F(X) \to F(V)$ is injective so $F(X) \to F(U)$ is injective. Suppose $\beta \in F(U)$ has equal pullbacks then $\pi^* \beta \in F(V)$ has equal projections and hence arises from a unique class $\alpha \in F(X)$ so that $\alpha \mapsto \pi^* \beta$. Since $F(U) \to F(V)$ is injective this means that $\alpha \mapsto \beta$ along $F(X) \to F(U)$ proving the claim.
\end{proof}

\begin{rmk}
The condition (b) in the corollary illustrates the utility of having our covers preserved via arbitrary base change in the definition of a Grothendieck topology. Indeed, we will use it essentially in the proof of the following.
\end{rmk}

\begin{cor}
Let $F$ be a sheaf on a site $\C$. If $U \to X$ is refined by a cover of $\C$ then $F$ satisfies the sheaf condition for $U \to X$.
\end{cor}

\begin{proof}
Indeed, suppose there is a cover $V \to X$ which factors as $V \to U \to X$. Then by definition, $F$ satisfies the sheaf condition for the covers $V \to X$ and $V \times_X U \to U$ using that covers are preserved under base change. In particular $F(U) \to F(V \times_X U)$ is injective so we can apply our previous result.
\end{proof}

\begin{cor}
Let $\tau$ and $\tau'$ be Grothendieck topologies on $\C$. If $F$ is a sheaf for $\tau$ and every morphism in $\tau'$ is refined by a $\tau$-cover then $F$ is a $\tau'$-sheaf.
\end{cor}

\begin{cor}
Let $\tau$ and $\tau'$ be Grothendieck topologies on $\C$. Suppose that $\tau$ and $\tau'$ are cofinal meaning every covering morphism in one is refined by a covering morphism in the other. Then $\tau$ and $\tau'$ have the same categories of sheaves (as subcategories of $\mathrm{PSh}(\C)$). 
\end{cor}

\section{April 20}

\subsection{Quasi-Coherent Sheaves on Algebraic Spaces}

For every $\Spec{A} \to X$ where $X$ is an algebraic space, I want an $A$-module $M$ such that for $\Spec{A'} \to \Spec{A}$ we have $M'$ is the pullback of $M$. 

\begin{prop}
If $\pi : X \to Y$ is a quasi-compact and quasi-separated morphism of algebraic spaces and $\F$ is quasi-coherent then $\pi_* \F$ is quasi-coherent.
\end{prop}

\subsection{Cech Cohomology}

For the Zariski topology (and also other cohomology) on $X$ an algebraic space with quasi-compact and affine diagonal then Cech cohomology works for quasi-coherent sheaves.

\section{Example}

Consider $\M^a_3$ the moduli space of genus $3$ curves with no nontrivial automorphisms over $\CC$. Let $\G = \Sch_{\CC}$ then $\M^a_3 \in \PSh_\G$ is the functor,
\[ S \mapsto \{ \pi :  C \to S \text{ relative dim 1 smooth geometrically intergral fibers of genus 3 with no automorphisms} \} \]
Consider $\pi_* \Omega_{C/S}$ is a rank $3$ vector bundle (by cohomology and base change) and thus we get a closed embedding,
\[ C \embed \P_S(\pi_* \Omega_{C/S}) \]
over $S$. These are all embedded as plane quartic curves. Quartic divisors are parametrized by $\P^{14}$ there is an open $U \subset \P^{14}$ where the associated plane quadric is a smooth irreducible curve. Then,
\[ \M^a_{3} = U / \PGL_{3} \]
To see this, consider,
\begin{center}
\begin{tikzcd}
B_U \arrow[r] \arrow[d] \pullback & U \arrow[d]
\\
B \arrow[r] & \M^a_3
\end{tikzcd}
\end{center}
Then $B_U$ is exactly $\Isom{\P_B(\pi_* \Omega_{C/S})}{\P^3_B}$ which is a Zarikis $\PGL_3$-torsor OR MAYBE where $C^{\text{univ}} \to U$ is the universal curve over $U$. Maybe it's actually correct to take the $\PGL_3$-torsor $D$ over $B$ of sections of $\P_S(\pi_* \Omega_{C/S})$ and tak
\bigskip\\
Let's check this. The $T$-points $T \to B_U$ are exactly given by the following data $(a, b, \gamma)$ where $a : T \to B$ and $b : T \to U$ and an isomorphism $\gamma : C_T \iso C^{\text{univ}}_T$ which is the data of a map 
\[ T \to \mathrm{Isom}_{B \times U}(C_{B \times U}, C^{\text{univ}}_{B \times U}) \]
However, since $C^{\text{text}} \to U$ is universal the maps $b : T \to U$ are exactly classified by isomorphism classes $[C_T]$ as closed subschemes since because these curves are canonically embedded, as $C \embed \P_B(\pi_* \Omega_{X/S})$ and $C^{\text{univ}} \embed \P^3_U$ the isomorphism $C_T \iso C_T^{\text{univ}}$ induces an isomorphism $\P(\pi_* \Omega_{X/S}) \iso \P^3$. For a fixed such isomorphism there is a unique map $T \to U$ defined by the image of $C_T$ in $\P^3$ therefore,
\[ B_U = \mathrm{Isom}_{B \times U}(C_{B \times U}, C^{\text{univ}}_{B \times U}) = \mathrm{Isom}_{B}(\P_B(\pi_* \Omega_{X/S}), \P^3_B) \]

\section{April 25}

\begin{defn}
A \textit{sieve} over $X$ is a sub-presheav of $h_X$.
\end{defn}

\begin{example}
$S_{U \to X}$ for any $U$
\end{example}

\begin{defn}
Given $\tau$, a sieve is called a \textit{covering sieve} if it contains some cover.
\end{defn}

\begin{example}
If $\C$ is the category of opens of a topological space $X$ then a sieve on $X$ is a collection of open in $X$ stable under subsets. 
\end{example}

\begin{thm}
A presheaf $\F$ satisfies,
\[ \F \text{ is a sheaf } \iff \Hom{}{S}{\F} = \Hom{}{h_X}{\F} \text{ for all covering sieves } S \subset h_X \]
\end{thm}

\begin{rmk}
This shows that the category of sheaves recovers the covering sieves because we can take the sieves to be those that satisfy,
\[ \Hom{}{S}{\F} = \Hom{}{h_X}{\F} \]
for all sheaves $\F$.
\end{rmk}

\subsection{Example}


We want to show that $\M_g^a$ is an algebraic space. First we need to show it is a sheaf in the \etale (or smooth) topology. In fact, this will work in the fpqcK topology. 

\begin{thm}
The data of $(X, \L)$ where $\L$ is ample descends because we can descend the algebra,
\[ \bigoplus_{n \ge 0} \L^{\ot n} \]
which determines $X$ via embedding into projective space.
\end{thm}

\section{April 27}

Suppose $G \acts X$ where $G$ is a geometric group. We want to define $X / G$ as an algebraic space. 

\begin{example}
We want,
\begin{enumerate}
\item $X \to Y$ is a $G$-bundle then $X/G = Y$
\item $\Z / 2 \acts \A^1$ via $x \mapsto -x$ is not a $G$-bundle and we want $\A^1 / (\Z / 2)$ to be the GIT quotient. 
\end{enumerate}
\end{example}
\noindent
What is the definition of $X/G$ as an algebraic space? There should be a $G$-invariant map $X \to X/G$ such that any $G$-invariant map $X \to Y$ factors uniquely through $X \to X / G \to Y$,
\begin{center}
\begin{tikzcd}
X \arrow[rd] \arrow[r] & X/G \arrow[d, dashed]
\\
& Y
\end{tikzcd}
\end{center} 
We call this the categorical quotient. There are some problems with this definition,
\begin{enumerate}
\item it might not exist
\item $X \to X / G$ might not be a $G$-bundle e.g. $\A^1 \to \A^1$ via $x \mapsto x^2$ is the quotient $\A^1 / (\Z / 2)$ but this is not a $\Z/2$-bundle (ramified over the origin). 
\end{enumerate}

Some possible answers,
\begin{enumerate}
\item define $X/G$ as the categorical quotient
\item define $X/G$ as the categorical quotient by only when $X \to X / G$ is a $G$-bundle
\item defined it as the presheaf (or the sheafification)
\[ Y \mapsto X(Y) / G(Y) \]
\item define it as the presheaf sending $Y$ to equivariant maps,
\begin{center}
\begin{tikzcd}
P \arrow[r] \arrow[d] & X
\\
Y 
\end{tikzcd}
\end{center}
where $P \to Y$ is a principal $G$-bundle. We would have to take this up to isomorphism and this will be bad if there are nontrivial automorphisms of the map $P \to X$.
\item Consider the presheaf $h^X / h^G$ and take the sheafification to get $X/G$. If $G \acts X$ freely then $h^X / h^G$ is a separated presheaf.
\end{enumerate}

\begin{example}
If $Y = *$ and $P$ is the trivial $\Z/2$-bundle over $Y$ then the two maps $P \to \A^1$ whose image is $\pm 1$ have no automorphisms but the map $P \to \A^1$ whose image is $0$ does have an automorphism because the action is not free. 
\end{example}

\begin{defn}
The action $G \acts X$ is \text{free} if it is free on $T$-points $G(T) \acts X(T)$.
\end{defn}

\begin{example}
An elliptic curve is $E = \CC / \Lambda$ analytically and indeed $h^E = (h^{\CC} / h^\Lambda)^{++}$. Algebraically we have $\Lambda \acts \Spec{\CC[t]} = \A^1_{\CC}$ where $\Lambda$ is a discrete group viewed as a scheme. And we can define $\A^1_{\CC} / \Lambda$ which is an algebraic space but not isomorphic to an elliptic curve! However $(\A^1_{\CC} / \Lambda)^{\an} \cong \CC / \Lambda$ so $(\A^1_{\CC} / \Lambda)^{\an} \cong E^\an$ but the isomorphism (the Weierstrass $\wp$-function) is not algebraic. 
\end{example}

\section{May 6}

Question: how many times do you need to plus $h^U / h^R$ to get a stack? 

\subsection{How do we tell if a stack is DM}

\begin{prop}
If $\M$ is DM then $\I_{\M} \to \M$ is unramified. 
\end{prop}

\begin{proof}
Consider,
\begin{center}
\begin{tikzcd}
R \arrow[r] \arrow[d] & U \times U \arrow[d]
\\
\M \arrow[r] & \M \times \M
\end{tikzcd}
\end{center}
where the donward maps are \etale. However, $R \to U$ is \etale and hence unramified so $R \to U \times U$ is unramified. 
\end{proof}

\begin{prop}
If $\M$ is an algebraic stack and $\I_{\M} \to \M$ is unramified then $\M$ is DM.
\end{prop}

\begin{proof}
Given a smooth cover $U \to \M$. For each point $p \in U$ where the relative dimension $n > 0$ we want to slice to find an \etale neighborhood. 
\end{proof}

\section{May. 9}

Let $\M$ be an algebraic stack (meaning a locally representable stack in the smooth topology on schemes). 

\begin{defn}
The inertia stack,
\begin{center}
\begin{tikzcd}
\I_{\M} \arrow[d, "\Delta"]  \arrow[r] & \M \arrow[d, "\Delta"]
\\
\M \arrow[r, "\Delta"] & \M \times \M
\end{tikzcd}
\end{center}
is defined by the above fiber product.
\end{defn}

\begin{prop}
Let $\M$ be an algebraic stack, then $\M$ is as an algebraic space if and only if $\I \to \M$ is an isomorphism. 
\end{prop}

\begin{prop}
The following are equivalent,
\begin{enumerate}
\item $\M$ is DM 
\item $\Delta$ is unramified
\item $\Omega_{\M / \M \times \M} = 0$
\item $\Omega_{\I/\M} = 0$.
\end{enumerate}
\end{prop}

\begin{theorem}
Let $\pi : X \to Y$ be proper flat, whose geometric fibers are reduced and connected with $Y$ locally noetherian. Let $\L$ be a line bundle on $\L$. Then the presheaf sending $Z \to Y$ to data $(\M_Z, \varphi)$ where $\varphi : \pi_Z^* \M_Z \iso \L|_Z$ is an isomorphism of line bundles on $X_Z$ for the diagram,
\begin{center}
\begin{tikzcd}
X_Z \arrow[d, "\pi_Z"] \arrow[r] & X \arrow[d, "\pi"]
\\
Z \arrow[r] & Y
\end{tikzcd}
\end{center}
This is represented by a locally closed subscheme of $Y$. If the fibers of $\pi$ are integral then a closed subscheme. 
\end{theorem}

\begin{rmk}
If we assume $\pi$ is finitely presented then we can drop noetherian assumptions because it is the pullback of a noetherian case and therefore the theorem holds (showing it can be pulled back from a \textit{flat} noetherian case is the tricky part). 
\end{rmk}

\newcommand{\Y}{\mathscr{Y}}

\begin{rmk}
The above theorem works for a schematic morphism $f : \X \to \Y$ of algebraic stacks because it holds after all base changes to schemes. 
\end{rmk}

\begin{prop}
For $g \ge 2$ the stack $\M_g$ sending $B$ to families of smooth genus $g$ curves (flat schematic smooth map of relative dimension $1$ with integral curves of genus $g$ fibers) is algebraic. It is a stack in the fpqcK topology. 
\end{prop}

\begin{proof}
Any family of curves $\C \to B$ has $\omega_{\C / B}$ relatively ample and hence we get an embedding into projective space locally. Using cohomology and base change,
\[ \C \embed \P(\pi_* \omega_{\C/B}^{\ot 3}) \]
Therefore, consider the locus of canonically embedded curves $H \embed \Hilb_{\P^n}$ and then $H \to \M_g$ is a $\PGL_n$-torsor.  
\end{proof}

\begin{prop}
The stack of dimension $n$ smooth projective varities $X$ where $\det{\Omega_X} = \omega_X$ is ample is an algebraic stack for the fpqcK topology.
\end{prop}

\begin{proof}
Let $X \to B$ be such a family. Then there is some $\omega_{X_0}^{\ot N}$ which is very ample with vanishing higher cohomology. Then we consider the open substack of $\M$ where $h^{>0}(X, \omega_X^{\ot N}) = 0$ and $h^0$ is constant which is open by semicontinuity. Then the very ample locus is open (for flat maps the closed embedding locus is open). Then we get $\M_{X_0} \subset \M$ open and $H \to \M_{x_0}$ is a $\PGL_n$-torsor. 
\end{proof}

\section{May 11}

\begin{prop}
Suppose that $\M$ is an algebraic stack. Then the following are equivalent,
\begin{enumerate}
\item $\Omega_{\I / \M} = 0$
\item $\Omega_{\Delta} = 0$
\item $\M$ has a representable \etale cover by a scheme so is DM.
\end{enumerate}
\end{prop}

\begin{proof}
By pullback (b) implies (a) and $\I \to \M$ has a section so (a) implies (b) by pullback as well. We showed previously that if $\M$ is DM then $\Omega_{\Delta} = 0$ since $\Delta$ admits an \etale cover by an unramified morphism. Therefore, we just need to show that if $\Omega_{\Delta}$ then $\M$ is DM. 
\bigskip\\
Start with a smooth cover $U \to \M$ by a scheme $U$. We need to slice $U$ to make it \etale. We can shrink and take disjoint usion so we may take $U = \Spec{A}$. Now consider,
\begin{center}
\begin{tikzcd}
R \arrow[d, "\pi_2"] \arrow[r, "\pi_1"] & U \arrow[d]
\\
U \arrow[r] & \M
\end{tikzcd}
\end{center}
To slice a smooth morphism we just need that its restriction to the fiber cuts down the dimension of the differentials by one. Consider the sequence for $R \to U \times U \to \M$,
\begin{center}
\begin{tikzcd}
\pi_1^* \Omega_{U/\M} \oplus \pi_2^* \Omega_{U/\M} \arrow[r] & \Omega_{R/\M} \arrow[r] & \Omega_{R / (U \times U)} \arrow[r] & 0
\end{tikzcd}
\end{center}
Since $R \to U \times U$ is unramified we see that $\Omega_{R / U \times U} = 0$ and thus any function $f \in \m \setminus \m^2$ for a point $\m \subset A$ then it has a nonzero differential using the sequence (with structure map $\pi_2 : R \to U$),
\begin{center}
\begin{tikzcd}
\pi_2^* \Omega_{U / \M} \arrow[r] & \Omega_{R/\M} \arrow[r] & \Omega_{R/U} \arrow[r] & 0
\end{tikzcd}
\end{center}
So we see from these sequences that $\pi_1^* \Omega_{U / \M} \onto \Omega_{R/U}$ therefore for any covector in $(\Omega_{R/U})_p$ arises locally from pullback of some form $\Omega_{U/\M}$ which locally $\d{f}$ for $f \in \m$ on $\Spec{A} \subset U$. 
\end{proof}

\subsection{Stack of Algebraic Curves}

If $g \ge 2$ then $H^0(C, \T_C) = 0$ and therefore there are no infinitessimal automorphisms. This proves that $\Omega_{\I / \M_g} = 0$ so $\M_g$ is DM. Furthermore, because the stabiliers are closed subschemes of $\PGL_n$ we see that $\M_g$ has affine diagonal (in particular quasi-compact). However, it is unramified so we see that every genus $g$ curve has finitely many automorphisms. 
\bigskip\\
Consider genus $g$ curves with distinct smooth marked points $p_1, \dots, p_n \in C$ such that,
\[ \struct{C}(p_1 + \cdots + p_n) \]
is very ample with $h^1 = 0$. We claim this is an Artin stack by exactly the same argument: embed into $\P^n$ in families. 

\section{May 13}

\begin{rmk}
Amazing theorem is that we don't need smooth covers, flat is enough.
\end{rmk}

\begin{theorem}[04S6]
Let $F$ be an fppf sheaf and $f : U \to F$ a representable (by algebraic spaces) morphism which is surjective flat and locally finitely presented. Then $F$ is an algebraic space.
\end{theorem}

\begin{theorem}[06DC]
Let $f : \X \to \Y$ be a morphism of stacks in the fppf topology. Suppose that $\X$ is an algebraic stack, $f$ is representable by algebraic stacks which is surjective locally of finite presentation and flat. Then $\Y$ is an algebraic space. 
\end{theorem}

Return to $\M^{vp}_{g,n}$ is the moduli stack of families $f : \C \to B$ with $f$ flat finitely presented with $n$ sections whose geometric fibers are $1$-dimensional schemes where the sections are distinct smooth points $p_1, \dots, p_n \in \C_{\bar{s}}$ and $\C_{\bar{s}}$ has arithmetic genus $g = 1 - \chi(C, \struct{C})$ and,
\[ \struct{}(p_1 + \cdots + p_n) \]
is very ample with vanishing $h^1$. 

\begin{rmk}
The genus can be weird for example $g(\P^1 \sqcup \P^1) = -1$. But this really is the right notion because $\chi$ is locally constant in flat families. 
\end{rmk}

\begin{prop}
Why is $B \embed \C$ via the sections $\sigma_i$ effective Cartier divisors. We need to show $\I_{B / \C}$ is invertible. There is a sequence,
\begin{center}
\begin{tikzcd}
0 \arrow[r] & \I_{B / \C} \arrow[r] & \struct{\C} \arrow[r] & \struct{B} \arrow[r] & 0
\end{tikzcd}
\end{center}
\end{prop}


We construct this as follows. Let $U \subset \Hilb(\P^N)$ be the locus where the intersection with a fixed hyperplane $H$ is degree $n$. Then $V = Z \cap H \to U$ is flat of degree $n$ where $Z$ is the universal familiy over $U$. Then we take our parameter space $V \times_U \cdots \times_U V \setminus \Delta$ where $\Delta$ is the locus where the points are equal. This gives canonical sections of $V$ pulled back to here thus labeling the points. Then we mod out by the $\PGL_n$ action fixing $H$ to get our stack.
\bigskip\\
Then $U \to \M_{g,n}$ where $U \subset \M_{g,N}$ is the smooth geometric fibers locus where $N = n + d$ where $d$ is large enough so that $N > 2 g + 2$. 

\begin{rmk}
Consider $M_g$ the stack of nodal genus $g$ curves with connected fibers. 
\end{rmk}  

\end{document}