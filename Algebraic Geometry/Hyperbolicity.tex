\documentclass[12pt]{article}
\usepackage{import}
\import{./}{AlgGeoCommands}


\newcommand{\cO}{\mathcal{O}}
\newcommand{\V}{\mathbb{V}}

\begin{document}

\section{The Kobayashi Pseudodistance}

\begin{defn}
A \textit{directed pair} $(X, V)$ is a pair of a complex mnifold $X$ and a holomorphic subbundle $V \subset T_X$. 
\end{defn}

Here let $\Delta$ be the unit disk in $\CC$ and $\rho$ the Poincare metric on $\Delta$.

\begin{defn}
Let $X$ be a complex manifold. The \textit{Kobayashi pseudodistance} is the pseduometric defined,
\[ d_X(p,q) = \inf_\alpha \ell(\alpha) \]
where $\alpha$ is a chain of holomorphic disk $f_i : \Delta \to X$ and points $p = p_0, p_1, \dots, p_k = q$ of $X$ and pairs $a_1, b_1, \dots, a_k, b_k \in \Delta$ such that,
\[ f_i(a_i) = p_{i-1} \quad f_i(b_i) = p_i \]
and the length $\ell(\alpha)$ of the chain is defined as,
\[ \ell(\alpha) := \rho(a_1, b_1) + \cdots + \rho(a_k, b_k) \]
where $\rho$ is the Poincare metric on $\Delta$.
\end{defn}

\begin{example}
Let $X = \CC$ then $d_X = 0$. Indeed, by choosing larger and larger discs containing $p,q$ their pullback to the unit disk is then closer and closer to the origin and hence have vanishing Poincare distance.  
\end{example}

\begin{rmk}
Recall the Schwartz-Pick lemma says that any holomorphic map $f : \Delta \to \Delta$ is a contraction for the Poincare metric. Therefore, $d_{\Delta} = \rho$.
\end{rmk}

\begin{lemma}
Let $f : X \to Y$ be holomorphic. Then $d_Y(f(x), f(y)) \le d_X(x,y)$ 
\end{lemma}

\begin{proof}
Indeed, choosing any chain of disks $g_i : \Delta \to X$ computing $d_X(x,y)$ we see that $f \circ g_i$ is a chain of disks connecting $f(x)$ and $f(y)$ of the same length. Therefore,
\[ d_Y(f(x), f(y)) = \inf_{\alpha} \ell(\alpha) \le d_X(x, y) \] 
\end{proof}

\begin{cor}
If $f : \CC \to X$ is an entire curve then for $x, y \in f(\CC)$ we have $d_X(x,y) = 0$ meaning if $f$ is nonconstant then $d_X$ is degenerate along the image of $f$. 
\end{cor}

\begin{proof}
Indeed, let $z_1, z_2 \in \CC$ map to $x,y$ respectively. Then,
\[ d_X(x,y) \le d_{\CC}(z_1, z_2) = 0 \]
\end{proof}

\newcommand{\kk}{\mathbf{k}}

Brody's theorem is a converse to this result. We start by considering an infinitesimal anlogue of the Kobayashi pseudodistance. Let $v \in T_{X, x_0}$ be a holomorphic tangent vector at $x_0 \in X$ and define,
\[ \kk_X(v) = \inf \{ \lambda > 0 \mid \exists f : \Delta \to X \text{ such that } f(0) = x_0 \text{ and } \lambda f'(0) = v \} \]
where $f : \Delta \to X$ is holomorphic. It is easy to check that holomorphic maps contract this pseduometric and for $X = \Delta$ it agrees with the Poincare metric.

\begin{theorem}
Let $X$ be a complex manifold. Then,
\[ d_X(p,q) = \inf_\gamma \int_\gamma \kk_X(\gamma'(t)) \d{t} \]
where the infimum is taken over all piecewise smooth curves joining $p$ and $q$.
\end{theorem}

\begin{defn}
A \textit{Brody curve} $f : \CC \to X$ is an entire curve which has bounded derivative (wrt to some/any hermitian metric).
\end{defn}

\begin{theorem}[Brody]
Let $X$ be a compact complex manifold. If $d_X$ is degenerate then there exists a Brody curve in $X$.
\end{theorem}

\begin{rmk}
Of course, in the case that $X$ is compact any entire curve is automatically Brody.
\end{rmk}



\section{Definitions}

\begin{defn}
We say a directed pair $(X, V)$ is,
\begin{enumerate}
\item \textit{Brody hyperbolic} if there does not exist a nonconstant entire map $f : \CC \to X$ tangent to $V$
\item \textit{Kobyashi hyperbolic} if the Kobayashi pseudodistance is nondegenerate (i.e. it is a metric).
\end{enumerate}
\end{defn}

\begin{theorem}[Brody]
Let $X$ be a compact complex manifold. Then $X$ is Kobayashi hyperbolic if and only if it is Brody hyperbolic.
\end{theorem}

\begin{rmk}
Therefore, we will call manifolds with this property just ``hyperbolic'' or ``analytically hyperbolic'' for emphasis. 
\end{rmk}

\begin{defn}
If $X$ is a complex projective algebraic variety we say $(X, V)$ is
\begin{enumerate}
\item \textit{algebraically hyperbolic} if there exists $\epsilon > 0$ such that for every complete intergral curve $C \subset X$ we have,
\[ 2 g(C) - 2 \ge \epsilon \deg_H{C} \]
where $g(C)$ is the geometric genus of $C$
\item \textit{algebraically quasi-hyperbolic} if $X$ contains finitely many genus $0$ and genus $1$ curves.
\end{enumerate}
\end{defn}

\begin{theorem}[Demailly]
Let $X$ be a smooth projective varitety. Then the following hold,
\[ X \text{ is hyperbolic} \implies X \text{ is algebraically hyperbolic} \] 
\end{theorem}

\begin{theorem}
If $X$ is algebraically hyperbolic then $X$ admits no nonconstant morphisms from an abelian variety.
\end{theorem}


Some references:
\begin{enumerate}
\item \chref{https://arxiv.org/pdf/math/0103084.pdf}{Xi Chen}
\item \chref{https://arxiv.org/pdf/1807.03665.pdf}{Javanpeykar}
\end{enumerate}

\subsection{The Green-Griffiths Locus and Jets}

\begin{theorem}[\chref{https://www-fourier.ujf-grenoble.fr/~demailly/manuscripts/hyperbolic.pdf}{Demilly's Notes} Theorem 7.9]
Let $(X, V)$ be a direct projective manifold and $A$ an ample line bundle. Then for any entire curve $f : \CC \to X$ tangent to $V$ and any $P \in H^0(X, E^{GG}_{k,m}(V^*) \ot A^{-1})$  we have $P(f', f'', \dots, f^{(k)}) = 0$ identically.
\end{theorem}

Therefore, if we fix an ample line bundle we can consider the locus cut out by all these differential equations.

\newcommand{\GG}{\mathrm{GG}}
\newcommand{\Exc}{\mathrm{Exc}}


\begin{defn}
The \textit{Green-Griffiths locus} $\GG_A(X, V)$ is the set $x \in X$ such that for all $k > 0$ there exists a $k$-jet $\varphi_k : (\CC, 0) \to (X, x)$ tangent to $V$ so that for all $m > 0$ every global jet differential $P \in H^0(X, E^{GG}_{k,m}(V^*) \ot A^{-1})$ satisfies $P(\varphi_k) = 0$.
\end{defn}

\begin{rmk}
The locus $\GG_A(X, V)$ is independent of the choice of ample line bundle (see \chref{https://arxiv.org/pdf/1302.4756.pdf}{Diverio and Rousseau} Lemma 2.2. This paper also gives many examples showing that $\Exc(X)$ can be strictly smaller than $\GG(X)$. However, it is conjectured that if $X$ is general type then $\GG(X) \subsetneq X$.
\end{rmk}


LOOK AT THE HILBERT MODULAR SURFACES FOR WHICH THE GG LOCUS IS EVERYTHING


\section{Conjectures}

\begin{conj}[Kobayashi]
For $n \ge 2$ and $D \subset \P^n$ a very general hypersurface of degree $\deg{D} \ge 2n + 1$ then,
\begin{enumerate}
\item $D$ is hyperbolic
\item $\P^n \sm D$ is hyperbolic.
\end{enumerate}
\end{conj}

\begin{conj}[Green-Griffiths-Lang]
Let $X$ be a projective variety of general type. Then there exists a proper algebraic subvariety containing all non-constant entire curves $f : \CC \to X$.
\end{conj}

\begin{conj}[Demailly]
If $X$ is algebraically hyperbolic then $X$ is hyperbolic.
\end{conj}

\begin{prop}
The Green-Griffiths-Lang conjecture implies the Demailly conjecture.
\end{prop}

WHY?
\begin{proof}
Suppose $X$ is algebraically hyperbolic. If $X$ is not of general type then $X$ has a fibration over its canonical model by varities of Kodaira dimension $0$. (I NEED THAT IF NOT GENERAL TYPE THEN NOT ALGEBRAICALLY HYPERBOLIC DOES THIS FOLLOW FROM MMP)
\end{proof}

\section{Theorems}

\begin{theorem}[Bogomolov]
Let $X$ is a smooth projective surface with $s_2(X) = c_1(X)^2 - c_2(X) > 0$ then $X$ has finitely many genus $0$ or genus $1$ curves (i.e. it is algebraically quasi-hyperbolic).
\end{theorem}

\begin{theorem}[McQuillian]
Let $X$ is a smooth projective surface with $s_2(X) = c_1(X)^2 - c_2(X) > 0$ and $X$ has \textit{no} genus $0$ or genus $1$ curves then $X$ is hyperbolic.
\end{theorem}



\section{Bogomolov's Theorem}


The notion of stability of a point on a space of linear representations of a reductive group, due to Mumford [10], leads to a notion of stabilite for fiber bundles over a curve, whose properties were studied in [13] and [19].

\begin{defn}
Over a smooth proper integral curve, a vector bundle $E$ of rank $r(E)$ and degree $d(E)$ is \textit{stable} (resp. \textit{semistable}) is for every nonzero proper subbundle $F \subsetneq E$ we have,
\[ \frac{d(F)}{r(F)} < \frac{d(E)}{r(E)} \quad \left( \text{resp.} \frac{d(F)}{r(F)} \le \frac{d(E)}{r(E)} \right) \]
A vector bundle is \textit{unstable} if it is not semistable.
\end{defn}

Now let $X$ be a smooth proper surface over a field $k$, and $E$ a vector bundle over rank $2$ over $X$. Then a linear representation $\rho : \GL_2 \to \GL(V)$ produces an associated bundle $E^{(\rho)} := E \times_{\GL_2} V$ of rank $\dim{V}$. 

\begin{defn}
We say a rank $2$ vector bundle is \textit{instable} if there exists a representation $\rho : \GL_2 \to \GL(V)$ with $\det{\rho} = 1$ such that $E^{(\rho)}$ admits a nonzero section which vanishes at some point. 
\end{defn}

If the characteristic of $k$ is zero, which we will assume for the remainder, then Bogomolov's instabilite criterion is simply expressed in terms of devissage of bundles of rank $2$ (WHAT?). It is interesting to note that we can here short-circuit the theory and prove directly using these simpler methods. 
\par
There are many applications. We quote from memory a proof, elegant and algebraic, of the vanishing theorem of Kodaira-Ramanujan. In the remaning section we prove the following:

\begin{theorem}[0.3]
Let $X$ be a proper smooth surface of general type. Then $\Omega_X$ is not unstable.
\end{theorem}

As a consequence, we obtain the inequality $c_1^2 \le 4 c_2$ where $c_1, c_2$ are the Chern classes of the sheaf $\Omega_X^1$ -- improved by Miyaoka [9] which is the best form possible $c_1^2 \le 3 c_2$ - and a geometric result that we will develop here.
\par 
The problem is the following: can we show 'bound'' the familly of curves of bounded geometric genus on a smooth proper surface $X$? We construct easily examples where the answer is negative. Bogomolov provides a partial solution in the case that $X$ is a surface of general type. We summarize briefly the method.
\par 
Let $\pi : P = \P(\Omega_X^1) \to X$ be the canonical projection from the projectiviation of the canonical bundle. We construct on $P$ a good linear system of divisors alowing it to be mapped to the projective space $\P^N$. If $C$ is a smooth proper curve and $f: C \to X$ is a nonconstant morphism there is a lift $t_f : C \to P$ via the differential defined over points $\alpha \in P$ where $f$ is unramified as $t_f(\alpha) = (f(\alpha), f(v_\alpha))$ where $v_\alpha$ is a nonzero tangent vector to $C$ at $\alpha$. We apply this to the normalizations of curves embedded in $X$ and study their images in $\P^N$.
\bigskip\\
We prove the following result:

\begin{theorem}
Let $X$ be a smooth proper surface minimal of general type.
\begin{enumerate}
\item If $c_1^2 > c_2$ then the curves of bounded geomeric genus on $X$ form a bounded family.
\item If $c_1^2 \le c_2$ and $\rank \NS{X} \ge 2$ then there exists a nonempty open cone $C \subset \NS{X}_{\RR}$ containing the cone $\{ z \mid z \in \NS{X}_{\RR} , z^2 \le 0 \}$ such that for any closed cone $C'$ contained in $C$ the family of curves of bounded geometric grnus on $X$ have image in $\NS{X}_{\RR}$ contained in $C'$ forms a bounded family. Moreover, any translate of $C$ parallel to $K_X$ has the same property. 
\end{enumerate} 
\end{theorem}

As a corollary, we obtain finiteness of curves with negative self-intersection and bounded geometric genus on surfaces of general type. In particular a solution to Mordell's problem.
\par 
Let's point out finally that Bogomolov uses a powerful result of Deidenberg on differential equations [18] but a recent paper of Jouanalou [5] alows us to avoid the use of this sledgehammer.

\subsection{Criteria for instability of vector bundles of rank $2$ on surfaces}

Considering the form of representations of $\PGL_2$ we give a definition equivalent to above.

\begin{defn}
A vector bundle $E$ of rank $2$ on a surfaces is \textit{unstable} if and only if there exists $n > 0$ such that $S^{2n} E \ot (\det{E})^{-n}$ has a nonzero section vanishing at some point of $X$.
\end{defn}

\subsubsection{Remark: devissage of vector bundles of rank $2$}

Let $E$ be a vector bundle of rank $2$ and $L$ an invertible sheaf and $s : L \to E$ a nonzero map. The bidual $M$ of $E/L$ is reflexive and hence invertible (since $X$ is a smooth surface), and the kernel $L_1$ of the homomorphism $E \to M$ is a larger invertible subsheaf of $E$ contining $L$. We say that it is a saturated line bundle of $E$. The cokernel $E/L_1$ is torsion-free in rank $1$, and hence of the form $I_Z \ot M$ for $M$ an invertible sheaf and $I_Z$ a sheaf of ideals defining a closed subscheme $Z$ of dimension $0$ outside of which $L'$ is a subbundle of $E$. We have a diagram of exact sequences,
\begin{center}
\begin{tikzcd}
0 \arrow[r] & L \arrow[d] \arrow[r] & E \arrow[r] \arrow[d, equals] & E/L \arrow[d] \arrow[r] & 0
\\
0 \arrow[r] & L_1 \arrow[r] & E \arrow[r] & I_Z \ot M \arrow[r] & 0
\end{tikzcd}
\end{center} 
We will say that the second line is a devissage of $E$. We can deduce the Chern classes of $E$,
\[ c_1(E) = c_1(L_1) + c_1(M) \quad \quad c_2(E) = c_1(L_1) \cdot c_1(M) + \deg{Z} \]

\begin{theorem}[Bogomolov-Mumford]
A vector bundle $E$ of rank $2$ over a surface $X$ is unstable if and only if there exists a devissage,
\[ 0 \to L \to E \to I_Z \ot M \to 0 \]
such that if $L' = L \ot M^{-1} = L^2 \ot (\det{E})^{-1}$ then either,
\begin{enumerate}
\item $L'$ is in the cone $C_+ \subset \NS{X}_{\Q}$ generated by positive divisors (IS THIS NEF?)
\item or $L' = \struct{X}$ and $Z$ is nonempty
\end{enumerate}
Moreoverm the devissage is unique.
\end{theorem}

We will prove this using only Mumford's theory of instability.
\par 
Let $P = \P(E)$ and $p : P \to X$ the projection and $\struct{P}(1)$ the canonical relatively ample bundle on $P$. A nonzero section $s \in H^0(X, S^{2n} E \ot (\det{E})^{-n})$ corresponds to a nonzero section $t \in H^0(P, \struct{P}(2n) \ot p^* (\det{E})^{-n}))$ . Let $\xi \in X$ be the generic point and $K = \kappa(\xi)$. If we chose a basis of $E_K$ then $s(\xi)$ corresponds to a homogeneous polynomial $F$ of degree $2n$ in two variables. Since $s$ is zero t some point of $X$, we know $s(\xi)$ is unstable for the action of $\PGL_2$ on $S^{2n} E_K \ot (\det{E_K})^{-n}$ (WHY?). We deduce from the stability criterion using 1-parameter subgroups [11] that $F$ has a root of order greater than $n$ in the algebraic closure of $K$, so also in $K$ (WHAT? WHY?), that's to say there exists an integer $r \ge 1$ and two polynomials $G, H$ homogeneous of degrees $1$ and $n-r$ respectively such that $F = G^{n+r} H$. Let $D$ be the divisor of $t$ and $\Delta$ the closure of the divisor defined over a generic point by $G$. We can write $D = (n + r) \Delta + \Delta'$ where has degree $1$ and $\Delta'$ has degree $n - r$ on $P$. Therefore, there exist invertible modules $L, L'$ on $X$ such that,
\[ \struct{P}(\Delta) = \struct{P}(1) \ot p^* L \quad \cO_P(\Delta') = \cO_P(n-r) \ot p^* L' \]
and hence,
\[ (\det{E})^{-n} = L^{n+r} \ot L' \]
The divisor $\Delta$ corresponds to a section of $E \ot L$ and thus an injection $L^{-1} \embed E$ which by construction is saturated in $E$. We verify that it provides the desired devissage. 


\subsection{Operations on unstable bundles}

Instability is preserved by passage to the dual and tensor product with a line bundle. 

\begin{enumerate}
\item Let $f : Y \to X$ be a surjective morphism of surfaces, $E$ a vector bundle of rank $2$ over $X$. Then $E$ is unstable if and only if $f^* E$ is. 

\item Let $f : Y \to X$ be a finite faithfully flat morphism of surfaces, $F$ a fiber bundle of rank $2$ on $Y$. Then if $F$ is unstable so is $f_* F$.
\end{enumerate}


\subsection{Proof of Theorem 0.3}

Suppose that $\Omega_X^1$ is unstable. Then there exists a devissage:
\[ 0 \to L \to \Omega^1_X \to I_Z \ot M \to 0 \]
and an integer $n > 0$ such that there is an injection $\cO_X \embed (L \ot M^{-1})^{\ot n}$. Note yhat $L \ot M^{-1}  = L^2 \ot (\det{\Omega^1_X})^{-1} = L^2 \ot (\Omega_X^2)^{\ot -1}$. Also, for $m \gg 0$,
\[ h^0(L^{2m}) = h^0((L \ot M^{-1})^{\ot m} \ot (\Omega^2_X)^{\ot m}) \ge h^0((\Omega_X^2)^{\ot m}) \in O(m^2) \]
Therefore, the theorem is a consequence of the following.

\begin{theorem}[Bogomolov]
Let $X$ be a smooth proper surface and $L \embed \Omega_X^1$ an invertible subsheaf. Then $h^0(L^n) \in O(n)$. 
\end{theorem}

First recall the pretty result of Castelnuovo and of Franchis which we will need for the proof.

\begin{lemma}[4, 12]
Let $\omega_1, \omega_2$ be two holomorphic $1$-forms on $X$ which are linearly independent over $k$ such that $\omega_1 \wedge \omega_2 = 0$. Then there exists a curve $C$ which is proper and smooth over $k$ of genus $g \ge 2$ and two holomorphic $1$-forms $\theta_1, \theta_2$ on $C$ and a morphism $u : X \to C$ such that $\omega_i = u^* \theta_i$ for $i = 1,2$.
\end{lemma}

There exists a meromorphic function $f : X \rat \P^1$ such that $\omega_2 = f \omega_1$. This defines a morphism $f : X' \to \P^1$ where $X'$ is a blowup of $X$. Let $u : X' \to C \to \P^1$ be the Stein factorization.  We have an exact sequence of modules of differentials,
\[ 0 \to u^* \Omega^1_C \to \Omega^1_{X'} \to \Omega^1_{X'/C} \to 0 \]
We know $\omega_2 = f \omega_1$ and $0 = \d{\omega_2} = \d{f} \wedge \omega_1$ (since $\omega_i$ are global holomorphic forms they are closed by Hodge theory). 

{\color{red} WHY DOES IT WORK ON AN OPEN}

But $\d{f}$ is pulled back from an open of $U$ so $\omega_1$ is also as it is parallel to $\d{f}$ hence also $\omega_2 = f \omega_1$. So above an open $U \subset C$ the forms $\omega_1, \omega_2$ are in the image of,
\[ H^0(u^{-1}(U), u^* \Omega_C^1) = H^0(U, \Omega^1_C) \to H^0(u^{-1}(U), \Omega^1_{X'}) \]
so we choose $\theta_1, \theta_2$ holomorphic forms on $U$ which pull back to $\omega_1, \omega_2$. However, $u_* \cO_{X'} = \cO_{C}$ so $\theta_1, \theta_2$ extend to global sections of $\Omega_C$ because $\omega_1, \omega_2$ are global sections of $\Omega_{X'}$. Indeed,
{\color{red} (WHY DOES IT EXTEND??) THIS SEEMS WRONG}


Since $\omega_1, \omega_2$ are $k$-independent so are $\theta_1, \theta_2$. Hence $g(C) \ge 2$ and therefore the map $u : X' \to C$ contracts all rational curves and hence factors through $X' \to X$ giving the requried map. 

\subsubsection{Interlude: regularizing meromorphic 1-forms via covers}

{\color{red} WHATIS THE CORRECT DEFINITION OF TAME?}

\begin{lemma}
Let $f : X \to Y$ be a morphism of locally noetherian schemes. If $Z \subset Y$ is an irreducible subset of codimension $\le r$ then either $f$ does not dominate $Z$ or there is some closed $Z' \subset X$ of codimension $\le r$. 
\end{lemma}

\begin{proof}
Using that $\codim{Z,Y} = \dim{\stalk{Y}{\xi}}$ where $\xi \in Z$ is the generic point we immediately reduce to the affine case. Either $\xi \notin f(X)$ and we are done or we can choose $f : U \to V$ a mp of affine schemes sending $\xi' \in U$ to $\xi \in V$. Let $\varphi : A \to B$ be a map of noetherian rings and $\p \subset A$ a prime of height $\le r$ in the image of $\Spec{B} \to \Spec{A}$. Passing to $A_\p \to B_\p$ we need to find a prime $\q$ of $B_\p$ of height $\le r$ with $\varphi^{-1}(\q)$ maximal. Then $\p$ is the unique minimal prime over an ideal of definition $(x_1, \dots, x_r) \subset A_\p$ generated by at most $r$ elements by \chref{https://stacks.math.columbia.edu/tag/00KQ}{Tag 00KQ}. Since $B_\p / \p B_\p$ is nonzero (the fiber is nonempty) the ideal $(x_1, \dots, x_r) B_\p$ is proper hence, by the Krull height theorem, there exists a prime $\q$ containing it of height $\le r$. Then each $x_i \in \varphi^{-1}(\q)$ so $\p \subset \varphi^{-1}(\q)$ and we conclude. 
\end{proof}

\begin{example}
Noetherianity is essential in the above. Indeed, we could take a domain $D$ with every nonzero prime of infinite height (as constructed in ``Anti-archimedean rings and power series rings'' by D.D. Anderson). Then for any nonzero nonunit $t \in D$ the map $k[t] \to D$ certainly falisfies the claim that the divisor $V(t)$ is in the image of a divisor (since there are none) although it is in the image of some prime.
\end{example}

\begin{prop}
Let $f : X \to Y$ be a proper dominant morphism of locally noetherian integral $S$-schemes that are smooth over $S$ at the generic points of all divisors. If $f$ is tame and $\omega \in (\Omega_{Y/S})_{\eta}$ is a meromorphic differential such that $f^* \omega \in H^0(X, \Omega_{X/S}^{\vee \vee})$ is a global reflexive differential then $\omega \in H^0(Y, \Omega_{Y/S}^{\vee \vee}$ is a global reflexive differential.
\end{prop}

\begin{proof}
Since $Y$ is regular in codimension $1$ it suffices to show that for each $\xi \in Y$ of height $1$ that $\omega_\xi \in (\Omega_Y)_{\xi}$. Since $f$ is proper and dominant it is surjective so we may choose $\xi' \in X$ mapping to $\xi$. The fiber over a divisor must contain a divisor of $X$ so we can choose  $\xi'$ in the smooth locus. locus hence $f^* \omega$ is a well-defined differential form over $\stalk{X}{\xi'}$.  Since $\stalk{X}{\xi'}$ is a noetherian local domain by [Hartshorne, Ex.4.11] there exists a DVR $R \subset \Frac{\stalk{X}{\xi'}}$ dominating $\stalk{X}{\xi'}$. 

{\color{red} FINISH}

\end{proof}

\begin{rmk}
For example, this holds for any tame dominant map of normal proper varities over a perfect field. 
\end{rmk}

{\color{red} COUNTEREXAMPLES}


\subsubsection{Completion of the Theorem}

Either, for all $n > 0$ we have $h^0(X, L^{\ot n}) \le 1$ or there exists $n > 0$ such that $h^0(X, L^{\ot n}) \ge 2$. In the second case, there is a standard method of extracting an $n^{\text{th}}$-root {\color{red} (WHAT THE HELL DOES THIS MEAN)} to get $h^0(X, L) \ge 2$. In this case, there are two forms $\omega_1, \omega_2 \in H^0(X, \Omega^1_X)$ such that $\omega_1 \wedge \omega_2 = 0$ since they arise from the same subsheaf of rank $1$. Therefore, by the lemma, there exists a curve $C$ and a morphism $u : X \to C$ and an invertible sheaf $L_0$ on $C$ such that,
\[ L \subset u^*(L_0) \]
{\color{red} AGAIN WHY?}
so we can conclude that,
\[ h^0(X, L^n) \le h^0(C, L_0^n) \in O(n) \]

\begin{cor}
If $c_1$ and $c_2$ are the Chern classes of $\Omega^1_X$ then $c_1^2 \le 4 c_2$. 
\end{cor}

\subsubsection{Curves of bounded genus on a minimal surface of general type}

We provide a few examples showing that $X$ being general type plays an essential role, and that in the contrary case, there can be unbounded families of curves of fixed geometric genus.

\begin{example}
Let $X = \P^2$ then $\NS{X} = \Z$. There exist in the projective plane curves of bounded geometric genus but arbitrarily large degree.
\end{example} 

\begin{example}
Let $E$ be an elliptic curve without complex multiplication and let $X = E \times E$. Then $\NS{X} = \Z f_1 \oplus \Z f_2 \oplus \Z \Delta$ where $f_i$ are the fiber classes and $\Delta$ is the diagonal. For every pair of integers $(m, n)$ the image in $X$ od the morphism $f_{m,n} : E \to X$ given by $f_{m,n}(\alpha) = (m \alpha, n \alpha)$ is a curve of class $m^2 f_1 + n^2 f_2 + (m-n)^2 \Delta$ and genus $1$.
\end{example}

\begin{example}
Let $B$ be a smooth proper curve and $\pi : X \to B$ a nonisotrivial (that is to say it does not become trivial after some finite base change $B' \to B$) minimal elliptic fibration admitting a section $\sigma : B \to X$ of infinite order. Let $\omega$ be the conormal bundle of $\sigma$. Then there exist global sections $g_2 \in H^0(X, \omega^4)$ and $g_3 \in H^0(X, \omega^6)$ such that $X$ is the minimal resolution of the surface $Y \subset \P_B(\omega^2 \oplus \omega^3 \oplus \cO_B)$ defined by the Weierstrass equation,
\[ y^2 z = x^3 - g_2 x z^2 - g_3 z^3 \]
Morover, $\omega$ is independent of the section $\sigma$ as has degree $- \sigma(B)^2$. If the degree is zero, then $g_2$ and $g_3$ are constant and the fibration $\pi$ is isotrivial. There exist infinitely many sections of negative self-intersection and the classes are algebraically distict. 
\end{example}

Notation: we write $K_X$ for a canonical divisor of $X$ and $T_X$ the tangent bundle and $\pi : \P(\Omega_X) \to X$ the canonical projection and $L = \cO_P(1)$ the relatively ample bundle for $\pi$.
\par 
Let $F$ be a invertible bundle on $X$. We note that $F$ is a divisor of some linear system (DOES HE MEAN $F$ IS THE ZERO LOCUS OF SOME SECTION). Moreover, for any rational number $\ell \in \QQ$, we allow ourselves to form the sheaf $\ell F$, extending that we consider the tensor powers $(\ell F)^{\ot m}$ for which $m$ is such that $m \ell$ is an integer.


\subsection{COnstruction of a good linear system of divisors on $P$}

\begin{prop}
Let $F$ be an invertible sheaf on $X$ and $\ell$ a rational positive number such that,
\begin{enumerate}
\item $K \cdot F \ge 0$
\item $(K + 2 \ell F)^2 > 0$
\item $c_1^2(\Omega_X \ot \ell F) - c_2(\Omega_X \ot \ell F) > 0$
\end{enumerate}
Then for $m \gg 0$ the linear system $(L \ot \pi^* (\ell F))^m$ defines a rational map $u_F : \P(\Omega_X^1) \rat \P^N$ birational onto its image.
\end{prop}

{\color{red} IT SEEMS WRONG THAT $\ell F$ IS INSIDE THE $S^m$ THIS GIVES $(\ell F)^{2m}$ NOT $(\ell F)^{m}$ AS SHOULD BE FROM PROJECTION FORMULA}

\begin{proof}
By the theorem of Iitaka [20], it suffices to show that for $m \gg 0$,
\[ h^0(P, (L \ot \pi^* (\ell F))^m) = h^0(X, S^m(\Omega_X \ot \ell F)) \ge O(m^3) \]
The Riemann-Roch formula for $E$ shows that,
\[ \chi(S^m E) = \frac{m (m+1)(m+2)}{24} (c_1^2(E) - 4 c_2(E)) + \frac{m+1}{2} \left[ \frac{m^2}{4} c_1^2(E) - \frac{m}{2} K_X \cdot c_1(E) \right]  + (m + 1) \chi(\cO_X) \]
and hence for $m \gg 0$,
\[ h^0(S^m(\Omega_X^1 \ot \ell F)) + h^2(S^m(\Omega_X^1 \ot \ell F)) \sim h^1(S^m(\Omega_X^1 \ot \ell F)) + \frac{m^3}{6} \left[ c_1^2(\Omega_X \ot \ell F) - c_2(\Omega_X^1 \ot \ell F) \right] \ge O(m^3) \]
By Serre duality, {\color{red} HOW DO I FIX THE DUAL AND $S^m$ IN POSITIVE CHAR}
\[ h^2(S^m(\Omega_X \ot \ell F)) = h^0(K \ot S^m(T_X \ot -\ell F)) \]
Chosing some divisors $D$ and $D'$ ample and smooth such that,
\[ \cO_X(-D') \subset K \subset \cO_X(D) \]
we find that,
\[ \bigg| h^0(K \ot S^m(T_X \ot -\ell F)) - h^0(S^m(T_X \ot -\ell F)) \bigg| \in O(m^2) \]
Therefore, we conclude by appealing to the following lemma.
\end{proof}

\begin{lemma}
For any $m > 0$ we have $H^0(S^m(T_X \ot - \ell F)) = 0$.
\end{lemma}

{\color{red} THE $m$ VS $2m$ DOESNT MAKE SENSE}

\begin{proof}
We showed that $T_X \ot -\ell F$ is not unstable. Hence, the only sections of $H^0(S^{2m}(T_X \ot -\ell F) \ot (\det{(T_X \ot -\ell F)})^{-m})$ are nowhere vanishing. If we show for $m \gg 0$ that $H^0(\det{(T_X \ot - \ell F})^{-m})$ has a nonzero section with a zero at some point $x \in X$ then its product with a section $H^0(S^{2m}(T_X \ot - \ell F))$ will give a contradiction.
Thus, the result will follow from the definition,
\[ \det{(T_X \ot - \ell F)^{-m}} = m (K + 2 \ell F) \]
and Riemann-Roch,
\[ \chi(m(K + 2 \ell F)) \sim \frac{m^2}{2} (K + 2 \ell F)^2 \in O(m^2) \]
and therefore,
\[ h^0(m(K + 2 \ell F)) + h^2(m(K + 2 \ell F)) \ge O(m^2) \]
by Serre duality,
\[ h^0(m(K + 2 \ell F)) = h^0(K - m(K + 2 \ell F)) \]
Since $K \cdot (K - m(K + 2 \ell F)) = K^2 - m K \cdot (K + 2 \ell F) < 0$ and $K$ is nef (we assumed that $X$ is minimal) $h^0(K - m (K + 2 \ell F)) = 0$ for $m \gg 0$ giving the result.
\end{proof}

Our any bundle $F$ verifying the conditions of the properosition, we fix, once and for all, $m$ and $\ell$ and let $Z_F$ be the closed subset of $\P(\Omega_X)$ outside of which $u_F$ is defined.

\begin{defn}
Let $C$ be a curve embedded in $C$ and $f : \wt{C} \to C$ its normalization. If $t_f(\wt{C})$ is not contained in (resp. is contained in) $Z_F$, we say that $C$ is $F$-regular (resp. $F$-irregular).
\end{defn}

\subsection{Proof of Theorem 0.4}

We suppose that $\L \embed \Omega_X^1$ is a invertible subsheaf. If $h^0(X, \L^{\ot n}) \le 1$ for all $n$ then we are done. Otherwise, there is some $n > 0$ such that $h^0(X, \L^{\ot n}) \ge 2$. In this case, by passing to a cyclic cover we may assume that $h^0(X, \L) \ge 2$. Therefore, there are two independent $1$-forms $\omega_1, \omega_2 \in H^0(X, \L) \subset H^0(X, \Omega^1_X)$ such that $\omega_1 \wedge \omega_2 = 0$ because they lie in the same $1$-dimensional subspace at the generic point $\L_{\eta} \subset \Omega_{X, \eta}^1$. Therefore, we may apply Castelnuovo's lemma to produce a morphism $f : X \to C$ to some curve of genus $g \ge 2$ with $\omega_1, \omega_2$ pulled back along $f$. By the proof of this lemma, we see that any local section of $\L$ is pulled back along $f$ hence $\L \embed f^* \Omega_C$.  

\subsubsection{Ramified Cyclic Covers}

Let $X$ be a scheme and $\L \in \Pic{X}$ a line bundle and $s \in H^0(X, \L^{\ot n})$ a nonzero section of some tensor power. Then we may form a finitely-presented sheaf of $\struct{X}$-algebras,
\[ \cA = \struct{X} \oplus t \L^{\ot -1} \oplus \cdots \oplus t^{n-1} \L^{\ot -(n-1)} \]
where multiplication is defined in the obvious manner,
\[ (t^a f_1) (t^b t_2)  = 
\begin{cases}
t^{a+b} f_1 f_2 & a + b < n
\\
t^{a + b - nk} [(s^\vee)^{\ot k} \ot \id](f_1 f_2) & nk \le a + b < (n+1)k
\end{cases}  \]
where $[(s^\vee)^{\ot k} \ot \id] : \L^{\ot -(a+b)} \to \L^{\ot -(a + b - nk)}$. Then we define $X_{\L, s} := \rSpec{X}{\cA}$. Over the locus where $s$ is nonvanishing it is clear that $X_{\L, s} \to X$ is a degree $n$ cyclic cover which is \etale for $n$ nonzero in the base scheme. 
\bigskip\\
Note that $\cA$ can also be described as follows. Consider the symmetric algebra,
\[ \Sym{\bullet}{\L^\vee} = \bigoplus_{n = 0}^\infty t^n \L^{\ot -n} \onto \cA \]
which is the quotient as a sheaf of algebras by the ideal generated by $(t^n f - s^\vee(f))$ for local sections $f$ of $\L^{\ot -n}$. Therefore, $X_{\L, s} \embed \V_X(\L)$ is a closed subscheme of the total space of the line bundle $\L$ which can be described as the locus of points $(x, v)$ such that $v^n = s(x)$.
\bigskip\\
Note that under $\pi : \V_X(\L) \to X$ we get a canonical section $t \in H^0(\V_X(\L), \pi^* \L)$ and hence for $f : X_{\L, s} \to X$ there is a canonical section $t \in H^0(X_{\L, s}, f^* \L)$ such that $t^n = f^* s$. 
\bigskip\\
Now suppose that $s_1, s_2 \in H^0(X, \L^{\ot n})$ are two independent sections. Then by passing to the iterating cyclic cover, $X' = (X_{\L, s_1})_{f^* \L, f^* s_2} \to X_{\L, s_1} \to X$ we get $\L' = f^* \L$ and two canonical sections $t_1, t_2 \in H^0(X', \L')$ such that $t_i^n = f^* s_i$ for $i = 1,2$. 
\bigskip\\
Furthermore, suppose that $n$ is invertible on the base and there is an injection $\L \embed \Omega_X^1$. Then passing to the cyclic cover (which is generically \etale) we get $f^* \L \embed f^* \Omega_X^1 \embed \Omega_{X'}^1$ which is injective because it is at the generic point. Hence we reduce to the situation that $h^0(X, \L) \ge 2$.

\section{Study of Curves}

\begin{defn}
An algebraic foliation on $X$ is defined by an invertible subsheaf $L$ of the sheaf $\Omega_X^1$. It is called \textit{saturated} if $L$ is a saturated subsheaf of $\Omega_X^1$.
\end{defn}

\begin{defn}
Let $C$ be a curve on $X$ and $\wt{C}$ the normalization, let $L \subset \Omega_X^1$ be an algebraic foliation. We say that $C$ is an integral curve of $L$ if the composition
\[ L|_{\wt{C}} \to \Omega^1_X |_{\wt{C}} \to \Omega^1_{\wt{C}} \]
is zero. The integral curves of a foliation are also integral curves of the associated saturated foliation and so are the loci where the map $L \to \Omega_X^1$ is zero. 
\end{defn}

\begin{prop}
There exsits a finite number of surfaces smooth proper surfaces $Z_i$ dominating $X$, each equipped with an algebraic foliation, and such that the $F$-irregular curves of $X$, except for finitely many, are the images in $X$ of integral curves of these foliations.
\end{prop}

\begin{proof}
The irreducible component of the closed set $Z_F \subset \P(\Omega_X^1)$ are curves or surfaces. The curves $C$ such that $t_f(\wt{C})$ is contained in the some component that does not dominate $X$ are finite in number.
\par 
Let $Z_i$ be the desginularization of the components that dominate $X$ and $\alpha_i : Z_i \to X$ the resoluting dominant morphisms, and $C$ a curve which lifts to $Z_i$
\begin{center}
\begin{tikzcd}
Z_i \arrow[r] \arrow[rd, "\alpha_i"] & \P(\Omega_X^1) \arrow[d, "\pi"] 
\\
\wt{C} \arrow[u, "g"] \arrow[r, "f"] & X
\end{tikzcd}
\end{center}
On $\P(\Omega_X^1)$, we have by construction an exact sequence
\[ 0 \to L' \t \pi^* \Omega_X^1 \to L \to 0 \]
where $L'$ is an invertible sheaf. 
\par 
The injection
\[ L'|_{Z_i} \to \alpha_i^* \Omega_X^1 \to \Omega^1_{Z_i} \]
defines an algebraic foliation and we easily verify that $g(\wt{C})$ is an integral curve.
\end{proof}

To finish the proof of Theorem 0.4, two methods are possilbe -- we can, as is done by Bogomolov, use the following analyitc result

\begin{theorem}[Seidenberg]
Lt $A$ and $B$ be two elements of the complete local ring $R = k[[x,y]]$. There exists a scheme $S$ ........
\end{theorem}
 
this shows that the famile of curves $(C_i)_{i \in I}$ .... one a surface is bounded. 
 
Or, one can use the .. .


\begin{theorem}[Jouanolou]
Let $L \to \Omega_X^1$ be an algebraic foliation on a smooth projective variety $X$ over $\CC$. If there exist infinitely many integral hypersurfaces, they are contained in the fibers of a contraction.
\end{theorem} 
 
\end{document}