\documentclass[12pt]{article}
\usepackage{import}
\import{./}{AlgGeoCommands}

\begin{document}

\begin{remark}
Unless otherwise stated, all rings are commutative and unital.
\end{remark}

\section{Definitions}

\begin{definition}
An element $p \in A$ is prime if $(p)$ is a prime ideal. Equivalently $p$ is prime if whenever $p \divides xy$ either $p \divides x$ or $p \divides y$.
\end{definition}

\begin{definition}
An element $r \in A$ which is nonzero and not a unit is irreducible if whenever $r = xy$ either $x \in A^\times$ or $y \in A^\times$. 
\end{definition}

\section{Domains}

\begin{definition}
A ring $A$ is a domain if $A$ has no zero divisors i.e. if $ab = 0$ then $a = 0$ or $b = 0$.
\end{definition}

\begin{proposition}
Let $A$ be a domain then any nonzero prime element is irreducible. 
\end{proposition}

\begin{proof}
Let $p \in A$ be a prime. Now suppose that $p = xy$ for $x,y \in A$. Thus, $p \divides xy$ so (WLOG) we have $p \divides x$ so $x = pz$ and thus $p = pzy$. However, $p$ is nonzero and $A$ is a domain so $zy = 1$ and thus $y \in A^\times$ proving that $p$ is irreducible. 
\end{proof}

\section{Principal Ideal Domains}

\begin{definition}
A principal ideal domain (PID) is a domain $A$ such that every ideal is principal. 
\end{definition}

\begin{lemma}
If $A$ is a PID then $A$ is Noetherian.
\end{lemma}

\begin{proof}
Every ideal is principal and thus finitely generated.
\end{proof}

\begin{lemma}
Let $A$ be a PID and $r \in A$ irreducible then $(r)$ is maximal and thus $r$ is prime. 
\end{lemma}

\begin{proof}
Consider an intermediate ideal $(r) \subset J \subset A$ then since $A$ is a PID we have $J = (a)$ so $r \in (a)$ and thus $r = ac$ so either $a \in A^\times$ in which case $J = A$ or $c \in A^\times$ in which case $J = (r)$ so $(r)$ is maximal and thus a prime ideal.
\end{proof}

\begin{theorem}
Let $A$ be a PID and not a field then $\dim{A} = 1$.
\end{theorem}

\begin{proof}
Any prime ideal $\p \subset A$ is principal so $\p = (p)$ and $p$ is prime. Either $p = 0$ which is prime since $A$ is a domain or $p$ is irreducible and so we have shown $(p)$ is maximal. So every prime ideal is zero or maximal and thus $\dim{A} \le 1$. If $\dim{A} = 0$ then $(0)$ is maximal so $A$ is local and any nonzero element is thus invertible so $A$ is a field. 
\end{proof}

\begin{theorem}[Kaplansky]
Let $A$ be Noetherian then $A$ is a principal ideal ring iff every maximal ideal is prime.
\end{theorem}

\begin{theorem}[Cohen]
A ring $A$ is Noetherian iff every prime ideal is finitely generated.
\end{theorem}

\begin{corollary}
A ring $A$ is a principal ideal ring iff every prime ideal is principal. 
\end{corollary}

\section{Unique Factorization Domains}

\begin{definition}
A domain $A$ is a unique factorization domain (UFD) if every nonzero element has a unique factorization into irreducible elements. 
\end{definition}

\begin{definition}
A factorization ring $A$ is a ring such that every nonzero element has a factorization into irreducible elements.
\end{definition}

\begin{lemma}
If $A$ is a Noetherian domain then it is a factorization domain.
\end{lemma}

\begin{proof}
Take $a_0 \in A$. If $a$ is irreducible, zero, or a unit then we are done. Then we can write, $a = a^{(1)}_1 a^{(1)}_2$ for $a_1, b_1 \notin A^\times$. Continuing in this manner we get,
\[ (a) \subsetneq (a^{(1)}_1, a^{(1)}_2) \subsetneq (a^{(2)}_1, a^{(2)}_2, a^{(2)}_3, a^{(3)}_4) \subsetneq \cdots \]
(CHECK THIS)
This sequence is proper since if $a = bc$ and $b \in (a)$ then $a = arc$ so $rc = 1$ and thus $c \in A^\times$ contradicting our construction. However, $A$ is Noetherian then the sequence must terminate so at some point the factorization must become irreducible. 
\end{proof}

\begin{theorem}
Let $A$ be a factorization domain. Then $A$ is a UFD iff every irreducible is prime. 
\end{theorem}

\begin{proof}
If $A$ is a UFD and $p$ an irreducible. Let $x, y \in A$ and $p \divides xy$ then $p$ is in the factorization of $xy$ and thus, by uniqueness must be in the factorization of either $x$ or $y$ so $p \divides x$ or $p \divides y$.
\bigskip\\
Conversely, if $A$ is a factorization domain and every irreducible is prime then given two factorizations of $x$ each irreducible must, by primality, divide an irreducible in the other factorization so they are equal. 
(DO THIS BETTER)
\end{proof}

\begin{corollary}
If $A$ is a PID then $A$ is a UFD.
\end{corollary}

\begin{proof}
If $A$ is a PID then it is Noetherian and thus a factorization domain. Furthermore, every irreducible is prime so $A$ is a UFD.
\end{proof}

\subsection{Height One Prime Ideals}

\begin{proposition}
Let $A$ be a Noetherian ring. Then any principal prime ideal has height at most one.
\end{proposition}

\begin{proof}
Let $\p = (p) \subset A$ be a principal prime ideal. Then consider the localization which is $A_{(p)}$ Noetherian and the unique maximal ideal $p A_{(p)}$ is principal. Take $N = \nilrad{A_{(p)}}$ then,
\[ \dim{A_{(p)}/N} = \dim{A_{(p)}} = \height{\p} \]
but $A_{(p)} / N$ is a Noetherian domain and the unique maximal ideal $p A_{(p)}$ is principal so $A_{(p)} / N$ is a PID and thus $\dim{A_{(p)} / N} \le 1$. 
\end{proof}

\begin{proposition}
If $A$ is a UFD then every prime ideal of height one is principal.
\end{proposition}

\begin{proof}
Let $\p \subset A$ be a prime ideal with $\height{\p} = 1$. Take any nonzero element $x \in \p$ and consider its factorization into irreducibles. Since $\p$ is prime some irreducible factor $p \divides x$ must be in $\p$ so $(p) \subset \p$. Since $A$ is a UFD all irreducibles are prime so $(p) \subset \p$ is prime. However $\height{\p} = 1$ and $(p) \neq (0)$ so $(p) = \p$ and thus $\p$ is principal.
\end{proof}

\begin{theorem}
Let $A$ be a Noetherian domain. Then $A$ is a UFD iff every height one prime ideal is principal. 
\end{theorem}

\begin{proof}
We showed one direction above. Conversely, suppose every height one prime ideal is principal. Since $A$ is a Noetherian domain, it suffices to show that each irreducible is prime. Let $r$ be irreducible and consider a minimal prime $\p \supset (r)$. Then by Krull's Hauptidealsatz, $\p$ has height one so by our assumption $\p = (p)$ is principal. However, $(r) \subset (p)$ so $p \divides r$ but $r$ is irreducible so we must have $(r) = (p) = \p$ and thus $r$ is prime.
\end{proof}

\begin{theorem}[Krull's Hauptidealsatz]
Let $I \subset A$ be an ideal in a Noetherian ring $A$ with $n$ generators then any minimal prime ideal $\p \supset I$ has height at most $n$.
\end{theorem}

\end{document}