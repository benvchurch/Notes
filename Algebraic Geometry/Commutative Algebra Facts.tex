\documentclass[12pt]{article}
\usepackage{import}
\import{./}{AlgGeoCommands}

\begin{document}

\title{Commutative Algebra Facts for Algebraic Geometry}

\maketitle

\tableofcontents

\begin{remark}
Unless otherwise stated, all rings are commutative and unital.
\end{remark}

\section{Definitions}

\begin{definition}
An element $p \in A$ is prime if $(p)$ is a prime ideal. Equivalently $p$ is prime if whenever $p \divides xy$ either $p \divides x$ or $p \divides y$.
\end{definition}

\begin{definition}
An element $r \in A$ which is nonzero and not a unit is irreducible if whenever $r = xy$ either $x \in A^\times$ or $y \in A^\times$. 
\end{definition}

\section{Domains}

\begin{definition}
A ring $A$ is a domain if $A$ has no zero divisors i.e. if $ab = 0$ then $a = 0$ or $b = 0$.
\end{definition}

\begin{proposition}
Let $A$ be a domain then any nonzero prime element is irreducible. 
\end{proposition}

\begin{proof}
Let $p \in A$ be a prime. Now suppose that $p = xy$ for $x,y \in A$. Thus, $p \divides xy$ so (WLOG) we have $p \divides x$ so $x = pz$ and thus $p = pzy$. However, $p$ is nonzero and $A$ is a domain so $zy = 1$ and thus $y \in A^\times$ proving that $p$ is irreducible. 
\end{proof}

\section{Principal Ideal Domains}

\begin{definition}
A principal ideal domain (PID) is a domain $A$ such that every ideal is principal. 
\end{definition}

\begin{lemma}
If $A$ is a PID then $A$ is Noetherian.
\end{lemma}

\begin{proof}
Every ideal is principal and thus finitely generated.
\end{proof}

\begin{lemma}
Let $A$ be a PID and $r \in A$ irreducible then $(r)$ is maximal and thus $r$ is prime. 
\end{lemma}

\begin{proof}
Consider an intermediate ideal $(r) \subset J \subset A$ then since $A$ is a PID we have $J = (a)$ so $r \in (a)$ and thus $r = ac$ so either $a \in A^\times$ in which case $J = A$ or $c \in A^\times$ in which case $J = (r)$ so $(r)$ is maximal and thus a prime ideal.
\end{proof}

\begin{theorem}
Let $A$ be a PID and not a field then $\dim{A} = 1$.
\end{theorem}

\begin{proof}
Any prime ideal $\p \subset A$ is principal so $\p = (p)$ and $p$ is prime. Either $p = 0$ which is prime since $A$ is a domain or $p$ is irreducible and so we have shown $(p)$ is maximal. So every prime ideal is zero or maximal and thus $\dim{A} \le 1$. If $\dim{A} = 0$ then $(0)$ is maximal so $A$ is local and any nonzero element is thus invertible so $A$ is a field. 
\end{proof}

\begin{theorem}[Kaplansky]
Let $A$ be Noetherian then $A$ is a principal ideal ring iff every maximal ideal is prime.
\end{theorem}

\begin{theorem}[Cohen]
A ring $A$ is Noetherian iff every prime ideal is finitely generated.
\end{theorem}

\begin{corollary}
A ring $A$ is a principal ideal ring iff every prime ideal is principal. 
\end{corollary}

\section{Unique Factorization Domains}

\begin{definition}
A domain $A$ is a unique factorization domain (UFD) if every nonzero element has a unique factorization into irreducible elements. 
\end{definition}

\begin{definition}
A factorization ring $A$ is a ring such that every nonzero element has a factorization into irreducible elements.
\end{definition}

\begin{lemma}
If $A$ is a Noetherian domain then it is a factorization domain.
\end{lemma}

\begin{proof}
Take $a_0 \in A$. If $a$ is irreducible, zero, or a unit then we are done. Then we can write, $a = a^{(1)}_1 a^{(1)}_2$ for $a_1, b_1 \notin A^\times$. Continuing in this manner we get,
\[ (a) \subsetneq (a^{(1)}_1, a^{(1)}_2) \subsetneq (a^{(2)}_1, a^{(2)}_2, a^{(2)}_3, a^{(3)}_4) \subsetneq \cdots \]
(CHECK THIS)
This sequence is proper since if $a = bc$ and $b \in (a)$ then $a = arc$ so $rc = 1$ and thus $c \in A^\times$ contradicting our construction. However, $A$ is Noetherian then the sequence must terminate so at some point the factorization must become irreducible. 
\end{proof}

\begin{theorem}
Let $A$ be a factorization domain. Then $A$ is a UFD iff every irreducible is prime. 
\end{theorem}

\begin{proof}
If $A$ is a UFD and $p$ an irreducible. Let $x, y \in A$ and $p \divides xy$ then $p$ is in the factorization of $xy$ and thus, by uniqueness must be in the factorization of either $x$ or $y$ so $p \divides x$ or $p \divides y$.
\bigskip\\
Conversely, if $A$ is a factorization domain and every irreducible is prime then given two factorizations of $x$ each irreducible must, by primality, divide an irreducible in the other factorization so they are equal. 
(DO THIS BETTER)
\end{proof}

\begin{corollary}
If $A$ is a PID then $A$ is a UFD.
\end{corollary}

\begin{proof}
If $A$ is a PID then it is Noetherian and thus a factorization domain. Furthermore, every irreducible is prime so $A$ is a UFD.
\end{proof}

\subsection{Height One Prime Ideals}

\begin{proposition}
Let $A$ be Noetherian. Then any principal prime ideal has height at most one.
\end{proposition}

\begin{proof}
Let $\p = (p) \subset A$ be a principal prime ideal. Then consider the localization which is $A_{(p)}$ Noetherian and the unique maximal ideal $p A_{(p)}$ is principal. Take $N = \nilrad{A_{(p)}}$ then,
\[ \dim{A_{(p)}/N} = \dim{A_{(p)}} = \height{\p} \]
but $A_{(p)} / N$ is a Noetherian domain and the unique maximal ideal $p A_{(p)}$ is principal so $A_{(p)} / N$ is a PID and thus $\dim{A_{(p)} / N} \le 1$. 
\end{proof}

\begin{proposition}
If $A$ is a UFD then every prime ideal of height one is principal.
\end{proposition}

\begin{proof}
Let $\p \subset A$ be a prime ideal with $\height{\p} = 1$. Take any nonzero element $x \in \p$ and consider its factorization into irreducibles. Since $\p$ is prime some irreducible factor $p \divides x$ must be in $\p$ so $(p) \subset \p$. Since $A$ is a UFD all irreducibles are prime so $(p) \subset \p$ is prime. However $\height{\p} = 1$ and $(p) \neq (0)$ so $(p) = \p$ and thus $\p$ is principal.
\end{proof}

\begin{theorem}
Let $A$ be a Noetherian domain. Then $A$ is a UFD iff every height one prime ideal is principal. 
\end{theorem}

\begin{proof}
We showed one direction above. Conversely, suppose every height one prime ideal is principal. Since $A$ is a Noetherian domain, it suffices to show that each irreducible is prime. Let $r$ be irreducible and consider a minimal prime $\p \supset (r)$. Then by Krull's Hauptidealsatz, $\p$ has height one so by our assumption $\p = (p)$ is principal. However, $(r) \subset (p)$ so $p \divides r$ but $r$ is irreducible so we must have $(r) = (p) = \p$ and thus $r$ is prime.
\end{proof}

\begin{theorem}[Krull's Hauptidealsatz]
Let $I \subset A$ be an ideal in a Noetherian ring $A$ with $n$ generators then any minimal prime ideal $\p \supset I$ has height at most $n$.
\end{theorem}

\section{Simple Modules}

\begin{defn}
A nonzero $R$-module is \textit{simple} if it has no nontrivial submodules.
\end{defn}

\begin{prop}
Let $R$ be a ring and $M$ an $R$-module. Then the following are equivalent,
\begin{enumerate}
\item $M$ is simple
\item $\ell_R(M) = 1$
\item $M = R / \m$ for some maximal ideal $\m \subset R$.
\end{enumerate}
\end{prop}

\begin{proof}
The first two are equivalent by definition. Clearly if $\m \subset R$ is maximal then $R / \m$ is simple. Now suppose that $M$ is simple and take a nonzero $x \in M$. Then $(x) = M$ by simplicity so consider $I = \ker{(R \xrightarrow{x} M)} = \Ann{A}{x} = \{r \in R \mid r x = 0\}$. Since $M = R x$ we know that $M \cong R / I$. However, by the lattice isomorphism theorem, submodules of $R / I$ correspond to ideals above $I$ so since $M$ is simple we must have $I$ maximal.  
\end{proof}

\section{Artinian Modules}

\begin{defn}
An $R$-module $M$ is \textit{noetherian/artinian} if it satisfies the ascending/descending chain condition on submodules.
\end{defn}

\begin{theorem}
An $R$-module $M$ has finite length iff it is both noetherian and artinian.
\end{theorem}

\begin{proof}
If $M$ has finite length then clearly it is noetherian and artinian since chains of submodules are bounded in length. Alternatively, simple modules are noetherian and artinian  so given a composition series we see that $M$ is noetherian and artinian by repeated extension. Now, conversely, assume that $M$ is noetherian and artinian. By the artinian property we can take a minimal nonzero submodule $M_1 \subset M$. Then $M_1$ is simple. Either $M / M_1$ is simple or we may repeat to get $M_2 \supset M_1$ and $M_2 / M_1$ is simple. Thus we get an ascending chain $0 = M_0 \subset M_1 \subset M_2 \subset M_3 \subset  \cdots$ with $M_{i+1}/M_i$ simple. Since $M$ is Noetherian, this must terminate at $M_n = M$ so we get a finite length composition series showing that $M$ has finite length.
\end{proof}

\section{Artinian Rings}

\begin{defn}
A ring $A$ is \textit{artinian} if it satisfies the descending chain condition on ideals: given a chain of ideals,
\[ I_0 \supset I_1 \supset I_2 \supset \cdots \]
the chain stabilizes $I_{n+i} = I_n$. 
\end{defn}

\begin{rmk}
$A$ is artinian iff it is artinian as a module over itself.
\end{rmk}

\begin{prop}
An artinian ring has finitely many maximal ideals.
\end{prop}

\begin{proof}
Let $\m_1, \m_2, \m_3, \dots$ be a list of maximal ideals. Then consider the chain,
\[ \m_1 \supset \m_1 \m_2 \supset \m_1 \m_2 \m_3 \supset \cdots \]
By the artinian condition, we must have $\m_1 \cdots \m_n = \m_1 \cdots \m_n \m_{n+1}$ for some $n$. But then by prime avoidence $\m_{n+1}$ must be one of $\m_1, \dots, \m_n$ since $\m_{n+1} \supset \m_1 \cdots \m_n$ so $\m_{n+1} \supset \m_i$ and $\m_i$ is maximal.
\end{proof}



\begin{prop}
Let $A$ be an artinian ring. Then every prime ideal is maximal so $\dim{A} = 0$.
\end{prop}

\begin{proof}
Let $\p$ be prime and $x \notin \p$. Consider the chain,
\[ (x) \supset (x^2) \supset (x^3) \supset \cdots \]
By the artinian condition $(x^n) = (x^{n+1})$ for some $n$ so $x^n = r x^{n+1}$ for some $r \in A$. Thus,
\[ x^n(rx - 1) = 0 \] 
However, $x^n \notin \p$ so $rx - 1 \in \p$ and thus $x \in A / \p$ is invertible so $A / \p$ is a field and thus $\p$ is maximal.
\end{proof}

\begin{prop}
Let $A$ be artinian. Then $\nilrad{A}$ is a nilpotent ideal.
\end{prop}

\begin{proof}
Let $I = \nilrad{A}$. Consider the chain of ideals,
\[ I \supset I^2 \supset I^3 \supset \cdots \]
By the artinian condition, $I^{n+1} = I^n$ for some $n$. Consider $J = \{ x \in A \mid x I^n = 0 \}$. If $J \neq R$ we can choose $J' \supsetneq J$ minimal (using the artinian property). Then take $y \in J'$ so by minimality $J' = J + (y)$. Suppose $J + I(y) = J'$ then, since $J \subset \Jac{A}$ and $(y)$ is finitely generated, by Nakayama, $J' = J + I(y) = J$ which is false so $J \subset J + I(y) \subsetneq J'$ and thus $J = J + I(y)$ by minimality so $I(y) \in J$. Therefore, $y \cdot I^{n+1} = 0$ but $I^{n+1} = I^n$ so $y \cdot I^n = 0$ and thus $y \in J$ contradicting our situation so $J = R$ and thus $I^n = 0$.
\end{proof}

\begin{prop}
Every artinian ring is a product of local artinian rings: $A_{\m_i} = A / \m_i^n$.
\end{prop}

\begin{proof}
Let $\m_1, \dots, \m_r$ be the maximal ideals. Then we know that $\m_1^{n_1} \cdots \m_r^{n_r} = 0$ for some integers $n_1, \dots, n_r \in \Z$. Therefore, by the Chinese remainder theorem,
\[ A = A / (\m_1^{n_1} \cdots \m_r^{n_r}) = \prod_{i = 1}^r A / \m_i^{n_i} \]
Furthermore, $A / \m_i^{n_i}$ is local because $\m_i$ is the only maximal ideal above $\m_i^{n_i}$. Furthermore, 
\[ A_{\m_i} = (A / \m_i^{n_i})_{\m_i} = A / \m_i^{n_i} \]
since $A \setminus \m_i$ is not contained in any maximal ideal of $A / \m_i^{n_i}$ and thus is invertible.   
\end{proof}

\begin{prop}
A ring $A$ is artinian iff it has finite length as a module over itself.
\end{prop}

\begin{proof}
If $A$ has finite length as an $A$-module then it satisfies both the ascending and descending chain conditions on $A$-submodules i.e. ideals thus $A$ is both noetherian and artinian. Conversely, let $A$ be artinian. Since $A$ is a finite product of local artinian rings we may reduce to the case that $A$ is local artinian with maximal ideal $\m$. Since $\nilrad{A} = \m$ then $\m^n = 0$ for some $n$ so we get a series,
\[ 0 = \m^n \subset \m^{n-1} \subset \cdots \subset \m \subset A \]
Then $\m^i / \m^{i+1}$ is a $A / \m$-module and,
\[ \ell_A(\m^i / \m^{i+1}) = \ell_{A/\m}(\m^i / \m^{i+1}) = \dim_{A/\m} \m^i / \m^{i+1} \]
which must be finite since $\m^i / \m^{i+1}$ is an artinian module and thus must have finite dimension else there would be a nonterminating descending chains. Thus from the series $A$ has finite length. 
\end{proof}

\begin{theorem}
A ring $A$ is artinian iff $A$ is noetherian and $\dim{A} = 0$.
\end{theorem}

\begin{proof}
If $A$ is artinian then it has finite length over itself and thus is noetherian. Also every prime is maximal so $\dim{A} = 0$. Conversely, suppose that $A$ is noetherian and $\dim{A} = 0$. Then $\Spec{A}$ is a noetherian topological space which has finitely many irreducible componets so $A$ has finitely many minimal primes which are also maximal since $\dim{A} = 0$. Thus $A$ has finitely many primes all of which are maximal. Since $\dim{A} = 0$ we have $I = \Jac{A} = \nilrad{A}$ so any $f \in I$ is nilpotent so $I$ is nilpotent because $A$ is noetherian so $I$ is finitely generated. Thus by the Chines remainder theorem $A$ is a finite product of local rings so we reduce to the case that $A$ is local with maximal ideal $\m$. Then we get a series,
\[ 0 = \m^n \subset \m^{n-1} \subset \cdots \subset \m \subset A \]
but $\m^i / \m^{i+1}$ is a finite $A / \m$-module since $A$ is noetherian so,
\[ \ell_A(\m^i / \m^{i+1}) = \ell_{A/\m}(\m^i / \m^{i+1}) = \dim_{A/\m} \m^i / \m^{i+1} \]
is finite and thus $\ell_A(A)$ is finite from the series showing that $A$ is artinian.
\end{proof}


\begin{prop}
Let $A$ be an artinian ring. Then,
\[ \ell_A(A) = \sum_{i = 1}^r \ell_{A_{\m_i}}(A_{\m_i}) \]
\end{prop}

\begin{proof}
We can write, $A = A_{\m_1} \times \cdots \times A_{\m_r}$ and thus the formula immediately follows.
\end{proof}


\begin{prop}
Any finite dimensional $k$-algebra is artinian.
\end{prop}

\begin{proof}
By dimensionality arguments every descending chain stabilizes. 
\end{proof}

\begin{prop}
Let $A \to B$ be a local map and $M$ an $B$-module of finite length. Then,
\[ \ell_A(M) = \ell_B(M) \cdot [ \kappa(\m_B) : \kappa(\m_A) ] \]
and in particular $\ell_A(M)$ is finite if $\kappa(\m_B)$ is a finite extension of $\kappa(\m_A)$.
\end{prop}

\begin{proof}
Consider a composition series,
\[ 0 = M_0 \subset M_{1} \subset \cdots \subset M_n = M \]
Then $M_i / M_{i - 1}$ is a simple $B$-module so $M_i / M_{i-1} \cong B / \m_B = \kappa(\m_B)$ since $B$ is local. Therefore,
\[ \ell_A(M) = \sum_{i = 1}^n \ell_A (M_i / M_{i-1}) = \sum_{i = 1}^n \ell_A(\kappa(\m_B)) = n \cdot [ \kappa(B_\m) : \kappa(A_\m) ] \]
where $\ell_A(\kappa(\m_B)) = \ell_{\kappa(\m_A)}(\kappa(\m_B))$ because $A \to B$ is local and,
\[ \ell_{\kappa(\m_A)}(\kappa(\m_B)) = \dim_{\kappa(\m_A)}(\kappa(\m_B)) = [\kappa(\m_B) : \kappa(\m_A)] \]
\end{proof}

\begin{cor}
If $A$ is a local artinian finite type $k$-algebra. Then,
\[ \dim_k{A} = \ell_A(A) \cdot \dim_k{(A / \m)} \]
in particular $A$ is a finite $k$-module. 
\end{cor}

\begin{proof}
Viewing $A$ as a module over itself we know it has finite length since $A$ is artinian. Furthermore, $A / \m$ is a field finitely generated over $k$ and thus a finite extension of $k$ by the Nullstellensatz. Then applying the previous result we conclude. 
\end{proof}

\begin{cor}
Let $A$ be an artinian finite type $k$-algebra. Then,
\[ \dim_k{A} = \sum_{i = 1}^r \ell_{A_{\m_i}}(A_{\m_i}) \cdot \dim_k{(A / \m_i)} \]
\end{cor}

\begin{proof}
Since $A$ is artinian we can write,
\[ A = \prod_{i = 1}^r A_{\m_i} \]
where $A_{\m_i}$ are the local artinian factors associated to the finitely many prime ideals $\m_1, \dots, \m_r$. The result follows from above by additivity of the dimensions.
\end{proof}

\begin{rmk}
We can generalize this to the following proposition.
\end{rmk}

\begin{prop}
Let $A$ be local with maximal ideal $\m$ and $B$ be semi-local with maximal ideals $\m_i$. Let $A \to B$ be a homomorphism of rings such that $\m_i$ lie over $\m$ and $[\kappa(\m_i) : \kappa(\m)]$ is finite. Let $M$ be a finite length $B$-module. Then,
\[ \ell_A(M) = \sum_{i = 1}^n \ell_{B_{\m_i}}(M_{\m_i}) \cdot [\kappa(\m_i) : \kappa(\m)] \]
\end{prop}

\section{Weakly Associated Points}

\newcommand{\WAss}[2]{\mathrm{WAss}_{#1}\left(#2 \right)}

\subsection{Weakly Associated Primes}

\begin{defn}
Let $A$ be a ring and $M$ an $A$-module. Then a prime $\p \subset A$ is \textit{weakly associated} to $M$ if $\p$ is minimal over $\Ann{A}{m}$ for some $m \in M$. We denote these primes $\WAss{A}{M}$.
\end{defn}

\begin{lemma}
Let $M$ be an $A$ module then the natural map,
\[ M \to \prod_{\p \in \WAss{A}{M}} M_\p \]
is injective.
\end{lemma}

\begin{proof}
Suppose that $m \in M$ maps to zero. Then $\p \not\subset \Ann{A}{m}$ for each $\p \in \WAss{A}{M}$ which implies $\Ann{A}{m} = A$ since otherwise some associated prime will be minimal over $\Ann{A}{m}$. Thus $m = 0$.
\end{proof}

\begin{lemma}
Let $M$ be an $A$-module. Then,
\[ M = (0) \iff \WAss{A}{M} = \empty \]
\end{lemma}

\begin{proof}
If $M = (0)$ then this is clear. Otherwise, by the previous lemma $M \embed (0)$ is injective so $M = (0)$. 
\end{proof}

\begin{lemma} \label{weak_ass_primes_localization}
Let $A$ be a ring and $M$ an $A$-module. Then,
\[ \p \in \WAss{A}{M} \iff \p A_\p \in \WAss{A_\p}{M_\p} \]
\end{lemma}

\begin{proof}
Consider the exact sequence for each $m \in M$,
\begin{center}
\begin{tikzcd}
0 \arrow[r] & \Ann{A}{m} \arrow[r] & A \arrow[r, "m"] & M 
\end{tikzcd}
\end{center}
Since localization is exact, we get an exact sequence,
\begin{center}
\begin{tikzcd}
0 \arrow[r] & (\Ann{A}{m})_\p \arrow[r] & A_\p \arrow[r, "m"] & M_\p
\end{tikzcd}
\end{center}
Therefore, $\Ann{A_\p}{m} = (\Ann{A}{m})_\p$. If $\p \supset \Ann{A}{m}$ is minimal then $\p A_\p \supset (\Ann{A}{m})_\p = \Ann{A_\p}{m}$ is minimal. Conversely, if $\p A_\p \supset \Ann{A_\p}{m/s}$ is minimal then,
\[ \Ann{A_\p}{m/s} = \Ann{A_\p}{m} = (\Ann{A}{m})_\p \]
which implies that $\p \supset \Ann{A}{m}$ is minimal because if $x \in \Ann{A}{m}$ and $x \notin \p$ then $(\Ann{A}{m})_\p = A_\p$ and any prime $\q$ such that $\p \subset \q \subset \Ann{A}{m}$ implies that $\q A_\p$ is intermediate.
\end{proof}

\begin{lemma} \label{minimal_supp_wass}
Let $A$ be a ring and $M$ an $A$-module. Then $\WAss{A}{M} \subset \Supp{A}{M}$ furthermore any minimal element of $\Supp{A}{M}$ is an element of $\WAss{A}{M}$.
\end{lemma}

\begin{proof}
Since $\p \supset \Ann{A}{m}$ we know $M_\p \neq 0$ since $m$ is nonzero in $M_\p$. Furthermore, suppose that $\p \in \Supp{A}{M}$ is minimal. Then $\Supp{A_\p}{M_\p} = \{ \p A_\p \}$ and $M_\p \neq 0$ so $\WAss{A_\p}{M_\p} = \{ A_\p \}$ and thus $\p \in \WAss{A}{M}$.
\end{proof}

\begin{prop}
Let $M$ be finite or $A$ finite-dimensional. Every element of $\Supp{A}{M}$ is contained in a minimal element. Likewise for $\WAss{A}{M}$ and the sets of minimal elements coincide.
\end{prop}

\begin{proof}
For Zorn's lemma, we need to show that every downward chain in $\Supp{A}{M}$ has a lower bound. If $\dim{A} < \infty$ then any downward chain of primes stabilizes. Alternatively, assume that $M$ is finite and consider a chain $\{ \p_i \}_{i \in I}$ then I claim that,
\[ \q = \bigcap_{i \in I} \p_i \in \Supp{A}{M} \]
First, $\q$ is prime because if $xy \in \q$ then $xy \in \p_i$ at each stage so either $x \in \p_i$ or $y \in \p_i$ but because $I$ is totally ordered either there is a maximal $i \in I$ at which $x$ appears in which case $y \in \q$ or $x$ lies is $\p_i$ for arbitrarily large $i$ meaning that $x \in \p_i$ for all $i$ so $x \in \q$. Now I claim that $M_\q \neq 0$. Let $m_1, \dots, m_r \in M$ generate. It suffices to show that $\q \supset \Ann{A}{m_j}$ for some $j$ or equivalently that $\p_i \supset \Ann{A}{m_j}$ for some fixed $j$ and all $i$. Indeed for each $i$ there is some $j$ so that $\p_i \supset \Ann{A}{m_j}$. Therefore, at least one $j$ must satisfy $\p_i \supset \Ann{A}{m_j}$ for unbounded $i$ and hence $\p_i \supset \Ann{A}{m_j}$ for all $i$. 
\bigskip\\
Now let $\p \in \WAss{A}{M}$ then $\p \in \Supp{A}{M}$ so choose $\q \subset \p$ minimal in $\Supp{A}{M}$ then we have shown that $\q \in \WAss{A}{M}$ and is minimal in $\WAss{A}{M}$ because $\WAss{A}{M} \subset \Supp{A}{M}$ and it is minimal in $\Supp{A}{M}$. We have shown that any minimal element of $\Supp{A}{M}$ is in $\WAss{A}{M}$ and hence is minimal in $\WAss{A}{M}$. This discussion shows the converse. 
\end{proof}

\begin{rmk}
The condition that $M$ is finite is necessary if $A$ is not finite dimensional (in which case downward chains of primes always stabilize). For example, let $A = k[x_0, x_1, \dots]$ and,
\[ M = \bigoplus_{i = 0}^\infty A / \p_i \text{ where } \p_i = (x_i, x_{i+1}, \dots) \]
Then,
\[ \Supp{A}{M} = \bigcup_{i = 0}^\infty V(\p_i) \]
Thus if $\q \in \Supp{A}{M}$ then $\q \supset \p_i$ for some $i$ but then $\q \supset \p_i \supsetneq \p_{i+1}$ so $\Supp{A}{M}$ has no minimal elements.  
\end{rmk}

\begin{rmk}
The set $\WAss{A}{M}$ need not be a downward set (even when every element is contained in a minimal element) even in the best situations of $A$ a finite-dimensional noetherian ring and $M$ a finite $A$-module. For example let $A = k[x,y,z] / (x^2, xy, xz)$ and $M = A$ then $\WAss{A}{M} = \{ (x), (x,y,z) \}$ so the intermediate prime $(x,y)$ is not associated. 
\end{rmk}

\begin{lemma} \label{ass_primes_localization}
Let $A$ be a ring and $M$ an $A$-module and $S \subset A$ a multiplicative subset. Then.
\begin{enumerate}
\item $\WAss{A}{S^{-1} M} = \WAss{S^{-1} A}{S^{-1} M}$ 
\item $\WAss{A}{M} \cap \Spec{S^{-1} A} = \WAss{A}{S^{-1} M}$.
\end{enumerate}
\end{lemma}

\begin{proof}
We have,
\[ \p \in \WAss{A}{S^{-1} M} \iff \p A_\p \in \WAss{A_\p}{S^{-1} M_\p}  \]
For $\p \in \Spec{S^{-1} A}$ (i.e. $S \subset A \setminus \p$) we have $S^{-1} M_\p = M_\p$ and $(S^{-1} A)_\p = A_\p$ so both equalities hold. Otherwise, $\p A_\p$ containes an element of $S$ so $\p A_\p$ has some nonzero divisor on $S^{-1} M_\p$ and thus $\p \notin \WAss{A}{S^{-1} M}$. 
\end{proof}


\begin{proposition}
Let $A$ be a ring $M$ an $A$-module then $\p \in \Supp{A}{M}$ if and only if there exists $\q \subset \p$ with $\q \in \WAss{A}{M}$. Therefore, \[ \bigcap_{\p \in \Supp{A}{M}} \p = \bigcap_{\p \in \WAss{A}{M}} \p \quad \text{ and } \quad \Supp{A}{M} = \bigcup_{\p \in \WAss{A}{M}} V(\p) \]
\end{proposition}

\begin{proof}
Take $\p \in \Supp{A}{M}$ so $M_\p \neq 0$ and then $\Ass{A}{M_\p} \neq \varnothing$. Using the previous lemma, there exists $\q \in \Ass{A}{M_\p} = \Ass{A}{M} \cap \{ \q \mid \q \subset \p \}$. Furthermore, the support is an upward set (if $\q \subset \p$ and $M_\q \neq 0$ then $M_\p \neq 0$ since $M_\p \to M_\q$ is localization). Thus, if we have $\q \subset \p$ with $\q \in \Ass{A}{M} \subset \Supp{A}{M}$ then $\p \in \Supp{A}{M}$.  
\end{proof}


\begin{lemma}
Let $M \embed N$ be an injection of $A$-modules. Then $\WAss{A}{M} \subset \WAss{A}{N}$.
\end{lemma}

\begin{proof}
This follows because the set of annihilators of elements of $M$ is a subset of the set of annihilators of elements of $N$.
\end{proof}

\begin{lemma} \label{exact_seq_weak_ass}
Consider an exact sequence of $A$-modules,
\begin{center}
\begin{tikzcd}
0 \arrow[r] & M_1 \arrow[r] & M_2 \arrow[r] & M_3 
\end{tikzcd}
\end{center}
then,
\[ \WAss{A}{M_2} \subset \WAss{A}{M_1} \cup \WAss{A}{M_3} \]
\end{lemma}

\begin{proof}
Let $\p \in \WAss{A}{M_2}$ and $\p \notin \WAss{A}{M_1}$. Using the previous lemma it suffices to consider the case that $A$ is local with maximal ideal $\p$ (since we may localize the exact sequnence at $\p$). 
Then $\p$ is minimal over $\Ann{A}{m}$ for some $m \in M_2$ not in the image of $M_1 \to M_2$ (else $\p \in \WAss{A}{M_1}$). Therefore $\bar{m} \in M_3$ is nonzero and $\Ann{A}{\bar{m}} \supset \Ann{A}{m}$ but $\Ann{A}{\bar{m}}$ is proper since $\bar{m}$ is nonzero and thus contained in $\p$. Since $\p$ is minimal over $\Ann{A}{m}$ it must also be minimal over $\Ann{A}{\bar{m}}$ and thus we conclude that $\p \in \WAss{A}{M_3}$.
\end{proof}

\begin{lemma}
Let $A$ be a ring and $M$ and $A$-module. Then,
\[ \bigcup_{\p \in \WAss{A}{M}} = \{ \text{zero divisors on } M \} \]
\end{lemma}

\begin{proof}
Let $m \in M$ have zero divisors then there is exists a minimal prime (by Zorn's Lemma) above $\Ann{A}{m}$ which must be associated. Conversely, if $f \in \p \in \WAss{A}{M}$ then $\p$ is minimal over $\Ann{A}{m}$ for some $m \in M$. Then $R = (A / \Ann{A}{m})_\p$ has a unique minimal prime $\p$ so $\p = \nilrad{R}$ and thus $g f^n \in \Ann{A}{m}$ for some least $n > 0$ and $g \notin \p$. Thus $g f^n m = 0$ so $f (g f^{n-1} m) = 0$ but $g f^{n-1} m \neq 0$ because $n$ is minimal so $f$ is a zero divisor.
\end{proof}

\begin{prop}
Let $A$ be reduced then $\WAss{A}{A}$ are exactly the minimal primes of $A$.
\end{prop}

\begin{proof}
The minimal primes are in $\WAss{A}{A}$ by Lemma \ref{minimal_supp_wass}. Because $\p \in \WAss{A}{A} \iff \p A_\p \in \WAss{A_\p}{A_\p}$ is suffices to consider the case of a reduced local ring $(R, \m)$ and $\m \in \WAss{R}{R}$. Then $\m$ is minimal over $\Ann{R}{x}$ for some $x \in \m$ so $\m = \sqrt{\Ann{R}{x}}$. Thus $x^n \in \Ann{R}{x}$ so $x^{n+1} = x^n \cdot x = 0$ so $x = 0$ because $R$ is reduced a contradiction unless $\m = 0$ so $R$ is a field so $\m$ is minimal showing that $\p A_\p \subset A_\p$ and thus $\p \subset A$ are minimal primes and that $A_\p$ is a field. 
\end{proof}

\begin{lemma}
Let $A$ be a ring and $\p \subset A$ a prime then $\WAss{A}{A / \p} = \{ \p \}$. 
\end{lemma}

\begin{proof}
For nonzero $a \in A / \p$ (i.e. $a \notin \p$) the set $\Ann{A}{a} = \{ r \in A \mid ra \in \p \} = \p$ since $\p$ is prime and therefore therefore $\p$ is the unique minimal prime over an annihilator.
\end{proof}

\begin{proposition} \label{noetherian_finite_wass}
Let $A$ be a ring and $M$ a Noetherian $A$-module. Then,
\begin{enumerate}
\item there exists a finite filtration,
\[ (0) = M_0 \subsetneq M_1 \subsetneq M_2 \subsetneq \cdots \subsetneq M_n = M \]
such that each $M_{i} / M_{i-1} \cong A / \p_{i}$ for some $\p_i \in \supp{A}{M}$
\item for any such filtration, $\WAss{A}{M} \subset \{ \p_1, \p_2, \dots, \p_n\}$
\item $\WAss{A}{M}$ is finite.
\end{enumerate}
\end{proposition}

\begin{proof}
Since $M \neq (0)$ there is some $\p \in \WAss{A}{M}$ so we have an injection $A / \p \to M$ let $M_1 \subset M$ be the image of this map so $M_1 / M_0 \cong A / \p_1$. Now take $M / M_1$ and $\p_2 \in \WAss{A}{M/M_1}$ then we have an injection $A / \p_2 \to M/M_1$ so take $\bar{M}_2$ to be the image inside $M/M_1$ and $M_2$ its preimage in $M$. Then $M_2 / M_1 \cong A / \p_2$ and continuing by induction we construct a sequence,
\[ 0 \subsetneq M_1 \subsetneq M_2 \subsetneq M_3 \subsetneq \cdots \]
with $M_{i}/M_{i-1} = A / \p_i$ and 
\[ \p_i \in \WAss{A}{M/M_{i-1}} \subset \Supp{A}{M/M_{i-1}} \subset \Supp{A}{M} \]
However, $M$ is Noetherian so this sequence must stabilize but it is striclty increasing when $M_i \subset M$ is proper. Thus, $M_n  = M$ for some $n$. 
\bigskip\\
For any such filtration we get exact sequences,
\begin{center}
\begin{tikzcd}
0 \arrow[r] & M_i \arrow[r] & M_{i+1} \arrow[r] & A / \p_{i+1} \arrow[r] & 0
\end{tikzcd}
\end{center}
Assume for induction that $\Ass{A}{M_i} \subset \{\p_1, \dots, \p_i \}$ then, by Lemma \ref{exact_seq_weak_ass},
\[ \WAss{A}{M_{i+1}} \subset \WAss{A}{M_i} \cup \WAss{A}{A / \p_{i+1}} \subset \{\p_1, \dots, \p_{i+1} \} \]
proving (b) by induction. (c) follows directly from (a) and (b). 
\end{proof}

\subsection{Associated Primes}

\begin{defn}
Let $A$ be a ring and $M$ an $A$-module. We say that $\p \subset A$ is an \textit{associated prime} of $M$ if $\p = \Ann{A}{m}$ for some $m \in M$. We write $\Ass{A}{M}$ for the set of associated primes of $M$.
\end{defn}

\begin{rmk}
Note $\p = \Ann{A}{m} \iff A / \p \embed M$ via $a \mapsto a \cdot m$.
\end{rmk}

\begin{rmk}
Clearly $\Ass{A}{M} \subset \WAss{A}{M}$. We will see equality holds when $A$ is Noetherian.
\end{rmk}


\begin{lemma}
Given an exact sequence of $A$-modules,
\begin{center}
\begin{tikzcd}
0 \arrow[r] & M_1 \arrow[r] & M_2 \arrow[r] & M_3
\end{tikzcd}
\end{center}
we have, 
\[ \Ass{A}{M_2} \subset \Ass{A}{M_1} \cup \Ass{A}{M_3} \]
\end{lemma}

\begin{proof}
If $\p \in \Ass{A}{M}$ then we have an embedding
\begin{center}
\begin{tikzcd}
A / \p \arrow[r, hook] & M_2
\end{tikzcd}
\end{center}
which is injective and $\iota(A / \p) \cap N_1 = (0)$
then we get an injective map $A / \p \to M_3$ so $\p \in \Ass{A}{M_3}$. If $\iota(A / \p) \cap M_1 \neq (0)$ then take nonzero $n \in \iota(A / \p) \cap M_1$. Then $\Ann{A}{n} = \Ann{A}{\iota(x)}$ for $x \in A / \p$ nonzero. However, if $a \cdot \iota(x) = 0$ then $\iota(a \cdot x) = 0$ but $\iota$ is injective so $a \cdot x = 0$ and thus $\Ann{A}{\iota(x)} = \Ann{A}{x} = \p$ because if $a \cdot x \in \p$ for $x \notin \p$ then $a \in \p$. 
\end{proof}

\begin{lemma}
Let $S_{M, \p} = \{ \Ann{A}{m} \subset \p \mid m \in M \setminus \{0\} \}$ then any maximal element in $S_{M, \p}$ is a prime ideal.
\end{lemma}

\begin{proof}
Let $\q \in S_{M, \p}$ be maximal with $\q = \Ann{A}{m}$ for $m \neq 0$. Suppose $ab \in \q$ and $a, b \notin \q$. Then $\q \subsetneq \Ann{A}{a m}$ since $b \in \Ann{A}{a m} \setminus \Ann{A}{m}$ so by maximality $\Ann{A}{am} \not\subset \p$. Choose $s \in \Ann{A}{am} \setminus \p$. Then $a \in \Ann{A}{sm}$ so $\Ann{A}{m} \subsetneq \Ann{A}{sm}$ and thus by maximality we can choose $t \in \Ann{A}{sm} \setminus \p$ so $st \in \Ann{A}{m} \subset \p$ but $s,t \notin \p$ contradicting the primality of $\p$. Thus $\q$ is prime. 
\end{proof}

\begin{prop}
Let $A$ be Noetherian and $M$ be an $A$-module. Then,
\[ \Ass{A}{M} = \WAss{A}{M}  \]
In particular, $\Ass{A}{M} \neq \empty$ and all other properties of $\WAss{A}{M}$ apply to $\Ass{A}{M}$.
\end{prop}

\begin{proof}
$\Ass{A}{M} \subset \WAss{A}{M}$ is obvious. If $\p \in \WAss{A}{M}$ then $\p \supset \Ann{A}{m}$ for some $m \in M$ and thus $m$ is nonzero in $M_\p$ so $\p \in \Supp{A}{M}$. Let $A$ be Noetherian then ascending chains in $S_{M, \p}$ stabilize and thus by Zorn's Lemma every annhilator $\Ann{A}{m} \subset \p$ is contained in some maximal $\Ann{A}{m'} \subset \p$. Thus, if $\p \in \WAss{A}{M}$ then $\p$ is a minimal prime over some $\Ann{A}{m}$ so $\p = \Ann{A}{m'}$ since $\Ann{A}{m'}$ is prime and $\Ann{A}{m} \subset \Ann{A}{m'} \subset \p$.
\end{proof}


\begin{lemma} \label{ass_primes_localization}
Let $A$ be a ring and $M$ an $A$-module and $S \subset A$ a multiplicative subset. Then.
\begin{enumerate}
\item $\Ass{A}{S^{-1} M} = \Ass{S^{-1} A}{S^{-1} M}$ 
\item $\Ass{A}{M} \cap \Spec{S^{-1} A} \subset \Ass{A}{S^{-1} M}$ with equality when $A$ is Noetherian.
\end{enumerate}
\end{lemma}

\begin{proof}
Tag 05BZ.
\end{proof}


\begin{proposition} \label{noetherian_finite_ass}
Let $A$ be a Noetherian ring and $M$ a finite $A$-module. Then,
\begin{enumerate}
\item there exists a finite filtration,
\[ (0) = M_0 \subsetneq M_1 \subsetneq M_2 \subsetneq \cdots \subsetneq M_n = M \]
such that each $M_{i} / M_{i-1} \cong A / \p_{i}$ for some $\p_i \in \supp{A}{M}$
\item for any such filtration, $\Ass{A}{M} \subset \{ \p_1, \p_2, \dots, \p_n\}$
\item $\Ass{A}{M}$ is finite.
\end{enumerate}
\end{proposition}

\begin{proof}
$M$ is a Noetherian module so this applies directly from Prop. \ref{noetherian_finite_wass}.
\end{proof}

\begin{prop}
Let $A$ be a Noetherian ring and $I \subset A$ an ideal and $M$ a finite $A$-module. Then the following are equivalent,
\begin{enumerate}
\item $I \subset \p$ for some $\p \in \Ass{A}{M}$
\item $I \subset \{ \text{zero divisors on } M \}$
\end{enumerate}
\end{prop}

\begin{proof}
If $I \subset \p$ for $\p \in \Ass{A}{M}$ then,
\[ I \subset \p  \subset \{ \text{zero divisors on } M \} \]
Conversely, if $I \subset \{\text{zero divisors on } M\}$ then,
\[ I \subset \{ \text{zero divisors on } M \} = \bigcup_{\p \in \Ass{A}{M}} \p \]
By Proposition \ref{noetherian_finite_ass}, the set $\Ass{A}{M}$ is finite so by prime avoidance $I \subset \p$ for some $\p \in \Ass{A}{M}$. 
\end{proof}

\begin{cor}
Let $\m \subset A$ be a maximal ideal with $A$ noetherian and $M$ a finite $A$-module. Then $\m \in \Ass{A}{M}$ if and only if $\m \subset \{ \text{zero divisors on } M\}$.
\end{cor}

\begin{cor}
Let $(A, \m)$ be a noetherian local ring then $\m \in \Ass{A}{A}$ iff $\m = \{ \text{zero divisors} \}$.
\end{cor}

\begin{proof}
Immediate from the above since zero divisors are not units and thus contained in $\m$.
\end{proof}


\begin{cor}
Let $A$ be noetherian and $M$ be a finite $A$-module then for all $\p \in \Spec{A}$,
\[ \p \in \Ass{A}{M} \iff \p A_\p = \{ \text{zero divisors on } M_\p \} \]
\end{cor}

\subsection{Primary Decomposition}

\begin{rmk}
In this section we let $A$ be a Noetherian ring.
\end{rmk}

\begin{definition}
An $A$-module $M$ is called coprimary if $\Ass{A}{M} = \{\p\}$ and if $N \subset M$ we say that $N$ is $\p$-primary if $M / N$ is coprimary with $\Ass{A}{M/N} = \{ \p \}$.  
\end{definition}

\begin{lemma}
$M$ is coprimary iff any zero divisor of $M$ is locally nilpotent i.e. if $a \cdot m = 0$ for some $m \in M \setminus \{0\}$ then $\forall m' \in M : a^n \cdot m' = 0$ for some $n$. 
\end{lemma}

\begin{proof}
Assume that $M$ is coprimary, $\Ass{A}{M} = \{ \p \}$. If $x \in M$ is nonzero then $Ax$ is a nonzero submodule of $M$ so $\Ass{A}{Ax} = \{ \p \}$ since it is nonempty. Therefore, $\p$ is a minimal element in $\Supp{A}{A x} = V(\Ann{A}{x})$
because $Ax \cong A / \Ann{A}{x}$. Thus, $\sqrt{\Ann{A}{x}} = \p$. If $a$ is a zero divisor of $M$ then $a \in \p$ so $a^n \in \Ann{A}{x}$ so $a$ is locally nilpotent. Converely, assume that the set of zero divisors equals the set of locally nilpotent elements. Take $\p$ to be the ideal of all locally nilpotents. Take $\q \in \Ass{A}{M}$ then $\q = \Ann{A}{x}$ for some $x$. If $a \in \p$ then $a^n \cdot x = 0$ for some $n$ implies that $a^n \in \q$ so $a \in \q$. so $\p \subset \q$. Furthermore,
\[ \bigcup_{\q \in \Ass{A}{M}} \q = \{ \text{zero divisors} \} = \p \]
so for any $\q \in \Ass{A}{M}$ we have $\q \subset \p$. Thus, $\p = \q$ so $\Ass{A}{M}$ constains a unique prime.
\end{proof}

\begin{corollary}
If $I \subset A$ is an ideal then $\Ass{A}{A / I} = \{ \p \}$ if and only if $I$ is a primary ideal and in that case $\sqrt{I} = \p$. 
\end{corollary}

\begin{proof}
Consider $I \subset A$ and $A / I$ is coprimary then take $x,y \in A$ such that $y \notin I$ and $\bar{x} \cdot \bar{y} = 0$ in $A / I$. Then $\bar{x}$ is a zero divisor of $A / I$ so it is locally nilpotent by the above. Thus, $\bar{x}^n \cdot 1 = 0$ for some $n$ so $x^n \in I$ so $x \in \sqrt{I}$ and thus $I$ is primary. Furthermore,
\[ \sqrt{I} = \bigcap_{\p \in V(I)} \p = \bigcap_{\p \in \Supp{A}{A / I}} \p = \bigcap_{\p \in \Ass{A}{A / I}} \p = \p \]
since $\Ass{A}{M}$ is the set of minimal primes of $\Supp{A}{M}$ and $\Ass{A}{A / I} = \p$.  
\end{proof}

\begin{definition}
Let $M$ be an $A$-module and $N \subset M$. We say $N$ has a primary decomposition if,
\[ N = Q_1 \cap Q_2 \cap \cdots \cap Q_n \]
where each $Q_i$ is primary. Moreover, we say that this decomposition is irredundant if 
\begin{enumerate}
\item if $i \neq j$ then $\Ass{A}{M / Q_i} \neq \Ass{A}{M / Q_j}$ 

\item we cannot remove any $Q_j$ from the intersection.
\end{enumerate}
\end{definition}

\begin{lemma}
Let $M$ be an $A$-module then,
\begin{enumerate}
\item If $Q_1, Q_2 \subset M$ are $\p$-primary then $Q_1 \cap Q_2$ is $\p$-primary.  

\item If $N = Q_1 \cap \cdots \cap Q_n$ is a irredundant primary decomposition and for each $i$, $Q_i$ is $\p_i$-primary then,
\[ \Ass{A}{M / N} = \{ \p_1, \dots, \p_n \} \] 
\end{enumerate}
\end{lemma}

\begin{proof}
Consider the injection,
\begin{center}
\begin{tikzcd}
0 \arrow[r] & M / Q_1 \cap Q_2 \arrow[r, hook] & M / Q_1 \oplus M / Q_2
\end{tikzcd}
\end{center}
which implies that,
\[ \Ass{A}{M / Q_1 \cap Q_2} \subset \Ass{A}{M / Q_1 \oplus M / Q_2} = \Ass{A}{M/Q_1} \cup \Ass{A}{M/Q_2} = \{ \p \} \]
proving the first.
For the second, consider the injection,
\begin{center}
\begin{tikzcd}
M / N \arrow[r, hook] & M / Q_1 \oplus \cdots \oplus M / Q_n 
\end{tikzcd}
\end{center}
which implies that,
\[ \Ass{A}{M / N} \subset \Ass{A}{M/Q_1} \cup \cdots \cup \Ass{A}{M/Q_n} \subset \{ \p_1, \dots, \p_n \} \]
We need to show that $\p_i \in \Ass{A}{M / N}$ for each $i$.
We have the exact sequence,
\begin{center}
\begin{tikzcd}
0 \arrow[r] & N \arrow[r] & Q_2 \cap \cdots \cap Q_n \arrow[r] & M / Q_1
\end{tikzcd}
\end{center}
and therefore,
\begin{center}
\begin{tikzcd}
(Q_2 \cap \cdots \cap Q_n) / N \arrow[r, hook] & M / Q_1
\end{tikzcd}
\end{center}
which implies that,
\[ \Ass{A}{(Q_2 \cap \cdots \cap Q_n) / N} \subset \Ass{A}{M / Q_1} = \{ \p_1 \} \]
so since it is nonempy we have,
\[ \{ \p_1 \} = \Ass{A}{(Q_2 \cap \cdots \cap Q_n) / N} \subset \Ass{A}{M / N} \]
where the inclusion holds via the exact sequence,
\begin{center}
\begin{tikzcd}
0 \arrow[r] & N \arrow[r] & Q_2 \cap \cdots \cap Q_n \arrow[r] & M / N
\end{tikzcd}
\end{center}
The same argument holds for each $i$. 
\end{proof}

\begin{theorem}
Let $M$ be Noetherian. For each $\p \in \Ass{A}{M}$, there exist $Q_{\p} \subset M$ which are $\p$-primary such that,
\[ \bigcap_{\p \subset \Ass{A}{M}} Q_{\p} = 0 \]
\end{theorem}

\begin{proof}
Fix $\p \in \Ass{A}{M}$ and consider the set $S_{\p} = \{ Q \subset M \mid \p \notin \Ass{A}{Q} \} \neq \varnothing$ since the zero module is contained in this set. Since $M$ is Noetherian ascending chains stabilize so by Zorn's lemma there exists a maximal element $Q_\p \in S_\p$. We know,
\[ \Ann{A}{M / Q_\p}  \neq \varnothing \]
since we have $M / Q_\p \neq (0)$. Otherwise, $M = Q_\p$ which implies $\p \in \Ass{A}{Q_\p}$ but $Q_\p \in S_\p$. Let $\p' \in \Ass{A}{M / Q_\p}$ and suppose that $\p' \neq \p$ then we have,
\begin{center}
\begin{tikzcd}
A / \p' \arrow[r, hook] & M / Q_\p
\end{tikzcd}
\end{center}    
The image of this embedding is a submodule, $Q_\p \subsetneq Q' \subset M$ such that $Q' / Q_\p \cong A / \p'$ implying that,
\[ \Ass{A}{Q' / Q_\p} = \{ \p' \} \]
Thus we have an exact sequence,
\begin{center}
\begin{tikzcd}
0 \arrow[r] & Q_\p \arrow[r] & Q' \arrow[r] & A / \p \arrow[r] & 0
\end{tikzcd}
\end{center}
which implies that $\Ass{A}{Q'} \subset \Ass{A}{Q_\p} \cup \Ass{A}{A / \p'} =  \Ass{A}{Q_\p} \cup \{ \p' \}$.
However, this contradicts the fact that $Q_\p$ is maximal in $S_\p$ since $Q' \in S_\p$ as long as $\p' \neq \p$. Therefore, $\p' = \p$ so $\Ass{A}{A / Q_\p} = \{ \p \}$. Now consider,
\[ \Ass{A}{\bigcap_{\p \in \Ass{A}{M}} Q_\p} \subset \bigcap_{\p \in \Ass{A}{M}} \Ass{A}{Q_\p} = \varnothing \]
because for any $\p$ we know $\p \notin \Ass{A}{Q_\p}$. Therefore,
\[ \bigcap_{\p \in \Ass{A}{M}} Q_\p = (0) \]
since it has no associated primes. 
\end{proof}

\begin{corollary}
If $M$ is a finite $A$-module then any submodule has a primary decomposition. 
\end{corollary}

\begin{proof}
Let $N \subset M$ be a submodule. 
Apply the theorem to $\bar{M} = M / N$ which has finite type so $\Ass{A}{M / N}$ is finite. Write, $\Ass{A}{M / N} = \{ \p_1, \dots, \p_r \}$. Therefore, there exist primary ideals $Q_i$ such that,
\[ Q_{\p_1} \cap \cdots \cap Q_{\p_r} = (0) \]
in $M / N$. Take $Q_i$ to be the preimage of $Q_{\p_i}$. Thus,
\[ Q_1 \cap \cdots \cap Q_r = N \]
and 
\[ M / Q_i \cong \bar{M} / Q_{\p_i} \implies \Ass{A}{M / Q_i} = \{ \p_i \} \]
\end{proof}


\subsection{Weakly Associated Points}

\begin{defn}
Let $X$ be a scheme and $\F$ a quasi-coherent $\struct{X}$-module. Then we define,
\begin{enumerate}
\item $x \in X$ is \textit{weakly associated} to $\F$ if $\m_x \subset \stalk{X}{x}$ is weakly associated to $\F_x$
\item $\WAss{\struct{X}}{\F}$ is the set of weakly associated points of $\F$
\item the (weakly) associated points of $X$ are $\WAss{\struct{X}}{\struct{X}}$.
\end{enumerate}
\end{defn}

\begin{prop}
Let $X = \Spec{A}$ and $\F = \wt{M}$ be a quasi-coherent $\struct{X}$-module then we have,
\[ \WAss{\struct{X}}{\F} = \WAss{A}{M} \]
\end{prop}

\begin{proof}
Immediate consequence of Lemma \ref{weak_ass_primes_localization}.
\end{proof}

\begin{prop}
Let $X$ be a scheme and $\F$ a quasi-coherent sheaf. Then,
\[ \F = 0 \iff \WAss{\struct{X}}{\F} = 0 \]
\end{prop}

\begin{proof}
Choose an affine open cover $U_i = \Spec{A_i}$ such that $\F|_{U_i} = \wt{M_i}$. Then $\WAss{A}{M_i} = \WAss{\struct{X}}{\F} \cap U_i = \empty$ so $M_i = 0$ and thus $\F = 0$. 
\end{proof}

\begin{prop}
Let $X$ be a scheme and $\F \to \G$ a morphism of quasi-coherent $\struct{X}$-modules. If $\F_x \to \G_x$ is injective for each $x \in \WAss{\struct{X}}{\F}$ then $\F \to \G$ is injective.
\end{prop}

\begin{proof}
Consider the sequence,
\begin{center}
\begin{tikzcd}
0 \arrow[r] & \K \arrow[r] & \F \arrow[r] & \G
\end{tikzcd}
\end{center}
Since $\F_x \to \G_x$ is an injection $\K_x = 0$ for each $x \in \WAss{\struct{X}}{\F}$. Furthermore, $\WAss{\struct{X}}{\K} \subset \WAss{\struct{X}}{\F}$ and thus $\WAss{\struct{X}}{\K} = \empty$ so $\K = 0$.
\end{proof}

\subsection{Associated Points: the Noetherian Case}

\begin{rmk}
By analogy, we might define an \textit{associated point} of $\F$ on $X$ to be a point $x \in X$ such that $\m_x \subset \stalk{X}{x}$ is an associated prime of $\F_x$. However, this definition is problematic because, in general, associated primes do not play nicely with localization. In particular $\p \in \Ass{A}{M} \implies \p A_\p \in \Ass{A_\p}{M_\p}$ but the converse may not hold. Therefore, we may have a scheme $X$ and a quasi-coherent sheaf $\F$ such that on an affine open $U = \Spec{A}$ with $\F |_U = \wt{M}$ we have $\p \in \Ass{A}{M}$ but $\p = x \in X$ is not as associated point of $\F$ on $X$. To recify this pathology, we only consider associated points on locally noetherian schemes in which case there is no difference between weakly associated points and associated points. 
\end{rmk}

\begin{defn}
Let $X$ be a locally noetherian scheme and $\F$ a quasi-coherent $\struct{X}$-module. We say $x \in X$ is an \textit{associated point} of $\F$ if $x$ is a \textit{weakly associated point}. Likewise we write, 
\[ \Ass{\struct{X}}{\F} = \WAss{\struct{X}}{\F} \]
\end{defn}

\begin{rmk}
Notice this definition is purely notational. In the locally noetherian case we simply will write $\Ass{\struct{X}}{\F}$ for $\WAss{\struct{X}}{\F}$ as a reminder that these sets behave as expected for associated points in the case of Noetherian rings.
\end{rmk}

\begin{prop}
Let $X$ be noetherian and $\F$ a coherent $\struct{X}$-module. Then $\Ass{\struct{X}}{\F}$ is finite.
\end{prop}

\begin{proof}
Since $X$ is quasi-compact we may choose a finite open cover $U_i = \Spec{A_i}$ with $A_i$ Noetherian on which $\F|_{U_i} = \wt{M_i}$ for finite $A_i$-modules. Then $\Ass{\struct{X}}{\F} \cap U = \Ass{A_i}{M_i}$ each of which is finite  since $M_i$ is a Noetherian module.
\end{proof}

\section{Depth}

\subsection{Definitions}

\begin{defn}
Let $A$ be a ring $I \subset A$ an ideal and $M$ a finite $A$-module. Then $x_1, \dots, x_r \in I$ are an $M$-\textit{regular sequence in} $I$ if
\begin{enumerate}
\item $x_i$ is a nonzerodivisor on $M / (x_1, \dots, x_{i-1}) M$ for each $i \in \{1, \dots, r \}$
\item $M / (x_1, \dots, x_r)M$ is nonzero. 
\end{enumerate}
We say that $\depth{I}{M}$ is the supremum of the lengths of $M$-regular sequence in $I$ unless $IM = M$ in which case $\depth{I}{M} = \infty$.
\end{defn}

\begin{rmk}
If $IM \subsetneq M$ then $\depth{I}{M} = 0$ iff $I \subset \{ \text{zero divisors on } M \}$.
\end{rmk}

\begin{rmk}
If $(A, \m)$ is a local ring then we define $\depth{}{M} := \depth{\m}{M}$.
\end{rmk}

\subsection{The Cohomological Criterion}

\begin{lemma}
Let $A$ be a Noetherian ring, $I \subset R$ an ideal, and $M$ a finite $A$-module with $IM \neq M$. Then the following are equivalent,
\begin{enumerate}
\item $\Ext{i}{A}{N}{M} = 0$ for all $i < n$ and all finite $A$-modules $N$ with $\Supp{A}{N} \subset V(I)$

\item $\Ext{i}{A}{A/I}{M} = 0$ for all $i < n$

\item there exists a finite $A$-module $N$ with $\Supp{A}{N} = V(I)$ and $\Ext{i}{A}{N}{M} = 0$ for all $i < n$

\item there exists an $M$-regular sequence $x_1, \dots, x_n \in I$ of length $n$
\end{enumerate}
and therefore $\depth{I}{M} = \inf \{ n \in \Z \mid \Ext{i}{A}{A/I}{M} \neq 0 \}$.
\end{lemma}

\begin{proof}
Clearly (a) $\implies$ (b) $\implies$ (c). Now we show that (c) $\implies$ (d).

Finally, we need to show that (d) $\implies$ (a).
(DOOOOOOOOOOOOOOOOOOOOOOO!! OR SPLIT UP THIS PROOF!!)
\end{proof}

\begin{rmk}
From here on, let $A$ be a Noetherian ring and $I \subset A$ an ideal and $M$ a finite $A$-module with $IM \neq M$.
\end{rmk}

\begin{lemma}
Consider an exact sequence of finite $A$-modules such that $I M_i \neq M_i$, 
\begin{center}
\begin{tikzcd}
0 \arrow[r] & M_1 \arrow[r] & M_2 \arrow[r] & M_3 \arrow[r] & 0
\end{tikzcd}
\end{center}
Then the following hold,
\begin{enumerate}
\item $\depth{I}{M_2} \ge \min \{ \depth{I}{M_1}, \depth{I}{M_3} \}$

\item $\depth{I}{M_1} \ge \min \{ \depth{I}{M_2}, \depth{I}{M_3} + 1 \}$

\item $\depth{I}{M_3} \ge \min \{ \depth{I}{M_1} - 1, \depth{I}{M_2} \}$
\end{enumerate}
\end{lemma}

\begin{proof}
Apply the functor $\Hom{A}{A/I}{-}$ to give the long exact sequence,
\begin{center}
\begin{tikzcd}
\Ext{i}{A}{A/I}{M_1} \arrow[r] & \Ext{i}{A}{A/I}{M_2} \arrow[r] & \Ext{i}{A}{A/I}{M_3} \arrow[r] & \Ext{i+1}{A}{A/I}{M_1}   
\end{tikzcd}
\end{center}
If $i < n = \min \{ \depth{I}{M_1}, \depth{I}{M_3} \}$ then $\Ext{i}{A}{A/I}{M_2} = 0$ applying the cohomological criterion and the exact sequence so $\depth{I}{M_3} \ge n$. The other parts follow similarly.
\end{proof}

\begin{lemma}
Let $x$ be a nonzerodivisor on $M$ then $\depth{I}{M/xM} = \depth{I}{M} - 1$.
\end{lemma}

\begin{proof}
Applying the previous Lemma to the exact sequence,
\begin{center}
\begin{tikzcd}
0 \arrow[r] & M \arrow[r, "\times x"] & M \arrow[r] & M / x M \arrow[r] & 0 
\end{tikzcd}
\end{center}
gives $\depth{I}{M/xM} \ge \depth{I}{M} - 1$. However, for any $M/xM$-regular sequence $x_1, \dots, x_n \in I$ we get a $M$-regular sequence $x, x_1, \dots, x_n \in I$ and thus $\depth{I}{M} \ge \depth{I}{M/xM} + 1$.
\end{proof}

\begin{cor}
Any $M$-regular sequence $x_1, \dots, x_r \in I$ can be extended to a regular sequence of length $\depth{I}{M}$ and thus all maximal regular sequences have the same length.
\end{cor}

\begin{proof}
Given an $M$-regular sequence $x_1, \dots, x_r \in I$ we apply the previous Lemma to show that,
\[ \depth{I}{M/(x_1, \dots, x_r)M} = \depth{I}{M} - r \]
and thus there exists a regular sequence $x_{r+1}, \dots, x_d \in I$ for $M/(x_1, \dots, x_r)M$ meaning that $x_1, \dots, x_r, \cdots, x_d \in $ gives a $M$-regular sequence of length $\depth{I}{M}$ extending $x_1, \dots, x_r$.
\end{proof}

\subsection{Vanishing Criteria on Ext}

(GRADE AND (Ischebeck))

\subsection{Locality of Depth}

\begin{prop}
Let $A$ be a noetherian ring, $I \subset A$ an ideal, and $M$ a finite $A$-module. Then,
\[ \depth{I}{M} = \inf \{ \depth{}{M_\p} \mid \p \in V(I) \} \]
\end{prop}

\begin{proof}
DOOOOOOOOOOO!!!!
\end{proof}

\subsection{Additional Lemmas}


\begin{prop}
Let $A$ be Noetherian ring, $I \subset A$ an ideal, and $M$ a finite $A$-module. Then there exists an exact sequence of finite $A$-modules,
\begin{center}
\begin{tikzcd}
0 \arrow[r] & K \arrow[r] & F_{r-1} \arrow[r] & \cdots \arrow[r] & F_1 \arrow[r] & F_0 \arrow[r] & M \arrow[r] & 0
\end{tikzcd}
\end{center}
where $F_i$ are finite free $A$-modules and $r = \depth{}{A} - \depth{}{M}$. Furthermore, given any such sequence, $\depth{}{K} = \depth{}{A}$. 
\end{prop}

\begin{proof}
There always exists a surjection $F_0 \onto M$ from a finite free module $F_0$ because $M$ is finite. Extending to an exact sequence,
\begin{center}
\begin{tikzcd}
0 \arrow[r] & K_1 \arrow[r] & F_0 \arrow[r] & M \arrow[r] & 0
\end{tikzcd}
\end{center}
gives $\depth{I}{K} \ge \min\{ \depth{I}{A}, \depth{I}{M}  + 1\}$ because $F_0$ is free so clearly $\depth{I}{F_0} = \depth{I}{A}$ by the cohomological criterion. Thus either $\depth{I}{K} \ge \depth{I}{A}$ already or  $\depth{I}{K} \ge \depth{I}{M} + 1$. Therefore, repeating this process $r$ times we see that $\depth{I}{K_r} \ge \depth{I}{M}$
\end{proof}

\subsection{Cohen-Macaulay Rings}


(IS THIS CORRECT AS STATED!!)
\begin{prop}
Let $A$ be a ring, $I \subset A$ an ideal, and $M$ a finite $A$-module. Then,
\[ \depth{I}{M} \le \min_{\p \in \WAss{A}{M}} \dim{A/\p} \le \dim{\Supp{A}{M}} \] 
\end{prop}

\begin{defn}
Let $A$ be a Noetherian local ring. A finite $A$-module $M$ is \textit{Cohen-Macaulay} if,
\[ \depth{}{M} = \dim{\Supp{A}{M}} \]
We say that $A$ is Cohen-Macaulay if it is Cohen-Macaulay as an $A$-module i.e. if $\depth{}{A} = \dim{A}$.
\end{defn}

\begin{lemma}
If $A$ is a Cohen-Macaualy Noetherian local ring then for any prime $\p \in \Spec{A}$ the local ring $A_\p$ is Cohen-Macaulay.
\end{lemma}

\begin{proof}
Tag 0AAG
\end{proof}

\begin{rmk}
This Lemma allows for the following definition.
\end{rmk}

\begin{defn}
A ring $A$ is Cohen-Macaulay if $A$ is Noetherian and $A_\p$ is Cohen-Macaualy for each $\p \in \Spec{A}$.
\end{defn}

(UNIVERSALLY CATENARY ETC..)


(FIX THIS STATEMENT!!)

\begin{prop}
Let $R$ be a regular local ring and $M$ a finite $A$-module. Then any exact sequence of finite $A$-modules
\end{prop}

\subsection{Dimension}

\begin{prop}
Let $(A, \m)$ be a Noetherian local ring and $f \in \m$. Then,
\[ \dim{A/(f)} \ge \dim{A} - 1 \]
with equality iff $f$ is a nonzero divisor.
\end{prop}

\begin{proof}
https://math.stackexchange.com/questions/2085779/the-dimension-modulo-a-principal-ideal-in-a-noetherian-local-ring
\end{proof}


\subsection{Properties}

\begin{prop}
Let $(A, \m)$ be a Noetherian local ring and $f \in \m$ a nonzero divisor. Then $A$ is Cohen-Macaulay iff $A / (f)$ is Cohen-Macaulay.
\end{prop}

\begin{proof}
We have $\depth{}{A/(f)} = \depth{}{A} - 1$ and $\dim{A/(f)} = \dim{A} - 1$.
\end{proof}

\section{Finite Projective Modules over Local Rings}

\begin{remark}
It is well know that if $\phi : M \to M$ is an endomorphism of Noetherian $R$-modules which is surjective then it is injective. However, we can remove the Noetherian hypothesis and only require $M$ to be finitely generated (which does not imply Noetherian unless $R$ is Noetherian). 
\end{remark}

\begin{remark}
The following proposition crucially only holds for \textit{commutative} rings.  
\end{remark}

\begin{theorem}
Let $M$ be a finite $R$-module and $\phi : M \to M$ a surjective endomorphism then $\phi$ is injective.
\end{theorem}

\begin{proof}
We consider $M$ as a $R[X]$-module with $X \cdot m  = f(m)$. Let $I = (X) \subset R[X]$ then $I \cdot M = M$ since $f$ is surjective. Thus, by Nakayama, $\exists P(X) \in I$ such that $(1 - P(X)) \cdot M = 0$. Thus, for all $m \in M$ we have $P(X) \cdot m = m$ i.e. $m = P(f)(m)$ so if $f(m) = 0$ then $m = 0$ since $P(X) \in I$ and thus has no constant terms.
\end{proof}

\begin{lemma}
Let $(R, \m, \kappa)$ be a local ring and $M$ a finite $R$-module with $M \otimes_R \kappa = 0$. Then $M = 0$.
\end{lemma}

\begin{proof}
If $M \otimes_R \kappa = M / \m M = 0$ then $\m M = M$. However, since $R$ is local $\m = \Jac{R}$ and $M$ is finite so by Nakayama, $M = 0$.
\end{proof}

\begin{lemma}
Let $(R, \m, \kappa)$ be a local ring and $\phi : M \to N$ a map of $R$ modules with $N$ finite such that $\phi \otimes \id_\kappa : M \otimes_R \kappa \to N \otimes_R \kappa$ is surjective. Then $\phi$ is surjective.
\end{lemma}

\begin{proof}
Consider the exact sequence,
\begin{center}
\begin{tikzcd}
M \arrow[r, "\phi"] & N \arrow[r] & \coker{\phi} \arrow[r] & 0
\end{tikzcd}
\end{center}
Since $- \otimes_R \kappa$ is right-exact, we get an exact sequence,
\begin{center}
\begin{tikzcd}
M \otimes_R \kappa \arrow[r, "\phi \otimes \id_\kappa"] & N \otimes_R \kappa \arrow[r] & \coker{\phi} \otimes_R \kappa \arrow[r] & 0
\end{tikzcd}
\end{center}
However, $\phi \otimes \id_\kappa$ is surjective so by exactness $\coker{\phi} \otimes_R \kappa = 0$. However, since $N$ is finite so is $\coker{\phi}$ and thus $\coker{\phi} = 0$ by the lemma showing that $\phi$ is surjective.
\end{proof}

\begin{lemma}
Let $(R, \m, \kappa)$ be a local ring. Suppose that $M$ is a finite $R$-module with an endomorphism $\phi : M \to M$ such that $\phi \otimes \id : M \otimes_R \kappa \to M \otimes_R \kappa$ is an isomorphism then $\phi$ is an isomorphism. 
\end{lemma}

\begin{proof}
Consider the exact sequence,
\begin{center}
\begin{tikzcd}
M \arrow[r, "\phi"] & M \arrow[r] & \coker{\phi} \arrow[r] & 0
\end{tikzcd}
\end{center}
and apply the right-exact functor $(-) \otimes_R \kappa$ to get,
\begin{center}
\begin{tikzcd}
M \otimes_R \kappa \arrow[r, "\phi \otimes \id"] & M \otimes_R \kappa \arrow[r] & (\coker{\phi}) \otimes_R \kappa \arrow[r] & 0
\end{tikzcd}
\end{center}
But $\phi \otimes \id$ is an isomorphism and the sequence is exact so $(\coker{\phi}) \otimes_R \kappa = 0$ and thus, by the previous lemma, $\coker{\phi} = 0$ so $\phi$ is surjective. Now we apply the previous theorem to get that $\phi$ is an isomorphism.
\end{proof}

\begin{lemma}
Let $M$ be a finite module over $R$ a local ring then bases of $M \otimes_R \kappa$ lift to generating sets $R^n \onto M$ giving,
\[ \mathrm{rank}(M) = \dim_{\kappa}{(M \otimes_R \kappa)} \]
\end{lemma}

\begin{proof}
If $M$ is generated by $m_1, \dots, m_n$ then $M \otimes_R \kappa = M / \m M$ is generated by $\bar{m}_1, \dots, \bar{m}_n$ over $\kappa = R / \m R$ since surjectivity of $R^n \to M$ is preserved after applying $(-) \otimes_R \kappa$. Thus,
\[ \mathrm{rank}(M) = \dim_{\kappa} M \otimes_{R} \kappa \le n \]
Now suppose that $v_1, \dots, v_n$ is a $\kappa$-basis of $M \otimes_{R} \kappa = M / \m M$ then choose lifts $m_1, \dots, m_n \in M$. I claim that $m_1, \dots, m_n$ generate $M$ as an $R$-module. Let $N \subset M$ be the $R$-submodule generated by the $m_1, \dots, m_n$ and let $K = M / N$. Then I claim that $\m K = K$. To see this it suffices to show that $K \subset \m K$. For any $m \in M$ we know that its image $\bar{m} \in M / \m M$ is in the span of the basis $v_1, \dots, v_n$ so,
\[ \bar{m} = r_1 v_1 + \cdots r_n v_n \]
for $r_i \in R$. Thus,
\[ m - (r_1 m_1 + \cdots r_n m_n) \in \m M \]
This implies that in $K$ we have $m \in \m K$ so $K = \m K$. Then since $\Jac{R} = \m$ (because $R$ is local) by Nakayama $K = 0$ so $M$ is generated by $m_1, \dots, m_n$. 
\end{proof}

\begin{theorem}
Every finite projective module over a local ring is free.
\end{theorem}

\begin{proof}
Let $P$ be a finite projective $R$-module where $(R, \m, \kappa)$ is a local ring. Then there is a surjection $R^n \to P$ which we may assume gives a basis $\kappa^n \xrightarrow{\sim} P \otimes_R \kappa$. We extend to a short exact sequence,
\begin{center}
\begin{tikzcd}
0 \arrow[r] & K \arrow[r] & R^n \arrow[r] & P \arrow[r] & 0
\end{tikzcd}
\end{center}
but $P$ is projective so the sequence splits giving $R^n \cong K \oplus P$ and a surjection $R^n \to K$ making $K$ finitely generated. Since split exact sequences are preserved under additive functors,
\begin{center}
\begin{tikzcd}
0 \arrow[r] & K \otimes_R \kappa \arrow[r] & \kappa^n \arrow[r] & P \otimes_R \kappa \arrow[r] & 0
\end{tikzcd}
\end{center}
but the second map is an isomorphism so $K \otimes_R \kappa = 0$ and $K$ is finite so by the lemma $K = 0$. Thus $R^n \xrightarrow{\sim} P$ is an isomorphism so $P$ is free.
\end{proof}

\begin{lemma}
Let $P$ be a projective $R$-module and $S \subset R$ a multiplicative subset. Then $S^{-1} P$ is a projective $S^{-1} R$-module.
\end{lemma}

\begin{proof}
Let $M, N$ be $S^{-1} R$-modules and consider a diagram in the category of $R$-modules,
\begin{center}
\begin{tikzcd}
& & M \arrow[d, two heads]
\\
P \arrow[r] \arrow[rru, dashed, "\phi"] & S^{-1} P \arrow[ru, dashed, "\tilde{\phi}"'] \arrow[r] & N
\end{tikzcd}
\end{center}
then $P \to N$ lifts to $\phi : P \to M$ since $P$ is projective. Now we define $\tilde{\phi} : S^{-1} P \to M$ via $\tilde{\phi}(x \otimes r/s) = (r/s) \cdot \phi(x)$ using the decomposition $S^{-1} P = P \otimes_R S^{-1} R$. This makes the diagram commute. 
\end{proof}

\begin{remark}
We can also use the fact that (See Tag 05G3),
\[ \Hom{S^{-1} R}{S^{-1} P}{-} = \Hom{S^{-1} R}{P \otimes_R S^{-1} R}{-} = \Hom{R}{P}{\mathrm{Res}^{S^{-1} R}_{R} (-)} \]
and that projectiveity of $P$ is equivalent to $\Hom{R}{P}{\mathrm{Res}^{S^{-1} R}_{R} (-)}$ being exact showing that $S^{-1} P$ is $S^{-1} R$-projective.
\end{remark}

\begin{lemma}
Let $M$ be a finitely-presented $R$-module such that $M_\p$ is a free $R_\p$-module at each prime $\p \in \Spec{R}$. Then $M$ is a localy free $R$-module.
\end{lemma}

\begin{proof}
Take a prime $\p \in \Spec{R}$ then $M_\p$ is a finite free $R_\p$-module say $M_\p \cong R_\p^n$. Lift the basis to give a map $R^n \to M$ and an exact sequence,
\begin{center}
\begin{tikzcd}
0 \arrow[r] & C \arrow[r] & R^n \arrow[r] & M \arrow[r] & K \arrow[r] & 0
\end{tikzcd}
\end{center}
Since $M$ is finitely-presented, both $K$ and $C$ are finitely generated. Futhermore, localizing at $\p$ gives,
\begin{center}
\begin{tikzcd}
0 \arrow[r] & C_\p \arrow[r] & R^n_\p \arrow[r] & M_\p \arrow[r] & K_\p \arrow[r] & 0
\end{tikzcd}
\end{center}
but $R^n_\p \to M_\p$ is an isomorphism so $C_\p = 0$ and $K_\p = 0$. Since they are finitely generated, there is an element $f \notin \p$ killing both generating sets and thus $C_f = 0$ and $K_f = 0$. Therefore, 
\begin{center}
\begin{tikzcd}
0 \arrow[r] & C_f \arrow[r] & R^n_f \arrow[r] & M_f \arrow[r] & K_f \arrow[r] & 0
\end{tikzcd}
\end{center}
is exact so $R^n_f \xrightarrow{\sim} M_f$ is an isomorphism so $M$ is free on $D(f) \subset \Spec{R}$ for $\p \in D(f)$ so $M$ is locally free.
\end{proof}

\begin{theorem}
Let $R$ be a ring. Then finite projective $R$-modules are exactly the finite locally free $R$-modules. 
\end{theorem}

\begin{proof}
If $P$ is finite projective then $P_\p$ is finite projective over $R_\p$ and thus free. Furthermore, $P$ is finitely presented because there is an exact sequence,
\begin{center}
\begin{tikzcd}
0 \arrow[r] & K \arrow[r] & R^n \arrow[r] & P \arrow[r] & 0
\end{tikzcd}
\end{center}
which splits $R^n \cong K \oplus P$ since $P$ is projective giving a surjection $R^n \to K$ thus showing that $K$ is finite and giving a finite presentation,
\begin{center}
\begin{tikzcd}
R^n \arrow[r] & R^n \arrow[r] & P \arrow[r] & 0
\end{tikzcd}
\end{center}
Therefore, by the previous lemma, $P$ is locally free. 
\bigskip\\
Conversely, if $P$ is locally free so there exists a finite ($\Spec{R}$ is quasi-compact) open cover $D(f_i)$ of $\Spec{R}$ such that $P_{f_i} \cong R_{f_i}^n$. Then we need to show that $\Hom{R}{P}{-}$ is exact. We use that $\Hom{R}{P}{-}_{f_i} = \Hom{R_{f_i}}{P_{f_i}}{(-)_{f_i}}$ which is exact since $P_{f_i}$ is free and localization $(-)_{f_i}$ is an exact functor. Then $\Hom{R}{P}{-}$ is exact since we can check exactness of the hom sequence locally. 
\end{proof}

\begin{remark}
Look at Tag 00NV for more detailed version.
\end{remark}

\section{Integral and Finite Extensions}

\begin{defn}
Let $\varphi : A \to B$ be a map of rings. We say that an element $x \in B$ is \textit{integral} over $A$ if it satisfies a monic polynomial,
\[ x^n + \varphi(a_{n-1}) x^{n-1} + \cdots + \varphi(a_0) = 0 \]
for $a_i \in A$. We say that $\varphi$ is \textit{integral} if every element $x \in B$ is integral over $A$.
\end{defn}
(DO THIS \chref{https://stacks.math.columbia.edu/tag/00GH}{STUFF}).

\section{Normal Domains}


\begin{defn}
Let $R$ be a domain. We say that $R$ is \textit{normal} if $R$ is integrally closed in $\Frac{R}$.
\end{defn}

\begin{lemma}
Let $R$ be a domain. The following are equivalent,
\begin{enumerate}
\item $R$ is a normal domain
\item for each multiplicative subset $S \subset R$, the localization $S^{-1} R$ is a normal domain
\item for each prime $\p \subset R$ the localization $R_\p$ is a normal domain
\item for each maximal ideal $\p \subset R$ the localization $R_\m$ is a normal domain.
\end{enumerate}
\end{lemma}

\begin{proof}
Let $R$ be a normal domain and $x \in K = \Frac{R}$ satisfying the monic polynomial,
\[ x^n + \frac{r_{n-1}}{s_{n-1}} x^{n-1} + \cdots + \frac{r_0}{s_0} \]
for $\frac{r_{i}}{s_i} \in S^{-1} R$. Then let $s = s_{n-1} \cdots s_0$ and,
\[ (sx)^n + s_0 \cdots s_{n-2} r_{n-1} (sx)^{n-1} + \cdots + s^{n-1} s_1 \cdots s_{n-1} r_{0} = 0 \]
and therefore $sx \in K$ is integral over $R$ so $sx \in R$ and thus $x \in S^{-1} R$ showing that $S^{-1} R$ is integrally closed.
\bigskip\\
Clearly, (b) $\implies$ (c) $\implies$ (d). Finally, suppose that each $R_{\m}$ is integrally closed. Then,
\[ R = \bigcap R_{\m} \]
inside $K$. Suppose that $x \in K$ is integral over $R$ then $x$ is integral over each $R_{\m}$ and thus $x \in R_{\m}$ for each $\m$ by integral closure so $x \in R$ proving that $R$ is an integrally closed domain. 
\end{proof}

\subsection{Normalization}

\begin{lemma}
Let $\varphi : A \to B$ be a ring map. Then,
\[ B' = \{ b \in B \mid b \text{ is integral over } A \} \]
is an integrally closed $A$-subalgebra of $B$ called the integral closure of $A$ in $B$.
\end{lemma}

\begin{proof}
(DO THIS!!!)
\end{proof}

\begin{prop}
Let $A$ be a noetherian normal domain with $K = \Frac{A}$ and $L/K$ a finite seperable extension. Let $A'$ be the normalization of $A$ in $L$. Then $A \subset A'$ is a finite extension of rank $n = [L : K]$.
\end{prop}

\newcommand{\inner}[2]{\left< #1, #2 \right>}

\begin{proof}
Consider the trace pairing,
\[ L \times L \to K \quad (x,y) \mapsto \inner{x}{y} := \mathrm{Tr}_{L/K}(xy) \]
Since $L/K$ is separable this is nondegenerate (see algebra review). Furthermore, if $x \in L$ is integral over $A$ then $\mathrm{Tr}_{L/K}(x) \in K$ is integral over $A$ so because $A$ is normal $\mathrm{Tr}_{L/K}(x) \in A$. Therefore, choosing an integral $K$-basis $x_1, \dots, x_n \in L$ (which we can always do by clearing denominators since $L/K$ is algebraic) then $A' \subset L$ is contained in,
\[ M = \{ \alpha \in L \mid \inner{\alpha}{x_i} \in A \text{ for all } i \} \]
which is an $A$-module because $\inner{-}{x_i}$ is linear. However, $M \cong A^{\oplus n}$ via choosing the dual basis of $x_1, \dots, x_n$. Thus $A' \subset A^{\oplus n}$ so $A'$ is a finite $A$-module since $A$ is noetherian. Furthermore,
\[ N = A x_1 \oplus \cdots \oplus A x_n \subset A' \]
by definition because each $x_i \in L$ is integral. Therefore, $A^{\oplus n} \subset A' \subset A^{\oplus n}$ so by tensoring with $K$ we see that $\rank{(A')} = n$.
\end{proof}

\section{Projective and Global Dimension}

\subsection{Projective Dimension}

\newcommand{\pd}[2]{\mathrm{pd}_{#1} \left( #2 \right)}

\begin{defn}
Let $M$ be an $A$-module. Then the projective dimension $\pd{A}{M}$ is the minimal length $r$ of a projective resolution of $M$,
\begin{center}
\begin{tikzcd}
0 \arrow[r] & P_r \arrow[r] & \cdots \arrow[r] & P_1 \arrow[r] & P_0 \arrow[r] & M \arrow[r] & 0
\end{tikzcd}
\end{center}
and $\pd{A}{M} = \infty$ if there does not exist a finite-length projective resolution of $M$.
\end{defn}

\begin{lemma}[Schanuel's lemma]
Let $A$ be a ring and $M$ an $A$-module. Let,
\begin{center}
\begin{tikzcd}
0 \arrow[r] & K \arrow[r, "c_1"] & P_1 \arrow[r, "p_1"] & M \arrow[r] & 0 & & 0 \arrow[r] & L \arrow[r, "c_2"] & P_2 \arrow[r, "p_2"] & M \arrow[r] & 0
\end{tikzcd}
\end{center}
be two short exact sequences of $A$-module where $P_i$ are projective. Then there exists an isomorphism of short exact sequences,
\begin{center}
\begin{tikzcd}
0 \arrow[r] & K \oplus P_2 \arrow[d] \arrow[r, "(c_1 \, \id)"] & P_1 \oplus P_2 \arrow[d] \arrow[r, "(p_1 \, 0)"] & M \arrow[r] \arrow[d, equals] & 0
\\
0 \arrow[r] & P_1 \oplus L \arrow[r, "(\id \, c_2)"'] & P_1 \oplus P_2 \arrow[r, "(p_2 \, 0)"'] & M \arrow[r] & 0
\end{tikzcd}
\end{center}
\end{lemma}

\begin{proof}
Using projectivity of $P_1$ and $P_2$ we get maps $a : P_1 \to P_2$ and $P_2 \to P_1$ over $M$ meaning that $p_2 \circ a = p_1$ and $p_1 \circ b = p_2$. Therefore, we get a diagram,
\begin{center}
\begin{tikzcd}
0 \arrow[r] & K \oplus P_2 \arrow[r, "(c_1 \, \id)"] & P_1 \oplus P_2 \arrow[r, "(p_1 \, 0)"] & M \arrow[r] \arrow[d, equals] & 0
\\
0 \arrow[r] & N \arrow[d, dashed] \arrow[u, dashed] \arrow[r] & P_1 \oplus P_2 \arrow[d, "s"] \arrow[u, "t"'] \arrow[r, "(p_1 \, p_2)"] & M \arrow[r] \arrow[d, equals] & 0
\\
0 \arrow[r] & P_1 \oplus L \arrow[r, "(\id \, c_2)"'] & P_1 \oplus P_2 \arrow[r, "(p_2 \, 0)"'] & M \arrow[r] & 0
\end{tikzcd}
\end{center}
where $t(x,y) = (x + b(y), y)$ and $s(x,y) = (x, y + a(x))$ such that,
\[ (p_1, 0) \circ t = p_1 \circ (\id + b) = p_1 + p_2 \quad \text{and} \quad (0, p_2) \circ s = p_2 \circ (\id + a) = p_1 + p_2 \]
so the diagram commutes inducing maps $N \to K \oplus P_2$ and $N \to P_1 \oplus L$ where $N = \ker{(P_1 \oplus P_2 \to M)}$. It is clear that $t$ and $s$ are isomorphisms and thus the induced maps are also isomorphisms proving the claim.
\end{proof}

\begin{lemma}
Let $A$ be a ring and $M$ an $A$-module with finite projective dimension. Then for any projective resolution,
\begin{center}
\begin{tikzcd}
\cdots \arrow[r] & P_2 \arrow[r] & P_1 \arrow[r] & P_0 \arrow[r] & M \arrow[r] & 0
\end{tikzcd}
\end{center}
the module $\ker{(P_k \to P_{k-1})}$ is projective for $k \ge \pd{A}{M} - 1$.
\end{lemma}

\begin{proof}
We proceed by induction on $\pd{A}{M}$. For the case $\pd{A}{M} = 0$ then $M$ is projective so the exact sequence,
\begin{center}
\begin{tikzcd}
0 \arrow[r] & K \arrow[r] & P_0 \arrow[r] & M \arrow[r] & 0
\end{tikzcd}
\end{center}
splits so $P_0 = K \oplus M$ proving that $K$ is also projective giving the case $k = 0$. Replacing $M$ by $K = \ker{(P_0 \to M)}$ we prove $\ker{(P_k \to P_{k-1})}$ is projective for all $k$.
\bigskip\\
Now for induction suppose $\pd{A}{M} = d + 1$ and let,
\begin{center}
\begin{tikzcd}
0 \arrow[r] & \tilde{P}_{d+1} \arrow[r] & \cdots \arrow[r] & \tilde{P}_1 \arrow[r] & \tilde{P}_0 \arrow[r] & M \arrow[r] & 0
\end{tikzcd}
\end{center}
be a minimal length projective resolution. By Schanuel's lemma,
\[ \tilde{P}_0 \oplus \ker{(P_0 \to M)} \cong P_0 \oplus \ker{(\tilde{P}_0 \to M)} \]
If $\pd{A}{M} = 1$ and $k = 0$ then $\ker{(\tilde{P}_0 \to M)}$ is projective meaning that $\ker{(P_0 \to M)}$ is projective as well. Otherwise let $k > 0$ and consider the projective resolutions,
\begin{center}
\begin{tikzcd}
\cdots \arrow[r] & P_3 \arrow[r] & P_2 \arrow[r] & P_1 \arrow[r] & \ker{(P_0 \to M)} \arrow[r] & 0
\\
0 \arrow[r] & \tilde{P}_{d+1} \arrow[r] & \cdots \arrow[r] & \tilde{P}_1 \arrow[r] & \ker{(\tilde{P}_0 \to M)} \arrow[r] & 0
\end{tikzcd}
\end{center}
We cannot directly apply induction because these are not resolutions of the same module. However, applying $- \oplus \tilde{P}_0$ to the first sequence and $- \oplus P_0$ to the second we get projective resolutions of $M' = \tilde{P}_0 \oplus \ker{(P_0 \to M)} \cong P_0 \oplus \ker{(\tilde{P}_0 \to M)}$
\begin{center}
\begin{tikzcd}
\cdots \arrow[r] & P_3 \oplus \tilde{P}_0 \arrow[r] & P_2 \oplus \tilde{P}_0 \arrow[r] & P_1 \oplus \tilde{P}_0 \arrow[r] & M' \arrow[r] & 0
\\
0 \arrow[r] & \tilde{P}_{d+1} \oplus P_0 \arrow[r] & \cdots \arrow[r] & \tilde{P}_1 \oplus P_0 \arrow[r] & M' \arrow[r] & 0
\end{tikzcd}
\end{center}
because direct sum is exact and preserves projectives. From the second sequence $\pd{A}{M'} \le d$ so we may apply induction and find that $\ker{(P_k \oplus \tilde{P}_0 \to P_{k-1} \oplus \tilde{P}_0)} = \ker{(P_{k+1} \to P_{k})} \oplus \tilde{P}_0$ is projective for $k \ge d-1$ and thus $\ker{(P_k \to P_{k-1})}$ is projective for $k \ge d$ completing the proof.
\end{proof}

\begin{lemma}
Let $A$ be a Noetherian ring and $M$ a finite $A$-module. Then the following are equivalent,
\begin{enumerate}
\item $\pd{A}{M} \le d$ 
\item there exists a resolution of $M$ by finite modules $F_i$ and $P_d$,
\begin{center}
\begin{tikzcd}
0 \arrow[r] & P_d \arrow[r] & F_{d-1} \arrow[r] & \cdots \arrow[r] & F_1 \arrow[r] & F_0 \arrow[r] & M \arrow[r] & 0
\end{tikzcd}
\end{center}
where the $F_i$ are finite free and $P_d$ is finite projective.
\end{enumerate}
\end{lemma}

\begin{proof}
Clearly the second implies the first since $F_i$ are projective. Given $\pd{A}{M} \le d$ we know $d - 1 \ge \pd{A}{M} - 1$. Since $A$ is Noetherian and $M$ is finite we can build a finite free resolution,
\begin{center}
\begin{tikzcd}
0 \arrow[r] & P_d \arrow[r] & F_{d-1} \arrow[r] & \cdots \arrow[r] & F_1 \arrow[r] & F_0 \arrow[r] & M \arrow[r] & 0
\end{tikzcd}
\end{center}
by taking a generating set for $M$ and the kernel $\ker{(F_k \to F_{k-1})}$ is again a finite $A$-module by the Noetherian property. Then let $P_d = \ker{(F_{d-1} \to F_{d-2})}$. Since the $F_k$ are projective, by the previous lemma $P_d$ is projective and finite as a submodule of a finite module.
\end{proof}

\begin{lemma}
Let $A$ be a Noetherian local ring and $M$ a finite $A$-module. Then the following are equivalent,
\begin{enumerate}
\item $\pd{A}{M} \le d$ 
\item there exists a resolution of $M$ by finite free modules $F_i$,
\begin{center}
\begin{tikzcd}
0 \arrow[r] & F_d \arrow[r] & F_{d-1} \arrow[r] & \cdots \arrow[r] & F_1 \arrow[r] & F_0 \arrow[r] & M \arrow[r] & 0
\end{tikzcd}
\end{center}
\end{enumerate}
\end{lemma}

\begin{proof}
This follows from above noting that finite projective $A$-modules are free because $A$ is local.
\end{proof}

\begin{prop}
Let $A$ be a ring and $M$ an $A$-module. Then the following are equivalent,
\begin{enumerate}
\item $\pd{A}{M} \le n$
\item $\Ext{i}{A}{N}{M} = 0$ for all $A$-modules $A$ and all $i \ge n + 1$
\item $\Ext{n+1}{A}{N}{M} = 0$ for all $A$-modules.
\end{enumerate}
\end{prop}

\begin{proof}
(DO THIS!!!)
\end{proof}

\begin{lemma}
Consider an exact sequence of $A$-modules,
\begin{center}
\begin{tikzcd}
0 \arrow[r] & M_1 \arrow[r] & M_2 \arrow[r] & M_3 \arrow[r] & 0
\end{tikzcd}
\end{center}
\begin{enumerate}
\item if $\pd{A}{M_2} \le n$ then $\pd{A}{M_1} \le n$ and $\pd{A}{M_3} \le n + 1$
\item if $\pd{A}{M_1} \le n$ and $\pd{A}{M_3} \le n$ then $\pd{A}{M} \le n$
\item if $\pd{A}{M_1} \le n$ and $\pd{A}{M} \le n + 1$ then $\pd{A}{M_3} \le n + 1$.
\end{enumerate}
\end{lemma}

\begin{proof}
Combine the long exact sequence of Ext groups and the previous result.
\end{proof}

\subsection{Global Dimension}

\newcommand{\gldim}[1]{\mathrm{gldim}\left( #1 \right)}

\begin{defn}
Let $A$ be a ring. The global dimension $\gldim{A}$ is the supremum of $\pd{A}{M}$ over all $A$-modules $M$.
\end{defn}

\begin{theorem}
Let $A$ be a ring. The following are equivalent,
\begin{enumerate}
\item $\gldim{A} \le n$
\item $\pd{A}{M} \le n$ for all $A$-modules $M$
\item $\pd{A}{M} \le n$ for all finite $A$-modules $M$
\item $\pd{A}{M} \le n$ for all cyclic $A$-modules $M$.
\end{enumerate}
\end{theorem}

\begin{proof}
Tag 065T.
\end{proof}

\begin{lemma}
Let $A$ be a ring, $M$ an $A$-module, and $S \subset A$ a multiplicative subset then,
\begin{enumerate}
\item $\pd{S^{-1}A}{S^{-1} M} \le \pd{A}{M}$
\item $\gldim{S^{-1} A} \le \gldim{A}$
\end{enumerate}
\end{lemma}

\begin{proof}
The functor $S^{-1}(-) : \Mod{A} \to \Mod{S^{-1} A}$ is exact and preserves projectives because it is left-adjoint to restriction which is also exact. Therefore, if $M$ has a projective $A$-resolution of length $n$ then $S^{-1} M$ has a projective $S^{-1} A$-resolution of length at most $n$ so $\pd{S^{-1} A}{S^{-1} M} \le \pd{A}{M}$. Notice that for any $S^{-1} A$-module $M$, we have $M = S^{-1} M_A$ viewing $M_A$ as an $A$-module under the restriction function. Thus, applying the first part
\begin{align*}
\gldim{S^{-1} A} & = \sup \{ \pd{S^{-1} A}{M} \mid M \in \Mod{S^{-1} A} \} \le \sup \{ \pd{A}{M_A} \mid M \in \Mod{S^{-1} A} \} 
\\
& \le \sup \{ \pd{A}{M} \mid M \in \Mod{A} \} = \gldim{A} 
\end{align*}
\end{proof}

\begin{prop}
Let $R$ be a Noetherian ring. Then,
\[ \gldim{R} = \sup \{ \gldim{R_\p} \mid \p \in \Spec{R} \} = \sup \{ \gldim{R_\m} \mid \m \in \mSpec{R} \} \]
\end{prop}

\begin{proof}
DOO!!!!!!!!!!
\end{proof}

\subsection{Auslander-Buchsbaum}

(MOST GENERAL VERSION!!)

\subsection{Regular Rings}

\begin{rmk}
Throughout let $(R, \m, \kappa)$ be a noetherian local ring. 
\end{rmk}

\begin{lemma}
We always have,
\[ \dim_\kappa \m / \m^2 \ge \dim{R} \]
\end{lemma}

\begin{proof}
By Nakayma, $n = \dim_\kappa \m / \m^2$ is the minimal number of generators of $\m$.   Then by Krull's ideal theorem, $\dim{R} = \height{\m} \le n$.
\end{proof}

\begin{cor}
When $R$ is a Noetherian local ring, $\dim{R}$ is finite.
\end{cor}

\begin{proof}
$\dim_\kappa \m / \m^2$ is finite because $\m$ is finitely generated since $R$ is Noetherian.
\end{proof}

\begin{defn}
We say that $R$ is a \textit{regular local ring} if $\dim_\kappa \m / \m^2 = \dim{R}$.
\end{defn}

\begin{prop}
Let $R$ be a regular local ring. Then $\gldim{R} \le \dim{R}$.
\end{prop}

\begin{proof}
DO!!!!
\end{proof}

\begin{prop}
Let $(R, \m, \kappa)$ be a noetherian local ring then $\pd{R}{\kappa} \ge \dim_\kappa \m / \m^2$.
\end{prop}

\begin{proof}
Tag 00OA.
\end{proof}

\begin{prop}
If $\pd{R}{\kappa} < \infty$ then $\dim{R} \ge \pd{R}{\kappa}$.
\end{prop}

\begin{proof}
Tag 00OB.
\end{proof}

\begin{prop}
Let $R$ be a Noetherian local ring. If $\pd{R}{\kappa} < \infty$ then $R$ is a regular local ring.
\end{prop}

\begin{proof}
The above propositions give $\dim{R} \ge \pd{R}{\kappa} \ge \dim_\kappa \m / \m^2$ but $\dim_\kappa \m / \m^2 \ge \dim{R}$.
\end{proof}

\begin{prop}
Let $(R, \m, \kappa)$ be a noetherian local ring. Then $\gldim{R} < \infty$ if and only if $R$ is a regular local ring in which case $\gldim{R} = \dim{R}$.
\end{prop}

\begin{proof}
If $R$ is regular local then $\gldim{R} \le \dim{R}$. Conversely, if $\gldim{R}$ is finite then $\pd{R}{\kappa} < \infty$ so $R$ is reglar local. In this case, $\pd{R}{\kappa} = \dim{R}$ and $\gldim{R} \le \dim{R}$ so $\gldim{R} = \dim{R}$.
\end{proof}

\begin{lemma}
If $R$ is reglar local then $R_\p$ is regular local for each prime $\p \in \Spec{R}$.
\end{lemma}

\begin{proof}
If $R$ is regular local then $\gldim{R} < \infty$ and thus $\gldim{R_\p} \le \gldim{R} < \infty$. Since $R_\p$ is local and noetherian, $R_\p$ is regular local as well.
\end{proof}

\begin{defn}
A noetherian ring $R$ is \textit{regular} if $R_\p$ is regular local for each $\p \in \Spec{R}$.
\end{defn}

\begin{rmk}
The preceeding Lemma says that a regular local ring is regular. 
\end{rmk}

\begin{rmk}
It suffices to check regularity at $R_\m$ for maximal ideals $\m \in \mSpec{R}$ since $R_\p$ is a localization of some $R_\m$ and we have shown that localization preserves being regular local.
\end{rmk}

\begin{prop}
Let $R$ be a Noetherian ring. The following are equivalent for each $n \in \N$,
\begin{enumerate}
\item $\gldim{R} \le n$

\item for each $\m \in \mSpec{R}$ the ring $R_\m$ is regular and $\dim{R_\m} \le n$

\item for each $\p \in \mSpec{R}$ the ring $R_\p$ is regular and $\dim{R_\p} \le n$.
\end{enumerate}
Therefore, if $\gldim{R} < \infty$ then $R$ is regular and if $R$ is regular then
\[ \gldim{R} = \sup \{ \dim{R_\m} \mid \m \in \mSpec{R} \} = \sup \{ \dim{R_\p} \mid \p \in \Spec{R} \} \]
\end{prop}

\begin{proof}
This follows from,
\[ \gldim{R} = \sup \{ \gldim{R_\m} \mid \m \in \mSpec{R} \} \]
and the fact that $\gldim{R_\m} < \infty$ is equivalent to regularity of $R_\m$ in which case $\gldim{R_\m} = \dim{R_\m}$. 
\end{proof}

\begin{rmk}
Notice that even when $R$ is regular $\gldim{R}$ may be infinite simply because the dimensions of $R_\m$ for $\m \in \mSpec{R}$ may be unbounded even when $R$ is Noetherian. In this case, $\dim{R} = \infty$ so if $\dim{R}$ is finite then $\gldim{R}$ is finite iff $R$ is regular.
\end{rmk}

\section{Pseudomorphisms}

\begin{lemma}
Let $f : X \to Y$ be a morphism of schemes such that for each weakly associated point $y \in Y$ there exists a point $x \in X$ such that $f(x) = y$ and $\stalk{Y}{y} \to \stalk{X}{x}$ is injective. Then the map on sheaves $\struct{Y} \to f_* \struct{X}$ is injective.
\end{lemma}

\begin{proof}
To show that $\struct{Y} \to f_* \struct{X}$ is injective, it suffices to show that $\stalk{Y}{y} \to (f_* \struct{X})_y$ is injective on each weakly associated point $y \in Y$. Furthermore, we know there exists $x \in X$ with $f(x) = y$ and the composition $\stalk{Y}{y} \to (f_* \struct{X})_y \to \stalk{X}{x}$ is injective and thus $\stalk{Y}{y} \to (f_* \struct{X})_y$ is injective.
\end{proof}

\begin{rmk}
In particular, if $f : X \to Y$ is a dominant map of integral schemes then $\struct{Y} \to f_* \struct{X}$ is injective.
\end{rmk}

\begin{example}
Consider the map $\Spec{k[x]} \to \Spec{k[x,y]/(xy, y^2)}$. Although this map hits the generic point $(y)$, it does not hit the embedded associated point $(x, y^2)$ at the origin and thus $k[x,y]/(xy, y^2) \to k[x]$ is not injective since $y \mapsto 0$.
\end{example}

\begin{defn}
We say an immersion $\iota : Y \embed X$ is \textit{scheme theoretically dense} if the scheme theoretic image is $X$. 
\end{defn}

\begin{lemma}
An open immersion $\iota : U \to X$ is scheme theoretically dense iff $U$ contained all weakly associated points of $X$.
\end{lemma}

\begin{proof}

\end{proof}

When can we ensure that the coker of $\struct{Y} \to f_* \struct{X}$ is supported in codimension one.

\subsection{Annhiliators}

\begin{rmk}
Here we let $X$ be a scheme. Warning: many of these results do not hold for arbitrary locally ringed spaces. In particular, the kernel of quasi-coherent sheaves need not be quasi-coherent on an arbitrary locally ringed space. However, this holds locally on schemes because kernels and cokerns of sheaves associated to modules are associated to modules.
\end{rmk}

\begin{defn}
Let $\F$ be a sheaf of $\struct{X}$-modules. Then we define the sheaf of annihilators: 
\[ \shAnn{\struct{X}}{\F} = \ker{(\struct{X} \to \shHom{\struct{X}}{\F}{\F})} \]
\end{defn}

\begin{lemma}
Let $\F, \G$ be quasi-coherent $\struct{X}$-modules with $\F$ finitely presented. Then $\shHom{\struct{X}}{\F}{\G}$ is quasi-coherent.
\end{lemma}

\begin{proof}
Locally on $U \subset X$ we have a presentation,
\begin{center}
\begin{tikzcd}
\bigoplus_{i = 1}^m \struct{U} \arrow[r] & \bigoplus_{j = 1}^n \struct{U} \arrow[r] & \F|_U \arrow[r] & 0
\end{tikzcd}
\end{center}
Applying the functor $\shHom{\struct{U}}{-}{\G}$ gives,
\begin{center}
\begin{tikzcd}
0 \arrow[r] & \shHom{\struct{U}}{\F|_U}{\G|_U} \arrow[r] & \bigoplus_{j = 1}^n \G|_U \arrow[r] & \bigoplus_{i = 1}^m \G|_U
\end{tikzcd}
\end{center}
Since $\G$ is quasi-coherent and finite sums and kernels of quasi-coherent sheaves are quasi-coherent we see that $\shHom{\struct{X}}{\F}{\G}$ is locally quasi-coherent and thus quasi-coherent.
\end{proof}

\begin{lemma}
If $\F$ is finitely presented then $\shAnn{\struct{X}}{\F}$ is quasi-coherent.
\end{lemma}

\begin{proof}
From the previous lemma, $\shHom{\struct{X}}{\F}{\F}$ is quasi-coherent. Therefore, the kernel,
\[ \shAnn{\struct{X}}{\F} = \ker{(\struct{X} \to \shHom{\struct{X}}{\F}{\F})} \]
is quasi-coherent.
\end{proof}

\begin{prop}
Let $\F$ be finitely presented. Then $\Supp{\struct{X}}{\F}$ is closed and the quasi-coherent sheaf of ideals $\shAnn{\struct{X}}{\F}$ gives a scheme structure on $\Supp{\struct{X}}{\F}$. Furthermore, $\F$ is naturally a $\struct{X} / \shAnn{\struct{X}}{\F}$ - module.
\end{prop}

\begin{lemma}
Let $f : X \to Y$ be a morphism of schemes. Assume that $\struct{Y}$ and $f_* \struct{X}$ are coherent on $Y$. Furthermore, for each generic point of an irreducible component $\xi \in Y$ assume that there exists some $x \in X$ with $f(x) = \xi$ and $\stalk{Y}{\xi} \to \stalk{X}{x}$ surjective. Then $\Csh = \coker{(\struct{Y} \to f_* \struct{X})}$ has $Z = \Supp{\struct{Y}}{\Csh}$ in positive codimension.
\end{lemma}




\section{Singularities of Curves}


\begin{defn}
NORMALIZATION
\end{defn}

\begin{prop}
Normalization of a curve exists and is regular. 
\end{prop}

(CAN WE GET $H^0(O_X)$ is the same?)


\section{Jacobson Rings}

\begin{definition}
A ring $A$ is \textit{weakly-Jacobson} if $\nilrad{A} = \Jac{A}$. 
\end{definition}

\begin{definition}
A topological space $X$ is \textit{weakly-Jacobson} if the closed points of $X$ are dense.
\end{definition}

\begin{proposition}
A ring $A$ is weakly-Jacobson iff $\Spec{A}$ is weakly-Jacobinson.
\end{proposition}

\begin{proof}
The closed points of $\Spec{A}$ are maximal ideals $\m$ which are dense iff there exists a maximal ideal in each nonempty principal open $\m \in D(f)$ where $f \notin \nilrad{A}$ since $D(f)$ is nonempty. Thus, $\Spec{A}$ is weakly-Jacobinson iff 
\[ f \notin \nilrad{A} \implies \exists \m \in \mSpec{A} : f \notin \m \]  
This is equivalent to $f \in \Jac{A} \implies f \in \nilrad{A}$ so $\Spec{A}$ is weakly-Jacobson iff $\Jac{A} \subset \nilrad{A}$ however $\Jac{A} \supset \nilrad{A}$ by definition so this is equivalent to $\nilrad{A} = \Jac{A}$.
\end{proof}

\begin{definition}
A ring $A$ is \textit{Jacobson} if for any ideal $I \subset A$,
\[ \sqrt{I} = \bigcap_{\m \supset I} \m \]
\end{definition}

\begin{proposition}
A ring is Jacobson iff every quotient is weakly Jacobinson.
\end{proposition}

\begin{proof}
For any ideal $I \subset A$ then consider $A / I$. We know $\nilrad{A / I} = \Jac{A / I}$ iff
\[ \sqrt{I} = \bigcap_{\m \supset I} \m \]
since $\Spec{A / I} = V(I)$ so the result follows.
\end{proof}

\begin{definition}
A topological space $X$ with closed points $X_0$ is Jacobson if for every closed subspace $Z \subset X$ we have $\overline{Z \cap X_0} = Z$ i.e. the closed points are dense in $Z$. 
\end{definition}

\begin{remark}
Clearly Jacobson rings and spaces are weakly Jacobson. 
\end{remark}

\begin{proposition}
A ring $A$ is Jacobson iff its spectrum $\Spec{A}$ is Jacobson.
\end{proposition}

\begin{proof}
Every closed subset $Z \subset \Spec{A}$ is of the form $Z = V(I) = \Spec{A / I}$ for some ideal $I \subset A$. Any closed point of $X$ is closed in $Z$ so if $Z \cap X_0$ is dense then $Z$ is weakly-Jacobson so $\Spec{A / I}$ is weakly-Jacobson for each ideal $I \subset A$ and thus $A$ is Jacobson. Conversely, if $A$ is Jacobson and $Z = V(I) \subset \Spec{A}$ is closed. Any nonempty open of $Z$ is of the form $U \cap Z \subset Z$ for an open $U \subset \Spec{A}$ which contains some principal affine open $D(f) \subset U$ such that $Z \cap D(f)$ is nonempty. Then, $f \notin \sqrt{I}$ but $A$ is Jacobson so,
\[ \sqrt{I} = \bigcap_{\m \supset I} \m \]
and thus $f \notin \m$ for some $\m \supset I$ so $\m \in D(f)$ and $\m \supset I$ i.e. $\m \in V(I) \cap D(f) \subset Z \cap U$ so closed points are dense in $Z$ so $X$ is Jacobson.
\end{proof}

\begin{theorem}
Finitely generated $k$-algebras are Jacobson.
\end{theorem}

\begin{proof}
Let $B$ be a finitely generated $k$-algebra. Then for any ideal $I \subset B$ we need to show that $\nilrad{B / I} = \Jac{B / I}$ which is equivalent to $\Jac{B / \sqrt{I}} = (0)$ since $\nilrad{B / \sqrt{I}} = (0)$. Since $B / \sqrt{I}$ is reduced and a finitely generated $k$-alegbra, it suffices to prove that if $A$ is a reduced finitely generated $k$-algebra then $\Jac{A} = 0$.
\bigskip\\
Let $f \in A$ be nonzero, we wish to show that $f \notin \Jac{A}$ i.e. there is a maximal ideal in $D(f)$. Since $A$ is reduced and $f$ is nonzero $f$ is not nilpotent so $A_f \neq 0$ meaning there exists a maximal ideal $\m \in \mSpec{A_f}$. Let $\p \subset A$ be the preimage of $\m$ under $A \to A_f$. Then there are inclusions,
\[ k \embed A / \p \embed A_f / \m \]
Since $\m \subset A_f$ is maximal $A_f / \m$ is a field and $A_f = A[f^{-1}]$ is a finite type $k$-algebra hence $A_f / \m$ is also so by the Nullstellensatz $A_f / \m$ is a finite extension of $k$. Thus, since $\p$ is prime, $A / \p$ is a domain and $A / \p \embed A_f / \m$ so $A / \p$ is a finite dimensional domain and thus a field so $\p$ is maximal with $f \notin \p$ so $f \notin \Jac{A}$ and thus $\Jac{A} = (0)$. 
\end{proof}


\section{Versions of Hilbert's Nullstellensatz}

\begin{rmk}
Versions of the Nullstellensatz are loosely divided into ``weak'' and ``strong'' depending on if they apply to showing an ideal is proper or to the structure of all radical ideals. Zariski's lemma is also sometimes called a Nullstellensatz because it features critically in many of its proofs. 
\end{rmk}

\begin{lemma}[Zariski, Weak Nullstellensatz I]
Let $E$ be a finitely generated $k$-algebra and a field then $E / k$ is a finite extension of fields.
\end{lemma}

\begin{proof}
This follows directly from Noetherian normalization.
\end{proof}

\begin{thm}[Weak Nullstellensatz II]
Let $k$ be algebraically closed. Then the maximal ideals of $k[x_1, \dots, x_n]$ are $\m_\lambda = (x_1 - \lambda_1, \dots, x_n - \lambda_n)$ for $\lambda = (\lambda_1, \dots, \lambda_n) \in k^n$.
\end{thm}

\begin{proof}
Let $A = k[x_1, \dots, x_n]$ and $\m \subset A$ be a maximal ideal. Then $A / \m$ is a finitely generated $k$-algebra and a field so $A / \m$ is a finite extension of $k$ but $k$ is algebraically closed so $A / \m = k$. The map $A / \m \to k$ must take $x_i \mapsto \lambda_i \in k$ and thus the kernel of this map is $\m = (x_1 - \lambda_1, \dots, x_n - \lambda_n)$. 
\end{proof}

\begin{theorem}[Strong Nullstellensatz I]
Finitely generated $k$-algebras are Jacobson.
\end{theorem}

\begin{proof}
By taking the quotient, it suffices to prove that $\nilrad{A} = \Jac{A}$ for finitely generated $k$-algebras. Let $a \notin \nilrad{A}$ then $A_a$ is nonzero because $a$ is not nilpotent. Choose a maximal ideal $\m_0 \subset A_a$. Under the map $\varphi : A \to A_a$ consider $\m = \varphi^{-1}(\m_0)$ which is a maximal ideal because $A/\m \embed A_a / \m_0$ and $A_a / \m_0$ is a finitely generated $k$-algebra and a field and thus a finite field extension of $k$ proving that $A / \m \subset A_a / \m_0$ is a finite dimensional domain and thus a field so $\m$ is maximal. However, if $a \in \m$ then $a \in \m_0$ but $a \in A_a$ is a unit and thus it cannot lie in any maximal ideal. Thus $a \notin \m$ so $a \notin \Jac{A}$. Therefore, $\Jac{A} \subset \nilrad{A}$ so $\Jac{A} = \nilrad{A}$.
\end{proof}

\begin{definition}
Let $I \subset k[x_1, \dots, x_n]$ be an ideal then $V(I) \subset \bar{k}^n$ is the common vanishing set,
\[ V(I) = \{ z \in k^n \mid \forall f \in I : f(z) = 0 \} \]
For a subset $Z \subset \bar{k}^n$ define the ideal $I(Z) \subset k[x_1, \dots, x_n]$ of polynomials vanishing on $Z$,
\[ I(Z) = \{ f \in k[x_1, \dots, x_n] \mid \forall z \in Z : f(z) = 0 \} \]
\end{definition}

\begin{remark}
For general $k$, the set $V(I) \subset k^n$ corresponds exactly to the closed points of $V(I) = \Spec{k[x_1, \dots, x_n]/I} \subset \A^n_k = \Spec{k[x_1, \dots, x_n]}$. 
\end{remark}

\begin{prop}
Let $Z \subset \bar{k}^n$ be a subset. Then, in the Zariski topology,
\[ V(I(Z)) = \overline{Z} \]
\end{prop}

\begin{proof}
The vanishing of functions is closed and the intersection of such sets is closed so $V(I)$ is always closed. Futhermore, it is clear that $V(I(Z)) \supset Z$ so $V(I(Z)) \supset \overline{Z}$. However, if $Y \supset Z$ is Zariski closed then $Y = V(J)$ for some ideal $J \subset k[x_1, \dots, x_n]$. However, $V(J) \supset Z$ so $J$ must vanish on $Z$ and thus $J \subset I(Z)$ so $V(J) \supset V(I(Z))$. Therefore, $V(I(Z)) = \overline{Z}$. 
\end{proof}

\begin{thm}[Weak Nullstellensatz III]
Let $k$ be algebraically closed and $J \subset k[x_1,\dots, x_n]$ be an ideal. Then, $V(J) = \varnothing \iff J = (1)$. 
\end{thm}

\begin{proof}
A point $\lambda \in V(J)$ is equivalent to $\m_\lambda \supset J$. If $J$ is proper then it is contained in some maximal ideal $\m$. When $k$ is algebraically closed every maximal ideal is of the form $\m_\lambda$ so $\m_\lambda \supset J$ and thus $\lambda \in V(J)$. Therefore, if $J \neq (1)$ then $V(J) \neq \varnothing$. If $J = (1)$ then clearly $V(J) = \varnothing$ since $1$ does not vanish anywhere.
\end{proof}

\begin{thm}[Strong Nullstellensatz II]
Let $k$ be algebraically closed and $J \subset k[x_1,\dots, x_n]$ be an ideal. Then, $I(V(J)) = \sqrt{J}$. 
\end{thm}

\begin{proof}
Since $k[x_1, \dots, x_n]$ is Jacobson we have,
\[ \sqrt{J} = \bigcap_{\m \supset I} \m \]
but since $k$ is algebraically closed every maximal ideal is of the form $\m_\lambda$ so,
\[ \sqrt{J} = \bigcap_{\m_\lambda \supset J} \m_\lambda = \bigcap_{\lambda \in V(J)} I(\lambda) = I(V(J)) \]
\end{proof}

\begin{remark}
These imply that $I(V(I(Z))) = I(Z)$ and $V(I(V(J))) = V(J)$.
\end{remark}

\begin{rmk}
The weak Nullstellensatz directly implies the strong version using the Rabinowitsch trick. Indeed, let $J = (f_1, \dots, f_m)$ and $f \in I(Z(J))$. Then the polynomials $f_1, \dots, f_m, 1 - x_0 f$ in $k[x_0, x_1, \dots, x_n]$ have no common solution so by the weak form there are $g_i \in k[x_0, x_1, \dots, x_n]$ such that,
\[ g_1 f_1 + \cdots + g_m f_m + g_{m+1}(1 - x_0 f) = 1 \]
This holds for all $x_0$ hence for $x_0 = f^{-1}$ in $K(x_1, \dots, x_n)$ so we get,
\[ \sum_{i = 1}^r g_i(f(x_1, \dots, x_n)^{-1}, x_1, \dots, x_n) f_i(x_1, \dots, x_n) = 1 \]
The only denominators are $f$ so clearing denominators,
\[ f^r = \sum_{i = 1}^n h_i f_i \]
where $h_i(x_1, \dots, x_n) = f(x_1, \dots, x_n)^r g_i(f(x_1, \dots, x_n)^{-1}, x_1, \dots, x_n)$ proving that $f \in \sqrt{J}$. 
\bigskip\\
This is secretly the same proof that goes from Weak Nullstellensatz I to Strong Nullstellensatz I by localizing $A \to A_f$ which is what the Rabinowitsch trick accomplishes. The condition $f \in Z(I(J))$ becomes that $J$ generates the unit ideal of $A_f$ since it does not lie in any maximal ideal by the weak form. Thus $f \in \sqrt{J}$ (this is the converse of choosing a maximal ideal assuming $f \notin \sqrt{J}$).
\end{rmk}

\begin{remark}
In the context of affine schemes, for a subspace $Z \subset \Spec{A}$, we define,
\[ I(Z) = \bigcap_{\p \in Z} \p \] 
it is clear $V(I(Z)) = \overline{Z}$ and $I(V(J)) = \sqrt{J}$ by definition giving the bijection between Zariski closed subsets and radical ideals. This makes the statements corresponding to classical versions of the Nullstellensatz for schemes definitions rather than theorems.
\end{remark}

\section{Jacobson Schemes}

\begin{rmk}
Here we use Chevalley's theorem to reprove the versions of the Nullstellensatz. 
\end{rmk}

\begin{theorem}[Chevalley]
Let $f : X \to Y$ be finite presentation morphism of schemes and $C \subset X$ locally construcible. Then $f(C)$ is locally constructible.
\end{theorem}

\begin{rmk}
In quasi-compact schemes, locally construcible and constructible coincide. 
\end{rmk}

\begin{thm}[Nullstellensatz]
Let $K$ be a finite type $k$-algebra and a field then $K/k$ is finite.
\end{thm}

\begin{proof}
Suppose not. Then there is an injection $k[t] \embed K$ because $K$ cannot be algebraic. Then $\Spec{K} \to \A^1_k$ so by Chevalley the image is constructible. But the image the generic point which is not constructible giving a contradiction. 
\end{proof}


\begin{defn}
$X$ is \textit{Jacobson} if every nonempty constructible subset has a closed (in $X$) point.
\end{defn}

\begin{rmk}
This is equivalent to the closed points being dense in every closed set.
\end{rmk}

\begin{cor}
$k$-schemes locally of finite type are Jacobson.
\end{cor}

\begin{proof}
Let $C \subset X$ be construtible then there is a closed $Z \subset X$ such that $Z \cap C$ contains a nonempty affine open $U = \Spec{A}$ of $Z$. Since $Z$ is closed it suffices to show that $U$ contains a closed point of $Z$. Since $A$ is nonzero there is some maximal ideal $\m \subset A$ and $A$ is a finite-type $k$-algebra so $A / \m$ is a finite extension of $k$. Hence the corresponding point $x \in X$ is closed.
\end{proof}


\begin{example}
Some (non) examples of Jacobson schemes,
\begin{enumerate}
\item finte type $k$-schemes are Jacobson
\item $\Spec{\Z}$ is Jacobson
\item if $R$ is a local ring of $\dim{R} \ge 1$ then not Jacobson
\item $X = \Spec{R} \setminus \{ \m_R \}$ is Jacobson.
\end{enumerate}
\end{example}

\begin{prop}
Let $S$ be Jacobson and $f : X \to S$ is finite type. 
\begin{enumerate}
\item If $x \in X$ is a closed point then $f(x)$ is closed.
\item $X$ is Jacobson.
\end{enumerate}
\end{prop}

\begin{proof}
For (a) let $x \in X$ be a closed point then Chevalley's theorem implies that $\{ f(x) \}$ is constructible so $\{ f(x) \}$ is closed because $S$ is Jacobson. For (b) let $C \subset X$ be constructible. Then Chevalley's theorem implies that $f(C) \subset S$ is constructible so there is a closed point $s \in f(C)$. Then $X_s \to \kappa(s)$ is finite type so $X_s$ is Jacobson. Then $X_s \cap C \subset X_s$ is nonempty constructible so it has a closed point $x \in C \cap X_s$ and $X_s$ is closed (because $s \in S$ is closed) so $x$ is a closed point.
\end{proof}

\begin{cor}
Let $X$ be a finite type $\Z$-scheme. Then $X$ is Jacobson and for each closed $x \in X$ the residue field $\kappa(x)$ is finite.
\end{cor}

\begin{proof}
The first part is immediate from the fact that $\Spec{\Z}$ is Jacobson. If $x \in X$ is a closed point then it lies over some $p \in \Spec{\Z}$ nonzero (because $x$ is closed) so $x \in X_p$ and $X_p$ is finite type over $\kappa(p) = \FF_p$ so $\kappa(x) / \FF_p$ is finite by the Nullstellensatz.
\end{proof}

\begin{prop}
If $X \to \Spec{\Z}$ is finite type and $X$ is reduced then there is a dense open such that $U \to \Spec{\Z}$ is smooth.
\end{prop}


\end{document}