\documentclass[12pt]{article}
\usepackage{import}
\import{./}{AlgGeoCommands}
\renewcommand{\U}{\mathfrak{U}}

\usepackage{dutchcal}
\DeclareMathAlphabet{\pazocal}{OMS}{zplm}{m}{n}


\newcommand{\fppf}{\mathrm{fppf}}
\newcommand{\Set}{\mathrm{Set}}
\newcommand{\Def}{\mathrm{Def}}
\newcommand{\cDef}{\mathcal{Def}}
\newcommand{\Inf}{\mathrm{Inf}}
\renewcommand{\X}{\pazocal{X}}
\newcommand{\Y}{\mathcal{Y}}
\newcommand{\Spf}[1]{\mathrm{Spf} \! \left(#1 \right)}
\renewcommand{\F}{\mathcal{F}}
\renewcommand{\G}{\mathcal{G}}
\newcommand{\cZ}{\pazocal{Z}}

\begin{document}

\title{Artin's Axioms and Formal Deformation Theory}
\author{Stanford-Berkeley Number Theory Learning Seminar}
\maketitle

\section{Definitions}


\begin{defn}
An \textit{algebraic space} is a functor $X : (\Sch_S)_{\fppf}^{\op} \to \Set$ such that,
\begin{enumerate}
\item $F$ is a sheaf in the $\fppf$ topology
\item the diagonal $\Delta_{X/S} : X \to X \times_S X$ is representable by schemes
\item there is a scheme $U$ and an \etale surjection $U \onto X$.
\end{enumerate}
\end{defn}

\begin{defn}
An \textit{algebraic stack} is a category fibered in groupoids $\X \to (\Sch_S)_{\fppf}$ such that,
\begin{enumerate}
\item $\X$ is a stack in the $\fppf$ topology
\item $\Delta_{\X / S} : \X \to \X \times_S \X$ is representable by algebraic spaces 
\item there is an algebraic space $U$ and an \etale surjection $U \onto \X$.
\end{enumerate}
\end{defn}

\begin{rmk}
The map $U \to \X$ is only necessarily representable by algebraic spaces so to express the property of being an \etale surjection consider any map from a scheme $T \to \X$ and an \etale cover from a scheme $V \to U \times_{\X} T$ in the diagram,
\begin{center}
\begin{tikzcd}
V \arrow[r] \arrow[rr, bend left, "\et \text{ surj}"] & U \times_{\X} T \arrow[d] \arrow[r] & T \arrow[d]
\\
& U \arrow[r] & \X  
\end{tikzcd}
\end{center}
This property is independent of the choice of \etale cover $V \to U \times_{\X} T$ by \etale descent for \etale surjective morphisms.
\end{rmk}

\begin{rmk}
Why do we only require that $\X$ be smooth locally an algebraic space and its diagonal be representable by only algebraic spaces? The diagonal is closely related to the automorphism groups of objects $\X$ parametrizes. When $\pi : X \to S$ a proper finitely presented map of schemes, $\Hilb_{X/S}$ is representable by an algebraic space but not generally by a scheme unless $\pi$ is projective. This shows that $\mathrm{Isom}_S(X, Y)$ between two proper $S$-schemes is usually only representable by an algebraic space. Therefore, we want to allow for $\Delta$ to be representable by algebraic spaces not just schemes to capture moduli of proper non-projective objects. 
\end{rmk}

\begin{defn}
Consider $f : \X \to \cZ$ and $g : \Y \to \cZ$ morphisms of categories fibered in groupoids. Then the $2$-fiber product $\X \times_{\cZ} \Y$ is defined as the category fibered in groupoids,
\begin{enumerate}
\item objects are $(x,y,\gamma)$ with $p(x) = p(y)$ and $\gamma : f(x) \to g(y)$ a morphism over $\id$
\item morphisms are $\varphi : (x,y,\gamma) \to (x',y', \gamma')$ are given by pairs $(\varphi_x : x \to x', \varphi_y : y \to y')$ such that the diagram,
\begin{center}
\begin{tikzcd}
f(x) \arrow[r, "\gamma"] \arrow[d, "\varphi_x"] & g(x) \arrow[d, "\varphi_y"]
\\
f(x') \arrow[r, "\gamma'"] & g(x')
\end{tikzcd}
\end{center}
commutes.
\end{enumerate}
\end{defn}

\begin{prop}
For any objects $x,y \in \X(U)$. There is a $2$-fiber product diagram,
\begin{center}
\begin{tikzcd}
\mathrm{Isom}(x,y) \arrow[r] \arrow[d] & U \arrow[d, "y"]
\\
U \arrow[r, "x"'] & \X
\end{tikzcd}
\end{center}
\end{prop}

\begin{defn}
The \textit{inertia stack} of $\X$ is the category fibered in groupoids $\I_{\X} = \X \times_{\X \times \X} \X$.
\end{defn}

\begin{prop}
For any $x \in \X(U)$ there is a $2$-fiber product diagram,
\begin{center}
\begin{tikzcd}
\mathrm{Isom}(x,x) \pullback \arrow[d] \arrow[r] & \I_X \arrow[d]
\\
U \arrow[r, "x"] & \X
\end{tikzcd}
\end{center}
\end{prop}

\section{Presentations}

\begin{prop}
Let $X$ be an algebraic space over $S$ and $f : U \onto X$ an \etale surjection from a scheme $U$. Set $R = U \times_X U$ in the pullback diagram,
\begin{center}
\begin{tikzcd}
R \arrow[d] \arrow[r] & U \arrow[d]
\\
U \arrow[r] & X 
\end{tikzcd}
\end{center}
then we have,
\begin{enumerate}
\item $j : R \to U \times_S U$ is a monomorphism and $R(T) \subset U(T) \times U(T)$ is an equivalence relation for all $T \to S$
\item the projections $s,t : R \to U$ are \etale
\item the diagram,
\begin{center}
\begin{tikzcd}
R \arrow[r, shift left, "s"] \arrow[r, shift right, "t"'] & U \arrow[r] & X
\end{tikzcd}
\end{center}
is a coequalizer in $\mathrm{Sh}((\Sch_S)_{\fppf})$. 
\end{enumerate}
\end{prop}

\begin{proof}
The first two are immediate The last holds in any category of sheaves given that $U \to X$ is surjective. 
\end{proof}

\begin{defn}
Let $(U, R, s, t, c)$ be a groupoid in algebraic spaces over $S$. The quotient stack,
\[ p : [U/R] \to (\Sch_S)_{\fppf} \]
is the stackification of the category fibered in groupoids,
\[ (T \to S) \mapsto (U(T), R(T), s, t, c) \]
\end{defn}

\begin{prop}[\chref{https://stacks.math.columbia.edu/tag/04T5}{04T5}]
Given an algebraic stack $\X$ there is a smooth morphism $U \to \X$ from a scheme. We recover the groupoid presentation by taking the $2$-fiber product,
\begin{center}
\begin{tikzcd}
R \pullback \arrow[r] \arrow[d] \pullback & U \arrow[d]
\\
U \arrow[r] & \X 
\end{tikzcd}
\end{center}
and $R$ is an algebraic space because we assumed that $\Delta_{\X}$ is representable by algebraic spaces. Then there is a natural equivalence $[U/R] \iso \X$.
\end{prop}

\section{Infinitesimal Deformation Theory}

\begin{rmk}
First we recall how to apply infinitesimal deformation theory in the relative setting. In the basic case, we want to probe properties of a morphisms of schemes $f : X \to S$ near a finite type point $x : \Spec{k} \to S$. There is some affine open $\Spec{\Lambda} \subset X$ containing $x$. Then we need to consider Artinian local rings $A$ and diagrams,
\begin{center}
\begin{tikzcd}
& & & X \arrow[d, "f"]
\\
\Spec{k} \arrow[r] & \Spec{A} \arrow[r] \arrow[rru, dashed] & \Spec{\Lambda} \arrow[r, hook] & S 
\end{tikzcd}
\end{center}
and consider the set of dashed arrows. This means our base category should be the category of local Artinian $\Lambda$-algebras with residue field $k$.
\end{rmk}

\begin{defn}
Let $\Lambda$ be a noetherian ring and $\Lambda \to k$ a finite ring map with $k$ a field. Let $\C_\Lambda$ be the category of,
\begin{enumerate}
\item $(A, \varphi)$ where $A$ is an Artinian local $\Lambda$-algebra and $\varphi : A / \m_A \to k$ a $\Lambda$-algebra isomorphism
\item morphisms $f : (B, \psi) \to (A, \varphi)$ are local $\Lambda$-algebra maps such that $\varphi \circ (f \text{ mod } \m_A) = \psi$
\end{enumerate}
\end{defn}

\begin{rmk}
As in the absolute case (which corresponds to $\Lambda = k$) we can factor any extension $B \onto A$ into \textit{small} extensions $\varphi : B' \onto A$ where $\ker{\varphi}$ is principal and annihilated by $\m_B$.
\end{rmk}

\begin{defn}
Let $\Lambda$ be a Noetherian ring and let $\Lambda \to k$ be a finite ring map where $k$ is a field. Define the category $\widehat{\C}_\Lambda$ of,
\begin{enumerate}
\item pairs $(R, \varphi)$ where $R$ is a Noetherian complete local $\Lambda$-algebra and $\varphi : R / \m_R \to k$ is a $\Lambda$-algebra isomorphism,
\item morphisms $f : (S, \psi) \to (R, \varphi)$ are local $\Lambda$-algebra map such that $\varphi \circ (f \text{ mod } \m_S) = \psi$.
\end{enumerate}
\end{defn}

\begin{rmk}
Then $\C_{\Lambda} \subset \widehat{\C}_{\Lambda}$ is naturally a full subcategory. 
\end{rmk}

\subsection{Deformation Functors}

\begin{defn}
A \textit{predeformation functor} is a functor $F : \C_{\Lambda} \to \Set$ such that $F(k) = \{ * \}$.
\end{defn}

\begin{rmk}
The condition $F(k) = \{ * \}$ corresponds to choosing a fixed base object for the deformations.
\end{rmk}

\begin{defn}
Given a predeformation functor $F : \C_{\Lambda} \to \Set$ we extend it to $\wh{F} : \wh{\C}_{\Lambda} \to \Set$ via,
\[ \wh{F}(R) = \varprojlim_n F(R/\m_R^n) \]
A functor $F$ is \textit{pro-representable} if $\wh{F}$ is representable.
\end{defn}

\begin{defn}
We say a morphism $\varphi : F \to G$ of functors on $\C_\Lambda$ is \textit{smooth} the map,
\[ F(B) \to F(A) \times_{G(A)} G(B) \]
induced by an extension $B \onto A$ is surjective.
\end{defn}

\begin{defn}
Let $F : \C_\Lambda \to \Set$ be a predeformation functor. The \textit{tangent space} of $F$ is the set $T F = F(k[\epsilon])$. We will see under some assumptions this set is naturally a $k$-vectorspace.
\end{defn}

\begin{defn}
Let $F : \C_\Lambda \to \Set$ be a predeformation functor. A \textit{hull}\footnote{Some authors use the terminology \textit{miniversal} formal object. However, in the deformation category setting, a minimal versal object may not induce an isomorphism of the tangent space so we reserve the term \textit{miniversal} for a minimal versal object see \chref{https://stacks.math.columbia.edu/tag/06T0}{Tag 06T0}.} for $F$ is a pair $(R, \eta)$ where $R \in \wh{\C}_{\Lambda}$ and $\eta \in \wh{F}(R)$ such that $h_R \to F$ is formally smooth and bijective on tangent spaces.
\end{defn}

\begin{rmk}
Let $k[\epsilon]$ be the ring $k[\epsilon]/(\epsilon^2)$ with the trivial $\Lambda$-algebra structure. 
\end{rmk}

\begin{defn}
Let $F : \C_\Lambda \to \Set$ be a predeformation functor. If $A' \to A$ and $A'' \to A$ are morphisms in $\C_\Lambda$ there is a natural map,
\[ (*) \quad F(A' \times_A A'') \to F(A') \times_{F(A)} F(A'') \]
Then Schlessinger's conditions are as follows,
\begin{enumerate}
\item[(H1)] if $A'' \onto A$ is a small thickening then $(*)$ is surjective
\item[(H2)] if $A = k$ and $A'' = k[\epsilon]$ then $(*)$ is bijective
\item[(H3)] $T F = F(k[\epsilon])$ is finite-dimensional
\item[(H4)] if $A'' = A'$ and $A' \onto A$ is a small thickening, then $(*)$ is bijective.
\end{enumerate}
\end{defn}

\begin{rmk}
$T F = F(k[\epsilon])$ has a canonical vectorspace structure when $F$ satisfies (H2) since we get,
\[ F(k[\epsilon]) \times F(k[\epsilon]) \iso F(k[\epsilon_1, \epsilon_2]) \to F(k[\epsilon]) \]
using the map $k[\epsilon_1, \epsilon_2] \to k[\epsilon]$ via $\epsilon_1 \mapsto \epsilon$ and $\epsilon_2 \mapsto \epsilon$. The scalar multiplication is defined by $F(k[\epsilon]) \to F(k[\epsilon])$ induced by the map $\epsilon \mapsto c \epsilon$. 
\bigskip\\
We cannot give $T F$ a vectorspace structure without (H2) so it is more correct to group the Schlessinger conditions into pairs (H1) + (H2) and (H3) + (H4) as we do in the sequel.
\end{rmk}

\begin{defn}
A pre-deformation functor $F : \C_{\Lambda} \to \Set$ is a \textit{deformation functor} if it satisfies (H1) and (H2).
\end{defn}

\begin{theorem}[Schlessinger]
Let $F : \C_{\Lambda} \to \Set$ be a deformation functor. Then,
\begin{enumerate}
\item $F$ admits a hull if and only if it satisfies (H3)
\item $F$ is pro-representable if and only if it satisfies (H3) and (H4).
\end{enumerate}
\end{theorem}

\begin{example}
Let $X$ be a $k$-scheme, the functor $\Def_X : \C_k \to \Set$ defined by,
\[ \Def_X : A \mapsto \{ (X', \psi) \mid X' \text{ flat } A\text{-scheme with } \psi : X' \ot_A k' \iso X \} / \cong \]
is a deformation functor. 
\end{example}

\begin{example}
Let $X = \Spec{k[x,y]/(xy)}$ and $F = \Def_X$. If $A$ is a finite type $k$-algebra and $P \onto A$ is a presentation from a polynomial ring with kernel $K$ then [H, Ex. 9.8] shows that,
\begin{center}
\begin{tikzcd}
\Hom{A}{\Omega_{P/k} \ot_k A}{A} \arrow[r] & \Hom{A}{J/J^2}{A} \arrow[r] & T \Def_A \arrow[r] & 0
\end{tikzcd}
\end{center}
arising from the conormal exact sequence,
\begin{center}
\begin{tikzcd}
J/J^2 \arrow[r] & \Omega_{P/k} \ot_P A \arrow[r] & \Omega_{A/k} \arrow[r] & 0
\end{tikzcd}
\end{center}
In our case, let $P = k[x,y]$ and $J = (xy)$. Then we have,
\begin{center}
\begin{tikzcd}
A \partial_x \oplus A \partial_y \arrow[r] & A \arrow[r] & T \Def_A \arrow[r] & 0
\end{tikzcd}
\end{center}
and therefore $T \Def_A = A/(x,y) = k$. Thus $\Def_X$ satisfies (H3) so it should have a hull. Indeed,
\[ (k\dbrac{t}, \, \Spf{k\dbrac{t}[x,y]/(xy - t)}) \]
is a hull (note the formal object is effective). Let's first understand why this hull is not a pro-representing object. For any map, $\varphi : k\dbrac{t} \to A$ the induced object,
\[ \varphi_*(\Spf{k\dbrac{t}[x,y]/(xy - t)}) = \Spec{A[x,y]/(xy - \varphi(t))} \]
is unchanged (in isomorphism class) if we replace $\varphi$ by $\varphi' = u \varphi$ for any unit $u \in A$ since then we can scale $x$ or $y$ to remove $u$. However, recall that a deformation $X'$ is equipped with a distinguished isomorphism $\varphi : X' \ot_A k \iso X$ with which isomorphisms of deformations must be compatible. Therefore, $\varphi' = u \varphi$ and $\varphi$ define the same deformation if $u \in A^\times$ is a unit and $u \equiv 1 \mod \m_A$. Therefore, the map, $h_R \to \Def_X$ is not injective for general $A$ but is injective for $A = k[\epsilon]$ (since $(1 +  a \epsilon) \cdot \epsilon = \epsilon$ so multiplication by such $a$ does nothing) as must be true for a hull.  
\bigskip\\
However $\Def_X$ is not pro-representable since it does not satisfy (H4). Indeed, let $A = k[\epsilon]/(\epsilon^3)$ and consider,
\[ \Def_X(A \times_k A) \to \Def_X(A) \times \Def_X(A) \]
I claim this is not injective. Indeed, $t = \epsilon_1 + \epsilon_2$ and $t = \epsilon_1 + \epsilon_2 + \epsilon_1 \epsilon_2$ map to the same pair of deformations but I claim they are not related by such a unit. Write,
\[ u = 1 + a \epsilon_1 + b \epsilon_2 + O(\epsilon^2) \]
then,
\[ u (\epsilon_1 + \epsilon_2) = \epsilon_1 + \epsilon_2 + a \epsilon_1^2 + (a + b) \epsilon_1 \epsilon_2 + b \epsilon_2^2 + O(\epsilon^3) \] 
and we cannot have $a = b = 0$ but $a + b = 1$.
\end{example}

\begin{rmk}
The above illustrates why it is necessary to define deformations of a scheme as equipped with a distinguished isomorphism $\varphi : X' \ot_A k \iso X$ otherwise $\Def_X$ will not be a deformation functor. Indeed, let $\Def'_X$ be the pre-deformation functor,
\[ \Def'_X : A \mapsto \{ X' \mid X' \text{ flat } A\text{-schemes such that } X' \ot_A k' \cong X \} / \cong \]
but forgetting the isomorphism. Then for $X = \Spec{k[x,y]/(xy)}$
\[ \Def_X'(k[\epsilon_1, \epsilon_2]) \to \Def'_X(k[\epsilon]) \times \Def'_X(k[\epsilon]) \]
is not injective. Indeed,  
\[ \Spec{k[\epsilon_1, \epsilon_2][x,y]/(xy + \epsilon_1 + \epsilon_2} \quad \text{and} \quad \Spec{k[\epsilon_1, \epsilon_2][x,y]/(xy + \epsilon_1 + 2 \epsilon_2)} \]  
have the same image but are not isomorphic. 
\end{rmk}


\subsection{Deformation Categories}

\begin{defn}
A \textit{predeformation category} is a category cofibered in groupoids $\F \to \C_{\Lambda}$ such that $\F(k)$ is equivalent to the trivial category. 
\end{defn}

\begin{rmk}
Let $\F$ be a predeformation category and $x_0 \in \F(k)$. Then for any $x \in \F$ over $A$ let $q : A \to k$ then there is a pushforward $x \to q_* x$ and $q_* x \in \F(k)$ so there is a unique isomorphism $q_* x \iso x_0$ and hence there is a canonical morphism $x \to x_0$ in $\F$. 
\end{rmk}

\begin{rmk}
If $F : \C_\Lambda \to \Set$ is a predeformation functor then the associated cofibered set $\F_F \to \C_{\Lambda}$ is a predeformation category. Likewise, if $\F \to \C_\Lambda$ is a predeformation category then the functor of isomorphism classes $\overline{\F} : \C_{\Lambda} \to \Set$ is a predeformation functor. 
\end{rmk}

\begin{defn}
Let $\F \to \C_\Lambda$ be a category cofibered in groupoids. The \textit{category of formal objects of} $\wh{\F}$ is the category of,
\begin{enumerate}
\item formal objects $(R, \xi_n, f_n)$ consists of an object $R \in \wh{\C}_\Lambda$, and objects $\xi_n \in \F(R/\m_R^n)$ and morphisms $f_n : \xi_{n+1} \to \xi_n$ over the projection $R / \m_R^{n+1} \to R / \m_R^n$

\item morphisms $a : (R, \xi_n, f_n) \to (S, \eta_n, g_n)$ consists of a map $a_0 : R \to S$ in $\wh{\C}_\Lambda$ and a collection $a_n : \xi_n \to \eta_n$ of morphisms in $\F$ lying over $R / \m_R^n \to S / \m_S^n$ such that the diagrams,
\begin{center}
\begin{tikzcd}
\xi_{n+1} \arrow[d, "a_{n+1}"] \arrow[r, "f_n"] & \xi_n \arrow[d, "a_n"]
\\
\eta_{n+1} \arrow[r, "g_n"] & \eta_n
\end{tikzcd}
\end{center} 
commute for each $n \in \N$.
\end{enumerate}
\end{defn}

\begin{prop}[\chref{https://stacks.math.columbia.edu/tag/06H4}{06H4}]
The formal objects forms a category cofibered in groupoids $\hat{p} : \wh{\F} \to \wh{\C}_\Lambda$.
\end{prop}

\begin{defn}
Let $p : \F \to \C_\Lambda$ be a category cofibered in groupoids. We say that $\F$ satisfies the \textit{Rim-Schlessinger (RS) condition} if for all $A_1 \to A$ and $A_2 \to A$ in $\C_\Lambda$ with $A_2 \onto A$ surjective,
\[ \F(A_1 \times_A A_2) \to \F(A_1) \times_{\F(A)} \F(A_2) \]
is an equivalence. A \textit{deformation category} is a predeformation category $\F$ satisfying (RS).
\end{defn}

\begin{lemma}[\chref{https://stacks.math.columbia.edu/tag/06J5}{06J5}]
Condition (RS) is equivalent to: for every diagram in $\F$,
\begin{center}
\begin{tikzcd}
& x_2 \arrow[d]
\\
x_1 \arrow[r] & x
\end{tikzcd}
\quad \text{lying over} \quad
\begin{tikzcd}
& A_2 \arrow[d]
\\
A_1 \arrow[r] & A
\end{tikzcd}
\end{center}
in $\C_\Lambda$ with $A_2 \to A$ surjective, there exists a fiber product $x_1 \times_x x_2$ in $\F$ such that the diagram,
\begin{center}
\begin{tikzcd}
x_1 \times_x x_2 \arrow[r] \arrow[d] & x_2 \arrow[d]
\\
x_1 \arrow[r] & x
\end{tikzcd}
\quad \text{lying over} \quad
\begin{tikzcd}
A_1 \times_A A_2 \arrow[r] \arrow[d] & A_2 \arrow[d]
\\
A_1 \arrow[r] & A
\end{tikzcd}
\end{center}
\end{lemma}

\begin{lemma}[\chref{https://stacks.math.columbia.edu/tag/07WQ}{07WQ}]
If $\X \to S$ is an algebraic stack then for any $\Spec{k} \to S$ and $x_0 \in \X(k)$ the deformation category $\F_{\X, k, x_0}$ satisfies (RS). 
\end{lemma}

\begin{rmk}
By Schlessinger's theorem, this is telling us that a deformation functor $F = D_{X,x_0}$ represented by some pointed finite-type quasi-separated\footnote{I don't know if these are the \textit{right} conditions but they make the discussion work.} algebraic space $x_0 \in X$ over a noetherian scheme $S$ is pro-representable. So even though $X$ does not have a canonical local ring it does have a formal local ring $\wh{\stalk{X}{x_0}}$. We can calculate it from the formal local ring of any \etale cover $U \to X$. This is well-defined because for two \etale covers $U_1 \to X$ and $U_2 \to X$ we have $U_1 \times_X U_2$ is an \etale cover of both and these maps identify the formal local rings. There is a subtly here about the residue field of the preimage of $x_0$ in these \etale covers meaning that the complete local rings will not be isomorphic until after a field extension. The technical assumptions ensure that $X$ is decent and then the discussion of \chref{https://stacks.math.columbia.edu/tag/0EMV}{Tag 0EMV} applies.
\end{rmk}

\begin{prop}
If a predeformation category $\F \to \C_\Lambda$ satisdies (RS) then the functor of isomorphism classes $\overline{\F}$ satisfies,
\begin{enumerate}
\item (H1)
\item (H2)
\item (H4) if and only if for every morphism $x' \to x$ in $\F$ over a small extension $A' \onto A$ the map $\Aut[A']{x'} \to \Aut[A]{x}$ is surjective.
\end{enumerate}
\end{prop}

\begin{rmk}
This condition on automorphisms is saying that $\mathrm{Isom}$ is formally smooth at the identity. 
\end{rmk}

\begin{rmk}
We can define comparable notions of (H1) and (H2) for deformation categories (see \chref{https://stacks.math.columbia.edu/tag/06HV}{Tag 06HV}) which are weaker than (RS) and suffice in the proof of Rim-Schlessinger (the existence of versal objects discussed in the next section) however in practice this condition turns out to be less useful. Indeed, we will need the full power of (RS) to prove Artin's criteria.
\end{rmk}

\begin{prop}
Let $X$ be a $k$-scheme. The pre-deformation category $\cDef_X \to \C_k$ given by deformations of $X$ over Artin local $k$-algebras is a deformation category.
\end{prop}

\begin{proof}
Given a diagram,
\begin{center}
\begin{tikzcd}
& A_2 \arrow[d]
\\
A_1 \arrow[r] & A
\end{tikzcd}
\end{center}
and deformations over each ring, we need to form the pushout,
\begin{center}
\begin{tikzcd}
\wt{X} \arrow[from=r] \arrow[from=d] & X_{A_2} \arrow[from=d]
\\
X_{A_1} \arrow[from=r] & X_A
\end{tikzcd}
\end{center}
Since $A_2 \onto A$ is surjective the map $X_A \embed X_{A_2}$ is an infinitesimal thickening and also $X_{A} \to X_{A_1}$ is affine. Therefore, the pushout exists by \chref{https://stacks.math.columbia.edu/tag/07RT}{Tag 07RT} which just reduces to the affine case.
\end{proof}

\begin{rmk}
Choosing a ring $\Lambda$ with a finite type map $\Lambda \to k$ I can replace $\cDef_X \to \C_k$ with $\cDef_X \to \C_\Lambda$ giving deformations over the infinitesimal neighborhood of $k$ in $\Lambda$. For example, deformations over $\FF_p$ preserves characteristic $p$ while deformations over $\ZZ \to \FF_p$ give formal lifts to characteristic zero.
\end{rmk}

\begin{prop}
Let $X$ be a $k$-scheme then,
\begin{enumerate}
\item $\Def_X$ satisfies (H1) and (H2)
\item $\Def_X$ satisfies (H3) if $X$ is proepr over $k$ or affine with isolated singularities
\item $\Def_X$ satisfies (H4) if and only if for every deformation $(X_{A'}, \varphi_{A'})$ over $A'$ and small extension $A' \to A$ every automorphism of $(X_A, \varphi_A)$ lifts to an automorphism of $(X_{A'}, \varphi_{A'})$.
\end{enumerate}
\end{prop}

\begin{proof}
Use that $\cDef_X$ satisfies (RS) and the above resutls. The tangent space $T \Def_X$ can be computed in terms of $T^1$ functor which has finite support if $X$ is affine with isolated singularities and cohomology which is finite for $X$ proper. See Hartshorne's deformation theory for details. 
\end{proof}

\begin{rmk}
Therefore, to ensure that $\Def_X$ is pro-representable we need that $\Aut{X}$ is smooth at the identity. Indeed, we saw that the node did not satisfy (H4) and this is explained by the failure of automorphisms to lift. Indeed let $A' = k[\epsilon]/(\epsilon^3) \onto A = k[\epsilon]/(\epsilon^2)$ and consider,
\[ \Spec{A'[x,y]/(xy - \epsilon)} \]
I claim that the automorphism of $\Spec{A[x,y]/(xy)}$ sending $y \mapsto (1 + \epsilon) y$ does not lift. Suppose it did, $x \mapsto x + \epsilon^2 f$ and $y \mapsto (1 + \epsilon) y + \epsilon^2 g$ then,
\[ xy - \epsilon \mapsto (1 + \epsilon) xy + \epsilon^2 (x g + y f) - \epsilon \]
which cannot equal $(1 + \epsilon) [xy - \epsilon]$ as it must to define an automorphism. Indeed, if it did, then evaluating at $x = y = 0$ gives $(1 + \epsilon) \epsilon = \epsilon$ contradicting the fact that $\epsilon^2 \neq 0$ in $A'$.
\end{rmk}


\subsection{Versality}

\begin{rmk}
A versal object is a universal object without the ``uni'' i.e. without the uniqueness. 
\end{rmk}

\begin{defn}
A morphism $\varphi : \F \to \G$ of categories cofibered in groupoids over $\C_\Lambda$ is \textit{smooth} if for every extension $B \onto A$ in $\C_{\Lambda}$ the map,
\[ \F(B) \to \F(A) \times_{\G(A)} \G(B) \]
is essentially surjective.
\end{defn}

\begin{rmk}
This is basically the formal lifting criterion for formal smoothness. Indeed, if these deformation categories are induced by the representable functors for a morphism of schemes $f : X \to Y$ then we get that,
\[ X(B) \to X(A) \times_{Y(A)} Y(B) \]
is surjective which is equivalent to there existing a dashed arrow in each lifting diagram,
\begin{center}
\begin{tikzcd}
\Spec{A} \arrow[d, hook] \arrow[r] & X \arrow[d, "f"]
\\
\Spec{B} \arrow[r]  \arrow[ru, dashed] & Y
\end{tikzcd}
\end{center} 
\end{rmk}

\begin{lemma}
Smoothness of $\varphi : \F \to \G$ is equivalent to the following explicit condition. For every surjection $B \onto A$ in $\C_{\Lambda}$ and $y \in \G(B)$ and $x \in \F(A)$ equipped with a map $y \to \varphi(x)$ over $B \onto A$ there is $x' \in \F(B)$ and a morphism $x' \to x$ over $B \onto A$ and a morphism $\varphi(x') \to y$ over $\id : B \to V$ such that,
\begin{center}
\begin{tikzcd}
\varphi(x') \arrow[r] \arrow[rd] & y \arrow[d]
\\
& \varphi(x)
\end{tikzcd}
\end{center}
\end{lemma}

\begin{defn}
Let $R \in \wh{\C}_{\Lambda}$. We say $\xi \in \wh{\F}(R)$ is \textit{versal} if the morphism $\xi : \underline{R}|_{\C_\Lambda} \to \F$ defined by $\xi$ is smooth.
\end{defn}

\begin{rmk}
The morphism is defined as follows. For any $A \in \C_{\Lambda}$ and map $\varphi : R \to A$ it will factor as $\varphi_n : R / \m^n \to A$ we send $(A, \varphi) \mapsto (\varphi_n)_* \xi_n$. The compatibility isomorphisms of the formal object $\xi$ make this well-defined. 
\end{rmk}

\begin{rmk}
Let $\xi$ be a formal object of $\F$. Versality of $\xi$ is equivalent to: the existence of a dashed arrow for any diagram,
\begin{center}
\begin{tikzcd}
& y \arrow[d]
\\
\xi \arrow[ru, dashed] \arrow[r] & x
\end{tikzcd}
\end{center}
in $\wh{\F}$ such that $y \to x$ lies over a surjective map $B \onto A$ of Artinian rings. 
\end{rmk}

\begin{theorem}[Rim-Schlessinger]
A deformation category $\F$ such that $T \F = \overline{\F}(k[\epsilon])$ is finite dimensional admits a versal formal object.
\end{theorem}

\begin{example}
Let $X$ be a $k$-scheme. The cofibered category of deformations $\cDef_X \to \C_{k}$ is a deformation category. Thus if $T \cDef_X = T \Def_X$ is finite dimensional then $X$ admits a versal formal deformation $\X \to \Spf{R}$.
\end{example}

\begin{defn}
Given a category fibered in groupoids,
\[ p : \X \to (\Sch_S)_{\fppf} \]
and a finite type point $\Spec{k} \to S$ and $x_0 \in \X(k)$. First factor $\Spec{k} \to \Spec{\Lambda} \embed S$ through some affine open such that $\Lambda \to k$ is finite. The category $\C_\Lambda$, up to canonical equivalence, does not depend of the choice of affine open $\Spec{\Lambda} \subset S$. Note that $\C_{\Lambda}$ is equivalent to the opposite category of factorizations,
\[ \Spec{k} \to \Spec{A} \to S \]
such that $A$ is Artin local and $A \to k$ identifies $k$ with the residue field. Now let $\F_{\X, k, x_0}$ be the category of,
\begin{enumerate}
\item morphisms $x_0 \to x$ of $\X$ over $\Spec{k} \to \Spec{A}$ as $S$-map in $\C_{\Lambda}$,
\item morphisms $(x_0 \to x) \to (x_0 \to x')$ are diagrams,
\begin{center}
\begin{tikzcd}
x \arrow[from=rr] & & x' 
\\
& x_0 \arrow[ru] \arrow[ul]
\end{tikzcd}
\end{center}
in $\X$ (notice the reversal of arrows). 
\end{enumerate}
Then $p : \F_{\X, k, x_0} \to \C_{\Lambda}$ is a predeformation category. We say that a formal object $\xi = (R, \xi_n, f_n)$ of $\X$ is \textit{versal} if $\xi$ is versal as a formal object of $\F_{\X, k, x_0}$ with $k = R / \m_R$ and $x_0 = \xi_1$. We say that $x \in \X(U)$ is versal at a finite type point $u_0 \in U$ if $\hat{x} \in \wh{\F}_{\X, \kappa(u_0), x_0}$ is versal where $x_0 : \Spec{k} \to U \to \X$ is the image. 
\end{defn}

\begin{defn}
Let $S$ be a locally noetherian scheme and $p : \X \to (\Sch_S)_{\fppf}$ a category fibered in groupoids. We say $\X$ satisfies \textit{openness of versality} if given a scheme $U$ locally of finite type over $S$, an object $x \in \X(U)$, and a finite type point $u_0 \in U$ such that $x$ is versal at $u_0$ then there is exists an open neighborhood $u_0 \in U' \subset U$ such that $x$ is versal at every finite type point of $U'$.
\end{defn}

\subsection{Effectivity}

\begin{defn}
A formal object $\xi = (R, \xi_n, f_n) \in \wh{\F}_{\X, k, x_0}$ is \textit{effective} if it arises from $\tilde{\xi} \in \X(R)$.
\end{defn}

\begin{lemma}[\chref{https://stacks.math.columbia.edu/tag/07X3}{07X3}]
If $\X \to S$ is an algebraic stack over a locally noetherian scheme $S$ then every formal object is effective.
\end{lemma}

\begin{proof}
First, if $X$ is a scheme then for all local rings $R$ factoring $\Spec{k} \to X$ the map corresponds to $\Spec{R} \to \Spec{\stalk{X}{x}} \to X$ so if $R$ is complete,
\[ X(R) = \Hom{\text{loc}}{\stalk{X}{x}}{R} = \varprojlim_{n} \Hom{\text{loc}}{\stalk{X}{x}}{R/\m_R^n} = \varprojlim_n X(R/\m_R^n) \]
The general case follows from an intricate descent argument.
\end{proof}

\section{Artin's Axioms}

\begin{theorem}[Artin Approximation]
Let $S$ be a locally noetherian scheme and a category fibered in groupoids $p : \X \to (\Sch_S)_{\fppf}$. Let $R$ be a Noetherian complete local ring with residue field $k$ with $\Spec{R} \to S$ finite type and $x \in \X(R)$. Let $s \in S$ be the image of $\Spec{k} \to \Spec{R} \to S$. Assume that,
\begin{enumerate}
\item $\stalk{S}{s}$ is a $G$-ring
\item $p$ is limit-preserving on objects.
\end{enumerate}
Then for every $N \ge 1$ there exist,
\begin{enumerate}
\item a finite type $S$-algebra $A$
\item a maximal ideal $\m_A \subset A$
\item an object $x_A \in \X(A)$
\item an $S$-isomorphism $R / \m_R^N \iso A / \m_A^N$
\item an isomorphism $x|_{R/\m_R^N} \iso x_A |_{A / \m_A^N}$ over the previous map
\item an isomorphism $\gr{\m_R}{R} \iso \gr{\m_A}{A}$ of graded $k$-algebras.
\end{enumerate}
\end{theorem}

\begin{lemma}
Let $S$ be a locally noetherian scheme and $p : \X \to (\Sch_S)_{\fppf}$ a category fibered in groupoids. Let $\xi$ be a formal object of $\X$ with $x_0 = \xi_1$ lying over $\Spec{k} \to S$ with image $s \in S$ such that,
\begin{enumerate}
\item $\xi$ is versal
\item $\xi$ is effective
\item $\stalk{S}{s}$ is a $G$-ring 
\item $p : \X \to (\Sch_S)_{\fppf}$ is limit-preserving
\end{enumerate}
then there exists a finite type morphism $U \to S$, a finite type point $u_0 \in U$ with residue field $k$ and $x \in \X(U)$ such that $x : U \to \X$ is versal at $u_0$ and $x|_{\Spec{\stalk{U}{u_0}}}$ induces $\xi$. 
\end{lemma}

\begin{proof}
Choose an object $x_R \in \X(R)$ whose completion is $\xi$. Apply Artin approximation with $N = 2$ to obtain $A, \m_A, x_A \in \X(A)$ approximating $\xi$. Let $\eta$ be the formal object completing $x_A |_{\Spec{\hat{A}}}$ (the completion of $A$ at $\m_A$). Then a lift for the diagram in $\wh{\F}_{\X, k, x_0}$,
\begin{center}
\begin{tikzcd}
& \eta \arrow[d]
\\
\xi \arrow[r] \arrow[ru, dashed] & \xi_2 = \eta_2 
\end{tikzcd}
\quad \text{lying over} \quad
\begin{tikzcd}
& \hat{A} \arrow[d]
\\
R \arrow[r] \arrow[ru, dashed] & R / \m_R^2 = A / \m_A^2
\end{tikzcd}
\end{center}
exists because $\xi$ is versal. Since the map $R \to \hat{A}$ induces an isomorphism on tangent spaces and by construction $\dim_k \m_R^n / \m_R^{n+1} = \dim_k \m_A^n / \m_A^{n+1}$ we conclude that $R \to \hat{A}$ is an isomorphism. Hence $\eta \cong \xi$ is versal so the map $x_A : \Spec{A} \to \X$ is versal at $\wh{x_A|_{\Spec{\hat{A}}}}  = \eta$. 
\end{proof}

\begin{theorem}
Let $S$ be a locally Noetherian base scheme and consider a category fibered in groupoids $p : \X \to (\Sch_S)_{\fppf}$. For each finite type morphism $\Spec{k} \to S$ with $k$ a field and $x_0 \in \X(\Spec{k})$  assume that,
\begin{enumerate}
\item $\X$ is a stack for the \etale topology
\item $\Delta_{\X / S} : \X \to \X \times_S \X$ is representable by algebraic spaces
\item $\X$ is limit preserving (preserves filtered colimits)
\item $\X$ satisfies the Rim-Schlessinger condition (RS)
\item $T \F_{\X, k, x_0}$ is finite dimensional for all $k$ and all $x_0 \in \F(k)$
\item every formal object of $\X$ is effective
\item $\X$ satisfies openness of versality
\item $\stalk{S}{s}$ is a $G$-ring for all finite type points $s \in S$ 
\item a set theoretic condition
\end{enumerate}
then $\X$ is an algebraic stack. 
\end{theorem}

\begin{proof}
It suffices to show that for each finite type $\Spec{k} \to S$ and $x_0 \in \X(k)$ there is a finite type morphism $U \to S$ and a smooth map $U \to \X$ such that there is a finite type point $u_0 : \Spec{k} \to U$ such that $x|_{u_0} \cong x_0$.
\bigskip\\
By Rim-Schlessinger $\F_{\X, k, x_0}$ admits a versal formal object $\xi$ which is then effective. Artin approximation allows us to approximate an effective formal object by a finite type object $U \to \X$ which is versal at $u_0 \in U$. By openness of versality, we can shrink $U$ such that $U \to \X$ is versal at every finite type point.  
\bigskip\\
Finally, prove that a representable morphism $f : \X \to \Y$ of limit preserving categories fibered in groupoids which is smooth on deformation categories is smooth (Tag \chref{https://stacks.math.columbia.edu/tag/07XX}{07XX}). Indeed, for $T \to \Y$ the condition says that $f : \X_T \to T$ is a formally smooth map of algebraic spaces\footnote{There is a subtly here with changing fields that requires the full strength of (RS) where as proving that a versal object exists only requires (S1) and (S2) and finite-dimensionality of tangent spaces} and the limit-preserving condition gives finitely presented. 
\end{proof}

\begin{rmk}
Usually most difficult to prove openness of versality. There a number of deformation-theoretic techniques for proving this but require effectivity of formal objects over more general formal schemes. There are also tangent-obstruction theory methods for proving openness of versality.
\end{rmk}

\section{References}

\begin{enumerate}
\item The Stacks Project: formal deformation theory and Artin's axioms.

\item Hartshorne: Deformation Theory.

\item \chref{https://math.stanford.edu/~vakil/727/class12.pdf}{Ravi's notes on deformation theory}.

\item \chref{https://mathoverflow.net/questions/37521/a-versal-deformation-of-a-simple-node}{This answer about deforming the node}.

\item \chref{https://lekili.duckdns.org/teaching/topics/Deformation.pdf}{basic notes}

\item \chref{https://cel.hal.science/cel-00392119/document}{Notes on deformation theory including versality of certain deformations of schemes}. 

\item \chref{https://www.msri.org/people/members/defthy07/lectures/brian.pdf}{Brian Osserman, gives good example with Def of a scheme}.
\end{enumerate}
\end{document}