\documentclass[12pt]{article}
\usepackage{import}
\import{../}{AlgGeoCommands}

\begin{document}

\newcommand{\R}{\mathbb{R}}
\newcommand{\Lie}[1]{\mathrm{Lie}\left( #1 \right)}
\newcommand{\der}{\mathrm{der}}
\newcommand{\ad}{\mathrm{ad}}
\renewcommand{\S}{\mathbb{S}}

\section{Flag Varieties}

\begin{definition}
Let $V$ be a vectorspace then the Grassmaniann $G_d(V)$ is the space of $d$-dimensional subspaces of $V$. For any subspace $S \subset V$ of complementary dimension $n - d$ we define,
\[ G_d(V)_S = \{ W \in G_d(V) \mid W \cap S = \{ 0 \} \} \]
Thus, writing $V =  W_0 \oplus S$ for some fixed $W_0$ then projection $W \to W_0$ defines an isomorphism,
\[ G_d(V)_S \xrightarrow{\sim} \Hom{}{W_0}{S} \]
These form an affine open cover of $G_d(V)$ with its variety structure with, 
\[ G_d(V)_S \cong \A(\Hom{}{W_0}{S}) \]
Furthermore this shows that $G_d(V)_S$ is smooth with tangent space,
\[ T_{W_0} G_d(V) = \Hom{}{W_0}{S} = \Hom{}{W_0}{V/W_0} \] 
\end{definition}

\begin{definition}
We extend the above discussion to chain s of subspaces. Let $\bf{d} = (d_1, \dots, d_r)$ be a sequence of integers with $n > d_1 > d_2 > \cdots > d_r > 0$ and let $G_{\bf{d}}(V)$ be the space of flags,
\[ F : \: V \supset V^1 \supset \cdots \supset V^r \supset 0 \]
with $\dim{V^i} = d_i$. The map,
\[ G_{\bf{d}}(V) \xrightarrow{F \mapsto (V^i)} \prod_{i} G_{d_i}(V) \subset \prod_i \P\left( \bigwedge^{d_i} V \right) \]
gives an embedding of $G_{\bf{d}}(V)$ inside $\prod_i G_{d_i}(V)$ showing that $G_{\bf{d}}(V)$ is a projective variety.
\end{definition}

\section{The Hodge Filtration}

Given any Hodge structure of weight $n$ there is a filtration,
\[ F^\bullet : \: F^{-n} \supset F^1 \supset \cdots \supset F^p \supset F^{p+1} \supset \cdots \quad \quad \quad F^p = \bigoplus_{r \ge p} V^{r,s} \subset V_\C \]
Note that when $p + q = n$ we have,
\[ \overline{F^q} = \bigoplus_{s \ge q} \overline{V^{s,r}} = \bigoplus_{S \ge q} V^{r,s} = \bigoplus_{r \le p} V^{r,s} \]
Therefore,
\[ V^{p,q} = F^p \cap \overline{F^q} \]
Recall that Hodge structures correspond to representations of the Deligne torus, $h : \S \to \Gm^\R$ which is a map of $\R$-algebraic groups. Recall that $h$ acts on $V^{p,q}$ on real points $z \in \S(\R) = \C^\times$ via $h(z) \cdot v = z^{-p} \bar{z}^{-q} \cdot v$ for $v \in V^{p,q}$. 

\section{Variations of Hodge Structures}

\section{Shimura Data}

\begin{prop}
Let $\varphi : G \to H$ be a surjective map of algebraic groups over $\R$. Then $\varphi(\R) : G(\R)^+ \to H(\R)^+$ is surjective.
\end{prop}

\begin{proof}
Viewing this as a map of Lie groups, it suffices to show that $\d{\varphi} : \Lie{G} \to \Lie{H}$ is surjective. In that case $\Im{\varphi}$ is open (since it is a local diffeomorphism) and closed since it is an open subgroup. Therefore,  $\Im{\varphi} \supset H(\R)^+$ the connected component of the identity.

(Q? Why is it surjective on tangent spaces?)
\end{proof}

\begin{rmk}
For a reductive group $G$ with center $Z = Z(G)$ and a torus $T$ there is an exact diagram,
\begin{center}
\begin{tikzcd}
& & 1 \arrow[d]
\\
& &G^{\der} \arrow[d] \arrow[dr] 
\\
1 \arrow[r] & Z \arrow[r] \arrow[dr] & G \arrow[r, "\ad"] \arrow[d, "\nu"] & G^\ad \arrow[r] & 1
\\
& & T \arrow[d]
\\
& & 1
\end{tikzcd}
\end{center}
Then $Z \cap G^\der$ is the center of $G^\der$ and $G^\der = \ker{(G \to T)}$ where $T$ is the maximal abelian quotient. (Q? why don't we call $G^\der = G^\ab$ in this case?)
\bigskip\\
This gives an exact sequence,
\begin{center}
\begin{tikzcd}
1 \arrow[r] & Z \cap G^\der \arrow[r] & Z \times G^\der \arrow[r] & G \arrow[r] & 1
\end{tikzcd}
\end{center}
\end{rmk}

\begin{defn}
We define the cohomology $H^1(\Q, G)$ as crossed modulo principal homomorphisms $\Gal{\Q^\ab / \Q} \to G(\Q^\ab)$. 
(Q? Is $\Q^\a = \Q^\ab$? How to make sense of this? Is this like taking an abelian version of the etale site and viewing $G$ as a sheaf on this site and taking cohomology?)
\end{defn}

\begin{prop}
For a reductive group $G$ over $\R$ the Lie group $G(\R)$ has only finitely many connected components. 
\end{prop}

\begin{proof}
Using the above map, an exact sequence of real algebraic groups,
\begin{center}
\begin{tikzcd}
1 \arrow[r] & N \arrow[r] & G' \arrow[r] & G \arrow[r] & 1
\end{tikzcd}
\end{center}
with $N \subset Z(G')$ gives rise to an exact sequence, 
\begin{center}
\begin{tikzcd}
\pi_0(G'(\R)) \arrow[r] & \pi_0(G(\R)) \arrow[r] & H^1(\R, N)
\end{tikzcd}
\end{center}
(Q? I don't understand this?? If it were ordinary cohomology which could be related to $\pi_1$ then I understand but how group cohomology? Are you using the fibration seqeunce?)
\end{proof}

\begin{theorem}[real approximation]
Let $G$ be a connected algebraic group over $\Q$. Then $G(\Q)$ is dense in $G(\R)$.
\end{theorem}

\subsection{The Data}

\begin{definition}
A Shimura datum is a pair $(G, X)$ with $G$ a reductive group over $\Q$ and $X$ a set of $G(\R)$-conjugacy class of homomorphisms $h : \S \to G_\R$ as real algebraic groups. The pair must satisfy the foloowing conditions,
\begin{enumerate}
\item for all $h \in X$ the Hodge structure defined on $\Lie{G_\R}$ via,
\begin{center}
\begin{tikzcd}
\S \arrow[r, "h"] & G_\R \arrow[r, "\mathrm{Ad}"] & \Aut{\Lie{G_\R}} 
\end{tikzcd}
\end{center}
has type $\{(-1,1), (0,0), (1,-1) \}$.
\item for all $h \in X$ the composition $\ad \circ h$ takes $\ad(h(i))$ to the Cartan involution of $G_\R^\ad$ ( i.e. if $-B(X, \ad(h(i)) \cdot Y)$ is a positive-definite bilinear form where $B$ is the killing form)
\item $G^\ad$ has no $\Q$-factor on which the projection of $h$ is trivial.
\end{enumerate}
\end{definition}


\begin{rmk}
The first condition says that $\mathrm{Ad} \circ h : \S \to \Aut{\Lie{G_\R}}$ acts only via the characters $z / \bar{z}, 1, \bar{z}/z$. 
\end{rmk}

\begin{rmk}
Note that in contrast to the connected case, $G$ is reductive (rather than semi-simple), $h$ has target $G_\R$ (rather than $G_\R^\ad$), and $X$ is a full $G(\R)$-conjugacy class (not a connected component).
\end{rmk}

\begin{proposition}
Let $(G, X)$ be a Shimura datum and $X^+$ be a connected component of $X' = \{\ad \circ h : \S \to \G_\R^\ad \mid h \in X \}$ (regarded as a $G(\R)^+$-conjugacy class  of homomorphisms $\S \to G_\R^\ad$). Then $(G^\der, X^+)$ is a connected Shimura datum. In particular, $X$ is a finite disjoint union of hermitian symmetric domains. 
\end{proposition}

\subsection{Shimura Varieties}

\renewcommand{\Sh}[2]{\mathrm{Sh}_{#1} \left( #2 \right)}

\begin{lemma}
Let $(G, X)$ be a Shimura datum, for every connected component $X^+$ of $X$ there is a natural bijection,
\[ G(\Q)_+ \backslash X^+ \times G(\A_f) \xrightarrow{\sim} G(\Q) \backslash X \times G(\A_f) \]
\end{lemma}

\begin{proof}
The map is surjective because $G(\Q)$ is dense in $G(\R)$. 
\end{proof}

\begin{lemma}
For every open subgroup $K \subset G(\A_f)$ the set $G(\Q)_+ \backslash G(\A_f) / K$, the set $G(\Q)_+ \backslash G(\A_f) / K$ is finite.
\end{lemma}

\begin{defn}
Gor a compact open subgroup $K \subset G(\A_f)$, consider the double coset space,
\[ \Sh{K}{G, X} : = G(\Q) \backslash X \times G(\A_f) / K \]
in which $G(\Q)$ acts on $X$ and $G(\A_f)$ on the left, and $K$ acts on $G(\A_f)$ on the right:
\[ q(x, a) k = (qx , q a k) \quad q \in G(\q) \quad x \in X \quad a \in G(\A_f) \quad k \in K \]
\end{defn}

\begin{lemma}
Let $C$ be a set of representatives for the double coset space,
\[ G(\Q)_+ \backslash G(\A_f) / K \]
and let $X^+$ be a commnected component of $X$. Then,
\[ G(\Q) \backslash X \times G(\A_f) / K \cong \bigsqcup_{g \in C} \Gamma_g \backslash X^+ \]
where $\Gamma_g$ is the subgroup $g K g^{-1} \cap G(\Q)_+$ of $G(\Q)_+$. When we endow $X$ with its usual topology and $G(\A_f)$ with its adelic topology this is a homeomorphism. 
\end{lemma}

\begin{rmk}
Because $\Gamma_g$ is a congrence subgroup of $G(\Q)$, its image in $G^\ad(\Q)$ is arithmetic (WHAT?) and so its image in $\Aut{X^+}$ is arithmetic. Moreove, when $K$ is sufficently small, $\Gamma_g$ will be neat for all $g \in C$ and so its image in $\Aut{X^+}^+$ will also be near and hence torsion-free. Then $\Gamma_G \backslash X^+$ is an arithmetic locally symmetric variety, and $\Sh{K}{G, X}$ is a finite disjoint union of such varieties. Moreover, for an inclusion $K' \subset K$ of sufficiently small compact open subgroups of $G(\A_f)$, the natural map $\Sh{K'}{G, X}
 \to \Sh{K}{G, X}$ is regular. Thus, when we vary $K$, we get an inverse system of algebraic vareties $(\Sh{K}{G, X})_K$. There is a natural action of $G(\A_f)$ on the system as follows: for $g \in G(\A_f)$ we send $K \mapsto g^{-1} K g$ which sends compact open subgroups to compact open subgroups then,
\[ T(g) : \Sh{K}{G, X} \to \Sh{g^{-1} K h}{G, X} \]
\end{rmk}

\begin{defn}
Let $(G, X)$ be a Shimura datum. A Shimura variety relative to $(G, X)$ is a variety of the form $\Sh{K}{G, X}$ for some compact open subgroup $K \subset G(\A_f)$.
\bigskip\\
The \textit{Shimura variety} $\Sh{}{G, X}$ attached to the Shimura datum $(G, X)$ is the inverse system of varieties $(\Sh{K}{G, X})_K$ endowed with the $G(\A_f)$ action.  
\end{defn}

\subsection{Morphisms of Shimura Varieties}

\begin{defn}
Let $(G, X)$ and $(G', X')$ be Shimura data, 
\begin{enumerate}
\item a morphism of Shimura data $(G, X) \to (G', X)$ is a homomorphism $G \to G'$ of algebraic groups sending $X \onto X'$
\item A morphism of Shimura varieties $\Sh{}{G, X} \to \Sh{}{G', X'}$ is an inverse system of regular maps of algebraic varieties equivariant with respect to the $G(\A_f)$-action.
\end{enumerate}
\end{defn}

\begin{thm}
A morphism of Shimura data $(G, X) \to (G', X')$ defines a morphism $\Sh{}{G, X} \to \Sh{}{G', X'}$ of Shimura varieties which is a closed immersion if $G \to G'$ is injection.
\end{thm}

\begin{rmk}
What does it mean here to be a closed immersion? Does this mean that each map in the sequence is a closed immersion? Okay, it means that for sufficiently small $K$ it becomes a closed immersion. 
\end{rmk}

\subsection{The Structure of a Shimura Variety}


\section{Symplectic Space}

\begin{defn}
Let $k$ be a field ($\mathrm{char}(k) \neq 2$) then a \textit{symplectic space} $(V, \psi)$ is a $2n$-dimensional $k$-vectorspace $V$ and a nondegenerate alternative $2$-form $\psi$.
\bigskip\\
We say that a subspace $W \subset V$ is totally isotropic if $\psi(W, W) = 0$. 
\bigskip\\
A symplectic basis of $V$ is a basis $\{ e_{\pm i} \}$ such that,
\begin{align*}
\psi(e_{+i}, e_{-i}) &= 1 \quad \text{for} 1 \le i \le n
\\
\psi(e_i, e_j) &= 0 \quad \text{for} j \neq \pm i
\end{align*} 
\end{defn}

\newcommand{\Gsp}[1]{\mathrm{GSp}\left( #1 \right)}
\newcommand{\GSp}[1]{\mathrm{GSp}\left( #1 \right)}
\newcommand{\Sp}[1]{\mathrm{Sp}\left( #1 \right)}
\newcommand{\fin}{\mathrm{fin}}
\renewcommand{\C}{\mathbb{C}}

\begin{defn}
For a symplectic space $(V, \psi)$ we define $\Gsp{\psi}$ to be endomorphisms of $V$ which preserve $\psi$ up to scaling,
\[ \Gsp{\psi}(k) = \{ g \in \GL{n}{V} \mid \psi(gu, gv) = v(g) \cdot \phi(u, v) \quad v(g) \in k^\times \} \]
\end{defn}

\begin{rmk}
Here $\Gsp{\psi}$ is a sort of conformal version of the symplectic group.
\end{rmk}

\begin{rmk}
There is a homomorphism $v : \Gsp{\psi} \to \Gm$ sending $g \mapsto v(g)$. Then clearly $\ker{\nu} = \Sp{\psi}$. Furthermore, $\Sp{\psi} \embed \GSp{\psi}$ is the derived subgroup giving a diagram,
\begin{center}
\begin{tikzcd}
& \Sp{\psi} \arrow[d] \arrow[rd]
\\
\Gm \arrow[rd] \arrow[r] & \Gsp{\psi} \arrow[d, "\nu"] \arrow[r, "\ad"] & \GSp{\psi}^\ad 
\\
& \Gm
\end{tikzcd}
\end{center}
\end{rmk}

\subsection{Shimura Datum Attached to a Symplectic Space}

Let $(V, \psi)$ be a symplectic space over $\Q$ and let $G(\psi) = \GSp{\psi}$ and $S(\psi) = \Sp{\psi} = G^\der$. 
\bigskip\\
Now let $J$ be a complex structure on $V(\R)$ such that $\psi(Ju, Jv) = \psi(u,v)$. Then clearly, $J \in S(\psi)(\R)$. For each $z \in \C^\times$ we get $h_J(z) \in G(\psi)(\R)$ and it lies in $S(\psi)(\R)$ if $|z| = 1$ (UNDERSTAND THIS?). We say that $J$ is \textit{positive} if $\psi_J(u, v) := \psi(u, Jv)$ is positive-definite. 
\bigskip\\
Let $X^{\pm}$ denote the set of positive and respectively negatively complex constructures on $V(\R)$ such that $\psi(J u, J v) = \psi(u, v)$ for all $u, v \in V$ and let $X(\psi) = X^{+} \sqcup X^{-}$. Then $G(\psi)(\R)$ acts on $X$ according to,
\[ (g, J) \mapsto g J g^{-1} \]
and the stabilizer in $G(\psi)(\R)$ of $X^{+}$ is,
\[ G(\psi)(\R)^+ = \{ g \in G(\psi)(\R) \mid v(g) > 0 \} \]

\begin{prop}
Then the pair $(G(\psi), X(\psi))$ satisfies the axioms of the Shimura datum and furthermore satisfies the futher additional axioms,
\begin{enumerate}
\item[SV2'] for all $h \in X$ then $\ad(h(i))$ is a Cartan involution on $G_\R / w_X(\Gm)$ (not just $G_\R^\ad$)
\item the weight map $w_X : \Gm \to G_\R$ is defined over $\Q$
\item the group $Z(\Q) \embed Z(\A_{\fin})$ is discrete
\item the torus $Z^\circ$ splits over a CM-field 
\end{enumerate}
where the weight homomorphism $w_X : \Gm{\R} \to Z(G)_\R^\circ \subset G_\R$ is a map over $\R$ of $\Q$-tori.
\end{prop}

\begin{rmk}
We call $(G(\psi), X(\psi))$ the Siegel Shimura datum associated to a symmetric space $(V, \psi)$. 
\end{rmk}

\subsection{The Siegel Modular Variety}

Let $(G, X) = (G(\psi), X(\psi))$ be the Shimura datum defined by a symplectic space $(V, \psi)$ over $\Q$. Then the Siegal modular variety of $(V, \psi)$ is simply $\Sh{}{G, X}$.
\bigskip\\
Now let $V(\A_{\fin}) = V \otimes_\Q \A_{\fin}$ thne $G(\A_{\fin})$ is the group of $\A_{\fin}$-linear automorphisms of $V(\A_{\fin})$ perserving $\psi$ up to multiplication by an element of $\A_{\fin}^\times$.
\bigskip\\
Let $K \subset G(\A_{\fin})$ be a compact open subgroup of $G(\A_{\fin})$ and let $\H_K$ denote the set of triples $((W, h), s, \eta K)$ where,
\begin{enumerate}
\item $(W, h)$ is a rational Hodge structure of type $(-1,0)$ and $(0, -1)$
\item one of $s$ or $-s$ is a polarization for $(W, h)$
\item $\eta K$ is a $K$-orbit of $\A_{\fin}$-linear isomorphism $V(\A_{\fin}) \to W(\A_{\fin})$ under which $\psi$ corresponds to $s$ up to $\A_{\fin}^\times$-scaling.
\end{enumerate}
Furthermore, an isomorphism $b : ((W, h), s, \eta K) \to ((W', h'), s', \eta'K)$ of triples is an isomorphism $b : (W, h) \to (W', h')$ of rational Hodge structures sending $s$ to $cs'$ for $c \in \Q^\times$ and $b \circ \eta = \eta'$ modulo $K$.

\section{Shimura Varieties of Hodge Type}

\begin{rmk}
For a symplectic space $(V, \psi)$ denote the associated Shimura data by $(G(\psi), X(\psi))$.
\end{rmk}

\begin{defn}
A Shimura datum $(G, X)$ is of \textit{Hodge type} if there is a symplective space $(V, \phi)$ over $\Q$ and an injection $\rho : G \embed G(\psi)$ sending $X$ into $X(\psi)$. The Shimura variety $\Sh{}{G, K}$ is then of \textit{Hodge type}. 
\end{defn}

\begin{rmk}
The embedding $\rho : G \embed G(\psi)$ composes with $\nu : G(\psi) \to \Gm$ to give a character $\nu : G \to \Gm$ of $G$.
\end{rmk}

\begin{prop}
Let $\Q(r)$ be the Hodge struture on $\Q$ with $G$ acting by $g \cdot x = \nu(g)^r \cdot g$. Then for each $h \in X$ we see that $(\Q(r), \nu \circ h)$ is a rational Hodge structure of type $(-r,-r)$ so this agrees with previous notation.
\end{prop}

\begin{prop}
There exist multilinear maps $t_1, \dots, t_n : V \times \cdots \times V \to \Q(r_i)$ such that $G$ is the subgroup of $G(\psi)$ fixing these maps.
\end{prop}

\begin{proof}
This is an application of Chevalley's theorem since this is equivalent to finding a set of tensors which give $G$ as the group fixing these tesnors. 
\end{proof}


\begin{rmk}
Let $(G, X)$ be of Hodge type an schoose an embedding $(G, X) \embed (G(\psi), X(\psi))$ for some symplectic space $(V, \psi)$ and  multilinear maps $t_1, \dots, t_n : V \times \cdots \times V \to \Q(r_i)$ with $G$ their stabilizing subgroup. 
\bigskip\\
We now define a set $\H_K$ of triples, $((W, h), (s_i), \eta K)$ as before satisfying,
\begin{enumerate}
\item $(W, h)$ is a rational Hodge structure of type $(-1,0), (0, -1)$
\item $s_0$ or $-s_0$ is a polarization for $(W, h)$
\item $s_1, \dots, s_n$ are multilinear maps $s_i : W \times \cdots \times W \to \Q(r_i)$
\item $\eta K$ is a $K$-orbit of isomorphisms $V(\A_{\fin}) \to W(\A_{\fin})$ under which $\psi$ is $s_0$ up to $\A_{\fin}^\times$-scaling and $t_i$ corresponds to $s_i$.
\end{enumerate}
and further require that, there is an isomorphism $a : W \to V$ sending $s_0$ to $\psi$ up to $\Q^\times$-scaling and $s_i$ corresponds to $t_i$ and $h$ to an element of $X$. 
\end{rmk}

\begin{prop}
The complex points $\Sh{K}{\mathbb{\C}}$ classify the elements of $\H_K$ up to isomorphism. 
\end{prop}

\begin{defn}
Now let $A$ be an abelian variety over $\C$ and $W = H_1(A, \Q)$. Then we have seen that,
\[ H^m(A, \Q) = \Hom{}{\bigwedge^m W}{\Q} \]
We say that $t \in H^{2r}(A, \Q)$ is a \textit{Hodge tensor for} $A$ if the corresponding map,
\[ W^{\otimes 2r} \to \bigwedge^{2r} W \to \Q(r) \]
is a morphism of Hodge structures.
\end{defn}

\begin{rmk}
Now we define a similar set of tuples whose moduli we can describe.
\end{rmk}

\begin{defn}
Let $(G, X) \embed (G(\psi), X(\psi))$ and $t_1, \dots, t_n$ as above. Then let $M_K$ be the following set of triples $(A, (s_i), \eta K)$ where,
\begin{enumerate}
\item $A$ is an abelian variety over $\C$
\item $s_0$ or $-s_0$ is a polarization for the rational Hodge structure $H_1(A, \Q)$
\item $s_1, \dots, s_n$ are Hodge tensors for $A$ (or its powers WHAT DOES THAT MEAN?)
\item $\eta K$ is a $K$-orbit of $\A_{\fin}$-linear isomorphisms $V(\A_{\fin}) \to V_f(A)$ (WHAT IS THIS) sending $\psi$ onto $s_0$ up to $\A_{\fin}^\times$-scaling and each $t_i$ to $s_i$
\end{enumerate}
which further satisfies the condition:
\begin{center}
there exists an isomorphisms $a : H_1(A, \Q) \to V$ sending $s_0$ to $\psi$ up to $\Q^\times$-scaling and $s_i$ to $t_i$ and $h$ to an element of $X$.
\end{center}
\end{defn}

\begin{theorem}
The complex points $\Sh{K}{\mathbb{C}}$ classify the elements of $M_K$ up to isomorphism. 
\end{theorem}

\begin{rmk}
Let $A(\C) = \C^g / \Lambda$ then,
\[ H^m(A, \Q) = \Hom{}{\bigwedge^m \Lambda}{\Q} \]
and for $T = T_0 A$ we have $\Lambda \otimes \C = T \oplus \bar{T}$. Therefore,
\[ H^m(A, \C) = \Hom{}{\bigwedge^m (\Lambda \otimes \C)}{\C} = \Hom{}{\bigoplus_{p + q = m} \bigwedge^p T \otimes \bigwedge^q \bar{T}}{\C} = \bigoplus_{p + q = m} H^{p, q} \]
where we identify,
\[ H^{p,q} = \Hom{}{\bigwedge^p T \otimes \bigwedge^q \bar{T}}{\C} \]
This construction of the Hodge structure on $H^m$ does agree with the usual abstract construction from Hodge theory. 
\end{rmk}

\begin{prop}
A Hodge tensor on $A$ is an element of,
\[ H^{2r}(A, \Q) \cap H^{r, r} \subset H^{2r}(A, \C) \]
Recall that the Hodge conjecture predicts the Hodge tensors are spaned over $\Q$ by the algebraic classes. We can show this here for $r = 1$. Consider the exponential sequence,
\begin{center}
\begin{tikzcd}
0 \arrow[r] & \underline{\Z} \arrow[r] & \struct{A} \arrow[r, "\exp"] & \struct{A}^\times \arrow[r] & 0 
\end{tikzcd}
\end{center}
which on cohomology gives,
\begin{center}
\begin{tikzcd}
H^1(A, \struct{A}^\times) \arrow[d, equals] \arrow[r] & H^2(A, \Z) \arrow[r] & H^2(A, \struct{A}) 
\\
\Pic{A} \arrow[ru, "c_1", dashed]
\end{tikzcd}
\end{center}
Note that $\alpha \in H^2(A, \Z)$ maps to zero in $H^2(A, \struct{A}) = H^{0,2}$ iff it maps to zero in $H^{2,0}$. Therefore,
\[ \Im{c_1} = H^2(A, \Z) \cap H^{1,1} \]
thus we see that the space $H^{1,1}$ in \textit{integral} (even better than rational) cohomology is exactly the subgroup of classes of algbraic line bundles or equivalently of divisors i.e. algebraic cylces.
\end{prop}

\begin{rmk}
Since degree zero bundles map to zero, the first map $\Pic{A} \to H^2(A, \Z)$ factos through $\Pic{A} / \mathrm{Pic}^0(A) = NS(A)$ the Neron-Severi group of connected components of $\Pic{A}$. Furthermore, from before we have,
\[ H^{2}(A, \Z) = \Hom{}{\bigwedge^2 H_1(A, \Z)}{\Z} \]
Then we get an injection,
\[ NS(A) \embed H^2(A, \Z) = \Hom{}{\bigwedge^2 H_1(A, \Z)}{\Z} \]
Note that a \textit{polarization} of $A$ is an element of $NS(A)$ mapping to a polarization of $H_1(A, \Z)$. (LOOK AT PREVIOUS DEF OF POLARIZATION AND COMPARE!!)
\end{rmk}

\end{document}