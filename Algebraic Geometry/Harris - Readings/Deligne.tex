\documentclass[12pt]{article}
\usepackage{import}
\import{../}{AlgGeoCommands}

\begin{document}

\section{Cohomology Review}

\newcommand{\CP}{\mathbb{CP}}
\renewcommand{\C}{\mathbb{C}}
\newcommand{\Hyper}{\mathbb{H}}


\begin{definition}
Let $X$ be a smooth complete variety over $\C$ (a smooth proper scheme over $\C$). There is a corresponding analytic manifold $X^{\an}$ whose exact topology depends on the structure map $X \to \Spec{\C}$. This gives us access to topological cohomology denoted $H_B^n(X) = H^{n}(X^\an, \Q)$. 
\end{definition}

\begin{definition}
For each embedding $\sigma : k_0 \to \C$ there is a corresponding $X^\sigma = X \times_\sigma \Spec{\C}$ and we write $H^p_\sigma(X) = H^p_B(X^\sigma) = H^p((X^\sigma)^\an, \Q)$. 
\end{definition}

\begin{remark}
In the case that $X$ is projective, a projective embedding $X \to \P^n$ defines an embedding $X^\an \to \CP^n$ which pulls back the canonical Kahler form on $\CP^n$ to give $X$ a Kahler structure. By Hodge theory, this gives a decomposition,
\[ H_B^n(X, \C) = H_{\dR}^n(X^\an) = \bigoplus_{p + q = n} H^{p,q}(X) \]
where $H^{p,q}(X)$ can be identified with a complex form of type $(p,q)$ and also with the sheaf cohomology,
\[ H^{p,q}(X) = H^p(X, \Omega^q) \]
\end{remark}

\begin{definition}
The algebraic deRham cohomology is given by the hyper cohomology of the deRham complex,
\[ H^n_\dR(X / k) = \Hyper^n(X, \Omega^\bullet) \]
\end{definition}

\begin{theorem}
There is a Hodge-to-deRham spectral sequence,
\[ E^{p,q}_1 = H^p(X, \Omega^q) \implies \Hyper^{p+q}(X, \Omega^\bullet) = H^{p + q}_{\dR}(X) \]
which gives a filtration on the algebraic deRham cohomology. Furthermore, the continuous map $X \to X^\an$ induces an isomorphism,
\[ H^n_\dR(X) \xrightarrow{\sim} H^n_{\dR}(X^\an) \]
which sends the filtration of the Hodge-to-deRham spectral sequence to the filtration of $H^n_{\dR}(X^\an)$ given by Hodge theory. 
\end{theorem}


\begin{remark}
In general, let $F : \mathcal{A} \to \mathcal{B}$ be an additive functor and $\Ch{\A}$ its category of complexes. Then there is a spectral sequence computing the hyperderived functor,
\[ E_1^{p,q} = R^q F(C^p) \implies \R^{p+q} F(C^\bullet) = \Hyper^{p + q}(C^\bullet) \] 
\end{remark}

\begin{proposition}
Consider a resolution (exact sequence) in an abelian category $\mathcal{A}$
\begin{center}
\begin{tikzcd}
0 \arrow[r] & A \arrow[r] & C^0 \arrow[r] & C^1 \arrow[r] & C^2 \arrow[r] & \cdots
\end{tikzcd}
\end{center}
and an additive functor $F : \mathcal{A} \to \mathcal{B}$. Then, the derived functors of $F$ on $A$ agree with the hyperderived functors of $F$ on $C^\bullet$,
\[ R^p F(A) = \R^p F(C^\bullet) \]
In pariticular, in the category of sheaves on $X$, given any resolution $\F \to \G^\bullet$ we have,
\[ H^p(X, \F) = \Hyper^p(X, \G^\bullet) \]
\end{proposition}

\begin{proof}
We choose a resolution of $C^\bullet$ which is an complex of injectives $I^\bullet$ and a quasi-isomorphism $\alpha : C^\bullet \to I^\bullet$. Consider the diagram,
\begin{center}
\begin{tikzcd}
& A \arrow[d, "\varepsilon"] 
\\
0 \arrow[r] & C^0 \arrow[d, "\alpha_0"] \arrow[r] & C^1 \arrow[r] \arrow[d, "\alpha_1"]  & C^2 \arrow[r] \arrow[d, "\alpha_2"] & \cdots
\\

0 \arrow[r] & I^0 \arrow[r] & I^1 \arrow[r] & I^2 \arrow[r] & \cdots
\end{tikzcd}
\end{center}
Since $A \xrightarrow{\varepsilon} C^\bullet$ is a resolution, the top row is exact except in degree zero where $\ker{(C^0 \to C^1)} = A$. Since $\alpha : C^\bullet \to I^\bullet$ is a quasi-isomorphism the complex $I^\bullet$ must also be exact in positive degree and at degree zero $\alpha_* : H^0(C^\bullet) \xrightarrow{\sim} H^0(I^\bullet)$ is an isomorphism so $\alpha_0 \circ \varepsilon : A \to \ker{(C^0 \to C^1)} \to \ker{(I^0 \to I^1)}$ is an isomorphism. Thus the complex $0 \to A \xrightarrow{\alpha_0 \circ \varepsilon} I^\bullet$ is exact so it is an injective resolution of $A$. Therefore,
\[ R^p F(A) = H^p(F(I^\bullet)) = \R^p F(C^\bullet) \] 
\end{proof}

\begin{remark}
When the resolution $A \to C^\bullet$ is acyclic then, applying the spectral sequence,
\[ E_1^{p,q} = R^q F(C^p) \implies \R^{p + q} F(C^\bullet) \]
we see that $E_1^{p,0} = F(C^p)$ and all others are zero. Thus, $E_2^{p,0} = H^p(F(C))$ so the spectral sequence converges giving,
\[ \R^{p} F(C^\bullet) = H^p(F(C^\bullet)) \]
Together with the previous proposition we conclude,
\[ R^p F(A) = H^p(F(C^\bullet)) \]
that we can compute derived functors on any acylcic resolution.
\end{remark}

\begin{remark}
Applying these remarks to the case of a complex manifold $X$, we consider the resolution of the constant sheaf $\underline{\C}_X$ by the holomorphic differential forms $\Omega^k_X$,
\begin{center}
\begin{tikzcd}
0 \arrow[r] & \underline{\C}_X \arrow[r] & \Omega^1_X \arrow[r] & \Omega^2_X \arrow[r] & \cdots
\end{tikzcd}
\end{center}
This complex is exact by the Poincare lemma. Thus we have an isomorphism,
\[ H^p_{\text{sing}.}(X; \C) = H^p(X, \underline{\C}_X) \xrightarrow{\sim} \Hyper^p(X, \Omega^\bullet_X) = H^p_{\dR}(X) \]
\end{remark}

\newcommand{\Afin}{\mathbb{A}_{\Q, \text{fin}.}}

\begin{definition}
When $k = \bar{k}$ we write the Etale cohomology as,
\[ H^n(X, \Afin) = \varprojlim H^n_{\text{et}}(X_{\text{et}}, \Z / m \Z) \]
\end{definition}

\begin{theorem}
For $k = \C$ there is a canonical isomorpihsm,
\[ H^n_B(X) \otimes \Afin \to H^n_{\text{et}}(X) \]
Therefore $H^n_B(X) \otimes \Afin$ is independent of the choice of structure map $X \to \Spec{\C}$. 
\end{theorem}

\newcommand{\Hdg}{\mathrm{Hdg}}
\newcommand{\cl}[1]{\mathrm{cl} \left( #1 \right)}

\begin{remark}
Recall that we have defined an algebraic cycle via the cohomology class of a smooth subvariety $Z \subset X$ of codimension $p$,
\[ \cl{Z} \in \Hdg^p(X) = H^{2p}_B(X) \cap H^{p}(X, \Omega^p) \]
We give an alternative definition in terms of Chern classes.
\end{remark}

\begin{definition}
First, we define a Chern class $c_1 : \Pic{X} \to H^2_{\dR}(X)$ via the following. Consider the map $\d{\log} : \struct{X}^\times \to \Omega^1_X$ which takes $f \mapsto f^{-1} \d{f}$. Then there is a map of complexes,
\begin{center}
\begin{tikzcd}
0 \arrow[r] \arrow[d] & 0 \arrow[r] \arrow[d] & \struct{X}^\times \arrow[d," \d{\log}"] \arrow[r] & 0 \arrow[r] \arrow[d] & \cdots
\\
0 \arrow[r] & \struct{X} \arrow[r, "\d{}"] & \Omega_X^1 \arrow[r, "\d{}"] & \Omega_X^2 \arrow[r] & \cdots
\end{tikzcd}
\end{center}
Which gives a map on hypercohomology,
\[ H^{n-1}(X, \struct{X}^\times) = \Hyper^n(X, 0 \to \struct{X}^\times \to 0 \to \cdots) \to \Hyper^n(X, \Omega^\bullet_X) = H^n_{\dR}(X) \]
Recall that $\Pic{X} = H^1(X, \struct{X}^\times)$ and therefore we have a map,
\[ c_1 : \Pic{X} \to H^2_{\dR}(X) \]
Then, note that we may extend this to $c_p : \Pic{X} \to H^{2p}_{\dR}(X)$ via splitting.
\end{definition}

\begin{definition}
For any smooth codimension $p$ subvariety $Z \subset X$ we can define,
\[ \cl{Z} = \frac{1}{(p - 1)!} c_p(\iota_* \struct{Z}) \]
To make this definition make any sense, we need to note that the Chern class is defined on the Grothendieck group of $X$ which, when $X$ is smooth is equivalent to the Grothendieck group of the category of coherent $\struct{X}$-modules. This correspondence defines $c_p(\iota_* \struct{Z})$ when $\iota_* \struct{Z}$ is not a vector bundle only a coherent sheaf. 
\end{definition}

\subsection{Basic Properties of Absolutly Hodge Cycles}

\newcommand{\E}{\mathcal{E}}

\begin{remark}
We first need to discuss algebraic conntections on bundles. The setup is $k_0$ is a field of characteristic zero and $S$ is a smooth $k_0$-scheme.  
\end{remark}

\begin{definition}
A $k_0$-connection on a coherent $\struct{S}$-nodule $\mathcal{E}$ is a morphism of sheaves of $k_0$-modules,
\[ \nabla : \mathcal{E} \to \Omega^1_{S} \otimes_{\struct{S}} \mathcal{E} \]
(not as $\struct{S}$-modules) which further satisfies the Leibniz rule, for $f \in \struct{S}(U)$ and $s \in \mathcal{E}(U)$,
\[ \nabla (f s) = \d{f} \otimes e + f \nabla (e) \]
where $\d{} : \struct{S} \to \Omega^1_S$ is the canonical map. We define the subsheaf of horizontal sections, $\mathcal{E}^\nabla = \ker{\nabla}$
\end{definition}

\begin{remark}
Any connection may be extended to $\mathcal{E}$-valued $k$-forms,
\[ \nabla_k : \Omega_S^k \otimes_{\struct{S}} \mathcal{E} \to \Omega^{k + 1}_S \otimes_{\struct{S}} \mathcal{E} \]
via,
\[ \nabla_k (\omega \otimes e) = \d{\omega} \otimes e + (-1)^k \omega \wedge \nabla e \]
\end{remark}

\begin{definition}
The connection $\nabla$ defines a corresponding curvature form,
\[ \omega_\nabla = \nabla_1 \circ \nabla : \E \to \Omega^2_S \otimes_{\struct{S}} \E \]
We say that $\nabla$ is flat or integrable if the curvature vanishes $\omega_\nabla = \nabla_1 \circ \nabla = 0$. 
\end{definition}

\begin{lemma}
The curvature $\omega_\nabla : \E \to \Omega^2_S \otimes_{\struct{S}} \E$ is a $\struct{S}$-module map.
\end{lemma}

\begin{proof}
Consider,
\begin{align*}
\omega_\nabla(f s) & = \nabla_1 (\d{f} \otimes s + f \nabla s) = \d{\d{f}} \otimes s - \d{f} \wedge \nabla s + \d{f} \wedge \nabla s + f \nabla_1 \circ \nabla 
\\
& = f \nabla_1 \circ \nabla s = f \omega_\nabla (s) 
\end{align*} 
\end{proof}

\begin{remark}
If we write locally,
\[ \nabla e = \sum_i f_i \d{g_i} \otimes s_i \]
then the curvature takes the form,
\[ \omega_\nabla (e) = \sum_{i} (\d{f_i} \wedge \d{g_i} \otimes e - f_i \d{g_i} \otimes \nabla s_i) \]
\end{remark}

\begin{proposition}
$\nabla$ is flat iff the $\struct{S}$-map $Q : \mathrm{Der}(\struct{S}, \struct{S}) \to \End{\mathcal{E}}$ given by sending $D$ to,
\[ \mathcal{E} \xrightarrow{\nabla} \Omega^1_S \otimes_{\struct{S}} \mathcal{E} \xrightarrow{D \otimes \id} \struct{S} \otimes_{\struct{S}} \mathcal{E} \xrightarrow{a \otimes b \mapsto ab} \mathcal{E} \]
is a morphism of Lie algebras.
\end{proposition}

\begin{remark}
Note that $Q(D)$ is in fact a $\struct{S}$-morphism using the universal property,
\[ \mathrm{Der}(\struct{S}, \struct{S}) \cong \Homover{\struct{S}}{\Omega^1_S}{\struct{S}} \]
\end{remark}

\begin{proof}
We need to check that $Q[D_1, D_2] = [Q(D_1), Q(D_2)]$ is equivalent to $\nabla_1 \circ \nabla = 0$. Now,
\[ [D_1, D_2] \in \Homover{\struct{S}}{\Omega^1_S}{\struct{S}} \]
is the unique $\struct{S}$-map such that,
\[ [D_1, D_2] \circ \d = D_1 \circ \d \circ D_2 \circ \d - D_2 \circ \d \circ D_1 \circ \d \]
Now consider this action locally,
\begin{align*}
[D_1, D_2] \otimes \id \circ \nabla = \sum_i f_i (D_1 \circ \d \circ D_2 \circ \d - D_2 \circ \d \circ D_1 \circ \d)(g_i) \cdot s_i 
\end{align*}
Furthermore, 
\[ [Q(D_1), Q(D_2)] = (D_1 \otimes \id) \circ \nabla \circ (D_2 \otimes \id) \circ \nabla - (D_2 \otimes \id) \circ \nabla \circ (D_1 \otimes \id) \circ \nabla \]
Again consider its local action,
\begin{align*}
Q(D_1) \circ  Q(D_2)(e) & = (D_1 \otimes \id) \circ \nabla \left( \sum_i f_i D_2(\d{g_i}) \cdot s_i \right) 
\\
& = \sum_i \Big( [D_2(\d{g_i}) D_1(\d{f_i}) + f_i D_1(\d{(D_2(\d{g_i}))}) ] \cdot s_i + f_i D_2(\d{g_i}) D_1( \nabla s_i ) \Big)
\end{align*}
Now consider,
\begin{align*}
\Big[ Q(D_1) \circ Q(D_2) & - Q(D_2) \circ Q(D_1)] - Q([D_1, D_2]) \Big](e)
\\
& = \sum_i \Big( D_1(\d{f_i}) D_2(\d{g_i}) - D_2(\d{f_i}) D_1(\d{g_i})  \Big) \cdot s_i
\\
& + \sum_i f_i \Big(D_1(\d{(D_2(\d{g_i}))}) - D_2(\d{(D_1(\d{g_i}))}) \Big) \cdot s_i
\\
& +  \sum_i  \Big( f_i D_2(\d{g_i}) D_1(\nabla s_i) - g_i D_1(\d{g_i}) D_2(\nabla s_i) \Big)
\\
& - \sum_i f_i (D_1 \circ \d \circ D_2 \circ \d - D_2 \circ \d \circ D_1 \circ \d)(g_i) \cdot s_i
\\
& = \sum_i \Big( D_1(\d{f_i}) D_2(\d{g_i}) - D_2(\d{f_i}) D_1(\d{g_i})  \Big) \cdot s_i 
\\
& +  \sum_i  \Big( f_i D_2(\d{g_i}) D_1(\nabla s_i) - g_i D_1(\d{g_i}) D_2(\nabla s_i) \Big)
\\
& = (D_1 \otimes D_2 - D_2 \otimes D_1) \otimes \id_{\E} \circ \omega_\nabla 
\end{align*}
which is defined on $(\Omega^1_S)^{\otimes 2} \otimes_{\struct{S}} \E$ but descends to $\Omega^2_S \otimes_{\struct{S}} \E$ since it sends the ideal $\omega \otimes \omega \mapsto 0$. Therefore, we see that $Q$ is a Lie algebra map iff
\[ \forall D_1, D_2 \in \Homover{\struct{S}}{\Omega^1_S}{\struct{S}} : (D_1 \otimes D_2 - D_2 \otimes D_1) \otimes \id_{\E} \circ \omega_\nabla = 0 \]
in particular when $\omega_\nabla = 0$. Furthermore when $Q$ is a Lie algebra map then we must have $\omega_\nabla = 0$ since, for any fixed form, there exists sections of $\Omega^1_S$ which do not kill it. 
\end{proof}

\begin{example}
For $\E = \struct{S}$ we have the universal connection $\d : \struct{S} \to \Omega^1_S$. Then the statment that $\d$ is flat is equivalent to $\d^2 = 0$. 
\end{example}

\begin{remark}
Recall that given $f : X \to S$ there is an exact sequence of $\struct{X}$-modules,
\begin{center}
\begin{tikzcd}
f^* \Omega^1_S \arrow[r] & \Omega^1_X \arrow[r] & \Omega^1_{X / S} \arrow[r] & 0
\end{tikzcd}
\end{center}
We may define,
\[ \Omega^k_{X / S} = \bigwedge^k \Omega^1_{X / S} \]
to give $\Omega^\bullet_{X / S}$, the relative deRham complex of $X$ over $S$,
\begin{center}
\begin{tikzcd}
0 \arrow[r] & \struct{X} \arrow[r, "\d{}"] & \Omega^1_{X / S} \arrow[r, "\d{}"] & \Omega^2_{X / S} \arrow[r] & \cdots 
\end{tikzcd}
\end{center}
\end{remark}

\begin{definition}
Now consider a proper smooth morphism $\pi : X \to S$ of smooth varieties. We define its sheaf of relative deRham cohomology by the hyperderived functors applied to the relative de Rham complex,
\[ \H^n_\dR(X / S) = \R^n \pi_*(\Omega^\bullet_{X / S}) \]
\end{definition}

\begin{remark}
Note that for the structure map $\pi : X \to \Spec{k_0}$ map we have $\pi_* \F = \Gamma(X, \F)$ and thus its hyperderived functors are simply hypercohomology of sheaves so,
\[ \H_{\dR}^n(X / k_0) = \Hyper^n(\Omega^\bullet_{S/ k_0}) = H^n_{\dR}(X / k_0) \]
recovering algebraic de Rham cohomology. 
\end{remark}


\begin{definition}
Let $S$ and $\pi : X \to S$ be smooth. Then there is a decreasing filtration,
\[ F^p \Omega_X^q = \bigoplus_{p \ge p'} \Im{( \pi^* \Omega_S^{p'} \otimes_{\struct{X}} \Omega^{q-p'}_X \to \Omega^q_X)} \]
There is always an exact sequence of sheaves of $k_0$-modules,
\begin{center}
\begin{tikzcd}
0 \arrow[r] & F^1 / F^2 \arrow[r] & F^0 / F^2 \arrow[r] & F^0 / F^1 \arrow[r] & 0
\end{tikzcd}
\end{center}
which, in this case, gives an exact sequence of complexes,
\begin{center}
\begin{tikzcd}
0 \arrow[r] & \Omega^{\bullet - 1}_{X/S} \otimes_{\struct{X}} \pi^* \Omega_S^1 \arrow[r] & \Omega^\bullet_X / F^2 \Omega_X^\bullet \arrow[r] & \Omega^\bullet_{X / S} \arrow[r] & 0 
\end{tikzcd}
\end{center}
The associated long exact sequence of hypercohomolgy,  
\begin{center}
\begin{tikzcd}
\R^n \pi_* (\Omega^{\bullet - 1}_{X / S}) \otimes_{\struct{S}} \Omega^1_S \arrow[r] \arrow[d, equals] & \R^n \pi_* (\Omega^\bullet_X / F^2 \Omega^\bullet_X) \arrow[r] & \R^n \pi_* (\Omega^\bullet_{X / S}) \arrow[r, "\nabla"] & \R^{n + 1} \pi_* (\Omega^{\bullet - 1}_{X / S}) \otimes_{\struct{S}} \Omega^1_S \arrow[d, equals]
\\
\R^{n-1} \pi_* (\Omega^{\bullet}_{X / S}) \otimes_{\struct{S}} \Omega^1_S & & & \R^{n} \pi_* (\Omega^{\bullet}_{X / S}) \otimes_{\struct{S}} \Omega^1_S
\end{tikzcd}
\end{center}
In partiuclar, the connecting map $\nabla : \R^n \pi_* (\Omega^\bullet_{X / S}) \to\R^{n} \pi_* (\Omega^{\bullet}_{X / S}) \otimes_{\struct{S}} \Omega^1_S$ 
is a flat connection on the relative deRham sheaf, $\H^n_\dR(X / S) = \R^n \pi_*(\Omega^\bullet_{X / S})$. We call this connection the Gauss-Manin connection.
\end{definition}

\begin{remark}
For example, if $f : X \to S$ is etale then we know that $f^* \Omega^1_S \to \Omega^1_X$ is an isomorphism and thus $\Omega^1_{X / S} = 0$. Therefore, the sheaf of relative deRham cohomology is,
\[ \R^n \pi_* (\Omega^\bullet_{X / S}) = \R^n \pi_* (0 \to \struct{X} \to 0 \to \cdots) = R^n \pi_*(\struct{X}) \]
Then the connecting map $\nabla : R^n \pi_*(\struct{X}) \to R^n \pi_*(\struct{X}) \otimes_{\struct{S}} \Omega^1_S$ is simply induced by the exerior derivative,
\[ \nabla  = R^n \pi_*(\d{} : \struct{X} \to \Omega^1_X) \]
where $\pi_* (\Omega^1_X) = \pi_* \struct{X} \otimes_{\struct{S}} \Omega^1_S$.
\end{remark}

\begin{remark}
If we take $k_0 = \C$ then GAGA implies that,
\[ \H^n_{\dR}(X / S)^\an \cong \H^n_{\dR}(X^\an / S^\an) \]
and $\nabla^\an$ is a flat connection on $\H^n_{\dR}(X^\an / S^\an)$ so there is a relative deRham complex,
\begin{center}
\begin{tikzcd}
0 \arrow[r] & \struct{X}^\an \arrow[r, "\d{}"] & (\Omega^1_{X / S})^\an \arrow[r, "\d{}"] & (\Omega^2_{X / S})^\an \arrow[r] & \cdots 
\end{tikzcd}
\end{center}
However, by Ehresmann's lemma, locally above $s \in S$ we may write $\pi^{-1}(U) = U \times X_s$ and choose $U$ to be contractible. Then, locally, $\Omega^\bullet_{X^\an / S^\an} = \underline{\C}_X \otimes (\Omega^\bullet_{X_s})^\an$ which, using the projection formula, gives,
\[ \H^n_{\dR}(X^\an / S^\an) = \R^n \pi_* (\underline{\C}_X \otimes (\Omega^\bullet_{X_s})^\an) = R^n \pi_*(\underline{\C}_X) \otimes \struct{S}^\an \] 
In particular, there is a natural connection on this analytic sheaf,
\begin{align*}
\nabla^\an & : R^n \pi_*(\underline{\C}_X) \otimes_{\underline{\C}_S} \struct{S} \to (R^n \pi_*(\underline{\C}_X) \otimes_{\underline{\C}_S} \struct{S}) \otimes_{\struct{S}} \Omega^1_S = R^n \pi_*(\underline{\C}_X) \otimes_{\underline{\C}_S} \Omega^1_S
\\
& \nabla^\an : (\alpha \otimes f) \mapsto \alpha \otimes \d{f} 
\end{align*}
Clearly this connection satisfies $\H^n_{\dR}(X^\an / S^\an)^{\nabla^\an} \cong R^n \pi_*(\underline{\C}_X)$. In fact, there is a unique connection satisfing this property which is the GAGA equivalent analytic connection to the algebraic Gauss-Manin connection.
\end{remark}


\section{Local Systems}

\begin{proposition}
Let $\E$ be a vector bundle on $X$ with a flat connection 
\[ \nabla : \E \to \Omega^1_X \otimes_{\struct{X}} \E   \]
Then $\E^\nabla = \ker{\nabla}$ is a local system. 
\end{proposition}

\begin{proof}
Since $\E$ is locally free, we can find a cover of trivializing neighbrohoods $U$ such that $\E|_U \cong \struct{U}^{\oplus n}$. Then $\nabla : \struct{U}^{\oplus n} \to (\Omega^1_U)^{\oplus n}$ is a connection. Define,
\[ \nabla e_j = \sum_{i = 1}^n \omega_{ij} \otimes e_i  \]
where $\omega_{ij} \in \Omega_X^1(U)$ is a form.
This uniquely defines the connection since,
\begin{align*}
\nabla (f_1, \dots, f_n) & = \nabla \left( \sum_{i = 1}^n f_i e_i \right) = \sum_{i = 1}^n \left( f_i \nabla e_i + \d{f_i} \otimes e_i \right)
\\
& = \sum_{i,j = 1}^n \omega_{ij} \otimes f_j e_i + (\d{f_1}, \dots, \d{f_n})
\end{align*}
Therefore, $\E^\nabla$ is given locally by $(f_1, \dots, f_n)$ solving the linear system of differential equations,
\[ \d{f_i} + \sum_{j = 1}^n \omega_{ij} f_j = 0 \]
The condition of flatness is that,
\[ \nabla_1 \circ \nabla = 0 \]
which locally is,
\begin{align*}
\nabla_1 \circ \nabla (f_1, \dots, f_n) & = \nabla_1 \left( \sum_{i,j = 1}^n \omega_{ij} \otimes f_j e_i + \sum_{j = 1}^n \d{f_j} \otimes e_j  \right) 
\\
& = \sum_{i,j = 1}^n \left[ \d{\omega_{ij}} \otimes f_j e_i - \omega_{ij} \wedge \nabla (f_j e_i) \right]  + \sum_{i = 1}^n \left[ \d \d f_i \otimes e_i - \d{f_j} \wedge \nabla e_j \right]
\\
& = \sum_{i,j = 1}^n \left[ \d{\omega_{ij}} \otimes f_j e_i - \omega_{ij} \wedge  \left(  \d{f_j} \otimes e_i  + f_j \sum_{k = 1}^n \omega_{ki} \otimes e_k \right) \right]  - \sum_{i,j = 1}^n \left[ \d{f_j} \wedge \omega_{ij} \otimes e_i  \right]
\\
& = \sum_{i,j = 1}^n  \left[ \d{\omega_{ij}} \otimes  e_i -  \sum_{k = 1}^n \omega_{ij} \wedge \omega_{ki} \otimes e_k \right] f_j
\\
& = \sum_{i,j = 1}^n \left[ \d{\omega_{ij}} + \sum_{k = 1}^n \omega_{ik} \wedge \omega_{kj} \right] \otimes f_j e_i
\end{align*}
So the curvature $\omega_\nabla$ is given by coefficients,
\[ \Theta_{ij} = \d{\omega_{ij}} + \sum_{k = 1}^n \omega_{ik} \wedge \omega_{kj} \]
This vanishing is exactly the criterion in Frobenius' theorem for integrability.
\end{proof}

\section{Principle B}

\newcommand{\et}{\mathrm{et}}
\newcommand{\fin}{\mathrm{fin}}

\begin{proposition}
Let $k_0 \subset \C$ have finite transcendence degree over $\Q$ and $X$ be a complete smooth variety over a field $k$ that is finitely generated over $k_0$. Let $\nabla$ be the Gauss-Manin connection on $\H^n_{\dR}(X)$ relative to $X \to \Spec{k} \to \Spec{k_0}$.
\bigskip\\
If $t \in H^n_{\dR}(X)$ is rational relative to all embeddings $k \embed  \C$ then $\nabla t = 0$. 
\end{proposition}

\begin{proof}
Let $A$ be a finite-type $k_0$-algebra and $\pi : X_A \to \Spec{A}$ a smooth proper map with generic fibre $X_{(0)} = X \to \Spec{k}$ and such that $t$ extends to $\Gamma(\Spec{A}, \H^n_{\dR}(X / \Spec{A})$. After bse change via $k_0 \embed \C$ to $S = \Spec{A_\C}$ there are maps,
\[ X_S \to S \to \Spec{\C} \]
and a global section $t' = t \otimes 1$ of $\H^n_{\dR}(X_S^\an / S^\an)$. We need to show that $(\nabla \otimes 1) t' = 0$. However, if we recall that,
\[ \H^n_{\dR}(X_S^\an / S^\an) = \R^n \pi_*^\an (\Omega^\bullet_{X_S^\an / S^\an}) = (R^n \pi_* \C) \otimes_{\underline{\C}} \struct{S^\an} = H^n(X^\an_S, \underline{\C}) \otimes_{\underline{\C}} \struct{S^\an} \]
and that the Gauss-Manin connection kills exactly those sections purely in,
\[ \H^n(X^\an_S, \underline{\C}_X) = R^n \pi_* (\underline{C}_X) \]
An embedding $\sigma : k \embed \C$ gives a point $\Spec{\C} \to \Spec{A}$ of $s$. Since $t$ is rational,
\[ t(s) \in H^n(X^\an_s, \Q) \subset H^n_{\dR}(X^\an_s) \]
Then locally on $S$ we have $\H^n_{\dR}(X^\an / S^\an) = R^n \pi_*(\underline{\C}_X) \otimes \struct{S}^\an$ which is locally free and $\H^n(X^\an, \underline{\C}_X)$ gives its sheaf of locally constant sections. However, $t$ takes rational vatonal values on the closed points which are dense so it must be locally constant and thus $t \in \H^n(X^\an, \underline{\C}_X)$ so $\nabla t = 0$.
\end{proof}

\begin{definition}
Let $\pi : X \to S$ e a proper smooth map of smooth varieties / $\C$ with $S$ connected. Then,
\[ \H^n_{\et}(X/S)(m) = \varprojlim_r (R^n \pi^*_{\et} \mu_r^{\otimes m} ) \otimes_Z \Q \]
and
\[ \H^n_{\A}(X/S)(m) = \H^n_{\dR}(X/S)(m) \times \H^n_{\et}(X/S)(m) \]
and
\[ \H_B^{2 p}(X/S)(p) = R^{2p} \pi_*^\an \Q(p) \]
\end{definition}

\begin{remark}
By Ehresmann's lemma we can locally write $\pi^{-1}(U) = U \times X_s$ with $U$ contractible. Therefore, by Kunneth,
\[ H_B^{2p}(X/S)(p)(U) = H^{2 p}(\pi^{-1}(U), \Q(p)|_U) = H^{2p}_B(X_s, \Q(p)) \otimes_\Q H^0(U, \Q(p) = H^{2p}_B(X_s, \Q(p)) \]
since $U$ is contractible. This is a constant sheaf so $H^{2p}_B(X/S)(p)$ is a local system. A similar arument holds for the other sheaves.
\end{remark}


\begin{theorem}[Principle B]
Let $t \in \Gamma(\H^{2 p}_{\A}(X/S)(p))$ such that $\nabla t_{\dR} = 0$. If $(t_{\dR})_s \in F^0 H^{2p}_{\dR}(X_s)(p)$ for each $s \in S$ and $t_s$ is an absolute Hodge cyclce in $H^{2p}_\A(X_s)(p)$ for some $s$ then it is an absolute Hodge cycle for every $s$.
\end{theorem}

\begin{proof}
We suppose that $t_s$ is an absolute Hodge cycle for some some $s \in S$. For any $s' \in S$ we need to show that $t_{s'}$ is absolutly Hodge meaning that it is rational relative to every isomorphism $\sigma : \C \to \C$. However, such an isomorphism gives a morphism $\sigma \pi : \sigma X \to \sigma S$ and a section $\sigma(t)$ of $\H^n_\A(\sigma X / \sigma S)(p)$. We know that $\sigma(t)_{\sigma s}$ is rationl and we must show that $\sigma(t)_{\sigma s'}$ is rational. It suffices to prove this for $\sigma = \id$ given that there is some $\sigma$ for which this global rationality holds.
\bigskip\\
First, consider the component $t_{\dR}$ of $t$ (relative the the construction of $\H^n_\A(\sigma X / \sigma S)(p)$ as a product. By assumption $\nabla t_{\dR} = 0$ so $t_{\dR}$ is a global section of $\H^{2p}(X^\an, \underline{\C}_X)$ which we have shown is the vanishing of the analytic Gauss-Manin connection. Since $t_{\dR}$ is rational at one point, it must be rational at every point since $\H^{2p}(X^\an, \underline{\C}_X)$ is locally constant and $X^\an$ is connected. 
\bigskip\\
Thus, it suffices to prove the rationality of the other factor $t_{\et}$. Since the relative cohomology sheaves defined above are local systems, for any point $s$ we have a monodromy action of $\pi_1(S, s)$ on their stalks at $s$ whose fixed points are those germs which extend globally. In particular, this induces isomorphism,
\begin{align*}
\Gamma(S, \H_B^{2p}(X/S)(p)) \cong H^{2p}_B(X_s)^{\pi_1(S,s)}
\\
\Gamma(S, \H_{\et}^{2p}(X/S)(p)) \cong H^{2p}_{\et}(X_s)^{\pi_1(S,s)}
\end{align*}
Then consider the diagram,
\begin{center}
\begin{tikzcd}
\Gamma(S, \H_B^{2p}(X/S)(p)) \arrow[d, "\sim"] \arrow[r, hook] & \Gamma(S, \H^{2p}_B(S/X)(p)) \otimes \A_\fin \arrow[r, "\sim"]  \arrow[d, "\sim"] & \Gamma(S, \H^{2p}_\et(X/S)(p))  \arrow[d, "\sim"]
\\
H^{2p}_B(X_s)(p)^{\pi_1(S,s)} \arrow[d, hook] \arrow[r, hook] & H^{2p}_B(X_s)(p)^{\pi_1(S, s)} \otimes \A_{\fin} \arrow[r, "\sim"]  \arrow[d, hook] & H^{2p}_{\et}(X_s)(p)^{\pi_1(S,s)}  \arrow[d, hook]
\\
H^{2 p}_B(X_s) \arrow[r, hook] & H^{2 p}_B(X_s) \otimes A_\fin \arrow[r, "\sim"] & H^{2 p}_{\et}(X_s)(p)
\end{tikzcd}
\end{center}
We have $t_{\et} \in \Gamma(S, \H^{2 p}_{\et}(X / S)(p))$ which is rational at $s$ so its image in $H^{2 p}_\et(X_s)(p)$ lies in $H^{2 p}_B(X_s)(p)$. Now we need the following lemma which alows us to conclude that $t_{\et} \in \Gamma(S, \H^{2p}_B(X/S)(p))$ and thus $(t_\et)_{s'} \in H^{2 p}_B(X_s)(p) \subset H^{2p}_{\et}(X_s)(p)$ for all $s'$ completing the theorem.
\end{proof}

\begin{lemma}
Let $W \embed V$ be an inclusion of vectorspaces. Let $Z$ be a third vectorspace and take nonzero $z \in Z$. Wmbed $V$ in $V \otimes Z$ via $v \mapsto v \otimes z$. Then, in $V \otimes Z$,
\[ (W \otimes V) \cap (V \otimes z) = W  \otimes z \]
\end{lemma}

\begin{proof}
This is clear if we choose a basis $e_i$ for $W$ which extends to a basis of $V$. Then any $x \in V \otimes Z$ has a unique expansion,
\[ x = \sum e_i \otimes z_i \]
If $x \in W \otimes Z$ then $z_i = 0$ for each $e_i$ not in $W$ and if $x \in V$ then $z_i = z$ for each nonzero $z_i$. 
\end{proof}

\begin{remark}
The proof of principle B concludes taking $Z = \A_{\fin}$ and $z = 1$ over the inclusion $H_B^{2 p}(X_s)^{\pi_1(S,x)(p)} \to H^{2 p}_B(X_s)(p)$. The lemma then implies that, in $H^{2 p}_\et(X_s)(p)$,
\begin{align*}
\Gamma(S, \H_B^{2 p}(X/S)(p)) \cap H^{2 p}_B(X_s)(p) & = [H^{2p}_B(X_s)(p)^{\pi_1(S, s)} \otimes \A_{\fin}] \cap H^{2 p}_B(X_s)(p) 
\\
& = H^{2 p}_B(X_s)(p)^{\pi_1(S,s)} = \Gamma(S, \H^{2p}_B(X/S)(p))
\end{align*}
so we get a global rational section.
\end{remark}


\section{The Main Theorem}

\begin{theorem}[Deligne]
Let $X$ be an abelian variety over an algebraically closed field $k$ and $t \in H^{2p}_{\A}(X)(p)$. If $t$ is a Hodge cycle relative to some embedding $\sigma : k \hookrightarrow \C$ then it is a Hodge cycle with repsect to every embedding. That is, every Hodge cycle is absolutly Hodge. 
\end{theorem}


\section{Hodge Structures and Mumford-Tate Groups}

\subsection{The Deligne Torus}

\renewcommand{\S}{\mathbb{S}}
\newcommand{\Res}{\mathrm{Res}}
\newcommand{\FR}{\mathcal{R}}
\newcommand{\Set}{\mathbf{Set}}
\newcommand{\Nm}{\mathrm{Nm}}

\begin{remark}
Let $T \to S$ be a morphism of schemes. Given an $S$-scheme $X$ and a $T$-scheme $Y$,
\[ \Hom{T}{Y}{X \times_S T} = \Hom{S}{Y}{X} \]
where,
\begin{center}
\begin{tikzcd}[row sep = huge]
Y \arrow[rd, dashed] \arrow[rrd, bend left] \arrow[ddr, dashed, bend right]
\\
& X \times_S T \arrow[r] \arrow[d] & T \arrow[d]
\\
& X \arrow[r] & S
\end{tikzcd}
\end{center}
\end{remark}

\begin{definition}
Let $T \to S$ be a morphism of schemes. Given an $T$-scheme $X$ we define the restriction of scalars functor $\FR_{T/S}(X) : \Sch_{S}^{\op} \to \Set$ via,
\[ Y \mapsto X(Y \times_S T) = \Hom{T}{Y \times_S T}{X}  \]
When the functor $\FR_{T/S}(X)$ is representable in $\Sch_S$ then we call the (unique up to unique isomorphism) $S$-scheme representing it $X' = \Res_{T/S}(X_T)$ such that,
\[ \FR_{T/S}(X) = \Hom{S}{-}{\Res_{T/S}(X)} \]
In this case, we have an isomorphism of functors,
\[  \Hom{T}{- \times_S T}{X} = \Hom{S}{-}{\Res_{T/S}(X)} \]
which makes $\Res_{T/S}(X)$ be right-adjoint to extension of scalars functor,
\[ Y_S \mapsto Y_S \times_S T \] 
\end{definition}

\begin{remark}
Starting with $\Gm^A = \Spec{A[z, z^{-1}]}$ we define some algebraic groups as follows.
\end{remark}


\begin{definition}
The Deligne torus $\S$ is an algebraic group over $\R$ defined as,
\[ \S = \Res_{\C / \R} \Gm^\C \]
where $\Res_{\C / \R}$ is restriction of scalars from $\C$ to $\R$. 
\end{definition}

\begin{remark}
We may characterize $\Res_{\C / \R}$ as the right-adjoint to base change so the $S$-points are,
\begin{align*}
\S(S) & = \Hom{\R}{S}{\Res_{\C / \R} \Gm^\C} = \Gm^\C(S \times_\R \C) = \Hom{\C}{S \times_\R \C}{\Gm^\C}
\\
& = \Hom{\C}{\C[z, z^{-1}]}{\Gamma(S \times_\R \C)} = \Gamma(S \times_\R \C)^\times
\end{align*}
In particular, the $\R$-points of $\S$ are,
\[ \S(\R) = \Gm^\C(\C) = \C^\times \]
Furthermore, the $\C$-points of $\S$ are,
\[ \S(\C) = \Gm^\C(\C \otimes_\R \C) = \Hom{\C}{\C[z, z^{-1}]}{\C \oplus i \C} = \C^\times \times i \C^\times \] 
\end{remark}

\begin{definition}
We define a set of characters and cocharacters of $\S$. First we define the character,
\[ \Nm : \S \to \Gm^\R  \]
on $\R$-points $\S(\R) \to \Gm^\R(\R)$ as $\C^\times \to \R^\times$ via $z \mapsto z \bar{z}$. 
\bigskip\\
Furthermore, we define the cocharacter,
\[ w : \Gm^\R \to \S \]
on $\R$-points $\Gm^\R(\R) \to \S(\R)$ by the natural inclusion $\R^\times \embed \C^\times$. 
\bigskip\\
Lastly, we define a $\C$-cocharacter,
\[ \mu : \Gm^\C \to \S_\C  \]
on $\C$-points via $\Gm^\C(\C) \to \S_\C(\C)$ as $\mu(z) = (z, i)$ where,
\[ \S_{\C}(\C) = \Hom{\C}{\C}{\S \times_\R \C} = \Hom{\R}{\C}{\S} = \S(\C) = \C \oplus i \C \]   
\end{definition}


\subsection{Hodge Structures}

\begin{definition}
Let $V$ be a finite-dimensional $\Q$-vectorspace. A $\Q$-\textit{rational Hodge structure} of weight $n$ on $V$ is a decomposition,
\[ V_\C = V \otimes_\Q \C = \bigoplus_{p + q = n} V^{p,q} \]
such that $V^{q,p} = \overline{V^{p,q}}$. 
\end{definition}

\begin{definition}
A Hodge structure defines a cocharacter,
\[ \mu : \Gm^\C \to \GL{}{V_\C} \]
via $\mu(z) v^{p,q} = z^{-p} v^{p,q}$ for $v^{p,q} \in V^{p,q}$. 
\bigskip\\
Furthermore, $\overline{\mu(z)} \cdot v^{p,q} = \bar{z}^{-q} v^{p,q}$ commutes with the action of $\mu(z)$. Therefore, we may take their product to give a map of real algebraic groups,
\[ h : \S \to \GL{}{V_\R} \]
via $h(z) v^{p,q} = z^{-q} \bar{z}^{-q} v^{p,q}$. 
where $\C^\times$ is the algebraic group,
\[ \Spec{\C[x, x^{-1}]} \to \Spec{\R} \]
\end{definition}

\begin{remark}
Conversely, any homomorphism of $\R$-algebraic groups $h : \S \to \GL{}{V_\R}$ which, on $\R$, restricts to $r \mapsto r^{-n} \id_V$ defines a Hodge structure of weight $n$ on $V$ by taking $V^{p,q}$ to be the eigenspace of eigenvalue $z^{-p} \bar{z}^{-q}$ for $h(z)$ i.e.,
\[ V^{p,q} = \{ v \in V_\C \mid \forall z \in \S(\R) : h(z) \cdot v = z^{-p} \bar{z}^{-q} v \} \] 
\end{remark}

\begin{definition}
The Weil operator $C \in \GL{}{V_\R}$ of a Hodge structure $(V, h)$ is $C = h(i)$. 
\end{definition}

\begin{proposition}
Given a Hodge structure on $V$ there is a decreasing filtration of $V_\C$ via,
\[ F^p V = \bigoplus_{p' \ge p} V^{p', n - p'} \]
\end{proposition}

(ASK RAYMOND ABOUT TATE TWISTS AND THIS HODE STRUCTURE)

\begin{example}
For any $m$ we define a Hodge structure of weight $-2m$ denoted $\Q(m)$ via taking $\Q(m)_\C = \Q(m)^{-m,-m}$ 
\end{example}

\subsection{Mumford-Tate Groups}


\begin{definition}
The Mumford-Take group $M(V)$ associatd to  Hodge structure $(V, h)$ is the smallest $\Q$-algebraic subgroup of $\GL{}{V}$ such that,
\[ \Im{h}(\R) \subset M(V)(\R) \]
\end{definition}

\begin{example}
For $\Q(m)$ as a Hodge structure the map $h : \C^\times \to \GL{1}{\R}$ is given by $h(z) = |z|^{-m}$ which is surjective for $m \neq 0$. Thus, for $n \neq 0$ we have,
\[ M_h = \mathbb{G}_{m}^\Q \]
and for $n = 0$ it is $\Spec{\Q}$ the trivial $\Q$-group scheme. 
\end{example}


(BADDD)

\begin{proposition}
Let $V$ be a $\Q$-vectorspace with Hodge structure $h$ of weight $n$. The tensor space,
\[ T = V^{\otimes m_1} \otimes V^{\vee \otimes m_2} \otimes \Q(1)^{\otimes m_3} \]
has a Hodge structure of weight $(m_1 - m_2)n  - 2 m_3$. Then the Mumford-Tate group $G$ of $(V, h)$ is the subgroup of $\GL{n}{V} \times \mathbb{G}_m$ fixing all rational tensors of type $(0, 0)$ in $T$. 
\end{proposition}

\begin{proof}
For any $t \in T$ the element $t$ is of type $(0, )$ iff it is fixed by $\mu(\mathbb{G}_m)$ so $M_h = H'$. We will now prove that characters of $H$ lift and thus $H = H'$. 
\end{proof}

\subsection{DO IT RIGHT}

\begin{remark}
Let $(V, h)$ be a Hodge structure of weight $d$. Then the tensor space,
\[ T^{m,n}(V) = \bigoplus_{j = 1}^n V^{\otimes m_j} \otimes (V^\vee)^{\otimes n_j} \]
is a Hodge structure of weight,
\[ N = \sum_{j = 1}^n (m_j  - n_j) d \]
Furthermore, let $M(V)$ be the Mumford-Tate group of $(V, h)$ i.e. the intersection of all $\Q$-algebraic subgroups of $\GL{}{V}$ whose $\R$-points contain $\Im{h}$. 
\end{remark}

\begin{lemma}
There are morphism of $\R$-algebraic subgroups,
\begin{center}
\begin{tikzcd}
\S \arrow[r, "\psi"] & M(V)_\R \arrow[r, hook] & \GL{}{V_\R}
\end{tikzcd}
\end{center}
Conversely, given any $\Q$-vectorspace $H$ with an algebraic representation,
\[ \rho : M(V) \to \GL{}{H} \]
gives $H$ a Hodge structure via,
\begin{center}
\begin{tikzcd}
\S \arrow[r, "\psi"] & M(V)_\R \arrow[r, "\rho"] & \GL{}{H_\R}
\end{tikzcd}
\end{center}
\end{lemma}

\begin{proposition}
Let $H \subset T^{m,n}(V)$ be any rational subspace. Then $H$ is a Hodge substructure iff $H$ is stable under $M(V)$. Furthermore, a rational vector $t \in T^{m,n}(V)$ is of type $(0,0)$ iff it is fixed by $M(V)$. 
\end{proposition}

\begin{proof}
If $H$ is stable under the action of the Mumford-Tate group then it becomes a representation $\rho : M(V) \to \GL{}{H}$ since it is rational this gives a Hogde structure on $H$. 
\bigskip\\
Conversely, suppose that $V \subset T^{m,n}(V)$ is a substructure then consider its stabilizer $G_H \subset \GL{}{V}$ which is a $\Q$-algebraic subgroup since $H$ is rational. Moreover, $(G_H)_\R$ contains $\Im{h}$ because as a Hodge structure it splits into eigenspaces of $h$ so is preserved by its image. Thus $M(V) \subset G_H$ by definition so $M(V)$ preserves $V$.
\bigskip\\
Likewise, it is clear that $t$ is fixed by the action of $\S(\R)$ iff $t$ is of Hodge type $(0,0)$. Thus it suffices to prove that $t$ is fixed by $\S(\R)$ iff it is fixed by $M(V)$. A similar argument will show this.
\bigskip\\
First, if $t$ is fixed by $M(V)$ then it is fixed by $M(V)(\R)$ which contains $\Im{h}$ and thus $t$ is fixed by $\S(\R)$. 
\bigskip\\
Conversely, if $t$ is fixed by $\S(\R)$ then its stabilizer $G_t \subset \GL{}{V}$ is a $\Q$-algebraic subgroup since $t$ is rational. Furthermore, by assumption, $\Im{h} \subset (G_t)(\R)$ and thus $M(V) \subset G_t$ by definition showing that $M(V)$ fixes $t$.
\end{proof}

\begin{corollary}
The space $\End{V}$ is an algebraic $M(V)$-rep and therfore a Hodge structure. Furthermore, the type-$(0,0)$ Hodge classes are exactly morphisms of Hodge structures since they must commute with the action of $\S$. Therefore,
\[ \Hom{\text{HS}}{V}{V} = \End{V}^{M(V)} \] 
\end{corollary}


\subsection{Polarization}

\begin{definition}
A polarization $\psi$ of $(V, h)$ is a morphism of Hodge structures,
\[ \psi : V \times V \to \Q(-n) \]
such that $\psi(x, C y)$ on $V_\R$ is an inner product where $C = h(i)$ is the Weil operator.  
\end{definition}

\begin{remark}
Under the canonical isomorphism,
\[ \Hom{}{V \otimes V}{\Q(-n)} \cong V^\vee \otimes V^\vee(-n) \]
a polarization is a tensor of bidegree $(0,0)$ because it is a morphism of Hodge structures and thus is fixed by the Mumford-Tate group $G$,
\[ \forall v,v' \in V : \forall (g_1, g_2) \in G(\Q) : \psi(g_1 v, g_1 v') = g_2^n \psi(v, v') \]
\end{remark}

\begin{remark}
Let $C = h(i)$ be a Weil operator. For $v^{p,q} \in V^{p,q}$ we have $C v^{p,q} = i^{-p + q} v^{p, q}$ and thus $C^2$ acts as $(-1)^n$ on all of $V$ where $n = p + q$ is the weight of $V$. 
\end{remark}

\begin{definition}
Let $H$ be a real algebraic group with an involution $\sigma$ of $H_\C$. Then a real-form of $H$ is a real algebraic group $H_\sigma$ and an isomorphism $H_\C \to (H_\sigma)_\C$ sending complex conjugation to the action of $\sigma$ on complex conjugation on $H(\C)$. 
\end{definition}

\begin{theorem}
The Mumford-Tate group $M(V)$ is connected and if $(V, h)$ is polarizable then $M(V)$ is reductive.
\end{theorem}

\begin{proof}
$M(V)$ is clearly connected else its connected component of the identity would be a smaller $\Q$-algebraic subgroup also satisfying the property that its $\R$-points contain $\Im{h}$ (because $\S$ is connected the image must lie in this connected component). Now, we use the fact that a connected algebraic group is reductive if it has a faithful semisimple representation. We will show that the tautological representation $M(V) \embed \GL{}{V}$ which is clearly faithful is also semisimple when $V$ is polarizable.  
\end{proof}

\begin{proposition}
If $V$ is polarizable then $M(V) \subset \GL{}{V}$ is semisimple. 
\end{proposition}

\newcommand{\inner}[2]{\left< #1, #2 \right>}

\begin{proof}
We will prove that a real algebraic group $H$ is semisimple if it has a \textit{compact} real-form. It suffices to show that $H_\sigma$ is semisimple. By the unitarian trick, any finite-dimensional $H$-rep has an $H_\sigma$-invariant positive definite symmetric form via,
\[ \inner{u}{v}_0 = \int_{H_\sigma} \inner{h \cdot u}{h \cdot v} \]
to conclude that every finite-dimensional $H_\sigma$-rep is semisimple. This implies that $H_\sigma$ is reductive. 
\bigskip\\
Thus, it suffices to prove that the Mumford-Tate group has a \textit{compact} real-form (the compactness here is the magic ingredient). Consider the special Mumford-Take group of $(V, h)$,
\[ G^0 = \ker{(G \to \Gm)} \]
and $G^1$ be the smallest $\Q$-reational subgroup of $\GL{}{V} \times \Gm$ (WHY THIS GROUP) such that $G^1_\R$ constains $h(U^1)$ where $U^1$ is the $\R$-algebraic groups whose $\R$-points are $S^1 \subset \C^\times$. Then, $G^1 \subset G^0 \subset G$ since,
\[ G^1_\R \cdot h(C^\times) = G_\R \text{ and } h(U^1) = \ker{(h(C^\times) \to \Gm} \]
so $G^0 = G^1$ and thus $G^0$ is connected since $G^1$ is.
\bigskip\\
Since $C = h(i)$ acts trivially on $\Q(1)$ we know $C \in G^0(\R)$. Furthermore $C^2$ acts as $(-1)^n$ on $V$ and thus is in the center of $G^0(\R)$. The inner automorphism $a_C : g \mapsto C^{-1} g C$ of $G_\R$ is therefore an involution since its square satisfies,
\[ a_C^2(g) = C^{-2} g C^{2} = g \]
because $C^2$ is in the center.
\bigskip\\
Now let $\psi$ be a polarization of $V$. For $u,v \in V_\C$ and $g \in G^0(\C)$ we have,
\[ \psi(u, C \bar{v}) = \psi(gu, g C \bar{v}) = \psi(g , C C^{-1} g C \bar{v}) = \psi(gu, C \overline{ a_C(\bar{g}) v}) \]
Thus, the positive-definition bilinear form $\phi(u,v) = \psi(u, C \bar{v})$ on $V_\R$ is invariant under the $G^0$-real-form $G^0_{a_C}$ since the action of $\bar{g}$ is sent to $a_C (\bar{g})$ under the the isomorphism $G^0_\C \to (G^0_{a_C})_\C$. Since $G^0_{a_C}$ has an invariant inner-product on $V$ it must be compact. (ASK HARRIS ABOUT THAT)
\end{proof}


\subsection{Characterizing Subgroups}

Here let $G$ be a reductive algebraic group over  a field $k$ of characteristic zero and let $V_\alpha$ be a faithful faimly of finite-dimensional representations of $G$ over $k$ such that $G \to \prod \GL{}{V_\alpha}$ is injecitve. We may define a tensor algebra,
\[ T^{m,n} = \bigotimes_\alpha V_\alpha^{\otimes m(\alpha)} \otimes \bigotimes_{\alpha} (V_\alpha^\vee)^{\otimes n(\alpha)} \]
which is also a finite $G$-rep. 

\begin{definition}
Then for any algebraic subgroup $H \subset G$ we write $H'$ for the subgroup fixing all tensors appearing in some $T$ fixed by $H$. That is, $H'$ is the largest subgroup $H \subset H'$ which fixes every tensor fixed by $H$. 
\end{definition}

\begin{definition}
Given an algebraic group $G$ over $k$ we define its character group,
\[ X_k(G) = \Hom{k}{G}{\Gm^k} \]
\end{definition}

\begin{theorem}
We have the following,
\begin{enumerate}
\item Every finite $G$-rep is contained in a sum of $T^{m, n}$
\item Every subgroup $H \subset G$ is the stabilizer of a line $D$ is some finite $G$-rep.
\item If $H \subset G$ is reductive or $X_k(G) \to X_k(H)$ is surjective then $H = H'$.
\end{enumerate}
\end{theorem}

\begin{proof}
Let $W$ be a finite $G$-rep and $W_0$ be the trivial rep on the underlying space of $W$. There is a morphism of $G$-reps, $W \to W_0 \otimes_k k[G] \cong k[G]^{\dim{W}}$ so it suffices to prove that the regular representation can be expressed in terms of tensors.
\bigskip\\
There must be a finite sum $V = \bigoplus_{\alpha} V_\alpha$ such that the action $G \to \GL{}{V}$ is faithful then embed,
\[ \GL{}{V} \to \End{V} \times \End{V^\vee} \]
identifying $\GL{}{V}$ with a closed subvariety of $\End{V} \times \End{V^\vee}$
(FIX)
\bigskip\\
Let $I \subset \Gamma(G, \struct{G})$ be the ideal of global functions on $G$ whose value is zero on $H$. Consider the regular $G$-representation $k[G]$ 
(FIX)
\bigskip\\
The subgroup $H$ is the stabilizer of a line $D$ in some $G$-representation $V$ which, by (a), we may take to be a direct sum of tensor representations $T^{m,n}$. Now suppose that $H$ is reductive then $V$ must be a semisimple $H$-representation so we can write $V = W \oplus D$ for some $H$-representation $W$. Furthermore, dualizing $V^\vee = W^\vee \oplus D^\vee$. Since $H$ is the stabilizer of $D$ 
\end{proof}

(WHAT IS THE POINT)

\begin{lemma}
Every $\Q$-character of $H$ (above) extends to $\GL{}{V} \times \Gm$ 
\end{lemma}

\begin{proof}
Any $\Q$-character restricted to $\Gm$ is $\Q(n)$ for some $n$. After tensoring with $\Q(-n)$ we find that the character is trivial on $\mu(\Gm)$. But $H$ as the minimal subgroup must act trivially then we use the fact that trivial characters extend. 
\end{proof}

(OF THIS)

\begin{theorem}
Let $G \subset \GL{}{V}$ be the subgroup of all elements which fix every $(0,0)$-hodge class in every tensor space $T^{m, n}(V)$. Then $M(V) = G$. 
\end{theorem}

\begin{proof}
We have shown that $M(V) \subset G$. Furthermore, $M(V)' = G$ since $(0,0)$-tensors are exactly the tensors fixed by the Mumford-Tate group and thus $G$ is the group of all elements fixing all tensors fixed by $M(V)$. Now we use the general fact about reductive groups that if $G$ is reductive and $H \subset G$ is a reductive subgroup then $H' = H$.
\end{proof}

\subsection{Back to Principle B}

\begin{remark}
We need a slightly stronger version of Principle B proved as a corellary.
\end{remark}

\begin{theorem}
Let $\pi : X \to S$ be a smooth proper map of smooth varieties over $\C$ with $S$ connected and let $V$ be a local subsyttem of $R^{2 p} \pi_* \Q(p)$ such that $V_s$ consists purely of $(0,0)$-cycles for all $s$ and consistens of absolute Hodge cycles at at least one $s \in S$. Then $V_s$ consists of absolute Hodge cycles for all $s \in S$.
\end{theorem}

\begin{proof}
If $V$ is constant i.e. if the map $\Gamma(S, V) \to V_s$ is bijective then this follows immedietly from the above argument. However, we may reduce the general case to this as follows.
\bigskip\\
By Hodge theory on $S^\an$, at each point $s \in S$ the stalk $(R^{2 p} \pi_* \underline{\Q}(p))_s$ has a Hodge structure and a polarization which, since $R^{2 p} \pi_* \Q(p)$ is a local system, glue to give a form,
\[ \psi : R^{2 p} \pi_* \underline{\Q}(p) \times R^{2 p} \pi_* \underline{\Q}(p) \to \underline{\Q}(-p) \]
which at each point is a polarization on the Hodge structure $(R^{2 p} \pi_* \Q(p))_s$. On the rational $(0,0)$-subspace,
\[ (R^{2 p} \pi_* \underline{\Q}(p))_S \cap (R^{2 p} \pi_* \underline{\C}(p))_s^{0,0} \]
the form is symmetric, bilinear, rational and positive definite. Since $V_{s}$ everywhere consists of $(0,0)$-cycles this is a form defined on $V_{s}$. Since monodromy preserves the form, the image of $\pi_1(S, s_0)$ in $\Aut{V_{s_0}}$ is finite because it is discrete and lies inside the compact group preserving the form. Therefore, after passing to a finite covering we can ensure that $\pi_1(S, s_0)$ acts trivially on $V_{s_0}$ implying that $V$ is globally constant. 
\end{proof}


\section{Principle A}

\begin{definition}
Let $X_\alpha$ be a family of complete smooth varieties over $k$. We define tensor spaces,
\begin{align*}
T_{\dR} & = \left( \bigotimes_{\alpha} H_{\dR}^{m(\alpha)} (X_\alpha) \right) \otimes \left( \bigotimes_{\alpha} H^{n(\alpha)}_{\dR}(X_\alpha)^\vee \right) (m)
\\
T_{\dR} & = \left( \bigotimes_{\alpha} H_{\et}^{m(\alpha)} (X_\alpha) \right) \otimes \left( \bigotimes_{\alpha} H^{n(\alpha)}_{\et}(X_\alpha)^\vee \right) (m)
\\
T_{\A} & = T_{\dR} \times T_{\et}
\end{align*}
Finally, given an inclusion $k \embed \C$ we get a Betti tensor space,
\[ T_{\sigma} = \left( \bigotimes_{\alpha} H_{\sigma}^{m(\alpha)} (X_\alpha) \right) \otimes \left( \bigotimes_{\alpha} H^{n(\alpha)}_{\sigma}(X_\alpha)^\vee \right) (m) \]
We say that an element $t \in T_{\A}$ is,
\begin{enumerate}
\item rational relative to $\sigma$ if its image in $T_\A \otimes_{k \times \A_{\text{fin}}} (\C \times \A_{\text{fin}})$ lies in the subspace $T_{\sigma}$
\item is a Hodge cycle relative to $\sigma$ if it is rational relative to $\sigma$ and its first component lies in $F^0$ meaning it lies in the subspace generated by,
\[ F^0 H_{\dR}^{2p}(X)(p) = H^{p,p}_{\dR}(X) \subset H_{\dR}^{2p}(X)(p) \times H^{2p}_{\et}(X)(m) \]
\item is absolutly Hodge if it is a Hodge cycle relative to each $\sigma : k \embed \C$.
\end{enumerate}
\end{definition}

\begin{theorem}[Principle A]
Let $X_\alpha$ be a family of varieties over $\C$ and,
\[ T = \bigotimes_{\alpha} H_B^{n_\alpha}(X_\alpha) \otimes H^{n_\alpha}_B(X_\alpha)^\vee \otimes \Q(1) \]
Let $t_i \in T_i$ be absolute Hodge cycles and let $G$ be the subgroup of,
\[ \prod_{\alpha, n_\alpha} \GL{}{H^{n_\alpha}_B(X_\alpha)} \times \Gm \]
fixing all $t_i$. If $t \in T$ and is fixed by $G$ then it is an aboslute Hodge cycle.  
\end{theorem}

\begin{remark}
We first need a lemma.
\end{remark}

(FIX THIS SECTION ON TORSORS)

\begin{lemma}
Let $G$ be an algebraic group over $\Q$ and $P$ be a $G$-torsor of isomorphism $H^\alpha_\sigma \to H^\alpha_\tau$ where these are families of $\Q$-rational $G$-reps. Let $T_\sigma$ and $T_\tau$ be tensor spaces of $H_\sigma$ and $H_\tau$. Then $P$ defines a map $T_\sigma^G \to T_\tau$. 
\end{lemma}

\begin{proof}
Locally, for the etale topology on $\Spec{\Q}$, (MEANING WE CAN CHOSE AN ETALE COVERING SUCH THAT THIS IS THE CASE?) points of $P$ give isomorphisms $T_\sigma \to T_\tau$. Furthermore, the restriction to $T^G_\sigma$ is idependent of the point since $P$ is a $G$-torsor. Therefore, this map descends to $T^G_\sigma \to T_\tau$. 
\end{proof}

\begin{proof}
We define our groups over $k$ with an isomorphism $\sigma : k \embed \C$. Let $\tau : k \embed \C$ be any other isomorphism. We may assume that $t$ and $t_i$ belong to the same tensor space $T$ then because the $t_i$ are absolute Hodge cylces, they lie in $T_\sigma$ for each $\sigma$. Then there are inclusions of cohomology,
\begin{center}
\begin{tikzcd}
H_\sigma(X_\alpha) \arrow[dr, hook] & & H_\tau(X_\alpha) \arrow[dl, hook]
\\
& H_\sigma(X_\alpha) \otimes (\C \times \A_{\text{fin}}) 
\end{tikzcd}
\end{center}
defined by these isomorphisms. These inclusions follow from the identification of $H_\sigma(X_\alpha) \otimes (\C \times \A_{\text{fin}})$ with the etale cohomology which is independent of the choice of embedding $k \embed \C$. These induce maps on the tensors,
\begin{center}
\begin{tikzcd}
T_\sigma \arrow[dr, hook] & & T_\tau \arrow[dl, hook]
\\
& T \otimes (\C \times \A_{\text{fin}}) 
\end{tikzcd}
\end{center}
Now, define a functor,
\[ P(R) = \{ p : H_\sigma \times R \xrightarrow{\sim} H_\tau \otimes R \mid p : \text{p preserves each absolute Hodge cylces} \} \]
Recall that, by definition, an absolute Hodge cycle corresponds to another absolute Hodge cycle for each embedding $k \embed \C$ so the condition above make sense, $p$ should itentify $t_i \in T_\sigma$ with its corresponding absolute Hodge cylce in $T_\tau$. 
\bigskip\\
The inclusions demonstrate that $P(\C \times \Afin)$ is nonempty and since $H_\sigma \otimes R$ and $H_\tau \otimes \R$ are $G$-representations we get a $G$-action on $P(R)$. Since $G$ is the group fixing exactly the absolute Hodge cycles, we can see that $P$ is a $G$-torsor.
\bigskip\\
If we apply the previous lemma we obtain a map $T^G_\sigma \to T_\tau$ making the following diagram commute,
\begin{center}
\begin{tikzcd}
T_\sigma^G \arrow[r] \arrow[d, hook] & T_\tau \arrow[d, hook]
\\
T_\sigma \arrow[r, hook] & T \otimes (\C \times \A_{\text{fin}})
\end{tikzcd}
\end{center}
Therefore, the map $T^G_\sigma \to T_\tau$ is injective we must have $t \in T_\tau$ since it lies in $T^G_\sigma$ by hypothesis. Thus $t$ is rational relative to all $\sigma$.
\bigskip\\
It remains to show that the component $t_{\dR}$ of $T \otimes \C = T_{\dR}$ lies in the filtration $F^0 T_{\dR}$. For a rational $s \in T_{\dR}$,
\[ s \in F^0 T_{\dR} \iff s \text{ is fixed by } \mu(\C^\times) \]
where $\mu(\C^\times)$ corresponds to the real action defining the Mumford-Tate group. Since, by hypothesis, $(t_i)_{\dR} \in F^0$ we know that $G \supset \mu(\C^\times)$ since $\mu(\C^\times)$ must fix all of them. Clearly then if $t$ is fixed by $G$ we must have $t$ fixed by $\mu(\C^\times)$ and thus $t_{\dR} \in F^0 T_{\dR}$. 
\end{proof}

\section{Construction of Some Absolute Hodge Cycles}

\subsection{Hermitian Forms}

\renewcommand{\Tr}{\mathrm{Tr}}

\begin{remark}
Recall that a number field $E$ is a CM-filed if for each embedding $E \embed \C$ complex conjugation induces a nontrivial automorphism on $E$ independently on the embedding. The fixed field is then a totally real field $F$ and $E / F$ has degree $2$. 
\end{remark}

\begin{definition}
If $E$ is a CM-field and $V$ is a $K$-vectorspace then a sesquilinear form $\phi : V \times V \to \E$ is Hermitian if $\phi(v,w) = \overline{\phi(w,v)}$. 
\end{definition}

\begin{remark}
For any embedding $\tau : F \embed \R$ we obtain a Hermitian form $\phi_\tau$ on $V_\tau = V \otimes_{\tau} \R$. Let $a_\tau$ and $b_\tau$ be the dimensios of the maximal subspaces of $V_\tau$ on which $\phi_\tau$ is positive definite and negatice dfinite respectively. 
\bigskip\\
Furthermore, $\phi$ defines a Hermitan form on the top forms $\Lambda^{\dim{V}} V \cong E$ which must be an $E$-Hermitian form on $E$ and thus is given by an element $f \in F$ defined up to $\Nm_{E/F} E^\times$. We call this the discriminant. 
\end{remark}

\begin{remark}
Let $(v_1, \dots, v_d)$ be an orthogonal basis for $\phi$ and $\phi(v_i, v_i) = c_i$. Then $a_\tau$ is the number of $i$ s.t. $\tau c_i > 0$ and $b_\tau$ is the number of $i$ s.t. $\tau c_i < 0$ and $f = c_1 \cdots c_n$. If $\phi$ is nondegenrate, then $f \in F^\times / \Nm_{E/F} E^\times$ and,
\[ a_\tau + b_\tau = \dim{V} \quad \quad \quad \mathrm{sign}{(\tau f)} = (-1)^{b_\tau} \]
\end{remark}

\begin{proposition}
Given, for each embedding  $\tau : F \embed \C$, a tripple $(a_\tau, b_\tau)$ and $f \in F^\times / \Nm_{E/F} E^\times)$ satisfying the above. Then there exists a unique pair $(V, \phi)$ a non-degenerate Hermitian form $\phi$ on an $E$-vectorspace $V$ with invariants $(a_\tau, b_\tau)$ with respect to $\tau : F \embed \R$ and $f$. 
\end{proposition}

\begin{definition}
A Hermitian space $(V, \phi)$ of dimension $d$ is \textit{split} if it satisfies the equivalent conditions,
\begin{enumerate}
\item $a_\tau = b_\tau$ for all $\tau$ and $f = (-1)^{d/2}$
\item there is a totally isotropic subspace of $V$ of dimnsion $d/2$ (for each $v \in W : \phi(v,v) = 0$).
\end{enumerate}
\end{definition}

\begin{lemma}
Let $k$ be a field, $k'$ an etale $k$-algebra (a finite product of finite separable extensions of $k$) and $V$ a f.g. free $k'$-module. Then,
\begin{enumerate}
\item The map,
\[ f \mapsto \Tr_{k'/k} \circ f : \Hom{k'}{V}{k'} \to \Hom{k}{V}{k} \]
is an isomorphism of $k$-vectorspaces.
\item $\exterior^n_{k'} V$ is a direct summand of $\exterior^n_k V$ naturally.
\end{enumerate}
\end{lemma}

\begin{proof}
The trace map $\Tr_{k'/k} : k' \times k' \to k$ is nondegenerate (HOW IS THIS A PAIRING). The map $f \mapsto \Tr_{k'/k} \circ f$ is injective and then onto because the spaces are of the same dimension.
\bigskip\\
There are obvious maps,
\begin{align*}
\exterior^n_k V & \to \exterior^n_{k'} V 
\\
\exterior^n_k V^\vee & \to \exterior^n_{k'} V^\vee
\end{align*}
where here we deine the dual of $k'$-modules as,
\[ V^\vee = \Hom{k'}{V}{k'} = \Hom{k}{V}{k} \]
(WHAT?)
\end{proof}

\subsection{Conditions to Consist of Absolute Hodge Cycles}

\begin{remark}
In this section we will be in the following situation. 
\end{remark}

\begin{definition}
Let $A$ be an abelian variety over $\C$ and $E$ a CM field with a homomorphism $\nu : E \to \End{A} \otimes_\Z \Q$. Let $ = \dim_{E} H_1(A, \Q)$ which has an $E$-vectorspace structure via $\nu$. Thus, $2 \dim{A} = d [E : \Q]$. 
\end{definition}

\newcommand{\g}{\mathfrak{g}}

\begin{proposition}
The analytic space $A^\an$ is a compact complex Lie group which is a complex torus. Let $\g$ be the lie Algebra then there is an $\R$-linear map $\g \to H_1(A^\an, \R)$ sending a tangent vector to the homology class defined by its geodesic (ASK HARRIS ABOUT THIS). Now $\g$ is a complex vectorspace so $H_1(A^\an, \R)$ inherents a complex structure given by an $\R$-linear endomorphism $J : H_1(A^\an, \R) \to H_1(A^\an, \R)$.   
\end{proposition}

\begin{proposition}
Hoge theory gives a hodge structure on $H^1(A^\an, \R)$ which is determined by a map $h : \S \to \GL{}{H^1(A, \R)}$.
\bigskip\\
Now, on a complex torus of $\dim_{\R}(A^\an) = 2g$ there are isomorphisms,
\[ H^1(A^\an, \R)^\vee \xrightarrow{\sim} \exterior^{2 g-1} H^1(A^\an, \R) \xrightarrow{\sim} H^{2g - 1}(A^\an, \R) \xrightarrow{\sim} H_1(X, \R) \]
This identification gives an isomorphism,
\[ \GL{}{H^1(A^\an, \R)} \cong \GL{}{H_1(A, \R)} \]
under which $h(i) \mapsto J$. 
\end{proposition}

\begin{proposition}
Consider the decomposition,
\begin{align*}
E \otimes_\Q \C & \xrightarrow{\sim} \prod_{\sigma \in \Hom{}{E}{\C}} \C
\\
e \otimes z & \mapsto (\sigma \mapsto \sigma(e) \cdot z)
\end{align*}
Tensoring by $H^1_B(A) = H^1(A^\an, \Q)$ we find,
\[ H^1_B(A) \otimes_\Q \C \cong \bigoplus_{\sigma \in \Hom{}{E}{\C}} H^1_{B}(A) \otimes_{\sigma} \C \]
where,
\[ H^1_B(A) = H^1(A^\an, \Q) \] 
is an $E$-vectorspace and $e \in E$ acts on $H^1_{B}(A) \otimes_\sigma \C$ via $\sigma(e)$. Since $E$ repsects the Hodge structure on $H^1_B(A)$ each $H^1_{E, \sigma}(A) = H^1(A^\an, \Q) \otimes_\sigma \C$ acquires a Hodge structure,
\[ H^1_{E, \sigma}(A) = H^{1,0}_{E, \sigma}(A) \oplus H^{0,1}_{E, \sigma}(A) \]
Define,
\[ a_\sigma = \dim_{\C} H^{1,0}_{E, \sigma}(A) \quad \text{and} \quad b_\sigma = \dim_{\C} H^{1,0}_{E, \sigma}(A) \quad \text{thus} \quad a_\sigma + b_\sigma = d \]
\end{proposition}

\begin{proposition}
The subspace,
\[ \exterior^d_E H^1_B(A) \subset H^d(A^\an, \Q) \]
has pure bidegree $(\tfrac{d}{2}, \tfrac{d}{2})$ iff $a_\sigma = b_\sigma$ for each $\sigma \in \Hom{}{E}{\C}$. 
\end{proposition}

\begin{proof}
For a complex torus, we have,
\[ H^d(A^\an, \Q) \cong \exterior^d_{\Q} H^1(A^\an, \Q) \] 
so a previous lemma identifies,
\[ \exterior^d_{E} H^1(A^\an, \Q) \subset \exterior^d_\Q H^1(A^\an, \Q) \]
as a direct summand. 
Then consider,
\begin{align*}
\left( \exterior^d_E H^1_B(A) \right) \otimes_\Q \C & \cong \exterior^d_{E \otimes_\Q \C} \left( H^1_B(A) \otimes_\Q \C \right) 
\\
& \cong \bigoplus_{\sigma \in \Hom{}{E}{\C}} \exterior^d_\C (H^1(A^\an, \Q) \otimes_{\sigma} \C) )
\\
& \cong \bigoplus_{\sigma \in \Hom{}{E}{\C}} \exterior^d_\C (H^{1,0}_{E, \sigma}(A) \oplus H^{0,1}_{E, \sigma}(A))
\\
& \cong \bigoplus_{\sigma \in \Hom{}{E}{\C}} \exterior^{a_\sigma}_\C H^{1,0}_{E, \sigma}(A) \oplus \exterior^{b_\sigma}_\C H^{0,1}_{E, \sigma}(A)
\end{align*}
Thus, we have decomposed this subspace into a sum of pure bidegree $(a_\sigma, 0)$ and $(0, b_\sigma)$ proving the proposition. 
\end{proof}

\begin{remark}
In the case $a_\sigma = b_\sigma$ then,
\[ \left( \exterior^d_E H^1_B(A) \right) (\tfrac{d}{2}) \]
(ASK HARRIS WHY TATE TWIST HERE?) consists of Hodge cycles. We want to know when this consists of absolute Hodge cycles.
\end{remark}

\begin{lemma}
If $A = A_0 \otimes_\Q E$ for some abelian variety $A_0$ of dimension $\tfrac{d}{2}$ then,
\[ \left( \exterior^d_E H^1_B(A) \right) (\tfrac{d}{2}) \subset H^d(A^\an, \Q) (\tfrac{d}{2}) \]
consists of absolute Hodge cycles. 
\end{lemma}

\begin{proof}
Consier the diagram,
\begin{center}
\begin{tikzcd}
H^d_B(A_0)(\tfrac{d}{2}) \otimes_\Q E \arrow[r] \arrow[d, "\sim"] & H^d_B(A_0)(\tfrac{d}{2}) \otimes_\Q E \arrow[d, "\sim"]
\\
\left( \exterior^d_E H^1_B(A_0 \otimes_\Q E) \right) (\tfrac{d}{2}) \arrow[r] & \left( \exterior^d_{E \otimes \A} H^1_\A(a_0 \otimes_\Q E) \right) (\tfrac{d}{2}) \arrow[r, hook] & H^d_{\A}(A_0 \otimes E)(\tfrac{d}{2}) 
\end{tikzcd}
\end{center}
The vertical maps are induced by the isomorphism $H^1_B(A_0) \otimes_\Q E \xrightarrow{\sim} H^1_B(A_0 \otimes_\Q E)$. There is a similar diagram for each embedding $\sigma : E \embed \C$ and thus the image of the bottom map must be stable with respect to a choice of $\sigma : E \embed \C$. Therefore, the Hodge cycles,
\[ \left( \exterior^d_E H^1_B(A_0 \otimes_\Q E) \right) (\tfrac{d}{2}) \subset H^d_B(A_q \otimes_Q E) (\tfrac{d}{2}) \]
are indeed absolutly Hodge. (ASK HARRIS ABOUT THIS PROOF)? I don't understand it. 
\end{proof}

\subsection{Riemann Forms}

\begin{definition}
A Hermitian form $H$ on a complex vectorspace $V$ is a complex bilinear form $H : \overline{V} \times V \to \C$ (sesquilinear on $H$) which satisfies,
\[ H(u, v) = \overline{H(v, u)} \]
\end{definition}

\begin{lemma}
Let $V$ be a complex vectorspace. There is a one-to-one correspondence between Hermitian forms $H$ on $V$ and real-valued skew-symmetric forms $E$ on $V$.
\end{lemma}

\begin{proof}
The correspondence is given by,
\begin{align*}
H & \mapsto E_H \quad \quad E_H(u,v) = \Im{H(u,v)} 
\\
E & \mapsto H_E \quad \quad H_E(u,v) = E(iu, v) + i E(u, v) 
\end{align*}
\end{proof}

\begin{definition}
A Riemann form $E : V \times V \to \R$ on a complex vectorspace $V$ is an antisymmetric $\R$-bilinear form such that,
\begin{enumerate}
\item $E(i u, i v) = E(u, v)$ 
\item the corresponding Hermitan form $H_E$ is positive definite.
\end{enumerate}
\end{definition}

\begin{definition}
A complex torus $X = V / \Lambda$ is \textit{polarizable} if there exists an antisymmetric form $E : \Lambda \times \Lambda \to \Z$ such that $E_\R : V \times V \to \R$ (using that $V = \Lambda \otimes_\Z \R$) is a Riemann form.  
\end{definition}

(IS THIS EQUIVALENT TO THE POLARIZATION OF THE HODGE STRUCTURE $H_1(X, \Q)$)

\begin{theorem}
A complex torus $X = V / \Lambda$ is of the form $A^\an$ for some abelian variety $A$ iff $X$ is polarizable.
\end{theorem}

(DOES THIS IMPLY THAT ALL ABELIAN VARIETIES ARE POLARIZABLE IN THE FOLLOWING SENSE)

\begin{definition}
A polarization of an abelian variety $A$ is an isogeny $\lambda : A \to A^\vee$ such that 
\end{definition}

\begin{remark}
We can identify, $A^\vee = \mathrm{Pic}^0\left( A \right)$.
\end{remark}

\begin{proposition}
For each line bundle $\L$ on $A / k$ there is an associated morphism $\phi_\L : A \to A^\vee$ which is an isogeny if $\L$ is ample.
\end{proposition}

\begin{proof}
We define a map $\phi_\L : A(\overline{k}) \to \Pic{A}$ via $\phi_\L(x) = t^*_x \L \otimes_{\struct{A}} \L^{-1}$. First, via the Theorem of the Square, for $x,y \in A(\overline{k})$,
\[ t_x^* \L \otimes_{\struct{A}} t_y^* \L = t^*_{x + y} \L \otimes_{\struct{A}} \L \]
Therefore,
\[ (t_x^* \L \otimes_{\struct{A}} \L^{-1}) \otimes_{\struct{A}} (t_y^* \L \otimes_{\struct{A}} \L^{-1}) = t^*_{x + y} \L \otimes_{\struct{A}} \L^{-1} \]
so $\phi$ is a group homomorphism. Furthermore, $\deg{t_x^* \L} = \deg{\L}$ since the map $t_x : A \to A$ is an isomorphism. (IS THIS TRUE?) Therefore, $\deg{\phi_\L(x)} = 0$ so the image is contained in $\mathrm{Pic}^0(A) = A^\vee(\overline{k})$. 
\end{proof}

\begin{definition}
A polarization of $A$ is an isogeny $\phi : A \to A^\vee$ such that $\phi_{\overline{k}} : A_{\overline{k}} \to A_{\overline{k}}^\vee$ is of the form $\phi_{\L}$ for some ample line bundle $\L$ on $A_{\overline{k}}$. Deriving from a line bundle gives symmetry $\phi = \phi^\vee$ and ampleness is a positivity condition. 
\end{definition}

\begin{definition}
Let $A$ be an abelian variety with a polarization $\phi : A \to A^\vee$. Since $\phi$ is an isogeny, it has an ``inverse element'' in the algebra $\phi^{-1} \in \Hom{}{A^\vee}{A} \otimes \Q$. (This follows from inverting the multiplication by $n$ maps). Then we define the Rosati involution of the endomorphism algebra $\End{A} \otimes_\Z \Q$ via,
\[ \alpha^\dagger = \phi^{-1} \circ \alpha^\vee \circ \phi \quad \text{for} \quad \alpha \in \End{A} \otimes_\Z \Q \]  
\end{definition}

\begin{remark}
The Rosati involution depends on the choice of polarization. 
\end{remark}

\begin{theorem}
A polarization $\theta$ on $A$ is determined by a Riemann form $\phi$ on $H_1(A^\an, \Q)$. Two forms $\phi, \phi'$ determine the same polarization iff $\exists a \in \Q^\times : \phi' = a \phi$. In this case, the Rosati involution is determined by,
\[ \forall u,v \in H_1(A^\an, \Q) :   \phi(\alpha(u), v) = \phi(u, \alpha^\dagger(v)) \quad \quad \alpha \in \End{A} \otimes_\Z \Q \]
\end{theorem}

\begin{proof}
(HOW DOES ONE PROVE THIS?)
\end{proof}

\begin{theorem}
Let $A$ be an abelian variety over $\C$ and $\nu : E \to \End{A} \otimes_\Z \Q$ the inclusion of a CM-field with $d = \dim_E H^1(A^\an, \Q)$. Suppose there exists a polarization $\theta$ for $A$ such that,
\begin{enumerate}
\item the Rosati involution of $\theta$ induces complex conjugation on $E \subset \End{A} \otimes_\Z \Q$
\item ther exists a split $E$-Hermitian form $\phi$ on $H_1(A^\an, \Q)$ and $f \in E^\times$ with $\overline{f} = - f$ such that $\phi(x, y) = \Tr_{E/\Q}(f\phi(x,y))$ is a Riemann form for $\theta$.
\end{enumerate}
Then the subspace,
\[ \left( \exterior^d_E H^1_B(A) \right) (\tfrac{d}{2}) \subset H^d(A^\an, \Q) (\tfrac{d}{2}) \]
consists of absolute Hodge cycles. 
\end{theorem}

\subsection{Shimura Varieties}

\section{The Proof for Abelian Varieties of CM Type}

\begin{definition}
The Mumford-Tate group $M(A)$ of an abelian variety $A$ is the Mumford tate group of the rational Hodge structure $H_1(A, \Q)$. 
\end{definition}

\begin{definition}
An abelian variety is of CM-type if $M(A)$ is abelian.  
\end{definition}

\begin{remark}
Any abelian variety $A$ is isogenous to a product of simple abelian varieties $A_\alpha$ and $A$ is CM-type iff each $A_\sigma$ is CM-type since the Mumford-Tate group of the product $M(A)$ is contained in the product of $M(A_\alpha)$ and projects fully onto each. Therefore, it will suffice to study simple abelian varieties of CM-type. 
\end{remark}

\begin{lemma}
Let $A$ be an abelian variety. Then $\End{A} \otimes_\Z \Q$ is isomorphic to the subalgebra of elements in $\End{H_1(A^\an, \Q)}$ preserving the Hodge structure. Furthermore, preserving the Hodge structure is equivalent to commuting with the image of $\mu : \Gm \to \GL{}{H_1(A^\an, \C)}$. 
\end{lemma}

\begin{proof}
(PROVE THIS)
\end{proof}

\begin{proposition}
A simple abelian varity over $\C$ is of CM-type iff $E = \End{A} \otimes_\Z \Q$ is a commutative field over which $H_1(A, \Q)$ has dimension $1$. In this case, $E$ is a CM-field and the Rosati involution on $E$ for any polarization of $A$ is complex conjugation on $E \subset \End{A} \otimes_\Z \Q$. 
\end{proposition}

\begin{proof}
Let $A$ be an abelian variety with $E \subset \End{A} \otimes_\Z \Q$ such that, 
\[ \dim_{E} H_1(A, \Q) = 1 \]
Then $\mu(\Gm)$ commutes with $E \otimes \R$ in $\End{H_1(A^\an, \R)}$ because the Hodge structure is compatible with the $E$-vectorspace structure. (WHY THOUGH)
The subspace $(E \otimes_\Q \R) \subset \GL{}{H_1(A^\an, \R)}$ is all diagonal matricies (since $H_1(A^\an, \R)$ is dimension one over $E$) and since anything that commutes with all diagonal matrices must itself be diagonal, we have $h(\S) \subset (E \otimes_\Q \R)^\times$ which implies that $M(A) \subset \mathbb{G}_{E^\times}$ where $\mathbb{G}_{E^\times} \subset \GL{}{H_1(A^\an, \Q)}$ is the commutative $\Q$-algebraic subgroup defined by $\mathbb{G}_{E^\times}(F) = (E \otimes_\Q F)^\times$ and thus whose $\R$-points are $(E \otimes_\Q \R)$ containing $h(\S)$. Therefore $M(A) \subset \mathbb{G}_{E^\times}$ is abelian since $\mathbb{G}_{E^\times}$ is a commutative group scheme. (I believe that $\mathbb{G}_{E^\times} = \Res_{E / \Q} (\Gm^E)$ IS THIS CORRECT?)
\bigskip\\
Conversely, let $A$ be simple and of CM-type an $\mu : \Gm \to \GL{}{H_1(A^\an, \C)}$ define the Hodge structure on $H_1(A^\an, \C)$. Since $A$ is simple, $E = \End{A} \otimes_\Z \Q$ is a division ring of degree $\le \dim_{\Q}{H_1(A^\an, \Q)}$ over $\Q$. 
(COMPLETE THIS PROOF!?)
\end{proof}


\subsection{The Proof For CM Case}

Let $A_\alpha$ be a finite family of abelian varieties of CM-type. We need to show that every Hodge cycle in,
\[ T_{\A} = \left( \bigotimes_\alpha H^1_\A(X_\alpha)^{\otimes m_\alpha} \right) \otimes \left( \bigotimes_\alpha H^1_{\A}(X_\alpha)^{\vee \otimes n_\alpha} \right)(m) \]
is an absolute Hodge cycle. According to Principal A the group $G^{AH}$ fixing all absolute Hodge cycles fixes exactly the absolute Hodge cycles. Thus it suffices to prove that the subgroup $G^H \subset G^{AH}$ fixing all Hodge cycles is equal to $G^{AH}$. 


\section{Proof of the Main Theorem}

Let $A$ be an abelian variety over $\C$ and $t_\alpha$ for $\alpha \in I$ be Hodge cycles on $A$. We need to show that these are absolute Hodge cycles. Since we know the result in the case that $A$ is CM-type it suffices to prove the following.

\begin{proposition}
There exists a connected smooth algebraic variety $S / \C$ and an abelian scheme $\pi : Y \to S$ such that,
\begin{enumerate}
\item for some $s_0 \in S$ the fibre $Y_{s_0} = A$
\item for some $s_1 \in S$ the fibre $Y_{s_1}$ is of CM-type
\item the cycles $t_\alpha$ extend to rational cycles of bidegree $(0,0)$ on $Y$. Explicitly, suppose that,
\[ t_\alpha \in H_B^1(A)^{\otimes m(\alpha)} \otimes H_B^1(A)^{\vee \otimes n(\alpha)} \]
then there is a section $t$ of,
\[ (R^1 \pi_* \underline{\Q})^{\otimes m(\alpha)} \otimes (R^1 \pi_* \underline{\Q})^{\otimes n(\alpha)} \]
over a finite cover $\tilde{S} \to S$ such that for some $\bar{s}_0$ over $s_0$ we have $t_{\bar{s}_0} = t_\alpha$ and for all $\tilde{s} \in \tilde{S}$ we have, 
\[ t_{\tilde{s}} \in H^1_B(Y_{\tilde{s}})^{\otimes m(\alpha)} \otimes H^1_B(Y_{\tilde{s}})^{\vee \otimes n(\alpha)} \]
is a Hodge cycle. 
\end{enumerate}
\end{proposition}

\newcommand{\Gr}{\mathrm{Gr}}
\newcommand{\Lie}{\mathrm{Lie}}

\begin{proof}
$S$ will be a Shimura Variety. Extend the set of AH cylces such that some $t_\alpha$ is a polarization of $A$ and let $H = H_1(A, \Q)$. Now we consider $G \subset \GL{H} \times \Gm$ fixing $t_\alpha$. Since the hodge character must act trivially on $t_\alpha$ then it defines a character $h_0 : \C^\times \to G(\R)$. 
\bigskip\\
Define,
\[ X = \{ h : \C^\times \to G(\R) \mid h \text{ is conjugate to } h_0 \in G(\R) \} \]
For each $h \in X$ we get a new Hodge structue of $H$ relative to which $t_\alpha$ has bidegree $(0,0)$ since $h$ fixes it. Let $F^0(h) = H^{0,-1} \subset H \otimes \C$ in this new Hodge structue. Sending $h \mapsto F^0(h)$ is a map $X \to \Gr_k(H \otimes \C)$ as real manifolds. The map is injective becaues the filtration completely determines a hodge structure. Consider the centralizer $K_\infty$ of $h_0$. Then,
\begin{center}
\begin{tikzcd}
T_{h_0} (X) \arrow[r, equals] & \Lie(G_\R) / \Lie(K_\infty) \arrow[r, hook] \arrow[d, equals] & \End{H \otimes \C} / F^0 \End{H \otimes \C} \arrow[r, equals] & T_{\phi(h_0)} \Gr_k(H \otimes \C)
\\
& \Lie(G_\C) / F^0 \Lie(G_\C) \arrow[ru, hook]
\end{tikzcd}
\end{center}
where the Filtration on $\End{H \otimes \C}$ is given by the Hodge structure $h_0$ on $H$. Then, $X$ is a complex manifold. 
\bigskip\\
To each $h \in X$ we attach a complex torus given by the double cosets $F^0(h) \setminus H \otimes \C / H(\Z)$ where $H(\Z)$ is a fixed lattice inside $H$. In particular, at $h_0$ we get,
\[ F^0(h_0) \setminus H \otimes \C / H(\Z) = T_0(A) / H(\Z) \]
These tori form a family $B \to X$. 
Then define the group,
\[ \Gamma_n = \{ g \in G(\Q) \mid (g - q) H(\Z) \subset n H(\Z) \} \]
for some. For sufficiently large $n$ Baily and Borel show that $S = X / \Gamma$ is an algebraic variety, in particular a Shimura variety. 
\end{proof}


\section{Ideal for Next Semester}

That paper on Slopes of powers of Frobenius on crystalline cohomology.

Course on crystalline cohomology.

Course on Shimura varieties.

Study supersingular curves or K3 surfaces.

\end{document}