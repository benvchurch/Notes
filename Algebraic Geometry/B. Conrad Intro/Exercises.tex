\documentclass[12pt]{article}
\usepackage{import}
\import{../}{AlgGeoCommands}


\begin{document}

\section{Homework 1}

Chapter I
\begin{enumerate}
\item 1.1
\item 1.2
\item 1.3
\item 1.5
\item 1.6
\item 1.7
\item 1.10
\item 1.11
\item 1.12
\end{enumerate}

\section{Homework 2}

Chapter II
\begin{enumerate}
\item 3.2
\item 3.3
\item 5.1
\item 5.2
\item 5.10
\end{enumerate}

\subsection{A}

\subsection{B}

\subsection{C}

\section{Homework 3}

\subsection{1}

\subsection{2}

\subsection{3}

\section{Homework 4}

Chapter II
\begin{enumerate}
\item 1.6
\item 1.7
\item 1.8
\item 1.9
\item 1.10
\item 1.11
\item 1.12
\item 1.15
\item 1.16
\item 1.18
\item 1.22
\end{enumerate}

\subsection{A}

\section{Homework 5}
Chapter II
\begin{enumerate}
\item 2.3
\item 2.4
\item 2.5
\item 2.6
\item 2.7
\item 2.8
\item 2.9
\item 2.11
\item 2.13
\item 2.17
\item 2.18
\item 2.19
\end{enumerate}
See homework for some hints.

\subsection{A}

\subsection{B}

\section{Homework 6}
TODO
\begin{enumerate}
\item 3.8
\item 3.11
\item 3.17
\item 3.18
\item 3.19
\item 3.20
\end{enumerate}

\subsection{A}

\subsection{B}

\section{Homework 7}

\subsection{A}

\subsection{B}

\subsection{C}

\subsection{D}

(ASK BRIAN ABOUT THIS TOMORROW HOW TO NOT USE ETALENESS)
(WHY CANT WE MAKE THE FOLLOWING ARGUMENT: 

we require that $A$ is a separable algebra and then every quotient is a separable algebra so in particular the residue fields are separable and thus each closed point should have residue field contained in separable closure by nullstellensatz (its a finite extension of groud field) and thus the $k^\sep$-points are dense because the closed points are dense for any locally finite type $k$-scheme.
\bigskip\\
Let $X$ be a scheme locally of finite type over a separably closed field $k$ such that $X$ is \textit{geometrically} integral over $k$. It suffices to show that every affine open $\Spec{A} \subset X$ contains a $k$-point. Since $A$ is geometrically irreducible we see that $K = \Frac{A}$ satisfies $K \otimes_k k^{\frac{1}{p}}$ is reduced and thus by [Mat, Thm. 26.2] admits a separating transcendence basis. 
\bigskip\\
Suppose that $X$ is only required to be integral. Then we may take $k = \FF_p(t)^\sep$ and $X = \Spec{\bar{k}}$ then $X$ has no $k$-points because $\bar{k}$ is a strict extension of $k$ but is clearly integral.
\bigskip\\
Suppose $X$ is required to be geometrically integral but $k$ is not separably closed. Then we may take $k = \RR$ and $X = V(x^2 + y^2 + z^2) \subset \P^2_{\RR}$ which is a conic with no $\RR$-points such that $X_{\CC} \cong \P^1_{\CC}$.

\end{document}