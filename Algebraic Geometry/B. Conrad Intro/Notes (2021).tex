\documentclass[12pt]{article}
\usepackage{import}
\import{../}{AlgGeoCommands}


\begin{document}

\section{Sep. 20}

\subsection{History}

\begin{enumerate}
\item 19th century
\begin{enumerate}
\item $Z(f_1, \dots, f_n) \subset \CC^n$ using analytic tools
\item Riemann's idea of moduli of algebraic curves (1857)
\end{enumerate}
\item 20th century
\begin{enumerate}
\item $Z(f_1, \dots, f_n) \subset \CC^n$ using algebraic tools (commutative algebra)
\item replace $\CC$ with algebraically closed field $k$
\item number theory: want $k = \F_p$ or $\Q$ not algebraically closed. Examples: Fermat's Last Theorem:
\[ u^n + v^n = 1 \]
want geometry for,
\[ \Q[u,v]/(u^2 + v^2 - 1) \]
but this only has finitely many points so how is there a ``geometry''. 
\end{enumerate}
\end{enumerate}

\subsubsection{Question: for any field $k$ is there a ``geometry'' for $k[X_1, \dots, X_n]/I$?}

First attempt (Weil and Zariski 1930s - 1940s) use Galois theory with algebraic sets in $\overline{k}^n$ for ideals $I \subset k[X_1, \dots, X_n]$. This only works for perfect fields (not $\F_p(t)$ which we want to consider generic families of equations over $\Z$). Weil's Foundations of Algebraic Geometry.

\subsubsection{The Weil Conjectures (1948)}

For $f_1, \dots, f_r \in \FF_p[x_1, \dots, x_n]$ then define,
\[ N_m = \{ x \in \FF_{p^m}^r \mid f_1(x) = \cdots = f_r(x) = 0 \} \]
Then we define a Zeta function,
\[ \zeta(s) = \exp{ \left( \sum_{m = 1}^\infty \frac{N_m}{m} p^{-sm} \right) } \]
which sould controll the behavior of $N_m$ as $m \to \infty$. Furthermore,
\[ Z_\CC(F_1, \dots, F_r) \subset \CC^n \quad \text{ and } \quad Z_p = \Z(F_1, \dots, F_r) \subset \overline{\FF_p}^n \]
are closely related where algebraic topology invariants of $Z_\CC$ gives formulas for counts of $Z_p$. 

\subsubsection{1950's: Chaos (K\"{a}hler, Shimura, Nagata, etc.)}

Proposal for algebraic geometry over Dedekind domains but chaotic and confusing. Then Schemes resolve all of these problems to give ``geometry over any commutative ring''. 

\subsection{Affine Algebraic Sets}

Let $k$ be an algebraically closed field. Let $\A^n = k^n$ and define a subset $Z \subset \A^n$ to be \textit{algebraic} if $Z = Z(\Sigma)$ where $\Sigma \subset k[X_1, \dots, X_n]$ is a set of polynomials. Then $Z(\Sigma) = Z(I)$ where $I$ is the ideal generated by $\Sigma$.

\begin{thm}
The algebraic sets form (the complement of) a topology on $\A^n$.
\end{thm}

\begin{rmk}
We call this the Zariski topology.
\end{rmk}
\noindent
There is a base of open sets given by,
\[ U_f = \{ x \in \A^n \mid f(x) \neq 0 \} \]

\subsubsection{Examples}

The Zariski topology on $\A^1$ has the cofinite topology. However, $\A^2 \neq \A^1 \times \A^1$ as a topological space. 

\begin{rmk}
Some weird properties of the Zariski topology,
\begin{enumerate}
\item In $\CC^n$ any nonzero open ball is Zariski dense.
\item $Z(f) = Z(f^n)$ and $Z(I) = Z(\sqrt{I})$.
\end{enumerate}
\end{rmk}

\begin{defn}
For any $Y \subset \A^n$ define,
\[ I(Y) = \{ f \in k[X_1, \dots, X_n] \mid \forall y \in Y : f(y) = 0 \} \]
which is a radical ideal.
\end{defn}

\begin{prop}[Nullstellensatz]
For a field $k$ and $\m \subset K[X_1, \dots, X_n]$ a maximal ideal. Then $K[X_1, \dots, X_n]/\m$ is a finite dimensional $K$-vector space.
\end{prop}

\begin{proof}
210B [Mat, Thm 5.3]
\end{proof}

\begin{cor}
If $K$ is algebraically closed then $K \to K[X_1, \dots, X_n]/\m$ is an isomorphism so we have $a_i \mapsto X_i$ and thus $X_i - a_i = 0$ in the quotient so,
\[ \m = (X_1 - a_1, \dots, X_n - a_n) \]
is the kernel of the map $K[X_1, \dots, X_n] \to K[X_1, \dots, X_n]/\m$. Therefore, points in $Z(J)$ correspond to $\m \supset J$. Therefore,
\[ I(Z(J)) = \bigcap_{\m \supset J} \m \]
\end{cor}

\begin{thm}
The following hold,
\begin{enumerate}
\item $I_1 \subset I_2 \implies Z(I_1) \supset Z(I_2)$
\item $Y_1 \subset Y_2 \implies I(Y_1) \supset I(Y_2)$
\item $I(Y_1 \cup Y_2) = I(Y_1) \cap I(Y_2)$
\item $Z(I_1 \cap I_2) = Z(Y_1) \cup Z(Y_2)$
\item $Z(I(Y)) = \overline{Y}$
\item $I(Z(J)) = \sqrt{J}$ (Hilbert's Nullstellensatz). 
\end{enumerate}
\end{thm}

\begin{proof}
The first three follow directly from definitions. Suppose that $x \notin Z(I_1)$ and $x \notin Z(I_2)$ then there is some $f_i \in I_i$ such that $f_i(x) \neq 0$ so $f_1(x) f_2(x) \neq 0$ but $f_1 f_2 \in I_1 \cap I_2$ so $x \notin Z(I_1 \cap I_2)$. 
\bigskip\\
Now $Y \subset Z(I(Y))$ so $\overline{Y} \subset Z(I(Y))$. Pick $x \notin \overline{Y}$ so there is some $x \in U_f$ such that $U_f \cap \overline{Y} = \empty$. Therefore, $U_f \cap Y = \empty$ so $f|_Y = 0$ since if $x \in Y$ then $x \notin U_f$. Thus $f \in I(Y)$ so $x \notin Z(I(Y))$ proving that $Z(I(Y)) \subset \overline{Y}$.
\bigskip\\
Since $J \subset I(Z(J))$ is radical we see that $\sqrt{J} \subset I(Z(J))$. The key is to apply the Nullstellensatz.
\end{proof}

\section{Sep. 22}

\begin{prop}
Let $K$ be a field and $A$ a finitely generated $K$-algebra, $J \subset A$ an ideal. Then,
\[ \sqrt{J} = \bigcap_{\m \supset J} \m = \Jac{J} \]
\end{prop}

\begin{proof}
Replace $A$ by $A / \sqrt{J}$ such that $J = (0)$ and $\nilrad{A} = (0)$. Choose $f \neq 0$ then $f$ is not nilpotent so $A_f$ is nonzero and $A_f = A[x]/(xf - 1)$ is a finitely generated $K$-algebra. Thus $A_f$ has a maximal ideal $\m \subset A_f$. Now under $\varphi : A \to A_f$ we see that $\varphi^{-1}(\m) \subset A$ is a prime. However, $A / \varphi^{-1}(\m) \embed A_f / \m$ but $A_f / \m$ is a finite field extension of $K$ so $A / \varphi^{-1}(\m)$ is a finite dimensional $K$-algebra and a domain so its a field and thus $\varphi^{-1}(\m)$ is maximal and $f \notin \varphi^{-1}(\m)$ and thus $f \notin \Jac{A}$.  
\end{proof}

\begin{rmk}
Any domain $D$ that is a finite dimensional $K$-algebra is a field because if $r \in D$ is nonzero then $D \xrightarrow{\times r} D$ is injective and thus surjective so $xr = 1$ for some $x \in D$ so $D$ is a field.
\end{rmk}

\begin{rmk}
Usually difficult to compute $\sqrt{J}$ given generators of $J$. 
\end{rmk}

\begin{defn}
Say that $f \in k[x_1, \dots, x_n]$ is \textit{radical} if $f \in k$ and no repeated irreducible factors. The hypersurface $Z(f)$ for non-constant $f$ is radical.
\end{defn}

\begin{defn}
A topological space $Y$ is \textit{irreducible} if $Y \neq \empty$ and $Y \neq Y_1 \cup Y_2$ for closed $Y_1, Y_2 \subsetneq Y$. Otherwise, $Y$ is \textit{reducible}.
\end{defn}

\begin{rmk}
If $Y$ is irreducible then every nonempty open $U \subset Y$ is dense. This is because $Y = (Y \setminus U) \cup \overline{U}$ but if $U$ is nonempty then $Y \setminus U$ is a proper subset so $\overline{U} = Y$.
\end{rmk}

\begin{defn}
A topological space $Y$ is \textit{noetherian} if it satisfies the DCC for closed sets meaning if,
\[ Z_1 \supset Z_2 \supset Z_3 \supset \cdots \]
is a descending chain then it stabilizes menaing $Z_n = Z_{n + 1}$ for all sufficiently large $n$.
\end{defn}

\begin{example}
$\A^n$ is noetherian because closed sets correspond to ideals and $k[x_1, \dots, x_n]$ is noetherian.
\end{example}

\begin{prop}
Let $Z \subset \A^n$ be an algebraic set. Then $Z$ is irreducible if and only if $I(Z)$ is prime.
\end{prop}

\begin{proof}
Irreducibles are nonempty and prime ideals $I$ are proper subsets. Thus consider the case that $I(Z) \neq (1)$ equivalently that $Z$ is nonempty. We see that,
\[ Z = Z_1 \cup Z_2 \iff I(Z) = I(Z_1) \cap I(Z_2) \]
and $Z_i \subsetneq Z$ iff $I(Z_1) \supsetneq I(Z)$. Therefore, irreducibility of $Z$ is equivalent to the condition that if $I(Z) = I_1 \cap I_2$ with $I_1$ and $I_2$ radical then $I_1 = I(Z)$ or $I_2 = I(Z)$ which is equivalent to in $A = k[x_1, \dots, x_n] / I(Z)$ the property that if $(0) = J_1 \cap J_2$ then either $J_1 = (0)$ or $J_2 = (0)$. Therefore, we reduce to showing the following: if $A$ is a nonzero reduced ring, then $A$ is a domain iff $J_1 \cap J_2 = (0)$ for radical ideals $J_1, J_2$ then either $J_1 = (0)$ or $J_2 = (0)$.
\bigskip\\
If $A$ is a domain then $J_1 J_2 \subset J_1 \cap J_2 = (0)$ so if $a_i \in J_i$ are nonzero then $a_1 a_2 \in J_1 J_2$ so $a_1 a_2 = 0$ contradicting the fact that $A$ is a domain. Now suppose that $A$ has this property. Choose $f,g \in A$ such that $fg = 0$ then let $Q = \sqrt{(f)} \cap \sqrt{(g)}$. If $a \in Q$ then $a^n = pf$ and $a^m = qg$ so $a^{n+m} = pq fg = 0$ and thus $a \in \nilrad{A}$ but $A$ is reduced so $Q = (0)$ and thus either $f = 0$ or $g = 0$ by the assumption.
\end{proof}

\begin{cor}
If $f$ is radical then $Z(f)$ is irreducible iff $f$ is irreducible.
\end{cor}

\begin{proof}
Both are equivalent to $(f)$ being prime.
\end{proof}

\begin{thm}
Every noetherian topological space is a finite union of irreducible closed sets,
\[ Y = Y_1 \cup \cdots \cup Y_r \]
which is unique if we require the irredudency,
\[ Y_i \not\subset \bigcup_{j \neq i} Y_i \]
Furthermore, in the irredudent case, the $Y_i$ are exactly the maximal irreducible subsets (i.e. irreducible components). 
\end{thm}

\begin{cor}
Every algebraic set $Z$ is a finite union of irreducible closed subsets.
\end{cor}

\begin{defn}
An \textit{affine variety} is a irreducible algebraic set.
\end{defn}

\section{Dimension and Regular Functions}

\begin{lemma}
If $Y \subset X$ is irreducible in the subspace topology and $Y \subset Z_1 \cup Z_2$ for closed $Z_j \subset X$ then $Z \subset Z_1$ or $Y \subset Z_2$.
\end{lemma}

\begin{proof}
Then $Y = (Y \cap Z_1) \cap (Y \cap Z_2)$.
\end{proof}

\begin{rmk}
This is why for an irredundant decomoposition,
\[ X = Z_1 \cup \cdots \cup Z_n \]
into its irreducible components then every irreducible $Y \subset X$ lies inside some $Z_i$. Therefore, the $Z_i$ are indeed maximal irreducible subsets.
\end{rmk}

\begin{defn}
The (combinatorial) \textit{dimension} of a topological space $X$ is,
\[ \dim(X) = \sup \{ n \ge 0 \mid Z_0 \subsetneq Z_1 \subsetneq \cdots \subsetneq Z_n \subset X \text{ of irreducible closed } Z_j \subset X \} \]
Furthermore we set $\dim(\empty) = -\infty$.
\end{defn}

\begin{rmk}
We may have $\dim(X) = \infty$.
\end{rmk}

\begin{defn}
For a commutative ring $A$,
\[ \dim{A} = \sup\{ n \ge 0 \mid \p_0 \subsetneq \p_1 \subsetneq \cdots \subsetneq \p_n \subset A \text{ for prime } \p_j \subset A \} \]
and we set $\dim{(0)} = - \infty$.
\end{defn}

\begin{defn}
For $Y \subset \A^n$ an algebraic set, we define the coordinate ring $k[Y] := k[x_1, \dots, x_n]/I(Y)$. Notice this depends on the embedding into affine space not necessarily the intrinsic structure of $Y$.
\end{defn}

\begin{rmk}
We see that there are inclusion reversing equivalences,
\[ \{ \text{Radical ideals of } k[Y] \} \iff \{ \text{closed subsets } Z \subset Y \} \]
and likewise,
\[ \{ \text{Prime ideals of } k[Y] \} \iff \{ \text{irreducible closed subsets } Z \subset Y \} \]
Therefore,
\[ \dim{Y} = \dim{k[Y]} \]
\end{rmk}

\begin{rmk}
For irreducible closed $Z \subset X$ where $X$ is an affine algebraic set, does there exist a maximal length chain,
\[ Z_0 \subsetneq Z_1 \subsetneq \cdots \subsetneq Z_d \subset X \]
with some $Z_j = Z$? If $X$ is reducible then the answer is no because we can have irreducible components of different dimensions. However, for irreducble algebraic sets the answer is yes. 
\end{rmk}

\begin{thm}
Let $B$ be a domain finitely generated over a field $k$. Then,
\begin{enumerate}
\item $\dim{B} = \trdeg{k}{\Frac{B}}$ which is, in particular, finite
\item For any prime $\p \subset B$,
\[ \dim{B} = \dim{B/\p} + \dim{B_\p} \]
\end{enumerate}
\end{thm}

\begin{rmk}
We interpret the second part of this theorem as follows. The primes of $B / \p$ are exactly the primes containing $\p$ and thus we consider maximal chains,
\[ \p = \p_0 \subsetneq \cdots \subsetneq \p_n \]
and the primes of $B_\p$ are exactly the primes contained in $\p$ and thus we consider maximal chains,
\[ \p = \q_m \supsetneq \cdots \subsetneq \q_0 \]
and thus splicing them together, by the theorem, gives a maximal length chain in $B$ containing $\p$.
\end{rmk}

\begin{proof}
For (a) [Mat. Thm. 5.6] for (b) [Mat, Ex. 5.1] (the solution is in the back of the book and uses part (a) and induction for non-algebraically closed fields even if you only care about the case of algebrcially closed fields).  
\end{proof}

\begin{theorem}[Krull]
For all local noetherian rings, $\dim{A} < \infty$.
\end{theorem}

\begin{proof}
See [Mat, Thm. 13.5] and [AM, Cor. 11.11].
\end{proof}

\begin{cor}
$\dim{\A^n} = \dim{k[x_1, \dots, x_n]} = \trdeg{k}{k(x_1, \dots, x_n)} = n$
\end{cor}

\begin{cor}
Let $Z \subset \Z^n$ be irreducble and closed. Then,
\begin{enumerate}
\item For nonempty open $U \subset Z$ (so $\overline{U} = Z$ because $Z$ is irreducible) then $\dim{U} = \dim{Z}$
\item If $Z = Z(f)$ for irreducible $f$ then $\dim{Z} = n - 1$
\item For each $x \in Z$ we have $\dim{k[Z]_{\m_z}} = \dim{Z}$.
\end{enumerate}
\end{cor}

\begin{rmk}
$\dim{k[Z]_{\m_z}}$ corresponds to chains of irreducibles beginning at $Z_0 = \{ z \}$.
\end{rmk}

\begin{proof}
(c) is immediate. Now we do (a). We see that $U$ is irreducible because $\overline{U} = Z$ since $Y \subset U$ is closed then $\overline{Y} \cap U = Y$. Suppose that,
\[ Y_0 \subsetneq \cdots \subsetneq Y_n \subset U \]
is a chain of closed irreducible subsets of $U$ then,
\[ \overline{Y}_0 \subsetneq \cdots \subsetneq \overline{Y}_n \subset Z \]
is a chain of closed irreducible subsets of $Z$ since $\overline{Y}_i \cap U = Y_i$ so the containments are proper. Therefore, $\dim{U} \le \dim{Z}$. Now, by the previous theorem, we can choose a maximal chain such that $Z_0 = \{ z \}$ with $z \in U$ (the point can be chosen arbitrarily) and get,
\[ Z_0 \subsetneq \cdots \subsetneq Z_n = Z \]
so take $Y_j = Z_j \cap U$ which is clearly closed in $U$ and irreducible. Since $z \in U \cap Z_j$ we see that $Y_j$ is nonempty but open in $Z_j$ and thus $\overline{Y}_j = Z_j$ and thus the containments must be proper since $Z_j \subsetneq Z_{j+1}$. 
\bigskip\\
Finally to show (b) we apply the dimension formula,
\[ \dim{Z(f)} + \height{(f)} = n \]
so it suffices to prove that $(0) \subsetneq (f)$ is a minimal nonzero prime. However, for any nonzer $\p \subset (f)$ take an irreducible element $g \in \p$ (factor any element and by primality its irreducible factors are inside $\p$) and thus $(g) \subset \p \subset (f)$ so $g = fr$ but $g$ is irreducible and $f$ is not a unit so $r$ is a unit and thus $(g) = (f)$ so $\p = (f)$.
\end{proof}

\begin{prop}
Let $A$ be a Noetherian domain and suppose that $f \in A$ is nonzero and $(f)$ is prime. Then $(f)$ is a minimal nonzero prime.
\end{prop}

\begin{proof}
 then take $x \in \p$ so $x = fr$ but $f \notin \p$ so $r \in \p$ and thus $f \p = \p$. Thus if $\p$ is finitely generated (it is because we are in a Noetherian ring) then there is $r \in (f)$ such that $(r - 1) \p = 0$ by Nakayama but in a domain this implies $\p = 0$ because $r - 1 \neq 0$.
\end{proof}

\begin{rmk}
The closed sets $Z \subset \A^n$ whose irreduclbe components are all of dimension $n - 1$ are \textit{exactly} $Z = Z(f)$ fo nonconstant $f \in k[x_1, \dots, x_n]$ (look at irreduclbe components $Z_j = Z(f_j)$).
\end{rmk}

\subsubsection{``Nice'' functions on algebraic sets}

We have $k[Z] = k[x_1, \dots, x_n] / I(Z) \embed \mathrm{Func}(Z, k)$ by sending $g \mapsto (z \mapsto g(z))$ because these functions by definition do not care about polynomials that vanish on $Z$. Consider $U = Z_f \subset Z$ then we get $\alpha_f : k[Z]_f \to \mathrm{Func}(Z_f, k)$ because $f$ is nonvanishing on $U$ and thus $f^{-1}$ makes sense as a function. 

\begin{defn}
For any open $U \subset Z$ nonempty we define,
\[ \struct{Z}(U) = \{ \varphi : U \to k \mid \forall u \in U : \exists u \in V \subset U : \varphi|_V = \tfrac{g}{h} \text{ for } g,h \in k[Z] \text{ and } h|_V \text{ nonvanishing } \} \]
\end{defn}

\begin{prop}
The map $\alpha_f : k[Z]_f \iso \struct{Z}(Y)$ is an isomorphism.
\end{prop}

\end{document}