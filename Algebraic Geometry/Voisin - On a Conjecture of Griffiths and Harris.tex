\documentclass[12pt]{article}
\usepackage{import}
\import{./}{AlgGeoCommands}

\begin{document}

\newcommand{\g}[2]{\mathfrak{g}^{#1}_{#2}}

\section{On a Conjecture of Griffiths and Harris}

In [4], Grifiths and Harrised proposed the following range of conjectures concerning curves contained in a general hypersurface in $\P^4$ of degree $d \ge 6$ (such that the canonical bundle is ample)
\begin{enumerate}
\item we have $d$ divides $\deg{C}$
\item the image of the Abel-Jacobi map
\[ \varphi : \mathrm{Hom}^2(X) / \mathrm{Rat}^2(X) \to J^2(X) \]
of $C$ is trivial
\item the group
\[ \mathrm{Hom}^2(X) / \mathrm{Alg}^2(X) \]
is trivial
\item the group
\[ \mathrm{Alg}^2(X) / \mathrm{Rat}^2(X) \]
is trivial
\item if $C$ is smooth, then $C$ is a complete intersection with a surface $\Sigma \subset \P^4$. 
\end{enumerate}
Mark Green has explained some recent progress on (b), this note proposes to show that (e) is false, we show this by showing that the variant (e'), is also false,
\begin{center}
the normal bundle sequence for $C \subset X \subset \P^4$ is split
\end{center}
I thank the CIRM and the university of Trento for excellent welcome given to us during this conference as well as C. Ciliberto and E.Ballico for having allowed me to include these remarks in their proceedings.

\subsection{The counterexample to (e)}

We suppose $d > 2$ since the case $d = 2$ is trivial since every quadric contains a line.
\par
Let $X \subset \P^n$ for $n \ge 4$ be a smooth hypersurface of degree $d$, and let $C \subset X$ be a smooth curve. We suppose there exists a surface $\Sigma \subset \P^n$ such that $C$ is the complete intersection of $X$ and $\Sigma$. Since $C$ is smooth, $\Sigma$ is smooth along $C$, so $\Sigma^{\text{sing}}$ is comprised of isolated points located outside of $C$ (I guess because $C \subset \Sigma$ is ample). Let $\tau : \ol{\Sigma} \to \Sigma$ be the designgularization of $\Sigma$. We get a natural inclusion $C \subset \ol{\Sigma}$ and $C$ is a member of the linear system $| \tau^* \struct{\Sigma}(d)|$ on $\ol{\Sigma}$. The class of $C$ in $H^2(\ol{\Sigma}, \Z)$ is divisible by $d$ which implies
\begin{enumerate}
\item $d$ divides $C \cdot_{\ol{\Sigma}} C$
\item $d$ divides $K_{\ol{\Sigma}} \cdot_{\ol{\Sigma}} C$
\end{enumerate}
and therefore the adjunction formula implies:
\begin{center}
$d$ divides $\deg{K_C}$ 
\end{center}

\subsection{•}

We consider now a curve with an ordinary double point $D$ constructed via two smooth tangent plances $P_1 \cap X = C_1$ and $P_2 \cap X = C_2$ of $X$ meeting transversely at a point $p$. One such curve exists when $n \ge 4$. 
\par 
We have, for $i = 1,2$
\begin{enumerate}
\item $d$ divides $\deg{C_i}$
\item $d$ divides $\deg{K_{C_i}}$
\end{enumerate}
from which we have
\[ \deg{K_D} = \deg{K_{C_1}} + \deg{K_{C_2}} + 2 \equiv 2 \mod d \]
Then if $S$ is a complete intersection surface $X \cap X_1 \cap \cdots \cap X_{n-3}$ containing $D$ and let $D' \subset S$ be a smooth member of the linear system $| m H + D |$ on $S$ ($D'$ exists for $m$ sufficiently large). 
\par 
We have
\begin{align*}
\deg{K_{D'}} &= (D')^2 + K_S \cdot D' = (D + m H)^2 + K_S \cdot (D + m H)
\\
& = \deg{K_D} + 2m H \cdot D + m^2 H^2 + m K_S \cdot H
\end{align*}
the last two terms are divisible by $d$; according to (a) $\deg{D}$ is also $2$ mod $d$ hence $\deg{K_{D'}}$ is $2$ mod $d$.
\par 
Since $d > 2$, $D'$ does not satisfy condition 1.2, and provides a counterexample to (v). 

\subsection{Counterexample to (e)}

We supppose from now on (for simplicity) that $n = 3$. We start again with the curve $D = C_1 \cup_p C_2$.
\par 
We will show the following facts 
\begin{enumerate}
\item[(A)] the normal bundle exact sequence of $D \subset X \subset \P^4$ is not split
\item[(B)] if $S = X \cap X'$ is a smooth surface containing $D$ with $\deg{X'} = k$ sufficiently large; let $D'$ be a smooth curve on $S$ in the linear system $| m H + D |$ for $m$ sufficiently large; then the normal bundle exact sequence of $D' \subset X \subset \P^4$ is not split.
\end{enumerate}
\begin{lemma}
(A) $\implies$ (b)
\end{lemma}

\begin{proof}
We denote $e_D \in H^1(C, N_{D|X}(-d))$ the class of the extension
\[ 0 \to N_{D|X} \to N_{D | \P^4} \to \struct{D}(d) \to 0 \]
and for any hypersurface $X'$ such that $X' \cap X = S$, we denote $F_D^{X'} \in H^1(D, N_{D|S}(-d))$ the class of the extension
\[ 0 \to N_{D|S} \to N_{D|X'} \to \struct{D}(d) \to 0 \]
and the same notations for $D'$
\par 
We consider the exact sequence
\[ 0 \to N_{D|S} \to N_{D|X} \to \struct{D}(k) \to 0 \]
which provides the map
\[ \alpha : H^1(D, N_{D|S}(-d)) \to H^1(D, N_{D|X}(-d)) \quad \text{ such that } \quad \alpha(F_D^{X'}) = e_D \]
\begin{enumerate}
\item[(i)] We suppose $S$ is fixed, and let $m$ be such that $H^1(S, \struct{S}(-D)(k- d - m)) = 0$. This implies: $H^0(S, \struct{S}(k-d)) \to H^0(D', \struct{D'}(k-d))$ is surjective, pour any curve $D'$ in the liniear system $| m H + D|$ on $S$. We immediately deduce: if $e_{D'} - 0$, there exists a hypersurface $X^{''}$ of degree $k$, such that $X \cap X'' = S$ and $F^{X''}_{D'} = 0$.

\item[(ii)] We consider the exact sequence
\[ 0 \to \struct{S}(-d) \to \struct{S}(D')(-d) \to N_{D'|S}(-d) \to 0 \]
This gives a map
\[ \delta_{D'} : H^1(D', N_{D' | S}(-d)) \to H^2(S, \struct{S}(-d)) \]
we also have the natural map given by cup-product:
\[ \beta : H^2(S, \struct{S}(-d)) \to \Hom{}{H^0(S, \struct{S}(d))}{H^2(S, \struct{S})} \]
we verify easily that $\beta$ is injective as soon as $K_S \ge 0$.
\par 
It is then well known that the image $\beta \circ \delta_{D'}(F^{X''}_{D'}) \in \Hom{}{H^0(\struct{S}(d))}{H^2(\struct{S})}$ is identified as the composite 
\[ H^0(\struct{S}(d)) \to H^1(T_S) \xrightarrow{\smile \lambda_{D'}} H^2(\struct{S}) \]
this map is the cup product with the class $\lambda_{D'} \in H^1(\Omega_S)$ corresponding to $D$, and the map $H^0(\struct{S}(d)) \to H^1(T_S)$ is provided by the exact sequence
\[ 0 \to T_S \to T_{X''|S} \to \struct{S}(d) \to 0 \]
It is also easy to verify that $\smile \lambda_{D'}$ only depends on the ``class in primitive cohomology'' of $D'$ i.e. if $\smile \lambda_{D'} = \smile \lambda_{D''}$ if $\lambda_{D'} = \lambda_{D''} + n \lambda_H$ for $n \in \Z$. We thus deduce $\smile \lambda_{D'} = \smile \lambda_D$. 

\item[(iii)] We choose some $k$ (and $S$) such that we have $H^1(\struct{S}(D)(-d)) = 0$ (it is easily seen that this condition is satisfied for $k$ big enough).

\item[(iv)] We suppose towards a contradiciton that $e_{D'} = 0$, or $D'$ is a smooth and chosen as in (i): there exists some $X''$ such that $F^{X''}_{D'} = 0$. We then deduce $\beta \circ \delta_{D'}(F^{X''}_{D'}) = 0$, and applying (ii) that $\smile \lambda_{D'} = 0$.  Still according to (ii),  it comes from $\smile \lambda_D$, where $\beta \circ \delta_D(F^{X''}_{D}) = 0$. But the choice of $k$, made in (iii), implies that $\delta_D$ is injective. Since $\beta$ is also injective, we have deduced $F^{X''}_{D} = 0$, an immediately $e_D = 0$, which contradicts (A).
\end{enumerate}
\end{proof}

\subsubsection{Proof of (A)}

The normal exact sequence of $D \subset X \subset \P^4$ write as 
\[ 0 \to N_{D|X} \to N_{D|\P^4} \to \struct{D}(d) \to 0 \]
it is clear that it is enough to prove that
\[ h^0(N_{D|\P^4}(-d)) = 0 \]
We consider the following exact sequences $(E_1)$ and $(E_2)$:
\[ 0 \to N_{D|\P^4}(-d) \to N_{D | \P^4}(-d)|_{C_1} \oplus N_{D | \P^4}(-d)|_{C_2} \to N_{D|\P^4}(-d)|_p \to 0 \] 
\[  0 \to N_{C_1 | \P^4}(-d) \to N_{D | \P^4}(-d)|_{C_1} \to \struct{p} \to 0 \]
We have $h^0(N_{C_i|\P^4}(-d)) = 1$.
\par 
It suffices to show
\begin{enumerate}
\item[(i)] $H^0(N_{D|\P^4}(-d)|_{C_1}) \cong H^0(N_{C_i | \P^4}(-d))$
\item $H^0(N_{C_1 | \P^4}(-d)) \oplus H^0(N_{C_2 | \P^4}(-d)) \embed H^0(N_{D | \P^4}(-d)_p)$
\end{enumerate}
\begin{enumerate}
\item[(i)] By Riemann-Roch an duality, $H^0(N_{D|\P^4}(-d)|_{C_i}) = H^0(N_{C_i | \P^4}(-d))$ if and only if the inclusion
\[ H^0(N_{D|\P^4}^\vee(-d) \ot K_{C_i}) \embed H^0(N_{C_i | \P^4}^\vee(d) \ot K_{C_i}) \]
is strict, or for $d \ge 3$, the sheaf $N_{C_i | \P^4}^\vee(d) \ot K_{C_i}$ is generated by global sections. The conslusion is then immediate, using the dual of $E_2$.

\item[(ii)] the section of $H^0(N_{C_i | \P^4}(-d))$ comes from the canonical section of $N_{C_i|P_i}(-d)$ (where $P_i$ are the planes defining $C_i$) for $i = 1,2$. The assertion results immediately from the fact that the tangent spaces of $P_1$ and $P_2$ are transverse at the point $p$, and the local description of $N_{D|\P^4}$. 
\end{enumerate}

\begin{rmk}
It is natural to think that a curve of type $C_1 \cup_p C_2$ furnishes counterexamples to (e) and (e'), indeed we consider the reducible surface $P = P_1 \cup_p P_2$, the union of planes $P_1$ and $P_2$ meeting transversally at the point $p$.  Then its scheme-theoretic intersection with $X$ is not the reducible curve $C_1 \cup_p C_2$, but has an embedded point, so that $D$ is not a global complete intersection $P \cap X$. 
\end{rmk}

Conserning the other points of the conjecture of Griffiths and Harris, we can make the (perhaps obvious) remark:

\begin{lemma}
(ii) $\implies$ (i)
\end{lemma}

\begin{proof}
We suppose that the general hypersurface $X$ contains a curve of degree $m$, and that its image under the Abel-Jacobi map $\varphi_X$ is not zero.
\par 
There exists an irreducible variety $W$ equipped with a proper map $p : W \to \X$ where $\X = \P(H^0(\P^4, \struct{}(d))$, such that the fiber over $X$ parametrizes curves of degree $m$ inside $X$; two such curves are homologous and for general $X$, we hace $\forall C, C' \in p^{-1}(X) : \varphi_X(C - C') = 0$; in fact, this remains true for all smooth $X$. In effect, if $H$ denotes a plane section of $X$, we have, for general $X$, $\varphi_X(d C - m H) = 0$ for all $C \in p^{-1}(X)$. By irreducibility of $W$, this remains true for all $X \in \X$. Then $\varphi_X(C - D')$ is a torsion point, constant over the connected components of $p^{-1}(X) \times p^{-1}(X)$. But the normality of $\X$ and irreducibility of $W$ imply that if $W \to W_1 \to X$ is the Stein factorization of $p$, each irreducible component of the product $W_1 \times_{\X} W_1$ dominates $\X$. This leads easily to the result.  
\par 
We fix a line $\Delta$ in $\P^4$ and we write $\X_{\Delta}$ for the family of hypersufraces of degree $d$ containig $\Delta$. We notation $W_{\Delta} := p^{-1}(\X_{\Delta})$; we have then a normal function $\nu_{\Delta}$ defined as follows on $\X_{\Delta}$: let $X$ be a smooth point of $\X_{\Delta}$ and let $C \in p^{-1}(X)$; then $\deg{(m \Delta - C)} = 0$ and we can set $\nu_{\Delta}(X) = \varphi_X(m \Delta - C)$.
\par 
But it is known that the group of normal functions on $\X_{\Delta}$ is cyclic generated by the normal unction $\nu_\Delta^H$ defined by: $\nu_{\Delta}^H(X) = \varphi_X(d \Delta - H)$ (it suffices to generalize the arugment of [3], section 3). We then deduce that there exist a $k$ such that for all smooth $X$ in $\X_{\Delta}$ we have $\Phi_{X}(m \Delta - C) = k \varphi_X(d \Delta - H)$ for $C \in p^{-1}(X)$.
\par 
As $\Delta$ is deformation equivalent to $\Delta'$ we have in fact $k = k'$; on $\X_{\Delta} \cap \X_{\Delta'}$, so $(m -kd) \Phi_X(\Delta - \Delta') = 0$. But by Griffiths [3], if $X$ is general in $\X_{\Delta} \cap \X_{\Delta'}$ then $\Phi_{X}(\Delta - \Delta') \in J(X)$ is not a torsion point. Then $m - kd = 0$ which proves (i).
\end{proof}

\subsection{Conclusion}

In paragraph 1 we cleared the necessary condition 1.2 for the curve $C$ to be a complete intersection $X \cap \Sigma$ of $X$ with a surface $\Sigma \subset \P^4$. If $d$ divides the degree of $C$, this condition is automatically saisfied when $C$ is sub-canical (i.e. $K_C = \struct{C}(m)$ for some $m$). Likewise, it seems difficult to construct by an analogous process to the one described in paragraphs 1 and 2, a sub-canonical curve for which the normal bundle sequence is not eact. It does therefore not exclude that (e) or (e') might hold for sub-canonical curves.

\end{document}