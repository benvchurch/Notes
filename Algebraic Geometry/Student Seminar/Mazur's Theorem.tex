\documentclass[12pt]{article}
\usepackage{hyperref}
\hypersetup{
    colorlinks=true,
    linkcolor=blue,
    filecolor=magenta,      
    urlcolor=blue,
}

\usepackage{import}
\import{../}{AlgGeoCommands}

\begin{document}

\section{The Formal Immersion Step (the new hotness on tiktak)}

\newcommand{\hatO}{\wh{\mathcal{O}}}
\newcommand{\TT}{\mathbb{T}}
\newcommand{\FFbar}{\overline{\mathbb{F}}}

\begin{theorem}
Let $N$ be a prime, either $11$ or $\ge 17$ (ensuring that $X_0(N)$ has genus $> 0$) then there are no elliptic curves over $\Q$ with a torsion point of order $N$. 
\end{theorem}

Kep points, 

\begin{enumerate}
\item if $E$ has good reduction at $3$ then $E[N](\Q) \embed \overline{E}(\FF_3)$ which has order at most $9$ by Hasse so $N < 9$.

\item if $E$ has multiplicative reduction we can get crazy polygons so no control on $N$

\item if $E$ has additive reduction: what can the special fiber of the minimal regular proper model be? From Kodaira's classification, there are a bounded number of components and hence a bound on $\# \overline{E}(\FF_3) \le 12$.  
\end{enumerate}
 
Assume from now on that $N = 11$ or $N > 17$. 
 
\begin{prop}
If $(E, C)$ is a pair of an elliptic curve over $\Q$ and a cyclic subgroup scheme $C \subset E$ of order $N$. Then $E$ has potentially good reduction away from $2N$.
\end{prop}

\begin{rmk}
This implies you can't have multiplicative reduction because potentially good reduction means the semistable reduction is good but multiplicative reduction is also semistable.
\end{rmk}

\begin{rmk}
Recall that,
\[ \text{good reduction} \iff T_\ell E \text{ is unramified} \]
\[ \text{mult. reduction} \iff I \to \GL(V_\ell E) \text{ is (nontrivial) unipotent} \]
\end{rmk}

\begin{prop}
Let $\cA$ be the Neron model over $\Z[1/2N]$ of the Eisenstein quotient $A$ of $J = \Jac{X_0(N)}$. Define,
\begin{center}
\begin{tikzcd}
X_0(N)_{\Q} \arrow[r] & J \arrow[r] & A
\end{tikzcd}
\end{center}
$f : X_0(N) \to \cA$ over $\Z[1/2N]$ sends $\infty \mapsto 0$.
Then if $p \ndivides 2 N$ then $\infty \in X_0(N)(\Z_{(p)})$ is the only $\Z_{(p)}$-point of $X_0(N)$ mapping to $0 \in \cA(\ZZ_{(p)})$ which reduces to $\infty \in X_0(N)(\FF_p)$. 
\end{prop}

\begin{defn}
Let $f : Y \to Z$ is lft and $Y,Z$ are locally noetherian. If $y\ in U$ say $f$ is a \textit{formal immersion at} $y$ if $\stalk{Z}{f(y)}^\wedge \onto \stalk{Y}{y}^\wedge$ is surjective.
\end{defn}

\begin{defn}
$Y, Z$ are ft + sep over a locally noetherian base $S$. If $f$ is an $S$-morphism and $y \in Y(S)$ is a section then $f$ is a \textit{formal immersion along} $y$ if,
\begin{enumerate}
\item $f$ is a formal immersion along all points of $y$
\item $f_s$ is a formal immersion at $y_s$ for all $s \in S$.  
\end{enumerate}
\end{defn}

\begin{rmk}
This is supposed to be equivalent to $\wh{Y}_y \embed \wh{Z}_{f(y)}$. 
\end{rmk}

\begin{lemma}
Let $A, B$ be complete noeth. local rings and $f : A \to B$ is a local map such that $f : A / \m_A \to B / \m_B$ and $f : \m_A / \m_A^2 \onto \m_B / \m_B^2$ is surjective. 
\end{lemma}

\begin{proof}
Approximate. 
\end{proof}

\begin{prop}
Let $Y$ be separated and $f : Y \to Z$ be a formal immersion at $y \in Y$. Let $T$ be an integral noetherian scheme with $p_1, p_@ \in Y(T)$ are s.t. $y = p_1(t) = p_2(t)$ at some $t \in T$ and $f \circ p_1 = f \circ p_2$ then $p_1 = p_2$. 
\end{prop}

\begin{lemma}
Let $A, B$ be complete noetherian local rings flat over a dvr $(R, \pi)$ with a map $A \to B$ such that $A /\m_A \to B / \m_A$ is an isomorphism. Then $A \to B$ is surjective iff $A / \pi \to B / \pi$ is surjective.
\end{lemma}

\begin{proof}
This follows from the fact that $\m_A / (\m_A^2 + \pi A) \onto \m_B / (\m_B^2 + \pi B)$ being surjective implies that it was surjective before moding by $\pi$.
\end{proof}

\begin{cor}
We can check formal immersions at the special fiber of a DVR. 
\end{cor}

\begin{proof}[Proof of Proposition]
$A = \{ x \in T  \mid p_1(x) = p_2(x) \}$ then $Y$ is separated implies $A \subset T$ closed and $T$ is integral so suffices to show $\Spec{\stalk{T}{t}} \to T$ factors through $A \embed T$. So assume $T$ is local with closed point $t$. Can assume $Y$ is local with closed point $y$. 
\begin{center}
\begin{tikzcd}
\stalk{T}{t} \arrow[r, hook] & \hatO_{T,t}
\\
\stalk{Y}{y} \arrow[u, shift left] \arrow[u, shift right] \arrow[r] & \hatO_{Y, y} \arrow[u, shift left] \arrow[u, shift right] & \hatO_{Z, f(y)} \arrow[l] 
\end{tikzcd}
\end{center}
thus the maps must agree on the local rings since they agree after composing with the surjection. 
\end{proof}

Goal show that if $T_\Q \onto A$ is any surjection of abelian varities with connected kernel (what we call an optimal quotient) then $X_0(N) \to J \to \cA$ over $\Z[1/2N]$ is a formal immerison. 
\bigskip\\
Setup $N$ is prime $ >2$ and $S = \Spec{\Z[1/2N]}$ and $X = X_0(N)$ then $J = J_0(N)$ and $\TT \embed \End{J}$ the Hecke algebra. 

\begin{rmk}
all optimal quotients of $J$ are of the form $J / IJ$ where $I \sub \TT$ is a \textit{saturated} ideal $(\TT / I$ is torsion-free). Then $J_\Q = J_0(N)^{\text{new}}_{\Q}$ so everything in Daniel's talk applies. In particular,
\[ J_{\Q} \sim \prod_{f \in C} A_f \]
with $C$ Galois orbits of cusp forms. Also,
\[ \End[\Q]{A_f} = K_f = \im{\TT} \]
with $[ K_f : \Q ] = \dim{A_f}$. Then any optimal quotient of $J_{\Q}$ is $\prod_{g \in C'} A_g$ with $C' \subset C$. 
\end{rmk}

\begin{theorem}
The tangent space $T_0(F)$ is a free $\T_{\Z[1/2N]}$-module of rank $1$ generated by $\deriv{}{q} |_0$. 
\end{theorem}

\begin{rmk}
This is saying,
\[ S_2(N)_R \cong H^0(J_R, \Omega^1_{J_R/R}) = T^*_0(J_R) \]
for any ring $R$. This is because level $N$ cusp $2$-forms are exactly given by forms on $X_0(N)$ and these are the same as forms on its Jacobian. 
\end{rmk}

\begin{cor}
If $A$ is an optimal quotient of $J$ then $X \to \cA$ sending $\infty \mapsto 0$ is a formal immersion over $S$.
\end{cor}

\begin{proof}
It suffices to show that $T_\infty X \embed T_0 \cA$ over each prime. Then in the a sequence,
\begin{center}
\begin{tikzcd}
0 \arrow[r] & B \arrow[r] & J \arrow[r] & A \arrow[r] & 0
\end{tikzcd}
\end{center}
since $J$ and $A$ have good reduction so does $B$ by Neron-Ogg-Shafarevich. Then Raynaud's theorem gives an exact sequence,
\begin{center}
\begin{tikzcd}
0 \arrow[r] & T_0(\mathcal{B}) \arrow[r] & T_0(J) \arrow[r] & T_0(\cA) \arrow[r] & 0
\end{tikzcd}
\end{center} 
(HMMM)
\end{proof}

Reduction, $M' = T_0(T) / (\TT_{\Z[1/2N]} \deriv{}{q})$. But $T_0(J)$ is finite over $\Z[1/2N]$ hence also $\TT_{\Z[1/2N]}$. Suffices to show that $M' / \m M' = 0$ for all $\m \subset \TT_{\Z[1/2N]}$ i.e. $\deriv{}{q}$ generated $T_0(T) / \m T_0(J)$. 

\begin{lemma}
$S_2(N)_{\Q}^{\text{new}}$ is a free $\TT_\Q$-module of rank  $1$ generated by $\deriv{}{q}|_0$.
\end{lemma}

\begin{lemma}
For $\m \subset \TT_{\Z[1/2N]}$ and $T_0(J) / \m T_0(J) = 0$. 
\end{lemma}

\begin{proof}
Finiteness of $T_0(J)$ and NAK and $T_0(J) \ot_{\Z} \Q \neq 0$.
\end{proof}

\begin{lemma}
For $\m \subset \TT_{\Z[1/2N]}$ then $\deriv{}{q}$ has nonzero image in $T_0(J) / \m T_0(J)$.
\end{lemma}

\begin{proof}
If $f \in S_2(N)_{\FFbar_\ell}$ has a $q$-expansion,
\[ f = \sum_{n = 1}^\infty a_n q^n \]
then $\deriv{}{q}(f) = a_1$ and we win by showing that if $f$ is an eigenform with $a_1 = 0$ then $f = 0$. This is because $\deriv{}{q} (T_n f) = a_n$ so if $T_n f = \lambda f$ for $\lambda \neq 0$ then we also have all $a_n = 0$. 
\bigskip\\
Let's do this in more detail. Let $\ell$ be the characteristic of $F = (\TT \ot \Z[1/2N]) / \m$ and $R = (\TT \ot \ZZ[1/2N]) \ot_{\ZZ} \FFbar_\ell$. And let $M = T_0(J) \ot_{\ZZ} \FFbar_\ell$. Then there is an exact sequence,
\begin{center}
\begin{tikzcd}
\m \ot_{\ZZ} \FFbar_\ell \arrow[r] & R \arrow[r] & F \ot_{\ZZ} \FFbar_\ell \arrow[d, equals] \arrow[r] & 0
\\
& & \prod_{i \in I} \FFbar_\ell \arrow[d, two heads]
\\
0 \arrow[r] & \m R \arrow[r] & R \arrow[r] & \FFbar_\ell \arrow[r] & 0
\\
& & & F \arrow[u, hook]
\end{tikzcd}
\end{center} 
by tensoring the inclusion $F \embed \FFbar_\ell$ we get $T_0(J) / \m T_0(J) \embed M / \m M$. As $R$-modules,
\[ (M / \m M)^\vee \cong M^\vee[\m] \cong H^0(X_{\FFbar_\ell}, \Omega^1_{X/ \FFbar_\ell})[\m] \]
\end{proof}

\begin{theorem}
if $f : X \to S$ is a smooth proper relative curve then $R^i f_* \Omega_{X/S}$ commutes with all base change.
\end{theorem}

\begin{proof}
If $S$ is reduced this comes from Grauert. Otherwise use cohomology and base change.
\end{proof}

In particular: if $f \in S_2(N)_{\FFbar_\ell}[\m]$ is nonzero can lift to char $0$ and then $\deriv{}{q} (T_n f) = a_n(f)$ follows from analysis. 

\begin{lemma}
For every $\m \subset \TT_{\Z[1/2N]}$. Then $T_0(J) / \m T_0(J)$ is free over $\TT_{\Z[1/2N]} / \m$ generated by $\deriv{}{q}$. 
\end{lemma}

\begin{proof}
$\dim_F T_0(J) / \m T_0(J) = \dim_{\FFbar_\ell} M^\vee[\m]$ then let $a_n$ be the image of $T_n$ in $R / \m = \FFbar_\ell$ then if $f \in S_2(N)_{\FFbar_\ell}[\m]$ and $T_n(f) = a_n(F)$ so $f$ is a multiple of $q + a_2 q^2 + \cdots$. 
\end{proof}

\section{Final Talk}

\subsection{Kodaira Classification of Special Fibers}

\begin{defn}
An \textit{elliptic surface} is a regular connected $2$-dimensional scheme $X$ equipped with a map proper map $\pi : X \to C$ to a regular connected $1$-dimensional scheme $C$ (e.g. a Dedekind scheme) whose generic fiber is a smooth geometrically-connected curve of genus $1$.
\end{defn}

\begin{rmk}
The map $\pi : X \to C$ is dominant (since the generic fiber is nonempty) and $X$ and $C$ are integral so we get an injection $K(C) \embed K(X)$ hence $\stalk{X}{x}$ are torsion-free $\stalk{C}{\pi(x)}$-modules and hence flat since the base is a DVR. Thus $\pi$ is flat. Since $X$ is irreducible the fibers must be all pure dimension $1$. Then $\pi$ is also proper so since $C$ is normal and the generic fiber is geometrically-connected $\pi_* \struct{X} = \struct{C}$ so $\pi$ has connected fibers. Thus every fiber of $\pi$ is a connected genus $1$ curve (not necessarily reduced) and $\pi$ is smooth iff these are smooth genus $1$ curves. 
\end{rmk}

\begin{defn}
We say that an elliptic surface $X$ is \textit{pointed} if it furthermore equipped with a section $\sigma : C \to X$ of $\pi$. We say that $\pi$ is \textit{relatively minimal} if the fibers contain no $(-1)$-curves. In this case we say that $X$ is a \textit{minimal elliptic surface}.
\end{defn}

\begin{rmk}
A minimal elliptic surface $\pi : X \to C$ is exactly the data of compatible minimal regular models of its generic fiber $E$ over each DVR $\stalk{C}{p}$. Therefore, to classify the fibers of minimal elliptic surfaces, it suffices to classify the special fibers of minimal regular models of genus $1$ curves and, in the equicharacteristic case, then exhibit these fibers in complete families with regular total spaces. (IT IS OBVIOUS THAT THIS CAN ALWAYS BE DONE??)
\end{rmk}

\begin{rmk}
DO I NEED TO ASSUME $S$ IS EXCELLENT FOR EVERYTHING I WANT TO BE TRUE.
\end{rmk}

\subsection{General Properties of Regular Models}

\begin{rmk}
Liu's book covers this topic well. 
\end{rmk}
\renewcommand{\Div}{\mathrm{Div}}

Let $X \to S = \Spec{R}$ be a regular proper model with special fiber $X_s$. Let $\Gamma_i$ be the irreducible components of $X_s$ appearing with multiplicity $d_i$. These are (possibly singular) proper curves over $\kappa = R / \m$. 
\bigskip\\
We want to define an intersection pairing on $X$.  For any nonzero horizontal divisor $D$ we should have $C \cdot X_s > 0$. However, $\struct{X}(X_s) = \struct{X}$ because it is cut out by a global function $\pi \in \Gamma(X, \struct{X})$. Thus we cannot have an intersection pairing invariant under linear equivalence. This is because $S$ is not ``complete'' so we can deform $X_s$ to the ``boundary'' where it vanishes. Arakalov theory solves this but instead we will just ask that $i(-,D)$ is invariant under linear equivalence in its first factor. However, there is still an issue if $D$ contains a horizontal divisor then $X_s \cdot D > 0$ because definition of the intersection pairing is symmetric. To fix this we restrict the second coordinate to only vertical divisors.

\begin{lemma}
There is a bilinear intersection pairing,
\[ i_s :  \Div(X) \times \Div_s(X) \to \Z \]
which satisfies the following properties,
\begin{enumerate}
\item when $D$ and $E$ share no components by,
\[ (D, E) \mapsto \sum_{x \in D \cap E} i_x(D, E) [\kappa(x) : \kappa] \]

\item if $D \sim D'$ then $i_s(D, E) = i_s(D', S)$

\item if FINDI
\end{enumerate}
When $E \subset X_s$ is an effective divisor in the special fiber this is equivalent to,
\[ i_s(D, E) = \deg_{\kappa} \struct{E}(D) \]
This is only defined for the second divisor $\Div_s(X)$ supported in the special fiber because the base is ``not complete''.  The following example shows the pathologies of this intersection pairing and why we can't define it for arbitrary divisors. 
\end{lemma}


\begin{rmk}
$i_s(-, E)$ is invariant under linear equivalence. However $i_s(D, -)$ is  \textit{not} invariant unless $D \in \Div_s(X)$. For example, $\struct{X}(X_s) = \struct{X}$ because it is cut out by a global function $\pi \in \Gamma(X, \struct{X})$. However, we will see that
\[ K_{X/S} \cdot X_s = 2 g(X_\eta) - 2 \]
which is nonzero. However, when by $D, E \in \Div_s(X)$ then $i_s$ is symmetric and hence invariant under linear equivalence in both components. This implies for example that $X_s^2 = 0$.
\end{rmk}

\begin{rmk}
The above example shows why we cannot define an intersection theory at all for arbitrary divisors. Indeed, suppose we had,
\[ i : \Div(X) \times \Div(X) \to \Z \]
and we just wanted $i(-,D)$ invariant under linear equivalence in the first coordinate. However, if $D = H + V$ where $H$ is a horizontal divisor and $V$ is a vertical divisor we have seen that the usual intersection product means that,
\[ i(X_s, H + V) = X_s \cdot H = H \cdot X_s > 0 \]
but $X_s \sim 0$.  
\end{rmk}

Since $X$ is regular and flat over a regular base, the fibers are Gorenstein (\chref{https://stacks.math.columbia.edu/tag/0BJJ}{Tag 0BJJ}). Therefore, there exists a relative dualizing line bundle $\omega_{X/S}$ (\chref{https://stacks.math.columbia.edu/tag/0E6R}{Tag 0E6R}) 

\begin{lemma}
Let $X$ be a regular surface flat and proper over $\Spec{R}$.
\begin{enumerate}
\item $X$ is minimal iff $K_{X/S}$ is numerically effective
\item 
\end{enumerate}
\end{lemma}

\begin{lemma}
Let $X$
\begin{enumerate}
\item $K_{X/S} \cdot X_s = 2 g(X_\eta) - 2$
\item 
\end{enumerate}
\end{lemma}

\subsection{The Relation to Neron Models}

\begin{prop}
Let $E$ be an elliptic curve over $K$ with $K = \Frac{R}$ and $R$ a DVR. Then let $X$ be the minimal regular model of $E$ over $R$. By properness, $X$ is a pointed elliptic surface with $\sigma_K = 0 \in E(K) = X(K)$. Let $\E \subset X$ be the open subscheme obtained by removing the points of $X$ which are singularities of the special fiber. Then $\E$ is the Neron model of $E$.
\end{prop}

\begin{proof}
We verify the Neron mapping property in two steps. First, if $\Spec{R'} \to \Spec{R}$ is a finite \etale cover then by properness,
\[ X(R') = E(K') \]
so it suffices to show that maps $\Spec{R'} \to X$ land in $\E$. Indeed, let $x \in X$ be the image of a closed point $\m \subset R'$ then we get a diagram,
\begin{center}
\begin{tikzcd}
R'_\m \arrow[from = r] & \stalk{X}{x}
\\
R \arrow[u] \arrow[ru]
\end{tikzcd}
\end{center} 
and $R \to R'_\p$ is \etale so the uniformizer lands in $\m \sm \m^2$. Thus by commutativity $\pi \in \m_x \sm \m_x^2$. Since $\stalk{X}{x}$ is regular this implies that $\stalk{X_s}{x} = \stalk{X}{x} / \pi$ is regular so $x \in \E$. Thus,
\[ \E(R') \to X(R') \to E(K') \]
are all isomorphisms. This can be used to show that the group law on $E$ extends uniquely to $\E$. Now, if $T \to \Spec{R}$ is a smooth $R$-scheme and $f_K : T_K \to E$ is a map then by properness $f_K$ extends to a rational map $f : T \rat X$ defined away from codimension $2$. However, $T \to \Spec{R}$ has sections through any point after some \etale extension $\Spec{R'} \to \Spec{R}$. Since we know maps $\Spec{R'} \to X$ land in $\E$ we conclude that $f$ factors through $\E \embed X$. Finally, since $\E$ is a group by a translation argument $f : T \rat \E$ is everywhere defined. Then,
\[ \Hom{R}{T}{\E} \to \Hom{K}{T_K}{E} \]
is surjective and since $\E \to \Spec{R}$ is separated it is injective.
\end{proof}


\begin{lemma}
Suppose that $E$ is an elliptic curve over $\Q$ with additive reduction at $p$ and $\E$ its Neron model over $\ZZ_{(p)}$. Then $\# \pi_0(\E_{\FF_p}) \le 4$.
\end{lemma}

\begin{proof}
Additive reduction means that the special fiber of $\E^0$ is $\Ga$ (since the residue field is perfect). Therefore, all the geometric components (IS IT POSSIBLE FOR $\pi_0(\E_{\FF_p})$ NONSPLIT??) of $\E^0$ are isomorphic to $\Ga$. From our construction of the Neron model this is equivalent to, in the special fiber of the minimal regular model, each genus $1$ component having a cusp and each genus $0$ component intersecting the other components in exactly one point. By inspection of the possible types (II, III, IV, HOW MANY OTHERS, there can be a maximum of $4$ geometric components under these restrictions. 
\end{proof}

\subsection{Completing the Proof}

\newcommand{\tors}{\mathrm{tors}}

We first need a lemma. (HAS THIS BEEN PROVEN PREVIOUSLY??)

\begin{prop}
Let $p \neq 2$ and $f : H \to G$ be a morphism of finite flat group schemes over a DVR $R$ with mixed characteristic $0$ and $p$. Let $K = \Frac{R}$. If $f_K : H_K \to G_K$ is a closed immersion then $f$ is a closed immersion.
\end{prop}

\begin{proof}
DO THIS!!
\end{proof}

\begin{theorem}
Let $(\E_\Q, C)$ be a pair of an elliptic curve over $\Q$ and a cyclic subgroup $C$ of order $N$ with $N$ an odd prime. Then $E$ has potentially good reduction at all odd primes $p \neq N$. 
\end{theorem}

The proof of this theorem relies on the following result from last time.

\begin{prop}
Let $\cA$ be the Neron model over $\Z[1/2N]$ of any nonzero optimal quotient $A$ of $J$. Define $X_0(N)_{\Q} \to J \to A$ by sending the cusp $\infty$ to $0$ and let $f$ denote the morphism extensing this over $\Spec{\Z[1/2N]}$. Then $\infty \in X_0(N)(\Z_{(p)})$ is the only point reducing to $\infty \in X_0(N)(\FF_p)$ that also maps to $0$ in $\cA(\ZZ_{(p)})$ under $f$.
\end{prop}

\begin{proof}[Proof of Theorem]
Let $A = \wt{J}$ DO THIS PROFO
\end{proof}

\begin{rmk}
Let $\E$ be the neron model of an elliptic curve $E$ over a DVR $R$. Then $\E[N]$ is finite flat over $R$ because $\E \xrightarrow{N} \E$ is finite flat so its base change along the zero section is also finite flat. To show $\E \to \E$ is finite flat we use miracle flatness check quasi-finiteness explicitly then show properness because (HOW TO DO!!)
\end{rmk}



Now we are ready to prove the main theorem. 

\begin{theorem}[Mazur]
Let $N$ be prime and $N \ge 11$ and not $13$. Then there are no elliptic curves over $\Q$ with a torsion subgroup of order divisible by $N$.
\end{theorem}

\begin{proof}
Suppose $E(\Q)[N]$ is nonempty. We proved that $E$ has potentially good reduction at $3$. We will now show that $N \le 7$. First, suppose that $E$ has good reduction at $3$ so its Neron model $\E$ over $\Spec{\Z_{(3)}}$ is proper. Then since $\E[N]$ is finite over $\Spec{\Z_{(3)}}$ we have,
\[ \Z / N \Z \embed E(\Q)_{\tors} \to \E(\FF_3) \]
is injective. But $\E_{\FF_3}$ is an elliptic curve so by the Hasse-Weil bound,
\[ \# \E_{\FF_3} \le \lfloor 4 + 2 \cdot \sqrt{3} \rfloor = 7 \]
Otherwise, $E$ has additive reduction at $3$ and let $\E$ be its Neron model over $\Spec{\Z_{(3)}}$. Consider the exact sequence,
\begin{center}
\begin{tikzcd}
0 \arrow[r] & \E^0_{\FF_3} \arrow[r] & \E_{\FF_3} \arrow[r] & \pi_0(\E_{\FF_3}) \arrow[r] & 0 
\end{tikzcd}
\end{center}
Since we are in the case of additive reduction, $\E^0_{\FF_3} = \Ga$. Consider the map of finite flat group schemes,
\[ \underline{\Z / N \Z} \to \E[N] \]
over $R = \Z_{(3)}$ arising from the inclusion $\Z / N \Z \embed E[N](\Q)$ which spreads out by the Neron mapping property since $\underline{\Z / N \Z}$ is smooth over $R$. This map is, by construction, a closed immersion on the generic fiber so by Prop 3.3 this map is a closed immerison. In particular,
\[ \Z / N \Z \embed \E[N]_{\FF_3} \]
By the above lemma, $\# \pi_0(\E_{\FF_3}) \le 4$ so if $N > 4$ then $\Z / N \Z$ lands in $\E^0_{\FF_3}$ but $\# \E^0_{\FF_3}(\FF_3) = \# \Ga(\FF_3) = 3$ so this is impossible. Therefore we conclude that $N \le 4$. 
\end{proof}

\end{document}

