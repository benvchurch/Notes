\documentclass[12pt]{article}
\usepackage{hyperref}
\hypersetup{
    colorlinks=true,
    linkcolor=blue,
    filecolor=magenta,      
    urlcolor=blue,
}

\usepackage{import}
\import{../}{AlgGeoCommands}

\begin{document}

\section{The Formal Immersion Step (the new hotness on tiktak)}

\newcommand{\hatO}{\wh{\mathcal{O}}}
\newcommand{\TT}{\mathbb{T}}
\newcommand{\FFbar}{\overline{\mathbb{F}}}

\begin{theorem}
Let $N$ be a prime, either $11$ or $\ge 17$ (ensuring that $X_0(N)$ has genus $> 0$) then there are no elliptic curves over $\Q$ with a torsion point of order $N$. 
\end{theorem}

Kep points, 

\begin{enumerate}
\item if $E$ has good reduction at $3$ then $E[N](\Q) \embed \overline{E}(\FF_3)$ which has order at most $9$ by Hasse so $N < 9$.

\item if $E$ has multiplicative reduction we can get crazy polygons so no control on $N$

\item if $E$ has additive reduction: what can the special fiber of the minimal regular proper model be? From Kodaira's classification, there are a bounded number of components and hence a bound on $\# \overline{E}(\FF_3) \le 12$.  
\end{enumerate}
 
Assume from now on that $N = 11$ or $N > 17$. 
 
\begin{prop}
If $(E, C)$ is a pair of an elliptic curve over $\Q$ and a cyclic subgroup scheme $C \subset E$ of order $N$. Then $E$ has potentially good reduction away from $2N$.
\end{prop}

\begin{rmk}
This implies you can't have multiplicative reduction because potentially good reduction means the semistable reduction is good but multiplicative reduction is also semistable.
\end{rmk}

\begin{rmk}
Recall that,
\[ \text{good reduction} \iff T_\ell E \text{ is unramified} \]
\[ \text{mult. reduction} \iff I \to \GL(V_\ell E) \text{ is (nontrivial) unipotent} \]
\end{rmk}

\begin{prop}
Let $\cA$ be the Neron model over $\Z[1/2N]$ of the Eisenstein quotient $A$ of $J = \Jac{X_0(N)}$. Define,
\begin{center}
\begin{tikzcd}
X_0(N)_{\Q} \arrow[r] & J \arrow[r] & A
\end{tikzcd}
\end{center}
$f : X_0(N) \to \cA$ over $\Z[1/2N]$ sends $\infty \mapsto 0$.
Then if $p \ndivides 2 N$ then $\infty \in X_0(N)(R)$ is the only $R$-point of $X_0(N)$ mapping to $0 \in \cA(R)$ which reduces to $\infty \in X_0(N)(\FF_p)$. 
\end{prop}

\begin{defn}
Let $f : Y \to Z$ is lft and $Y,Z$ are locally noetherian. If $y\ in U$ say $f$ is a \textit{formal immersion at} $y$ if $\stalk{Z}{f(y)}^\wedge \onto \stalk{Y}{y}^\wedge$ is surjective.
\end{defn}

\begin{defn}
$Y, Z$ are ft + sep over a locally noetherian base $S$. If $f$ is an $S$-morphism and $y \in Y(S)$ is a section then $f$ is a \textit{formal immersion along} $y$ if,
\begin{enumerate}
\item $f$ is a formal immersion along all points of $y$
\item $f_s$ is a formal immersion at $y_s$ for all $s \in S$.  
\end{enumerate}
\end{defn}

\begin{rmk}
This is supposed to be equivalent to $\wh{Y}_y \embed \wh{Z}_{f(y)}$. 
\end{rmk}

\begin{lemma}
Let $A, B$ be complete noeth. local rings and $f : A \to B$ is a local map such that $f : A / \m_A \to B / \m_B$ and $f : \m_A / \m_A^2 \onto \m_B / \m_B^2$ is surjective. 
\end{lemma}

\begin{proof}
Approximate. 
\end{proof}

\begin{prop}
Let $Y$ be separated and $f : Y \to Z$ be a formal immersion at $y \in Y$. Let $T$ be an integral noetherian scheme with $p_1, p_2 \in Y(T)$ are s.t. $y = p_1(t) = p_2(t)$ at some $t \in T$ and $f \circ p_1 = f \circ p_2$ then $p_1 = p_2$. 
\end{prop}

\begin{lemma}
Let $A, B$ be complete noetherian local rings flat over a dvr $(R, \pi)$ with a map $A \to B$ such that $A /\m_A \to B / \m_A$ is an isomorphism. Then $A \to B$ is surjective iff $A / \pi \to B / \pi$ is surjective.
\end{lemma}

\begin{proof}
This follows from the fact that $\m_A / (\m_A^2 + \pi A) \onto \m_B / (\m_B^2 + \pi B)$ being surjective implies that it was surjective before moding by $\pi$.
\end{proof}

\begin{cor}
We can check formal immersions at the special fiber of a DVR. 
\end{cor}

\begin{proof}[Proof of Proposition]
$A = \{ x \in T  \mid p_1(x) = p_2(x) \}$ then $Y$ is separated implies $A \subset T$ closed and $T$ is integral so suffices to show $\Spec{\stalk{T}{t}} \to T$ factors through $A \embed T$. So assume $T$ is local with closed point $t$. Can assume $Y$ is local with closed point $y$. 
\begin{center}
\begin{tikzcd}
\stalk{T}{t} \arrow[r, hook] & \hatO_{T,t}
\\
\stalk{Y}{y} \arrow[u, shift left] \arrow[u, shift right] \arrow[r] & \hatO_{Y, y} \arrow[u, shift left] \arrow[u, shift right] & \hatO_{Z, f(y)} \arrow[l] 
\end{tikzcd}
\end{center}
thus the maps must agree on the local rings since they agree after composing with the surjection. 
\end{proof}

Goal show that if $T_\Q \onto A$ is any surjection of abelian varities with connected kernel (what we call an optimal quotient) then $X_0(N) \to J \to \cA$ over $\Z[1/2N]$ is a formal immerison. 
\bigskip\\
Setup $N$ is prime $ >2$ and $S = \Spec{\Z[1/2N]}$ and $X = X_0(N)$ then $J = J_0(N)$ and $\TT \embed \End{J}$ the Hecke algebra. 

\begin{rmk}
all optimal quotients of $J$ are of the form $J / IJ$ where $I \sub \TT$ is a \textit{saturated} ideal $(\TT / I$ is torsion-free). Then $J_\Q = J_0(N)^{\text{new}}_{\Q}$ so everything in Daniel's talk applies. In particular,
\[ J_{\Q} \sim \prod_{f \in C} A_f \]
with $C$ Galois orbits of cusp forms. Also,
\[ \End[\Q]{A_f} = K_f = \im{\TT} \]
with $[ K_f : \Q ] = \dim{A_f}$. Then any optimal quotient of $J_{\Q}$ is $\prod_{g \in C'} A_g$ with $C' \subset C$. 
\end{rmk}

\begin{theorem}
The tangent space $T_0(F)$ is a free $\T_{\Z[1/2N]}$-module of rank $1$ generated by $\deriv{}{q} |_0$. 
\end{theorem}

\begin{rmk}
This is saying,
\[ S_2(N)_R \cong H^0(J_R, \Omega^1_{J_R/R}) = T^*_0(J_R) \]
for any ring $R$. This is because level $N$ cusp $2$-forms are exactly given by forms on $X_0(N)$ and these are the same as forms on its Jacobian. 
\end{rmk}

\begin{cor}
If $A$ is an optimal quotient of $J$ then $X \to \cA$ sending $\infty \mapsto 0$ is a formal immersion over $S$.
\end{cor}

\begin{proof}
It suffices to show that $T_\infty X \embed T_0 \cA$ over each prime. Then in the a sequence,
\begin{center}
\begin{tikzcd}
0 \arrow[r] & B \arrow[r] & J \arrow[r] & A \arrow[r] & 0
\end{tikzcd}
\end{center}
since $J$ and $A$ have good reduction so does $B$ by Neron-Ogg-Shafarevich. Then Raynaud's theorem gives an exact sequence,
\begin{center}
\begin{tikzcd}
0 \arrow[r] & T_0(\mathcal{B}) \arrow[r] & T_0(J) \arrow[r] & T_0(\cA) \arrow[r] & 0
\end{tikzcd}
\end{center} 
(HMMM)
\end{proof}

Reduction, $M' = T_0(T) / (\TT_{\Z[1/2N]} \deriv{}{q})$. But $T_0(J)$ is finite over $\Z[1/2N]$ hence also $\TT_{\Z[1/2N]}$. Suffices to show that $M' / \m M' = 0$ for all $\m \subset \TT_{\Z[1/2N]}$ i.e. $\deriv{}{q}$ generated $T_0(T) / \m T_0(J)$. 

\begin{lemma}
$S_2(N)_{\Q}^{\text{new}}$ is a free $\TT_\Q$-module of rank  $1$ generated by $\deriv{}{q}|_0$.
\end{lemma}

\begin{lemma}
For $\m \subset \TT_{\Z[1/2N]}$ and $T_0(J) / \m T_0(J) = 0$. 
\end{lemma}

\begin{proof}
Finiteness of $T_0(J)$ and NAK and $T_0(J) \ot_{\Z} \Q \neq 0$.
\end{proof}

\begin{lemma}
For $\m \subset \TT_{\Z[1/2N]}$ then $\deriv{}{q}$ has nonzero image in $T_0(J) / \m T_0(J)$.
\end{lemma}

\begin{proof}
If $f \in S_2(N)_{\FFbar_\ell}$ has a $q$-expansion,
\[ f = \sum_{n = 1}^\infty a_n q^n \]
then $\deriv{}{q}(f) = a_1$ and we win by showing that if $f$ is an eigenform with $a_1 = 0$ then $f = 0$. This is because $\deriv{}{q} (T_n f) = a_n$ so if $T_n f = \lambda f$ for $\lambda \neq 0$ then we also have all $a_n = 0$. 
\bigskip\\
Let's do this in more detail. Let $\ell$ be the characteristic of $F = (\TT \ot \Z[1/2N]) / \m$ and $R = (\TT \ot \ZZ[1/2N]) \ot_{\ZZ} \FFbar_\ell$. And let $M = T_0(J) \ot_{\ZZ} \FFbar_\ell$. Then there is an exact sequence,
\begin{center}
\begin{tikzcd}
\m \ot_{\ZZ} \FFbar_\ell \arrow[r] & R \arrow[r] & F \ot_{\ZZ} \FFbar_\ell \arrow[d, equals] \arrow[r] & 0
\\
& & \prod_{i \in I} \FFbar_\ell \arrow[d, two heads]
\\
0 \arrow[r] & \m R \arrow[r] & R \arrow[r] & \FFbar_\ell \arrow[r] & 0
\\
& & & F \arrow[u, hook]
\end{tikzcd}
\end{center} 
by tensoring the inclusion $F \embed \FFbar_\ell$ we get $T_0(J) / \m T_0(J) \embed M / \m M$. As $R$-modules,
\[ (M / \m M)^\vee \cong M^\vee[\m] \cong H^0(X_{\FFbar_\ell}, \Omega^1_{X/ \FFbar_\ell})[\m] \]
\end{proof}

\begin{theorem}
if $f : X \to S$ is a smooth proper relative curve then $R^i f_* \Omega_{X/S}$ commutes with all base change.
\end{theorem}

\begin{proof}
If $S$ is reduced this comes from Grauert. Otherwise use cohomology and base change.
\end{proof}

In particular: if $f \in S_2(N)_{\FFbar_\ell}[\m]$ is nonzero can lift to char $0$ and then $\deriv{}{q} (T_n f) = a_n(f)$ follows from analysis. 

\begin{lemma}
For every $\m \subset \TT_{\Z[1/2N]}$. Then $T_0(J) / \m T_0(J)$ is free over $\TT_{\Z[1/2N]} / \m$ generated by $\deriv{}{q}$. 
\end{lemma}

\begin{proof}
$\dim_F T_0(J) / \m T_0(J) = \dim_{\FFbar_\ell} M^\vee[\m]$ then let $a_n$ be the image of $T_n$ in $R / \m = \FFbar_\ell$ then if $f \in S_2(N)_{\FFbar_\ell}[\m]$ and $T_n(f) = a_n(F)$ so $f$ is a multiple of $q + a_2 q^2 + \cdots$. 
\end{proof}

\section{Final Talk}


\subsection{The Eisenstein Quotient}

Last time we discussed how Mazur's theorem follows if we can show that all elliptic curves over $\Q$ with an order $N$ subgroup have potentially good reduction away from $2 N$. To prove this, we are going to need to produce a rank zero optimal quotient $A$ of $J_0(N)$ such that $X_0(N) \to J_0(N) \to A$ separates the two cusps $0$ and $\infty$ (in fact, I think it suffices to just know that $A$ is nonzero). First, since $X_0(N)$ has positive genus $[\infty] - [0]$ is nonzero in $J_0(N)$ and furthermore,
\[ (N-1) ([\infty] - [0]) = \div{\Delta(z)} - \div{\Delta(N z)} \]
so this point is torsion of order dividing $N-1$. The following result allows us to construct such a quotient.

\begin{theorem}
Let $A$ be an abelian variety over $\Q$ and $N, p$ be distinct odd primes. Suppose,
\begin{enumerate}
\item $A$ has good reduction away from $N$
\item $A$ has completely toric reduction at $N$
\item the Jordan-H\"{o}lder factors of $A[p](\overline{\Q})$ are $1$-dimensional and either trivial or cyclotomic characters. 
\end{enumerate}
Then $\rank A(\Q) = 0$.
\end{theorem}

\newcommand{\fppf}{\mathrm{fppf}}

\begin{proof}
Let $\cA$ be the Neron model of $A$ over $\Z$ and consider the sequence,
\begin{center}
\begin{tikzcd}
0 \arrow[r] & K_n \arrow[r] & \cA^\circ \arrow[r, "p^n"] & \cA^\circ \arrow[r] & 0
\end{tikzcd}
\end{center}
which is exact in the fppf topology (though may not be in the \etale toplogy say if $\cA^\circ$ contains a $\Gm$ factor) and $K_n$ is quasi-finite (and finite over $\Z[1/N]$) flat (note that $[p^n] : \cA \to \cA$ need not be surjective if $\pi_0(\cA_{\FF_N})$ has $p$-torsion). This gives an injection of fppf cohomology,
\[ H^0_{\fppf}(\Spec{\Z}, \cA^\circ) \ot \Z / p^n \Z \embed H^1_{\fppf}(\Spec{\Z}, K_n) \]
Now you do a lot of hard work using the Jordan-H\"{o}lder factors to show that $H^1_{\fppf}(\Spec{\Z}, K_n)$ is bounded independently of $n$. Therefore $\rank \cA^\circ(\Z) = 0$ and hence $\rank A(\Q) = 0$ because $\cA^\circ(\Z) \subset \cA(\Z) = A(\Q)$ has finite index. 
\end{proof}

Recall that the simple isogeny factors of $J_0(N)$ correspond to Galois orbits of eigenforms $f \in S_2(N)$. Therefore, optimal quotients are in correspondence with subsets of these Galois orbits. Let,
\[ S_p = \{ f \mid A_f \text{ satisifes hypotheses of above theorem for } (N, p) \} \]
For the above theorem, fix $p$ dividing the order of $[\infty] - [0] \in J_0(N)$. Then $J_0(N)[p](\Q) \neq 0$ so some $A_f[p]$ contains a trivial representation. This will alow us to conclude that $A_f[p]$ satisfies the hypotheses of the theorem (e.g. if $f$ is defined over $\Q$ then $A_f$ is an elliptic curve so by the Weil pairing the other term in $A_f[p]$ is cyclotomic). Therefore, $S$ is nonempty. Thus the largest quotient of $J_0(N)$ we can show has rank zero it given by taking the simple factors in $S_p$. Explicitly, let $\p_f = \ker{(\T \to \Z)}$ given by taking the eigenvalues $T_\ell \mapsto a_\ell(f)$. Then $A_f = J_0(N)/ \p_f J_0(N)$. Then we define an ideal $\gamma_\a = \bigcap\limits_{f \in S_p} \p_f$ and the corresponding optimal quotient $J^\a = J_0(N) / \gamma_\a J_0(N)$ is called the \textit{Eisenstein quotient}. Up to isogeny,
\[ J^\a \simeq \prod_{f \in S_p} A_f \]
Now we explain the notation and the name Eisenstein quotient. 

\begin{defn}
The \textit{$p$-Eisenstein ideal} $\a$ is the ideal of $\T$ generated by $p$ and,
\[ T_\ell - (\ell + 1) \]
for each prime $\ell \neq N$.
\end{defn}

\begin{prop}
We have,
\begin{enumerate}
\item $f \in S_p \iff a_\ell(f) \equiv \ell + 1 \mod p \iff \p_f \subset \a$
\item \[ \gamma_\a = \bigcap\limits_{f \in S_p} \p_f = \bigcap\limits_{\p_f \subset \a} \p_f = \bigcap\limits_{\substack{ \p \subset \a \\ \p \text{ minimal}}} \]
\item $\a$ is maximal and $\TT / \a = \FF_p$
\end{enumerate}
\end{prop}

\begin{rmk}
Since the level $2$ Eisenstein series is an eigenform with $a_\ell(E_2) = \ell + 1$ we see why called the $p$-Eisenstein ideal and that $S_p$ consists exactly of those eigenforms $f$ congruent mod $p$ to $E_2$ away from $N$ meaning $a_\ell(f) \equiv a_\ell(E_2) \mod p$. 
\end{rmk}

\begin{lemma}
$J_0(N)[\a^\infty] \iso J^\a[\a^\infty]$
\end{lemma}

\begin{proof}
DO THIS!!!
\end{proof}


Finally, we compute $T_\ell ([\infty] - [0]) = (\ell + 1) ([\infty] - [0])$ which shows that $[\infty] - [0] \in J_0(N)[\a^\infty]$ and is nonzero hence is nonzero in $A = J^\a$. Therefore, we have found an optimal quotient satisfying the required properties. 

\subsection{Completing the Proof}

\newcommand{\tors}{\mathrm{tors}}
\newcommand{\fin}{\mathrm{fin}}

We first need a lemma. (HAS THIS BEEN PROVEN PREVIOUSLY??)

\begin{prop}
Let $p \neq 2$ and $f : H \to G$ be a morphism of finite flat group schemes over a DVR $R$ with mixed characteristic $0$ and $p$. Let $K = \Frac{R}$. If $f_K : H_K \to G_K$ is a closed immersion then $f$ is a closed immersion.
\end{prop}

\begin{proof}
DO THIS!!
\end{proof}

\begin{theorem}
Let $(E_\Q, C)$ be a pair of an elliptic curve over $\Q$ and a cyclic subgroup $C$ of order $N$ with $N$ an odd prime. Then $E$ has potentially good reduction at all odd primes $p \neq N$. 
\end{theorem}

The proof of this theorem relies on the following result from last time.

\begin{prop}
Let $\cA$ be the Neron model over $\Z[1/2N]$ of any nonzero optimal quotient $A$ of $J$. Define $X_0(N)_{\Q} \to J \to A$ by sending the cusp $\infty$ to $0$ and let $f$ denote the morphism extensing this over $\Spec{\Z[1/2N]}$. For any $p \ndivides 2 N$ the point $\infty \in X_0(N)(\ZZ_{(p)})$ is the only point reducing to $\infty \in X_0(N)(\FF_p)$ that also maps to $0$ in $\cA(R)$ under $f$.
\end{prop}

\begin{proof}[Proof of Theorem]
We are going to take $A$ in the above theorem to be the Eisenstein quotient. Let $R = \ZZ_{(p)}$. We consider the reduction of $(E, C)$, by the Neron mapping property the subgroup $C$ spreads out to $\E$ (since $p \neq N$ the subgroup can be split after an unramified extension of $R$ and the generator spreads out by the Neron mapping property). This gives a morphism,
\begin{center}
\begin{tikzcd}
\Spec{\Q} \arrow[d] \arrow[r] & Y_0(N)_{\Q} \arrow[r] & X_0(N)_{\Q} \arrow[d]
\\
\Spec{R} \arrow[rr, "\xi"] & & X_0(N)_{R} 
\\
\Spec{\FF_p} \arrow[u] \arrow[rr, "\bar{\xi}"] & & X_0(N)_{\FF_p} \arrow[u]
\end{tikzcd}
\end{center}
having potentially multiplicative reduction at $p$ means exactly that $\bar{\xi}$ hits one of the cusps of $X_0(N)_{\FF_p}$. Since the Atkin-Lehner involution permutes the two cusps $0$ and $\infty$ we may assume that $\bar{\xi}$ hits $\infty$ (otherwise replace $(E, C)$ by $(E/C, E[N]/C)$). Then we get a diagram,
\begin{center}
\begin{tikzcd}
\Spec{\Q} \arrow[r, "\xi_{\Q}"] \arrow[d] & X_0(N)_{\Q} \arrow[r] \arrow[d] & J_0(N) \arrow[r] \arrow[d] & A \arrow[d]
\\
\Spec{R} \arrow[r, "\xi"] & X_0(N)_{R} \arrow[r] & J_0(N)_{R} \arrow[r] & \cA
\end{tikzcd}
\end{center}
Since the section $\infty_{R}$ and the map $\xi$ both reduce mod $p$ to $\infty_{\FF_p} \in X_0(N)_{\FF_p}$ they both map to $0 \in \cA_{\FF_p}$. We will use the previous results to show that $\xi = \infty_{R}$ which contradicts the fact that $E$ is smooth so its generic point factors through the open $Y_0(N)_{\Q}$. 
\bigskip\\
To apply the previous theorem, we need to show that the bottom row composes to the zero morphism. Call this point $P \in \cA(R) = A(\Q)$ which is torsion because we know $A$ has rank $0$. Let $m$ be the order of $P$ then consider the map of finite flat group schemes,
\[ \underline{\Z / m \Z} \to (\cA[m])_{\fin} \]
over $R$ arising from the inclusion $\Z / m \Z \embed A[m](\Q) = \cA[m](R)$ which spreads out by the Neron mapping property since $\underline{\Z / N \Z}$ is smooth over $R$ and lands in the finite part\footnote{The notes claim that $\cA[N]$ is finite, I don't think this is true.}. This is a closed immersion on the generic fiber and therefore a closed immersion. Therefore, $\Z / m \Z \embed \cA[m]_{\FF_p}$ but we know that $P$ reduces to $0$ mod $p$ so $P = 0$ proving the claim.   
\end{proof}


Now we are ready to prove the main theorem. 

\begin{theorem}[Mazur]
Let $N$ be prime and $N \ge 11$ and not $13$. Then there are no elliptic curves over $\Q$ with a torsion subgroup of order divisible by $N$.
\end{theorem}

\begin{proof}
Suppose $E(\Q)[N]$ is nonempty. We proved that $E$ has potentially good reduction at $3$. We will now show that $N \le 7$. First, suppose that $E$ has good reduction at $3$ so its Neron model $\E$ over $\Spec{\Z_{(3)}}$ is proper. Then since $\E[N]$ is finite over $\Spec{\Z_{(3)}}$ we have,
\[ \Z / N \Z \embed E(\Q)_{\tors} \to \E(\FF_3) \]
is injective. But $\E_{\FF_3}$ is an elliptic curve so by the Hasse-Weil bound,
\[ \# \E_{\FF_3} \le \lfloor 4 + 2 \cdot \sqrt{3} \rfloor = 7 \]
Otherwise, $E$ has additive reduction at $3$ and let $\E$ be its Neron model over $\Spec{\Z_{(3)}}$. Consider the exact sequence,
\begin{center}
\begin{tikzcd}
0 \arrow[r] & \E^0_{\FF_3} \arrow[r] & \E_{\FF_3} \arrow[r] & \pi_0(\E_{\FF_3}) \arrow[r] & 0 
\end{tikzcd}
\end{center}
Since we are in the case of additive reduction, $\E^0_{\FF_3} = \Ga$. Consider the map of finite flat group schemes,
\[ \underline{\Z / N \Z} \to (\E[N])_{\fin} \]
over $R = \Z_{(3)}$ arising from the inclusion $\Z / N \Z \embed E[N](\Q)$ which spreads out by the Neron mapping property since $\underline{\Z / N \Z}$ is smooth over $R$ and lands in the finite part\footnote{The notes claim that $\E[N]$ is finite, I don't think this is true.} since the generator is a section over $R$. This map is, by construction, a closed immersion on the generic fiber so by Prop 3.3 this map is a closed immerison. In particular,
\[ \Z / N \Z \embed \E[N]_{\FF_3} \]
We will then prove, in the case of additive reduction, that $\# \pi_0(\E_{\FF_3}) \le 4$ so if $N > 4$ then $\Z / N \Z$ lands in $\E^0_{\FF_3}$ but $\# \E^0_{\FF_3}(\FF_3) = \# \Ga(\FF_3) = 3$ so this is impossible. Therefore $N \le 4$. 
\end{proof}

\subsection{Kodaira Classification of Special Fibers}

\begin{defn}
An \textit{elliptic surface} is a regular connected $2$-dimensional scheme $X$ equipped with a map proper map $\pi : X \to C$ to a regular connected $1$-dimensional scheme $C$ (e.g. a Dedekind scheme) whose generic fiber is a smooth geometrically-connected curve of genus $1$.
\end{defn}

\begin{rmk}
The map $\pi : X \to C$ is dominant (since the generic fiber is nonempty) and $X$ and $C$ are integral so we get an injection $K(C) \embed K(X)$ hence $\stalk{X}{x}$ are torsion-free $\stalk{C}{\pi(x)}$-modules and hence flat since the base is a DVR. Thus $\pi$ is flat. Since $X$ is irreducible the fibers must be all pure dimension $1$. Then $\pi$ is also proper so since $C$ is normal and the generic fiber is geometrically-connected $\pi_* \struct{X} = \struct{C}$ so $\pi$ has connected fibers. Thus every fiber of $\pi$ is a connected genus $1$ curve (not necessarily reduced) and $\pi$ is smooth iff these are smooth genus $1$ curves. 
\end{rmk}

\begin{defn}
We say that an elliptic surface $X$ is \textit{pointed} if it furthermore equipped with a section $\sigma : C \to X$ of $\pi$. We say that $\pi$ is \textit{relatively minimal} if the fibers contain no $(-1)$-curves. In this case we say that $X$ is a \textit{minimal elliptic surface}.
\end{defn}

\begin{rmk}
A minimal elliptic surface $\pi : X \to C$ is exactly the data of compatible minimal regular models of its generic fiber $E$ over each DVR $\stalk{C}{p}$. Therefore, to classify the fibers of minimal elliptic surfaces, it suffices to classify the special fibers of minimal regular models of genus $1$ curves and, in the equicharacteristic case, then exhibit these fibers in complete families with regular total spaces. (IT IS OBVIOUS THAT THIS CAN ALWAYS BE DONE??)
\end{rmk}

\begin{rmk}
DO I NEED TO ASSUME $S$ IS EXCELLENT FOR EVERYTHING I WANT TO BE TRUE.
\end{rmk}

\subsection{General Properties of Regular Models}

\begin{rmk}
Liu's book covers this topic well. 
\end{rmk}
\renewcommand{\Div}{\mathrm{Div}}

Let $X \to S = \Spec{R}$ be a regular proper model with special fiber $X_s$. Let $\Gamma_i$ be the irreducible components of $X_s$ appearing with multiplicity $d_i$. These are (possibly singular) proper curves over $\kappa = R / \m$. 
\bigskip\\
We want to define an intersection pairing on $X$.  For any nonzero horizontal divisor $D$ we should have $C \cdot X_s > 0$. However, $\struct{X}(X_s) = \struct{X}$ because it is cut out by a global function $\pi \in \Gamma(X, \struct{X})$. Thus we cannot have an intersection pairing invariant under linear equivalence. This is because $S$ is not ``complete'' so we can deform $X_s$ to the ``boundary'' where it vanishes. Arakalov theory solves this but instead we will just ask that $i(-,D)$ is invariant under linear equivalence in its first factor. However, there is still an issue if $D$ contains a horizontal divisor then $X_s \cdot D > 0$ because definition of the intersection pairing is symmetric. To fix this we restrict the second coordinate to only vertical divisors.

\begin{lemma}
There is a unique bilinear intersection pairing,
\[ i_s :  \Div(X) \times \Div_s(X) \to \Z \]
which satisfies the following properties,
\begin{enumerate}
\item when $D$ and $E$ share no components then,
\[ (D, E) \mapsto \sum_{x \in D \cap E} i_x(D, E) [\kappa(x) : \kappa] \]

\item if $D \sim D'$ then $i_s(D, E) = i_s(D', S)$

\item the restriction $i_s : \Div_s(X) \times \Div_s(X) \to \Z$ is symmetric

\item if $E$ is effective and $E \subset X_s$ as a subscheme then,
\[ i_s(D, E) = \deg_{\kappa(s)} \struct{X}(D)|_E \]
\end{enumerate}
\end{lemma}

Since $X$ is regular and flat over a regular base, the fibers are Gorenstein (\chref{https://stacks.math.columbia.edu/tag/0BJJ}{Tag 0BJJ}). Therefore, there exists a relative dualizing line bundle $\omega_{X/S}$ (\chref{https://stacks.math.columbia.edu/tag/0E6R}{Tag 0E6R}). Choose a divisor $K_{X/S}$ such that $\struct{X}(K_X) \cong \omega_{X/S}$.  

\begin{lemma}
Let $X$ be a regular surface flat and proper over $\Spec{R}$.
\begin{enumerate}
\item $X$ is minimal iff $K_{X/S}$ is numerically effective
\item $K_{X/S} \cdot X_s = 2 p_a(X_\eta) - 2$
\item if $X$ is minimal and $p_a(X_\eta) = 1$ then $K_{X/S} \sim 0$. 
\end{enumerate}
\end{lemma}

\begin{lemma}
Let $X$ be a regular surface flat and proper over $\Spec{R}$. Let $\Gamma \subset X_s$ be an irreducible component and $k' = H^0(\Gamma, \struct{\Gamma})$. Then,
\[ K_{X/S} \cdot \Gamma = 0 \]
if and only if one of the following is true,
\begin{enumerate}
\item $H^1(\Gamma, \struct{\Gamma}) = 0$ and $\Gamma^2 = -2 [k' : \kappa(s)]$ in which case $\Gamma$ is a conic over $k'$
\item $p_a(X_\eta) = 1$ and $\Gamma$ is a connected component of $X_s$. 
\end{enumerate}
\end{lemma}

These together give us numerical conditions on the structure of the special fiber of a minimal regular model of $E$. Indeed, since $X$ is proper over $\Spec{R}$ and $X_\eta$ is geometrically reduced and connected $X_s$ is connected. Therefore, either $X_s = d \Gamma$ and $p_a(\Gamma) = 1$ or every component is a conic of self-intersection $\Gamma^2 = - [k' : \kappa(s)]$. Moreover, $X_s \sim 0$ so $X_s \cdot \Gamma = 0$. Since,
\[ X_s = \sum_{i} m_i \Gamma_i \]
this gives a formula,
\[ \Gamma_j^2 = - \frac{1}{m_j} \sum_{i \neq j} m_i \Gamma_i \cdot \Gamma_j \]
This gives a restriction on the number of components $\Gamma_i$ can meet given its self intersection. These two conditions limit the possible intersection graphs. The remainder of the classification of special fibers is purely enumerating the possible combinatorial types of intersection graphs with multiplicities. 

\begin{theorem}[Kodaira]
WRITE OUT THE TYPES
\end{theorem}

\subsection{The Relation to Neron Models}

\begin{prop}
Let $E$ be an elliptic curve over $K$ with $K = \Frac{R}$ and $R$ a DVR. Then let $X$ be the minimal regular model of $E$ over $R$. By properness, $X$ is a pointed elliptic surface with $\sigma_K = 0 \in E(K) = X(K)$. Let $\E \subset X$ be the open subscheme obtained by removing the points of $X$ which are singularities of the special fiber. Then $\E$ is the Neron model of $E$.
\end{prop}

\begin{proof}
We verify the Neron mapping property in two steps. First, if $\Spec{R'} \to \Spec{R}$ is a finite \etale cover then by properness,
\[ X(R') = E(K') \]
so it suffices to show that maps $\Spec{R'} \to X$ land in $\E$. Indeed, let $x \in X$ be the image of a closed point $\m \subset R'$ then we get a diagram,
\begin{center}
\begin{tikzcd}
R'_\m \arrow[from = r] & \stalk{X}{x}
\\
R \arrow[u] \arrow[ru]
\end{tikzcd}
\end{center} 
and $R \to R'_\p$ is \etale so the uniformizer lands in $\m \sm \m^2$. Thus by commutativity $\pi \in \m_x \sm \m_x^2$. Since $\stalk{X}{x}$ is regular this implies that $\stalk{X_s}{x} = \stalk{X}{x} / \pi$ is regular so $x \in \E$. Thus,
\[ \E(R') \to X(R') \to E(K') \]
are all isomorphisms. This can be used to show that the group law on $E$ extends uniquely to $\E$. Now, if $T \to \Spec{R}$ is a smooth $R$-scheme and $f_K : T_K \to E$ is a map then by properness $f_K$ extends to a rational map $f : T \rat X$ defined away from codimension $2$. However, $T \to \Spec{R}$ has sections through any point after some \etale extension $\Spec{R'} \to \Spec{R}$. Since we know maps $\Spec{R'} \to X$ land in $\E$ we conclude that $f$ factors through $\E \embed X$. Finally, since $\E$ is a group by a translation argument $f : T \rat \E$ is everywhere defined. Then,
\[ \Hom{R}{T}{\E} \to \Hom{K}{T_K}{E} \]
is surjective and since $\E \to \Spec{R}$ is separated it is injective.
\end{proof}


\begin{lemma}
Suppose that $E$ is an elliptic curve over $\Q$ with additive reduction at $p$ and $\E$ its Neron model over $R$. Then $\# \pi_0(\E_{\FF_p}) \le 4$.
\end{lemma}

\begin{proof}
Additive reduction means that the special fiber of $\E^0$ is $\Ga$ (since the residue field is perfect). Therefore, all the geometric components (IS IT POSSIBLE FOR $\pi_0(\E_{\FF_p})$ NONSPLIT??) of $\E^0$ are isomorphic to $\Ga$. From our construction of the Neron model this is equivalent to, in the special fiber of the minimal regular model, each genus $1$ component having a cusp and each genus $0$ component intersecting the other components in exactly one point. By inspection of the possible types (II, III, IV, HOW MANY OTHERS, there can be a maximum of $4$ geometric components with multiplicity $1$ under these restrictions. 
\end{proof}


\end{document}

