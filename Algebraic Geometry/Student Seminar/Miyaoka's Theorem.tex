\documentclass[12pt]{article}
\usepackage{hyperref}
\hypersetup{
    colorlinks=true,
    linkcolor=blue,
    filecolor=magenta,      
    urlcolor=blue,
}

\usepackage{import}
\import{../}{AlgGeoCommands}

\begin{document}
\section{Pseudo-effective}

\begin{defn}
A divisor class $D \in N^1(X)_{\RR}$ is \textit{pseudo-effective} if it is in the closure of the cone of effective divisors. 
\end{defn}

\begin{defn}
A class $\alpha \in N_1(X)_{\RR}$ is \textit{movable} if $\alpha \cdot D \ge 0$ for any effective Cartier divisor $D$.
\end{defn}

\begin{prop}
If $D$ is pseudo-effective if and only if $D \cdot \alpha \ge 0$ for all movable classes $\alpha$. 
\end{prop}

\begin{proof}
If $D$ is pseudo-effective then by definition,
\[ D = \lim_{t \to 0} D_t \]
for $D_t$ effective $\RR$-divisors. If $\alpha$ is movable then by definition $D_t \cdot \alpha \ge 0$ for $t > 0$. Since intersection products are continuous (they are really polynomials in the coefficients) we have $D \cdot \alpha \ge 0$. The coverse holds for duals of cones in finite-dimensional vector spaces. Indeed, if $D$ is not pseudo-effective, the separating hyperplane theorem ensures the existence of a numerical curve class $\alpha$ such that $E \cdot \alpha \ge 0$ on all effective divisors, i.e. $\alpha$ is movable, but $D \cdot \alpha < 0$.
\end{proof}



\section{Miyaoka's Theorem}

\section{Relations Between Notions of Semipositiviity}

\begin{theorem}[Mehta-Ramanathan]
Let $X$ be a normal projective variety of dimension $\ge 2$ and $H$ an ample divisor. Let $\E$ be torsion-free sheaf. Then for $m \gg 0$ the restriction of $\E$ to a gneral member $Y \in |mH|$ is $H|_Y$-semistable if and only if $\E$ is $H$-semistable.
\end{theorem}

Therefore, we can reduce to sufficently large degree complete intersection curves.


\section{The Main Theorem}

\begin{prop}
Let $X$ be a smooth projective variety over a field of characteristic $p > 0$. Assume there is a $\Q$-divisor $D$ with $\deg{D} > 0$ such that when restricted to a generated complete intersection curve $\F(-D)$ ample and $(\T_X/F)(-D)$ negative. Then on the open $U$ where $\F \subset \T_X$ is a subbundle we have that $\F$ is a $p$-closed foliation.
\end{prop}

\begin{proof}
The bracket defines an $\struct{X}$-linear map $\bigwedge^2 \F \to \T_X / \F$. This must be zero because $(\wedge^2 \F)(-D)$ is ample but $(\T_X/\F)(-2D)$ is negative if restricted to a general curve. Hence $\F$ is a foliation. 
\bigskip\\
The $p^{\text{th}}$-power map induces $F^* \F \to (\T_X / \F)$ then $F^* \F(-D)$ is ample on a generic curve but $(\T_X/\F)(-D)$ is negative so the map is zero. 
\end{proof}


\begin{theorem}
Let $X$ be a normal projective variety over an algebraically closed field of characteristic zero. If $\T_X$ is not generically semi-negative then $X$ is uniruled.
\end{theorem}

\begin{proof}
Let $\F \subset \T_X$ be the maximal destabilizer and we assume $\mu(\F) > 0$. Then let $D = c H$ with $\mu(\F) > c > \mu_{\max}(\T_X/\F)$ so that $\F(-D)$ is ample and $(\T_X/\F)(-D)$ is negative on the generic complete intersection curve. Then applying the previous result we get modulo almost all primes a $p$-closed foliation $\F \subset \T_X$. Then we apply the previous theorem so for almost all $p$ the reduction of $X$ is uniruled by rational curves $C$ of degree bounded uniformly by,
\[ C \cdot H \le \frac{3 H^n}{(\det{\F}) \cdot K_X} \]
because $\mu(\F) > 0$ so the denominator is nonzero. Therefore, because the Hom scheme is finite type $X$ must be uniruled.
\end{proof}

{\color{red} Is it true that $X$ uniruled implies $\Omega_X$ not generically semipositive?}

\begin{proof}
Let $X$ be uniruled by $f : \P^1 \times B \rat X$ and $\Omega_X$ be generically semipositive. Consider a generic complete intersection curve $C \subset X$ and its preimage $C' \subset \P^1 \times B$. Then $g : C' \to C$ is finite. Since $\Omega_X |_C$ is semipositive $g^* \Omega_X|_C$ is semipositive so $\Omega_X|_C \to \Omega_{\P^1 \times B}|_{C'}$ which is generically injective and of the same rank means that $\Omega_{\P^1 \times B}|_{C'}$ must also be semipositive. However, $\Omega_{\P^1 \times B} = \Omega_{\P^1} \boxtimes \Omega_{B}$ and $C'$ is a generic complete intersection curve so $\Omega_{\P^1}|_{C'}$ is negative giving a contradiciton.
\end{proof}

\section{Supplementary Lemmas}

\newcommand{\Nef}{\mathrm{Nef}}
\newcommand{\Eff}{\mathrm{Eff}}
\newcommand{\NE}{\mathrm{NE}}

\begin{prop}
Let $C$ be a smooth projective curve over an algebraically closed field of characteristic zero. Let $\E$ be a locally free sheaf of rank $r$ and $\pi : \P(\E) \to C$ the projective bundle. Let $M = c_1(\struct{\P(\E)}(1)) - (1/\rank{\E}) \pi^* c_1(\E)$. Then the following are equivalent,
\begin{enumerate}
\item for any finite $f : C' \to C$ then $f^* \E$ is $\mu$-semistable
\item $M$ is nef
\item $\Nef(X) = \RR_{+} M + \RR_{+} \pi^* P$  for $P \in N^1(C)$ a generator
\item $\ol{\NE}(X) = \RR_{+} M^{r-1} + \RR_{+} M^{r-2} \pi^* P$
\item $\ol{\Eff}(X) = \Nef(X)$
\item $\ol{\Eff}(X) \subset \Nef(X)$
\item $M - \pi^* D$ is not pseduo-effective for any $\Q$-divisor $D$ with $\deg{D} > 0$
\item $M + \pi^* D$ is ample for some $\Q$-divisor $D$ with $0 < \deg{D} < 1/r!$
\item $M - \pi^* D$ is not pseudo-effective, where $D$ is some $\Q$-divisor with $0 < \deg{D} < 1/r!$
\item $\E$ is $\mu$-semistable.
\end{enumerate}
\end{prop}

\begin{proof}
Let $r = \rank{\E}$ and $X = \P(\E)$. 
By the canonical bundle formula, setting $\xi := c_1(\struct{\P(\E)}(1))$ we get
\[ \xi^{r} = \xi^{r-1} \pi^* c_1(\E) \]
Therefore,
\[ M^r = (\xi - 1/r \, \pi^* c_1(\E))^r = \xi^r - \xi^{r-1} \pi^* c_1(\E) = 0 \]
since $(\pi^* c_1(\E))^i = 0$ for $i > 1$. This implies that,
\[ M^{r-2} \cdot (M + \pi^* D) \cdot (M - \pi^* D) = M^r - M^{r-2} (\pi^* D)^2 = 0 \]
since the square of any pullback divisor is zero.


Note that $\ol{\NE}(X) \subset N_1(X)$ is the dual cone of $\Nef(X) \subset N^1(X)$ basically by definition. Let $P \in N^1(C)$ be a generator. We know that $N^1(X)$ has a basis $M$ and $P$.

Suppose $D = a M + b \pi^* P$ is nef. Since $\pi^* P \cdot M^{r-2}$ is a line in a fiber which is an effective curve we see $a = D \cdot (\pi^* P) \cdot M^{r-2} \ge 0$. Furthermore, $D^r = a^{r-1} b \ge 0$ so for $a > 0$ this implies $b \ge 0$ (which is also clear for $a = 0$). Since $\pi^* P$ is nef we see $(b) \iff (c)$.


Lets show that $M^{r-1}$ and $M^{r-2} \pi^* P$ form a basis of $N_1(X)$. Indeed, against the basis $M, \pi^*P \in N^1(X)$ the intersection pairing is given by the matrix
\[ 
\begin{pmatrix}
0 & 1
\\
1 & 0
\end{pmatrix} \]
which is nondegenerate. Therefore $(c) \iff (d)$ using the intersection pairing. 


If $M$ is nef then $M + \epsilon \pi^* D$ by $(c)$ is in the interior of the nef cone hence is ample. If $D = a M + b \pi^* P$ is pseduo-effective then $D \cdot (M + \epsilon \pi^* D)^{r-2} \in \ol{\NE}(X)$ and so is its limit $\epsilon \to 0$ so $D \cdot M^{r-2} = a M^{r-1} + b M^{r-2} \pi^* P \in \ol{\NE}(X)$ hence $a, b \ge 0$ by $(d)$. If $a, b > 0$ then $D$ is ample and hence effective so we conclude $(e)$. 

\end{proof}

\begin{lemma}
Let $f : C' \to C$ be a separable surjective $k$-map of smooth complete curves. Let $\E$ be a bundle on $C$. Then the Harder-Narishiman filtration of $f^* \E$ is the pullback of the Harder-Narishiman filtration of $\E$.
\end{lemma}

\begin{proof}
Note that $\deg{f^* \E} = \deg{f^* \det{\E}} = (\deg{f}) \cdot (\deg{\E})$. By factoring the morphism it suffices to consider the case where $f$ is Galois with galois group $G$. We need to show that if
\[ 0 = \E_0 \subsetneq \E_1 \subsetneq \cdots \subsetneq \E_r = \E \]
is the Harder-Narisimhan filatration then $f^* \E_i$ is the Harder-Narisimhan filatration of $\E$. Since the slopes of the graded parts are still strictly decreasing after applying $f^*$, it suffices to show that if $\E$ is semistable then $f^* \E$ is semistable and then we apply this to the graded parts (here we use flatness of $f$ to ensure that $f^*$ is exact). Let $\F \subset f^* \E$ be the maximal destabilizer. Consider the $G$-action on $f^* \E$ then $\sigma_g : f^* \E \to f^* \E$ must preserve $\F$ since it is canonical (there is a unique maximal subbundle containing all subbundles of maximal slope) and hence $\F$ descends to $\F_0 \subset \E$ but $\mu(\F_0) = \deg{f} \mu(\F)$ so since $\mu(\F_0) \le \mu(\E)$ we must have $\mu(\F) = \mu(f^* \E)$ and hence $f^* \E = \F$. 
\end{proof}

{\color{red} IS THE FOLLOWING TRUE}

\begin{prop}
Let $C$ be a smooth projective curve over an algebraically closed field of characteristic zero. Let $\E$ be a locally free sheaf of rank $r$ and $\pi : \P(\E) \to C$ the projective bundle. Let $M = c_1(\struct{\P(\E)}(1))$. Then the following are equivalent,
\begin{enumerate}
\item for any finite $f : C' \to C$ then $f^* \E$ is semipositive
\item $M$ is nef
\item $M - \pi^* D$ is not pseduo-effective for any $\Q$-divisor $D$ with $\deg{D} > 0$
\item $M + \pi^* D$ is ample for some $\Q$-divisor $D$ with $0 < \deg{D} < 1/r!$
\item $M - \pi^* D$ is not pseudo-effective, where $D$ is some $\Q$-divisor with $0 < \deg{D} < 1/r!$
\item $\E$ is semipositive.
\end{enumerate}
\end{prop}

\begin{proof}
Notice that $M^2 = c_1(\E)$
\end{proof}


\begin{cor}
Let $(X, H)$ be a normal, projective, polarized scheme over a ring $R$ of characteristic zero, finitely generated over $\ZZ$. Let $\E$ be a torsion free sheaf on $X$. Let $K = \ol{\Frac{R}}$. If $\E_K$ is $H$-semistable on $X_K$ then $\E$ is $H$-semistalbe on reduction mod $p$ for almost all $p$.
\end{cor}

\begin{proof}
Let $C \sim m H^{n-1}$ be a general complete intersection curve on $X$ of large degree. Then we may assume that $\E|_C$ is $\mu$-semistable on $C_K$ hence using the above notation $M + c \pi^* H$ is ample on $\P(\E_C)_K$ but ampleness is an open condition for projective morphisms so this is satisfied for $\E|_C$ modulo almost every $p$, which implies $H$-semistability modulo almost every prime. 
\end{proof}

\begin{lemma}
Let $C$ be a smooth curve and $\E$ a vector bundle. Then $\E$ is $\mu$-semistable if and only if $\E(-\mu)$ is semipositive.
\end{lemma}

\begin{proof}
This is almost immediate from the definition. Semistable means that for any $\E \onto \L$ we have $\mu(\L) \ge \mu(\E)$ and semipositive means $\mu(\L) \ge 0$ so shifting by $- \mu(\E)$ these are the same condition. 
\end{proof}

\begin{cor}
Over a field of characteristic zero, if $\E$ is $H$-semistable then $\E^{\ot n}$ is $H$-semistable. Hence the direct summands $S^m \E$ and $\bigwedge^m \E$ are $H$-semistable. More generally, if $\E_1, \E_2$ are $H$-semistable then $\E_1 \ot \E_2$ are $H$-semistable. 
\end{cor}

\begin{proof}
We can reduce to a complete intersection curve of sufficiently divisible degree. Suppose $\E_1, \E_2$ are $\mu$-semistable this means that $\struct{\P(\E_i)}(1)$ are nef for $i = 1,2$.
\end{proof}

\begin{cor}

\end{cor}

\begin{proof}
We can reduce to a complete intersection curve of sufficiently divisible degree. 
Then we just need to show that if $\E_1, \E_2$ are semipositive then $\E_1 \ot \E_2$ is semipositive. Consider $\E_1 \ot \E_2 \onto \F$ where $\F$ is a vector bundle. {\color{red} I ONLY SEE HOW TO DO THIS IF ONE IS GLOBALLY GENERATED?}
\end{proof}

\begin{defn}
Let $X$ be a projective variety and $\F$ a torsion-free coherent sheaf. We say that $\F$ is \textit{generically $H$-semipositive} if $\mu_{\min}(\F) \ge 0$.
\end{defn}

\begin{rmk}
This is equivalent to ``generically nef''. {\color{red} WHY?}
\end{rmk}

\newpage

\section{Talk}

\section{Harder-Narasimhan Filtration}

Let $X$ be a smooth projective variety of dimension $n$ with an ample divisor $H$. Then for any torsion-free coherent sheaf $\E$ define,
\[ \mu(\E) := \frac{c_1(\E) \cdot H^{n-1}}{\rank{\E}} \]

\begin{defn}
We say that a torsion-free sheaf $\E$ on $X$ is $H$\textit{-stable (resp. semistable)} if every subsheaf $0 \subsetneq \F \subsetneq \E$ satisfies,
\[ \mu(\F) < (\text{resp.} \le) \, \mu(\E) \] 
\end{defn}

\begin{rmk}
Note that it is not true that a nonzero map $\varphi : \E \to \F$ of vector bundles implies that $c_1(\E) \cdot H^{n-1} \le c_1(\F) \cdot H^{n-1}$ unless both have the same rank. For example, consider on $\P^1$ the map $\struct{X}(1) \to \struct{X}(1) \oplus \struct{X}(-1)$. However, if $X$ is smooth $\varphi : \E \to \F$ is a nonzero map of torsion-free sheaves of the same rank $r$ then there is a map $\det{\varphi} : \det{\E} \to \det{\F}$ and hence we get that $c_1(\F) - c_1(\E) = c_1(\det{\F}) - c_1(\det{\E})$ is effective. 
\end{rmk}

\begin{prop}
Fix a torsion-free sheaf of rank $r$ on the projective polarized variety $(X, H)$. Then the set of slopes $\{ \mu(\F) \mid 0 \neq \F \subset \E \} \subset \frac{1}{r!} \Z$ is bounded above. Let $\mu_1$ be the maximum then $\{ \F \subset \E \mid \mu(\F) = \mu_1 \}$ contains the largest element with respect to the inclusion relation (the maximal destabilizer).
\end{prop}

\begin{proof}
Because $\E$ is torsion-free there are injections,
\[ \E \embed \E^{\vee \vee} \embed \struct{X}(mH)^N \]
for some integers $m, N$. Therefore, it suffices to show that slopes of subsheaves of $\struct{X}(mH)^N$ are bounded. Let $\F \subset \E$ be a rank $s$ subsheaf. At the generic point the matrix corresponding to $\F \embed \struct{X}(mH)^N$ has $s$ independent columns (because it is full rank) and hence we can choose $\F \embed \struct{X}(mH)^N \to \struct{X}(mH)^s$ such that the composition is injective. Then taking determinants we get $\deg{\F} \le s m H^n$ and hence $\mu(\F) \le m H^n$ proving a uniform bound.
\bigskip\\
Now suppose that $\F_1, \F_2 \subset \E$ are two subsheaves with $\mu(\F_1) = \mu(\F_2) = \mu_1$. It suffices to show that $\mu(\F_1 + \F_2) = \mu_1$. Consider the exact sequence,
\begin{center}
\begin{tikzcd}
0 \arrow[r] & \F_1 \cap \F_2 \to \arrow[r] & \F_1 \oplus \F_2 \arrow[r] & \F_1 + \F_2 \arrow[r] & 0
\end{tikzcd}
\end{center}
and the additivity of Chern classes,
\[ r \mu(\F_1 + \F_2) = r_1 \mu(\F_1) + r_2 \mu(\F_2) - r' \mu(\F_1 \cap \F_2) \]
where $r = \rank{(\F_1 + \F_2)}$ and $r_i = \rank{\F_i}$ and $r' = \rank{(\F_1 \cap \F_2)}$. By definition of $\mu_1$ we have $\mu(\F_1 \cap \F_2) \le \mu_1$ and thus,
\[ r \mu(\F_1 + \F_2) \ge (r_1 + r_2 - r') \mu_1 \]
and thus $\mu(\F_1 + \F_2) \ge \mu_1$ but trivially $\mu(\F_1 + \F_2) \le \mu_1$ so we win. 
\end{proof}

\begin{defn}
By the above result, setting $\mu_{\max}(\E) = \mu_1$ is a well-defined invariant of $(X, H, \E)$ and so is the maximal destabilizer. By maximality, the maximal destabilizer is saturated and $H$-semistable.  
\end{defn}

\begin{lemma}
Let $\E$ be torsion-free and $\F \subset \E$ the maximal destabilizer. Then $\E$ is $H$-semistable iff $\F = \E$ iff $\mu(\E) = \mu_{\max}(\F)$. If $\E$ is not $H$-semistable then $\mu_{\max}(\E/\F) < \mu_{\max}(\E) = \mu(\F)$.
\end{lemma}

\begin{proof}
Indeed, $\E$ is $H$-semistable iff $\mu_{\max}(\E) = \mu(\E)$ since this exactly means that every subsheaf has slope at most $\mu(\E)$ but this is equivalent to $\F = \E$ since $\F$ is maximal amoung subsheaves with $\mu(\F) = \mu_{\max}(\E)$. 
\bigskip\\
Suppose that $\mu_{\max}(\E) > \mu(\E)$. Then if $0 \neq \F' \subset (\E / \F)$ is the maximal destabilizer then its preimage $\F'' \subset \E$ must satisfy $\mu(\F'') < \mu_{\max}(\E)$ because $\F''$ strictly contains $\F$ then consider,
\[ 0 \to \F \to \F'' \to \F' \to 0 \]
we have,
\[ r \mu(\F) + r' \mu(\F') = r'' \mu(\F'') < r'' \mu(\F) \]
and therefore,
\[ r' \mu(\F') < (r'' - r) \mu(\F) \]
but $r' = r'' - r$ so we conclude.
\end{proof}

\begin{cor}
There exists a filtration,
\[ 0 = \F_0 \subsetneq \F_1 \subsetneq \cdots \subsetneq \F_s = \E \]
where $\F_{i+1}$ is the preimage in $\E$ of the maximal destabilizer of $\E / \F_{i}$. Therefore, $\F_{i+1} / \F_i$ is $H$-semistable and the slopes satisfy,
\[ \mu_{\max}(\E) = \mu(\F_1 / \F_0) > \mu(\F_2 / \F_1) > \cdots > \mu(\F_s / \F_{s-1}) = \mu_{\min}(\E) \]
Furthermore, $\mu_{\min}(\E) = - \mu_{\max}(\E^\vee)$ is the minimal slope of a torsion-free quotient of $\E$. 
\end{cor}

\begin{proof}
We need to show that the slopes of $\E^\vee$ are negative of the slopes of $\E$. Consider the filtration $\F'_r := \ker{\E^\vee \to \F_r^\vee}$. Note that $c_1$ is insensitive to sheaves of support in codimension $\ge 2$ and therefore we may dualize torsion-free bundles without worry. Thus we get $\deg(\F'_r) = -\deg(\E^\vee) + \deg(\F_r)$ and there is a correspondence between saturated subsehaves $\F \subset \E$ and torsion-free quotient sheaves $\E^\vee \onto \im \subset \F^\vee$ under which the slopes are inverted. 
\end{proof}

\newcommand{\cQ}{\mathcal{Q}}

\begin{lemma}
Let $\phi : \E_1 \to \E_2$ be a morphism of torsion-free sheaves with $\mu_{\min}(\E_1) > \mu_{\max}(\E_2)$. Then $\phi = 0$.
\end{lemma}

\begin{proof}
Assume $\phi \neq 0$ so $\I = \im{\phi}$ is nonzero. Consider the exact sequences
\[ 0 \to \K \to \E_1 \to \I \to 0 \]
and
\[ 0 \to \I \to \E_2 \to \cQ \to 0 \]
We know that $\mu(\I) \ge \mu_{\min}(\E_1)$ since $\I$ is a torsion-free quotient. But also $\mu(\I) \le \mu_{\max}(\E_2)$ and hence 
\[ \mu_{\min}(\E_1) \le \mu(\I) \le \mu_{\max}(\E_2) \]
\end{proof}

Now we need two results about semistability that I don't have time to prove.

\begin{lemma}
If $k$ has characteristic zero and $\E, \E'$ are semistable then $\E \ot \E'$ is semistable. Hence 
\begin{enumerate}
\item $\E^{\ot n}$ 
\item $\nSym{n}{\E}$
\item $\wedge^n \E$
\end{enumerate} 
are all semistable.
\end{lemma}

\begin{theorem}
Semistability is open in flat families. In particular, let $(X, H)$ be a normal, projective, polarized scheme over a ring $R$ of characteristic zero, finitely generated over $\ZZ$. Let $\E$ be a torsion free sheaf on $X$. Let $K = \ol{\Frac{R}}$. If $\E_K$ is $H$-semistable on $X_K$ then $\E$ is $H$-semistalbe on reduction mod $p$ for almost all $p$.
\end{theorem}

\subsection{Foliations and Bend and Break}

\begin{defn}
A subsheaf $\F \subset \T_X$ is a \textit{foliation} if
\begin{enumerate}
\item $\F$ is closed under $[-,-]$ 
\item $\F$ is saturated meanign $\T_X / \F$ is torsion-free
\end{enumerate}
If $X$ has characteristic $p$ then we say $\F$ is $p$-closed if it is closed under the map $\partial \mapsto \partial^p$.
\end{defn}

\begin{theorem}[Ekedahl]
Let $X$ be a smooth variety over a perfect field $k$. There is a 1-1 correspondence,
\[ \{ p\text{-closed foliations } \F \subset \T_X \} \iff \{ X \to Y \text{ purely inseperable of height } 1 \text{ with } Y \text{ normal} \} \]
Given by\footnote{Notice that $X \to Y$ is a homeomorphism so we just need to specify $\struct{Y}$ as a sheaf of rings on $X$.},
\[ \F \mapsto \struct{Y} := \Ann{\struct{X}}{\F} = \{ x \in \struct{X} \mid \forall \partial \in \F : \partial x = 0 \} \]
and 
\[ Y \mapsto \F := \ker{(\T_X \to \T_Y)} \]
Furthermore, $\F \subset \T_X$ is a subbundle if and only if $X \to Y$ is flat (hence $Y$ is also smooth). 
\end{theorem}

\begin{lemma}
Let $X \to Y$ correspond to a foliation $\F \subset \T_X$. Then there is an exact sequence,
\begin{center}
\begin{tikzcd}
0 \arrow[r] & \F \arrow[r] & \T_X \arrow[r] & \T_Y \arrow[r] & \Frob^* \F \arrow[r] & 0
\end{tikzcd}
\end{center}
hence $\omega_X = \omega_Y \ot (\det{\F})^{p-1}$.
\end{lemma}

\begin{theorem}[Miyaoka-Mori]
Let $k$ be an algebraically closed field of any characteristic and $(X, H)$ a normal projective polarized variety defined over $k$. Assume that there exists an irreducible curve $C \subset X$ contained in the smooth locus of $X$ such that $C \cdot K_X < 0$. Then for any closed point $x \in C$ there exists a rational curve $\Gamma \subset X$ through $x$ such that 
\[ \Gamma \cdot H \le \frac{2 C \cdot H}{- C \cdot K_X} \]
\end{theorem}

\subsection{Main Theorem}

\begin{theorem}
Let $(X, H)$ be a smooth, polarized projective variety over a field of characteristic $p > 0$. Assume that there is a $p$-closed foliation $\F \subset \T_X$ such that,
\[ (-K_X + (p-1) \det{\F}) \cdot H^{n-1} > 0 \]
Then $X$ contains a rational curve $C$ through a general point of $X$ such that,
\[ C \cdot H \le \frac{2 p H^n}{(-K_X + (p-1) \det{\F}) \cdot H^{n-1}} \]
\end{theorem}

\begin{proof}
Let $\pi : X \to Y$ be the quotient by $\F$. Let $H^{(1)}$ be an ample divisor on $X^{(1)}$ such that $\varphi^* H^{(1)} = p H$. Let $m H^{(1)}$ be very ample and $\Gamma^{(1)} \subset X^{(1)}$ be a general complete intersection curve cut out by $m H^{(1)}$ and $\Gamma^* \subset Y$ and $\Gamma \subset X$ its inversel image with reduced structure. The natural projection $\Gamma \to \Gamma^{(1)}$ is Frobenius and $\Gamma$ is numerically equivalent to $m^{n-1} H^{n-1}$ as a $1$-cycle on $X$. Let $d$ be the degree of $\pi : \Gamma \to \Gamma^*$ which is either $1$ or $p$. Then we have,
\[ d (\Gamma^* \cdot (-K_Y)) = \Gamma \cdot (-\pi^*K_Y) = \Gamma \cdot (-K_X + (p-1) \det{\F}) = m^{n-1} H^{n-1} \cdot (-K_X + (p-1) \det{\F}) \]
Since this is positive, by Bend-and-Break through a general point of $Y$ there exists a rational curve $C'$ such that,
\[ C' \cdot \pi_* H \le 2 \frac{\Gamma^* \cdot \pi_* H}{\Gamma^* \cdot (-K_Y)} \]
Then its image under $Y^{(-1)} \to X$ produces a rational curve $C$ through a general point of $X$ of degree at most,
\[ C \cdot H \le \frac{2d(\Gamma \cdot H)}{\Gamma \cdot (-\pi^* K_Y)} = \frac{2 p m^{n-1} H^n}{m^{n-1} (-K_X + (p-1) \det{\F}) \cdot H^{n-1}} = \frac{2 p H^n}{(-K_X + (p-1) \det{\F}) \cdot H^{n-1}} \]
\end{proof}

\begin{theorem}
Let $X$ be a normal projective variety over an algebraically closed field of characteristic zero. If $X$ is not uniruled then all HS-slopes of $\Omega_X$ are non-negative i.e. $\mu_{\min}(\Omega_X) \ge 0$.
\end{theorem}

This is saying that to have any positivity in $\T_X$ at all requires that $X$ is covering by rational curves.


\begin{proof}
Towards contradiction we can assume that $\mu_{\max}(\T_X) > 0$.
Let $\F \subset \T_X$ be the maximal destabilizer and we assume $\mu(\F) > 0$. By maximality, $\F$ is saturated. We claim it is a foliation. Indeed, $[-,-]$ defines a map $\wedge^2 \F \to \T_X / \F$ but $\F$ is semistable so $\wedge^2 \F$ is also semistable (using characteristic zero here)
\[ \mu(\wedge^2 \F) = \frac{(r-1) H^{n-1} \cdot c_1(\F)}{{r \choose 2}} = \frac{H^{n-1} \cdot c_1(\F)}{r/2} = 2 \mu(\F) > \mu(\T_X / \F) \]
because $\mu(\F) > 0$ so $2 \mu(\F) > \mu(\F)$. Therefore the map is zero so $\F$ is a foliation.
\bigskip\\
Now we spread out $(X, H, \F)$ to $(X_A, H_A, \F_A)$ over $A$ finite type over $\ZZ$ and consider the reduction in characteristic $p \gg 0$ (abusing notation to replace $(X, H, \F)$ by $(X_\p, H_\p, \F_\p)$ for $\p \subset A$ with $A / \p$ a fintie field of characteristic $p$). The claim is that $\F$ is $p$-closed. Indeed, the $p$-curature map
\[ \psi_p : \Frob_p^* \F \to \T_X / \F \]
is linear and $\mu(\Frob_p^* \F) = p \mu(\F) > \mu(\F)$ and $\mu(\F) > \mu(\T_X / \F)$ so again since $\F$ is semistable for $p \gg 0$ the map is zero. Then we apply the previous theorem so for almost all $p$ the reduction of $X$ is uniruled by rational curves $C$ of degree bounded asymtotically by,
\[ C \cdot H \le \frac{2 p H^n}{(-K_X + (p-1) \det{\F}) \cdot H^{n-1}} \sim \frac{2 H^n}{(\det{\F}) \cdot H^{n-1}} \]
because $\mu(\F) > 0$ so the denominator is positive. The hom scheme $\Hom{A}{\P^1_A}{X_A}$ of maps of bounded $H$-degree is finite type over $A$. Therefore, $X$ must be uniruled.
\end{proof}

\end{document}
