\documentclass[12pt]{article}
\usepackage{hyperref}
\hypersetup{
    colorlinks=true,
    linkcolor=blue,
    filecolor=magenta,      
    urlcolor=blue,
}

\usepackage{import}
\import{../}{AlgGeoCommands}

\begin{document}

\newcommand{\inner}[2]{\left< #1, #2 \right>}

\section{Borel-Weil-Bott}

Let $G/k$ be a split connected reductive group. Let $T \subset \ol{B} \subset G$ be a split maximal torus and $\ol{B}$ a borel containing it. Let $X^*(T) = \Hom{}{T}{\Gm}$. Let $X^*(T)^+$ be the cone of dominant characters which is
\[ \{ \lambda : \inner{\lambda}{\alpha^\vee} \ge 0 \} \]
where $\alpha^\vee$ are coroots for the Borel $B$ opposite to $\ol{B}$ (for example this means that $P_\alpha = B$ for the dynamical thing). Write $B = TN$ where $N$ is the unipotent radical of $B$.

\begin{theorem}[Highest Weight]
Suppose $k$ has characteristic zero. Let $W$ be an irreducible finite dimensional algebraic representation of $G$. Then $\dim_{k} W^N = 1$ and $T \acts W^N$ by some $\chi_W \in X^\bullet(T)^+$
\begin{enumerate}
\item the association $W \mapsto \chi_W$ induces a bijection,
\[ \{ \text{irred. fin dim reps} \} / \text{isom} \iso X^\bullet(T)^+ \]
\item The category of algebraic finite dimensional reps of $G$ is semisimple. 
\end{enumerate} 
\end{theorem}

Borel-Weil-Bott gives an explicit way to construct these highest-weight representations algebraically.
\bigskip\\
Let $X = G / \ol{B}$ be the flag variety. A $\ol{B}$-representation $E$ induces a vector bundle $\L_E$ on $X$ where,
\[ \Gamma(U, \L_E) = \{ f : \pi^{-1}(U) \to E \mid \forall g \in \ol{B} : f(gb) = b^{-1} f(g) \} \]

\begin{theorem}[Borel-Weil]
For any $\lambda \in X^\bullet(T)$ consider $\L_{\lambda}$ by induction to $\ol{B}$
$H^0(\lambda) := H^0(G/B, \L_\lambda)$ is either:
\begin{enumerate}
\item the irreducible of heighest weight $\lambda$ if $\lambda \in X^\bullet(T)^+$
\item $0$ otherwise.
\end{enumerate}
\end{theorem}

\begin{theorem}[Bott]
First for $w \in W := N(T)/T$ consider the action on $X^\bullet(T)$ by $w \bullet \lambda = w (\lambda + \rho) - \rho$ where,
\[ \rho = \tfrac{1}{2} \sum_{\alpha \in \Phi^+} \alpha \]
Then $H^{i + \ell(w)}(G/B, \L_{w \bullet \lambda}) \cong H^i(G/B, \L_\lambda)$ where $\ell(w)$ is the number of simple reflections needed to write $w$. In particular $H^i(G/B, \L_\ell) = 0$ unless $i = i(\lambda)$ where $i(\lambda)$ is the minimal length of a $w$ such that $\lambda = w \bullet \mu$ for $\mu \in X^\bullet(T)^+$. 
\end{theorem}

\begin{rmk}
$\L_{-2 \rho} = \omega_{G/B}$. 
\end{rmk}

\subsection{Proof of Borel-Weil}

First recall the Bruhat decomposition,
\[ G = \bigcup_{w \in W} \ol{N} w B \]
and $\codim{\ol{N} w B} = \ell(w)$. Therefore, $\ol{N}B$ is an open subset which is rational. 
\bigskip\\
Work over $\CC$.
\bigskip\\
Consider $H^0(\lambda) := H^0(G/B, \L_\lambda) = \{ f : G \to \CC \mid f(g b) = \lambda(b)^{-1} f(b) \}$. If $B = TN$ so $\ol{B} = T \ol{N}$ then,
\[ H^0(\lambda)^N = \{ f : G \to \CC \mid f(g b) = \lambda(b^{-1}) f(g) \text{ and } f(ab) = f(b) \} \]
For any $f \in H^0(\lambda)$ it is determined on the open cell $\ol{N}B \subset G$ of the Bruhat decomposition via $f(nb) = \lambda(b)^{-1} f(1)$ therefore $\dim_{\CC} H^0(\lambda)^N \le 1$ determined by $f$. If $g \in H^0(\lambda)^N$ such that $g(1) = 1$  note that if $t \in T$ then $(t \cdot f)(1) = f(t^{-1}) = \lambda(t) f(1) = \lambda(t)$ so $T \acts H^0(\lambda)^N$ by the character $\chi_W$. Therefore, if $H^0(\lambda)^N \neq 0$ then $\lambda \in X(T)^+$ by the theorem of highest weight. For this we used that $G/B$ is proper so $H^0(\lambda)$ is a finite-dimensional algebraic representation. Moreover, we need to show that if $\lambda$ is dominant then $H^0(\lambda)^N \neq 0$. 
\bigskip\\
To construct nonzero $f \in H^0(\lambda)^N$ we need to satisfy $f(gb) = \lambda(b)^{-1} f(g)$ and $f(ng) = f(g)$. Using the second equation and $f(1) = 1$ we get a well-defined function on the open cell. We need to show this extends to $G$. Consider,
\[ S = \ol{N} B \cup \ol{N} s_\alpha B = R B \]
where $s_\alpha$ is a simple reflection and $R$ is the root subgroup associated to $\alpha$ and $\ol{N}$. Idea: $R$ has some Levi $L$, by passing to the universal cover of $L$ and by pulling back, one reduces to the case $G = \SL_2$. On $\SL_2$ we immediately check that for the characters,
\[ \lambda_n : \begin{pmatrix}
t & 0 
\\
0 & t^{-1}
\end{pmatrix} \]
that the function
\[ f : \begin{pmatrix}
a & b
\\
c & d
\end{pmatrix}
\mapsto d^n f(1) \]
is a holomorphic function satisfying the required properties. 
\bigskip\\
Matt's suggestion: shouldn't the order of vanishing along some divisor be governed by $\inner{\lambda}{\alpha^\vee}$ and these are nonnegative by the definition of dominant.


\subsection{Bott}

The main theorem follows from iterating the following result. 
\begin{theorem}
If $\alpha$ is a simple root and $\lambda \in X^\bullet(T)$ with $\inner{\lambda + \rho}{\alpha^\vee} \ge 0$. Then there exists a $G$-equivariant isomorphism,
\[ H^i(G/B, \L_\lambda) = H^{i+1}(G/B, \L_{\s_\alpha \bullet \lambda} \]
\end{theorem}

\begin{proof}
Let $P_\alpha$ be a parabolic contaning $\ol{B}$ and the root subgroup generated by $\alpha$. Construct a $P_\alpha$-rep $V_\alpha^\lambda$ such that as a $T$-module it is the direct sum of characters: $\lambda, \lambda - \alpha, \cdots, s_\alpha(\lambda)$. One way to do this is $H^0(P_\alpha/B, \L_\alpha)$. If $\inner{\lambda}{\alpha^\vee} \ge $ then there are two short exact sequence,
\begin{center}
\begin{tikzcd}
0 \arrow[r] & K \arrow[r] & V^\lambda_\alpha \arrow[r] & (\lambda) \arrow[r] & 0
\\
0 \arrow[r] & s_\alpha(\lambda) \arrow[r] & K \arrow[r] & V^{\lambda \bullet \alpha}_\alpha \arrow[r] & 0
\end{tikzcd}
\end{center}
Reindexing, we get 
\begin{center}
\begin{tikzcd}
0 \arrow[r] & K \arrow[r] & V^{\lambda + \rho}_\alpha \ot (-\rho) \arrow[r] & (\lambda) \arrow[r] & 0
\\
0 \arrow[r] & s_\alpha \bullet \lambda \arrow[r] & K \ot (-\rho) \arrow[r] & V^{\lambda + \rho - \alpha}_\alpha \ot (-\rho) \arrow[r] & 0
\end{tikzcd}
\end{center}
Claim,
\[ R \Gamma(X, V_\alpha^{\lambda + \rho} \ot (-\rho)) = R \Gamma(X, K \ot (-\rho)) = 0 \]
Consider the projection: 
\[ G/B \to G/P_\alpha \]
which is a $\P^1$-bundle. Then $V_\alpha^{\lambda + \rho}$ and $V^\lambda_\alpha$ are $P_\alpha$-reps so in fact on fibers are trivial. But $\inner{-\rho}{\alpha^\vee} = -1$ so $V_\alpha^{\lambda+ \rho \ot (-\rho)$ restricted to any fiber of $\Pi_\alpha$ vas vanishing cohomology since it restricts to $\struct{\P^1}(-1)$. Then by the Leray spectral sequence,
\[ R \Pi_{\alpha *} V_\alpha^{\lambda + \rho} \ot (-\rho) = 0 \]
implies the vanishing. Then taking the two connecting maps,
\[ H^i(G/B, \L_\lambda} \to H^{i+1}(G/B, K \ot (-\rho)) \rightarrow H^i(G/B, \L_{s_\alpha \bullet \lambda}) \]
these maps are isomorphisms by the vanishing in the long exact sequences. The composition of these isomorphisms gives the result. 

\end{proof}

\end{document}


