\documentclass[12pt]{article}
\usepackage{import}
\import{./}{AlgGeoCommands}

\begin{document}

\section{Examples of stable $\infty$-cagtegories}

\subsection{dg-Categories}

\newcommand{\DK}{\mathrm{DK}}
\newcommand{\fin}{\mathrm{fin}}
\newcommand{\Fun}[3]{\mathrm{Fun}_{#1}\left(#2, #3\right)}
\newcommand{\Map}[3]{\mathrm{Map}_{#1}\left(#2, #3\right)}
\newcommand{\Mult}[3]{\mathrm{Mult}_{#1}\left(#2, #3\right)}
\newcommand{\Fin}{\mathrm{Fin}}
\newcommand{\Ob}{\mathrm{Ob}}
\renewcommand{\O}{\mathcal{O}}

Review, let $\cA$ be an abelian cagtegory with enough projectives. Then there is a category $D^-(\cA)$ the derived category of bounded-above.


$\Ch(\cA)$ is naturally a dg-Category meaning it is enriched in $\Ch(\Ab)$. Indeed, if $A, B \in \Ch(\cA)$ then we define,
\[ \Hom{}{A_\bullet}{B_\bullet}_n = \prod_{k} \Hom{}{A_k}{B_{k-n}} \]
with differential,
\[ (\d f)_k = f_{k+1} \circ \d_B + (-1)^{n+1} \d_B \circ f_k \]

Then we check that,
\[ H^0(\Hom{}{A}{B}_\bullet) = \Hom{\Ch{\cA}}{A}{B} \]

\subsection{Program}

Given a dg-category, we get a simplicially enriched category by truncating and then applying Dold-Kan and then apply the homotopy-coherent nerve. 
\bigskip\\
However, this requires checking many steps. We can instead go directly with the following construction. 

\subsection{Construction}

\newcommand{\dg}{\mathrm{dg}}

\begin{defn}
A \textit{dg-category} is a category enriched in $\Ch(\cA)$. This includes the requirement that,
\[ \d{(g \circ f)} = \d{g} \circ f + (-1)^{\deg{g}} g \circ \d{f} \]
which arises from needing to preserve the monoidal structure on $\Ch(\cA)$ given by graded tensor $A \ot B$. 
\end{defn}

We will define a dg-Nerve which goes directly from dg-categories to $\infty$-categories which does not require passing through simplically-enriched categories.

\begin{defn}
Let $C$ be a dg-category. Then the \textit{dg-Nerve} is the simplicial set $N_{\dg}(C)$ where $N_{\dg}(C)_n$ is the set of objects $X_i$ for $i \in \{ 0, \dots, n \}$ and for each ordered set $I = \{ i_- < i_m < \cdots < i_1 < i_+ \}$ for $m \ge 0$ of elements in $\{ 0, \dots, n \}$ we have map,
\[ f_I \in \Map{}{X_{i_-}}{X_{i_+}} \]
such that,
\[ \d{f_I} = \sum_{1 \le j \le m} (-1)^i (f_{I \sm \{ i_j \}} - f_{\{i_j < \cdots < i_+ \}} \circ f_{\{ i_- < \cdots < i_j \}}) \]
and the maps work as follow. If $\alpha : [m] \to [n]$ is monotone. Then the induced map $\alpha^*$ is given by,
\[ \alpha^* (\{ X_i \}, \{ f_I \}) = (\{ X_{\alpha(i)} \}, \{ g_I \}) \]
where,
\[ 
g_J = 
\begin{cases}
f_{\alpha(J)} & \alpha|_J \text{ injective} 
\\
\id_{X_i} & J = \{ j, j'\} \text{ and } \alpha(j) = j'
\\
0 & \text{ else}
\end{cases} \]
\end{defn}

\begin{prop}
For any dg-category $C$ the dg-Nerve $N_{\dg}(C)$ is an $\infty$-category. 
\end{prop}

\begin{proof}
We need to show that inner horns can be filled. However, $\Lambda^n_i \to N_{\dg}(C)$ this is the same data as specifying $\Delta^n \to N_{\dg}(C)$ except we haven't specified all the maps $f_I$. In fact, this has specified the maps for all $I$ except for $I = [n]$ and $I = [n] \sm \{ i \}$. Then we set $f_{[n]} = 0$ and,
\[ f_{[n] \sm \{ i \}} = \sum_{0 < p < n} (-1)^{p-i} f_{\{ p, \dots, n \}} \circ f_{\{0, \dots, p\}} - \sum_{p < p < n, p \neq i} (-1)^{p-i} f_{[n] \sm \{ 0 \}} \]
\end{proof}

\begin{rmk}
$\Hom{N_{\dg}(C)}{X}{Y} = \DK(\tau_{\ge 0} \Map{C}{X}{Y})$ which is the same result we would have gotten from the program applying Dold-Kan to the simplicially-enriched category.
\end{rmk}

\begin{rmk}
Under what conditions do we get a stable $\infty$-category from $N_{\dg}(C)$?
\end{rmk}

\begin{defn}
Let $\cA$ be an abelian category with enough projectives,
$D^+(\cA) = N_{\dg}((\Ch(\cA_{\text{proj}})_{\ge 0})$.
\end{defn}

\begin{prop}[Prop 1.3.2.10]
$D^{-}(\cA)$ is stable. 
\end{prop}

\begin{defn}
We say that $\cA$ is a \textit{Grothendieck abelian category} if $A$ has a generator and small filtered colimits of monomorphisms are monomorphisms.
\end{defn}

\begin{example}
The category of $R$-modules is Grothendieck.
\end{example}

\begin{theorem}
Let $\cA$ be Grothendieck. Then $\Ch(\cA)$ has a model structure where,
\begin{enumerate}
\item cofibrations $M \to N$ are those morphisms which are injective termwise
\item equivalences $M \to N$ are quasi-isomorphisms
\item fibrations are those satisfying the right lifting property wrt the acyclic cofibrations.
\end{enumerate}
\end{theorem}

\begin{prop}[1.3.5.6]
\begin{enumerate}
\item If $M_n$ is injective for all $n$ and $M_n \cong 0$ for $n \gg 0$ then $M_\bullet$ is fibrant.
\item If $M_\bullet$ is fibrant then each $M_n$ is injective. 
\end{enumerate}
\end{prop}

If $M \cA$ and $M' = M[0]$ then there is a fibrant replacement $M[0] \to Q$ where the map is a trivial cofibration this proves enough injectives.

\begin{defn}
Let $\Ch(\cA)^\circ$ be the full subcategory of fibrat objects. Let $D(A) = N_{\dg}(\Ch(\cA)^\circ)$.
\end{defn}

\begin{prop}
$D(\cA)$ is stable and is the $\infty$ unbounded derived category. 
\end{prop}

\newcommand{\dlim}{\varinjlim}
\newcommand{\ilim}{\varprojlim}
\newcommand{\plim}{\varprojlim}

\subsection{Spectra}

\begin{defn}
We say an $\infty$-category is \textit{pointed} if it admits a zero object. 
\end{defn}

Motivation: stable maps,
\[ [X, Y]_s : = \varinjlim_n [\Sigma^n X, \Sigma^n Y] \]
There is a category of topological spaces with stable maps. This gives a triangulated category with $\Sigma$ acting via shift. Constructing this category ``formally'' we have some options,
\begin{enumerate}
\item objects are $(X, n)$ with $X \in \Top$ and $n \in \ZZ$ and morphisms are,
\[ \Hom{}{(X, n)}{(Y, m)} = \dlim_k [\Sigma^{k+n} X, \Sigma^{k+m} Y] \]
which works even for negative $n, m$ becasuse we can choose $k$ large enough. This ``formally inverts'' suspension by giving elemens like $(X, -1)$ which is the de-suspension of $X$

\item construting inifinte loop spaces: a sequence $E_n$ with maps $E_n \iso \Omega E_{n+1}$ so these are progressive deloopings. These are also giving inverses of $\Sigma$.
\end{enumerate}

\begin{defn}
Let $F : C \to D$ be a functor of $\infty$-categories then,
\begin{enumerate}
\item $F$ is \textit{excisive} if psuhout diagrams map to pullback diagrams
\item $F$ is \textit{reduced} if $F(*) = *$. 
\end{enumerate}
\end{defn}

\begin{defn}
Let $\C$ admit finite limits. A \textit{spectrum object} is a reduced excisive functor,
\[ F : S^\fin_* \to \C \]
The $\infty$-category of spectra is,
\[ \Sp(\C) = \Fun{}{S^\fin_*}{\C}_{\text{exc, red}} \subset \Fun{}{S^\fin_*}{\C} \]
where $S^\fin_*$ is the pointed category of finite spaces (full subcategory of $\infty$-category of pointed spaces). 
\end{defn}

\begin{prop}[1.4.2.11]
Let $C$ be a pointed $\infty$-category with finite limits and colimits then $\Omega : \C \to \C$ is an equivalence then $\C$ is stable. 
\end{prop}

\begin{prop}
If $\C$ is a pointed $\infty$-category with finite limits and colimts then $\Sp(\C)$ is stable. 
\end{prop}

\begin{prop}
Let $\C$ be a pointed $\infty$-category with finite limits then there is a tower of $\infty$-categories,
\begin{center}
\begin{tikzcd}
\cdots \arrow[r] & C \arrow[r, "\Omega"] & C \arrow[r, "\Omega"] & C
\end{tikzcd}
\end{center}
and $\Sp(\C)$ is the homotopy limit. 
\end{prop}

Let $\cA$ be an abelian group. Then the Eilenberg-Maclain spaces satisfy $K(A, n-1) \cong \Omega K(A, n)$ and hence defines a spectrum. 

\begin{prop}
Let $\C$ be an $\infty$-category which admits finite limits. The following are equivalent,
\begin{enumerate}
\item $\C$ is a stable $\infty$-category
\item the functor $\Omega^\infty : \Sp(\C) \to \C$ is an equivalence of $\infty$-categories
\end{enumerate}
where $\Omega^\infty$ is the map sending $F \mapsto F(S^0)$ where $S^0 = * \coprod *$.
\end{prop}

\section{$\infty$-operads}

References: Higher Algebra

\subsection{Motivations}

Want the symmetric monoidal category in the $\infty$-setting. 
\bigskip\\
Recall that the definition of a symmetric monidal category is annoying:

\begin{defn}
A symmetric monidal category $(\C, \ot, 1, \alpha, \nu, \beta)$ is,
\begin{enumerate}
\item a $1$-category $\C$
\item a functor $\otimes : \C \times \C \to \C$
\item a natural isomorphism,
\[ \alpha_{X,Y,Z} : X \ot (Y \ot Z) \iso (X \ot Y) \ot Z \]
\item a unit object $1$ with a fixed isomorphism $1 \ot 1 \iso 1$
\item a natural isomorphsim,
\[ \beta_{X,Y} : X \ot Y \iso Y \ot X \]
\end{enumerate}
(this is the easy part, now the bad part) satisfying,
\begin{enumerate}
\item $A \mapsto 1 \ot A$ and $A \mapsto A \ot 1$ are equivalences of categories (then from $A \ot 1 \iso A \ot (1 \ot 1) \iso (A \ot 1) \ot 1$ we see that $A \iso A \ot 1$ by equivalence) 
\item unit should be compatible with associativity,
\begin{center}
\begin{tikzcd}
X \ot (1 \ot Y) \arrow[rr, "\alpha_{1,Y}"] \arrow[rd] & & (X \ot 1) \ot Y \arrow[ld] 
\\
& X \ot Y
\end{tikzcd}
\end{center}

\item some huge nasty diagram making associativity compatible for 4-fold products
\end{enumerate}
\end{defn}

\subsection{Construction}

Let $(\C, \ot)$ be a symmetric monoidal category. Then we can construct a new category $C^{\ot}$ whose objects are $\{ [C_1, \dots, C_n] \}_{C_i \in \C}$ with $n \ge 0$. And the morphisms,
\[ [C_1, \dots, C_n] \xrightarrow{(S, \alpha, f)} [C_1', \dots, C_m'] \]
for $S \subset \{ 1, \dots, n \}$ and $\alpha : S \to \{ 1, \dots, m \}$ and maps $\{ f_j \}^m_{j = 1}$ which are maps $f_j : \ot_{i \in \alpha^{-1}(j)} C_i \to C_j$. The composition law is given by $(S, \alpha, f) \circ (S', \alpha', f')$ in the only reasonable way (WRITE DOWN). 

\begin{rmk}
Notice that the ordering of the object $[C_1, \dots, C_n]$ does not matter. 
\end{rmk}

\begin{defn}
New category $\Fin_*$ whose objects are $\left< n \right> := \{ 1, \dots, n \} \sqcup \{ * \}$ and the morphisms are $\left< n \right> \to \left< m \right>$ with $* \mapsto *$. This is just the category of finite sets with a disjoint basepoint added and the maps must be basepoint preserving. 
\end{defn}

\begin{rmk}
$\Fin_*$ is the category of pointed finite sets and basepoint preserving maps. 
\end{rmk}

\begin{defn}
We have special maps $\rho^i : \left< n \right> \to \left< 1 \right>$ which sends $i \mapsto 1$ and $k \mapsto *$ for $k \neq i$. Then we write $\left< n \right>^\circ = \{1, \dots, n\} = \left< n \right> \sm \{ * \}$.
\end{defn}

\begin{rmk}
The data of a symmetric monoidal category is equivalent to a functor $\C^\ot \to \Fin_*$ given by $[C_1, \dots, C_n] \mapsto \left< n \right>$ satisfying some properties. 
\end{rmk}

\subsection{Colored Operads}

\begin{defn}
A symmetric monoidal $\infty$-category should be a cocartesian fibration $p : \C^\ot \to N(\Fin_*)$ satisfying $\forall n \ge 0$ the collection of $\{ \rho^i \}$ induces $\rho^i_* : \C_{\left< n \right>}^{\ot} \to C^{\ot}_{\left< 1 \right>}$ determining $\C^{\ot}_{\left< n \right>} \cong (\C_{\left< 1 \right>}^{\ot})^{\oplus n}$.
\end{defn}

\begin{rmk}
We will come back and motivate this definition. First we discuss operads which are like categories where we have hom spaces have arbitrary arity, they are not just 2-adic.
\end{rmk}

\begin{defn}
A \textit{colored operad} $\O$ is the following data,
\begin{enumerate}
\item a collection of objects $\Ob(\O)$
\item Hom spaces: given any finite set $I$ and a list of objects $(\{ X_i \}_{i \in I}, Y)$ we get a space $\Mult{\O}{\{ X_i \}_{i \in I}}{Y}$ if $I = \empty$ we get a set $\Mult{\O}{}{Y}$ which is a unit (or something)
\item a composition law, for any map $I \to J$ we get,
\[ \prod_{j \in J} \Mult{\O}{\{ X_i \}_{i \in I_j}}{Y_j} \times \Mult{\O}{\{ Y_j \}_{j \in J}}{Z} \to \Mult{\O}{ \{X_i \}_{i \in J}}{Z} \]
\item $\id_Y \in \Mult{\O}{\{Y \}}{Y}$
\item associativity (DO THIS)
\end{enumerate}
\end{defn}

\begin{rmk}
We use the terminology $\mathrm{Mult}$ because in the case of a symmetric monoidal category we will get an operad with,
\[ \Mult{\O}{\{ X_i \}_{i \in I}}{Y} = \Hom{}{\bigotimes_{i \in I} X_i}{Y} \]
\end{rmk}

\begin{rmk}
From $\O$ we get a $1$-category $\C$ with Objects $\Ob(\C) = \Ob(\O)$ and $\Hom{}{X}{Y} = \Mult{\O}{ \{ X \}}{Y}$.
\end{rmk}

\begin{example}
Given any $1$-category $\C$ we can produce an operad $\O$ on the same objects with,
\[ \Mult{\O}{ \{ X_i \}_{i \in I} }{Y} = 
\begin{cases}
\empty & \# I \neq 1
\\
\Hom{\C}{X}{Y} & \# = 1
\end{cases} \]
This is left-adjoint to the ``underlying category'' functor.
\end{example}

\begin{example}
Given a symmetric monoidal category $(\C, \ot)$ we get an operad on the same objects with,
\[ \Mult{\O}{ \{X_i \}_{i \in I}}{Y} = \Hom{\C}{ \bigotimes_{i \in I} X_i}{Y} \]
\end{example}

\begin{example}
Given a colored operad $\O$, we get a new category $\O^{\ot}$ whose objects are $\{ X_i \}_{i \in I}$ and whose mapping sets are,
\[ \Hom{\O^{\ot}}{ \{ X_i \}_{i \in I} }{ \{Y_j \}_{j \in J} } = \prod_{j \in J} \Mult{\O}{ \{ X_i \}_{i \in I} }{Y_j} \]
In fact, $\O$ is equivalent to the data of the fibration,
\[ \pi : \O^{\ot} \to \Fin_* \]
with some requirements on $\pi$. Indeed we can recover, $\O = \pi^{-1}(\left< 1 \right>)$ and we get $\Mult{\O}{ \{ X_i \}_{i \in I} }{Y}$ by considering maps in $\O^{\ot}$ between the object $\{ X_i \}_{i \in I} \in \pi^{-1}(\left< n \right>)$ and $Y \in \pi^{-1}9\left< 1 \right>)$. 
\end{example}


\subsection{Cocartesian morphisms}

\begin{defn}
Let $p : \C \to \D$ be a functor of $1$-categories. Then $g : X \to Y$ in $\C$ is cocartesian if for all $Z$,
\begin{center}
\begin{tikzcd}
\Hom{\C}{Y}{Z} \arrow[r] \arrow[d] & \Hom{\C}{X}{Z} \arrow[d]
\\
\Hom{\D}{p(Y)}{p(Z)} \arrow[r] & \Hom{\D}{p(X)}{p(Z)}
\end{tikzcd}
\end{center}
is a pullback. 
\end{defn}

\begin{defn}
Let $p : \C \to \D$ be a functor of $\infty$-categories. Then a morpism $g : X \to Y$ in $\C$ is cocartesian if $p$ is an inner fibration and (WHAT)
\end{defn}


\subsection{$\infty$-Operads}

\begin{defn}
$f : \left< m \right> \to \left< n \right>$ in $\Fin_*$ is \textit{inertia} if for $i \in \left< n \right>^\circ$ then $f^{-1}(i)$ consists of \textit{exactly} one element. This means that two elements can only be mapped to the same place if their image is $*$ and also the map is surjective.
\end{defn}

\begin{defn}
An $\infty$-operad is a functor $p : \O^\ot \to N(\Fin_*)$ from an $\infty$-category $\O^\ot$ where we write $\O^{\ot}_{\left< n \right>} = \pi^{-1}(\left< n \right>)$ such that,
\begin{enumerate}
\item if $f : \left< m \right> \to \left< n \right>$ is inertia then every object $C \in \O^{\ot}_{\left< m \right>}$ there exists a $p$-cocartesian morphism $\bar{f} : \C \to \C'$ lifting $f$

\item for $C \in \C^{\ot}_{\left< n \right>}$ and $C' \in \C^{\ot}_{\left< m \right>}$ and $f : \left< m \right> \to \left< n \right>$ in $\Fin_*$ then,
\[ \Map{f}{C}{C'} := (\Map{}{C}{C'})^\circ \]
is the connected component lying over $f$ then,
\[ \Map{f}{C}{C'} \cong \prod_{1 \le i \le n}^f \Map{\rho_i \circ f}{C}{C'} \]
\item $\forall n \ge 0$ the maps $\{ \rho^i_! : \O_{\left< n \right>} \to \O_{\left< 1 \right>}$ induces an equivaence of $\infty$-categories,
\[ \O_{\left< n \right>}^{\ot} \iso \left( \O_{\left< 1 \right>}^\ot \right)^{\oplus n} \]
\end{enumerate}
\end{defn}

\begin{rmk}
We write $\O := \O^{\ot}_{\left< 1 \right>}$ which is called the underlying $\infty$-category of $\O^{\ot}$. 
\end{rmk}

\begin{rmk}
If $\O$ is a colored ($1$-categorical) operad then $N(\O^{\ot}) \to N(\Fin_*)$ is an $\infty$-operad. 
\end{rmk}

\begin{example}
The trivial operad is on the trivial category,
\[ \underline{\mathrm{Triv}} \subset \Fin_* \]
which is the full subcategory on the inertia maps. Then the inclusion map,
\[ N(\underline{\mathrm{Triv}}) \to N(\Fin_*) \]
is an $\infty$-operad. 
\end{example}

\subsection{The Associative Operad}

Let $\E_0^\ot$ be the operad whose objects are $\left< n \right>$ and whose morphism $f : \left< m \right> \to \left< n \right>$ such that $\# f^{-1}(i) \le 1$ for $1 \le i \le n$ (weaker than inertia since it does not have to be surjective). This is the first of our associative operads. 
\bigskip\\
The commutative operad is supposed to be $N(\Fin_*) \to N(\Fin_*)$.

\begin{defn}
A morphism $F : \O_1 \to \O_2$ of operads is a functor $F : \O_1^\ot \to \O_2^\ot$ of $\infty$-categories and a homotopy making the diagram,
\begin{center}
\begin{tikzcd}
\O^\ot_1 \arrow[rr, "F"] \arrow[rd] & & \O^{\ot}_2 \arrow[ld]
\\
& N(\Fin_*)
\end{tikzcd}
\end{center}
\end{defn}

\section{Algebras and Modules}

\subsection{Operad Review}


Perspectives on operads:
\begin{enumerate}
\item Categories with ``many-to-one'' structure: meaning there are higher airity maps $Mult( \{ X_i \}, Y)$

\item For every operad $\O$ get $\O$-monoidal category 

\item For every operad $\O$ get $\O$-algebra object in a symmetric monoidal category

\item for $\O' \to \O$ a map of operads get $\O'$-algebras in $\O$-monoidal categories.
\end{enumerate}

Motivation: $(\Ab, \ot_\Z)$ be the symmetric monoidal category of abelian groups with tensor. A commutative ring is $A \in \Ab$ equipped with maps,
\begin{enumerate}
\item $e : \Z \to A$
\item $m : A \ot A \to A$
\end{enumerate}
such that some diagrams commute,
\begin{center}
\begin{tikzcd}
A \ot A \to A \arrow[r] \arrow[d] & A \ot A \arrow[d]
\\
A \ot A \arrow[r] & A
\end{tikzcd}
\end{center}
etc. We can package all these together into the following construction.

\begin{defn}
A \textit{unital commutative ring} is,
\begin{enumerate}
\item an object $A \in \Ab$
\item for each finite set $I$ a map $m_I : A^{\ot I} \to A$
\end{enumerate}
such that,
\begin{enumerate}
\item if $\# I = 1$ then $m_I = \id$
\item for $\varphi : I \to J$ then $m_I$ is the composition,
\[ A^{\ot I} = \bigotimes_{i \in J} A^{\ot \varphi^{-1}(j)} \xrightarrow{\bigotimes_{j \in J} \m_{\varphi^{-1}(j)}} A^{\ot J} \xrightarrow{m_J} A \]
\end{enumerate}
\end{defn}

\begin{rmk}
Commutativity arrises from the swapping map $\varphi : \{ 1, 2\} \to \{ 1, 2 \}$.
\end{rmk}

\newcommand{\comb}{\mathrm{comb}}
\newcommand{\comp}{\mathrm{comp}}

\begin{defn}
$\ord(I)$ is the set of linear orders on $I$. There is a ``combine orderings'' map,
\[ \comb : \ord(J) \times \prod_{j \in J} \ord(\varphi^{-1}(j)) \to \ord(I) \]
given $\varphi : I \to J$.
\end{defn}

\begin{defn}
A \textit{unital associative ring} is,
\begin{enumerate}
\item an object $A \in \Ab$
\item for each finite set $I$ and $o \in \ord(I)$ a map $m_{I,o} : A^{\ot I} \to A$
\end{enumerate}
such that,
\begin{enumerate}
\item if $\# I = 1$ then $m_I = \id$
\item for $\varphi : I \to J$ then $m_{I, \comb(o, o_j)}$ is the composition,
\[ A^{\ot I} = \bigotimes_{i \in J} A^{\ot \varphi^{-1}(j)} \xrightarrow{\bigotimes_{j \in J} \m_{\varphi^{-1}(j)}, o_j} A^{\ot J} \xrightarrow{m_J, o} A \]
\end{enumerate}
\end{defn}

\begin{rmk}
What if we want to define a \textit{non-unital} associative ring? In this case we just require our sets $I$ to be nonempty. Or alternatively, let,
\[ \ord'(I) = 
\begin{cases}
\ord(I) & I \neq \empty 
\\
\empty & I = \empty 
\end{cases}\]
Replacing $\ord$ by $\ord'$ gives the definition of a non-unital associative ring. 
\end{rmk}

\begin{defn}
For a colored operad $\O^{\ot}$ which consists of,
\begin{enumerate}
\item a set of colors $\O$ 
\item for each collection $\{ X_i \}_{i \in i}$ and $Y$ of colors there is a set,
\[ \Mult{\O}{ \{ X_i \}_{i \in I}}{Y} \]
satisfying some associativity and unital relations. 
\end{enumerate} 
and a symmetric monoidal category $(\C, \ot)$ an \underline{$\O$-algebra object} in $(\C, \ot)$ is the data of,
\begin{enumerate}
\item for each $X \in \O$ an object $A_X \in \C$
\item for each $m : \{ X_i \}_{i \in I} \to Y$ a morphism,
\[ f_m : \bigotimes_{i \in I} A_{X_i} \to A_Y \]
\end{enumerate}
satisfying,
\begin{enumerate}
\item if $m = \id_X \in \Mult{\O}{ \{ X \} }{X}$ then $f_m = \id_{A_X}$

\item if $\varphi : I \to J$ and $m \in \Mult{\O}{ \{ Y_j \}_{j \in J}}{Z}$ and $m_j \in \Mult{\O}{ \{ X_i \}_{i \in \varphi^{-1}(j)}}{Y_j}$ then $f_{\comp(m, \{ m_j \})}$ equals the composition,
\[ \bigotimes_{i \in I} A_{X_i} \xrightarrow{\otimes_{j \in J} f_{m_j}} \bigoplus_{j \in J} A_{Y_j} \xrightarrow{f_m} A_Z \]
\end{enumerate}
\end{defn}

\begin{rmk}
An $\O$-algebra object in $\C$ is the same as a map of operads $\O^\ot \to \C^\ot$. 
\end{rmk}

\begin{rmk}
There is a notion of maps of operads,
\[ \{ \text{symmetric monoidal cat} \} \embed \{ \text{operads} \} \]
An $\O^\ot$-algebra object of $\C^\ot$ is a map of operads $\O^\ot \to \C^\ot$. Maps of operads are diagrams,
\begin{center}
\begin{tikzcd}
\O^\ot \arrow[r] \arrow[d] & \O^{\ot'} \arrow[d]
\\
N(\Fin_*) \arrow[r, equals] & N(\Fin_*)
\end{tikzcd}
\end{center}
that takes cotartesian lifts of inert maps to cocartesian lifts. 
\end{rmk}

\subsection{Little Cubes Operad}

\renewcommand{\E}{\mathbb{E}}

For $n \ge 0$ construct an $\infty$-operad $\E_n^{\ot}$. Consider $D^n \subset \RR^n$ unit disk. There is a unique color $* \in \O$ and then define (writing $I$ for $\{ * \}_I$),
\[ \Mult{}{I}{*} = \{ \text{embeddings } \coprod_{i \in I} D^n \to D^n \text{ arising from scaling and translation} \} \]
There is a composition map, for $\varphi : I \to J$,
\[ \Mult{}{J}{*} \times \prod_{j \in J} \Mult{}{\varphi^{-1}(j)}{*} \to \Mult{}{I}{*} \]
which is continuous. Then we can take the topological nerve to get an $\infty$-operad $\E_n^{\ot}$. 

\begin{rmk}
$\E_1^\ot$ is discrete and $\E^{\ot}_1 \cong \mathrm{Assoc}^\ot$ (defining unital associative algebra) which is the operad of finite sets with orderings. 
\bigskip\\
Then $\E_0^{\ot}$ is trivial which just imposes the existence of an object $u : \Z \to A$ so we get pointed objects.
\end{rmk}

\begin{defn}
$\E^\ot_{\infty} = [N(\Fin_*) \to N(\Fin_*)]$ as an operad. Meaning $\Mult{\E_\infty^{\ot}}{I}{*} = \{ * \}$. This is because as we take the limit of $n$ for the morphism sets of $\E_n^{\ot}$ with fixed $I$ the spaces become weakly contractible.
\end{defn}

There are maps of operads $\E_0^{\ot} \to \E_1^{\ot} \to \E_2^{\ot} \to \cdots \to \E^{\ot}_{\infty}$. Therefore, I can always restrict an $\E_n^\ot$-algebra to a lower-order algebra. 

\newcommand{\Alg}{\mathrm{Alg}}
\newcommand{\gp}{\mathrm{gp}}

\begin{rmk}
Recall that $\S$ is the $\infty$-category of spaces.
\end{rmk}

\begin{theorem}[May]
Inside the monoidal category $(\S, \times)$. Then there is a functor,
\[ \S_* \xrightarrow{\Omega^n} \Alg_{\E^\ot_n}(\S) \] 
This induces an equivalence for $n \ge 1$,
\[ \S_{*, \ge n} \iso \Alg_{\E^\ot_n}^{\gp}(\S) \]
where $\S_{*, \ge n}$ is the $\infty$-category of $n$-connected pointed spaces and $\Alg_{\E^\ot_n}^{\gp}(\S)$ is the $\infty$-category of grouplike $\E_n^{\ot}$-algebras in spaces where we say that an algebra $A$ is grouplike if $\pi_0(A)$ with its induced monoid structure is a group. Furthermore, there is an equivalence,
\[ \Sp(\S_*)_{\ge 0} \iso \Alg_{\E_\infty^\ot}(\S) \]
between connective spectra and $\E_\infty^\ot$-algebras. 
\end{theorem}

\subsection{Limits and Colimits}

\begin{theorem}
If $\C$ has all small limits then $\Alg_{\O}(\C)$ also has all small limits and limits are computed ``objectwise''. 
\end{theorem}

\begin{theorem}
If $\C$ has all small colimits and $X \ot -$ preserves them then $\Alg_{\E^\ot_{\infty}}(\C)$ has small colimits. 
\end{theorem}

\subsection{Modules}

\newcommand{\CAlg}{\mathrm{CAlg}}

If $A \in \CAlg(\C)$ then there is a category $\Mod{A}{\C}$.

\begin{rmk}
$\Mod{A}{\C}$ does \textit{not} have a symmetric monoidal structure but only an operad structure. 
\end{rmk}

\begin{theorem}
If $A \in \CAlg(\C)$ then,
\[ \CAlg(\Mod{A}{(\C)}) \cong \CAlg(\C)_{A / } \]
If $B$ is an algebra over $\Mod{A}{(\C)}$ corresponding to some $\\overline{B} \in \CAlg(\C)_{A / }$ then,
\[ \Mod{B}{(\Mod{A}{(\C)})} \cong \Mod{\overline{B}}{(\C)} \]
\end{theorem}

\begin{theorem}
\begin{enumerate}
\item Limits in $\Mod{A}{(\C)}$ can be computed in $\C$
\item of $\C$ is presentable and $X \ot -$ preserves al l small colimits then $\Mod{A}{(\C)}$ is symmetric monoidal, presentable, and $\ot$ commutes with colimits. 
\end{enumerate}
\end{theorem}

\section{Ring Spectra}

Recalll a spectrum object in $S_*$ os an $\infty$-functor,
\[ F : S_*^\fin \to S_* \]
such that,
\begin{enumerate}
\item $F$ sends homotopy pushouts to pullbacks
\item $F(*) = *$.
\end{enumerate}

\begin{rmk}
For each $n$ there is a pushout square,
\begin{center}
\begin{tikzcd}
S^n \arrow[r] \arrow[d] & * \arrow[d]
\\
* \arrow[r] & S^{n+1}
\end{tikzcd}
\end{center}
which gives a pullback square,
\begin{center}
\begin{tikzcd}
F(S^n) \arrow[r] \arrow[d] & * \arrow[d]
\\
* \arrow[r] & F(S^{n+1})
\end{tikzcd}
\end{center}
and thus $F(S^n) = \Omega F(S^{n+1})$. Since every object of $S_*^{\fin}$ is a finite colimit of spheres therefore the data of $F$ is equivalent up to homotopy to the sequence $\{ F(S^n) \}$ along with the data of equvalenes $F(S^n) \iso \Omega F(S^{n+1})$. This is classically what is known as an $\Omega$-spectrum.
\end{rmk}

\begin{defn}
A \textit{spectrum} is a sequence of pointed spaces $\{ X _n \}_{n \ge 0}$ along with maps $\Sigma X_n \to X_{n+1}$ (equivalently $X_n \to \Omega X_{n+1}$).
\end{defn}

\begin{example}
Some spectra,
\begin{enumerate}
\item $\S = \{ S^n \}_{n \ge 0}$ is not an $\Omega$-spectrum
\item for $A \in \Ab$ we have $H A = \{ K(A, n) \}_{h \ge 0}$ is an $\Omega$-spectrum

\item of $\{ Y_n \}_{n \ge 0}$ is any sequence of spaces then define $X_0 = Y_0$ and $X_{n+1} = \Sigma X_n \vee Y_{n+1}$ and this defines a spectrum.
\end{enumerate}
\end{example}

How do we make this into a category of specta? If $X, Y$ are $\Omega$-spectra then,
\[ \Hom{\mathrm{hSp}}{X}{Y} = \left\{ f_n : X_n \to Y_n \middle| 
\begin{tikzcd}
X_n \arrow[r, "f_n"] \arrow[d] & Y_n \arrow[d]
\\
\Omega X_{n+1} \arrow[r, "\Omega f_{n+1}"] & \Omega Y_{n+1} 
\end{tikzcd} \right\} \]

If $X, Y$ are \textit{not} $\Omega$-spectra then,
\[ \Hom{\text{hSp}}{X}{Y} = \dlim_{X' \subset X} \Hom{}{X'}{X} \]
where $X' \subset X$ is a weak homotopy equivalence where we define weak homotopy equivalence using the following notion of homotopy groups.

\begin{defn}
if $X$ is spectrum $n \in \ZZ$ then define,
\[ \pi_n(X) = \dlim_k \pi_{n+k}(X_k) \]
\end{defn}

\begin{example}
$\pi_n(\S) = \pi_n^s(\S) = \pi_{2n + 2}(S^{n+2}$ by Freudenthal suspension.
\end{example}

\begin{example}
\[ \pi_n(HA) = 
\begin{cases}
A & n = 0
\\
0
\end{cases} \]
\end{example}

\begin{example}
Homotopy groups may be supported in negative degrees. Indeed consider,
\[ X_n = S^n \vee S^{n-1} \vee \cdots \vee S^1 \]
then $\pi_k(X_n) \neq 0$ for all $k \in \Z$ using Hilton-Milnor theorem. 
\end{example}

\begin{prop}
The inclusion $N(\Ab) \to \Sp$ sending $A \mapsto HA$ is fully faithful.
\end{prop}

\begin{proof}
Let's just check this $\mathrm{hSp}$. Point is to compute $[K(A, n), K(B, n)]$. This is,
\[ [K(A, n), K(B, n)] = H^n(K(A, n), B) = \Hom{}{H_n(A, n), \Z}{B} = \Hom{}{\pi_n(K(A, n))}{B} = \Hom{}{A}{B} \]
using Hurewicz's theorem. 
\end{proof}

\begin{defn}
There is a natural $t$-structure on $\Sp$ which is $\Sp^{\ge 0} = \{ X \mid \pi_n(X) = 0 \quad n < 0 \}$ with $\Sp^{\le 0}$ defined similarly. 
\end{defn}

\newcommand{\heart}{\heartsuit}
\newcommand{\EE}{\mathbb{E}}

\begin{prop}
$(\Sp^{\ge 0}, \Sp^{\le 0})$ is a $t$-structure and its heart $\Sp^{\heart} = N(\Ab)$ meaning $X \in \Sp^{\heart}$ satisfies $X \cong H \pi_0(X)$.
\end{prop}

\begin{proof}
This is just because its heart is objects with trivial higher homotopy groups (setlike). 
\end{proof}

\subsection{Smash Product}

For $X, Y \in \Top_*$ then $X \wedge Y = (X \times Y) / (X \vee Y)$.

\begin{example}
$S^0 \wedge X \cong X$ and $S^1 \wedge X \cong \Sigma X$ (by definition). Then $S^m \wedge S^n \cong S^{m+n}$ because $(\RR^m)_{\infty} \wedge (\RR^n)_{\infty} \cong (\RR^{m+n})_{\infty}$.
\end{example}

\begin{example}
There is a weird sign,
\begin{center}
\begin{tikzcd}
S^1 \wedge S^1 \arrow[d] \arrow[r, "\text{flip}"] & S^1 \wedge S^1 \arrow[d]
\\
S^2 \arrow[r, "-1"] & S^2
\end{tikzcd}
\end{center}
because it is orientation reversing. 
\end{example}

\begin{prop}
Some properties,
\begin{enumerate}
\item \[ \Hom{\Top_*}{X \wedge Y}{Z} \cong \Hom{\Top_*}{X}{\Hom{\Top_*}{Y}{Z}} \]

\item $X \wedge Y \cong Y \wedge X$

\item $(X \wedge Y) \wedge Z \cong X \wedge (Y \wedge Z)$. 
\end{enumerate}
\end{prop}


\begin{theorem}[HA, 4.8.2.19]
There exists a symmetric monoidal structure $\ot : \Sp \times \Sp \to \Sp$ with unit $\S$ which commutes with small colimits in both variables. If $\C$ is a symmetric monoidal $\infty$-cat, stable and presentable and $\ot : \C \times \C \to \C$ preserves small colimits in each variable then there exists a unique up to homotopy symmetric monoidal functor $F : \Sp^\ot \to \C^\ot$ such that the underlying $\Sp \to \C$ preserves small colimts. 
\end{theorem}

\begin{rmk}
Note that $H A \ot H B \not\cong H(A \ot B)$ this acts more like a derived tensor product.
\end{rmk}

\begin{example}
$\S \wedge \S \to \S$ is an isomorphism and gives the trivial ring spectrum. 
\end{example}

If $R$ is an $\EE_1$-ring then,
\[ \pi_* R = \bigoplus_{n \in \ZZ} \pi_* R \]
is a graded ring. Then $\pi_n R = [\S[n], R]$ with $\S[n]_k = \S_{n+k}$ and we have,
\[ [\S[n], R] \times [\S[m], R] \to [\S[n] \ot \S[m], R \ot R] \to [\S[n+m], R] = \pi_{n+m}(R) \]
a multiplication map. 

\begin{rmk}
Note that if $R$ is an $\EE_\infty$-ring then $\pi_* R$ is graded commutative from the fact that $\S[n] \ot \S[m]$ is antisymmetric. 
\end{rmk}

Notions of left (right) module. All certain algebra objects in $\Sp$. The left-module operad $\underline{\text{LM}}$ has two colors $A, M$ and,
\[ \Mult{\text{LM}}{ \{ X_i \}_{i \in I}}{A} = 
\begin{cases}
\ord(I) & X_i = A \text{ for all } i
\\
\empty & \text{else} 
\end{cases} \]
and likewise,
\[ \Mult{\text{LM}}{\{X_i\}_{i \in I}}{M} = 
\begin{cases}
\ord(I) & M \text{ is largest if all but exactly one are } M
\\
\empty & \text{else}
\end{cases} \]

A left module is an $\text{LM}$-algebra object of $\Sp$,
\begin{center}
\begin{tikzcd}
\LMod_R \arrow[r] \arrow[d] & \LMod \arrow[d]
\\
\{ R \} \arrow[r, hook] & \Alg_{\EE_1}(\Sp) 
\end{tikzcd}
\end{center}

\begin{prop}
We have,
\begin{enumerate}
\item $\LMod_R$ is stable
\item natural $t$-structure (connective and anti-connected)
\item if $\pi_n R = 0$ for $n \neq 0$ then $\LMod_R \cong \D(\Mod_{\pi_0(R)})$ preserving $t$-structures.
\end{enumerate}
\end{prop}

Let $R$ be a discrete commutative ring then $\Alg^{\dg}(R)$ has a model structure with,
\begin{enumerate}
\item weak equivalences are quasi-isomorphisms
\item fibrations are levelwise surjective 
\end{enumerate}

\begin{prop}
$N(\Alg^{\dg}(R)^c][W^{-1}] \cong \Alg_{\EE_1}(R)$ where $c$ means the category of cofibrant objects. If $\Q \subset R$ then,
\[ N(\CAlg^{\ad}(R)^c)[W^{-1}] \cong \Alg_{\EE_\infty}(R) \]
with graded commutative on the left.
\end{prop}

\section{Feb. 23}

\begin{defn}
Let $R$ be connecteive then $P \in \LMod_R^\cn$ is \textit{projective} if $\Hom{}{P}{-} : \LMod_R^\cn \to \S$ preserve gemoetric realization.
\end{defn}

\begin{rmk}
\begin{enumerate}
\item in classical setting geometric realization is coequalizer so this recoveres the usual definition of preserving colimits

\item $\LMod_R$ of all not necessarily connective objects has no nonzero projective objects. 
\end{enumerate}
\end{rmk}

\begin{prop}
The following are equivalent,
\begin{enumerate}
\item $P$ is projective

\item for all $Q \in \LMod_R^\cn$ and $i > 0$ we have $\Ext{i}{R}{P}{Q} := \pi_0(\Hom{}{P}{Q[i]}) = 0$

\item Given fiber sequence,
\[ N' \to N \to N'' \]
the map $\Ext{0}{R}{P}{N} \to \Ext{0}{R}{P}{N''}$ is surjective.
\end{enumerate}
\end{prop}

\begin{prop}
TFAE:
\begin{enumerate}
\item $P$ is projective

\item there exists a free module $R$-module $M$ with $P$ a retract of $M$ meaning $P \to M \to P$ with $P \to P$ an equivalence. 
\end{enumerate}
\end{prop}

\begin{proof}
First show (a) $\implies$ (b). There exists $R^{\oplus I} \to P$ such that $\pi_0(R^{\oplus n}) \onto \pi_0(P)$ is a surjection. Consider the fiber sequence,
\[ N \to R^{\oplus I} \to P \]
then $N$ is connective by the surjection of $\pi_0$. Consider $\Hom{R}{P}{-}$ applied to the fiber sequence, by surjection on $\Ext{0}{R}{P}{-}$ we get $P \to R^{\oplus I}$ satisfying the required properties. 
\bigskip\\
For (b) $\implies$ (a) we have projectivity is preserved by retract. For,
\[ P \to S \to P \]
then $\Ext{i}{R}{P}{Q}$ is a retract of $\Ext{i}{R}{S}{Q}$ then STP free module are projective.  
\end{proof}

\begin{rmk}
What are the examples of ring spectra:
\begin{enumerate}
\item $K(A)$ for $A$ a discrete ring
\item $\S$
\item simplicial commutative rings.
\end{enumerate}
\end{rmk}

\begin{rmk}
In the followign definition we don't need any connectivity assumptions.
\end{rmk}

\begin{defn}
$M$ is flat over $R$ if,
\begin{enumerate}
\item $\pi_0(M)$ is a flat $\pi_0(N)$-module
\item $\pi_m(R) \ot_{\pi_0(R)} \pi_0(M) \iso \pi_n(M)$ for all $n$.
\end{enumerate}
\end{defn}

\begin{rmk}
Flatness is closed under coproduct, retract, filtered colimits. If $R$ is connective then projective implies flat. 
\end{rmk}

\begin{prop}
Let $N, R$ be connective. The following are equivalent,
\begin{enumerate}
\item $N$ is flat
\item if $M$ is a discrete right $R$-module then $M \ot_R N$ is discrete.
\end{enumerate}
\end{prop}

\begin{proof}
Spectral sequence,
\[ \Tor{\pi_* R}{p}{\pi_* M}{\pi_* N}_q = \pi_{p+q}(M \ot_R N) \]
Then (a) $\implies$ (b) because LHS = 0 if $p \neq 0$ and thus,
\[ \pi_p(M \ot_R N) = (\pi_* M \ot_{\pi_* R} \pi_* N)_p = \cong (\pi_* M \ot_{\pi_0(R)} \pi_0(N))_p \]
For (b) $\implies$ (a) we use $- \ot_R N : \RMod_R \to \S$ and that $\RMod_R^{\heart} = \Mod_{\pi_0(R)}$. Then (b) says that we restrict to,
\[ - \ot_R N : \Mod_{\pi_0(R)} \to \Set \]
which in particular is a map to $\Set$. However, $-\ot_R N$ is exact (preserves fiber sequences) and hence is exact on the heart. Then $- \ot_R N = - \ot_{\pi_0(R)} \pi_0(N)$ (using the spectral sequence and the fact that $M \ot_R N$ is discrete for $M$ discrete) and thus is exact meaning $\pi_0(N)$ is flat over $\pi_0(R)$. The rest uses the spectral sequence. 
\end{proof}

\begin{lemma}
A map $f : M \to N$ of flat $R$-modules is an equivalence iff $\pi_0(f) : \pi_0(M) \to \pi_0(N)$ is an isomorphism.
\end{lemma}

\begin{prop}
Let $R$ be connective and $M / R$ is flat then,
\[ M \text{ is projective} \iff \pi_0(M) \text{ is projective over } \pi_0(R) \]
\end{prop}

\begin{proof}
We show a weaker version. If $\pi_0(M)$ is free over $\pi_0(R)$ then can find a free module $R^{\oplus n} \to M$ which is an isomorphism on $\pi_0$. Then apply lemma to conclude. 
\end{proof}

\subsection{Localization}

Let $R$ be an $\EE_\infty$-ring then $\pi_*(R)$ is graded-commutative. Consider $S \subset \pi_*(R)$ set of homog. elements closed under multiplication and containing $1$. Then,
\begin{enumerate}
\item $M$ is $S$-nilp if all $x \in \pi_m(M)$ are killed by some $s \in S$

\item $M$ is $S$-local if for each $s \in S$ the map $\pi_* M \xrightarrow{s} \pi_* M$ is an isomorphism (not necessarily a graded map)
\end{enumerate}
This produces two subcategories $\LMod_R^{\text{S-nilp}}$ and $\LMod_R^{\text{S-Loc}}$. We want some adjoints which will be localizations.
\begin{enumerate}
\item $\LMod_R^{\text{S-nilp}}$ is stable $\infty$-category, closed under small colimits generated under colimits by $R / Rs [n]$ for $s \in S$. If $s \in \pi_d(R)$ then this arises from,
\[ R[d] \xrightarrow{s} R \to R / R s \]

\item $M \in \LMod_R^{\text{S-loc}} \iff $ for all $s \in S$ and $n \in \ZZ$ then $\Hom{}{R/R_s[n]}{N}$ is contactible iff $\forall M \in \LMod^{\text{S-nilp}}$ we have $\Hom{R}{M}{N}$ contractible.
\end{enumerate}

(1) gives a right adjoint $G : \LMod_R \to \LMod_R^{\text{S-nilp}}$ to the inclusion. Then we take a fiber sequence,
\[ G(M) \to M \to S^{-1} M \]
defining $S^{-1} M$ and,
\[ \Hom{}{N}{G(M)} \iso \Hom{}{N}{M} \to \Hom{}{N}{S^{-1} M} \]
so the last term is contractible proving that $S^{-1} M \in \Mod_R^{\text{S-loc}}$. 

\begin{rmk}
The functor $S^{-1}(-)$ gives a left adjoint to the inclusion. 
\end{rmk}

\begin{prop}
$(\LMod_R^{\text{S-nilp}, \LMod_R^{\text{S-loc}})$ gives a $t$-structure on $\LMod_R$ with trivial heart. 
\end{prop}

\begin{rmk}
$\pi_*(S^{-1} M) = S^{-1} \pi_*(M)$. 
\end{rmk}
\end{document}