\documentclass[12pt]{article}
\usepackage{import}
\import{./}{AlgGeoCommands}

\DeclareMathOperator{\Exc}{\mathrm{Exc}}

\begin{document}

The issue with $J^k_X$ is that if $X$ has no symmetric forms then it has no $J^k_X$ sections. The same is not true for $E^GG$ or $P$ the Green-Griffiths or Semple jets.
\bigskip\\
Desiderata: logarithmic GG or Semple jets and an extension result for them. 

\section{Strategy of RR14}

Let $\X \to X$ be the coarse space of a smooth stack such that $X$ has only ADE singularities. 

\subsection{Orbifold Riemann-Roch}

By the standard computations,
\[ \chi(\X, P^{2,m}_\X) = \frac{m^3}{6} (13 c_1^2 -  9 c_2) + O(m^2) \]
{\color{red} DO THIS!! GET NUMERICS CORRECT}

\subsection{Vanishing Theorems}

We need Bogomolov's vanishing theorem for smooth stacks. If $\X$ is general type then,
\[ H^0(\X, S^m \T_{\X} \ot K_{\X}^p) = 0 \]
whenever $m > 2 p$. {\color{red} RR claim this is because of an orbifold Kahler-Einstein metric and Bochner identities? Does Bogomolov's Original Proof Work ALSO?}

{\color{red} WE REALLY NEED THE FOLLOWING THEOREM OF BOGOMOLOV}

\begin{theorem}[Bogomolov]
Let $X$ be a surface of general type and $i_1 + \cdots + i_k > 2q$ integers then,
\[ H^2(X, S^{i_1} \Omega_X \ot \cdots \ot S^{i_k} \Omega_X \ot K_X^{\ot -q}) = 0 \]
\end{theorem}

\begin{prop}
Let $X$ be a surface of general type then $H^2(\X, E^{k,m}_{\X} \ot K_{\X}^{\ot -q}) = 0$ for $m > 2k$ and $\floor{m/k} > 2 q$ with $k \ge 1$.
\end{prop}


\begin{proof}
We use induction and the fundamental filtration,
\[ E^{k-1,m} = F^0 E^{k,m} \subsetneq F^1 E^{k,m} \subsetneq \cdots \subsetneq F^{\floor{k/m}} E^{k,m} = E^{k,m} \]  
where the quotients are given by,
\[ F^p / F^{p-1} \cong S^p \Omega_X \ot E^{k-1, m-kp} \]
Therefore, $E^{k,m}$ has a composition series by terms,
\[ S^{i_1} \Omega_X \ot \cdots \ot S^{i_k} \Omega_X \]
where each $(i_1, \dots, i_k)$ such that $i_1 + 2 i_2 + \cdots + k i_k = m$ appears exactly once. Therefore, because $\dim{X} = 2$ by Bogomolov's vanishing condition we conclude. 
\end{proof}

\subsection{Log Jets and Extension (PROBLEM)}

We need to construct a log jet bundle $P_X^{k,m}(\log{D})$ and we need the following result,
\[ H^0(Y \sm E, P_X^{k,m}) = H^0(Y, P_X^{k,m}(\log{D})) \]
This is the jet version of the extension theorem of Miyaoka.

\subsection{The Proof (PROBLEM)}

Let $g_k(X)$ be the leading term in the asymtotic Riemann-Roch computation of $P^{k, m}_X$. {\color{red} For $k = 2$ IT IS SOMETHING LIKE $g_k(X) = \frac{1}{12}(13 c_1^2 - 9 c_2)$ OR SOMETHING}

\begin{theorem}
Suppose that $g_k(Y) + g_k(\X) > 0$ then,
\[ h^0(Y, P^{k,m}_X) \ge \frac{g_k(Y) + g_k(\X)}{2} m^3 + O(m^2) \]
in particular there are enough $k$-jets.
\end{theorem}

\begin{proof}
Consider the exact sequences,
\[ 0 \to P^{k,m}_Y \to P^{k,m}_Y(\log{D}) \to Q_m \to 0 \]
where $Q_m$ is supported on the exceptional divisor $E$. Since the singularities are ADE, the minimal resolution is crepant so there exists a neighborhood $U$ of $E$ such that the canonical bundle is tirival. Therefore, we get the sequence
\[ 0 \to P^{k,m}_Y \ot K_Y^{\ot (1 - m)} \to P^{k,m}_Y(\log{D}) \ot K_Y^{\ot (1 - m)} \to Q_m \to 0 \]
The proof will distinguish two cases according to the value of $\limsup \frac{h^0(Q_m)}{m^3}$.
Let us first suppose that,
\[ \limsup \frac{h^0(Q_m)}{m^3} \le \frac{g_k(\X) - g_k(Y)}{2} \]
As expalined above, Bogomolov's vanishing theorem implies that,
\[ \limsup \frac{1}{m^3} h^0(\X, P^{k,m}_{\X}) \ge g_k(\X) \]
Then the exact sequence implies that,
\begin{align*}
\limsup \frac{1}{m^3} h^0(Y, P^{k,m}_Y) & \ge \limsup \frac{1}{m^3} h^0(Y, P^{k,m}_Y(\log{E}) - \limsup \frac{h^0(Q_m)}{m^3} \ge g_k(\X) - \frac{g_k(\X) - g_k(Y)}{2} 
\\
&= \frac{g_k(\X) + g_k(Y)}{2}
\end{align*}
by the extension property. Alternatively, if
\[ \limsup \frac{h^0(Q_m)}{m^3} > \frac{g_k(\X) - g_k(Y)}{2} \]
then the extension property and the triviality of $K_Y$ near $E$ gives,
\begin{align*}
h^0(Y, P^{k,m}_Y(\log{E}) \ot K_Y^{\ot (1-m)}) = h^0(Y \sm E, P^{k,m}_Y \ot K_Y^{\ot (1 - m)}) = h^0(\X, P^{k,m}_{\X} \ot K_{\X}^{\ot (1 - m)})
\end{align*}
{\color{red} IN THIS CASE WE NEED AN UPPER BOUND ON $h^0(Q_m)$ IN TERMS OF $h^1(Y, P^{k,m}_Y)$ }
PROBLEM SERRE DUALITY DOESNT SEND $P^{k,m}$ TO ITSELF TIMES THE CANONICAL DOES IT
{\color{red} THIS FAILS BADLY DONT SEE HOW TO CONCLUDE WITHOUT THIS TRICK}
\end{proof}

\section{Strategy of Bruin}

Let $X$ be an ADE surface and $\tau : Y \to X$ its minimal resolution. Let $\cA = E^{k,m}$ consider $\tau_* \cA$ and its reflexive hull $\hat{\cA} = (\tau_* \cA)^{\vee \vee}$. The Leray spectral sequence,
\[ E_2^{p,q} = H^p(X, R^q \tau_* \cA) \implies H^{p+q}(Y, \cA) \]
gives rise to the 6-term exact sequence,
\begin{center}
\begin{tikzcd}[column sep = tiny]
0 \arrow[r] & H^1(X, \tau_* \cA) \arrow[r] & H^1(Y, \cA) \arrow[r] & H^0(X, R^1 \tau_* \cA) \arrow[r] & H^2(X, \tau_* \cA) \arrow[r] & \ker{(H^2(Y, \cA) \to H^0(X, R^2 \tau_* \cA))} \arrow[r] & H^1(X, R^1 \tau_* \cA) 
\end{tikzcd}
\end{center}
The sheaf $R^1 \tau_* \cA$ has 0-dimensional support on the singular locus $S \subset X$ and thus,
\[ H^1(X, R^1 \tau_* \cA) = H^2(X, R^1 \tau_* \cA) = 0 \]
Furthermore, $\tau$ has $1$-dimensional fibers only and hence $R^2 \tau_* \cA = 0$. Therefore, the exact sequence becomes,
\begin{center}
\begin{tikzcd}
0 \arrow[r] & H^1(X, \tau_* \cA) \arrow[r] & H^1(Y, \cA) \arrow[r] & H^0(X, R^1 \tau_* \cA) \arrow[r] & H^2(X, \hat{\cA}) \arrow[r] & H^2(Y, \cA) \arrow[r] & 0
\end{tikzcd}
\end{center}
where we use the fact that $\hat{\cA} / \tau_* \cA$ and $\ker{(\tau_* \cA \to \hat{\cA})}$ are both supported on $S$ which is 0-dimensional so $H^2(X, \tau_* \cA) = H^2(X, \hat{\cA})$. Therefore, via the vanishing result $H^2(X, \hat{\cA}) = 0$ we get.
\[ h^1(Y, \cA) = h^1(X, \tau_* \cA) + h^0(X, R^1 \tau_* \cA) \]

{\color{red} DOES THE REFLEXIVE VANISHING RESULT HOLD FOR JETS?}

Note that we use that $\hat{\cA}$ is exactly reflexive symmetric forms on $X$. This is obvious because we can compute everything on $X \sm S$ which is isomorphic to $Y \sm E$ so $\tau_* \cA|_U = S^m \Omega_X |_U$ and therefore $\hat{\cA} = j_* (\tau_* \cA|_U) = (S^m \Omega_X)^{\vee \vee}$. The same argument should work for the jets since it works for any canonical object. 

\subsection{Bogomolov's Vanishing Lemma for Reflexive Forms}

\newcommand{\reg}{\mathrm{reg}}

\begin{prop}
Let $X$ be a surface with nodal singularities whose resolution $Y$ is a surface of general type. Then for $m > 0$,
\[ H^2(X, \hat{S}^m \Omega_X) = 0 \]
\end{prop}

\begin{proof}
From Serre duality for reflexive sheaves on normal surfaces:
\[ h^0(X, \hat{S}^m \Omega) = h^0(X, ((\hat{S}^m \Omega_X)^\vee \ot \omega_X)^{\vee \vee}) \]
The normality propery plus the reflexive nature on $X_{\reg}$ give,
\[ h^0(X, ((\hat{S}^m \Omega_X)^\vee \ot \omega_X)^{\vee \vee}) = h^0(X_{\reg}, (S^m \Omega_X)^\vee \ot \omega_X) = 0 \]
But on $X_{\reg}$ we have the quality $\Omega_X^\vee = \Omega_X \ot \omega_X$ and hence,
\[ h^2(X, \hat{S}^m \Omega_X) = h^0(X_{\reg}, S^m \Omega_X \ot \omega_X^{1-m}) \]
Suppose there is a $m > 2$ such that this is positive. Let $\ell = qm$ and,
\[ s \in H^0(X_{\reg}, S^m \Omega_X \ot \omega_X^{(1 - m)}) \]
a nonzero section and $w \in H^0(X_{\reg}, \omega_X^{(m-2)q})$ then
\[ s^{2q} \ot w \in H^0(X_{\reg}, S^{2 \ell} \Omega_X \ot \omega_X^{-\ell}) \]
Since $m > 2$ this implies that if $q \gg 0$ then $S^{2 \ell} \Omega_X \ot \omega_X^{-\ell}$ has a nontrivial section vanishing at some point of $X_{\reg}$. 
\bigskip\\
The existence of a nontrivial section of $S^{2 \ell} \Omega_X \ot \omega_X^{-\ell}$ vanishing at some point of $X_{\reg}$ implies instability properties of $\Omega_X$ on $X_{\reg}$. Applying the stability theory of Bogomolov-Mumford it follows that there is a line bundle $L \subset \Omega_X$ on $X_{\reg}$ such that the line bundle $(L^2 \ot \omega_X^{-1})^k$ has nontrivial sections for some $k$. The bigness of $K_X$ on $X_{\reg}$ then implies that $L$ is big over $X_{\reg}$. Now we apply the following.
\end{proof}

\begin{lemma}
Let $X$ be a projective surface with nodal singularities. Then there is a smooth cover of $X$ meaning a smooth projective surface $S$ and a finite map $f : S \to X$. In particular, the pre-image of the singular points consists of a finite set of smooth points. 
\end{lemma}


Completing the proof: there is a smoothing cover $f : S \to X$. Denote by $T$ the preimage on $S$ of the set of nodal points of $X$. Let $\d{f} : f^* \Omega_X^1 \to \Omega_S^1$ be the differential. Consider the rank one subsheaf of $\Omega_{X}|_{S\sm T}$ on $S \sm T$ which is the image of the sheaf $\d{f}(L)$. Since $T$ has codimension $2$ on $S$, this extends to a rank one coherent subsheaf of $\Omega_S^1$ and let $L' \subset \Omega_S^1$ be its saturation. A saturated subsheaf of a reflexive sheaf is reflexive and since it is rank $1$ we see that $L'$ is a line bundle.


\begin{lemma}
Let $\F \subset \G$ be a saturated subsheaf and $\G$ reflexive. Then $\F$ is reflexive. In fact $\F$ is reflexive if and only if it is a saturated subsheaf of a vector bundle. 
\end{lemma}

\begin{proof}
Consider the sequence,
\[ 0 \to \F \to \G \to \H \to 0 \]
where $\H$ is torsion-free. Then applying the dual we find,
\[ 0 \to \H^\vee \to \G^\vee \to \F^\vee \to \shExt{1}{}{\H}{\struct{X}} \] 
but the last term is torsion so we get
\[ 0 \to \F^{\vee \vee} \to \G^{\vee \vee} \to \H^{\vee \vee} \to \shExt{1}{}{\F^\vee}{\struct{X}} \]
but now consider the diagram,
\begin{center}
\begin{tikzcd}
0 \arrow[r] & \F^{\vee \vee} \arrow[r] & \G^{\vee \vee} \arrow[r] & \H^{\vee \vee} \arrow[r] & \shExt{1}{}{\F^\vee}{\struct{X}}
\\
0 \arrow[r] & \F \arrow[r] \arrow[u] \arrow[r] & \G \arrow[r] \arrow[u] \arrow[r] & \H \arrow[u] 
\end{tikzcd}
\end{center}
but $\G \to \G^{\vee \vee}$ is an isomorphism and $\H \to \H^{\vee \vee}$ is injective so $\F \to \F^{\vee \vee}$ is an isomorphism via a diagram chase. 
\end{proof}

Now we finish up: we conclude by pulling back that $L'$ is big. However, it is a theorem of Bogomolov that if $L' \subset \Omega_Y$ for a smooth surface $Y$ then $L'$ is not big giving a contradiction. 
\bigskip\\
The following seems like an easier proof: the orbifold cover $\pi : \X \to X$ is \etale away from the singular points. Then we consider $\pi^* \pi_* S^k \Omega_{\X} \to S^k \Omega_{\X}$. We want to show that,
\[ H^2(X, \hat{S}^k \Omega_X) = 0 \]
However, I claim that $\pi_* S^k \Omega_{\X} = \hat{S}^k \Omega_X$. Indeed, if $\E$ is a vector bundle on $\X$ then $\pi_* \E = j_* \E|_U$ where $U$ is the open over which $X$ is smooth and hence isomorphic to $\X$. Indeed, it suffices to show that for $j' : U \to \X$ we have $\E \to j'_* \E|_U$ is an isomorphism. But this can be checked \etale locally on schematic covers where it is clear because $\X$ is smooth and $U$ is big. Therefore,
\[ H^2(X, \hat{S}^k \Omega_X) = H^2(X, \pi_* S^k \Omega_\X) \]  
But $\pi : \X \to X$ is a good moduli space morphism and hence $\pi_*$ is exact on quasi-coherent sheaves. Since $\X$ and $X$ have affine diagonal\footnote{Indeed, locally this map is modeled on $[\A^n / G] \to \A^n / G$ for a finite group quotient. To show that $[\A^n / G]$ has affine diagonal notice that the diagonal is covered by $G \times \A^n \to \A^n \times \A^n$ given by projection and action which is affine since it is a map of affine schemes (here I need $G$ to be a finite or more generally affine group scheme)} this implies that $R^i \pi_* = 0$ for $i > 0$ so by the Leray spectral sequence,
\[ H^2(X, \hat{S}^k \Omega_X) = H^2(\X, S^k \Omega_{\X}) = 0 \]
by Bogomolov's vanishing result for stacks. This proof has the advantage of working directly for products of symmetric powers and hence for $k$-jets. 

\section{Some Questions}

Look at this paper \chref{https://arxiv.org/pdf/1603.02225.pdf}{by Ya Deng}

Questions:
\begin{enumerate}
\item why is ``every entire curve is algebraically degenerate'' enough to imply the GGL conjecture? Why couldn't for each one the locus it maps into be different so that the union is Zariski dense. 
\item Do the conjectures hold for $2$-jets i.e. for resolving foliations on $3$-folds? McQuillan seems to think the answer is yes. 
\end{enumerate}

\section{Note on Horikawa Surfaces}

\chref{https://arxiv.org/pdf/1201.5822.pdf}{In this paper} they prove that certain Horikawa surfaces are hyperbolic just using big orbifold cotangent bundle. Maybe this could give some example?

Specifically look at Proposition 56.


\section{Demailly - Goul}

\begin{defn}
A Riemann surfaces is \textit{parabolic} if it does not admit a positive non constant superharmonic function (equivalently it does not admit a Green function). 
\end{defn}

\begin{rmk}
This is the same as the $1$-dimensional case of Bost's \textit{Liouville property} that every bounded above plurisubharmonic function is constant.
\end{rmk}

\begin{theorem}[Liouville]
Let $M$ be a compact complex manifold. Then any plurisubharmonic function on $M$ is constant. 
\end{theorem}

\begin{theorem}
If a connected complex manifold $M$ satisfies the Liouville property and $F \subset M$ is a closed analytic subset of positive codimension then $M \sm F$ satisfies the Liouville property.
\end{theorem}

{\color{red} IS THIS TRUE?}

\begin{lemma}
Let $\Omega \subset \CC$ be a domain. Let $f : \Omega \sm \{ 0 \} \to \RR$ be a bounded below superharmonic function. Then $f$ extends to $f : X \to \RR$.
\end{lemma}

\begin{proof}
\chref{https://mathoverflow.net/questions/461150/a-compact-riemann-surface-with-a-finite-set-of-points-removed-is-parabolic}{See this answer}.
\end{proof}

Therefore, an open algebraic curve is parabolic if and only if its compactification is parabolic.

\begin{lemma}
Let $X$ be a compact Riemann surface minus finitely many points, then $X$ is parabolic.
\end{lemma}

\begin{proof}
This follows from the extension theorem and Liouville for subharmonic functions.
\end{proof}


\subsection{Notation}

\begin{enumerate}
\item $D_2 := \P(T_{X_1/X}) \subset X_2 = \P(V_1)$ is the zero divisor of the map,
\[ \struct{X_2}(-1) \to \pi_2^* \struct{X_1}(-1) \]
induced by the canonical maps $\struct{X_2}(-1) \embed \pi_2^* V_1$ and $V_1 \to \struct{X_1}(-1)$.
\end{enumerate}

\subsection{Multifoliations}

\begin{defn}
An \textit{algebraic multi-foliation} on $X$ is a rank $1$ subsheaf $\F \subset S^m \Omega_X$ locally generated by a jet differentials of order $1$.
\end{defn}

\begin{theorem}
Let $X$ be a nonsingular surface of general type and let $\theta_k$ be the $k$-jet threshold. Assume that either $\theta_1 < 0$ or that the following three conditions are satisfies,
\begin{enumerate}
\item $\theta_1 \ge 0$ and $\theta_2 < 0$
\item $\Pic{X} = \Z$
\item $\frac{c_1^2}{c_2} > \frac{9}{13 + 12 \theta_2}$
\end{enumerate}
then every nonconstant holomorphic map $f : \CC \to X$ is the leaf of an algebraic multi-foliation on $X$.
\end{theorem}

\begin{prop}
Let $X$ be a minimal surface of general type, equipped with an algebraic multi-foliation $\F \subset S^m \Omega_X$. Assume that
\[ m (c_1^2 - c_2) + c_1 \cdot c_1(\F) > 0 \]
Then there is a curve $\Gamma$ in $X$ such that all parabolic leaves of $\F$ are contained in $\Gamma$.
\end{prop}

\begin{proof}
Note that any rank $1$ torsion-free {\color{red} reflexive} sheaf on a surface is locally free. The inclusion $\F \embed S^m \Omega_X$ viewed as a section of $S^m \Omega_X \ot \F^{\vee}$ defines a section of $\struct{X_1}(m) \ot \pi^* \F^{-1}$ whose zero divisor $Z \subset X_1$ is precisely the divisor associated with the foliaiton. Therefore $Z = m u_1 - \pi^* c_1(\F)$ in $\Pic{X_1}$ and our calculuations imply that $\struct{X_1}(1)|_Z$ is big as soon as,
\[ (u_1|_Z)^2 = u_1^2 \cdot Z = m (c_1^2 - c_2) + c_1 \cdot c_1(\F) > 0 \quad (u_1|_Z) \cdot (-c_1) = m c_1^2 + c_1 \cdot c_1(\F) > 0 \]
However, $X$ is minimal so $c_2 \ge 0$ hence the first inequality implies the second proving the claim.
\end{proof}

\begin{prop}
Let $X$ be a surface of general type equipped with a multifoliation $\F \subset S^m \Omega_X$ and let $\sigma \in H^0(X_1, \struct{X_1}(m) \ot \pi^*_{1,0} \F^\vee)$ be the associated canonical section. Let $G_1$ be the divisorial part of the subscheme defined by $\ker{\d{\sigma}|_{V_1|Z}}$ then if,
\[ m^2 (4 c_1^2 - 3 c_2) + m(5 c_1^2 - 3 c_2) + (8m + 4)c_1 \cdot \F + 3 \F^2 - (3 u_1 - c_1) \cdot G_1  > 0\]
all parabolic leaves of $\F$ are algebraically degenerate.  
\end{prop}

{\color{red} why does algebraically degneracy of each leaf imply it for the union?}

\section{Proof}

\begin{theorem}[Dem95]
Let $X$ be an algebraic surface of general type and $A$ an ample line bundle. Then,
\[ h^0(X, P^{2,m} \ot A^{-1}) \ge \frac{m^4}{648} (13 c_1^2 - 9 c_2) + O(m^3) \]
\end{theorem}

Consider the Semple tower $X_2 \to X_1 \to X$ which are iterated $\P^1$-bundles. Thus,
\[ \Pic{X_2} = \Pic{X} \oplus \Z u_1 \oplus \Z u_2 \]
so every line bundle is of the form,
\[ \pi_{2,1}^* \struct{X_1}(a_1) \ot \struct{X_2}(a_2) \ot \pi_{2,0}^* \L \]
where we write,
\[ u_1 = \pi_{2,1}^* \struct{X_1}(1) \quad u_2 = \struct{X_2}(1) \]
The canonical exact sequence,
\begin{center}
\begin{tikzcd}
0 \arrow[r] & T_{X_1/X} \arrow[r] & V_1 \arrow[r] & \struct{X_1}(-1) \arrow[r] & 0
\end{tikzcd}
\end{center}
which serves as the definition of $V_1$ and the canonical injection,
\[ \struct{X_2}(-1) \embed \pi_2^* V_1 \]
yield the morphism
\[ \struct{X_2}(-1) \embed \pi_2^* \struct{X_1}(-1) \]
which admits the hyperplane section $D_2 := \P(T_{X_1/X}) \subset X_2 = \P(V_1)$ as its zero divisor. Therefore,
\[ \struct{X_2}(-1) = \pi_2^* \struct{X_1}(-1) \ot \struct{}(-D_2) \]

\begin{lemma}
With respect to the projection $\pi_{2,0} : X_2 \to X$ the weighted line bundle $\struct{X_2}(a_1, a_2)$ is
\begin{enumerate}
\item relatively effective iff $a_1 + a-2 \ge 0$ and $a_2 \ge 0$
\item relativiely big iff $a_1 + a_2 > 0$ and $a_2 > 0$
\item relatively nef iff $a_1 \ge 2 a_2 \ge 0$
\item relatively ample iff $a_1 > 2 a_2 > 0$.
\end{enumerate}
Moreover, the following properties hold,
\begin{enumerate}
\item for $m = a_1 + a_2 \ge 0$ there is an injection
\[ (\pi_{2,0})_* \struct{X_2}(a_1, a_2) \embed P^{2,m} \]
which is an isomorphism for $a_1 - 2 a_2 \le 0$
\item let $Z \subset X_2$ be an irreducible divisor such that $Z \neq D_2$ then in $\Pic{X_2}$ 
\[ Z \sim a_1 u_1 + a_2 u_2 + \pi_{2,0}^* L \]
where $a_1 \ge 2 a_2 \ge 0$
\item Let $F \in \Pic{X}$ be any divisor class. In $H^*(X_2) = H^*(X)[u_1, u_2]$ the following hold,
\[ u_1^4 = 0 \quad u_1^3 u_2 = c_1^2 - c_2 \quad u_1^2 u_2^2 = c_2 \quad u_1 u_2^3 = c_1^2 - 3 c_2 \quad u_2^4 = 5 c_2 - c_1^2 \]
and 
\[ u_1^3 \cdot F = 0 \quad u_1^2 u_2 \cdot F = - c_1 \cdot F \quad u_1 u_2^2 \cdot F = 0 \quad u_2^3 \cdot F = 0 \]
\end{enumerate}
\end{lemma}

\begin{proof}
{\color{red} DO THIS}
\end{proof}

Let $B_2$ be the base locus of $\struct{X_2}(m)$ on $X_2$.

\begin{prop}
Let $X$ be a minimal surface of general type. If $c_1^2 - 9/7 c_2 > 0$ then the restriction of $\struct{X_2}(1)$ to every irreducible $3$-dimensional component $Z$ of $B_2 \subset X_2$ which projects onto $X_1$ and differs from $D_2$ is big.
\end{prop}

\begin{proof}
Because $Z$ is effective we can write,
\[ Z = a_1 u_1 + a_2 u_2 - \pi_{2,0}^* F \quad (a_1, a_2) \in \Z^2 \quad a_1 \ge 2 a_2 > 0 \]
where $F$ is some divisor in $X$. Our strategy is to show that $\struct{X_2}(2,1)|_Z$ is big. By Lemma 3.3(e) we find,
\[ (2 u_1 + u_2)^3 \cdot Z = (_1 + z_2) (13 c_!^2 - 9 c_2) + 12 c_1 \cdot F \]
The multiplication morphism by the canonical section of $\struct{}(Z)$ defines a sheaf injection
\[ \struct{}(\pi_{2,0}^* F) \embed \struct{X_2}(a_1, a_2) \]
\end{proof}

\subsection{2-jets}

Let $Z \subset X_1$ be the divisor associated with a given foliation $\F$ and $\sigma \in H^0(X_1, \struct{X_1}(m) \ot \pi^* \F^{-1})$ the corresponding section. Let $\T_Z$ be the tangent sheaf of $Z$ defined by the exact sequence,
\begin{center}
\begin{tikzcd}
0 \arrow[r] & \T_Z \arrow[r] & \T_{X_1}|_Z \arrow[r, "\d{\sigma}"] & \struct{X_1}(m)|_Z \ot \pi^* \F^{-1}|_Z \arrow[r] & 0
\end{tikzcd}
\end{center}
If we define $S = \T_Z \cap \struct{}(V_1)$ then there is an exact sequence
\begin{center}
\begin{tikzcd}
0 \arrow[r] & S \arrow[r] & V_1|_Z \arrow[r, "\d{\sigma}"] & \struct{X_1}(m)|_Z \ot \pi^* \F|_Z^{-1}
\end{tikzcd}
\end{center}
and $S$ is an invertible sheaf so there is a dual sequence
\begin{center}
\begin{tikzcd}
0 \arrow[r] & \struct{X_1}(-m)|_Z \ot \pi^* \F|_Z \arrow[r] & V^*_1 |_Z \arrow[r] & S^* 
\end{tikzcd}
\end{center}
We can then lift $Z$ into a surface $\wt{Z} \subset X_2$ in such a way that the projection map $\pi_{2,1} : \wt{Z} \to Z$ is a modification, at a generic point $x \in Z$, the point of $\wt{Z}$ lying above $x$ is taken to be $(x, [S_x]) \in X_2$ (meaning we take the induced morphism from the generic quotient map $V_1^*|_Z \to S^*$ which gives a lift to the projectiviation). We need to compute the cohomology class of this $2$-cycle $\wt{Z}$ in $H^\bullet(X_2)$. One problem is that the cokernel of the map
\[ \d{\sigma}|_{V_1|_Z} : V_1|_Z \to \struct{X_1}(m)|_Z \ot \F^{-1}|_Z \]
may have torsion along a $1$-cycle $G_1 \subset Z$. If the foliation is generic, however, the cokerne of $\d{\sigma}|_{V_1|_Z}$ will have no torsion in codimension $1$ and $\d{\sigma}$ then induces a section of
\[ \struct{X_2}(1) \ot \pi_{2,1}^* \struct{X_1}(m) \ot \pi^*_{2,0} \F^{-1} \]
whose zero locus in $\wt{Z}$. As $Z \sim m u_1 - c_1(\F)$ the cohomology class of $\wt{Z}$ is given by,
\[ [\wt{Z}] = (m u_1 - c_1(\F)) \cdot (u_2 + m u_1 - \F) = m^2 u_1^2 + m u_1 \cdot u_2 - 2 m u_1 \cdot c_1(\F) - u_2 \cdot c_1(\F) + c_1(\F)^2 \]
A Chern class computation yields
\[ (2 u_1 + u_2)^2 \cdot \wt{Z} = m^2 (4 c_1^2 - 3 c_2) + m (5 c_1^2 - 3 c_2) + (8 m + 4) c_1 \cdot c_1(\F) + 3 c_1(\F)^2 \]
If the $1$-cycle $G_1$ is nonzero, the numerical formula becomes,
\[ [\wt{Z}] = (m u_1 - \F) \cdot (u_2 + m u_1 - c_1(\F)) - \pi_{2,1}^* [G_1] \]
The generical formula for $(2 u_1 + u_2)^2 \cdot [\wt{Z}]$ is therefore,
\[ (2 u_1 + u_2)^2 \cdot [\wt{Z}] = m^2 (4 c_1^2 - 3 c_2) + (5 c_1^2 - 3 c_2) + (8m + 4) c_1 \cdot c_1(\F) + 3 c_1(\F)^2 - (3 u_1 - c_1) \cdot [G_1] \]
Using the obvious exact sequences $H^2(\wt{Z}, \struct{X_2}(2m, m)|_{\wt{Z}})$ is a quotient of
\[ H^2( \pi_{2,1}^{1}(Z), \struct{X}(2m,m) |_{\pi_{2,1}^{-1}(Z)}) \]
which is controlled by $H^2(X_2, \struct{X_2}(2m,m))$ and $H^3(X_2, \struct{X_2}(2m,m) \ot \struct{}(-Z))$ a direct image argument shows that the latter groups are controlled by groups of the form $H^2(X, E_{2,3m} \Omega \ot L)$ for suitable $L$. s in the proof of Theorem 3.4 one can check that the latter $H^2$ groups vanish. The positivity of $(2 u_1 + u_2)^2 \cdot [\wt{Z}]$ therefore implies that $\struct{X_2}(2,1)|_{\wt{Z}}$ is big and therefore all parabolic leave of the multifoliation $\F$ are algebraically degenerate because the parabolic leaves of $\F$ must lie in the base locus of all invariant differentials and this cuts out a $1$-dimensional space on $\wt{Z}$ by the bigness and the fact that $\wt{Z}$ is $2$-dimensional.

\end{document}