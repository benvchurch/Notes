\documentclass[12pt]{article}
\usepackage{import}
\import{./}{AlgGeoCommands}
\renewcommand{\U}{\mathfrak{U}}
\newcommand{\Frob}{\mathrm{Frob}}

\newcommand{\st}{\mathrm{st}}

\begin{document}

\section{Math 245B Topics in algebraic geometry: Deligne-Lustzig Theory}

Note: no class week of Jan 29th and zoom the week after. 

The course is about $\CC$-rep theory of finite groups of Lie type e.g. $\GL_3(\FF_8)$ or $\Sp_8(\FF_{27})$ or $\SO_5(\FF_3)$. The goal is to construct all the (irreducible) representations. 

\begin{example}
Consider $G = \SL_2(\FF_q)$ for $p > 2$. Then $T(\FF_q) \subset B(\FF_q) \subset \SL_2(\FF_q)$ be the torus and upper-triangular Borel. Given a character $\theta : T(\FF_q) \to \CC^\times$ consider the map $B \to T$ quotienting by the unipotent part then get a $G$-rep $\Ind{G}{B(\FF_q)} \theta$. If $\theta$ is trivial then $\Ind{G}{B(\FF_q)}{\theta} = \mathrm{Fun}(\P^1(\FF_q), \CC)$ with the standard $\SL_2(\FF_q)$-action. This has a subrep of the constant functions giving an exact sequence,
\begin{center}
\begin{tikzcd}
0 \arrow[r] & \CC \arrow[r] & \Ind{G}{B(\FF_q)}{1} \arrow[r] & \text{st} \arrow[r] & 0
\end{tikzcd}
\end{center}
where $\text{st}$ is the Steinberg. This is irreducible (exercise). Does this proceedure give all representations? No. 
\end{example}

\begin{example}
If $\theta^2 \neq 1$ then $\Ind{G}{B(\FF_q)}{\theta} = \Ind{G}{B(\FF_q)}{\theta^{-1}}$ so we get fewer representations. If $p > 2$ and $q = p^r$ then we get $\frac{q+5}{2}$ irreps of $\SL_2(\FF_q)$ from this proceedure. However, there are $q + 4$ conjugacy classes and thus irreps. 
\end{example}

The other half of the reps must come from a different construction. Frobenius was able to write these down in the 1890s but we want a general proceedure for all groups of Lie type. Macdonald conjectured that these are related to characters of $T^1(\FF_q) \subset \SL_@(\FF_q)$ which is the nonsplit torus $\FF_{q^2}^\times \subset \GL_2(\FF_q)$ intersected with $\SL_2$. Problem, is there is no $\FF_q$-stable Borel containing this. Drinfeld gives us the solution. Consider the curve,
\[ C = \{ xy^q - yx^q = 1 \} \subset \A^2_{\FF_q} \]
which has commuting actions of $\SL_2(\FF_q)$ are $\mu_{q+1}$ given by,
\[ \begin{pmatrix}
a & b
\\
c & d
\end{pmatrix} 
\cdot (x,y) \mapsto (ax + by, cx + d y) \]
and
\[ \zeta \cdot (x,y) \mapsto (\zeta x, \zeta y) \]
Then for $\theta : \mu_{q+1} \to \overline{\Q_{\ell}}$ (which is abstractly isomorphic to $\CC$) then we get a representation,
\[ \SL_2(\FF_q) \acts H^1_{\et}(C_{\overline{\FF}_q}, \overline{\Q_\ell})[\theta] \]
where this is the part where $\mu_{q+1}$ acts by $\theta$. These give the remaining representations. 

\begin{rmk}
Notice that $C$ is a $\mu_{q+1}$-cver of $\P^1_{\FF_q} \sm \P^1(\FF_q)$. 
\end{rmk}

\section{Representation Theory of Finite Groups}

\newcommand{\Vect}{\mathrm{Vect}}

\begin{defn}
Let $G$ be a finite group and $k$ a field. A $k$-representation of $G$ is a pair $(V, \pi)$ where $V$ is a finite-dimensional $k$-vectorspace and $\pi : G \times V \to V$ is a $k$-linear action of $G$. A morphism of representations $f : (V, \pi) \to (V', \pi')$ is a linear map $f : V \to V'$ such that,
\begin{center}
\begin{tikzcd}
G \times V \arrow[d, "\pi"] \arrow[r, "\id \times f"] & G \times V' \arrow[d, "\pi'"]
\\
V \arrow[r, "f"] & V' 
\end{tikzcd}
\end{center}
This category is called $\Rep{k}{G}$.
\end{defn}

\begin{prop}
$\Rep{k}{G}$ is abelian and $F : \Rep{k}{G} \to \Vect_k$ commutes with all limits and colimits. Furthermore, $\Rep{k}{G}$ is monoidal and $F$ is a monoidal functor with the usual $\otimes$ on $\Vect_k$. 
\end{prop}

\begin{prop}[Maschke]
If $\# G \in k^\times$ then $\Rep{k}{G}$ is semisimple.
\end{prop}

\begin{defn}
Given $(V, \pi, \rho)$ there is a function $\chi_V : G \to k$ via $g \mapsto \tr{\rho(g)}$ called the \textit{character}. 
\end{defn}

\begin{theorem}[Orthogonality]
If $\# G = k^\times$ and $V, V'$ are $G$-reps then,
\[ \frac{1}{\# G} \sum_{g \in G} \chi_V(g) \chi_{V'}(g^{-1}) = \dim \Hom{G}{V}{V'} \]
inside $k$.
\end{theorem}

\begin{proof}
The LHS is,
\[ \frac{1}{\# G} \sum_{g \in G} \tr{\left( g | \Hom{}{V}{V'} \right)} \]
and for any $w \in \Rep{k}{G}$ we have,
\[ \frac{1}{\# G} \sum_{g \in G} \tr{\left( g | W \right)} = \dim{W^G} \]
\end{proof}

\begin{prop}
Let $\# G \in k^\times$ and $k = \bar{k}$. Then $\{ \chi_V \}$ for $V$ irreps span the space of conjugation invariant functions $G \to k$. 
\end{prop}

\section{Jan 11}

Fix a finite group $G$ and a field $k$ s.t. $\# G \in k^\times$ and $k = \bar{k}$. If $H \subset G$ is a subgroup, then there is a functor,
\[ \Res{G}{H}{-} : \Rep{k}{G} \to \Rep{k}{H} \]
which has both a left and a right adjoint given by 
\[ \Ind{G}{H}{-} : \Rep{k}{H} \to \Rep{k}{G} \]
which is defined by,
\[ V \mapsto \{ f : G \to V \mid \forall h \in H, g \in G : f(h g) = \rho_V(h) f(g) \} \]

\begin{rmk}
$\dim \Ind{G}{H}{V} = [G : H] \dim{V}$.
\end{rmk} 

\begin{rmk}
A goal of Mackey theory is to understand when induced representations are irreducible.
\end{rmk}

\newcommand{\inner}[2]{\left< #1, #2 \right>}

\begin{defn}
We notate the induced character,
\[ \chi^G_V = \chi_{\Ind{G}{H}{V}} \]
so therefore Frobenius reciprocity (the adjunction) is given by the corresponding statement for pairing characters,
\[ \inner{\chi_V^G}{\chi_V^G}_G = \inner{\chi_V}{\chi^G_V|_H}_H \]
Recall, by character theory $\Ind{G}{H}{V}$ is absolutely irreducible iff the above pairing is $1$.
For $g \in G$ we write $H^g$ for $g H g^{-1} \subset G$ and $\rho : H \to \GL(V)$ I write $\rho^g : g H g^{-1} \to \GL(V)$ with $g h g^{-1} \mapsto \rho(h)$. Note that $H \cap H^g$ only depends, up to isomorphism, on $[g] \in H \backslash G / H$.
\end{defn}

\begin{thm}
\[ \Res{G}{H}{\Ind{G}{H}{\rho}} = \bigoplus_{[g] \in H \backslash G / H} \Ind{H}{H \cap H^g}{\Res{H^g}{H \cap H^g}{\rho^g}} \]
\end{thm}

\begin{cor}
$\Ind{G}{V}{V}$ is irreducible iff $V$ is irreducble and $\Res{H^g}{H^g \cap H}{\chi}$ and $\Res{H^g}{H^g \cap H}{\rho^g}$ share no common irreducible factors (other than $g = 1$).
\end{cor}

\begin{proof}
\begin{align*}
\inner{\chi_V^G}{\chi_V^G}_G & = \inner{\chi_V}{(\chi_V^G)_H}_H = \sum_{g \in H \backslash G / H} \inner{\chi_V}{\chi_{\Ind{H}{H \cap H^g}{\Res{H^g}{H \cap H^g}{\rho^g}}}} 
\\
& = \sum_{g \in H \backslash G / H}\inner{\Res{H}{H \cap H^g}{\chi}}{\Res{H^g}{H \cap H^g}{\chi^g}}
\end{align*}
Each term in the sum is a positive integer so we must have exactly one of them is equal to $1$. 
\end{proof}

\begin{example}
Apply this to $G = \SL_2(\FF_q)$ and $H = B(\FF_q)$. Let,
\[ s = \begin{pmatrix}
0 & - 1
\\
1 & 0
\end{pmatrix} \]
then,
\[ s 
\begin{pmatrix}
a & b
\\
c & d
\end{pmatrix}
s^{-1} = 
\begin{pmatrix}
d & -c
\\
-b & a
\end{pmatrix} \]
Conjugation by $s$ preserves $T(\FF_q)$ and axts as inversion on it. Then $B(\FF_q) \cap s B(\FF_q) s^{-1} = T(\FF_q)$. 
\end{example}

\begin{lemma}
$\SL_2(\FF_q) = B(\FF_q) \cup B(\FF_q) s B(\FF_q)$ is the Bruhat decomposition. 
\end{lemma}

If we start with $\theta_1, \theta_2 : T(\FF_q) \to \CC^\times$ and consider them as representations of $B(\FF_q) \to T(\FF_q)$ then,
\[ \inner{\Ind{SL_2(\FF_q)}{B(\FF_q)}{\theta_1}}{\Ind{\SL_2(\FF_q)}{B(\FF_q)}{\theta_2}}_G = \inner{\theta_1}{\theta_2}_T + \inner{\theta_1}{\theta_2^s}_T \]

\begin{cor}
If $\theta_1 = \theta_2$ we find $\Ind{\SL_2(\FF_q)}{B(\FF_q)}{\theta}$ is irred if $\theta_1 \neq \theta_1^{-1}$. If $\theta_1 \in \{ \theta_2, \theta_2^{-1} \}$ then $\Ind{-}{-}{\theta_1}$ and $\Ind{-}{-}{\theta_2}$ shrea no common factors. 
\end{cor}

If $p > 2$ then there are $q - 3$ characters $\theta$ with $\theta \neq \theta^{-1}$ and therefore $\frac{q-3}{2}$ irreps of $\SL_2(\FF_q)$. Then,
\[ \Ind{-}{-}{1} = 1 + \text{st} \]
and for $\alpha \neq 1$ with $\alpha^2 = 1$
\[ \Ind{-}{-}{\alpha} = R(\alpha)_+ + R(\alpha)_+ \]
with $R(\alpha)_+$ and $R(\alpha)_-$ are nonisomorphic representations of the same dimension. Therefore we have found,
\[ \frac{q - 3}{2} + 4 = \frac{q + 5}{2} \]
representations.

\begin{defn}
A representation of $\SL_2(\FF_q)$ that does not contain any of the previous representation as a summand is called \textit{cuspidal}. 
\end{defn}

\begin{example}
Consider $\SL_2(\ZZ_p) \embed \SL_2(\QQ_p)$ and $\SL_2(\ZZ_p) \to \SL_2(\FF_p)$ and let $\SL_2(\ZZ_p)$ act on $V$ via a cuspidal rep of $\SL_2(\FF_p)$ then c-Ind to $\Q_p$ is cuspidal. 
\end{example}

\section{$\ell$-adic Cohomology}

Let $X$ be a smooth projective $\FF_q$-variety. Then can define,
\[ \zeta_X(T) = \exp{ \left( \sum_{n \ge 1} \# X(\FF_{q^n}) \frac{T^n}{n} \right)} \in \Q\dbrac{T} \]
 
\begin{example}
$X = \Spec{\FF_q}$ then, 
\[ \zeta_X(T) = \frac{1}{1 - T} \]
If $X = \P^1_{\FF_q}$ then,
\[ \zeta_X(T) = \frac{1}{(1 - T)(1 - q T)} \]
If $X = E$ is an elliptic curve over $\FF_q$ then,
\[ \zeta_X(T) = \frac{(1 - \alpha T)(1 - \beta T)}{(1 - T)(1 - q T)} \]
\end{example}

\begin{conj}[Weil]
$\zeta_X$ is a rational function. 
\end{conj}

\begin{proof}
Weil's idea: we are counting fixed points of $\Frob^r_q$ on $X_{\overline{\FF}_q}$. Now, if $M$ is a compact oriented manifold and $\psi : M \to M$ continuous with isolated fixed points then,
\[ \# \mathrm{fix}(\psi) = \sum_{i} (-1)^i \tr{( \psi_* | H^i_{\text{sing}}(M, \RR))} \]
This implies that the exponential generating function for $\# \mathrm{fix}(\psi^n)$ is a rational function. 
\end{proof}

Is there an ``algebraic definition'' of singular cohomology for $X$ smooth projective over $\CC$. Then $H^0_{\text{sing}}(X(\CC), \Z) = \pi_1(X(\CC))^\ab$ but $\CC^\times$ has a $\ZZ$-cover $\exp : \CC \to \CC^\times$ which is not algebraic. However, Riemann existence proves that all \textit{finite} covering spaces \textit{are} algebraic. Therefore, $H^1_{\text{sing}}(X(\CC), \Z / n \Z)$ has an algebraic definnition. 
\bigskip\\
Serre gives a simple argument that shows there cannot exist a cohomology theory for smooth projective $\FF_q$-varities which is valued in $\Q$-vectorspaces such that $H^1(E, \Q)$ is a two-dimensional $\Q$-vectorspace. This is because $\End{E}$ is a quaternion algebra and this cannot act on $\Q^2$ in the necessary way. 
\bigskip\\
So we could hope to define a cohomology theory with values in $\Z / \ell^n \Z$ for $\ell \neq p$ this gives a theorey with values in $\varprojlim \ZZ / \ell^n \ZZ = \ZZ_\ell$ and thus in $\ZZ_\ell[\ell^{-1}] = \QQ_\ell$.

\begin{theorem}[Grothendieck-Deligne-Artin] 
Yes this is possible. There is a functor \[ H^i_{\et}(-, \QQ_\ell) : \{ \text{sm proj varities over }^\op \overline{\FF}_p \} \to \{ \text{fin dim } \QQ_{\ell}\text{-vector spaces} \} \] 
such that,
\begin{enumerate}
\item $H_{\et}^i(X, \Q_\ell) = 0$ unless $0 \le i \le 2 \dim{X}$
\item $H^0_{\et}(X, \Q_\ell) = \Q_\ell[\pi_0(X)]$

\item If $X$ lift to $\wt{X}$ over $\CC$ then,
\[ H^i_{\text{sing}}(\wt{X}(\CC), \Q_\ell) = H^i_{\et}(X, \Q_\ell) \]

\item $H^i_{\et}(X, \Q_\ell) = H^{2 d - i}(X, \Q_\ell)^\vee$ if $X$ is equidimensional of dimension $d$

\item if $\psi : X \to X$ has isolated fixed points then,
\[ \# \text{fix}(\psi) = \sum_i (-1)^i \tr{( \psi_* | H^i_{\et}(X, \Q_\ell))} \]

\item if $X$ is over $\FF_q$ then,
\[ \# X(\FF_{q^n}) = \sum_{i} (-1)^i \tr{( \Frob_{q}^n | H^i_{\et}(X_{\overline{\FF}_{q}}, \Q_\ell))} \]
\end{enumerate}
\end{theorem}



\begin{theorem}
There are also functors,
\[ H^i_{c}(-, \QQ_\ell) : \{ \text{varities over }^\op \overline{\FF}_p \text{ with proper maps} \} \to \{ \text{fin dim } \QQ_{\ell}\text{-vector spaces} \} \] 
such that,
\begin{enumerate}
\item $H_c^i(X, \Q_\ell) = H^iX, \Q_\ell)$ if $X$ is proper / projective
\item $H_{c}^i(X, \Q_\ell) = 0$ unless $0 \le i \le 2 \dim{X}$

\item If $X$ is smooth and affine then $H^i_c(X, \Q_\ell) = 0$ for $0 \le i \le \dim{X}$

\item If $Z \subset X$ is closed then is the a LES,
\begin{center}
\begin{tikzcd}
\cdots \arrow[r] & H^i_c(U, \Q_\ell) \arrow[r] & H^i_c(X, \Q_\ell) \arrow[r] & H^i_c(Z, \Q_\ell) \arrow[r] & H^{i+1}_c(U, \Q_\ell) \arrow[r] & \cdots
\end{tikzcd}
\end{center}

\item if $\psi : X \to X$ has isolated fixed points then,
\[ \# \text{fix}(\psi) = \sum_i (-1)^i \tr{( \psi_* | H^i_{c}(X, \Q_\ell))} \]

\item if $X$ is over $\FF_q$ then,
\[ \# X(\FF_{q^n}) = \sum_{i} (-1)^i \tr{( \Frob_{q}^n | H^i_{c}(X_{\overline{\FF}_{q}}, \Q_\ell))} \]
\end{enumerate}
\end{theorem}

Let $C$ be the Drinfeld curve over $\FF_q$ equipped with actions of $\SL_2(\FF_q)$ and $\mu_{q+1}$. Let $\theta$ be a character of $\mu_{q+1}$ with values in $\Q_\ell$. 

\begin{defn}[Deligne-Lustzig induction]
Let $[\theta]$ denote $\Hom{\mu_{p+1}}{\theta}{-}$ then let,
\[ R(\theta) = H^0_c(C_{\overline{\FF}_p}, \Q_\ell) [ \theta] - H^1_c(C_{\overline{\FF}_p}, \Q_\ell)[\theta] + H^2_c(C_{\overline{\FF}_p}, \Q_\ell)[\theta] \]
in the grothendieck group of representations. 
\end{defn}

\section{Jan. 18}

\newcommand{\FFbar}{\overline{\FF}}
\newcommand{\lbar}[1]{\overline{#1}}
\newcommand{\Qbar}{\lbar{\QQ}}

Recall the Drinfeld curve $C$ (for fixed $q = p^r$) given by,
\[ \{ X Y^q - Y X^q = 1 \} \subset \A^2_{\FF_q} \]
This has an action of $\SL_2(\FF_q)$ given by,
\[ \begin{pmatrix}
a & b
\\
c & d
\end{pmatrix}
\cdot (x, y) = (a x + by, c x + d y) \]
and by $\mu_{q + 1}$ given by,
\[ \zeta \cdot (x, y) = (\zeta x, \zeta y) \]
Observation: $C(\FF_q) = \empty$. For some character, 
\[ \theta : \mu_{q + 1} \to \Qbar_\ell^\times \]
we define the virtual representation,
\[ R'(\theta) = H^2_c(C_{\overline{\FF}_q}, \Qbar_\ell)[\theta] - H^1_c(C_{\overline{\FF}_q}, \Qbar_\ell)[\theta] \]
Here for $W \in \Rep{\mu_{q+1}}$ we write,
\[ W[\theta] = \{ w \in W \mid \zeta \cdot w = \theta(\zeta) \cdot w \} \]
We start by computing,
\[ R'(1) = H^i_c(C_{\FFbar_q}, \Qbar_\ell)^{\mu_{q+1}} = H^i_c(C_{\FFbar_{q}} / \mu_{q+1}, \Qbar_\ell) \]

\begin{lemma}
The map $C \to \P^1_{\FF_q} \sm \P^1_{\FF_q}(\FF_q)$ is a quotient map by the $\mu_{q+1}$-action. 
\end{lemma}

\begin{proof}
Since $[\zeta \cdot X, \zeta \cdot Y] = [X, Y]$ the map is $\mu_{q+1}$-invariant.
\bigskip\\
The action is clearly free since $(0, 0)$ is not on the curve.
\bigskip\\
Claim that the map is surjective. Indeed, given $[1 : T] \in \P^1(\FFbar_q) \sm \P^1(\FF_q)$. We want to find some $\lambda \in \FFbar_q^\times$ such that $[ \lambda : \lambda T]$ is on the curve:
\[ \lambda^{q+1} (T^q - T) = 1 \]
which solvable since $T^q \neq T$ and $\FFbar_q^\times$ has all $(q+1)$-roots. 
\bigskip\\
If $(\lambda, \lambda T)$ and $(\lambda', \lambda' T)$ are two different solutions then $\lambda = \zeta \lambda'$ for $\zeta \in \mu_{q+1}$ which is true because the solutions are exactly the $(q+1)$-roots of $(T^q - T)^{-1}$. 
\bigskip\\
Therefore, $C(\FFbar_q) / \mu_{q+1} = \P^1(\FFbar_q) \sm \P^1(\FF_q)$. In fact, this is an isomorphism of schemes. 
\end{proof}

Now we compute! Let $U = \P^1_{\FF_q} \sm \P^1(\FF_q)$. Take the long-exact sequence,
\begin{center}
\begin{tikzcd}[column sep = small]
0 \arrow[r] & H^0_c(U_{\FFbar_q}, \Qbar_\ell) \arrow[r] & H^0(\P^1, \Qbar_\ell) \arrow[d, equals] \arrow[r] & H^0(Z_{\FFbar_q}, \Qbar_\ell) \arrow[d, equals] \arrow[r] & H^1_c(U_{\FFbar_q}, \Qbar_\ell) \arrow[r] & H^1(\P^1, \Qbar_\ell) \arrow[d, equals] \arrow[r] & 0
\\
& & \Qbar_\ell & 1 \oplus \text{st} & & 0
\end{tikzcd}
\end{center} 
and furthermore $H^2_c(U_{\FFbar_q}, \Qbar_\ell) = H^2(\P^1, \Qbar_\ell) = \Qbar_\ell(-1)$. The map $H^0(\P^1) \to H^0(Z)$ is injective so we see that,
\[ H^0_c(U_{\FFbar_q}, \Qbar_\ell) = 0 \quad \text{ and } \quad H^1_c(U_{\FFbar_q}, \Qbar_\ell) = \text{st} \]
Therefore,
\[ R'(1) = \st - 1 \]
Because there are no $\mu_{q+1}$-fixed points, the trace formula tells us that,
\[ \tr{(\zeta | H^2_c(C))} - \tr{(\zeta | H^1_c(C))} = 0 \]
This characterizes the regular representation of $\mu_{q+1}$. So the character of the virtual representation, $H^1_c(C) - H^2_c(C)$ is a multiple of the regular representation of $\mu_{q+1}$.
\bigskip\\
If we then apply $[\theta]$ for $\theta \neq 1$ we get an actual representation since $H^2_c(C)$ is trivial as an $\SL_2(\FF_q)$-representation. The degree of $H^1_c(C)[\theta]$ us then the same as the degree of $H^1_c(C)[1] - H^2_c(C)[1] = \st - 1$ which has dimension $q-1$. This argument works because this virtual character is the same as the regular representation and thus contains every irrep with equal degree. 

\begin{theorem}
If $\theta \neq 1$ then $H^1_c(C_{\FFbar_q}, \Qbar_\ell)[\theta]$ is cuspidal. 
\end{theorem}

\begin{proof}
Consider,
\[ U = \left< 
\begin{pmatrix}
1 & b
\\
0 & 1 
\end{pmatrix} \right> \subset \SL_2(\FF_q) \]
Then,
\[ \Rep_{\Qbar_\ell}{T} \to \Rep_{\Qbar_\ell}{B} \to \Rep{\Qbar_\ell}{\SL(\FF_q)} \]
where the first map is given by quotienting by $U$ and the second by induction. To show that our given representation is orthogonal to the image, it suffices to show it restricted to $B$ is orthogonal to $\Rep{\Qbar_\ell}{T}$. Therefore, it suffices to show that,
\[ (H^1_c(C)[\theta])_U = (H^1_c(C)[\theta])^U = 0 \]
So we need to understand $H^1_c(C/U, \Qbar_\ell)$ with the action on $\mu_{q+1}$. What is the quotient by $U$. Notice that,
\[ \begin{pmatrix}
1 & b 
\\
0 & 1 
\end{pmatrix} \cdot (x, y) = (x + b y, y) \]
so we expect that $C \to \Gm$ sending $(x,y) \mapsto y$ is the quotient map with fiber $\FF_q$.
\end{proof}

\section{Jan. 20}

Since the actions of $\SL_2$ and $\mu_{q+1}$ commute we see that,
\[ ( H^1_c(C)[\theta])_U = (H^1_c(C)[\theta])^U = (H^1_c(C))^U[\theta] = H^1_c(C/U)[\theta] \]

\begin{lemma}
The map $f : C \to \A^1 \sm \{ 0 \}$ via $(x,y) \mapsto y$ induces the quotient by $U$. 
\end{lemma}

\begin{proof}
The map is $U$ invariant. Now the action is $(x, y) \mapsto (x + by, y)$. Surjectivity, given $y \neq 0$ there is always a root of,
\[ x^q y - y^q x - 1 = 0 \]
in $\FFbar_q$. Given two solutions we need to show they are related by the action. Given two solutions $x_1, x_2$ we want to find $b$ such that $x_2 = x_1 + b$. Let $b = y^{-1}(x_2 - x_1) \in \FFbar_q$ this is the unique choice of $b$. Thus we need to show that $b \in \FF_q$. This is equivalent to showing that $b^q = b$. Indeed,
\[ b^q = y^{-q} (x_2^q - x_1^q) = y^{-(q+1)} y(x_2^q - x_1^q) = y^{-(q+1)} y^q (x_2 - x_1) = b \]
using the defining equations. Now some algebraic geometry facts will tell us $C/U \to \A^1 \sm \{ 0 \}$ is an isomorphism. 
\end{proof}

Now,
\[ H^i_c(\A^n) = 
\begin{pmatrix}
1 & i = 2n
\\
0 & i \neq 2n
\end{pmatrix} \]
Furthermore, the $\mu_{q+1}$ action on $\A^n$ is trivial on cohomology. Since $\{ 0 \} \embed \A^1$ is $\mu_{q+1}$-equivariant so by the LES all of the cohomology $H^i_c(\A^1 \sm \{ 0 \})$ has the trivial $\mu_{q+1}$-reepresentation. 

\begin{cor}
For $\theta \neq 1$ then $H^1_c(C) [\theta]$ is cuspidal.
\end{cor}

\begin{rmk}
Assume $p > 2$ then there are $q+4$ irreps of $\SL_2(\FF_q)$ we have already found $\frac{q+5}{2}$ of them, missing $\frac{q+3}{2}$ of them. However, there are $q$ nontrivial $\theta$. Notice that,
\[ \frac{q+3}{2} = \frac{q-1}{2} + 2 \]
We claim that $\theta$ and $\theta^{-1}$ give the same irrep and $\theta$ of order two gives two irreps. 
\end{rmk}

\begin{rmk}
The map $\Frob : C \to C$ is $\SL_2(\FF_q)$-equivariant but not $\mu_{q+1}$-equivariant since $(\zeta \cdot x, \zeta \cdot y) \mapsto (\zeta^{q} \cdot x^q, \zeta^{q} \cdots y^q) = (\zeta^{-1} x^q, \zeta^{-1} y^q)$ so $F$ indues an $\mu_{q+1}$-invariant map $\Frob : C \to C'$ where $C'$ is given the inverse $\mu_{q+1}$-representation. Thus $F$ induces and $\SL_2(\FF_q)$-equivariant isomorhism,
\[ H^1_c(C) \to H^1_c(C) \]
which takes $H^1_c(C)[\theta] \iso H^1_c(C)[\theta^{-1}]$. Next, Mackey formula. 
\end{rmk}

\begin{theorem}[Geometric Mackey formula]
Let $\theta_1, \theta_2$ be nontrivial then,
\[ \inner{H^1_c(C)[\theta_1]}{H^1_c(C)[\theta_2]}_{\SL_2} = \inner{\theta_1}{\theta_2}_{\mu_{q+1}} + \inner{\theta_1}{\theta_2^{-1}}_{\mu_{q+1}} \]
\end{theorem}

\begin{prop}
We have isomorphisms as $\SL_2(\FF_q)$-representations,
\[ H^1_c(C)[\theta_1] \cong H^1_c(C) [\theta_1^{-1}] \cong (H^1_c(C)[\theta_1])^\vee \]
\end{prop}

Define $R'(\theta) = H_c^1(C)[\theta] - H^2_c(C)[\theta]$. Then we have, using duality,
\[ \inner{R'(\theta_1)}{R'(\theta_2)} = \inner{1}{R'(\theta_1) \ot R'(\theta_2)} = \dim (H^1_c(C)[\theta_1] \ot H^1_c(C)[\theta_2])^{\SL_2(\FF_q)} \]
Now we write,
\[ H^*_c(X) := \sum_{i = 0}^{2 \dim{X}} (-1)^i H^i_c(X) \]
for the virtual representation. This behaves well with respect to Kunneth. Then consider,
\[ H^*_c(C \times C)[\theta_1 \times \theta_2]^{\SL_2(\FF_q)} \] 
We want to compute,
\[ H^*_c(C \times C / \SL_2(\FF_q)) \]
as a virtual $\mu_{q+1} \times \mu_{q+1}$-represetation. 
\bigskip\\
Let $Z = C \times C \subset \A^4$. Then we decompose $Z = Z_0 \cup Z_{\neq 0}$ where $Z_0$ is cut out by $xw - yz = 0$. 

\begin{lemma}
$Z_0$ is $\mu_{q+1} \times \mu_{q+1} \times \SL_2(\FF_q)$-stable. 
\end{lemma}

\begin{proof}
This is clear for the $\mu_{q+1} \times \mu_{q+1}$. Then, we can compute,
\[ xw - yz \mapsto (ax + by)(cz + dw) - (cx + dy)(az + bw) = xw - yz \]
\end{proof}

\section{Jan 25}

Deligne-Lustzig induction for $\SL_2$ part IV.
\bigskip\\
We have the Drinkfeld curve $C$ and need to prove the Mackey formula for $H^1_c(C)[\theta]$ for $\theta : \mu_{p+1} \to \Qbar_\ell^\times$. Geometrically, this mseans understanding,
\[ H^*_c(C \times C / \SL_2(\FF_q)) \]
as a virtual representation of $\mu_{q+1} \times \mu_{q+1}$. We broke up,
\[ C \times C = Z_0 \cup Z_{\neq 0} \]
into $\SL_2(\FF_Q) \times \mu_{q+1}$ stable parts. We showed last time that,
\[ Z_{\neq 0} / G \sim \{ U^{q+1} - ab = 1 \} \subset \A^1 \sm \{ 0 \} \times \A^2 \] 
with action,
\[ (\zeta_1, \zeta_2) \cdot (u, a, b) = (\zeta_1 \zeta_2 U, \zeta_1 \zeta_2^{-1} a, \zeta_1^{-1} \zeta_2 b) \]
Question is, how to compute $H^*_c(V)$ as a virtual representation. This is equivalent to computing traces,
\[ \Tr{(\zeta_1, \zeta_2) \mid H^*_c(V)} \]
Use the $\Gm$-action $\lambda \cdot (U, a, b) = (U, \lambda^{-1} a, \lambda b)$ and compare to $\Tr{-  \mid H^*_c(V^\Gm)}$.

\begin{prop}
Since $V$ is affine $\exists t \in \Gm(\FFbar_q)$ such that $V^\Gm = V^t$.
\end{prop}

\begin{prop}
If $\gamma$ is a finite-order automorphism of a variety, $\gamma = s u$ with $u$ having $p$-power order and $s$ prime-to-$p$-order and $su = us$. Then,
\[ \Tr{\gamma \mid H^*_c(V)} = \Tr{u \mid H^*_c(V^s)} \]
\end{prop}

\begin{lemma}
Suppose that $\Gamma \times \Gm$ acts on an affine variety $V$. Then,
\[ \Tr{\gamma \mid H^*_c(V)} = \Tr{\gamma \mid H^*_c(V^\Gm)} \]
\end{lemma}

\begin{proof}
Choose $t \in \Gm(\FFbar_q)$ such that $V^t = V^\Gm$. Then for each $\gamma \in \Gamma$ we have,
\[ \Tr{\gamma \mid H^*_c(V^\Gm)} = \Tr{\gamma \mid H^*_c(V^t)} \]
Then write $\gamma = su$ as before. We see that,
\[ \Tr{\gamma \mid H^*_c(V^\Gm)} = \Tr{u \mid (V^t)^s} = \Tr{u \mid (V^s)^t} \]
Then let $g = ut$ and $t$ has prime-to-$p$-order so we get,
\[  \Tr{u \mid (V^s)^t} = \Tr{g \mid V^s} \]
However, $\Gm$ acts trivially on cohomology since $\Gm$ is connected. Therefore, 
\[ \Tr{g \mid V^s} = \Tr{u \mid V^s} = \Tr{us \mid V} = \Tr{\gamma \mid V} \]
\end{proof}

Now we apply this to $V \subset \A^1 \sm \{ 0 \} \times \A^2$ with $\Gm$-action is $\lambda \cdot (U, a, b) = (U, \lambda^{-1} a, \lambda b)$. Then $V^\Gm = \mu_{q+1} \times \{ 0 \} \times \{ 0 \}$ with $\mu_{q+1} \times \mu_{q+1}$ acting via,
\[ (\zeta_1, \zeta_2) \cdot \zeta = \zeta_1 \zeta_2 \zeta \]
Now consider,
\[ Z_0 \subset C \times C \subset \A^4 \]
cut out by the equations,
\begin{align*}
x y^q - y x^q  & = 1
\\
z t^q - t z^q & = 1
\\
x t - y z & = 0
\end{align*}

\begin{lemma}
The map $\mu_{q+1} \times C \to Z_0$ given by,
\[ (\zeta, x, y) \mapsto (x, y, \zeta x, \zeta y) \]
is a $\SL_2(\FF_q)$-equivariant isomorphism. 
\end{lemma}

\begin{proof}
Given $(x,y) \in C$ we want to show there are at most $q+1$ options for $(z,t)$ s.t $(x,y,z,t) \in Z_0(\FFbar_q)$. Write,
\[ t = \frac{y z}{x} \]
and then $z$ to satisfy,
\[ z^{q+1} \left( \frac{y}{x} \right)^{q - 1} z^{q+1} \left( \frac{y}{x} \right) = 1 \]
which has $q+1$ roots. Therefore $\varphi$ is a bijection on $\FFbar_q$-points. We can easily verify smoothness and then conclude. 
\end{proof}

\begin{cor}
$Z_0 / G \cong \mu_{q+1} \times \A^1$ with $\mu_{q+1} \times \mu_{q+1}$ acting via,
\[ (\zeta_1, \zeta_2) \cdot (\zeta, z) = (\zeta_1^{-1} \zeta_2 \zeta, \zeta_1^2 z) \]
\end{cor}

\begin{theorem}[Mackey]
Let $\theta_1, \theta_2$ be nontrival characters of $\mu_{q+1}$. Then the pairing,
\[ \inner{H^1_c(C)[\theta_1]}{H^1_c(C)[\theta_2]}_{\SL_2(\FF_q)} = \inner{\theta_1}{\theta_2}_{\mu_{q+1}} + \inner{\theta_1}{\theta_2^{-1}}_{\mu_{q+1}} \]
\end{theorem}

\begin{proof}
As discussed before,
\[ \inner{-}{-} = \dim H^*_c(C \times C)^{ \SL_2(\FF_q)} [\theta_1 \times \theta_2] \]
We can break this up into,
\[ \dim H^*_c(Z_0)^{\SL_2(\FF_q)}[\theta_1 \times \theta_2] + \dim H^*_c(Z_{\neq 0})^{\SL_2(\FF_q)}[\theta_1 \times \theta_2] \]
which equals,
\[ = \dim H^*_c(Z_0 / \SL_2(\FF_q))[\theta_1 \times \theta_2] + \dim H^*_c(Z_{\neq 0} / \SL_2(\FF_q))[\theta_1 \times \theta_2] \]
which is by our computations,
\[ \Ind{\mu_{q+1} \times \mu_{q+1}}{\mu_{q+1}^{(1)}}{1} [\theta_1 \times \theta_2] + \dim \Ind{\mu_{q+1} \times \mu_{q+1}}{\mu_{q+1}^{(2)}}{1} [\theta_1 \times \theta_2] \] 
where the first is embedded by the diagonal and the second by the anti-diagonal. By Frobenius reciprocity,
\[ = \inner{1}{\theta_1 \ot \theta_2}_{\mu_{q+1}} + \inner{1}{\theta_1 \ot \theta_2^{-1}}_{\mu_{q+1}} \]
\end{proof}

\begin{cor}
$H^1_c(C)[\theta]$ is an irrep of $\dim = q - 1$ if $\theta^2 \neq 1$. Then,
\[ - H^1_c(C) [ \theta_0 ] = (C)_+ + (C)_- \]
is a sum of two irreps. By a counting arugment this has produced all the irreps for $p > 2$. 
\end{cor}

\begin{rmk}
We can also reinterpret parabolic induction in terms of Deligne-Lustzig induction. Indeed, 
\[ H^0_c \left( \frac{\SL_2(\FF_q)}{U(\FF_q)} \right) [\alpha] \]
gives the parabolic induction so we consider $\frac{\SL_2(\FF_q)}{U(\FF_q)}$ a $0$-dimensional variety with $\SL_2(\FF_q) \times \mu_{q-1}$-formula. 
\end{rmk}

\section{Jan 27}

$\GL_3(\FF_q)$ does act on $F$ but since the action extends to $\GL_3$ nothing interesting happens on cohomoloy. Therefore we need a different construction.

\subsection{A general theory of ``relative position''}

Either we choose the condition,
\[ (L_1 \subsetneq L_2, L_1^* \subsetneq L_2^*) \]
We want a relative position map,
\[ F(\FFbar_q) \times F(\FFbar_q) \to Q \]
For $\P^1$ we have,
\[ \P^1(\FFbar_q) \times \P^1(\FFbar_q) \to \{ 0, 1 \} \]
where this measures just if two lines are equal. However, this cannot be made an algebraic map because there are connectivity issues. 
\bigskip\\
A better way is to use the Bruhat decomposition: there are 2 left $B$-orbits on $G/B$. In the above case, there are $2$ left $B$-orbits on $\P^1$. There are also $2$ left $G$-orbits on $\P^1 \times \P^1$ or $G \backslash C^G / G \times G / B = B \backslash G / B$. 
\bigskip\\
In general, for $\GL_3$ want to look at $\GL_3 \backslash F \times F / \GL_3$. 

\begin{exercise}
Let $S_n$ be the symmetric group and conflate $\sigma \in S_n$ with the corresponding permutation matrix in $\GL_n$. Let $B$ be the upper triangular Borel. Then,
\[ \GL_n = \bigsqcup_{\sigma \in S_n} B \sigma B \]
Geometrically, this translates to,
\[ \GL_n / B \times \GL_n / B = \bigsqcup_{\sigma \in S_n} O(s) \]
where $O(s)$ is the $\GL_n$-orbit of $(1, \sigma)$. 
\end{exercise}

We get two decompositions of $\GL_n / B$,
\begin{enumerate}
\item $\GL_n  / B \to \GL_n / B \times \GL_n / B$ given by $x \mapsto (x,1)$ and take inverse image of $O(\sigma)$ to get $(\GL_n/B)^\sigma$ the Schubert varieties. 

\item $\GL_n / B \to \GL_n / B \times \GL_n / B$ given by $x \mapsto (x, F_q x)$ where $F$ is Frobenius. Then pullback $O(\sigma)$ to get $(\GL_n / B)(\sigma)$ the Deligne-Lustzig varities.  
\end{enumerate}
Observe that $X(w) = (\GL_n / B)(w)$ are $\GL_n(\FF_q)$-stable. This is the equivalent of the base $\P^1 \sm \P^1(\FF_q)$ that appeared as the quotient of the Drinfeld curve. We need to build the cover $Y(w)$, the analog of the Drinfeld curve itself. Also we have not seen the torus that should act on the $Y(w)$. We will also construct a torus $T_w$ such that $T_w(\FF_q) \times \GL_n(\FF_q) \acts Y(w)$. 

\begin{prop}
\begin{enumerate}
\item all the $X(w)$ are smooth of pure dim $\ell(w)$ (like the $X^w$ Schubert)

\item the closures have the same singularities as the closure of $X^w$

\item $X(w)$ is usually not connected (like we saw for $\P^1(\FF_q)$ in the case of the split torus)
\end{enumerate}
\end{prop} 

\begin{rmk}
The Deligne-Lustzig varities are related to the Schubert varities via the Lang Isogeny. Indeed,
\begin{center}
\begin{tikzcd}
G \arrow[r, "F(x) x^{-1}"] \arrow[d] & G \arrow[d]
\\
G / B  & G / B
\end{tikzcd}
\end{center}
takes the Schubert variety to the Deligne-Lusztig variety. 
\end{rmk}

\section{Feb. 13}

\newcommand{\TT}{\mathbb{T}}

Define $A_w \to X \times X$ as the moduli space,
\[ (\Spec{R} \to X \times X) \mapsto \{ g \in G_0(R) \mid g B_0 = x \quad g w B_0 = y \} \]
where $x,y$ are the $X$-points defined by $\Spec{R} \to X \times X$. 
\bigskip\\
What is really going is,
\begin{center}
\begin{tikzcd}
G \arrow[d, "\Delta"] \arrow[r] & G / (U \cap w U w^{-1}) \arrow[r, "\pi"] & \struct{}(w) \arrow[d, "1, w"] 
\\
G \times G \arrow[r] & G / U \times G / w U w^{-1} \arrow[r] & G / B \times G / w B w^{-1}
\end{tikzcd}
\end{center}
The upshot is,
\[ Y \times X \xrightarrow{\pi_1} X \times X \]
and also
\[ X \times Y \xrightarrow{\pi_2} X \times X \]
are right $\TT$-torsors are overe $\struct{}(w) \subset R \times R$ there is an isomorphism between them which is $\TT$-equivalrian via the map,
\[ \TT \xrightarrow{\Ad \omega^{-1}} \TT \]
Moreover, over the graph of Frobenius,
\[ \Gamma_F \subset X \times X \]
there is a similar map,
\[ Y \times X |_{\Gamma_F} \to X \times Y |_{\Gamma_F} \]
which is $\TT$-equivariant for,
\[ F : \TT \to \TT \]

\begin{defn}
$\wt{X}(w) \to X(w)$ is the equalizer of $F$ and $\tilde{w}$ inside $Y \times X |_{X(w)}$.
\end{defn}

Let $\TT(w)^F$ be the equalizer of $F, \Ad w^{-1} : \TT \to \TT$> This acts on $\wt{X}(w) \to W(w)$ and in fact this is a $\TT(w)^F$-torsor. One checks that $G_0(\FF_Q)$ acts on $\wt{X}(\wt{w})$ such that $\wt{X}(\wt{w}) \to W(w)$ is equivariant. 

\begin{defn}
For a character $\theta : T(w)^F \to \Qbar_\ell$ we define,
\[ R^\theta(w) = \sum (-1)^i H^i_c(\wt{X}(w), \QQbar_\ell)[\theta] \]
as a virtual $G_0(\FF_q)$-representation.
\end{defn}

\begin{rmk}
$R^1(w) = H^*_c(X(w), \Qbar_\ell)$.
\end{rmk}

This definition is unsatisfactory because $\TT(w)^F$ is not a priori a $T_0(\FF_q)$ for some $T_0 \subset G_0$.

\section{Feb. 15}

Let the notation $(G, \TT_0 \subset B_0, U_0, W)$. Let $T_0 \subset G_0$ be a maximal torus and $B$ a borel $U \subset B$ unipotent radical.

\begin{defn}
\[ S_{T_0, B} = \{g \in G \mid g^{-1} F(g) \in F(U) \} \]
Then we get $L : S_{T_0, B} \to F(U)$ is a $G_0(\FF_q)$-covering and $h \in G_0(\FF_q)$ acts via $h \cdot g = hg$. 
\end{defn}

In the $\SL_2$ case, we get an $\SL_2(\FF_q)$-covering of $\AA^1$ (which should be the Drinfeld curve). Observe that $T_0(\FF_q)$ acts on $S_{T_0, B}$ on the right, by $t \cdot g = gt$ ($T_0$ normalizeses $U$ hence $T_0(\FF_q)$ normalises $F(U)$. So for a character $\theta : T_0(\FF_q) \to \Qbar_\ell^\times$ we can define,
\[ R^\theta_{T_0, B} = \sum_{i} (-1)^i H_c^i(S_{T_0, B}, \Qbar_\ell^\times)[\theta] \]
(considered as a virtual $G_0(\FF_Q)$-representation). Note that $S_{T_0, B}$ has dimension equal to $\dim{U} = \dim{X}$ while $\wt{X}(w)$ had dimensino $\ell(w)$.

\begin{rmk}
$U \cap F(U)$ acts on $S_{T_0, B}$ on the right, equivariant for $G_0(\FF_q) \times T_0(\FF_q)$ then,
\[ \wt{X}_{T_0, B} = S_{T_0, B} / (U \cap F(U)) \]
``does not change cohomology''. 
\end{rmk}

Now $\wt{X}_{T_0, B}$ has dimension $\ell(w)$, where $w = \mathrm{Rel}(B, F B)$ and $FB = \dot{w} B w^{-1}$ for some 

\[ R^\theta_{T_0, B} = \sum_i (-1)^i H^i_c(\hat{X}_{T_0, B}, \Qbar_\ell)[\theta] \]

When $T_0 = \TT_0$ then,
\[ \wt{X}_{\TT_0, B_0} = G_0(\FF_q) / U(\FF_q) \leftarrow S_{\TT_0, B_0} = \A^1 \times (G_0(\FF_q) / U_0(\FF_q)) \]

\begin{example}
When $G_0 = \SL_2$ and $T_0$ is \textit{not} conjugate to $\TT_0$ then $S_{T_0, B} = \wt{X}_{T_0, B}$ is the DRinfeld curve.
\end{example}

Nice definition, but no link to $X(w)$. Hard to show it is independent of the choice of $B$. Let's try to relate it to the construction of $X(w)$. 
\skip
Choose $x \in G(\FFbar_q)$ s.t. $x(\TT_0 \subset B)x^{-1} = (T \subset B)$ then $FB = F(x) B_0 F(x)^{-1}$ and so $x^{-1} F(x) = \dot{w}$. Moreover,
\[ \Ad_x :  \TT_0 \to T \]
is an isomorphism. Then $t \mapsto F(t)$ induces,
\[ t \mapsto x^{-1} F(x t x^{-1}) x = \dot{w} F(t) \dot{w}^{-1} \]
In other words, $T_0$ is just $\TT(w)$. Moreover, the map $g \mapsto g \cdot x$ identifies,
\[ \wt{X}_{T_0, B} = \{ g \in G \mid F(g) \in F(U) \} / (U \cap FU) \]
this maps under $g \mapsto g t$ to,
\[ \{ h \in G \mid (h x^{-1})^{-1} F(h x^{-1}) \in FU \} / (\Ad x^{-1} U \cap \Ad x^{-1} F U) \]
The condition $x h^{-1} F(h) F(x)^{-1} \in F(U) = F(x) U_0 F(x)^{-1}$ equivlaently,
\[ h^{-1} F(h) \in x^{-1} F(x) U_0 = w U_0 \]
Then we ote $\Ad x^{-1} U = U_0$ and,
\[ \Ad x^{-1} FU = x^{-1} F(x) U_0 F(x^{-1}) = \dot{w} U_0 \dot{w}^{-1} = \{ h \in G \mid h^{-1} F(h) \in \dot{w}U_0 \} / (U_0 \cap \dot{w} U_0 \dot{w}^{-1}) \]
Then we get a diagram,
\begin{center}
\begin{tikzcd}
\{ g \in G \mid g^{-1} F(g) \in \dot{w} U \} / (U_0 \cap \Ad \dot{w} U) \arrow[d, hook] \arrow[r] & \{ g \in G \mid g^{-1} F g \in \dot{w} B \} / (B \cap \Ad \dot{w} B) \arrow[d, hook] 
\\
G / (U \cap \Ad \dot{w} U) \arrow[r, "\pi"] \arrow[d] & G / (B \cap \Ad \dot{w} B) \arrow[d]
\\
G / U \times G / \Ad \dot{w} U_0 \arrow[r, "\TT \times \TT"] & G / B \times G / \Ad \dot{w} B
\end{tikzcd}
\end{center}
Claim: this identifies $Z$ with $X(w) \embed \mathcal{O}(w)$ and $\hat{Z} \to Z$ with $\wt{X} \embed \mathcal{O}(w)$ and $\hat{Z} to G / (U \cap \Ad \dot{w} U)$ with $\hat{X}(\dot{w})$ equivariant for the actions of $G_0(\FF_q)$ and $T(w)^F$. Indeed, $\wt{X}(\dot{w}) \subset G / U \cap \Ad{\dot{w}} U$ is cut out by $g \dot{w} U \dot{w} g^{-1} = F(g) U g(g^{-1})$ iff $g^{-1} F(g) \in \dot{w} U$ 

\begin{cor}
$R^\teta_{T_0, B} = R^\theta(w)$ for $w = \mathrm{Rel}(B, FB)$.
\end{cor}

\section{Feb 17}

\begin{theorem}
Independene of $V$ of $R^1_{T_0, B}$ iff $R^1(w) = R^1(w')$ if there exists $w_1 \in W(\FF_q)$ such that $w_1 w F(w_1)^{-1} = w'$ if and only if,
\[ R^1(w) = \sum_i (-1)^i H^i_c(X(w), \Qbar_\ell) =: H^*_c(X(w))  \]
as a $G_0(\FF_q)$ virtual representation.
\end{theorem}


\begin{proof}
WLOG, may assume that $\ell(w_1) = 1$ and that either $\ell(w) = \ell(w')$ or $\ell(w') = \ell(w) + 2$. If $w = w_1 w_2$ and $\ell(w) = \ell(w_1) + \ell(w_2)$ then there is a map,
\[ (X \times X) \times_X (X \times X) \to X \times X \]
sending
\[ ((x,y), (y,z)) \mapsto (x,z) \]
induces a map,
\[ \cO(w_1) \times_X \cO(w_2) \iso \cO(w) \]
Furthermore, if $s \in W(\FF_q)$ has length $1$ then $\cO(s) \times_X \cO(s) \iso \cO(s) \cup \cO(1) = \overline{\cO(s)}$.
Using fact $A$, we get maps $(X(w) \subset X)$
\[ X(w) \rightrightarrows X \]
called $\delta, \gamma$ such that,
$(x, \gamma(x)) \in \cO(s)$ and $(\gamma(x), \delta(x)) \in \cO(w)$ and $(\delta(x), F(x)) \in \cO(F(s))$. We would like to compare $X(w')$ to $X(w)$ so maybe we could hope that $\gamma(x) \in X(w)$ or that $F \gamma(x) = \delta(x)$. This is not true. We saw that,
\[ ((\delta(x), F(x)), (F x, F \gamma(x))) \in \cO(F(s)) \times_X \cO(F(s)) \]
and therefore,
\[ (\delta(x), F(\gamma(x)) \in \overline{\cO(F(s))} = \cO(F(s)) \cup \cO(1) \]
so there is a possibility the equation doesn't hold but we can just pass to the closed subvariety on which it does hold. Therefore, we define,
\[ X_1 = \{ x \in X(w') \mid \delta(x) = F(\gamma(x)) \} \]
and likewise,
\[ X_ = X(w') \sm X_1  = \{ x \in X(w') \mid (\delta(x), F \gamma(x)) \in \cO(F(s)) \} \]
Then there is a map $\gamma : X_1 \to X(w)$ sending $x \mapsto \gamma(x)$ which is an $\A^1$-fibration. Indeed, the map,
\[ \cO(s) \xrightarrow{\pi_2} X \]
is an $\A^1$-fibration so there is an $\A^1$-worth of choices of $x$ such that $\gamma(x) = y$ for fixed $y$ and $(x, y) \in \cO(s)$. This implies that $(Fx, Fy) \in \cO(F(s))$ and thus by uniqueness,
\[ (x,y,F(y), F(x)) = (x, \gamma(x), \delta(x), F(x))) \]
so all these choices of $x$ lie in $X(w')$. This can also be seen from showing that the following diagram,
\begin{center}
\begin{tikzcd}
X_1 \arrow[r, "\gamma"] \arrow[d] \pullback & X(w) \arrow[d]
\\
\cO(s) \arrow[r, "\pi_2"] & X
\end{tikzcd]
\end{center}
is cartesian. Upshot, 
\[ H^*_c(X_1) = H^*_c(X(w)) \]
as virtual $G_0(\FF_q)$-reps. It remains to show that $H^*_c(X_2) = 0$. Recall that $(\delta(x), F(\gamma(x)) \in \cO(F(s))$ 
For $x \in X_2$ ($\ell(F(sw)) = \ell(F s)$. Then by the first fact,
\[ (\delta(x), F(\delta(x)) \in \cO(F(w)) \]
Now we study the map $\delta : X_2 \to X(F(sw))$. Define,
\[ X'_2 \subset X_@ \times X(sw) = \{ (x,y) \mid F(y) = \delta(x) \} \]
we get a Cartesian diagram,
\begin{center}
\begin{tikzcd}
X_2' \arrow[r, "\pi_2"] \arrow[d, "\pi_1"] & X(sw) \arrow[d, "F"]
\\
X_2 \arrow[r, "\delta"] & X(F(sw)) 
\end{tikzcd}
\end{center}
Let $\delta'$ be the map $\pi_2 : X_2' \to X(sw)$. Ckaim that the map $\delta'$ is a $\Gm$-fibration, in fact it is the complement of the zero section of a line bundle which is equivariant for a $G_0(\FF_q)$-action which is fiberwise trivial. $(\delta(x), F x) \in \cO(F(s))$. Once we have the claim, then,
\[ H^*_c(X_2) = 0 \]
as a $G_0(\FF_q)$-representation,
\[ H_c^*(X_2) = H^*_c(X(F9sw))) - H^*_c(X(F(sw))) = 0 \]
\end{proof}

\section{Feb. 22}

\begin{rmk}
I finally realized $G_0$ is the group over $\FF_q$ and $G$ is its base change to $\overline{\FF}_q$.
\end{rmk}

A character formula for $R^\theta_{T_0, B}$ Recall that on Friday we shosed that,
\[ R^1(w) = R^1(w') \text{ if } \exists w_1 : w' = w_1 w F(w_1^{-1}) \]
iff $R^1_{T_0,B}$ does not depend on $B$. Today: we show the same thing for all $\theta$. 

\begin{defn}
For $T_0 \subset G_0$ over $\FF_q$ we define the Green function $Q_{T_0, G_0}$ as the restriction to the unipotent elements in $G_0(\FF_q)$ of the trace function of $R^1_{T_0, B}$ (for some choice of $B$). 
\end{defn}

\begin{theorem}
Let $x = su$ in $G_0(\FF_q)$ with $u$ unipotent and $s$ semisimple. Then,
\[ \Tr{x | R^\theta_{T_0, B}} = \frac{1}{\# Z^0(s)(\FF_q)} \sum_{\substack{ g_0 \in G_0(\FF_q) \\ \Ad_{g_0} T_0 \subset Z^0(s)}} Q_{\Ad g_0 T_0, Z^0(s)}(u) \theta(g_0^{-1} s g_0) \]
\end{theorem}

\begin{rmk}
$u$ commutes with $s$, hence $u \in Z(s)(\FF_q)$ but (Fact) $u \in Z^0(s)$. Since $g_0 T_0 g_0^{-1} \subset Z^0(s)$ we have $s \in g_0 T_0 g_0^{-1}$ hence $g_0^{-1} s g_0 \in T_0(\FF_q)$.
\end{rmk}

\begin{rmk}
This is clearly independent of $B$ but it could be zero.
\end{rmk}

\begin{theorem}
Recall that $G(\FF_q) \acts \wt{X}_{T_0, B} = \{ g \in G \mid g^{-1} F(g) \in F(u) \} / (U \cap F U)$
\end{theorem}

\begin{rmk}
To elaborate, $u$ commutes 


We need the folloing fact.
\end{rmk}

\begin{prop}
Let $I(s) = \pi_0(X^s_{T_0, B})$ then $s$ acts on $\pi^{-1}(y)$ for $y \in X^s_{T_0, B}$. This gives us an element, $t(s, y) \in T_0(\FF_q)$. For all $x \in \pi^{-1}(y)$ we have $s x = x t(s,y)$. Then,
\begin{enumerate}
\item $t(s,y)$ is constant on connected components of $X^s_{T_0, B}$ and,
\[ \tr{su | H_c^*(\wt{X}_{T_0, B})[\theta]} = \sum_{y \in I(s)} \tr{u | H^*_c(X^s_{T_0, B})} \theta(t(s,y)) \]
\end{enumerate}
\end{prop}

Now we prove the theorem using the fact.

\begin{proof}
A borel is $s$ fixed iff $s B_1 s^{-1} = B_1$ iff $s \in B_1$. For such $B_1$ we can always find $h$ s.t. $h B h^{-1} = B_1$ and $h T h^{-1} \subset Z^0(s)$ for $s \in h T h^{-1}$. The set of such $h$ is a $Z^0(s)$ homogeneous space in $G$ (meaning a coset of $Z^0(s)$ which a priori may not be split). Also since $T$ is defined over $\FF_q$ this is defined over $\FF_q$. By Lang, it has a rational point $g_0$ so it is of the form $z \cdot g_0$ for $g_0 \in G_0(\FF_q)$. 
\bigskip\\
Now for $g_0 \in G_0(\FF_q)$ s.t. $g_0 T g_0^{-1} \subset Z^0(s)$ we look at,
\[ X_{\Ad g_0 T_0, \Ad g_0 B \cap Z^0(s)} = 
\begin{cases}
z \in Z^0(s)
\\
z^{-1} F(z) \in 
\end{cases} \]




We get a morphism,
\[ \bigsqcup_{\substack{ g_0 \in G_0(\FF_q) \\ g_0 T_0 g_0^{-1} \subset Z^0(s)}} X_{\Ad_{g_0} T_0, \Ad_{g_0} B \cap Z^0(s)} \to X^s_{T_0, B} \]
Is surjective on $\overline{\FF}_p$-points and we are ``overcounting'' by a factor of $\frac{1}{Z^0(s)(\FF_q)}$. Then we make two claims,
\begin{enumerate}
\item This map is an isomorphism after restricting to one $g_0$ from each $Z^0(s)(\FF_q)$-coset in $G_0(\FF_Q)$ [or after quotienting by $Z^0(s)$]. 

\item The function $t(s, y)$ on the image of $X_{g_0}$ is equal to $g_0^{-1} s g_0 \in T_0(\FF_q)$. 
\end{enumerate}
\end{proof}

\begin{rmk}
These are probably closed immersions but at the level of cohomology is doesn't matter. The only thing that matters is 
\end{rmk}

\section{Feb. 24}

Mackey formula for DL induction.

Let $G_0$ be as before. Our goal is to compute,
\[ \inner{R^\theta_{T_0}}{R^{\theta'}_{T_0'}} \]
Try to show that $R^\theta_{T_0}$ and $R^{\theta'}_{T_0'}$ are disjoint as virtual characters under good assumptions

\subsection{Strategy}

Study the $T_0(\FF_q) \times T_0'(\FF_q)$-action on,
\[ H^i_c((S_{T_0, B} \times S_{T_0', B'})/G_0(\FF_q), \Qbar_\ell) \]
(keep in mind, for $\SL_2$ we broke up this variety into two pieces corresponding to the two terms in the Mackey formula). 
\bigskip\\
For $(B \supset T, B' \supset T')$ with unipotent radicals $U, U'$. Then recall,
\[ S_{T_0, B} = \{ g \in G \mid g^{-1} F(g) \in F U \} \]
same for $T_0'$ and $B'$. Want to write own the quotient,
\begin{center}
\begin{tikzcd}
S_{T_0, B} \arrow[rr] \arrow[rd] &  & \overline{S} \arrow[dl]
\\
& F U \times F U'
\end{tikzcd}
\end{center}
where,
\[ \overline{S} = \{ (x, x', y) \in FU \times F U' \times G \mid x F(y) = y x' \} \]
(MAYBE!!) This is a $G_0(\FF_q)$-torsor over $F U \times F U'$ using the action,
\[ g_0 \cdot (x, x', y) = (f_0 x g_0^{-1}, x', g_0 y) \]
(NOT STABLE UNDER)
We have $S_{T_0, B} \times S_{T_0, B} \xrightarrow{\alpha} \overline{S}$ sending,
\[ (g, g') \mapsto (g^{-1} F(g), g'^{-1} F(g'), g^{-1} g') \]
this is certainly invariant under $(g_0, g_0) \cdot (g, g')$.

\begin{rmk}

\end{rmk}
\end{document}
