\documentclass[12pt]{article}
\usepackage{hyperref}
\hypersetup{
    colorlinks=true,
    linkcolor=blue,
    filecolor=magenta,      
    urlcolor=cyan,
}
 
\urlstyle{same}
\usepackage{import}
\import{../}{AlgGeoCommands}


\begin{document}
\begin{defn}
Let $X$ be a compact irreducible hyperkahler manifold: $X$ a compact KAhler manifold and $\pi_1(X) = 0$ and $H^0(X, \Omega^2_X) = \CC \sigma$ where $\sigma$ is nowhere degenerate.
\end{defn}

\begin{example}
\begin{enumerate}
\item if $\dim{X} = 2$ then $X$ is just a K3 surface
\item $S^{[g]}$ the Hilbert scheme of $g$ points on a K3
\item $M_v(S)$ is the moduli space sof stable sheaves on $S$ with the pairing,
\[ \omega : \Ext{1}{S}{E}{E} \times \Ext{1}{S}{E}{E} \to \CC \]
gives a holomorphic symplectic form on the tangent space
\item If $H$ is an able divisor on $S$ and $\C / |H|$ then $\overline{\mathrm{Pic}^d}_{\C/|H|}$

\item noncompact: $T^* C$ and $M_v(T^* C)$ is the moduli space of Higgs bundles.
\end{enumerate}
\end{example}

\begin{theorem}[Matsushita]
Let $X$ be compact irreducible HK manifold $\dim{X} = 2g$. The only fibrations $X$ can have are $f : X \to B$ with $\sigma$-Lagrangian fibers which are generically Lagrangian $g$-dim abelian varieties. 
\end{theorem} 

\begin{example}
For $S$ is a K3 surface then every fibration is of the form $S \to \P^1$ making $S$ an elliptic surface. Then either,
\begin{enumerate}
\item isotrivial (constant $j$-invariant) for example $\mathrm{Kum}(E_1 \times E_2) \to E_2 / \pm 1 = \P^1$
\item fibers vary in moduli
\end{enumerate}
Then we get $S^{[g]} \to \P^g$ which is fibered in abelian varieties which are products of elliptic curves. In this case either,
\begin{enumerate}
\item $S \to \P^1$ is isotrivial then $S^{[g]} \to \P^g$ is isotrivial
\item $S^{[g]} \to \P^g$ has $g$-dimensional moduli variation (since the fiber is $E_{z_1} \times \cdots \times E_{z_g}$ have moduli varying independently). 
\end{enumerate}
\end{example}

\begin{example}
We also have fibrations,
\begin{enumerate}
\item $\overline{\Pic}^d_{\C / |H|} \to |H|$
\item $M_v(T^* C) \to \bigoplus_{0 < k \le r} H^0(\omega^k_C)$ 
\end{enumerate}
\end{example}

\begin{theorem}
Any Lagrangian fibration of a compact irred. HK manifold is either isotrivial or maximal variation in moduli. 
\end{theorem}

\begin{rmk}
This is not true for $M_v(T^* C) \to \A^1$ since $\CC^\times \times \Pic^0_C$ acts on it. 
\end{rmk}

Now assume that $X$ is compact $f : X \to Y$ a Lagrangian fibration and $f^\circ : X^\circ \to B^\circ$ the smooth locus.

\begin{lemma}
Let $f : X \to B$ be a lagrangian fibration and,
\[ V_{\Q} = R^1 f^\circ_* \Q_{X^\circ}$ then $R_{\RR}$ is irreducible real variation of Hodge structures. 
\end{lemma}

\begin{proof}
For $b \in B^\circ$ then the map $H^2(X, \RR) \to H^2(X_b, \RR)$ has $1$-dim stabilizer under monodormy. Therefore, using Deligne global cycles $H^2(X, \RR) \onto H^0(B^0, R^2 f_*^\circ \RR)$ has rank $1$. But $R^2 f_*^\circ \RR = \wedge^2_ V_{\RR}$ because the fibers are abelian varieties. If $V_{\RR} = W_1 \oplus W_2$ then $h^0(\wedge^2 V_{\RR}) > 1$ arising from the two polarizations on the factors. 
\end{proof}

When is a polarizable $\RR$-variation $V_{\RR}$ irreducible. Clearly if $V$ is an irreducible local system but this is not necessary. 



If $X$ admits purely insep finite map to $\P^n$ then it is unirational. This is just saying its a Zariski surface. 
\end{document}
