\documentclass[12pt]{article}
\usepackage{import}
\import{../}{AlgGeoCommands}

\begin{document}

\section{Interpolation for Brill-Noether Curves}

Question: when can we pass a given type of curve through a given number of points in general position? 

Antiquity: (Eudlid, c. 300 bc),
\begin{enumerate}
\item two points - unique line (1st postulate)
\item three points - unique circe (Book IV, prop 5)
\end{enumerate}

\begin{defn}
A circle is a conic through $[1 : \pm \sqrt{-1} : 0]$ 
\end{defn}

Pappus (340 AD): conic section - through $5$ points.
\\
18th century - invention of algebraic geometry 
\\
Cramer (1750): plane curves of degree $n$ can pass through $\frac{n(n+3)}{2}$ points. 
\\
Waring (1779): graph of degree $n$ polynomial can pass through $n+1$ points (Lagrange interpolation)

\subsection{Modern Era: Brill-Noether Theory}

Way of making precise a ``given type of curve''.

\begin{theorem}[Griffiths, Harris, Eisenbud, Gieseker, Fulton, Lazasfeld]
Let $C$ be a general curve of genus $g$. Then $C$ admits a non-degenerate degree $d$ map to $\P^r$ iff,
\[ \rho(g,r,d) = g - (r+1)(g+r-d) \ge 0 \]
In this case, the \textit{universal family} of such curves is irreducible.
\end{theorem}

\begin{defn}
Call such a curve \textit{BN-curve}. 
\end{defn}


Main Question: let $d,g,n \ge 0$ satisfy $\rho(g,r,d) \ge 0$. Then can we pass a BN-curve of degree $d$ and genus $g$ through $n$ general points $p_1, \dots, p_n \in \P^r$. 

\subsection{Applications}

\begin{enumerate}
\item Maximal rank theorem (L): let $C \subset \P^r$ be a general BN-curve of degree $d$ and genus $g$. Then,
\[ h_C(k) := \min \{ kd + 1 - g \mid { k + r \choose k } \} \]
ideal: through some points on a curve find another curve through those points, take the union and degenerate to this reducible curve

\item smoothing curve singularities (Ian stephans): idea want to find a smooth 

\item Moving curves in $\M_g$: idea arrange for there to be a $1$-dimensional set of curves $C$ through $p_1, \dots, p_n$. 
\end{enumerate}

Equivalent formulation: consider $M^\circ_{g,n}(\P^r, d) \subset M_{g,n}(\P^r, d)$ the Brill-Noether locus. Consider the evaluation map,
\[ \pi : M^\circ_{g, n}(\P^r, d) \to (\P^r)^n \]
When is $\pi$ dominant is the question. 
\bigskip\\
Necessary condition is the dimension count,
\[ r n = \dim{(\P^r)^n} \le \dim{ M^\circ_{g,n}(\P^r, d) } = (r+1) d - (r-e)(g-1) + n \]
This is equivalent to,
\[ (r-1) n \le (r+1) d - (r-3)(g-1) \]

Conjecture 1: this is sufficient. 
\\
Conjecture 1+$\epsilon$: can pass through linear spaces $\{ \Lambda_i \}$ if,
\[ \sum_{i} (r-1 - \dim{\Lambda_i}) \le (r+1) d - (r-3)(g-1) \]

\begin{rmk}
Not true that $M^\circ$ is the largest dimensional component of $M_{g,n}(\P^r, d)$. If we allow the entire moduli space (passing in the question from BN-curves to all smooth curves) then the necessary condition (and the optimal answer) might be larger. 
\end{rmk}

\subsection{Counterexamples}

$(d,g,r) = (6,4,3)$ is a canonical curve: intersection of quadric and cubic. Conjecture says $n \le 12$ but a quadric can only pass through $9$ points. 
\\
$(d,g,r) = (10,6,5)$ is a quadric intersect a quintic del-Pezzo. Conjecture says $n \le 12$ but the del-Pezzo can only pass through $11$ points (by blunt dimension count, moduli of the points in $\P^2$ blown up). 
\\
$(d, g) = (r+2, 2)$ these are hyperelliptic. Using the hyperelliptic involution, the lines between the conjugate points sweep out a scroll. The dimension of the space of such scrolls is $r^2 + 2r - 6$. 

\begin{center}
\begin{tabular}{c|c|c}
$r$ & conj. predicts & dim count for scrolls predicts
\\
$3$ & $n \le 10$ & $n \le 9$
\\
$4$ & $n \le 9$ (and can meet a line) & $n \le 9$ (and can not meet a line: $18$-dim space with $9 \times 2$ conditions for the points no conditions left over for the line)
\\
$5$ & $n \le 10$ & $n \le 9$
\end{tabular}
\end{center}
So for $r = 3,5$ we get counterexamples to conjecture $1$ and $r = 4$ is a counterexample to conjecture $1+\epsilon$.
\bigskip\\
Conjecture $1 + \epsilon - \delta$: true except in these five cases.

\subsection{Prior Work}

Sacchiew 1980 + Ran 2007 $\implies g = 0$
\\
Stevens (1989, 1996) $\implies$ canonical curves $(d,g,r) = (2g - 2, g, g - 1)$
\\
Perrin Vogt (2018) $r = 3$ 
\\
Atarevsov, L, Yang (2019) $\implies d \ge g + r$
\\
L, Vogt (2021) $\implies r = 4$

\begin{theorem}[V, Vogt]
Conjecture $1 + \epsilon - \delta$ is true in any characteristic. 
\end{theorem}

In positive characteristic we have annother question: when is $\pi$ generically smooth. 
\[ \pi \text{ generically smooth} \iff H^1(N_C(-p_1 - \cdots - p_n)) = 0 \]
and the necessary condition is equivalent to,
\[ \chi(N_C(-p_1 - \cdots - p_n)) \ge 0 \]

\begin{defn}
A vector bundle $\E \to C$ \textit{satisfies interpolation} if for $D \subset C$ a general effective divisor of any degree,
\[ H^0(\E(-D)) = 0 \text{ or } H^1(\E(-D)) = 0 \]
\end{defn}

\begin{rmk}
This is a good property since,
\begin{enumerate}
\item it means if euler characteristic is $\ge 0$ then $H^1 = 0$

\item this inducts well

\item it will help with linear spaces as well as points with this
\end{enumerate}
\end{rmk}

Conjecture 2: $N_C$ satisfies interpolation except in the $5$ cases.

\begin{rmk}
Conj 2 implies Conj $1+\epsilon-\delta$. If we weakened interpolation to: euler char $\ge 0$ implies $H^1 = 0$ then this would just give Conj $1 - \delta$.
\end{rmk} 

\subsection{Characteristic $2$}

However! Conj $2$ is false in char = 2.

\begin{center}
\begin{tikzcd}
0 \arrow[r] & N_C^\vee(1) \arrow[r]  & \struct{C}^{r+1} \arrow[r] \arrow[d, equals] & J^1(\struct{C}(1)) \arrow[r] \arrow[d, equals] & 0
\\
& F^* \struct{C'}^{\oplus r+1} \arrow[r] & F^* F_* \struct{C}(1) 
\end{tikzcd}
\end{center}
If $g = 0$ then,
\[ N_C^\vee(1) = \bigoplus \struct{}(a_i) \]
with all $a_i$ even since Frob has degree $2$. Interpolation on $\P^1$ is basically the same as being ballanced. Thus $N_C$ can only satisgy interpolation if $\mu(N^\vee_C(1)) \in 2 \Z$ iff $d \equiv 1 \mod (r-1)$. 

\begin{theorem}[L, Vogt]
For $C$ a general BN-curve: $N_C$ satisfies interpolation unless,
\begin{enumerate}
\item $(d,g,r) = \{ (5,2,3), (6,2,4), (7,2,5), (6,4,3), (10,6,5) \}$

\item $\mathrm{char}(k) = 2$ and $g = 0$ and $\d \neq\equiv 1 \mod (r-1)$.
\end{enumerate}
\end{theorem} 


\begin{rmk}
Complete intersections in $\P^3$ of degree $d$ has much larger size than BN-component. 
\end{rmk}

\end{document}
