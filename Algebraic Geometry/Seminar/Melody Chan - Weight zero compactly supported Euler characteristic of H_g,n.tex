\documentclass[12pt]{article}
\usepackage{import}
\import{../}{AlgGeoCommands}

\begin{document}

\renewcommand{\H}{\mathcal{H}}

Work over $\CC$ and let $g \ge 2$. Let $\H_{g,n} \subset \M_{g,n}$ be the space of hyperelliptic curves with $n$ marked points. The pointed are labeled and may or may not lie in hyperelliptic pairs or at the intersections. 

\[ \chi^{S_n}_{0,c}(\H_{g,n}) = \sum_{i \ge 0} (-1)^i W_0 H^i_c(\H_{g,n}, \Q) \in K_0(\Q[S_n]) \]
for the mixed Hodge structure weight filtration. We view this in the Grothendieck group of $S_n$-representations as a virtual representation or character.
\bigskip\\
The main theorem (BCK 2023+) gives a sum-of-graphs formula,
\[ z_g = \sum_{n \ge 0} \ch_n(\chi^{S_n}_{0,c}(\H_{g,n})) \in \hat{\Lambda}_{\QQ} = \ilim \QQ \dbrac{x_1, \dots, x_n}^{S_n} \]
in the degree-completed ring of symmetric functions where,
\[ \ch_n \left( \sum_{\lambda \proves n} a_\lambda S^\lambda \right) = \sum_{\lambda \proves n} a_\lambda s_\lambda \in \Lambda_n \]
where $s_\lambda$ is the Shur function,
\[ s_\lambda = \sum_{T \in SSYT(\lambda)} x_1^{c_1(T)} x_2^{c_2(T)} \cdots \]
these are semistandard fillings of the Young tableux. Or equivalently,
\[ \ch_n(W) = \frac{1}{n!} \sum_{\sigma \in S_n} \tr{\sigma|W} \cdot \psi(\sigma) \]
where,
\[ \psi(\sigma) = p_{a_1} \cdots p_{a_k} \]
where $\sigma$ has cycle type $a_1 \ge a_2 \ge \cdots \ge a_k$ and,
\[ p_i = x_1^i + x_2^i + \cdots \]
Sample theorem $g = 2$ (Faber, '08),
\[ z_2 = - \frac{1}{12} P_1 + \tfrac{1}{2} \frac{P_1}{P_2} - \frac{1}{6} \frac{P_1^2}{P_3} - \frac{1}{12} \frac{P_1}^3}{P_2^2} - \frac{1}{6} \frac{P_2 P_3}{P_6} \]
where $P_i = 1 + p_i$. 
\bigskip\\
Related work:
\begin{enumerate}
\item $\chi(\H_{g,n})$ and $\chi(M_{g,n})$ Bini '07 Bini-Harer 11, $\chi^{\orb}(\M_{g,n})$ Harer-Zagier '86. 

\item $\chi^{S_n}(\H_{g,n})$ and $\chi^{S_n}(M_{g,n})$ Gorsky '09, '14.
\end{enumerate}

\subsection{Boundary Complexes}

Let $U \subset X$ be a nc compactification of smooth varieties. Then there is a combinatorial object $\Delta(U \subset X)$ called the boundary complex. If $D = D_1 \cup \cdots \cu[ D_t$ with $D = X \sm U$ with each $D_i$ smooth (snc case) is a simplicial complex whose $p$-simplices are irreducible components of $D_{i_0} \cap \cdots \cap D_{i_p}$ for $i_0, \dots, i_p$. 

\begin{example}
$U = \Gm^2$. Then we can consider $\P^2$ or $\P^1 \times \P^1$ are toric compactifications. We could also blow up $\P^2$ at not a torus fixed point to get a non-toric example. These give a triangle, a square, and a triangle with a tail respectively.
\end{example}

It turns out that the homotopy type of this complex is an invariant of $U$. From Deligne Hodge II,III we get,
\begin{theorem}
$\wt{H}^{k-1}(\Delta(U \subset X), \Q) \cong \gr^W_0 H^k_c(U, \Q)$.
\end{theorem}

\subsection{Admissible Covers (Harris - Mumford)}

Abramovich-Corti-Vistoli '03. Let $G$ be a group. Then, the moduli space of admissible $G$-covers,
\[ \Adm^G_{g,n} \]
parametrizes some compactification of $G$-covers of smooth curves. Consider $G \acts P$ where $P$ is a nodal curve and $(C, p_1, \dots, p_n) \in \Mbar_{g,n}$ and there is an equivariant map $\pi : G \to P$ which is a torsor except possibly over the markings and nodes and nodes have to map to nodes. And the $G_x$-action at $x \in P^{\text{sing}}$ is \textit{balanced} (the eigenvalues are inverses). This stack is roughly $\Mbar_{g,n}^{\text{bal}}(BG, 0)$ of stacky curves mapping to $BG$. Then there is $\Mbar^G_{g,n}$ moduli of pointed admissible curves where we have marked points lifing the marked points downstairs. This comes along with an operadic structure corresponding to composiing the sets with $G$-actions. Consider the open substack of smooth curves,
\[ \M^G_{g,n} \subset \Mbar^G_{g,n} \]
is a nc compactification of stacks. 

\subsection{Hyperelliptic curves}

Let $G = \Z / 2 \Z = \{ 0, 1 \}$. Pass to $\wt{\H}_{g,n}$ moduli of hyperelliptic curves with labeled (i.e. ordered) Weirestrass points. Let $\wt{\H}_{g,n}^\circ$ be the locus where all the Weierstrass points and marked points have distinct images in $\P^1$. 

\end{document}
