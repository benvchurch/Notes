\documentclass[12pt]{article}
\usepackage{import}
\import{../}{AlgGeoCommands}

\begin{document}

\section{Introduction}

\begin{theorem}[Faltings]
If $C$ is a curve over a number field $K$ of genus $g \ge 2$ and $L/K$ is finite then $C(L)$ is a finite set.
\end{theorem}

We ask if there are infinitely points of a fixed degree.

\begin{defn}
If $P \in C(\overline{K})$ then $\deg{P} = [\kappa(P) : K]$. 
\end{defn}

\begin{theorem}[Bhargava-Gross]
There exist curves $C$ and integers $n > 1$ such that $C$ has infinitely many algebraic points of degree $n$.
\end{theorem}

\begin{rmk}
However, these points arrise in different fields of the same degree so it does not contradict Falting's theorem.
\end{rmk}

\begin{defn}
\[ N_n(X) = \# \{ L/K \mid [L : K] = n \text{ and } | \mathrm{Disc}(L/K) | \le x \} \]
We say that,
\[ N_{n, L/K}(X, G) = \# \{ K(P) / K \} \]
\end{defn}


Let $E$ be an elliptic curve over $\Q$ then Lemke Oliver-Thorne:
\[ N_{n,E/\Q}(X, S_n) \gg x^{c_n - \epsilon} \]
where $c_n \to \frac{1}{4}$ from below as $n \to \infty$. 
\bigskip\\
Keyes showed that if $C$ is a hyperelliptic curve then,
\[ N_{n, C/\Q}(S, S_n) \gg x^{c_n} \]
as $c_n \to \frac{1}{4}$ as $n \to \infty$.

\subsection{Superelliptic Curves}

Fix $m \ge 2$, a superelliptic curve $C/\Q$ of exponent $m$ is given by,
\[ V(y^m - f(x)) \]
where,
\[ f(x) = \sum_{i = 0}^d c_i x^i \]
is a square-free polynomial of degree $d$.

\begin{thm}
Let $C$ be a superelliptic curve of exponent $m$ and suppose that $(m, d) \divides n$ and $n$ satisfies,
\[ n \ge \max \{ d, \mathrm{lcm}(m,d) - m + d + 1, 2 m^2 - m \} \]
Then,
\[ N_{n,C}(X, S_n) \gg x^{c_n} \]
for $c_n \to \frac{1}{m^2}$ as $n \to \infty$.
\end{defn}

\end{thm}
\end{document}
