\documentclass[12pt]{article}
\usepackage{import}
\import{../}{AlgGeoCommands}

\begin{document}

\section{The Embedding Theorem in Hurwitz-Brill-Noether Theory}

joint with Kaelin Cook-Powell, Dave Jensen, Eric Larson, and Isabel Vogt. 

\begin{defn}
$W^r_d(C) = \{ L \in \mathrm{Pic}^d(C) \mid h^0(C, L) \he r + 1 \}$ line bundles with enough sections to give nondegenerate map to $\P^r$ of degree $d$. These fit together to a universal Brill-Noether stack,
\[ \W^r_d \to \M_g \]
\end{defn}

\begin{defn}
Let $\rho(g,r,d) = g - (r+1)(g - d + r)$. 
\end{defn}

\begin{theorem}[Griffiths-Harris, Fulton-Lazarsfeld, Gieseker, Eisenbud-Harris[
If $\rho(g,r,d) \ge 0$ then $\W^r_d$ has a unique irreducible component dominating $\M_g$ whose general fiber has dimension $\rho(g,r,d)$. Call this distinguished component $\W^r_{d, \BN}$.
\end{theorem}

\begin{rmk}
When we say a ``general map'' to $\P^r$ we mean a general line bundle in $\W^r_{d, \BN}$. 
\end{rmk}

\begin{theorem}[Eisenbud-Harris]
Assume $\rho \ge 0$. If $r \ge 3$ then a general degree $d$ map $C \to \P^r$ is an embedding. 
\end{theorem}

\begin{rmk}
Also necessary away from,
\begin{enumerate}
\item $(g,r,d,) \in \{ (0,1,1), (0,2,2), (1,2,3), (3,2,4) \}$
\end{enumerate}
\end{rmk}

\begin{defn}
We say that $\L$ is $p$-very ample if for all divisors $D \subset C$ of degree $p+1$,
\[ h^0(L(-D)) = h^0(L) - (p+1) \]
\end{defn}

\begin{rmk}
\begin{enumerate}
\item base-point free $\iff$ $0$-very ample
\item very ample $\iff$ $1$-very ample.
\end{enumerate}
\end{rmk}

\begin{rmk}
Geometrically, this means the span of the image of $D$ is dimension $p$.
\end{rmk}

\begin{theorem}[Farkas]
A general $L \in \W^r_{d, \BN}$ is $p$-very ample if $r \ge 2p + 1$.
\end{theorem}

We want to extend this to Hurwitz Brill-Noether Theory,
\[ \W^{\vec{e}} \to \H_{k,g} \]
where $\H_{k,g}$ parametrizes degree $k$ genus $g$ covers $C \to \P^1$. 

\subsection{Hurwitz-Brill-Noether Theory}

If $f : C \to \P^1$ has degree $k$. Then $W^r_d(C)$ is often reducible (meaning for varying over curves of fixed gonality I guess). But we can refine the space as follows. If $L$ is a line bundle on $C$, then $f_* L$ is a rank $k$ vector bundle so we can decompose it,
\[ f_* L = \struct{}(e_1) \oplus \cdots \oplus \struct{}(e_k) \]
with $e_1 \le \cdots \le e_k$. 

\begin{defn}
We say that $\vec{e}'$ specializes $\vec{e}$ if,
\[ e_1' + \cdot + e_j' \le e_1 + \cdots + e_j \]
for all $j$. Then,
\[ W^{\vec{e}}(C) = \{ L \in \Pic{C} \mid f_* L \cong \struct{}(\vec{e}) \text{ or a specialization } \} \]
\end{defn}

\begin{prop}
We see,
\[ W^r_d(C) = \bigcup_{\substack{ h^0(\struct{}(\vec{e}) \ge r + 1 // e_1 + \cdots + e_k = d - g + 1 - k}} W^{\vec{e}}(C) \]
\end{prop}

\begin{defn}
$\rho'(g, \vec{e}) = g - \sum_{i,j} \max \{ 0, e_i \cdots e_j - 1 \}$.
\end{defn}

\begin{rmk}
Given $\vec{e}$ we recover the pair $(r,d)$ as,
\[ r = h^0(\struct{}(\vec{e})) - 1 \quad \text{and} \quad d = e_1 + \cdots + e_k + g - 1 + k \]
\end{rmk}

\begin{theorem}
If $\rho'(g, \vec{e}) \ge 0$ then $\W^{\vec{e}}$ has a unique irreducible component dominating $\H_{k,g}$. Call it $\W^{\vec{e}}_{\BN}$.
\end{theorem}

\begin{defn}
A \textit{general map} of type $\vec{e}$ meaning the map defined by a general point $L \in \W^{\vec{e}}_{\BN}$.
\end{defn}

\begin{rmk}
Note, $r \ge 3$ is NOT sufficient to gaurantee a general $L \in \W^{\vec{e}}_{\BN}$ defines an embedding. The reason: maps of type $\vec{e}$ naturally factor through certain scrolls. Let $E = f_* L$. Then, $f^* E = f^* f_* L \to L$. Dualizing gives,
\[ L^\vee \embed f^* E^\vee \]
This gives an embedding,
\begin{center}
\begin{tikzcd}
C \arrow[r, hook] \arrow[rd, "f"] & \P(E^\vee) \arrow[d]
\\
& \P^1
\end{tikzcd}
\end{center}
such that $\struct{\P(E^\vee)}(1)|_C = L$. Then $\struct{\P(E^\vee)}(1)$ defines a rational map (defined everywhere except the negative subbundle) so $C \to \P^r$ factors through projection $\P(E^\vee) \rat \P(E^\vee/E_{-}) \to \P^r$. 
\end{rmk}

\begin{example}
$g = 5$ and $k = 3$ and $\vec{e} = (-2,1,1)$. Then,
\[ r = h^0(\struct{}(\vec{e})) - 1 = 4 - 1 = 3 \]
and
\[ d = g - 1 + k = 7 \]
Then we get,
\[ C \embed \P(E^\vee) \rat \P(\struct{}(1) \oplus \struct{}(1)) \cong \P^1 \times \P^1 \embed \P^3 \]
Total degree $7$ and the second map has degree $2$ so $\im{C} \subset \P^1 \times \P^1$ has bidegree $(3,4)$ and thus $\im{C}$ has arithmetic genus $6$ so we must have produced an node. Thus, this is not an embeding. 
\end{example}

\begin{rmk}
Therefore, you would expect the condition is that the projection to the nonnegative scroll has ``enough room'' i.e. is dimension at least $3$. 
\end{rmk}

\begin{theorem}
Assume $\rho'(g, \vec{e}) \ge 0$. If $e_{k-2} \ge 0$ and $r = h^0(\struct{}(\vec{e}) - 1 \ge 3$ then a general map of type $\vec{e}$ is an embedding.
\end{theorem}

\begin{rmk}
This is again almost necessary. They write down a complete list of other examples where the generic member is an embedding. 
\end{rmk}

\begin{defn}
We say $L$ is \textit{relatively $p$-very ample} if for all effective divisors $D \subset C$ contained in a fiber of $f : C \to \P^1$, we have,
\[ h^0(L(-D)) = h^0(L) - (p+1) \]
\end{defn}

\begin{rmk}
This also seems to be necessary outside of a finite list of examples.
\end{rmk}

\begin{theorem}
If $e_{k-p-1} \ge 0$ then a general $L \in \W_{\BN}^{\vec{e}}$ is relatively $p$-very ample. 
\end{theorem}

\begin{cor}
Suppose,
\begin{align*}
e_{k-1} & \ge p
\\
e_{k-2} & \ge p-1
\\
& \vdots
\\
e_{k-p-1} & \ge 0
\end{align*}
then a general $L \in \W^{\vec{e}}_{\BN}$ is $p$-very ample. 
\end{cor}

\begin{rmk}
However, here it does not seem to be a necessary condition.
\end{rmk}

\begin{conj}
If $e_{k-1} \ge 0$ and $h^0(\struct{}(\vec{e}) \ge 2 p + 2$ then a general $L \in \W^{\vec{e}}_{\BN}$ is $p$-very ample. 
\end{conj}

\begin{proof}[Proof Technique]
Embedded degeneration of $C \subset \P(E^\vee)$. 
\end{proof}


\end{document}