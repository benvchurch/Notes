\documentclass[12pt]{article}
\usepackage{import}
\import{../}{AlgGeoCommands}

\begin{document}

\section{Moduli Spaces of Cubic Threefolds}

We can define such a moduli space as follows,
\[ M = \P (H^0(\struct{\P^4}(3)))^{\text{smooth}} // \SL{5}{\C} \]
Then $\dim{M} = 35 - 1 - 24 = 10$. This is quasi-projective but not projective. We can find a projective model as follows,
\[ M^{\text{SIT}} = \P(H^0(\struct{\P^4}(3))^{\text{SS}} // \SL{5}{\C} \]
which has finite quotient singularities. Now,
\[ M^{\text{SIT}} = M^K \setminus K \]
where $K$ is the $K$-stabilit

\subsection{Period Maps}

Jacobians,
\[ IJ(x) = H^{2,1}(X)^* / H_3(X, \Z) \in A_5 \]
Clemes - Griffith: $X$ is not rational. Then we get a map $M \to A_f$. 

\begin{theorem}[Clemens-Griffiths, Mumford]
Torelli holds and $M \embed A_5$. 
\end{theorem}

There is another period map associated to the Boll quooteint via Alloch, Toledo. Suppose we have a cubic threefold,
\[ X = \{ f_3(x_0, \dots, x_4) = 0 \} \subset \P^4 \]
Then we associate to it a cubic fourfold,
\[ Z = \{ z^3 - f_3 = 0 \} \subset \P^5 \]
If $X$ is smooth then $Z$ is also smooth. There is a 3 to 1 map $Z \to \P^4$ birational along $X$. Then the Fano variety of lines,
\[ F(Z) = \{ \ell \subset \P^5 \mid \ell \subset Z \} \]
Then $\omega_{F(Z)} = \struct{F(Z)}$. Then $F(Z)$ is a smooth four-fold and a holomorphic symplectic manifold i.e. a hyperKahler manifold. The Hodge theory on hyper-Kahlers alows $\omega = \omega (F(H)) \in \Omega$ a period point 
where $\Omega$ is period domain $\dim{\Omega} = 20$. Furthermore, $Z$ has $\Z / 32 \Z$-action so the period point lie $\omega \in B^{10}$ where $B^{10}$ is a $10$-dimensional ball. Then $M \embed B^{10} / \Gamma$ where $\Gamma$ is arithmetic group. 

\begin{rmk}
We have $B^{10} / \Gamma - M \cong B^g / \Gamma_h$ which is the moduli space of hyperelliptic curves of genus $5$. 
\end{rmk}
\end{document}
