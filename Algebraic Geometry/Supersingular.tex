\documentclass[12pt]{article}
\usepackage{import}
\import{./}{AlgGeoCommands}
\renewcommand{\U}{\mathfrak{U}}

\begin{document}

\section{Kodaira Dimension}

\begin{defn}
Let $X$ be a smooth projective variety over $k$ with canonical bundle $\omega_X$. Then we define the \textit{plurigenera} of $X$ to be,
\[ p_n(X) = \dim_k H^0(X, \omega_X^{\otimes n}) \]
Furthermore, we define the \textit{Kodaira} dimension $\kappa(X)$ as the minimal integer $d$ such that the plurigenera satisfy $p_n(X) \in O(n^d)$ and $\kappa(X) = - \infty$ if $p_n(X) = 0$ for all $n > 0$.
\end{defn}

\begin{defn}
We say that a variety is \textit{of general type} if $\kappa(X) = \dim{X}$.
\end{defn}

\begin{prop}
For smooth projective curves $X$ over $k$ of genus $g$ we have,
\[ p_\ell(X) = 
\begin{cases}
0 & g = 0
\\
1 & g = 1
\\
g & \ell = 1
\\
(2 \ell - 1)(g - 1) & g \ge 2, \ell > 1
\end{cases} \]
and therefore,
\[ \kappa(X) = 
\begin{cases}
- \infty & g = 0
\\
0 & g = 1
\\
1 & g \ge 2
\end{cases} \]
\end{prop}

\begin{proof}
For $g = 0$ we know $\deg{\omega_X^{\otimes \ell}} = - \ell < 0$ so $H^0(X, \omega_X^{\otimes \ell}) = 0$ and thus $\kappa(X) = -\infty$. For $g = 1$ we know $\omega_X = \struct{X}$ and thus $p_\ell(X) = 1$ for all $\ell$ so $\kappa(X) = 0$.
\bigskip\\
Now consider $g \ge 2$. For any $\L \in \Pic{X}$ we know that $H^0(X, \L) = 0$ if $\deg{\L} < 0$ so if $\deg{\L} > 2g - 2$ then $H^0(X, \omega_X \otimes_{\struct{X}} \L^\vee) = 0$ and thus by Riemann-Roch,
\[ \dim_k H^0(X, \L) = \deg{\L} + 1 - g \]
In particular, for $\L = \omega_X^{\otimes \ell}$ we have $\deg{\omega_X^{\otimes \ell}} = (2 g - 2) \ell$ so for $g \ge 2$ and $\ell > 1$ we have,
\[  p_\ell(X) = \dim_k H^0(X, \omega_X^{\otimes \ell}) = (2 g - 2) \ell + (1 - g) = (2 \ell - 1)(g - 1) \]
Also, for $\ell = 1$ we get $H^0(X, \omega_X) = g$. Therefore, $p_\ell(X) \sim \ell$ so $\kappa(X) = 1$. 
\end{proof}

\begin{rmk}
In particular, a curve is general type iff $g \ge 2$. 
\end{rmk}

\begin{prop}
Let $X \subset \P^n_k$ be a smooth hypersurface of degree $d$. Then,
\[ \kappa(X) =
\begin{cases}
-\infty & d < n + 1
\\
0 & d = n + 1
\\
n - 1 & d > n + 1
\end{cases} \]
Therefore, $X$ is of general type iff $d > n + 1$. 
\end{prop}

\begin{proof}
Let $X \subset \P^n_k$ be a smooth hypersurface of degree $d$. Then $\omega_X = \struct{X}(d - n - 1)$. Thus,
\[ \omega_X^{\otimes \ell} = \struct{X}((d - n - 1) \ell)  \]
There is an exact sequence,
\begin{center}
\begin{tikzcd}
0 \arrow[r] & \struct{\P}(-d) \arrow[r] & \struct{\P} \arrow[r] & \struct{X} \arrow[r] & 0 
\end{tikzcd}
\end{center}
then twisting we get,
\begin{center}
\begin{tikzcd}
0 \arrow[r] & \struct{\P}(d(\ell - 1) - \ell(n + 1)) \arrow[r] & \struct{\P}((d - n - 1) \ell)) \arrow[r] & \omega_X^{\otimes \ell} \arrow[r] & 0
\end{tikzcd}
\end{center}
Then we can compute,
\[ p_\ell(X) = H^0(X, \omega_X^{\otimes \ell})  \]
from the long exact sequence,
\begin{center}
\begin{tikzcd}
H^0(\P^n_k, \struct{\P}(a)) \arrow[r] & H^0(\P^n_k, \struct{\P}(b)) \arrow[r] & H^0(X, \omega_X^{\otimes \ell}) \arrow[r] & H^1(X, \struct{\P}(a)) 
\end{tikzcd}
\end{center}
where $a = d(\ell - 1) - (n + 1)\ell$ and $b = (d - n - 1) \ell$. For the case $n > 2$ we get vanishing of $H^1$ for line bundles in general. In the case $n = 2$ we can apply our results for curves to conclude. Now,
\[ p_\ell(X) = h^0(\P^n, \struct{\P}(b)) - h^0(\P^n, \struct{\P}(a)) = { (d - n - 1) \ell + n  \choose n } - { (d - n - 1) \ell + n - d \choose n } \]
Therefore, we have three cases depending on the sign of $d - n - 1$. If $d = n + 1$ then $\omega_X \cong \struct{X}$ is the trivial bundle so $p_\ell(X) = 1$ and thus $\kappa(X) = 0$. If $d < n + 1$ then $p_\ell(X) = 0$ for $\ell > 0$ and thus $\kappa(X) = - \infty$. If $d > n + 1$ then $p_\ell(X)$ is a degree $n-1$ polynomial in $\ell$ so $\kappa(X) = \dim{X} = n - 1$.
\end{proof}

\section{Weighted Projective Spaces}

\begin{defn}
For an $(r+2)$-tuple $(q_0, \dots, q_{r+1})$ we define the \textit{weighted projective space} over $k$,
\[ \P_k(q_0, \dots, q_{r+1}) = \Proj{k[x_0, \dots, x_{r+1}]} \]
where the ring $R = k[x_0, \dots, x_{r+1}]$ is graded with $\deg{x_i} = q_i$. Clearly, $\P_k(d q_0, \dots, q_{r+1}) \cong \P_k(q_0, \dots, q_{r+1})$ by scaling degrees. Thus we may assume that the ideal $(q_0, \dots, q_{r+1}) = \Z$.
\end{defn}

\subsection{Line Bundles and Divisors}

\subsection{Toric Construction}

\begin{prop}
The weighted projective space $\P_k(q_0, \dots, q_{r+1})$ is a toric variety with torus,
\[ D_+(x_0 \cdots x_{r+1}) = \Spec{k[y_1^{\pm 1}, \dots, y_{r+1}^{\pm 1}]} \]
where $y_i$ is a monomial in $x_j^{\pm 1}$. 
\end{prop}

\begin{proof}
We know that,
\[ D_+(x_0 \cdots x_{r+1}) = \Spec{k[x_0, \dots, x_{r+1}] \left[ \frac{1}{x_0 \cdots x_{r+1}} \right]_{0}} \]
Then monomials in, 
\[ \tilde{R} = k[x_0, \dots, x_{r+1}] \left[ \frac{1}{x_0 \cdots x_{r+1}} \right]_{0} \]
are of the form,
\[ y = \prod_{i = 0}^{r+1} x_i^{a_i} \]
for some $(r+2)$-tuple $a_i \in \Z$ such that,
\[ \deg{y} = \sum_{i = 0}^{r + 1} q_i a_i = 0 \]
Thus, the allowed monomials are given by $(r+2)$-tuples in $K \subset \Z^{r+2}$ the kernel of $\Z^{r+2} \to \Z$. There is an exact sequence,
\begin{center}
\begin{tikzcd}
0 \arrow[r] & K \arrow[r, hook] & \Z^{r + 2} \arrow[r, two heads] & \Z \arrow[r] & 0
\end{tikzcd}
\end{center}
where the map $\Z^{r+2} \onto \Z$ via $(q_i) \mapsto \sum q_i a_i$ is surjective since the ideal $(q_0, \dots, q_{r+1}) = \Z$. Since $\Z$ is projective, this sequence splits so,
\[ \Z^{r + 2} = K \oplus \Z \]
and thus $K$ is free of rank $r + 1$ which implies that there are generators $y_i$ for $1 \le i \le r + 1$ such that $\tilde{R} = k[y_1^{\pm 1}, \dots, y_{r+1}^{\pm 1}]$.
\end{proof}
\noindent\\
Now we consider how weighted projective space may be constructed from combinatorial toric fan or polytope data. 

\subsection{Weighted Hypersurfaces}

\end{document}