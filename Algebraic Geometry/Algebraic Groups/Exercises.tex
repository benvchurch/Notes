\documentclass[12pt]{article}
\usepackage{import}
\import{../}{AlgGeoCommands}


\begin{document}

\section{Homework 1}

\subsection{1}

\subsubsection{1}

Maps $\Hom{k}{\Ga}{\Ga} = \Hom{k}{\Spec{k[t]}}{\Spec{k[t]}} = \Hom{k}{k[t]}{k[t]} = k[t]$. Therefore we need $f \in k[t]$ which are cogroup maps for $k[t] \to k[t] \otimes_k k[t]$ meaning that $f(x + y) = f(x) + f(y)$. 
\bigskip\\
Consider $\Hom{k}{\Gm}{\Gm} = \Hom{k}{\Spec{k[t]}}{\Spec{k[t]}} = \Hom{k}{k[t,t^{-1}]}{k[t,t^{-1}]} = (k[t, t^{-1}])^\times$ such that $f(xy) = f(x) f(y)$.

\subsubsection{2}

If $k$ is a $\Q$-algebra then any $f \in k[t]$ satisfying $f(x + y) = f(x) + f(y)$ must be linear with zero constant term and thus $f = a t$ for $a \in k$ so we get $\End{k}{\Ga} \cong k$. 
\bigskip\\
If $k$ is a field of characteristic $p > 0$ then if $f(x+y) = f(x) + f(y)$ we must have that $f(ax) = a f(x)$ for each $a \in \FF_p$ which implies that,
\[ f(t) = \sum c_j t^{p^j} \]
Suppose that $k = \Z / (p^2)$. (DO THIS CASE!!)

\subsubsection{3}

If $k$ is a field then $(k[t, t^{-1}])^\times$ consists of elements of the form $f(t) = a t^n$ and if $f(xy) = f(x) f(y)$ then $a = 1$ so $\End{\Gm} \cong \Z$.
\bigskip\\
Now suppose that $A$ is an Artinian local ring and $\kappa = A / \m$ its residue field. It suffices to prove that every $f \in (A[t, t^{-1}])^\times$ such that $f(xy) = f(x) f(y)$ is of the form $f = t^n$ for some $t \in \Z$. We have shown that the image of $f$ in $(\kappa[t, t^{-1}])^\times$ is of the form $t^{n}$ for some $n \in \Z$. Then we can consider $g = f t^{-n}$ which is $1$ when reduced to the special fiber. To conclude that $g = 1$ we appeal to induction on the length of $A$. If $\ell_A(A) = 1$ then $A$ must be a field in which case we are done. Since $A$ is Artinian, $\m^{N+1} = 0$ but $\m^N \neq 0$ for some $N$. Then,
\begin{center}
\begin{tikzcd}
0 \arrow[r] & \m^{N} \arrow[r] & A \arrow[r] & A / \m^{N} \arrow[r] & 0
\end{tikzcd}
\end{center}
However, $\m^{N}$ is a $\kappa$-module. Then $A' = A / \m^{N}$ has smaller length so the image of $g$ in $A'[t. t^{-1}]$ equals $1$ and thus $g - 1 \in \m^{N}[t, t^{-1}]$. However, $g(xy) = g(x)g(y)$ and thus $g(1) = g(1)^2$ but $g(1) \in A^\times$ so $g(1) = 1$. Furthermore, 
\[ (g(x) - 1)(g(y) - 1) = g(xy) + 1 - g(x) - g(y) = (g(xy) - 1) - (g(x) - 1) - (g(y) - 1) \]
and thus since $(g(x) - 1)(g(y) - 1) = 0$ letting $h = g - 1$,
\[ h(xy) = h(x) + h(y) \]
But this is impossible for degree reasons since $h(t^2) = 2 h(t)$ so if,
\[ h = \sum_{n = -k}^k c_n t^n \]
then $c_k = 0$ and $c_{-k} = 0$ since $h(t^2) = 2 h(t)$ and thus $h = 0$.

\subsubsection{4}

Let $A$ be a complete Noetherian local ring and $f \in (A[t, t^{-1}])^\times$. Then under $A \to A / \m^k$ we see that $f \mapsto t^{n}$ for a fixed $n$ by using (iii) (the fixedness of $n$ comes from the composition $A \to A / \m^k \to \kappa$ and $f = t^n$ in $\kappa[t,t^{-1}]$). However, the maps $A \to A / \m^n$ are mutually injective because $A$ is complete so $f = t^n$.
\bigskip\\
Now let $A$ be any Noetherian local ring and $f \in (A[t, t^{-1}])^\times$. Consider the injection $A \to \hat{A}$ (this is injective because if $x \mapsto 0$ under each $A \to A / \m^k$ then $x = \m^k$ for all $k$ so $x = 0$ by the Krull intersection theorem) then we see that $f = t^n$ since it is in $\hat{A}[t,t^{-1}]$.
\bigskip\\
Now let $A$ be any local ring and $f \in (A[t,t^{-1}])^\times$ and consider $A'$ to be the ring generated by the coefficients of $f$ and $f^{-1}$ over $\Z$. Then localizing at $\m \cap A'$ we get a Noetherian local subring $A'' \subset A$ such that $f \in (A''[t,t^{-1}])^\times$ and thus $f = t^n$.
\bigskip\\
Now consider any ring $A$ and $f \in (A[t,t^{-1}])^\times$. Then for any ideal $\p \in \Spec{A}$ we have $f_\p \in (A_\p[t, t^{-1}])^\times$ so $f_\p = t^{n_\p}$ giving a function $n : \Spec{A} \to \Z$ taking $\p \mapsto n_\p$. However, if $f_\p = t^{n_\p}$ then there is some $u \in A \setminus \p$ such that $u(f_\p - t^{n_\p}) = 0$ which implies that $f = t^n$ in $A_u[t, t^{-1}]$ and thus $n$ is constant on $D(u)$ with $\p \in D(u)$. 

\subsection{2}

Let $V$ be a finite-dimensional vector space over a field $k$. 

\subsubsection{1}

Should this be $\Sym{}{V^*}$ or $\Sym{}{V}^*$? It is clear that $\Sym{}{V^*}$ is the set of funtorial functions on $V$ that are sums of products of linear maps while $\Sym{}{V}^*$ contains divided power structures.

\subsubsection{2}

Consider the functor,
\[ \End{V}(R) = \End{V_R} \]
However, $\Sym{}{-}$ is the left adjoint to the forgetful functor from $k$-algebras to $k$-modules. Then,
\[ \Hom{k\text{-alg}}{\Sym{}{\End{V}^*}}{R} = \Hom{k}{\End{V}^*}{R} = \End{V} \otimes_k R = \End{V_R} \]
functorially and thus $\Sym{}{\End{V}^*}$ represents the functor $\End{}{V}$.

\subsubsection{3}

Define $\det \in \Sym{}{\End{V}^*}$ as follows. Let $n = \dim{V}$ then define the ring map $\End{V} \to \End{\bigwedge^n V}$ via $\varphi \mapsto \wedge^n \varphi$. This defines a natural map of algebras,
\[ \Sym{}{\End{\bigwedge^n V}^*} \to \Sym{}{\End{V}^*} \]
However, $\bigwedge^n V$ is one-dimensional and thus,
\[ \End{\bigwedge^n V} \cong k \]
canonically via the canonical basis element $\id \in \End{\bigwedge^n V}$. Then the map,
\[ \Sym{}{\End{\bigwedge^n V}^*} \to \Sym{}{\End{V}^*} \]
sends $\id \mapsto \det$.
\bigskip\\
Now consider the algebra,
\[ A = \Sym{}{\End{V}^*}[\det^{-1}] \]
Then we see,
\[ \Hom{k\text{-alg}}{A}{R} = \{ \varphi : \End{V}^* \to R \mid \varphi(\det) \in R^\times \} = \{ \varphi \in \End{V} \otimes_k R \mid \det{\varphi} \in R^\times \} \]
which is exactly $\Aut{V}(R)$.

\subsubsection{4}

Let $B : V \times V \to k$ be a bilinear form. Consider the subfunctor $\Aut{V,B} \subset \Aut{V}$ of points preserving $B$. It is clear that $B(g \cdot v, g \cdot u) = B(v,u)$ is a closed condition.
\bigskip\\
Let $B$ be nondegenerate. We say that $T : V_R \to V_R$ is a $B$-similitude if $B_R(T v, T u) = \mu(T) B_R(v,u)$ for all $v,u \in V_R$ and some $\mu(T) \in R^\times$. Since $B$ is nondegenerate, for each $v \in V$ there is $v' \in V$ such that $B(v,v') = 1$ and thus $B_R(v \otimes 1, v' \otimes 1) = 1$. Then, $B_R(T v \otimes 1, T v' \otimes 1) = \mu(T) \cdot B_R(v \otimes 1, T v' \otimes 1) = \mu(T)$ so $\mu(T)$ is uniquely determined by $T$ and $B$. Consider the functor sending $R \mapsto \{ B\text{-similitudes} \}$. Consider the closed subscheme,
\[ H \subset {}{V} \times \Gm \]
defined in coodinates,
\[ {}{V} \times \Gm = k[x_{ij}][\det^{-1}][t, t^{-1}] \]
as the vanishing of (let $B$ be represented by $S_{ij}$) the equations,
\[ x_{\ell i} S_{ij} x_{jk} = t S_{\ell k} \]
for each pair $(\ell, k)$.

\subsection{3 dO THIS}

\subsubsection{1}

Let $X$ be a connected scheme of finite type over a field $k$ and $x : \Spec{k} \to X$ is a rational point. Let $k' / k$ be a finite extension. Since $\Spec{k'} \to \Spec{k}$ is flat and finite we see that $X_{k'} \to X$ is flat and finite and thus open and closed. Consider the fiber over $x$,
\begin{center}
\begin{tikzcd}
\Spec{k'} \arrow[r] \arrow[d] & \Spec{k} \arrow[d]
\\
X_{k'} \arrow[r] \arrow[d] & X \arrow[d]
\\
\Spec{k'} \arrow[r] & \Spec{k}
\end{tikzcd}
\end{center}
since the bottom square is cartesian and the outer square is trivially cartesian we see that the top square is cartesian. Therefore, the fiber over $x$ consists of a single point with residue field $k'$. Now if $X_{k'} = C_1 \cup C_2$ is a disjoint union of clopen sets. Then under $f : X_{k'} \to X$ we see that $f(C_1) \cup f(C_2) = X$ and $f(C_i)$ are clopen so we need to show that $f(C_1) \cap f(C_2) = \empty$. Since the fibers over $X(k)$ are single poitns we see that each fiber is contained in exactly one of $C_1$ or $C_2$ so $f(C_1) \cap f(C_2) \cap X(k) = \empty$. Without loss of generality, the fiber over $x$ is contained in $C_1$ and thus $x \notin f(C_2)$ but $X$ is connected and $f(C_2)$ is clopen so $f(C_2) = \empty$ and thus $C_2 = \empty$ meaning that $X_{k'}$ is connected.
\bigskip\\
Now for every extension of fields $k' / k$ we know that,
\[ \varinjlim_{k' \supset k'' \supset k} k''  = k' \]
over the finite extensions $k'' / k$. However, $\Spec{-}$ is a right adjoint $\mathbf{Ring}^\op \to \mathbf{Sch}$ and thus preserves limits (this is a limit in $\mathbf{Ring}^\op$) and thus we see that.
\[ \Spec{k'} = \varprojlim_{k' \supset k'' \supset k} \Spec{k''} \]
and products commute with limits so we see that,
\[ X_{k'} = \varprojlim_{k' \supset k'' \supset k} X_{k''} \]
Since for each $k_1 \supset k_2$ the map $X_{k_1} \to X_{k_2}$ is surjective and furthermore the map $X_{k'} \to X_{k_1}$ is surjective  

\subsubsection{2}

Let $X$ and $Y$ be geometrically connected of finite type over $k$. Then it suffices to show that $(X \times_k Y) \times_k \bar{k}$ is connected. However,
\[ (X \times_k Y) \times_k \bar{k} = (X \times_k \bar{k}) \times_{\bar{k}} (Y \times_k \bar{k}) \]
and then I claim that if $X$ and $Y$ are connected and finite type over an algebraically closed field $k$ then $X \times_k Y$ is connected. Since $X$ and $Y$ are connected the standard affine cover $U_i \times V_j$ overlap eachother and thus it suffices to show the claim for affine $X$ and $Y$. Thus we need to show that if $A$ and $B$ are finitely generated $k$-algebras with prime nilradical then $A \otimes_k B$ has prime nilradical. 
\bigskip\\
Suppose that $X$ and $Y$ are connected but not necessarily geometrically connected over $k = \Q$. We can take $X = \Spec{\Q(i)}$ and $Y = \Spec{\Q(i)}$ and then 
\[ X \times_k Y = \Spec{\Q(i) \otimes_\Q \Q(i)} = \Spec{\Q(i)} \sqcup \Spec{\Q(i)} \]
is not connected.

\subsection{4 DO THIS}

Let $G$ be a group scheme of finite type over $k$.

\subsubsection{1}

Let $(G_{\overline{k}})_\red$ be the closed subscheme of $G_{\overline{k}}$. To show that various maps factor through $(G_{\overline{k}})_\red \embed G_{\overline{k}}$ it suffices to show that $(G_{\overline{k}})_\red \times (G_{\overline{k}})_\red$ is reduced (since obviously $(G_{\overline{k}})_\red$ and $\Spec{\overline{k}}$ are reduced). Since reducedness and smothness are local properties we reduce to the affine case that $A$ is a finite type $k$-algebra then $B = (A_{\overline{k}})_\red$ then then tensor product of reduced $\bar{k}$-algebras is reduced. To see this, consider,
\[ B \to \prod_{\p \text{ minimal}} B_\p \]
which is injective because for a reduced ring the associated primes are exactly the mininal primes. Then $B_\p$ is a field because $\p$ is a minimal prime. Since $B$ is flat over $k$ we can suppose that $B$ is a finite product of fields ($B$ is finite type and thus noetherian) so it suffices to reduce to the case of a domain and we know that the tensor product of domains over an algebraically closed field is a domain. 
\bigskip\\
Now we need to show that $H = (G_{\bar{k}})_\red$ is smooth. However, since $\bar{k}$ is algebraically closed and $H$ is reduced, by generic smoothness there is a smooth point and then by translation every point is smooth. 

\subsubsection{2}

Let $k$ be an imperfect field $k$ and $G$ an algebraic group scheme over $k$. Then $G_\red$ need not be a closed algebraic subgroup of $G$. This happens when $G_\red \times_k G_\red$ is not reduced and thus it does not map into $G_\red$. (FIND EXAMPLE)

\subsubsection{3}

Let $k$ be imperfect and characteristic $p > 0$. Choose $\alpha \in k \setminus k^p$ then let,
\[ f = x_0^0 + a x_1^p + \cdots + a^{p-1} x_{p-1}^p - 1 \]
and consider,
\[ G = \Spec{k[x_0, \dots, x_{p-1}]/(f)} \]
with the group operation,
\[ (x \cdot y)_n = \sum_{p + q = n} x_p y_q \]
This works because,
\[ G = \ker{(\mathrm{Nm} : \Res{k(a^{\frac{1}{p}})}{k}{\Gm} \to \Gm)} \]
I claim that $f$ is not a power over $k$. Indeed, because the degree of $f$ is prime it would have to be $f = g^p$ but this implies that $a = \alpha^p$ which is not true by hypothesis. Therefore, $G$ is reduced.
\bigskip\\
Now after base changing to $k(a^{\frac{1}{p}})$ we see that,
\[ f = (x_0 + a^{\frac{1}{p}} x_1 + \cdots + a^{\frac{p-1}{p}} x_{p-1} - 1)^p \]
and thus $G \times_k k(a^{\frac{1}{p}}) = \Spec{k[x_0, \dots, x_{p-1}]/(f)}$ is not reduced and thus not smooth. 

\renewcommand{\GL}{\mathrm{GL}}
\renewcommand{\SL}{\mathrm{SL}}

\subsubsection{4}

Consider the subscheme,
\[ \mu_n = \ker{(\Gm \xrightarrow{x \mapsto x^n} \Gm)} \]
which is a subgroup because kernels always are. Clearly, the kernel is closed. Let $K = \ker{(G \to H)}$. Then $K \times K \subset G \times G \to G$ maps to $K$ because,
\[ K \times K \to G \to H \]
is the map $K \times K \to \Spec{k} \to H$ and thus factors through $K \to G$ by the universal property of the kernel which is a fiber product. 
\bigskip\\
Consider the map $\deg : \GL_N \to \Gm$ and the preimage $G = \det^{-1}{\mu_n} \subset \GL_N$ is always a $k$-subgroup by the universal property (its the exact same argument as for the kernel). Expliclty, let $K \subset H$ be a closed subgroup and $f : G \to H$ a morphism of algebraic groups. Let $\tilde{K} = f^{-1}(K)$ the pullback which is a closed subscheme of $G$ then consider the diagram,
\begin{center}
\begin{tikzcd}
\tilde{K} \times \tilde{K}  \arrow[rr, bend left, "m"] \arrow[d] \arrow[r, dashed] & \tilde{K} \arrow[r] \arrow[d] & G \arrow[d, "f"]
\\
K \times K \arrow[r, "m"] & K \arrow[r] & H
\end{tikzcd}
\end{center}
therefore multiplication factors through $\tilde{K} \times \tilde{K} \to \tilde{K}$. The same trick works for inversion. 
\bigskip\\
It is clear that $\SL_N \subset G$ and thus assuming that $\SL_N$ is connected we see that $\SL_N \subset G^0$. Furthermore, as long as $p \ndivides n$ we see that $\mu_n$ is disconnected with a reduced point at each root of unity contained in $k$ and $\mu_n^0 = \Spec{k}$ the trivial group scheme at the origin. Thus, $G^0 = \ker{\det} = \SL_N$. 
\bigskip\\
For $k = \Q$ and $n = 5$ the group scheme $G \setminus G^0$ is the fiber over the one nonidentity point of $\mu_5 = \Spec{k[x]/(x^5 - 1)}$ which is the point $\eta = \Spec{k[x]/(x^4 + x^3 + x^2 + x + 1)}$. Therefore, the preimage is isomorphic as a scheme over $k$ to $\SL_N \times_k \eta$ which is connected because it is just $\SL_N$ over $\eta$ viewed as a $k$-scheme. However, over $\overline{\Q}$ we see that $\eta$ splits into four points and thus $G \setminus G^0 = \det^{-1}(\eta)$ must have at least four components after base change to $\overline{\Q}$.
\end{document}