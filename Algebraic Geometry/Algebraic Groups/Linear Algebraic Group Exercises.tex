\documentclass[12pt]{article}
\usepackage{import}
\import{./}{AlgGroupsCommands}


\begin{document}

\section{Homework 1}

\tableofcontents

\subsection{1}

\subsubsection{1}

Maps $\Hom{k}{\Ga}{\Ga} = \Hom{k}{\Spec{k[t]}}{\Spec{k[t]}} = \Hom{k}{k[t]}{k[t]} = k[t]$. Therefore we need $f \in k[t]$ which are cogroup maps for $k[t] \to k[t] \otimes_k k[t]$ meaning that $f(x + y) = f(x) + f(y)$. 
\bigskip\\
Consider $\Hom{k}{\Gm}{\Gm} = \Hom{k}{\Spec{k[t]}}{\Spec{k[t]}} = \Hom{k}{k[t,t^{-1}]}{k[t,t^{-1}]} = (k[t, t^{-1}])^\times$ such that $f(xy) = f(x) f(y)$.

\subsubsection{2}

If $k$ is a $\Q$-algebra then any $f \in k[t]$ satisfying $f(x + y) = f(x) + f(y)$ must be linear with zero constant term and thus $f = a t$ for $a \in k$ so we get $\End{\Ga} \cong k$. 
\bigskip\\
If $k$ is a field of characteristic $p > 0$ then if $f(x+y) = f(x) + f(y)$ we must have that $f(ax) = a f(x)$ for each $a \in \FF_p$ which implies that,
\[ f(t) = \sum c_j t^{p^j} \]
Suppose that $k = \Z / (p^2)$. For $f \in k[t]$ we see that reducing to $\bar{f} \in \FF_p[t]$ that,
\[ f(t) = \sum \bar{c}_j t^{p^j} \]
for some $c_j \in \FF_p$. Therefore, $f$ must be of the form,
\[ f(t) = \sum c_j t^{p^j} \]
for $c_j \in k$. However, because $(k, +)$ is generated by $1$ we must have $f(at) = a f(t)$ for $a \in k$ and thus $c_j p^{p^j} = c_j$ but $p^2 = 0$ so if $j > 0$ we get $c_j = 0$ and therefore, $f(t) = c t$ for any $c \in k$.

\subsubsection{3}

If $k$ is a field then $(k[t, t^{-1}])^\times$ consists of elements of the form $f(t) = a t^n$ and if $f(xy) = f(x) f(y)$ then $a = 1$ so $\End{\Gm} \cong \Z$.
\bigskip\\
Now suppose that $A$ is an Artinian local ring and $\kappa = A / \m$ its residue field. It suffices to prove that every $f \in (A[t, t^{-1}])^\times$ such that $f(xy) = f(x) f(y)$ is of the form $f = t^n$ for some $t \in \Z$. We have shown that the image of $f$ in $(\kappa[t, t^{-1}])^\times$ is of the form $t^{n}$ for some $n \in \Z$. Then we can consider $g = f t^{-n}$ which is $1$ when reduced to the special fiber. To conclude that $g = 1$ we appeal to induction on the length of $A$. If $\ell_A(A) = 1$ then $A$ must be a field in which case we are done. Since $A$ is Artinian, $\m^{N+1} = 0$ but $\m^N \neq 0$ for some $N$. Then,
\begin{center}
\begin{tikzcd}
0 \arrow[r] & \m^{N} \arrow[r] & A \arrow[r] & A / \m^{N} \arrow[r] & 0
\end{tikzcd}
\end{center}
However, $\m^{N}$ is a $\kappa$-module. Then $A' = A / \m^{N}$ has smaller length so the image of $g$ in $A'[t, t^{-1}]$ equals $1$ and thus $g - 1 \in \m^{N}[t, t^{-1}]$. However, $g(xy) = g(x)g(y)$ and thus $g(1) = g(1)^2$ but $g(1) \in A^\times$ so $g(1) = 1$. Furthermore, 
\[ (g(x) - 1)(g(y) - 1) = g(xy) + 1 - g(x) - g(y) = (g(xy) - 1) - (g(x) - 1) - (g(y) - 1) \]
and thus since $(g(x) - 1)(g(y) - 1) = 0$ letting $h = g - 1$,
\[ h(xy) = h(x) + h(y) \]
But this is impossible for degree reasons since $h(t^2) = 2 h(t)$ so if,
\[ h = \sum_{n = -k}^k c_n t^n \]
then $c_k = 0$ and $c_{-k} = 0$ since $h(t^2) = 2 h(t)$ and thus $h = 0$.

\subsubsection{4}

Let $A$ be a complete Noetherian local ring and $f \in (A[t, t^{-1}])^\times$. Then under $A \to A / \m^k$ we see that $f \mapsto t^{n}$ for a fixed $n$ by using (iii) (the fixedness of $n$ comes from the composition $A \to A / \m^k \to \kappa$ and $f = t^n$ in $\kappa[t,t^{-1}]$). However, the maps $A \to A / \m^n$ are mutually injective because $A$ is complete so $f = t^n$.
\bigskip\\
Now let $A$ be any Noetherian local ring and $f \in (A[t, t^{-1}])^\times$. Consider the injection $A \to \hat{A}$ (this is injective because if $x \mapsto 0$ under each $A \to A / \m^k$ then $x = \m^k$ for all $k$ so $x = 0$ by the Krull intersection theorem) then we see that $f = t^n$ since it is in $\hat{A}[t,t^{-1}]$.
\bigskip\\
Now let $A$ be any local ring and $f \in (A[t,t^{-1}])^\times$ and consider $A'$ to be the ring generated by the coefficients of $f$ and $f^{-1}$ over $\Z$. Then localizing at $\m \cap A'$ we get a Noetherian local subring $A'' \subset A$ such that $f \in (A''[t,t^{-1}])^\times$ and thus $f = t^n$.
\bigskip\\
Now consider any ring $A$ and $f \in (A[t,t^{-1}])^\times$. Then for any ideal $\p \in \Spec{A}$ we have $f_\p \in (A_\p[t, t^{-1}])^\times$ so $f_\p = t^{n_\p}$ giving a function $n : \Spec{A} \to \Z$ taking $\p \mapsto n_\p$. However, if $f_\p = t^{n_\p}$ then there is some $u \in A \setminus \p$ such that $u(f_\p - t^{n_\p}) = 0$ which implies that $f = t^n$ in $A_u[t, t^{-1}]$ and thus $n$ is constant on $D(u)$ with $\p \in D(u)$. 

\subsection{2}

Let $V$ be a finite-dimensional vector space over a field $k$. 

\subsubsection{1}

Should this be $\Sym{}{V^*}$ or $\Sym{}{V}^*$? It is clear that $\Sym{}{V^*}$ is the set of funtorial functions on $V$ that are sums of products of linear maps while $\Sym{}{V}^*$ contains divided power structures.

\subsubsection{2}

Consider the functor,
\[ \End{V}(R) = \End{V_R} \]
However, $\Sym{}{-}$ is the left adjoint to the forgetful functor from $k$-algebras to $k$-modules. Then,
\[ \Hom{k\text{-alg}}{\Sym{}{\End{V}^*}}{R} = \Hom{k}{\End{V}^*}{R} = \End{V} \otimes_k R = \End{V_R} \]
functorially and thus $\Sym{}{\End{V}^*}$ represents the functor $\End{V}$.

\subsubsection{3}

Define $\det \in \Sym{}{\End{V}^*}$ as follows. Let $n = \dim{V}$ then define the ring map $\End{V} \to \End{\bigwedge^n V}$ via $\varphi \mapsto \wedge^n \varphi$. This defines a natural map of algebras,
\[ \Sym{}{\End{\bigwedge^n V}^*} \to \Sym{}{\End{V}^*} \]
However, $\bigwedge^n V$ is one-dimensional and thus,
\[ \End{\bigwedge^n V} \cong k \]
canonically via the canonical basis element $\id \in \End{\bigwedge^n V}$. Then the map,
\[ \Sym{}{\End{\bigwedge^n V}^*} \to \Sym{}{\End{V}^*} \]
sends $\id \mapsto \det$.
\bigskip\\
Now consider the algebra,
\[ A = \Sym{}{\End{V}^*}[\det^{-1}] \]
Then we see,
\[ \Hom{k\text{-alg}}{A}{R} = \{ \varphi : \End{V}^* \to R \mid \varphi(\det) \in R^\times \} = \{ \varphi \in \End{V} \otimes_k R \mid \det{\varphi} \in R^\times \} \]
which is exactly $\Aut{V}(R)$.

\subsubsection{4}

Let $B : V \times V \to k$ be a bilinear form. Consider the subfunctor $\Aut{V,B} \subset \Aut{V}$ of points preserving $B$. It is clear that $B(g \cdot v, g \cdot u) = B(v,u)$ is a closed condition.
\bigskip\\
Let $B$ be nondegenerate. We say that $T : V_R \to V_R$ is a $B$-similitude if $B_R(T v, T u) = \mu(T) B_R(v,u)$ for all $v,u \in V_R$ and some $\mu(T) \in R^\times$. Since $B$ is nondegenerate, for each $v \in V$ there is $v' \in V$ such that $B(v,v') = 1$ and thus $B_R(v \otimes 1, v' \otimes 1) = 1$. Then, $B_R(T v \otimes 1, T v' \otimes 1) = \mu(T) \cdot B_R(v \otimes 1, T v' \otimes 1) = \mu(T)$ so $\mu(T)$ is uniquely determined by $T$ and $B$. Consider the functor sending $R \mapsto \{ B\text{-similitudes} \}$. Consider the closed subscheme,
\[ H \subset {}{V} \times \Gm \]
defined in coodinates,
\[ {}{V} \times \Gm = k[x_{ij}][\mathrm{det}^{-1}][t, t^{-1}] \]
as the vanishing of (let $B$ be represented by $S_{ij}$) the equations,
\[ x_{\ell i} S_{ij} x_{jk} = t S_{\ell k} \]
for each pair $(\ell, k)$.

\subsection{3}

\subsubsection{1 DO!!}

Let $X$ be a connected scheme of finite type over a field $k$ and $x : \Spec{k} \to X$ is a rational point. Let $k' / k$ be a finite extension. Since $\Spec{k'} \to \Spec{k}$ is flat and finite we see that $X_{k'} \to X$ is flat and finite and thus open and closed. Consider the fiber over $x$,
\begin{center}
\begin{tikzcd}
\Spec{k'} \arrow[r] \arrow[d] & \Spec{k} \arrow[d]
\\
X_{k'} \arrow[r] \arrow[d] & X \arrow[d]
\\
\Spec{k'} \arrow[r] & \Spec{k}
\end{tikzcd}
\end{center}
since the bottom square is cartesian and the outer square is trivially cartesian we see that the top square is cartesian. Therefore, the fiber over $x$ consists of a single point with residue field $k'$. Now if $X_{k'} = C_1 \cup C_2$ is a disjoint union of clopen sets. Then under $f : X_{k'} \to X$ we see that $f(C_1) \cup f(C_2) = X$ and $f(C_i)$ are clopen so we need to show that $f(C_1) \cap f(C_2) = \empty$. Since the fibers over $X(k)$ are single poitns we see that each fiber is contained in exactly one of $C_1$ or $C_2$ so $f(C_1) \cap f(C_2) \cap X(k) = \empty$. Without loss of generality, the fiber over $x$ is contained in $C_1$ and thus $x \notin f(C_2)$ but $X$ is connected and $f(C_2)$ is clopen so $f(C_2) = \empty$ and thus $C_2 = \empty$ meaning that $X_{k'}$ is connected.
\bigskip\\
Now for every extension of fields $k' / k$ we know that,
\[ \varinjlim_{k' \supset k'' \supset k} k''  = k' \]
over the finitely generated extensions $k'' / k$. We have shown that $X_{k'}$ is connected for a finite extension $k' / k$ so we also need to show this for a purely transcendental finitely generated extension. Thus it suffices to consider the case $k' = k(x)$. However, $A \otimes_k k(x)$ is a localization of $A[x]$ so if $A$ is a domain then $A \otimes_k k'$ is also a domain (HOW TO CONCLUDE).
\bigskip\\
Now, $\Spec{-}$ is a right adjoint $\mathbf{Ring}^\op \to \mathbf{Sch}$ and thus preserves limits (this is a limit in $\mathbf{Ring}^\op$) and thus we see that.
\[ \Spec{k'} = \varprojlim_{k' \supset k'' \supset k} \Spec{k''} \]
and products commute with limits so we see that,
\[ X_{k'} = \varprojlim_{k' \supset k'' \supset k} X_{k''} \]
A projective limit of schemes with affine transition maps has its underlying topological space equal to the projective limit of the underlying topological spaces. Therefore, it suffices to show that a projective limit of connected spaces with surjective transition maps is connected. (DO THIS!!)

\subsubsection{2}

Let $X$ and $Y$ be geometrically connected of finite type over $k$. Then it suffices to show that $(X \times_k Y) \times_k \bar{k}$ is connected. However,
\[ (X \times_k Y) \times_k \bar{k} = (X \times_k \bar{k}) \times_{\bar{k}} (Y \times_k \bar{k}) \]
and then I claim that if $X$ and $Y$ are connected and finite type over an algebraically closed field $k$ then $X \times_k Y$ is connected. Since $X$ and $Y$ are connected the standard affine cover $U_i \times V_j$ overlap eachother and thus it suffices to show the claim for affine $X$ and $Y$. Thus we need to show that if $A$ and $B$ are finitely generated $k$-algebras with prime nilradical then $A \otimes_k B$ has prime nilradical. 
\bigskip\\
Suppose that $X$ and $Y$ are connected but not necessarily geometrically connected over $k = \Q$. We can take $X = \Spec{\Q(i)}$ and $Y = \Spec{\Q(i)}$ and then 
\[ X \times_k Y = \Spec{\Q(i) \otimes_\Q \Q(i)} = \Spec{\Q(i)} \sqcup \Spec{\Q(i)} \]
is not connected.

\subsection{4 DO THIS}

Let $G$ be a group scheme of finite type over $k$.

\subsubsection{1}

Let $(G_{\overline{k}})_\red$ be the closed subscheme of $G_{\overline{k}}$. To show that various maps factor through $(G_{\overline{k}})_\red \embed G_{\overline{k}}$ it suffices to show that $(G_{\overline{k}})_\red \times (G_{\overline{k}})_\red$ is reduced (since obviously $(G_{\overline{k}})_\red$ and $\Spec{\overline{k}}$ are reduced). Since reducedness and smothness are local properties we reduce to the affine case that $A$ is a finite type $k$-algebra then $B = (A_{\overline{k}})_\red$ then then tensor product of reduced $\bar{k}$-algebras is reduced. To see this, consider,
\[ B \to \prod_{\p \text{ minimal}} B_\p \]
which is injective because for a reduced ring the associated primes are exactly the mininal primes. Then $B_\p$ is a field because $\p$ is a minimal prime. Since $B$ is flat over $k$ we can suppose that $B$ is a finite product of fields ($B$ is finite type and thus noetherian) so it suffices to reduce to the case of a domain and we know that the tensor product of domains over an algebraically closed field is a domain. 
\bigskip\\
Now we need to show that $H = (G_{\bar{k}})_\red$ is smooth. However, since $\bar{k}$ is algebraically closed and $H$ is reduced, by generic smoothness there is a smooth point and then by translation every point is smooth. 
\bigskip\\
Then $G^\circ$ is connected and has a rational point so by 3. $G^\circ$ is geometrically connected. However, $(G^\circ_{\bar{k}})_{\red}$ is smooth and connected (reduction does not change the underlying topological space) and therefore irreducible (because regular rings are domains). Therefore $G^\circ$ is geometrically irreducible.

\subsubsection{2}

Let $k$ be an imperfect field $k$ and $G$ an algebraic group scheme over $k$. Then $G_\red$ need not be a closed algebraic subgroup of $G$. This happens when $G_\red \times_k G_\red$ is not reduced and thus it does not map into $G_\red$. For example, see \chref{https://mathoverflow.net/questions/38891/is-there-a-connected-k-group-scheme-g-such-that-g-red-is-not-a-subgroup}{this mathoverflow answer}.

\subsubsection{3}

Let $k$ be imperfect and characteristic $p > 0$. Choose $\alpha \in k \setminus k^p$ then let,
\[ f = x_0^0 + a x_1^p + \cdots + a^{p-1} x_{p-1}^p - 1 \]
and consider,
\[ G = \Spec{k[x_0, \dots, x_{p-1}]/(f)} \]
with the group operation,
\[ (x \cdot y)_n = \sum_{p + q = n} x_p y_q \]
This works because,
\[ G = \ker{(\mathrm{Nm} : \Res{k(a^{\frac{1}{p}})}{k}{\Gm} \to \Gm)} \]
I claim that $f$ is not a power over $k$. Indeed, because the degree of $f$ is prime it would have to be $f = g^p$ but this implies that $a = \alpha^p$ which is not true by hypothesis. Therefore, $G$ is reduced.
\bigskip\\
Now after base changing to $k(a^{\frac{1}{p}})$ we see that,
\[ f = (x_0 + a^{\frac{1}{p}} x_1 + \cdots + a^{\frac{p-1}{p}} x_{p-1} - 1)^p \]
and thus $G \times_k k(a^{\frac{1}{p}}) = \Spec{k[x_0, \dots, x_{p-1}]/(f)}$ is not reduced and thus not smooth. 


\subsubsection{4}

Consider the subscheme,
\[ \mu_n = \ker{(\Gm \xrightarrow{x \mapsto x^n} \Gm)} \]
which is a subgroup because kernels always are. Clearly, the kernel is closed. Let $K = \ker{(G \to H)}$. Then $K \times K \subset G \times G \to G$ maps to $K$ because,
\[ K \times K \to G \to H \]
is the map $K \times K \to \Spec{k} \to H$ and thus factors through $K \to G$ by the universal property of the kernel which is a fiber product. 
\bigskip\\
Consider the map $\deg : \GL_N \to \Gm$ and the preimage $G = \det^{-1}{\mu_n} \subset \GL_N$ is always a $k$-subgroup by the universal property (its the exact same argument as for the kernel). Expliclty, let $K \subset H$ be a closed subgroup and $f : G \to H$ a morphism of algebraic groups. Let $\tilde{K} = f^{-1}(K)$ the pullback which is a closed subscheme of $G$ then consider the diagram,
\begin{center}
\begin{tikzcd}
\tilde{K} \times \tilde{K}  \arrow[rr, bend left, "m"] \arrow[d] \arrow[r, dashed] & \tilde{K} \arrow[r] \arrow[d] & G \arrow[d, "f"]
\\
K \times K \arrow[r, "m"] & K \arrow[r] & H
\end{tikzcd}
\end{center}
therefore multiplication factors through $\tilde{K} \times \tilde{K} \to \tilde{K}$. The same trick works for inversion. 
\bigskip\\
It is clear that $\SL_N \subset G$ and thus assuming that $\SL_N$ is connected we see that $\SL_N \subset G^0$. Furthermore, as long as $p \ndivides n$ we see that $\mu_n$ is disconnected with a reduced point at each root of unity contained in $k$ and $\mu_n^0 = \Spec{k}$ the trivial group scheme at the origin. Thus, $G^0 = \ker{\det} = \SL_N$. 
\bigskip\\
For $k = \Q$ and $n = 5$ the group scheme $G \setminus G^0$ is the fiber over the one nonidentity point of $\mu_5 = \Spec{k[x]/(x^5 - 1)}$ which is the point $\eta = \Spec{k[x]/(x^4 + x^3 + x^2 + x + 1)}$. Therefore, the preimage is isomorphic as a scheme over $k$ to $\SL_N \times_k \eta$ which is connected because it is just $\SL_N$ over $\eta$ viewed as a $k$-scheme. However, over $\overline{\Q}$ we see that $\eta$ splits into four points and thus $G \setminus G^0 = \det^{-1}(\eta)$ must have at least four components after base change to $\overline{\Q}$.

\section{Homework 2}

\subsection{1}

\subsubsection{1}

Wait, if $k$ is perfect then doesn't regular imply geometrically regular in general and thus (since $G$ is finite type) smooth over $k$.
\bigskip\\
Thus if $X$ is the regular compactification of $G$ then $X$ is smooth over $k$. Since we are in the additive case, $G_{\bar{k}} \cong \A^1_{\bar{k}}$ and thus $X_{\bar{k}}$ being a regular compactification (which is unique) must be isomorphic to $\P^1_{\bar{k}}$ with $X_{\bar{k}} \setminus G_{\bar{k}}$ a single point. This implies that $X \setminus G$ is a single point $\Spec{k'}$.

\subsubsection{2}

Since $k$ is perfect, $k' \otimes_k \bar{k}$ is reduced and thus is a product of $[k' : k]$ copies of $\bar{k}$ thus giving $[k' : k]$ points in the fiber of $X_{\bar{k}} \to X$ over $\Spec{k'} \to X$. However, this fiber contains a single point and thus $[k' : k] = 1$ so $k' = k$. Therefore, $X$ has a rational point and has genus zero (arithmetic genus is preserved under flat base change) and thus is isomorphic to $\P^1$ by taking the line bundle $\struct{X}(p)$ where $p$ is the rational point. By Riemann-Roch, we have that $\struct{X}(p)$ has exactly two sections since $\omega_X(-p)$ has negative degree and $1 - g + 1 = 2$. Therefore it defines a degree one map to $\P^1$ which must be an isomorphism because it is birational. 
\bigskip\\
Therefore $G \cong \A^1_k$ as a $k$-scheme so it suffices to check that $\A^1_k$ has a unique group law which is checked in the notes.

\subsection{2}

Let $T$ be a torus of dimension $r \ge 1$ over a field $k$. Therefore, $T_{k'} \cong \Gm^r$ over some finite extension $k'/k$. Here we claim that the extension can always be chosen to be separable.

\subsubsection{1}

Suppose we treat the case $k = k_s$ (i.e. $k$ is separably closed). This implies that if $T$ is a torus over $k_s$ then $T$ is split over $k_s$ (because all finite separable extensions of $k$ are trivial). In particular, given a torus over $k$, we see that $T_{k_s}$ is a torus over $k_s$ and thus is split by assumption. However, there must be a finite extension $k'/k$ such that $T_{k'}$ is split and $k' \subset k_s$ so $k'/k$ is finite separable proving the claim.

\subsubsection{2}

We can assume that $k = k_s$. Then for a finite extension $k'/k$ there is an isomorphism $f : T_{k'} \iso \Gm^r$ as $k'$-groups. Let $A = k' \otimes_k k'$ and $p_1, p_2 : \Spec{A} \to \Spec{k'}$ the two projections. Because $k'/k$ is purely inseparable, we see that $A$ is an Artin local ring with residue field $k'$. $A$ is Artinian because it has finite length as a $k$-module since $k'/k$ is finite. Furthermore, if $A$ were not local, then it would be a product of local Artinian rings so there is some non-nilpotent zero divisor which gives a separable extension. Consider the diagonal $\Delta  : \Spec{k'} \to \Spec{A}$ which must hit the unique point of $\Spec{A}$. Thus its residue field embeds in $k'$. However, $A$ is a $k'$-algebra so thus $A / \m = k'$. Now, the two maps $p_1^* f, p_2^* f : T_A \iso \Gm^r$ both restrict to $f$ over the single point because $p_1 \circ \Delta = p_2 \circ \Delta$ by construction so $\Delta^* p_1^* = \Delta^* p_2^*$.
\bigskip\\
Since these are isomorphisms, it suffices to consider $(p_1^* f) \circ (p_2^* f)^{-1}$ and show this is the identity. This is an automorphism of $\Gm^r$ as an $A$-algebraic group such that when restricted to the special fiber it is trivial. It suffcies to check that the component maps $\Gm \to \Gm$ (which are $A$-endomorphisms) are either trivial or the identity respectively. However, on the previous assignment we showed that the map $\End{\Gm \times A} \to \End{\Gm \times A / \m}$ is bijective so we are indeed done by noting that the component maps $\Gm \to \Gm$ are what I required when restricted to the special fiber.

\subsubsection{3}

Let $V$ be a $k$-vectorspace. Consider the two projections $p_1, p_2 : k' \otimes_k V \to A \otimes_k V$ and their equalizer $K \subset k' \otimes_k V$. We want to show that $K = V$. This follows immediately from the following more general lemma.

\begin{lemma}
Let $R \to S$ be a faithfully flat ring map. Suppose that $M$ is an $R$-module. Then the following sequence is exact,
\begin{center}
\begin{tikzcd}
0 \arrow[r] & M \arrow[r] & M \otimes_R S \arrow[r] & M \otimes_R (S \otimes_R S) 
\end{tikzcd}
\end{center}
where $M \otimes_R S \to M \otimes_R (S \otimes_R S)$ is the difference of the projections: $m \otimes s \mapsto m \otimes s \otimes 1 - m \otimes 1 \otimes s$.
\end{lemma}

\begin{proof}
By faithful flatness, it suffices to show that,
\begin{center}
\begin{tikzcd}
0 \arrow[r] & M \otimes_R S \arrow[r] & M \otimes_R S \otimes_R S \arrow[r] & M \otimes_R (S \otimes_R S) \otimes_R S
\end{tikzcd}
\end{center}
is exact and thus we may replace $R$ by $S$ and $S$ by $S \otimes_R S$ so there is a section $\delta : S \otimes_R S \to S$. Then I claim that the complex,
\begin{center}
\begin{tikzcd}
0 \arrow[r] & S \arrow[r] & S \otimes_R S \arrow[r] & S \otimes_R S \otimes_R S 
\end{tikzcd}
\end{center}
is nullhomotopic as a comples of $S$-modules and thus is ``universally exact'' in the sense that it is exact under the application of an additive functor (for example $M \otimes_R -$. To see this, we already have the map $\delta : S \otimes_R S \to S$ and $S \to 0$ is the zero map. We see that $S \to S \otimes_R S \to S$ is the identity. Furthermore take the section $\delta \otimes \id : S \otimes_R S \otimes_R S \to S \otimes_R S$. This is a special case of the exactness of the Amister complex for faithfully flat maps. 
\end{proof}
\noindent
Now I claim that $T$ is affine. Indeed, $T_{k'} \cong \Gm^r$ is affine and thus by Serre's cohomological criterion and flat base change $T$ is affine. Therefore, $T = \Spec{A}$ and $\Gm^r = \Spec{B}$ and we consider $V = \Hom{k}{A}{B}$. Notice that $V \otimes_k k' = \Hom{k'}{A \otimes_k k'}{B \otimes_k k'}$ and there is a map coming from the isomorphism $f : T_{k'} \iso \Gm^r$ such that the projections to $V \otimes_k A$ agree. Thus by the lemma this comes from a unique $g \in \Hom{k}{A}{B}$ such that $f = g_{k'}$. Since $f$ is an isomorphism, by a previous part we see that $g$ is an isomorphism proving the claim.

\subsection{3}

\subsubsection{1 ACTUALLY DO THIS}

The map $\Spec{K} \to \Spec{k}$ is a fpqc covering and therefore is a universal effective epimorphism by Tag 023Q and thus, in particular, is a universal epimorphism so $X_K \to X$ is an epimorphism for any $k$-scheme $X$. Then consider the diagram,
\begin{center}
\begin{tikzcd}
X_K \arrow[r, "f_K"] \arrow[d, "\pi"] & X_Y \arrow[d]
\\
X \arrow[r, "f"] & Y
\end{tikzcd}
\end{center}
so we see that if $f_K = g_K$ then $f \circ \pi =  g \circ \pi$ and thus $f = g$ since $\pi$ is an epimorphism.
\bigskip\\
Suppose that $Z, Z' \subset X$ are closed subschemes such that $Z_K = Z'_K$ as closed subschemes of $X_K$. 

\subsubsection{2}

Let $f : X \to Y$ be a morphism of affine schemes and $f_K$ is an isomorphism. We need to consider a map of $k$-algebras $f : A \to B$ such that $f_K : A \otimes_k K \to B \otimes_k B$ is an isomorphism. By the previous problem, it suffices to show that $f_K^{-1}$ is defined over $k$. Notice that $A \subset A \otimes_k K$ maps to $B \otimes_k K$ under $f_K$ and therefore $g = f_K^{-1}|_B : B \to A$ is a ring map and $g_K = f_K^{-1}$. 
\bigskip\\
Now let $f : X \to Y$ be an arbitrary morphism of schemes such that $f_K$ is an isomorphism. If $f$ is affine then locally on $Y$ we see that $f$ is an isomorphism by the previous part. Thus it suffices to show that if $f_K$ is affine then $f$ is affine. Because forming pre-images commutes with base change, it suffices to show that if $X_K$ is affine then $X$ is affine. We apply Serre's criterion. For any quasi-coherent $\struct{X}$-module $\F$ we see that,
\[ H^i(X_K, \F_K) = 0 \]
for $i > 0$ because $X_K$ is affine. However, by flat base change,
\[ H^i(X_K, \F_K) = H^i(X, \F) \otimes_k K \]
and therefore $H^i(X, \F) = 0$ so $X$ is affine. 

\subsubsection{3 DO THIS ACTUALL}

Let $K / k$ be Galois with $\Gamma = \Gal{K/k}$. Suppose that $F : X_K \to Y_K$ satisfies $\gamma^* F = F$ for all $\gamma \in \Gamma$. Since $K / k$ is Galois we see that $K \otimes_k K \cong \prod_{\gamma \in \Gamma} K^\gamma$ as $\Gamma$-modules. Therefore,
\[ X_K \otimes_k K = \prod_{\gamma \in \Gamma} X_{\gamma, K} \]
There are two projections $(K \otimes_k K) \to K$ and the condition exactly says that pulling back along these projections $F$ defines the same map either way. Therefore, by descent, as justified earlier, we see that $F$ comes from a unique map $f : X \to Y$. Descent for closed subschemes follows from descent for quasi-coherent sheaves of ideals which follows from the property above. Furthermore, descent for open subschemes corresponds exactly to descent for the reduced scheme structure on their closed complements (an open set has a unique open subscheme structure). 

\subsection{4}

Let $q : V \to k$ be a quadratic form on a finite-dimensional vector space $V$ of dimension $d \ge 2$, and let $B_q : V \times V \to k$ be the corresponding symmetric bilinear form. Let,
\[ V^\perp = \{ v \in V \mid B_q(v,-) = 0 \} \]
We call $\delta_q = \dim{V^\perp}$ the defect of $q$.

\subsubsection{1}

Notice that for $v \in V$ we have $B_q(v, -) = B(-,v) = 0$ and therefore $B_q$ factors through
\[ \bar{B}_q : V / V^\perp \times V / V^\perp \to k \]
Furthermore, $\bar{B}_q$ is obviously symmetric. Furthermore, suppose that $\bar{B}_q(v, -) = 0$ then $v \in V^\perp$ so it is zero in $V/V^\perp$ and thus $\bar{B}_q$ is non-degenerate. Therefore, $B_q$ is nondegenerate exactly when $\delta_q = 0$. If $\mathrm{char}(k) = 2$ then notice that,
\[ B_q(v,v) = q(v + v) - q(v) - q(v) = q(2v) - 2q(v) = 0 \]
so we see that $B_q$ is alternating. However, $\bar{B}_q$ is alternating and nondegenrate so $\dim{V/V^\perp}$ has even dimension (it must pair up a basis) and thus $\dim{V} \equiv \dim{V^\perp} \mod 2$. Therefore, if $\dim{V}$ is odd then $\delta_q > 0$. 

\subsubsection{2}

Suppose that $\delta_q = 0$. Assume that $k$ is algebraically closed.
\bigskip\\
We let $\dim{V} = 2n + 1$ or $\dim{V} = 2n$. If $n = 0$ we are clearly done in either case because either $V = (0)$ and $q = 0$ or $V = k$ and $q(v) = v^2$. Now we proceed by induction. Choose linearly independent $v_1, v_2 \in V$. Now,
\[ q(\alpha v_1 + \beta v_2) = \alpha^2 q(v_1) + \alpha \beta B(v_1, v_2) + \beta^2 q(v_2) \]
By algebraic closure there exists $\alpha, \beta$ not both zero such that $q(\alpha e_1 + \beta e_2) = 0$. 
\bigskip\\
Either way now we have $q(e_1) = 0$. Choose $\tilde{e}_2 \in V$ such that $B_q(e_1, e_2) = 1$ (which we can do because $B_q$ is nondegenerate) then we must have $e_1, \tilde{e}_2$ be independent because $B_q(e_1, e_1) = 0$. Now let, $e_2 = \tilde{e}_2 + \alpha e_1$
and consider,
\[ q(e_2) = q(\tilde{e}_2) + \alpha^2 q(e_1) + \alpha B_q(e_1, \tilde{e}_2) = q(\tilde{e}_2) + \alpha \]
so we may choose $\alpha = - q(\tilde{e}_2)$ such that $q(e_2) = 0$ and also $B_q(e_1, e_2) = B_q(e_1, \tilde{e}_2) + \alpha B_q(e_1, e_1) = 1$ because $B_q(e_1, e_1) = 2 q(e_1) = 0$. 
\bigskip\\
Consider the map $\varphi : V \to k^2$ by sending $v \mapsto (B_q(v, e_2), B_q(v, e_1))$. For $v = \alpha e_1 + \beta e_2$ we see that $\varphi(v) = (\alpha, \beta)$ so $\varphi$ is surjective\footnote{Indeed, for any nondegenerate bilinear form $B$ and independent vectors $e_1, e_2$, consider the map $\varphi : V \to k^2$ by sending $v \mapsto (B(e_1, v), B(e_2, v))$. I claim this map is surjective. Otherwise, there would be some $\alpha, \beta \in k$ such that $\alpha B(e_1, v) + \beta B(e_2, v) = 0$ for all $v$ and thus $B(\alpha e_1 + \beta e_2, -) = 0$ so $\alpha e_1 + \beta e_2 = 0$ and thus by independence $\alpha = 0$ and $\beta = 0$ proving surjectivity.} and $e_1, e_2$ maps to the standard basis of $k^2$. Therefore, $\dim{\ker{\varphi}} = \dim{V} - 2$ and if $v \in \ker{\varphi} \cap \vspan{e_1, e_2}$ then $v = \alpha e_1 + \beta e_2$ but then $\varphi(v) = (\beta, \alpha) = 0$ so $v = 0$. Thus, for any $v \in \ker{\varphi}$ there is some $w \in V$ such that $B_q(v, w) = 1$ but then $w = \alpha e_1 + \beta e_2 + w'$ for $w' \in \ker{\varphi}$ and $B_q(v, e_1) = B_q(v, e_2) = 0$ so $B_q(v, w') = 1$ and thus $B_q |_{\ker{\varphi}}$. Thus by the induction hypothesis, $q |_{\ker{\varphi}}$ has the required form meaning there is a basis $v_1, \dots, v_{n-1}, w_1, \dots, w_{n-1}$ such that,
\[ q(v_i) = q(w_i) = 0 \quad \text{and} \quad B(v_i, v_j) = B(w_i, w_j) = 0 \quad \text{and} \quad B(v_i, w_j) = \delta_{ij} \]
with possibly $v_0$ such that $q(v_0) = 1$ and $B(v_0, v_i) = B(v_0, w_i) = 0$ for $i > 0$. Then let $v_n = e_1$ and $w_n = e_2$ and we see that $q$ takes on the required form. 
\bigskip\\
Now suppose that $\delta_q = 1$ and $k$ has characteristic $2$. In this case $\dim{V} = 2n + 1$ because $\delta_q$ is odd. Let $e_0 \in V^\perp$ be a generator of $V^\perp$ then $B(e_0, -) = 0$. Let $W$ be a complement to $V^\perp$ so $V = V^\perp \oplus W$. Then $B_q |_W$ is nondegenerate because since $W \cap V^\perp = (0)$ for nonzero $w \in W$ we see that $B_q(w, -) \neq 0$ so there exists some $v \in V$ such that $B_q(w, v) = 1$ but we can write $v = v_0 + w'$ for $v_0 \in V^\perp$ and $w' \in W$ so $B_q(w, w') = B_q(w, v) - B_q(w, v_0) = 1$ because $v_0 \in V^\perp$. Furthermore, $\dim{W} = 2n$ so we see that,
\[ q|_W(x_i w_i) = \sum_{i = 1}^n x_i x_{i + n} \]
and $B(e_0, -) = 0$ so we see that for any $w \in W$,
\[ q(\alpha e_0 + w) = \alpha^2 q(e_0) + q(w) + \alpha B(e_0, w) = \alpha^2 q(e_0) + q(w) \]
and therefore using that $k$ is algebraically closed to scale $q(e_0)$ if it is nonzero and setting $e_i = w_i$ for $i > 0$ we have either,
\[ q(x_i e_i) = x_0^2 + \sum_{i = 1}^n x_i x_{i+n} \quad \text{or} \quad q(x_i e_i) = \sum_{i = 1}^n x_i x_{i + n} \]
In the first case, $q|_{V^\perp} \neq 0$ while in the second case $q|_{V^\perp} = 0$. The reason that $\delta_q \neq 0$ in the first case is that $B_q(e_0, e_0) = 0$ even though $q(e_0) = 1$ because $B_q(e_0, e_0) = 2 q(e_0)$ so this is the ``problem'' in the odd dimensional case is that we cannot get rid of the ``problem'' that $B_q$ is automatically alternating by pairing up dimensions $x_i x_{i+n}$ so that $B(e_i, e_{i+n}) = 1$ when there are an odd number of dimensions.

\subsubsection{(iii)}

Let $q \neq 0$ and consider $V(q) \subset \P^{d-1}$. Giving $V$ a basis $e_1, \dots, e_n$ we see that for $\CC$,
\[ \pderiv{}{x_j} q \left( \sum_i x_i e_i \right) = \lim_{h \to 0} \frac{q(x_i e_i + h x_j) - q(x_i e_i)}{h}  =  \lim_{h \to 0} \frac{h^2 q(e_j) + B_q(x_i e_i, h e_j)}{h} = B_q(x_i e_i, e_j) = \sum_i x_i B(e_i, e_j) \]
Therefore, the Jacobian matrix has full rank at $v$ if and only if $B_q(v, -) \neq 0$ and thus $V(q)$ is smooth if and only if $\delta_q = 0$.
\bigskip\\
Now for any algebraically closed field $k$ with characteristic not $2$ we can write,
\[ q(x_i e_i) = \sum_{i,j} \tfrac{1}{2} x_i x_j B_q(e_i, e_j) \]
and therefore, 
\[ \pderiv{}{x_k} q \left( \sum_i x_i e_i \right) = \pderiv{}{x_k} \sum_{i,j} \tfrac{1}{2} x_i x_j B_q(e_i, e_j) = \sum_i B(x_i e_i, e_k) \]
giving the same answer as before and thus $V(q)$ is smooth if and only if $\delta_q = 0$.
\bigskip\\
Now let $k$ be an algebraically closed field with characteristic equal  to $2$. First suppose that $\delta_q \le 1$ and $q|_{V^\perp} \neq 0$ if $\delta_q = 1$. In the case $\delta_q = 0$ we have the form,
\[ q(x_i e_i) = \sum_{i = 1}^n x_i x_{i + n} \]
then,
\[ \pderiv{}{x_j} q(x_i e_i) = 
\begin{cases}
x_{j + n} & j \le n
\\
x_j & j > n
\end{cases} \]
and therefore these all vanish only at the origin so $V(q)$ is smooth.  If $\delta_q = 1$ and $q|_{V^\perp} \neq 0$ then,
\[ q(x_i e_i) =  x_0^2 + \sum_{i = 1}^n x_i x_{i + n} \]
then,
\[ \pderiv{}{x_j} q(x_i e_i) = 
\begin{cases}
0 & j = 0
\\
x_{j + n} & j \le n
\\
x_j & j > n
\end{cases} \]
and therefore this vanishes when all but $x_0$ vanish however if $q(x_i e_i) = x_0^2 = 0$ then all must vanish so $V(q)$ is actually smooth. However, if $\delta_q = 1$ and $q|_{V^\perp} = 0$ then we have the same derivative but $q(x_0) = 0$ and thus $V(q)$ is not smooth at $[1 : 0 : \dots : 0]$.
\bigskip\\
Converesely, if $V(q)$ is smooth we must show that $\delta_q \le 1$ and then we have already concluded that if $\delta_q = 1$ we must have $q|_{V^\perp} \neq 0$. Suppose that $\delta_q > 1$ then there are independent $e_1, e_2 \in V^\perp$ and thus,
\[ q(x_1 e_1 + x_2 e_2) = x_1^2 q(x_1) + x_2^2 q(x_2) \]
so by algebraic closure there is a nontrivial solution with $x_1, x_2$ not both zero. However, if we set all the other variables equal to zero it is clear that the derivative vanishes at $(x_1, x_2, 0, \dots, 0)$ and thus $[x_1 : x_2 : 0 : \dots : 0]$ is a singular point proving that $\delta_q \le 1$. 

\subsection{5}

\subsubsection{25.3 DO THIS!!!}

Let $A$ be a ring and $I \subset A$ an ideal. Let $\hat{A}$ be the $I$-adic completion of $A$. Let $D \in \Der{}{A}{A}$. Then consider $D(I^n)$. Because $I^n$ is generated by terms of the form $a_1 \cdots a_n$ with $a_1, \dots, a_n \in I$ it suffices to prove that $D(a_1 \cdots a_n) \in I^{n-1}$. Indeed, 
\[ D(a_1 \cdots a_n) = D(a_1) a_2 \cdots a_n + a_1 D(a_2 \cdots a_n) \]
Clearly, $a_2 \cdots a_n \in I^{n-1}$ and by induction we can assume that $D(a_2 \cdots a_n) \in I^{n-1}$ proving the claim. Finally, the base chase $D(I^1) \subset A$ is trivial.
\bigskip\\
For any multiplicative subset $S \subset A$ consider the formula,
\[ D(a/s) = (D(a)s - a D(s))/s^2 \]
I claim this is a derivation $A_S \to A_S$. First, we show that it is well-defined. If $a/s = b/u$ then there is some $t \in S$ such that $t(au - bs) = 0$ in $A$. Therefore,
\[ D(t) (au - bs) + t (D(a) u + a D(u) - D(b) s - b D(s)) = 0 \]
Therefore, 
\[ t^2 [u^2 (D(a) s - a D(s)) - s^2(D(b) u - b D(u))] = ust (D(a) u + a D(u) - D(b) s - b D(s)) \]

\subsubsection{25.4}

Let $k$ be a ring and $k'$ and $A$ two $k$-algebras and set $A' = k' \otimes_k A$. Let $S \subset A$ be a multiplicative set. Notice that, for any $A'$-module $M$,
\[ \Hom{A'}{\Omega_{A'/k'}}{M} = \Der{k'}{A'}{M} = \Der{k}{A}{M} = \Hom{A}{\Omega_{A/k}}{M} = \Hom{A'}{\Omega_{A/k} \otimes_A A'}{M} \]
by tensor-hom adjunction since this preserves being a derivation. Therefore, Yoneda, there is a natural isomorphism of $A'$-modules, $\Omega_{A'/k'} \cong \Omega_{A/k} \otimes_A A'$. Furthermore,
\[ \Omega_{A/k} \otimes_A A' = \Omega_{A/k} \otimes_A (A \otimes_k k') = (\Omega_{A/k} \otimes_A A) \otimes_k k' = \Omega_{A/k} \otimes_k k' \]
Likewise, for any $A_S$-module,
\begin{align*}
\Hom{A_S}{\Omega_{A_S/k}}{M} & = \Der{k}{A_S}{M} = \Der{k}{A}{M} = \Hom{A}{\Omega_{A/k}}{M} 
\\
& = \Hom{A_S}{\Omega_{A/k} \otimes_A A_S}{M}
\end{align*}
using that $M$ is an $A_S$-module. Therefore, by Yoneda, there is a natural isomorphism of $A_S$-modules,
\[ \Omega_{A_S / k} = \Omega_{A/k} \otimes_A A_S \]

\subsubsection{2}

Let $X$ be a finite type reduced $k$-scheme. Smoothness implies separability of each field map and thus $K / k$ must be separable for each function field.
\bigskip\\
Conversely, suppose that for each generic point $\xi \in X$ we have $\kappa(\xi) / k$ is separable. Then $\kappa(\xi) / k$ is separably generated meaning that $K = \kappa(\xi)$ is a separable algebraic extension of $k(x_1, \dots, x_n)$. There is an affine open of $X$ containing $\xi$ such that $\xi$ is the unique minimal prime (take the complement of the other components (finitely many because $X$ is noetherian) and shrink until affine) call this open $\Spec{A}$. Then $A$ is a domain and $K = \Frac{A}$. By shrinking $A$ further (i.e. by localizing at the denominators of $x_i$) we may assume that $x_i \in A$. Since the $x_i$ are algebraically independent over $k$ inside $K$ we see that $k[x_1, \dots, x_n] \subset A$ is a subring. Since $k[x_1, \dots, x_n]$ is smooth over $k$ it suffices to show that we can localize $A$ such that it is smooth over $k[x_1, \dots, x_n]$. 


\subsubsection{3}

Let $X$ be smooth on a dense open $U \subset X$. To show that $X(k^{\sep})$ is dense in in $X_{k^{\sep}}$ it suffices to base change to the case $k = k^{\sep}$. To show  that $X(k)$ is dense, we just need to show that on ever affine open $V \subset X$ there is some rational point. Furthermore, since $U \cap V$ is nonempty since $U$ is dense it suffices to show that $U \cap V$ contains a rational point so we can reduce to the case of a smooth irreducible (thus integral) affine scheme $\Spec{A}$ over $k$. Let $K = \Spec{A}$ then $K / k$ is separable so there a transcendence basis $x_1, \dots, x_n \in K$ over $k$ such that $K$ is a separable finite extension of $k(x_1, \dots, x_n)$ (finiteness comes from the fact that $K$ is finitely-generated as a field extension so any algebraic extension is automatically finite). Thus, by the primitive element theorem, $K = k(x_1, \dots, x_n)(t)$ where $t \in K$ is separable and algebraic over $L = k(x_1, \dots, x_n)$ thus satisfying some separable polynomial $f \in B[t]$ (meaning $f'(t) \neq 0$) where $B = k[x_1, \dots, x_n]$ (we can clear denominators from $f$ if necessary).
\bigskip\\
Clearing denominators, we can assume that $t$ is integral over $k[x_1, \dots, x_n]$ and that $t \in A$. Let $B = k[x_1, \dots, x_n]$. I claim that $A_q = B_q[t]$ for some $q \in B$.
\bigskip\\
Indeed, because $A$ is finite type over $k$, it is finite type over $B$ so we can write $A = B[y_1, \dots, y_r]$ for some $y_i \in A \subset K$. So each $y_i$ is a polynomial in $t$ with coefficients in $L$ (since $K = L(t)$) then we can take $q$ to be the product of these denominators in $B$ and thus each $y_i \in B_q[t]$ but $A_q \supset B_q[t]$ and thus $A_q = B_q[t]$.
\bigskip\\
Therefore, $A$ is an integral extension of $B_q$ with $A \cong B_q[x]/(f)$. Furthermore, $f'(t) \neq 0$ so $\frac{1}{f'(t)}$ exists in $K$ and thus is a polynomial in $t$ with coefficients in $L$. Localizing at the denominators of these coefficients, we shrink to $B' \subset A'$ such that $A' = B'[x]/(f)$ with $f'$ a unit in $A'$ which is standard \etale and finite. This gives a map $\Spec{A'} \to U \subset \A^n_k$ which is finite \etale and dominant and thus surjective. Because $U \subset \A^n_k$ is a dense open, there exists some rational point $x \in U$. Then $\Spec{A'} \to \Spec{B'}$ is surjective so there is some point $y \in \Spec{A'}$ with $y \mapsto x$. Thus $k = \kappa(x) \embed \kappa(y)$ is a separable extension since $B' \subset A'$ is \etale. Since $k$ is separably closed we see that $\kappa(y) = \kappa(x) = k$ and thus $y \in \Spec{A'}$ is a $k$-rational point.

\section{Homework 3}

\subsection{1}


Let $\PGL_n = \Spec{k[x_{ij}]_{(\det)}}$. For any homogeneous element $f \in k[x_{ij}]$ we know that the open subscheme $D_+(f) = \{ f \neq 0 \} \subset \P^{n^2 - 1}$ is isomorphic to the affine scheme $\Spec{k[x_{ij}]_{(f)}}$ and thus,
\[ \PGL_n = \Spec{k[x_{ij}]_{(\det)}} \cong D_+(\det) = \{ \det \neq 0 \} \subset \P^{n^2 - 1} \]
The map $k[x_{ij}]_{(\det)} \to k[x_{ij}]_{\det}$ gives a natural map $\GL_n \to \PGL_n$ and therefore a natural map $\GL_n(R) \to \PGL_n(R)$ sending a set of elements $a_{ij} \in R$ with $\det{a_{ij}} \in R^\times$ to the function $\varphi : k[x_{ij}]_{(\det)} \to R$ sending $x_{ij} \mapsto a_{ij}$. Notice that, since elements of $k[x_{ij}]_{(\det)}$ are homogeneous fractions, the $R^\times$-orbits are sent to a single map $k[x_{ij}]_{(\det)} \to R$. Furthermore, suppose that $a_{ij} \in R$ is an invertible matrix inducing the trivial map $\varphi : k[x_{ij}]_{(\det)} \to R$ which sends $\frac{x_{ii}^n}{\det} \mapsto 1$ and all other monomials to zero. Since $a_{ii}^n = \det(a_{ij})$ we see that $a_{ii} = \lambda_i$ is an $n$-th root of $d = \det(a_{ij})$ and thus a unit. Furthermore, $a_{11}^{n-1} a_{ij} = 0$ so because $a_{11}$ is a unit we see that $a_{ij} = 0$. Then $d = a_{11} \cdots a_{nn} = \lambda_1 \cdots \lambda_n$. Furthermore, $a_{ii} a_{11}^{n-1} = d$ so $\lambda_i \lambda_1^n = \lambda_1 d$ but $\lambda_1^n = d$ so since $d \in R^\times$ we see that $\lambda_i = \lambda_1$ and thus they are all equal so $a_{ij} = \lambda \delta_{ij}$ proving that the kernel $\ker{(\GL_n(R) \to \PGL_n(R))} = R^\times$ so the map $\GL_n(R)/R^\times \to \PGL_n(R)$ is injective.

\subsection{2}

If we fix a variable $x_{ab}$ then I claim that the natural map,
\[ k[x_{ij}]_{(x_{ab} \det)} \to k[x_{ij}, \det{}^{-1}]/(x_{ab} - 1) \]
is an isomorphism. First, notice that this map is well-defined because $x_{ab}$ us a unit in the target. Furthermore, notice that,
\[ \frac{x_{ij} \det}{x_{ab} \det} = \frac{x_{ij}}{x_{ab}} \mapsto x_{ij} \]
and this is a degree zero homogeneous term so the map is surjective. Furthermore, no nonzero homogeneous term can possibily be a multiple of $x_{ab} - 1$ so the preimage of $(x_{ab} - 1)$ is zero and thus the map is injective. 
\bigskip\\
Now for a $k$-algebra map $\varphi : k[x_{ij}]_{(\det)} \to R$ let $M$ be the set of monomials of degree $n$ in the variables $x_{ij}$. For $m \in M$ let $f_m = \varphi(\frac{m}{\det}) \in R$. Because,
\[ \sum_{m \in M} \mathrm{sign}(m) \cdot m = \det \]
we see that,
\[ \sum_{m \in M} \mathrm{sign}(m) \cdot \frac{m}{\det} = 1 \]
and therefore,
\[ \sum_{m \in M} \mathrm{sign}(m) \cdot f_m = 1 \]
in $R$ proving that $(f_m)_{m \in M}$ generate the unit ideal and thus $\{ \Spec{R_{f_m}} \}$ form an affine open cover of $\Spec{R}$. Furthermore, for each $m$ choose some variable $x_{ab}$ that appears in $m = x_{ab} z$. Then the map $k[x_{ij}]_{(\det)} \to R \to R_{f_m}$ factors through $k[x_{ij}]_{(\det)} \to k[x_{ij}]_{(x_{ab} \det)}$ via sending,
\[ \frac{g}{x_{ab}^r \det^r} = \frac{g z^r}{x_{ab}^r z^r \det^r} = \frac{g z^r}{m^r \det^r} \mapsto f_m^{-r} \cdot \varphi \left( \frac{g z^r}{\det^{2r}} \right) \]
which makes sense because $\deg{(g z^r)} = r + n r + (n-1)r = 2nr$. It is straightforward to check that this is a ring map. Therefore, we get a diagram,
\begin{center}
\begin{tikzcd}
k[x_{ij}, \det^{-1}] \arrow[r] & k[x_{ij}, \det^{-1}]/(x_{ab} - 1) \arrow[r, equals] & k[x_{ij}]_{(x_{ab} \det)} \arrow[r, dashed] &  R_{f_m}
\\
& k[x_{ij}]_{(\det)} \arrow[lu] \arrow[ru] \arrow[rr, "\varphi"] & & R \arrow[u] 
\end{tikzcd}
\end{center}
Therefore, $\varphi : k[x_{ij}]_{(\det)} \to R_{f_m}$ lifts to $k[x_{ij}, \det^{-1}]$ which implies that $\varphi |_{R_{f_m}} \in \GL_n(R)/R^\times$. Therefore, the map of sheaves $\GL_n / \Gm \to \PGL_n$ is surjective and also injective (by the previous part) and therefore an isomorphism of sheaves of groups. Therefore $\PGL_n$ is isomorphic to the sheafification of the presheaf $R \mapsto \GL_n(R)/R^\times$. Therefore, the $k$-map $\GL_n \to \PGL_n$ uniquely determines the $k$-group structure on $\PGL_n$ by the property that $\GL_n \to \PGL_n$ is a $k$-homomorphism because the group structure is determined by the group structure on the functor and the group structue on $\PGL_n(R)$ is determined by the group structue on $\PGL_n(R_{f_m})$ but $\GL_n(R_{f_m}) \onto \PGL_n(R_{f_m})$ so if this is a $k$-homomorphism of groups it uniquely determines the group structue on $\PGL_n(R_{f_m})$. 

\subsubsection{3}

Let $R$ be local with maximal ideal $\m$. If $U \subset \Spec{R}$ is an open continaing the unique closed point $\m$ then $U = \Spec{R}$ since $U^C$ is closed and thus either is empty or $\m \in U^C$. Therefore, for each $\varphi \in \PGL_n(R)$ some element of the open cover from (2) must be all of $\Spec{R}$ and therefore $\varphi$ lifts to $\GL_n(R)$ so the map $\GL_n(R) / R^\times \iso \PGL_n(R)$ is an isomorphism.
\bigskip\\
Let $R$ be a Dedekind domain and $I \in \Pic{R}$ be a non-trivial 2-torsion element. Then I claim that $I \oplus I \cong R^2$. I will use the following lemma.

\begin{lemma}
Every ideal $I \subset R$ in a Dedekind domain is generated by at most two elements. 
\end{lemma}

\begin{proof}
Take a nonzero $x \in I$ then it suffices to show that $I/(x)$ is a principal $R/(x)$-module. However, I claim that $R/(x)$ is a PRI. Notice that $\dim{R/(x)} = 0$ and $R$ is noetherian so $R/(x)$ is an Artinian ring. Thus, there are finitely many prime ideals $\p_i \supset (x)$ and we get,
\[ R / (x) = \prod_{i = 1}^r \left( \frac{A}{(x)} \right)_{\p_i} = \prod_{i = 1}^r A_{\p_i}/(x) \]
However, each $A_{\p_i}$ is a DVR so we see that the ideal $I/(x)$ is isomorphic to,
\[ I/(x) = \prod_{i = 1}^r I A_{\p_i} / (x) \]
but since $A_{\p_i}$ is a DVR we see that $I A_{\p_i}$ is principal and thus $I/(x)$ is principal.
\end{proof}
\noindent
Now consider the exact sequence,
\begin{center}
\begin{tikzcd}
0 \arrow[r] & J \arrow[r] & R^2 \arrow[r] & I \arrow[r] & 0
\end{tikzcd}
\end{center}
where $J = \ker{(R^2 \to I)}$. Because $I$ is invertible and thus locally free, it is projective so the sequence splits and thus $R^2 \cong I \oplus J$. In particular, $J$ is also projective and locally free of rank $1$ and thus invertible. Furthermore, taking top exterior powers we see that $I \otimes J \cong R$ so $I$ and $J$ are inverse in the Picard group. In particular, if $I$ is 2-torsion in $\Pic{R}$ then $J \cong I$ because,
\[ J \cong J \otimes (I \otimes I) = (J \otimes I) \otimes I \cong R \otimes I = I \]
Therefore, in our case $I \oplus I \cong R^2$. 

\begin{rmk}
The corresponding statement is false for smooth projective curves (i.e. the phenomenon only appears in the affine case). For example, let $E$ be an elliptic curve over $\CC$ and $p \in E$ a 2-torsion point. Then let $\L = \struct{E}(p)$. We know that $\L \otimes \L = \struct{E}(2p) \cong \struct{E}$ so $\L \in \Pic{E}$ is 2-torsion. However, 
\[ \dim H^0(E, \L \oplus \L) = 2 \dim H^0(E, \L) = 4 \]
by Riemann Roch since $\deg{\L} = 1$. However, $H^0(E, \struct{E} \oplus \struct{E}) = \CC^{\oplus 2}$ so we see that,
\[ \L \oplus \L \not\cong \struct{E} \oplus \struct{E} \]
\end{rmk}
\noindent
Now I want to associate the isomorphism $\varphi : R^2 \to I \oplus I$ with an element of $\PGL_2(R)$ that does not lift globally to $\GL_2(R)$. Notice that because $I$ is flat, $I \otimes I \to I^2$ is an isomorphism and thus we can fix an isomorphism $\psi : I^2 \to R$. The map $\varphi$ is given by a matrix $a_{ij} \in I$ so for the degree 2 monomials $m_{ij,k\ell} = a_{ij} a_{k \ell} \in I^2$ we get $f_{ij,k\ell} = \psi(m_{ij,k\ell}) \in R$ defining a map $\tilde{\varphi} : k[x_{ij}]_{(\det)} \to R$ sending,
\[ \frac{m}{\det} \mapsto \frac{f_m}{f_{11,22} - f_{12,21}} \]
However, I claim that $\tilde{\varphi}$ recovers $I$ and thus if $\tilde{\varphi}$ lifts to a matrix $a_{ij} \in R$ then we would get $I \cong R$ which cannot happen if $I$ is not principal. Consider the $k[x_{ij}]_{(\det)}$-module $M = \left( k[x_{ij}]_{\det} \right)_1$ the degree one part (this is $\struct{}(1)$). Then I claim that $I \cong R \otimes_{\bar{\varphi}} M$ as $R$-modules. Indeed, we send $1 \otimes x_{ij} \mapsto a_{ij} \in I$ which is surjective because $\varphi$ is surjective and injective because $M$ is invertible so $ R \otimes_{\bar{\varphi}} M$ is also invertible and thus the map is an isomorphism on stalks (surjective maps of locally free modules of the same rank are injective). However, if $k[x_{ij}]_{(\det)} \to R$ factors through $k[x_{ij}]_{(\det)} \to k[x_{ij}]_{\det}$ then,
\[ R \otimes_{\varphi} M = R \otimes_{k[x_{ij}]_{\det}} k[x_{ij}]_{\det} \otimes_{k[x_{ij}]_{(\det)}} M \cong R \]
because $k[x_{ij}]_{\det} \otimes M \cong k[x_{ij}]_{\det}$ via,
\[ 1 \mapsto \frac{x_{11}}{\det} \otimes x_{22} - \frac{x_{12}}{\det} \otimes x_{21} \]
\bigskip\\
We can explain this example more generally. Let $X$ be a $k$-scheme. The exact sequence,
\begin{center}
\begin{tikzcd}
1 \arrow[r] & \Gm \arrow[r] & \GL_n \arrow[r] & \PGL_n \arrow[r] & 1
\end{tikzcd}
\end{center}
gives rise to an exact sequence,
\begin{center}
\begin{tikzcd}
1 \arrow[r] & \Gm(X) \arrow[r] & \GL_n(X) \arrow[r] & \PGL_n(X) \arrow[r] & H^1(X, \Gm) \arrow[r] & H^1(X, \GL_n)
\end{tikzcd}
\end{center}
Therefore, an element $\varphi \in \PGL_n(X)$ is in the image of $\GL_n(X) \to \PGL_n(X)$ (furthermore $\GL_n(X) = \GL_n(\Gamma(X, \struct{X}))$ are everywhere invertible matrices of functions) if and only if its image in $H^1(X, \Gm)$ is the trivial bundle (recall that $H^1(X, \GL_n)$ classifies $\GL_n$-torsors or equivalently rank $n$ vector bundles). Furthermore, a line bundle $\L \in H^1(X, \Gm) = \Pic{X}$ comes from $\PGL_n(X)$ if and only if $\L \mapsto \L^{\oplus n} \in H^1(X, \GL_n)$ is the trivial bundle.
\bigskip\\
We can explain these constructions a bit more explicitly. A morphism $\varphi : X \to \PGL_n \subset \P^{n^2 - 1}$ is given by a line bundle $\L = \varphi^* \struct{\P}(1)$ and a matrix of $n^2$ sections $s_{ij} \in \Gamma(X, \L)$ that globally generate it. Furthermore, that the morphism lands inside the locus $\PGL_n = D_+(\det) \subset \P^{n^2}$ is exactly the condition that $\varphi^{-1} \det$ is nonvaishing as a section of $\varphi^* \struct{\P}(n) = \L^{\otimes n}$. Therefore, $\varphi^{-1} \det : \struct{X} \to \L^{\otimes n}$ is a trivialization so $\L$ is an $n$-torsion line bundle. However, more is true because $\varphi^{-1} \det$ is not an arbitrary section. Indeed, $\varphi^{-1} \det = \det{\varphi^{-1} x_{ij}} = \det{s_{ij}}$. Instead of thinking of the matrix of sections $s_{ij} : \struct{X} \to \L$ as a map $\struct{X}^{\oplus n^2} \onto \L$ we may think of it as a map $(s_{ij}) : \struct{X}^{\oplus n} \to \L^{\oplus n}$ determined by the matrix. Since $\det{s_{ij}} : \struct{X} = \bigwedge^n \struct{X}^{\oplus n} \to \bigwedge^n \L^{\oplus n} = \L$ is nowhere vanishing this map is an isomorphism on fibers and thus an isomorphism of vector bundles so $\L^{\oplus n}$ is trivialized by the sections. Conversely, given $\L \in \Pic{X}$ such that $\L^{\oplus n}$ is trivial, choose a trivialization $(s_{ij}) : \struct{X}^{\oplus n} \iso \L^{\oplus n}$ given by a matrix of sections $s_{ij}$ then $\det{s_{ij}}$ is nonvanishing so $s_{ij}$ are not all simultaneously vanishing (indeed there are always at least $n$ nonzero sections at each point such that the determinant is nonzero) so $s_{ij}$ and $\L$ define a map $\varphi : X \to \P^{n^2}$ that factors through $\PGL_n$ because $\det{s_{ij}}$ is nowhere vanishing so $\varphi^{-1} \det = \det{s_{ij}}$ is nowhere vanishing. Finally, given a map $\varphi : X \to \PGL_n$ we get a line bundle $\L = \varphi^* \struct{\P}(1) \in \Pic{X} = H^1(X, \Gm)$. If $\L \cong \struct{X}$ then up to a choise of isomorphism $\psi : \struct{X} \iso \L$ we can choose our sections as $s_{ij} \in \Gamma(X, \struct{X})$ and $\det{s_{ij}}$ is nowhere vanishing so $(s_{ij}) \in \GL_n(X)$ forms an everywhere invertible matrix of global functions so $\varphi$ is induced via $\GL_n \to \PGL_n$. Whenever $\varphi$ comes from $X \to \GL_n \to \PGL_n$ then $\L = \varphi^* \struct{\P}(1)$ is trivial because under $\GL_n \to \PGL_n$ we see that $\struct{\P}(1)$ gets a nonvanishing global section from the cofactor matrix.

\subsection{2}

\subsubsection{1}

Let $f : G' \to G$ be a $k$-homomorphism of $k$-groups.
By definition,
\begin{align*}
(\ker{f})(R) &= \Hom{k}{\Spec{R}}{\ker{f}} = \Hom{k}{\Spec{R}}{G'} \times_{\Hom{k}{\Spec{R}}{G}} \Hom{k}{\Spec{R}}{\Spec{k}} 
\\
&= G'(R) \times_{G(R)} e = \ker{(G'(R) \to G(R))}
\end{align*}
just by the universal limit property of the pullback. Therefore, $\ker{f}$ is normal because normality can be checked on the sheaf values.

\subsubsection{2}

We proved that the diagram,
\begin{center}
\begin{tikzcd}
k[x_{ij}, \det^{-1}] \arrow[r, two heads] & k[x_{ij}, \det^{-1}]/(x_{ab} - 1) \arrow[r, equals] & k[x_{ij}]_{(x_{ab} \det)} 
\\
& k[x_{ij}]_{(\det)} \arrow[lu] \arrow[ru] 
\end{tikzcd}
\end{center}
commutes and therefore,
\begin{center}
\begin{tikzcd}
\GL_n \arrow[rd] & & U_{\alpha \beta} \arrow[ll, hook, "\text{closed}"'] \arrow[ld, hook, "\text{open}"]
\\
& \PGL_n
\end{tikzcd}
\end{center}
Since the $U_{\alpha \beta}$ cover $\PGL_n$ because the $x_{\alpha \beta}$ together generate the irrelevant ideal, we see that $\GL_n \to \PGL_n$ is surjective since $U_{\alpha \beta} \to \GL_n \to \PGL_n$ is surjective onto $U_{\alpha \beta}$. 
\bigskip\\
Consider the diagonal embedding $\Gm \to \GL_n$. Clearly $\Gm \to \GL_n \to \PGL_n$ is trivial because $R^\times \to \GL_n(R) \to \PGL_n(R)$ is trivial. Thus $\Gm \to \GL_n$ factors as $\Gm \to \ker{f} \to \GL_n(R)$. Furthermore, when $R$ is local we know that $\GL_n(R) \to \PGL_n(R)$ has kernel exactly $R^\times$ embedded diagonally so $\Gm(R) \to (\ker{f})(R)$ is an isomorphism on the stalks and thus $\Gm \to \ker{f}$ is an isomorphism.

\subsubsection{3} 


Let $\mu_n = \ker{(\Gm \xrightarrow{n} \Gm)} = \Spec{k[t]/(t^n - 1)}$. Then $\mu_n(R) = \ker{(R^\times \xrightarrow{n} R^\times)}$ is exactly the group of $n^{\text{th}}$ roots of unity of $R$. Consider the map $\SL_n \to \GL_n \to \PGL_n$ where $\SL_n = \ker{\det}$. Then there is a diagram,
\begin{center}
\begin{tikzcd}
\mu_n \pullback \arrow[d] \arrow[r] & \Gm \pullback \arrow[d] \arrow[r] & \Spec{k} \arrow[d]
\\
\SL_n \pullback \arrow[d] \arrow[r] & \GL_n \arrow[d, "\det"] \arrow[d] \arrow[r] & \PGL_n
\\
\Spec{k} \arrow[r] & \Gm
\end{tikzcd}
\end{center}
Since the small squares are all Cartesian thus the two rectangles are also Cartesian and thus $\mu_n = \ker{(\Gm \to \GL_n \to \Gm)}$ is also the kernel of $\SL_n \to \GL_n \to \PGL_n$. Furthermore, consider the diagram of rings,
\begin{center}
\begin{tikzcd}
k[x_{ij}]/(\det - 1) \arrow[r, two heads] & k[x_{ij}]_{x_{ab}}/(\det - 1)
\\
k[x_{ij}, \det^{-1}] \arrow[u]  & k[x_{ij}]_{(x_{ab} \det)}  \arrow[u] 
\\
& k[x_{ij}]_{(\det)} \arrow[lu] \arrow[u] 
\end{tikzcd}
\end{center}
Where the natural map $k[x_{ij}]_{(\det)} \to k[x_{ij}]_{x_{ab}}/(\det - 1)$ factors through $k[x_{ij}]_{(x_{ab}\det)}$ because $x_{ab} \det$ is a unit in the target. However, notice that we cannot choose a ``section'' $k[x_{ij}, \det^{-1}] \to k[x_{ij}]_{(x_{ab} \det)}$ making the diagram commute (I call this a section because it corresponds to a closed embedding $U_{ab} \embed \GL_n$) because this would correspond to a choice for a value of $x_{ab}$ which remains somewhat undetermined in $k[x_{ij}]_{x_{ab}}/(\det - 1)$. Then I claim that $k[x_{ij}]_{(x_{ab} \det)} \to k[x_{ij}]_{x_{ab}}/(\det - 1)$ is injective and finite meaning that the corresponding diagram of schemes,
\begin{center}
\begin{tikzcd}
\SL_n \arrow[d] & V_{ab} \arrow[l] \arrow[d]
\\
\GL_n \arrow[dr] & U_{ab} \arrow[d, hook]
\\
& \PGL_n
\end{tikzcd}
\end{center}
and since $U_{ab}$ cover $\PGL_n$ and $V_{ab} \onto U_{ab}$ is finite and dominant and thus surjective we see that $\SL_n \to \PGL_n$ is surjective. 
\bigskip\\
Wait, it is actually easier to justify the finitness on the whole map and ignore this cover. I claim that $k[x_{ij}]_{(\det)} \to k[x_{ij}]/(\det - 1)$ is finite and injective. The map sends,
\[ \frac{f}{\det^r} \mapsto f \]
for any homogeneous polynomial $f$ of degree $nr$. This is clearly injective. Furthermore, for any $x_{ij}$ notice that $x_{ij}^n - \varphi(\frac{x_{ij}^n}{\det}) = 0$ so $x_{ij}$ satisfies a monic polynomial over $k[x_{ij}]_{(\det)}$ and thus the extension must be finite. Therefore, $\SL_n \to \PGL_n$ is finite and dominant and thus surjective with finite fibers. Indeed, we showed that $\mu_n = \ker{(\SL_n \to \PGL_n)}$ is indeed finite.

\subsection{3 DO THIS}

Let $G$ be a $k$-group of finite type with an action on a $k$-scheme $V$ of finite type. Assume that $W \subset V$ is a closed subscheme that is geometrically reduced and separated over $k$.

\subsubsection{1}

From HW2 Exercise 5 we know that $W$ is smooth on a dense open and thus $W(k^{\sep})$ if and only if its function fields are separable over $k$. However, since $W$ is geometrically reduced, every residue field is separable over $k$.
\bigskip\\
We can also argue as follows. If $k = k_s$ then $W_{\bar{k}} \to W$ is a homeomorphism so we are done. In general, a smooth dense open will be defined over some finite Galois extension. Applying the automorphisms to the dense open, their intersection (because of finiteness) is also a dense open which descends to a smooth dense open by Galois descent because it is Galois-invariant. 

\subsubsection{2}

For each $v \in V(k)$ let $\alpha_v : G \to V$ be the orbit map $g \mapsto g \cdot v$. Define $Z_G(v) = \alpha_v^{-1}(v)$. By the definition of a limit,
\[ Z_G(v)(S) = \Hom{k}{S}{Z_G(v)} = \Hom{k}{S}{G} \times_{\Hom{k}{S}{V}} \Hom{k}{S}{v} = G(S) \times_{V(S)} v_S \]
which is the set of $g \in G(S)$ such that $g \cdot v_S = v_S$ in $V(S)$.

\subsubsection{3}

If $k = k^{\sep}$. Notice that, for any $k$-scheme $S$,
\[ \underline{Z}_G(W)(S) = \{g \in G(S) \mid  g \text{ acts trivially on } W_S \} \subset \{ g \in G(S) \mid \forall w \in W(S) : g \cdot w = w \} \subset \bigcap_{w \in W(k)} Z_G(w)(S) \]
Consider the parallel maps $g, \id : W_S \to W_S$. Let $Z \subset W_S$ be the equalizer of $g, \id$ which is a closed subscheme because $W_S \to S$ is separated since $Z \to W_S$ is the pullback of $\Delta_{W_S/S} : W_S \to W_S \times_S W_S$ along $(g, \id) : W_S \to W_S \times_S W_S$. Suppose that $g$ and $\id$ agree on $W(k) \subset W(S)$. I claim that the preimage $Q \subset W_S$ of $W(k)$ under $W_S \to W$ is dense. Indeed any open $U \subset W_S$ contains a small affine open $V_1 \times_k V_2$ with $V_1 \subset W$ affine open and $V_2 \subset S$ affine open. Then $V_1$ contains a $k$-point and $V_2$ contains some $k'$-point so while $V_1 \times_k V_2$ may not contain a $k$-point it contains a point mapping into $W(k)$. Furthermore, for and $q \in Q$ consider $x = \pi(q) \in W(k)$ and $\Spec{\kappa(q)} \to S$ its projection to $S$. Then the diagram,
\begin{center}
\begin{tikzcd}
S \arrow[rd] \arrow[rdd, "\id"'] \arrow[rr] & & \Spec{k} \arrow[d, "x"]
\\
& W_S \pullback \arrow[d] \arrow[r] & W \arrow[d]
\\
\Spec{\kappa(q)} \arrow[uu] \arrow[r] & S \arrow[r] & \Spec{k}
\end{tikzcd}
\end{center}
shows that $\Spec{\kappa(q)} \to W_S$ factors through $S \to W_S$ coming from $W(k) \subset W(S)$. Since, by hypothesis, $S \to W_S \xrightarrow{g,\id} W_S$ give the same map we see that $g, \id$ agree at each $q \in Q$ and therefore $Z$ is a dense closed subset.

\begin{rmk}
Note: unless $W_S$ is reduced, this does not imply that $Z = W_S$ \textit{as subschemes.} For example, the maps $\Spec{k[\epsilon]/(\epsilon^2)} \to \Spec{k[x]}$ sending $x \mapsto 0$ and $x \mapsto \epsilon$ respectively have scheme theoretic equalizer $\Spec{k} \to \Spec{k[\epsilon]}$ which is the entire subset (showing that the maps on topological spaces act identically) but this is not equal to $\Spec{k[\epsilon]/(\epsilon^2)}$ as a subscheme.
\end{rmk}

(HOW TO CONCLUDE)
\bigskip\\
Now for general $k$

\subsection{4}

If $G$ is a smooth algebraic $k$-group then let $G \acts G$ by conjugation and set $Z_G = Z_G(G)$. Since $G$ is separated, this definition makes sense.

\subsubsection{1 FINISH THIS!!}

Let $G = \GL_n$ and $T \subset G$ the diagonal torus. It is clear that $T \subset Z_G(T)$ by observing that $T(S)$ acts trivially on $T(S)$ for any $S$ because $T$ is abelian. Then, 
\[ T(S) \subset Z_G(T) = \underline{Z}_G(T)(S) = T(S) \]
because it is an elementary fact that the stabilizer of $R^\times \subset \GL_n(R)$ under conjugation is $R^\times$. Therefore, $Z_G(T) = T$.
\bigskip\\
Now consider the case $G = \PGL_n$ and $T$ is the diagonal torus. Because $\pi : \GL_n \to \PGL_n$ is surjective, we get an action of $\GL_n$ on $\PGL_n$ inducing the full conjugation action of $\PGL_n$ on itself. Again because $T$ is abelian, clearly $T \subset Z_G(T)$ so it suffices to show that $\pi^{-1}(T) = \pi^{-1}(Z_G(T))$. 
 
\subsubsection{2}

We showed that $Z_G(T) = T$ where $T$ is the diagonal torus. For any ring $R$, the even permutation matrices lie in $\SL_n(R) \subset \GL_n(R)$ and thus (because even permutations are transitive) $G(R)$ fixes exactly the scalar matrices in $T(R)$. Therefore, $Z_{\GL_n} = \Gm$ is the torus of diagonal scalar matrices. Furthermore, $Z_{\SL_n} = \ker{\det : T \to T} = \mu_n$ embedded diagonally. Now, $Z_{\PGL_n} = 1$ (WHY?)

\subsubsection{3}

Let $V \subset G$ be a smooth closed subscheme and $k'/k$ an extension of fields. For any $k'$-algebra $S$,
\begin{align*}
\Hom{k'}{S}{Z_{G_{k'}}}{V_{k'}} = \underline{Z}_{G_{k'}}(V_{k'})(S) & = \{ g \in G_{k'}(S) \mid g \text{ acts trivially on } (V_{k'})_{S} \} 
\\
& = \{ g \in G(S) \mid g \text{ acts trivially on } V_{S} \}  = \underline{Z}_{G}(V)(S) 
\\
& = \Hom{k}{S}{Z_G(V)} = \Hom{k'}{S}{(Z_G(V))_{k'}} 
\end{align*}
and thus $Z_{G_{k'}}(V_{k'}) = Z_G(V)_{k'}$ naturally. The exact same argument works for $N_G$.

\section{Homework 4}

\subsection{1}

Let $T \subset \Sp_{2n}$ be the torus defined by matrices 
$\begin{pmatrix}
t & 0
\\
0 & t^{-1} 
\end{pmatrix}$ for diagonal $t \in \GL_{n}$. To be precise it represents the functor sending $R$ to such matrices defined over $R$. To show that $Z_{\Sp_{2n}}(T) = T$. It is clear that $T$ is abelian and therefore $T$ acts trivially on itself by conjugation. Therefore,
\[ T \subset Z_{\Sp_{2n}}(T) \]
Now we need to show that they are equal. It suffices to prove that their $R$-points agree for all $R$ (so they define the same group functor). Ineed, if a matrix $M \in \Sp_{2n}(R)$ fixes $T_R$ then it must fix all $R$-points of $T_R$ but this already implies it must be diagonal (the only $R$-matrices commuting with every diagonal matrix is itself diagonal) and the only diagonal matrices in $\Sp_{2n}$ are in $T$. Thus $T(R) = Z_{\Sp_{2n}}(T)(R)$ for all $R$ so we see that $T = Z_{\Sp_{2n}}(T)$. 
\bigskip\\
We can give a more conceptual proof as follows. Let $T' \subset \GL_{2n}$ be the maximal torus of diagonal elements. It is clear that $T$ is the preimage of $T'$ under the canonical map $\pi : \Sp_{2n} \to \GL_{2n}$. Furthermore, $Z_{\GL}(T') = T'$ and $\pi^{-1}(Z_{\GL_{2n}}(T')) \subset Z_{\Sp_{2n}}(T)$ because any element mapping to an element that fixes $T$ must fix its preimage proving that $T' \subset Z_{\Sp_{2n}}(T)$.
\bigskip\\
Now we need to consider $Z_{\Sp_{2n}}$. Because $Z_{\Sp_{2n}}(T) = T$ we know that $Z_{\Sp_{2n}} \subset T$ but must preserve not just $T$ but all of $\Sp_{2n}$ under conjugation. Notice that $\GL_n \subset \Sp_{2n}$ and any element of the center must also lie in the center of $\GL_n$ which is $\Gm$ (embeded diagonally) so we get $Z_{\Sp_{2n}} \subset \Gm \cap T = \mu_2$. This is because the embedding sends $t \in \Gm(R)$ to the block matrix 
\[ \begin{pmatrix}
t I & 0
\\
0 & t^{-1} I
\end{pmatrix} \] 
however to commute with,
\[ \begin{pmatrix}
0 & 1
\\
-1 & 0
\end{pmatrix} \]
we must have $t = t^{-1}$ and thus $t^2 = 1$ provign that $Z_{\Sp_{2n}}(R) \subset \mu_2(R)$.
Finally, it is obvious that $\mu_2 \subset Z_{\Sp_{2n}}$ because these are scalar matrices and thus $Z_{\Sp_{2n}} = \mu_2$.

\subsection{2}

Let $G = \PGL_n$. To prove that $G$ is smooth via the infinitessimal lifting criterion, we need to consider an extension of Artin rings $R \onto R'$ by an ideal $J \subset R$ of square zero. Then we require that $G(R) \to G(R')$ is surjective. Because $R$ and $R'$ are local rings we proved that $G(R) = \GL_n(R)/R^\times$. Thus the map $G(R) \to G(R/J)$ is surjective simply because $\GL_n(R) \to \GL_n(R/J)$ is surjective since any lift of an univertible matrix will have determinant not in $\m \subset R$ else it will map into the maximal ideal of $R/J$ ($R \to R/J$ is a local ring map) and thus not be invertible. 
\bigskip\\
For connectedness, we need to find a smooth, geometrically connected $k$-scheme $X$ with a $G$-action that is transitive on $\bar{k}$-points and has connected stabilizer. In our case we choose $X = \P^n_k$ which is clearly geometrically connected and the standard action $\PGL_{n+1}$ on $\P^n$. This is transitive on $\bar{k}$-points by inspection. Now the stabilizer of a point is a copy of $\GL_n \times \A^{n-1} \subset \PGL_{n+1}$ because it is the quotient of the stabilizer of a single line in $\A^{n+1}$ which is scheme theoretically isomorphic to $\Gm \times \GL_n \times \A^{n} \subset \GL_{n+1}$ because it is the set of matrices whose first column is some scaling of the first basis vector. Then under the quotient we get $\GL_n \times \A^{n} \subset \PGL_{n+1}$ which is connected therefore we have shown that $\PGL_{n+1}$ is connected. The case $\PGL_1$ is trivial (litterally the trivial group).
\bigskip\\
It would probably be better to act on the projective flag variety but it works basically as in the affine case.

\subsection{3}

Let $X$ be a scheme over a field $k$ and $x \in X(k)$. 

\begin{enumerate}
\item For $c \in k$ consider the $k$-algebra endomorphism $k[\epsilon]$ sending $\epsilon \mapsto c \epsilon$. This gives an endomorphism $X(k[\epsilon]) \to X(k[\epsilon])$ over $X(k)$. Recall that a map $\eta : \Spec{k[\epsilon]} \to X$ over $x \in X(k)$ factors through $\Spec{k[\epsilon]} \to \Spec{\stalk{X}{x}}$ and thus is equivalent to a local map $\stalk{X}{x} \to k[\epsilon]$ which is therefore determined by $\varphi : \m_x / \m_x^2 \to k$ because any map $\stalk{X}{x} \to k[\epsilon]$ factors through $\stalk{X}{x} / \m_x^2$ because $\m_x$ maps inside $(\epsilon)$ and as $k$-modules,
\[ \stalk{X}{x} / \m_x^2 \cong \m_x / \m_x^2 \oplus k \]
because $x$ is a rational point. Therefore, to show that it corresponds to multiplication by $c$ it suffices to show that $\stalk{X}{x} \to k[\epsilon] \to k[\epsilon]$ acts by multiplying the map $\m_x / \m_x^2 \to k$ by $c$ which is clear because this $k$ is $(\epsilon) \subset k[\epsilon]$ which is multiplied by $c$ in $k[\epsilon] \to k[\epsilon]$.

\item The quotient maps $k[\epsilon,\epsilon'] \onto k[\epsilon]$ give a map,
\[ X(k[\epsilon, \epsilon']) \to X(k[\epsilon]) \times_{X(k)} X(k[\epsilon']) \]
To show this is bijective, it suffices to prove that,
\begin{center}
\begin{tikzcd}
k[\epsilon, \epsilon'] \pullback \arrow[r] \arrow[d] & k[\epsilon'] \arrow[d]
\\
k[\epsilon] \arrow[r] & 0
\end{tikzcd}
\end{center}
is a pullback diagram in the category of rings because the functor $X(-)$ preserves limits. Indeed, this is clear using the standard construction of the fibered product of two rings. Now consider the quotient map $k[\epsilon, \epsilon'] \onto k[\epsilon]$ via killing the ideal $(\epsilon - \epsilon')$. Given a map $\varphi : \stalk{X}{x} \to k[\epsilon, \epsilon']$ recovering two tangent vectors $\eta_1, \eta_2 : \m_x / \m_x^2 \to k$ via the projections then consider the above quotient,
\[ \stalk{X}{x} \xrightarrow{\eta} k[\epsilon, \epsilon'] \onto k[\epsilon] \]
via killing $(\epsilon - \epsilon')$. Notice that $t \in \m_x$ maps to $\eta_1(t) \epsilon + \eta_2(t) \epsilon' \mapsto [\eta_1(t) + \eta_2(t)] \epsilon$ in the quotient so this defines the tangent vector $\eta_1 + \eta_2$.

\item Now let $(X, x) = (G, e)$ be a $k$-group. From the multiplication map $m : G \times G \to G$ we get an induced map $m : T_e(G) \times T_e(G) \to T_e(G)$. We want to show that this is exactly addition. To do so, it suffices to check that,
\[ \stalk{G}{e} \to \stalk{G}{e} \otimes_k \stalk{G}{e} \]
is just $t \mapsto t \otimes 1 + 1 \otimes t$ ``to first order''. What this really means (seriously, we're not physicists here) is that,
\[ \stalk{G}{e}/\m_e^2 \to \stalk{G}{e} / \m_e^2 \otimes_k \stalk{G}{e} / \m_e^2 \]
is given exactly by $t \mapsto t \otimes 1 + 1 \otimes t$ for $t \in \m_e$. This suffices because maps to $k[\epsilon]$ always factor through this quotient.
\bigskip\\
We have seen that, $\stalk{G}{e}  / \m_e^2 \cong k \oplus \m_e / \m_e^2$ and therefore we get,
\[ k \oplus \m_e / \m_e^2 \to k \oplus (\m_e / \m_e^2 \otimes_k k) \oplus (k \otimes_k \m_e / \m_e^2) \oplus (\m_e / \m_e^2 \otimes_k \m_e / \m_e^2) \]
The $k$ part is boring (we get $k \mapsto k$ and not to any other part this is $1$ in the algebra structure). The interesting part is the map on $\m_e / \m_e^2$. Since $e \in G(k)$ is the identity, this map is supposed to be trivial on either quotient (killing one of the $\m_e / \m_e^2$) and therefore we just get the identity $\m_e / \m_e^2 \to \m_e / \m_e^2$ on each of the two factors isomorphic to $\m_e / \m_e^2$. The interesting part is the map,
\[ \m_e / \m_e^2 \to \m_e / \m_e^2 \otimes_k \m_e / \m_e^2 \]
which is basically the dual of the Lie bracket. However, that does not actually matter for us here because a local map from the right to $k[\epsilon]$ must be zero on the ``squared'' component because each maximal ideal maps into $(\epsilon)$ and thus their product to zero. Equivalently this is because a $k$-algebra map $\stalk{G}{e} / \m_e^2 \otimes_k \stalk{G}{e} / \m_e^2 \to k[\epsilon]$ is equivalent to two $k$-algebra maps $\stalk{G}{e} / \m_e^2 \to k[\epsilon]$. Thus we have show that multiplication induces addition of tangent vectors.
\bigskip\\
We can give an alternative (much simpler) proof as follows. Consider the maps $G \times e \to G \times G \to G$ and $e \times G \to G \times G \to G$ which compose to the identity. These induce $T_e(G) \to T_e(G) \oplus T_e(G) \to T_e(G)$ into the two factors but compose to the identity. Since the maps $T_e(G) \to T_e(G) \oplus T_e(G)$ are the canonical inclusions from the closed embeddings $G \times e \to G \times G$ and $e \times G \to G \times G$ we see that $T_e(G) \oplus T_e(G) \to T_e(G)$ is the unique map determined by the identity on each factor (the codiagonal) and thus is exactly $(x,y) \mapsto x + y$. 
\end{enumerate}

\subsection{4}

\newcommand{\Nm}{\mathrm{N}}
\newcommand{\Nrd}{\mathrm{Nrd}}

Let $A$ be a finite-dimensional associative (unital) algebra over a field $k$. Ddefine the ring functor $\underline{A}$ on $k$-algebras via,
\[ \underline{A}(R) = A \otimes_k R \]
and the group functor $\underline{A^\times}$ via,
\[ \underline{A^\times}(R) = (A \otimes_k R)^\times \]

\begin{enumerate}
\item 
Let $e_1, \dots, e_n \in A$ be a basis and let $e_i e_j = f_{ijk} e_k$ be structure constants. Then we can set $X = \A^n_k$. As an (additive) group scheme it is obious that $X$ represents $\underline{A}$ as an additive group functor (because when you forget about multiplication, $A$ is just a vectorspace!). Now we put a weird multiplication map $m : X \times X \to X$ given by the comultiplication,
\[ x_k \mapsto \sum_{ij} f_{ijk} x_i \otimes x_j \]
It is tedius but easy to check that if $A$ is associative then comultiplication I just wrote down is also (co)associative blah blah so we get a ring scheme $X$ representing $\underline{A}$. 
\bigskip\\
Note, if you hate coordinates, you can set,
\[ X = \Spec{\Sym{k}{A^*}} \]
Then the bilinear multiplication form $B : A \otimes A \to A$ has dual $A^* \to A^* \otimes A^*$ inducing a map $\Sym{k}{A^*} \to \Sym{k}{A^* \otimes A^*} = \Sym{k}{A^*} \otimes \Sym{k}{A^*}$ and thus a map $X \to X \times X$. Then,
\[ \Hom{}{\Spec{R}}{X} = \Hom{k}{\Sym{k}{A^*}}{R} = \Hom{k}{A^*}{R} = A \otimes_k R \]
and the multiplication map clearly induces $B$ under these identifications.
\bigskip\\
Now consider the map $X \to \End{A}$ (the endomorphism scheme representing $R \mapsto \End{A \otimes_k R}$ given at the level of functors by sending $A \otimes_k R \mapsto \End{R}{A \otimes_k R}$ sending an element to its action as an automorphism. Then, consider,
\[ \Nm_{A/k} : X \to \mathrm{End}(A) \xrightarrow{\det} \A^1_k \]
Then consider the preimage of the submonoid $\Gm \subset \A^1_k$ which is the closed subscheme
\[ Y = \Nm_{A/k}^{-1}(\Gm) \subset X \]
This represents the following functor,
\begin{align*}
\Hom{}{\Spec{R}}{\Nm_{A/k}^{-1}(\Gm)} & = \Hom{}{\Spec{R}}{X} \times_{\Hom{}{\Spec{R}}{\A^1}} \Hom{}{\Spec{R}}{\Gm} 
\\
& = (A \otimes_k R) \times_{R} R^\times = (A \otimes_k R)^\times 
\end{align*}
because it is exactly the elements whose action is a matrix with unit determinant and thus is invertible as a matrix. Why does this mean that $u \in \End{A \otimes_k R}^\times$ and in the image of $A \otimes_k R$ is actually a unit. This is because if $u$ is invertible in $\End{A \otimes_k R}$ then it is injective as a linear map $A \to A$ and therefore surjective (because $\dim{A}$ is finite) and thus there is some $u' \in A$ such that $uu' = 1$ meaning that $u$ is actually invertible.

\item Let $A = M_n(k)$ then $A \otimes_k R = M_n(R)$ so $(A \otimes_k R)^\times = \GL_n(R)$ so $\underline{A^\times}$ represents the functor $R \mapsto \GL_n(R)$ and therefore $\underline{A^\times} = \GL_n$. 
\bigskip\\
Notice that $\underline{A} \embed \End{A}$ is a closed embedding because it is the map,
\[ \Spec{\Sym{k}{A^*}} \embed \Spec{\Sym{k}{\Hom{k}{A}{A}^*}} \]
given by applying the symmetric algebra functor to the dual of $a \mapsto B(a, -)$ which is injective because $A$ is unital and therefore its dual is surjective proving that the ring map is a quotient and thus the scheme map is a closed immersion. Therefore, consider the diagram,
\begin{center}
\begin{tikzcd}
\underline{A^\times} \arrow[d, hook] \pullback \arrow[r, hook] & \GL(A) \pullback \arrow[r] \arrow[d, hook] & \Gm \arrow[d, hook]
\\
\underline{A} \arrow[r, hook] & \End{A} \arrow[r, "\det"] & \A^1_k
\end{tikzcd}
\end{center}
which proves by the symmetry of pullback that $\underline{A^\times} \embed \GL{A}$ is a closed embedding.
\bigskip\\
For $k = \Q$ and $A = \Q(\sqrt{d})$ we choose the basis $e_1 = 1$ and $e_2 = \sqrt{d}$. Then $x e_1 + y e_2$ maps to $\End{A}$ as the matrix,
\[ \begin{pmatrix}
x & y d 
\\
y & x
\end{pmatrix} \]
and thus,
\[ \underline{A^\times} = V(\alpha - \delta, d \beta - \gamma) \]
for coordinates,
\[ \begin{pmatrix}
\alpha & \beta
\\
\gamma & \delta
\end{pmatrix} \in \Gamma(\GL_n, \struct{\GL_n}) \]

\item For $A = M_n(k)$ the map $\Nm_{A/k} : \underline{A^\times} \to \Gm$ is given by sending a matrix $M \in M_n(k) \otimes_k R = M_n(R)$ to the matrix representation for multiplication by $M$ as a map $M_n(R) \to M_n(R)$ (viewed just as a free $R$-module). Choosing the standard $n^4$ basis elements $e_{ij}$ the ``big matrix'' representation of $M_n(R)$ has determinant $(\det{M})^n$ (CHECK THIS).
\bigskip\\
On HW1 4(iii) the scheme we wrote down was exactly the kernel of $\Nm_{A/k} : \underline{A^\times} \to \Gm$ where $A = k(a^{\frac{1}{p}})$ for some $a \in k$ not in the image of Frobenius (and thus of course $k$ is not perfect). 
\end{enumerate}

\subsection{5}


Let $A$ be a finite-dimensional central simple algebra over $k$. 

\begin{enumerate}
\item We proceed by length induction on Artin local rings $R$. If $\ell_R(R) = 1$ then $R$ is a field and we know that $\Aut{M_n(R)}$ is given by inner automorphisms (conjugation by units). Now for the inductive step. Suppose that $R$ is an Artin local ring and $\ell_R(R) = n$. Let $\m \subset R$ be the maximal ideal which must be finitely generated and consist of nilpotents since $R$ is Artinian. If $R$ is not a field then there is some $y \in \m$ such that $(y) \neq 0$ but $y \m = 0$ (say this is a small extension). Then there is an exact sequence,
\begin{center}
\begin{tikzcd}
0 \arrow[r] & (y) \arrow[r] & R \arrow[r] & R' \arrow[r] & 0
\end{tikzcd}
\end{center}
where $R' = R / (y)$ has length less than $n$ and $(y) \cong \kappa$ is naturally a $\kappa = R / \m$-module because $Y \m = 0$. If $\varphi \in \Aut{M_n(R)}$ then it image in $\Aut{M_n(R')}$ is given by conjugation by some unit $u \in \GL_n(R')$ by the induction hypothesis. Then we can lift $\tilde{u} \in \GL_n(R)$ because $\det{u} \in R'^\times$ and thus $\det{\tilde{u}} \notin \m$ because otherwise it would map into the maximal ideal (the map $R \to R'$ is local) and thus $\det{\tilde{u}} \in R^\times$ so it is invertible. Now consider $\tilde{\varphi}(x) = u^{-1} \varphi(x) u$ which is an automorphism of $M_n(R)$ which acts via the identity on $M_n(R')$ after quotienting. Therefore, $\eta = \varphi' - \id : M_n(R) \to M_n((y))$ is a derivation. To see this, consider,
\[ \eta(ab) = (\tilde{\varphi} - \id)(ab) = \tilde{\varphi}(a) \tilde{\varphi}(b) - ab = (\tilde{\varphi}(a) - a)b + a(\tilde{\varphi}(b) - b) \]
because the difference is,
\[ [\tilde{\varphi}(a) \tilde{\varphi}(b) - ab] - [(\tilde{\varphi}(a) - a)b + a(\tilde{\varphi}(b) - b)] = (\tilde{\varphi}(a) - a)(\tilde{\varphi}(b) - b) = 0 \] 
because the multiplication in $(y)$ and thus in $M_n((y))$ is trivial. 
Furthermore, the derivation $\eta$ is $R$-linear and thus kills $\m M_n(R) \subset M_n(R)$ because for any $r \in \m$ and $a \in M_n(R)$ we have,
\[ (\tilde{\varphi} - \id)(ra) = r \cdot (\tilde{\varphi}(a) - a) = 0 \]
because $r \in \m$ and $(\tilde{\varphi}(a) - a) \in M_n((y))$ and $y \m = 0$. Notice that $\m M_n(R) = M_n(\m)$ because $M_n(\m)$ is generated by the matrices with one entry all of which are clearly in $\m M_n(R)$. Therefore, it descends to a derivation $\bar{\eta} : M_n(R)/M_n(\m) = M_n(\kappa) \to M_n(y)$. However, a derivation does not care about the algebra structure on its target only on the $M_n(\kappa)$-module structure but $M_n(y) \cong M_n(\kappa)$ as $M_n(\kappa)$-modules because $(y) \cong \kappa$ as $\kappa$-modules (note that $y \m = 0$ so $(y)$ is automatically a $\kappa = R / \m$-module). Now, by the Noether-Skolem theorem, every derivation $\eta : M_n(\kappa) \to M_n(\kappa)$ is inner menaing it is of the form $\eta = [u, -]$ for some $u \in M_n(\kappa)$. Now consider $\tilde{u} \in M_n(y)$ under the isomorphism $\kappa \iso (y)$ (i.e. we can write $\tilde{u} = y u$). I claim that $\tilde{\varphi}(x) = (1 + \tilde{u}) x (1 + \tilde{u})^{-1}$. Indeed, because $\tilde{u}^2 = 0$ since $y^2 = 0$ we see that $(1 + \tilde{u})^{-1} = 1 - \tilde{u}$. Therefore,
\[ (1 + \tilde{u}) x (1 + \tilde{u})^{-1} = x + \tilde{u} x - x \tilde{u} + \tilde{u} x \tilde{u} \]
but $y^2 = 0$ so $\tilde{u} x \tilde{u} = 0$ and thus,
\[ (1 + \tilde{u}) x (1 + \tilde{u})^{-1} = x + [\tilde{u}, x] \]
Now, $\tilde{\varphi} - \id = y [u, -] = [\tilde{u}, -]$ and therefore we see that $\tilde{\varphi}(x) = x + [\tilde{u}, x]$ proving that,
\[ \tilde{\varphi}(x) = (1 + \tilde{u}) x (1 + \tilde{u})^{-1} \]
and thus $\tilde{\varphi}$ and thus $\varphi$ is inner. Therefore by induction on $\ell_R(R)$ we have proven that $\Aut{M_n(R)}$ is inner for all Artin local rings $R$.

\item Notice that because these are $k$-vectorspaces, there is a natural isomorphism,
\[ \Hom{R\text{-alg}}{A_R}{M_n(R)} \iso \Hom{k\text{-alg}}{A}{M_n(k)} \otimes_k R \]
because the condition of being a map of algebras is preserved under the natural isomorphism of $R$-modules. Therefore,
\[ \mathrm{Iso}_R(A_R, M_n(R)) \]
should correspond to the ``units''. To make this precise, choosing a $k$-linear isomorphism $\varphi : A \to M_n(k)$ we get a map,
\[ \Hom{R\text{-alg}}{A_R}{M_n(R)} \embed \GL{(M_n(R))} \xrightarrow{\det} R^\times \]
which gives a map,
\[ \Spec{\Sym{k}{\Hom{k\text{-alg}}{A}{M_n(k)}}} \embed \GL(M_n(k)) \xrightarrow{\det} \Gm \]
Let $I$ be the kernel of this map (as a map of pointed scheme since the first one is not given a monoid structure since we dont have an algebra structure on the vectorspace of maps). Then it is clear that $I$ represents the functor,
\[ R \mapsto \mathrm{Iso}_R(A_R, M_n(R)) \]
because it represents the invertible elements in the affine space representing maps,
\[ A_R \to M_n(R) \] 
Because $A_{\bar{k}} \iso M_n(\bar{k})$ as $\bar{k}$-algebras we see that $I(\bar{k})$ is nonempty. Furthermore, after choosing a $\bar{k}$-point $I_{\bar{k}} \cong \Aut{A_{\bar{k}}}$ because a choice of a point $x \in I_{\bar{k}}$ corresponds to a choice of an isomorphism $A_{\bar{k}} \iso M_n(\bar{k})$ which fixes an isomorphism between the functor $I_{\bar{k}}$ represents, namely $R \mapsto \mathrm{Iso}_R(A_R, M_n(R))$, and the functor $R \mapsto \Aut{M_n(R)}$. Because now $\End{M_n(\bar{k})}$ is actually an algebra, we can apply the construction of problem 3 to get a scheme $\underline{\End{M_n(\bar{k})}^\times} = \Aut{M_n(\bar{k})}$ representing that second functor. Then by Yoneda it is isomorphic to $I_{\bar{k}}$. 
\bigskip\\
Thus to show that $I$ is smooth it suffices to show that $I_{\bar{k}}$ is smooth and thus that $\Aut{M_n(\bar{k})}$ is smooth. We check the infinitessimal lifting property. Let $R$ be a local Artin ring with $J \subset R$ an ideal of square zero (in fact any nontrivial ideal will work because it will be nilpotent since the maximal ideal is nilpotent). Let $G = \Aut{M_n(\bar{k})}$. In the previous part we saw that,
\[ G(R) = \PGL_n(R) \]
given by $\GL_n(R)$ acting on $M_n(R)$ by conjugation (note that $\PGL_n(R) = \GL_n(R) / R^\times$ since $R$ is local). Therefore, consider,
\[ G(R) \to G(R/J) \]
and thus the map $\PGL_n(R) \to \PGL_n(R/J)$ which we know is surjective because any invertible matrix lifts to an invertible matrix since $R$ is local (its determinant is not in the maximal ideal of $R/J$ and thus the dertminant of a lift is not in the maximal ideal of $R$ and thus a unit). Therefore, by the infinitessimal lifting criterion, $G$ is smooth. 
\bigskip\\
Because $I$ is finite type over $k$ and smooth by HW 5 (iii) we know that $I(k^\sep)$ is dense (in particular nonempty) and therefore $I(k')$ is nonempty for some finite separable $k'/k$. Therefore, there exists some isomorphism $A_{k'} \iso M_n(k')$ over a finite separable extension.


\item Therefore, we can choose a finite Galois extension $K/k$ and a $K$-algebra isomorphism $\theta : A_K \iso M_n(K)$ which is, by Skolem-Noether, unique up to conjugation by a unit. Therefore, for any $R$-algebra,
\[ A_R \xrightarrow{\theta} M_n(R) \xrightarrow{\det} R^\times \]
does not depend on the choice of $\theta$ because for any other choice $\theta'$ we know that $\theta' = c_u \circ \theta$ where $c_u$ is conjugation by some element $u \in M_n(K)$ which does not change the determinant. This defines a morphism $\underline{A}_K \to \A^1_K$ which is undependent of the choice of $\theta$. The $\Gal{K/k}$-action takes $\theta$ to some other choice of an isomorphism $A_K \to M_n(K)$ which thus defines the name determinant map. Thus $\underline{A}_K \to \A^1_K$ is Galois invariant and therefore arises from a unique map $\Nrd_{A/k} : \underline{A} \to \A^1_k$.
\bigskip\\
Furthermore, by definition, if $F/k$ is an extension such that $A_F \iso M_n(F)$ when we see that $\Nrd_{A/k}$ agrees with $\det$ under any such isomorphism.
\bigskip\\
Furthermore, because $\Nrd_{A/k}^n$ and $\Nm_{A/k}$ are Galois equivariant we just need to show that they are equal after some extension $K/k$. For any $K$-algebra, there is a commutative diagram,
\begin{center}
\begin{tikzcd}
A_R \arrow[d, "\theta"] \arrow[r, "\theta"] & M_n(R) \arrow[d, hook] \arrow[r, "\det"] & R^\times \arrow[r, "x \mapsto x^n"] & R^\times \arrow[d, equals]
\\
M_n(R) \arrow[r, hook] & \mathrm{End}_R(M_n(R)) \arrow[rr, "\det"] & & R^\times 
\\
A_R \arrow[u, "\theta"'] \arrow[r, hook] & \mathrm{End}_R(A_R) \arrow[u, "\theta"'] \arrow[rr, "\det"] & & R^\times \arrow[u, equals]
\end{tikzcd}
\end{center}
by what we did in problem 4 where the top row is $\Nrd_{A/k}$ and the bottom row is $\Nm_{A/k}$ and clearly the leftmost map $A_R \to A_R$ is the identity. We should ask why the botom right square actually commutes. This is just something silly about vector spaces. If $T : V \to W$ is an isomorphism of vector spaces then we get a map $T : \End{V} \to \End{W}$ sending $\varphi \mapsto T \circ \varphi \circ T^{-1}$ which clearly preserves the determinant. Then the bottom left square commutes because $\theta(a)$ acts on $M_n(R)$ via $\theta \circ (\theta(a) \cdot -) \circ \theta^{-1}$ because $\theta$ is an algebra isomorphism which is exactly $A_R \to \mathrm{End}_R(A_R) \to \mathrm{End}_R(M_n(R))$.
\bigskip\\
Therefore, there is a commutative diagram,
\begin{center}
\begin{tikzcd}
\underline{A^\times} \pullback \arrow[d, hook] \arrow[r] & \Gm \pullback \arrow[d, hook] \arrow[r, "x \mapsto x^n"] & \Gm \arrow[d, hook]
\\
\underline{A} \arrow[r, "\Nrd_{A/k}"] & \A^1 \arrow[r, "x \mapsto x^n"] & \A^1
\end{tikzcd}
\end{center}
where the big rectangle is a pullback by definition because the bottom composition is $\Nm_{A/k}$. Therefore, because the right square is a pullback this means that the left square is also a pullback proving that $\underline{A^\times} = \Nrd_{A/k}^{-1}(\Gm)$ scheme theoretically.

\item Let $\SL(A)$ denote the scheme-theoretic kernel of $\Nrd_{A/k} : \underline{A^\times} \to \Gm$. The formation of $\SL(A)$ as a kernel commutes with extensions of the ground field $k$ simply because the formation of $\Nrd_{A/k}$ commutes with taking extensions since it is Galois descended from a map that commutes with taking extensions. Finally, kernels and base change commute because they are both limits so indeed we see that $\SL(A_{k'}) = \SL(A)_{k'}$. In particular, because $\theta : A_{\bar{k}} \iso M_n(\bar{k})$ such that the diagram,
\begin{center}
\begin{tikzcd}[row sep = small]
\underline{A_{\bar{k}}^\times} \arrow[dd, "\theta"] \arrow[rd, "\Nrd_{A/k}"] 
\\
& \Gm
\\
\GL_n \arrow[ru, "\det"']
\end{tikzcd}
\end{center}
commutes we get immediately isomorphisms,
\[ \SL_n = \ker{\det} \cong \ker{\Nrd_{A/k}} = \SL(A)_{\bar{k}} \]
showing that $\SL(A)$ is a form of $\SL_n$. In particular, $\SL(A)$ is smooth and connected because these are geometric properties. However, notice that $\ker{\Nm_{A/k}}$ need not be smooth in positive characteristic basically because it is the composition of $\Nrd_{A/k}$ with the map $x \mapsto x^n$ whose kernel $\mu_n$ need not be smooth if $p \divides n$. For a concrete example, consider HW1 problem 4(iii).
\end{enumerate}

\section{Homework 5}

\subsection{1 DO}

We show that there is no nontrial $k$-group scheme isomorphic to closed $k$-subgroups of $\Ga$ and $\Gm$. 
\bigskip\\
Let $Z$ be a $k$-group scheme equipped with closed group embeddings $\iota_1 : Z \embed \Ga$ and $\iota_2 : Z \embed \Gm$. Since we have shown that $\Gm \not\cong \Ga$ we know that the subgroup must be proper and therefore $Z$ must be a finite scheme (has finitely many points) of finite type over $k$.
\bigskip\\
If $k$ has characteristic zero then for any \textit{finite-dimensional} $k$-algebra $A$ the set $Z(A) \subset \Ga(A) = (A, +)$ must be either trivial or infinite\footnote{I need $A$ to be finite-dimensional else $Z(A)$ need not be finite even though $Z$ is a finite scheme. Indeed, consider $Z = \Spec{\Q(i)}$ as a $\Q$-scheme and $A = \prod\limits_{n \in \N} \Q(i)$ where there are infinitely many maps $\Spec{A} \to \Spec{\Q(i)}$ because I can send $i \mapsto (\epsilon_n i)_n$ where $\epsilon_n \in \{ \pm 1 \}$.} because an nonzero element of $A$ has infinite additive order since it is a $k$-vectorspace and $k$ has characteristic zero. Therefore $Z(A) = \{ e \}$ and thus $Z$ must be trivial because we need only to check on finite field extension of $k$ (to check the number of points) and finite dimensional Artin local rings (to check smoothness). We have actually showed that $\Ga$ has no nontrivial subgroups in positive characteristic.
\bigskip\\
If $k$ has characteristic $p > 0$ then $Z(\bar{k}) \subset \Ga(\bar{k}) = \bar{k}$ must be a finite $p$-torsion group. However, $Z(\bar{k}) \subset \Gm(\bar{k}) = \bar{k}^\times$ and any finite multiplicative subgroup of a field is cyclic. Therefore, the only possibility is that  $Z(\bar{k})$ is cyclic of order $p$. However, the only soution in $\bar{k}^\times$ to $x^p = 1$ is $1$ and thus $Z(\bar{z}) = \{ e \}$.
\bigskip\\
Now for any $k$-algebra $A$, we see that $Z(A)$ 


Filter and intersect will smallest filtered part then take quotient.

\subsection{2}

Let a smooth finite type $k$-group $G$ act linearly on a finite-dimensional $V$. Let $\underline{V}$ denote the affine space $\Spec{\Sym{k}{V^*}}$ representing the functor $A \mapsto V_A$. Define,
\[ \underline{V^G}(A) = \{ v \in V_A \mid G_A \text{ acts trivially on } v \} \]

\subsubsection{(i) DO }

Suppose that $(\underline{V^G})_{k^\sep}$ is representable by $Z$ as a subscheme of $V_{k^\sep}$. Then I claim that $Z$ is Galois equivariant. Indeed,   for any $\gamma \in \Gal{k^\sep/k}$ consider the closed subscheme $\gamma^*(Z)$. But notice that,
\begin{align*}
\Hom{k^\sep}{\Spec{A}}{\gamma^*(Z)} & = \Hom{k^\sep}{\Spec{A^{\gamma}}}{Z} = Z(A^\gamma) = \underline{V^G}(A^\gamma)
\\
& = \{ v \in V_{A^\gamma} \mid G_{A^\gamma} \text{ acts trivially on }  v \} = \{ v \in V_{A} \mid G_A \text{ acts trivially on } v \} 
\end{align*}
Therefore, $\gamma^*(Z)$ and $Z$ represent the same functor of subsets of $\underline{V_{k^\sep}}$ and therefore $\gamma^*(Z) = Z$ as subschemes. Therefore, by Galois descent, there is a closed subscheme $V^G \subset V$ such that $(V^G)_{k^\sep} = Z$ as a closed subscheme of $V_{k^\sep}$. Now I claim that $V^G$ represents $\underline{V^G}$. Indeed, this follows by uniqueness for Galois descent for sheaves on the \etale site.
\bigskip\\
Now I reduce to the case $k = k^\sep$. Consider the subspace $W = V^{G(k)} \subset V$ at the vectorspace level. Then we get a closed immersion 

\subsubsection{(ii)}

Let $K/k$ be an extension. Then I claim that $(V_K)^{G_K}$ and $(V^G)_K$ represent the same functor of subspaces and are thus the same as subspaces of $V_K$. Indeed, for any $K$-algebra $A$,
\[ (V_K)^{G_K}(A) = \{ v \in (V_K)_A \mid (G_K)_A \text{ acts trivially on } v \} = \{ v \in V_A \mid G_A \text{ acts trivially on } v \} \]
and likewise,
\[ (V^G)_K(A) = V^G(A) = \{ v \in V_A \mid G_A \text{ acts trivially on } v \} \]
giving the same functor of subspaces.

\subsection{3}

Let $k$ be a field and $k'$ a finite commutative $k$-algebra (UGH) and $X'$ an affine $k'$-scheme of finite type. Consider the functor,
\[ R_{k'/k}(X') : A \mapsto X'(k' \otimes_k A) \]
on $k$-algebras.

\subsubsection{(i) DO}

First consider the case of $X' = \A^n_{k'}$. Clearly, $R_{k'/k}(X')$ is representable by $X = \A^{n d}_k$ where $d = \dim_k k'$ because,
\[ R_{k'/k}(X')(A) = (k' \otimes_k A)^{\oplus n} \cong A^{\oplus n d} \]
as a $k$-module naturally in $A$ by choosing a basis of $k'$ as a $k$-vector space. 
\bigskip\\
Now consider $X' \embed \A^n_{k'}$ a closed immersion. 

(DO I HAVE TO WRITE DOWN EQUATIONS HERE?)

\subsubsection{(ii)}

Suppose that $X' = \Spec{k'}$ then $R_{k'/k}(X') : A \mapsto \Hom{k'}{k'}{k' \otimes_k A} = \{ 1 \}$ which is thus represented by $\Spec{k}$ since $\Spec{k}(A) = \Hom{k}{k}{A} = \{ 1 \}$. Therefore $R_{k'/k}(\Spec{k'})$ and $\Spec{k}$ are uniquely isomorphic. 
\bigskip\\
Now suppose that $X'$ is a $K'$-group. We need to show that $m : X' \times X' \to X'$ descends to $R_{k'/k}(X')$. Because $R_{k'/k}$ is a funtor preserving products it preserves the diagrams defining a group object and thus $R_{k'/k}$ is a group object in the category of $k$-schemes. To see why $R_{k'/k}$ preserves products, notice that $R_{k'/k}$ is right adjoint to base change meaning,
\[ \Hom{k}{T}{R_{k'/k}(X')} = \Hom{k'}{T_{k'}}{X'} \]
and therefore $R_{k'/k}$ actually preserves all limits and thus preserves products. This adjunction holds for all schemes but we only really need the affine case because,
\begin{align*}
R_{k'/k}(X'_1 \times_{k'} X'_2)(A) & = (X'_1 \times_{k'} X'_2)(k' \otimes_k A) 
\\
& = \Hom{k'}{\Spec{k' \otimes_k A}}{X'_1 \times_{k'} X'_2} 
\\
& = \Hom{k'}{\Spec{k' \otimes_k A}}{X'_1} \times \Hom{k'}{\Spec{k' \otimes_k A}}{X'_2} 
\\
& = (R_{k'/k}(X'_1)(A)) \times_{k} (R_{k'/k}(X'_2)(A)) 
\end{align*}
and therefore since these functors are equal the schemes representing them are uniquely isomorphic (and the map from schemes to representable functors preserves products because they are limits by the definition of a limit) so we see that,
\[ R_{k'/k}(X'_1 \times_{k'} X'_2) = R_{k'/k}(X'_1) \times_{k} R_{k'/k}(X'_2) \]

\subsubsection{(iii)}

For any extension of fields $K/k$, let $K' = K \otimes_k k'$. Then for any $K$-algebra $A$ consider,
\begin{align*}
\Hom{K}{\Spec{A}}{R_{K'/K}(X'_{K'})} & = X'_{K'}(K' \otimes_K A) = \Hom{K'}{\Spec{K' \otimes_K A}}{X'_{K'}} 
\\
& = \Hom{k'}{\Spec{K' \otimes_K A}}{X'} = X'(K' \otimes_K A)
\end{align*}
However, $K' \otimes_K A = k' \otimes_k K \otimes_K A = k' \otimes_k A$ and therefore,
\begin{align*}
\Hom{K}{\Spec{A}}{R_{K'/K}(X'_{K'})} & = X'(k' \otimes_k A) = R_{k'/k}(X')(A) = \Hom{k}{\Spec{A}}{R_{k'/k}(X')} 
\\
& = \Hom{K}{\Spec{A}}{R_{k'/k}(X')_K}
\end{align*}
where the last line follows because $A$ is already equipped with a $K$-algebra structure. Therefore, by Yoneda the $K$-schemes are uniquely isomorphic,
\[ R_{K'/K}(X'_{K'}) = R_{k'/k}(X')_K \]
Now let $K = \bar{k}$ and suppose that $k'$ is a field. We want to prove that if $X'$ is smooth over $k'$ then $R_{k'/k}(X')$ is smooth over $k$. To do this, we need to show that for any extension of Artin local  $R \onto R'$ that $R_{k'/k}(X')(R) \to R_{k'/k}(X')(R')$ is surjective. However, from the digram,
\begin{center}
\begin{tikzcd}
R_{k'/k}(X')(R) \arrow[d, equals] \arrow[r] & R_{k'/k}(X')(R') \arrow[d, equals]
\\
X'(k' \otimes_k R) \arrow[r] & X'(k' \otimes_k R')
\end{tikzcd}
\end{center}
it suffices to show that $X'(k' \otimes_k R) \to X'(k' \otimes_k R')$ is surjective. However, if $R$ is a local Artin $k$-algebra then $k' \otimes_k R \onto k' \otimes_k R'$ is still an infinitessimal extension of Artin rings and thus $X'(k' \otimes_k R) \to X'(k' \otimes_k R')$ is surjective because $X'$ is smooth and thus formally smooth. Notice, we don't actually need that $k'$ is a field here at all just a finite commutative $k$-algebra (Brian just didn't talk about smoothness not over a field).
\bigskip\\
Let's give a proof using only the smoothness criteria in the notes. Because smoothness is a geometric condition, we can base change to $K = \bar{k}$ and check that $(R_{k'/k}(X'))_K = R_{K'/K}(X'_{K'})$ is smooth. However, for any Artin local $\bar{k}$-algebras $R \onto R'$,
\begin{center}
\begin{tikzcd}
R_{K'/K}(X'_{K'})(R) \arrow[d, equals] \arrow[r] & R_{K'/K}(X'_K)(R') \arrow[d, equals]
\\
X'(k' \otimes_{k} R) \arrow[r] & X'(k' \otimes_k R')
\end{tikzcd}
\end{center}

\subsubsection{(iv) DO}

If $k'/k$ is a separable extension then for any $k^\sep$-algebra $A$ consider,
\[ \Hom{k^\sep}{\Spec{A}}{R_{k'/k}(X')_{k^\sep}} = R_{k'/k}(X')(A) =  X'(k' \otimes_k A) \]
However, because $k'/k$ is separable,
\[ k' \otimes_k A = k' \otimes_k k^\sep \otimes_{k^\sep} A = \prod_{\sigma \in \Hom{k}{k'}{k^\sep}} (k^\sep)^\sigma \otimes_{k^\sep} A = \prod_{\sigma} A^\sigma \]
via the natural isomorphism,
\[ k' \ot_k k^\sep \iso \prod_{\sigma} (k')^\sigma \quad \text{ via } \quad \alpha \otimes \beta \mapsto (\sigma(\alpha) \beta)_\sigma \]
where $(k^\sep)^\sigma$ is $k^\sep$ viewed as a $k'$-algebra through a $\sigma$-twist with $k'$-action given by $\lambda \cdot \alpha = \sigma(\lambda) \alpha$ and likewise $A^\sigma = (k^\sep)^\sigma \otimes_\sigma A$ is $A$ with $k'$-action given by $\lambda \cdot \alpha = \sigma(\lambda)\alpha$. This can alternatively be described as the $k'$-algebra $k' \xrightarrow{\sigma} k^\sep \to A$.
\bigskip\\
Now,
\[ X'(k' \otimes_k A) = \prod_{\sigma} X'(A^\sigma) = \prod_{\sigma} (\sigma^* X')(A) \]
from the diagram,
\begin{center}
\begin{tikzcd}[column sep = small]
\Spec{A} \arrow[d, dashed] \arrow[dd, bend right] \arrow[r] & \Spec{A^\sigma} \arrow[d]
\\
\sigma^* X' \pullback \arrow[r] \arrow[d] & X' \arrow[d]
\\
\Spec{k^\sep} \arrow[r, "\sigma"] & \Spec{k'}
\end{tikzcd}
\end{center}
where because of the commutative square comming from the definition of $A^\sigma$,
\begin{center}
\begin{tikzcd}
A & A^\sigma \arrow[l, "\id"']
\\
k^\sep \arrow[u] & k' \arrow[l, "\sigma"'] \arrow[u]
\end{tikzcd}
\end{center}
we see that dotted maps $\Spec{A} \to \sigma^* X'$ over $\Spec{k^\sep}$ are equivalent to maps $\Spec{A^\sigma} \to X'$ over $\Spec{k'}$. Therefore, by Yoneda, there is a natural isomorphism,
\[ R_{k'/k}(X')_{k^\sep} \cong \prod_{\sigma \in \Hom{k}{k'}{k^\sep}} \sigma^* X' \]
The natural $\Gal{k^\sep/k}$-action on $R_{k'/k}(X')_{k^\sep}$ is given by taking $\gamma \in \Gal{k^\sep/k}$ and applying $\id \times \Spec{\gamma^{-1}}$ on $R_{k'/k}(X') \times_k \Spec{k^\sep}$. This gives a map,
\[ R_{k'/k}(X')_{k^\sep}(A) \xrightarrow{\gamma} R_{k'/k}(X')_{k^\sep}(A^{\gamma^{-1}}) \quad \text{ via } \quad (\eta : \Spec{A} \to R_{k'/k}(X')) \mapsto (\id \times \Spec{\gamma^{-1}}) \circ \eta \]
Via the above isomorphism, this is equivalent to the map,
\[ \prod_{\sigma} X'(A^\sigma) \to \prod_{\sigma} X'(A^{\gamma^{-1} \sigma}) \quad \text{ via } \quad (\eta_\sigma)_\sigma \mapsto (\gamma \circ \eta \]

\subsection{4}

Let $\Gamma = \Gal{k^{\sep}/k}$. A $\Gamma$-lattice is a finite free $\Z$-module equiped with a $\Gamma$-action making an open subgroup act trivially. 

\subsubsection{(i)}

Let $T$ be a $k$-torus then define the character group as,
\[ X(T) = \Hom{k^{\sep}}{T_{k^{\sep}}}{\Gm} \]
Since $T$ is a Torus, $T_{k^{\sep}} \cong \Gm^n$ and therefore,
\[ X(T) \cong \Hom{k^{\sep}}{\Gm^n}{\Gm} \cong \Hom{k^{\sep}}{\Gm}{\Gm}^{\oplus n} = \Z^n \]
by a previous homework problem. Thus $X(T)$ is a finite free $\Z$-module with rank $\dim{T}$. Furthermore, there is a natural $\Gamma$ action on $T_{k^{\sep}}$ and on $(\Gm)_{k^\sep}$ and therefore on $X(T)$ via conjugating by $\Gamma$. Explicitly, for $\sigma \in \Gamma$ and $\varphi \in X(T)$ we see that $\sigma \circ \varphi \circ \sigma^{-1}$ is $k^\sep$-linear. Because there is a finite Galois extension $k' / k$ such that $T_{k'} \cong \Gm^n$ we see that any $\sigma \in \Gamma$ fixing $k'$ must preserve $X(T)$ since the Galois action preserves the degree of maps $\Gm \to \Gm$. Therefore, the action is trivial on the open subgroup $\Gal{k^\sep/k'}$.

\subsubsection{(ii)}

Let $\Lambda$ be a $\Gamma$-lattice. 
\bigskip\\
If the $\Gamma$-action on $\Lambda$ trivial then we construct the torus,
\[ D_k(\Lambda) = \Spec{k[\Lambda]} \]
with multiplication given by the standard comultiplication map on the Hopf algebra $k[\Lambda]$. Then,
\[ \Hom{k}{\Spec{R}}{D_k(\Lambda)} = \Hom{k}{k[\Lambda]}{R} = \Hom{\mathbf{Grp}}{\Lambda}{R^\times} = \Hom{\mathbf{Grp}}{\Lambda}{R^\times_{k^\sep}}^{\Gamma} \]
Now we do the general case. Suppose that $\Gamma \acts \Lambda$ factors through a finite $\Gal{k'/k}$. Then consider,
\[ \Spec{k'[\Lambda]} \]
which is going to be the split torus $D_k(\Lambda)_{k'}$. There is a $\Gal{k'/k}$-action on $\Spec{k'[\Lambda]}$ by the action on $k'$ and the action on $\Lambda$. Therefore, by Galois descent for algebras we get a $k$-algebra $A$ such that $A \otimes_k k' \iso k'[\Lambda]$ \textit{as} $k'[\Gal{k'/k}]$-modules. Then let,
\[ D_k(\Lambda) = \Spec{A} \]
Now I claim that $D_k(\Lambda)$ represents the correct functor. Indeed,
\begin{align*}
\Hom{k}{\Spec{R}}{D_k(\Lambda)} & = \Hom{k}{A}{R} = \Hom{k'}{A \otimes_k k'}{R \otimes_k k'}^{\Gal{k'/k}} = \Hom{k'}{k'[\Lambda]}{R_{k'}}^{\Gal{k'/k}} 
\\
& = \Hom{}{\Lambda}{R_{k'}}^{\Gal{k'/k}} = \Hom{}{\Lambda}{R_{k^\sep}}^{\Gamma} 
\end{align*}  
where the final equality follows because $\Gal{k^\sep/k'}$ acts trivially on $\Lambda$.
\bigskip\\
Since it was constructed via Galois descent, we see that $D_k(\Lambda)_{k^\sep} = k^\sep[\Lambda]$ canonically with the $\Gamma$-action induced on $k^\sep$ and the given $\Gamma$-action on $\Lambda$. Therefore,
\[ X(D_k(\Lambda)) = \Hom{k^\sep}{D_k(\Lambda)_{k^\sep}}{\Gm} = \Hom{k^\sep}{k^\sep[t,t^{-1}]}{k^\sep[\Lambda]} = \Lambda \]
where the last set is maps of Hopf algebras. Furthermore, we recover the correct $\Gamma$-module structure by construction of the $\Gamma$-module structure on $k^\sep[\Lambda]$.

\subsubsection{(iii)}

Because $T$ is a sheaf in the \etale topology on $\Spec{k}$ (it is affine so its $A$ points are litterally maps of $k$-algebras which makes the following also easy) we see that for any $k$-algebra $A$,
\[ T(A) = T_{k^\sep}(k^\sep \otimes_k A)^\Gamma \]
Now notice that,
\begin{align*}
\Hom{k^\sep}{\Spec{A_{k^\sep}}}{\Gm^n} &= \Hom{k^\sep}{\Spec{A_{k^\sep}}}{\Gm}^n = (A_{k^\sep}^\times)^n = \Hom{}{\Z^n}{A_{k^\sep}^\times} 
\\
& = \Hom{}{X(\Gm^n)}{A_{k^\sep}^\times}
\end{align*}
via the canonical map sending $T(A) \to \Hom{}{X(T)}{A^\times_{k^\sep}}$ via $\varphi \mapsto (\psi \mapsto \psi \circ \varphi)$ where $\psi \in X(T)$ is a map $T_{k^\sep} \to \Gm$ and thus $T(A_{k^\sep}) \to \Gm(A_{k^\sep}) = A_{k^\sep}^\times$. Using an isomorphism $T_{k^\sep} \iso \Gm^n$ we get a diagram,
\begin{center}
\begin{tikzcd}
T_{k^\sep}(A_{k^\sep}) \arrow[d] \arrow[r] & \Hom{}{X(T)}{A_{k^\sep}^\times} \arrow[d]
\\
\Gm^n(A_{k^\sep}) \arrow[r, "\sim"] & \Hom{}{X(\Gm^n)}{A_{k^\sep}}
\end{tikzcd}
\end{center}
which commutes. Since the downward maps are isomorphism and the bottom map is an isomorphism we see that the top natural map is an isomorphism as well. Therefore,
\[ T(A) = T_{k^\sep}(k^\sep \otimes_k A)^\Gamma = \Hom{}{X(T)}{A_{k^\sep}}^\Gamma = D_k(X(T))(A) \]
naturally. Therefore, by Yoneda, we get a (natural in $T$) isomorphism of $k$-groups $T \iso D_k(X(T))$.
\bigskip\\
Now for an extension $k'/k$ of fields, we see that this is compatible with $T_{k'} \iso D_{k'}(X(T_{k'}))$. Notice that $X(T_{k'})$ is the same lattice as $X(T)$ but with the action restricted to the subgroup 
\[ \Gal{k^\sep/k'} \subset \Gamma \]
Therefore, scalar extension corresponds to restriction of the group action in the anti-equivalence between $k$-Tori and $\Gamma$-lattices.
\bigskip\\
Finally, suppose that $T$ is split. Then $T \iso \Gm^n$ and therefore $X(T) \iso X(\Gm^n) = \Z^n$ with the trivial $\Gamma$-action meaning that $X(T) = X(T)^\Gamma$. Furthermore, suppose that $X(T)^\Gamma = X(T)$ meaning that the $\Gamma$-action is trivial. Then $X(T) \iso \Z^n$ (equivariantly) meaning that $T \cong D_k(X(T)) \cong D_k(\Z^n) = \Gm^n$ so $T$ is split.

\subsubsection{(iv) DO}

Because $\Hom{k}{-}{\Gm} : \mathbf{CAlgGrp}_k^\op \to \mathbf{CMon}$ is left-exact we immediately see that,
\[ X(\coker{(T' \to T)}) = \ker{(X(T) \to X(T'))} \]
and therefore if $T' \to T$ is surjective (in the category of algebraic groups or equivalently in the category of sheaves of sets) then $\ker{(X(T) \to X(T'))} = 0$ so $X(T) \to X(T')$ is injective. To show the other direction, we will make use of the following result.
\bigskip\\
For any map of tori $T' \to T$ we have $\coker{(T' \to T)}$ is a torus. Indeed, $\ker{(X(T) \to X(T'))}$ is a $\Gamma$-lattice so we get a torus $K = D_k(\ker{(X(T) \to X(T'))})$ and thus by the anti-equivalence,
\[ T' \to T \to K \]
is a cokernel in the category of tori (WHAT ARE KERNELS AND COKERNELS IN CATEGORY OF TORI??)



Now suppose that $\ker{(T' \to T)}$ is a torus. Then because of the anti-equivalence, $X(\ker{(T' \to T)}) = \coker{(X(T) \to X(T'))}$ is free and thus torsion-free. Conversely, if $\Lambda = \coker{(X(T) \to X(T'))}$ is torsion-free then it is a $\Gamma$-lattice and therefore defines a torus $D_k(\Lambda)$. By the anti-equialence $D_k(\Lambda) = \ker{(T' \to T)}$ is a torus (WHAT ARE THE KERNELS IN THE CATEGORY OF TORI?)
\bigskip\\




SOME NOTES: WE DO HAVE THAT $X(D(\Lambda)) = \Lambda$ ALWAYS. AND I THINK WE EVEN HAVE,
\[ \Hom{}{G}{D(\Lambda)} \embed \Hom{}{\Lambda}{X(G)} \]
However, the torusification $G \mapsto D(X(G)/\text{Torsion})$ is not going to go anywhere because $X(\GL_n) = \Z$ so we're not going to recover anything like a maximal torus (SAD).
\bigskip\\


Use that $\Lambda = \ker{(X(T') \to X(T))}$ is a $\Gamma$-lattice and therefore by the antiequivalence,
\[ T \to T' \to D_k(\Lambda) \]
is an epimorphism of tori (WHY IS THAT THE COKERNEL!!)

Use 
\[ X(\coker{(T' \to T)}) = \ker{(X(T) \to X(T'))} \]
on algebraic groups / schemes / functors

\subsubsection{(v)}

Let $k'/k$ be a finite separable extension. Suppose that $T'$ is a $k'$-torus. Then consider $T = R_{k'/k}(T')$. We proved that,
\[ T_{k^\sep} = \prod_{\sigma \in \Hom{k}{k'}{k^\sep}} \sigma^*(T') = \prod_{\sigma \in \Hom{k}{k'}{k^\sep}} (T'_{k^\sep})^\sigma \cong \prod_{\sigma \in \Hom{k}{k'}{k^\sep}} \Gm^n \]
and therefore $T$ is a $k$-torus. Now consider the character lattice $X(R_{k'/k}(T'))$. Let $\Gamma' = \Gal{k^\sep/k'}$. For any $\Gamma$-lattice $\Lambda$ we can consider,
\begin{align*}
\Hom{\Gamma}{X(R_{k'/k}(T'))}{\Lambda} & = \Hom{k}{D_k(\Lambda)}{R_{k'/k}(T')} = \Hom{k'}{D_{k}(\Lambda)_{k'}}{T'} = \Hom{k'}{D_{k'}(\mathrm{Res}^\Gamma_{\Gamma'} (\Lambda))}{T'} 
\\
& = \Hom{\Gamma'}{X(T')}{\mathrm{Res}^\Gamma_{\Gamma'} (\Lambda)}= \Hom{\Gamma}{\mathrm{Ind}^\Gamma_{\Gamma'} X(T')}{\Lambda}
\end{align*}
Therefore, by Yoneda, there is a natural isomorphism,
\[ X(R_{k'/k}(T')) \cong \mathrm{Ind}^\Gamma_{\Gamma'} X(T') \]
Now to construct a morphism of the required form, we need to find an injection,
\[ X(T) \embed \bigoplus_{i = 1}^n \mathrm{Ind}^\Gamma_{\Gamma'_i} \Z = \bigoplus_{i = 1}^n \Z[\Gamma_i] \]
where $\Gamma_i' = \Gal{k^\sep/k'_i}$ for finite separable $k'_i / k$ and $\Gamma_i = \Gamma / \Gamma'_i$. Thus we need to inject any $\Gamma$-lattice $\Lambda$ into a finite sum of regular representations. The trick is that, over $\Q$, representations are semi-simple and every simple representation (say factoring through $\Gamma_i$) embedds in the regular representation $\Q[\Gamma_i]$ by character computations\footnote{It suffices to check on characters because any nonzero map from a simple representation has trivial kernel and therefore we just need a nonzero element of $\Hom{G}{V}{W}$ which is computed as $(\Hom{}{V}{W})^G$ by the character inner product because it is the trace of the averaging projection map $p : \Hom{}{V}{W} \to \Hom{G}{V}{W}$ which of course needs characteristic zero. This works perfectly fine over $\Q$}. Now we apply the following fact. If $\Lambda_1, \Lambda_2$ are finite torsion-free (and thus free) $\Z$-modules with a $G$-action and $\varphi : \Lambda_1 \otimes_\Z \Q \to \Lambda_2 \otimes_\Z \Q$ is a $G$-map then consider the diagram,
\begin{center}
\begin{tikzcd}
\Lambda_1 \otimes_\Z \Q \arrow[r, hook, "n \cdot \varphi"] & \Lambda_2 \otimes_{\Z} \Q
\\
\Lambda_1 \arrow[u, hook] \arrow[r, hook, dashed] & \Lambda_2 \arrow[u, hook]
\end{tikzcd}
\end{center}
where the upward maps are injective because $\Lambda_i$ are torsion-free and we can ensure that $n \varphi$ maps into the $\Z$-lattice by choosing sufficiently large $n$ to clear denominators. Because the $\Q$-representation $\Lambda \otimes_\Z \Q$ embedds in a product of $\Q[\Gamma_i]$ we get an injection,
\[ \Lambda \embed \bigoplus_{i = 1}^n \Z[\Gamma_i] \]
Applying this construction to $X(T)$, this corresponds to a surjective map of tori,
\[ \prod_{i = 1}^n X(R_{k_i'/k} (\Gm)) = \prod_{i = 1}^n D_k(\mathrm{Ind}^{\Gamma}_{\Gamma_i'} \Z)\onto T \] 
Furthermore, because $\Gm \embed \A^1$ is an open immersion $R_{k'/k}(\Gm) \embed R_{k'/k}(\A^1) = \A^{[k' : k]}$ is an open immersion so $R_{k'/k}(\Gm)$ is rational and therefore $T$ is unirational.

\subsubsection{(vi) DO}

\subsection{5}

Let $T \subset \GL(V)$ be a $k$-torus, with $k$ an infinite field. Let $A_T \subset \End{V}$ be the commutative $k$-subalgebra generated by $T(k)$ (which is Zariski-dense in $T$ since $k$ is infinite using unirationality from Exercise 4(iv)).

\subsubsection{(i)}

Because $T_{\bar{k}} \cong \Gm^n$ we can choose an embedding $T_{\bar{k}} \embed \GL_{n}$ sending $T_{\bar{k}}$ to the diagonal maximal torus which is semisimple (its diagonaliziable!) by definition. Therefore, since the Jordan decomposition is independent of the embedding we find that $T(\bar{k})$ is semi-simple and therefore $T(\bar{k}) \subset \GL(V) \subset \End{V_{\bar{k}}}$ is also semi-simple by the uniqueness.

\subsubsection{(ii) DO}

Let $k = k^\sep$. Then $T$ is a split torus so it is isomorphic to $\Gm^k \subset \GL(V)$. Then $T(k) \subset \End{V}$ is a commutative and semi-simple group of operators which thus can be simultaneously diagonalized so we may choose a basis $e_1, \dots, e_n \in V$ such that $T(k) = \{ \mathrm{diag}(t_{i_1}, \cdots, t_{i_r}) \mid t_i \in k^\times \}$ and thus
\[ A_T = \{ \mathrm{diag}(t_1, \cdots, t_r, \lambda, \cdots, \lambda) \mid t_i, \lambda \in k^\times \} = k^r \times k \]
unless $r = n$ in which case $A_T \cong k^n$ where the last factor is diagonal in $e_{r+1}, \dots, e_n$ (DO THIS BETTER) if $r < n$ and otherwise $A_T = k^n$. Then if $T$ is maximal menaing $r = n$ then $A_T^\times = T(k)$ by direct computation.

\subsubsection{(iii) DO}

It is clear that $(A_T)_{k^\sep} \subset A_{T_{k^\sep}}$. Because $T$ is unirational if we work in $T_{k^\sep}$ then $T(k)$ is dense in $T(k^\sep)$. Therefore, since $(A_T)_{k^\sep}$ is a closed subalgebra of $\End{V_{k^\sep}}$ containing $T(k)$ we see that it contains $T(k^\sep)$ and thus $A_{T_{k^\sep}} \subset (A_T)_{k^\sep}$ because $A_{T_{k^\sep}}$ is the algebra generated by $T(k^\sep)$. Therefore, $(A_T)_{k^\sep} = A_{T_{k^\sep}}$.
\bigskip\\
If $T$ is maximal then $T_{k^\sep}$ is still maximal  (WHY, ASK BRIAN??)
\bigskip\\
Then $(A_T)_{k^\sep} = A_{T_{k^\sep}}$ and $T_{k^\sep}(k^\sep) = (A_{T_{k^\sep}})^\times$. Therefore,
\[ (A_T)^\times_{k^\sep} = T(k^\sep) \]
Then taking Galois invariants we recover,
\[ (A_T)^\times = T(k) \]


\subsubsection{(iv) DO}

Let $T$ be a smooth connected commutative affine $k$-group. If $T$ is a torus then $T(\bar{k})$ is semisimple. Converely, suppose that $T(\bar{k})$ is semisimple. Then choose an embedding $T \embed \GL(V)$. Then consider $T(\bar{k}) \subset \End{V_{k^\sep}}$ which is a set of commuting semisimple operators and therefore we can diagonalize it such that $T(\bar{k}) \subset \Gm^n(\bar{k})$ therefore $T \embed \GL(V)$ factors through $T \embed \Gm^n \embed \GL(V)$. Thus since $T$ is a smooth and connected subgroup of a torus we see by a previous problem that $T$ is a torus.

\section{Homework 6}

\subsection{1 DO THIS}

\begin{lemma}
Let $X$ be a qcqs scheme over $\Spec{R}$ and $A$ a flat $R$-algebra. Then,
\[ \Gamma(X_A, \struct{X_A}) = \Gamma(X, \struct{X}) \otimes_R A \]
\end{lemma}

\begin{proof}
There is a fintie affine cover $U_i \subset X$ with $U_i = \Spec{B_i}$  for $R$-algebras $B_i$ such that $U_i \cap U_j$ is covered by finitely many affine opens $W_{ijk} = \Spec{B_{ijk}}$. Then,
\[ \Gamma(X, \struct{X}) = \ker{\left( \bigoplus_{i} \struct{X}(U_i) \to \bigoplus_{ijk} \struct{X}(W_{ijk}) \right)} \]
over the evident restriction maps. Notice that finiteness alows me to write these as direct sums. Then $U_i \times_R A$ gives an open affine cover of $X_A$ whose intersections are covered by $W_{ijk} \times_R A$ and therefore,
\begin{align*}
\Gamma(X_A, \struct{X_A}) & = \ker{\left( \bigoplus_i \struct{X}(U_i \times_R A) \to \bigoplus_{ijk} \struct{X}(W_{ijk} \times_R A) \right)}
\\
& = \ker{\left( \bigoplus_i \struct{X}(U_i) \otimes_R A \to \bigoplus_{ijk} \struct{X}(W_{ijk}) \otimes_R A\right)}
\\
& = \ker{\left( \bigoplus_i \struct{X}(U_i) \to \bigoplus_{ijk} \struct{X}(W_{ijk}) \right)} \otimes_R A = \Gamma(X, \struct{X}) \otimes_R A
\end{align*}
where $- \otimes_R A$ commutes with direct sums and kernels because $A$ is flat.
\end{proof}

\begin{prop}
Let $k$ be a field and $G$ a smooth unipotent affine $k$-group equpped with a lect action on a quasi-affine $k$-scheme $V$ of finite type over $k$. For any $v \in V(k)$ the smooth locally closed image of the orbit map $\alpha_v : G \to V$ defined by $g \mapsto g \cdot v$ is actually closed in $V$.
\end{prop}

\begin{proof}
Since $G$ is finite type, then the orbits of $G$ are the union of finitely many translates of the orbits of $G^\circ$ so we may assume that $G$ is connected. By the closed orbit lemma, $\alpha_v(G) \subset \overline{\alpha_v(G)} = W$ is open and let $F = W \setminus \alpha_v(G)$ with the reduced induced structure. Let $\I \subset \struct{W}$ be the sheaf of ideals of $F$ and $J \subset k[W]$ its global sections. Now I claim that $G$ acts on $k[W]$. We see immediately that $G(R)$ acts on $W_R$ and thus on $k[W_R] = k[W] \otimes_k R$ giving a map of functors $\underline{G} \to \underline{\mathrm{Aut}}(k[W])$. Let $\bar{V}$ be the closure in affine space and $\bar{F}$ be the closure of $F$ in $V$. Since $\bar{V} \cap V = F$ we see that any $v \in V \setminus F$ does not belong to $\bar{F}$, hence there exists a regular function of $\bar{F}$ which vanihes on $\bar{F}$ and is equal to $1$ at $v$ becasuse $\bar{V}$ is affine. Its restriction $f \in k[W]$ is then a nonzero element of $J$. The ideal $J$ is $G$-stable and is the union of $G$-invariant finite dimensional subspaces (REF 1.9 DONT UNDERSTAND). By 4.8 any such subspace contains a non-zero element fixed under $G$. However, the $G$-invariants in $k[W]$ are the constant functions because $G$ acts transitively on $\alpha_v(G)$ so $J \cap k^\times$ is nonempty and thus $J = (1)$ proving that $F = \emptyset$ so $\alpha_v(G)$ is closed.
\end{proof}

\subsection{2}

A $k$-morphism $f : G' \to G$ of finite type $k$-groups is an \textit{isogeny} it is it surjective, flat, and $\ker{f}$ is finite.

\subsubsection{(i)}

Let $f : G' \to G$ be a surjective homomorphism of smooth finite type $k$-groups whose dimensions $\dim{G'} = \dim{G} = n$ are equal. 
\bigskip\\
I claim that every fiber of $f$ has dimension zero and thus because $G', G$ are smooth and hence regular and Cohen-Macaulay, by miracle flatmess, $f : G' \to G$ is flat. Furthermore, $\dim{\ker{f}} = 0$ and $\ker{f}$ is a finite type $k$-scheme so $\ker{f}$ is finite. Therefore $f$ is an isogeny.
\bigskip\\
Because dimension is preserved under field extensions, it suffices to prove that $f_{k^\sep} : G'_{k^\sep} \to G_{k^\sep}$ has all fibers zero dimensional. Because surjectivity, finite-type, and homomorphisms are preserved under base change, we can just let $k = k^\sep$. I claim there is a nonempty open set $U \subset G$ such that,
\[ U \subset \{ y \in G \mid \dim{G'_y} = 0 \} \]
Therefore, because $k = k^\sep$-points are dense $U$ contains a $k$-point and thus all $k$-points are contained in a translate of $U$ by $G(k)$. Furthermore, because $f' : G' \to G$ is surjective, it is surjective on $k = k^\sep$-points so the fiber over a $G(k)$-translate is a $G'(k)$-translate of the fiber since $f' : G' \to G$ is a homomorphism. Therefore,
\[ G(k) \subset \bigcup_{g \in G(k)} g \cdot U \subset \{ y \in G \mid \dim{G'_y} = 0 \} \]
However, $G$ is Jacboson (it is a finite type $k$-scheme) and therefore $G(k)$-points are dense in every closed subset so the only open containing every $G(k)$-point is $G$ proving that all fibers have dimension $0$.
\bigskip\\
Now, to justify the claim, consider the map $f : f^{-1}(G^\circ) \to G^\circ$ which alows us to assume that $G$ is connected and smooth and thus integral. Now we apply generic flatness to get a nonempty open $U \subset G^\circ$ such that $f : f^{-1}(U) \to U$ is flat. Then flatness and surjectivity shows that the fibers over $U$ all have dimension zero. (DO THIS)

(ASK ABOUT A SLICKER PROOF)

\subsubsection{(ii)}

We have shown that a map of $k$-tori $f : T' \to T$ is surjective if and only if $X(T) \to X(T')$ is injective. We have also shown that $f : T' \to T$ has finite kernel if and only if $X(T) \to X(T')$ has finite cokernel.
\bigskip\\
Therefore, if $f$ is an isogeny then $X(T) \to X(T')$ is surjective with finite cokernel.
\bigskip\\
Suppose that $X(T) \to X(T')$ is surjective with finite cokernel. In this case $X(T)$ and $X(T')$ must have the same rank because their kernel and cokernel are both torsion (so tensoring by $\Q$ gives equality of the ranks). Thus $\dim{T} = \dim{T'}$ and $T' \to T$ is surjective because $X(T) \to X(T')$ is injective so by the previous part $f$ is an isogeny. 

\subsubsection{(iii)}

Consider the following statements about a $k$-torus $T$,
\begin{enumerate}
\item $T$ contains a $\Gm$ as a $k$-subgroup
\item there is a surjective $k$-homomorphism $T \onto \Gm$
\item the lattice $X(T)$ is $k$-\textit{isotropic} menaing $X(T)_\Q$ has a nonzero $\Gal{k^\sep/k}$-invariant vector.
\end{enumerate}

First, (b) and (c) are equivalent as follows. A surjection $T \onto \Gm$ gives an injection $\Z \embed X(T)$ giving a $\Gal{k^\sep/k}$-invaraint vector in $X(T)$ and thus in $X(T)_\Q$. Conversely, because $X(T)$ is torsion-free, $X(T) \embed X(T)_\Q$ so and a generating set for $X(T)$ gives a basis of $X(T)_\Q$ so clearing denominators any $\Gal{k^\sep/k}$-invariant vector in $X(T)_{\Q}$ gives a $\Gal{k^\sep/k}$-invariant vector in $X(T)$ and thus a map $\Z \embed X(T)$ and thus a surjection $T \onto \Gm$ of $k$-groups.
\bigskip\\
Now I show that (a) and (c) are equivalent. Given a $k$-subgroup $\Gm \subset T$ we get a surjection $X(T) \onto \Z$ and thus $X(T)_\Q \onto \Q$ of $\Gal{k^\sep/k}$-representations. However, (becuase these representations factor through finite quotient $\Gal{k'/k}$) semi-simplicitly gives that $X(T)_\Q$ has an irreducible component $\Q$ and thus there is a $\Gal{k^\sep/k}$-invaraint vector in $X(T)$. Conversely, given such a $\Gal{k^\sep/k}$-invariant vector in $X(T)_\Q$, by semi-simplicitly, $\Q$ is a direct factor of $X(T)_\Q$ so there is a surjective morphism $X(T)_\Q \onto \Q$. The image of $X(T) \subset X(T)_\Q$ must be a nontrivial $\Z$-module inside $\Q$ and thus isomorphic to $\Z$ giving a Galois equivariant map $X(T) \onto \Z$ and thus an injective $k$-homomorphism $\Gm \embed T$.

\subsubsection{(iv)}

Let $\Lambda$ be a $\Gamma$-lattice. Consider the subspace $\Lambda^\Gamma \subset \Lambda$. Then I claim that $\Lambda_a = \Lambda / \Lambda^\Gamma$ is torsion-free and thus a $\Gamma$-lattice. Indeed, this is because $\Lambda^\Gamma$ is saturated because if $n v \in \Lambda^\Gamma$ then $v \in \Lambda^\Gamma$ for any $n \in \Z^+$ because multiplication by $n$ is injective on $\Lambda$. Furthermore, it is clear that $\Lambda_a$ is anisotropic because any $\Gamma$-invariant vector is by definition in $\Lambda^G$. Finally, if $\Lambda \to \Lambda'$ is a map with $\Lambda'$ isotropic then $\Lambda^G \mapsto 0$ so it factors through $\Lambda \to \Lambda_a$. Therefore, let $T_a = D_k(X(T)_a)$ giving $T_a \subset T$ and every anisotropic torus mapping to $T$ factors through $T_a$. In particular, $T_a$ is the largest anisotropic $k$-subtorus containing all others.
\bigskip\\
We have an exact sequence of $\Gamma$-lattices,
\begin{center}
\begin{tikzcd}
0 \arrow[r] & \Lambda^G \arrow[r] & \Lambda \arrow[r] \arrow[l, bend right] & \Lambda_a \arrow[r] & 0
\end{tikzcd}
\end{center}
consider the map $q : \Lambda \to \Lambda^G$ given by,
\[ v \mapsto \sum_{\sigma \in \Gamma/\Gamma_\Lambda} \sigma \cdot v \]
Warning, this is not a section over $\Z$ but over $\Q$ can be made into a section by dividing by $\Gamma / \Gamma_\Lambda$. However, it is clear that $\Lambda^G \to \Lambda \to \Lambda$ is multiplication by $n = | \Gamma / \Gamma_\Lambda |$ (which is finite by the definition of a $\Gamma$-lattice so everything makes sense). Notice that,
\[ \ker{q} = \Lambda \cap (\Lambda \otimes_\Z \Q)_\Gamma \]
where $(\Lambda \otimes_\Z \Q)_\Gamma$ is the coinvariants of the $\Q$-representation $\Lambda \otimes_\Z \Q$.
\bigskip\\
Let $\Lambda_s = \im{q} \subset \Lambda^G$. Now, if $\varphi : \Lambda \to \Lambda'$ and $\Lambda'$ has a trivial $\Gamma$-action then $\varphi(\ker{q}) = 0$ because if $v \in \ker{q}$ then because the action on $\Lambda'$ is trivial,
\[ n \varphi(v) = \sum_{\sigma \in \Gamma / \Gamma_\Lambda} \sigma \cdot \varphi(v) = \varphi \left( \sum_{\sigma \in \Gamma / \Gamma_\Lambda} \sigma \cdot v \right) = \varphi(q(v)) = 0 \implies \varphi(v) = 0 \]
because $\Lambda'$ is torsion-free. Therefore, $\varphi$ factors uniquely through the quotient $\Lambda / \ker{q} \iso \Lambda_s$. Thus, $\Lambda_s$ is the largest $\Gamma$-invariant quotient of $\Lambda$. Therefore, let $T_s = D_k(X(T)_s)$ giving $T_s \subset T$ because $X(T) \onto X(T)_s$ and every split torus mapping to $T$ factors through $T_s$ because every map $X(T) \to \Lambda$ where $\Lambda$ has a trivial $\Gamma$-action factors through $X(T) \onto X(T)_s$. In particular, $T_s$ is the largest split subtorus containing every other split subtorus.  
\bigskip\\
Now let $\Lambda = X(T)$. Consider the canonical map,
\[ \Lambda \to \Lambda_s \times \Lambda_a \]
given by sending $v \mapsto (q(v), [v])$ with $[v] \in \Lambda / \Lambda^G$. This corresponds to the canonical map of tori via multiplication,
\[ T_s \times T_a \to T \]
To show that this map is an isogeny, it suffices to show that $\Lambda \to \Lambda_s \times \Lambda_a$ is injective with finite cokernel. First suppose that $(q(v), [v]) = 0$ then $v \in V^G$ so $q(v) = n v$ but $\Lambda$ is torsion-free so $v = 0$ proving injectivity. Tensoring by $\Q$ gives that the sequence splits so this map becomes an isomorphism proving that its kernel is torsion but is finitely generated so thus is finite. Explicitly, I claim that the image $C$ contains an index $n$ sublattice of a larger group as follows,
\[ n(\Lambda^G \times \Lambda_a) \subset C \subset \Lambda_s \times \Lambda_a \subset \Lambda^G \times \Lambda_a \]
because for any element $(n w, [nv]) \in (n \Lambda^G \times \Lambda_a)$ choose $v \in \Lambda$ representing $[v]$ then because $u = w - q(v) \in \Lambda^G$ we see that $[n v + u] = [n v]$ and $q(n v + u) = n q(v) + n u = nw$ so $n v + u \mapsto (nw, [nv])$. Therefore, $|C| \le n^{\mathrm{rank}(\Lambda)}$.

\subsection{3}

\subsubsection{(i)}

\renewcommand{\Aut}{\mathrm{Aut}}

Let $T$ be a $k$-torus. First suppose that $T$ is $k$-split of dimension $r$. Then I claim that the constant $k$-group $\GL_r(\Z)$ (which is \etale but only locally finitely presented) represents the functor,
\[ \Aut_{T/k} : S \mapsto \Aut_S(T_S) \]
By the same argument as in HW1 1(iv) we see that $\Aut_{S}(T_S)$ is given by locally constant functions $S \to \GL_r(\Z)$ and thus by $\underline{\GL_r(\Z)}(S)$.
\bigskip\\
In general, let $k'/k$ be a finite Galois extension such that $T_{k'}$ is $k'$-split. Then $\Aut_{T_{k'}/k'}$ is represented by $\underline{\GL(\Lambda)}$ where $\Lambda = X(T) = X(T_{k'}) \cong \Z^r$ without the $\Gamma$-action. Then putting the $\Gamma$-action back on $X(T)$ induces a $\Gamma$-action on $\underline{\GL(\Lambda)}$ which we descend to an \etale $k$-scheme $A$. Now,
\[ \Aut_{T/k}(S) = \Aut_{S}(T_S) = \Aut_{S_{k'}}(T_{S_k'})^G = A_{k'}(S_{k'})^G = A(S) \]
so $A$ represents $\Aut_{T/k}$. 
\bigskip\\
For Galois descent we can take the open sets of $\underline{\GL_r(\Z)}$ given by taking the Galois orbits which are finite and thus affine schemes. (WHAT IS THE POINT OF GOING UP TO KSEP??)

\subsubsection{(ii)}


Let $G$ be a connected $k$-group scheme equipped with an action on $T$ by group homomorphisms. This is equivalent to a morphism of $k$-groups $G \to \Aut_{T/k}$. However, $G_{k'} \to \Aut_{T_{k'}/k'} = \underline{\GL_r{\Z}}$ and $\underline{\GL_r(\Z)}$ is totally disconnected so the image of the connected set $G_{k'}$ ($G$ is connected and thus geometrically connected see HW1 3(i)) is trivial. Since $G$ is reduced, the image must be the trivial group. Thus by the uniqueness part of Galois descent $G \to \Aut_{T/k}$ is the trivial morphism. 
\bigskip\\
If $T$ is a normal $k$-subgroup of a connected finite type $k$-group $G$ then $G \acts T$ by group homomorphisms. Therefore, by above the action is trivial. Therefore, by definition, $T$ is central.
\bigskip\\
Consider the unipotent group of upper triangular matrices $U \subset \SL_2$. It is easy to check that $U \cong \Ga$. Furthermore, let $G \subset \SL_2$ consist of all upper triangular matrices. Then $U \subset V$ is normal. However, it is easy to check that,
\[ 
\begin{pmatrix}
a & 0 
\\
0 & a^{-1} 
\end{pmatrix} \]
acts on $\Ga$ via scaling by $a^{-2}$ and thus $\Ga$ is not central.

\subsection{4}

Let $T$ be a $k$-torus in a $k$-group $G$ of finite type.

\subsubsection{(i) DO THIS!!}

To construct a $k$-morphism $N_G(T) \to \Aut_{T/k}$ it suffices to product a natural transformation between the functors they represent. This is done via,
\[ N_G(T)(S) = \{ g \in G_S \mid g : T_S \iso T_S \} \to \Aut_{S}(T_S) = \Aut_{T/k}(S) \]
and clearly $Z_G(T)$ which represents the functor,
\[ Z_G(T)(S) = \{ g \in G_S \mid g = \id : T_S \to T_S \} \]
is exactly its kernel. Since $N_G(T)$ is a finite type $k$-group it has finitely many connected components. However, $(\Aut_{T/k})_{\bar{k}} = \underline{\Aut_{\Z}(X(T))} \cong \underline{\GL_r(\Z)}$ is discrete as a toplogical space so the image of $N_G(T)_{\bar{k}}$ must be a finite set of points. Furthermore, from the exact sequence,
\begin{center}
\begin{tikzcd}
0 \arrow[r] & Z_G(T) \arrow[r] & N_G(T) \arrow[r] & \Aut_{T/k} 
\end{tikzcd}
\end{center} 
we see that,
\begin{center}
\begin{tikzcd}
0 \arrow[r] & Z_G(T)(\bar{k}) \arrow[r] & N_G(T)(\bar{k}) \arrow[r] & \Aut_{T/k}(\bar{k}) \arrow[r, equals] & \Aut_{\Z}(X(T)) 
\end{tikzcd}
\end{center} 
and therefore the image is isomorphic to $W(G,T) = N_G(T)(\bar{k}) / Z_G(T)(\bar{k})$ and is a finite subgroup of $\Aut_{\Z}(X(T))$ because the scheme theoretic image over $\bar{k}$ contains finitely many points.
(DO THIS BETTER)
\bigskip\\
Suppose that $f : G' \to G$ is surjective with finite kernel and $T'$ is a $k$-torus in $G'$ containing $\ker{f}$ with $f(T') = T$. If we pass to $\bar{k}$ then everything is actually surjective on $\bar{k}$-points. First, choose some $g \in N_{G'}(T')(\bar{k})$ and suppose that $f(g) \in Z_G(T)(\bar{k})$. Then for any $t \in T(\bar{k})$ we have $f(gtg^{-1}t^{-1}) = f(g) f(t) f(g)^{-1} f(t)^{-1} = e$ and thus $gtg^{-1} t^{-1} \in (\ker{f})(\bar{k})$. Now the conjugation action $G'_{\bar{k}} \times T_{\bar{k}} \to G_{\bar{k}}$ with $g \in G'(\bar{k})$ fixed gives a map $T_{\bar{k}} \to G_{\bar{k}}$ which we showed factors through $\ker{f}$ on the level of $\bar{k}$-points which is enough to show that it factors $T_{\bar{k}} \to ((\ker{f})_{\bar{k}})_{\red} \to G_{\bar{k}}$ through the closed subscheme because $T$ and $G$ are smooth and thus reduced. However, $\ker{f}$ is finite and $T$ is geometrically connected so its image is a single point and thus it factors through the trivial subgroup so the action of $g$ on $T_{\bar{k}}$ is trivial proving that $g \in Z_{G'}(T')(\bar{k})$. Therefore, $W(G',T') \to W(G,T)$ is injective.
\bigskip\\
Now, let $g \in N_G(T)(\bar{k})$. Because $f : G' \to G$ is surjective we can find some $g' \in G'(\bar{k})$ such that $f(g') = g$. It suffices to show that $g' \in N_{G'}(T')(\bar{k})$. We know that for any $t' \in T'(\bar{k})$ we know that $f(g' t' g'^{-1}) = g f(t') g^{-1}$ and $f(t') \in t$ so $f(g' t' g'^{-1}) = g f(t') g^{-1} \in T(\bar{k})$ because $g \in N_G(T)(\bar{k})$ and thus $g' t' g'^{-1} \in f^{-1}(T(\bar{k}))$. Because $\ker{f} \subset T'$ we see that $f^{-1}(T(\bar{k})) = T'(\bar{k})$ and thus $g' t' g'^{-1} \in T(\bar{k})$. Therefore, the conjugation morphism given by $g' : T'_{\bar{k}} \to G'_{\bar{k}}$ sending $T(\bar{k}) \to T(\bar{k})$ and therefore factors through $T_{\bar{k}}$ so $g' \in N_{G'}(T')$. Thus, $W(G',T') \to W(G,T)$ is surjective so
\[ f : W(G',T') \iso W(G,T) \]
is an isomorphism.


\subsubsection{(ii) DO THIS}

Choosing a maximal torus $T$ is equivalent to choosing a basis of $V$. The normalizier of the torus must send the lines generated by the basis to those same lines. Therefore, any $g \in N_{G}(T)(\bar{k})$ must be a permutation matrix times an element of the torus. For $G = \GL_n, \SL_n, \PGL_n$ we are allowed any permutation matrix and thus $W(G,T) = N_G(T)(\bar{k})/Z_G(T)(\bar{k}) = N_G(T)(\bar{k})/T(\bar{k}) = S_n$ are exactly the permutation matrices since $T$ is maximal so $Z_G(T) = T$. 
\bigskip\\
For the symplectic case, $G = \Sp_{2n}$, we need to consider the block permutation matrices which give $S_n \subset W(\Sp_{2n}, T)$. However, there is also the matrix,
\[
\begin{pmatrix}
0 & - 1
\\
1 & 0 
\end{pmatrix} \]
in the normalizer for each pair of coodinates giving $W(\Sp_{2n}, T) = S_n \rtimes \left< - 1 \right>^n$ which is the group of signed permuations and equal to the Weil group of $\Sp_{2n}$.

\subsection{5}

Let $(V, q)$ be a non-degenerate quadratic space over a field $k$ with $\dim{V} \ge 2$. 

\subsubsection{(i)}

Suppose that $v \in V$ such that $q(v) = 0$. If characteristic of $k$ is not two then $B_q$ is nondengenerate so there exists $\tilde{w} \in V$ such that $B_q(v, w) = 1$. Otherwise, $q_{\bar{k}}$ is nondegenerate and thus if $v \in V^\perp$ then $q(v) \neq 0$ (see HW2 4(iii)) or else $q_{\bar{k}}(v) = 0$ and $(q_{\bar{k}}|_{V_{\bar{k}}^\perp} = 0$ so because $q(v) = 0$ we see that $v \notin V^\perp$. Therefore, $B_q(v, -) \neq 0$ so again there exist some $\tilde{w} \in V$ such that $B_q(v, w) = 1$. Then we must have $v, \tilde{w}$ be independent because $B_q(v, v) = 0$. Now let, $w = \tilde{w} + \alpha v$
and consider,
\[ q(w) = q(\tilde{w}) + \alpha^2 q(v) + \alpha B_q(v, \tilde{w}) = q(\tilde{w}) + \alpha \]
so we may choose $\alpha = - q(\tilde{w})$ such that $q(w) = 0$ and also $B_q(v, w) = B_q(v, \tilde{w}) + \alpha B_q(v, v) = 1$ because $B_q(v, v) = 2 q(v) = 0$. Therefore $H = \vspan{v,w}$ is a hyperbolic plane. Furthermore, 
\[ H^\perp = \{ x \in V \mid B_q(x, H) = 0 \} = \ker{\varphi} \quad \text{where} \quad \varphi : V \to k^2 \quad \text{via} \quad x \mapsto (B(x,w), B(x,v)) \] 
Then $\varphi|_H : H \to k^2$ takes $\alpha v + \beta w \mapsto (\alpha, \beta)$ is clearly an isomorphism so by the Rank-Nullty construction,
\[ V = H \oplus H^\perp \]
Therefore, 
\[ \Gm \subset \SO(H) \subset \SO(q) \]
where the embedded $\Gm \subset \SO(H)$ is given by sending,
\[ t \mapsto \begin{pmatrix}
t & 0
\\
0 & t^{-1}
\end{pmatrix}
\]
in the basis $v, w \in H$ because this preserves the form,
\[ q(\alpha v + \beta w) = \alpha \beta \]

\subsubsection{(ii) DO THIS}

Suppose that $\SO(q)$ contains a $k$-subgroup $S$ isomorphic to $\Gm$. Consider the $2$-dimensional $k$-split $k$-torus $T$ generated in $\GL{V}$ by $S$ and the central $\Gm$. Let $A_T \subset \End{V}$ be the corresponding commutative $k$-subalgebra. (WAIT HERE DOES $r = 2$!)  Then $A \cong k^2$ (SHOW THIS) and the inclusion,
\[ \Gm = S \embed T = R_{A/k}(\Gm) = \Gm^2 \]
is given by $t \mapsto (t^{h_1}, t^{h_2})$ because all maps $\Gm \to \Gm^2$ are of this form (WHAT THE HECK IS THIS ABOUT)
\bigskip\\
(SOMEHOW SHOW) 
Simultaneously diagonalize $S$ then we get via $\mathrm{diag}(t^{n_1}, \dots, t^{n_d})$. Then the fact that $\sum n_i = 0$ comes just from $S \subset \SO(q) \subset \SL_n$. Then we get $S \to \Gm^n$ and therefore its projections onto 
\bigskip\\
Then in the basis $\{ e_i \}$ of $V$ the subgroup $S$ acts via $\mathrm{diag}(t^{n_1}, \dots, t^{n_d})$ for $n_1 \le \cdots \le n_d$ with $\sum n_i = 0$. Since the sum is zero, clearly $n_1 < 0 < n_d$ since they are not all zero (else $S$ would be trivial). Then write,
\[ q(x_i e_i)  = \sum_{i \le j} a_{ij} x_i x_j \]
in these coodinates. Acting via $S \subset \SO(q)$ we see that,
\[ q(x_i e_i) = q(x_i t^{n_i} e_i) = \sum_{i \le j} a_{ij} t^{n_i + n_j} x_i x_j \]
Since these must be the same form we see that $a_{ij} t^{n_i + n_j} = a_{ij}$ and thus if $a_{ij} \neq 0$ we must have $n_i + n_j = 0$ so that this holds for all $t$. Let,
\[ V_{-} = \vspan{e_i \mid n_i < 0} \quad \text{and} \quad V_+ = \vspan{e_i \mid n_i > 0} \]
Then for $v \in V_{\pm}$ we see that $n_i + n_j < 0$ for all $i,j$ such that $n_i, n_j < 0$ so $a_{ij} = 0$. Therefore, 
\[ q(v) = q(v_i e_i) = \sum_{i \le j} a_{ij} v_i v_j = 0 \]
because $v_i$ is only nonzero if $n_i < 0$ and thus for $v_i v_j \neq 0$ we have $a_{ij} = 0$. Similarly, if $v \in V_+$ then $q(v) = 0$.

\section{Homework 7}

\subsection*{Cartier's Theorem}

\begin{lemma}
Let $R$ be a ring and $I \subset R$ an ideal and $D : R \to R$ a derivation. Then there exists a canonical derivation $D' : \widehat{R} \to \widehat{R}$ extending $D$ to the $I$-adic competion,
\begin{center}
\begin{tikzcd}
\widehat{R} \arrow[r, "D'"] & \widehat{R} 
\\
R \arrow[u] \arrow[r, "D"] & R \arrow[u]
\end{tikzcd}
\end{center}
\end{lemma}

\begin{proof}
Because $D$ is a derivation $D(I^n) \subset I^{n-1}$. Therefore, we get a well-defined map,
\[ D'_n : \widehat{R} \to R / I^n \xrightarrow{D} R / I^{n-1} \]
Taking the limit gives $D' : \widehat{D} \to \widehat{D}$ it is clear that $D'$ extends $D$ and is a derivation.  
\end{proof}

\begin{lemma}
A local continuous map of complete local $k$-algebras $\alpha : A \to B$ both with residue field $k$ is surjective if and only if $\alpha : \m_A/\m_A^2 \to \m_B / \m_B^2$ is surjective.
\end{lemma}

\begin{proof}
If $\alpha$ is surjective then $\alpha(A) \cap \m_B = \m_B$ but if $a \notin \m_A$ then $a$ is a unit so $\alpha(a) \notin \m_B$ and thus $\alpha(A) \cap \m_B = \alpha(\m_A)$ so $\alpha(\m_A) = \m_B$. Since $\alpha(\m_A^2) \subset \m_B^2$ this defines a surjective map $\alpha : \m_A / \m_A^2 \to \m_B / \m_B^2$.
\bigskip\\
Conversely, suppose that $\alpha : \m_A / \m_A^2 \to \m_B / \m_B^2$ is surjective. Consider,
\begin{center}
\begin{tikzcd}
0 \arrow[r] & \m_B^n / \m_B^{n+1} \arrow[r] & B / \m_{B}^{n+1} \arrow[r] & B / \m_B^n \arrow[r] & 0
\\
0 \arrow[r] & \m_A^n / \m_A^{n+1} \arrow[u] \arrow[r] & A / \m_{A}^{n+1} \arrow[u] \arrow[r] & A / \m_A^n \arrow[u] \arrow[r] & 0
\end{tikzcd}
\end{center}
since $\m_B^n$ is generated by $n$-fold products of elements of $\m_B$ we see that $\m_A^n / \m_A^{n+1} \to \m_B^n/\m_B^{n+1}$ is surjective. Explicitly, for $b_1 \cdots b_n \in \m_B^n$ lift to $a_i \in \m_A$ such that $a_i \mapsto b_i \mod \m_B^2$. Then $a_1 \cdots a_n \mapsto b_1 \cdots b_n \mod \m_{B}^{n+1}$. For the case $n = 1$ we  know $A / \m_A \to B / \m_B$ is an isomorphisms since these are both $k$ and $\alpha$ is a $k$-isomorphism. Thus, by induction we see that $\alpha : A / \m_A^n \to B / \m_B^n$ is surjective for all $n$. Then for $b \in B$ condier the projections $b_n \in B / \m_B^n$ which form a compatible sequence so we can lift to $a_n \in A / \m_A^n$ a compatible sequence (compatibility is given by lifting using the above diagram first along $A / \m_A^{n+1} \to A / \m_A^n$ and then correcting by an element of $\m_A^n / \m_A^{n+1}$ so make the image in $B / \m_B^{n+1}$ correct). Because $A$ is complete this gives an element $a \in A$ such that $\alpha(a)_n = b_n$ and thus because $B$ is complete $\alpha(a) = b$ so $\alpha$ is surjective.
\end{proof}

\begin{thm}[Cartier]
Any finite type $k$-group scheme over a field of characteristic zero is smooth.
\end{thm}

\begin{proof}
We will show that if $X$ is any scheme over $k$ and $x \in X(k)$ (I THINK WE NEED IT TO BE A RATIONAL POINT) such that there exists vector fields $D_1, \dots, D_n$ on a neighborhood of $x$ with $n = \dim_k \m_x / \m_x^2$ that induce a basis of $T_x = (\m_x / \m_x^2)^\vee$ then $X$ is smooth at $x$.
\bigskip\\
Choose $x_i$ in $\m_x$ such that they form a basis of $\m_x / \m_x^2$ and $(D_i x_j)(x) = \delta_{ij}$ (this is the dual basis). These extend to derivations on the completion $D_i : \widehat{\stalk{X}{x}} \to \widehat{\stalk{X}{x}}$. The elements $x_i$ define a local ring map $k[t_1, \dots, t_n]_{(t_1, \dots, t_n)} \to \stalk{X}{x}$. By the universal property of completion for local rings we get a unique continuous ring map,
\[ \alpha : k[[t_1, \dots, t_n]] \to \widehat{\stalk{X}{x}} \]
Furthermore, the derivations give a continuous ring map $\beta : \widehat{\stalk{X}{x}} \to k[[t_1, \dots, t_n]]$ via,
\[ \beta(f) = \sum_{\nu_1, \dots, \nu_n \ge 0} \frac{D^\nu f}{\nu!}(x) \cdot t^{\nu} \]
where $D^\nu f = D_1^{\nu_1} \cdots D_n^{\nu_n} f$ and $\nu! = \nu_1 ! \cdots \nu_n !$ and $t^\nu = t_1^{\nu_1} \cdots t_n^{\nu_n}$. This is a $k$-morphism of rings by the exponential series formulae. Furthermore, $\beta$ is continuous and local (CHECK THIS) Defining this map requires characteristic zero. Because $\stalk{X}{x} /\m_x = k$ and $t_i \mapsto x_i$ which generate $\m_x$ we see that $\alpha$ is surjective by the lemma. Furthermore, 
\[ \beta(x_i) = t_i \mod (t_1, \dots, t_n)^2 \]
because $D_j(x_i) = \delta_{ij}$ on $\m_x / \m_x^2$. Therefore, by the lemma, $\beta$ is surjective so $\beta \circ \alpha$ is a surjective endomorphism of $k[[t_1, \dots, t_n]]$ and thus is an isomorphism so $\alpha$ is also injective and thus an isomorphism proving that $\widehat{\stalk{X}{x}} \cong k[[t_1, \dots, t_n]]$ and therefore $\stalk{X}{x}$ is regular. Since $k$ has characteristic zero this means that $X$ is smooth at $x$.


\end{proof}

\subsection{1}

\subsubsection{(i)}

Consider the derivation $D = \partial_x : R[x] \to R[x]$ on $(\Ga)_R$. We need to check that $D(f \circ \ell_g) = D(f) \circ \ell_g$ for $g \in \Ga(R) = R$ and $f \in R[x]$. Indeed,
\[ \partial_x (f(x + g)) = (\partial_x f)(x + g) \]
by the chain rule.
\bigskip\\ 
Likewise consider the derivation $D = t \partial_t : R[t, t^{-1}] \to R[t,t^{-1}]$ on $(\Gm)_R$. We need to check that $D(f \circ \ell_g) = D(f) \circ \ell_g$ for $g \in \Gm(R) = R^\times$ and $f \in R[t,t^{-1}]$. Indeed,
\[ t \partial_t (f(gt)) = gt (\partial_t f)(gt) = (t \partial_t f)(gt)  \]
by the chain rule.

\subsubsection{(ii)}

Let $A$ be a finite dimensional associative $k$-algebra, and $\underline{A^\times}$ the associated $k$-group of units. This scheme represents the functor $R \mapsto (A \otimes_k R)^\times$. Thus we consider,
\[ T_e(\underline{A}^\times) = \ker{(\underline{A}^\times(k[\epsilon]) \to \underline{A}^\times(k))} = \ker{(A[\epsilon]^\times \to A^\times)} \]
We can write any element of $A[\epsilon]$ as $a + b \epsilon$ for $a, b \in A$ and to be in the kernel we require that $a = 1$. Then notice that $(1 + b \epsilon)(1 - b \epsilon) = 1$ so $T_e(\underline{A}^\times) = A$ as a $k$-vectorspace. Now we need to determine the Lie bracket. Applying Lemma A.7.4 of PRG, for any $v,w \in \g$ inside $G(k[\epsilon, \epsilon'])$ we have,
\[ (1 + v \epsilon)(1 + w \epsilon')(1 - v \epsilon)(1 - v \epsilon') = 1 + \epsilon \epsilon' [v, w] \]
In our case, $G(k[\epsilon, \epsilon']) = (A[\epsilon, \epsilon')^\times$ and $v = a, w = a'$ for $a,a' \in A$ embedded in the obvious way so,
\begin{align*}
(1 + a \epsilon)(1 + a' \epsilon')(1 - a \epsilon)(1 - a' \epsilon') & = (1 + a \epsilon + a' \epsilon' + aa' \epsilon \epsilon') (1 - a \epsilon - a' \epsilon' - a'a \epsilon \epsilon') 
\\
& = 1 + (aa' - a'a) \epsilon \epsilon'
\end{align*}
and therefore, $[a, a'] = aa' - a'a \in A$.
\bigskip\\
Apply this to $A = \End{V}$ for some vectorspace $V$. Then $\underline{A^\times} = \GL(V)$ so we see that $\Lie(\GL(V)) = \End{V}$ with the bracket $[X, Y] = XY - YX$ for $X,Y \in \End{V}$. We call this Lie algebra $\gl(V) \cong \Lie(\GL(V))$. 
\bigskip\\
Now consider the closed subgroup $\SL(V) \embed \GL(V)$ then $\Lie(\SL(V)) \embed \Lie(\GL(V))$. From the functor of points, 
\[ T_e(\SL(V)) = \{ I + \epsilon X \in \End{V_{k[\epsilon]}} \mid \det{(I + \epsilon X)} = 1 + \epsilon \tr{X} = 1 \} = \{ X \in \End{V} \mid \tr{X} = 0 \} = \sl(V) \]
then the inclusion of Lie algebras determines the bracket to be $[X,Y] = XY - YX$ and $\tr{(XY - YX)} = \tr{XY} - \tr{YX} = 0$ so this make sense\footnote{Notice that if $X,Y \in \sl(V)$ we may have $XY \notin \sl(V)$ for example $A = \begin{pmatrix}
0 & 1
\\
1 & 0
\end{pmatrix} \in \sl(k^2)$ but $X^2 = I \notin \sl(k^2)$ so the commutator not the algebra structure is what is preserved in the inclusion}. Next, we have $\GL(V) \onto \PGL(V)$. From the functor of points it is easy to compute,
\[ \pgl(V) = \Lie(\PGL(V)) = \gl(V) / \{ \lambda I \} \]
where the bracket structure is induced on $\pgl(V)$ via the surjection $\gl(V) \onto \pgl(V)$. Next, the symplectic algebra has an inclusion $\sp(B) \subset \gl(V)$ so the bracket structue is given by $[A,B] = AB - BA$ and the functor of points tells us that,
\[ \sp(B) = T_e(\Sp(B)) = \{ X \in \GL(V) \mid B(X-,-) + B(-,X-) = 0 \} \]
Next, let $\GSp(B) \subset \GL(V)$ be the algebraic group representing the functor,
\[ \GSp(B) : A \mapsto \{ X \in \End{V_A} \mid B(Xv,Xw) = \lambda(X) B(v,w) \text{ for all } v,w \in V_A \text{ and some } \lambda(X) \in A \} \]
Then from the functor of points,
\begin{align*}
\gsp(B) & = \Lie(\GSp(B)) 
\\
& = \{ I + \epsilon X \in \End{V_{k[\epsilon]}} \mid B(-,-) + B(X-,-) + B(-,X-) = (1 + \epsilon \lambda(X)) B(-,-) \}
\\
& = \{ X \in \End{V} \mid B(X-,-) + B(-,X-) = \lambda(X) B(-,-) \}
\end{align*}
with the standard bracket structure since $\GSp(B) \embed \GL(V)$. Finally, $\SO(q) \embed \GL(V)$ and from the functor of points,
\begin{align*}
\so(q) & = \Lie(\SO(q)) = \{ I + \epsilon X \in \End{V_{k[\epsilon]}} \mid q(-) + \epsilon B_q(X-,-) = q(-) \} 
\\
& = \{ X \in \End{V} \mid B_q(X-,-) = 0 \}
\end{align*}
with the bracket $[X,Y] = XY - YX$ because $\SO(q) \embed \GL(V)$.

\subsubsection{(iii) HMMM}

By embedding each in $\GL_n$ 
we compute the $p$-Lie algebra structure. First, $\underline{A^\times} \embed \GL(A)$ then $\Lie(\underline{A^\times}) = A \to \End{A} \cong \gl_n$ is an algebra map as well as a Lie algebra map (JUTISFY). Therefore $\Lie_p(\underline{A^\times}) = A$ with the $p$-Lie structure $a \mapsto a^p$ since it is contained in $\gl(V)$ on which $X \mapsto X^p$ is the $p$-Lie structure.
\bigskip\\
We know that $\Gm = \GL(k)$ and thus $\Lie_p(\Gm) = k$ with $a \mapsto a^p$ being the $p$-Lie structure. In particular, if $k = \FF_p$ then the $p$-map is trivial. For $\Ga \embed \GL(k^2)$ the $p$-lie structue is given via the embedding,
\[ t \mapsto 
\begin{pmatrix}
1 & t 
\\
0 & 1
\end{pmatrix} \]
Then,
\[ \begin{pmatrix}
1 & t
\\
0 & 1
\end{pmatrix}
\mapsto 
\begin{pmatrix}
1 & t
\\
0 & 1
\end{pmatrix}^p
= \begin{pmatrix}
1 & pt
\\
0 & 1
\end{pmatrix}
= \begin{pmatrix}
1 & 0
\\
0 & 1
\end{pmatrix} \]
becuase $k$ has characteristic $p$ and therefore the $p$-structue is given by $t \mapsto 0$ on $\Lie_p(\Gm) = k$. 
\bigskip\\
We can calcuate these explicitly in terms of the left-invariant derivations. On $\Gm$ for any $a \in k$ we have,
\[ (a t \partial_t)^p = a^p (t \partial_t)^p = a^p (t \partial_t) \]
To see this, I claim that,
\[ (t \partial_t)^n = \left( \sum_{k = 1}^n c_{n,k} t^k \partial_t^k  \right) \]
where $c_{n,k}$ are the Stirling numbers of the second kind. We prove this formula by induction. For the case $n = 1$ this is trivial. Now,
\begin{align*}
(t \partial_t)^{n+1} & = (t \partial_t) (t \partial_t)^n = \sum_{k = 1}^n c_{n,k} (t \partial_t) (t^k \partial^k_t) 
\\
& = \sum_{k = 1}^n c_{n,k} (t^{k+1} \partial_t^{k+1} + k t^k \partial_t^k) = \sum_{k = 1}^{n+1} (k c_{n,k} + c_{n,k-1}) t^k \partial_t^k
\\
& = \sum_{k = 1}^{n+1} c_{n+1,k} \, t^k \partial_t^k
\end{align*} 
However, because the characteristic is $p$, the operator $\partial_t^p = 0$ and furthermore, $p \divides c_{p,k}$ for $1 < k < p$ (CITE THIS) so as an operator, $(t \partial_t)^p = t \partial_t$.
\bigskip\\
Likewise, the left-invariant forms on $\Ga$ are generated by $\partial_x$ and therefore for $a \in k$,
\[ (a \partial_x)^p = a^p \partial_x^p = 0 \]
because $\partial_x^p = 0$ is zero as an operator and therefore the $p$-map is zero on $\Lie_p(\Ga)$.

\subsection{2}

Let $G$ be a smooth group of dimension $d > 0$ over $k$.

\subsubsection{(i) WRITE DOWN THE FORMS}

A left-invariant differential $i$-form $\omega \in \Omega^{i, \ell}_{G}(G)$ is a global section $\omega \in \Gamma(G, \Omega^i_G)$ such that for each $k$-algebra $R$ and $g \in G(R)$ consider the morphism $\ell_g : G_R \to G_R$ which must satisfy,
\[ \ell_g^* \omega_R = \omega_R \]
This is (CHECK) the same as requring $m^* \omega = p_2^* \omega$ as forms on $G \times G$ under the maps $m : G \times G \to G$ and $p_2 : G \times G \to G$.
There is a restriction map,
\[ \Omega^{i, \ell}_{G}(G) \to \Omega^i_{G, e} \otimes_{\stalk{G}{e}} \kappa(e) \cong \bigwedge^i \g^* \]
where the second isomorphism holds because wedges commute with taking stalks and tensor products. I claim that this map is an isomorphism. Given this, $\dim_k{\Omega^{i,\ell}_G(G)} = {\dim{\g^*} \choose i} = {d \choose i}$. 
\bigskip\\
To prove the restriction map is an isomorphism we refer to the exact proof given in PRG for the $i = -1$ case of derivations. The idea is that the value of the form at $e$ uniquely determines the form everywhere by left translating around. To fully justify this notice that $\Omega_G$ is the trivial bundle so choosing a global frame we see that $\bigwedge^i \Omega_G$ is a global wedge.
\bigskip\\
For $\GL(V)$ we have top form,
\[ \frac{\bigwedge\limits_{i,j} \d{x_{ij}}}{(\det{x_{ij}})^n} \]

\subsubsection{(ii)}

Let $g \in G(R)$ and $\ell_g$ and $r_g$ be the left and right translation maps $G_R \to G_R$. Because $\ell_g$ and $r_g$ commute, we see that for any $\omega \in \Omega^{i, \ell}_{G}(G) \otimes_k R$ that the form $r_g^* \omega$ is a left-invariant form because,
\[ \ell_{g'}^* (r_g^* \omega) = r_g^* (\ell_{g'}^* \omega) = r_g^* \omega \]
for any $g' \in G(R)$ and thus $r^* \omega \in \Omega^{i, \ell}_{G}(G) \otimes_k R$. However, because $\ell_g \neq r_g$ in general we see that $r_g^*$ acts nontrivially on $\Omega^{i, \ell}_{G}(G) \otimes_k R$ while for any left-invariant form $\ell_g^* \omega = \omega$ by definition. 
\bigskip\\
This defines a functor $\underline{G} \to \underline{\mathrm{Aut}(\Omega^{i, \ell}_{G}(G))}$ so we get a $G$-representation on $\Omega^{i, \ell}_{G}(G)$. For the case $i = d$ we know that $\Omega^{d, \ell}_G(G)$ is one-dimensional and therefore $\underline{\mathrm{Aut}(\Omega^{d, \ell}_{G}(G))}$ is represented by $\Gm$ canonically (via the trace). Therefore, we get a character $\chi_G : G \to \Gm$. Furthermore, if $g \in Z_G(R)$ then $\ell_g = r_g$ because it commutes with everything and therefore $r_g^* \omega = \ell_g^* \omega = \omega$ because $\omega$ is left-invariant. Therefore, $\chi|_{Z_G} = 1$ because we showed that this holds at the level of functors. Now suppose that $G / Z_G = \D(G/Z_G)$. Since $\chi_G : G / Z_G \to \Gm$ is a map to an abelian group it automatically kills the derived subgroup so $\chi_G = 1$.
\bigskip\\
To see why any map $G \to A$ to an abelian $k$-group $A$ kills $\D(G)$ look at $\bar{k}$-points. The derived subgroup has the expected geometric points, $\D(G)(\bar{k}) = [G(\bar{k}), G(\bar{k})]$ so the image of the group homomorphism $\D(G)(\bar{k}) \to A(\bar{k})$ is trivial because $A(\bar{k})$ is abelian. Therefore, since $\D(G)$ is reduced this means that the map $\D(G) \to A$ factors through the trivial map.

\subsubsection{(iii) DO THIS!!}

\subsection{3}

Let $K/k$ be a degree-$2$ finite \etale algebra (i.e. separable quadratic field extension or $k \times k$) and let $\sigma$ be the unique non-trivial $k$-automorphism of $K$. Notice that $(K)^\sigma = k$. A $\sigma$-hermitian space is a pair $(V, h)$ consisting of a finite free $K$-module equipped with a perfect $\sigma$-semilinear form $h : V \times V \to K$ meaning,
\[ h(cv, v') = ch(v,v') \quad \quad h(v,cv') = \sigma(c) h(v,v') \quad \quad h(v',v) = \sigma(h(v,v')) \]
Notice that $v \mapsto h(v,v)$ is a quadratic form $q_h : V \to k$ over $k$ such that,
\[ q_h(cv) = N_{K/k}(c) q_h(v) \]
for $c \in K$ and $v \in V$ also notice that $\dim_k{V}$ is even. The unitary group $\U(h)$ over $k$ is the subgroup of $R_{K/k}(\GL(V))$ preserving $h$. Using $R_{K/k}(\SL(V))$ gives the special unitary group $\SU(h)$. For exmaple let $V = F$ be finite \etale over $K$ with an involution $\sigma'$ lifting $\sigma$ and $h(v,v') := \mathrm{Tr}_{F/K}(v \sigma'(v'))$ (e.g. if $F,K$ are CM fields, $k$ is totally real, and $\sigma'$ and $\sigma$ are complex conjugation).

\subsubsection{(i)}

If $K = k \times k$ let $e = (1,0)$ and $e' = (0,1)$ denote the two nontrivial idempotent elements. Let $V_0 = e V$ and $V_1 = e' V$.
Because $1 = e + e'$ we see that $V_0 + V_1 = V$. Furthermore, suppose that $v \in V_0 \cap V_1$ then $v = e w = e' w'$ then $ev = e^2 w = ew = v$ but $ev = ee'w' = 0$ and thus $v = 0$. Therefore $V = V_0 \oplus V_1$. For $v_0 \in V_0$ and $v_1 \in V_1$, define $B(v_0, v_1) = h(v_0, v_1)$. I claim this is a perfect pairing. Notice that,
\[ h(v_0, v_1) = h(e v_0, v_1) = e h(v_0, v_1) \]
so $B(v_0, v_1) \in e K = k$ in the first factor. Now,
\begin{align*}
h(v, w) &= h(ev, ew) + h(ev, e'w) + h(e'v, ew) + h(e'v, e'w)
\\
& = e e' h(v,w) + h(ev, e'w) + \sigma(h(ew,e'v)) + e' e h(v, w) 
= (B(v_0, w_1), B(w_0, v_1))
\end{align*} 
therefore because $h$ is a perfect pairing we see that $B : V_0 \times  V_1 \to k$ must be perfect such that every $v_0$ has some $w_1$ such that $B(v_0, w_1) = 1$ and visa versa. This naturally (in $h$) identifies $V_1 = V_0^\vee$ via $v_1 \mapsto B(-,v_1)$ under which $h$ becomes the form,
\[ h((v, \ell), (v', \ell')) = (\ell'(v), \ell(v')) \]
Then,
\begin{align*}
\U(h)(A) & = \{ g \in \GL(V_A) \mid h(gv,gw) = h(v,w) \}
\\
& = \{ g \in \GL((V_0)_A \oplus (V_0)_A^\vee) \mid ((g \ell')(gv), (g \ell)(gv')) = (\ell'(v), \ell(v')) \}
\end{align*}
and thus $g$ is block diagonal with the second block the inverse contragradient of the first meaning that $\U(h) \iso \GL(V_0)$. Furthermore, $\SU(h) \subset R_{K/k}(\SL(V))$ contains only those matrices with $K$-trace equal to $1$ which means that the trace of the first block (because a $k$-basis of $V_0$ is a $K$-basis of $V$) is one and thus $\SU(h) \iso \SL(V_0)$ under this isomorphism.
Then $q_h(v,\ell) = (\ell(v), \ell(v))$ which lies in $k \to k \times k$ the $\sigma$-invariant subspace so we identify $q_h(v,\ell) = \ell(v)$ and thus $q_h : V_0 \times V_0^\vee \to k$ is the canonical perfect pairing of $V_0$. If we consider the associated form,
\[ B_{q}((v, \ell), (v', \ell')) = q_h(v + v', \ell + \ell') - q_h(v,\ell) - q_h(v',\ell') = \ell(v') + \ell'(v) \]
this is always nondegenerate because for nonzero $(v, \ell)$ either $v \neq 0$ in which case take $(v',\ell')$ such that $\ell'(v) = 1$ and $\ell(v') = 0$ or $\ell \neq 0$ in which case take $(v', \ell')$ such that $\ell'(v) = 0$ and $\ell(v') = 1$.

\subsubsection{(ii)}
Let $K$ be a field. Notice that $K \otimes_k K \cong K \times K$ as a $K$-algebra and therefore we reduce the above case for $K \otimes_k K$ over $K$. Indeed, for any $K$-algebra $A$,
\begin{align*}
\U(h)_K(A) & = \{ g \in \GL(V \otimes_K K \otimes_k A) \mid h(gv,gw) = h(v,w) \} 
\\
& = \{ g \in \GL(V \otimes_K (K \otimes_k K) \otimes_K A) \mid h(gv, gw) = h(v,w) \} = \U(h_K)(A)
\end{align*}
for the above case of a split \etale $K$-algebra. Thus by the previous part, $\U(h)_K \cong \GL_n$ and $\SU(h)_K \cong \SL_n$ via the same isomorphism by the above construction. Therefore, $\U(h)$ and $\SU(h)$ are forms of $\GL_n$ and $\SL_n$ respectively. Since these are geometric properties, $\U(h)$ and $\SU(h)$ are smooth and connected (and thus geometrically connected). Furthermore, 
\[ \D(\U(h))(\bar{k}) = [\U(h)(\bar{k}), \U(h)(\bar{k})] = [\GL_n(\bar{k}), \GL_n(\bar{k})] = \SL_n(\bar{k}) = \SU(h)(\bar{k}) \]
and thus the closed $k$-subgroups $\D(\U(h))$ and $\SU(h)(\bar{k})$ have the same geometric points and thus $\D(\U(h)) = \SU(h)$.
\bigskip\\
Now consider,
\[ Z = \ker{(N_{K/k} : R_{K/k}(\Gm) \to \Gm)} \]
We compute,
\[ Z(A) = \ker{(N_{K/k} : (A \otimes_k K)^\times \to A^\times)} \]
Now via Galois descent,
\[ Z(A) = Z(A \otimes_k K)^\sigma = \ker{(N_{K/k} : (A \otimes_K (K \otimes_k K))^\times \to (A \otimes_k K)^\times)}^\sigma  \]
Therefore, it suffices to look at $K$-algebras $B$ and,
\[ Z(B) = \ker{(N_{K/k} : (B \otimes_K (K \otimes_k K))^\times \to B^\times} = \ker{(N_{K/k} : B^\times \times B^\times \to B^\times)} = \{ (\gamma, \gamma^{-1}) \mid \gamma \in B^\times \} \]
Furthermore the identification $\GL_n \to \U(h)_K \subset (R_{K/k} (\GL(V)))_K$ via $X \mapsto (X, (X^\vee)^{-1})$ alows us to conclude that the central diagonal $\Gm$ inside $\GL_n$ maps isomorphically to $Z(B)$ via $\gamma \mapsto (\gamma, \gamma^{-1})$ at the level of $B$-points. Because, $\GL_n \to \U(h)_K$ is an isomorphism we see that $Z_{\U(h)}(B) = Z_{\U(h)_K}(B) = Z(B)$ (formation of $Z$ commutes with base changes for fields) inside $\U(h)(B)$ and therefore $Z_{\U(h)} = Z$ as closed $k$-subgroups of $\U(h)$ by the uniqueness part of Galois descent.
\bigskip\\
Consider $V(q_h) \subset \P(V)$ then $V(q_h)_K = V(q_{h,K}) \subset \P(V_K)$ and $q_{h,K}$ is identified with the above form on $V_0 \oplus V_0^\vee$ which we showed explicitly is nondegenerate and thus $V(q_{h,K})$ is smooth so $V(q_h)$ is also smooth and hence $q_h$ is nondegenerate. Finally, by Corollary A.7.6 in PRG, we see that $\Lie(R_{K/k}(\GL(V))) = \Lie(\GL(V)) = \gl(V)$ viewed as a $k$-Lie algebra. From the closed embedding $\U(h) \embed R_{K/k}(\GL(V))$ we obtain an inclusion of lie algebras $\Lie(\U(h)) \embed \gl(V)$ so the bracket is given explicitly by commutator. Moreover,
\begin{align*}
T_e(\U(h)) & = \ker{(\U(h)(k[\epsilon]) \to \U(h)(k))} = \{ 1 + \epsilon X \mid \forall u,v \in V_{k[\epsilon]} : h((1 + \epsilon X)u, (1 + \epsilon X)v) = h(v,u) \}
\\
& = \{ X \in \gl(V) \mid \forall u,v \in V_k : h(Xu,v) + h(u,Xv) = 0 \}
\end{align*}

\subsubsection{(iii) DO!!}

It is clear that for any $g \in \U(h)(A)$ that by definition $h(gv,gw) = h(v,w)$ for $v,w \in V_A$ (notice $g$ is defined inside the Weil restriction but $V_A = V \otimes_K (K \otimes_k A) = V \otimes_k A$ and thus this makes sense inside $\GL(V)$ with $V$ viewed as a $k$-algebra as well where in the first case we are taking $K \otimes_k A$-linear maps and in the second case we are allowing just $A$-linear maps) and thus in paticular $q_h(gv) = h(gv,gv) = h(v,v) = q_h(v)$ so $g \in \SO(q_h)(A)$ (here thinking of $V$ as a $k$-algebra). 
\bigskip\\
In the split case we know that $\GL_n \iso \U(h)$ via identifying $V \cong V_0 \oplus V_0^\vee$ and 
\[ h((v, \ell), (v', \ell')) = (\ell'(v), \ell(v')) \]
Therefore, the form is equal to $q_h((v, \ell)) = \ell(v)$. Then we write $X \in \SO(q_h)(A)$ as block matrices,
\[ 
\begin{pmatrix}
X_{11} & X_{12}
\\
X_{21} & X_{22}
\end{pmatrix} \]
such that $(v, \ell) \mapsto (X_{11} v + X_{12} \ell, X_{21} v + X_{22} \ell)$ and thus we require that,
\[ \ell(v) = (X_{21} v + X_{22} \ell)(X_{11} v + X_{12} \ell) = \ell(X_{22}^\vee X_{11} v) + (X_{21} v)(X_{12} \ell) + (X_{21} v)(X_{11} v) + \ell(X_{22}^\vee X_{12} \ell)  \]
Setting $\ell = 0$ we see that $(X_{21} v)(X_{11} v) = 0$. Setting $v = 0$ we see that $\ell(X_{22}^\vee X_{12} \ell) = 0$. Therefore our condition becomes,
\[ \ell(X_{22}^\vee X_{11} v) + (X_{21} v)(X_{12} \ell) \]
and the matrices in $X \in \U(h)(A)$ are exactly the block diagonal ones which must then have $X_{22}^\vee X_{11} = I$ so $X_{22} = (X_{11}^\vee)^{-1}$ as we saw before.
\bigskip\\
Consider the case $k = \RR$. For the split case we have identified the closed subgroup $\U(h) \subset \SO(q_h)$ so we need to consider the nonsplit case of $K = \CC$. This gives a closed $\RR$- subgroup $\U(h) \subset \SO(q_h)$ whose $\RR$-points are,
\[ \U(n) \subset \SO(2n) \]
as real Lie groups and whose $\CC$-points correspond to the split case above for $k = \CC$. (WHAT MORE SHOULD I SAY?)

\subsection{4}

Let $H$ be a smooth $k$-group acting on a separated finite type $k$-scheme $Y$. For a $k$-scheme $S$, let $Y^H(S)$ be the set of $y \in Y(S)$ invariant under the $H_S$-action on $Y_S$. 

\subsubsection{(i) HMM}

Let $k = k^\sep$. Consider the closed subscheme,
\[ Z = \bigcap_{h \in H(k)} Y^h = \bigcap_{h \in H(k)} \alpha_h^{-1}(\Delta_{Y/k}) \]
for $\alpha_h : Y \to Y \times Y$ via $y \mapsto (y, h \cdot y)$. It suffices to prove that $Z(A)$ and $Y^H(A)$ are the same functor on $k$-algebras,
\begin{align*}
Y^H(A) = \{ y \in Y(A) \mid H(A') \cdot y_{A'} = y_{A'} \text{ for all } A' \text{ over } A \} 
\end{align*}
furthermore,
\begin{align*}
Z(A) = \bigcap_{h \in H(k)} \{ y \in Y(A) \mid h \cdot y = y \} 
\end{align*}
Thus, clearly $Y^H(A) \subset Z(A)$. Now the $k$-points are dense in $H$ (because $k = k^\sep$). I claim that the points $\Spec{A} \to H_A$ comming from $H(k)$ are schematically dense. Given that if the two maps $H_A \to Y_A$ given by the orbit map and the constant map agree on $H(k)$ then they are equal.
\bigskip\\
To show the maps $\Spec{A} \to H_A$ are schematically dense we d  

\subsubsection{(ii) DO}

For $y \in Y^H(k)$ notice that $y \in Y(k)$ is fixed by the $H$-action on $Y$. Therefore, $H(A)$ acts on $Y(A)$ fixing $y$ meaning that $H(A)$ acts on 
\[ T_{y_A}(Y_A) = \ker{(Y(A[\epsilon]) \to Y(A))} = \ker{(Y(k[\epsilon]) \to Y(k))} \otimes_k A = T_y(Y) \otimes_k A \]
where the equality follows from $Y$ being locally of finite presentation (Remark 2.1.7 in PRG). This gives a functorial representation $\underline{H} \to \underline{\mathrm{Aut}}(T_y(Y))$. Now we want to compute the invariants. 

\subsubsection{(iii) DO}


Assume that $H$ is a closed subgroup of a $k$-group $G$ of finite type. Let $\g = \Lie(G)$ and $\h = \Lie(H)$. Then $G$ acts on itself by conjugation so consider $Z_G(H)$. I claim that $Z_G(H) = G^H$ because,
\[ Z_G(H)(A) = \{ g \in G(A) \mid g \text{ acts trivially on } H_A \} \]
and likewise,
\[ G^H(A) = \{ g \in G(A) \mid H_A \text{ acts trivially on } g \} \]
which are the same because $ghg^{-1} = h$ iff $hgh^{-1} = g$. Therefore,
\[ T_e(Z_G(H)) = T_e(G^H) = (T_e G)^H = \g^H \]
Furthermore, 

\subsection{5}

A diagram,
\begin{center}
\begin{tikzcd}
1 \arrow[r] & G' \arrow[r, "j"] & G \arrow[r, "\pi"] & G'' \arrow[r] & 1
\end{tikzcd}
\end{center}
of finite type $k$-groups is called exact if $\pi$ is faithfully flat and $(j : G' \to G) = \ker{\pi}$. 

\subsubsection{(i)}

By definition, $\pi : G \to G''$ is flat surjective and a group homomorphism with kernel $G'$ and thus is $G'$-invariant. Thus it suffices to show that,
\[ G \times G' \to G \times_{G''} G \]
is an isomorphism. We show this via Yoneda. Because $G' = \mathrm{eq}(\pi, e)$. Given a map $S \to G \times_{G''} G$ meaning a pair of maps $a,b : S \to G$ such that $\pi \circ a = \pi \circ b$ then $q = m \circ (\iota, \id) \circ (a,b) : S \to G$ satisfies $\pi \circ q = e$ and thus factors through $j : G' \to G$ so we write $q = q' \circ j$. Then I claim that $(a, q') : S \to G \times G'$ maps to $(a,b)$. Indeed, 
\[ (\id, m \circ (\id, j)) \circ (a, q') = (a, m \circ (a, q)) = (a, b) \]
because,
\[ m \circ (a,q) = m \circ (a, m \circ (\iota, \id) \circ (a,b)) = m \circ (m \circ (a, \iota a), b) = m \circ (e, b) = b \]
This gives a natural inverse to $\Hom{}{S}{G \times G'} \to \Hom{}{S}{G \times_X G}$ proving that $G \times H \iso G \times_X G$ is an isomorphism and thus $G''$ is the quotient of $G' \to G$. 
\bigskip\\
Consider a diagram of tori,
\begin{center}
\begin{tikzcd}
1 \arrow[r] & T' \arrow[r] & T \arrow[r] & T'' \arrow[r] & 1
\end{tikzcd}
\end{center}
and the associated diagram of $\Gamma$-lattices,
\begin{center}
\begin{tikzcd}
1 \arrow[r] & X(T'') \arrow[r] & X(T) \arrow[r] & X(T') \arrow[r] & 1
\end{tikzcd}
\end{center}
Because $X$ is an anti-equivalence of categories, we know that $T' = \ker{(T \to T'')}$ if and only if $X(T') = \coker{(X(T) \to X(T''))}$. Furthermore, we prove that $T \to T''$ is surjective if and only if $X(T'') \to X(T)$ is injective. Therefore the sequence of tori is exact if and only if the sequence of $\Gamma$-modules is exact. 


\subsubsection{(ii) DO THIS}

If $G"$ is finite then $\pi : G \to G''$ is surjective and flat (this is just what faithfully flat means) with fintie kernel (by definition) and thus is an isogeny. Thus. it is clear that exact with finite kernel is equivalent to our previous definition of isogeny.
\bigskip\\
Let $\pi : G \to G''$ be an isogeny. Then use Claim 12.1.3 to see that $\pi$ is actually finite flat of constant degree (DO THIS OUT)
\bigskip\\
We already proved that $\pi_n : \SL_n \to \PGL_n$ is surjective with kernel $\mu_n$ in HW3 2(iii). Since $\dim{\SL_n} = n^2 - 1 = \dim{\PGL_n}$ and both are smooth, we see that by HW6 2(i) that $\pi_n : \SL_n \to \PGL_n$ is an isogeny. Looking forward to the next part, we expect that $\Lie(\pi_n)$ is surjective when the kernel $\mu_n$ is smooth which happens exactly when $p \ndivides n$ where $p$ is the characteristic of $k$. I claim this is exactly when $\Lie(\pi_n)$ is surjective.
\bigskip\\
We have shown that $\Lie(\SL_n) = \{ A \in M_n(k) \mid \tr{A} = 0 \}$ and $\Lie(\PGL_n)  = M_n(k) / \{ \lambda I \}$ with the additive structure and $\Lie(\pi_n)$ is $A \mapsto [A]$ via the inclusion and then the quotient map. Thus $\Lie(\pi_n)$ is surjective exactly when every matrix $A \in M_n(k)$ differs by $\lambda I$ from a traceless matrix. Notice that,
\[ \tr{(A - \lambda I)} = \tr{A} - n \lambda \]
so we can set $\lambda = n^{-1} \tr{A}$ when $n$ is invertible in $k$. Otherwise, $p \divides n$ so $n = 0$ in $k$ and thus take any $A$ with $\tr{A} \neq 0$ then,
\[ \tr{(A - \lambda I)} = \tr{A} - n \lambda = \tr{A} \neq 0 \]
so $[A]$ cannot be in the image of $\Lie(\pi_n)$. For example, we can always take $A$ to be the matrix with a single $1$ in the upper left-hand corner and all other entries zero.


\subsubsection{(iii)}

Consider a short exact sequence of finite type $k$-groups
\begin{center}
\begin{tikzcd}
1 \arrow[r] & G' \arrow[r, "j"] & G \arrow[r, "\pi"] & G'' \arrow[r] & 1
\end{tikzcd}
\end{center}
Then we get a diagram,
\begin{center}
\begin{tikzcd}
& 1 \arrow[d] & 1 \arrow[d] & 1 \arrow[d]
\\
1 \arrow[r] & T_e(G') \arrow[d] \arrow[r] & T_e(G) \arrow[r] \arrow[d] & T_e(G'') 
\\
1 \arrow[r] & G'(k[\epsilon]) \arrow[r] \arrow[d] & G(k[\epsilon]) \arrow[r] \arrow[d] & G''(k[\epsilon]) \arrow[d]
\\
1 \arrow[r] & G'(k) \arrow[r] & G(k) \arrow[r] & G''(k) 
\end{tikzcd}
\end{center}
which implies that the sequence of kernels is left-exact,
\begin{center}
\begin{tikzcd}
1 \arrow[r] & T_e(G') \arrow[r] & T_e(G) \arrow[r] & T_e(G'')
\end{tikzcd}
\end{center}
we have shown that these maps on tangent spaces are additive and because $j$ and $\pi$ are group maps these preserve the Lie bracket. Therefore, we get a left-exact sequence of Lie algebras,
\begin{center}
\begin{tikzcd}
1 \arrow[r] & \Lie(G') \arrow[r] & \Lie(G) \arrow[r] & \Lie(G'')
\end{tikzcd}
\end{center}
To show exactness, it suffices to prove that $T_e(G) \onto T_e(G'')$ is surjective. Given that $G'$ is smooth, I claim that the morphism $\pi : G \to G''$ is smooth. It is automatically flat and the smooth locus on $G$ is open so by translation we just need to show that the smooth locus is nonempty. However, since $\pi$ is flat and finitely presented (everything in sight is finite type over $k$ and thus Noetherian), and $G' = \ker{\pi}$ is smooth then $\pi$ is smooth at each point of $G' \subset G$ by Tag 01V9. Therefore, the smooth locus of $\pi$ is nonempty so by translating it must be all of $G$ (we don't actually need $G$ is smooth here at all).


\subsubsection{(iv) DO THIS}

Consider the defining diagram of the relative Frobenius,
\begin{center}
\begin{tikzcd}[row sep = large]
X \arrow[rr, "F", bend left] \arrow[rd] \arrow[r, dashed, "F_{X/k}"] & X^{(p)} \arrow[d] \arrow[r] & X \arrow[d]
\\
& \Spec{k} \arrow[r, "F"] & \Spec{k}
\end{tikzcd}
\end{center}
Since $F : \Spec{k} \to \Spec{k}$ is finite flat, we see that $X^{(p)} \to X$ is finite flat by base change. (AM I SUPPOSED TO JUST USE PROP A.3.1??)
\bigskip\\
For $\Lie(F_{G/k}) = 0$ do I work \etale locally? In this case, it is clear because \etale locally this is just the map $\A^d \to \A^d$ sending $x_i \mapsto x_i^p$ which is zero on tangent spaces. 

\section{Homework 8}

\subsection{1 DO THIS}

Let $A$ be a central simple algebra over a field $k$, and $T$ a $k$-torus un $\underline{A}^\times$.

\subsubsection{(i)}

Define $A_T$ via Galois descent from the algebra generated by $T(k^\sep) \subset A$. Since $A_{k^\sep} \cong M_n(k^\sep)$ we see that $\underline{A}^\times_{k^\sep} \cong \GL_n$. Given a maximal torus $T \subset \underline{A}^\times$ 

 via base change and using Grothendieck's theorem there is a correspondence between


\subsection{2}

Suppose that $k$ is a local field and $T$ is a torus. If $T$ is not anisotropic then there exists some $\Gm \embed T$ over $k$ and thus $\Gm(k) \subset T(k)$ as a closed subset with the subspace topology. However, $k^\times$ is always noncompact and therefore if $T(k)$ were compact then the closed subspace $k^\times = \Gm(k) \subset T(k)$ would also have to be compact. Therefore $T(k)$ is not compact.
\bigskip\\
Conversely, suppose that $T$ is anisotropic. Let $\Lambda = X(T)$ and $k'$ be a finite Galois extension of $k$ so that the $\Gamma$-action on $X(T)$ factors through $\Gamma \to G = \Gal{k^\sep/k'}$. Now, $T \cong D_k(\Lambda)$ so,
\[ T(k) = \Hom{}{\Lambda}{(k^\sep)^\times}^\Gamma = \Hom{}{\Lambda}{k'^\times}^G \]
Now consider the rational section to $\Lambda^G \embed \Lambda$ given by $p : \Lambda \to \Lambda^G$ via,
\[ v \mapsto \sum_{\sigma \in G} \sigma \cdot v \]
Because $\Lambda$ is anisotropic, we see that $\Lambda^G = (0)$. Notice that for any $x \in T(k)$ defines a $G$-invariant group map $\Lambda \to k'^\times$. Composing with $N_{k'/k}$ gives a group map $\Lambda \to k^\times$ but,
\[ N_{k'/k}(x(v)) = \prod_{\sigma \in G} \sigma(x(v)) = \prod_{\sigma \in G} x(\sigma \cdot v) = x \left( \sum_{\sigma \in G} \sigma \cdot v \right) = x(p(v)) = 1 \]
Choosing a $\Z$-basis of $\Lambda$ gives an inclusion $T(k) \subset (k'^\times)^n$. Explicitly, a $\Z$-basis gives $\Z^n \onto \Lambda$ and therefore a $G$-equivariant map $\Z[G]^n \onto \Lambda$. This gives $T \embed (\Res{k'}{k}{\Gm})^n$. Therefore, $T(k) \subset (k'^\times)^n$ which is closed with respect to the induced topology. Furthermore, the above calculation shows that $T(k)$ is contained in the closed subgroup $U_1^n$ where $U_1 \subset k'^\times$ is the subset of field norm $1$. Thus $T(k) \subset U_1^n$ is a closed subgroup so it suffices to prove that $U_1 \subset k'^\times$ is compact. In the case $k = \RR$ and $k' = \RR$ this is trivial since $U_1 = \{ 1 \}$ in the case that $k' = \CC$ then $U_1 = S^1$ which is compact. In the nonarchimedean case, for any finite extension of nonarchimedean local fields $L/K$ of degree $n = [L : K]$, the unique way to extend the valuation on $K$ to a valuation on $L$ is,
\[ \nu_L(x) = \frac{1}{n} \nu_K(N_{L/K}(x)) \]
Therefore, 
\[ U_1(L) \subset \ker{N_{L/K}} \subset \struct{L}^\times = \{ x \in L \mid v_L(x) = 0 \} \]
and $U_1(L) \subset L^\times$ is closed since $N_{L/K}$ is continuous and $\struct{L}^\times$ is compact so $U_1(L)$ is compact.

\subsection{3 DO THIS} 

Let $Y$ be a smooth separated $k$-scheme locally of finite type, and $T$ a $k$-torus with a left action on $Y$. We want to show that $Y^T$ is smooth.

\subsubsection{(i)}


By functorial consideration, $(Y^T)_{\bar{k}} = (Y_{\bar{k}})^{T_{\bar{k}}}$ and $Y^T$ is smooth if and only if $(Y^T)_{\bar{k}}$ is so we can reduce to the case that $k = \bar{k}$. Fix a finite local $k$-algebra $R$ with residue field $k$, and an ideal $J \subset R$ with $J \m_R = 0$. Choose $\bar{y} \in Y^T(R/J)$ and for $R$-algebras $A$ let $E(A)$ be the fiber of $Y(A) \onto Y(A/JA)$ over $\bar{y}_{A/JA}$. Let $y_0 = \bar{y} \mod \m_R \in Y^T(k)$ and $A_0 = A / \m_R A$.
\bigskip\\
Because $Y$ is smooth, by the infinitessimal lifting property, $E(A) \neq \empty$ because $A \to A / JA$ is an infinitessimal extension since $J \m_R = 0$ and $J \subset \m_R$ meaning $(JA)^2 = 0$.
\bigskip\\
Consider the $A_0$-module $F(A) = JA \ot_k \Tan_{y_0}(Y) = JA \ot_{A_0} (A_0 \otimes \Tan_{y_0}(Y))$. Because $A \to A/JA$ is an infinitessimal extension maps $\Spec{A} \to Y$ are determined topologically by the map $g : \Spec{A/JA} \to \Spec{R/J} \to Y$. Therefore  $Y(A)$ points are determined by lifts,
\begin{center}
\begin{tikzcd}
& & & g^{-1}\struct{Y} \arrow[d] \arrow[ld, dashed]
\\
0 \arrow[r] & JA \arrow[r] & A \arrow[r] & A / JA \arrow[r] & 0
\end{tikzcd}
\end{center}
(here these modules denote the sheaves on $\Spec{A/JA}$). Any two lifts differ by a derivation $g^{-1} \struct{Y} \to JA$. Therefore, we get,
\[ \Der{k}{g^{-1} \struct{Y}}{JA} = \Hom{A/JA}{g^* \Omega_{Y/k}}{JA} \]
However, $Y$ is smooth so $g^* \Omega_{Y/k}$ is finite locally free so,
\[ \Der{k}{g^{-1} \struct{Y}}{JA} = (g^* \Omega_{Y/k})^\vee \otimes_{A/JA} JA \]
Now $JA$ is an $A_0$-module because $J \m_R = 0$. Then, because $g$ factors through $\Spec{R/J} \to Y$, 
\[ \Der{k}{g^{-1} \struct{Y}}{JA} = (g^* \Omega_{Y/k})^\vee \otimes_{R/J} A/JA \otimes_{A/JA} A_0 \otimes_{A_0} JA = (g^* \Omega_{Y/k})^\vee \otimes_{R/J} JA \]
However, $R/J$ acts on $JA$ through $R/\m_R = k$ and thus.
\[ \Der{k}{g^{-1} \struct{Y}}{JA} = (g^* \Omega_{Y/k})^\vee \otimes_{R/\m_A} JA = (\Omega_{Y/k})^\vee_{y_0} \otimes_k JA = (\m_{y_0} / \m_{y_0}^2)^\vee \otimes_k JA = \Tan_{y_0}(Y) \otimes_k JA \]
because $(\Omega_{Y/k})_{y_0} \otimes_{\stalk{Y}{y_0}} k = \m_y / \m_y^2$ since $y_0 \in Y(k)$ is a rational point. Therefore, $E(A)$ is a $F(A)$-torsor.

\subsubsection{(ii)}

Because $y_0 \in Y^T(k)$ we see that $T(A_0)$-acts on $\Tan_{y_0}(Y)$ via an $A_0$-linear action since $T(A_0)$ acts on $\ker{(Y(A_0[\epsilon]) \to Y(A_0))} = A_0 \otimes_k \Tan_{y_0}(Y)$ by the same argument. (WRITE DOWN $T$ ACTION ON $F$).
\bigskip\\
Futhermore, $T(A)$ acts on $Y(A) \onto Y(A/JA)$ but $T(A)$ acts trivially on $\bar{y}_{A/JA} \in Y(A/JA)$ and therefore $T(A)$ stabilizes $E(A)$. Furthermore, for $t \in T(A)$ and $y \in E(A)$ and $v \in F(A)$ then,
\[ t \cdot (v + y) = t_0 \cdot v + t \cdot y \]
which follows from $A$-linearity of the action $T(A) \acts E(A)$ since addition by $v$ is given by lifting to $E(A)$ and $t \cdot v = t_0 \cdot v$ for $t_0 = t \mod \m_R$ because $T(A_0) \acts Y(A_0)$ compatibly with $T(A) \acts Y(A)$ and the action of $v$ on $y$ factors through $A_0$ (which is why the lifted action was well-defined).
\bigskip\\
Via $T(A) \to T(A_0)$ we get an action of $T_A$ on $F$ and therefore a $T_R$ action on $F$.

\subsubsection{(iii)}

Choose $\xi \in E(R)$ and define $h : T_R \to F$ via $t \cdot \xi = h(t) + \xi$ for points $t \in T_R(A)$. Now,
\[ h(t_1 t_2) = t_1 t_2 \cdot \xi - \xi = (t_1 \cdot \xi - \xi) + t_1 \cdot (t_2 \cdot \xi - \xi)  = h(t_1) + t_1 \cdot h(t_2) \]
so $h$ is a $1$-cocycle (we showed its a crossed homomorphism). Now, $h$ is a $1$-coboundary if and only if there is $\eta \in E(A)$ such that $h(t) = t \cdot \eta - \eta$. In that case, $t \cdot \xi = t \cdot \eta - \eta + \xi$ and thus $t \cdot (\xi - \eta) = \xi - \eta$ meaning $\xi - \eta \in E^{T_R}(A)$. Conversely, given $\xi' \in E^{T_R}(A)$ then $\eta = \xi - \xi'$ satisfies,
\[ t \cdot (\xi - \eta) = h(t) + \xi - t \cdot \eta = \xi - \eta \]
and therefore,
\[ h(t) = t \cdot \eta - \eta \]
Now let $V_0 = J \otimes_k \Tan_{y_0}(Y)$. For any $k$-algebra $B$ we have,
\[ V_0 \otimes_k B = (J \otimes_k B) \otimes_k \Tan_{y_0}(Y) \]
and $J \otimes_k B$ is the kernel of the $R$-algebra extension $R \ot_k B \onto R/J \otimes_k B$ by flatness. Therefore, $F(R \otimes_k B) = V_0 \otimes_k B$. Therefore, we get a $1$-cocycle $h_0 : T \to \underline{V_0}$. Then I claim that $t \cdot (v,c) = (t \cdot v + c h_0(t), c)$ is a $k$-linear representation of $T$ on $V_0 \oplus k$. Indeed,
\[ (t_1 t_2) \cdot (v, c) = (t_1 t_2 \cdot v + c h_0(t_1 t_2), c) = (t_1 t_2 \cdot v + c h_0(t_1) + c t_1 \cdot h_0(t_2), c) = t_1 \cdot (t_2 \cdot v + c h_0(t_2), c) = t_1 \cdot t_2 \cdot (v, c) \]
Then we get an exact sequence of $T$-representations,
\begin{center}
\begin{tikzcd}
0 \arrow[r] & V_0 \arrow[r] & V_0 \oplus k \arrow[r] & k \arrow[r] & 0
\end{tikzcd}
\end{center}
Because $T$-representations are semi-simple there exists a $T$-equivariant splitting $k \to V_0 \oplus k$ and thus an element $(v_0, 1)$ such that $t \cdot (v_0, 1) = (v_0, 1)$ and thus $t \cdot v_0 + h_0(t) = v_0$ meaning that $h_0(t) = t \cdot v_0 - v_0$ meaning that $h_0$ is a $1$-coboundary. Therefore, $v_0 \in V_0 = J \otimes_k \Tan_{y_0}(Y) = F(R)$ so we can lift to $v \in E(R)$ such that $h(t) = t \cdot v - v$  showing that $h$ is a $1$-coboundary (CHECK THIS LIFTING!!). Therefore, $E^{T_R}(R)$ is nonempty showing that $Y^T(R) \onto Y^T(R/J)$ is surjective. Therefore, $Y^T$ is smooth.
 
\subsubsection{4}

Let $G$ be a smooth $k$-group of finite tpye, and $T$ a $k$-torus equipped with a left action on $G$. 

\subsubsection{(i)}

There is some confustion in this problem. We will prove the followig: $G^T$ is a subgroup and that for connected $G$ that $T$ acts trivially on $G$ if and only if $T$ acts triviall on $\g = \Lie(G)$.
\bigskip\\
By the previous problem, $G^T$ is smooth. Furthermore, 
\[ \Tan_e(G^T) = \Tan_e(G)^T = \g^T \]
If $T$ acts trivally on $G$ then clearly $T$ acts trivially on $\g$. Conversely, suppose that $T$ acts trivially on $\g$ then $\Tan_e(G^T) = \g$. Because $G^T$ is smooth, we see that $\dim{G^T} = \dim{\Tan_e(G^T)} = \dim{\g} = \dim{G}$ because $G$ is smooth. Since $G$ is connected and smooth then it is irreducible. Since $G^T \subset G$ is a closed subscheme of full dimension it must contain an irreducible component and thus $G^T = G$ scheme theoretically because $G$ is reduced and irreducible. 

\subsubsection{(ii)}

Let $T$ be a $k$-subgroup of $G$ acting by conjugation. Then consider $\Ad_{G}|_T : T \to G \to \GL{\g}$ which is semisimple. Because these are stable under base change, it suffices to consider the case $k = \bar{k}$. The linear maps $\Ad_G(t)$ for $t \in T(k)$ are simultaneously diagonalizabe. Choose a basis of eigenvectors $v_1, \dots, v_n \in \g$ where $v_1, \dots, v_r$ form a basis of $\Lie(T)$. Then we know,
\[ T_e(N_G(T)) = \bigcap_{t \in T(k)} (\Ad(t) - 1)^{-1}(\Lie(T)) \]
For any $v \in \Tan_e(N_G(T))$ we write,
\[ v = \alpha_1 v_1 + \cdots + \alpha_n v_n \quad \text{and then} \quad [\Ad(t) - 1]v = \alpha_{r+1} (\lambda_{r+1}(t) - 1) v_{r+1} + \cdots + \alpha_n (\lambda_{n}(t) - 1) v_{n} \in \Lie(T) \]
However, since $v_1, \dots, v_r$ form a basis of $\Lie(T)$ the other vectors are complementary and thus $\alpha_i (\lambda_i(t) - 1) = 0$ for all $i$. Thus either $\alpha_i = 0$ or $\lambda_i(t) = 1$ for all $t$ meaning that $v_i \in \g^T$. This proves that,
\[ \Lie(N_G(T)) \subset \g^T = \Lie(Z_G(T)) \]
However, $Z_G(T) \subset N_G(T)$ is a closed subgroup so we also have $\Lie(Z_G(T)) \subset \Lie(N_G(T))$ and thus $\Lie(Z_G(T)) = \Lie(N_G(T))$. Because $Z_G(T)$ is smooth, $\dim{Z_G(T)} = \dim_k \Lie(Z_G(T)) = \dim_k \Lie(N_G(T)) \ge \dim{N_G(T)}$ but $Z_G(T) \subset N_G(T)$ is a closed subscheme and thus $\dim{N_G(T)} = \dim_k \Lie(N_G(T))$ which implies that $N_G(T)$ is regular at the origin and hence regular and hence smooth because $k = \bar{k}$. Thus, since $N_G(T)$ is reduced and $Z_G(T) \subset N_G(T)$ is a closed subscheme of the same dimension, we see that it is an open subscheme. Since $Z_G(T)$ is smooth we see that $N_G(T)$ is also smooth. Then by Galois descent we see that $Z_G(T) \subset N_G(T)$ is open over $k$ and $N_G(T)$ is smooth over $k$.
\bigskip\\
Since $N_G(T)$ and $Z_G(T)$ are smooth and of the same dimension, the quotient $N_G(T)/Z_G(T)$ is also smooth and of dimension $0$ and thus $N_G(T) / Z_G(T)$ is finite \etale over $\Spec{k}$ (finiteness follows from finite type and dimension zero).

\subsubsection{(iii)}

\renewcommand{\End}[2]{\mathrm{End}_{#1}\left( #2 \right)}


Let $W = W(G,T)$ be the Weil group. Suppose that $T$ is $k$-split. Then $\End{}{T} = \End{}{T_{k^\sep}}$ because these are just given by $n \times n$ matrices with entires $\Z$. On HW6 4(i) we constructed a map $N_G(T) \to \mathrm{Aut}_{T/k}$ with kernel $Z_G(T)$ (this is clear because $Z_G(T)$ represents the elements with trivial scheme theoretic action on $T$). Thus, we get an embedding $W = N_G(T)/Z_G(T) \embed \mathrm{Aut}_{T/k}$ it factors through $N_G(T)/Z_G(T)$ by functorial properties of the quotient and this map has trivial kernel and thus is a closed embedding. Then $W(k^\sep) \embed \mathrm{Aut}_T(k^\sep)$ but $\mathrm{Aut}_T(k^\sep) = \mathrm{Aut}_T(k)$ and so $W(k^\sep) = W(k)$ because this is an embedding of Galois modules so the Galois action on $W(k^\sep)$ is trivial so $W(k) = W(k^\sep)^G = W(k^\sep)$.
\bigskip\\
Now, because $W(k) = W(k^\sep)$ we see that all its points are defined over $k$ and therefore since $W$ is an \etale $k$-group it is just a finite product of copies of $k$ (since \etale algebras are finite products of fields and here all the fields must be $k$ since $W(k) = W(k^\sep)$). Therefore $W$ is the constant group scheme $\underline{W(k)}$. 
\bigskip\\
However, let $k$ be infinite and $K$ is a separable quadratic extension of $k$ such that $-1 \notin N_{K/k}(K^\times)$ with $G = \SL(K) \cong \SL_2$ and $T$ the non-split maximal $k$-torus corresponding to the norm-$1$ part of $K \subset \mathrm{End}_k(K)$. Then $W(k^\sep) = S_2$ is nontrival by HW6 4(ii). However, I claim that $W(k) = S_2$ and $N_G(T)(k) = 1$.
\bigskip\\
First, notice that $-1 \notin N_{K/k}(K^\times)$ implies that $-1 \neq 1$ and therefore $\mathrm{char}(k) \neq 2$. Therefore, we may choose a $k$-basis $1, \alpha \in K$ such that $\alpha^2 = q$ for $q \in k$ a nonsquare. Then the matrices $A \in \SL(K)(k)$ that preserve $T(k)$ are exactly the matrices $A$ preserving $\ker{N_{K/k}} \subset \mathrm{End}_k(K)$. From Hilbert 90, every element of norm $1$ is of the form $\sigma(\beta)/\beta$ for $\beta \in K$ therefore we can directly compute for $\beta = a - b \alpha$ that,
\[ \frac{\alpha(\beta)}{\beta} = \frac{a^2 + b^2 q + 2 ab \alpha}{a^2 - b^2 q} \]
For example, choosing $a = b = 1$ we see that,
\[ \frac{1 + q + 2 \alpha}{1 - q} \]
has norm $1$ and is $k$-independent from $1$ since $\frac{2}{1-q} \neq 0$. Therefore the norm $1$ elements span $K \subset \mathrm{End}_k(K)$ meaning that $A$ preserves $K$. Since $\SL(K)(k)$ acts on $\mathrm{End}_k(K)$ via algebra automorphism we see that $A : K \to K$ is an algebra automorphism and therefore must be given by $\tau \in \Gal{K/k} \subset \End{k}{K}$ meaning that for all $\varphi \in K \subset \End{k}{K}$ we have $A \varphi A^{-1} = \tau \varphi \tau^{-1}$ and therefore $\tilde{A} = \tau^{-1} A$ fixes $K$ pointwise. Therefore $\tilde{A}$ commutes with $K$ so it is $K$-linear and therefore $\tilde{A} = \lambda$ for some $\lambda \in K$. Thus $A = \sigma \lambda$. However, $\sigma(1) = 1$ and $\sigma(\alpha) = -\alpha$ so,
\[ \sigma = 
\begin{pmatrix}
1 & 0
\\
0 & -1
\end{pmatrix} \]
and thus $\det{\sigma} = -1$. However, $\det{A} = \det{(\lambda \tau)} = N_{K/k}(\lambda) \det{\tau}$ but since $-1 \notin N_{K/k}(K^\times)$ we cannot have $\tau = \sigma$ and $\det{A} = 1$ and thus $A = \lambda$. Therefore $N_G(T)(k) = K^\times = T(k) = Z_G(T)(k)$ meaning that $N_G(T)(k) / Z_G(T)(k) = 1$. 
\bigskip\\
However, because $\SL(K)$ is a form of $\SL_2$ we see that $W(k^\sep) = W(\SL_2, T) = S_2$. Furthermore, the Galois action on $W(k^\sep)$ must be trivial because it acts by group automorphisms which cannot take the nontrivial element to the identity element. Therefore $W(k) = W(k^\sep)^G = W(k^\sep) = S_2$ so we see that $N_G(T)(k) \to W(k)$ is not surjective.

\subsubsection{(iv)}

It is easy to see for the split groups $\GL_n, \PGL_n, \SL_n, \Sp_n$ that the (signed)permutation matrices generating $W$ are all defined over $k$ (in fact they are defined over the prime subfield). 


\renewcommand{\End}[1]{\mathrm{End}\left( #1 \right)}


\subsubsection{5 DO THIS}

\subsubsection{(i)}

It is a general fact that, assuming Weil restriction exists, if $T$ is an $S$-scheme and $X$ is an $S$-scheme there is a natural map,
\[ X \to \Res{T}{S}{X_T} \]
this is just the unit of the adjunction $- \otimes_S T \dashv \Res{T}{S}{-}$. On the functor of points, this corresponds to the map,
\[ X(Z) \to X(Z \times_S T) = X_T(Z \times_S T) \]
defined by the projection map. This clearly commutes with base change and is actually a natural transformation.
\bigskip\\
In the case of a field extension $k' / k$ and $X$ is an affinite $k$-scheme of finite type I want to show that $j : X \to \Res{k'}{k}{X_{k'}}$ is a closed immersion. We can check being a closed immersion after faithfully flat base change by $\Spec{k'} \to \Spec{k}$. Then,
\[ X_{k'} \to \Res{k'}{k}{X_{k'}} = \prod_{\sigma \in \Hom{k}{k'}{k^\sep}} (X_{k'})^\sigma \]
which is the diagonal embedding (note that $(X_{k'})^\sigma \cong X_{k'}$ canonically as a $k$-scheme (but not as $k'$-schemes!) because  it is the base change of a $k$-scheme $X$) and thus a closed immersion.

\subsubsection{(ii)}

The functor, $\Res{T}{S}{-}$ is a right adjoint and therefore preserves products. Furthermore, I claim that,
\begin{center}
\begin{tikzcd}
X \times_S X \arrow[d, equals] \arrow[r, "j_X \times j_X"] & \Res{T}{S}{X_T} \times_S \Res{T}{S}{X_T} 
\\
X \times_S X \arrow[r, "j_{X \times_S X}"] & \Res{T}{S}{(X \times_S X)_T} \arrow[u]
\end{tikzcd}
\end{center}
commutes because both the top and bottom are equivalent to $(X \times_S X)_T \to (X \times_S X)_T$ via adjunction. Then we use the following lemma to show that $j_G : G \to \Res{T}{S}{G_T}$ is a group homomorphism. 

\begin{lemma}
Let $\eta : F \to G$ be a natural transformation of product preserving functors such that $\eta_{X \times X} = \eta_X \times \eta_X$. Then for any group object $X$,
\[ \eta_X : F(X) \to G(X) \]
is a homomorphism. 
\end{lemma}

\begin{proof}
Because $F, G$ are product preserving, we see that $F(X)$ and $G(X)$ are group objects. Then consider the diagram,
\begin{center}
\begin{tikzcd}
F(X) \times F(X) \arrow[d, "\eta_X \times \eta_X"'] \arrow[from=r, "\sim"'] & F(X \times X) \arrow[d, "\eta_{X \times X}"'] \arrow[r, "F(m_X)"] & F(X) \arrow[d, "\eta_X"]
\\
G(X) \times G(X) \arrow[from=r, "\sim"'] & G(X \times X) \arrow[r, "G(m_X)"] & G(X)
\end{tikzcd}
\end{center}
proving that $\eta_X : F(X) \to G(X)$ is a homomorphism of group objects.
\end{proof}

\subsubsection{(iii)}

A \textit{vector group} over $k$ is a $k$-group $G$ admitting an isomorphism $G \iso \Ga^n$. Then a \textit{linear structure} on $G$ is the resulting $\Gm$-action pulled back from $\Ga^n$. A \textit{linear homomorphism} $G' \to G$ between vector groups equipped with linear structures is a $k$-homomorphism which respects the linear structures.
\bigskip\\
Any vector group has a unique linear structure up to unique linear isomorphism. However, for $\Ga$ we get a completely unique action which is stronger\footnote{this is false for $\Ga^2$ because the example shows we can use a nonlinear (in the standard structure) isomorphism $f : \Ga^2 \to \Ga^2$ to put a weird linear structure on $\Ga^2$ that is linearly isomorphic to the original via $f$ (by definition $f$ is linear in the induced structure) but is a different $\Gm$ action on the underlying $\Ga^2$}.
Suppose that $f : \Ga \to \Ga$ is an isomorphism. We want to show that the induced $\Gm$ action via pulling back the standard structure is independent of the choice of $f$. The map corresponds to $f \in k[x]$ such that $f(x + y) = f(x) + f(y)$ and therefore,
\[ f(x) = \sum_{j \ge 0} c_j x^{p^j} \]
as we saw on HW1. However, for this to be an isomorphism it must have degree $1$ else the composition would still have positive degree and thus cannot be the identity which corresponds to $f(x) = x$. Thus $f(x) = c_0 x$. Now the induced action $\rho : k[x] \to k[x] \otimes k[t,t^{-1}]$ is given by,
\[ \rho(x) = f^{-1}(\rho^{\text{std}}(f(x))) = f^{-1}(c_0 x) \otimes t = x \otimes t \]
which is exactly thestandard action. Therefore, there is a unique $\Gm$ action on $\Ga$ arising from an isomorphism $\Ga \to \Ga$ and thus a completely unique linear structure on $\Ga$.
\bigskip\\
Now $\End{\Ga}$ with its linear structure consists of polynomials $f \in k[x]$ such that $f(x + y) = f(x) + f(y)$ and, due to the action, under the map $\rho : k[x] \to k[x] \otimes k[t,t^{-1}]$ sending $x \mapsto x \otimes t$ we have $\rho(f(x)) = f(x) \otimes t$. If we write,
\[ f(x) = \sum_{n \ge 0} c_n x^n \]
then,
\[ \rho(f(x)) = \sum_{n \ge 0} c_n x^n \otimes t^n \]
and,
\[ f(x) \otimes t = \sum_{n \ge 0} c_n x^n \otimes t \]
For these to b equal in $k[x] \otimes k[t,t^{-1}] = k[x,t,t^{-1}]$ we must have $c_n = 0$ if $n \neq 1$. Therefore, $f(x) = c x$ so $\End{\Ga} = k$. 
\bigskip\\
To show that linear $k$-homomorphisms $\Ga^n \to \Ga^m$ correspond to $m \times n$ matrices, it suffices (via the universal property of the product) to show that maps $\Ga^n \to \Ga$ are given by a list $(a_1, \dots, a_n) \in k^n$ such that $x \mapsto a_1 x_1 + \cdots + a_n x_n$. Indeed, such a map is given by sending $x \mapsto f$ for $f \in k[x_1, \dots, x_n]$. Furthermore, we require that $f(x_1, \dots, x_n) \otimes t = \rho(f(x_1, \dots, x_n))$. If we write,
\[ f(x_1, \dots, x_n) = \sum_{i_1, \dots, i_n} c_{i_1, \dots, i_n} x_1^{i_1} \cdots x_n^{i_n} \]
then,
\[ \rho(f(x_1, \dots, x_n)) = \sum_{i_1, \dots, i_n} c_{i_1, \dots, i_n} x_1^{i_1} \cdots x_n^{i_n} \otimes t^{i_1 + \cdots + i_n} \]
and therefore we must have $c_{i_1, \dots, i_n} = 0$ unless $i_1 + \cdots + i_n = 1$ meaning that,
\[ f(x_1, \dots, x_n) = a_1 x_1 + \cdots + a_n x_n \]
(this even shows its automatically a group map if its linear) proving the claim.
\bigskip\\
Let $\mathrm{char}(k) = 0$. We ask if there are any non-linear homomorphism $\Ga^n \to \Ga^m$. Again, it suffices to consider the case $m = 1$ by the universal property of the product. A homomorphism $\Ga^n \to \Ga$ is given by $f \in k[x_1, \dots, x_n]$ such that,
\[ f(x+y) = f(x) + f(y) \quad \text{or more accurately} \quad f(x \otimes 1 + 1 \otimes x) = f(x) \otimes 1 + 1 \otimes f(x) \]
If we write, 
\[ f(x_1, \dots, x_n) = \sum_{i_1, \dots, i_n} c_{i_1, \dots, i_n} x_1^{i_1} \cdots x_n^{i_n} \]
then we require that,
\[ \sum_{i_1, \dots, i_n} c_{i_1, \dots, i_n} (x_1 + y_1)^{i_1} \cdots (x_n + y_n)^{i_n} = \sum_{i_1, \dots, i_n} c_{i_1, \dots, i_n} (x_1^{i_1} \cdots x_n^{i_n} + y_1^{i_1} \cdots y_n^{i_n}) \]
Since $\Q \subset k$ we can check this a map $\Q^n \to \Q$ and it is easy to see that this linearity property must imply that $c_{i_1, \dots, i_n} = 0$ unless $i_1 + \cdots + i_n = 1$ and therefore,
\[ f(x_1, \dots, x_n) = a_1 x_1 + \cdots + a_n x_n \]
meaning that $f$ is linear.

\section{Homework 9}

\subsection{1}

\subsection{2}

Let  $U$ be a smooth connected commutative affine $k$-group, and assume $U$ is $p$-torsion if $\ch{k} = p$ is positive.

\begin{enumerate}
\item Let $\ch{k} = p$ be positive and $U$ be $k$-split. We prove that $U$ is a vector group by induction on the dimension of $U$. Consider a composition series $\{ U_i \}$ with $U_{n+1} = 1$. Then $U_n \cong \Gm$ or $\Ga$ but $U$ is $p$-torsion so $U_n$ is $p$-torsion and thus $U_n \cong \Ga$ since $\Gm$ is $p-1$-torsion which is coprime to $p$. Then we apply PRG Cor. B.1.12 to conclude the $U_n$ is a direct factor of $U$ meaning we can write,
\[ U = U' \times \Ga \]
Therefore, it suffices to prove that $U'$ is a $k$-split smooth connected commutative $p$-torsion affine $k$-group since $\dim{U'} = \dim{U} - 1$ so we would then conclude by induction. Since $U'$ is a closed subgroup of $U$ it is clearly affine commutative and $p$-torsion. Furthermore, if $U'$ is not smooth or connected then $U$ would not be. Finally, we need to show that $U'$ is $k$-split. However, $U \onto U'$ is a surjective $k$-map of smooth connected affine $k$-groups and $U$ is $k$-split solvable so $U'$ is also $k$-split (Example 22.1.6).

\item Now let $\ch{k} = 0$. Consider a short exact sequence,
\begin{center}
\begin{tikzcd}
0 \arrow[r] & \Ga \arrow[r] & G \arrow[r] & \Ga \arrow[r] & 0
\end{tikzcd}
\end{center}
Because $\Ga$ is a normal subgroup this gives an action $\Ga \acts \Ga$ via group homomorphisms. However, since $\End{\Ga} = \Gm$ and there are no nontrivial maps $\Ga \to \Gm$ we see that $\Ga \acts \Ga$ trivially and therefore $G$ is commutative. Furthermore, extensions of affine group schemes by affine group schemes are affine (we can use arguments with torsors see \chref{https://mathoverflow.net/questions/109456/are-extensions-of-linear-algebraic-groups-over-a-field-themselves-linear-algeb}{here}).
\bigskip\\
Then we can embed $G \embed \GL{V}$ and to show that $G$ is unipotent we pass to $k = \bar{k}$ and then it suffices to show that $G$ does not contain any copy of $\Gm$. Otherwise, $\Gm \to G \to \Ga$ would be zero so the map $\Gm \to G$ would factor through $\Gm \to \Ga \to G$ but any map $\Gm \to \Ga$ is again trivial giving a contradiciton. Therefore, by Cor. 24.1.4 we see that $G$ is unipotent. Therefore, the exponential map $\underline{\g} \to G$ exsits and is invertible by $\log$. Thus we get an exact sequence,
\begin{center}
\begin{tikzcd}
1 \arrow[r] & \Ga \arrow[r] & G \arrow[r] & \Ga \arrow[r] & 1
\\
1 \arrow[r] & \Ga \arrow[u, equals] \arrow[r] & \underline{\g} \arrow[u, "\exp"] \arrow[r] & \Ga \arrow[r] \arrow[u, equals] & 1
\end{tikzcd}
\end{center}
However, because $G$ is commutative $\exp$ is a group map and $\g$ has a trivial Lie bracket so the bottom sequence splits as an exact sequence of vector groups. Thus the sequence,
\begin{center}
\begin{tikzcd}
1 \arrow[r] & \Ga \arrow[r] & G \arrow[r] & \Ga \arrow[r] & 1
\end{tikzcd}
\end{center}
is isomorphic to a split sequence and thus splits.
\bigskip\\
Now we apply this to the smooth connected commutative affine unipotent $k$-group $U$. Then $U$ is solvable so it has a filtration with quotients $\Gm$ and $\Ga$ but $U$ is unipotent so the factors are $\Ga$. Choose the first term of the filtration $\Ga \subset U$
\begin{center}
\begin{tikzcd}
0 \arrow[r] & \Ga \arrow[r] & U \arrow[r] & U' \arrow[r] & 0 
\end{tikzcd}
\end{center} 
then by the induction hypothesis $U' \cong \Ga^n$. Now we adapt the previous argument to show that $U \cong \Ga^{n+1}$ finishing the proof by induction.
\end{enumerate}

\subsection{3}

Let $k'/k$ be a degree-$p$ purely inseparable extension of a field $k$ of characteristic $p > 0$.

\subsubsection{(i)}

Let $U = \Res{k'}{k}{\Gm} / \Gm$. Since $\Res{k'}{k}{\Gm} \onto U$ is faithfully flat (or by quotient construction) it suffices to show that $G = \Res{k'}{k}{\Gm}$ is smooth and connected. Smoothness is immediate from the lifting criterion because if $R' \onto R$ is an infinitessimal extension of Artin local rings then so is $R' \otimes_k k' \onto R \otimes_k k'$. For connectedness, we use that $\Gm$ is geometrically connected over $k'$ and smooth and this implies that $\Res{k'}{k}{\Gm}$ is connected. Alternatively, we can check connectedness directly from the structure of,
\[ \Res{k'}{k}{\Gm} \times_k k' \]
by writing as a subset of upper triangular matrices in $\GL(V)$.
\bigskip\\
Furthermore, to show that $U$ is $p$-torsion it suffices to check on geometric points. We see that,
\[ U(\bar{k}) = (k' \otimes_k \bar{k})^\times / \bar{k}^\times = (\bar{k}[\epsilon]/(\epsilon^{p+1}))^\times / \bar{k}^\times \cong \{ 1  + a_1 \epsilon + a_2 \epsilon^2 + \cdots + a_{p-1} \epsilon^{p-1} \mid a_i \in \bar{k} \} \]
is $p$-torsion because these are ``additive''. Explicitly,
\[ (1 + a_1 \epsilon + \cdots + a_{p-1} \epsilon^{p-1})^p = 1 + \sum_{i = 1}^p a_i^i {p \choose i } \epsilon^i = 1 \]
because $p \divides { p \choose i }$ for $0 < i < p$.
\bigskip\\
Therefore, under any embedding $U \embed \GL(V)$ we see that $t \in U(\bar{k})$ satisfies $t^p = 1$ as an operator on $V$ and thus has a polynomial $(t - 1)^p$ so $t$ is unipotent. Therefore $U$ is unipotent.

\subsubsection{(ii)}

Consider,
\[ \Res{k'}{k}{\Gm}(k^\sep)[p] = \ker{((k' \otimes_k k^\sep)^\times \xrightarrow{(-)^p} (k' \otimes_k k^\sep)^\times)} \]
However, because $k'/k$ is purely inseparable, we see that $k' \otimes_k k^\sep$ is a field (in our case $k' = k(a^{\frac{1}{p}})$ and then $k' \otimes_k k^\sep = k^\sep[x]/(x^p - a)$ which is a field because $x^p - a$ is irreducible in $k^\sep$ since $a^{\frac{1}{p}} \notin k^\sep$) and thus if $a^p = 1$ then $(a - 1)^p = 0$ so $a = 1$.
\bigskip\\
Suppose now that $\Ga \subset U$ as a $k$-subgroup. Then consider the diagram,
\begin{center}
\begin{tikzcd}
1 \arrow[r] & \Gm \arrow[d, equals] \arrow[r] & G \pullback \arrow[d, hook] \arrow[r] & \Ga \arrow[d, hook] \arrow[r] & 1
\\
1 \arrow[r] & \Gm \arrow[r] & \Res{k'}{k}{\Gm} \arrow[r] & U \arrow[r] & 1
\end{tikzcd}
\end{center}
However, since every extension of $\Ga$ by $\Gm$ is split over $k$ and thus $G \cong \Gm \times \Ga$. However, then $\Ga \embed G \embed \Res{k'}{k}{\Gm}$ which implies that,
\[ k^\sep = \Ga(k^\sep)[p] \subset \Res{k'}{k}{\Gm}(k^\sep)[p] \]
contradicting our previous calculation and therefore showing that $U$ cannot contain any $\Ga$ factor.

\subsection{4}

Let $G$ be a smooth group of finite type over a field $k$, and $N$ a commutative normal $k$-subgroup scheme. 
\bigskip\\
First recall that the quotient of a subgroup $H \subset G$ is initial for right $H$-invariant maps $f : G \to Y$ for any $k$-scheme $Y$. In particular, if $Y$ is a $k$-group and $f : G \to Y$ a $k$-homomorphism then $f$ is $H$-invariant iff $H \to G \to Y$ is trivial and thus in that case $f : G \to Y$ factors uniquely through $\pi : G \to G/H$ and remains a homomorphism if $H$ is normal.

\subsubsection{(i)}

Let $G \acts N$ via conjugation. Since $N$ is commutative we see that $N \acts N$ trivially and thus the action naturally factors through $\pi : G \to G/N$ because it is a coequalizer and $\rho : N \times N \to N$ is the trivial map. Suppose that $N$ is central in $G$. Then by the same reasoning $N \times G \to G$ is trivial so the conjugation action $G \times G \to G$ factors through $\pi : G \to G/N$.
\bigskip\\
Let $G  = \SL_n$ and $N = \mu_n$ over any field $k$. Since $N$ is normal we get an action of $G/N = \PGL_n$ on $\mu_n$ but this is trivial because $\mu_n$ is central. In this case however, we get an action of $G / N = \PGL_n$ on $G = \SL_n$ by conjugation. We describe this action functorially. Consider $A \in \PGL_n(k)$. Then it comes from a matrix $A' \in \GL_n(k)$ then we act on $\SL_n$ by conjugation by $A'$ which is well-defined because scalars do not effect the result of conjugation.

\subsubsection{(ii)}

Consider the commutator map $c : G \times G \to G$. It is clear that there is a factorization,
\[ c : G \times G \to \D(G) \to G \]
by definition. Furthermore, $c : Z_G \times G \to \D(G)$ is clearly trivial so we get a unique factorization $c : G / Z_G \times G \to \D(G)$. Likewise, $c : G/Z_G \times Z_G \to \D(G)$ is trivial because $G \times Z_G \to \D(G)$ is trivial and therefore we get a unique map $c : (G / Z_G) \times (G / Z_G) \to \D(G)$.

\subsection{5}

Let $B$ be a smooth connected solbable group over a field $k$.

\subsubsection{(i)}

If $B = \Gm \ltimes \Ga$ with the standard semi-direct product structure, we need to show that $Z_B(t,0)$ under the conjugation action is the left factor for any $t \in k^\times \setminus \{ 1 \}$. 
\bigskip\\
We do this functorially on $R$-points. Now,
\[ B(R) = R^\times \rtimes R \]
with the standard direct product structure. Notice that if $(x,y) \in B(R)$ centralizes $Z_B(t,0)$ then  \[ (x,y) \cdot (t,0) = (xt, x \cdot 0 + y) (x^{-1}, - x \cdot y) = (t, y - t \cdot y) \]
So if $(x,y) \cdot (t,0) = (t,0)$ then we must have $t \cdot y = y$ so because $t \neq 1$ so $t - 1 \in k$ is a unit we see that $y = 0$ in $R$. Furthermore, it is clear that $(x, 0)$ stabilizes $(t,0)$ scheme-theoretically since the action is just the conjugation action of $\Gm \acts \Gm$ which is scheme-theoretically trivial. Therefore, we see that $Z_B(t,0) = \Gm$ as closed subschemes of $B$. 

\subsubsection{(ii)}

Let $k = \bar{k}$ and $S \subset B(k)$ is a commutative subgroup of semisimple elements. I claim by induction on $\dim{B}$ that $S \subset T$ for some maximal torus $T \subset B$. 
\bigskip\\
We proceed by induction on $\dim{B}$. Since $B$ is solvable we can take an exact sequence,
\begin{center}
\begin{tikzcd}
1 \arrow[r] & A \arrow[r] & B \arrow[r, "\pi"] & B' \arrow[r] & 1
\end{tikzcd}
\end{center}
where $A$ is either $\Gm$ or $\Ga$. By induction we can assume that $\pi(S) \subset T'(k)$ for some maximal torus $T' \subset B'$. Then we reduce to the case,
\begin{center}
\begin{tikzcd}
1 \arrow[r] & A \arrow[r] & \pi^{-1}(T') \arrow[r] & T' \arrow[r] & 1
\end{tikzcd}
\end{center}
If $A = \Gm$ then $\pi^{-1}(T')$ is a torus so we are done. Therefore we may assume that $A = \Ga$. Therefore we get an action of $T$ on $\Ga$ and we get $\pi^{-1}(T') = T' \rtimes \Ga$. Choose an element $t \in S$, then by Prop. G.1.1 there is a $G(k)$ conjugate $g t g^{-1}$ of $t$ with $g t g^{-1} \in T(k)$. By a similar argument as part (i) we see that $Z_B((t, 0)) = T$ and thus because $S$ is commutative $g S g^{-1} \subset Z_B((t, 0))(k) = T(k)$ and therefore $S$ is contained in the maximal torus $g^{-1} T g$ completing the induction.


\subsubsection{(iii)}

Assume that $\ch{k} \neq 2$ with $k = \bar{k}$. Let $G = \SO_n$ for $n \ge 3$ and $\mu \cong \mu_2^{n-1}$ be the ``diagonal'' $k$-subgroup $\{(\zeta_i) \in \mu_2^n \mid \prod \zeta_i = 1 \}$. Notice that $\mu$ contains all diagonal matrices in $G$. Therefore, if $\mu \subset H$ where $H$ is a smooth commutative subgroup then $\mu(k) \subset H(k)$ but if $H(k)$ contains a nondiagonal matrix then conjugating with some  element of $\mu(k)$ will reverse the sign of the non-diagonal term (if the $(i,j)$ entry with $i \neq j$ is nonzero then conjugating by the matrix with a $-1$ in the $(i,i)$ or in the $(j,j)$ position will reverse the sign) and thus $\mu(k) = H(k)$. Because $H$ is smooth (hence reduced) and $k = \bar{k}$-points are dense we see that $\mu = H$ so $\mu$ is a maximal as a commutative smooth $k$-subgroup of $G$.
\bigskip\\
Suppose that $\mu$ is contained in a maximal torus $T \subset \SO_n$. Because $k = \bar{k}$ we see that $T \cong \Gm^{k}$ for $k = \lfloor \frac{n}{2} \rfloor < n-1$ since $n \ge 3$ (alternatively notice that $\SO_n \subset \SL_n$ so any maximal torus in $\SO_n$ is a torus in $\SL_n$ and thus has dimension at most $n-1$) Then consider $T(k)[2] = (k^\times)^{k}[2] = \{ \pm 1 \}^k$ which is smaller than $\mu(k) \cong \{ \pm 1 \}^{n-1}$. Therefore $\mu$ cannot be contained in any maximal torus.
\bigskip\\
Suppose that $\mu$ were contained in a Borel $k$-subgroup $B \subset G$. Because $B$ is by definition a smooth connected solvable $k$-subgroup and $\mu(k) \subset B$ we see, by the previous problem, that $\mu(k)$ is contained in a maximal torus of $B$ which can thus be extended to a maximal torus of $G$ (in fact I think it should already be a maximal torus of $G$). This contradicts what we just showed proving that $\mu$ cannot lie in any Borel subgroup. Therefore, a maximal solvable smooth subgroup of $G_{\bar{k}}$ containing $\mu_{\bar{k}}$ which exists for dimension and Zariski-closure reasons is not contained in a Borel subgroup. 

\subsection{6 DO THIS}

Let $G$ be a quasi-split smooth connected affine $k$-group and $B \subset G$ a Borel $k$-subgroup. Let $T$ be a maximal $k$-torus in $B$.

\subsubsection{(i)}

Let $S$ denote the set of Borel subgroups of $G_{\bar{k}}$ containing $T_{\bar{k}}$.
\bigskip\\
Consider the map $N_G(T)(\bar{k}) \to S$ sending $g \mapsto g B g^{-1}$. It is clear that $g B_{\bar{k}} g^{-1}$ is a Borel subgroup containing $T_{\bar{k}}$ because conjugation is an isomorphism and thus preserves being smooth connected and solvable and $g \in N_G(T)(\bar{k})$ so $g T_{\bar{k}} g^{-1} = T_{\bar{k}}$ so $g B_{\bar{k}} g^{-1}$ is a Borel containing $T_{\bar{k}}$.
\bigskip\\
Let $H$ be any $\bar{k}$-Borel subgroup containing $T_{\bar{k}}$. Then we know that $\bar{k}$-Borel subgroups are conjugate over $\bar{k}$ so there is some $g \in G(\bar{k})$ such that $H = g B_{\bar{k}} g^{-1}$ but this may not fix $T_{\bar{k}}$. However, $g T_{\bar{k}} g^{-1}$ and $T_{\bar{k}}$ are two maximal tori of $H$ and therefore are conjugate in $H$ so there is some $h \in H(\bar{k})$ such that $h g T_{\bar{k}} g^{-1} h^{-1} = T_{\bar{k}}$ and clearly $h g B_{\bar{k}} g^{-1} h^{-1} = h H h^{-1} = H$ so we see that $hg \in N_G(T)(\bar{k})$ and thus $N_G(T)(\bar{k}) \to S$ is surjective. 
\bigskip\\
Now suppose that $g \in N_G(T)(\bar{k})$ fixes $B_{\bar{k}}$ meaning $g \in N_G(B)(\bar{k})$. However, $N_G(B) = B$ because parabolic subgroups are self-normalizaing. Thus, $g \in B(\bar{k})$. Now because $B$ is solvable we see that,
\[ B \cong U \rtimes T \]
for some torus $T$ (which by conjugation we can assume is $T$) and some unipotent group $U$. Now consider, $g = (u,t)$ and we know that for any $g' \in B(\bar{k})$ we have $g' = (u',t')$ then,
\[ g \cdot g' \cdot g^{-1} = (u,t) \cdot (u',t') \cdot (\varphi(t^{-1}) u^{-1}, t^{-1}) = (u \varphi(t) u', tt') \cdot (\varphi(t^{-1}) u^{-1}, t^{-1}) = (u \varphi(t) u' \varphi(t') u^{-1}, t') \]
and therefore $g \cdot (0, t') \cdot g^{-1} = (0, t')$ so $g \in Z_G(T)(\bar{k})$ and if $g$ centralizes $T$ then (HOW DOES THE OTHER WAY WORK??). Therefore,
\[ N_G(T)(\bar{k}) / Z_G(T)(\bar{k}) \iso S \]
is a bijection so we see that $S$ is finite.

\subsubsection{(ii)}

We showed that $W = N_G(T) / Z_G(T)$ is a constant group if $T$ is $k$-split. Since $T$ splits over $k^\sep$ we may work in the split case. In HW8 4, we showed that $W$ is a constant finte \etale group scheme and thus has its points defined over $k^\sep$. Since $N_G(T) \onto W$ is faithfully flat with smooth fibers $Z_G(T)$ we see that the map is smooth. Therefore, since $T_{k^\sep}$ is split we see that $W_{k^\sep}$ is a constant finite \etale group scheme and thus every point of $W$ is defined over $k^\sep$ (alternatively, $W$ is finite \etale and thus every point is given by a finite separable extension of $k$ and thus is defined over $k^\sep$). Therefore, for each $x \in W(k^\sep) = W(\bar{k})$ the fiber of $N_G(T) \to W$ over $x$ is isomorphic to $Z_G(T)_{k^\sep}$ which is smooth and nonempty and thus has a $k^\sep$ point proving that $N_G(T)(k^\sep) \onto W(k^\sep)$ is surjective. Alternatively, we can use that $N_G(T)(k^\sep)$ is dense in $N_G(T)_{k^\sep}$ and thus its image in $W_{k^\sep}$ is dense but $W_{k^\sep}$ is disconnected so a dense set is the entire space proving surjectivity at the level of $k^\sep$-points. Since the sequence,
\begin{center}
\begin{tikzcd}
0 \arrow[r] & Z_G(T)(A) \arrow[r] & N_G(T)(A) \arrow[r] & W(A)
\end{tikzcd}
\end{center}
is left exact for any $k$-algebra $A$ because $\Hom{}{\Spec{A}}{-}$ commutes with limits, therefore we see that,
\[ N_G(T)(k^\sep)/Z_G(T)(k^\sep) \to N_G(T)(\bar{k}) / Z_G(T)(\bar{k}) \]
is bijective. In particular $N_G(T)(k^\sep) \to S$ is surjective showing that every Borel subgroup of $G_{\bar{k}}$ containing $T_{\bar{k}}$ is a $k^\sep$ conjugate of $B$ and therefore defined over $k^\sep$.

\subsubsection{(iii)}

Let $T$ be $k$-split and $Z_G(T) = T$. Then we have that $W$ is a constant group with $W(k^\sep) = N_G(T)(k^\sep)/Z_G(T)(k^\sep)$. Furthermore, since $T$ is $k$-split we have $W$ is a finite \etale constant group and thus $W(k^\sep) = W(k)$. Notice that $Z_G(T)(k^\sep) = T(k^\sep) = (k^\sep)^\times$. Therefore, we get an exact sequence of Galois modules,
\begin{center}
\begin{tikzcd}
1 \arrow[r] & (k^\sep)^\times \arrow[r] & N_G(T)(k^\sep) \arrow[r] & W(k^\sep) \arrow[r] & 1
\end{tikzcd}
\end{center}
Taking Galois cohomlogy,
\begin{center}
\begin{tikzcd}
1 \arrow[r] & k^\times \arrow[r] & N_G(T)(k^\sep)^G \arrow[r] & W(k^\sep)^G \arrow[r] & H^1(\Gal{k^\sep/k}, (k^\sep)^\times) 
\end{tikzcd}
\end{center}
but $H^1(\Gal{k^\sep/k}, (k^\sep)^\times) = 1$ by Hilbert 90 so we get an exact sequence,
\begin{center}
\begin{tikzcd}
1 \arrow[r] & k^\times \arrow[r] & N_G(T)(k^\sep)^G \arrow[r] & W(k^\sep)^G \arrow[r] & 1
\end{tikzcd}
\end{center}
However, by Galois descent $N_G(T)(k^\sep)^G = N_G(T)(k)$ and $W(k^\sep)^G = W(k)$ since these are sheaves on $(\Spec{k})_{\et}$. Therefore we see that,
\[ W(k) = N_G(T)(k) / Z_G(T)(k) \]
and because we $T$ is split so we have $W(k) = W(k^\sep)$ we also see that,
\[ N_G(T)(k) / Z_G(T)(k) = N_G(T)(k^\sep) / Z_G(T)(k^\sep) \]
Therefore, using the same argument as above, since $N_G(T)(k) \to N_G(T)(\bar{k}) \to S$ is surjective and therefore every Borel subgroup of $G_{\bar{k}}$ containing $T_{\bar{k}}$ is a $k$-conjugate of $B$ and therefore is defined over $k$. Thus, all the Borel subgroups containing $T_{\bar{k}}$ are defined over $k$ and are conjugate over $k$.
\bigskip\\
(DO THE EXAMPLES)

\subsubsection{(iv)}

(WHY SAID MAXIMAL??)

(MAXIMAL NOT NEEDED) 

Let $U \subset G_{\bar{k}}$ be a smooth unipotent subgroup. First I claim that $U$ is connected. This follows from the fact that $U$ admits a filtration by $\Ga$ and that extensions of connected groups by connected fibers are connected.Therefore, by definition $U$ is contained in a maximal smooth connected solvable subgroup i.e. a borel $B_{\bar{k}}$. 
\bigskip\\
Now suppose that $B \cap B' = T$ for some Borel subgroups $B$ and $B'$. Then the unipotent radical $U$ is a smooth normal unipotent subgroup and thus is contained in a Borel $B$. However, because it is normal and the Borel subgroups are conjugate $U$ is contained in every Borel. Therefore $U \subset B \cap B' = T$ but $T$ is semi-simple so $U = 1$ and thus $G$ is reductive.
\bigskip\\
Applying the above to the Borel subgroups of upper and lower triangular matrixes we see that $\GL_n$ and $\SL_n$ and $\PGL_n$ and $\Sp_{2n}$ and $\SO_n$ are reductive.

\begin{lemma}
Consider an exact sequence,
\begin{center}
\begin{tikzcd}
1 \arrow[r] & G' \arrow[r] & G \arrow[r] & G'' \arrow[r] & 1
\end{tikzcd}
\end{center}
of $k$-groups. If $G'$ and $G''$ are connected then $G$ is connected.
\end{lemma}

\begin{proof}
We apply Prop. 8.1.2 with $G$ acting on $G''$ transitively. Notice that we don't require $G''$ to be smooth but this is only needed to show that $G \to G''$ is flat using miracle flatness but from the exact sequence we have $G \to G''$ is faithfully flat. Then the stabilizer is $G'$ which we assume is connected. 
\end{proof}

\section{Homework 10}

\subsection{1}

Let $G$ be a smooth connected affine group over a field $k$.

\subsubsection{(i) DO THIS!!}

Let $T \subset G$ be a maximal $k$-torus and $N \subset G$ a smooth connected $k$-subgroup such that $N$ is normalized by $T$ meaning $T \subset N_G(N)$. Therefore we get an action $T \acts N$ so the fixed points of the action $N^T = Z_G(T) \cap N$ is smooth by HW 8 Exercise 3. 
\bigskip\\
We will make use of the correspondence between closed subgroups of $G/H$ and closed subgroups containing $H$. 
\bigskip\\
By Theorem 26.2.5, we see that $Z_G(T)$ is smooth and connected. If we can show that $Z_G(t) \cap N$ is connected then we can replace $G$ by $Z_G(T)$ and $N$ by $N \cap Z_G(T)$ to reduce to the case that $T$ is central in $G$.
\bigskip\\
However, consider the group $H = T \rtimes N$ then $Z_H(T) = T \rtimes (Z_G(T) \cap N)$ which is smooth and connected because it is $H^T$ for a torus acting on a smooth connected group $H$. Therefore the projection map (not a group map) $Z_H(T) \onto (Z_G(T) \cap N)$ and therefore $Z_G(T) \cap N)$ is connected. 
\bigskip\\
Now we can assume that $T$ is central. Therefore $G/T$ is unipotent because first base changing to $\bar{k}$ then $\Gm \embed G/T$ would correspond to a closed subgroup $H \subset G$ which is an extension of $\Gm$ by $T$ and thus $H$ is a torus containing $T$ (WHY IS EXTENSION OF TORI A TORI) but $T$ is maximal so we see that $G/T$ contains no nontrivial tori and therefore must be unipotent. Then we get a map,
\[ N / (N \cap T) \to G / T \]
which may not be a closed immersion but is a monomorphism because it is injective on the functor of points (IS THIS RIGHT). Therefore, any torus $T' \subset N / (N \cap T)$ gives a nonconstant map $T' \to G/T$ from a torus to a unipotent group giving a contradiction unless $T'$ is trivial. However, any torus $T' \subset N$ containing $N \cap T$ gives rise to a torus $\overline{T'} \subset N/(N \cap T)$ (WHY IS THE QUOTIENT A TORUS) and therefore must be trivial meaning $T' = N \cap T$. Therefore, it suffices to show that $N$ contains a torus containing $N \cap T$. Indeed, $N \cap T$ is a commutative semisimple subgroup and therefore is contained in a torus (CHECK THIS!!)
\bigskip\\
Finally, we need to show that any maximal torus of $N \cap Z_G(T)$ is maximal in $N$.
\bigskip\\
When $T$ is not maximal, we can consider the diagonal $\Gm \subset \GL_{n}$ which is contained in the center and therefore is normalized by any subgroup. Then take $\SO(q) \subset \GL{n}$ for the form,
\[ q(x) = x_1^2 + \cdots + x_n^2 \]
therefore we have $\Gm \cap \SO(q) = \mu_2$ which is disconnected for $\ch{k} \neq 2$. 

\subsubsection{(ii)}

\newcommand{\R}{\mathcal{R}}

Let $H \subset G$ be a smooth connected normal $k$-subgroup and $P \subset G$ a parabolic $k$-subgroup. Let $k = \bar{k}$ so that $(P \cap H)^0_{\red}$ is automatically a subgroup of $P \cap H$. Since $(P \cap H)^0_{\red}$ is smooth because it is reduced and $k = \bar{k}$, it suffices to show that $H / (P \cap H)^\circ$ is proper.


It suffices to show that $(P \cap H)^\circ_{\red}$ contains a Borel of $H$. To do so it suffices to show that it contains a maximal torus and the unipotent radical. By the previous part, if $T \subset P$ is a maximal torus in $G$ then $T \cap H$ is a maximal torus in $H$. Furthermore, let $\R(H) \subset H$ be the unipotent radical of $H$. Since $H$ is normal $G$ acts on $H$ by automorphisms but $\R(H) \subset H$ is characteristic and thus is fixed under this action and therefore $\R(H)$ is normal in $G$ and therefore $\R(H) \subset \R(U) \subset B$. Therefore, $\R(H) \subset (P \cap H)^0_{\red}$ since $\R(H)$ is smooth and connected. Therefore $(P \cap H)^0_{\red}$ contains a Borel proving that it is parabolic since it is also smooth and connected.
\bigskip\\
Now $(P \cap H)^0_{\red} \subset P \cap H$ is a normal subgroup. Because $(P \cap H)^0_{\red}$ is parabolic we see that $N_{H}((P \cap H)^0_{\red}) = (P \cap H)^0_{\red}$ on geometric points and therefore $P \cap H \subset N_{H}((P \cap H)^0_{\red}) = (P \cap H)^0_{\red}$ on geometric points proving that $(P \cap H)^0 = P \cap H$ so $P \cap H$ is connected. 

\subsubsection{(iii)}

Furthermore, if we know that for parabolics $Q = N_H(Q)$ scheme-theoretically then we can actually say that $P \cap H \subset N_{H}((P \cap H)^0_{\red}) = (P \cap H)^0_{\red}$ and therefore $P \cap H$ is reduced and therefore smooth as well as connected. 
\bigskip\\
In particular, if $B \subset G$ is a Borel then $B \cap H$ is a Borel $k$-subgroup since we have shown it is smooth connected and parabolic but it is also solvable because this is preserved by intersection (CHECK THIS). 

\subsection{2}

\subsubsection{(i)}

The conjugation action $\SL_2 \acts \SL_2$ is trivial on $Z(\SL_2) = \mu_2$ by definition and thus factors uniquely through the quotient $\SL_2 / Z(\SL_2) = \PGL_2$ giving an action $\PGL_2 \acts \SL_2$.
\bigskip\\
Let $G = \PGL_2$ or $\SL_2$. Let $\phi : G \to G$ be a $k$-automorphism preserving the standard upper triangular Borel and the diagonal $k$-torus. Now $G / B \iso \P^1$ we can describe the map,
\[ \begin{pmatrix}
a & b 
\\
c & d 
\end{pmatrix}
\mapsto [a : c] \]
where this is the orbit of $\infty$ under the standard action $t \mapsto \frac{a t + b}{ct + d}$ then $B$ is the stabilizer of $\infty$ since $[a : c] = [1 : 0]$ iff $c = 0$. Therefore, $\phi$ descends to a $k$-automorphism of $\P^1 \to \P^1$ and therefore is given by a $k$-point $A \in \PGL_2(k)$ acting on $\P^1$ and thus acting on $\SL_2$ through conjugation. However, since $\phi$ preserves the diagonal we must have $A$ diagonal because every diagonal matrix is a sum of a scalar matrix and a matrix on the diagonal of $\SL_2$ meaning that $A$ must commute with all diagonal matrices and thus be diagonal (since the diagonal torus of $\GL_2$ is self centralizing). 

\subsubsection{(ii)}

\renewcommand{\Aut}{\mathrm{Aut}}

Consider the map $\PGL_2(k) \to \Aut_k(G)$ where $G = \SL_2$ or $\PGL_2$. Since $\PGL_2$ has trivial center this map is injective. Let $\phi \in \Aut_k(G)$. Because the $k$-Borel subgroups are all conjugate we can conjugate $\phi$ so that it preserves $B$. Then $\phi(T) \subset B$ is a maximal torus and since all maximal tori in $B$ are $k$-conjugate we see that we can conjugate by an element of $B$ such that $\phi$ preserves $B$ and $T$ and therefore $\phi$ is in the image by part (i) proving that this map is an isomorphism. 
\bigskip\\
Therefore, every $k$-automorphism of $\PGL_2$ is inner. However, because $\SL_2(k) \to \PGL_2(k)$ may fail to be surjective this does not immediately show that every automorphism of $\SL_2$ is inner. Indeed, because $\PGL_2(k) \iso \Aut_k(\SL_2)$ we see that $\SL_2(k) \to \PGL_2(k)$ is surjective if and only if $\SL_2(k) \to \Aut_k(\SL_2)$ is surjective if and only if every $k$-automorphism of $\SL_2$ is inner. 
\bigskip\\
Now the sequence,
\begin{center}
\begin{tikzcd}
0 \arrow[r] & \mu_2 \arrow[r] & \SL_2 \arrow[r] & \PGL_2 \arrow[r] & 0
\end{tikzcd}
\end{center}
is exact and thus exact in the fppf topology (CHECK THAT IMPLICATION) of $\Spec{k}$ so there is an exact sequence,
\begin{center}
\begin{tikzcd}
0 \arrow[r] & \mu_2(k) \arrow[r] & \SL_2(k) \arrow[r] & \PGL_2(k) \arrow[r] & H^1(k, \mu_2) \arrow[r] & H^1(k, \SL_2)
\end{tikzcd}
\end{center}
however, $H^1(k, \SL_2) = 0$ by Hilbert 90. Reall I should use the exact sequence,
\begin{center}
\begin{tikzcd}
0 \arrow[r] & \SL_n \arrow[r] & \GL_n \arrow[r] & \Gm \arrow[r] & 0
\end{tikzcd}
\end{center}
which gives an exact sequence,
\begin{center}
\begin{tikzcd}
\GL_n(k) \arrow[r] & \Gm(k) \arrow[r] & H^1(k, \SL_n) \arrow[r] & H^1(k, \GL_n)
\end{tikzcd}
\end{center}
and then apply Hilbert 90 to get $H^1(k, \GL_n) = 0$ but also $\GL_n(k) \onto \Gm(k)$ is surjective for any field (even any ring just taking diagonal entires $1$ and the element you want) so $H^1(k, \SL_n) = 0$. Therefore, $\SL_2(k) \to \PGL_2(k)$ is surjective iff $H^1(k, \mu_2) = 0$. Furthermore, there is a fppf exact sequence,
\begin{center}
\begin{tikzcd}
0 \arrow[r] & \mu_n \arrow[r] & \Gm \arrow[r, "n"] & \Gm \arrow[r] & 0
\end{tikzcd}
\end{center}
and therefore we see that,
\begin{center}
\begin{tikzcd}
\Gm(k) \arrow[r, "n"] & \Gm(k) \arrow[r] & H^1(k, \mu_n) \arrow[r] & H^1(k, \Gm)
\end{tikzcd}
\end{center}
is exact but $H^1(k, \Gm) = 0$ by Hilbert 90 and therefore,
\[ H^1(k, \mu_n) = k^\times / (k^\times)^n \]
this is ``Kummer theory''. Thus $H^1(k, \mu_2) = (k^\times)/(k^\times)^2$ vanishes iff $k^\times = (k^\times)^2$ proving our claim.
\bigskip\\
Alternatively, if $k^\times = (k^\times)^2$ then every matrix $\PGL_2(k) = \GL_2(k) / \Gm(k)$ can be scaled to have trace $1$ so it is in the image of $\SL_2(k) \to \PGL_2(k)$. Conversely, if $k^\times \neq (k^\times)^2$ then choose $a \in k^\times \setminus (k^\times)^2$ so we can take the diagonal matrix,
\[ 
A = \begin{pmatrix}
a & 0
\\
0 & 1
\end{pmatrix} \]
which cannot be in the image of $\SL_2$ because any scalar multiple of $A$ by $\lambda$ has determinant $\lambda^2 a \notin (k^\times)^2$ and thus $\lambda^2 a \neq 1$. 


\subsection{3}

Let $\lambda : \Gm \to G$ be a $1$-parameter $k$-subgroup of a smooth affine $k$-subgroup $G$. Define,
\[ \mu : U_G(\lambda^{-1}) \times P_G(\lambda) \to G \]
via the multiplication. Let $\g = \Lie{(G)}$ we are going to show that $\mu$ is an open immersion.

\subsubsection{(i)}

For $n \in \Z$ we define $\g_n$ as the $n$-weight space for $\lambda$ meaning the set of $X \in \g$ such that
\[ \Ad(\lambda(t)) \cdot X = t^n X \]
on $R$-points. This means that for $(1 + \epsilon X) \in G(k[\epsilon])$ the orbit morphism $\alpha_X : \Gm \to G_{k[\epsilon]}$ defined by $t \mapsto \lambda(t) (1 + \epsilon X) \lambda(t)^{-1}$ satisfies,
\[ t \mapsto \lambda(t) (1 + \epsilon X) \lambda(t)^{-1} = 1 + \epsilon \Ad(\lambda) \cdot X = 1 + \epsilon t^n X \]
Therefore, identifying the scheme representing $\vspan{X}$ with $\A^1$ we have,
\begin{center}
\begin{tikzcd}
\Gm \arrow[d, hook] \arrow[r, "x \mapsto x^n"] & \A^1 \arrow[d, equals]
\\
\A^1 \arrow[r, "\alpha'"] & \A^1
\end{tikzcd}
\end{center}
This extension is possible if and only if $n \ge 0$. To see this, there is a unique extension to a map $\P^1 \to \P^1$ which sends $0$ into $\A^1$ if and only if $n \ge 0$ since for $n < 0$ we have $0 \mapsto \infty$. Furthermore $0 \mapsto 0$ if and only if $n > 0$. 
\bigskip\\
Now if $X \in \g_n \cap \Lie{(P_G(\lambda))}$ then $\Gm \to G$ will extend across $\Gm \to \A^1$ which implies that the action on tangent spaces extends also meaning that either $X = 0$ or $n \ge 0$ and if $X \in \Lie{(U_G(\lambda))}$ then either $X = 0$ or $n > 0$. Now we need the other direction. First, if $n = 0$ then $\Gm \acts G$ induces a trivial action on the tangent space and therefore by HW 8 Ex. 4(i) we see that $\Gm \acts G$ is trivial and therefore $\Gm \to G_A$ extends to $\A^1_A$ for any $g \in G(A)$. Now if $n > 0$ we want to show that $\Gm \to G_{k[\epsilon]}$ given by the orbit map at $1 + \epsilon X$ extends by sending $0 \mapsto e$. Because $G$ is smooth, there exists a closed curve $X \subset G$ passing through $e \in G$ with 



(HOW TO RELATE THIS TO EXTENSION TO $G$??)


Then we compute,
\[ T_{(e,e)} \mu : \Lie{(U_G(\lambda^{-1}))} \times \Lie{(P_G(\lambda))} \to \Lie{(G)} \]
on the inclusion of the two factors giving by the inclusions of $e \times P_G(\lambda)$ and $U_G(\lambda^{-1}) \times e$ map to $G$ via the standard inclusion under multiplication because $e$ acts trivially. Therefore, we have shown that on each factor this map is $\g_{\lambda^{-1} > 0} \embed \g$ and $\g_{\lambda \ge 0} \embed \g$. However, $\g_{\lambda^{-1} > 0} = \g_{\lambda < 0}$. Therefore, by HW 4 we have shown that multiplication induces a sum map on tangent spaces and therefore,
\[ T_{(e,e)} \mu : \g_{\lambda < 0} \oplus \g_{\lambda \ge 0} \to \g \]
which is an isomorphism by the weight-space decomposition of $\g$.

\subsubsection{(ii) DO THIS}

(WHAT DOES THIS MEAN)

\subsubsection{(iii)}

Let $H = U_G(\lambda^{-1}) \times P_G(\lambda)$. For each $\xi \in H(\bar{k})$ we can translate by $\xi$ to see that $\d{\mu}(\xi)$ is a translate of the isomorphism $T_{(e,e)}$ and thus is an isomorphism on tangent spaces. Therefore, if $H$ is smooth then $\mu$ is \etale because it is an isomorphism on tangent spaces [EGA IV.4, 17.11.1]. Therefore it has open image and is quasi-finite because $\mu$ is finite type (it is a product of open immersions of noetherian schemes) and \etale. 
\bigskip\\
Okay, let's actually do it the intended way. Let $A \to B$ be a local map of regular local rings such that $\m_A / \m_A^2 \to \m_B / \m_B^2$ and $A / \m_A \to B / \m_B$ are isomorphisms. Then we have,
\begin{center}
\begin{tikzcd}
0 \arrow[r] & \m_B^n / \m_B^{n+1} \arrow[r] & B / \m_B^{n+1} \arrow[r] & B / \m_B^n \arrow[r] & 0 
\\
0 \arrow[r] & \m_A^n / \m_A^{n+1} \arrow[u] \arrow[r] & A / \m_A^{n+1} \arrow[r] \arrow[u] & A / \m_A^n \arrow[r] \arrow[u] & 0 
\end{tikzcd}
\end{center}
However, by regularity the leftward maps in the following diagram are isomorphisms,
\begin{center}
\begin{tikzcd}
\nSym{n}{\m_B/\m_B^2} \arrow[r, "\sim"] & \m_B^n / \m_B^{n+1}
\\
\nSym{n}{\m_A/\m_A^2} \arrow[u] \arrow[r, "\sim"] & \m_A^n / \m_A^{n+1} \arrow[u]
\end{tikzcd}
\end{center}
Since $\m_A / \m_A^2 \to \m_B / \m_B^2$ is an isomorphism, we also see that $\m_A^n / \m_A^{n+1} \to \m_B^n / \m_B^{n+1}$ is an isomorphism. Therefore by inductively applying the 5-lemma we get isomorphisms $A / \m_A^{n} \iso B / \m_B^{n}$ for all $n$ and therefore $\hat{A} \to \hat{B}$ is an isomorphism.
\bigskip\\
Therefore, given $f : X \to Y$ for smooth $k$-schemes with $Tf$ an isomorphism on all $\bar{k}$-points we see that there are isomorphisms $\widehat{\stalk{Y}{y}} \to \widehat{\stalk{X}{x}}$ for $y = f(x)$. This implies that $f$ is flat and quasi-finite by the following. 
\bigskip\\
First, because $\widehat{\stalk{Y}{y}} \to \widehat{\stalk{X}{x}}$ is an isomrphism we see that,
\[ \m_y \widehat{\stalk{X}{x}} = \m_x \widehat{\stalk{X}{x}} \]
Therefore, because for any local ring $A$, the map $A \to \hat{A}$ is local we see,
\[ \m_y \stalk{X}{x} = \m_y \widehat{\stalk{X}{x}} \cap \stalk{X}{x} = \m_x \widehat{\stalk{X}{x}} \cap \stalk{X}{x} = \m_x \]
Therefore, each $\bar{k}$-point is isolated in its fiber (equivalently $f$ is unramified) and therefore $f$ has finite fibers because it is finite type and thus quasi-compact because a discrete closed subset of a quasi-compact set discrete and quasi-compact and thus finite.
\bigskip\\
Furthermore, there is a diagam,
\begin{center}
\begin{tikzcd}
\stalk{Y}{y} \arrow[r] \arrow[d] & \stalk{X}{x} \arrow[d]
\\
\widehat{\stalk{Y}{y}} \arrow[r] & \widehat{\stalk{X}{x}}
\end{tikzcd}
\end{center}
and the downward maps are faithfully flat. Since the bottom map is an isomorphism and thus flat we see that $\stalk{Y}{y} \to \stalk{X}{x}$ is flat by the following lemma.

\begin{lemma}
Let $A \to B$ and $B \to C$ be ring maps with $B \to C$ faithfully flat with $A \to B \to C$ flat. Then $B \to C$ is flat.
\end{lemma}

\begin{proof}
Consider an exact sequence of $A$-modules,
\begin{center}
\begin{tikzcd}
0 \arrow[r] & M_1 \arrow[r] & M_2 \arrow[r] & M_3 \arrow[r] & 0
\end{tikzcd}
\end{center}
Then by flatness of $A \to B \to C$ we get an exact sequence,
\begin{center}
\begin{tikzcd}
0 \arrow[r] & M_1 \ot_A C \arrow[d, equals] \arrow[r] & M_2 \ot_A C \arrow[d, equals] \arrow[r] & M_3 \ot_A C \arrow[d, equals] \arrow[r] & 0
\\
0 \arrow[r] & (M_1 \ot_A B) \ot_B C \arrow[r] & (M_2 \ot_A B) \ot_B C \arrow[r] & (M_3 \ot_A B) \ot_B C \arrow[r] & 0
\end{tikzcd}
\end{center}
However, since $B \to C$ is faithfully flat a sequence is exact after applying $- \ot_B C$ if and only if it was exact to start with. Therefore, 
\begin{center}
\begin{tikzcd}
0 \arrow[r] & M_1 \ot_A B \arrow[r] & M_2 \ot_A B \arrow[r] & M_3 \ot_A B  \arrow[r] & 0
\end{tikzcd}
\end{center}
is exact so $A \to B$ is flat.
\end{proof}

\begin{rmk}
We say that ``flatness descends under faithfully flat descent''. What we mean is the following proposition.
\end{rmk}

\begin{prop}
If $f : X \to Y$ is an $S$-morphism and $g : S' \to S$ is faithfully flat such that $f_{S'}$ is flat then $f$ is flat.
\end{prop}

\begin{proof}
Because flatness is local on the source and target, we immediately reduce to the affine case. Then $\Spec{B} \to \Spec{A}$ over a base $\Spec{R}$ and $g : \Spec{R'} \to \Spec{R}$ faithfully flat. Then consider the diagram,
\begin{center}
\begin{tikzcd}
A \arrow[d] \arrow[r] & B \arrow[d]
\\
A \ot_R R' \arrow[r] & B \ot_R R'
\end{tikzcd}
\end{center}
with $A \ot_R R' \to B \ot_R R'$ flat. Because faithfull flatness is preserved under base change $A \to A \ot_R R'$ and $B \to B \ot_R R'$ are faithfully flat and thus $A \to A \ot_R R' \to B \ot_R R'$ is flat but this equals $A \to B \to B \ot_R R'$ and the second map is faithfully flat so by the lemma $A \to B$ is flat proving the claim.
\end{proof}

\subsubsection{(iv)}

(DO I NEED FINITE TYPE FOR EXAMPLE WHAT ABOUT??)

$\Spec{k(t)} \to \Spec{k}$ is not proper. 

(HMM)

Let $f : X \to Y$ be a flat quasi-finite separated map of noetherian schemes such that $X_y \to \Spec{\kappa(y)}$ is a finite scheme of rank $d$ for all $y \in Y$. Because these schemes are noetherian and $f$ is separated (DO I NEED FINITE TYPE), to show that $f$ is proper, it suffices to prove that $f$ satifies the existence part of the valuative criterion for discrete valuation rings. Let $R$ be a DVR and $K = \Frac{A}$. Consider a diagram,
\begin{center}
\begin{tikzcd}
\Spec{K} \arrow[d] \arrow[r] & X \arrow[d]
\\
\Spec{R} \arrow[r] & Y
\end{tikzcd}
\end{center}
We need a morphism $\Spec{R} \to X$ making the diagram commute therefore giving a section of $X_R \to \Spec{R}$ which is also flat quasi-finite separated and $X_R$ is noetherian and has constant rank fibers. Therefore we may replace $X$ with $X_R$ to get a diagram,
\begin{center}
\begin{tikzcd}
\Spec{K} \arrow[r] \arrow[d] & X \arrow[d]
\\
\Spec{R} \arrow[r, equals] & \Spec{R}
\end{tikzcd}
\end{center}
Let $\xi \in \Spec{R}$ and $\eta \in \Spec{R}$ be the generic and special points respectively. Let $S \subset X(K)$ be the set of points $\Spec{K} \to X$ which extend to some $\Spec{R} \to K$. By the valuative criterion for separatedness, the map $X(R) \to X(K)$ is injective and $S \to X(R) \to S$ is the identity by definition so $S \to X(R)$ is injective. Therefore, $S \iso X(R)$. 
\bigskip\\
Let $\kappa = \kappa(\eta)$ then we have a map $X(R) \to X(\kappa)$. However, I claim that $X(K) \to X(\kappa)$ 

\subsection{4}

Let $\lambda : \Gm \to G$ be a 1-parameter $k$-subgroup of a smooth affine $k$-group. Let $n \ge 1$ be an integer. We need to consider the closed subschemes $P_G(\lambda^n)$ and $U_G(\lambda^n)$ and $Z_G(\lambda^n)$. To show that $P_G(\lambda^n) = P_G(\lambda)$ (and the rest) it suffices to show that they represent the same subfunctor of $A \mapsto G(A)$. Then,
\[ P_G(\lambda^n)(A) = \{ g \in G(R) \mid \lim_{t \to 0} t \cdot g \text{ exists } \} \]
Meaning that the $A$-map $\alpha_g' : \Gm \to G_A$ extends along $\Gm \to \A^1_A$. However, $\lambda^n$ gives the orbit map,
\[ \alpha'_g : \Gm \tolabel{n} \Gm \tolabel{\alpha_g} G \]
therefore if $\alpha_g$ extends then so does $\alpha'_g$ because there is a diagram,
\begin{center}
\begin{tikzcd}
\Gm \arrow[d] \arrow[r, "x \mapsto x^n"] & \Gm \arrow[d] \arrow[r, "\alpha_g"] & G 
\\
\A^1_A \arrow[r, "x \mapsto x^n"] & \A^1_A \arrow[ru, dashed]
\end{tikzcd}
\end{center}
meaning that $P_G(\lambda)(A) \subset P_G(\lambda^n)(A)$. To prove the converse, notice that this square is a pushout in the category of affine schemes because,
\begin{center}
\begin{tikzcd}
A[x, x^{-1}] \arrow[from=r] \arrow[from=d] & A[x^n,x^{-n}] \arrow[from=d]
\\
A[x] \arrow[from=r] & A[x^n]
\end{tikzcd}
\end{center}
is a fiber product diagram (i.e. an intersection in $A[x, x^{-1}]$). Therefore, $\alpha_g$ extends if and only if $\alpha_g'$ extends.
\bigskip\\
Again it is clear that $U_G(\lambda) \subset U_G(\lambda^n)$ and $Z_G(\lambda) \subset Z_G(\lambda^n)$. Because $\A^1_A \to \A^1_A$ sends $0 \mapsto 0$ we see that if $\alpha_g'$ extends sends $0 \mapsto 1$ then by commutativity of the diagram $\alpha_g$ also extends sending $0 \mapsto 1$ so $U_G(\lambda) = U_G(\lambda^n)$. Finally, we consider $Z_G(\lambda^n)$. Then $\alpha_g'$ is the trivial map so the image of $\alpha_g$ is trivial on the \etale neighborhood $\Gm \tolabel{n} \Gm$. Therefore $\alpha_g = 1$ by \etale descent for subschemes since if $Z$ is the subscheme on which $\alpha_g$ and $1$ agree then $Z$ pulled back along $\Gm \tolabel{n} \Gm$ is $\Gm$ so $Z = \Gm$.

\subsection{5 DO THIS}

(IS THERE A REASON CLOSED IS SPECIFIED ISNT THAT PART OF OUR DEFINITION?)


This is lemma 25.5.3. I will reproduce the proof.
\\bigskip\\
Let $G$ be a reductive group over $k$ and $N \subset G$ a smooth closed normal $k$-subgroup.  Reductivity is a geometric notion so we base change to $\bar{k}$ and assume that $k$ is algebraically closed. Consider $\R(N) \subset N$ which is the maximal normal smooth connected unipotent subgroup all these are intrinsic properties to $\R(N)$ except for normality so because $\R(G)$ is trivial it suffices to prove that $\R(N)$ is normal in $G$. Because the image of a unipotent element is unipotent we see that $\R(N)$ is preserved in $N$ under any automorphism of $N$ (we say that $\R(N)$ is a characteristic subgroup). Since $N \subset G$ is normal conjugation induces an automorphism on $N$ and therefore preserves $\R(N)$ so $\R(N)$ is normal in $G$ and therefore $\R(N)$ is trivial completing the proof.

\subsection{6 DO THIS}

If $d \divides n$ then factoring the map $n : \Gm \to \Gm$ into $\Gm \tolabel{d} \Gm \tolabel{n/d} \Gm$ gives a diagram,
\begin{center}
\begin{tikzcd}
\mu_d \arrow[d] \arrow[r] & 0 \arrow[d]
\\
\mu_n \arrow[d] \arrow[r, "d"] & \mu_n \arrow[d]
\\
\Gm \arrow[r, "d"] & \Gm
\end{tikzcd}
\end{center}
because the entire rectangle is Cartesian we see that the top square must also be Cartesian proving that $\mu_d = \mu_n[d]$.
\bigskip\\
There is a canonical map $\underline{\Z/n\Z} \to \End{\mu_n}$ defined functorially on a $k$-algebra $A$ by sending $r \mapsto (x \mapsto x^r)$. It is straightforward to check that this a ring map. Furthermore, I claim that $\Z / n \Z \to \End{\mu_{n,A}}$ is an isomorphism. Injectivity is clear because if $x^a = x^b$ then $x^{a-b} = 1$ since $x$ is a unit so $n \divides a - b$ so $[a] = [b]$ in $\Z / n \Z$. For surjectivity we need to consider endomorphisms of $A[x]/(x^n - 1)$ as a Hopf algebra over $A$. These are determined by the image of $x$ say $x \mapsto f(x)$. Then to be a Hopf algebra map we must have,
\[ f(\Delta x) = \Delta f(x) \]
meaning that if,
\[ f(x) = \sum_{i = 0}^d c_i x^i \]
then we have,
\[ \sum_{i = 0}^d c_i x^i \ot x^i = \sum_{i,j \ge 0} c_i c_j x^i \ot x^j \] 
so we need off diagonal terms to vanish meaning $c_i c_j = 0$ for $i \neq j$. If $A$ is a domain then $c_i$ and $c_j$ can only be supported in one degree (mod $n$) so we find that $f(x) = c_i x^i$ and $c_i = c_i^2$. Furthermore, $f(x)^n = 1$ meaning that $(c_i x^i)^n = c_i^n x^{in} = c_i^n = 1$ so $c_i \in A$ is a root of unity with $c_i^2 = c_i$ so $c_i$ is a unit and therefore $c_i = 1$ so we see that $f(x) = x^i$. 
\bigskip\\
For the general case, (DO GENERAL CASE!! REDUCTION TO LOCAL CASE THEN MORE WORK)


\subsection{7}

Let $\phi : G \rat H$ be a rational homomorphism of smooth connected groups over $k$. Because $G$ is irreducible, there is a maximal open locus $U \subset G$ on which $\phi$ is defined. It suffices to show that $U = G$ so that $\phi$ extends to a $k$-homomorphism. To show that $U = G$ we can pass to $k^\sep$ by Galois descent for open subschemes. Furthermore, I claim that $U_{k^\sep}$ is the maximal locus on which $\phi$ is defined. If $\phi'$ extends $\phi$ over a larger $U'$ then it also extends over the galois orbit of $U'$ because $\sigma(\phi')$ agrees with $\sigma(\phi) = \phi$ on $U$ and thus by the uniqueness of rational map $\sigma(\phi') = \sigma(\phi)$ on the overlap of their domains so we get $\phi'$ defined on the Galois orbit on $U'$ and $\phi'$ is Galois-invariant because $\sigma(\phi')$ and $\phi'$ both restrict to $\sigma(\phi) = \phi$ on $U$ and therefore are equal. Thus by Galois descent, $\phi$ extends to $\phi' : U' \to H$ over $k$. Therefore, we may assume that $k = k^\sep$.
\bigskip\\
Suppose that a tranlate of $U$ by $g \in U(k)$ is not contained in $U$. Then we can define $\phi$ on $U \cup g \cdot U$ via $\phi : U \to H$ and $\phi' : g \cdot U \to H$ where $\phi'(g \cdot x) = \phi(g) \phi(x)$. These agree on the overlap because $\phi : U \to H$ is a homomorphism wherever multiplication on $U$ is well-defined. Thus these glue to extned $\phi$ on $U \cup g \cdot U$. However, $U$ is the maximal set on which $\phi$ is defined so $g \cdot U \subset U$. 
\bigskip\\
Therefore, we may assume that all translates of $U$ by $U(k)$ are all contained in $U$. Then the multiplication map $m : U \times U \to X$ must factor through $U \subset X$ because the $k$-points in $U$ are dense. Therefore $U$ is an open subgroup and therefore also closed meaning that $U = G$ because $G$ is connected. 

\subsection{8}

Let $G$ be a smooth connected affine group over an algebraically closed field $k$ with $\ch{k} = 0$. 

\begin{lemma}
Let $G$ be an algebraic group and $N \subset G$ a normal subgroup. Then any finite-dimensional semisimple $G$-representation restricts to a semisimple $N$-representation.
\end{lemma} 

\begin{proof}
It suffices to consider an irreducible $G$-representation. It suffices to show this for the group $G_A$ acting on $V_A$ compatbility so I will supress the $A$ and work with actual groups (DOES THIS WORK?). Suppose that $W \subset V$ is an $N$-irreducible representation. Because $V$ is irreducible, the $G$-translates of $V$ span $W$. Therefore, choosing a set of representatives $g_i$ for $G/N$ we see that,
\[ V = \sum_{i} g_i \cdot W \]
Because $N$ is normal, $g_i \cdot W$ is a $N$-representation because $n g_i w = g_i (g_i^{-1} n g_i) w$ but $g_i^{-1} n g_i \in N$ by normality so $(g_i^{-1} n g_i) w \in W$ proving that this is actually an action. It is the action $N \acts W$ twisted by conjugation by $g_i$. Since the conjugation map $N \to N$ is surjective $N \to N \to \GL{(W)}$ is also irreducible. Now, $g_i \cdot W \cap g_j \cdot W$ is an $N$-invariant subspace of both $g_i \cdot W = g_j \cdot W$ therefore by irreducibility, either $g_i \cdot W = g_j \cdot W$ (both subspaces are everything) or $g_i \cdot W \cap g_j \cdot W = (0)$. Therefore, we can reduce out set of coset representatives to hit each of these subspaces exactly once (which is a finite set by dimension reasons) and therefore,
\[ V = \bigoplus_{i} g_i \cdot W \]
with $W$ irreducible proving that $V$ is a semisimple $N$-representation. I did not use the base ring anywhere so this works totally at the level of functors and by dimension reasons there are the same number of $g_i$ at each step and thus they are compatible. 
\end{proof}

\subsubsection{(i)}

Suppose that all finite-dimensional representations of $G$ are completely reducible (i.e. semi-simple). To show that $G$ is reducitive, we need to show that it does not contain any nontrivial normal smooth connected unipotent subgroups. 
\bigskip\\
Let $N \subset G$ be a normal smooth connected unipotent subgroup. Let $\rho : G \to \GL{(V)}$ be an embedding of $G$ into $\GL{(V)}$ which exists because $G$ is affine and finite type over $k$.
\bigskip\\
Suppose I can show that after restriction to $N$ then every representation remains semisimple. However, $N$ is solvable (and split because $k = \bar{k}$) so given a representation $\rho : G \to \GL{(V)}$ we get a representation $\rho : N \to \GL{(V)}$ which can be conjugated into the upper triangular subgroup of $\GL{(V)}$ by Lie-Kolchin. Let $T \subset \GL{(V)}$ be the diagonal torus, then $N \cap T \subset N$ consists of semisimple elements so $N \cap T = 1$ since $N$ is unipotent its image is unipotent and upper triangular and therefore embeds in the strictly upper triangular matrices. However this representation cannot be semisimple if $N$ is nontrivial (see the lemma). Therefore $N = 1$ provint that $G$ is reductive.

\begin{lemma}
Let $U$ be a unipotent group and $\rho : U \to \GL{(V)}$ a representation. Then $\rho$ is semisimple iff $\rho$ is the trivial action on $V$.
\end{lemma}

\begin{proof}
We proceed by induction on $\dim{V}$. By Theorem 16.1.8, $\rho$ has a fixed point $v \in V$ and thus assuming that $V$ is semisimple we have a complement $W \subset V$ to $\vspan{v}$. By the induction hypothesis, $\rho$ acts trivially on $W$ since $W$ is semi-simple and thus $U$ acts trivially on $V = \vspan{v} \oplus W$ as $U$-modules. For the base case, if $\dim{V} = 1$ then $V$ is automatically irreducible and thus semisimple but since $U$ is unipotent $\rho$ has a fixed point so $(\rho, V)$ is the trivial representation. 
\end{proof}

\subsubsection{(ii)}

Suppose that $G$ is reductive. Then $\Lie{(\D(G))}$ is a semi-simple Lie algebra and a subspace of a finite-dimension linear representation space for $G$ is $G$-stable if and only if it is $\g$-stable under the induced $\g \to \End{V}$ since $\ch{k} = 0$.
\bigskip\\
We need to prove that $\g$ is semi-simple. 
\bigskip\\
Now all finite-dimensional $G$-representations are semi-simple. Given $\rho : G \to \End{V}$ and a $G$-invariant subspace $W \subset V$ we need to produce a $G$-invariant complement. We know that $W$ is $\g$-invariant so we can choose a $\g$-invariant complement $W' \subset V$ because $\g$ is semi-simple. Therefore $W'$ is a $G$-invariant complement proving the claim. 


\begin{lemma}[Weyl]
Let $\g$ be a semi-simple lie algebra. Then any finite $\g$-module is semisimple.
\end{lemma}

\begin{proof}
(DO THIS!!!)
\end{proof}

\section{Questions}



\subsection{Questions}

\begin{enumerate}
\item For Cartier's theorem, do we need $x \in X$ to be a rational point? I think we need this for the map $k[[x_1, \dots, x_n]] \to \widehat{\stalk{X}{x}}$ to be surjective.

\item How to do the examples in 1(ii) from the result?

\item What am I supposed to do in 3(iii).

\item In 4(iv) am I just supposed to use Prop A.3.1. Also how to duo the Lie algebra part (just \etale locally)?

\item did he define $\gsp$

\item IS SURJECTIVITY OF GROUPS THE SAME AS SURJECTIVITY OF SHEAVES??
\end{enumerate}

\subsection{When is Weil Restriction of $\Gm$ a Torus}

Only in generally in the Galois case. For example, let $k$ have characteristic $2$ and $t \in k \setminus k^2$ then let $k' = k(t^{\frac{1}{2}})$. Then,
\[ (R_{k'/k} \Gm)_k'(A) = (A \otimes_k k')^\times \]
but since we base changed to $k'$ this means we only consider $k'$-algebras $A$ and then $A \otimes_k k' = A \otimes_{k'} (k' \otimes_k k') \cong A[\epsilon]$ because $k' \otimes k' = k'[x]/(x^2 - t) = k'[x]/(x - t^{\frac{1}{2}})^2 \cong k'[\epsilon]$. Therefore,
\[ (R_{k'/k} \Gm)_k'(A) = (A \otimes_k k')^\times \cong (A[\epsilon])^\times \cong A^\times \oplus \epsilon A \]
with multiplication $(a + b \epsilon)(a' + b' \epsilon) = aa' + (a b' + a' b) \epsilon$. Therefore, $b \mapsto 1 + b \epsilon$ gives an embedding of $\Ga \embed (R_{k'/k} \Gm)_{k'}$ so there is no way that $R_{k'/k} \Gm$ can be a torus.


\subsection{Lemma 7.2.1.}

How does this work? What about $k = \FF_p(t)$ and $A = k[x]/(x^p - t)$. Then $A$ is a reguar (it is the field $k' = k(t^{\frac{1}{p}})$) noetherian local ring but $A \otimes_k k' = k'[x]/(x - t^{\frac{1}{p}})$ is already local and is noetherian but is not regular. What gives? Ahhhh! It has residue field $k'$ not $k$. Therefore, having a $k$-rational point here is the essential part as we showed on homework $1$. 

\subsection{NOW!}

\subsection{Example 7.2.2.}

Is it actually true that ``You only get power series whose coefficients all live in a finite dimensional $\Q$-subspace of $\CC$. What about if $\alpha \in \CC$ is transcendental then $1 - \alpha x \in A_{\CC}$ but not in $\m'$ and thus,
\[ \frac{1}{1 - \alpha x} = 1 + \alpha x + \alpha^2 x^2 + \cdots \]
is in $A'$ although $\Q(\alpha)$ is infinite dimensional.

\subsection{Is it Enough to Check Geometrically Reduced at one Point}

To show that an algebriac $k$-group is smooth, does it suffice to show that $\stalk{G}{e}$ is geometrically reduced?

\subsection{When does an Artin local ring contain its residue field?}

First of all, clearly no because of $\Z / p^2 \onto \Z / p$ but $\Z / p^2$ does not have any subfields. However, there is a submodule $\Z / p \embed \Z / p^2$ sending $1 \mapsto p$ but this does not map isomorphically onto $\Z / p$ (indeed it maps to zero).
\bigskip\\
Why did I think this was true then?!
\bigskip\\
Let's consider some cases in which it is true. Let $(A, \m, \kappa)$ be an Artinian local $k$-algebra with $\m^2 = (0)$ and $\kappa / k$ separably generated. Then $\kappa / k$ is formally smooth so there is a lifting,
\begin{center}
\begin{tikzcd}
k \arrow[r] \arrow[d] & A \arrow[d]
\\
\kappa \arrow[ru, dashed] \arrow[r, equals] & A / \m
\end{tikzcd}
\end{center}
giving a subfield of $A$ such that $\kappa \to A \to A / \m$ is an isomorphism. This is a shadow of a much stronger result.

\begin{theorem}
Let $(A, \m, \kappa)$ be a complete local $k$-algebra with $\kappa / k$ separably generated. Then there is a subfield $K \subset A$ such that $K \subset A \to \kappa$ is an isomorphism.
\end{theorem}
\noindent\
Okay but where does the lifting as a module come from?

\subsection{Does Infinitessimal Lifting alow lifing along any quotient of an Artin ring?}

\section{Questions for Oct. 19}

\begin{enumerate}
\item For smoothness in 3(iii) do we actually need that $k'$ is a field? What I don't understand is that $K'/K$ will not be an extension of fields anymore for $K = \bar{k}$.
\item Good proof that $\mu_p$ is not a subgroup of $\Ga$?
\item How to prove that smooth connected $k$-subgroup is a torus (JUST BECAUSE WE DIDN'T DICSUSS SMOTHNESS OVER $k$)
\item Why is $R_{k'/k}(\Gm)$ unirational?
\item Why does $\Lambda$ embed in the product of regular representations??
\item Why is 
\end{enumerate}


\begin{rmk}
NOTICE: Neron model book does the Moichizuki Hartshorne thing with affine space.
\end{rmk}

\end{document}