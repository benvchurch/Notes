% Typesetting Packages

\usepackage[utf8]{inputenc}
\usepackage[english]{babel}
\usepackage[a4paper, total={7in, 9.5in}]{geometry}

% Math Packages
 
\usepackage{amsthm, amssymb, amsmath, centernot, graphicx}

% Fonts

\usepackage{calligra, mathrsfs}
\usepackage{lmodern}
\usepackage[T2A,T1]{fontenc}

% Other Font Options

% \usepackage{mathpazo} % The Rising Sea
% \usepackage{euler}

% \usepackage{euler} % Vaughan's notes
% \usepackage{eulervm}

%\usepackage{kpfonts}  

% \usepackage{libertine} 
% \usepackage{libertinust1math}

\usepackage{amsfonts} % Stacks project math font

% Make a not ugly version of widetilde

\DeclareSymbolFont{kplargesymbols}{OMX}{jkp}{m}{n}
\DeclareMathAccent{\wt}{\mathalpha}{kplargesymbols}{"65}
\DeclareMathAccent{\wtt}{\mathord}{largesymbols}{"65}
\newcommand{\wh}[1]{\widehat{#1}}

% Graphics Packages

\usepackage{tikz-cd}
\usepackage{graphicx}

\newsavebox{\pullbacksymbol}
\sbox\pullbacksymbol{%
\begin{tikzpicture}%
\draw (0,0) -- (1ex,0ex);%
\draw (1ex,0ex) -- (1ex,1ex);%
\end{tikzpicture}}

\newsavebox{\pushoutsymbol}
\sbox\pushoutsymbol{%
\begin{tikzpicture}%
\draw (0,0) -- (-1ex,0ex);%
\draw (-1ex,0ex) -- (-1ex,-1ex);%
\end{tikzpicture}}

\newcommand{\pullback}{\arrow[dr, phantom, "\usebox\pullbacksymbol" , very near start, color=black]}
\newcommand{\cartesian}{\arrow[dr, phantom, "\usebox\pullbacksymbol" , very near start, color=black]}
\newcommand{\pushout}{\arrow[lu, phantom, "\usebox\pushoutsymbol" , very near start, color=black]}

\newcommand{\connectingmap}[1]{
\arrow[draw=none]{d}[name=Z, shape=coordinate]{}
\arrow[#1,
rounded corners, crossing over,
to path={ -- ([xshift=2ex]\tikztostart.east)
|- (Z) [near end]\tikztonodes
-| ([xshift=-2ex]\tikztotarget.west)
-- (\tikztotarget)}]}

% General Utilities 

\renewcommand{\bf}[1]{\mathbf{#1}}
\newcommand{\notimplies}{%
  \mathrel{{\ooalign{\hidewidth$\not\phantom{=}$\hidewidth\cr$\implies$}}}}
\renewcommand{\labelenumi}{(\alph{enumi})}
\usepackage{hyperref}
\newcommand{\chref}[2]{\href{#1}{\color{blue}{\underline{#2}}}}

% Set-Theoretic Commands

\newcommand{\sm}{\! \setminus \!}
\renewcommand{\empty}{\varnothing}
\newcommand{\sub}{\subset}
\newcommand{\sups}{\supset}

% Functions

\newcommand{\embed}{\hookrightarrow}
\newcommand{\hook}{\hookrightarrow}
\newcommand{\into}{\hookrightarrow}
\newcommand{\inj}{\hookrightarrow}
\newcommand{\onto}{\twoheadrightarrow}
\newcommand{\surj}{\twoheadedrightarrow}
\newcommand{\iso}{\xrightarrow{\sim}}
\newcommand{\tolabel}[1]{\xrightarrow{#1}}
\newcommand{\lmap}[1]{\xrightarrow{#1}}


% Commands on Functions

\newcommand{\id}{\mathrm{id}}
\DeclareMathOperator{\coker}{\mathrm{coker}}
\DeclareMathOperator{\eq}{\mathrm{eq}}
\DeclareMathOperator{\coeq}{\mathrm{coeq}}
\renewcommand{\Im}[1]{\mathrm{Im}\left( #1 \right)}
\DeclareMathOperator{\im}{\mathrm{im}}
\DeclareMathOperator{\coim}{\mathrm{coim}}
\newcommand{\invI}[2]{#1^{-1} \left( #2 \right)}
\newcommand{\ev}{\mathrm{ev}}

% Complex Analysis Commands

\newcommand{\real}[1]{\mathfrak{Re}\left( #1 \right)}
\newcommand{\imag}[1]{\mathfrak{Im}\left( #1 \right)}

% Category-Theoretic Commands

\newcommand{\Hom}[3]{\mathrm{Hom}_{#1} \left( #2, #3 \right)}
\newcommand{\Homover}[3]{\mathrm{Hom}_{#1} \left( #2, #3 \right)}
\newcommand{\Der}[3][]{\mathrm{Der}_{#1} \left( #2, #3 \right)}
\newcommand{\End}[2][]{\mathrm{End}_{#1} \left( #2 \right)}
\newcommand{\Aut}[2][]{\mathrm{Aut}_{#1} \left( #2 \right)}

\newcommand{\op}{\mathrm{op}}
\DeclareMathOperator{\colim}{\mathrm{colim}}

% Commands for Groups and Actions

\newcommand{\ab}{\mathrm{ab}}
\DeclareMathOperator{\ord}{\mathrm{ord}}
\newcommand{\fix}[2]{\mathrm{Fix}_{#1} (#2)}

\usepackage{bbm}
\newcommand{\acts}{\; \rotatebox[origin=c]{-90}{$\circlearrowright$} \;}



\DeclareMathOperator{\orb}{\mathrm{Orb}}
\DeclareMathOperator{\stab}{\mathrm{Stab}}
\DeclareMathOperator{\Orb}{\orb}
\DeclareMathOperator{\Stab}{\stab}

\newcommand{\galgroup}[1]{\mathrm{Gal}\left( #1 \right)}
\newcommand{\Gal}[1]{\mathrm{Gal}\left( #1 \right)}
\newcommand{\gal}[1]{\mathrm{Gal}\left( #1 \right)}
\newcommand{\sep}{\mathrm{sep}}

% Linear Algebra Commands

\DeclareMathOperator{\rank}{\mathrm{rank}}
\newcommand{\vspan}[1]{\mathrm{span}\! \left\{#1 \right\}}
\newcommand{\Tr}[1]{\mathrm{Tr}\left( #1 \right)}
\DeclareMathOperator{\tr}{\mathrm{tr}}
\newcommand{\exterior}{\bigwedge\nolimits}

% Topology Commands

\newcommand{\ball}[2]{B_{#1} \! \left(#2 \right)}
\newcommand{\topo}{\mathcal{T}}
\newcommand{\base}{\mathcal{B}}

\newcommand{\RP}{\mathbb{RP}}
\newcommand{\rp}{\mathbb{RP}}
\newcommand{\CP}{\mathbb{CP}}
\newcommand{\cp}{\mathbb{CP}}

% Number Theory Commands

\newcommand{\Z}{\mathbb{Z}}
\newcommand{\ZZ}{\mathbb{Z}}
\newcommand{\N}{\mathbb{N}}
\newcommand{\NN}{\mathbb{NN}}
\newcommand{\Zplus}{\mathbb{Z}^{+}}
\newcommand{\Primes}{\mathbb{P}}
\newcommand{\Q}{\mathbb{Q}}
\newcommand{\QQ}{\mathbb{Q}}
\newcommand{\FF}{\mathbb{F}}
\newcommand{\RR}{\mathbb{R}}
\newcommand{\CC}{\mathbb{C}}
\newcommand{\divides}{\mid}
\newcommand{\ndivides}{\centernot \mid}
\newcommand{\modeq}[3]{#1 \equiv #2 \quad (\mathrm{mod} \: {#3})}
\DeclareMathOperator{\ch}{\mathrm{char}}

% Homological Algebra Commands

\newcommand{\dsum}{\oplus}
\newcommand{\bop}{\bigoplus}
\newcommand{\ot}{\otimes}
\newcommand{\tensor}{\otimes}
\renewcommand{\bot}{\bigotimes}
\newcommand{\btensor}{\bigotimes}
\newcommand{\Tor}[4]{\mathrm{Tor}^{#1}_{#2} \left( #3, #4 \right)}
\newcommand{\Ext}[4]{\mathrm{Ext}^{#1}_{#2} \left( #3, #4 \right)}

% Representation Theory Commands

\newcommand{\Res}[3]{\mathrm{Res}^{#1}_{#2} \left( #3 \right)}
\newcommand{\Ind}[3]{\mathrm{Ind}^{#1}_{#2} \left( #3 \right)}
\newcommand{\Rep}{\mathrm{Rep}}

% Calculus Commands

\renewcommand{\d}{\mathrm{d}}
\newcommand{\dc}{\mathrm{d}^c}
\newcommand{\dn}[2]{ \mathrm{d}^{#1} #2 \:}
\newcommand{\deriv}[2]{\frac{\d{#1}}{\d{#2}}}
\newcommand{\nderiv}[3]{\frac{\dn{#1}{#2}}{\d{#3}^{#1}}}
\newcommand{\pderiv}[2]{\frac{\partial{#1}}{\partial{#2}}}
\newcommand{\parsq}[2]{\frac{\partial^2{#1}}{\partial{#2}^2}}
\newcommand{\npar}[3]{\frac{\partial^{#1} #2}{\partial {#3}^{#1}}}

% Environments and Proofs

\renewcommand{\qedsymbol}{$\square$}
\newcommand{\cont}{$\boxtimes$}
 
\theoremstyle{remark}
\newtheorem*{remark}{Remark}
\newtheorem*{rmk}{Remark}

\theoremstyle{definition}
\newtheorem{theorem}{Theorem}[subsection]
\newtheorem{thm}[theorem]{Theorem}
\newtheorem{lemma}[theorem]{Lemma}
\newtheorem{lm}[theorem]{Lemma}
\newtheorem{proposition}[theorem]{Proposition}
\newtheorem{prop}[theorem]{Proposition}
\newtheorem{corollary}[theorem]{Corollary}
\newtheorem{cor}[theorem]{Corollary}
\newtheorem{example}[theorem]{Example}
\newtheorem{ex}[theorem]{Example}
\newtheorem{definition}[theorem]{Definition}
\newtheorem{defn}[theorem]{Definition}
\newtheorem{conjecture}[theorem]{Conjecture}
\newtheorem{conj}[theorem]{Definition}

\newcommand{\newmarkedtheorem}[1]{%
  \newenvironment{#1}
    {\csname inner@#1\endcsname}{
    \vspace{5pt}
	\hrule
	\vspace{6pt}
	}%
  \newtheorem{inner@#1}%
} % WHAT DOES THIS DO??


\newmarkedtheorem{exercise}[theorem]{Exercise}
\newmarkedtheorem{exr}[theorem]{Exercise}

\newenvironment{defnolab}[1][Definition]{\begin{trivlist}
\item[\hskip \labelsep {\bfseries #1}]}{\end{trivlist}} % WHAT DOES THIS DO??

\newenvironment{lproof}{\begin{proof} \renewcommand{\qedsymbol}{}}{\end{proof}} % WHAT DOES THIS DO??
