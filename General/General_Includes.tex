% Typesetting Packages

\usepackage[utf8]{inputenc}
\usepackage[english]{babel}
\usepackage[a4paper, total={7in, 9.5in}]{geometry}

% Math Packages
 
\usepackage{amsthm, amssymb, amsmath, centernot, graphicx}

% Fonts

\usepackage{calligra, mathrsfs}

% Graphics Packages

\usepackage{tikz-cd}
\usepackage{graphicx}

% General Utilities 

\renewcommand{\bf}[1]{\mathbf{#1}}
\newcommand{\notimplies}{%
  \mathrel{{\ooalign{\hidewidth$\not\phantom{=}$\hidewidth\cr$\implies$}}}}
\renewcommand{\theenumi}{(\alph{enumi})}

% Set-Theoretic Commands

\newcommand{\sm}{\! \setminus \!}
\renewcommand{\empty}{\varnothing}
\newcommand{\embed}{\hookrightarrow}
\newcommand{\hook}{\hookrightarrow}
\newcommand{\onto}{\twoheadrightarrow}

% Commands on Functions

\newcommand{\id}{\mathrm{id}}
\DeclareMathOperator{\coker}{\mathrm{coker}}
\renewcommand{\Im}[1]{\mathrm{Im}\left( #1 \right)}
\DeclareMathOperator{\im}{\mathrm{Im}}
\DeclareMathOperator{\coim}{\mathrm{coim}}
\newcommand{\invI}[2]{#1^{-1} \left( #2 \right)}
\newcommand{\ev}{\mathrm{ev}}

% Category-Theoretic Commands

\newcommand{\Hom}[3]{\text{Hom}_{#1}\left( #2, #3 \right)}
\newcommand{\Homover}[3]{\text{Hom}_{#1}\left( #2, #3 \right)}
\newcommand{\End}[1]{\text{End}\left( #1 \right)}
\newcommand{\Aut}[1]{\text{Aut}\left( #1 \right)}
\newcommand{\op}{\mathrm{op}}

% Commands on Groups and Actions

\newcommand{\ab}{\mathrm{ab}}
\newcommand{\ord}{\mathrm{ord}}
\newcommand{\fix}[2]{\mathrm{Fix}_{#1} (#2)}

\usepackage{bbm}
\newcommand{\acts}{ \;  \rotatebox[origin=c]{-90}{$\circlearrowright$} \;  }

\newcommand{\orb}[1]{\mathrm{Orb}(#1)}
\newcommand{\stab}[1]{\mathrm{Stab}(#1)}

\newcommand{\Orb}[1]{\mathrm{Orb}(#1)}
\newcommand{\Stab}[1]{\mathrm{Stab}(#1)}

\newcommand{\galgroup}[1]{\mathrm{Gal}\left( #1 \right)}
\newcommand{\Gal}[1]{\mathrm{Gal}\left( #1 \right)}
\newcommand{\gal}[1]{\mathrm{Gal}\left( #1 \right)}

% Linear Algebra Commands

\newcommand{\rank}{\mathrm{rank}}
\newcommand{\vspan}[1]{\mathrm{span}\! \left\{#1 \right\}}
\newcommand{\Tr}[1]{\mathrm{Tr}\left( #1 \right)}
\newcommand{\tr}[1]{\mathrm{tr}\left( #1 \right)}
\newcommand{\exterior}{\bigwedge\nolimits}

% Topology Commands

\newcommand{\ball}[2]{B_{#1} \! \left(#2 \right)}
\newcommand{\topo}{\mathcal{T}}
\newcommand{\base}{\mathcal{B}}

\newcommand{\RP}{\mathbb{RP}}
\newcommand{\rp}{\mathbb{RP}}
\newcommand{\CP}{\mathbb{CP}}
\newcommand{\cp}{\mathbb{CP}}

% Number Theory Commands

\newcommand{\Z}{\mathbb{Z}}
\newcommand{\N}{\mathbb{N}}
\newcommand{\Zplus}{\mathbb{Z}^{+}}
\newcommand{\Primes}{\mathbb{P}}
\newcommand{\Q}{\mathbb{Q}}
\newcommand{\divides}{\mid}
\newcommand{\ndivides}{\centernot \mid}
\newcommand{\modeq}[3]{#1 \equiv #2 \quad (\mathrm{mod} \: {#3})}

% Homological Algebraic Commands

\newcommand{\Tor}[4]{\mathrm{Tor}^{#1}_{#2} \left( #3, #4 \right)}
\newcommand{\Ext}[4]{\mathrm{Ext}^{#1}_{#2} \left( #3, #4 \right)}

\DeclareMathOperator{\colim}{\mathrm{colim}}

% Calculus Commands

\renewcommand{\d}{\mathrm{d}}
\newcommand{\dn}[2]{ \mathrm{d}^{#1} #2 \:}
\newcommand{\deriv}[2]{\frac{\d{#1}}{\d{#2}}}
\newcommand{\pderiv}[2]{\frac{\partial{#1}}{\partial{#2}}}
\newcommand{\parsq}[2]{\frac{\partial^2{#1}}{\partial{#2}^2}}

% Environments and Proofs

\renewcommand{\qedsymbol}{$\square$}
\newcommand{\cont}{$\boxtimes$}
 
\theoremstyle{remark}
\newtheorem*{remark}{Remark}
\newtheorem*{rmk}{Remark}

\theoremstyle{definition}
\newtheorem{theorem}{Theorem}[section]
\newtheorem{thm}{Theorem}[section]
\newtheorem{lemma}[theorem]{Lemma}
\newtheorem{lm}[theorem]{Lemma}
\newtheorem{proposition}[theorem]{Proposition}
\newtheorem{prop}[theorem]{Proposition}
\newtheorem{corollary}[theorem]{Corollary}
\newtheorem{cor}[theorem]{Corollary}
\newtheorem{example}[theorem]{Example}
\newtheorem{ex}[theorem]{Example}
\newtheorem{definition}[theorem]{Definition}
\newtheorem{defn}[theorem]{Definition}

\newcommand{\newmarkedtheorem}[1]{%
  \newenvironment{#1}
    {\csname inner@#1\endcsname}{
    \vspace{5pt}
	\hrule
	\vspace{6pt}
	}%
  \newtheorem{inner@#1}%
}

\newmarkedtheorem{exercise}[theorem]{Exercise}
\newmarkedtheorem{exr}[theorem]{Exercise}

\newenvironment{defnolab}[1][Definition]{\begin{trivlist}
\item[\hskip \labelsep {\bfseries #1}]}{\end{trivlist}}



\newenvironment{lproof}{\begin{proof} \renewcommand{\qedsymbol}{}}{\end{proof}}
