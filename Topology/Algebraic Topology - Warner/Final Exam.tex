\documentclass[12pt]{extarticle}
\usepackage{import}
\import{./}{Includes}

\usetikzlibrary{decorations.markings}

\begin{document}
{\title{% 
	\large \textbf{Mathematics GU4053 Algebraic Topology
	\\ Final Exam} \vspace{-2ex}}
\author{Benjamin Church }
\maketitle}

Note. My order of path concatenation follows the convention used in class rather than Hatcher,
\[\gamma * \delta(x) = \begin{cases}
\delta(2x) & x \le \tfrac{1}{2} \\
\gamma(2x - 1) & x \ge \tfrac{1}{2}
\end{cases}\]
I will use $\cong$ to denote homeomorphism and $\simeq$ to denote homotopy equivalence of spaces and maps.
 
\section*{Problem 1.}

Since the space $X$ is a CW complex with two $2$-cells (called $A$ and $B$), one $1$-cell (called $a$) and one $0$-cell (called $\alpha$), its cellular homology complex is,
\begin{center}
\begin{tikzcd}
0 \arrow[r] & A \Z \oplus B \Z \arrow[r, "\partial_2"] & a \Z \arrow[r, "\partial_1"] & \alpha \Z \arrow[r] & 0
\end{tikzcd}
\end{center}
To calculate the homology of this complex, we need to know the boundary maps. First, since $X$ is formed by first gluing $a$ to $\alpha$ to form a copy of $S^1$ both endpoints of $a$ are $\alpha$ so $\partial_1(a) = 0$ and thus $\partial_1 = 0$. Next, we need to compute the second boundary map. Using the cellular boundary formula,
\[ \partial_n(e_{\alpha}^n) = \sum_{\beta} d_{\alpha \beta} e_{\beta}^{n-1} \]
where $e_{\alpha}^n$ enumerates the $n$-cells and thus the generators of the cellular homology complex and $d_{\alpha \beta}$ is the degree of the map $S_{\alpha}^{n-1} \to X^{n-1} \to S_{\beta}^{n-1}$ given by attaching and then collapsing $X^{n-1} \sm e_{\beta}^{n-1}$ to a point. In this case, these variables run over $e_{\alpha}^2 = A, B$ and $e_{\beta}^1 = a$. Furthermore, we are given that the attaching map of $A$ has degree $p$ and the attaching map of $B$ has degree $q$ viewed as a map from the boundary $\partial D^2$ to $X^1 \cong S^1$. Therefore,
\[ \partial_2 (A) = p a \quad \text{and} \quad \partial_2(B) = q a \]
Calculating the homology requires knowing the kernel and image of $\partial_2$,
\[ \ker{\partial_2} = \{ (x,y) \in \Z^2 \mid px + qy = 0 \} = \{ (z \tilde{q}, -z\tilde{p}) \mid z \in \Z \} \]
where letting $g = \gcd{(p,q)}$, set $\tilde{p} = \frac{p}{g}$ and $\tilde{q} = \frac{q}{g}$. Next, by Bezout's identity,
\[ \Im{\partial_2} = \{ px + qy \mid (x,y) \in \Z^2 \} = \{ zg \mid z \in \Z \} \] 
Finally, the homology becomes,
\begin{align*}
H_0(X) &= \ker{\partial_0} / \Im{\partial_1} = \ker{\partial_0} \cong \Z
\\
H_1(X) & = \ker{\partial_1} / \Im{\partial_2} \cong \Z / g \Z
\\
H_2(X) & = \ker{\partial_2} / \Im{\partial_3} =  \{ (z \tilde{q}, -z\tilde{p}) \mid z \in \Z \}  \cong \Z 
\end{align*}   
Since there are no $n$-cells with $n > 2$ all higher homology vanishes. Since homology is a homotopy invariant, if $X \cong S^2$ then we know that $H_1$ must vanish since $H_1(S^2) = 0$. This occurs exactly when $g = \gcd{(p,q)} = 1$ and thus exactly when $p$ and $q$ are coprime. I claim this is also a sufficient condition for $X \simeq S^2$ by a homotopy equivalence.
\bigskip\\
Following Hatcher, I use the notation $X_0 \sqcup_{f} X_1$ to mean the space $X_0$ with $X_1$ attached via the map $f$. We know \footnote{Hatcher Chapter 0, page 16.} that if $f \simeq g$ as maps $A \to X_0$ where $(X_1, A)$ is a CW pair then $X_0 \sqcup_f X_1 \simeq X_0 \sqcup_g X_1$ rel $X_0$. Consider the spaces $X_0 = S^1 \sqcup_{f_p} D^2$ where $f_p$ is the order $p$ attaching map and the CW pair $(D^2, S^1)$ with maps $f_r : S^1 \to X_0$ being an order $r$ attaching map. I claim that $\pi_1(X_0) \cong \Z / p \Z$. This is easily seen because $X_0$ has a single $1$-cell copy of $S^1$ so $\pi_1(X_0)$ has one generator and $\pi_1(X_0)$ has one relation given by contracting the boundary of the $D^2$ which is glued by the order $p$ map. This relation gives, $[\gamma]^p = e$ so $\pi_1(X_0) = \left< g \mid g^p = e \right> \cong \Z / p \Z$.  Therefore, given two attaching maps $f_{r}, f_{r'} : S^1 \to X_0$ we know that $f_{r} \simeq f_{r'}$ if $\mod{r}{r'}{p}$ since the order one map generates $\pi_1(X_0)$ and thus $[f_r] = [f_1]^r$ implying that $[f_r] = [f_{r'}]$ if $\mod{r}{r'}{p}$ since the fundamental group has order $p$. Let $r$ be the reduced residue of $q$ modulo $p$ i.e., via the division algorithm, write $q = pq' + r$ where $0 \le r < p$. Then, $f_{q} \simeq f_{r}$ and thus,
\[ (S^1 \sqcup_{f_p} D^2) \sqcup_{f_{q}} D^2 \simeq (S^1 \sqcup_{f_p} D^2) \sqcup_{f_{r}} D^2 \] 
However, since the image of both attaching maps is the $S^1$ which is left invariant under the quotienting and homotoping, we can shift the order of attaching,
\[ (S^1 \sqcup_{f_p} D^2) \sqcup_{f_{r}} D^2  \simeq (S^1 \sqcup_{f_r} D^2) \sqcup_{f_{p}} D^2 \]
Therefore, in total we have reduced,
\[ (S^1 \sqcup_{f_p} D^2) \sqcup_{f_{q}} D^2 \simeq (S^1 \sqcup_{f_r} D^2) \sqcup_{f_{p}} D^2  \] 
I will now introduce the notation for this double attached space,
\[ [p, q] = (S^1 \sqcup_{f_p} D^2) \sqcup_{f_{q}} D^2 \]
Thus, I have proven that $[p, q] \simeq [r, p]$ where $q = pq' + r$ and $0 \le r < p$. Therefore, I can reduce the pair of integers defining the space $[p,q]$ via the Euclidean algorithm. Furthermore, if I start with $\gcd{(p,q)} = 1$ then we know that the Euclidean algorithm (allowing swaps or reductions) can reduce the pair to $[1, 1]$.  However, $[1, 1] \simeq S^2$ where $[1,1]$ is a standard complex structure for $S^2$ given by taking an $S^1$ and gluing two disks $D^2$ with degree $1$ attaching maps to the copy of $S^1$. Therefore, via the Euclidean algorithm, $[p, q] \simeq [1, 1] \simeq S^2$ if $\gcd{(p,q)} = 1$. In summary, 
\[ [p,q] \cong S^2 \iff \gcd{(p,q)} = 1 \]

\section*{Problem 2.}
Let $f : S^n \to S^n$ be a continuous map. Suppose that the map $f : S^n \to S^n$ has no fixed points. Then, the map $(-f) : S^n \to S^n$ satisfies the hypothesis of Lemma \ref{antifixed} because, 
\[f(x) \neq x \iff (-f)(x) \neq -x\]
and thus $\deg{(-f)} = 1$
However, $\deg{(-f)} = \deg{(-\mathbf{1})} \cdot \deg{f} = (-1)^{n+1} \deg{f}$ where $-\mathbf{1}$ is the antipodal map. Therefore, $\deg{f} = (-1)^{n+1}$. Taking the contrapositive, if $\deg{f} \neq (-1)^{n+1}$ then $f$ must have a fixed point. 

\section*{Problem 3.}

Let $F : C \to C'$ and $G: C' \to C$ be equivalences of categories with associated natural isomorphisms $\epsilon : F \circ G \implies \id_{C'}$ and $\eta : G \circ F \implies \id_{C}$. Suppose that $C$ has all limits. Let $\alpha : D \to C$ be a diagram in $C$ where $D$ is a small category. Since $C$ has all limits, $\lim{\alpha} \in C$ so we have universal cones,
\begin{center}
\begin{tikzcd}[column sep = huge, row sep = huge]
& A \arrow[ddl, bend right] \arrow[ddr, bend left] \arrow[d, dashed, "\exists ! f"] & \\
& \lim{\alpha} \arrow[dl, swap, "\pi(d_i)"] \arrow[dr, "\pi(d_j)"] & \\
\alpha(d_i) \arrow[rr] & & \alpha(d_j) 
\end{tikzcd}
\end{center}   
with projection maps $\pi(d) : \lim{\alpha} \to \alpha(d)$ for each $d \in \mathrm{Ob}(D)$ such that any cone factors uniquely through the universal cone. In particular, we will take $A = G(B)$ where $B$ is some object in $C'$. Consider the cone in $C'$,
\begin{center}
\begin{tikzcd}[ row sep = huge]
F \circ \alpha(d_i) \arrow[rr] & & F \circ \alpha(d_j)
\\
& B \arrow[ul] \arrow[ur]
\end{tikzcd}
\end{center}
and apply the functor $G$,
\begin{center}
\begin{tikzcd}[column sep = huge, row sep = huge]
G \circ F \circ \alpha(d_i) \arrow[ddd, "\eta_{\alpha(d_i)}"'] \arrow[rr] & & G \circ F \circ \alpha(d_j) \arrow[ddd, "\eta_{\alpha(d_j)}"]
\\
& G(B) \arrow[ru] \arrow[lu] \arrow[ddr, bend left] \arrow[ddl, bend right] \arrow[d, dashed, "\exists ! f"] & \\
& \lim{\alpha} \arrow[dl, swap, "\pi(d_i)"] \arrow[dr, "\pi(d_j)"] & 
\\
\alpha(d_i) \arrow[rr] & & \alpha(d_j)
\end{tikzcd}
\end{center} 
where the natural transformation $\eta$ gives maps $G \circ F \circ \alpha(d) \to \alpha(d)$ which compose to produce maps $G(B) \to \alpha(d)$ commuting with the map $\alpha(d_i) \to \alpha(d_j)$ by naturality. Finally, the cone at $B(G)$ to $\alpha$ factors through $\lim{\alpha}$. Now, apply the functor $F$, (drawing $\eta^{-1}$ rather than $\eta$ in the diagram in $C$ which exists because $\eta$ is a natural isomorphism),
\begin{center}
\begin{tikzcd}[column sep = huge, row sep = huge]
F \circ \alpha(d_i) \arrow[rr] \arrow[from=dd, "\epsilon_{F \circ \alpha(d_i)}"] & & F \circ \alpha(d_j) \arrow[from=dd, "\epsilon_{F \circ \alpha(d_j)}"']
\\
& B \arrow[ul] \arrow[ur]
\\
F \circ G \circ F \circ \alpha(d_i) \arrow[from=ddd, "F(\eta_{\alpha(d_i)}^{-1})"] \arrow[rr] & & F \circ G \circ F \circ \alpha(d_j) \arrow[from=ddd, "F(\eta_{\alpha(d_j)}^{-1})"']
\\
& F \circ G(B) \arrow[from=uu, "\epsilon^{-1}_{B}" near start, crossing over] \arrow[ru] \arrow[ul] \arrow[ddr, bend left] \arrow[ddl, bend right] \arrow[d, "F(f)"] & \\
& F(\lim{\alpha}) \arrow[dl, swap, "F(\pi(d_i))"] \arrow[dr, "F(\pi(d_j))"] & 
\\
F \circ \alpha(d_i) \arrow[rr] & & F \circ \alpha(d_j)
\end{tikzcd}
\end{center} 
The top three squares (parallelograms) commute by naturality of $\epsilon$.
Chasing this diagram, we see that the cone at $B$ to $F \circ \alpha(d)$ factors through $F(\lim{\alpha})$ via the map $F(f) \circ \epsilon_B^{-1}$ since each square commutes. The projections are given by $\epsilon_{F \circ \alpha(d)} \circ F(\eta_{\alpha(d)}^{-1}) \circ F(\pi(d))$. We have shown existence of a map $B \to F(\lim{\alpha})$ but to prove that $F(\lim{\alpha})$ is actually a limit, we need to show that this map is unique. Given two maps $f, g : B \to F(\lim{\alpha})$ which factor the cone through the universal cone consider the commutative diagram in $C'$,
\begin{center}
\begin{tikzcd}[column sep = huge, row sep = huge]
& B  \arrow[ddr, bend left] \arrow[ddl, bend right] \arrow[d, "f", bend left] \arrow[d, "g"', bend right] & \\
& F(\lim{\alpha}) \arrow[dl, swap, "F(\pi(d_i))"] \arrow[dr, "F(\pi(d_j))"] & 
\\
F \circ \alpha(d_i) \arrow[rr] & & G \circ F \circ \alpha(d_j)
\end{tikzcd}
\end{center}  
and then apply the functor $G$ where the maps $G(B) \to \alpha(d)$ are given by composing the image of the maps $B \to F \circ \alpha(d)$ under $G$ with the natural transformation $\eta$.
\begin{center}
\begin{tikzcd}[column sep = huge, row sep = huge]
\alpha(d_i) \arrow[ddd, "\eta_{\alpha(d_i)}^{-1}"'] \arrow[rr] & & \alpha(d_j) \arrow[ddd, "\eta_{\alpha(d_j)}^{-1}"]
\\
& G(B) \arrow[ru] \arrow[lu] \arrow[ddr, bend left] \arrow[ddl, bend right] \arrow[d, "G(f)", bend left] \arrow[d, "G(g)"', bend right] & \\
& G \circ F(\lim{\alpha}) \arrow[dl, swap, "\pi(d_i)"] \arrow[dr, "\pi(d_j)"] & 
\\
G \circ F \circ \alpha(d_i) \arrow[rr] & & G \circ F \circ \alpha(d_j)
\\
& \lim{\alpha} \arrow[rd, "\pi(d_i)"] \arrow[dl, "\pi(d_j)"] \arrow[from=uu, "\eta_{\lim{\alpha}}" near end, crossing over]
\\
\alpha(d_i) \arrow[from=uu, "\eta_{\alpha(d_i)}"'] \arrow[rr] & & \alpha(d_j) \arrow[from=uu, "\eta_{\alpha(d_j)}"]
\end{tikzcd}
\end{center}  
However, there exists a unique map factoring through $\lim{\alpha}$ such that the cones at $G(B)$ commute with the projection maps. Therefore, $\eta_{\lim{\alpha}} \circ G(f) = \eta_{\lim{\alpha}} \circ G(g)$ which because $\eta$ is an isomorphism implies that $G(f) = G(g)$ by left-multiplying with $\eta_{\lim{\alpha}}^{-1}$. Finally, consider the naturality diagram in $C'$,
\begin{center}
\begin{tikzcd}[column sep = huge, row sep = huge]
F \circ G(B) \arrow[r, "F \circ G(f)", bend left] \arrow[r, "F \circ G(g)", bend right] \arrow[d, "\epsilon_{B}"] & F \circ G \circ F(\lim{\alpha}) \arrow[d, "\epsilon_{F(\lim{\alpha})}"]
\\
B \arrow[r, "f", bend right] \arrow[r, "g", bend left] & F(\lim{\alpha})
\end{tikzcd}
\end{center}
Therefore, $f \circ \epsilon_B = \epsilon_{F(\lim{\alpha})} \circ (F \circ G)(f)$ and $g \circ \epsilon_B = \epsilon_{F(\lim{\alpha})} \circ (F \circ G)(g)$ but $G(f) = G(g)$ so $f \circ \epsilon_B  = g \circ \epsilon_B$ and since $\epsilon$ is a natural isomorphism $f = g$ by right-multiplying with $\epsilon_B^{-1}$. Therefore, for any cone at $B$ there is a unique map $f : B \to F(\lim{\alpha})$ which makes the diagram,
\begin{center}
\begin{tikzcd}[column sep = huge, row sep = huge]
& B \arrow[ddl, bend right] \arrow[ddr, bend left] \arrow[d, dashed, "\exists ! f"] & \\
& F(\lim{\alpha}) \arrow[dl, swap, "F(\pi(d_i))"] \arrow[dr, "F(\pi(d_j))"] & \\
F \circ \alpha(d_i) \arrow[rr] & & F \circ \alpha(d_j) 
\end{tikzcd}
\end{center}   
commute. Therefore, $F(\lim{\alpha})$ with the maps $F(\pi(d))$ forms a universal cone and is the limit of the diagram $F \circ \alpha$. However, limits are unique to to unique isomorphism so $F(\lim{\alpha}) \cong \lim{(F \circ \alpha)}$.
\bigskip\\
Now, we need to show that $C'$ has all limits. Let $\beta : D \to C'$ be a diagram where $D$ is a small category. We know that $G \circ \beta$ has a limit in $C$ since $C$ has all limits. By the previous proposition, $\lim{(F \circ G \circ \beta)} \cong F(\lim{G \circ \beta})$. In particular, $\lim{(F \circ G \circ \beta)}$ exists in $C'$. Let $A$ form a cone over $\beta$ then consider the naturality diagram, 
\begin{center}
\begin{tikzcd}
& & & A  \arrow[dddr, bend left] \arrow[ddll, dashed, bend right = 15, swap, "\exists ! f"]
\\
& & & \lim{(F \circ G \circ \beta)} \arrow[ddl] \arrow[ddr] & 
\\
& \lim{(F \circ G \circ \beta)} \arrow[urr, crossing over, "\id"] \arrow[ddl, swap, "F \circ G(\pi(d_i))"']  & 
\\
& & \beta(d_i) \arrow[from=uuur, bend right] \arrow[rr] & & \beta(d_j) 
\\
F \circ G \circ \beta(d_i) \arrow[from=urr, "\epsilon_{\beta(d_i)}^{-1}"] \arrow[rr] & & F \circ G \circ \beta(d_j) \arrow[from=urr, "\epsilon_{\beta(d_i)}^{-1}"] \arrow[from=uul, "F \circ G(\pi(d_j))"', crossing over]
\end{tikzcd}
\end{center}
Since $A$ forms a cone to $F \circ G \circ \beta$ under the composition of its maps to $\beta$ and the natural transformation $\eta$, this map factors uniquely through $\lim{(F \circ G \circ \beta)}$. But since $\epsilon$ is an isomorphism the map $A \to \lim{(F \circ G \circ \beta)}$ commutes with the cone from $A$ to $\beta$ and the cone from $\lim{(F \circ G \circ \beta)}$ to $\beta$. Furthermore, any map $A \to \lim{(F \circ G \circ \beta)}$ which satisfies this property must also commute with the cone from $\lim{(F \circ G \circ \beta)}$ to $F \circ G \circ \beta(d)$ since the bottom square commutes and the forward maps are isomorphisms. Thus, by the limit property of $\lim{(F \circ G \circ \beta)}$ such $f$ is unique. Thus, $\lim{(F \circ G \circ \beta)}$ is a limit of the diagram $\beta$ and because all limits are unique up to unique isomorphism we have $\lim{\beta} \cong \lim{(F \circ G \circ \beta)}$. Therefore, we have proven that the limit of any diagram $\beta : D \to C'$ exists in $C'$ and moreover that $\lim{\beta} \cong \lim{(F \circ G \circ \beta}) \cong F(\lim{G \circ \beta})$ so $F$ is surjective on limits.   



\section*{Problem 4.}

Let $X = S^1 \times S^1 \times S^1$ with the product CW structure. Take the map $f : S^3 \to S^2$ which is the Hopf fibration and the $g : X \to S^3$ given by shrinking the $2$-skeleton of $X$ to a point. Consider the composition $f \circ g : X \to S^2$. By functoriality, the induced map $(f \circ g)_* = f_* \circ g_*$ on both homotopy and homology groups. Therefore, the induced map factors as,
\begin{center}
\begin{tikzcd}
\pi_n(X) \arrow[r] & \pi_n(S^3) \arrow[r] & \pi_n(S^2)
\end{tikzcd}
\end{center}
However, 
\[ \pi_n(X) = \pi_n(S^1 \times S^1 \times S^1) \cong \pi_n(S^1) \times \pi_n(S^1) \times \pi_n(S^1) = 
\begin{cases}
\Z^3 & n = 1
\\
0 & n > 1
\end{cases}\]
Therefore, for $n > 1$ we have $(f \circ g)_* = 0$ because $\pi_n(X) = 0$. For $n = 1$ we know $\pi_1(S^3) = 0$ since $\pi_k(S^m) = 0$ for $k < m$ and thus $(f \circ g)_*$ factors through zero. Thus, $(f \circ g)_* = 0$ on all homotopy groups. 
\bigskip\\
Likewise on homology groups the induced map $(f \circ g)_*$ factors as,
\begin{center}
\begin{tikzcd}
\tilde{H}_n(X) \arrow[r] & \tilde{H}_n(S^3) \arrow[r] & \tilde{H}_n(S^2)
\end{tikzcd}
\end{center}
However, $\tilde{H}_n(S^m) = \Z$ if $n = m$ and zero otherwise. Thus, for each $n$ either $\tilde{H}_n(S^3) = 0$ or $\tilde{H}_n(S^2) = 0$. Therefore, the map $(f \circ g)_*$ either has zero codomain or factors through zero. Either way, $(f \circ g)_* = 0$ on all homology.
\bigskip\\
Consider the homology of $X$. Since all $H_n(S^1)$ are free abelian, $\Tor{\Z}{1}{H_{p}(S^1)}{H_q(S^1)} = 0$ for all $p$ and $q$. Thus, by the K\"{u}nneth theorem there is an exact sequence,
\begin{center}
\begin{tikzcd}
0 \arrow[r] & \bigoplus\limits_{p + q = n} H_p(S^1) \otimes_{\Z} H_q(S^1) \arrow[r] & H_n(S^1 \times S^1) \arrow[r] & 0
\end{tikzcd}
\end{center}  
Therefore, using the facts that $\Z \otimes_{\Z} \Z = \Z$ and $H_n(S^1) \cong \Z$ if $n = 0,1$ and zero otherwise,
\[ H_n(S^1 \times S^1) \cong 
\begin{cases}
\Z & n = 2
\\
\Z \oplus \Z & n = 1
\\
\Z & n = 0 
\\
0 & n > 2
\end{cases}
\]
Because at least one of (actually both) groups are free abelian, $\Tor{\Z}{1}{H_p(S^1)}{H_q(S^1 \times S^1)} = 0$ so applying the K\"{u}nneth theorem again we have an exact sequence,
\begin{center}
\begin{tikzcd}
0 \arrow[r] & \bigoplus\limits_{p + q = n} H_p(S^1) \otimes_{\Z} H_q(S^1 \times S^1) \arrow[r] & H_n(S^1 \times S^1 \times S^1) \arrow[r] & 0
\end{tikzcd}
\end{center}  
and thus,
\[ H_n(S^1 \times S^1 \times S^1) \cong \bigoplus\limits_{p + q = n} H_p(S^1) \otimes_{\Z} H_q(S^1 \times S^1) \]
Therefore,
\begin{align*}
H_0(X) & \cong H_0(S^1) \otimes_{\Z} H_0(S^1 \times S^1) = \Z \otimes_{\Z} \Z = \Z
\\
H_1(X) & \cong [H_1(S^1) \otimes_{\Z} H_0(S^1 \times S^1)] \oplus [ H_0(S^1) \otimes_{\Z} H_1(S^1 \times S^1)] \cong \Z \oplus [\Z \oplus \Z ] 
\\
H_2(X) & \cong [H_0(S^1) \otimes_{\Z} H_2(S^1 \times S^1)] \oplus [H_1(S^1) \otimes_{\Z} H_1(S^1 \times S^1)] \cong \Z \oplus [\Z \oplus \Z] 
\\
H_3(X) & \cong H_1(S^1) \otimes_{\Z} H_2(S^1 \times S^1) \cong \Z
\end{align*}
and all higher homology vanishes because $S^1$ is a 1-dimensional CW complex and therefore the product complex, $X = S^1 \times S^1 \times S^1$ is 3-dimensional. Consider the long exact sequence of homology associated to the pair, $(X, X^2)$ where $X^2$ is the two skeleton of $X$,
\begin{center}
\begin{tikzcd}
\cdots \arrow[r] & H_3(X^2) \arrow[r] & H_3(X) \arrow[r] & H_3(X, X^2) \arrow[r] & \cdots
\end{tikzcd}
\end{center}
However, $H_3(X^2) = 0$ and since $(X, X^2)$ is a good pair,
\[H_3(X, X^2) \cong \tilde{H}_3(X/X^2) = H_3(X^3 / X^2) \cong H_3(S^3)\]
because $X = X^3$ so $H_3(X / X^2)$ is the quotient space given by taking the 3-cells and collapsing the 2-skeleton. Furthermore, $X$ has exactly one 3-cell, namely the product $e_1^1 \times e_2^1 \times e_3^1$ of the $1$-cells of the copies of $S^1$. However, $g : X \to S^3$ is defined exactly to be the map collapsing the $2$-skeleton of $X$ to a point. Thus the exact sequence becomes,
\begin{center}
\begin{tikzcd}
0 \arrow[r] & H_3(X) \arrow[r, "g_*"] & \arrow[r] H_3(S^3) \arrow[r] & \cdots
\end{tikzcd}
\end{center}
So $g_* : H_3(X) \to H_3(S^3) \cong \Z$ is injective. 
\bigskip\\
Although $f \circ g$ induces the trivial map on all $\pi_n$ and $\tilde{H}_n$ I claim that $f \circ g$ is not nullhomotopic. Suppose $f \circ g$ is nullhomotopic with a homotopy $H : X \times I \to S^2$ such that $H(x, 0) = f \circ g(x)$ and $H(x, 1) = x_0$. In this case, I have a commutative diagram,
\begin{center}
\begin{tikzcd}
X \arrow[r, "g"] \arrow[d, "\iota_0"] & S^3 \arrow[d, "f"]
\\
X \times I \arrow[ru, dashed, "\tilde{H}"] \arrow[r, "H"] & S^2 
\end{tikzcd}
\end{center}
which commutes because $H(x, 0) = f \circ g(x)$. Thus, because $f : S^3 \to S^2$ is a fibration we get a lift $\tilde{H}$ making the diagram commute. In particular, $\tilde{H}(x,0) = g(x)$ and $f \circ \tilde{H}(x, t) = H(x, t)$ so $f \circ \tilde{H}(t, 1) = H(x, 1) = x_0$ so $\tilde{H}(x, 1) \in f^{-1}(x_0) \cong S^1$ under the Hopf bundle. Thus, $\tilde{H}$ is a homotopy from $g$ to $\tilde{H}(-,1)$. The subset $f^{-1}(x_0) \subset S^3$ is proper because some points of $S^3$ are mapped to points of $S^2$ other than $x_0$ so by Lemma \ref{not-surjective-sphere} we know that $\tilde{H}(-, 1)$ is a nullhomotopic map $X \to S^3$ since it is not surjective (also, $S^3$ is simply connected so any loop $S^1$ in $S^3$ can be contracted to a point). Therefore, because homotopy is an equivalence relation, $g$ is also nullhomotopic which contradicts the fact that $g_* : H_3(X) \to H_3(S^3)$ is injective since we calculated $H_3(X) \cong \Z \neq 0$ but any nullhomotopic map must be trivial on homology. Therefore, the assumption that $f \circ g : X \to S^2$ was nullhomotopic is false.     

\section*{Problem 5.}

Let $\iota : A \to X$ be the inclusion map with $A \subset X$. Suppose that $\iota$ is nullhomotopic. Then, take a homotopy $H : A \times I \to X$ such that $H(a, 0) = a$ and $H(a, 1) = x_0$. Now, define a map $r : C_\iota \to X$ by $r(x) = x$ if $x \in X \subset C_{\iota}$ the image of $\iota$ and $r(a, t) = H(a, 1 - t)$ if $(a, t) \in C_\iota$ is in the cone part. Since $H(a, 1-1) = a = \iota(a)$ by the gluing lemma, $r$ is continuous. Furthermore, $r$ is well-defined because $(a, 1) \sim (a', 1)$ in $C_\iota$ but we also have $r(a,1) = H(a,1) = x_0$ ad $r(a', q) = H(a', 1) = x_0$ so $r$ is constant on glued equivalence classes and thus descends to a continuous map on the quotient space. However, by construction, $r|_{X} = \id_X$ so $r : C_\iota \to X$ is a retract.  
\bigskip\\
Conversely, suppose there exists a retract $r : C_\iota \to X$. Consider the map $H : A \times I \to X$ given by $r(a, 1 - t)$ where $(a, 1 - t) \in C_\iota$ is in the cone part of $C_\iota$. Then, $H(a, 0) = r(a, 1) = r(\iota(a)) = a$ because $(a, 1) \sim \iota(a) = a$ in $C_\iota$ and by definition, $r$ is the identity on $X \subset C_\iota$. Furthermore, $H(a, 1) = r(a, 0)$ but $(a,0) \sim (a', 0)$ inside $C_{\iota}$ by the quotient definition of $C_{\iota}$. Thus, $r(a, 0) = r(a', 0)$ so $H(a, 0) = x_0$ for some fixed point. We know that $x_0 \in X$ because $r$ is a retract. Therefore, $H$ is a homotopy from $\iota$ to a constant map so $\iota$ is nullhomotopic. 
\bigskip\\
Given that $\iota : A \to X$ is null homotopic, consider the long exact sequence of reduced homology associated to the pair $(X,A)$,
\begin{center}
\begin{tikzcd}
\tilde{H}_{n+1}(C_\iota) \arrow[r] & \tilde{H}_{n}(A) \arrow[r, "\iota_*"] & \tilde{H}_{n}(X) \arrow[r] & \tilde{H}_{n}(C_\iota) \arrow[r] & \tilde{H}_{n-1}(A) \arrow[r, "\iota_*"] & H_{n-1}(X) \arrow[r] & \tilde{H}_{n-1}(C_\iota)
\end{tikzcd}
\end{center}  
Since $\iota : A \to X$ is nullhomotopic, $\iota_*$ is the zero map. Therefore, the long exact sequence breaks up into short exact sequences,
\begin{center}
\begin{tikzcd}
0 \arrow[r] & \tilde{H}_{n}(X) \arrow[r, "j_*"] & \tilde{H}_n(C_\iota) \arrow[r] & \tilde{H}_{n-1}(A) \arrow[r] & 0
\end{tikzcd}
\end{center}
However, since $\iota$ is nullhomotopic, we have shown above that $C_\iota$ must retract onto $X$. Therefore, there must exist a map $r : C_f \to X$ such that $r \circ j = \id_{X}$ where $j : X \to C_f$ is the inclusion. Therefore, on homology we get a map $r_* : \tilde{H}_n(C_f) \to \tilde{H}_n(X)$ satisfying $r_* \circ j_* = \id_{\tilde{H}_n(X)}$. Therefore, the exact sequence,
\begin{center}
\begin{tikzcd}
0 \arrow[r] & \tilde{H}_{n}(X) \arrow[r, "j_*"] & \tilde{H}_n(C_\iota) \arrow[l, bend left, "r_*"] \arrow[r] & \tilde{H}_{n-1}(A) \arrow[r] & 0
\end{tikzcd}
\end{center}
splits on the left. Thus,
\[ \tilde{H}_n(C_\iota) \cong \tilde{H}_n(X) \oplus \tilde{H}_{n-1}(A) \]
However, we know that the natural map $M_\iota \to X$ is a deformation retract where $M_\iota$ is the mapping cylinder. Furthermore, $(M_\iota, A \times \{0\})$ is a good pair because the inclusion $A \times \{0\} \hookrightarrow M_\iota$ is a cofibration. Furthermore, $C_f = M_\iota / (X \times \{0\})$ so
\[ \tilde{H}_n(C_f) \cong \tilde{H}_n(M_\iota / (A \times \{0\})) \cong H_n(M_\iota, A \times \{0\}) \cong H_n(X, A) \] 
where the last line follows by the deformation retract of $M_\iota$ to $X$ which takes $A \times \{0\}$ to $A \subset X$ since $\iota(a) = a$.  
Putting everything together,
\[ H_n(X, A) \cong \tilde{H}_{n}(X) \oplus \tilde{H}_{n-1}(A) \]

\section*{Problem 6.}

Let $X$ be path-connected, locally path-connected, and semi-locally simply-connected, and suppose that $G_1 \subset G_2$ are subgroups of $\pi(X, x_0)$. Let $p_1 : X_{G_1} \to X$ and $p_2 : X_{G_2} \to X$ be covering spaces corresponding to $G_1$ and $G_2$. Consider the diagram,
\begin{center}
\begin{tikzcd}[column sep = large, row sep = large]
 & X_{G_2} \arrow[d, "p_2"]
\\ 
X_{G_1} \arrow[ru, dashed, "\tilde{p}_1"] \arrow[r, "p_1"] & X 
\end{tikzcd}
\end{center} 
$X_{G_1}$ is path-connected and locally path-connected by construction. Furthermore, we know that ${p_1}_*(\pi_1(X_{G_1}, \tilde{x}_1)) = G_1$ and ${p_2}_*(\pi_2(X_{G_2}, \tilde{x}_2)) = G_2$. Therefore, ${p_1}_*(\pi_1(X_{G_1}, \tilde{x}_1)) \subset {p_2}_*(\pi_2(X_{G_2}, \tilde{x}_2))$ so by the lifting lemma there is a lift, 
$\tilde{p}_1 : (X_{G_1}, \tilde{x}_1) \to (X_{G_2}, \tilde{x}_2)$ making the diagram above commute. Thus $p_2 \circ \tilde{p}_1 = p_1$ so it suffices to show that $f = \tilde{p}_1$ is a covering map. 
\bigskip\\
Because $p_2 : X_{G_2} \to X$ is a covering map, for each $x \in X$ there exists an open neighborhood $x \in U$ such that $p_2^{-1}(U) = \sqcup_{\alpha} \tilde{U}_{\alpha}$ is a union of disjoint sets each of which is mapped homeomorphically onto $U$ by $p_2$. Likewise, $p_1$ is a covering map so there exists a neighborhood $x \in V$ such that $p_1^{-1}(V) = \sqcup_{\alpha} \tilde{V}_{\alpha}$ and each $\tilde{V}_{\alpha}$ is mapped homeomorphically onto $V$ by $p_1$. We can take $W$ to be the connected component of $V \cap U$ containing $x$ (which is open in $X$ since $X$ is locally connected) and use the fact that $p_1 |_{\tilde{V}} : \tilde{V} \to V$ is a homeomorphism and $p_2 |_{\tilde{U}} : \tilde{U} \to U$ is a homeomorphism to write both $p_1^{-1}(W)$ and $p_2^{-1}(W)$ as disjoint unions of sets homeomorphically mapped onto $W$ by $p_1$ and $p_2$ respectively since the restriction of a homeomorphism is also a homeomorphism onto its image. Since $p_1 = p_2 \circ f$ we know that $f^{-1}(p_2^{-1}(W)) = p_1^{-1}(W)$ and $f$ restricts to a map $p_1^{-1}(W) \to p_2^{-1}(W)$ because,
\[x \in p_1^{-1}(W) \iff p_2 \circ f(x)= p_1(x) \in W \iff f(x) \in p_2^{-1}(W) \]
However, write $p_1^{-1}(W) = \sqcup_{\alpha} \tilde{W}^1_{\alpha}$ then $\tilde{W}_{\alpha}^1 \cong W$ so $\tilde{W}_{\alpha}^1$ is connected. Furthermore, we can write $p_2^{-1}(W) = \sqcup_{\alpha} \tilde{W}^2_{\alpha}$. By continuity, $f(\tilde{W}^1_{\alpha})$ is connected and thus must lie entirely inside some $\tilde{W}_{\beta}^2$ in the disjoint union else its intersections with the sections of the disjoint union would form a disconnection of $f(\tilde{W}_{\alpha}^2)$. Since $\Im{f|_{\tilde{W}^2_{\alpha}}} \subset \tilde{W}^2_{\beta}$ we can write the restriction as, 
\[p_1 |_{\tilde{W}^1_{\alpha}} = (p_2 \circ f)|_{\tilde{W}^1_{\alpha}} = p_2|_{\tilde{W}^2_{\beta}} \circ f|_{\tilde{W}^1_{\alpha}} \] 
However, both $p_1 |_{\tilde{W}^1_{\alpha}}$ and $p_2|_{\tilde{W}^2_{\beta}}$ are homeomorphisms onto $W$ and thus $f|_{\tilde{W}^1_{\alpha}}$ is a homeomorphism onto $\tilde{W}^2_{\beta}$. Furthermore, $f^{-1}(\tilde{W}^2_{\beta}) \subset f^{-1}(p_2^{-1}(W)) = p_1^{-1}(W) = \sqcup_{\alpha} \tilde{W}^1_{\alpha}$ so we can write the preimage as a disjoint union of open sets $\sqcup_{\alpha} f^{-1}(\tilde{W}^2_{\beta}) \cap \tilde{W}^1_{\alpha}$ (since $\tilde{W}_{\beta}^2$ is open and $f$ is continuous we know that $f^{-1}(\tilde{W}_{\beta}^2)$ is open). Furthermore, we showed that $f$ restricted to each section (which is mapped inside $\tilde{W}^2_{\beta}$ by $f$) is a homeomorphism onto $\tilde{W}^2_{\beta}$. Furthermore, every $y \in X_{G_2}$ is contained in some $\tilde{W}^2_{\beta}$ for $x = p_2(y)$ since $x \in W$ and $\tilde{W}^2_{\beta}$ cover the preimage of $W$. By the above connectivity argument, either $f$ does not map into $\tilde{W}^2_{\beta}$ or $f^{-1}(\tilde{W}^2_{\beta})$ is the disjoint union of sets $\tilde{W}^1_{\alpha}$ mapped homeomorphically onto $\tilde{W}^2_{\beta}$ by $f$. Thus, $\tilde{W}^2_{\beta}$ ranging over $\beta$ and $W$ for each $x \in X$ forms an open cover of $X_{G_2}$ each set of which is disjoint from $\Im{f}$ or is evenly covered by $f$. Thus, by Hatcher's definition\footnote{Hatcher does not require a covering map to be surjective.} $f$ is a covering map.  

\section*{Problem 7.}

\newcommand{\SU}[1]{\mathrm{SU}(#1)}
\newcommand{\SO}[1]{\mathrm{SO}(#1)}

I claim that the given space is homeomorphic to $\rp^3$. First, note that this space, $X$, is  homeomorphic to a ball $D^3$ with antipodal points on its boundary identified. This follows simply by ``blowing up'' the space until it becomes a ball while maintaining the identifications of opposite points. Now, $X \cong \SO{3}$ the rotation group via the map which takes a vector $\vec{v} \in D^3 \subset \R^3$ to the rotation about that axis by an angle $\pi |\vec{v}|$. This is bijective because if $\vec{v}$ is a unit vector, that is it lies on the boundary, then $\vec{v} \sim -\vec{v}$ but rotations by $\pi$ about opposite directions are equivalent. This map is clearly continuous and thus is a continuous bijection of compact Hausdorff spaces which implies that it is a homeomorphism. Furthermore, there is a double cover\footnote{One parametrization of this map is via Cayley-Klein parameters. Equivalently, the unit quaternions give a double cover of the rotation group acting via conjugation $\vec{p} \mapsto q \vec{v} q^{-1}$ where $q$ and $-q$ clearly induce the same transformation.} $p : \SU{2} \to \SO{3}$ but $\SU{2} \cong S^3$ and thus, 
\[ X \cong \SO{3} \cong \SU{2} / \{\pm 1\} \cong S^3 / \{ \pm 1 \} \cong \rp^3\]
However, we computed the homology of $\rp^n$ in class using cellular homology. Thus,
\[ H_n(X) \cong H_n(\rp^3) \cong 
\begin{cases}
\Z & n = 0,3 
\\
\Z / 2 \Z & n = 1 
\\
0 & \text{else}
\end{cases}\] 
Furthermore, there is a double cover $p : S^3 \to X$ and $S^3$ is simply connected so we know that $\pi_1(X)$ is isomorphic to the group of Deck transformations of $p$. However, the deck transformations act freely and transitively on the fibers of $p$ which have two elements. Therefore, the fundamental group has order two so $\pi_1(X) \cong \Z / 2 \Z$. This group is generated by a loop traversing a long diagonal between two opposite corners which are identified. 
\bigskip\\
For completeness, I will also compute these groups explicitly. Consider the space $X$,



\begin{center}
\begin{tikzpicture}[scale=3]
\begin{scope}[very thick,decoration={
    markings,
    mark=at position 0.5 with {\arrow{>}}}
    ] 
  \draw[thick,postaction={decorate}](0,2,0)--(2,2,0) node[midway, above] {c};
  \draw[thick,postaction={decorate}](0,2,0)--(0,2,2) node[midway, left] {e};
  \draw[thick,postaction={decorate}](0,2,2)--(2,2,2) node[midway, below] {d};
  \draw[thick,postaction={decorate}](2,2,2)--(2,2,0) node[midway, below right] {f};
  \draw[thick,postaction={decorate}](2,2,0)--(2,0,0) node[midway, right] {a};
  \draw[thick,postaction={decorate}](2,0,2)--(2,0,0) node[midway, below right] {e};
  \draw[thick,postaction={decorate}](2,0,2)--(0,0,2) node[midway, below] {c};
  \draw[thick,postaction={decorate}](0,0,2)--(0,2,2) node[midway, left] {a};
  \draw[thick,postaction={decorate}](2,2,2)-- (2,0,2) node[midway, right] {b};
  \draw[gray,postaction={decorate}](2,0,0)--(0,0,0) node[midway, below] {d};
  \draw[gray,postaction={decorate}](0,0,0)--(0,2,0) node[midway, left] {b};
  \draw[gray,postaction={decorate}](0,0,0)--(0,0,2) node[midway, below right] {f};
\end{scope}

  \draw(0,0,0) node[below]{\delta};
  \draw(0,0,2) node[below left]{\alpha};
  \draw(0,2,0) node[above]{\beta};
  \draw(2,0,0) node[right]{\gamma};
  \draw(2,0,2) node[below]{\beta};
  \draw(2,2,0) node[above right]{\alpha};
  \draw(0,2,2) node[left]{\gamma};
  \draw(2,2,2) node[above]{\delta};

  \draw(1,1,2) node{A};
  \draw(1,2,1) node{B};
  \draw(2,1,1) node{C};
  \draw[gray!20](1,0,1) node{B};
  \draw[gray!20](0,1,1) node{C};
  \draw[gray!20](1,1,0) node{A};
\end{tikzpicture} 
\end{center}
which gives a cellular complex,
\begin{center}
\begin{tikzcd}
0 \arrow[r] & \Z \arrow[r, "\partial_3"] & \Z^3 \arrow[r, "\partial_2"] & \Z^6 \arrow[r, "\partial_1"] & \Z^4 \arrow[r] & 0
\end{tikzcd}
\end{center}  
Let $V$ represent the $3$-cell ``volume''
where because the opposite faces are forced to have opposite orientation,
\[ \partial_3 V = A + B + C - A - B - C = 0 \]
so $\partial_3 = 0$.
Furthermore,
\begin{align*}
\partial_2 A & = a + d + b + c
\\
\partial_2 B & = c - f - d - e
\\
\partial_2 C & = f + a - e - b
\end{align*}
and likewise,
\begin{align*}
\partial_1 a = \gamma - \alpha
\\
\partial_1 b = \beta - \delta
\\
\partial_1 c = \alpha - \beta
\\
\partial_1 d = \delta - \gamma
\\
\partial_1 e = \gamma - \beta
\\
\partial_1 f = \alpha - \delta
\end{align*}
Now, we need to calculate the kernels and images of each of the boundary maps. We see $\ker{\partial_3} = \Z$ and $\Im{\partial_3} = 0$. Furthermore,
\begin{align*}
\Im{\partial_2} = \left< (1, 1, 1, 1, 0, 0), \, (0, 0, 1, -1, -1, -1), \, (1, -1, 0, 0, -1, 1) \right>
\end{align*}
Consider the kernel of $\partial_2$ consisting of elements $(x,y,z) \in \Z^3$ such that,
\begin{align*}
(a + d + b + c) & x + (c - f - d - e) y + (f + a - e - b)z = 0 
\\
& \implies x = -z \text{ and } x = y \text{ and } y = z \implies x = y = z = 0
\end{align*}
Thus, $\ker{\partial_2} = 0$. Next, consider the image,
\begin{align*}
\Im{\partial_1} & = \left< (-1,0,1,0), \, (0,1,0,-1), \, (1,-1,0,0), \, (0,0,-1,1), \, (0,-1,1,0), \, (1,0,0-1) \right> 
\\
& = \{(t_1, t_2, t_3, t_4) \mid \sum t_i = 0 \} \subset \Z^4 
\end{align*}
and finally, the kernel consisting of elements $(x_1, \dots, x_6) \in \Z^6$ such that,
\begin{align*}
(\gamma - \alpha)  x_1 + (\beta - \delta)x_2 + & (\alpha - \beta) x_3 + (\delta - \gamma) x_4 + (\gamma - \beta) x_5 + (\alpha - \delta) x_5 = 0
\\
\implies \quad \quad \quad \quad \quad \quad \quad &
\\
 -x_1 + x_3 + x_6 & = 0
\\
 x_2 - x_3 - x_5 & = 0
\\
 x_1 - x_4 + x_5 & = 0
\\
 -x_2 + x_4 - x_6 & = 0  
\end{align*}
by matrix algebra, the kernel equals,
\begin{align*}
\ker{\partial_1} = \left< (1,1,1,1,0,0), \, (0,-1,-1,0,0,1), \, (1,0,1,0,-1,0) \right> 
\end{align*}
Therefore,
\[ H_3(X) = \ker{\partial_3} = V \Z \cong \Z \]
and similarly,
\[ H_2(X) = \ker{\partial_2} / \Im{\partial_3} = \ker{\partial_2} = 0 \] 
Next, write the basis of $\ker{\partial_1}$ as,
\[ v_1 = (1,1,1,1, 0, 0) \quad v_2 = (0,-1,-1,0,0,1) \quad v_2 = (1,0,1,0,-1,0) \]
Therefore, we can express the basis of $\Im{\partial_2}$ in this basis. Again using some matrix algebra,
\[ (1,1,1,1,0,0) = v_1 \quad (0,0,1,-1,-1,-1) = -v_1 - v_2 + v_3 \quad (1, -1 , 0, 0, -1, 1) = v_2 + v_3 \] 
Therefore, identifying $\ker{\partial_1} = \vspan{v_1, v_2, v_3} \cong \Z^3$ we have,
\begin{align*} 
H_1(X) & = \ker{\partial_1} / \Im{\partial_2} = \Z^3 / \left< (1, 0, 0), \, (-1,-1,1), \, (0,1,1) \right>
\\
& \cong \Z^2 / \left< (-1,1), \, (1,1) \right>
\\
&  \cong \Z / 2 \Z 
\end{align*}
The last step follows by considering the, clearly surjective, homomorphism $p : \Z^2 \to \Z / 2 \Z$ by taking $(x,y) \mapsto [x + y]$ modulo $2$. The kernel of this map is all $(x, y)$ such that $\mod{x}{y}{2}$ which is spanned by $(1,1)$ and $(-1,1)$ since $(1,1) + (-1,1) = (0,2)$ and $(1,1) - (-1,1) = (2,0)$. Therefore, by the first isomorphism theorem, 
\[ \Z^2 / \left< (1, 1), \, (1, -1) \right> \cong \Z / 2 \Z\]
Finally, 
\[ H_0(X) = \ker{\partial_0} / \Im{\partial_1} = \Z^4 / \{(t_1, t_2, t_3, t_3) \mid \sum t_i = 0 \} \cong \Z \] 
via the map $s : \Z^4 \to \Z$ taking $(t_1, t_2, t_3, t_4) \mapsto \sum t_i$ then $\Im{\partial_1} = \ker{s}$ and $\Z^4 / ker{s} \cong \Z$ since $s$ is surjective. Thus, in summary,
\[ H_n(X) \cong H_n(\rp^3) \cong 
\begin{cases}
\Z & n = 0,3 
\\
\Z / 2 \Z & n = 1 
\\
0 & \text{else}
\end{cases}\] 
\bigskip\\
We can calculate $\pi_1(X)$ by van Kampen's theorem. Let $U \subset X$ be an open ball containing the center of the cube and not touching any face. Take $V \subset X$ to be the complement in $X$ of a closed ball inside $U$. Thus, $U \cong D^3$ and $V$ deformation retracts onto the boundary of the cube since it contains the boundary but not the center. However, antipodal points of the boundary are identified so $V$ deformation retracts to a copy of $S^2$ with antipodal points identified and thus $V \simeq \rp^2$. Finally, $U \cap V \cong S^2 \times (0, 1)$ since it equals $U \cong D^3$ with an interior closed 3-disk removed. By a deformation retract, $U \cap V \simeq S^2$. Therefore, $\pi_1(U) \cong \pi_1(D^3) = 1$ and $\pi_1(U \cap V) \cong \pi_1(S^2) = 1$ and $\pi_1(V) \cong \pi_1(\rp^2) = \Z / 2 \Z$ by the double cover $p : S^2 \to \rp^2$. Therefore, by van Kampen's theorem,
\[ \pi_1(X) \cong \pi_1(U) *_{\pi_1(U \cap V)} \pi_1(V) = 1 *_{1} \Z / 2 \Z = \Z / 2 \Z \]      

\section*{Problem 8.}

Let $f : Y \to X$ be a fibration and $\alpha : I \to X$ a path from $x$ to $y$. Since the map $f : X \to Y$ is a fibration, whenever the diagram,
\begin{center}
\begin{tikzcd}[column sep = large, row sep = large]
Z \arrow[r] \arrow[d, hook, "\iota_0"] & Y \arrow[d, "f"] \\
Z \times I \arrow[r] \arrow[ru, dashed, "\exists g"] & X
\end{tikzcd}
\end{center}
commutes there exists a (not necessarily unique) map $g: Z \times I \to X$. Take $Z = f^{-1}(x)$ and the maps, $\iota : Z \hookrightarrow Y$ which is the inclusion and $h : Z \times I \to X$ given by $(z, t) \mapsto \alpha(t)$. Therefore, $h \circ \iota_0(z) = h(z, 0) = \alpha(0) = x$ and $f \circ \iota(z) = f(z) = x$ since $Z = f^{-1}(x)$. Therefore, $h \circ \iota_0 = f \circ \iota$ so the following diagram commutes, 
\begin{center}
\begin{tikzcd}[column sep = large, row sep = large]
Z \arrow[r, "\iota"] \arrow[d, hook, "\iota_0"] & Y \arrow[d, "f"] \\
Z \times I \arrow[r, "h"] \arrow[ru, dashed, "\exists g"] & X
\end{tikzcd}
\end{center}
and therefore by the fibration property we get a map $g : Z \times I \to Y$ which makes the diagram commute. In particular, $f \circ g = h$ and thus $f \circ g(z, t) = h(z, t) = \alpha(t)$ which implies that $g(z,t) \in f^{-1}(\alpha(t))$. Furthermore, $g \circ \iota_0(z) = \iota(z)$ so $g(z, 0) = z$. 
\bigskip\\
Define the map $\alpha_* : f^{-1}(x) \to f^{-1}(y)$ via $\alpha_*(z) = g(z, 1)$.   
\bigskip\\
I claim that if $\alpha, \beta$ are path-homotopic then $\alpha_*$ and $\beta_*$ are homotopic. Let $H : I \times I \to X$ be a path-homotopy from $\alpha$ to $\beta$ i.e. $H(0, s) = \alpha(s)$ and $H(1, s) = \beta(s)$ and $H(t, 0) = x$ and $H(t, 1) = y$. Using the previous construction we get maps $g_\alpha, g_\beta : Z \times I \to Y$ which satisfy $f \circ g_{\alpha}(z, t) = h_{\alpha}(z, t) = \alpha(t)$ and $f \circ g_{\beta}(z, t) = h_{\beta}(z, t) = \beta(t)$. Also, $g(z, 0) = \iota(z) = z$ for either $\alpha$ or $\beta$. Consider the diagram, 
\begin{center}
\begin{tikzcd}[column sep = huge, row sep = huge]
Z \times (I \times \{0\} \cup \partial I \times I) \arrow[d, "\iota"] \arrow[r, "\sim"] \arrow[rr, "r", bend left] &
Z \times I  \arrow[d, "\iota_0"] \arrow[r] & Y \arrow[d, "f"] 
\\
Z \times I \times I \arrow[rr, "\tilde{H}", bend right] \arrow[r, "\sim"] & Z \times I \times I \arrow[r] \arrow[ru, dashed, "\exists G"] & X
\end{tikzcd}
\end{center}
Where $\tilde{H}(z, s, t) = H(s, t)$. I define the map, $r : Z \times (I \times \{0\} \cup \partial I \times I)$ via sections, 
\[r|_{Z \times \{0\} \times I}(x, 0, s) = g_{\alpha}(z, s)\] and similarly,
\[r|_{Z \times \{1\} \times I}(x, 1, s) = g_{\beta}(z, s)\]
and lastly,
\[r|_{Z \times I \times \{0\}}(z, t, 0) = z\]
Now, as a subset of $I^2$ write this map $r(z, s, t)$ where we need to remember that either $s = 0,1$ or $t = 0$ for the map to be defined. Furthermore, $r(z, 0, t) = z = g_{\alpha}(z, 0) = g_{\beta}(z, 0)$ so $r$ is continuous by the gluing lemma. The isomorphisms of the above diagram are given as $I \cong I \times \{0\} \cup \partial I \times I$ by wrapping the interval around three of the four sides of a square. 
\bigskip\\
I claim the previous diagram commutes because on sections,
\begin{align*}
f \circ r(z, 0, s) & = f \circ g_{\alpha}(z, s) = \alpha(s) = H(0, s) = \tilde{H}(z, 0, s)
\\
f \circ r(z, 1, s) &= f \circ g_{\beta}(z, s) = \beta(s) = H(1, s) = \tilde{H}(z, 1, s)
\\
f \circ r(z, t, 0) &= f(z) = x = H(t, 0) = \tilde{H}(z, t, 0)
\end{align*}
since $z \in Z = f^{-1}(x)$. Therefore, the outer square commutes. The inner squares commute as well if we define the unlabeled maps  as compositions of the outer maps and the two isomorphisms. By the fibration property we get a lift $G : Z \times I \times I \to Y$ commuting with the inner square and thus a map $\tilde{G} : Z \times I \times I \to Y$ commuting with the outer map by the isomorphisms. Thus, $\tilde{G}(z, t, 0) = r(z, t, 0) = x$ and $\tilde{G}(z, 0, s) = r(z, 0, s) = g_{\alpha}(z, s)$ and $\tilde{G}(z, 1, s) = g_{\beta}(z, s)$. Furthermore, 
\[f \circ \tilde{G}(z, t, s) = \tilde{H}(z, t, s) = H(t, s)\] 
In particular, define the map $K : f^{-1}(x) \times I \to f^{-1}(y)$ by $K(z, t) = \tilde{G}(z, t, 1)$. Then, $K(z, 0) = \tilde{G}(z, 0, 1) = g_{\alpha}(z, 1) = \alpha_*(z)$ and $K(z, 1) = \tilde{G}(z, 1, 1) = g_{\beta}(z, 1) = \beta_*(z)$. Thus, $K$ is a homotopy between $\alpha_*$ and $\beta_*$. Therefore, the assignment $[\alpha] \mapsto [\alpha_*]$ is well-defined with respect to homotopy.
\bigskip\\
Let $\alpha : I \to X$ be a path from $x$ to $y$ and $\beta : I \to X$ a path from $y$ to $z$ in $X$. Now take the induced map $(\beta * \alpha)_* : f^{-1}(x) \to f^{-1}(z)$ and consider the diagram, 
\begin{center}
\begin{tikzcd}[column sep = large, row sep = large]
Z \arrow[r, "\iota"] \arrow[d, hook, "\iota_0"] & Y \arrow[d, "f"] \\
Z \times I \arrow[r, "h_{\beta * \alpha}"] \arrow[ru, dashed, "\exists g"] & X
\end{tikzcd}
\end{center}
where $h_{\beta * \alpha} (z, t) = (\beta * \alpha)(t)$. As before, $h_{\beta * \alpha}(z, 0) = \alpha(0) = x = f(z)$. I claim that the map,
\[ 
g(z, t) = g_{\beta}(\alpha_*(z), -) * g_{\alpha}(z,-) 
= 
\begin{cases}
g_{\alpha}(z, 2t) & t \le \tfrac{1}{2}
\\
g_{\beta}(\alpha_*(z), 2t - 1) & t \ge \tfrac{1}{2}
\end{cases} \]
is a lift which makes the diagram commute. First, $g$ is continuous by the gluing lemma because $g_{\alpha}(z, 1) = \alpha_*(z)$ and $g_{\beta}(\alpha_*(z), 0) = \alpha_*(z)$. Furthermore, 
\[f \circ g(z, t) = (f \circ g_{\alpha}(z,-)) * (f \circ g_{\beta}(\alpha_*(z), -))(t) = (\alpha * \beta)(t) = h(z, t)\]
and likewise $g(z, 0) = g_{\alpha}(z, 0) = z = \iota(z)$. Since $g$ is a possible lift we know that, up to homotopy, $(\beta * \alpha)_*(z) \sim g(z, 1)$ because we have shown that up to homotopy our definition is invariant to a choice of lift\footnote{We have proven that if $\alpha \sim \beta$ then for any lift of $\alpha$ and $\beta$ that $\alpha_* \sim \beta_*$. In particular, any path is homotopic to itself so any lifts of $\alpha$ give rise to homotopic induced maps.}. However, $g(z, 1) = g_{\beta}(\alpha_*(z), 1) = \beta_*(\alpha_*(z))$. Therefore, 
\[[(\beta * \alpha)_*] = [ \beta_* \circ \alpha_*] = [\beta_*] \circ [\alpha_*]\]
\bigskip\\
Now, let $e_{x} : I \to X$ be the constant path at $x$. Then let $h_{e} : Z \times I \to X$ be the constant map $h_e(z, t) = x$. Then the diagram commutes, 
\begin{center}
\begin{tikzcd}[column sep = large, row sep = large]
Z \arrow[r, "\iota"] \arrow[d, hook, "\iota_0"] & Y \arrow[d, "f"] \\
Z \times I \arrow[r, "h_e"] \arrow[ru, dashed, "\exists g"] & X
\end{tikzcd}
\end{center}
because $f \circ \iota(z) = f(z) = x = h(z, 0)$. Furthermore, the map $\tilde{h}(z, t) = z$ is a lift which makes this diagram commute because $\tilde{h}(z, 0) = z = \iota(z)$ and $f \circ \tilde{h}(z, t) = f(z) = x = h(z, t)$. Therefore, this construction gives an induced map $(e_x)_*(z) = h(z, 1) = z$ and thus $[(e_x)_*(z)] = [\id_Z]$ for any lift since the construction is invariant up to homotopy with respect to the choice of lift. 
\bigskip\\
Finally, define the functor, $L : \Pi(X) \to \mathrm{hTop}$ where $\Pi(X)$ is the fundamental groupoid of $X$ and $\mathrm{hTop}$ is the homotopy category. For any $x \in X$ define $L(x) = f^{-1}(x)$ and for any map (path) $[\alpha] : x \to y$ define $L([\alpha]) = [\alpha_*] : f^{-1}(x) \to f^{-1}(y)$. We have proven that we have maps $\alpha : x \to y$ and $\beta : y \to z$ then $L([\beta] * [\alpha]) = L([\beta * \alpha]) = [\beta_*] \circ [\alpha_*]$ where $*$ is function composition in the category $\Pi(X)$. Furthermore, $L(\id_{x}) = L([e_x]) = [\id_{f^{-1}(x)}] = \id_{L(x)}$. Therefore, $L : \Pi(X) \to \mathrm{hTop}$ is an honest-to-god functor.
\bigskip\\
Now we can prove that if $f : Y \to X$ is a fibration then all the fibers $f^{-1}(x)$ in the same path component are homotopy equivalent. Take $x, y \in X$ in the same path component. Thus, there exists a path $\alpha : I \to X$ from $x$ to $y$. Therefore, there is a `map' $[\alpha] : x \to y$ in $\Pi(X)$. However, $\Pi(X)$ is a groupoid so every map is an isomorphism and thus there is an inverse map $[\beta] : y \to x$ (given by the path $\beta(t) = \alpha(1 - t)$) such that $[\beta] * [\alpha] = \id_{x}$ and $[\alpha] * [\beta] = \id_y$. Therefore, using the functoriality of $L$,
\[ [\beta_* \circ \alpha_*] = [\beta_*] \circ [\alpha_*] = L([\beta]) \circ L([\alpha]) = L([\beta] * [\alpha]) = L(\id_{x}) = \id_{f^{-1}(x)} \]
and likewise, 
\[ [\alpha_* \circ \beta_*] = [\alpha_*] \circ [\beta_*] = L([\alpha]) \circ L([\beta]) = L([\alpha] * [\beta]) = L(\id_{y}) = \id_{f^{-1}(y)} \]
Therefore, $\beta_* \circ \alpha_* \sim \id_{f^{-1}(x)}$ and $\alpha_* \circ \beta_* = \id_{f^{-1}(y)}$ which implies that the map $\alpha_* : f^{-1}(x) \to f^{-1}(y)$ is a homotopy equivalence (an isomorphism in the category $\mathrm{hTop}$). Therefore, $f^{-1}(x)$ and $f^{-1}(y)$ are homotopy equivalent when $x$ and $y$ lie in the same path-component. 

\newpage

\section*{Problem 9.}

Let $X$ be any space. Suppose that $H_n(X; \Z) = 0$ for all $n > 0$. We know that $H_0(X; \Z) \cong \Z^{\pi_0(X)}$ which is the free abelian group on the connected components. Via the universal coefficient theorem there is an exact sequence,
\begin{center}
\begin{tikzcd}
0 \arrow[r] & H_n(X) \otimes G \arrow[r] & H_n(X ; G) \arrow[r] & \Tor{\Z}{1}{H_{n-1}(X)}{G} \arrow[r] & 0
\end{tikzcd}
\end{center} 
for any abelian group $G$. However, $H_n(X)$ is free because it is either zero or $\Z^{\pi_0(X)}$. Thus, $\Tor{\Z}{1}{H_{n-1}(X)}{G} = 0$ so we have isomorphisms,
\[ H_n(X ; G) \cong H_n(X) \otimes G\]
In particular, for $n > 0$ we know that $H_n(X) = 0$ and thus, $H_n(X ; G) = 0$ for any abelian group. In particular, $H_n(X; \Q) = 0$ and for each prime $H_n(X ; \Z / p \Z) = 0$. 
\bigskip\\
Conversely, 
suppose that $H_n(X ; \Q) = 0$ and for each prime $H_n(X ; \Z / p \Z) = 0$ when $n > 0$. Again, consider the universal coefficient theorem with $G = \Q$,
\begin{center}
\begin{tikzcd}
0 \arrow[r] & H_n(X) \otimes \Q \arrow[r] & H_n(X ; \Q) \arrow[r] & \Tor{\Z}{1}{H_{n-1}(X)}{\Q} \arrow[r] & 0
\end{tikzcd}
\end{center} 
However, $H_n(X ; \Q) = 0$ so we know that $H_n(X) \otimes \Q = 0$ and $\Tor{\Z}{1}{H_n(X)}{\Q} = 0$. Similarly, the universal coefficient theorem applied to $G = \Z / p \Z$ gives, 
\begin{center}
\begin{tikzcd}
0 \arrow[r] & H_n(X) \otimes (\Z / p \Z) \arrow[r] & H_n(X ; \Z / p \Z) \arrow[r] & \Tor{\Z}{1}{H_{n-1}(X)}{\Z / p \Z} \arrow[r] & 0
\end{tikzcd}
\end{center} 
but $H_n(X ; \Z / p \Z) = 0$ by hypothesis so $H_n(X) \otimes (\Z / p \Z) = 0$ and $\Tor{\Z}{1}{H_n(X)}{\Z / p \Z} = 0$ for each prime $p$. The group  $\Tor{\Z}{1}{\Z / p \Z}{M} = \Tor{\Z}{1}{M}{\Z / p \Z} = 0$ is the $p$-torsion of $M$. Therefore, it suffices to show that if an abelian group $M = H_n(X)$ has zero $p$-torsion and $M \otimes_{\Z} \Q = 0$ then $M = 0$. 
\bigskip\\ 
$M$ is torsion-free because if $m \in M$ has order $n$ then for any $p \divides n$ the element $m$ will be in the $p$-torsion subgroup which is empty. However, $M \otimes_{\Z} \Q = 0$ so take any element $m \in M$ then $m \otimes 1 = 0$. Since $m \otimes 1 = n \cdot m \otimes \frac{1}{n} = 0$ for each $n \in \Z \setminus \{0\}$ we must have that $n \cdot m$ or $\frac{1}{n}$ is zero for the tensor product to be zero. Since $\frac{1}{n} \neq 0$, there must be some $n$ such that $n \cdot m = 0$ so $M$ is entirely torsion. However, we have also shown that $M$ is torsion-free so $M = 0$. In particular, we have that $H_n(X) = 0$ for each $n$. 

\newpage

\section*{Problem 10.}

Let $(X, x_0)$ and $(Y, y_0)$ be pointed topological spaces. Consider the pair $(X \times Y, X \vee Y)$ which gives rise to a long exact sequence of homotopy groups,
\begin{center}
\begin{tikzcd}
\pi_{n+1}(X \times Y) \arrow[r] & \pi_{n+1}(X \times Y, X \vee Y) \arrow[r] & \pi_n(X \vee Y) \arrow[r] & \pi_n(X \times Y) \arrow[r] & \pi_n(X \times Y, X \vee Y)
\end{tikzcd}
\end{center}
I claim that the map, $\pi_{n}(X \times Y) \to \pi_{n}(X \times Y, X \vee Y)$ is the zero map for $n \ge 2$. Consider an arbitrary element $[(\gamma, \delta)] \in \pi_n(X \times Y)$ given by based maps $\gamma : S^n \to X$ and $\delta: S^n \to Y$. By Lemma \ref{product_decomp}, we can write $[(\gamma, \delta)] = [(\gamma, y_0)] * [(x_0, \delta)]$. However, the images of the maps $(\gamma, y_0)$ and $(x_0, \delta)$ are entirely contained within $X \vee Y$ so they map to the zero element of the group $\pi_{n}(X \times Y, X \vee Y)$. Therefore, since the connecting maps are homomorphisms\footnote{The object $\pi_n(X, A)$ is only a group when $n \ge 2$ otherwise it is just a pointed set.} for $n \ge 2$, $[(\gamma, \delta)]$ maps to zero proving my claim. Therefore, for $n \ge 2$, the long exact sequence breaks up into short exact sequences,
\begin{center}
\begin{tikzcd}
0 \arrow[r] & \pi_{n+1}(X \times Y, X \vee Y) \arrow[r] & \pi_n(X \vee Y) \arrow[r, "\iota_*"] & \pi_n(X \times Y) \arrow[r] & 0
\end{tikzcd}
\end{center}
Again using Lemma \ref{product_decomp}, we can define a map $j : \pi_n(X \times Y) \to \pi_n(X \vee Y)$. Take the element $[(\gamma, \delta)] \in \pi_n(X \times Y)$ and define $j([(\gamma, \delta)]) = [\gamma] * [\delta]$ via the composition of maps,
\begin{center}
\begin{tikzcd}
\pi_n(X \times Y) \arrow[r, "\sim"] & \pi_n(X) \times \pi_n(Y) \arrow[r] & \pi_n(X \vee Y)
\end{tikzcd}
\end{center}
which is clearly well-defined because if $([\gamma], [\delta]) = ([\gamma'], [\delta'])$ then $\gamma \sim \gamma'$ in $X$ and $\delta \sim \delta'$ in $Y$ and thus $\gamma * \delta \sim \gamma' * \delta'$ in $X \vee Y$. Furthermore, take any class $[\Gamma ] \in \pi_n(X \times Y)$. since this group is isomorphic to $\pi_n(X) \times \pi_n(Y)$ we know that $\Gamma$ splits up into loops in the coordinates $X$ and $Y$ such that $[\Gamma] = [(\gamma, \delta)]$. Thus,
\[ \iota_* \circ j ([\gamma], [\delta]) = \iota_*([\gamma] * [\delta])  \]
However, $\iota$ is simply the inclusion map of $X \vee Y \to X \times Y$ applied to loops which includes $X \hook X \times \{y_0\}$ and $Y \hook \{x_0\} \times Y$. Thus, by Lemma \ref{product_decomp},
\[ \iota_* \circ j([\Gamma]) = \iota_* \circ j ([\gamma], [\delta]) = \iota_*([\gamma] * [\delta]) = [(\gamma, \delta)] = [\Gamma] \]
Therefore, $\iota_* \circ j = \id_{\pi_n(X \times Y)}$. Therefore, the short exact sequence is right split so,
\[ \pi_n(X \vee Y) \cong \pi_n(X \times Y) \oplus \pi_n(X \times Y, X \vee Y) \cong \pi_n(X) \oplus \pi_n(Y) \oplus \pi_n(X \times Y, X \vee Y) \]
where I have used the fact that $\pi_n(X \times Y) \cong \pi_n(X) \oplus \pi_n(Y)$ and the direct sum is justified since both groups are abelian when $n \ge 2$.  

\section{Lemmata}

\begin{lemma} \label{antifixed}
Let $f : S^n \to S^n$ have no point at which $f(x) = - x$ then $\deg{f} = 1$. 
\end{lemma}


\begin{proof}
Suppose $\forall x : f(x) \neq -x$. Thus, the line $f(x)$ to $x$ does not pass though the origin. Therefore, the map $H(x, t) = [(1 - t) f(x) + t x] / | (1 - t) f(x) + t x |$ is a homotopy between $f$ and the identity since the denominator is never zero. Therefore, $\deg{f} = \deg{\mathbf{1}} = 1$. (Hatcher: Basic properties of degree, p.134)
\end{proof}

\begin{lemma} \label{not-surjective-sphere}
Let $f : X \to S^n$ be not surjective. Then $f$ is nullhomotopic.
\end{lemma}

\begin{proof}
Suppose that there exists some $s_0 \in S^n$ not in the image of $f$. Therefore, $f$ factors through the superset of its image, $S^n \sm \{s_0\} \cong D^n$,
\begin{center}
\begin{tikzcd}
X \arrow[r, "f'"] & D^n \arrow[r, "\iota"] & S^n
\end{tikzcd}
\end{center}
However, $D^n$ is contractible so $f'$ is nullhomotopic and thus $f = \iota \circ f'$ is nullhomotopic. 
\end{proof}

\begin{lemma} \label{product_decomp}
Let $\gamma : S^n \to X$ and $\delta : S^n \to Y$ be based maps. Then, in $\pi_n(X \times Y)$ for any $n > 0$ we have $[(\gamma, y_0)] * [(x_0, \delta)] = [(\gamma, \delta)] = [(x_0, \delta)] * [(\gamma, y_0)]$
\end{lemma}

\begin{proof}
This follows immediately from the proof that,
\[\pi_n \left( \prod_\alpha X_\alpha \right) \cong \prod_{\alpha} \pi_n(X_\alpha)\]
For two spaces, the isomorphism acts on a map $\Gamma : S^n \to X \times Y$ which is equivalent to a pair of maps $(\gamma, \delta) : S^n \to X \times Y$ via,
\[ [(\gamma, \delta)] \mapsto ([\gamma], [\delta]) \]
However,
\[ ([\gamma], [\delta]) = (x_0, [\delta]) * ([\gamma], y_0) = ([\gamma], y_0) * (x_0, [\delta]) \]
where I represent the constant loops at $x_0$ and $y_0$ which are the identity elements by $x_0$ and $y_0$ respectively. Therefore, under the isomorphism,
$[(\gamma, y_0)] * [(x_0, \delta)] = [(\gamma, \delta)] = [(x_0, \delta)] * [(\gamma, y_0)]$
\end{proof}

\end{document}
