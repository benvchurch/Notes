\documentclass[12pt]{extarticle}
\usepackage{import}
\import{./}{Includes}

\begin{document}
\atitle{7}
 
\section*{Problem 1.}

Let $X$ be an $n$-connected $n$-dimensional CW complex. Consider the map $f : X \to \ast$. Because $X$ is $n$-connected, we know that $\pi_i(X) = 0$ for $i \le n$ and therefore $f_* : \pi_n(X) \to \pi_n(\ast) = 0$ is an isomorphism because both a trivial. Furthermore, $f_* : \pi_{n + 1}(X) \to \pi_{n + 1}(\ast) = 0$ is a surjection because the codomain is trivial. Therefore, $f$ is an $n+1$-equivalence. However, $X$ and $\ast$ are both CW complexes of dimension less than $n + 1$, namely of dimension $n$ and $0$ respectively. Therefore, by Whitehead's theorem, $f$ is a homotopy equivalence so $X$ is homotopic to a single point and is therefore contractible.  

\section*{Problem 2.}

Let $(X, A)$ be a CW pair with $A$ contractible. Thus, $\pi_n(A) = 0$ for all $n$. By the entension lemma, any map $f : A \to Y$ can be extended to a map $f : X \to Y$ if $\pi_{n-1}(Y) = 0$ for any $n$ such that $X \sm A$ has at least one cell of dimension $n$. However, $\pi_{n}(A) = 0$ for all $n$ so any map $f : A \to A$ satisfies this criterion. In particular, take the identity $\id : A \to A$. Therefore, there is an extension $f : X \to A$ such that $f \circ \iota_A = \id$. Therefore, $f : X \to A$ is a retract.

\section*{Problem 3.}

Suppose that $X$ and $Y$ have a common CW approximation. Then, there exsits a CW complex $Z$ such that there exist weak equivalences $f : Z \to X$ and $f : Z \to X$. Therefore, there is a sequence $X, Z, Y$ such that each adjacent pair is related by a weak equivalence so, by definition, $X \simeq_W Y$.  \bigskip\\
Conversely, suppose that $X \simeq_W Y$ then there exists a sequence $X = X_1, X_2, \cdots, X_n = Y$ such that either $X_i \to X_{i+1}$ or $X_{i+1} \to X_i$ is a weak equivalence. Suppose that $g : A \to B$ is a weak homotopy equivalence. We know that $A$ has a CW approximation $Z$ and thus there is a weak equivalence $f : Z \to A$ and thus $g \circ f : Z \to B$ is a weak equivalence because $(g \circ f)_* = g_* \circ f_*$ and the composition of isomorphisms is an isomorphism. Therefore $Z$ is also a CW approximation of $B$. Furthermore, given any CW approximation $Z'$ of $B$ we know that CW approximations are unique up to homotopy equivalence so $Z \simeq Z'$ because $Z$ is also a CW approximation of $B$. Take a homotopy equivalence $h : Z' \to Z$ (which exists because homotopy equivalence is symmetric) and thus $f \circ h : Z' \to A$ is a weak equivalence because $(f \circ h)_* = f_* \circ h_*$ is a composition of isomorphisms and thus an isomorphism on every $\pi_n$. Therefore $Z'$ is a CW approximation of $A$. Therefore, any CW approximation of $A$ or $B$ is a CW approximation of both. \bigskip\\
In particular, let $Z$ be a CW approximation of $X_i$ then since either $X_{i} \to X_{i+1}$ or $X_{i+1} \to X_{i}$ is a weak equivalence we know that $Z$ is also a CW approximation of $X_{i+1}$. By induction, if $Z$ is a CW approximation of $X$ then $Z$ is a CW approximation of $X_{i}$ for all $i$ and thus also for $X_n = Y$. Therefore, $X$ and $Y$ have a common CW approximation.  

\section*{Problem 4.}

Let $n > k > 0$ and suppose there were a retract $\rp^n \to \rp^k$. The retract would induce a surjection on the the homotopy groups $\pi_i(\rp^n) \to \pi_i(\rp^k)$. However, for $i = k$ we know that $\pi_k(\rp^k) \cong \pi_k(S^k) \cong \Z$ and if $k \neq 2$ then $\pi_k(\rp^n) \cong \pi_k(S^n) = 0$ since $n > k$ and if $k = 2$ then $\pi_2(\rp^n) = \Z/2\Z$ Either way, there cannot be a surjection $\pi_k(\rp^n) \to \pi_k(\rp^k)$ and thus there cannot be a retract $\rp^n \to \rp^k$. 

\section*{Problem 5.}

We define a sequence of CW complexes,
\[X_n = \prod_{k = 1}^n K(G_k, k) \]
where $G_k$ is an enumeration of groups. We have subcomplex inclusion maps $\iota_n : X_n \to X_{n + 1}$ by setting the $n+1$ coordinate equal to the basepoint of $K(G_{n+1}, n + 1)$. Then define a new CW complex to be the direct limit or equivalently the union of these $X_n$,
\[ X = \bigcup X_n = \lim\limits_{\to} X_n \] Now, $K(G_n, n)$ is $n-1$-connected we know that it can be realized as a CW complex with no $k$-cells for $0 < k \le n - 1$ so the $n$-skeleton of $X$ is contained in $X_{n}$. Furthermore, for $n > i$,
\[ \pi_i(X_n) \cong \prod_{k = 1}^n \pi_i(K(G_k, k))   = 1 \times \cdots \times G_i \times \cdots \times 1 \cong G_i\]
However, we know that the $\pi_n$ of a CW complex only depends on its $n+1$-skeleton. Thus,
\[ \pi_n(X) = \pi_n(X_{n+1}) \cong G_n \]
because the entrie $n+1$-skeleton of $X$ is contained within $X_{n+1}$ and $X_{n+1} \subset X$. 

\section*{Problem 6.}

Let 
\begin{center}
\begin{tikzcd}
F \arrow[r, "\iota", hook] & E \arrow[r, "p"] & B
\end{tikzcd}
\end{center}
be a fibration such that $\iota$ is nullhomotopic. We can write the long exact sequence induced by this fibration as,
\begin{center}
\begin{tikzcd}
\cdots \arrow[r] & \pi_3(F) \arrow[r, "0"] & \pi_3(E) \arrow[r] & \pi_3(B) \arrow[r] & \pi_2(F) \arrow[r] & \pi_2(E) \arrow[draw=none]{d}[name=Z, shape=coordinate]{} \arrow[r] & \pi_2(B)
\arrow[dlllll,
rounded corners, crossing over,
to path={ -- ([xshift=2ex]\tikztostart.east)
|- (Z) [near end]\tikztonodes
-| ([xshift=-2ex]\tikztotarget.west)
-- (\tikztotarget)}]
\\ 
& \pi_1(F) \arrow[r] & \pi_1(E) \arrow[r] & \pi_1(B) \arrow[r] & \pi_0(F) \arrow[r] & \pi_0(E) \arrow[r] & \pi_0(B)
\end{tikzcd}
\end{center}
where the map $\iota_* = 0$ since $\iota$ is nullhomotopic. Since the zero map factors through the trivial group in the category of groups, the sequence breaks up into short exact sequences,
\begin{center}
\begin{tikzcd}
0 \arrow[r] & \pi_n(E) \arrow[r, "p_*"] & \pi_n(B) \arrow[r] & \pi_{n-1}(F) \arrow[r] & 0
\end{tikzcd}
\end{center}
We can show directly that $p_*$ is injective.
Take $f : S^n \to E$ and suppose that $p_*([f]) = [0]$ then $p \circ f \simeq e$ via a homotopy $h : S^n \times I \to B$ were $e : S^n \to B$ is some constant map. Consider the commutative diagram,
\begin{center}
\begin{tikzcd}
S^n \arrow[r, "f"] \arrow[d, hook] & E \arrow[d, "p"]\\
S^n \times I \arrow[r, "h"] \arrow[ru, "\tilde{h}", dashed] & B 
\end{tikzcd}
\end{center}
then there is a homotopy $\tilde{h} : S^n \times I \to E$ because $p$ is a fibration. Therefore, $\tilde{h}(x, 0) = f(x)$ and $p \circ \tilde{h} = h$ and therefore $p \circ \tilde{h}(x, 1) = h(x, 1) = x_0$ is constant so $\tilde{h}(x, 1) \in p^{-1}(x_0) = F$. Therefore the map $\tilde{h}(x,1)$ factors through the inclusion $F \hook E$. However, $\iota$ is nullhomotopic so $\tilde{h}(x,1)$ is nullhomotopic. Therefore, $\tilde{h}(x,0) = f(x)$ is homotopic to a nullhomotopic map so $[f]$ is trivial. Therefore $p_*$ is an injection. \bigskip\\
Now, we need a left inverse $g : \pi_n(B) \to \pi_n(E)$ of the map $p_*$ such that $g \circ p_* = \id_{\pi_n(E)}$ (which I have not been able to find). Given this, via problem 7. of the previous problem set, we see that each short exact sequence is split and thus,
\[ \pi_n(B) = \pi_n(E) \oplus \pi_{n-1}(F) \]


\section*{Problem 7.}

Suppose that $S^k \hook S^m \hook S^n$ is a fiber bundle. Since locally the fiber bundle is a product of the base with the fiber, by the Invariance of Dimension, we know that $m = k + n$. Since all spheres are paracompact, this is also a fibration so we can consider the exact sequence defined by a fibration,

\begin{center}
\begin{tikzcd}
\cdots \arrow[r] & \pi_4(S^k) \arrow[r] & \pi_4(S^m) \arrow[r] & \pi_4(S^n) \arrow[r] & \pi_3(S^k) \arrow[r] & \pi_3(S^m) \arrow[draw=none]{d}[name=Z, shape=coordinate]{} \arrow[r] & \pi_3(S^n)
\arrow[dlllll,
rounded corners, crossing over,
to path={ -- ([xshift=2ex]\tikztostart.east)
|- (Z) [near end]\tikztonodes
-| ([xshift=-2ex]\tikztotarget.west)
-- (\tikztotarget)}]
\\ 
& \pi_2(S^k) \arrow[r] & \pi_2(S^m) \arrow[r] & \pi_2(S^n) \arrow[r] & \pi_1(S^k) \arrow[r] & \pi_1(S^m) \arrow[r] & \pi_1(S^n) \arrow[r] & \pi_0(S^k)
\end{tikzcd}
\end{center}
however for $i < m = n + k$ we know that $\pi_i(S^m) = 0$ so we get exact sequences,
\begin{center}
\begin{tikzcd}
0 \arrow[r] & \pi_i(S^n) \arrow[r] & \pi_{i-1}(S^k) \arrow[r] & 0
\end{tikzcd}
\end{center}
and therefore $\pi_i(S^n) \cong \pi_{i-1}(S^k)$ for $ i < k + n$ and for $i = k + n$ one size gives zero so we have a surjection $\pi_{n + k}(S^n) \twoheadrightarrow \pi_{n + k -1}(S^k)$. First assume that $n, k > 0$. Suppose that $k < n - 1$ then since $k + 1 < k + n$ set $i = k + 1$ and we have $\pi_{k}(S^k) \cong \pi_{k+1}(S^n) = 0$ which contradicts the fact that $\pi_k(S^k) \cong \Z$. Similarly, if $k > n - 1$ then since $n < n + k$ set $i = n$ and we have $\pi_n(S^n) \cong \pi_{n - 1}(S^k) = 0$ which again contradicts the fact that $\pi_n(S^n) \cong \Z$. Therefore, $k = n - 1$ and thus $m = n + k = 2 n - 1$. In the case that $k = 0$ and $n > 0$ we get an exact sequence,
\begin{center}
\begin{tikzcd}
0 \arrow[r] & \pi_1(S^m) \arrow[r] & \pi_1(S^n) \arrow[r] & \pi_0(S^k) \arrow[r] & 0
\end{tikzcd}
\end{center}
but since $\pi_0(S^k)$ is nontrivial we must have $\pi_1(S^m)$ be nontrivial so $m = 1$ but $m = n + k = n$ so $n = 1$. This case satisfies the formula $k = n - 1$ and $m = 2 n - 1$. Next, consider the case $n = 0$ and $k > 0$. Since $m = n + k > 0$ we know that $S^m$ is connected by $S^k = S^0$ is non connected. Therefore, the map $S^m \hook S^k$ cannot be surjective and therefore the fiberes of $S^k$ cannot all be homoeomorphic. Therefore there are no fiber bundles in this case. Finally, in the most trivial case, $k = 0$ and $n = 0$ we would have to have a bundle $S^0 \hook S^0 \hook S^0$ which would require the preimage of both points in $S^0$ to have size two (since the fiber has two elements) which would mean the middle space would have to have at least four elements which it does not. Therefore, there cannot be any fiber bundles in this case either. \bigskip\\
Therefore, any fiber bundle $S^k \hook S^m \hook S^n$ must satsify $k = n - 1$ and $m = 2n - 1$. 

\end{document}
