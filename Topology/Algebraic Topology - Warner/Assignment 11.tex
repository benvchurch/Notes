\documentclass[12pt]{extarticle}
\usepackage{import}
\import{./}{Includes}

\begin{document}
\atitle{11}

\section*{Problem 1.}

\begin{enumerate}
\item Let $X = S^2$ and let $A \subset X$ be the set containing the north and south poles of $S^2$. Then, $(X, A)$ is a good pair so we  have a long exact sequence of reduced homology,
\begin{center}
\begin{tikzcd}
\cdots \arrow[r] & \tilde{H}_n(A) \arrow[r] & \tilde{H}_n(X) \arrow[r] & \tilde{H}_n(X/A) \arrow[r] & \tilde{H}_{n-1}(A) \arrow[r] & \tilde{H}_{n-1}(X) \arrow[r] & \cdots
\end{tikzcd}
\end{center} 
Since $A$ is a discrete space on two points, 
\[ \tilde{H}_n(A) \cong 
\begin{cases}
\Z & n = 0
\\
0 & n > 0
\end{cases} \]
Furthermore, we know the reduced homology of spheres so,
\[ \tilde{H}_n(X) = \tilde{H}_n(S^2) \cong 
\begin{cases}
\Z & n = 2
\\
0 & n \neq 2
\end{cases} \]
Therefore, at $n = 0$ the long exact sequence gives,
\begin{center}
\begin{tikzcd}
\cdots \arrow[r] & \tilde{H}_0(A) \arrow[r] & 0 \arrow[r] & \tilde{H}_0(X/A) \arrow[r] & 0
\end{tikzcd}
\end{center}
and thus $\tilde{H}_0(X / A) = 0$ which we knew since $X/A$ is path-connected. Furthermore, at $n = 1$, the long exact sequence gives,
\begin{center}
\begin{tikzcd}
\cdots \arrow[r] & \tilde{H}_1(A) \arrow[r] & 0 \arrow[r] & \tilde{H}_1(X/A) \arrow[r] & \tilde{H}_0(A) \arrow[r] & 0
\end{tikzcd}
\end{center}
which implies that $\tilde{H}_1(X/A) \cong \tilde{H}_0(A) \cong \Z$. Finally, whenever $n > 1$ since $A$ is the disjoint union of contractible spaces, $\tilde{H}_n(A) = \tilde{H}_{n-1}(A) = 0$ so from the long exact sequence,
\begin{center}
\begin{tikzcd}
0 \arrow[r] & \tilde{H}_n(X) \arrow[r] & \tilde{H}_n(X/A) \arrow[r] & 0
\end{tikzcd}
\end{center}
and thus $\tilde{H}_n(X/A) \cong \tilde{H}_n(X)$. Putting these together and using the fact that $\tilde{H}_n(X) \cong H_n(X)$ for $n > 0$ and $\tilde{H}_0(X) \cong H_0(X) \oplus \Z$ we find that the homology of $X/A$ is,
\[ H_n(X) \cong 
\begin{cases}
\Z & n = 0, 1, 2
\\
0 & n > 2
\end{cases} \]

\item Let $X = S^1 \times (S^1 \vee S^1)$. Since $X$ is path-connected, $H_1(X) \cong \Z$. Computing the fundamental group,
\[ \pi_1(X) \cong \pi_1(S^1) \times \pi_1(S^1 \vee S^1) \cong \Z \times (\Z * \Z) \] 
Therefore, since $X$ is $0$-connected, by Hurewicz's theorem,
\[ H_1(X) \cong \pi_1(X)^{\text{ab}} \cong \Z \oplus \Z \oplus \Z \]
Next, we can decompose $X = A \cup B$ where $A$ and $B$ are the two tori which are glued together to form $S^1 \times (S^1 \vee S^1)$ such that $A \cap B \cong S^1$. Applying the Mayer-Vietoris sequence,
\begin{center}
\begin{tikzcd}
\cdots \arrow[r] & H_n(A \cap B) \arrow[r] & H_n(A) \oplus H_n(B) \arrow[r] & H_n(X) \arrow[r] & H_{n-1}(A \cap B) \arrow[r] & \cdots
\end{tikzcd}
\end{center}
Since $A \cap B = S^1$ and $H_{n-1}(S^1) = 0$ for $n > 2$ we have,
\[ H_n(X) \cong H_n(A) \oplus H_n(B) = 0 \]
since the homology of a torus vanishes for $n > 2$. For $n = 2$ the map $H_{n-1}(A \cap B) \to H_{n-1}(A) \oplus H_{n-1}(B)$ so the map $H_n(X) \to H_{n-1}(A \cap B)$ is zero. 
\begin{center}
\begin{tikzcd}
0 \arrow[r] & \Z \oplus \Z \arrow[r] & H_2(X) \arrow[r] & 0
\end{tikzcd}
\end{center}
where $H_2(A) \cong H_2(B) \cong \Z$. 
Thus, $H_2(X) \cong \Z \oplus \Z$. Putting everything together,
\[ 
H_n(X) \cong 
\begin{cases}
\Z \oplus \Z & n = 2 
\\ 
\Z \oplus \Z \oplus \Z & n = 1 
\\
\Z & n = 0
\\
0 & n > 2
\end{cases}
\]
\newpage

\item

\newpage


\item The identification space of a torus where points which differ by a rotation of $2 \pi / m$ or $2 \pi / m$ about the two principal directions simply gives a single square identified in the same way as a torus. Therefore, this space is homeomorphic to a torus so it has homology,
\[ H_n(X) \cong 
\begin{cases}
\Z & n = 2
\\
\Z \oplus \Z & n = 1
\\
\Z & n = 1
\\
0 & n > 2
\end{cases} \]
Alternatively, if the question is asking to fix a base point $x_0$ and mod out by points on the \textit{fixed} circles $S^1 \times \{ x_0 \}$ and $\{ x_0 \} \times S^1$ then we can use the relative homology of $S^1 \times S^1$ with $A$ a finite number (in particular $k$) of points,
\[ H_r(S^1 \times S^1, A) \cong \begin{cases}
\Z & r = 2 \\
\Z^{k+1} & r = 1 \\
0 & r \neq 1, 2
\end{cases} \]
which we calculated on assignment $9$. In this case, we have $m + n  - 1$ points on the boundary of the identification square. Thus, since $(X, A)$ is a good pair,
\[ \tilde{H}_r(S^1 \times S^1/A) \cong \tilde{H}_r(S^1 \times S^1, A) \cong \begin{cases}
\Z & r = 2 
\\
\Z^{n + m} & r = 1 
\\
0 & r \neq 1, 2
\end{cases} \]
Thus,
\[ H_r(S^1 \times S^1/A) \cong 
\begin{cases}
\Z & r = 2 \\
\Z^{n + m} & r = 1 
\\
\Z & r = 0
\\
0 & r > 2
\end{cases} \]

\end{enumerate}

\section*{Problem 2.}

Consider the commutative diagram formed from portions of the long exact sequences for the pairs $(X^{n+1},X^n)$, and $(X^{n}, X^{n-1})$, and $(X^{n-1}, X^{n-2})$. 
\begin{center}
\begin{tikzcd}[column sep = tiny, row sep = large]
& & & H_{n}(X^{n+1}, X^n) \cong 0 
\\
H_n(X^{n-1}) \cong 0 \arrow[rd] & & H_n(X^{n+1}) \cong H_n(X) \arrow[ru]
\\
& H_n(X^n) \arrow[rd, "j_n"] \arrow[ru, "\iota"] &
\\
H_{n+1}(X^{n+1}, X^n) \arrow[ru, "\delta_{n+1}"] \arrow[rr, "d_{n+1}"] & & H_n(X^{n}, X^{n-1}) \arrow[rd, "\delta_n"] \arrow[rr, "d_n"] & & H_{n-1}(X^{n-1}, X^{n-2})
\\
& & & H_{n-1}(X^{n-1}) \arrow[ru, "j_{n-1}"]
\\
& & H_{n-1}(X^{n-2}) \cong 0 \arrow[ru]
\end{tikzcd}
\end{center}
The maps $d_{n+1}$ and $d_{n}$ are defined such as $d_{n+1} = j_n \circ \delta_{n+1}$ and $d_{n} = j_{n-1} \circ \delta_n$ such that the diagram commutes. By exactness, we have that $\Im{j_n} \cong \ker{\delta_n}$ but $\ker{\delta_n} = \ker{d_n}$ because $j_{n-1}$ is injective by exactness. Therefore, $\ker{n_n} = \Im{j_n}$. However, by exactness, $\Im{j_n}$ is injective so it is an isomorphism onto its image. Therefore, $j_n : H_n(X^n) \xrightarrow{\sim} \ker{d_n} \subset H_n(X^n, X^{n-1})$. However, the group $H_n(X^n, X^{n-1})$ is the free abelian group on the $n$-cells of $X$. Thus, since $H_n(X^n)$ is isomorphic to a subgroup of a free abelian group, the group $H_n(X^n)$ is itself a free abelian group.  

\section*{Problem 3.}

Let $X = A_1 \cup A_2 \cup \cdots \cup A_n$ such that all intersections are either empty or have trivial reduced homology. Define the sequence of topological spaces,
\[Y_k = A_1 \cup A_2 \cup \cdots \cap A_k\]
and likewise,
\[Z_k = A_k \cap A_{k+1} \cap \cdots \cap A_r\] 
Using the Mayer-Vietoris sequence we will prove by induction that $\tilde{H}_r(Y_k \cap Z_{k+1}) = 0$ for $r \ge k - 1$. Consider the base case, $k = 1$. We have $Y_1 = A_1$ and $Z_{2} = A_2 \cap A_3 \cap \cdots \cap A_r$ so $Y_1 \cap Z_{2} = A_1 \cap A_2 \cap \cdots \cap A_r$. Therefore,
\[ \tilde{H}_r(X_{1} \cap Z_{2}) = \tilde{H}_r(A_1 \cap A_2 \cap \cdots \cap A_r) = 0 \]
for all $r$ since the intersections have trivial reduced homology. Now take the induction hypothesis, $\tilde{H}_r(Y_{k} \cap Z_{k+1}) = 0$ for $r \ge k - 1$. We need to write the term $Y_{k+1} \cap Z_{k+2}$ in terms of the spaces we know more about. 
\[ Y_{k+1} \cap Z_{k+2} = (A_1 \cap Z_{k+2}) \cup (A_2 \cap Z_{k+2}) \cup \cdots \cup (A_{k+1} \cap Z_{k+2}) = (Y_k \cap Z_{k+1}) \cup Z_k \] 
All these spaces are open so we can consider the Mayer-Vietoris sequence,
\begin{center}
\begin{tikzcd}[column sep = small]
\tilde{H}_r((Y_k \cap Z_{k+1}) \cap Z_k) \arrow[r] & \tilde{H}_r(Y_k \cap Z_{k+1}) \oplus \tilde{H}_r(Z_{k}) \arrow[r] & \tilde{H}_r((Y_k \cap Z_{k+1}) \cup Z_{k}) \arrow[r] & \tilde{H}_{r-1}((Y_k \cap Z_{k+1}) \cap Z_{k})
\end{tikzcd}
\end{center}
If we take $r \ge k-1$ then $\tilde{H}_r(Y_k \cap Z_{k+1}) = 0$ by hypothesis and $\tilde{H}_r(Z_k) = 0$ because $Z_k$ is an intersection. Furthermore, if $r \ge k$ then $r-1 \ge k - 1$ so again by the induction hypothesis,
\[ \tilde{H}_{r-1}((Y_k \cap Z_{k+1}) \cap Z_k) = \tilde{H}_{r-1}(Y_k \cap Z_{k+1}) = 0 \]
Therefore, if $r \ge (k+1)-1$ the above long exact sequence is zero except for the middle left which is then forced to be zero,
\[ \tilde{H}_r(Y_{k+1} \cap Z_{k + 2}) = \tilde{H}_r((Y_k \cap Z_{k+1}) \cup Z_k) = 0 \]
so the claim holds by induction. Therefore, the claim holds for $k = n$. Thus, for $r \ge n - 1$ we have that $\tilde{H}_r(X) = \tilde{H}_r(Y_n \cap Z_{n + 1}) = 0$. 
\bigskip\\
Furthermore, we know that $\tilde{H}_{n}(S^n) \cong \Z$. However, we can decompose $S^n$ into $n + 2$ open sets with homologically trivial intersections. To do this, view the sphere $S^n$ as the boundary of an $n+1$-simplex $\partial \Delta^{n+1}$. Take the open sets to be the faces of $\partial \Delta^{n+1}$ of which there are $n+2$. \footnote{These sets are not actually open. However, we can take open sets which are $\epsilon$-neighborhoods of the faces which deformation retract onto the faces.} Furthermore, the intersections of these faces are lower dimensional (solid) simplices which are all contractible. Therefore, the theorem requires that $\tilde{H}_r(S^n) = 0$ for $r \ge (n+2)-1 = n+1$ which is strict because $\tilde{H}_n(S^n) \neq 0$.  

\section*{Problem 4.}

Let $A, B$ be abelian groups and let $F$ be a free abelian group. Consider the diagram in $\mathbf{AbGrp}$,

\begin{center}
\begin{tikzcd}
& F \arrow[d, "f"] 
\\
A \arrow[r, two heads, "g"] & B
\end{tikzcd}
\end{center}
where $g : A \to B$ is surjective. Since $F$ is a free abelian group it is in the image of the free functor $\mathbf{Free} : \mathbf{Set} \to \mathbf{AbGrp}$. Let $F = \mathbf{Free}(S)$ for some set $S$. Furthermore, $\mathbf{Free}$ is right adjoint to the forgetful functor $\mathbf{Forget} : \mathbf{AbGrp} \to \mathbf{Set}$. Consider the diagram in $\mathbf{Set}$,
\begin{center}
\begin{tikzcd}[column sep = huge, row sep = huge]
& S \arrow[d, "f'"] \arrow[dl, "h'"']
\\
\mathbf{Forget}(A) \arrow[r, two heads, "\mathbf{Forget}(g)"] & \mathbf{Forget}(B) 
\end{tikzcd}
\end{center}  
where the map $f' : S \to \mathbf{Forget}(B)$ is defined by $f'(s) = f(s) \in B$. Clearly, the map $\mathbf{Forget}(f)$ is still surjective because it acts on elements of the underlying sets identically to $g$. Therefore, there is a map $h' : S \to \mathbf{Forget}(A)$ in the category $\mathbf{Set}$ such that the diagram commutes because $\mathbf{Forget}(f)$ is surjective so there is a right inverse $i : \mathbf{Forget}(B) \to \mathbf{Forget}(A)$ and we can take $h' = i \circ f'$ such that $\mathbf{Forget}(g) \circ h' = (\mathbf{Forget}(g) \circ i) \circ f' = f'$. Becauset $\mathbf{Forget}$ and $\mathbf{Free}$ are adjoints,
\[ \mathrm{Hom}_{\mathbf{AbGrp}}(\mathbf{Free}(S), A) \cong  \mathrm{Hom}_{\mathbf{Set}}(S, \mathbf{Forget}(A))  \] 
naturally. Therefore, the map $h' \in \mathrm{Hom}_{\mathbf{Set}}(S, \mathbf{Forget}(A))$ corresponds to $h \in \mathrm{Hom}_{\mathbf{AbGrp}}(\mathbf{Free}(S), A)$ such that the diagram,
\begin{center}
\begin{tikzcd}[column sep = large, row sep = large]
& \mathbf{Free}(S) \arrow[d, "f"]  \arrow[dl, "h"']
\\
A \arrow[r, two heads, "g"] & B
\end{tikzcd}
\end{center}
commutes because the maps $g$ and $f$ correspond to $\mathbf{Forget}(g)$ and $f'$ under the natural adjointness relation. Therefore, $F = \mathbf{Free}(S)$ is a projective object in $\mathbf{AbGrp}$. 

\section*{Problem 5.}

Let $f : A \to F$ be a surjective map of abelian groups where $F$ is a free abelian group. By the previous problem, $F$ is a projective object so we have a commutative diagram which I have extended to an exact sequence,
\begin{center}
\begin{tikzcd}
&& & F \arrow[d, "\id_F"] \arrow[dl, dashed, "h"']
\\
0 \arrow[r] & \ker{f} \arrow[r, hook] & A \arrow[r, "f", two heads] & F \arrow[r] & 0
\end{tikzcd}
\end{center}
However, $f \circ h = \id_F$ so the exact sequence splits on the right. Therefore, $A \cong \ker{f} \oplus F$. 

\section*{Problem 6.}

Let $(C, d)$ be a chain complex of abelian groups such that $C_n$ is free. For each $n$, consider the exact sequence,
\begin{center}
\begin{tikzcd}
0 \arrow[r] & \ker{d_n} \arrow[r, "\iota", hook] & C_n \arrow[r, "d_n", two heads] & \Im{d_n} \arrow[r] & 0 
\end{tikzcd}
\end{center}
Since $\Im{d_n} \subset C_{n-1}$ which is a free group, we know that $\Im{d_n}$ is free. Therefore, by the above problem, $C_n \cong \ker{d_n} \oplus \Im{d_n}$ since $\Im{d_n}$ is free and $d_n : C_n \to \Im{d_n}$ is surjective. Define the map, $f_n : C_n \to H_n(C)$ by the composition,
\begin{center}
\begin{tikzcd}
C_n \arrow[r, "\sim"] & \ker{d_n} \oplus \Im{d_n} \arrow[r, "\pi_1"] & \ker{d_n} \arrow[r, "\pi_{H_n}"] & H_n(C)
\end{tikzcd}
\end{center}
We need to show that $f : C \to H$ is a chain map and a quasi-isomorphism where $H$ is the chain complex $(H_n(C), 0)$. 
Consider the diagram,
\begin{center}
\begin{tikzcd}
\cdots \arrow[r] & C_{n+1} \arrow[r, "d_{n+1}"] \arrow[d, "f_{n+1}"] & C_n \arrow[r, "d_n"] \arrow[d, "f_n"] & C_{n-1} \arrow[r, "d_{n-1}"] \arrow[d, "f_{n-1}"] & \cdots
\\
\cdots \arrow[r] & H_{n+1} \arrow[r, "0"] & H_n \arrow[r, "0"] & H_{n-1} \arrow[r, "0"] & \cdots
\end{tikzcd}
\end{center}
However, $f_n \circ d_{n+1} = 0 = 0 \circ f_{n+1}$ since $\Im{d_{n+1}} \subset \ker{\pi_{H_n}} \subset \ker{f_n}$. Therefore, $f_n$ is a chain map. Consider the induced map,
\[ f_* : H_n(C) \to H_n(H) = H_n(C) \]
where $H_n(H) = \ker{0_n} / \Im{0_{n+1}} = H_n = H_n(C)$. The induced map acts on $a \in \ker{d_n}$ via,
\[ f_*(a + \Im{d_n}) = f(a) + \Im{0_n} \]
However, $a \in \ker{d_n}$ so $\pi_1(a) = a$ and thus $f(a) = \pi_{H_n}(a) = a + \Im{d_n}$. Therefore, 
\[ f_*(a + \Im{d_n}) = a + \Im{d_n} \]
so $f_*$ is the identity map on $H_n(C) \to H_n(H)$ under the identification $H_n(H) \cong H_n(C)$. Thus, $f$ is a quasi-isomorphism.   

\end{document}
