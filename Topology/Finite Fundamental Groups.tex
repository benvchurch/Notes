\documentclass[12pt]{article}
\usepackage[utf8]{inputenc}
\usepackage[english]{babel}
\usepackage[a4paper, total={6in, 9in}]{geometry}
\usepackage{tikz-cd}
 
\usepackage{amsthm, amssymb, amsmath, centernot}

\newcommand{\notimplies}{%
  \mathrel{{\ooalign{\hidewidth$\not\phantom{=}$\hidewidth\cr$\implies$}}}}

\renewcommand\qedsymbol{$\square$}
\newcommand{\cont}{$\boxtimes$}
\newcommand{\divides}{\mid}
\newcommand{\ndivides}{\centernot \mid}
\newcommand{\Z}{\mathbb{Z}}
\newcommand{\N}{\mathbb{N}}
\newcommand{\C}{\mathbb{C}}
\newcommand{\Zplus}{\mathbb{Z}^{+}}
\newcommand{\Primes}{\mathbb{P}}
\newcommand{\ball}[2]{B_{#1} \! \left(#2 \right)}
\newcommand{\Q}{\mathbb{Q}}
\newcommand{\R}{\mathbb{R}}
\newcommand{\Rplus}{\mathbb{R}^+}
\newcommand{\invI}[2]{#1^{-1} \left( #2 \right)}
\newcommand{\End}[1]{\text{End}\left( A \right)}
\newcommand{\legsym}[2]{\left(\frac{#1}{#2} \right)}
\renewcommand{\mod}[3]{\: #1 \equiv #2 \: \mathrm{mod} \: #3 \:}
\newcommand{\nmod}[3]{\: #1 \centernot \equiv #2 \: \mathrm{mod} \: #3 \:}
\newcommand{\ndiv}{\hspace{-4pt}\not \divides \hspace{2pt}}
\newcommand{\finfield}[1]{\mathbb{F}_{#1}}
\newcommand{\finunits}[1]{\mathbb{F}_{#1}^{\times}}
\newcommand{\ord}[1]{\mathrm{ord}\! \left(#1 \right)}
\newcommand{\quadfield}[1]{\Q \small(\sqrt{#1} \small)}
\newcommand{\vspan}[1]{\mathrm{span}\! \left\{#1 \right\}}
\newcommand{\galgroup}[1]{Gal \small(#1 \small)}
\newcommand{\aut}[1]{\mathrm{Aut} \small(#1 \small)}
\newcommand{\ints}[1]{\mathcal{O}_{#1}}
\newcommand{\sm}{\! \setminus \!}
\newcommand{\norm}[3]{\mathrm{N}^{#1}_{#2}\left(#3\right)}
\newcommand{\qnorm}[2]{\mathrm{N}^{#1}_{\Q}\left(#2\right)}
\newcommand{\quadint}[3]{#1 + #2 \sqrt{#3}}
\newcommand{\pideal}{\mathfrak{p}}
\newcommand{\inorm}[1]{\mathrm{N}(#1)}
\newcommand{\tr}[1]{\mathrm{Tr} \! \left(#1\right)}
\newcommand{\delt}{\frac{1 + \sqrt{d}}{2}}
\newcommand{\ch}[1]{\mathrm{char} \: #1}
\renewcommand{\Im}[1]{\mathrm{Im}(#1)}
\newcommand{\minimal}[2]{\mathrm{Min}(#1;#2)}
\newcommand{\fix}[2]{\mathrm{Fix}_{#1} (#2)}
\newcommand{\id}{\mathrm{id}}
\renewcommand{\empty}{\varnothing}
\newcommand{\SU}[1]{\mathrm{SU}(#1)}

\theoremstyle{remark}
\newtheorem*{remark}{Remark}

\theoremstyle{definition}
\newtheorem{theorem}{Theorem}[section]
\newtheorem{lemma}[theorem]{Lemma}
\newtheorem{proposition}[theorem]{Proposition}
\newtheorem{corollary}[theorem]{Corollary}


\newenvironment{definition}[1][Definition:]{\begin{trivlist}
\item[\hskip \labelsep {\bfseries #1}]}{\end{trivlist}}


\newenvironment{lproof}{\begin{proof} \renewcommand{\qedsymbol}{}}{\end{proof}}


\begin{document}
\section{Manifolds with any Finite Fundamental Group}

\begin{remark}
For loops $\gamma_1, \gamma_2 : I \to X$ we will use the notation $\gamma_1 * \gamma_2$ to denote the loop, \[h(t) = \begin{cases} \gamma_1(2t) & t \le \frac{1}{2} \\ \gamma_2(2t - 1) & t \ge \frac{1}{2} \end{cases}\]
\end{remark}

\begin{definition}
An action of of a group $G$ on a topological space $X$ is a homomorphism $A : G \to \mathrm{Homeo}(X)$. Equivalently, one may define a map $\varphi : G \times X \to X$ and let $\varphi_g(x) = \varphi(g, x)$ such that $\varphi_e = \mathrm{id}_X$ and $\varphi_{gh} = \varphi_g \circ \varphi_h$ and $\varphi_g$ is a continuous map. Because $\varphi_{g^{-1}}$ is also continuous and $\varphi_g \circ \varphi_{g^{-1}} = \varphi_{g^{-1}} \circ \varphi_{g} = \varphi_e = \mathrm{id}_e$ then each map is a homeomorphism of $X$ to itself so $g \mapsto \varphi_g$ is a homomorphism from $G$ to $\mathrm{Homeo(X)}$.  
\end{definition}

\begin{definition}
Let $G$ be a group acting on a topological space $X$ then $X/G$ is the quotient space under the equivalence relation $x \sim y \iff \exists g \in G : g \cdot x = y$. 
\end{definition}

\begin{remark}
For $x \in X$, let $[x]_G$ denote the equivalence class under a group action and for $\gamma : I \to X$ let $[\gamma]$ denote the equivalence class under path-homotopy. 
\end{remark}

\begin{definition}
A group $G$ acts freely on a set $X$ if every stabilizer is trivial. Equivalently, if for some $x \in G$ we have $g \cdot x = h \cdot x$ then $(h^{-1}g) \cdot x = x$ so $g = h$. 
\end{definition}

\begin{definition}
A group action on $X$ is properly discontinuous if for any $x \in X$ there exists an open neighborhood $x \in U$ such that $(g \cdot U) \cap U = \empty$ for each $g \neq e$. 
\end{definition}

\begin{lemma}
Let the action of $G$ on $X$ be properly discontinuous, then $X$ is a covering space of $X/G$ with the covering map $\pi : X \to X/G$.
\end{lemma}

\begin{proof}
Take an open set $U \subset X$ and consider $\pi(U)$. Then, $\pi(x) \in \pi(U)$ if and only if $\exists y \in U$ such that $x \sim y \iff \exists g \in G : x = g \cdot y \iff x \in g \cdot U$. Therefore,
\[\pi^{-1}(\pi(U))  = \bigcup_{g \in G} g \cdot U\] which is open because each $g$ acts as an open map (in fact a homeomorphism). By the definition of $X/G$ then $\pi(U)$ is open so $\pi$ is an open map. Take a point $x_0 \in X$ and because the action is properly discontinuous, there exists an open $x_0 \in U$ such that $(g \cdot U) \cap U = \empty$ for each $g \neq e$. Consider $V = \pi(U) \subset X/G$ which is open. Since for $g \neq h$, we have $(h^{-1} g) \cdot U \cap U = \empty$ then $(g \cdot U) \cap (h \cdot U) = \empty$ so the slices are disjoint. Finally, take $x, y \in g \cdot U$ then if $\pi(x) = \pi(y)$ we have $[x] = [y]$ so $x = h \cdot y$ for some $g \in G$. But since $y \in g \cdot U$ then $x \in hg \cdot U$ and $x \in g \cdot U$ so $h = e$ and thus $x = y$ because for $g \neq e$ the sets $hg \cdot U$ and $g \cdot U$ are disjoint since $hg \neq g$. Therefore, $\pi|_{g \cdot U}$ is injective but it is trivialy surjective onto $V = \pi(U) = \pi(g \cdot U)$. Furthermore, $\pi$ is an open continuous map and thus a homeomorphism when restricted to $U$. Therefore, $V$ is an openly covered neighborhood of $[x]_G$ so $\pi$ is a covering map of $X/G$.  
\end{proof}

\newpage

\begin{theorem}
Let $X$ be a simply connected and a let the action of $G$ on $X$ be free and properly discontinuous. Then $\pi_1(X/G, [x_0]_G) \cong G$.
\end{theorem} 

\begin{proof}
Fix $x_0 \in X$, then take $g \in G$ and let $\gamma_g : I \to X$ be a path from $x_0$ to $g \cdot x_0$. Such a path exists because $X$ is path-connected. Take the projection map $\pi : X \to X/G$ given by $\pi(x) = [x]_G$. These paths project to loops in the quotient space, $\eta_g = \pi \circ \gamma_g$ which is a loop because $\eta_g(0) = \pi(x_0) = [x_0]$ and $\eta_g(1) = \pi(g \cdot x_0) = [g \cdot x_0] = [x_0]$ and action by $g$ is a continuous map. \\\\
Define the map $\phi : G \to \pi_1(X/G, [x_0]_G)$ given by $\phi : g \mapsto [\pi \circ \gamma_g]$. Take $g, h \in G$ and consider the path $\delta = \gamma_g * (g \cdot \gamma_h)$  where $(h \cdot \gamma_g)(t) = h \cdot \gamma_g(t)$ with endpoints:
\[\gamma_g * (g \cdot \gamma_h)(0) = \gamma_g(0) = x_0 \text{ and } \gamma_g * (g \cdot \gamma_h)(1) = (g \cdot \gamma_h)(1) = g \cdot (h \cdot x_0) = (gh) \cdot x_0\]
Therefore, because $X$ is simply connected, $\delta \sim \gamma_{gh}$ and thus, 
\[\pi \circ \delta = (\pi \circ \gamma_g) * (\pi \circ (g \cdot \gamma_h)) \sim \pi \circ \gamma_{gh} = \eta_{gh}\]
However, $\pi \circ \gamma_g = \eta_g$ and $\pi \circ (g \cdot \gamma_h)(t) = \pi(g \cdot \gamma_h(t)) = [g \cdot \gamma_h(t)]_G = [\gamma_h(t)]_G = \eta_h(t)$ because the orbits are equivalence classses. Thus, $\pi \circ (g \cdot \gamma_h) = \eta_h$ so $\eta_g * \eta_h \sim \eta_{gh}$. 
Therefore, $\phi(gh) = [\eta_{gh}] = [\eta_g * \eta_h] = [\eta_g] [\eta_h] = \phi(g) \phi(h)$ so $\phi$ is a homomorphism. It remains to show that $\phi$ is a bijection. \bigskip \\
$X$ is the universal cover of $X/G$ so any path $\delta : I \to X/G$ can be lifted to a a unique path $\gamma : I \to X$ up to a choice of initial point. Thus, if $\delta$ is a loop at $[x_0]_G$ then there exists a unique path $\gamma : I \to X$ such that $\pi \circ \gamma = \delta$ and $\gamma(0) = x_0$. However, $\pi \circ \gamma(1) = \delta(1) = [x_0]_G$ so $[\gamma(1)]_G = [x_0]_G$ thus $\exists g \in G : \gamma(1) = g \cdot x_0$. Because $X$ is simply connected, $\gamma \sim \gamma_g$ since they share endpoints. Finlly, $\phi(g) = [\pi \circ \gamma_g] = [\pi \circ \gamma] = [\delta]$ so the map $\phi$ is surjective. Finally, take $g, h \in G$ and suppose that $\phi(g) = \phi(h)$ then $\pi \circ \gamma_g \sim \pi \circ \gamma_h$. By the homotopy lifting lemma, these loops lift to unique path-homotopic paths in $X$ with initial point $x_0$. However, $\gamma_g$ and $\gamma_h$ already satisfy the projection property and therefore must be the unique lifts so $\gamma_g \sim \gamma_h$. In particular, $\gamma_g(1) = \gamma_h(1)$ because they are path homotopic so $g \cdot x_0 = h \cdot x_0$ but because $G$ acts freely on $X$ this implies that $g = h$. Therefore, $\phi$ is a bijection.      
\end{proof}

\begin{lemma}
A free action of a finite group on a Hausdorff space is properly discontinuous.  
\end{lemma}

\begin{proof}
Take $x \in X$ and, because the action is free, for each $g \neq e$ we have $g \cdot x \neq x$ so because $X$ is Hausdorff, there exist open sets $U_g$ and $V_g$ such that $x \in U_g$ and $g \cdot x \in U_g$ and $U_g \cap V_g$. Now, let,
\[U = \bigcap_{g \in G \setminus \{e\}} (U_g \cap g^{-1} \cdot V_g)\]
which is open because the intersection is finite. Also, for each $g$, $x \in U_g$ and $g \cdot x \in V_g$ so $x \in g^{-1} \cdot V_g$. Thus, $x \in U$. Now, take any $g \neq e$. We have $U \subset U_g$ and $U \subset g^{-1} \cdot V_g$ so $g \cdot U \subset V_g$. However, $U_g$ and $V_g$ are disjoint so $U$ and $g \cdot U$ are disjoint.  
\end{proof}


\begin{lemma}
Any quotient of a compact connected space is compact and connected.
\end{lemma}

\begin{proof}
Let $X$ be compact and connected. Then, $\pi : X \to X / \sim$ is continuous and surjective. Therefore, $X / \sim$ is the image of a compact and conected set and is thus compact and connected.
\end{proof}


\begin{theorem}
For any finite group $G$, there exists a compact connected manifold with fundamental group $G$.  
\end{theorem}

\begin{proof}
By considering $n \times n$ permutation matrices in $\SU{n}$ we get an embedding of $S_n$ inside $\SU{n}$. However, by Cayley's theorem, any group with order $n$ can be embedded as a subgroup of $S_n$. Therefore, we have the embeddings,
\begin{center}
\begin{tikzcd}[column sep=large]
G \arrow[r, hook] & S_n \arrow[r, hook] & \SU{n}
\end{tikzcd}
\end{center}
Let $\Gamma \subset \SU{n}$ be the embedded copy of $G$. $\Gamma$ acts on $\SU{n}$ by left multiplication which is a topological action because $\SU{n}$ is a topological group and thus has continuous multiplcation. Futhermore, if $g \cdot h = h$ then $gh = h$ so $g = e$. Thus, $\Gamma$ acts freely on $\SU{n}$. However, $\SU{n}$ is a simply connected compact manifold. In particular, $\SU{n}$ is a Hausdorff space and $\Gamma$ is finite so the action is properly discontinuous. Therefore, since $\SU{n}$ is simply connected, $\pi_1(\SU{n}/\Gamma, [x_0]) \cong \Gamma \cong G$. Furthermore $\SU{n}/\Gamma$ is a compact connected space because it is a quotient of $\SU{n}$ which is compact and connected. Finally, we must show that $\SU{n}/\Gamma$ is a manifold. (WIP)    
\end{proof}

\begin{theorem}
For any finite cyclic group $G$, there exists a compact connected 3-manifold with fundamental group $G$.  
\end{theorem}

\begin{proof}
Consider the matrix,
\[
M = \begin{pmatrix}
e^{\frac{2 \pi i}{n}} & 0 \\
0 & e^{-\frac{2 \pi i}{n}}
\end{pmatrix} \in \SU{2}\]
Let $\Gamma = \left< M \right> \subset \SU{2}$. Since $M$ has order $n$, $\Gamma \cong C_n$. By an identical argument to above, $\pi_1(\SU{2}/\Gamma, [x_0]) \cong \Gamma \cong C_n$ and $\SU{2}/\Gamma$ is compact and connected. It remains to show that $\SU{2}/\Gamma$ is a 3-manifold. (WIP)
\end{proof}

\end{document}