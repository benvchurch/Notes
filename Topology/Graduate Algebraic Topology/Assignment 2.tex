\documentclass[12pt]{extarticle}
\usepackage{import}
\import{./}{Includes}

\begin{document}
\atitle{2}

\section*{Hatcher 4.2}

\subsection{22}

Let $G$ be an abelian group and $n > 1$. Then let $X = M(G, n)$ be a CW complex representing the Moore space. We may assume that $X$ has no $k$-cells for $0 < k < n$ so $X$ is $(n-1)$-connected. Since $X$ is simply-connected, by Hurewicz, the first nonzero $\pi_i(X)$ is isomorphic to $H_i(X; \Z)$ but $\tilde{H}_i(X; \Z) = 0$ for $i < n$ so we get,
\[ h_n : \pi_n(X) \xrightarrow{\sim} H_n(X; \Z) = G \]
Then we need to kill $\pi_{n+1}(X)$ which can be accomplished by adding $(n+2)$-cells along nontrivial generating loops of $X$. By adding such cells inductively following the construction done in class we can kill all homotopy groups $\pi_i(X)$ for $i > n$ to get $X' = K(G, n)$ and $(X')^{n+1} = X^{n+1}$. Now we apply the LES to the CW pair $(X', X)$ to get,
\begin{center}
\begin{tikzcd}
H_{n+1}(X) \arrow[r] & H_{n+1}(X') \arrow[r] & H_{n+1}(X', X)
\end{tikzcd}
\end{center} 
Since $X = M(G, n)$ is a Moore space then $H_{n+1}(X) = 0$. Furthermore, $H_{n+1}(X', X) = 0$ because $(X')^{n+1} = X^{n+1}$ and this cellular homology group in generated on the $X$ skeleton (see Lemma \ref{pair_equal_skeleton_zero_rel_homology} for proof). Therefore, using the above exact sequence,
\[ H_{n+1}(K(G, n) ; \Z) = H_{n+1}(X' ; \Z) = 0 \]
\iffalse
Alternatively, we may apply the Hurewicz theorem to $K(G, n)$ as follows. Since $K(G, n)$ is a $(n-1)$-connected CW complex then $h_{n+1} : \pi_{n+1}(K(G, n)) \onto H_{n+1}(K(G, n); \Z)$ but $\pi_{n+1}(K(G, n)) = 0$ so $H_{n+1}(K(G, n); \Z) = 0$. 
\fi
\subsection{23}

Let $X$ be a $(n-1)$-connected CW complex. Then let $\pi_n(X) = G$ but $\pi_{i}(X)$ will not generically be zero for $i > n$.  We can attach cells in dimensions $n+2$ and above inductively to kill higher homotopy giving a CW complex $Y$ such that $Y^{n+1} = X^{n+1}$ and $Y$ is of $K(G, n)$ type. Now, consider the LES of the CW pair $(Y, X)$ under homology and homotopy using the naturality of the Hurewicz map,
\begin{center}
\begin{tikzcd}
\pi_{n+2}(Y) \arrow[d, "h_{n+2}"]  \arrow[r] & \pi_{n+2}(Y, X) \arrow[d, "h_{n+2}"] \arrow[r] & \pi_{n+1}(X) \arrow[d, "h_{n+1}"] \arrow[r] & \pi_{n+1}(Y) \arrow[d, "h_{n+1}"] \arrow[r] & \pi_{n+1}(Y, X) \arrow[d] \arrow[d, "h_{n+1}"]
\\
H_{n+2}(Y) \arrow[r] & H_{n+2}(Y, X) \arrow[r] & H_{n+1}(X) \arrow[r] & H_{n+1}(Y) \arrow[r] & H_{n+1}(Y, X) 
\end{tikzcd}
\end{center}
First, because $Y \sim K(G, n)$ we know that $\pi_{n+2}(Y) = \pi_{n+1}(Y) = 0$ and thus $\pi_{n+2}(Y, X) \xrightarrow{\sim} \pi_{n+1}(X)$ is an isomorphism. Recall that the map $h_{n+1} : \pi_{n+1}(X) \to H_{n+1}(X)$ sends a map $f : S^{n+1} \to X$ to the class $f_*([S^{n+1}])$. Now, because $Y$ is obtained from $X$ by attaching $n+2$-cells (and higher but $H_{n+2}$ is generated only on the $n+2$-cells) using cellular homology, the map $H_{n+2}(Y, X) \to H_{n+1}(X)$ sends generating $(n+2)$-cells to the homology classes of their boundary $(n+1)$-cells given by the attaching maps i.e. its image is generated by the homology classes $f_*([S^n])$ of the attaching maps $f : S^{n+1} \to X$ for the $(n+2)$-cells (this is usually written in terms of the degree and the generating $(n+1)$-cells in the image but we do not require an explicit form here). Since the Hurewicz map $h_{n+1} : \pi_{n+1}(X) \to H_{n+1}(X)$ takes \textit{every} class of map $f : S^n \to X$ to $f_*([S^n])$ we see that,
\[ \Im{H_{n+2}(Y, X) \to H_{n+1}(X)} \subset \Im{h_{n+1} : \pi_{n+1}(X) \to H_{n+1}(X)} \]
Furthermore, since $\pi_{n+2}(Y, X) \xrightarrow{\sim} \pi_{n+1}(X)$ is an isomorphism, using commutativity of the diagram,
\begin{align*}
\Im{h_{n+1} : \pi_{n+1}(X) \to H_{n+1}(X)} & = \Im{\pi_{n+2}(Y, X) \to \pi_{n+1}(X) \to H_{n+1}(X)} 
\\
& = \Im{\pi_{n+2}(Y, X) \to H_{n+2}(Y, X) \to H_{n+1}(X)}
\\
& \subset \Im{H_{n+2}(Y, X) \to H_{n+1}(X)} 
\end{align*}
Thus, 
\[ \Im{H_{n+2}(Y, X) \to H_{n+1}(X)} = \Im{h_{n+1} : \pi_{n+1}(X) \to H_{n+1}(X)} \]
Now, in the case that $n > 1$, we know that $H_{n+1}(Y ; \Z) = 0$ by the previous problem since $Y$ is a CW complex with homotopy type $K(G, n)$. Thus, the map $H_{n+2}(Y, Z) \to H_{n+1}(X)$ is surjective so $h_{n+1} : \pi_{n+1}(X) \to H_{n+1}(X)$ must be surjective as well. 
\bigskip\\
Now consider the case $n = 1$. By the above image formula we see that, 
\[ H_2(X) / h_2(\pi_2(X)) =  H_2(X) / \Im{H_{3}(Y, X) \to H_{2}(X)} \]
However, by exactness, $\Im{H_3(Y, X) \to H_{2}(X)} = \ker{(H_{2}(X) \to H_{2}(Y))}$. 
Consider the section of the exact sequence,
\begin{center}
\begin{tikzcd}
H_2(X) \arrow[r] & H_2(Y) \arrow[r] & H_2(Y, X) \arrow[r] & H_1(X) \arrow[r] & H_1(Y)
\end{tikzcd}
\end{center}
and $Y^2 = X^2$ so $H_1(X) \xrightarrow{\sim} H_1(Y)$ is an isomorphism since these groups are determined on the $2$-skeleton. Thus $H_{2}(Y, X) \to H_1(X)$ is the zero map and furthermore $H_2(Y, X)$ is generated by the $2$-skeleton $Y^2 = X^2$ so in fact $H_2(Y, X) = 0$ (see Lemma \ref{pair_equal_skeleton_zero_rel_homology} for proof). Therefore, $H_2(X) \to H_2(Y)$ is surjective so,
\[  H_2(X) / h_2(\pi_2(X)) = H_2(X) / \ker{(H_2(X) \to H_{2}(Y))} \cong \Im{H_2(X) \to H_{2}(Y)} = H_2(Y) \]
Finally, $Y$ is of homotopy type $K(G, 1)$ with $G = \pi_1(X)$ so we find,
\[ H_2(X) / h_2(\pi_2(X)) \cong H_2(K(\pi_1(X), 1)) \]

\subsection{31}

Consider a fiber bundle $F \embed E \to B$ such that the inclusion of the fiber $F \embed E$ is nullhomotopic. Then, from the long exact homology sequence, there is an exact sequence,
\begin{center}
\begin{tikzcd}
\pi_n(F) \arrow[r] & \pi_n(E) \arrow[r] & \pi_n(B) \arrow[r, "\partial"] & \pi_{n-1}(F) \arrow[r] & \pi_{n-1}(E)
\end{tikzcd}
\end{center}
However, the map $\pi_n(F) \to \pi_n(E)$ is zero because it is induced by the nullhomotopic inclusion $F \embed E$. Therefore, for $n > 1$ we get a short exact sequence,
\begin{center}
\begin{tikzcd}
1 \arrow[r] & \pi_n(E) \arrow[r] & \pi_n(B) \arrow[r, "\partial"] & \pi_{n-1}(F) \arrow[r] & 1
\end{tikzcd}
\end{center} 
Furthermore, I claim that such sequences are split. To show that such a sequence splits, it suffices to construct a map $g : \pi_{n-1}(F) \to \pi_n(B)$ right-inverse to $\pi_n(B) \to \pi_{n-1}(F)$. 
\bigskip\\
Since $F \embed E$ is nullhomotopic any $f : S^{n-1} \to F$ composes with $F \embed E$ to a nullhomotopic map i.e. $f$ extends to a map $\tilde{f} : D^n \to E$ sending $(D^n, \partial D^n) \to (E, F)$. Now recall that the long exact fibration sequence is proven by using the LES of the pair $(E, F)$ and showing via the homotopy lifting property that $\pi : E \to B$ induces an isomorphism $\pi_* : \pi_n(E, F) \xrightarrow{\sim} \pi_n(B, x_0)$. Thus the extension construction,
\begin{center}
\begin{tikzcd}
S^{n-1} \arrow[d, hook] \arrow[r, "f"] & F \arrow[d, hook]
\\
D^n \arrow[dr, "g(f)"'] \arrow[r, "\tilde{f}"] & E \arrow[d, "\pi_*"]
\\
& B
\end{tikzcd}
\end{center} 
defines a map $g : \pi_{n-1}(F) \to \pi_n(E, F) \xrightarrow{\pi_*} \pi_n(B)$. Then the following commutes,
\begin{center}
\begin{tikzcd}
\pi_{n-1}(F) \arrow[d, "\id"] \arrow[r, "g"] & \pi_n(E, F) \arrow[d, "\pi_*"] \arrow[ld, "\partial"']
\\
\pi_{n-1}(F) & \pi_n(B) \arrow[l, "\partial"]
\end{tikzcd}
\end{center}
And therefore $\pi_* \circ g : \pi_{n-1}(F) \to \pi_n(B)$ gives a splitting map since $\partial \circ \pi_* \circ g = \partial \circ g = \id$ and thus,
\[ \pi_n(B) = \pi_n(E) \oplus \pi_{n-1}(F) \]
\bigskip\\
Now it suffices to show that $g : \pi_{n-1}(F) \to \pi_n(E, F)$ is a well-defined (DO THIS)

\subsection{32}


Suppose that $S^k \hook S^m \hook S^n$ is a fiber bundle. Since locally the fiber bundle is a product of the base with the fiber, by the Invariance of Dimension for open subsets of $\R^n$ (and thus for manifolds), we know that $m = k + n$. This fiber bundle is a Serre fibration so we can consider the exact sequence defined by a fibration,

\begin{center}
\begin{tikzcd}
\cdots \arrow[r] & \pi_4(S^k) \arrow[r] & \pi_4(S^m) \arrow[r] & \pi_4(S^n) \arrow[r] & \pi_3(S^k) \arrow[r] & \pi_3(S^m) \arrow[draw=none]{d}[name=Z, shape=coordinate]{} \arrow[r] & \pi_3(S^n)
\arrow[dlllll,
rounded corners, crossing over,
to path={ -- ([xshift=2ex]\tikztostart.east)
|- (Z) [near end]\tikztonodes
-| ([xshift=-2ex]\tikztotarget.west)
-- (\tikztotarget)}]
\\ 
& \pi_2(S^k) \arrow[r] & \pi_2(S^m) \arrow[r] & \pi_2(S^n) \arrow[r] & \pi_1(S^k) \arrow[r] & \pi_1(S^m) \arrow[r] & \pi_1(S^n) \arrow[r] & \pi_0(S^k)
\end{tikzcd}
\end{center}
however for $i < m = n + k$ we know that $\pi_i(S^m) = 0$ so we get exact sequences,
\begin{center}
\begin{tikzcd}
0 \arrow[r] & \pi_i(S^n) \arrow[r] & \pi_{i-1}(S^k) \arrow[r] & 0
\end{tikzcd}
\end{center}
and therefore $\pi_i(S^n) \cong \pi_{i-1}(S^k)$ for $ i < k + n$ and for $i = k + n$ one side gives zero so we have a surjection $\pi_{n + k}(S^n) \twoheadrightarrow \pi_{n + k -1}(S^k)$. First assume that $n, k > 0$. Suppose that $k < n - 1$ then since $k + 1 < k + n$ set $i = k + 1$ and we have $\pi_{k}(S^k) \cong \pi_{k+1}(S^n) = 0$ which contradicts the fact that $\pi_k(S^k) \cong \Z$. Similarly, if $k > n - 1$ then since $n < n + k$ set $i = n$ and we have $\pi_n(S^n) \cong \pi_{n - 1}(S^k) = 0$ which again contradicts the fact that $\pi_n(S^n) \cong \Z$. Therefore, $k = n - 1$ and thus $m = n + k = 2 n - 1$. In the case that $k = 0$ and $n > 0$ we get an exact sequence,
\begin{center}
\begin{tikzcd}
0 \arrow[r] & \pi_1(S^m) \arrow[r] & \pi_1(S^n) \arrow[r] & \pi_0(S^k) \arrow[r] & 0
\end{tikzcd}
\end{center}
but since $\pi_0(S^k)$ is nontrivial we must have $\pi_1(S^m)$ be nontrivial so $m = 1$ but $m = n + k = n$ so $n = 1$. This case satisfies the formula $k = n - 1$ and $m = 2 n - 1$. Next, consider the case $n = 0$ and $k > 0$. Since $m = n + k > 0$ we know that $S^m$ is connected but $S^n = S^0$ is non connected. Therefore, the map $S^m \hook S^n$ cannot be surjective and therefore the fibers of $S^k$ cannot all be homoeomorphic. Therefore there are no fiber bundles in this case. Finally, in the most trivial case, $k = 0$ and $n = 0$ we would have to have a bundle $S^0 \hook S^0 \hook S^0$ which would require the preimage of both points in $S^0$ to have size two (since the fiber has two elements) which would mean the middle space would have to have at least four elements which it does not. Therefore, there cannot be any fiber bundles in this case either. \bigskip\\
Therefore, any fiber bundle $S^k \hook S^m \hook S^n$ must satisfy $k = n - 1$ and $m = 2n - 1$. 

\section*{Hatcher 4.3}

\subsection{1}

We require a map $f : \RP^\infty \to \CP^\infty$. Since $\CP^\infty = K(\Z, 2)$ we know that,
\[ [\RP^\infty, \CP^\infty] = H^2(\RP^\infty; \Z) = \Z / 2 \Z \]
so we choose a map $f : \RP^\infty \to \CP^\infty$ representing the class of nontrivial maps. 
But $H_2(\RP^\infty ; \Z) = 0$ so the map $f_* : H_2(\RP^\infty ; \Z) \to H_2(\CP^\infty, \Z)$ must be trivial. 
\bigskip\\
This does not contradict the universal coefficient theorem. By naturality of the universal coefficient sequence with respect to the map $\RP^\infty \to \CP^{\infty}$ gives,
\begin{center}
\begin{tikzcd}
0 \arrow[r] & \Ext{1}{\Z}{H_{1}(\CP^\infty; \Z)}{\Z} \arrow[r] \arrow[d, "f^*"] & H^2(\CP^\infty; \Z) \arrow[r] \arrow[d, "f^*"] & \Hom{H_2(\CP^\infty; \Z)}{\Z} \arrow[r] \arrow[d, "f^*"] & 0
\\
0 \arrow[r] & \Ext{1}{\Z}{H_{1}(\RP^\infty; \Z)}{\Z} \arrow[r] & H^2(\RP^\infty; \Z) \arrow[r] & \Hom{H_2(\RP^\infty; \Z)}{\Z} \arrow[r] & 0
\end{tikzcd}
\end{center}
We can compute these groups and find,

\begin{align*}
H_{1}(\RP^\infty; \Z) = \Z/2\Z \quad & \quad H_{1}(\CP^\infty; \Z) = 0
\\
H_{2}(\RP^\infty; \Z) = 0 \quad & \quad H_{2}(\CP^\infty; \Z) = \Z
\\
H^2(\RP^\infty; \Z) = \Z / 2 \Z \quad & \quad H^2(\CP^\infty; \Z) = \Z
\\
\Ext{1}{\Z}{H_{1}(\RP^\infty; \Z)}{\Z} = \Z / 2 \Z \quad & \quad \Ext{1}{\Z}{H_1(\CP^\infty}{\Z} = 0
\end{align*}

Therefore, the above complex gives,
\begin{center}
\begin{tikzcd}
0 \arrow[r] & 0 \arrow[d] \arrow[r] & \Z \arrow[d] \arrow[r] & \Z \arrow[d] \arrow[r] & 0
\\
0 \arrow[r] & \Z / 2 \Z \arrow[r] & \Z / 2 \Z \arrow[r] & 0 \arrow[r] & 0
\end{tikzcd}
\end{center}
which is a perfectly reasonable morphism of exact sequences in which the central downward map (induced on cohomology) is nontrivial yet the rightmost downward map (induced on dual homology) is trivial since the nontriviality is captured in the difference of the Ext groups between these sequences. 

\subsection{3}

Suppose we have a CW complex $X$ and a subcomplex $\iota : S^1 \embed X$ such that the induced map $\iota_* : H_1(S^1; \Z) \to H_1(X;\Z)$ is injective onto a direct summand of $H_1(X; \Z)$. Then consider the projection onto the summand,
\[ \pi : H_1(X; \Z) \to H_1(S^1; \Z) = \Z \]
However, $S^1 = K(\Z, 1)$ we have that,
\[ [X, S^1] = H^1(X; \Z) \]
By the universal coefficient theorem (since $H_0(X;\Z)$ is free) $H^1(X;\Z) = \Hom{H_1(X;\Z)}{\Z}$ so $\pi : H_1(X; \Z) \to \Z$ defines an element $[\pi] \in H^1(X; \Z)$ and thus a map $p : X \to S^1$ up to homotopy such that $p^* [\id] = [\pi]$. Now, the map $p \circ \iota : S^1 \to S^1$, using the identification,
\[ [S^1, S^1] \xrightarrow{\sim}  H^1(S^1; \Z) \]
is uniquely determined by its class $(p \circ \iota)^* [\id] = \iota^* \circ p^* [\id] = \iota^* [\pi] = [\pi \circ \iota_*] = [\id]$ since,
\[ \pi \circ \iota_* = \id : H^1(S^1; \Z) \to H^1(S^1; \Z) \]
Therefore, by injectivity of $[S^1, S^1] \xrightarrow{\sim} H^1(S^1; \Z)$ we get $p \circ \iota \sim \id$ by a homotopy $h : S^1 \times I \to S^1$. By the homotopy extension property of the CW pair $(X, S^1)$ we can extend,
\begin{center}
\begin{tikzcd}
S^1 \arrow[dd, hook, "\id \times \{ 0 \}"'] \arrow[rr, "\iota", hook] & & X \arrow[dd, "p"] \arrow[ld, hook, "\id \times \{ 0 \}"']
\\
& X \times I \arrow[rd, "\tilde{h}", dashed] 
\\
S^1 \times I \arrow[ru, hook, "\iota \times \id"] \arrow[rr, "h"] & & S^1
\end{tikzcd}
\end{center}  
Then $\tilde{p} : \tilde{h}(-, 1) : X \to S^1$ is a map such that $\tilde{p} \circ \iota = h(-,1) = \id$ so $\tilde{p} : X \to S^1$ is a retract. 

\subsection{4}

Given abelian groups $G, H$ and CW complexes $K(G, n)$ and $K(H, n)$. Then consider the map,
\[ \left< K(G, n), K(H, n) \right> \to \Hom{G}{H} \]
via $[f] \mapsto f_* : \pi_n(K(G, n)) \to \pi_n(K(H, n))$. We know that,
\[ \left< K(G, n), K(H, n) \right> \xrightarrow{\sim} H^n(K(G, n); H) \]
Then, by the universal coefficients theorem,
\[ H^n(K(G, n); H) = \Hom{H_n(K(G, n) ; \Z)}{H} = \Hom{G}{H} \]
using that $H_{n-1}(K(G, n); \Z) = 0$ and $\pi_n(K(G,n)) \xrightarrow{\sim} H_n(K(G, n); \Z)$ by Hurewicz. Therefore,
\[ \left< K(G, n), K(H, n) \right> = H^n(K(G, n); H)  = \Hom{G}{H} \]
so it suffices to check that these maps agree with $[f] \mapsto f_*$. First, note that by Hurewicz we have a commutative diagram,
\begin{center}
\begin{tikzcd}
\pi_n(K(G, n)) \arrow[d, "\sim"] \arrow[r, "f_*"] & \pi_n(K(H, n)) \arrow[d, "\sim"]
\\
H_n(K(G, n); \Z) \arrow[r, "f_*"] & H_n(K(H, n); \Z) 
\end{tikzcd}
\end{center}
so it suffices to check that $[f] \mapsto f_*$ on cohomology rather than homotopy groups. Now the composition,
\[ \left< K(G, n), K(H, n) \right> \xrightarrow{\sim} H^n(K(G, n); H) \xrightarrow{\sim} \Hom{H_n(K(G, n) ; \Z)}{H} \xrightarrow{\sim} \Hom{G}{H} \]
is given by $[f] \mapsto f^* [\id] \mapsto \id \circ f_* \mapsto f_*$

\subsection{6}

Note that $K(G, n) \times K(G, n) = K(G^2, n)$ since $\pi_k(X \times Y) \cong \pi_k(X) \times \pi_k(Y)$ naturally. Therefore, it suffices to give a multiplication map $\mu : K(G^2, n) \to K(G, n)$. By the preceding problem, we know,
\[ \left< K(G^2, n), K(G, n) \right> = \Hom{G^2}{G} \]
Now consider the map $\mu_*: G^2 \to G$ via $\mu(g_1, g_2) = g_1 + g_2$ which is a group homomorphism for abelian groups (amusingly since abelian groups are group objects in the category of groups). By the above identification, this determines, up to homotopy, a multiplication map $\mu : K(G, n) \times K(G, n) \to K(G, n)$. To show this structure makes $K(G,n)$ an H-space it suffices to show that $\mu$ is unital up to homotopy. 
\bigskip\\
For the inclusion of the base-point $e : * \to K(G, n)$ we have the following $\mu \circ (e \times \id) : K(G, n) \to K(G, n)$ maps to $\mu_* \circ (e \times \id)$ under the above identification which is the map $g \mapsto \mu_*(e, g) = g$. Thus $\mu_* \circ (e \times \id) = \id$ so by bijectivity of $\left< K(G, n) , K(G, n) \right> \xrightarrow{\sim} \Hom{G}{G}$ we see $\mu \circ (e \times \id) \sim \id$ as based maps. The same argument shows that $\mu \circ (\id \times e) \sim \id$ as based maps so,
\[ \mu : K(G, n) \times K(G, n) \to K(G, n) \]
is a based map unital up to based homotopy giving $K(G, n)$ the structure of an H-space.

\subsection{10}

Let $F \embed E \xrightarrow{p} B$ be a (strong) fibration.
First, consider a path $\gamma : I \to B$ from $x_1$ to $x_2$ and then the diagram,
\begin{center}
\begin{tikzcd}[row sep = large, column sep = large]
F_{x_1} \arrow[r, hook] \arrow[d, hook] & E \arrow[d, "p"]
\\
F_{x_1} \times I \arrow[ru, dashed, "\tilde{\gamma}"] \arrow[r, "\gamma"] & B
\end{tikzcd}
\end{center}
By homotopy lifting we get a map $\tilde{\gamma} : F_{x_1} \times I \to E$ lifting $\gamma : F_{x_1} \times I \to B$. Then $p \circ \tilde{\gamma} = \gamma$ so $\tilde{\gamma}(-, 1) \subset F_{x_2}$ since $p \circ \tilde{\gamma}(-,1) = \gamma(1) = x_2$. Therefore we get a map $[\gamma] : F_{x_1} \to F_{x_2}$ via $[\gamma](x) = \tilde{\gamma}(x, 1)$. 
\bigskip\\
I claim that two lifts of homotopic paths are homotopic. Given two paths $\gamma_1, \gamma_2 : I \to B$ and a path homotopy $h : I^2 \to B$ and two lifts $\tilde{\gamma_1}, \tilde{\gamma_2} : F_{x_1} \times I \to E$ we want a map $F_{x_1} \times I^2 \to E$ above $h : F_{x_1} \times I^2 \to B$. This map is defined on $F_{x_1} \times (I \times \{ 0, 1 \} \cup \{ 0 \} \times I)$ via $\tilde{\gamma}_1$ on $F_{x_1} \times I \times \{0\}$ and $\tilde{\gamma}_2$ on $F_{x_1} \times I \times \{0\}$ any by inclusion of the fiber $F_{x_1}$ on $F_{x_1} \times \{ 0 \} \times I$ (constant on $I$) since $h_ {\{0\} \times I}$ is constant since it is a path homotopy. Then by homotopy lifting, we get $\tilde{h} : F_{x_1} \times I \times I \to E$ such that $p \circ \tilde{h} = h$ and thus $\tilde{h}(-, 1, -) : F_{x_1} \times I \to F_{x_2}$ gives a homotopy from $[\gamma_1] : F_{x_1} \to F_{x_2}$ to $[\gamma_2] : F_{x_1} \to F_{x_2}$. 
\bigskip\\
Therefore, we have a representation of $\Pi(B)$ on $\mathbf{hTop}$ sending $x \mapsto F_x$ and $\gamma \mapsto [\gamma]$. Now, given a loop $\gamma : I \to E$ at $\tilde{x}_1$ above $x_1$ we consider $[p \circ \gamma] : F_{x_1} \to F_{x_1}$. 
\bigskip\\
Given a loop $\gamma : I \to E$ we will apply a slight modification, 
\begin{center}
\begin{tikzcd}[row sep = large, column sep = large]
F_{x_1} \times \{0\} \cup b_0 \times I \arrow[d, hook] \arrow[r, "\iota \cup \gamma"] & E \arrow[d, "p"]
\\
F_{x_1} \times I \arrow[ru, dashed, "\tilde{\gamma}"] \arrow[r, "p \circ \gamma"] & B
\end{tikzcd}
\end{center}
with $b_0 \in F$ the base-point to get a lift $\tilde{\gamma} : F_{x_1} \times I \to E$. This gives an action $\pi_1(E) \to \Aut{F}$ in the homotopy category. Thus we get an action, $\pi_1(E) \to \Aut{F} \to \Aut{\pi_n(F)}$. We need to check that that inclusion,
\[ \pi_1(F) \embed \pi_1(E) \to \Aut{F} \to \Aut{\pi_n(F)} \]
gives the usual action $\pi_1(F) \acts \pi_n(F)$.
\bigskip\\
For any loop $\gamma : I \to F$ at the base point $b_0 \in F$, the lift $\tilde{\gamma}$ satisfies $p \circ \tilde{\gamma} = p \circ \gamma = x_1$ and thus $\tilde{\gamma}$ has image contained in $F$. By construction, $\tilde{\gamma}(b_0, t) = \gamma(t)$ recovering the original loop $\gamma : I \to F$.
\bigskip\\
For a map $f : S^n \to F$ defining some class $[f] \in \pi_n(F)$ the action defines $[\gamma] \cdot [f] = [ \tilde{\gamma}(-,1) \circ f]$. Furthermore, $\tilde{\gamma}$ gives a based homotopy $h : (S^n \vee I) \times I \to F$ from $S^n \vee I \xrightarrow{f \vee \gamma} F$ to $\tilde{\gamma}(-,1)$ via,
\[ h(x, s) = \big( \tilde{\gamma}(f(x), s) \quad \vee \quad h(t, s) = \tilde{\gamma}(b_0, t(1 - s) + s) = \gamma(t(1-s) + s) \big) \]
where $p_0 \in S^n$ is the base-point. This is well-defined since,
 $f(p_0) = b_0$ so 
\[ h(b_0, s) = h(0, s) = \tilde{\gamma}(b_0, s) = \gamma(s) \]
Furthermore, $h(-,0) : S^n \vee I \to F$ is the pair of maps, 
\[ h(x, 0) = \big( \tilde{\gamma}(f(x), 0) = f(x) \quad \vee \quad \gamma(t) \big) \]
which is the standard definition of $[\gamma] \cdot [f] \in \pi_n(F)$ (to be exact we should compose with a homotopy equivalence $S^n \to S^n \vee I$ which can be accomplished by squeezing one hemisphere down to an interval and shrinking the equator to the base-point.)
and $h(-, 1) : S^n \vee I \to F$ is the pair of maps,
\[ h(x, 1) = \big( \tilde{\gamma}(f(x), 1) \vee \gamma(1) \big) \]
which is out definition of the action $[\gamma] \cdot [f]$ (again under the homotopy equivalence $S^n \vee I \to S^n$). Finally, $h$ is a based homotopy (with base point of $S^n \vee I$ chosen to be the unglued end of $I$) since, $h(1, s) = \gamma(1) = b_0$ using that $\gamma : B \to I$ is a loop at $b_0$. Thus, we have shown that the above construction agrees on based homotopy classes of maps $f : S^n \to F$ with the standard action $\pi_1(F) \acts \pi_n(F)$ under $\pi_1(F) \to \pi_1(E) \to \Aut{\pi_n(F)}$.
\bigskip\\
Finally, if $\pi_1(E) = 0$ then the map $\pi_1(F) \to \pi_1(E) \to \Aut{\pi_n(F)}$ must be zero  but this composition recovers the action $\pi_1(F) \acts \pi_n(F)$ so this action must be trivial.


\subsection{15}

Let $p : E \to B$ be a (strong) fibration and also a homotopy equivalence. Then there exists $f : B \to E$ such that $f \circ p \sim \id_E$ and $p \circ f \sim \id_B$.  First consider the homotopy $h : B \times I \to B$ from $p \circ f$ to $\id_B$. By the homotopy lifting property applied to,
\begin{center}
\begin{tikzcd}[row sep = large, column sep = large]
B \arrow[d, hook] \arrow[r, "f"] & E \arrow[d, "p"] 
\\
B \times I \arrow[ru, dashed, "\tilde{h}"] \arrow[r, "h"] & B
\end{tikzcd}
\end{center}
Then $p \circ \tilde{h}(-,1) = h(-,1) = \id_B$ so we get $\tilde{f} = \tilde{h}(-,1) : B \to E$ s.t. $p \circ \tilde{f} = \id_B$. Since $\tilde{h}(-,0) = f$ then $\tilde{h}$ is an explicit homotopy $f \sim \tilde{f}$ so we still have $\tilde{f} \circ p \sim f \circ p = \id_B$.
\bigskip\\
Now consider the homotopy $k : E \times I \to E$ from $\tilde{f} \circ p$ to $\id_E$. Then $p \circ \tilde{f} \circ p = p$ and $p \circ \id_E = p$ so we have $p \circ k(-,0) = p \circ k(-,1) = p$.

Now suppose that a homotopy $q$ satisfied $q(-, t) = k(-, 1-t)$ for $t \le \tfrac{1}{2}$. Then we could make the following construction,
\[ \ell(x, t, s) =
\begin{cases}
q(x, t) & t \le \tfrac{1}{2}(1 - s)
\\
q(x, 1-t) & t \ge \tfrac{1}{2}(1 + s)
\\
q(x, \tfrac{1}{2}(1-s)) & \text{else}
\end{cases} \]
Then at $t = \tfrac{1}{2}(1 - s)$ we have $q(x, t) = q(x, \tfrac{1}{2}(1 - s))$ and at $t = \tfrac{1}{2}(1 + s)$ we have $q(x, t) = q(x, \tfrac{1}{2}(1 - s)) = q(x, \tfrac{1}{2}(1 + s))$. Furthermore, 
\begin{align*}
\ell(x, t, 0) & = q(x, t)
\\
\ell(x, 0, s) & = q(x, 0) = q(x, 1)
\\
\ell(x, 1, s) & = q(x, 0) = q(x, 1)
\\
\ell(x, t, 1) & = q(x, 0) = q(x, 1)
\end{align*}
\bigskip\\
We would like to apply this to $p \circ k$ to get a homotopy between $p \circ k$ and the constant homotopy at $p$ but $k$ does not satisfy this property. Luckily, we can modify it via,
\[ k'(x, t) = 
\begin{cases}
\tilde{f} \circ p \circ k(x, 1-2t) & t \le \tfrac{1}{2}
\\
k(x, 2t-1) & t \ge \tfrac{1}{2} 
\end{cases} \]
Note that $k(x, 0) = \tilde{f} \circ p$ and,
\[ \tilde{f} \circ p \circ k(x, 0) = \tilde{f} \circ p \circ \tilde{f} \circ p = \tilde{f} \circ p \]
Furthermore, $k'(x,0) = \tilde{f} \circ p \circ k(x,1) = \tilde{f} \circ p$ and $k'(x, 1) = k(x, 1) = \id_E$ then $k'$ also is a homotopy $\tilde{f} \circ p$ to $\id_E$. 
However, note that
\begin{align*}
p \circ k'(x, t) & = 
\begin{cases}
p \circ \tilde{f} \circ p \circ k(x, 1-2t) & t \le \tfrac{1}{2}
\\
p \circ k(x, 2t-1) & t \ge \tfrac{1}{2} 
\end{cases}
\\ & =
\begin{cases}
p \circ k(x, 1-2t) & t \le \tfrac{1}{2}
\\
p \circ k(x, 2t-1) & t \ge \tfrac{1}{2} 
\end{cases}
\end{align*}
and thus $p \circ k'$ satisfies out property $p \circ k'(-, t) = q \circ k'(-, 1 - t)$. And thus we can use the above construction to get $\ell : E \times I \times I \to B$ such that,
\begin{align*}
\ell(x, t, 0) &= p \circ k'(x, t)
\\
\ell(x, t, 1) &= p \circ k'(x, 0) = p
\\
\ell(x, 0, s) &= p \circ k'(x, 0) = p
\\
\ell(x, 1, s) &= p \circ k'(x, 0) = p
\end{align*}
Thus $\ell$ gives a homotopy $p \circ k'$ to $p$. 
Therefore, consider the homotopy lifting diagram,
\begin{center}
\begin{tikzcd}[row sep = huge, column sep = huge]
E \times (I \times \{ 0 \} \cup \{ 0\} \times I \cup \{1\} \times I) \arrow[d, hook] \arrow[r, "r"] & E \arrow[d, "p"]
\\
E \times I \times I \arrow[ru, dashed, "\tilde{\ell}"] \arrow[r, "\ell"] & B
\end{tikzcd}
\end{center} 
where $r : E \times (I \times \{ 0 \} \cup \{ 0\} \times I \cup \{1\} \times I) \to E$ is given by $r = k' \cup \tilde{f} \circ p \cup \id_E$ so 
\begin{align*}
r(x, t, 0) & = k'(x, t)
\\
r(x, 0, s) & = \tilde{f} \circ p
\\
r(x, 1, s) & = \id_E
\end{align*}
which is continuous since $k'$ is a homotopy from $\tilde{f} \circ p$ to $\id_E$ and thus,
\begin{align*}
p \circ r(x, t, 0) & = p \circ k'(x, t)
\\
p \circ r(x, 0, s) & = p \circ \tilde{f} \circ p = p
\\
p \circ r(x, 1, s) & = p \circ \id_E = p
\end{align*}
so $p \circ r$ agrees with $\ell$. Then $\tilde{\ell} : E \times I \times I \to E$ satisfies
\begin{align*}
\tilde{\ell}(x, t, 0) & = k'(x, t)
\\
\tilde{\ell}(x, 0, s) & = \tilde{f} \circ p
\\
\tilde{\ell}(x, 1, s) & = \id_E
\\
p \circ \tilde{\ell}(x, t, 1) & = \ell(x, t, 1) = p 
\end{align*}
Thus, $s = \tilde{\ell}(x, t, 1) : E \times I \to E$ is a homotopy $s(-,0) = \tilde{f} \circ p$ to $s(-,1) = \id_E$ such that $p \circ s = p$ i.e. a fiber homotopy equivalence $\tilde{f} \circ p$ to $\id_E$. Furthermore, $p \circ \tilde{f} = \id_B$ so $p : E \to B$ is a homotopy fiber equivalence,
\begin{center}
\begin{tikzcd}
E \arrow[rd, "p"] \arrow[r, "p"] & B \arrow[d, "\id"]
\\
& B
\end{tikzcd}
\end{center}
between the bundles $p : E \to B$ and $\id : B \to B$. 

\section{Lemmas}

\begin{lemma} \label{pair_equal_skeleton_zero_rel_homology}
Suppose that $(X, A)$ is a CW pair with $X^n = A^n$ then $H_n(X, A) = 0$. 
\end{lemma}

\begin{proof}
Since $(X, A)$ is a good pair $H_n(X, A) \cong \tilde{H}_n(X/A)$. But $X/A$ has trivial $n$-skeleton so $\tilde{H}_n(X/A) = 0$ since it is generated by the $n$-cells. This will also follow from the same result for the pair $(D^{n+1}, S^n)$ which satisfies the assumption. From the LES of the pair,
\begin{center}
\begin{tikzcd}
H_{n}(S^n) \arrow[r] & H_n(D^{n+1}) \arrow[r] & H_{n}(D^{n+1}, S^n) \arrow[r] & H_{n-1}(S^n) \arrow[r] & H_{n-1}(D^{n+1})
\end{tikzcd}
\end{center}
but $H_n(D^{n+1}) = 0$ and $H_{n-1}(S^n) = 0$ so $H_{n}(D^{n+1}, S^n) = 0$ and we may use this result to build up inductively to show that  $H_n(X, A) = 0$. 
\end{proof}

\section{The Yoneda Embedding}

\newcommand{\C}{\mathcal{C}}
\newcommand{\D}{\mathcal{D}}
Natural transformations are the morphism in the category of functors between to fixed categories i.e. they are maps between functors.

\begin{defn}
Let $F, G : \C \to \D$ be functors. Then a natural tranformation $\eta : F \to G$ is for each object $A \in \C$ a morphism $\eta_A : F(A) \to G(A)$ such that for any morphism $f : A \to B$ in $\C$ there is a commutative square,
\begin{center}
\begin{tikzcd}
F(A) \arrow[d, "\eta_A"] \arrow[r, "f_*"] & F(B) \arrow[d, "\eta_B"]
\\
G(A) \arrow[r, "f_*"] & G(B)
\end{tikzcd}
\end{center}
\end{defn}

\begin{rmk}
The basic idea behind the Yoneda Embedding is the following special fact about natural transformations between hom functors. 
\end{rmk}

\begin{lemma}
Let $\eta : \Hom{A}{-} \to \Hom{B}{-}$ be a natural transformation. Then $\eta$ is uniquely determined by $\eta_A(\id_A)$ via $\eta_X(f) = f \circ \eta_A(\id_A)$ for any $f \in \Hom{A}{X}$.  
\end{lemma}

\begin{proof}
Let $f : A \to X$ be some map.
Consider the naturality diagram,
\begin{center}
\begin{tikzcd}
\Hom{A}{A} \arrow[r, "f_*"] \arrow[d, "\eta_A"] & \Hom{A}{X} \arrow[d, "\eta_X"] 
\\
\Hom{B}{A} \arrow[r, "f_*"] & \Hom{B}{X} 
\end{tikzcd}
\end{center}
Consider the element $\id_A \in \Hom{A}{A}$ which, under the upper path, maps to $\eta_X (f_*(\id_A)) = \eta_X(f \circ \id_A) = \eta_X(f)$ and, under the lower path, $f_*(\eta_A(\id_A)) = f \circ \eta_A(\id_A)$. Therefore,
\[ \eta_X(f) = f \circ \eta_A(\id_A) \]
\end{proof}

\begin{corollary}
Natural transformations $\eta : \Hom{A}{-} \to \Hom{B}{-}$ are in one-to-one correspondence with functions $\Hom{B}{A}$. We say $f^*$ is the natural transformation $f^*_X(g) = g \circ f$ for any $g \in \Hom{A}{X}$.  
\end{corollary}

\begin{rmk}
This remark shows that the functor $\Hom{A}{-}$ preserves properties of the object $A$ in some sense which allows us to define an embedding (i.e. an injective and fully faithful functor) of the category $\C$ into a larger category of functors.
\end{rmk}

\begin{theorem}
Let $\C$ be any category. The functor $Y : \C^{\op} \to \Set^{\C}$ sending $A \mapsto h^A$ where $h^A = \Hom{A}{-}$ and $f \mapsto f^*$ described above is fully faithful.
\end{theorem}

\begin{proof}
Clearly $(\id_A)^* = \id_{h^A}$ since $(\id_A)^*(f) = f \circ \id_A = f$ and for $f \in \Hom{B}{A}$ and $g \in \Hom{C}{B}$ then $(f \circ g)^* = g^* \circ f^*$ since for any $q \in \Hom{A}{X}$ we send,
\[ (f \circ g)^*(q) = q \circ (f \circ g) = (q \circ f) \circ g = g^*(f^*(q)) \]
The above corollary proves that $Y$ is fully faithful.    
\end{proof}

\begin{rmk}
The Yoneda embedding can be viewed as a generalization of Cayley's theorem from group theory. Indeed, any group $G$ is a Category on one element and $\Set^G$ is the Category of $G$-sets. The Yoneda embedding sends the single point $* \in G$ to the set $G$ with the left $G$-action given by multiplication on the left. Therefore, we see that $G$ is embedded into the set of permutations of $G$. 
\end{rmk}

\begin{rmk}
The fact that the Yoneda map is fully faithful is very important for the following reason.
\end{rmk}

\begin{lemma}
Let $F : \C \to \D$ be fully faithful then $X \cong Y \iff F(X) \cong F(Y)$.
\end{lemma}

\begin{proof}
If $F(X) \cong F(Y)$ then there are morphisms $f \in \Hom{F(X)}{F(Y)}$ and $g \in \Hom{F(Y)}{F(X)}$ which are inverses. However, since $F$ is full there exist morphisms $\tilde{f} : \Hom{X}{Y}$ and $g \in \Hom{Y}{X}$ such that $F(\tilde{f}) = f$ and $F(\tilde{g}) = g$. Then,
\[ F(\tilde{f} \circ \tilde{g}) = F(\tilde{f}) \circ F(\tilde{g}) = f \circ g = \id_{F(Y)} \quad \text{and} \quad F(\tilde{g} \circ \tilde{f}) = F(\tilde{g}) \circ F(\tilde{f}) = g \circ f = \id_{F(X)} \]
However, since $F$ is faithful then,
\[ \tilde{f} \circ \tilde{g} = \id_Y \quad \text{and} \quad \tilde{g} \circ \tilde{f} = \id_X \]
proving that $X \cong Y$. 
\end{proof}

\begin{rmk}
In particular, applying this Lemma to $Y : \C \to \Set^\C$ alows us to check if two objects are equal by looking at their functor of maps.
\end{rmk}

\begin{example}
For example, if $F : \C \to \D$ has a left adjoint $G : \D \to \C$ then $G$ preserves colimits. To see this we use the Yoneda lemma,
\begin{align*}
\Homover{\C}{G(\colim J)}{-} & = \Homover{\C}{\colim J}{F(-)} = \lim \Homover{\C}{J}{F(-)} 
\\
& = \lim \Homover{\C}{F(J)}{-} = \Homover{\C}{\colim G(J)}{-} 
\end{align*}
Since their associated hom functors are equivalent, we find that $G(\colim J) \cong \colim G(J)$.
\end{example}

\begin{definition}
We say a functor $F : \C \to \Set$ is \textit{representable} if $F \cong h^A$ for some $A \in \C$. 
\end{definition}

\begin{rmk}
The subcategory of representable functors in $\Set^\C$ is exactly the image of $Y$.
\end{rmk}

\end{document}
