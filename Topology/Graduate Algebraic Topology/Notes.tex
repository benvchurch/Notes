\documentclass[12pt]{extarticle}
\usepackage{import}
\import{./}{Includes}

\newcommand{\GL}[2]{\mathrm{GL}_{#1} \left( #2 \right)}
\renewcommand{\H}{\mathbb{H}}

\begin{document}

\section{Topics}

\begin{enumerate}
\item Basic homotopy theory
\item Obstruction theory
\item Characteristic Classes
\item The Serre spectral sequence
\item The Steenrod operations
\item K-theory
\end{enumerate}

References: Fuchs - Fomenko: homotopical topology, Hatcher's books

Six homeworks (one per topic)

\section{Homotopy Theory}

Basic Questions: 

\begin{enumerate}
\item given maps $f, g : X \to Y$ are they homotopy equivalent?
\item given spaces $X$ and $Y$ are they homotopy equivalent? 
\end{enumerate}

\begin{remark}
All spaces will be connected and locally connected. 
\end{remark}

\begin{definition}
The set $[X, Y] = \Hom{\mathbf{hTop}}{X}{Y}$. Given based spaces $X, Y$ we define $\left< X, Y \right> = \Hom{\mathbf{hTop}_{\bullet}}{X}{Y}$ where morphisms in $\mathbf{hTop}_\bullet$ are continuous maps preserving the basepoint up to homotopy. Note that homotopies in $\mathbf{Top}_\bullet$ are basepoint preserving.
\end{definition}

\begin{example}
Consider $S^n$. Given $f : S^n \to X$ we can construct, $X \sqcup_f D^{n+1}$ by gluing along $f$. This is the coproduct,
\begin{center}
\begin{tikzcd}
D^{n+1} \arrow[r] & X \sqcup_f D^{n+1}
\\
S^n \arrow[r, "f"] \arrow[u] & X \arrow[u]
\end{tikzcd}
\end{center}
Now if $f \sim f'$ then $X \sqcup_f D^{n+1} \sim X \sqcup_f D^{n+1}$. 
\end{example}

\begin{definition}
Given a based space $(X, x_0)$ we define the $n^{\mathrm{th}}$ homotopy group,
\[ \pi_n(X, x_0) = \left< (S^n, p_0), (X, x_0) \right> \]
The group structure is given by the equator squeezing map $s : S^n \to S^n \vee S^n$. Then we define $f * g = (f \vee g) \circ s$. 
\end{definition}

\begin{proposition}
$\pi_n(X, x_0)$ is abelian for $n \ge 2$. 
\end{proposition}


\begin{theorem}
$\pi_n(S^m) = 0$ if $n < m$.
\end{theorem}

\begin{theorem}
$\pi_n(S^n) = \Z$
\end{theorem}

\begin{theorem}
$\pi_3(S^2) = \Z$ generated by the Hopf fibration $\eta : S^3 \to S^2$. 
\end{theorem}

\begin{theorem}
For sufficiently large $n$,
\[ \pi_{n+1}(S^n) = \Z / 2 \Z \quad \quad \pi_{n + 2}(S^n) = \Z / 2 \Z \quad \quad \pi_{n+3}(S^3) = \Z / 2 4 \Z \]
\end{theorem}

\begin{remark}
Given $f : X \to Y$ we get $f_* : \pi_n(X) \to \pi_n(Y)$. 
\end{remark}

\begin{theorem}
Given a path $\gamma : x_1 \to x_2$ in $X$ we get a map,
\[ \gamma_{\#} : \pi_n(X, x_1) \to \pi_n(X, x_2) \]
depending only on the homotopy class of $\gamma$. 
In particular we have a $\pi_1(X, x_0)$-action on $\pi_n(X, x_0)$.
\end{theorem}

\begin{remark}
In the case $n = 1$ this is the conjugation action of $\pi_1(X, x_0)$ on itself.  
\end{remark}

\begin{proposition}
Given the previous proposition, we have,
\[ [S^n, X] = \pi_n(X, x_0) / \pi_1(X, x_0) \]
\end{proposition}

\begin{proposition}
If $p : \tilde{X} \to X$ is a covering map then for $n \ge 2$ the induced map,
\[ p_* : \pi_n(\tilde{X}) \to \pi_1(X) \]
is an isomorphism.
\end{proposition}

\begin{proof}
Injectivity is the homotopy lifting property. Furthermore given $f : S^n \to X$ we can lift it to $\tilde{f} : S^n \to \tilde{X}$ provided that $f_*(\pi_1(S^n)) \subset p_*(\pi_1(\tilde{X}))$. In the case $n \ge 2$, we have $\pi_1(S^n)$ thus such a lift always exists proving surjectivity. 
\end{proof}

\begin{example}
Let $\Sigma_g$ be a genus $g$ surface. For $g \ge 1$ then $\Sigma_g$ has universal cover $\R^2$ which is contractible and thus $\pi_n(\Sigma_g) = \pi_n(\R^2) = 0$ for $n \ge 2$. 
\end{example}

\begin{example}
For $n \ge 2$ we have $\pi_n(\RP^k) = \pi_n(S^k)$. 
\end{example}

\subsection{Basic Operations on Spaces}

\begin{definition}
The suspension of $X$ is $\Sigma X = X \vee S^1$.
\end{definition}

\begin{definition}
The loops space of $X$ is $\Omega X = \Hom{\mathbf{Top}_\bullet}{S^1}{X}$ with the compact-open topology. 
\end{definition}

\begin{theorem}[Adjunction]
\[ \left< \Sigma X, Y \right> = \left< X, \Omega Y \right> \]
\end{theorem}

\begin{example}
$\Sigma S^n = S^{n + 1}$
\end{example}

\begin{proposition}
$\pi_{n+1}(Y) = \left< S^{n+1}, Y \right> = \left< \Sigma S^n, Y \right> = \left< S^n, \Omega Y \right> = \pi_n(\Omega Y)$
\end{proposition}

\begin{proposition}
The space $\Omega X$ is a group object in the category $\mathbf{hTop}_\bullet$. 
\end{proposition}

\begin{remark}
The following definition is due to Hatcher. 
\end{remark}

\begin{definition}
A pointed space $(X, e, \mu)$ is an H-space is there is a map $\mu : X \times X \to X$ such that $\mu(-, e) \sim \id$ and $\mu(e, -) \sim \id$ as pointed maps (relative to the basepoint).  
\end{definition}

\begin{remark}
Any topological group (group object in $\Top$) is an H-space (pointed at the identity element). 
\end{remark}

\begin{remark}
Loop spaces are H-spaces since they are group objects in $\phTop$. 
\end{remark}

\begin{theorem}[Adams]
The spheres $S^n$ admitting an H-space structure are exactly $S^0, S^1, S^3, S^7$. 
\end{theorem}

\begin{corollary}
$\R^n$ has a unital division $\R$-algebra structure iff $n = 1,2,4,8$. 
\end{corollary} 

\begin{proof}
Consider the unit length elements $U = S^{n-1}$. Then a division algebra on $\R^n$ gives a multiplication $U \times U \to U$ (well defined since $xy = 0 \implies x = 0 \text{ or } y = 0$ and thus the result can be scalled to lie in $U$). 
\end{proof}

\section{Relative Groups}

\begin{definition}
Given a space $X$ a subspace $A \subset X$ and a point $x_0 \in A$ we denote the pointed pair as $(X, A, x_0)$. 
\end{definition}

\begin{definition}
For a pointed pair $(X, A, x_0)$ we define $\pi_n(X, A, x_0)$ as maps,
\[ f : (D^n, S^{n-1}, p_0) \to (X, A, x_0) \]
modulo homotopy through maps of this form. 
\end{definition}

\begin{remark}
Suppose $[f] \in \pi_n(X, A, x_0)$ is zero if it is homotopic to a map with image inside $A$. In fact if this is the case then $f$ may be homotoped relative to the boundary. Compression Lemma. 
\end{remark}

\begin{theorem}
There is a long exact sequence for the pointed pair $(X, A, x_0)$,
\begin{center}
\begin{tikzcd}
\cdots \arrow[r] & \pi_{n}(A, x_0) \arrow[r] \arrow[draw=none]{d}[name=Z, shape=coordinate]{} & \pi_{n}(X, x_0) \arrow[r] & \pi_{n}(X, A, x_0) 
\arrow[dll,
rounded corners, crossing over,
to path={ -- ([xshift=2ex]\tikztostart.east)
|- (Z) [near end]\tikztonodes
-| ([xshift=-2ex]\tikztotarget.west)
-- (\tikztotarget)}]
& &
\\ 
& \pi_{n-1}(A, x_0) \arrow[r] & \pi_{n-1}(X, x_0) \arrow[r] & \pi_{n-1}(X, A, x_0) \arrow[r] & \cdots
\end{tikzcd}
\end{center}
\end{theorem}

\section{Results on CW Complexes}

\begin{definition}
A CW pair is a CW complex $X$ with a subcomplex $A \subset X$ (a closed subset which is a cunion of cells e.g.$X^k$ the $k$-skelleton). 
\end{definition}

\begin{theorem}[homotopy extension]
Let $(X, A)$ be a CW pair. Then $(X, A)$ has the homotopy extension property i.e. $\iota : A \to X$ is a cofibration. 
\end{theorem}

\begin{proof}
Working cell-by-cell we can reduce to the case $(X, A) = (D^n, S^{n-1})$. In this case we are given a map on $D^n \times \{ 0 \} \cup S^{n-1} \times I$ which is a deformation retract of $D^n \times I$ so any map can be extended. 
\end{proof}

\begin{definition}
A map $f : X \to Y$ between CW complexes is \textit{cellular} if $f(X^k) \subset Y^k$.
\end{definition}

\begin{theorem}[cellular approximation]
Any map $f : X \to Y$ of CW complexes is homotopic to a cellular map.
\end{theorem}

\begin{corollary}
If $n < m$ then $\pi_n(S^m) = 0$. 
\end{corollary}

\begin{theorem}
If $\pi_i(X, x_0) = 0$ for $i \le n$ (i.e. $X$ is $n$-connected) then $X$ is homotopic to a CW complex with a single zero $0$-cell and no $i$-cells for $1 \le i \le n$. 
\end{theorem}

\begin{lemma}
If $(X, A)$ is a CW-pair and $A$ is contractible then $X \to X / A$ is a homotopy equivalence. 
\end{lemma}


\section{More Results on CW Complexes (01/29)}

\begin{theorem}[Whitehead]
Let $f : X \to Y$ be a map of CW complexes such that $f_* : \pi_n(X, x_0) \to \pi_n(Y, y_0)$ is an isomorphism for each $n$ then $f$ is a homotopy equivalence. 
\end{theorem}


\begin{example}
If $\pi_n(X, x_0) = 0$ for all $n \ge 0$ and $X$ is a CW complex then $X$ is contractible. To see this consider the constant map $X \to *$. 
\end{example}

\begin{example}
Consider $S^\infty = \varinjlim S^n$ where we consider $S^n \subset S^{n+1}$ as the equator. Then $\pi_n(S^\infty) = 0$ since any map $S^n \to S^{\infty}$ can be deformed to a point using the copy of $S^{n+1}$. Thus $S^\infty$ is contractible. 
\end{example}

\begin{remark}
In Whitehead's theorem, simply knowing $\pi_n(X) \cong \pi_n(Y)$ for each $n \ge 0$ does not imply $X \sim Y$ we need these isomorphisms to be induced by a single topological map $f : X \to Y$. 
\end{remark}

\begin{example}
Quotienting by the natural involution on $S^\infty$ we get a double cover $p : S^\infty \to \RP^{\infty}$. Using covering theory we find,
\[ \pi_n(\RP^\infty) 
= \begin{cases}
\Z / 2 \Z & n = 1
\\
0 & n > 1
\end{cases} \]
Furthermore, consider $X = S^2 \times \RP^\infty$ whose universal cover is $\tilde{X} = S^2 \times S^{\infty} \sim S^2$ and thus,
\[ \pi_n(X) 
= \begin{cases}
\Z / 2 \Z & n = 1
\\
\Z & n = 2
\\
0 & n > 1
\end{cases} \]
This has exactly the same homotopy groups as $Y = \RP^2$ whose universal vover is also $\tilde{X} = S^2$ and also has a two-fold cover. However, 
$H_*(\RP^2, \Z / 2 \Z)$ is finite dimensional and $H_*(S^2 \times \RP^\infty, \Z / 2 \Z)$ is infinite dimensional so they cannot be homotopy equivalent. 
\end{example}


\begin{definition}
The mapping cylinder of a morphism $f : X \to Y$ is the pushout,
\[ Mf = Y \coprod_f (X \times I) \]
There is a natural inclusion $\iota : X \embed Mf$ and a deformation retract $j : Mf \to Y$.  
\end{definition}

\begin{remark}
If $X$ and $Y$ are CW complexes then we may homotope $f : X \to Y$ to a cellular map in which case $Mf$ is a CW complex and $\iota : X \embed M(f)$ makes $(Mf, X)$ a CW pair. 
\end{remark}

\begin{definition}
If $X$ and $Y$ are any spaces $f : X \to Y$ is a \textit{weak homotopy equivalence} if $f_* : \pi_n(X) \to \pi_n(Y)$ is an isomorphism for all $n \ge 0$. 
\end{definition}

\begin{theorem}
Any space is weakly homotopy equivalent to a CW complex. 
\end{theorem}

\begin{remark}
Suspension is a functor: given $f : X \to Y$ we get $\Sigma f : \Sigma X \to \Sigma Y$ given by $\Sigma f(t, x) = (t, f(x))$. 
\end{remark}

\begin{remark}
The unit of the suspension-looping adjunction gives a map $X \to \Omega \Sigma X$ given by $x \mapsto (t \mapsto (t, x))$. Applying the functor $\pi_n$ gives the Freudenthal map $\sigma_n : \pi_n(X) \to \pi_{n+1}(\Sigma X)$. 
\end{remark}

\begin{theorem}[Freudenthal Suspension]
Let $X$ be an $n$-connected pointed space. Then the Freudenthal map $\Sigma_k : \pi_k(X) \to \pi_{k+1}(\Sigma X)$ is an isomorphism if $k \le 2n$ and an epimorphism if $k = 2 n + 1$. 
\end{theorem}

\begin{corollary}
$\pi_n(S^n) = \Z$.
\end{corollary}

\begin{proof}
We show this by induction. For $n = 1$ the result $\pi_1(S^1) = \Z$ is a simple application of covering space theory. Now we assume the result for $S^n$. Then since $S^n$ is $(n-1)$-connected, by the Fruedenthal suspension theorem we get an isomorphism $\pi_k(S^n) \xrightarrow{\sim} \pi_{k+1}(S^{n+1})$ for $k < 2n - 1$. Setting $k = n$ we see that $\pi_{n+1}(S^{n+1}) \cong \pi_{n}(S^n)$ for $n > 1$. However, for the case $n = 1$ we only get an epimorphism $\pi_1(S^1) \to \pi_2(S^2)$ since $1 = 2 - 1$. However, there is a surjective degree map $\pi_2(S^2) \to \Z$ and thus $\pi_2(S^2) = \Z$.  
\end{proof}

\section{Spectra}

\begin{definition}
A spectrum is a sequence $X_n$ of CW complexes along with structure maps $s_n : \Sigma X_n \to X_{n+1}$. 
\end{definition}

\begin{definition}
Let $X$ be a spectrum then we define the homotopy groups of $X$ via,
\[ \pi_k(X) = \varinjlim_n \pi_{k + n}(X_n) \]
where the maps $\Sigma X_n \to X_{n+1}$ induce $\pi_{k + n}(X_n) \to \pi_{k + n +1}(X_{n+1})$ by adjunction making the groups $\pi_{k + n}(X_n)$ a directed system.
\end{definition}

\begin{remark}
Spectra may have homotopy in negative dimension i.e. $\pi_k(X) \neq 0$ for $k \le 0$ in general. 
\end{remark}

\begin{definition}
We say a spectrum is stable if the structure maps are eventually all weak homotopy equivalences. 
\end{definition}

\begin{example}
Given a CW complex $X$ we can form the suspension specturm $X_n = \Sigma^n X = S^n \wedge X$ with identity maps $\Sigma X_n \to X_{n+1}$. This is clearly a stable spectrum.
\end{example}

\renewcommand{\S}{\mathbf{S}}

\begin{example}
The suspension spectrum of $S^0$ is the sphere spectrum $\S$ given by $\S_n = S^n$ with the natural homeomorphisms $\Sigma S^n \to S^{n+1}$. 
\end{example}

\begin{definition}
An $\Omega$-spectrum is a specturm $X$ such that the adjunction of the structue map $X_n \to \Omega X_{n+1}$ is a weak homotopy equivalence. 
\end{definition}

\section{Feb 12}

\begin{theorem}
Two CW complexes of type $K(G, n)$ are homotopy equivalent.
\end{theorem}

\begin{proof}
Let $X, Y$ be CW complexes. Assume that $X$ has no $1, \dots, (n-1)$-cells (since it is $(n-1)$-connected) and one $0$-cell (since it is connected). Then,
\[ X^n = \bigvee_{i \in I} S^n \]
each of these spheres represents an element $\pi_n(X) = G$. Construct $f_n : X^n \to Y$ by sending each $S^n$ to the corresponding element in $\pi_n(Y) = G$. Next construct $f_{n+1} : X^{n+1} \to Y$ so that each $\partial D^{n+1} = S^n \xrightarrow{f_n} Y$ represents $0 \in \pi_n(Y)$ (since the $(n+1)$-cells give the relations on $G$) then $\partial D^{n+2} = S^{n+1} \xrightarrow{f_{n+1}} Y$ is nullhomotopic because $\pi_{n+1}(Y) = 0$. Repeating, we can extend to all $X$. 
\end{proof}

\begin{remark}
Key point: $\pi_n(X)$ is generated by $n$-cells and has relations by $(n+1)$-cells. This is a first glimpse of obstruction theory. We ask the following questions:
\begin{enumerate}
\item[Q1] Given a CW pair $(X, A)$ and $f : A \to Y$ can we extend this to $\tilde{f} : X \to Y$?
\item[Q2] Given a giber bundle $p : E \to B$ and a map $f : X \to B$ can we lift it to $\tilde{f} : X \to E$?
\end{enumerate}
For Q1, assume that $\pi_1(Y) \acts \pi_n(Y)$ trivially (i.e. $Y$ is simple so we need not worry about basepoints!). Given $f : X^n \to Y$ can we extend it to $X^{n+1}$? Gluing a disk $D^{n+1}$ then $f$ extends to $D^{n+1}$ iff $f|_{S^n} : S^n \to Y$ is nullhomotopic i.e. is zero in $\pi_n(Y)$. In general, to each $(n + 1)$-cell $e$, $[f_e] \in \pi_n(Y)$ then we can construct $c_f \in C^{n+1}(X, \pi_n(Y))$ a cellular cochain called the obstruction cochain. Then $f$ extends to $X^{n+1} \iff c_f = 0$. 
\end{remark}

\begin{lemma}
$\delta c_f = 0$ i.e. $c_f$ is a cocycle. Therefore, $O_f := [c_f] \in H^{n+1}(X; \pi_n(Y))$ is the obstructuon class. 
\end{lemma}

\begin{theorem}
$f|_{X^{n-1}}$ extends to $X^{n+1}$ iff $O_f = 0$.  
\end{theorem}

\begin{proof}
First we prove the Lemma. Consider the diagram,
\begin{center}
\begin{tikzcd}
C_{n+2}(X) \arrow[dd, "\partial"] \arrow[r, equals] & H_{n+2}(X^{n+2}, X^{n+1}) \arrow[r, "h^{-1}"] & \pi_{n + 2}(X^{n+2}, X^{n+1}) \arrow[d, "\partial"]
\\
& & \pi_{n+1}(X^{n+1}) \arrow[d, "\iota"]
\\
C_{n+1}(X) \arrow[rrdd, "c_f"] \arrow[r, equals] & H_{n+1}(X^{n+1}, X^n) \arrow[r, "h^{-1}"] & \pi_{n+1}(X^{n+1}, X^n) \arrow[d, "\partial"] 
\\
& & \pi_n(X^n) \arrow[d, "f_*"]
\\
& & \pi_n(Y)
\end{tikzcd}
\end{center}
The piece of the LES,
\begin{center}
\begin{tikzcd}
\pi_{n+1}(X^{n+1}) \arrow[r] & \pi_{n+1}(X^{n+1}, X^n) \arrow[r] & \pi_n(X^n)
\end{tikzcd}
\end{center}
composes to zero so by the commutativity of the above diagram $c_f \circ \partial = 0$. 
\end{proof}

\begin{definition}
Suppose there are two maps $f, g : X^n \to Y$ that agree on $X^{n-1}$ then for each $n$-cell $D^n$ if we glue two $D^n$ along the boundary on which $f,g$ agree then we get a map $(f, g) : S^n \to Y$ and thus an element $\pi_n(Y)$ for each $n$-cell. This gives a difference cochain $d_{f,g} \in C^n(X ; \pi_n(Y))$ and $d_{f,g} = 0$ iff $f,g : X^n \to Y$ are homotopic relative to $X^{n+1}$.  
\end{definition}

\begin{lemma}
$\delta d_{f,g} = c_g - c_f$.
\end{lemma}

\begin{lemma}
Given $f : X^n \to Y$ for any $d \in C^n(X; \pi_n(Y))$ there is $g : X^n \to Y$ with $f|_{X^{n-1}} = g|_{X^{n-1}}$ s.t. $d_{f,g} = d$. 
\end{lemma}

\begin{proof}
For $d \in C^n(X; \pi_n(Y))$ then for an $n$-cell $e$ we have $d(e) \in \pi_n(Y)$ then consider the sum of maps $f$ and $d(e)$ using the sum structure on $e$ contracting the equator. 
\end{proof}

\begin{proof}
Now we prove the theorem. Suppose that $O_f = 0$ then $c_f = \delta d$ for some $d \in C^n(X ; \pi_n(Y))$. Now there exists $g : X^n \to Y$ with $f|_{X^{n-1}} = f|_{X^{n-1}}$ and $d_{f,g} = -d$. Also, $\delta d_{f,g} = c_g - c_f$ and thus $c_g = c_f + \delta d_{f,g} = c_f - \delta d = 0$ therefore $c_g = 0$ so $g$ can extend to $X^{n+1}$ and $f|_{X^{n-1}} = g|_{X^{n-1}}$. 
\end{proof}

\begin{theorem}
Let $f,g : X^n \to Y$ be maps with $f|_{X^{n-2}} = g|_{X^{n-2}}$. Then $[d_{f,g}] = 0$ iff they are homotopic relative to $X^{n-2}$. 
\end{theorem}

\subsection{Cohomology of $K(G, n)$}

Let $n \ge 2$ and $G$ abelian. Consider a map $f : X \to K(G, n)$. By Hurewicz, $H_n(K(G, n), \Z) = \pi_n(K(G, n)) = G$ and $H_{n-1}(K(G, n), \Z) = 0$. Now, by the universal coefficient theorem,
\[ H^n(K(G, n), G) = \Hom{H_n(K(G, n), \Z)}{G} = \Hom{G}{G} \]
Therefore, there is a canonical element $\mathbbm{1} \in H^n(K(G, n), G)$ which is the class of $\id : G \to G$. 
\bigskip\\
Also, via $f : X \to K(G, n)$, we also get $f^*(\mathbbm{1}) \in H^n(X ; G)$, which depends only on the homotopy class of $f$. 

\begin{theorem}
The map $[X, K(G, n)] \to H^n(X, G)$ sending $[f] \mapsto f^*(\mathbbm{1})$ is an isomorphism. 
\end{theorem}

\begin{rmk}
Notice that (at least for $n > 1$) there is no distinction between the unbased mapping classes $[X, K(G, n)]$ and based mapping classes $\left<X, K(G, n) \right>$. This is because, in general,
\[ [X, K(G, n)] = \left< X, K(G, n) \right> / \pi_1(K(G, n)) \]
but $\pi_1(K(G, n)) = 0$ by definition so $[X, K(G, n)] = \left< X, K(G, n) \right>$. We need to be a bit more careful in the $n = 1$ case. Then $\pi_1(K(G,1)) = G$ acts on $\left< X, K(G, 1) \right> = \Hom{\pi_1(X)}{G}$ by conjugation. Therefore, if $G$ is abelian, the action is trivial so again $[X, K(G, 1)] = \left< X, K(G, 1) \right> = \Hom{\pi_1(X)}{G} = H^1(X, G)$. In general, for nonabelian $G$, we have $\left< X, K(G, 1) \right> = \Hom{\pi_1(X)}{G}$ and $[X, K(G, 1)]$ are the conjugacy classes of maps $\pi_1(X) \to G$ (which equals the set $H^1(X, G)$ although this isn't defined in the standard definition unless $G$ is abelian).
\bigskip\\
We can improve this to coefficients in a sheaf of groups $\mathcal{G}$. Then $H^1(X, \mathcal{G})$ is defined to classify $\mathcal{G}$-torsors. If $G$ is a topological group we get a sheaf of groups $\mathcal{G} = \mathcal{C}^0(-,G)$ and define $H^1(X, G) := H^1(X, \mathcal{G})$ which exactly classifies principal $G$-bundles. The classifying space $BG$ is defined such that $H^1(-, G) = \mathrm{Bun}_G(-) = [-, BG]$. Therefore, we need to be careful for topological groups because $BG \not\simeq K(G,1) \simeq B G_{\text{disc}}$ and $H^1(X, G) \neq H^1(X, G_{\text{disc}}) = \Hom{\pi_1(X)}{G}$.
\end{rmk}

\begin{remark}
We say that $K(G, n)$ classifies $H^n(-,G)$ meaning that the functor,
\[ H^n(-,G) : \{ \text{CW-complexes} \} \to \mathbf{Set} \]
 is represented by $[-, K(G,n)]$.
\end{remark}

\begin{definition}
Given a contravariant functor $h : \{ \text{CW-complexes} \} \to \mathbf{Set}$ we say that $C$ classifies $h$ if there is a natural isomorphism $h \cong [-, C]$ in this case we say that $h$ is representable and the pair $(C, \id \in h(C))$ is a representation of $h$. 
\end{definition}

\section{Feb 17}

Recall that the isomorphism $H^n(K(G, n), G) \cong \Hom{G}{G}$ gives a canonical element $\id \in H^n(K(G,n), G)$ the fundamental class.

\begin{thm}
We have $[X, K(G, n)] = H^n(X, G)$ given by $[f] \mapsto f^*(\id) \in H^n(X, G)$. 
\end{thm}



\begin{rmk}
We can describe the fundamental class explicitly as follows. Wrtting $K(G, n)$ as a wedge of $n$-spheres each for a generator $g \in G$. Then there is a class $c_0 \in C^n(K(G,n), G)$ sending each $S^n$ to the corresponding element of $g$. Then, in terms of two important maps $K(G, n) \to K(G,n)$, the identity and the constant map, we get $c_0 = d_{\text{const, id}}$ and then $[c_0]$ is the fundamental class.
\end{rmk}


\begin{proof}
First we prove surjectivity. Given $[c] \in H^n(X, G)$ we want $f : X \to K(G, n)$ s.t. $f^*[c_0] = [c]$. Let $f |_{X^{n-1}}$ be constant. Last time we showed that given any $h : X^n \to K(G, n)$ and $d \in C^n(X, G)$ there is some $f$ such that $d_{h, f} = d$ and $f|_{X^{n-1}} = h|_{X^{n-1}}$. Pick $f$ such that $d_{\text{const}, f} = c$ and we may assume $f : X^n \to K(G, n)$ is ceulluar.  Now consider,
\begin{center}
\begin{tikzcd}
X^n \arrow[r, "f"] & K(G, n) \arrow[r, "\id"] & K(G, n)
\end{tikzcd}
\end{center}
then,
\[ f^*(d_{\text{const}, \id}) = d_{\text{const} \circ f, \id \circ f} \]
But $d_{\text{const}, \id} = c_0 \in C^n(K(G, n), G)$ and thus,
\[ f^*(c_0) = d_{\text{const}, g} = c \in C^n(X, G) \]
Therefore, we get $f : X^n \to K(G, n)$ such that $f^* [c_0] = [c]$. Furthermore, we can extend to $X^{n+1}$ since the homotopy of $K(G, n)$ vanishes above $n$. 
\bigskip\\
Now we show injectivity. Suppose that $f^*[\id] = g^*[\id]$ for two maps $f, g : X \to K(G, n)$. We may assume that both maps are cellular. Then $[d_{\text{const}, f} ] - [d_{\text{const}, g}] = \pm [d_{f,g}]$. But $d_{\text{const}, f} = f^*[\id]$ and likewise for $g$ so $[d_{f,g}] = 0$. This implies that $f, g |_{X^n}$ are homotopic which may be extended a homotopy on $X$ since $K(G, n)$ has no higher homotopy. 
\end{proof}

\begin{ex}
We have the following explicit calculations,
\begin{enumerate}
\item $K(\Z, 1) = S^1 \implies H^1(X; \Z) = [X, S^1]$
\item $K(\Z, 2) = \CP^\infty \implies H^2(X; \Z) = [X, \CP^\infty]$
\item $K(\Z/2 \Z, 1) = \RP^\infty \implies H^1(X, \Z / 2 \Z) = [X, \RP^\infty]$
\end{enumerate}
\end{ex}

\begin{thm}[Hopf]
For every CW complex with $\dim{X} \le n$ we have $[X, S^n] \xrightarrow{\sim} H^n(X, \Z)$ via $[f] \mapsto f^*(1)$ for $1 \in H^n(S^n, \Z)$. 
\end{thm}

\begin{proof}
We know that $H^n(X, \Z) \xrightarrow{\sim} [X, K(\Z, n)]$. We can construct $K(\Z, n)$ as one $0$-cell, and one $n$-cell with no relations i.e. no $(n+1)$-cells. Thus, $K(\Z, n)^{n+1} = S^n$. Therefore, if $\dim{X} = n$ then homotopy is controlled only by the $(n+1)$-skeleton. Explicitly, by cellular approximation $f : X \to K(G, n)$ is homotopic to $f : X \to K(G, n)^{n} = S^n$. Furthermore, if $f, g : X \to K(G, n)$ are homotopic then there is a homotopy $h : X \times I \to K(G, n)$ and again by cellular approximation a homotopy $h : X \times I \to K(G, n)^{n+1} = S^n$. Thus $[X, K(G, n)] = [X, S^n]$ proving the result. 
\end{proof}

\begin{definition}
The cohomotopy groups of $X$ are $\pi^n(X) = [X, S^n]$ which also have a pointed version $\pi^n(X, x_0) = \left< X, S^n \right>$. 
\end{definition}

\begin{rmk}
Since $H^n(X, \Z) = 0$ for $n > \dim{X}$ we immediately see that $\pi^n(X) = 0$ for $n > \dim{X}$ and $\pi^n(X) = H^n(X, \Z)$ for $\dim{X} = n$. In particular, cohomotopy is bounded unlike homotopy. 
\end{rmk}

\subsection{Generalizations}

\begin{rmk}
Given a fibre bundle $F \embed E \to B$ can we extend / find sections? For example, is there a nonvanishing vector field on $S^n$? Can you find $k$ linearly independent vector field on $S^n$. 
\end{rmk}

\begin{rmk}
Here we make some simplifying assumptions: $F$ is simple i.e. ($\pi_1 \acts \pi_n$ trivially) and $B$ is simply connected (this can be weakened by using local coefficients).  
\bigskip\\
The main questions will be, given a section $s : B^n \to E$ can you extend to $B^{n+1}$ ?
\end{rmk}

Choose $D^{n+1}$, an $(n+1)$-cell of $B$ then $E|_{D^{n+1}} \cong F \times D^{n+1}$ since $D^{n+1}$ is contractible. The section $s$ gives $\partial D^{n+1} \to F$ giving $[s] \in \pi_n(F)$. Then $s$ extends to $D^{n+1}$ iff $[s] = 0$. Since on each $(n+1)$-cell we get a $[s] \in \pi_n(F)$ and thus we patch these together to get a cochain,
\[ c_s \in C^{n+1}(B, \pi_n(F)) \]
To do this we require that $B$ is simply connected such that there is a unique identification of the fibres $\pi_n(F_b)$. 

\begin{thm}
$O_s \in H^{n+1}(B, \pi_n(F))$ vanishes iff $s|_{B^{n-1}}$ extends to $B^{n+1}$. 
\end{thm}

\subsubsection{Primary Obstruction}

Assume that $\pi_0(F) = \pi_1(F) = \cdots = \pi_{n-1}(F) = 0$. Then there is no obstruction to finding a section $s : B^n \to E$ (since the obstructions are zero at the cochain level). Then $O_s \in H^{n+1}(B, \pi_n(F))$ we get the first nonzero obstruction cochain.

\begin{prop}
This first nonvanishing obstruction $O_s$ does not depend on the choice of a section $s : B^n \to E$ over the $n$-skeleton.
\end{prop}

\begin{proof}
It sufficies to show that the HEP holds for sections of bundles and that if $s \sim s'$ on $B^n$ then $O_S = O_{S'}$. 
\bigskip\\
By induction we will 
\end{proof}

\begin{definition}
The \text{primary obstruction} of a fibre bundle $F \embed E \to B$ is the above obstruction class $O_B \in H^{n+1}(B, \pi_n(F))$ which an invariant of the bundle. 
\end{definition}

\section{Vector Bundles}

\begin{definition}
A vector bundle is a fibre bundle whose fibres are vectorspaces and fibre preserving maps are assumed to be linear.
\end{definition}

\begin{lemma}
A rank $n$ vector bundle is trivial iff there are $n$ everywhere linearly independent sections.
\end{lemma}

\begin{rmk}
Assuming $B$ is paracompact (e.g. a CW complex) for any vector bundle $p : E \to B$ we can assume $E$ comes with a fibrewise Euclidean structure i.e. a section of $E^* \otimes E^* \to B$. 
\end{rmk}

\begin{defn}
Given a vector bundle $p : E \to B$ with a metric $g \in \Gamma(B, E^* \otimes E^*)$ then we define $p_k : E_k \to B$ to be the bundle of $g$-orthonormal $k$-frames. Thus $p_k^{-1}(x) \cong V(n, k)$ the Stiefl-manifold of orthonormal $k$-frames of $\R^n$ (whose $O(k)$-quotient is the Grasmannian $G(n, k)$). Then $E$ has $k$ linearly independent section iff $E_k$ admits a section. 
\end{defn}

\begin{lemma}
$\pi_i(V(n, k)) = 0$ for $i < n - k$
\end{lemma}

\begin{proof}
Furthermore, for the base case, $V(n, 1) = S^{n-1}$ so $\pi_i(V(n,1)) = \Z \delta_{i,n-1}$.
\bigskip\\
There is a fibre bundle $V(n, k) \to S^{n-1}$ via $(v_1, \dots, v_k) \mapsto v_k$ with fibre $V(n-1, k-1)$. Therefore, $V(n-1, k-1) \embed V(n,k) \to S^{n-1}$. Then by the LES,
\begin{center}
\begin{tikzcd}
\cdots \arrow[r] & \pi_{i+1}(S^{n-1}) \arrow[r] & \pi_i(V(n-1,k-1)) \arrow[r] & \pi_i(V(n,k)) \arrow[r] & \pi_i(S^{n-1}) \arrow[r] & \cdots
\end{tikzcd}
\end{center}
Then for $i < n - 2$ we have $\pi_{i+1}(S^{n-1}) = \pi_{i}(S^{n-1}) = 0$ so $\pi_i(V(n, k)) = \pi_i(V(n-1, k -1))$.
\bigskip\\
Suppose that $i < n - k$ and $k \ge 2$ then $i < n - 2$ and $i < (n - k + 2) - 2$ so we have,
\[ \pi_i(V(n, k)) = \pi_i(V(n-1, k-1)) = \cdots \pi_i(V(n - k + 2, 2)) = \pi_i(V(n-k + 1, 1)) = \pi_i(S^{n-k}) = 0 \]
For $i = n - k$ we get,
\[ \pi_i(V(n, k)) = \pi_i(V(n-1, k-1)) = \cdots = \pi_i(V(n - k + 2, 2)) \]
but $i = (n - k + 2) - 2$ so we cannot reduce further
so we need to compute $\pi_{n-2}(V(n,2))$. However, $V(n,2)$ is the unit tangent bundle of $S^{n-1}$. Consider the fibration $S^{n-2} \embed V(n,2) \to S^{n-1}$ gives the LES,
\begin{center}
\begin{tikzcd}
\pi_{n-1}(S^{n-1}) \arrow[d, equals] \arrow[r, "\partial_*"] & \pi_{n-2}(S^{n-2}) \arrow[d, equals] \arrow[r] & \pi_{n-2}(V(n,2)) \arrow[r] & \pi_{n-2}(S^{n-1}) \arrow[d, equals]
\\
\Z \arrow[r, "d"] & \Z & & 0
\end{tikzcd}
\end{center}
Now the map $\partial_*$ is $\pi_{n-1}(S^{n-1}) \cong \pi_{n-1}(T_1 S^{n-1}, S^{n-2}) \to \pi_{n-2}(S^{n-2})$. Then $d : \Z \to \Z$ computes the number of zeros of a generic vector field on $S^{n-1}$ so,
\[ d = 
\begin{cases}
0 & n - 1 \text{ odd}
\\
2 & n - 1 \text{ even}
\end{cases} \]
Therefore,
\[ \pi_{n-2}(V(n, 2)) = 
\begin{cases}
\Z & n \text{ even}
\\
\Z / 2 \Z & n \text{ odd}
\end{cases} \] 
\end{proof}

\subsection{Stiefel-Whitney Classes}

\newcommand{\E}{\mathcal{E}}

The primary obstruction is $W_i(E) \in H^i(B, \tilde{\Z})$ of the bundle $E_k$ with $k = n + 1 -i$ is the obstruction to find $k$ linearly independent sections on the $i$-skeleton. 
\bigskip\\
Reduction mod $2$ to get, $w_i(E) \in H^i(B, \Z/2\Z)$ the Stiefel-Whitney classes. Furthermore, we define $w_0(E) = 1 \in \Z/2\Z = H^0(B, \Z_2)$. 

\begin{proposition}
$w_1(E)$ is zero iff $E$ is orientable. 
\end{proposition}

\begin{lemma}
Naturality of the Stiefel-Whitney Classes. Given $f : B' \to B$ and $p : E \to B$ we have,
\[ f^* w_i(E) = w_i(f^*E) \in H^i(B', \Z / 2 \Z) \] 
\end{lemma}

\begin{theorem}
If $E, E'$ are bundles on $B$ then,
\[ w_i(E \oplus E') = \sum_{p + q = i} w_p(E) \smile w_q(E') \]
\end{theorem}

\begin{defn}
The full Stiefel-Whitney Class is $w(E) = \sum_{i = 1}^n w_i(E) \in H^*(B, \Z/2\Z)$.
\end{defn}

\begin{rmk}
The the sum formula reduces to,
\[ w(E \oplus E') = w(E) \cdot w(E') \]
in the total cohomology ring. 
\end{rmk}

\begin{theorem}
The previous properties,
\begin{enumerate}
\item for the mobius bundle $\mu \to S^1$ then $w_1(\mu) = \Z / 2 \Z$
\item for $f : B' \to B$ then $f^* w_i(E) = w_i(f^* E)$
\item $w(E \oplus E') = w(E) \cdot w(E')$ in the cohomology ring
\end{enumerate}
uniquely characterize $w$ as a map from vector bundles to cohomology rings. 
\end{theorem}

\begin{example}
Consider the tautological bundle $\gamma^n \to \RP^n$ defined by,
\[ \gamma^n = \{ (\ell, v) \mid v \in \ell \} \subset \RP^n \times \R^{n+1} \]
then $\gamma^1 \to S^1$ is the Mobius bundle. 
\end{example}

\section{Characteristic Classes}

\begin{example}
Recall that $\gamma_1 \to \RP^1$ is the Modius bundle so $w(\gamma_1) = 1 + \alpha \in H^*(S^1)$ where $H^*(\RP^n, \Z / 2 \Z) = \mathbb{F}_2[\alpha]/(\alpha^{n+1})$. 
\bigskip\\
Then for $\iota : \RP^1 \embed \RP^n$ we find $\iota^*(\gamma_n) = \gamma_1$. Also we can compute,
\[ \iota^* : H^*(\RP^n) \to H^*(\RP^1) \]
must sent $\alpha \mapsto \alpha$. Since $w(\gamma_1) = w(\iota^*(\gamma_n)) = \iota^*(w(\gamma_n)) = 1 + \alpha$ we find that $w(\gamma_n) = 1 + \alpha$. 
\end{example}

\begin{example}
Consider the orthogonal bundle to $\gamma_n$,
\[ \gamma_n^\perp = \{ (\ell, b) \mid v \in \ell^\perp \} \subset \RP^n \times \R^n \]
It is clear that $\gamma_n \oplus \gamma_n^\perp$ is the trivial rank-$n+1$ bundle over $\RP^n$. Thus, 
\[ w(\gamma_n \oplus \gamma_n^\perp) = w(\gamma_n) \cdot w(\gamma_n^\perp) = 1 \]
But $w(\gamma_n) = 1 + \alpha$ so,
\[ w(\gamma_1) = \frac{1}{1 + \alpha} = 1 + \alpha + \alpha^2 + \cdots + \alpha^n \]
\end{example}

\begin{example}
Consider the tangent bundle $T \RP^n \to \RP^n$. First, consider,
\[ T S^n = \{ (x, v) \mid |x| = 1 \quad x \cdot v = 0 \} \subset S^n \times \R^{n+1} \]
Then,
\[ T \RP^n = T S^n / (x, v) \sim (-x, -v) \]
Then $x,-x \in \ell_x$ and $v, -v \in \ell_x^\perp$. Such an element $(x, v) \in T \RP^n$ is a pair $(x, L)$ where $L : \ell_x \to \ell_x^\perp$ is a linear map. Therefore,
\[ T \RP^n = \Hom{\gamma_n}{\gamma_n^\perp} \]
First, note that,
\[ \Hom{\gamma_n}{\gamma_n} = \underline{\R} \]
since there is the section $\id : \gamma_n \to \gamma_n$. Therefore,
\begin{align*}
T \RP^n \oplus \underline{\R} & = \Hom{\gamma_n}{\gamma_n^\perp} \oplus \Hom{\gamma_n}{\gamma_n} = \Hom{\gamma_n}{\gamma_n^\perp \oplus \gamma_n} = \Hom{\gamma_n}{\underline{\R}^{n+1}} 
\\
& = \bigoplus_{i = 1}^{n+1} \Hom{\gamma_n}{\underline{\R}}
\end{align*} 
However, for real bundles, a choice of metric gives an isomorphism $\Hom{\E}{\underline{\R}} \cong \E$. Thus, we find,
\[ T \RP^n \oplus \underline{\R} \cong \bigoplus_{i = 1}^{n+1} \gamma_n \]
Then,
\[ w( T \RP^n) = w( T \RP^n) \cdot w(\R) = (w(\gamma_n))^{n+1} = (1 + \alpha)^{n+1} = \sum_{k = 0}^{n+1} { n + 1 \choose k } \alpha^k \: \mod 2 \]
\end{example}

\begin{cor}
$w(T \RP^n) = 1$ iff $n + 1$ is a power of $2$ and thus if $ T \RP^n$ is trivial then $n = 2^k - 1$. 
\end{cor}

\section{Classifying Spaces}

\newcommand{\Vect}{\mathrm{Vect}}

Consider the functor $V : \{ \text{CW complex} \} \to \Set$ via $X \mapsto \Vect{X}$ to isomorphism classes of vector bundles on $X$. Is this functor representable in the homotopy category i.e. is $\Vect(X) = [X, C]$ for come classifying space $C$?
\bigskip\\
Recall the Grasmannian, $G(n, k)$ which classifies $k$-dimensional subspaces of $\R^n$. There is a natural inclusion $G(n, k) \to G(n+1, k)$ which alows us to construct,
\[ G_k = \varinjlim_{n} G(n, k) \]
In particular, $G_1 = \RP^\infty$. We will see $G_n = BO(n)$. Furthermore, there is a tautological bundle $\gamma^k_n \to G(n, k)$ which is a rank $k$-vector bundle,
\[ \gamma^k_n = \{ (x, v) \mid x \in G(n, k) \: v \in V_x \} \subset G(n, k) \times \R^{n} \]
This gives a bundle $\gamma^k \to G_k$ which we call $EO(n) \to BO(n)$. 

\begin{theorem}
Let $X$ be a finite CW complex then,
\[ [X, G_k] \to \Vect^k(X) \] 
via $f : X \to G_k \mapsto f^* \gamma^k$ is an isomorphism. 
\end{theorem}

\begin{proof}
Look at Milne.
\end{proof}

\begin{theorem}
$H^*(G_k, \Z / 2 \Z) = \mathbb{F}_2[w_1, \dots, x_n]$ 
with $w_i$ graded in degree $i$ where $w_i = w_i(\gamma^k)$.
\end{theorem}

\begin{cor}
If $E = f^* \gamma^n$ then $w_i(E) = f^* w_i(\gamma^n)$ so the Stiefel-Whitney classes of $E$ detect nontriviality in $\Z / 2 \Z$ cohomology of the classifying map $f : X \to G_k$. 
\end{cor}

\section{K-Theory}

\begin{prop}
Every class $\alpha \in K(X)$ can be represented as $[E] - [\varepsilon^n]$. 
\end{prop}

\begin{prop}

\end{prop}

\begin{defn}
We say that vector bundles $E$ and $E'$ are \textit{stably equivalent} if $E \oplus \varepsilon^n \cong E' \oplus \varepsilon^{m}$ for $n,m \in \N$. Then define reduced K-theory,
\[ \wt{K}(X, x_0) = \ker{(K(X) \to K(x_0))} \]
for connected $X$ we have $\wt{K}(X) = \mathrm{Vect}_{\CC}(X) / \text{stable equivalence}$. 
\end{defn}

\newcommand{\rk}{\mathrm{rk}}

\begin{rmk}
Notice that $\wt{K}(X) \subset K(X)$ is an ideal and in particular a $K(X)$-module but $\wt{K}(X)$ is not itself a ring under $\otimes$. From the viewpoint of stable equivalence, this is because the $\otimes$ unit $\varepsilon$ is set to zero and we need to modify our definition of multiplication. Notice that $\otimes$ does not respect stable equivalence. Indeed, if $E \sim E'$ then,
\[ E \oplus \varepsilon^n \cong E' \oplus \varepsilon^m \]
for some $n$ and $m$ so,
\[ (E \oplus \varepsilon^n) \otimes F \cong (E' \oplus \varepsilon^m) \otimes F \]
which gives that,
\[ (E \otimes F) \oplus F^n \cong (E' \otimes F) \oplus F^m \]
but this does not necessarily mean that $E \otimes F$ and $E \otimes F'$ are stably equivalent. Thus we define,
\[ E \otimes^{\mathrm{stab}} E' = E \otimes E' \oplus (E^\perp)^{\rk(E')} \oplus (E'^\perp)^{\rk(E)} \]
which comes from embedding $\wt{K}(X) \embed K(X)$ and inducing the product structure. Now if $E \sim E'$ then,
\begin{align*}
(E \oplus \varepsilon^n) \otimes^{\text{stab}} F & = E \otimes F \oplus F^n \oplus (E^\perp)^{\rk(F)} \oplus \varepsilon^{n \rk(F)} \oplus (F^\perp)^{n + \rk(E)}
\\
& \cong E \otimes F \oplus (E^\perp)^{\rk(F)} \oplus (F^\perp)^{\rk(E)} \oplus \varepsilon^{n \rk(F) + n}
\end{align*}
\begin{align*}
(E' \oplus \varepsilon^m) \otimes^{\text{stab}} F & = E' \otimes F \oplus F^m \oplus (E'^\perp)^{\rk(F)} \oplus \varepsilon^{m \rk(F)} \oplus (F^\perp)^{m + \rk(E')}
\\
& \cong E' \otimes F \oplus (E'^\perp)^{\rk(F)} \oplus (F^\perp)^{\rk(E')} \oplus \varepsilon^{m \rk(F) + m}
\end{align*}
and therefore because $E \oplus \varepsilon^n \cong E' \oplus \varepsilon^m$ we see that,
\[ E \otimes F \oplus (E^\perp)^{\rk(F)} \oplus (F^\perp)^{\rk(E)} \oplus \varepsilon^{n \rk(F) + n} \cong E' \otimes F \oplus (E'^\perp)^{\rk(F)} \oplus (F^\perp)^{\rk(E')} \oplus \varepsilon^{m \rk(F) + m} \]
which implies that,
\[ E \otimes^{\text{stab}} F = E \otimes F \oplus (E^\perp)^{\rk(F)} \oplus (F^\perp)^{\rk(E)} \sim E' \otimes F \oplus (E'^\perp)^{\rk(F)} \oplus (F^\perp)^{\rk(E')} = E' \otimes^{\text{stab}} F \]
which proves that $\otimes^{\text{stab}}$ is well-defined on stable equivalence classes since it is symmetric the exact same argument works for the second argument. 
\end{rmk}

\begin{prop}
For connected $X$ we have $\wt{K}(X) = [X : BU]$ where $BU = \varinjlim_{n} BU(n)$ since homotopy equivalence to $BU$ is equivalent to stable equivalence 
\end{prop}

\subsection{Mar 2}

Let $X$ be a finite CW complex and $\Vect_\CC(X)$ complex vector-bundles on $X$. Then we define,  $K(X)$ is the group completion of the monoid $\Vect_{\CC}(X)$ under $\oplus$. Then $K(X)$ is a ring under $\oplus$ and $\otimes$.
\bigskip\\
Then $X \mapsto K(X)$ is a functor: $f : X \to Y$ gives $f^* : K(Y) \to K(X)$ from pulling back bundles. Then $f^*$ depends only on the homotopy class $f \in [X, Y]$.
\bigskip\\
\begin{rmk}
Is $X$ is contractible then $K(X) \cong K(*) = \Z$.
\end{rmk}

\begin{prop}
For any vector bundle $E$ on $X$ there exists a vector bundle $E'$ on $X$ with $E \oplus E' = \varepsilon^{N}$. 
\end{prop}

\begin{cor}
In the $K$-group we have $[E] - [F] = [E'] - [\varepsilon^N]$ and furthermore $[E] - [\varepsilon^N] = [E'] - [\varepsilon^M]$ if and only if $\dim{E} - N = \dim{E'} - M$ and $E \sim_s E'$ are stably equivalent i.e. $E \oplus \varepsilon^n \cong E' \oplus \varepsilon^m$.  
\end{cor}

\begin{defn}
For a point $x_0 \in X$ we have $\wt{K}(X) = \ker{(K(X) \to K(x_0))}$ this gives bundles up to stable equivalence (assume that $X$ is connected). 
\end{defn}

\begin{prop}
$\wt{K}(X) = [X : BU]$ with $BU = \varinjlim_n BU(n)$.
\end{prop}

\begin{rmk}
Consider the Grassmannian,
\[ G_\CC(N, n) = U(N) / U(n) \times U(N-n) \]
and the Steifl manifold,
\[ V_\CC(N, n) = U(N) / U(N-n) \]
then there is a fibration,
\[ U(n) \embed V_\CC(N, n) \to G_\CC(N, n) \]
\end{rmk}

\begin{prop}
The homotopy groups of the Stiefl manifold satisfies,
\[ \pi_r(V_\CC(N, n)) = 0 \quad \text{ for } \quad r \le 2 (N - n) \]
Furthermore, the fibration,
\[ U(u) \embed V_\CC(\infty, n) \to G_\CC(\infty, n) = BU(n) \]
Then $\pi_r(V_\CC(\infty, n)) = 0$ by above and thus $V_\CC(\infty, n)$ is contractible. Therefore, from the long exact sequence,
\[ \pi_r(U(n)) \cong \pi_{r+1}(BU(n)) \]
\end{prop}

\begin{defn}
Now we have $BU = \varinjlim_n BU(n)$ and $U = \varinjlim_n U(n)$
\end{defn}

\begin{rmk}
Our previous proposition says that $\pi_r(U) \cong \pi_{r+1}(BU)$. 
\end{rmk}

\begin{prop}
Then $\wt{K}(S^r) = [S^r, BU] = \pi_r(BU) = \pi_{r-1}(U)$. 
\end{prop}

\begin{proof}
There is no issuse with based vs unbased maps because $\pi_1(BU) \acts \pi_n(BU)$ is trivial since the above is a fibration and $\pi_1(V_\CC(\infty, n)) = 0$ (use the homework). Alos $\pi_1(BU) = 0$ so we can conclude.
\end{proof}

\begin{example}
$\wt{K}(S^1) = \pi_0(U) = 0$ and $\wt{K}(S^2) = \pi_1(U) = \pi_1(U(1)) = \Z$ 
\end{example}

\begin{rmk}
The generator of $\wt{K}(S^2) = \Z$ is the tautological bundle $\gamma \to \CP^1 = S^2$ and we write its reduced class as $[\gamma] - [\CC]$. To see this, note that $\wt{K}(S^2) \xrightarrow{\sim} \Z$ is given by the Chern class $c_1$ and $c_1(\gamma) = -1 \in H^2(S^2; \Z)$.
\end{rmk}

\begin{rmk}
We can describe $K(S^2) = \wt{K}(S^2) \oplus \Z = \Z \oplus \Z$ via $\xi \mapsto (c_1(\xi), \dim{\xi})$. Consider in particular,
\[ (\gamma \otimes \gamma) \oplus \CC \mapsto (-2, 2) \quad \text{ and thus } \quad (\gamma \otimes \gamma) \mapsto (-2, 2) \]
Then in $K(S^2)$ we have $\gamma^2 \oplus \varepsilon = 2 \gamma$ and thus $\gamma^2 + 1 = 2 \gamma$ therefore $([\gamma] - 1)^2 = 0$ in $K(S^2)$. Therefore, $K(S^2) = \Z[x]/x^2$. 
\end{rmk}

\subsection{Long Exact Sequence of a Pair}

\begin{lemma}
Let $(X, A)$ be a finite CW pair. Then consider the sequence $A \embed X \to X / A$. Then we get induced maps,
\[ \wt{K}(X/A) \xrightarrow{p^*} \wt{K}(X) \xrightarrow{\iota^*} \wt{K}(A) \]
which is exact meaning $\ker{\iota^*} = \Im{p^*}$.
\end{lemma}

\begin{proof}
We know that $\wt{K}(-) = [-, BU]$ so the above sequence is simply,
\[ [X/A, BU] \to [X, BU] \to [A, BU]  \]
if a map $f : X \to BU$ restricts to a nullhomotopic map $f|_A : [A, BU]$ then by the homotopy extension property of the pair $(X, A)$ we get a homotopy on $f : X \to BU$ making $f |_A$ trivial i.e. $f$ descends to the quotient $\tilde{f} : X / A \to BU$.  
\end{proof}

\begin{rmk}
Using the homotopy equivalence $X \cup CA \simeq X/A$ we get a diagram,
\begin{center}
\begin{tikzcd}
A \arrow[r, hook] & X \arrow[r, two heads] \arrow[d, equals] & X / A \arrow[d, "\simeq"]
\\
& X \arrow[r, hook]  & X \cup CA \arrow[d, equals] \arrow[r, two heads] & \Sigma A \arrow[d, "\simeq"]
\\
& & X \cup CA \arrow[r, hook] & (X \cup CA) \cup CX \arrow[r, two heads] & \Sigma X
\end{tikzcd}
\end{center}
and therefore an exact sequence,
\begin{center}
\begin{tikzcd}
\wt{K}(\Sigma X) \arrow[r] & \wt{K}(\Sigma A) \arrow[r] & \wt{K}(X/A) \arrow[r] & \wt{K}(X) \arrow[r] & \wt{K}(A)
\end{tikzcd}
\end{center}
\end{rmk}

\section{Mar 4}

\begin{defn}
We define $\wt{K}^{-q}(X) = \wt{K}(\Sigma^q X)$. Furthermore, $K(X, A) = \wt{K}(X/A)$.
\end{defn}

\begin{prop}
Given a finite CW pair $(X, A)$ there is a long exact sequence,
\begin{center}
\begin{tikzcd}
\cdots \arrow[r] & \wt{K}^{q-2}(A) \arrow[r] & \wt{K}^{q-1}(X/A) \arrow[r] & \wt{K}^{q-1}(X) \arrow[r] & K^{q-1}(A) \connectingmap{dll} 
\\
& & \wt{K}^q(X/A) \arrow[r] & \wt{K}^q(X) \arrow[r] & \wt{K}^q(A) \arrow[r] & \wt{K}^{q+1}(X/A) \arrow[r] & \cdots
\end{tikzcd}
\end{center}
\end{prop}

\begin{thm}[Bott]
The map $K(X) \otimes K(S^2) \to K(X \times S^2)$ is an isomorphism. Explicitly,
\begin{align*}
K(X) \otimes K(S^2) & = K(X) \otimes \Z[H]/(H-1)^2 \cong K(X) \oplus K(X)
\\
K(X) \oplus K(X) \xrightarrow{\sim} & K(X \times S^2)
\\
(\alpha_1, \alpha_2) \mapsto & p_1^* \alpha_1 \oplus p_1^* \alpha_2 \otimes p_2^* \gamma  
\end{align*}
\end{thm}

\begin{cor}
$\wt{K}(X) = \wt{K}(\Sigma^2 X) = \wt{K}^{-2}(X)$. 
\end{cor}

\begin{proof}
Consider,
\[ \Sigma^2 X = X \times S^2 / X \vee S^2 \]
Now we consider the pair $(X \times S^2, X \vee S^2)$ which gives an exact sequence,
\begin{center}
\begin{tikzcd}
K^{-1}(X \times S^2) \arrow[r, two heads] & K^{-1}(X \vee S^2) \arrow[r] & K(X \times S^2, X \vee S^2) \arrow[r] & K(X \times S^2) \arrow[r, two heads, "\varphi"] & K(X \vee S^2)
\end{tikzcd}
\end{center}
where the maps $K(X \times Y) \to K(X \vee Y)$ are surjective since $\wt{K}(X \vee Y) = \wt{K}(X) \oplus \wt{K}(Y)$ and there is a pullback $K(X) \to K(X \times Y)$.
Note that there is a diagram,
\begin{center}
\begin{tikzcd}
K(X \times S^2) \arrow[r, two heads] & K(X \vee S^2) \arrow[d, equals]
\\
\wt{K}(X) \oplus \wt{K}(S^2) \oplus \Z \arrow[u, dashed] \arrow[r, equals] & \wt{K}(X \vee S^2) \oplus \Z
\end{tikzcd}
\end{center}
Therefore,  $K(X \times S^2, X \vee S^2) = \ker{\varphi}$. By Bott's theorem, we see that,
\[ \alpha_1 \otimes 1 + \alpha_2 \otimes \gamma = \alpha \otimes (\gamma - 1) + \beta \otimes \gamma \]
then $\varphi(x) = 0$ iff both restrictions to $X$ and $S^2$ are zero. 
Therefore, 
\[ \wt{K}(X) = K(X \times S^2, X \vee S^2) \cong \wt{K}(\Sigma^2 X) \]
\end{proof}

\subsection{Proof of Bott's Theorem}

First we consider vector bundles on $S^2$. Such vector bundles must be trivial on the upper and lower hemispheres but can have a nontrivial attaching function $f : S^1 \to \GL{n}{\CC}$.  Then, noting that $\pi_1(\GL{n}{\CC}) = \Z$  we see immediately that $\wt{K}(S^2) = \Z$. 
\bigskip\\
Now consider $E |_{X \times D^2_{\pm}}$ is the pullback of a bundle $E_0$ on $X$ since $D^2$ is contractible. Then $E$ is obtained by gluing these together via,
\[ f : X \times S^1 \to \Aut{E_0} \]
so $E$ can be described as a pair $[E, f]$. 
\begin{example}
The tautological bundle $\gamma \to S^2$ is represented by the pair $[\varepsilon, z^{-1}]$. 
\end{example}
\noindent
We have some basic facts,
\[ [E_1, f_1] \oplus [E_2, f_2] = [E_1 \oplus E_2, f_1 \oplus f_2] \]
and
\[ [E_1, f_1] \otimes [E_2, f_2] = [E_1 \otimes E_2, f_1 \otimes f_2] \]
in particular,
\[ [E_0, z^n f] = [E_0, f] \otimes \gamma^{-n} \]
\bigskip\\
Bott's theorem is that every bundle on $X \times S^2$ can be stabily represented as $\alpha_1 \otimes 1 + \alpha_2 \otimes \gamma$ i.e. every element in $K(X \times S^2)$ is stabily represented by,
\[ [E_0, \id] \oplus [E_1, z^{-1} ] \]
\bigskip\\
First, given $[E_0, f]$ we have $f : X \times S^1 \to \Aut{E_0}$ so fixing $x \in X$ we get,
\[ f(x, -) : S^1 \to \Aut{(E_0)_x} = \GL{n}{\CC} \]
and thus we can use Fourier analysis to conclude that there is a convergent series,
\[ f(x, z) = \sum_{n \in \Z} a_n(x) z^n \] 
for each $x \in X$. Furthermore, since $f$ is continuous on $X \times S^1$ we get that $a_n : X \to \CC$ are continuous since they are computed by a continuous family of integrals. However, $X$ is compact we get that,
\[ \sum_{n = -N}^N a_n(x) z^n \to f(x, z) \]
converges uniformly. Then we can choose sufficiently large $M$ such that,
\[ f_M = \sum_{n = - M}^M a_n(x) z^n \]
is uniformly close to $f$ and thus are homotopic by a linear homotopy. Thus we can assume that $f$ is a Laruent series. After multiplying by $z^M$ we can assume that $f$ is a polynomial,
\[ f(x) = p(x) = \sum_{n = 0}^N a_n(x) z^n \]
so we can assume that our bundle is of the form $[E_0, p]$ for a polynomial function. 
\bigskip\\
Now we claim that,
\[ [E_0, p] \oplus [E_0^n, \id] \cong [E_0^{\oplus (n+1)}, b(x) + z a(x)] \]
and show this with an explicit matrix computation.
\bigskip\\
Now we have our bundle of the form $[F, b(x) + z a(z)]$ and we need to show that,
\[ [F, b(x) + z a(z)] \sim [F_{+}, \id] \oplus [F_{-}, z ] \]
with $F = F_{+} \oplus F_{-}$. In fact, we can assume $a(z) = \id$ so we have $[F, b(x) + z]$. Fix $x \in X$ then for any $z \in S^1$ and we know that $b(x) + z \in \GL{n}{\CC}$ and thus $b(x)$ has no eigenvalue of norm $|\lambda| = 1$ else $b(x) + z$ for $z = \lambda$ would not be invertible. 

\begin{lemma}
Given $b(x)$ acting on $F \to x$ with no eigenvalue $|\lambda| = 1$ then,
\[ F = F_{+} \oplus F_{-} \]
such that $b(F_{\pm}) \subset F_{\pm}$ and $b|_{F_{+}}$ has eigenvalues $|\lambda| > 1$ and $b|_{F_{-}}$ has eigenvalues $|\lambda| < 1$. 
\end{lemma}
\noindent
Finally, 
\[ F, b + z] \sim [F_+, b + z] \oplus [F_{-}, b + z] \]
Now we may homotope $[F_+, b + z] \sim [F_+, \id]$ using the homotopy $b + t \xrightarrow{b + t z} b \xrightarrow{} \id$ where the first map is a homotopy since $b + t z$ is always invertible since $b$ has all eigenvalues $|\lambda| > 1$ and the second map is just a change of coordinates on fibers over $X$ which is fine since it is constant on $S^1$. Furthermore, we homotopy $[F_{-}, b + z] \sim [F_{-}, z]$ via the homotopy $b+z \xrightarrow{t b + z} z$ which gives a homotopy since each $b t + z$ is invertable since $b$ has all eigenvalues $|\lambda| < 1$.

\subsection{K-Theory in the Real Case}

\begin{defn}
$KO(X) = K(\Vect_\R(X))$
\end{defn}

\begin{example}
For spheres we can compute $\wt{KO}(S^n)$,
\begin{center}
\begin{tabular}{ c | c | c | c | c | c | c | c | c }
n & 0 & 1 & 2 & 3 & 4 & 5 & 6 & 7 \\
\hline \hline
$\wt{KO}(S^n)$ & $\Z$  & $\Z_2$ & $\Z_2$ & $0$ & $\Z$ & $0$ & $0$ & $0$
\end{tabular}
\end{center}
and these groups are $8$-periodic.
\end{example}

\section{K-Theory and Cohomology}

Let $X$ be a finite CW complex then we have ring cohomology theories $H^*(X)$ and $K(X)$. 
\bigskip\\
We have the total Chern class $c(E) = 1 + c_1(E) + c_2(E) + \cdots + c_n(E) \in H^*(X; \Z)$. 
Then given line bundles $L_1, L_2$ we have $c(L_i) = 1 + c_1(L_i)$ so,
\[ c(L_1 \oplus L_2) = 1 + c_1(L_1) + c_1(L_2)+ c_1(L_1) \smile c_1(L_2) \neq c(L_1) + c(L_2) \]
and
\[ c(L_1 \otimes L_2) = 1 + c_1(L_1) + c_1(L_2) \neq c(L_1) \cdot c(L_2) \]
Therefore, the total Chern class is poorly behaved as a ring map. 
\bigskip\\
\begin{theorem}[Splitting]
Given a vector bundle $E \to X$ then there exists $f : X' \to X$ s.t. $f^* E = L_1 \oplus \cdots \oplus L_n$ is a direct sum of line bundles with $n = \rank{(E)}$. Furthermore, the pullback maps  $f^* : H^*(X; \Z) \to H^*(X'; \Z)$ and $f^* : K(X) \to K(X')$ are injective. 
\end{theorem}

\begin{rmk}
In particular, 
\[ f^* c(E) = c(f^* E) = c(L_1 \oplus \cdots \oplus L_n) = (1 + c_1(L_1)) \cdots (1 + c_1(L_n)) \]
\end{rmk}

\begin{rmk}
Thus we can suppose that $E = L_1 \oplus \cdots \oplus L_n$ and write $c_1(L_i) = x_i \in H^2(X; \Z)$. Therefore, we have,
\[ c(E) = (1 + x_1) \cdots (1 + x_n) = \sum_{i = 0}^n e_i(x_1, \dots, x_n) \]
where $e_i(x_1, \dots, x_n)$ are the elementary symmetric polynomials in $x_1, \cdots, x_n$.
\end{rmk}

\newcommand{\CCh}{\mathrm{ch}}

\begin{defn}
In the above case, the Chern character of $E$ is,
\[ \CCh(E) = \sum_{i = 0}^n e^{x_i} \in H^*(X; \Q) \]
This gives,
\[ \CCh(E) = \rank{E} + x_1 + \cdots x_n + \tfrac{1}{2!} (x_1^2 + \cdots + x_n^2) + \tfrac{1}{3!} (x_1^3 + \cdots x_n^3) + \cdots \]
Therefore, we may write this in terms of the elementary symmetric polynomials. We have Newton sums,
\[ x_1^2 + \cdots + x_n^2 = (x_1 + \cdots + x_n)^2 - 2(x_1 x_2 + x_1 x_3 + \cdots + x_{n-1} x_n) \]
In genereral, there is a formula for,
\[ s_k(x_1, \dots, x_n) = x_1^k + \cdots + x_n^k \]
giving,
\[ s_k(x_1, \dots, x_n) = \sum_{i = 0}^{\ell_k} f_{k i} e_i(x_1, \dots, x_n) \]
these may be computed via Newton formulae. Then we see that,
\[ \CCh(E) = \sum_{k = 0}^n \frac{1}{k!} s_k(x_1, \cdots, x_n) = \sum_{k = 0}^n \sum_{i = 0}^{\ell_k} \frac{k_{ki}}{k!} e_i(x_1, \dots, x_n) \]
but $c_i(E) = e_i(x_1, \dots, x_n)$ so we have,
\[ \CCh(E) =  \sum_{k = 0}^n \sum_{i = 0}^{\ell_k} \frac{k_{ki}}{k!} c_i(E) \]
this is the definition we take for an arbitrary bundle.
\end{defn}

\begin{lemma}
$\CCh(E \oplus F) = \CCh(E) + \CCh(F)$ and $\CCh(E \otimes F) = \CCh(E) \cdot \CCh(F)$ in $H^{\text{even}}(X; \Q) \subset H^*(X; \Q)$.
Therefore, we get a ring map,
\[ \CCh : K(X) \to H^{\text{even}}(X; \Q) \] 
via $\CCh([E] - [F]) = \CCh(E) - \CCh(F)$. 
\end{lemma}

\begin{proof}
Use the splitting principle and properties of $\sum e^{x_i}$. 
\end{proof}

\begin{theorem}
Let $X$ be a finite CW complex. Then,
$\CCh_\Q : K(X) \otimes_\Z \Q \xrightarrow{\sim} H^{\text{even}}(X; \Q)$ is a ring isomorphism. 
\end{theorem}


\begin{rmk}
Thus $K(X)$ and $H^{\text{even}}(X; \Z)$ agree up to torsion. 
\end{rmk}

\begin{example}
[Atiyah-K-Theory] 
\[ K(\RP^{2m - 1}) = \Z \oplus \Z / 2^{m-1} \Z \] but 
\[ H^{\text{even}}(\RP^{2m - 1}; \Z) = \Z \oplus (\Z / 2 \Z)^{\oplus (m - 1)} \]
so the torsion of these groups may not agree.
\end{example}

\begin{theorem}
There is a diagram,
\begin{center}
\begin{tikzcd}
K(\Sigma X) \otimes_\Z \Q \arrow[r, "\CCh"] \arrow[d, "\sim"] & H^{\text{even}}(\Sigma X; \Q) \arrow[d, equals]
\\
K^1(X) \otimes_\Z \Q \arrow[r, "\CCh"] & H^{\text{odd}}(X; \Q)
\end{tikzcd}
\end{center}
\end{theorem}

\begin{proof}
We proceed by induction on $n = \dim{X}$. Suppose it holds for CW complexes of $\dim{Y} < n$ and let $\dim{X} = n$ then $X = X^n$ and look at the pair $(X^n, X^{n-1})$ which gives a morphism of long exact sequences,
\begin{center}
\begin{tikzcd}
K^{p-1}(X^{n}, X^{n-1}) \arrow[d, "\CCh"] \arrow[r] & K^p(X^{n-1}) \arrow[r] \arrow[d, "\CCh"] & K^p(X^n) \arrow[d, "\CCh"] \arrow[r] & K^p(X^{n}, X^{n-1}) \arrow[d, "\CCh"] \arrow[r] & K^{p+1}(X^{n-1}) \arrow[d, "\CCh"]
\\
H^{p-1}(X^{n}, X^{n-1}) \arrow[r] & H^p(X^{n-1}) \arrow[r] & H^p(X^n) \arrow[r] & H^p(X^{n}, X^{n-1}) \arrow[r] & H^{p+1}(X^{n-1}) 
\end{tikzcd}
\end{center}
where $p = \text{even}$ or $p = \text{odd}$ and $H^p(X) = H^{\text{even}}(Y, \Q)$ or $H^{\text{odd}}(X; \Q)$ and the K-groups are given rational coefficnents. The outer maps are isomorphisms by indunction and the two inner maps $\CCh : K^p(X^n, X^{n-1}) \to H^p(X^n, X^{n-1})$ are isomorphism by our computation for $S^n$ since $X^n / X^{n-1} = \bigvee_i S^n$. Thus the central map is an isomorphism by the five lemma proving the proposition by induction.
\end{proof}

\begin{example}
$K(\CP^n) \cong \Z[x]/(x^n)$ so $K(\CP^n) \cong \Z^{n+1}$ as a $\Z$-module. 
\end{example}

\subsection{The Splitting Principle}

\renewcommand{\P}{\mathbb{P}}

Consider a vector bundle $b : E \to X$ with $\rank{E} = n$. Consider the projective bundle $p : \P(E) \to X$. Then we get a pullback bundle,
\begin{center}
\begin{tikzcd}
p^* E \arrow[r] \arrow[d] & E \arrow[d, "b"]
\\
\P(E) \arrow[r, "p"] & X
\end{tikzcd}
\end{center}
Then $p^* E$ has a natural line sub-bundle, namely the tautological bundle of $\P(E)$. A point in $\P(E)$ is a pair $x \in X$ and $[L] \in \P(E_x)$ thus the tautological bundle is,
\[ \gamma = \{(x, [L], v) \mid L \subset E_x \text{ and } v \in L \} \]
Then $\gamma \subset p^* E$ since we get a map $(x, [L], v) \mapsto v \subset L \subset E_x$. This defines a line sub-bundle.
\bigskip\\
We then apply this proceedure to $L^\perp \subset p^* \P(E)$ to decompose $E$ into line bundles. 

\section{The Hopf Invariant}

\newcommand{\HP}{\mathbb{HP}}
\newcommand{\OP}{\mathbb{OP}}

Hatcher ch. 4

For a map $f : S^{2n - 1} \to S^n$ we define:

\begin{defn}
$C_f = S^n \cup_f D^{2n}$ so there is no difference in Cellular cohomology. Then $H^n(C_f; \Z) = \Z$ and $H^{2n}(C_f; \Z) = \Z$. Pick generators $\alpha$ and $\beta$. Then,
\[ \alpha^2 \in H^{2n}(C_f; \Z) \]
which implies that $\alpha^2 = h(f) \beta$ and we call $h(f) \in \Z$ the Hopf invariant. 
\end{defn}

\begin{prop}
The Hopf invariant gives a homomorphism $h : \pi_{2n - 1}(S^n) \to \Z$ with the following properties,
\begin{enumerate}
\item if $n$ is odd then $h = 0$ (since $\alpha \smile \alpha = 0$ in odd $n$).
\item for the Hopf fibration $H : S^3 \to S^2$ then $C_f = S^2 \cup_H D^4 = \CP^2$ and $H^*(\CP^2; \Z) = \Z[x]/(x^3)$ so the generator of $H^2(\CP^2; \Z)$ squares to the generator of $H^4(\CP^2; \Z)$ which implies that $h(H) = 1$. In particular, $h : \pi_3(S^2) \xrightarrow{\sim} \Z$ sending $H \mapsto 1$. 
\item There is a generalized Hopf fibration $f : S^7 \to S^4 = \HP^1$ then $\HP^2 = S^4 \cup_f D^8$ but $H^*(\HP^2; \Z) = \Z[x]/(x^3)$ so $h(f) = 1$ giving $h : \pi_7(S^4) \xrightarrow{\sim} \Z$ sending $f \mapsto 1$.
\item Furthermore, $\OP^2 = S^8 \cup_f D^{18}$ via a sphere fibration $f : S^{15} \to S^8$ then $h(f) = 1$ giving $h : \pi_{15}(S^8) \xrightarrow{\sim} \Z$ sending $f \mapsto 1$.
\end{enumerate}
\end{prop}

\begin{proof}
For all $n \ge 1$ there exists a map $f : S^{4n - 1} \to S^{2n}$ with Hopf invariant $2$.
\end{proof}

\begin{proof}
Indeed, there is the Whitehead map $S^{a+b-1} \to S^a \vee S^b$ used in the construction of the Whitehead product. Then $S^a \times S^b$ is constructed by attaching $D^{a+b}$ along this map. Thus we take $f : S^{4n-1} \to S^n$ as,
\[ S^{2n+2n-1} \to S^{2n} \vee S^{2n} \to S^{2n} \]
Then the Hopf invariant is defined by attaching $D^{4n}$ along this map to get,
\[ C_f = S^{2n} \cup_f D^{4n} \]
which is $S^{2n} \times S^{2n}$ with $S^{2n} \times *$ and $* \times S^{2n}$ identified. Then,
\[ \alpha \in H^{2n}(C_f;\Z) \]
pulls back to the class $x + y \in H^{2n}(S^{2n} \times S^{2n}; \Z)$ where $x,y$ are the Kunneth generators. Therefore, $x^2 = 0$ and $y^2 = 0$ and $xy$ generates $H^{4n}(S^{2n} \times S^{2n}; \Z)$ and the generator $\beta \in H^{4n}(C_f; \Z)$ pulls back to $xy$. Therefore $\alpha^2 \mapsto (x+y)^2 = 2 xy$ so $\alpha^2 = 2 \beta$ proving that $h(f) = 2$. Note we used that $2n$ is even in computing,
\[ (x+y)^2 = x^2 + xy + yx y^2 = 2 xy \]
otherwise $xy + yx = 0$ and we get $(x+y)^2 = 0$ as we must for odd-dimensional classes.
\end{proof}

\begin{rmk}
Therefore, the map $h : \pi_{4n-1}(S^{2n}) \to \Z$ is never trivial. It was easy to construct maps $f : S^{4n - 1} \to S^{2n}$ with $h(f) = 2$. However we have the following theorem,
\end{rmk}

\begin{theorem}[Adams]
Suppose there exists $f : S^{4n - 1} \to S^{2n}$ with $h(f) = 1$ then $n = 1, 2, 4$. 
\end{theorem}

\begin{rmk}
This is related to the following fact.
\end{rmk}

\begin{theorem}
Real division algebras has dimensions $n = 1,2,4,8$. 
\end{theorem}

\section{Mar 30}

Since the Hopf invariant $h : \pi_{4n - 1}(S^{2n}) \to \Z$ is a homomorphism and in general there exists $f$ with $h(f) = 2$ so surjectivity of the Hopf map is equivalent to $h(f)$ being odd for some $f$. 
\bigskip\\
If $S^{2n-1}$ has the structure of an H-space with a strict unit (i.e the operation is unital not unital up to homotopy) then there is a map $f : S^{4n - 1} \to S^{2n}$ with $h(f) = 1$. 

\begin{prop}
For any $n \ge 1$ the space $S^{2n}$ does not admit an $H$-space structure.
\end{prop}

\begin{proof}
Suppose $\mu : S^{2n} \times S^{2n} \to S^{2n}$ were such a structure. This would give a map on cohomology,
\[ H^{2n}(S^{2n};\Z) \to H^{2n}(S^{2n} \times S^{2n}; \Z) \]
which must be such that when composed with the inclusions $S^{2n} \to S^{2n} \times S^{2n}$ we get the identity thus $[S^{2n}] \mapsto x + y$ but by naturality of the cup product we must then have $(x+y)^2 = 0$ which is false since $2n$ is even we have $(x+y)^2 = 2 xy$ is nonzero since $xy$ is a generator.
\end{proof}

\begin{theorem}
If $\R^n$ has a division algebra structure on it then $S^{n-1}$ admits the structure of an H-space with a strict unit.
\end{theorem}

\begin{proof}
WLOG we may assume that the division algebra is unital. Then define $\mu  : S^{n-1} \times S^{n-1} \to S^{n-1}$ via $\mu(x,y) = \frac{x \cdot y}{|x \cdot y|}$ which is well-defined since $x \cdot y = 0$ iff $x = 0$ or $y = 0$ because it is a division algebra structure and $x,y \in S^{n-1}$ are nonzero.
\end{proof}

\begin{corollary}
If $\R^n$ admits a division algebra structure then $n = 1,2,4,8$ these are $\R, \CC, \mathbb{H}, \mathbb{O}$.
\end{corollary}

\begin{proof}
By above we must have an H-space structue on $S^{n-1}$ so $n-1$ is odd, take $n = 2k$. Then $h : \pi_{4k - 1}(S^{2k}) \to \Z$ must be surjective. By Adam's theorem, $k = 1,2,4$ and thus $n = 2,4,8$. 
\end{proof}

\begin{cor}
If $S^{n-1}$ is parallelizable then $S^{n-1}$ is an H-space with strict unit.
\end{cor}

\begin{proof}
Given indpendent vector fields $v_1, \dots, v_{n-1}$ on $S^{n-1}$. By Gram-Schmidt we can assume these everywhere form an orthonormal basis and at some point $e_n$ is the standard basis $e_1, \dots, e_{n-1}$. Then at each $x \in S^{n-1}$ we get an orthonormal basis of $\R^n$ via $\{v_1(x), \dots, v_{n-1}(x), x \}$ which forms a matrix $\alpha_x =  \, v_1(x) \, \dots \, v_{n-1}(x)] \in \mathrm{SO}(n)$ note it is determinant one by our choice at $P$ and connectedness of $S^{n-1}$. Then define $\mu : S^{n-1} \times S^{n-1} \to S^{n-1}$ via $\mu(x,y) = \alpha_x y \in S^{n-1}$ since $\alpha_x$ is an automorphism of the sphere. Furthermore, $\mu(e_n, x) = \mu(x, e_n) = x$ giving an $H$-space structure on $S^{n-1}$ with strict unit. 
\end{proof}

\begin{cor}
$S^n$ is parallelizable iff $n = 0,1,3,7$ with trivialization explicitly given by the multiplication structure on the units in the $\R$-algebras $\R, \CC, \mathbb{H}, \mathbb{O}$.
\end{cor}

\subsection{Proof of Adam's Theorem}

We reinterpret $h(f)$ in terms of K-theory. If $f : S^{4n - 1} \to S^{2n}$ is a map then $C_f = S^{2n} \cup_f D^{4n}$. Then $C_f / S^{2n} = S^{4n}$. Then we look at the LES of the pair $(C_f, S^{2n})$ in $K$-theory,
\begin{center}
\begin{tikzcd}
0 \arrow[r] & \wt{K}(S^{4n}) \arrow[r] & \wt{K}(C_f) \arrow[r] & \wt{K}(S^{2n}) \arrow[r] & 0
\end{tikzcd}
\end{center}
using that $\wt{K}^1(S^{4n}) = \wt{K}^1(S^{2n}) = 0$. Then the canonical class $(H - 1)^2 \mapsto \beta$ in $\wt{K}(C_f)$ and $\alpha \mapsto (H - 1)^n$ then $\alpha^2 \mapsto 0$ so it is in the image and thus $\alpha^2 = h(f) \beta$ where $h(f)$ is the Hopf invariant.

\subsection{Adams Operations}

Given $k \in \N$ there is an operation $\psi^k : K(X) \to K(X)$ which is a ring homomorphism which satisfy,
\begin{enumerate}
\item $\psi^k(L) = L^{\otimes k}$ where $L$ is a line bundle
\item naturality in $f : X \to Y$ then $f^* \circ \psi^k = \psi^k \circ f^*$ 
\item $\psi^k \circ \psi^\ell = \psi^{k \ell}$
\item $\psi^p(\alpha) \equiv \alpha^p \mod{p}$ for $p$ prime
\end{enumerate}

\begin{theorem}
These Adams operations exist.
\end{theorem}

\begin{proof}
Given a vector bundle $E$ apply the functor $E \mapsto \bigwedge^\ell E$. Notice that,
\[ \bigwedge^\ell (L_1 \oplus \cdots \oplus L_n) = \bigoplus_{i_1 < \dots < i_\ell} L_{i_1} \otimes \cdots \otimes L_{i_j} = \sigma_j(L_1, \dots, L_n) \]
Furthermore, 
\[ \psi^k(L_1 \oplus \cdots \oplus L_n) = L_1^k \oplus \cdots \oplus L^k_n \]
Now, since the polynomial $t_1^k + \cdots + t_n^k$ is symmetric we get a Newton decomposition,
\[ t_1^k + \cdots + t^k_n = S_k(\sigma_1(t_1, \dots, t_n), \dots, \sigma_k(t_1, \dots, t_n)) \]
Therefore, we define,
\[ \psi^k(E) := S_k(\wedge^1(E), \dots, \wedge^k(E)) \]
which gives the required form on line bundles and is clearly natural. Using the splitting principle, it suffices to check that the properties hold for sums of line bundles. 
\bigskip\\
Clealy on sums of line bundles $\psi^k$ is a ring homomorphism and $\psi^k \circ \psi^\ell = \psi^{k \ell}$ is clear since it is simply exponentiation. Furthermore, assuming $E = L_1 \oplus \cdots \oplus L_n$ we see,
\[ \psi^p(E) = L_1^p \oplus \cdots \oplus L_n^p \equiv (L_1 \oplus \cdots \oplus L_n)^p = E^p \mod{p} \]
since the multinomial coefficients are all divisible by $p$. 
\end{proof}

\section{Proof of Adams Theorem}

Let $f : S^{4n - 1} \to S^{2k}$ be some map with odd Hopf invariant and take $C_f = S^{2n} \cup_f D^{4n}$. Then consider the exact sequence,
\begin{center}
\begin{tikzcd}
0 \arrow[r] & \wt{K}(S^{4n}) \arrow[r] & \wt{K}(C_f) \arrow[r] & \wt{K}(S^{2n}) \arrow[r] & 0
\end{tikzcd}
\end{center}
Choose generators $\alpha \in \wt{K}(S^{4n})$ and $\gamma \in \wt{K}(S^{2n})$ and $\beta \in \wt{K}(C_f)$ mapping to $\gamma$. Then, since $\gamma^2 = 0$ we must have $\beta^2$ in the kernel so by exactness $\beta^2 = h(f) \alpha$. By the proposition $\psi^{k}(\alpha) = k^{2n} \alpha$. Then $\psi^k(\beta)$ maps to $k^n \cdot \gamma$. Therefore,
\[ \psi^k(\beta) = k^n \beta  + \mu_k \alpha \]
for $\mu_k \in \Z$. However, $\psi^k \circ \psi^\ell = \psi^{k\ell}$ applying to $\alpha$ and $\beta$ gives,
\[ (k^{2n} - k^n) \mu_\ell = (\ell^{2n} - \ell^n) \mu_k \]
Furthermore, $\psi^2(\beta) \equiv \beta^2 \mod{2}$ but $\beta^2 = h(f) \alpha$. Thus $s^n \beta + \mu_2 \alpha$ but $h(f)$ is odd so $\mu_2$ must be odd. Plugging in $k = 3$ and $\ell = 2$ we get,
\[ (e^{2n} - 3^n) \mu_2 = (2^{2n} - 2^n) \mu_3 \]
But $\mu_2$ is odd and thus $2^n \divides 3^n - 1$ which implies $n = 1,2,4$. 

\begin{prop}

\end{prop}

\section{Homotopy Groups of Spheres}



\section{How Many Lineary Independent Vector Fields are on $S^{n-1}$}

We showed that if $S^{n-1}$ is parallelizable then $n = 1,2,4,8$. Furthermore, since $\chi(S^{n-1}) = 2$ when $n-1$ is even which implies there are no nonvanishing vector fields.

\begin{thm}[Adams]
Let $n = (2a + 1) 2^{4b + c}$ for $0 \le c \le 3$. Then there are exactly $2^c + 8b - 1$ linearly indpendent vector fields on $S^{n-1}$. This is the Roda-Hurewicz number of $n$.
\end{thm}


\subsection{Clifford Algebras}

\newcommand{\Cl}[1]{\mathrm{C}\ell \left( #1 \right)}

\begin{defn}
Let $(V, Q)$ be a vector space with a quadratic form. We define the clifford algebra, 
\[ \Cl{V,Q} = T(V) / (v \otimes v - Q(v)) \] 
which is a unital associative algebra but not a division algebra. 
\end{defn}

\begin{example}
\begin{enumerate}
\item If $Q = 0$ then $\Cl{V, Q} = \bigwedge^{*} V$.
\item If $Q_{\mathrm{std}}(x_1, \dots, x_n) = -(x_1^2 + \cdots + x_n^2)$ on $V = \R^n$. Then,
\[ \Cl{\R^n, Q_{\mathrm{std}}} = \frac{\R[e_1, \dots, e_n]}{(e_i^2 + 1, e_i e_j + e_j e_i)} \] Setting $\Cl{n} = \Cl{\R^n, Q_{\mathrm{std}}}$ we get,
\begin{enumerate}
\item $\Cl{0} = \R$
\item $\Cl{1} = \CC$
\item $\Cl{2} = \H$
\item $\Cl{3} = \H \oplus \H$
\item $\Cl{4} = M_{2}(\H)$
\item $\Cl{5} = M_{4}(\CC)$
\item $\Cl{6} = M_8(\R)$
\item $\Cl{7} = M_8(\R) \oplus M_8(\R)$
\item $\Cl(8) = M_{16}(\R)$
\end{enumerate}
\end{enumerate}
\end{example}

\begin{rmk}
In general, $\dim{\Cl{V,Q}} = 2^{\dim{V}}$ with basis $e_1^{i_1} \dots e_n^{i_n}$ for $i_k = 0,1$. Furthermore, the composition $V \embed T(V) \onto \Cl{V, Q}$ is injective.
\end{rmk}

\begin{prop}
The Clifford algebra $\Cl{V, Q}$ satisfies the following universal property. For any linear map $\phi : V \to A$ with $A$ an associative algebra with $\phi(v)^2 = q(v) \cdot 1$ then we get a diagram,
\begin{center}
\begin{tikzcd}
V \arrow[d, hook] \arrow[r, "\phi"] & A
\\
\Cl{V,Q} \arrow[ru, dashed]
\end{tikzcd}
\end{center}
\end{prop}

\begin{thm}
Periodicty theorem if $\Cl{k} \cong M_n(F)$ then $C_{k+8} \cong M_{16 n}(F)$. Likewise, if $\Cl{k} \cong M_n(F) \oplus M_n(F)$ then $\Cl{k+8} = M_{16n}(F) \oplus M_{16n}(F)$. 
\end{thm}

\begin{rmk}
$M_n(F)$ has a unique irreducible representation $F^n$. 
\end{rmk}

\end{document}
