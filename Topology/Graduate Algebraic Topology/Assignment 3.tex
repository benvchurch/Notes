\documentclass[12pt]{extarticle}
\usepackage{import}
\import{./}{Includes}
\newcommand{\C}{\mathbb{C}}
\newcommand{\F}{\mathbb{F}}

\begin{document}
\atitle{3}

\section{Fomenko-Fuchs Chapter 18}

\subsection{9}

Let $X$ be a based CW complex (at $x_0$) with $\pi_0(X) = \pi_1(X) = \pi_2(X) = \cdots = \pi_{n-1}(X) = 0$ and $\pi_n(X) \neq 0$. Then consider the Serre fibration $E X \to X$ which has fiber $\Omega X$. Recall there is a canonical isomorphism $\pi_{i}(\Omega X) = \pi_{i + 1}(X)$ and thus we see,
\[ \pi_0(\Omega X) = \pi_1(\Omega X) = \cdots = \pi_{n-2}(\Omega X) = 0 \]
and $\pi_{n-1}(\Omega X) \neq 0$ so the primary obstruction of $EX \to X$ is given by a section $s : X^{n-1} \to EX$ and lies in $H^n(X; \pi_{n-1}(\Omega X)) = H^n(X; \pi_n(X))$. Note, that because $X$ is $(n-1)$-connected we can replace (up to homotopy equivalence) $X$ with a CW complex with one $0$-cell and no $k$-cells for $k = 1, \dots, n-1$. Then $X^{n-1}$ is a point so the section $s : X^{n-1} \to EX$ is simply sending the base point to the trivial loop at the base-point. The primary obstruction of $EX \to X$ is then,
\[ O_s \in H^n(X; \pi_n(X)) \]
defined as follows. For each $n$-cell $D^n$ with attaching map $f : S^{n-1} \to X^{n-1}$ (which must be the constant map) and we get maps $h : D^n \to X$ including each $n$-cell. Now, consider the pullback $h^* E X$ over $h : D^n \to X$ and the pullback of the section $s : X^{n-1} \to E X$ gives a section,
\[ h^* s : \partial D^n \to h^* EX \]
First, we need to trivialize $h^* EX$. Choose some base point $\tilde{x}_0 \in \partial D^n$ then for any $x \in D^n$ let $\gamma_x$ denote the linear path in $D^n$ from $x$ to $\tilde{x}_0$ which gives $h \circ \gamma_x$ a canonical path on $S^n \subset X$ from each point on $S^n$ to the base-point $x_0$ (note that these paths are continuous in $x \in D^n$ but not (and in fact multi-valued) as a function of $x \in S^n$).
\bigskip\\
Such choices gives an isomorphism $h^* EX \cong D^n \times \Omega X$ as follows. Note that,
\[ h^* EX = \{ (x, \gamma) \mid x \in D^n \text{ and } \gamma : I \to X \text{ with } \gamma(0) = x_0 \text{ and } \gamma(1) = x \} \]
Now consider the map $(x, \gamma) \mapsto (x, (h \circ \gamma_x) * \gamma)$ giving a loop at $x_0$, $(h \circ \gamma_x) * \gamma \in \Omega X$. In particular, consider the section $h^* s : \partial D^n \to h^* E X$ which sends $x \mapsto (x, e_{x_0})$ where $e_{x_0}$ is the constant path at $x_0$. Under the isomorphism we get $h^* s : \partial D^n \to D^n \times \Omega X$ given by $x \mapsto (x, (h \circ \gamma_x) * e_{x_0}) = (x, h \circ \gamma_x)$. Using $\partial D^n = S^{n-1}$, the section $h^* s : S^{n-1} \to D^n \times \Omega X$ defines a class $[h^* s] \in \pi_{n-1}(D^n \times \Omega X) = \pi_{n-1}(\Omega X)$ via $S^{n-1} \to \Omega X$ sending $x \mapsto h \circ \gamma_x$. This map is homotopic to the adjunction $S^{n-1} \to \Omega X$ of the inclusion $h : S^n \to X$. Therefore, $[h^* s] \in \pi_{n-1}(\Omega X) = [h] \in \pi_n(X)$. Therefore, the obstruction class,
\[ O_s \in H^n(X; \pi_n(X)) \]
is the class sending each cell $h : D^n \to X$ to $[h] \in \pi_n(X)$ in particular, it sends the generators $[h] \in H_n(X; \Z)$ to $[h] \in \pi_n(X)$ (the inverse of the Hurewicz isomorphism $h_n : \pi_n(X) \to H_n(X; \Z)$) thus $O_s = [X]$ the fundamental class of $X$.

\section{Fomenko-Fuchs Chapter 19}

\subsection{1}

First, note that if $F_\C = E \otimes_\R \C$ for a real bundle $F$ then the complex conjugation map $\C \to \overline{\C}$ which is an isomorphism of $\C$-vector spaces induces a $\C$-isomorphism $F \otimes_\R \C \to F \otimes_\R \overline{\C}$ i.e. $F_{\C} \cong \overline{F_{\C}}$ as complex vector bundles.
\bigskip\\
Now suppose that $E$ is a rank $n$ complex vector bundle with $\sigma : E \xrightarrow{\sim} \overline{E}$. I claim that we may assume that $\sigma$ is involutive i.e. $E \xrightarrow{\sigma} \overline{E} \xrightarrow{\sigma} \overline{\overline{E}} = E$ is the identity (I will justify this at the end). Now recall that $\overline{E} = E$ as real bundles but the $\C$-action is conjugate. Thus, consider the real sub-bundle $F = \{ x \in E \mid \sigma(x) = \id(x) \}$ thinking of $\id : E \to \overline{E}$ as the $\R$-linear identity. Note this is not a complex sub-bundle because if $x \in F$ then $\sigma(\lambda x) = \lambda \cdot \sigma(x) = \lambda \cdot x = \bar{\lambda} x$ which is not, in general, equal to $\lambda x$ unless $\lambda \in \R$.
\bigskip\\
Viewing $E$ as a real bundle with a complex structure $J : E \to E$ (i.e. a bundle automorphism $J$ with $J^2 = - \id$) we know that the complex structure, $\bar{J} : \overline{E} \to \overline{E}$ defining $\overline{E}$ is $-J : E \to E$. Now, $\sigma : E \to \overline{E}$ is $\C$-linear meaning $\sigma \circ J = \bar{J} \circ \sigma$ and thus $\sigma \circ J = - J \circ \sigma$.
\bigskip\\
Notice, on the fiber that for any $v \in E_x$ we can write $v = v^{+} + v^{-}$ with $v^{\pm} = \tfrac{1}{2}(v \pm \sigma v)$ where $\sigma v^{\pm} = \pm v$ since $\sigma$ is involutive (the same holds for sections). Thus we get a decomposition,
\[ E_{x} = E_{x}^+ \oplus E_{x}^{-} \]
into eigenspaces of $\sigma$ and $F_x = E_{x}^{+}$ is the $+1$-eigenspace of $\sigma$. Since $\sigma \circ J = - J \circ \sigma$, we see that $J$ swaps the sign of $\sigma$-eigenvalues so $J : E \to E$ acts on the fiber-wise decomposition as $J : E_x^{\pm} \to E_x^{\mp}$. In particular, since $J^2 = -\id$ it is invertible so $\dim_{\R}{E_x^{+}} = \dim_{\R}{E_x^{-}} = \tfrac{1}{2} \dim_{\R}{E_x} = n$. Therefore, $F \subset E$ is a rank $n$ \textit{real} sub-bundle.
\bigskip\\
Consider the map $F \otimes_\R \C \to E$ given by $v \otimes \lambda \mapsto \lambda v$. On the fibers this gives $F_x \otimes_\R \C \to E = E_x^{+} \oplus E_x^{-}$ and $F_x = E_x^{+}$ so for any $v \in E_x^{+}$ then $v \otimes 1 \mapsto v$. Furthermore, since $J : E_{x}^{+} \to E_{x}^{-}$ is an isomorphism, we can write any $v' \in E_x^{-}$ as $i v$ for $v \in E_x^{+}$ and thus $v \otimes i \mapsto i v = v'$ so $F_x \otimes_\R \C \to E$ is surjective. Since $\dim_{\C}(F \otimes_\R \C) = n$ ($\dim_{\R}(F) = n$) these bundles have equal rank over $\C$ so any fiber-wise $\C$-linear surjection is an isomorphism. Thus $E \cong F \otimes_\R \C$ for the real sub-bundle $F \subset E$. 
\bigskip\\
Now I justify why $\sigma : E \to \overline{E}$ may be chosen to be involutive. Given any isomorphism $\varphi : E \to \overline{E}$ we can consider $\varphi \circ \varphi : E \to \overline{E} \to \overline{\overline{E}} = E$. It suffices to show that $\varphi^2$ has a $\C$-linear square root $\xi : E \to E$ commuting with $\varphi$ since then we can take $\sigma = \varphi \circ \xi^{-1}$ and,
\[ \sigma^2 = \varphi \circ \xi^{-1} \circ \varphi \circ \xi^{-1} = \varphi^2 \circ \xi^{-2} = \id \]
Choose a Hermitian metric on $E$. Now, on each fiber $E_x$ we can choose square roots $\xi_x$ of $\varphi_x^2 : E_x \to E_x$ (one construction uses the surjectivity of the exponential map $\exp : \mathfrak{gl}(n, \C) \to \mathrm{GL}(n, \C)$ writing $\varphi_x^2 = e^{M}$ then take $\xi_x = e^{\tfrac{1}{2} M}$). Choosing some isomorphism $E_x \cong \C^n$ compatible with the Hermitian metric we can always choose a square root such that $\varphi_x(z) = \overline{\xi_x(z)}$ (where this complex conjugation is non-canonical it is induced by the choice of  isomorphism $E_x \cong \C^n$) and thus $\left< \varphi_x(z), \xi_x(z) \right> = \left< \overline{\xi_x(z)}, \xi_x(z) \right> = |\xi_x(z)|^2 \ge 0$ so the quadratic form $\left< \varphi(-), \xi(-) \right>$ is positive-definite on each fiber and, in fact, there is a unique choice of square root making the form positive-definite (since any other square root would negate some direction relative to $\varphi_x$). This gives a consistent choice of square roots on the fibers, and these $\xi_x$ are clearly continuous on local charts so they glue to give a global automorphism $\xi : E \to E$ satisfying $\xi^2 = \varphi^2$. Furthermore, $\xi \circ \varphi = \varphi \circ \xi$ since the equality $\xi_x \circ \varphi_x = \varphi_x \circ \xi_x$ holds on fibers because, up to some choice of an isomorphism $E_x \cong \C^n$ the maps $\xi_x$ and $\varphi_x$ differ only by complex conjugation. 

\subsection{7}

Consider a complex vector bundle $E$ which we may view as a real vector bundle $E_\R$ of double the rank. Consider the map,
\[ \C \otimes_\R E_\R \to E \oplus \overline{E} \]
as the sum of $\lambda \otimes x \mapsto \lambda x$ and $\lambda \otimes x \mapsto \lambda \cdot x = \bar{\lambda} x$ in $\overline{E}$. I claim this map is an isomorphism. First, note that it is clearly $\C$-linear (using the fact that $\overline{E}$ has conjugate $\C$-linear structure). Now, this map is surjective because for $(x, y) \in E \oplus \overline{E}$ consider,
\[ \tfrac{1}{2} \otimes x + \tfrac{1}{2} i \otimes (-i x) + \tfrac{1}{2} \otimes y - \tfrac{i}{2} i \otimes (-iy) \mapsto (\tfrac{1}{2} x + \tfrac{1}{2} x + \tfrac{1}{2} y - \tfrac{1}{2} y, \tfrac{1}{2} x - \tfrac{1}{2} y + \tfrac{1}{2} y + \tfrac{1}{2} y) = (x, y) \]
Thus, the map $\C \otimes_\R E_\R \to E \oplus \overline{E}$ is surjective and by construction it is $\C$-linear of fibers. Since both sides are rank $2 n$ complex vector bundles this map is an isomorphism since it is given fiber-wise by an invertible linear map.

\subsection{9}

Let $\iota : X \to Y$ be an immersed (or embedded) sub-manifold and $T Y$ be the tangent bundle of $Y$. Since $\iota : X \to Y$ is an immersion, we can an injection $\d{\iota} : TX \to \iota^* TY$ whose quotient is a vector bundle $N_Y X$ called the normal bundle so the canonical exact sequence,
\begin{center}
\begin{tikzcd}
0 \arrow[r] & TX \arrow[r] & \iota^* TY \arrow[r] & N_Y X \arrow[r] & 0
\end{tikzcd}
\end{center}
splits to give $\iota^* TY = TX \oplus N_Y X$. 
\bigskip\\
In particular, consider the case that $Y = \R^m$ for some $m$ i.e. $\iota$ gives an immersion (or embedding) into Euclidean space. Then $TY$ is a trivial bundle so we find, $\iota^* T Y = \underline{\R}^m$ is a trivial bundle on $X$. Therefore, $N_Y X \oplus TX = \underline{\R}^m$. There exists a perpendicular bundle $E$ to $TX$ such that $E \oplus TX = \underline{\R}^{2n}$ so we find,
\[ N_Y X \oplus \underline{\R}^{2n} = E \oplus \underline{\R}^m \]
Therefore, $N_Y X$ is stably equivalent to $E$ which is some bundle defined intrinsically on $X$ (it is the complement to the tangent bundle $TX$) so for any choice of immersion $\iota : X \to Y$ into a Euclidean space the normal bundles $N_Y X$ are stably equivalent.

\subsection{10}

Let $X$ be a $n$-dimensional oriented surface and $\iota : X \to \R^{n+1}$ an immersion. We know that $TX \oplus N X = \underline{\R}^{n+1}$. First, note that the first Stiefel-Whitney class $w_1$ is actually additive since,
\[ w_1(E_1 \oplus E_2) = \sum_{p + q} w_p(E_1) \smile w_q(E_2) = w_1(E_1) \smile 1 + 1 \smile w_1(E_2) = w_1(E_1) + w_2(E_2) \]
Furthermore, recall that $w_1(E) = 0$ if and only if $E$ is orientable. Since $X$ is oriented $w_1(TX) = 0$ and clearly $w_1(\underline{\R}^{n+1}) = 0$ so we have,
\[ w_1(T X \oplus N X) = w_1(TX) + w_1(NX) = w_1(\underline{\R}^{n+1}) = 0 \]
but $w_1(TX) = 0$ so $w_1(NX) = 0$ proving that $NX$ is orientable. However, $\rank{(TX)} = n$ so $NX$ is a line bundle since $\rank{(TX \oplus NX)} = \rank{\underline{\R}^{n+1}} = n+1$. Then we conclude that $NX$ is trivial since it is an orientable real line bundle (Lemma \ref{orientation}). 
\bigskip\\
Finally, we have $TX \oplus N X = \underline{\R}^{n+1}$ but we have shown the normal bundle is trivial, $N X \cong \underline{\R}$ so we find,
\[ TX \oplus \underline{\R} \cong \underline{\R}^{n+1} \]
and thus $TX$ is stably trivial so $X$ is stably parallelizable. 

\subsection{12}

First, we prove that $w(E) = 0$ iff $E$ is orientable iff $E$ is trivial for bundles on $S^1 = \RP^1$. Vector bundles on $I = [0, 1]$ are trivial since $I$ is contractible so, for a bundle $E \to S^1$ we have,
\begin{center}
\begin{tikzcd}
I \times \R^n \arrow[r] \arrow[d] & E \arrow[d]
\\
I \arrow[r] & S^1
\end{tikzcd}
\end{center} 
and thus $E$ is determined by a gluing function $\phi \in \mathrm{GL}(n, \R)$ identifying fibers across the glued point. Now, a path $\gamma : I \to \mathrm{GL}(n, \R)$ gives a map $(t, x) \mapsto (t, \gamma(t)x)$ from $E_{\gamma(0)}$ to $E_{\gamma(1)}$ which is well-defined since $(0, x) \sim (1, \gamma(0) x)$ which map to $(0, \gamma(0) x)$ and $(1, \gamma(1) \gamma(0) x)$ which are equivalent in $E_{\gamma(1)}$ so this is a well-defined map. An inverse path gives the inverse so we see that the bundle is defined by the path-component of $\phi \in \mathrm{GL}(n, \R)$. Therefore, there are only two isomorphism classes for rank $n$ bundles, those with positive determinant and those with negative determinant. The first class is trivial $E = \underline{\R}^n$ (which are clearly orientable) since we can take $\phi = \id$. For the second class, we can take,
\[ \phi = \mathrm{diag}(-1, 1, \dots, 1) \]
and thus $E = \gamma \oplus \underline{\R}^{n-1}$ with $\gamma$ the M\"{o}bius bundle. Thus $E$ is non-orientable and has no non-vanishing sections. Clearly, $w(\underline{\R}^n) = 0$ and, for the second class, since $E$ has no non-vanishing sections, there must be an obstruction on the $1$-skeleton $S^1$ so $w(E) \neq 0$ proving the claim for $S^1$. 
\bigskip\\
Now, we use the following fact: a bundle $E$ on $X$ is orientable if and only if its restriction to any loop $f^* E$ for $f : S^1 \to X$ is trivial (Lemma \ref{loop_restriction}). For a bundle $E$ on $X$ we know that, for any $f : S^1 \to X$ we have $f^* w(E) = w(f^* E)$ with $f^* : H^*(X, \Z / 2 \Z) \to H^*(S^1; \Z / 2 \Z)$.
\bigskip\\
If $w(E) = 0$ then $w(f^* E) = 0$ so we have $f^* E$ is trivial for each loop $f : S^1 \to X$ so $E$ is orientable.
\bigskip\\
Conversely, if $E$ is orientable then we use the fact that the Hurewicz map $h_1 : \pi_1(X) \to H_1(X; \Z)$ is the abelianization (in particular surjective) so, using the universal coefficient theorem,
\begin{center}
\begin{tikzcd}
0 \arrow[r] & \Ext{1}{\Z}{\Z}{\Z/2\Z} \arrow[r] & H^1(X; \Z/2\Z) \arrow[r] & \Hom{H_1(X; \Z)}{\Z / 2 \Z} \arrow[r] & 0
\end{tikzcd}
\end{center}
we see $\Ext{1}{\Z}{\Z}{\Z/2\Z} = 0$ and thus,
\[ H^1(X; \Z / 2 \Z) = \Hom{H_1(X; \Z)}{\Z / 2 \Z} = \Hom{\pi_1(X)}{\Z / 2 \Z} \]
where the map sends a cohomology class $c$ to the map $[f : S^1 \to X] \mapsto c(f_*[S^1])$. 
Now, for any $f : S^1 \to X$ we have $w(f^* E) = 0$ since $E$ is orientable so $f^* w(E) = 0$ and thus
\[ w(E)(f_*[S^1]) (f^* w(E))([S^1]) = w(f^* E)([S^1]) = 0 \]
for any loop so the class $w(E) = 0$ by the above isomorphism. 

\subsection{13}

First, let $E$ be a rank $n$ complex vector bundle on $B$. First, we construct the Euler class of $E_\R$. Let $F$ be the unit bundle of $E_\R$ (note $F$ is also the unit bundle of $E$ as a complex vector bundle since the complex norm and real norm coincide). Then $e(E_\R)$ is the primary obstruction class of $F$ constructed as follows. The fiber of $F$ is $S^{2n - 1}$ so we have no obstruction to giving a section $s : B^{2n-1} \to F$ which gives an obstruction class for extending this section to $B^{2n}$,
\[ e(E_\R) = O_s \in H^{2n}(B; \pi_{2n-1}(S^{2n - 1})) = H^{2n}(B; \Z) \]
where the isomorphism $\pi_{2n - 1}(S^{2n - 1}) = \Z$ is given by the orientation on $E$ (and thus $F$) induced by the complex structure. The Chern class $c_n(E)$ is defined by the primary obstruction of the bundle $E_1 = F$ the unit bundle (i.e. the bundle of unitary $1$-frames),
\[ c_n(E) = O_s \in H^{2n}(B; \pi_{2n - 1}(V_\C(n, 1))) = H^{2n}(B; \Z) \]
where the isomorphism $\pi_{2n-1} (V_\C(n,1)) = \pi_{2n-1}(S^{2n - 1}) = \Z$ is fixed by the orientation given by the complex structure. Thus we immediately see that $e(E_\R) = c_n(E)$. 
\bigskip\\
Now we consider the Stiefel-Whitney class of $E_\R$. First, we just seen that,
\[ w_{2n}(E_\R) = \rho_2 e(E_\R) = \rho_2 c_n(E) \]
using the previous equalities. Now, let $F_k$ be the bundle of orthonormal $k$-frames of $E_\R$. We need to consider the relation between the bundle $F_k$ and the bundle of unitary orthonormal $k$-frames $E_k$. There is an inclusion map $E_k \embed F_{2k}$ because we can choose the Euclidean metric on $E_\R$ to be the real part of the Hermitian metric on $E$ and thus unitary orthonormal frames are real orthogonal (although not vice-versa) and each unitary orthonormal $k$-frame defines a real orthonormal $2k$-frame via sending each $e_j$ to $e_j, i e_j$. Recall that these bundles have fibers $V(2n, 2k)$ and $V_\C(n, k)$ respectively with,
\begin{align*}
\pi_i(V(2n, 2k)) &= 0 \quad \text{ for } \quad i < 2(n - k) 
\\
\pi_i(V_\C(n, k)) &= 0 \quad \text{ for } \quad i < 2 (n - k) + 1
\end{align*} 
Therefore, there is no obstruction to giving a section $s : B^{2 (n - k) + 1} \to E_k$ which will define an obstruction class,
\[ c_{n - k + 1}(E) = O_s \in H^{2(n - k + 1)}(B; \pi_{2(n - k) + 1}(V_\C(n, k))) \]
Via the inclusion $E_k \embed F_{2k}$, we get a section $s : B^{2 (n - k) + 1} \to F_{2k}$ but the primary obstruction of the bundle $F_{2k}$ occurs on the $2(n - k)$-skeleton since $\pi_{2(n-k)}(V(2n, 2k)) \neq 0$ so the primary obstruction vanishes since we have produced an extension to the $2(n-k) + 1$-skeleton. Recall that $w_{j}(E_\R)$ is defined as follows: consider the bundle $F_{\ell}$ with $j = 2n - \ell + 1$ and its primary obstruction occurs for a section $s : B^{2n-\ell} \to F_{\ell}$ which is the class,
\[ w_{j}(E_\R) = O_{s} \in H^{2n - \ell + 1}(B; \Z / 2 \Z) \]
which is the obstruction to finding a section $s' : B^{2n - \ell + 1} \to F_{\ell}$. Therefore, $w_{2(n - k) + 1}(E_\R) = 0$ since then $\ell = 2 k$ and $F_{2k}$ admits a section $B^{2(n - k) + 1} \to F_{2k}$ as demonstrated above so the odd Stiefel-Whitney classes vanish for a complex vector bundle.
\bigskip\\
Now we compute the even classes, $w_{2(n-k + 1)}$ which are defined by the obstruction class of $F_{2k - 1}$. There is a map $E_{k} \embed F_{2k} \to F_{2k - 1}$ by throwing out some element of the basis. There is no obstruction to finding a section $s : B^{2(n - k) + 1} \to E_k$ (using the vanishing of the homology groups above) which gives an obstruction class,
\[ c_{n-k+1}(E) = O_s \in H^{2(n - k + 1)}(B; \pi_{2(n-k) + 1}(V_\C(n, k))) = H^{2(n - k + 1)}(B; \Z) \]
Now, via the map $E_{k} \to F_{2k - 1}$, this gives a section $\tilde{s} : B^{2(n - k) + 1} \to F_{2k - 1}$. Since,
\[ \pi_{(2 n - k) + 1}(V(2n, 2k - 1)) \neq 0 \]
the obstruction class of this section gives the primary obstruction of the bundle $F_{2k - 1}$,
\[ w_{2(n - k + 1)} = O_{\tilde{s}} \in H^{2(n - k + 1)}(B; \Z / 2 \Z) \]
Furthermore, the map $E_k \to F_{2k - 1}$ on the fiber induces,
\[ \pi_{2(n - k) + 1}(V_\C(n, k)) \to \pi_{2(n - k) + 1}(V(2n, 2k - 1)) \]
I claim this map is nontrivial so it is the unique nonzero map $\Z \to \Z / 2 \Z$, namely $\rho_2$, reduction modulo 2. Therefore, 
\[ w_{2(n - k + 1)}(E_\R) = O_{\tilde{s}} = \rho_2 O_{s} = \rho_2 c_{n - k + 1}(E) \]
proving that $w_{2j}(E_\R) = \rho_2 c_{2j}(E)$. 
\bigskip\\
Now I justify that the map,
\[ \pi_{2(n - k) + 1}(V_\C(n, k)) \to \pi_{2(n - k) + 1}(V(2n, 2k - 1)) \]
is nontrivial. Consider the diagram,
\begin{center}
\begin{tikzcd}
V_\C(n, k) \arrow[r] & V(2n, 2k - 1)
\\
V_\C(n - k + 1, 1) \arrow[u] \arrow[d, equals] \arrow[r] & V(2(n-k + 1), 1) \arrow[u] \arrow[d, equals]
\\
S^{2(n - k) + 1} \arrow[r, "\id"] & S^{2(n - k) + 1}
\end{tikzcd}
\end{center}
where the top map is the fiber map of $E_k \to F_{2k - 1}$ the middle map is given by considering a unit vector $u \in \C^{n - k + 1}$ and sending it to the orthonormal $2$-frame $u, iu$ of $\R^{2(n - k + 1)}$ and then forgetting the second vector to get a $1$-frame i.e. $u \mapsto u$ so the identity on $S^{2(n - k) + 1} \to S^{2(n - k) + 1}$. Therefore, taking homotopy groups we get,
\begin{center}
\begin{tikzcd}
\pi_{2(n-k) + 1}(V_\C(n, k)) \arrow[r] & \pi_{2(n-k) + 1}(V(2n, 2k - 1))
\\
\pi_{2(n-k) + 1}(V_\C(n - k + 1, 1)) \arrow[u] \arrow[d, equals] \arrow[r] & \pi_{2(n-k) + 1}(V(2(n-k + 1), 1)) \arrow[u] \arrow[d, equals]
\\
\pi_{2(n-k) + 1}(S^{2(n - k) + 1}) \arrow[r, "\id"] & \pi_{2(n-k) + 1}(S^{2(n - k) + 1})
\end{tikzcd}
\end{center}
The bottom map is the identity $\id : \Z \to \Z$. The left-hand upward map $\pi_{2(n-k) + 1}(V_\C(n - k + 1, 1)) \to \pi_{2(n-k) + 1}(V_\C(n, k))$ is an isomorphism from repeated application of the LES of the fibration
\[ V_\C(n - 1, k - 1) \embed V_\C(n, k) \to S^{2n - 1} \]
(and using induction, see the next problem for details). Finally, the right-hand upward map is a surjection from repeated application of the LES of the fibration,
\[ V(n - 1, k - 1) \embed V(n, k) \to S^{n - 1} \] Explicitly, we get from the LES,
\begin{center}
\begin{tikzcd}
\pi_{i+1}(S^{n-1}) \arrow[r] & \pi_{i}(V(n - 1, k - 1)) \arrow[r] & \pi_i(V(n, k)) \arrow[r] & \pi_i(S^{n-1}) 
\end{tikzcd}
\end{center}
Thus, for $i + 1 < n - 1$ we get an isomorphism $\pi_i(V(n - 1, k - 1)) \xrightarrow{\sim} \pi_i(V(n, k))$. In particular, for $i = n - k$ this works when $k > 2$ so we have $\pi_{n-k}(V(n - k + 2, 2)) = \pi_{n-k}(V(n, k))$. Now, the last step would use a fibration $V(j + 1, 1) \embed V(j + 2, 2) \to S^{j+1}$ where $j = n - k$ so, applying the LES gives,
\begin{center}
\begin{tikzcd}
\pi_{j+1}(S^{j + 1}) \arrow[d, equals] \arrow[r] & \pi_j(V(j + 1, 1)) \arrow[r] & \pi_j(V(j + 2, 2)) \arrow[r] & \pi_j(S^{j+1}) \arrow[d, equals] 
\\
\Z & & & 0
\end{tikzcd}
\end{center} 
showing that $\pi_j(V(j + 1, 1)) \to \pi_j(V(j + 2, 2))$ is surjective and thus, in combination with the isomorphism above, $\pi_{n-k}(V(n - k + 1, 1)) \to \pi_{n-k}(V(n - k + 2, 2)) \to \pi_{n - k}(V(n, k))$ is a surjection. In the case we were interested in, we had $n \mapsto 2n$ and $k \mapsto 2k - 1$ giving a surjection,
\[ \pi_{2(n - k) + 1}(V(2(n - k + 1), 1)) \onto \pi_{2(n - k) + 1}(V(2n, 2k - 1)) \]
as desired. 
\bigskip\\
Therefore, $\pi_{2(n-k) + 1}(V_\C(n, k)) \to \pi_{2(n-k) + 1}(V(2n, 2k - 1))$ must be surjective since it is part of a commutative square of surjective maps. This completes the proof.

\subsection{15}

Consider a complex rank $n$ vector bundle $E$ on $B$. Recall that the Chern class is defined by considering the bundle of unitary $k$-frames $E_k$ which has fiber $V_\C(n, k)$. Then $\pi_i(V_\C(n, k)) = 0$ for $i < 2(n - k) + 1$ and thus we have no obstruction to finding a section $s : B^{2(n - k) + 1} \to E_k$ the obstruction is in extending this section to the $2(n - k + 1)$-skeleton which gives a class,
\[ c_{n-k+1}(E) = O_{s} \in H^{2(n - k + 1)}(B; \pi_{2(n - k) + 1} V_\C(n, k)) \]
We have an isomorphism, $\pi_{2 (n - k + 1)} (V_\C(n, k)) \cong \Z$ which I claim is canonically given by orientation of the complex structure. Note that, for the conjugate bundle $\overline{E}$, we have the bundle $\overline{E}_k$ which has fiber $\overline{V_\C(n, k)}$. Note that the underlying real vector bundle of $\overline{E}$ and $\overline{E}_k$ agree with $E$ and $E_k$ so we may choose the same section $\bar{s} = s : B^{2(n - k) + 1} \to \overline{E}_k$. Therefore,
\[ c_{n-k+1}(\overline{E}) = O_{\bar{s}} \in H^{2(n - k + 1)}(B; \pi_{2(n - k) + 1} \overline{V_\C(n, k)}) \]
is an identical class but there may be a different canonical generator of $\pi_{2(n - k) + 1} \overline{V_\C(n, k)}$.
\bigskip\\
There is a fibration $V_\C(n-1, k-1) \embed V_\C(n, k) \to S^{2n - 1} \subset \C^{n}$ given by sending an orthonormal frame to its first unit vector. Then, from the LES we get,
\begin{center}
\begin{tikzcd}
\pi_{i+1}(S^{2n - 1}) \arrow[r] & \pi_i(V_\C(n - 1, k - 1)) \arrow[r] & \pi_i(V_\C(n, k)) \arrow[r] & \pi_i(S^{2n - 1}) 
\end{tikzcd}
\end{center}
When $k > 1$ take $i = 2(n - k) + 1$ we get an isomorphism,
\[ \pi_{2(n-k)+1}(V_\C(n, k)) = \pi_{2(n - k) + 1}(V_\C(n-1, k-1)) \]
inductively, we get, 
\[ \pi_{2(n-k)+1}(V_\C(n,k)) = \pi_{2(n-k)+1}(V_\C(n-k+1,1)) = \pi_{2(n-k)+1}(S^{2(n-k) + 1}) \cong \Z \]
All the isomorphisms are canonical except for the last $\pi_{2(n-k)+1}(S^{2(n-k)+1}) \cong \Z$ which depends on the orientation. There is a canonical orientation on $\C^{n-k+1}$ which induces a canonical generator of $\pi_{2(n-k) + 1}(S^{2(n-k) + 1})$. Conjugating the complex structure on $\C^{n-k+1}$ induces a factor of $(-1)^{n-k+1}$ on the orientation (since it corresponds to inverting the $n-k+1$ complex directions) and thus our oriented choice of isomorphism gives $\pi_{2(n-k) + 1}(\overline{V_\C(n, k)}) = (-1)^{n-k+1} \Z$. Therefore, we have,
\[ c_{n-k+1}(\overline{E}) = O_{\bar{s}} \in H^{2(n-k+1)}(B; (-1)^{n-k+1} \Z) \]
Notice that $O_{\bar{s}} \mapsto O_{s}$ under the isomorphism induced by $V_\C(n, k) = \overline{V_\C(n, k)}$ (ignoring the complex orientation) so once we introduce the orientation which may give opposite signs to the generators of the homotopies of the above two identified spaces we get,
\[ c_{n-k+1}(\overline{E}) = O_{\bar{s}} = (-1)^{n-k+1} O_{s} \in H^{2(n-k+1)}(B; \Z) \]
therefore,
\[ c_j(\overline{E}) = (-1)^j c_j(E) \in H^{2j}(B; \Z) \]
Finally, if $E$ is a rank $n$ real line bundle. Then $E \otimes_\R \C$ is isomorphic to its dual and thus,
\[ 2 c_{2j+1}(E \otimes_\R \C) = 0 \]

\subsection{18}

We will use the splitting principle to prove these statements which says it suffices to check $\Z / 2 \Z$-characteristic class relations on bundles of the form $E = \zeta \oplus \cdots \oplus \zeta$ on $X = \RP^\infty \times \cdots \times \RP^\infty$. This holds because $\Z/2\Z$-characteristic classes are polynomials in the Stiefel-Whitney classes and $w_i(\zeta \oplus \cdots \oplus \zeta) = e_i(x_1, \dots, x_n)$ so a polynomial in $w_i$ can only vanish on these bundles if it is the zero polynomial. Similar statements hold for the tautological bundle on $\CP^\infty$ for Chern classes.
\bigskip\\
Consider, the class,
\begin{align*}
w_n(\xi \otimes \zeta) - \sum_{i = 0}^{n} w_i(\xi) \times x^{n-i} \in H^n(X \times \RP^\infty; \Z /2 \Z) 
\end{align*}
where $x \in H^1(\RP^\infty; \Z/2\Z)$ is the generator in the cohomology ring $H^*(\RP^\infty; \Z / 2 \Z) = \F_2[x]$.
Using the splitting principle and naturality, it suffices to check this vanishes for $X = \RP^\infty \times \cdots \times \RP^\infty$ and $\xi = \zeta \oplus \cdots \oplus \zeta$. We write $y_i$ for the generators of the cohomology $H^*(\RP^\infty; \Z/2\Z)$ in the left-hand factor. Now,
\begin{align*}
w_n \left( (\zeta \oplus \cdots \oplus \zeta) \otimes \zeta \right) & = w_n(\zeta \otimes \zeta \oplus \cdots \oplus \zeta \otimes \zeta)
\\
& = \big( w(\zeta \otimes \zeta) \cdots w(\zeta \otimes \zeta) \big)_n
\\
& = \big( (1 + y_1 + x) \cdots (1 + y_n + x) \big)_n
\end{align*}
using $w_1(L_1 \otimes L_2) = w_1(L_1) + w_1(L_2)$ for line bundles and thus $w(L_1 \otimes L_2) = 1 + w_1(L_1) + w_1(L_2)$. 
Therefore,
\begin{align*}
w_n \left( (\zeta \oplus \cdots \oplus \zeta) \otimes \zeta \right) & = \big( (1 + y_1 + x) \cdots (1 + y_n + x) \big)_n
\\
& = (y_1 + x) \cdots (y_n + x) = \sum_{i = 0}^n e_i(y_1, \dots, y_n) x^{n-i} 
\\
& = \sum_{i = 0}^n w_i(\zeta \oplus \cdots \oplus \zeta) \times x^{n-i} 
\end{align*}
proving the formula.
Therefore, for any rank $n$ bundle $\xi$ on $X$ we find,
\[ \rho_2 e(\xi \otimes \zeta) = w_n(\xi \otimes \zeta) = \sum_{i = 0}^n w_i(\xi) \times x^{n-i} \in H^n(X \times \RP^\infty; \Z / 2 \Z) \]
\bigskip\\
Now we prove the corresponding formula for Chern classes. Let $\zeta_\C$ be the tautological bundle on $\CP^\infty$ and let $x \in H^2(\CP^\infty;\Z)$ be the generator of $H^*(\CP^\infty ; \Z) = \Z[x]$ (with $x$ having degree $2$). For a rank $n$ complex vector bundle $\xi$ on $X$, consider the class,
\[ c_n(\xi \otimes \zeta_\C) - \sum_{i = 0}^n c_i(\xi) \times x^{n-i} \]
It suffices to check that this class vanishes on bundles $\xi = \zeta_\C \oplus \cdots \oplus \zeta_\C$ on $X = \CP^\infty \times \cdots \times \CP^\infty$. We write $y_i$ for the generators $y_i \in H^2(\CP^\infty; \Z)$. Now consider,
\begin{align*}
c_n((\zeta_\C \oplus \cdots \oplus \zeta_\C) \otimes \zeta_\C) & = c_n(\zeta_\C \otimes \zeta_\C \oplus \cdots \oplus \zeta_\C \otimes \zeta_\C) 
\\
& = \big( c(\zeta_\C \otimes \zeta_\C) \cdots c(\zeta_\C \otimes \zeta_\C) \big)_n
\\
& = \big( (1 + y_1 + x) \cdots (1 + y_n + x) \big)_n
\\
& = (y_1 + x) \cdots (y_n + x) 
\\
& = \sum_{i = 0}^n e_i(y_1, \dots, y_n) \cdot x^{n-i}
\\
& = \sum_{i = 0}^n c_i(\zeta_\C \oplus \cdots \oplus \zeta_\C) \times x^{n-i}
\end{align*}
so the class vanishes on bundles of the form $\xi = \zeta_\C \oplus \cdots \oplus \zeta_\C$.
Therefore, for any rank $n$ complex vector bundle $\xi$ on $X$ we find,
\[ e(\xi \otimes \zeta_\C) = c_n(\xi \otimes \zeta_\C) = \sum_{i = 0}^n c_i(\xi) \times x^{n-i} \in H^{2n}(X \times \CP^\infty; \Z) \]

\section{Milnor-Stasheff}

\subsection{4C}

Let $\xi \subset T \RP^n$ be a rank-$2$ sub-bundle (we will restrict the case to $n = 4,6$). Since $T \RP^n$ is a real bundle on a compact manifold, we may give it a metric and thus the sub-bundle $\xi \subset T \RP^n$ admits a orthogonal complement which decomposes $T \RP^n = \xi \oplus \xi^\perp$. Therefore,
\[ w(T \RP^n) = w(\xi) \cdot w(\xi^\perp) \]
We have already computed,
\[ w(T \RP^n) \in H^*(\RP^n, \Z/2\Z) = \F_2[\alpha]/(\alpha^{n+1}) \]
to be,
\[ w(T \RP^n) = \sum_{k = 0}^{n+1} { n + 1 \choose k } \alpha^k \mod 2 \]
Furthermore, 
\[ w(\xi) = 1 + a_1 \alpha + a_2 \alpha^2 \]
Now, first restrict to the case $n = 4$ we see,
\[ w(T \RP^4) = 1 + \alpha + \alpha^4 \]
and since $\xi^\perp$ is also of rank $2$ we gave,
\[ w(\xi^\perp) = 1 + b_1 \alpha + b_2 \alpha^2 \]
We need,
\[ (1 + a_1 \alpha + b_1 \alpha^2) \cdot (1 + b_1 \alpha + b_2 \alpha^2) = 1 + \alpha + \alpha^4 \]
However, expanding,
\[ 1 + (a_1 + a_2) \alpha + (a_1 b_1 + a_2 + b_2) \alpha^2 + (a_1 b_2 + a_2 b_1) \alpha^3 + a_2 b_2 \alpha^4 \]
with $a_1, a_2, b_1, b_2 \in \F_2$. Thus we need $a_2, b_2 = 1$ since $a_2 b_2 = 1$ and $a_1 + b_1 = 1$ so $a_1 b_2 + a_2 b_1 = a_1 + b_1 = 1$ so the coefficient of $\alpha^3$ is nonzero showing that the above factorization is impossible and thus no such $\xi$ exists.
\bigskip\\
Now consider the case $n = 6$. We have,
\[ w(T \RP^6) = 1 + \alpha + \alpha^2 + \alpha^3 + \alpha^4 + \alpha^5 + \alpha^6 \]
Furthermore we have,
\[ w(\xi) = 1 + a_1 \alpha + a_2 \alpha^2 \]
and
\[ w(\xi^\perp) = 1 + b_1 \alpha + b_2 \alpha^2 + b_3 \alpha^3 + b_4 \alpha^4 \]
Therefore,
\begin{align*}
w(\xi) \cdot w(\xi^\perp) & = (1 + a_1 \alpha + a_2 \alpha^2) \cdot (1 + b_1 \alpha + b_2 \alpha^2 + b_3 \alpha^3 + b_4 \alpha^4) 
\\
& = 1 + (a_1 + b_1) \alpha + (a_1 b_1 + a_2 + b_2) \alpha^2 + (a_1 b_2 + a_2 b_1 + b_3) \alpha^3 + (a_2 b_2 + a_1 b_3 + b_4) \alpha^4
\\
& + (a_1 b_4 + a_2 b_3) \alpha^5 + a_2 b_4 \alpha^6 
\end{align*}
Then 
\begin{align*}
a_2 b_4 = 1 \quad & \implies a_2 = b_4 = 1
\\
a_1 b_4 + a_2 b_3 = 1 \quad & \implies a_1 + b_3 = 1
\\
a_1 + b_1 = 1 \quad & \implies a_1 = 1 \text{ or } b_1 = 1 \text{ and } a_1 b_1 = 0
\\
a_1 b_1 + a_2 + b_2 = 1 \quad & \implies b_2 = 0
\\
a_1 b_2 + a_2 b_1 + b_3 = 1 \quad & \implies b_1 + b_3 = 1
\\
a_2 b_2 + a_1 b_3 + b_4 = 1 \quad & \implies a_1 b_3  = 0
\end{align*}
Thus we have,
\begin{align*}
a_1 + b_1 & = 1
\\
b_1 + b_3 & = 1
\\
a_1 + b_3 & = 1
\end{align*}
Then $2 a_1 + b_1 + b_3 = 0$ and $2 a_1 = 0$ so $b_1 + b_3 = 0$ contradicting the middle equation giving a contradiction. Therefore, such a decomposition $\gamma \oplus \gamma^\perp = T \RP^6$ is impossible. 

\section{Lemmas}

\begin{prop} \label{orientation}
Any orientable real line bundle is trivial.
\end{prop}

\begin{proof}
In general, an orientation on $E$ is a $\mathrm{GL}^+(n, \R)$ structure on $E$ but $\mathrm{SL}(n, \R)$ is a deformation retract of $\mathrm{GL}^+(n, \R)$ so we get an $\mathrm{SL}^+(n, \R)$-structure on $E$ which determines a non-vanishing section $\omega \in \Gamma(X, \bigwedge^n E)$ of the top exterior power. Restricting to line bundles, $\bigwedge^n E = E$ so we get a non-vanishing section of $E$ which trivializes the line bundle. 
\bigskip\\
Alternatively, note that $w_1$ gives a bijection from line bundles to $H^1(X; \Z / 2 \Z)$ and $w_1(E) = 0$ if and only if $E$ is orientable meaning that $E$ is trivial if and only if it is orientable. 
\end{proof}

\begin{prop} \label{loop_restriction}
A vector bundle $E$ on $X$ is orientable if and only if its restriction to any loop $f^* E$ for $f : S^1 \to X$ is trivial 
\end{prop}

\begin{proof}
The orientability of $E$ is equivalent to the the top exterior power $\bigwedge^n E$ being a trivial line bundle. Thus it suffices to show the equivalent statement for triviality or equivalently orientability of line bundles. One direction is clear, if $L$ is a trivial line bundle on $X$ then for any $f : S^1 \to X$ we have $f^* L$ is a trivial line bundle. Conversely, suppose that $f^* L$ is orientable for any loop $f : S^1 \to X$. We need to show that $L$ is orientable. Consider a local trivialization $U_i$ with index set $I$ of $L$ on which $L |_{U_i} \xrightarrow{\varphi_i} \R \times U_i$. Then consider the graph $\Gamma$ on vertices $I$ and an edge for each nonempty intersection $U_i \cap U_j \neq \empty$. Each edge is given a sign $s_{ij}$ which is the sign of the determinant of the transition map $\varphi_i \circ \varphi_j^{-1}$ (note $\varphi_I \circ \varphi_j^{-1} : \R \times U_i \to \R$ gives a map $U_i \to \mathrm{GL}(1, \R) = \R^\times$ and $U_i$ is connected so the image has well-defined sign). Then an orientation of $L$ is equivalent to a choice of sign (equivalent to a choice of fiberwise-linear isomorphism $L|_{U_i} \xrightarrow{\varphi_i} \R \times U_i$) for each $U_i$ which is compatible with the signs of the edges. We can do this by choosing some orientation on some base $U_0$ and for each path in the graph choosing signs according to the edges in the path. This method only goes wrong when there are two paths from $U_0$ to $U_i$ which disagree with the correct choice of sign on $U_i$. Such paths give a loop in $\Gamma$ which cannot be given an orientation since the signs induced by the edges come back to disagree with the starting choice. Then, choosing a path $\gamma : S^1 \to X$ which induces the problematic path on $\Gamma$ then $\gamma^* L$ cannot be orientable since a non-vanishing section of $\gamma^* L$ would give a consistent choice of signs on the $U_i$ which $\gamma$ passes through. Therefore, $\gamma^* L$ is non-orientable for some loop $\gamma : S^1 \to L$. 
\end{proof}

\end{document}
