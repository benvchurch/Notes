\documentclass[12pt]{extarticle}
\usepackage{import}
\import{./}{Includes}

\usetikzlibrary{decorations.markings}

\begin{document}
{\title{% 
	\large \textbf{Mathematics GR 6307 Algebraic Topology
	\\ Exercises} \vspace{-2ex}}
\author{Benjamin Church }
\maketitle}

\newcommand{\Sing}[2]{\mathrm{Sing}_{#1}\left( #2 \right)}
\renewcommand{\Hom}[3]{\mathrm{Hom}_{#1}\left( #2, #3 \right)}

\newtheorem{theorem*}{Theorem}
\theoremstyle{definition}
\newtheorem{exercise}[section]{Exercise}


\section{Homological Algebra}

\begin{exercise}
Consider the one point space $X = *$. Then $\Sing{n}{X} = \Z$ generated by the constant map $\Delta^n \to X$. Now, the boundary map $\partial_n : \Sing{n}{X} \to \Sing{n-1}{X}$ is given by,
\[ \partial \sigma_n = \sum_{i = 0}^n (-1)^i \sigma_n |_i \]
However, $\sigma_n |_i = \sigma_{n-1}$ is simply the constant map so the sum is telescoping and thus,
\[ \partial \sigma_n  = \begin{cases}
0 & n \text{ odd}
\\
\sigma_{n-1} & n \text{ even}
\end{cases} \]
Now, consider the complex of singular chains,
\begin{center}
\begin{tikzcd}
\cdots \arrow[r] & \Sing{n}{X} \arrow[r, "\partial_n"] & \Sing{n-1}{X} \arrow[r] & \cdots \arrow[r] & \Sing{1}{X} \arrow[r, "\partial_1"] \arrow[r] & \Sing{0}{X} \arrow[r] & 0
\end{tikzcd}
\end{center}
We have established,
\[ \ker{\partial_n} = 
\begin{cases}
\Sing{n}{X} & n \text{ odd or } n = 0
\\
0 & n \text{ even}
\end{cases} \quad \quad \quad \Im{\partial_n} = 
\begin{cases}
0 & n \text{ odd}
\\
\Sing{n-1}{X} & n \text{ even}
\end{cases}  \]
Thus,
\[ H_n(X) = \ker{\partial_n} / \Im{\partial_{n+1}} =
\begin{cases}
0 & n > 0
\\
\Z & n = 0
\end{cases} \]
\end{exercise}

\begin{exercise}
Given two chain complexes $A, B$ consider their sum $A \oplus B$ which is the complex whose objects are $(A \oplus B)_n = A_n \oplus B_n$ and bounary maps are $\partial^{A \oplus B}_n = \partial_n^A \oplus \partial_n^B$. This is a complex because,
\[ \partial^{A \oplus B}_{n-1} \circ \partial^{A \oplus B}_n = (\partial_{n-1}^A \oplus \partial_{n-1}^B) \circ (\partial_n^A \oplus \partial_n^B) = (\partial_{n-1}^A \circ \partial_n^A) \oplus (\partial_{n-1}^B \circ \partial_n^B) = 0 \]
since $A$ and $B$ are complexes. 
\end{exercise}

\begin{exercise}
Consider $Z = X \coprod Y$ and $\Sing{n}{Z}$ which is generated by all maps $\Delta^n \to Z$. However, since $\Delta^n$ is connected any map $\Delta^n \to Z$ has image contained in either $X$ or $Y$ and thus we may parition the generators into the generators of $\Sing{n}{X}$ and $\Sing{n}{Y}$ so the free group has a partition,
\[ \Sing{n}{Z} = \Sing{n}{X} \oplus \Sing{n}{Y} \]
Furthermore it is clear that the boundary map on $\Sing{n}{Z}$ is the sum of the boundary maps on $\Sing{n}{X}$ and $\Sing{n}{Y}$ because if the image of $\Delta^n \to Z$ lies in $X$ then the map restricted to its faces also lies in $X$ so the boundary on such chains restricts to the bounary on $\Sing{n}{X}$. 
\end{exercise}

\begin{exercise}
The fact that $H_n(A \oplus B) = H_n(A) \oplus H_n(B)$ follows directly from the fact that kernels, images, and quotients all comute with direct products i.e.,
\[ \ker{(f \oplus g)} = \ker{f} \oplus \ker{g} \quad \quad \Im{(f \oplus g)} = \Im{f} \oplus \Im{g} \quad \quad (C_1 \oplus C_2)/(D_1 \oplus D_2) = (C_1 / D_1) \oplus (C_2 / D_2) \]
Therefore, taking homology also commutes with products. We can prove these facts by noting that $\oplus$ is a biproduct in any abelian category so its preserves limits and colimits (since it has a left and a right adjoint) and noting that $H_n(C) = \coker{(C_{n+1} \to \ker{\partial_n})}$. 
\end{exercise}

\begin{proposition}
$H_n(X \coprod Y) = H_n(X) \oplus H_n(Y)$
\end{proposition}

\begin{proof}
We have shown that $\Sing{*}{X \coprod Y} = \Sing{*}{X} \oplus \Sing{*}{Y}$ and thus,
\[ H_n(X \coprod Y) = H_n(\Sing{*}{X} \oplus \Sing{*}{Y}) = H_n(\Sing{*}{X}) \oplus H_n(\Sing{*}{Y}) = H_n(X) \oplus H_n(Y) \]
\end{proof}

\begin{exercise}
Let $X$ be a finite discrete space. Then we can write $X = \coprod\limits_{i = 1}^N *$. By the previous proposition and induction, we find,
\[ H_n(X) = \bigoplus_{i = 1}^N H_n(*) =
\begin{cases}
\Z^N & n = 0
\\
0 & n \neq 0
\end{cases} \]
Now let $X$ be any discrete space so $X = X = \coprod\limits_{i \in I} *$ but here is a subtlety in the above argument. For infinite disjoint unions it is still true that,
\[ \Sing{n}{X} = \bigoplus_{i \in I} \Sing{n}{*} \]
because a continuous map from a connected space to a discrete space is constant.
However, now the direct sum is not a biproduct only a coproduct when $I$ is infinite so we cannot naively conclude that infinite direct sums commute with homology because they may not preserve limits. However, in the category of abelian groups infinite direct sum is exact so it preserves kernels and cokernels which we showed was adequate to preserve homology when applied to a complex. Thus we again find,
\[ H_n(X) = H_n(\Sing{n}{X}) = \bigoplus_{i \in I} H_n(*) = 
\begin{cases}
\Z^I & n = 0
\\
0 & n \neq 0
\end{cases} \]
\end{exercise}

\begin{exercise}
Let $X$ be path-connected. Note that $\Delta^1 \cong I$ the closed interval and $\Delta^0 = *$. Thus $\Sing{0}{X}$ is the free group on points of $X$ and $\Sing{1}{X}$ is the free group on paths in $X$. Now given a path $\gamma : \Delta^1 \to X$ its boundary is $\partial_1 \gamma = [\gamma(0)] - [\gamma(1)]$.  Now,
\[ H_0(X) = \Sing{0}{X} / \Im{\partial_1} \]
Since $X$ is path connected, for any two points $x, y \in X$ there exists a path $\gamma$ from $x$ to $y$ so $x - y \in \Im{\partial_1}$. Now consider the degree map $\varepsilon : \Sing{0}{X} \to \Z$ which acts via,
\[ \varepsilon \left( \sum_{x \in X} n_x [x] \right) = \sum_{x \in X} n_x \]
Now $\ker{\varepsilon} = \Im{\partial_1}$ because,
\begin{align*}
\varepsilon \left( \sum_{x \in X} n_x [x] \right)  = 0 \implies \sum_{x \in X} n_x  = 0 \implies \sum_{x \in X} n_x [x] = \sum_{x \in X} n_x [x] - \sum_{x \in X} n_x [x_0] = \sum_{x \in X} n_x ([x] - [x_0]) \in \Im{\partial_1} 
\end{align*}
and clearly $\Im{\partial_1} \subset \ker{\varepsilon}$. 
Thus,
\[ H_n(X) = \Sing{0}{X} / \Im{\partial_1} = \Sing{0}{X} / \ker{\varepsilon} = \Z \]
\end{exercise}

\begin{exercise}
Let $M$ be a connected manifold. Consider the deRham complex of $M$,
\begin{center}
\begin{tikzcd}
0 \arrow[r] & \Omega_M^0 \arrow[r, "d"] & \Omega_M^1 \arrow[r, "d"] & \Omega_M^2 \arrow[r] & \cdots  
\end{tikzcd}
\end{center}
Now,
\[ H^0_{\text{dR}}(M) = \ker{d^0} \]
which is the group of smooth functions on $M$ with zero derivative. The mean value theorem tells us that if $f' = 0$ everywhere then it is locally constant. Thus the set of points $\{ x \in M \mid f(x) = c \} = f^{-1}(c)$ is open but $f^{-1}(c)$ is closed since $M$ is $T_2$ so points are closed and $f$ is continuous. Thus, since $M$ is connected $f^{-1}(c) = M$ or $f^{-1}(c) = \varnothing$ proving that $f$ is globally constant so,
\[ H^0_{\text{dR}}(M) = \ker{d^0} = \R \]
\end{exercise}

\begin{exercise}
Let $C$ and $D$ be chain complexes. Now consider the groups,
\[ \Hom{n}{C}{D} = \{ (\varphi_r)_{r \in \Z} \mid \varphi_r : C_r \to D_{r + n} \text{ is a homomorphism} \}  \] 
Now define a boundary,
\[ (\partial_n \varphi)_r = \partial_{r + n} \circ \varphi_r - (-1)^n \varphi_{r-1}\circ \partial_r  \]
First, note, $\partial_{r + n} \circ \varphi_r : C_r \to D_{r + n} \to D_{r + n - 1}$ and $\varphi_{r-1}\circ \partial_r  : C_r \to C_{r-1} \to C_{r + n - 1}$
and therefore,
\[ \partial_n \varphi \in \Hom{n-1}{C}{D} \]
Clearly is boundary is additive. It remains to show that this sequence is a chaim complex. Consider,
\begin{align*}
(\partial_{n-1} \circ \partial_{n} \varphi)_r & = \partial_{r + n - 1} \circ (\partial_n \varphi)_r - (-1)^{n-1} (\partial_n \varphi)_{r-1} \circ \partial_r
\\
& = \partial_{r + n - 1} \circ [ \partial_{r + n} \circ \varphi_r - (-1)^{n} \varphi_{r-1} \circ \partial_r c ] 
\\
& \quad - (-1)^{n-1} \circ [ \partial_{r - 1 + n} \circ \varphi_{r-1} \circ \partial_r  - (-1)^{n} \varphi_{r-2} \circ \partial_{r-1} \circ \partial_r ] 
\\
& = - (-1)^n \partial_{r + n - 1} \circ  \varphi_{r-1} \circ \partial_r  - (-1)^{n - 1} \partial_{r-1 + n} \circ \varphi_{r-1} \circ \partial_r = 0
\end{align*}
since these two terms cancel. Thus $\partial_{n-1} \circ \partial_n \varphi = 0$ so $\Hom{*}{C}{D}$ is a complex under these boundary maps.
\bigskip\\
Conisder the cycles in degree zero, i.e. $\Hom{0}{C}{D}$ such that $\partial_0 \varphi = 0$. Since $\varphi \in \Hom{0}{C}{D}$ is a family of maps $\varphi_r : C_r \to D_r$.  The fact that $\varphi$ is a cycle is exactly that,
\[ (\partial_0 \varphi)_r = \partial_r \circ \varphi_r - \varphi_{r-1} \circ \partial_r = 0 \]
which is exacly the condition,
\[ \partial_r \circ \varphi_r = \varphi_{r-1} \circ \partial_r \] 
which says that $\varphi : C \to D$ is a chain map. Furthermore, suppose $\varphi, \psi \in \Hom{0}{C}{D}$ differ by a boundary $\partial_1 \gamma$ for $\gamma \in \Hom{1}{C}{D}$ i.e. a family of maps $\gamma_r : C_r \to D_{r + 1}$. Thus,
\[ \varphi_r - \psi_r = (\partial_1 \gamma)_r = \partial_{r+1} \circ \gamma_r + \gamma_{r - 1} \circ \partial_r \]
which is exactly the condition that $\gamma$ be a chain homotopy between $\varphi$ and $\psi$.
\end{exercise}

\begin{exercise}
Suppose that $C$ has finitely generated homology groups $H_n(C)$. By the structure theorem for finitely generated abelian groups, we can find generators such that,
\[ H_n(C) \cong \Z [\gamma_n^1] \oplus \cdots \oplus \Z [ \gamma_n^r] \oplus \Z / k_n^1 \Z [\delta_n^1] \oplus \cdots \oplus \Z / k_n^s \Z [\delta_n^s] \]
and choose representative cycles $\gamma^i_n, \delta^i_n \in C_n$. For each finite generator let $k^i_n = \ord{[\delta^i_n]}$ be the order of its class so we know that $k^i_n \delta^i_n = \partial_{n+1} \gamma^i_{n} \in \Im{\partial_{n+1}}$. Now there is a map of chain complexes $\Z[-n] \to C$ via sending $1 \mapsto \gamma_n^i$ (this is a morphism of complexes because $\gamma_n^i \in \ker{\partial_n}$). Let $T^{k,n}$ denote the complex $\Z[-(n+1)] \xrightarrow{k} \Z[-n]$. Then there is a map of chain comples $T^{n,k^i_n} \to C$ as follows,
\begin{center}
\begin{tikzcd}[column sep = huge, row sep = huge]
\cdots \arrow[r] & 0 \arrow[d] \arrow[r] & \Z \arrow[r, "k^i_n"] \arrow[d, "1 \mapsto \gamma^i_n"] \arrow[r] & \Z \arrow[d, "1 \mapsto \delta^i_n"] \arrow[r] & 0 \arrow[d] \arrow[r] & \cdots 
\\
\cdots \arrow[r] & C_{n+2} \arrow[r, "\partial_{n+2}"] & C_{n+1} \arrow[r, "\partial_{n+1}"] \arrow[r] & C_{n} \arrow[r, "\partial_{n}"] & C_{n-1} \arrow[r] & \cdots 
\end{tikzcd}
\end{center} 
This is a morphism of chain complexes because $\partial_n \delta_n^i = 0$ since it is a cycle and $\partial_{n+1} \gamma^i_n = k^i_n \gamma^i_n$ so the above diagram commutes. 
\bigskip\\
Now take the sum of these morphisms to give,
\[ f : \bigoplus_{n \in \Z} \left( \bigoplus_{i = 1}^r \Z[-n] \oplus \bigoplus_{i = 1}^s T^{n, k^i_n} \right) \to C \] 
It is clear that on homology $f_*$ is surjective since it maps $1$ in the various elementary chain complexes to each generator of $H_n(C)$ in each degree. (SHOW THAT THIS WORKS EXPLICILY)
\end{exercise}

\section{Singular Homology}

We are asked to show the following using only the  Eilenberg-Steenrod Axioms for Homology.


\begin{proposition}
$H_n(\varnothing) = 0$
\end{proposition}

\begin{proof}
Take $X = A = U = *$ and apply excision. Then there is an isomorphism,
\[ H_n(\varnothing) = H_n(X \setminus A, U \setminus A) \to H_n(X, A) \]
Furthermore, by the long exact sequence for the pair $(X, A)$,
\begin{center}
\begin{tikzcd}
\cdots & \arrow[r] & H_n(A) \arrow[r] & H_n(X) \arrow[r] & H_n(X, A) \arrow[r] & H_{n-1}(A) \arrow[r] & \cdots
\end{tikzcd}
\end{center}
Since the map $H_n(A) \to H_n(X)$ is induced by the inclusion $A \to X$ which is the identity in this case it is an isomorphism. Thus, $H_n(X, A) = 0$ since all maps at it must be zero and the sequence is exact.
Thus,
\[ H_n(\varnothing) = H_n(X, A) = 0 \]
\end{proof}

\begin{proposition}
$H_n(X \coprod Y) = H(X) \oplus H(Y)$
\end{proposition}

\begin{proof}
Consider the open sets $X, Y \subset X \coprod Y$ which seperate $Z = X \coprod Y$ i.e. $X \cap Y = 0$ and $X \cup Y = Z$. Then, applying the Mayer-Vietoris sequence,
\begin{center}
\begin{tikzcd}
\cdots & H_n(X \cap Y) \arrow[r] & H_n(X) \oplus H_n(Y) \arrow[r] & H_n(X) \arrow[r] & H_n(Z) \arrow[r] & H_{n-1}(X \cap Y) \arrow[r] & \cdots
\end{tikzcd}
\end{center}
Since $H_{n}(X \cap Y) = H_{n}(\varnothing) = 0$, the kernel and cokernel of $H_n(X) \oplus H_n(Y) \to H_n(Z)$ are zero (in degree $n = 0$ the cokernel is automatically zero) and this is an isomorphism. 
\end{proof}

\begin{proposition}
\[ H_k(D^n, \partial D^n) = \begin{cases}
\Z & k = n
\\
0 & k \neq n
\end{cases}
\quad \quad \quad 
H_k(S^n) = \begin{cases}
\Z & k = n, 0
\\
0 & k \neq n 
\\
\Z \oplus \Z & k = n = 0
\end{cases} \]
\end{proposition} 

\begin{proof}
First, since $D^n$ is contractable, by the homotopy axiom, 
\[ H_k(D^n) = H_k(*) = 
\begin{cases}
\Z & k = 0
\\
0 & k \neq 0
\end{cases} \]
We proceed by induction on $n$. 
For $n = 0$, $\partial D^0 = \varnothing$ so \[ H_k(D^0, \partial D^0) = H_k(D^0) = \begin{cases}
\Z & k = 0
\\
0 & k \neq 0
\end{cases} \]
and $S^0 = * \coprod *$ so by above,
\[ H_k(S^0) = H_k(*) \oplus H_k(*) = 
\begin{cases}
\Z \oplus \Z & k = 0
\\
0 & k \neq 0 
\end{cases} \]
so we satisfy the base case. Now we assume the theorem holds for some $n$. Consider the long exact sequence of the pair $(D^{n+1}, \partial D^{n+1})$,
\begin{center}
\begin{tikzcd}
\cdots \arrow[r] & H_k(\partial D^{n+1}) \arrow[r] & H_k(D^{n+1}) \arrow[r] & H_k(D^{n+1}, \partial D^{n+1}) \arrow[r] & H_{k-1}(\partial D^{n+1}) \arrow[r] & \cdots 
\end{tikzcd}
\end{center}
Now we know $\partial D^{n+1}$ is homotopy equivalent to $S^n$ so we can apply the induction hypothesis. 
\end{proof}

\begin{proposition}

\end{proposition}

\begin{proposition}

\end{proposition}

\begin{proposition}

\end{proposition}

\begin{proposition}

\end{proposition}

\begin{proposition}

\end{proposition}

\begin{proposition}
Let $\iota : S^{n-1} \to D^n$ be the inclusion into the boundary. There does not exists a map $r : D^n \to S^{n-1}$ such that $r \circ \iota = \id_{S^{n-1}}$. 
\end{proposition}

\begin{proof}
Suppose there were such a map $r : D^n \to S^{n-1}$. Then, by functoriality on homology, $(r \circ \iota)_* = r_* \circ \iota_* = \id_* = \id_{H_*(S^{n-1})}$. 
Therefore, the maps $\iota_* : H_*(S^{n-1}) \to H_*(D^n)$ and $r_* : H_*(D^{n}) \to h_*(S^{n-1})$ compose to the identity $H_*(S^{n-1}) \to H_*(S^{n-1})$. In particular, in dimension $n - 1$,
\[ H_{n-1}(S^{n-1}) \cong \Z \]
but $D^n$ is contractible so $H_*(D^n) = 0$ which implies that $\iota_*$ is the zero map contradicting the fact that $r_* \circ \iota_* = \id_{H_{n-1}(S^{n-1})}$ since it must send the generator to zero. 
\end{proof}


\begin{theorem*}[Brower]
Any map $f : D^n \to D^n$ has a fixed point.
\end{theorem*}

\begin{proof}
Suppose that $f : D^n \to D^n$ has no fixed point. Then consider the map $\tilde{f} : D^n \to S^{n-1}$ defined by the ray of the line from $\vec{x} \in D^n$ to $f(\vec{x})$ and the boundary $\partial D^n = S^{n-1}$. This map is continuous and clearly constant on $S^{n-1}$ contradicting the above proposition.
\end{proof}

\begin{proposition}
Let $U \subset \R^n$ and $V \subset \R^m$ be open. If $U$ and $V$ are homeomorphic then $n = m$. 
\end{proposition}

\begin{proof}
Suppose that $f : U \to V$ is an homeomorphism. Choose $x \in U$ and consider the map of pairs $f : (U, U \setminus \{x \}) \to (V, V \setminus \{ f(x) \})$ which is a homeomorphism of pairs. Consider the closed set $U^C \subset \R^n \setminus \{ x \}$ then we can perform excision to this pair to get an isomorphism,
\[ H_*(U, U \setminus \{x \}) = H_*(\R^n \setminus U^C, \R^n \setminus \{ x \} \setminus U^C)  \xrightarrow{\sim} H_8(\R^n, \R^n \setminus \{ x \}) \]
By the same argument,
\[ H_*(V, V \setminus \{ f(x) \}) \xrightarrow{\sim} H_*(\R^m, \R^m \setminus \{ f(x) \}) \]
Now, consider the long exact sequence associated to the pair $(\R^n, \R^n \setminus \{ x \})$,
\begin{center}
\begin{tikzcd}
\cdots \arrow[r] & \tilde{H}_k(\R^n \setminus \{ x \}) \arrow[r] & \tilde{H}_k(\R^n) \arrow[r] & H_k(\R^n, \R^n \setminus \{ x \}) \arrow[r] & \tilde{H}_{k-1}(\R^n \setminus \{ x \}) \arrow[r] & \cdots
\end{tikzcd}
\end{center}
Since $\R^n$ is contractible $\tilde{H}_k(\R^n) = 0$ which implies that,
\[ H_k(\R^n, \R^n \setminus \{ x \}) = \tilde{H}_{k-1}(\R^n \setminus \{ x \}) \]
By homotopy invariance, $\tilde{H}_{k-1}(\R^n \setminus \{ x \}) = \tilde{H}_{k - 1}(S^{n-1})$ which implies that,
\[ H_k(\R^n, \R^n \setminus \{ x \}) = \tilde{H}_{k-1}(S^{n-1}) = \begin{cases}
\Z & n = k
\\
0 & n \neq k 
\end{cases} \]
valid for $n > 0$. Putting everything together, the assumed homeomorphism gives, an isomorphism of graded groups,
\[ f_* : H_*(U, U \setminus \{ x \}) \xrightarrow{\sim}  H_*(V, V \setminus \{ f(x) \}) \]
But, 
\[ H_n(U, U \setminus \{ x \}) = H_n(\R^n, \R^n \setminus \{ x \}) = \Z \]
and 
\[ H_n(V, V \setminus \{ f(x) \}) = H_n(\R^m, \R^m \setminus \{ f(x) \}) = \begin{cases}
\Z & m = n
\\
0 & m \neq n 
\end{cases} \]
which forces $m = n$. 
\end{proof}

\end{document}
