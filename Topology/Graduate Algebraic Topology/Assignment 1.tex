\documentclass[12pt]{extarticle}
\usepackage[utf8]{inputenc}
\usepackage[english]{babel}
\usepackage[utf8]{inputenc}
\usepackage[english]{babel}
\usepackage[a4paper, total={7in, 9.5in}]{geometry}
\usepackage{tikz-cd}

 
\usepackage{amsthm, amssymb, amsmath, centernot, graphicx}
\usepackage{accents}
\DeclareMathAccent{\wtilde}{\mathord}{largesymbols}{"65}
\newcommand{\orb}[1]{\mathrm{Orb}(#1)}
\newcommand{\stab}[1]{\mathrm{Stab}(#1)}
\newcommand{\rp}{\mathbb{RP}}
\newcommand{\cp}{\mathbb{CP}}

\newcommand{\notimplies}{%
  \mathrel{{\ooalign{\hidewidth$\not\phantom{=}$\hidewidth\cr$\implies$}}}}
 
\renewcommand\qedsymbol{$\square$}
\newcommand{\cont}{$\boxtimes$}
\newcommand{\divides}{\mid}
\newcommand{\ndivides}{\centernot \mid}
\newcommand{\Z}{\mathbb{Z}}
\newcommand{\N}{\mathbb{N}}
\newcommand{\C}{\mathbb{C}}
\newcommand{\Zplus}{\mathbb{Z}^{+}}
\newcommand{\Primes}{\mathbb{P}}
\newcommand{\ball}[2]{B_{#1} \! \left(#2 \right)}
\newcommand{\Q}{\mathbb{Q}}
\newcommand{\R}{\mathbb{R}}
\newcommand{\Rplus}{\mathbb{R}^+}
\newcommand{\invI}[2]{#1^{-1} \left( #2 \right)}
\newcommand{\End}[1]{\text{End}\left( A \right)}
\newcommand{\legsym}[2]{\left(\frac{#1}{#2} \right)}
\renewcommand{\mod}[3]{\: #1 \equiv #2 \: \mathrm{mod} \: #3 \:}
\newcommand{\nmod}[3]{\: #1 \centernot \equiv #2 \: mod \: #3 \:}
\newcommand{\ndiv}{\hspace{-4pt}\not \divides \hspace{2pt}}
\newcommand{\finfield}[1]{\mathbb{F}_{#1}}
\newcommand{\finunits}[1]{\mathbb{F}_{#1}^{\times}}
\newcommand{\ord}[1]{\mathrm{ord}\! \left(#1 \right)}
\newcommand{\quadfield}[1]{\Q \small(\sqrt{#1} \small)}
\newcommand{\vspan}[1]{\mathrm{span}\! \left\{#1 \right\}}
\newcommand{\galgroup}[1]{Gal \small(#1 \small)}
\newcommand{\sm}{\! \setminus \!}
\newcommand{\topo}{\mathcal{T}}
\newcommand{\base}{\mathcal{B}}
\renewcommand{\bf}[1]{\mathbf{#1}}
\renewcommand{\Im}[1]{\mathrm{Im} \: #1}
\renewcommand{\empty}{\varnothing}
\newcommand{\id}{\mathrm{id}}
\newcommand{\Hom}[2]{\mathrm{Hom}\left( #1, #2 \right)}
\newcommand{\Tor}[4]{\mathrm{Tor}^{#1}_{#2} \left( #3, #4 \right)}

\renewcommand{\theenumi}{(\alph{enumi})}

\newcommand{\atitle}[1]{\title{% 
	\large \textbf{Mathematics GU4053 Algebraic Topology
	\\ Assignment \# #1} \vspace{-2ex}}
\author{Benjamin Church }
\maketitle}

\newcommand{\hook}{\hookrightarrow}


\theoremstyle{remark}
\newtheorem*{remark}{Remark}

\theoremstyle{definition}
\newtheorem{theorem}{Theorem}[section]
\newtheorem{lemma}[theorem]{Lemma}
\newtheorem{proposition}[theorem]{Proposition}
\newtheorem{corollary}[theorem]{Corollary}
\newtheorem{example}[theorem]{Example}


\newenvironment{definition}[1][Definition:]{\begin{trivlist}
\item[\hskip \labelsep {\bfseries #1}]}{\end{trivlist}}

\usepackage{subcaption}
\usepackage{float}
\floatplacement{figure}{H}

\begin{document}
\atitle{1}

Note. My order of path concatenation follows Hatcher,
\[\gamma * \delta(x) = \begin{cases}
\gamma(2x) & x \le \tfrac{1}{2} \\
\delta(2x - 1) & \ge \tfrac{1}{2}
\end{cases}\]
 
\section*{Problem 1.}
Let $X$ be a contractible space. Then, there exists a homotopy $H : X \times I \to X$ between $\id_X$ and constant map $f : X \to \{x_0\} \subset X$. For any $x \in X$ consider the path $\gamma : I \to X$ given by $\gamma(t) = H(x, t)$ which satisfies $\gamma(0) = H(x, 0) = \id_X(x) = x$ and $\gamma(1) = H(x, 1) = x_0$. Therefore, any $x$ is path connected to $x_0$. However, because path connection is an equivalence relation on points, any $x, y \in X$ are path connected by transitivity.  

\section*{Problem 2.}
Let $f, f' : X \to Y$ and $g, g' : Y \to Z$ be pairs of homotopic maps. Then, there exist homotopies, $F : X \times I \to Y$ and $G : Y \times I \to Z$ between these maps. Consider the function $H : X \times I \to Z$ given by $H(x, t) = G(F(x, t), t)$ which is continuous by composition of continuous maps. Now, $H(x, 0) = G(F(x, 0), 0) = G(f(x), 0)  = g \circ f (x)$ and $H(x, 1) = G(F(x, 1), 1) = G(f'(x), 1) = g' \circ f'(x)$. Therefore, $H$ is a homotopy between $g \circ f$ and $g' \circ f'$.  

\section*{Problem 3.}
\begin{enumerate}
\item Let $f : X \to Y$ and $g : Y \to Z$ be homotopy equivalences with homotopy ``inverses'' such that the compositions are homotopy equivalent to identity maps, $f' : Y \to X$ and $g' : Z \to Y$. Consider the maps $g \circ f$ and $f' \circ g'$. Now, using the result of problem 2,
\[(g \circ f) \circ (f' \circ g') = g \circ ((f \circ f') \circ g') \simeq g \circ (\id_Y \circ g') = g \circ g' \simeq \id_Y\]
and similarly,
\[(f' \circ g') \circ (g \circ f) = f' \circ ((g' \circ g) \circ f) \simeq f' \circ (\id_Y \circ f) = f' \circ f \simeq \id_X\]
therefore $g \circ f$ is a homotopy equivalence. Therefore, $\simeq$ is an equivalence relation on topological spaces because $X \simeq X$ under the identity map. If $X \simeq Y$ then there are maps $f : X \to Y$ and $g : Y \to X$ which are homotopy ``inverses'' and thus $Y \simeq X$ by swapping $f$ and $g$. And finally, if $X \simeq Y$ and $Y \simeq Z$ then by above the composition of homotopy equivalences gives a homotopy equivalence $X \simeq Z$ so the relation is transitive. 

\item Consider the maps from $X$ to $Y$ under homotopy. Clearly, $f \simeq f$ under the homotopy $H(x, t) = f(x)$. If $f \simeq g$ then there exists a homotopy $H : X \times I \to Y$ then consider the map $H'(x, t) = H(x, 1 - t)$. Now, $H'(x, 0) = H(x, 1) = g(x)$ and $H'(x, 1) = H(x, 0) = f(x)$ so $g \simeq f$. Finally, let $f \simeq g$ and $g \simeq h$. Then, we have homotopies $F, G : X \times I \to Y$ between $f$ and $g$ and between $g$ and $h$ respectively. Define the map $H : X \times I \to Y$ given by,
\[ H(x, t) = \begin{cases}
F(x, 2t) & t \in [0, \tfrac{1}{2}] \\
G(x, 2t - 1) & t \in [\tfrac{1}{2}, 1]
\end{cases}\]
At $t = \tfrac{1}{2}$ the maps $F(x, 1) = g(x) = G(x, 0)$ so the map $H$ is continous by the gluing lemma. Furthermore, $H(x, 0) = F(x, 0) = f(x)$ and $H(x, 1) = G(x, 1) = h(x)$ so $H$ is a homotopy between $f$ and $h$. Thus, $f \simeq h$ so $\simeq$ is transitive and then an equivalence relation on maps with common domains and codomains. 
\item Let $f : X \to Y$ be a homotopy equivalence with homotopy ``inverse'' $g : Y \to X$ and let $h \simeq f$. Then, by problem 2, $h \circ g \simeq f \circ g \simeq \id_Y$ so by transitivity, $h \circ g \simeq \id_Y$. Similarly, $g \circ h \simeq g \circ f \simeq \id_X$ so $g \circ h = \id_X$. Therefore, $h$ is a homotopy equivalence with homotopy ``inverse'' $g$. 
\end{enumerate}

\section*{Problem 4.}

If every map $f : X \to Y$ for any $Y$ is nullhomotopic then in particular, $\id_X : X \to X$ is nullhomotopic so $X$ is contractable. Conversely, if $X$ is contractable then $\id_X : X \to X$ is homotopic to some constant map $g : X \to X$. For any map, $f : X \to Y$ we have $f = f \circ \id_X \simeq f \circ g$ which is a constant map because $g$ is constant. Thus, $f$ is nullhomotopic.  \bigskip \\

If every map $f : Y \to X$ for any $Y$ is nullhomotopic then in particular, $\id_X : X \to X$ is nullhomotopic so $X$ is contractable. Conversely, if $X$ is contractable then $\id_X : X \to X$ is homotopic to some constant map $g : X \to X$. For any map, $f : Y \to X$ we have $f = \id_X \circ f \simeq g \circ f$ which is a constant map because $g$ is constant. Thus, $f$ is nullhomotopic. \bigskip \\

\section*{Problem 5.}

Suppose there exist map $f : X \to Y$ and $g,h : Y \to X$ such that $f \circ g \simeq \id_Y$ and $h \circ f \simeq \id_X$. Then consider the composition,
\[h = h \circ \id_Y \simeq h \circ (f \circ g) = (h \circ f) \circ g \simeq \id_X = g\]
Therefore, by transitivity, $h \simeq g$. Thus, $g \circ f \simeq h \circ f \simeq \id_X$. However, $f \circ g \simeq \id_Y$ so $f$ is a homotopy equivalence. \bigskip \\
Let $f \circ g$ and $h \circ f$ be homotopy equivalences with homotopy ``inverses'' $a : Y \to X$ and $b : X \to Y$ respectively. Therefore, $f \circ (g \circ a) = (f \circ g) \circ a \simeq \id_Y$ and $(b \circ h) \circ f = b \circ (h \circ f) \simeq \id_X$. Therefore, by the above argument, $f$ is a homotopy equivalence.
 
\section*{Problem 6.}
Let $X$ be path-connected. Suppose that $\pi_1(X)$ is abelian and thus $\pi_1(X, x)$ is abelian at any point $x \in X$ because these groups are isomorphic on path-connected points. Now, let $h, h' : I \to X$ be paths with equal endpoints $x_0, x_1 \in X$ and let $\beta_h$ and $\beta_{h'}$ be the respective basepoint change isomorphisms. Take any loop $[\gamma] \in \pi_1(X, x_1)$. The maps $\bar{h} * h'$ and $\bar{h'} * h$ are loops at $x_1$ satisfying, \[(\bar{h} * h') * (\bar{h'} * h) = \bar{h} * ((h' * \bar{h'}) * h) \simeq \bar{h'} * h' \simeq e_{x_1}\]    
Therefore, $[\gamma] = [(\bar{h} * h') * (\bar{h'} * h) * \gamma] = [(\bar{h} * h') * \gamma * (\bar{h'} * h)]$ using the commutativity of $\pi_1(X, x_1)$. Then,
\[\beta_h([\gamma]) = \beta_h([(\bar{h} * h') * \gamma * (\bar{h'} * h)]) = [h *(\bar{h} * h') * \gamma * (\bar{h'} * h) * \bar{h}] = [h' * \gamma * \bar{h'}] = \beta_{h'}([\gamma])\]
and therefore, $\beta_h = \beta_{h'}$. \bigskip \\
Conversely, suppose that for any two paths with equal endpoints $h$ and $h'$ the change of basepoint maps are equal i.e. $\beta_h = \beta_{h'}$. In particular, take $x_0 \in X$ and let $h$ be any loop at $x_0$. Also, set $h' = e_{x_0}$ the constant loop at $x_0$. Then, for any loop $[\gamma] \in \pi_1(X, x_0)$ we know that,
\[ \beta_h([\gamma]) = [h * \gamma * \bar{h}] = [h] [\gamma] [h]^{-1} = \beta_{h'}([\gamma]) = [e_{x_0} * \gamma * \bar{e_{x_0}}] = [\gamma]\]
because $e_{x_0} * \gamma * \bar{e_{x_0}} \simeq \gamma$. Therefore, conjugation by $[h] \in \pi_1(X, x_0)$ is trivial for any $h$ so the group is abelian.
\section*{Problem 7.}

To show that the three conditions are equivalent, I will show that $(a) \implies (b) \implies (c) \implies (a)$. \bigskip \\

\subsection*{$(a) \implies (b)$}
Suppose that every map $f : S^1 \to X$ is homotopic to a constant map $g : S^1 \to \{p\}$. Then, there exists a homotopy $F : S^1 \times I \to X$ such that $F(x, 0) = f(x)$ and $F(x, 1) = p$. Now, identify all the points $S^1 \times {1}$ in the cylinder $S^1 \times I$. Under this identification of gluing together one end of the cylinder, the quotient space is the disk $D^2$. Now, $F(x, 1) = p$ so $F$ is constant on $S^1 \times \{1\}$ and thus constant on all equivalence classes.
\begin{center}
\begin{tikzcd}[column sep = huge, row sep = large]
S^1 \arrow[r, "\iota"] \arrow[rr, "f", bend left = 20] \arrow[rd, "\tilde{\iota}"] & S^1 \times I \arrow[d, "\pi"] \arrow[r, "F"] & X \\
& D^2 \cong S^1 \times I / \sim \arrow[ru, "\tilde{F}"]
\end{tikzcd}
\end{center}
Therefore, $F$ descends to the quotient space giving a map $\tilde{F} : D^2 \to X$ such that, \[\tilde{F}|_{S^1 \times \{0\} } = \tilde{F} \circ \tilde{\iota} = \tilde{F} \circ \pi \circ \iota = F \circ \iota = f\]
where $\iota : S^1 \to S^1 \times I$ is the inclusion onto $S^1 \times \{0\}$ on which $F(x, 0) = f(x)$ and $\pi : S^1 \times I \to D^2$ is the projection onto the quotient. \bigskip \\
\subsection*{$(b) \implies (c)$}
Suppose that every map $f : S^1 \to X$ extends to a map $F : D^2 \to X$. For any loop $\gamma : I \to X$ based at $x_0$, because $\gamma(0) = \gamma(1)$ the map $\gamma : I \to X$ descends to a map $f : I/\sim \to X$ on quotient space under the identification $0 \sim 1$. However, $I/\{0, 1\} \cong S^1$ so $f : S^1 \to X$ maps the generator of the fundamental group of $S^1$ to $\gamma$. Now, let $\iota : S^1 \to D^2$ be the inclusion onto the boundary of $D^2$. Then, $F \circ \iota(x) = f(x)$ because $F$ is an extension of $f$. The functor $\pi_1$ takes this diagram in Top to the analogous diagram in Grp,
\begin{figure}
\centering
\begin{subfigure}{.5\textwidth}
	\begin{center}
	\begin{tikzcd}[column sep = huge, row sep = large]
	S^1 \arrow[d, "\iota"] \arrow[r, "f"] & X \\
	D^2 \arrow[ru, "F"]
	\end{tikzcd}
	\end{center}
	\caption{Top}
\end{subfigure}%
\begin{subfigure}{.5\textwidth}
	\begin{center}
	\begin{tikzcd}[column sep = huge, row sep = large]
	\pi_1(S^1, s_0) \arrow[d, "\iota_*"] \arrow[r, "f_*"] & \pi_1(X, x_0) \\
	\pi_1(D^2, \iota(s_0)) \arrow[ru, "F_*"]
	\end{tikzcd}
	\end{center}
	\caption{Grp}
\end{subfigure}
\end{figure}
\noindent However, $D^2$ is homeomorphic to a convex subset of $\R^2$ and is thus contractable. Therefore, $\pi_1(D^2) = 0$ and thus $i_*(\pi_1(S^1, s_0)) \subset \pi_1(D^2, \iota(s_0)) = 0$ so $i_*(\pi_1(S^1, s_0)) = 0$. Therefore, $f_*(\pi_1(S^1, s_0)) = F_* \circ \iota_*(\pi_1(S^1, s_0)) = 0$. However, letting $[1]$ generate $\pi_1(S^1, s_0) \cong \Z$, we have $f_*([1]) = [\gamma]$ so $[\gamma] = [e_{x_0}]$ because $f_*$ is the zero map. Therefore, $[\gamma]$ is trivial so $\pi_1(X, x_0) = 0$.
\bigskip \\
\subsection*{$(c) \implies (a)$}
Suppose that $\pi_1(X, x_0) = 0$ for any $x_0 \in X$. Given any map $f : S^1 \to X$, take the map $\pi : I \to S^1$ given by the quotient map under the identification $0 \sim 1$. Then, $f \circ \pi$ is a loop in $X$ at some basepont $f \circ \pi(0) = x_0 = f \circ \pi(1)$. Because $X$ is simply connected, this loop is path-homotopic to the constant loop at $x_0$ under a homotopy $H : I \times I \to X$. Because $H(0, t) = H(1, t) = x_0$ the map descends to a map $\tilde{H} : S^1 \times I \to X$ on the quotient space under the same identification. $\tilde{H}$ is a homotopy between $f$ and a constant map, $\tilde{H}(x, 1) = x_0$. Thus, every map $f : S^1 \to X$ is homotopic to a constant map.  
\bigskip \\
Therefore, \[(a) \iff (b) \iff (c)\] 
\subsection*{simply connected $\iff$ all maps $S^1 \to X$ are homotopic:}
If all maps $f : S^1 \to X$ are homotopic then, in particular, every map $f : S^1 \to X$ is homotopic to a constant map. Using $(a) \implies (c)$ we conclude that $\pi(X, x_0) = 0$ at any basepoint. Furthermore, all constant maps from $S^1$ are homotopic which implies that $X$ is path-connected. Thus, $X$ is simply connected. 
\bigskip \\
Conversely, if $X$ is simply connected then $\pi_1(X, x_0) = 0$ for any basepoint $x_0 \in X$. From the result, $(c) \implies (a)$ we have that every map $f : S^1 \to X$ is homotopic to some constant map $f_c : S^1 \to \{c\} \subset X$. However, since $X$ is path connected, all constant maps are homotopic. Therefore, given two maps $f_1, f_2 : S^1 \to X$, we know that $f_1 \simeq f_{c_1}$ and $f_2 \simeq f_{c_2}$ and $f_{c_1} \simeq f_{2}$ because both are constant maps. Thus, $f_1 \simeq f_{c_1} \simeq f_{c_2} \simeq f_{c_2}$ because homotopy is an equivalence relation on maps. Therefore, any two maps $f : S^1 \to X$ are homotopic. \bigskip \\    

At last, we have shown that $X$ is simply-connected iff all maps $f : S^1 \to X$ are homotopic. 

\section*{Problem 8.}
Let $\gamma : I \to X$ be a loop at $x_0$ and $\delta : I \to Y$ be a loop at $y_0$. Then, consider the map, $H : I \times I \to X \times Y$ given by,
\[H(x, t) = \begin{cases}
(\gamma(3xt), y_0) & x \le \tfrac{1}{3} \\
(\gamma(t), \delta(3x - 1)) & x \in [\tfrac{1}{3}, \tfrac{2}{3}] \\ 
(\gamma((3x - 2)(1 - t) + t), y_0) & x \ge \tfrac{2}{3}
\end{cases}\]
First, consider the overlaps. At $x = \tfrac{1}{3}$, we have, $(\gamma(3xt), y_0) = (\gamma(t), y_0)$ and $(\gamma(t), \delta(0)) = (\gamma(t), y_0)$. At $x = \tfrac{2}{3}$, we have, $(\gamma(t), \delta(1)) = (\gamma(t), y_0)$ and $(\gamma((2 -2)(1 - t) + t), y_0) = (\gamma(t), y_0)$ so by the glueing lemma, $H$ is a continuous map. Futhermore, $H(0, t) = (\gamma(0), y_0) = (x_0, y_0)$ and $H(1, t) = (\gamma(1), y_0) = (x_0, y_0)$. Also,
\[H(x, 0) = \begin{cases}
(x_0, y_0) & x \le \tfrac{1}{3} \\
(x_0, \delta(3x - 1)) & x \in [\tfrac{1}{3}, \tfrac{2}{3}] \\ 
(\gamma(3x - 2), y_0) & x \ge \tfrac{2}{3}
\end{cases}\]
which is the path (using a triple concatenation with time divided into 1/3 intervals) $e_{(x_0, y_0)} * (\{x_0\} \times \delta) * (\gamma \times \{y_0\}) \simeq (\{x_0\} \times \delta) * (\gamma \times \{y_0\})$. Likewise,
\[H(x, 1) = \begin{cases}
(\gamma(3x), y_0) & x \le \tfrac{1}{3} \\
(x_0, \delta(3x - 1)) & x \in [\tfrac{1}{3}, \tfrac{2}{3}] \\ 
(x_0, y_0) & x \ge \tfrac{2}{3}
\end{cases}\]
which is the path $(\gamma \times \{y_0\}) * (\{x_0\} \times \delta) * e_{(x_0, y_0)} \simeq  (\gamma \times \{y_0\}) * (\{x_0\} \times \delta)$.
Thus, $H$ is a path-homotopy from $e_{(x_0, y_0)} * (\{x_0\} \times \delta) * (\gamma \times \{y_0\})$ to $e_{(x_0, y_0)} * (\{x_0\} \times \delta) * (\gamma \times \{y_0\}) $ \bigskip \\ These paths are themselves easily equivalent via reparametrization to $(\{x_0\} \times \delta) * (\gamma \times \{y_0\})$ and $(\{x_0\} \times \delta) * (\gamma \times \{y_0\})$. 

\section*{Problem 9.}

Let $p : \tilde{X} \to X$ be a covering map and $A \subset X$ have the subspace topology. Then, consider $\tilde{A} = p^{-1}(A)$ and $p' = p|_{\tilde{A}} : \tilde{A} \to A$. For each $x \in X$ there is an evenly covered neighborhood $U$ such that $\invI{p}{U}$ is a disjoint union of sets $W_\lambda$ each of which is homeomorphic to $U$ under $p$. Now, for $x \in A$ consider $\invI{p|_{\tilde{A}}}{U \cap A} = \invI{p|_{\tilde{A}}}{U} \cap \invI{p|_{\tilde{A}}}{A} = \invI{p}{U} \cap \tilde{A} = \bigsqcup\limits_{\lambda \in \Lambda} W_\lambda \cap \tilde{A}$. The sets $W_\lambda \cap \tilde{A}$ are disjoint because $W_\lambda$ are. Also, $p$ is a homeomorphism on $W_\lambda$ to $U$ and thus $p|_{\tilde{A}}$ is a homeomorphism restricted to $W_\lambda \cap \tilde{A}$ to its image $p(W_\lambda \cap \tilde{A}) = p(W_\lambda) \cap p(\tilde{A}) = U \cap A$ by properties of a bijection. Thus, $X \cap A$ is evenly covered by $p|_{\tilde{A}}$. Thus, $p|_{\tilde{A}} : \tilde{A} \to A$ is a covering map.

\section*{Problem 10.}

Let $X$ and $Y$ be path-connected and locally path-connected and let $\tilde{X}$ and $\tilde{Y}$ be simply-connected covering spaces with covering maps $p : \tilde{X} \to X$ and $q : \tilde{Y} \to Y$. Also let $f : X \to Y$ be a homotopy equivalence with homotopy ``inverse'' $g : Y \to X$. Now, by Lemma \ref{locpathcover}, the covering spaces, $\tilde{X}$ and $\tilde{Y}$ are locally path-connected. Since they are also simply-connected, all maps from $\tilde{X}$ or $\tilde{Y}$ to $X$ or $Y$ satisfy the lifting criterion. This is because $f_*(\pi_1(\tilde{X}, \tilde{x_0})) = 0$ which is trivially a subgroup of any group. \bigskip \\
Now, consider lifts of the maps $f \circ p : \tilde{X} \to Y$ and $g \circ q : \tilde{Y} \to X$, namely, $\wtilde{f \circ p} : \tilde{X} \to \tilde{Y}$ and $\wtilde{g \circ q} : \tilde{X} \to \tilde{Y}$ which satisfy 
\[p \circ \wtilde{g \circ q} = g \circ q \quad \quad q \circ \wtilde{f \circ p} = f \circ p\]
\begin{center}
\begin{tikzcd}[column sep = huge, row sep = large]
\tilde{X} \arrow[d, "p"] \arrow[dashed, r, "\wtilde{f \circ p}", bend left = 20]  & \tilde{Y} \arrow[d, "q"] \arrow[dashed, l, "\wtilde{g \circ q}", bend left = 20 ]\\
X \arrow[r, "f", bend left = 20] & Y \arrow[l, "g", bend left = 20]
\end{tikzcd}
\end{center}
Now, consider the composition,
\[p \circ (\wtilde{g \circ q} \circ \wtilde{f \circ p}) = g \circ q \circ \wtilde{f \circ p} = g \circ f \circ p = (g \circ f) \circ p \simeq \id_X \circ p = p\]
Therefore, by homotopy lifting, $(\wtilde{g \circ q} \circ \wtilde{f \circ p})$ is homotopic to some lift of $p$, namely, $r_p : \tilde{X} \to \tilde{X}$. Because $r_p$ is a lift of $p$, we must have that $p \circ r_p = p$ so $r$ is a deck transformation. However, the deck transformations form a group so if $(\wtilde{g \circ q} \circ \wtilde{f \circ p}) \simeq r_p$ then $(r_p^{-1} \circ \wtilde{g \circ q}) \circ \wtilde{f \circ p} \simeq \id_{\tilde{X}}$. \bigskip \\
Similarly,
\[q \circ (\wtilde{f \circ p} \circ \wtilde{g \circ q} ) = f \circ p \circ \wtilde{g \circ q} = f \circ g \circ q = (f \circ g) \circ q \simeq \id_Y \circ q = q\]
Therefore, by homotopy lifting, $(\wtilde{f \circ p} \circ \wtilde{g \circ q})$ is homotopic to some lift of $q$, namely, $r_q : \tilde{Y} \to \tilde{Y}$. Because $r_q$ is a lift of $q$, we must have that $q \circ r_q = q$ so $r$ is a deck transformation. 
However, the deck transformations form a group so if $(\wtilde{f \circ p} \circ \wtilde{g \circ q}) \simeq r_q$ then $\wtilde{f \circ p} \circ (\wtilde{g \circ q} \circ r_q^{-1}) \simeq \id_{\tilde{Y}}$. \bigskip \\
Therefore, by problem 5, we know that $\wtilde{f \circ p}$ is a homotopy equivalence.


\section*{Problem 11.}

\begin{enumerate}
\item Let $p : \tilde{X} \to X$ be a covering map and let $X$ be path-connected, locally path-connected, and semi-locally simply-connected. Since $X$ is locally path-connected, the path-components and components correspond. Let $x \sim y$ iff there is a path connecting $x$ and $y$ in $X$. Take $\tilde{x} \in \invI{p}{x_0}$ and consider the orbit $\orb{\tilde{x}}$ under the action of $\pi_1(X, x_0)$ via $[\gamma] \cdot \tilde{x} = \wtilde{\gamma}(1)$ where $\wtilde{\gamma}$ is the unique lift of $\gamma$ with initial point $\tilde{x}$. Now, assosicate, $\orb(\tilde{x})$ with $[\tilde{x}]$ under $\sim$. We need to show that this assoication is well-defined and one-to-one. \bigskip \\
If $\orb{\tilde{x}} = \orb{\tilde{x}'}$ then there must exist a path $\gamma$ in $X$ such that $[\gamma] \cdot \tilde{x} = \tilde{x}'$ because they lie in the same orbit. Thus, $\tilde{\gamma}(0) = \tilde{x}$ and $\tilde{\gamma}(1) = \tilde{x}'$ so the lift is a path between $\tilde{x}$ and $\tilde{x}'$. Thus, $\tilde{x} \sim \tilde{x}'$ and equivalently $[\tilde{x}] = [\tilde{x}']$. Conversely, if $[\tilde{x}] = [\tilde{x}']$ then these points must be equivalent under path-connection i.e. there exists a path $\delta : I \to \tilde{X}$ taking $\tilde{x}$ to $\tilde{x}'$. Consider, $p \circ \delta$ which is a loop in at $x_0$ in $X$ because $\delta(0) = \tilde{x} \in \invI{p}{x_0}$ and $\delta(1) = \tilde{x}' \in \invI{p}{x_0}$ so $p \circ \delta(0) = p \circ \delta(1) = x_0$. However, $[p \circ \delta] \cdot \tilde{x} = \tilde{x}'$ because $\delta$ is already the unique lift of $p \circ \delta$ at $\tilde{x}$ and thus $\orb{\tilde{x}} = \orb{\tilde{x}'}$.

\item Take $Z \subset \tilde{X}$ to be the component containing $\tilde{x}_0$. Under the Galois correspondence, $Z$ corresponds to $p_*(\pi_1(Z, \tilde{x}_0))$. Now, take $[\gamma] \in p_*(\pi_1(Z, \tilde{x}_0))$ then $[\gamma] = [p \circ \delta]$ for some loop $[\delta] \in \pi_1(Z, \tilde{x}_0)$. Consider, $[\gamma] \cdot \tilde{x}_0 = \tilde{\gamma}(1)$. However, $\delta$ is already the unique lift at $\tilde{x}_0$ because $\gamma = p \circ \delta$ and $\delta$ is based at $\tilde{x}_0$. Thus, $\tilde{\gamma} = \delta$ and $\delta$ is a loop at $\tilde{x}_0$ so $\tilde{\gamma}(1) = \tilde{x}_0$. Therefore, $[\gamma] \in \stab{\tilde{x}_0}$. \bigskip \\ 
Conversely, if $[\gamma] \in \stab{\tilde{x}_0}$ then $[\gamma] \cdot \tilde{x}_0 = \tilde{x}_0$ so the lift $\tilde{\gamma}$ at $\tilde{x}_0$ is a loop at $\tilde{x}_0$ because $\tilde{\gamma}(1) = [\gamma] \cdot \tilde{x}_0 = \tilde{x}_0$. Furthermore, $\tilde{\gamma}$ must be resticted to $Z$ because the image of any path must be contained in a single path component. Therefore, $[\tilde{\gamma}] \in \pi_1(Z, \tilde{x}_0)$ and thus, $[p \circ \tilde{\gamma}] \in p_*(\pi_1(Z, \tilde{x}_0))$ but $p \circ \tilde{\gamma} = \gamma$ so $[\gamma] \in p_*(\pi_1(Z, \tilde{x}_0))$. Therefore,
\[p_*(\pi_1(Z, \tilde{x}_0)) = \stab{\tilde{x}_0}\]

\end{enumerate}
     
              
\section*{Lemmas}

\begin{lemma} \label{locpathcover}
If $p : \tilde{X} \to X$ is a covering map and $X$ is locally path-conected then $\tilde{X}$ is locally path connected. 
\end{lemma}

\begin{proof}
Take $\tilde{x} \in \tilde{X}$ and an open $\tilde{x} \in A \subset \tilde{X}$. Now, consider $x = p(\tilde{x}) \in X$ which has an evenly covered neighborhood $x \in U$. Furthermore, because $X$ is locally path-connected, there is a path-connected neighborhood $V$ of $x$ such that, $x \in V \subset U \cap p(A)$ because $p(A)$ is open since every covering map is an open map. However, $\invI{p}{U}$ is a disjoint union of $W_\alpha$ on each of which $p$ restricts to a homeomorphism. Therefore, since $\tilde{x} \in \invI{p}{U \cap p(A)}$ take $W_\lambda$ to be the slice containing $\tilde{x}$. Then, $p$ restricted to $W_\lambda$ is a homeomorphism and therefore must take the path connected neighborhood $V$ of $x$ to a path connected neighborhood $\tilde{x} \in p|_{W_\lambda}^{-1}(V) \subset A$ where the final inclusion follows because $V \subset p(A)$ and $p$ is a homeomorphism on $W_\lambda$.  
\end{proof}



\end{document}
