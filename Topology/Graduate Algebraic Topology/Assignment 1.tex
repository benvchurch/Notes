\documentclass[12pt]{extarticle}
\usepackage{import}
\import{./}{Includes}

\begin{document}
\atitle{1}

\section{Chapter 0}

\subsection{Problem 26}

Let the pair $(X, A)$ have the homotopy extension property. I will use the following proposition, 

\begin{proposition}[Hatcher 0.20]
If $(X, A)$ satisfies the homotopy extension property and the inclusion map $A \embed X$ is a homotopy equivalence then it is a homotopy equivalence $\mathrm{rel} \: A$ or equivalently that $A$ is a deformation retract of $X$. 
\end{proposition}
\noindent
In particular, I will apply this proposition to the pair $(X \times I, X \times \{ 0 \} \cup A \times I)$ to show that $X \times I$ deformation retracts onto $X \times \{ 0 \} \cup A \times I$. By the proposition, it suffices to show that $(X \times I, X \times \{ 0 \} \cup A \times I)$ has the homotopy extension property and that $X \times \{ 0 \} \cup A \times I \embed X \times I$ is a homotopy equivalence. 
\bigskip\\
The homotopy extension property for $(X \times I, X \times \{ 0 \} \cup A \times I)$ follows directly from the homotopy extension property for $(X, A)$. Recall that the homotopy extension property is equivent to the existence of a retract $r : X \times I \to X \times \{ 0 \} \cup A \times I$. So we need to find a retract,
\[ r' : X \times I \times I \to X \times I \times \{ 0 \} \cup (X \times \{ 0 \} \cup A \times I) \times I = X \times (I \times \{ 0 \} \cup \{ 0 \} \times I) \cup A \times I \times I \]
However, we may deform the corner $(I \times \{ 0 \} \cup \{ 0 \} \times I)$ to a single edge $I \times \{ 0 \}$ so the subspace becomes,
\[ X \times I \times \{ 0 \} \cup A \times I \times I \]
so it suffices to show that $(X \times I, A \times I)$ has the homotopy extension property which is clear since we can take the retract $X \times I \to X \times \{0\} \cup A \times I$ and multiply by $I$ to get a retract $X \times I \times I \to X \times I \times \{ 0 \} \cup A \times I \times I$. 
\bigskip\\
Now we need to show that $\iota : X \times \{0 \} \cup A \times I \embed X \times I$ is a homotopy equivalence. We already have a retract $r : X \times I \to X \times \{ 0 \} \cup A \times I$ via the homotopy extension property such that $r \circ \iota = \id$. Thus it suffices to show that $\iota \circ r \sim \id$. Consider the following homotopy $h : X \times I \times I \to X \times I$,
\[ h(x, s, t) =  
\begin{cases}
(x, s (1 - 2t)) &  t \in [0, \tfrac{1}{2}] 
\\
\iota \circ r(x, s (2 t - 1)) & t \in [\tfrac{1}{2}, 1]
\end{cases} \] 
which is continuous at $t = \tfrac{1}{2}$ since $\iota \circ r(x, 0) = \iota(x, 0) = (x, 0)$ because $(x, 0) \in X \times \{ 0 \} \cup A \times I$ on which it is trivial since it is a retract. Thus $h$ is a homotopy from $h(-,-,0) = \id$ to $h(-,-,1) = \iota \circ r$. Thus $\iota : X \times \{ 0 \} \cup A \times I \embed X \times I$ is a homotopy equivalence. 
\bigskip\\
Now let $(X_1, A)$ be a pair satisfing the homotopy extension property and $f,g : A \to X_0$ are homotopic attaching maps. Consider a homotopy $F : A \times I \to X_0$ between $f$ and $g$ then $X_1 \sqcup_F (A \times I)$ contains $X_0 \sqcup_f X_1$ and $X_0 \sqcup_g X_1$. We have shown that there is a deformation retract,
\[ X_1 \times I \to X_1 \times \{ 0 \} \cup A \times I \]
This deformation retract gives a pair of deformation retracts $X_0 \sqcup_F (X_1 \times I) \to X_0 \sqcup_f X_1$ and $X_0 \sqcup_F (X_1 \times I) \to X_0 \sqcup_g X_1$ and thus a homotopy equivalence $X_0 \sqcup_f X_1 \sim X_0 \sqcup_g X_1 \: \: \mathrm{rel} \: X_0$ proving proposition 0.18 in general for a pair $(X, A)$ with the homotopy extension property. 

\subsection{Problem 29}

Let $X$ be a CW complex obtained from a subcomplex $A$ by attaching an $n$-cell $D^n$ via an attaching map $g : S^{n-1} \to A$. Consider a homotopy $f_t : A \to Y$ and a map $f_0' : D^n \to Y$ such that $f_0 \circ g$ and $f_0' |_{S^{n-1}}$ agree. Using the projection idea to form a deformation retract $D^n \times I \to D^n \times \{ 0 \} \cup S^{n-1} \times I$ we get the following explicit homotopy extension to $f' : D^n \times I \to Y$ via,
\[ f'_t(x) = 
\begin{cases}
f_{t'(x)} \circ g \left( \tfrac{x}{|x|} \right) & 2|x| > (2 - t)
\\
f_0' \left( \tfrac{2x}{2 - t} \right) & 2|x| \le (2 - t)
\end{cases} \]
where we have defined,
\[ t'(x) = \frac{2|x| - (2 - t)}{|x|} \]
Note that when $2 |x| = (2 - t)$ then $y = \frac{2 x}{2 - t} = \frac{x}{|x|} \in S^{n-1}$ so $f_0'(y) = f_0 \circ g(y)$ and $t' = 0$ and thus the maps glue to give a continuous homotopy extension. 

\section{Chapter 4.1}

\subsection{Problem 3}

Let $X$ be an H-space i.e. a pointed space $(X, x_0)$ with a multiplication map $\mu : X \times X \to X$ which is unital in the homotopy category meaning $\mu \circ (\id \times e) \sim \mu \circ (e \times \id) \sim \id$ where $e : * \to X$ is the inclusion of the identity at the basepoint $x_0$.
\bigskip\\
Now consider the induced group map $\mu_* : \pi_n(X \times X) \to \pi_n(X)$ which gives,
\begin{center}
\begin{tikzcd}
\pi_n(X \times X) \arrow[r, "\mu_*"] \arrow[r] \arrow[d, "\sim"] &  \pi_n(X)
\\
\pi_n(X) \oplus \pi_n(X) \arrow[ru, "\mu_*"] 
\end{tikzcd}
\end{center}
Since $\mu_*$ is a group homomorphism, we have (denoting multiplication / addition in $\pi_n$ as $*$),
\[ \mu_*((\gamma_1, \gamma_2) * (\delta_1, \delta_2)) = \mu_*(\gamma_1, \gamma_2) * \mu_*(\delta_1, \delta_2) \]
However, by definition of the product group,
\[ (\gamma_1, \gamma_2) * (\delta_1, \delta_2) = (\gamma_1 * \delta_1, \gamma_2 * \delta_2) \]
which gives,
\[ \mu_*(\gamma_1 * \delta_1, \gamma_2 * \delta_2) = \mu_*(\gamma_1, \gamma_2) * \mu_*(\delta_1. \delta_2) \]
To clarify the notation, I will write $\gamma_1 \circ \gamma_2 : = \mu_*(\gamma_1. \gamma_2)$ which denotes the map sending a pair of classes of maps to the class of their ``product'' under $\mu$ i.e. $(\gamma_1 \circ \gamma_2)(t) = \gamma_1(t) \circ \gamma_2(t) = \mu(\gamma_1(t), \gamma_2(t))$. Using this notation, we have,
\[ (\gamma_1 * \delta_1) \circ (\gamma_2 * \delta_2) = (\gamma_1 \circ \gamma_2) * (\delta_2 \circ \delta_2) \]
Furthermore, both operations are unital, letting $e$ denote the class of the constant map at $x_0$ then $\gamma * e = \gamma * e = \gamma$ (here these represent classes of paths) and we know $\mu_* \circ (\id \times e) = \mu_* \circ (e \times \id) = \id$ so $\gamma \circ e = e \circ \gamma = \gamma$ (since $* \to X$ gives the map including the class of the constant map into $\pi_n(X)$). 
\bigskip\\
The Eckmann-Hilton argument (proved in Lemma \ref{Eckmann_Hilton}) shows that any pair of unital operations which satisfy the above intertwining relation must agree and further must be commutative and associative. Therefore, $\circ = *$ so we find that $\gamma_1 * \gamma_2 = \mu_*(\gamma_1, \gamma_2)$ and we also get for free that $\pi_1(X)$ is abelian for any H-space $X$. 

\subsection{Problem 10}


Consider the quasi-circle $X$ which contains an arc and $S$ the graph of $\sin{(1/x)}$ on $(0, \pi]$ completed with the interval $L = [-1, 1]$ at zero. Thus the map $\R \to X$ sarting at $0$ going around the arc and tracing the $\sin{(1/x)}$ curve hits everything except for $L = [-1,1]$. Call the image of this map $C \subset X$ then $X = C \cup L$. 
\bigskip\\
Consider a map $f : S^n \to X$. I claim that any such map has image contained in $X_a = X \setminus \{ (x, \sin{(1/x)}) \mid 0 < x < a \}$ for some $a > 0$ which is contractible and thus $f$ is nullhomotopic implying that $\pi_n(X) = 0$. To show this suppose that $S^n$ hits  points $(x, \sin{(1/x)})$ for arbitrarily small $x$. By connectivity of $S^n$ then $f$ must hit $\{ (x, \sin{(1 / x)} \mid 0 < x \le 1 \}$. Choose a point $y \in L$ and an open neighborhood $V$ of $y$ which is not connected (this will hold as long as it does not wrap around the arc) such that $X \setminus X_a \subset V$. Then consider the connected components $U_i$ of $f^{-1}(V)$ which are open since $S^n$ is locally path-connected. Now $\overline{U}_i$ is also path-connected so $f(\overline{U_i})$ must be contained in a path component of $V$ and thus either hits $L$ or is contained entirely in $S$. However, in the latter case, since $\overline{U}_i$ is compact then $f(\overline{U}_i) \subset \{ (x, \sin{(1/ x)} \mid [a_i, b_i] \}$ for some closed interval $[a_i, b_i]$ with $a_i, b_i > 0$ since it must be compact and thus closed so $f(\overline{U}_i)$ must hit any limit points so it cannot get arbitrarily close to $L$ since $L$ is not in the image. Finally, let $W = f^{-1}(X_{a/2}^\circ)$ then since $X_{a/2}^\circ$ and $V$ cover $X$ we get an open cover $U_i$ and $W$ of $S^n$ such that $f(U_i) \subset X_{a_i}$ and $f(W) \subset X_{a/2}$. Since $S^n$ is compact we cana take this cover to be finite. Then taking $b = \min \{ a_i, a/2 \}$ which is positive for finitely many $a_i$ we find that $f(U_i) \subset X_{a_i} \subset X_{b}$ and $f(W) \subset X_{a/2} \subset X_b$. Since these opens form a cover of $S^n$ we find that $f(S^n) \subset X_b$ which is contractible and thus $f : S^n \to X$ is nullhomotopic so $\pi_n(X) = 0$ for $n > 0$. 
\bigskip\\
However, I claim that $X$ is not contractible showing, by Whitehead's theorem that $X$ is not homotopy equivalent to a CW complex. If $X$ we contractible then the map $f : X \to S^1$ given by projecting down the sine curve and contracting $L$ would be nullhomotopic. However, such a nullhomotopy would imply the existence of a lift $\tilde{f} : X \to \R$ over $p : \R \to S^1$ by the homotopy lifting property (since constant maps trivially have lifts). However, exercise 7 of $\S 1.3$ asks us to show this is impossible. $f : C \to S^1$ is the identity loop minus the base point and thus must lift to the injective increasing path $(0, 2\pi) \subset \R$ (up to choice of basepoint). However, $X$ is compact (closed and bounded in $\R^2$) so its image under $\tilde{f}$ must be compact and thus contain $[0, 2\pi] \subset \R$ so $\tilde{f}(L) \supset \{ 0, 2 \pi \}$. However, $f(L) = x_0$ is the basepoint of $S^1$ so $\tilde{f}(L) \subset p^{-1}(x_0)$ which is discrete and $L$ is connected so $\tilde{f}$ must be a single point contradicting the fact that $\tilde{f}(L) \supset \{ 0, 2 \pi \}$ showing that no lift,
\begin{center}
\begin{tikzcd}
& \R \arrow[d, "p"]
\\
X \arrow[ru, dashed, "\tilde{f}"] \arrow[r, "f"'] & S^1
\end{tikzcd}
\end{center}
can exist. Thus $X$ is not contractible. 

\subsection{Problem 14}

Let $f : X \to Y$ be a homotopy equivalence between CW complexes with no $(n+1)$-cells and let $g : Y \to X$ be a homotopy inverse. By cellular approximation we may take these maps to be cellular $f : X^n \to Y^n$ and $g : X^n \to Y^n$. Furthermore, we have homotopies $h_1 : X \times I \to X$ and $h_2 : Y \times I \to Y$ taking $g \circ f$ and $f \circ g$ to the identities. Again by cellular approximation we may take these maps to be cellular (note that on $X \times \{ 0, 1 \} \subset X \times I$ this map is cellular since it is $g \circ f$ and $\id$ then we can take our homotopy rel $X \times \{ 0, 1 \}$ meaning that our new cellular homotopy still takes $g \circ f$ to $\id$). Thus we get a homotopy $h_1 : (X \times I)^{n+1} \to X^{n+1}$. But since $X$ has no $(n + 1)$-cells we have $X^{n+1} = X^n$ and $(X \times I)^{n+1} = X^n \times I$ so we have a homotopy $h_1 : X^n \times I \to X^n$ from $g \circ f$ to $\id$. We similarly get a cellular homotopy $h_2 : Y^n \times I \to Y^n$ from $f \circ g$ to $\id$. Thus $X^n \sim Y^n$. 

\subsection{Problem 20}

Suppose that $X$ is a finite connected CW complex of dimension $\dim{X} = n$ and $\pi_i(Y)$ is finite for $i \le n$. First note that $X$ is path-connected so its image must land in a path component $Y_i \in \pi_0(Y)$. Thus,
\[ [X, Y] = \bigcup_{Y_i \in \pi_0(Y)} [X, Y_i] \]
Since $\pi_0(Y)$ is finite by assumption, it suffices to show that each $[X, Y_i]$ is finite so we may reduce to the case that $Y$ is path-connected. Further, when $Y$ is path-connected $\left< X, Y \right> = [X, Y] / \pi_1(Y, y_0)$ but $\pi_1(Y, y_0)$ is finite so $[X, Y]$ is finite iff $\left< X, Y \right>$ is finite so it suffices to show the pointed version. We will induct on the number of (positive dimensional) cells of $X$. If $X$ has a single (positive dimensional) then $X \cong S^n$ in which case $\left< X, Y \right> = \pi_n(Y)$ which we assume is finite. Now suppose that $X$ is constructed by attaching a $k$-cell  $D^k$ for $k \le n$ to a subcomplex $A$ via an attaching map $a : S^{k-1} \to A$. Then $X = A \sqcup_a D^k$ and we get maps $S^{k-1} \to A \to X$. The inclusion induces a map,
\[ [X, Y] \xrightarrow{\Phi} [A, Y] \]
Consider the kernel $\ker{\Phi}$, if a map $f: X \to Y$ restricts to $f|_A : A \to Y$ which is homotopic tvia $(f|_A)_t : A \times I \to Y$ then by the homotopy extension property of the CW pair $(X, A)$ we may extend this to a homotopy $f_t : X \times I \to Y$ such that $f_1 |_A$ is constant i.e. $f : X \to Y$ is homotopic to a pair $(f', c)$ with $c : A \to Y$ constant and $f' : D^k \to Y$ restricted such that $f'|_{S^{k-1}} = c \circ a$ a constant. Therefore, $f'$ defines a class in $\pi_k(Y)$ since as we homotope it we must fix its value on $\partial D^{k} = S^{k-1}$ to be exactly the point $c(A)$. Thus, $\ker{\Phi} = [S_k, Y]$. A similar argument shows that the fibres of $[X, Y] \to [A, Y]$ correspond exactly to classes $[S^k, Y]$. Since $\pi_k(Y)$ is finite by assumption then the fibre $[S^k, Y]$ is also finite (using $Y$ path-connected) and thus the map $[X, Y] \to [A, Y]$ has finite fibres. Furthermore, by induction we assume that $[A, Y]$ is finite which implies that $[X, Y]$ is finite as well since it maps onto a finite set with finite fibres proving by induction that $[X, Y]$ is finite for any finite connected CW complex $X$ with $\dim{X} = n$.  

\section{Chapter 4.2}

\subsection{Problem 2}

First note that the action of paths $[\gamma]$ on $\pi_n$ is natural in the following sense: given a map $f : X \to Y$ we get a commutative diagram,
\begin{center}
\begin{tikzcd}
\pi_n(X, x_0) \arrow[d, "f_*"] \arrow[r, "\gamma"] & \pi_n(X, x_1) \arrow[d, "f_*"]
\\
\pi_n(Y, f(x_0)) \arrow[r, "f_*(\gamma)"] & \pi_n(Y, f(x_1))
\end{tikzcd}
\end{center}
We might say that this forms a functor of groupoid actions. Restricting to the case of a covering map $p : \tilde{X} \to X$ and paths $\gamma$ between points on the same fibre $\tilde{x}_0, \tilde{x}_1 \in p^{-1}(x_0)$ we get,
\begin{center}
\begin{tikzcd}
\pi_n(\tilde{X}, \tilde{x}_0) \arrow[d, "p_*"] \arrow[r, "\gamma"] & \pi_n(\tilde{X}, \tilde{x}_1) \arrow[d, "p_*"]
\\
\pi_n(X, x_0) \arrow[r, "p_*(\gamma)"] & \pi_n(X, x_0)
\end{tikzcd}
\end{center}
However, the two vertical isomorphisms are not clearly commensurable since they have non-canonically isomorphic source groups. However, if the vertical maps on $\pi_1$ have the same image (which always happens for the universal cover since $\pi_1(\tilde{X}) = 0$) i.e. $p_* \pi_1(\tilde{X}, \tilde{x}_0) = p_* \pi_1(\tilde{X}, \tilde{x}_1)$ there there is a deck transformation $f : \tilde{X} \to \tilde{X}$ which takes $\tilde{x}_1$ to $\tilde{x}_0$. Furthermore, since by definition, $p \circ f = p$ we may extend the diagram to,
\begin{center}
\begin{tikzcd}
\pi_n(\tilde{X}, \tilde{x}_0) \arrow[d, "p_*"] \arrow[r, "\gamma"] & \pi_n(\tilde{X}, \tilde{x}_1) \arrow[d, "p_*"] \arrow[r, "f_*"] & \pi_n(\tilde{X}, \tilde{x}_0) \arrow[d, "p_*"]
\\
\pi_n(X, x_0) \arrow[r, "p_*(\gamma)"] & \pi_n(X, x_0) \arrow[r, "\id"] & \pi_n(X, x_0)
\end{tikzcd}
\end{center}
Then the leftmost and rightmost maps are, for $n \ge 2$, the fixed isomorphism $p_* : \pi_n(\tilde{X}, \tilde{x}_0) \to \pi_n(X, x_0)$ and the bottom map is exactly the action of the loop $p_*(\gamma) \in \pi_1(X, x_0)$ on $\pi_n(X, x_0)$ so using the isomorphism we can describe this action as the composition $f_* \circ \gamma$ acting on $\pi_n(\tilde{X}, \tilde{x}_0)$. Finally, by the path-lifting property, $\pi_1(X, x_0)$ is surjected onto by loops of the form $p_*(\gamma)$ for paths with endpoints in the fibre $p^{-1}(x_0)$.
\bigskip\\
Now consider the covering map $p : S^n \to \RP^n$ and choose a basepoint $x_0 \in \RP^n$ in the image of the north and south poles $N,S \in S^n$. There are two homotopy classes of paths with endpoints in the fibre $p^{-1}(x_0) = \{ N, S \}$, constant loops which clearly act trivially, and paths which go from one pole to the other. Let $\gamma$ be a path from $N$ to $S$ in $S^n$ and let $f : S^n \to S^n$ be the corresponding deck transformation which swaps these poles, namely the antipodal map $f = - \id$. Thus we only need to compute $f_* \circ [\gamma] : \pi_n(S^n, N) \to \pi_n(S^n, N)$. The group $\pi_n(S^n, N)$ is generated by the class $g = [ \id : S^n \to S^n ]$ so it suffices to check where this generator ends up. First, $[\gamma](g)$ corresponds to the map $R_\pi : S^n \to S^n$ which rotates the $n$-sphere by a half-turn as to swap $N$ and $S$ this corresponds to inverting a pair of axes one through the poles and one perpendicularly. Then the map $f_*([\gamma](g)) = f_* \circ R_\pi$ applies another inversion to each of the $n+1$ coordinates such that $f_* \circ R_\pi : S^n \to S^n$ is based at $N$. Furthermore, since inverting any coordinate of the class $g$ sends it to its inverse $-g$ then we see that $f_* \circ R_\pi = (-1)^{n + 3} g$. Therefore, when $n$ is odd the action of $[p \circ \gamma]$ on $\pi_n(\RP^n)$ is trivial and when $n$ is even it is the negation map $\Z \to \Z$. Since $\pi_1(\RP^n) = \Z / 2 \Z$ generated by this loop $[p \circ \gamma]$ we see that $\pi_1(\RP^n) \acts \pi_n(\RP^n)$ trivially when $n$ is odd and via negation when $n$ is even. 

\subsection{Problem 12} 

Let $f : X \to Y$ be a map of connected CW complexes such that $f_* : \pi_1(X) \xrightarrow{\sim} \pi_1(Y)$ is an isomorphism. Consider universal covers $p_X : \tilde{X} \to X$ and $p_Y : \tilde{Y} \to Y$  which we may give CW complex structures. Since $f \circ p_X : \tilde{X} \to Y$ is a map from a simply-connected (connected and locally-path-connected) space we get a lift,
\begin{center}
\begin{tikzcd}[column sep = large, row sep = large]
\tilde{X} \arrow[d, "p_X"'] \arrow[r, dashed, "\tilde{f}"] & \tilde{Y} \arrow[d, "p_Y"] 
\\
X \arrow[r, "f"'] & Y
\end{tikzcd}
\end{center}
We further assume that $\tilde{f}_* : H_i(\tilde{X}, \Z) \xrightarrow{\sim} H_i(\tilde{Y}, \Z)$ is an isomorphism. Since $\tilde{X}$ and $\tilde{Y}$ are simply-connected CW complexes we may apply the homological version of Whithead's theorem to conclude that $\tilde{f}$ is a homotopy equivalence. Now consider the diagram,
\begin{center}
\begin{tikzcd}[column sep = large, row sep = large]
\pi_n(\tilde{X}) \arrow[d, "(p_X)_*"'] \arrow[r, dashed, "\tilde{f}_*"] & \pi_n(\tilde{Y}) \arrow[d, "(p_Y)_*"] 
\\
\pi_n(X) \arrow[r, "f_*"'] & \pi_n(Y)
\end{tikzcd}
\end{center}
but $\tilde{f} : \tilde{X} \to \tilde{Y}$ is a homotopy equivelnce so $\tilde{f}_* : \pi_n(\tilde{X}) \xrightarrow{\sim} \pi_n(\tilde{Y})$ is an isomorphism. Furthermore since $p_X$ and $p_Y$ are covering maps, for $n \ge 2$ they induce isomorphism $(p_X)_* : \pi_n(\tilde{X}) \xrightarrow{\sim} \pi_n(X)$ and $(p_Y)_* : \pi_n(\tilde{Y}) \xrightarrow{\sim} \pi_n(Y)$. Thus by commutativity of the diagram $f_* : \pi_n(X) \xrightarrow{\sim} \pi_n(Y)$ is an isomorphism for $n \ge 2$ and since we already know this map is an isomorphism for $n = 1$ then we may apply the standard Whitehead's theorem to show that $f : X \to Y$ is a homotopy equivalence. 
 
\subsection{Problem 14}

Let $X$ be a CW complex of dimension $\dim{X} = n$ and $Y$ a subcomplex of $X$ which is homotopy equivalent to $S^n$. Then consider the natruality square of the Hurewicz map,
\begin{center}
\begin{tikzcd}
\pi_n(Y) \arrow[r] \arrow[d, "h_n"] & \pi_n(X) \arrow[d, "h_n"]
\\
H_n(Y) \arrow[r] & H_n(X)
\end{tikzcd}
\end{center}
However, since $Y \sim S^n$ we know that $Y$ is $(n-1)$-connected so $h_n : \pi_n(Y) \xrightarrow{\sim} H_n(Y)$ is an isomorphism. Furthermore, the homological LES associated to the pair $(X, Y)$ gives,
\begin{center}
\begin{tikzcd}
H_{n+1}(X, Y) \arrow[r] & H_n(Y) \arrow[r] & H_n(X) \arrow[r] & H_n(X, Y) 
\end{tikzcd}
\end{center} 
But since $(X, Y)$ is a CW pair (and thus good) we have $H_{n+1}(X, Y) = \tilde{H}_{n+1}(X / Y) = 0$ since $\dim{X / Y} = n$. Therefore the map $H_{n}(Y) \hookrightarrow H_n(X)$ is an injection. Returning to the Hurewicz diagram, the left and bottom maps are injective which implies that the top map $\pi_n(Y) \to \pi_n(X)$ must be injective as well. 

\subsection{Problem 15}

Let $M$ be a closed simply-connected 3-manifold. First I argue that $M$ must be orientable since $M$ is simply-connected via Lemma \ref{orientable_if_simply_connected}.  By Hurewicz theorem for $1$-connected spaces we find that $H_1(M, \Z) = 0$ and $h_2 : \pi_2(M) \xrightarrow{\sim} H_2(M, \Z)$ is an isomorprhism. Furthermore, by Poincare duality, there is a fundamental class $[M] \in H_d(M, \Z)$ and the map $H^i(M, \Z) \xrightarrow{\sim} H_{d-i}(M, \Z)$ defined by the cap product $\alpha \mapsto [M] \frown \alpha$ is an isomorphism. Furthermore, the universal coefficent theorem for cohomology gives an exact sequence,
\begin{center}
\begin{tikzcd}
0 \arrow[r] & \Ext{1}{\Z}{H_{i-1}(M, \Z)}{\Z} \arrow[r] & H^i(X, \Z) \arrow[r] & \Hom{\Z}{H_i(X, \Z)}{\Z} \arrow[r] & 0
\end{tikzcd}
\end{center}
Thus, since $H_1(X, \Z) = 0$ and $H_0(X, \Z) = \Z$ and $\Ext{1}{\Z}{\Z}{\Z} = 0$ because $\Z$ is projective as a $\Z$-module the exact sequence gives $H^1(X, \Z) = 0$. Then applying Poincare duality shows that $H_2(M, \Z) = 0$ so by Hurewicz $\pi_2(M) = 0$. Therefore, $M$ is 2-connected so we get a Hurewicz isomorphism $h_3 : \pi_3(M) \to H_3(M, \Z) = [M] \Z$. 
\bigskip\\
Recall that the Hurewicz map $h_n : \pi_n(X) \to H_n(X, \Z)$ is defined via $[f] \mapsto f_*([S^n])$. Since $h_3 : \pi_3(M) \to H_3(M, \Z)$ is an isomorphism there is some $f : S^3 \to M$ such that $h_3([f]) = f_*([S^n]) = [M]$. Since $H_3(S^3, \Z) = [S^n] \Z$ then $f_* : H_3(S^3, \Z) \to H_3(M, \Z)$ takes the generator to the generator and thus is an isomorphism. Furthermore, for $0 < i < 3$ we know that $H_i(S^3, \Z) = H_i(M, \Z) = 0$ so $f_* : H_i(S^3, \Z) \xrightarrow{\sim} H_i(M, \Z)$ is an isomorphism for all $i$ (since both vanish above $\dim{S^3} = \dim{M} = 3$). Now we use the fact that every manifold has the homotopy type of a CW complex to replace $M$ by some homotopy equivalent CW complex $X$ thus $f_* : H_i(S^3, \Z) \xrightarrow{\sim} H_i(X, \Z)$ is an isomorphism of simply-connected CW complexes and thus $f : S^3 \to X$ is a homotopy equivalence by the homological version of Whitehead's theorem. Therefore, $M \sim X \sim S^3$. 

\section{Lemmas}

\label{Eckmann_Hilton}
\begin{lemma}[Eckmann-Hilton]
Suppose that $*$ and $\circ$ are untial operations satisfing the following intertwining relation,
\[ (a * b) \circ (c * d) = (a \circ c) * (b \circ d) \]
Then $* = \circ$ and both are commutative and associative. 
\end{lemma}

\begin{proof}
First we need to show that the units agree i.e. $1_* = 1_\circ$. We simply plug into the relation,
\[ (1_* * 1_\circ) \circ (1_\circ * 1_*) = (1_* \circ 1_\circ) * (1_\circ \circ 1_*) \]
The left-hand side is $1_\circ \circ 1_\circ = 1_\circ$ and the right-hand side is $1_* * 1_* = 1_*$ so indeed $1_\circ = 1_*$. 
\bigskip\\
Now apply the relation to get,
\[ (a * 1) \circ (1 * b) = (a \circ 1) * (1 \circ b) \]
but since the units agree this is simply $a \circ b = a * b$ so the operations agree. For commutativity, note that,
\[ (1 \circ a) \circ (b \circ 1) = (1 \circ b) \circ (a \circ 1) \]
so applying the unit properties $a \circ b = b \circ a$. Finally, for associativity, note that,
\[ (a \circ b) \circ (1 \circ c) = (a \circ 1) \circ (b \circ c) \]
gives,
\[ (a \circ b) \circ c = a \circ (b \circ c) \]
\end{proof}


\label{orientable_if_simply_connected}
\begin{lemma}
Let $M$ be a simply-connected manifold then $M$ is orientable.
\end{lemma}

\begin{proof}
If $M$ were not orientable then it would have a connected orientation double cover $p : \tilde{M} \to M$ but since $\pi_1(M) = 0$, by the classiciation of covering spaces, $M$ has no nontrivial (in paricular two-sheeted) connected covering spaces (in the case that $M$ is orientable the orientation cover degenerates to the trivial disconnected two-sheeted cover). In particular, if $M$ is not orientable then $\pi_1(M)$ must have an index two subgroup corresponding to orientation preserving loops. 
\end{proof}

\end{document}
