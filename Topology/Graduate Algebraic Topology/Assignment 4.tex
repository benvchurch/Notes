\documentclass[12pt]{extarticle}
\usepackage{import}
\import{./}{Includes}
\newcommand{\C}{\mathbb{C}}
\newcommand{\F}{\mathbb{F}}
\newcommand{\Vect}{\mathrm{Vect}}

\begin{document}
\atitle{4}

\section{Maps of Hopf Invariant Two}

Recall that the Hopf invariant is a integer $h(f) \in \Z$ defined for maps $f : S^{2n - 1} \to S^n$ as follows.

\begin{defn}
Let $f : S^{2n - 1} \to S^n$ be a continuous map. Then consider $C_f = D^{2n} \cup_f S^n$. Choosing generators we have $H^n(C_f ; \Z) = \alpha \Z$ and $H^{2n}(C_f; \Z) = \beta \Z$. Then,
\[ \alpha^2 \in H^{2n}(C_f ; \Z) \implies \alpha^2 = h(f) \beta \] 
\end{defn} 

\begin{rmk}
Notice that when $n$ is odd $\alpha^2 = \alpha \smile \alpha = 0$ since $\alpha$ has odd degree. Therefore, we may restrict our consideration to maps $f : S^{4n - 1} \to S^{2n}$. 
\end{rmk}

\begin{prop}
The Hopf invariant gives a homomorphism $h : \pi_{2n - 1}(S^n) \to \Z$ with the following properties,
\begin{enumerate}
\item if $n$ is odd then $h = 0$ (since $\alpha \smile \alpha = 0$ in odd $n$).
\item for the Hopf fibration $H : S^3 \to S^2$ then $C_f = S^2 \cup_H D^4 = \CP^2$ and $H^*(\CP^2; \Z) = \Z[x]/(x^3)$ so the generator of $H^2(\CP^2; \Z)$ squares to the generator of $H^4(\CP^2; \Z)$ which implies that $h(H) = 1$. In particular, $h : \pi_3(S^2) \xrightarrow{\sim} \Z$ sending $H \mapsto 1$. 
\end{enumerate}
\end{prop}
\noindent\\
Our main result is the following.

\begin{theorem}
For all $n$, there exists a map $f : S^{4n - 1} \to S^{2n}$ with Hopf invariant: $h(f) = 2$. 
\end{theorem}
\noindent\\
To prove this theorem, we consider the following spaces.

\subsection{The James Restricted Product}

\begin{defn}
Let $(X, e)$ be a based topological space. Define the \textit{James restricted product} as the following quotient space,
\[ J_k(X) = X^k / \sim \]
where we identify $(x_1, \dots, x_i, e, \dots, x_k) \sim (x_1, \dots, e, x_i, \dots, x_k)$. Furthermore, we can define the total James space, $J(X) = \varinjlim J_m(X)$. 
\end{defn}

\begin{example}
We have $J_1(X) = X$ and $J_2(X) = X \times X / (x, e) \sim (e, x)$. 
\end{example}
\noindent\\
When $X$ is a CW complex, $J_m(X)$ inherits a CW complex structure from the product CW structure on $X$. Explicitly, we glue together the sub-complexes with one coordinate fixed at $e$. These James restricted products are especially interesting for us in the case of spheres in which case the cohomology is particularly easy to understand.

\begin{theorem} \label{cup_relation}
Fix even $n > 0$. Then,
\[ H^q(J(S^n); \Z) = 
\begin{cases}
\Z & n \divides q
\\
0 & \text{else}
\end{cases} \]
Let $\alpha_k \in H^{nk}(J(S^n); \Z)$ be a generator. If $n$ is even then for each $k \ge 1$ we have $\alpha_1^k = \pm k! \cdot \alpha_k$. 
\end{theorem} 

\begin{proof}
Let $S^n$ have its usual CW structure $e^0 \cup e^n$. Then we get a product-quotient CW structure on $J(S^n)$ which is $e^0 \cup e^n \cup e^{2n} \cup e^{3n} \cup \cdots$. Therefore, we immediately see that $H^q(J(S^n); \Z) = 0$ whenever $n \ndivides q$. Furthermore, assuming $n > 1$ (we only need this case) the cellular chain complex is $C_{nk} = \Z$ and otherwise $C_q = 0$ so the complex has segments,
\begin{center}
\begin{tikzcd}
\cdots \arrow[r] & 0 \arrow[r] & C_{nk} \arrow[r] & 0 \arrow[r] & \cdots 
\end{tikzcd}
\end{center}
Therefore, $H^{nk}(J(S^n); \Z) = \alpha_k \Z$ generated by $\alpha_k$ which is dual to the $nk$-cell $e^{nk}$. It remains to compute the cup product structure. 
\bigskip\\
Consider the quotient map $q : (S^n)^k = S^n \times \cdots \times S^n \to J_k(S^n)$. Now consider $H^n((S^n)^k; \Z)$. By Kunneth,
\[ H^n((S^n)^k; \Z) = \bigoplus_{i = 1}^k x_i \cdot H^n(S^n ; \Z) = x_1 \Z \oplus \cdots \oplus x_k \Z \]
where $x_i \in H^n((S^n)^k ; \Z)$ is the generator dual to the $n$-cell of $(S^n)^k$ corresponding to each $S^n$. Since the map $q : (S^n)^k \to J_k(S^n)$ glues these $n$-cells to form the singular $n$-cell $e^n$ we find that,
\[ q^*(\alpha_1) = x_1 + \cdots + x_k \]
Furthermore, by the Kunneth formula,
\[ H^*((S^n)^k ; \Z) = \bigotimes_{i = 1}^k H^*(S^n ; \Z) = \bigotimes_{i = 1}^k \Z[\alpha_i]/(\alpha_i^2) = \Z[\alpha_1, \dots, \alpha_k]/(\alpha_1^2, \dots, \alpha_k^2) \]
which. when $n$ is even, is a commutative ring (we need not worry about factors of $-1$ in definition of the product on tensors) with $\alpha_i$ in degree $n$. 
Therefore, for $1 \le \ell \le k$ we find,
\[ q^*(\alpha_k) = \sum_{i_1 < \dots < i_\ell} x_{i_1} \smile \cdots \smile x_{i_\ell} \]
because the unique $n\ell$-cell $e^{n\ell}$ of $J_k(S^n)$ is the gluing of the $nk$-cells of $(S^n)^k$,
\[ \{ e_{i_1}^{n} \times \cdots \times e_{i_\ell}^n \mid i_1 < \cdots < i_{\ell} \} \]
where the other factors are $e^0$. Furthermore,
\[ q^*(\alpha_1^k) = (x_1 + \cdots + x_k)^k = k! \cdot x_1 \smile \cdots \smile x_k = k! \cdot q^*(\alpha_k) \]
The map $(S^n)^k \to J_k(S^n) \embed J(S^n)$ induces an isomorphism $q^* : H^{nk}(J(S^n); \Z) \iso H^{nk}((S^n)^k ; \Z)$ because $H^{nk}(J(S^n); \Z) = \alpha_k \Z$ and $q^*(\alpha_k) = \alpha_1 \smile \cdots \smile \alpha_k$ which generates $H^{nk}((S^n)^k; \Z)$. Therefore,
\[ \alpha_1^k = k! \cdot \alpha_k \] 
\end{proof}

\subsection{Proof of the Main Theorem}

Let $n > 0$ be an even number. We wish to construct a map $f : S^{2n - 1} \to S^n$ with $h(f) = \pm 2$. We consider, explicitly, the space $J_2(S^n) = S^n \times S^n / (x, e) \sim (e, x)$. Consider the cell structure,
\[ S^n = \{ e \} \cup D^n \] 
Then we get a cell decomposition,
\[ J_2(S^n) = \{ e \} \cup D^n \cup D^{2n} = S^n \cup D^{2n} \]
since the product cells $\{ e \} \times D^n$ and $D^n \times \{ e \}$ are glued together. Therefore, the map,
\[ f : S^{2n - 1} = \partial D^{2n} \to J_2(S^n) \to (J_2(S^n))^{2n - 1} = S^n \]
gives a presentation $J_s(S^n) = C_f = S^n \cup_f D^{2n}$. I claim that $h(f) = \pm 2$. Indeed, consider a generator $\alpha_1 \in H^n(C_f; \Z) = H^n(J_2(S^n); \Z)$ and a generator $\alpha_2 \in H^{2n}(C_f; \Z) = H^{2n}(J_2(S^n); \Z)$. Then by Theorem \ref{cup_relation}, we have $\alpha_1 \smile \alpha_1 = \pm 2 \alpha_1$ showing that $h(f) = \pm 2$. 

\section{K-Theory of Projective Space}

\subsection{K-Theory}

Recall that for a (paracompact) space $X$, we define the $K$-theory of $X$, $K(X)$ to be the Grothendieck group of the exact category of complex vector bundles on $X$ with short exact sequences (which automatically split). Then $K(X)$ becomes a ring under the tensor product operation. Then $K$ becomes a contravariant functor from spaces to rings. We make the following definitions of the $K$-groups.

\begin{defn}
For a (connected paracompact) space $X$, define,
\begin{enumerate}
\item $\tilde{K}(X) = \ker{(K(X) \to K(*) = \Z)}$ 
\item $\tilde{K}^{-q}(X) = \tilde{K}(\Sigma^q X)$
\item $K^{-q}(X) = \tilde{K}^{-q}(X \sqcup *)$
\item $K(X, A) = \tilde{K}(X/A)$
\end{enumerate}
\end{defn}
\noindent
Then we have the following important results about $K$-theory.

\begin{prop}
Let $(X, A)$ be a CW pair with $A \xrightarrow{\iota} X \xrightarrow{q} X/A$. Then there is an associated long exact sequence of $K$-theory,
\begin{center}
\begin{tikzcd}
\cdots \arrow[r] & K^{-n}(X, A) \arrow[r, "q^*"] & K^{-n}(X) \arrow[r, "\iota^*"] & K^{-n}(A) \arrow[r] & K^{-n+1}(X, A) \arrow[r, "q^*"] & K^{-n+1}(X) \arrow[r] & \cdots
\end{tikzcd}
\end{center}
\end{prop}

\begin{theorem}[Bott]
There is a periodicity of $K$-theory, $\tilde{K}(X) \iso \tilde{K}(\Sigma^2 X) = \tilde{K}^{-2}(X)$.
\end{theorem}

\begin{rmk}
This periodicity allows us to define $\tilde{K}^q(X) = \tilde{K}^{q - 2k}(X)$ for $2k > q$. Furthermore, by periodicity, only $\tilde{K}^0(X)$ and $\tilde{K}^1(X)$ are important thus motivating the following definition.
\end{rmk}

\begin{defn}
$K^*(X) = K^0(X) \oplus K^1(X)$. Furthermore, we can give $K^*(X)$ a $K^0(X)$-algebra structure. Then $K^*(X) \cong K(X \times S^1)$.
\end{defn}

\begin{prop}
We have the following explicit computations,
\begin{enumerate}
\item $K(S^{2n}) = \Z[H]/(H - 1)^2$ so $\tilde{K}^0(S^{2n}) = \Z$ 
\item $K(S^{2n + 1}) = \Z$ so $\tilde{K}^0(S^{2n + 1}) = 0$.
\end{enumerate}
\end{prop}

\subsection{$G$-Spaces}

\begin{defn}
Let $G$ be a topological group. A $G$-\textit{space} is a topological space along with a continuous action $\rho : G \times X \to X$. A \textit{morphism} of $G$-spaces is a continuous map $f : X \to Y$ which commutes with the $G$-action. We say a vector bundle $\pi : E \to X$ is a $G$-\textit{bundle} if $E$ is a $G$-space with a linear action and $\pi : E \to X$ is a morphism of $G$-spaces.
\end{defn}

\begin{prop}
Suppose that $G \acts X$ freely. Then there is an equivalence of categories between the category of $G$-vector bundles on $X$ and the category of vector bundles on $X / G$. 
\end{prop}

\begin{proof}
We give a sketch. Given a $G$-vector bundle $E \to X$ the projection is $G$-equivariant and thus we get a quotient $E / G \to X / G$ which is a vector bundle since $G$ acts freely so $E/G \to X/G$ is locally isomorphic to $E \to X$. Conversely, given a vector bundle $V \to X/G$ consider the map $\pi : X \to X / G$ and take the vector bundle $\pi^* V \to X$. However, $\pi^* \embed X \times V$ and $X \times V$ has a natural $G$-action via $g \cdot (v, x) = (v, g \cdot x)$ giving an action of $\pi^* V$ compatible with the projection $\pi^* V \to X$. These constructions are inverse.  
\end{proof}

\begin{defn}
Let $G$ be a finite discrete group and $X$ a $G$-space. Let $\Vect_G(X)$ denote the category of $G$-vector bundles on $X$. The set of isomorphism classes forms a commutative monoid under $\oplus$. Then let $K_G(X)$ be the group completion which is a ring under $\otimes$.
\end{defn}

\begin{example}
If $G = 1$ then $K_G(X) = K(X)$. 
\end{example}

\begin{example}
If $X = *$ then $\Vect_G(X)$ is the category of finite dimensional $G$-representations. Then $K_G(X) = R(G)$ which is the Grothendieck group of $G$-representations. 
\end{example}

\subsection{Thom Isomorphism} 

\begin{defn}
Let $E \to X$ be a vector bundle. Then we define the unit sphere bundle $S(E)$ and the unit ball bundle $B(E)$. Then the \textit{Thom space} is $X^E = B(E) / S(E)$. Note that,
\[ K(B(E), S(E)) = \tilde{K}(X^E) \]
Furthermore, the exterior bundle $\Lambda^*(E)$ defines a vector bundle $\lambda_E \in \tilde{K}(X^E)$. 
\end{defn}

\begin{prop}
Let $E$ be a decomposable vector bundle over $X$. Then $\tilde{K}_G^*(X^E)$ is a free $K_G^*(X)$-module with $\lambda_E$ as generator.
\end{prop}

\begin{proof}
Atiyah Proposition 2.7.2.
\end{proof}

\begin{theorem}
Let $X$ be a $G$-space such that $K^1_G(X) = 0$ and $E$ be a decomposable $G$-vector bundle. Let $S(E)$ be the associated sphere bundle then there is an exact sequence,
\begin{center}
\begin{tikzcd}
0 \arrow[r] & K^1_G(S(E)) \arrow[r] & K^0_G(X) \arrow[r, "\varphi"] & K_G^0(X) \arrow[r] & K^0_G(S(E)) \arrow[r] & 0
\end{tikzcd}
\end{center}
where $\varphi$ is multiplication by,
\[ \lambda_E^{-1} = \sum (-1)^i [ \Lambda^i E] \]
\end{theorem}

\begin{proof}
Consider the pair $(B(E), S(E))$ where $B(E)$ is the unit ball bundle. Then there is a long exact sequence in $K$-theory,
\begin{center}
\begin{tikzcd}
\cdots \arrow[r] & K^{-1}(B(E), S(E)) \arrow[r] & K^{-1}_G(B(E)) \arrow[r] \arrow[draw=none]{d}[name=Z, shape=coordinate]{} & K^{-1}_G(S(E)) 
\arrow[dll,
rounded corners, crossing over,
to path={ -- ([xshift=2ex]\tikztostart.east)
|- (Z) [near end]\tikztonodes
-| ([xshift=-2ex]\tikztotarget.west)
-- (\tikztotarget)}]
\\
& K^{0}_G(B(E), S(E)) \arrow[draw=none]{d}[name=ZZ, shape=coordinate]{}  \arrow[r] & K^{0}_G(B(E)) \arrow[r] & K^0_G(S(E)) 
\arrow[dll,
rounded corners, crossing over,
to path={ -- ([xshift=2ex]\tikztostart.east)
|- (ZZ) [near end]\tikztonodes
-| ([xshift=-2ex]\tikztotarget.west)
-- (\tikztotarget)}]
\\
& K^{1}_G(B(E), S(E)) \arrow[r] & K^{1}_G(B(E)) \arrow[r] & \cdots
\end{tikzcd}
\end{center}
but $B(E)$ is homotopy equivalent to $X$. Therefore, we get,
\begin{align*}
K^1_G(B(E)) & = K^1_G(X) = 0
\\
K^0_G(B(E)) & = K^0_G(X)
\end{align*}
which gives an exact sequence (using Bott periodicity),
\begin{center}
\begin{tikzcd}[column sep = small]
0 \arrow[r] & K^{1}_G(S(E)) \arrow[r] & K_G^0(B(E), S(E)) \arrow[r] & K^{0}_G(X) \arrow[r] & K^0_G(S(E)) \arrow[r] & K^{1}_G(B(E), S(E)) \arrow[r] & 0
\end{tikzcd}
\end{center}
However, the Thom space $X^E = B(E) / S(E)$ gives $K^*(B(E), S(E)) = \tilde{K}^*(X^E)$ which we have shown is a graded free $K^*(X)$-module with $\lambda_E$ generating. Therefore, 
\begin{align*}
K^0_G(B(E), S(E)) & = \lambda_E \cdot K^0_G(X)
\\
K^1_G(B(E), S(E)) & = 0 
\end{align*}
so we get the required exact sequence,
\begin{center}
\begin{tikzcd}
0 \arrow[r] & K^1_G(S(E)) \arrow[r] & K^0_G(X) \arrow[r, "\lambda_E^{-1}"] & K_G^0(X) \arrow[r] & K^0_G(S(E)) \arrow[r] & 0
\end{tikzcd}
\end{center}
\end{proof}

\begin{lemma}
Let $X$ be a point then $K^1_G(X) = 0$. 
\end{lemma}

\begin{proof}
Since $K_G^*(X) = K_G(X \times S^1)$ it suffices to show that the map $K_G(S^1) \to K_G(*)$ is an isomorphism where $S^1$ is given a trivial $G$-action. Then.
\[ K_G(S^1) \cong K(S^1) \otimes R(G) \cong K(*) \otimes R(G) \cong K_G(*) \]
where we used $K(S^1) \cong K(*) = \Z$. 
\end{proof}

\begin{cor}
Let $G$ be a cyclic group and $E$ a $G$-module with $S(E)$ having a free $G$-action. Then there is an exact sequence,
\begin{center}
\begin{tikzcd}
0 \arrow[r] & K^1(S(E) / G) \arrow[r] & R(G) \arrow[r] & R(G) \arrow[r] & K^0(S(E)/G) \arrow[r] & 0 
\end{tikzcd}
\end{center}
\end{cor}

\begin{proof}
Note that finite $G$-representations are automatically semi-simple so the $G$-module $E$ is a decomposable bundle over a point. Then the result follows by applying the previous exact sequence to a point using that $K^1_G(X) = 0$. Furthermore, we use that $K^*_G(X) = K^*(X/G)$ when $G$ acts freely on $X$.
\end{proof}

\subsection{Application to the Case of Projective Space}

\begin{rmk}
For $E = \C^n$ we have $S(E) = S^{2n - 1}$. Let $G = \Z/2\Z$ which acts freely on $E$ via $x \mapsto -x$. Then $G$ acts on $S(E)$ freely via $x \mapsto -x$, the antipodal action. Therefore, $S(E)/G = \RP^{2n - 1}$. This will allow us to apply the above sequence. First we need to understand the representation theory of $G$. First, recall that by Maschke's theorem, $G$-representations are semi-simple so need only understand irreducible representations. 
\end{rmk}

\begin{theorem}
Let $G$ be a finite abelian group. Then all irreducible $G$-representations are one-dimensional i.e. are characters. 
\end{theorem}

\begin{proof}
Let $\rho : G \to \Aut{V}$ be an irreducible $G$-representation. Then for any $g, h \in G$ we have,
\[ \rho(g) \circ \rho(h) = \rho(gh) = \rho(hg) = \rho(h) \circ \rho(g) \]
Therefore, $\rho(g) : V \to V$ is a $G$-morphism. Since $V$ is irreducible, by Shur's Lemma, $\rho(g) = \lambda_g \id$ and thus $\rho : G \to \C^\times$ is a character.   
\end{proof}


\begin{example}
Representations of $G = \Z / 2 \Z$ are thus direct sums of characters. The characters $\rho : G \to \C^\times$ are determined by the image of $1$. We must have $\rho(1) = \pm 1$. These options are $1$ the trivial character and $\rho$ the nontrivial character. Furthermore, $\rho \otimes \rho : G \to \C^\times$ is trivial since $(-1)^2 = 1$. Therefore, representations are sums,
\[ n + m \rho : = 1 \oplus \cdots 1 \oplus \rho \oplus \cdots \oplus \rho \]
for $n, m \ge 0$ with the relation $\rho^{\otimes 2} = 1$. Thus, taking the group completion we find,
\[ R(G) = \Z[\rho]/(\rho^2 - 1) \]
Furthermore, the map $\lambda_{-1} : R(G) \to R(G)$ is given by multiplication by,
\[ \lambda_{-1} = \sum (-1)^i \rho^i = (1 - \rho)^n \]
\end{example}

\begin{prop}
We have $\tilde{K}^0(\RP^{2n - 1}) = \Z / 2^{n-1} \Z$ and $K^1(\RP^{2n-1}) = \Z$.
\end{prop}

\begin{proof}
Applying the exact sequence,
\begin{center}
\begin{tikzcd}
0 \arrow[r] & K^1(\RP^{2n-1}) \arrow[r] & \Z[\rho]/(\rho^2 - 1) \arrow[r] & \Z[\rho]/(\rho^2 -  1) \arrow[r] & K^0(\RP^{2n-1}) \arrow[r] & 0 
\end{tikzcd}
\end{center}
We change variables $\rho = \sigma - 1$ then $\sigma^2 = 2 \sigma$ and the map sends $1 \mapsto \sigma^n = 2^{n-1} \sigma$. Then the kernel is given by elements killed by $2^{n-1} \sigma$ which are of the form $(\sigma - 2) \Z$ and thus,
\[ K^1(\RP^{2n - 1}) \cong \Z \]
Finally, the cokernel is,
\[ K^0(\RP^{2n - 1}) = \Z[\sigma]/(\sigma^2 - 2 \sigma, 2^{n-1} \sigma) = \Z \oplus \Z/2^{n-1} \Z \]
\end{proof}

\begin{prop}
We have $K^0(\RP^{2n}) = \Z \oplus \Z / 2^n \Z$ and $K^1(\RP^{2n}) = 0$.
\end{prop}

\begin{proof}
Consider the exact sequences of the pairs $(\RP^{2n}, \RP^{2n - 1})$ and $(\RP^{2n + 1}, \RP^{2n})$. First,
\begin{center}
\begin{tikzcd}
K^1(\RP^{2n}, \RP^{2n - 1}) \arrow[r] & K^1(\RP^{2n}) \arrow[r] & K^1(\RP^{2n})
\end{tikzcd}
\end{center}
but $K^1(\RP^{2n}, \RP^{2n - 1}) = \tilde{K}^1(\RP^{2n} / \RP^{2n - 1}) = \tilde{K}^1(S^{2n}) = 0$ and thus $K^1(\RP^{2n}) \embed K^1(\RP^{2n - 1})$ is injective. Furthermore,
\begin{center}
\begin{tikzcd}
K^1(\RP^{2n+1}) \arrow[r] & K^1(\RP^{2n}) \arrow[r] & K^2(\RP^{2n + 1}, \RP^{2n}) 
\end{tikzcd}
\end{center}
but $K^2(\RP^{2n + 1}, \RP^{2n}) = \tilde{K}^0( \RP^{2n + 1} / \RP^{2n}) = \tilde{K}^0(S^{2n + 1}) = 0$
and thus $K^1(\RP^{2n + 1}) \onto K^1(\RP^{2n})$ is surjective. Furthermore, the composition $K^1(\RP^{2n + 1}) \onto K^1(\RP^{2n}) \embed K^1(\RP^{2n - 1})$ may be computed from the morphism $\RP^{2n - 1} \to \RP^{2n + 1}$ applied to the previous exact sequence in the cases $n$ and $n + 1$ to give a diagram,
\begin{center}
\begin{tikzcd}
0 \arrow[r] & K^1(\RP^{2n+1}) \arrow[d] \arrow[r] & \Z[\rho]/(\rho^2 - 1)  \arrow[d, "\sigma"] \arrow[r, "\sigma^{n+1}"] & \Z[\rho]/(\rho^2 -  1) \arrow[d, "\id"] \arrow[r] & K^0(\RP^{2n+1}) \arrow[d] \arrow[r] & 0 
\\
0 \arrow[r] & K^1(\RP^{2n-1}) \arrow[r] & \Z[\rho]/(\rho^2 - 1) \arrow[r, "\sigma^{n}"] & \Z[\rho]/(\rho^2 -  1) \arrow[r] & K^0(\RP^{2n-1}) \arrow[r] & 0 
\end{tikzcd}
\end{center}
However, $\ker{\sigma^{n+1}} = (\sigma - 2) \Z$ and thus $\sigma \ker{\sigma^{n+1}} = 0$ so the map $K^1(\RP^{2n + 1}) \to K^1(\RP^{2n - 1})$ is zero. Therefore, using the above factorization, $K^1(\RP^{2n}) = 0$. Furthermore, the exact sequence of the pair $(\RP^{2n + 1}, \RP^{2n})$ gives,
\begin{center}
\begin{tikzcd}
K^0(\RP^{2n + 1}, \RP^{2n}) \arrow[r] & K^0(\RP^{2n+1}) \arrow[r] & K^0(\RP^{2n}) \arrow[r] & K^1(\RP^{2n + 1}, \RP^{2n}) \arrow[r] & K^1(\RP^{2n + 1})
\end{tikzcd}
\end{center}
However, $K^0(\RP^{2n + 1}, \RP^{2n}) = \tilde{K}^0(\RP^{2n + 1} / \RP^{2n}) = \tilde{K}^0(S^{2n + 1}) = 0$. Furthermore,
\[ K^1(\RP^{2n + 1}, \RP^{2n}) = \tilde{K}^1(\RP^{2n + 1} / \RP^{2n}) = \tilde{K}^1(S^{2n + 1}) = \Z \]
and we showed that $K^1(\RP^{2n + 1}) = \Z$. Therefore, the sequence becomes,
\begin{center}
\begin{tikzcd}
0 \arrow[r] & K^0(\RP^{2n+1}) \arrow[r] & K^0(\RP^{2n}) \arrow[r] & \Z \arrow[r] & \Z
\end{tikzcd}
\end{center}
However, every map $\Z \to \Z$ is injective so $K^0(\RP^{2n}) \to \Z$ is zero. Thus, $K^0(\RP^{2n + 1}) \iso K^0(\RP^{2n})$ is an isomorphism showing that $K^0(\RP^{2n}) = \Z \oplus \Z / 2^n \Z$.

\end{proof}


\end{document}
