\documentclass[12pt]{extarticle}
\usepackage[utf8]{inputenc}
\usepackage[english]{babel}
\usepackage[a4paper, total={7in, 9.5in}]{geometry}
\usepackage{tikz-cd}

 
\usepackage{amsthm, amssymb, amsmath, centernot, graphicx}
\usepackage{accents}
\DeclareMathAccent{\wtilde}{\mathord}{largesymbols}{"65}
\newcommand{\orb}[1]{\mathrm{Orb}(#1)}
\newcommand{\stab}[1]{\mathrm{Stab}(#1)}
\newcommand{\rp}{\mathbb{RP}}
\newcommand{\cp}{\mathbb{CP}}

\newcommand{\notimplies}{%
  \mathrel{{\ooalign{\hidewidth$\not\phantom{=}$\hidewidth\cr$\implies$}}}}
 
\renewcommand\qedsymbol{$\square$}
\newcommand{\cont}{$\boxtimes$}
\newcommand{\divides}{\mid}
\newcommand{\ndivides}{\centernot \mid}
\newcommand{\Z}{\mathbb{Z}}
\newcommand{\N}{\mathbb{N}}
\newcommand{\C}{\mathbb{C}}
\newcommand{\Zplus}{\mathbb{Z}^{+}}
\newcommand{\Primes}{\mathbb{P}}
\newcommand{\ball}[2]{B_{#1} \! \left(#2 \right)}
\newcommand{\Q}{\mathbb{Q}}
\newcommand{\R}{\mathbb{R}}
\newcommand{\Rplus}{\mathbb{R}^+}
\newcommand{\invI}[2]{#1^{-1} \left( #2 \right)}
\newcommand{\End}[1]{\text{End}\left( A \right)}
\newcommand{\legsym}[2]{\left(\frac{#1}{#2} \right)}
\renewcommand{\mod}[3]{\: #1 \equiv #2 \: \mathrm{mod} \: #3 \:}
\newcommand{\nmod}[3]{\: #1 \centernot \equiv #2 \: mod \: #3 \:}
\newcommand{\ndiv}{\hspace{-4pt}\not \divides \hspace{2pt}}
\newcommand{\finfield}[1]{\mathbb{F}_{#1}}
\newcommand{\finunits}[1]{\mathbb{F}_{#1}^{\times}}
\newcommand{\ord}[1]{\mathrm{ord}\! \left(#1 \right)}
\newcommand{\quadfield}[1]{\Q \small(\sqrt{#1} \small)}
\newcommand{\vspan}[1]{\mathrm{span}\! \left\{#1 \right\}}
\newcommand{\galgroup}[1]{Gal \small(#1 \small)}
\newcommand{\sm}{\! \setminus \!}
\newcommand{\topo}{\mathcal{T}}
\newcommand{\base}{\mathcal{B}}
\renewcommand{\bf}[1]{\mathbf{#1}}
\renewcommand{\Im}[1]{\mathrm{Im} \: #1}
\renewcommand{\empty}{\varnothing}
\newcommand{\id}{\mathrm{id}}
\newcommand{\Hom}[2]{\mathrm{Hom}\left( #1, #2 \right)}
\newcommand{\Tor}[4]{\mathrm{Tor}^{#1}_{#2} \left( #3, #4 \right)}

\renewcommand{\theenumi}{(\alph{enumi})}

\newcommand{\atitle}[1]{\title{% 
	\large \textbf{Mathematics GU4053 Algebraic Topology
	\\ Assignment \# #1} \vspace{-2ex}}
\author{Benjamin Church }
\maketitle}

\newcommand{\hook}{\hookrightarrow}


\theoremstyle{remark}
\newtheorem*{remark}{Remark}

\theoremstyle{definition}
\newtheorem{theorem}{Theorem}[section]
\newtheorem{lemma}[theorem]{Lemma}
\newtheorem{proposition}[theorem]{Proposition}
\newtheorem{corollary}[theorem]{Corollary}
\newtheorem{example}[theorem]{Example}


\newenvironment{definition}[1][Definition:]{\begin{trivlist}
\item[\hskip \labelsep {\bfseries #1}]}{\end{trivlist}}


\begin{document}
\atitle{5}

\section*{Problem 1.}
Let $f_1 : V_1 \to Y$ and $f_2 : V_2 \to Y$ be functions on sets $V_1$ and $V_2$ with $V_1 \cup V_2 = X$. Also, let $f_1 |_{V_1 \cap V_2} = f_2 |_{V_1 \cap V_2}$ so that the function, $f : X \to Y$ given by, 
\[ f(x) =  
\begin{cases}
f_1(x) & x \in V_1 \\
f_2(x) & x \in V_2
\end{cases}
\]
is well defined. The following fact will be of use: 
\[\invI{f}{U} = \invI{f_1}{U} \cup \invI{f_2}{U}\]
This equality hold because:
\[x \in \invI{f}{U} \iff f(x) \in U \iff f_1(x) \in U \text{ or } f_2(x) \in U \iff x \in \invI{f_1}{U} \cup \invI{f_2}{U}\] 
Suppose that $f$ is continuous on any (not necessarily open or closed) sets $V_1$ and $V_2$. Then for any open $U \subset Y$ the set $\invI{f}{U} = \invI{f_1}{U} \cup \invI{f_2}{U}$ is open in $X$. Thus, $\invI{f}{U} \cap V_1$ is open in $V_1$. However, \[\invI{f_2}{U} \cap V_1 \subset \invI{f_1}{U} \cap V_1\] because if $x \in \invI{f_2}{U} \cap V_1$ then $x \in V_1 \cap V_2$ so $f_1(x) = f_2(x) \in U$ so $x \in \invI{f_1}{U}$. Thus, \[\invI{f}{U} \cap V_1 = (\invI{f_1}{U} \cap V_1) \cup (\invI{f_2}{U} \cap V_1) = \invI{f_1}{U} \cap V_1 = \invI{f_1}{U}\] because $\invI{f_1}{U} \subset V_1$ and thus $\invI{f_1}{U}$ is open in $V_1$. Thus, $f_1$ is continuous. The continuity of $f_2$ follows identically. Now, we will prove the converse in the cases that $V_1$ and $V_2$ are both closed or both open. 

\begin{enumerate}
\item Suppose that $V_1$ and $V_2$ are open and that $f_1 : V_1 \to Y$ and $f_2 : V_2 \to Y$ are continuous. For an open $U \subset Y$, consider \[\invI{f}{U} = \invI{f_1}{U} \cup \invI{f_2}{U}\]
However, by continuity, $\invI{f_1}{U}$ is open in $V_1$ and $\invI{f_2}{U}$ is open in $V_2$. Thus, there are sets $S_1, S_2 \subset X$ which are open in $X$ s.t. $\invI{f_1}{U} = S_1 \cap V_1$ and $\invI{f_2}{U} = S_2 \cap V_2$. Therefore, these sets are open in $X$ because $V_1$ and $V_2$ are open. Thus, \[\invI{f}{U} = \invI{f_1}{U} \cup \invI{f_2}{U}\] is open in $X$ so $f$ is continuous.

\item Suppose that $C_1$ and $C_2$ are closed and that $f_1 : V_1 \to Y$ and $f_2 : V_2 \to Y$ are continuous. For a closed $D \subset Y$, consider \[\invI{f}{D} = \invI{f_1}{D} \cup \invI{f_2}{D}\]
However, by continuity, $\invI{f_1}{D}$ is closed in $C_1$ and $\invI{f_2}{D}$ is closed in $C_2$. Thus, there are sets $S_1, S_2 \subset X$ which are closed in $X$ s.t. $\invI{f_1}{D} = S_1 \cap C_1$ and $\invI{f_2}{D} = S_2 \cap C_2$. Therefore, these sets are closed in $X$ because $C_1$ and $C_2$ are closed. Thus, \[\invI{f}{D} = \invI{f_1}{D} \cup \invI{f_2}{D}\] is closed in $X$ so $f$ is continuous.

\end{enumerate}

\section*{Problem 2.}
Let $\{A_n \mid n \in \N \}$ be a sequence of connected subsets of $X$ s.t. for every $n \in \N : A_n \cap A_{n+1} \neq \empty$. Consider the sequence of sets, $C_n = \bigcup\limits_{i = 0}^{n} A_i$. By induction, I will show that these sets are connected. $C_0 = A_0$ which is by hypothesis connected. Suppose that $C_n$ is connected then $C_{n + 1} = C_n \cup A_{n+1}$ and $A_n \subset C_n$ but $A_n \cap A_{n+1} \neq \empty$ so $C_n \cap A_{n+1} \neq \empty$. Thus, $C_{n+1}$ is the union of two intersecting connected sets and is therefore connected. Since $A_n \subset C_n$ and $C_n \subset \bigcup\limits_{i = 0}^{\infty} A_i$ we have, \[\bigcup_{n = 0}^{\infty} A_n = \bigcup_{n = 0}^{\infty} C_n\] and $A_0 \subset C_n$ so $A_0 \subset \bigcap\limits_{n = 0}^{\infty} C_n$ which is therefore not empty because $A_0 \cap A_1 \neq \empty$ so $A_0$ is not empty. Thus, because every $C_n$ is connected and the total intersection in nonempty, the union is also connected.  

\section*{Problem 3.}

I claim that every proper nonempty subset $A \subset X$ has $\partial A \neq \empty$ if and only if $X$ is connected. Note, I claim ``yes'' to both the question and its converse and I use $\partial A = \mathrm{Bd} A$.
\begin{proof}
Suppose that some $A \subset X$ has $\bar{A} \sm A^\circ = \empty$. Thus, $x \in \bar{A} \implies x \in A^\circ$ so $\bar{A} \subset A^\circ$ but $A^\circ \subset A \subset \bar{A}$ so $A^\circ = A = \bar{A}$. Furthermore, $A^\circ$ is open and $\bar{A}$ is closed so $A$ is clopen. Thus, if $X$ is connected then $A$ must be nonproper or empty. Converseley, if $X$ is disconnected then there exists a proper nonempty clopen set $U \subset X$ then $U = U^\circ = \bar{U}$ because both $U$ and $X \sm U$ are closed thus $\partial U = \bar{U} \sm U^\circ = \empty$.  
\end{proof} 

\section*{Problem 4.}

\begin{enumerate}
\item Suppose that $f : [0, 1] \to (0, 1)$ is a homeomorphism. Take $A = [0,1] \sm \{1\} = [0, 1)$  then by bijectivity, $f(A) = (0, 1) \sm \{f(1) \}$. However, $0 < f(1) < 1$ so $f(A)$ is not an interval and thus disconnected. However, $A = [0,1)$ is connected and by assumption $f$ is continuous so $f(A)$ must be connected which is a contradiction.  \\ \\
Suppose that $f : (0, 1] \to (0, 1)$ is a homeomorphism. Take $A = (0,1] \sm \{1\} = (0, 1)$  then by bijectivity, $f(A) = (0, 1) \sm \{f(1) \}$. However, $0 < f(1) < 1$ so $f(A)$ is not an interval and thus disconnected. However, $A = (0, 1)$ is connected and by assumption $f$ is continuous so $f(A)$ must be connected which is a contradiction.  \\ \\
Suppose that $f : [0, 1] \to (0, 1]$ is a homeomorphism. Take $A = [0,1] \sm \{0\} = (0, 1]$  then by bijectivity, $f(A) = (0, 1] \sm \{f(0) \}$. However, $0 < f(0) \le 1$. In the case $f(0) < 1$, we proceed as above, since $f(A)$ is not an interval and thus disconnected. However, $A = [0,1)$ is connected and by assumption $f$ is continuous so $f(A)$ must be connected which is a contradiction. In the case $f(0) = 1$, we have $A = (0, 1]$ and $f(A) = (0, 1)$ which we allready know are not homeomorphic contradicting Lemma \ref{subsethomeo}.

\item Suppose that $f : \R \to \R$ for $n > 1$ is a homeomorphism. Take $A = \R \sm \{0 \}$ then $f(A) = \R^n \sm \{f(0)\}$ but by Lemma \ref{subrnconnected}, $\R^n \sm \{f(0)\}$ is connected. However, $\R \sm \{0\}$ is not an interval so it is diconnected. However, by Lemma \ref{subsethomeo}, $A$ and $f(A)$ are homeomorphic which is a contradiction because connectedness is preserved by homeomorphism. 
\end{enumerate}

\section*{Problem 5.}

Let $S \subset \R^2$ be countable. Now, consider two points $\bf{v}, \bf{u} \in \R^2 \sm S$. Consider the set of lines passing through a given point: \[\mathcal{L}(\bf{w}) = \{L(\bf{w}, \theta) \mid \theta \in [0, \tfrac{\pi}{2}] \} \text{ with } L(\bf{w}, \theta) = \{\bf{r} \in \R^2 \mid (r_x - w_x) \sin{\theta} = (r_y - w_x) \sin{\theta}  \}\]

$L(\bf{w}, \theta)$ contains $(\cos{\theta}, \sin{\theta}) + \bf{w}$. Also, no two distinct lines intersect at more than one point so the number of lines about any point is uncountable since it is in bijection with the points on a half circle surrounding $\bf{w}$. Thus, every point has a line through it which does not intersect $S$. If this were false, we could construct a map $f : \mathcal{L}(\bf{w}) \to S$ given by mapping a line $L$ to the smallest $s \in S$ ($S$ is in bijection to a set of integers and thus can be well-ordered) that intersects $L$. This map would be an injection because two distinct lines through the same point cannot intersect but at that point. However, there cannot exist an injection from a uncountable set to a countable set so there must exist some (uncountably many in fact) lines which do not intersect $S$. Choose $\tilde{L}(\bf{v})$ and $\tilde{L}(\bf{u})$ to be two such lines which intersect eachother at $\bf{r}$ which is always possible because there is exactly one line though $\bf{u}$ which is parallel to $\tilde{L}(\bf{v})$ so take any other of the uncoutably many options for $\tilde{L}(\bf{u})$. Define $\gamma : [0, 1] \to \R^2 \sm S$ by 
\[ \gamma(t) =  
\begin{cases}
\gamma_1(t) = \bf{v} + 2t(\bf{r} - \bf{v}) & x \in [0, \tfrac{1}{2}] \\
\gamma_2(t) = \bf{r} + (2t - 1)(\bf{u} - \bf{r}) & x \in [\tfrac{1}{2}, 1]
\end{cases}
\]
Since $[0, \tfrac{1}{2}]$ and $[\tfrac{1}{2}, 1]$ are closed and intersect only at $\frac{1}{2}$ where $\gamma_1(\frac{1}{2}) = \bf{v} + (\bf{r} - \bf{v}) = \bf{r}$ and $\gamma_2(\frac{1}{2}) = \bf{r}$ so by the glueing lemma, $\gamma$ is continuous since $\gamma_1$ and $\gamma_2$ are continuous. Also, $\gamma(0) = \bf{v}$ and $\gamma(1) = \bf{r} + (\bf{u} - \bf{r}) = \bf{u}$. Finally, $\gamma$ is well defined because $\gamma_1(t) \in \tilde{L}(\bf{v}) \subset \R^2 \sm S$ and $\gamma_2(t) \in \tilde{L}(\bf{u}) \subset \R^2 \sm S$ . Thus, $\gamma(t) \in S$ so $\gamma$ is a path from $\bf{u}$ to $\bf{v}$ proving that $\R^2 \sm S$ is path connected.    

\section*{Problem 6.}

Let $A \subset \R^n$ be connected and open. Take $\bf{x}_0 \in A$ and consider \[U = \{ \bf{x} \in A \mid \exists \text{ path from } \bf{x}_0 \text{ to } \bf{x} \}\] Consider $\bf{z} \in U$, then $\bf{z} \in A$ which is open so $\exists \delta > 0$ s.t. $\bf{z} \in \ball{\delta}{\bf{z}} \subset A$. Also, there exists a continuous map $\gamma : [0, 1] \to A$ s.t. $\gamma(0) = \bf{x}_0$ and $\gamma(1) = \bf{z}$. For any $\bf{x} \in \ball{\delta}{\bf{z}}$, take the function $\gamma_G : [0, 2] \to A$ given by: 
\[\gamma_G(t) = 
\begin{cases}
\gamma(t) & t \in [0,1] \\
r(t) = \bf{z} + (t - 1)(\bf{x} - \bf{z}) & t \in [1,2]
\end{cases}\]
$\gamma_G$ is well defined because $|r(t) - \bf{z}| = (t - 1) |\bf{x} - \bf{z}| < \delta$ so $r(t) \in \ball{\delta}{\bf{z}} \subset A$.
Since $\gamma$ and $r(t)$ are continuous and $[0, 1] \cap [1, 2] = \{1\}$ with $\gamma(1) = \bf{z} = \left[\bf{z} + (t - 1)(\bf{x} - \bf{z}) \right]_{t = 1}$ so by the gluing lemma, $\gamma_G$ is continuous. Because $f : [0, 1] \to [0,2]$ given by $f(x) = 2x$ is a homeomorphism, $\tilde{\gamma} = \gamma_G \circ f : [0, 1] \to A$ is a continuous function with $\tilde{\gamma}(0) = \gamma(0) = \bf{x}_0$ and $\tilde{\gamma}(1) = \gamma_G(f(1)) = \gamma_G(2) = \bf{x}$. Thus, $\tilde{\gamma}$ is a path from $\bf{x}_0$ to $\bf{x}$ so $\bf{x} \in U$. Thus, $\ball{\delta}{\bf{z}} \subset U$ so $U$ is open. \\ \\
 
Likewise, consider $\bf{z} \in A \sm U$, then $\bf{z} \in A$ which is open so $\exists \delta > 0$ s.t. $\bf{z} \in \ball{\delta}{\bf{z}} \subset A$. Suppose that there exists a continuous map $\gamma : [0, 1] \to A$ s.t. $\gamma(0) = \bf{x}_0$ and $\gamma(1) = \bf{x}$ with $\bf{x} \in \ball{\delta}{\bf{z}}$. Then take the function $\gamma_G : [0, 2] \to A$ given by: 
\[\gamma_G(t) = 
\begin{cases}
\gamma(t) & t \in [0,1] \\
r(t) = \bf{x} + (t - 1)(\bf{z} - \bf{x}) & t \in [1,2]
\end{cases}\]
$\gamma_G$ is well defined because $|r(t) - \bf{x}| = (t - 1) |\bf{z} - \bf{x}| < \delta$ so $r(t) \in \ball{\delta}{\bf{z}} \subset A$.
Since $\gamma$ and $\bf{x} + t(\bf{z} - \bf{x})$ are continuous and $[0, 1] \cap [1, 2] = \{1\}$ with $\gamma(1) = \bf{x} = \left[\bf{x} + (t - 1)(\bf{z} - \bf{x}) \right]_{t = 1}$ so by the gluing lemma, $\gamma_G$ is continuous. Because $f : [0, 1] \to [0,2]$ given by $f(x) = 2x$ is a homeomorphism, $\tilde{\gamma} = \gamma_G \circ f : [0, 1] \to A$ is a continuous function with $\tilde{\gamma}(0) = \gamma(0) = \bf{x}_0$ and $\tilde{\gamma}(1) = \gamma_G(f(1)) = \gamma_G(2) = \bf{z}$. Thus, $\tilde{\gamma}$ is a path from $\bf{x}_0$ to $\bf{z}$ so $\bf{z} \in U$ a contradiction. Thus, $\bf{x} \notin U$ so $\ball{\delta}{\bf{z}} \subset A \sm U$ so $A \sm U$ is open. Thus, $U$ is clopen but $\bf{x}_0 \in U$ so because $A$ is connected, $U = A$ and therefore $A$ is path-connected.

\section*{Lemmas}


\begin{lemma} \label{subrnconnected}
For any $\bf{x}_0 \in \R^n$ with $n > 1$, the set $\R^n \setminus \{\bf{x}_0\}$ with the subspace topology is connected. 
\end{lemma}
\begin{proof}
Take $\bf{x}, \bf{y} \in \R^n \setminus \{\bf{x}_0 \}$. Suppose that $\bf{x}_0 - \bf{x} \in \vspan{\bf{y} - \bf{x}}$ then define $\gamma : [0, 1] \to \R^n \sm \{\bf{x}_0\}$ to be the map: \[\gamma(t) = \bf{x} + t (\bf{y} - \bf{x}) + t(1 - t) \bf{b}\] Where $\bf{b}$ is any vector not in the span of $\bf{y} - \bf{x}$. Such a $\bf{b}$ exists because $n > 1$. Now, $\gamma(0) = \bf{x}$ and $\gamma(1) = \bf{x} + (\bf{y} - \bf{x}) = \bf{y}$. Also, $\gamma$ is well defined because if $\gamma(t) = \bf{x}_0$ then, \[\bf{b} = \frac{1}{t(1-t)} (\bf{x}_0 - \bf{x}) - \frac{1}{1 - t} (\bf{y} - \bf{x}) \in \vspan{\bf{y} - \bf{x}}\] which contradicts the definition of $\bf{b}$. The previous formula is well defined because $t \neq 0$ and $t \neq 1$ since $\gamma(0) = \bf{x} \neq \bf{x}_0$ and $\gamma(1) = \bf{y} \neq \bf{x}_0$. Thus, $\Im{\gamma} \subset \R^n \setminus \{\bf{x}_0 \}$  \\ \\ Otherwise, if $\bf{x}_0 - \bf{x} \notin \vspan{\bf{y} - \bf{x}}$ then define $\gamma : [0, 1] \to \R^n \sm \{\bf{x}_0\}$ to be the map: \[\gamma(t) = \bf{x} + t (\bf{y} - \bf{x})\]  Now, $\gamma(0) = \bf{x}$ and $\gamma(1) = \bf{x} + (\bf{y} - \bf{x}) = \bf{y}$. Also, $\gamma$ is well defined because if $\gamma(t) = \bf{x}_0$ then $\bf{x}_0 - \bf{x} = t(\bf{y} - \bf{x})$ conntradicting the fact that $\bf{x}_0 - \bf{x} \notin \vspan{\bf{y} - \bf{x}}$. Thus, $\Im{\gamma} \subset \R^n \setminus \{\bf{x}_0 \}$. These maps $\gamma$ are continuous with respect to the Euclidean metric by $\epsilon-\delta$ arguments. Therefore, $\R^n \setminus \{\bf{x}_0\}$ is path connected and thus connected.  
\end{proof}

\begin{lemma} \label{subsethomeo}
If $(X, \topo_X)$ and $(Y, \topo_Y)$ are homeomorphic topological spaces with homeomorphism $f : X \to Y$ then for any $A \subset X$, $A$ is homeomorphic to $f(A)$ with the subspace topologies.  
\end{lemma}
\begin{proof}
For $A \subset X$ define $g : A \rightarrow f(A)$ by $g : x \mapsto f(x)$ which is trivially a surjection because $\Im{g} = f(A)$. Since $f$ is a bijection, $f$ is injective so $g(x) = g(y) \implies f(x) = f(y) \implies x = y$ so $g$ is a bijection. We must check that $g$ and $g^{-1}$ are continuous. If $U$ is open in $f(A)$ then $\exists V \in \topo_Y$ s.t. $U = V \cap f(A)$ then, \[x \in \invI{g}{U} \iff g(x) \in U \text{ and } x \in A \iff f(x) \in V \cap f(A) \text{ and } x \in A \iff x \in \invI{f}{V} \cap A\]  so $\invI{g}{U} = \invI{f}{V} \cap A$ which is open in $A$ because $f$ is continuous and $V \in \topo_Y$. Also, if $U$ is open in $A$ then $U = A \cap V$ with $V$ open in $X$ and consider $\invI{(g^{-1})}{U}$. \[x \in \invI{(g^{-1})}{U} \iff g^{-1}(x) \in U \iff x \in g(U) = f(U))\] 
Thus, $\invI{(g^{-1})}{U} = f(U) = f(A \cap U) = f(A) \cap f(U)$ which is open in $f(A)$. In the last line I have used $f(A \cap B) = f(A) \cap f(B)$ which follows from injectivity. Thus, $g^{-1}$ is a continuous function.
\end{proof}


\end{document}