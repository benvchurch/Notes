\documentclass[12pt]{extarticle}
\usepackage[utf8]{inputenc}
\usepackage[english]{babel}
\usepackage[utf8]{inputenc}
\usepackage[english]{babel}
\usepackage[a4paper, total={7in, 9.5in}]{geometry}
\usepackage{tikz-cd}

 
\usepackage{amsthm, amssymb, amsmath, centernot, graphicx}
\usepackage{accents}
\DeclareMathAccent{\wtilde}{\mathord}{largesymbols}{"65}
\newcommand{\orb}[1]{\mathrm{Orb}(#1)}
\newcommand{\stab}[1]{\mathrm{Stab}(#1)}
\newcommand{\rp}{\mathbb{RP}}
\newcommand{\cp}{\mathbb{CP}}

\newcommand{\notimplies}{%
  \mathrel{{\ooalign{\hidewidth$\not\phantom{=}$\hidewidth\cr$\implies$}}}}
 
\renewcommand\qedsymbol{$\square$}
\newcommand{\cont}{$\boxtimes$}
\newcommand{\divides}{\mid}
\newcommand{\ndivides}{\centernot \mid}
\newcommand{\Z}{\mathbb{Z}}
\newcommand{\N}{\mathbb{N}}
\newcommand{\C}{\mathbb{C}}
\newcommand{\Zplus}{\mathbb{Z}^{+}}
\newcommand{\Primes}{\mathbb{P}}
\newcommand{\ball}[2]{B_{#1} \! \left(#2 \right)}
\newcommand{\Q}{\mathbb{Q}}
\newcommand{\R}{\mathbb{R}}
\newcommand{\Rplus}{\mathbb{R}^+}
\newcommand{\invI}[2]{#1^{-1} \left( #2 \right)}
\newcommand{\End}[1]{\text{End}\left( A \right)}
\newcommand{\legsym}[2]{\left(\frac{#1}{#2} \right)}
\renewcommand{\mod}[3]{\: #1 \equiv #2 \: \mathrm{mod} \: #3 \:}
\newcommand{\nmod}[3]{\: #1 \centernot \equiv #2 \: mod \: #3 \:}
\newcommand{\ndiv}{\hspace{-4pt}\not \divides \hspace{2pt}}
\newcommand{\finfield}[1]{\mathbb{F}_{#1}}
\newcommand{\finunits}[1]{\mathbb{F}_{#1}^{\times}}
\newcommand{\ord}[1]{\mathrm{ord}\! \left(#1 \right)}
\newcommand{\quadfield}[1]{\Q \small(\sqrt{#1} \small)}
\newcommand{\vspan}[1]{\mathrm{span}\! \left\{#1 \right\}}
\newcommand{\galgroup}[1]{Gal \small(#1 \small)}
\newcommand{\sm}{\! \setminus \!}
\newcommand{\topo}{\mathcal{T}}
\newcommand{\base}{\mathcal{B}}
\renewcommand{\bf}[1]{\mathbf{#1}}
\renewcommand{\Im}[1]{\mathrm{Im} \: #1}
\renewcommand{\empty}{\varnothing}
\newcommand{\id}{\mathrm{id}}
\newcommand{\Hom}[2]{\mathrm{Hom}\left( #1, #2 \right)}
\newcommand{\Tor}[4]{\mathrm{Tor}^{#1}_{#2} \left( #3, #4 \right)}

\renewcommand{\theenumi}{(\alph{enumi})}

\newcommand{\atitle}[1]{\title{% 
	\large \textbf{Mathematics GU4053 Algebraic Topology
	\\ Assignment \# #1} \vspace{-2ex}}
\author{Benjamin Church }
\maketitle}

\newcommand{\hook}{\hookrightarrow}


\theoremstyle{remark}
\newtheorem*{remark}{Remark}

\theoremstyle{definition}
\newtheorem{theorem}{Theorem}[section]
\newtheorem{lemma}[theorem]{Lemma}
\newtheorem{proposition}[theorem]{Proposition}
\newtheorem{corollary}[theorem]{Corollary}
\newtheorem{example}[theorem]{Example}


\newenvironment{definition}[1][Definition:]{\begin{trivlist}
\item[\hskip \labelsep {\bfseries #1}]}{\end{trivlist}}



\begin{document}
\atitle{8}
 
\section*{Problem 1.}
Consider 
\[S = \{ (x, y) \in \R^2 \mid x^2 + y^2 = 1 \} \] and $Y = S \sm \{(0,1)\}$. Now, define the function $f : \R \to Y$ by \[f : t \mapsto \left( \frac{2 t}{t^2 + 1}, \frac{t^2 - 1}{t^2 + 1}  \right)\]
which is well-defined because $\frac{4t^2}{(t^2 + 1)^2} + \frac{(t^2 - 1)^2}{(t^2 + 1)^2} = \frac{(t^2 + 1)^2}{(t^2 + 1)^2} = 1$ and $\frac{t^2 - 1}{t^2 + 1} < 1$. Also define the map $g : Y \to \R$ given by,
\[g : (x, y) \mapsto \frac{x}{1 - y}\]
which is well-defined  because for $y \in Y$, we have $y \neq 1$. I claim these are inverse functions. This can be checked explicitly,
\begin{align*}
f \circ g (x, y) & = \left( \frac{\frac{2x}{1-y}}{\frac{4x^2}{(1-y)^2} + 1}, \frac{\frac{x^2}{(1-y)^2} - 1}{\frac{x^2}{(1-y)^2} + 1}  \right)  \\
& = \left( \frac{2x(1- y)}{x^2 + (1-y)^2}, \frac{x^2 - (1-y)^2}{x^2 + (1-y)^2}  \right)  \\
 & = \left( \frac{2x(1- y)}{x^2 + y^2 + 1 - 2y}, \frac{x^2 - 1 - y^2 + 2y}{x^2 + y^2 + 1 - 2y}  \right) \\
  & = \left( \frac{2x(1- y)}{2(1-y)}, \frac{2y(1-y)}{2(1-y)}  \right) \\
  & = \left(x, y \right)
\end{align*}
in which I have used the fact that $x^2 + y^2 = 1$.
Furthermore, 
\begin{align*}
g \circ f (t) &= \frac{\frac{2t}{t^2 + 1}}{1 - \frac{t^2 - 1}{t^2 +1}}
= \frac{2t}{(t^2 + 1) - (t^2 - 1)} = \frac{2t}{2} = t
\end{align*}
Thus, $f$ and $g$ are inverse functions so both are bijections. Also, because they are rational functions with everywhere nonzero denominators on subsets of $\R^n$, they are continuous. Thus, $f : \R \to Y$ is a homeomorphism. $\R$ is Hausdorff and $S$ is a closed bounded subset of $\R^2$ so it is compact Hausdorff. $S$ is clearly bounded by $1$ and is closed because it is the preimage of the closed set $\{1\}$ under the map $(x,y) \mapsto x^2 + y^2$ which is continuous. Finally, $\R \cong Y = S \sm \{(0,1)\}$ and therefore, $\hat{\R} \cong S$. 

\section*{Problem 2.}

Suppose that $C \subset \Q$ contains $(a, b) \cap \Q$ with $a < b$. This interval must contain an irration number, i.e. $ \exists r \in (a, b) \cap (R \sm \Q)$. Then $a < r < b$ so let $\delta = b - r$. Consider the sequence of intervals 
\[I_n = (r, r + \tfrac{\delta}{n}) \subset (a, b)\] 
where the last inclusion holds because $r + \frac{\delta}{n} < r + \delta = b$. Because $ \tfrac{\delta}{n} > 0$ each interval is nonempty and must contain some rational, $\exists q_n \in I_n \cap \Q \subset (a, b) \cap \Q \subset C$. Take any point $x \neq r$ then take $\epsilon = |r - x| > 0$ so we can choose $N \in \N$ s.t. $N > 2 \frac{\delta}{\epsilon}$. Thus, $\frac{\delta}{N} < \frac{\epsilon}{2}$ so for $n > N$ we have $q_n \in I_n \subset (r, r + \epsilon/2)$ so $|x - q_n| > |x - r| - |r - q_n| > \epsilon - \frac{\epsilon}{2} = \frac{\epsilon}{2}$. Therefore, there are at most $N$ values of $n$ for which $q_n$ is within distance $\frac{\epsilon}{2}$ of $x$. Therefore, no subsequence can converge to $x$ if $x \neq r$. However, $r \notin C \subset \Q$ because $r$ is irrational by construction. Therefore, $\{q_n\}$ is a sequence in $C$ with no subsequence which converges in $C$. Because $C \subset \R$ is a metric space, sequential compactness is equivalent to compactness so $C$ cannot be compact. \\\\
Suppose that $\Q$ were locally compact. Then for any $x \in \Q$ there would exist an open set $U$ and a compact set $C$ such that $x \in U \in C$. However, $U$ is open so $\exists \delta > 0$ such that $x \in \ball{\delta}{x} \subset U \subset C$ and $\ball{\delta}{x} = (x - \delta, x + \delta) \cap \Q$ in the metric space $\Q$. Therefore, $(x - \delta, x + \delta) \subset C$ and $C$ is a compact subset of $\Q$ which is a conradiction. Therefore, $\Q$ is not locally compact. 

\section*{Problem 3.}
\subsection*{(a)}

In this problem, we will use the fact that a continuous $f : \R \to \R$ satisfies the condition,
\[ \lim_{x \to \infty} f(x) = \pm \infty \quad \quad \lim_{x \to -\infty} f(x) = \pm \infty \] 
if and only if $\forall M \in \R : \exists c \in \R : |x| > c \implies |f(x)| > M$. First, suppose that $f : \R \to \R$ is proper. Given $M \in \R$, consider the set $[-M, M] \subset \R$ which is compact because it is closed and bounded. Then, because $f$ is proper, the set $\invI{f}{[-M, M]}$ is compact. In partiuclar, it is bounded by $c$. Thus, if $x \in \invI{f}{[-M, M]}$ then $|x| \le c$. Therefore, if $|x| > c$ then $x \notin \invI{f}{[-M,M]}$ so $f(x) \notin [-M, M]$ and therefore, $|f(x)| > M$ so the function, which is continuous by assumption, satisfies the above limit condition. \bigskip \\
Conversely, let $f$ be a continuous function satisfying the above limit properties. Let $C \subset \R$ be compact. Then by Heine-Borel, $C$ is closed and bounded. Since $C$ is closed and $f$ is continuous then $\invI{f}{C}$ is closed. Take a bound $M$ for $C$. By the limit property, $\exists c \in \R : |x| > c \implies |f(x)| > M$ thus,
\[x \in \invI{f}{C} \implies f(x) \in C \implies |f(x)| \le M \implies |x| \le c\]
Therefore, $\invI{f}{C}$ is closed and bounded so by Heine-Borrel it is compact. Therefore, $f$ is proper.        

\subsection*{(b)}

Let $f(x) = a_n x^n + \cdots + a_1 x + a_0$ be a nonconstant polynomial with $a_n \neq 0$. Then, 
\[ \lim_{x \to \pm \infty} \frac{f(x)}{a_n x^n} = \lim_{x \to \pm \infty} \frac{a_n x^n + \cdots + a_1 x + a_0}{a_n x^n} = \lim_{x \to \pm \infty} \left(1 + \frac{a_{n-1}}{a_n x} + \cdots + \frac{a_0}{a_n x^n} \right) = 1 \]
Therefore, $f(x)$ and $a_n x^n$ have the same asymptotics. In particular, 
\[\lim_{x \to \infty} f(x) = \pm \infty \quad \text{ and } \quad \lim_{x \to - \infty} f(x) = \pm \infty\]

because these conditions hold for $a_n x^n$. From analysis, $f(x)$ is continuous because each term is contiuous. Thus, $f(x)$ is a proper map. 

\section*{Problem 4.}
              
Let $f : X \to Y$ be a proper map and let $X$ and $Y$ be Hausdorff spaces. Define the map $\hat{f} : \hat{X} \to \hat{Y}$ by,

\[\hat{f}(x) =
\begin{cases}
f(x) & x \in X \\
\infty & x = \infty
\end{cases}
\]
Let $C \subset \hat{Y}$ be a closed set. Then, either $\infty \notin C$ and $C$ is compact or $\infty \in C$ and $C \cap Y$ is closed in $Y$. In the first case, $C$ is compact so becaues $f$ is proper and $\infty$ does not map into $C$, $\invI{\hat{f}}{C} = \invI{f}{C}$ is a compact set not containing $\infty$ and thus is closed in $\hat{X}$. In the second case, $C \cap Y$ is closed in $Y$ and $\infty \in C$ so $\invI{\hat{f}}{C} = \invI{f}{C \cap Y} \cup \{\infty\}$. By continuity, $\invI{f}{C \cap Y}$ is closed in $X$ so $\invI{f}{C \cap Y} \cup \{\infty\}$ is closed in $\hat{X}$. \bigskip \\
Conversely, suppose the function $\hat{f} : \hat{X} \to \hat{Y}$ given by,

\[\hat{f}(x) =
\begin{cases}
f(x) & x \in X \\
\infty & x = \infty
\end{cases}
\]  
is continuous. Then, take a closed set $C \subset Y$ and consider the set $D = C \cup \{\infty\} \subset \hat{Y}$. Because $C = D \cap Y$ is closed in $Y$ and $\infty \in D$ then $D$ is closed in $\hat{Y}$. Therefore, by continuity, $\invI{\hat{f}}{D} = \invI{f}{C} \cup \{\infty\}$ is closed in $\hat{X}$. Becuase the inverse image contains $\infty$, $\invI{\hat{f}}{D} \cap X = \invI{f}{C}$ must be closed in $X$. Therefore, $f : X \to Y$ is continuous. Likewise, take a compact set $C \subset Y$ then $C$ is closed in $\hat{Y}$ so because $\infty \notin C$ and by continuity, $\invI{\hat{f}}{C} = \invI{f}{C}$ is closed in $\hat{X}$. However, $\infty \notin \invI{f}{C}$ so the set must be compact in $X$ to be closed in $\hat{X}$. Therefore, $\invI{f}{C}$ is compact so $f$ is a proper map.
           
\section*{Problem 5.}  

Let $f_1 : X_1 \to Y_1$ and $f_2 : X_2 \to Y_2$ continuous with $X_1, X_2$ nonempty and $Y_1, Y_2$ Hausdorff. Suppose that $f_1 \times f_2 : X_1 \times X_2 \to Y_1 \times Y_2$ is proper. Take compact $C_1 \subset Y_1$ and $C_2 \subset Y_2$. Now, $\invI{(f_1 \times f_2)}{C_1 \times C_2} = \invI{f_1}{C_1} \times \invI{f_2}{C_2}$ is compact because $f_1 \times f_2$ is proper. Now, by Lemma \ref{prodcompac}, this implies that $\invI{f_1}{C_1}$ and $\invI{f_2}{C_2}$ are compact and therefore, $f_1$ and $f_2$ are proper. \bigskip \\
Conversely, let $f_1$ and $f_2$ be proper. Let $C \subset Y_1 \times Y_2$ be compact. The maps $\pi_1 : Y_1 \times Y_2 \to Y_1$ and $\pi_2 : Y_1 \times Y_2 \to Y_2$ are continuous so $\pi_1(C)$ and $\pi_2(C)$ are compact. Therefore, because $f_1$ and $f_2$ are proper, $\invI{f_1}{\pi_1(C)}$ and $\invI{f_2}{\pi_2(C)}$ are compact and therefore, $\invI{f_1}{\pi_1(C)} \times \invI{f_2}{\pi_2(C)}$ is compact. Then, because $Y_1$ and $Y_2$ is Hausdorff, $Y_1 \times Y_2$ is Hausdorff so $C$ is closed. Thus, $\invI{(f_1 \times f_2)}{C}$ is closed. \bigskip \\ Now, if $(x, y) \in \invI{(f_1 \times f_2)}{C}$ then $(f_1(x), f_2(y)) \in C$ so $f_1(x) \in \pi_1(C)$ and $f_2(y) \in \pi_2(C)$ so $x \in \invI{f_1}{\pi_1(C)}$ and $y \in \invI{f_2}{\pi_2(C)}$ so finally, $(x, y) \in \invI{f_1}{\pi_1(C)} \times \invI{f_2}{\pi_2(C)}$. Therefore, \[\invI{(f_1 \times f_2)}{C} \subset \invI{f_1}{\pi_1(C)} \times \invI{f_2}{\pi_2(C)}\] However, the former is closed and the latter is compact so $\invI{(f_1 \times f_2)}{C}$ is compact. Thus, $f_1 \times f_2$ is proper. 

\section*{Problem 6.}

Let $f : X \to Y$ and $g : Y \to Z$ be continuous and $g \circ f$, proper and let $Y$ be Hausdorff. Let $C \subset Y$ be compact. By continuity, $g(C)$ is compact and since $g \circ f$ is proper, $\invI{(g \circ f)}{g(C)} = \invI{f}{\invI{g}{g(C)}}$ is compact. However, $C \subset \invI{g}{g(C)}$ and $C$ is compact in a Hausdorff space so $C$ is closed. Thus, $\invI{f}{C}$ is closed and $\invI{f}{C} \subset \invI{f}{\invI{g}{g(C)}}$ which is compact. Therefore, $\invI{f}{C}$ is closed in a compact set and thus compact so $f$ is a proper map.   

\section*{Problem 7.}   

Let $f : X \to Y$ and $g : Y \to Z$ be continuous and $g \circ f$, proper and let $f$ be surjective. Let $C \subset Z$ be compact. Since $g \circ f$ is proper, $\invI{(g \circ f)}{C} = \invI{f}{\invI{g}{C}}$ is compact. Thus, $f(\invI{f}{\invI{g}{C}})$ is compact because $f$ is continuous. However, since $f$ is surjective, by Lemma \ref{surj}, $f(\invI{f}{\invI{g}{C}}) = \invI{g}{C}$ is compact. Therefore, $g$ is proper.         
              
\section*{Lemmas}

\begin{lemma} \label{prodcompac}
If $X$ and $Y$ are nonempty and $X \times Y$ is compact then $X$ and $Y$ are compact. 
\end{lemma}

\begin{proof}
Let $\{U_\lambda \mid \lambda \in \Lambda \}$ be an open cover of $X$. Then, $\{U_\lambda \times Y \mid \lambda \in \Lambda \}$ is an open cover of $X \times Y$ so there exists a finite subcover indexed by $\Lambda'$. Take any $x \in X$ and some $y \in Y$ (which exists because $Y \neq \empty$) then because $\Lambda'$ indexes a finite cover, $\exists \lambda \in \Lambda' : (x, y) \in U_\lambda \times Y$ so $x \in U_\lambda$ thus, $\{U_\lambda \mid \lambda \in \Lambda' \}$ is a finite subcover of $X$ so $X$ is compact. The argument for $Y$ is identical. 
\end{proof}

\begin{lemma} \label{surj}
If $f : X \to Y$ is surjective, then for any $A \subset Y$ we have $f(\invI{f}{A}) = A$.
\end{lemma}

\begin{proof}
If $a \in A$ then by surjectivity, $\exists a \in X$ s.t. $f(x) = a$ so $x \in \invI{f}{A}$ thus $f(a) = x \in f(\invI{f}{A})$ so $A \subset f(\invI{f}{A})$. If $a \in f(\invI{f}{A})$ then $\exists x \in \invI{f}{A}$ s.t. $f(x) = a$ but $f(x) \in A$ so $a \in A$ thus, $f(\invI{f}{A}) \subset A$ so $f(\invI{f}{A}) = A$.   
\end{proof}


\end{document}
