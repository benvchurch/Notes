\documentclass[12pt]{extarticle}
\usepackage[utf8]{inputenc}
\usepackage[english]{babel}
\usepackage[utf8]{inputenc}
\usepackage[english]{babel}
\usepackage[a4paper, total={7in, 9.5in}]{geometry}
\usepackage{tikz-cd}

 
\usepackage{amsthm, amssymb, amsmath, centernot, graphicx}
\usepackage{accents}
\DeclareMathAccent{\wtilde}{\mathord}{largesymbols}{"65}
\newcommand{\orb}[1]{\mathrm{Orb}(#1)}
\newcommand{\stab}[1]{\mathrm{Stab}(#1)}
\newcommand{\rp}{\mathbb{RP}}
\newcommand{\cp}{\mathbb{CP}}

\newcommand{\notimplies}{%
  \mathrel{{\ooalign{\hidewidth$\not\phantom{=}$\hidewidth\cr$\implies$}}}}
 
\renewcommand\qedsymbol{$\square$}
\newcommand{\cont}{$\boxtimes$}
\newcommand{\divides}{\mid}
\newcommand{\ndivides}{\centernot \mid}
\newcommand{\Z}{\mathbb{Z}}
\newcommand{\N}{\mathbb{N}}
\newcommand{\C}{\mathbb{C}}
\newcommand{\Zplus}{\mathbb{Z}^{+}}
\newcommand{\Primes}{\mathbb{P}}
\newcommand{\ball}[2]{B_{#1} \! \left(#2 \right)}
\newcommand{\Q}{\mathbb{Q}}
\newcommand{\R}{\mathbb{R}}
\newcommand{\Rplus}{\mathbb{R}^+}
\newcommand{\invI}[2]{#1^{-1} \left( #2 \right)}
\newcommand{\End}[1]{\text{End}\left( A \right)}
\newcommand{\legsym}[2]{\left(\frac{#1}{#2} \right)}
\renewcommand{\mod}[3]{\: #1 \equiv #2 \: \mathrm{mod} \: #3 \:}
\newcommand{\nmod}[3]{\: #1 \centernot \equiv #2 \: mod \: #3 \:}
\newcommand{\ndiv}{\hspace{-4pt}\not \divides \hspace{2pt}}
\newcommand{\finfield}[1]{\mathbb{F}_{#1}}
\newcommand{\finunits}[1]{\mathbb{F}_{#1}^{\times}}
\newcommand{\ord}[1]{\mathrm{ord}\! \left(#1 \right)}
\newcommand{\quadfield}[1]{\Q \small(\sqrt{#1} \small)}
\newcommand{\vspan}[1]{\mathrm{span}\! \left\{#1 \right\}}
\newcommand{\galgroup}[1]{Gal \small(#1 \small)}
\newcommand{\sm}{\! \setminus \!}
\newcommand{\topo}{\mathcal{T}}
\newcommand{\base}{\mathcal{B}}
\renewcommand{\bf}[1]{\mathbf{#1}}
\renewcommand{\Im}[1]{\mathrm{Im} \: #1}
\renewcommand{\empty}{\varnothing}
\newcommand{\id}{\mathrm{id}}
\newcommand{\Hom}[2]{\mathrm{Hom}\left( #1, #2 \right)}
\newcommand{\Tor}[4]{\mathrm{Tor}^{#1}_{#2} \left( #3, #4 \right)}

\renewcommand{\theenumi}{(\alph{enumi})}

\newcommand{\atitle}[1]{\title{% 
	\large \textbf{Mathematics GU4053 Algebraic Topology
	\\ Assignment \# #1} \vspace{-2ex}}
\author{Benjamin Church }
\maketitle}

\newcommand{\hook}{\hookrightarrow}


\theoremstyle{remark}
\newtheorem*{remark}{Remark}

\theoremstyle{definition}
\newtheorem{theorem}{Theorem}[section]
\newtheorem{lemma}[theorem]{Lemma}
\newtheorem{proposition}[theorem]{Proposition}
\newtheorem{corollary}[theorem]{Corollary}
\newtheorem{example}[theorem]{Example}


\newenvironment{definition}[1][Definition:]{\begin{trivlist}
\item[\hskip \labelsep {\bfseries #1}]}{\end{trivlist}}



\theoremstyle{remark}
\newtheorem*{remark}{Remark}

\begin{document}
\atitle{10}
\begin{remark}
For loops $\gamma_1, \gamma_2 : I \to X$ I will use the (old) notation $\gamma_1 * \gamma_2$ to denote the loop, \[h(t) = \begin{cases} \gamma_2(2t) & t \le \frac{1}{2} \\ \gamma_1(2t - 1) & t \ge \frac{1}{2} \end{cases}\]
\end{remark} 

\section*{Problem 1.}
Let $X$ be a locally euclidean connected space. If $X = \empty$ then $X$ is vacuously path connected because there are no two points to connect.
By Lemma \ref{patheq}, path-connectedness is an equivalence relation denoted by $\sim$. Suppose that $X \neq \empty$ then choose some point $x_0 \in X$ and define the set,
\[U = [x_0] = \{ x \in X \mid x_0 \sim x \}\]
For $x \in X$, by the locally Euclidean property, $\exists$ open $x \in V_x \subset X$ such that $f: V_x \to B_1(0) \subset \R^n$ is a homoemorphism. Take any point $y \in V_x$ then $f(x)$ and $f(y)$ are points in $\ball{1}{0}$ which is convex in $\R^n$ and therefore path-connected. Thus, there exists a path $\gamma : I \to \ball{1}{0}$ from $f(x)$ to $f(y)$. Then, take $\delta = f^{-1} \circ \gamma$ which is continuous because $f$ is continuous with continuous inverse. Then, $\delta(0) = f^{-1}(f(x)) = x$ and $\delta(1) = f^{-1}(f(y)) = y$. Thus, there is a path from $x$ to $y$. \bigskip \\
If $x \in U$ then $x_0 \sim x$ and $\forall y \in V_x : x \sim y$ thus $x_0 \sim y$ so $y \in U$. Therefore, $V_x \subset U$ so $U$ is open because every point has an open neighborhood contained in $U$. \bigskip \\
If $x \in X \setminus U$ then $x_0 \not\sim x$ and $\forall y \in V_x : x \sim y$, however, if $x_0 \sim y$ then by transitivity $x_0 \sim x$ which we assumed was false. Thus, $x_0 \not\sim y$ so $y \notin U$ and so $V_x \subset X \setminus U$. Therefore, $X \setminus U$ is open because every point has an open neighborhood contained in $U$. \bigskip \\
Therefore, $U$ is a clopen set but $x_0 \sim x_0$ so $x_0 \in U$ and thus $U \neq \empty$. However, $X$ is connected so the only nonempty clopen set is $X$. Thus $U = X$. Thus, $\forall x,y \in X : x, y \in U$ so $x \sim x_0$ and $y \sim x_0$ so by transitivity, $x \sim y$. Thus, $X$ is path connected. Becausere path-connected always implies connected, the converse holds as well. 

\section*{Problem 2.}
Let $X$ be path connected and take some $x_0 \in X$. First, suppose that all paths with equal endpoints are path-homotopic. In particular, every loop at $x_0$ is path-homotopic to the trivial loop so the group $\pi_1(X, x_0)$ contains only the identity i.e. $\pi_1(X, x_0) \cong \{e\}$. Conversely, suppose that $\pi_1(X, x_0) \cong \{e\}$. Take $x, y \in X$ and any two paths $\gamma_1, \gamma_2$ between $x$ and $y$. Because $X$ is path connected, there is an isomorphism given by conjugation with a path from $x_0$ to $x$ such that $\pi_1(X, x) \cong \pi_1(X, x_0) \cong \{e\}$. Let $\sim$ denote path-homotopy. Now, $(\gamma_2 * \hat{\gamma}_2) * \gamma_1  \sim \gamma_1$ because $\gamma_2 * \hat{\gamma}_2 \sim e_y$ and $e_y * \gamma_1 \sim \gamma_1$ by reparameterization. Again by reparameterization, $(\gamma_2 * \hat{\gamma}_2) * \gamma_1  \sim \gamma_2 * (\hat{\gamma}_2 * \gamma_1)$ but $\hat{\gamma}_2 * \gamma_1$ is a path from $x$ to $x$ and thus a loop. However, the fundamental group at $x$ is trivial so $\hat{\gamma}_2 * \gamma_1 \sim e_x$. Thus, $\gamma_2 * (\hat{\gamma}_2 * \gamma_1) \sim \gamma_2 * e_x \sim \gamma_2$. Therefore, \[\gamma_1 \sim (\gamma_2 * \hat{\gamma}_2) * \gamma_1  \sim \gamma_2 * (\hat{\gamma}_2 * \gamma_1) \sim \gamma_2\]
Therefore, any two loops with equal endpoints are path-homotopic. 

\section*{Problem 3.}

From problem 4 on assignment 3, $\C \sm \{0\} \cong \R^2 \sm \{(0,0)\} \cong \R \times S$ where \[S = \{z \in \C \mid |z| = 1\}\]
Let the homeomorphism between these spaces be denoted by $e : \C \sm \{0\} \to \R \times S$. \bigskip \\
Now, $f : \C \sm \{0\} \to \C \sm \{0\}$ given by $z \mapsto z^n$ induces a map $\tilde{f} : \R \times S \to \R \times S$ given by $\tilde{f} = e \circ f \circ e^{-1}$ which takes $(x, z) \mapsto (nx, z^n)$. Therefore, $\tilde{f} = m_n \times p$ where $p = f|_S$ and $m_n : x \mapsto nx$. We proved in lecture that $f|_S$ and $m_n$ are covering maps of $S$ and $\R$ respectively. Therefore, by Lemma \ref{prods}, $\tilde{f} = m_n \times f|_S$ is a covering map of $\R \times S$. 

\begin{center}
\begin{tikzcd}[column sep=small, row sep = large]
\C \setminus \{0\} \arrow[d, "f"] \arrow[r, "e"] & \R \times S \arrow[d, "\tilde{f}"]\\
\C \setminus \{0\} \arrow[r, "e"] & \R \times S 
\end{tikzcd}
\end{center}
Any homeomorphism is a $1$-fold covering map so $e$ and $e^{-1}$ are covering maps. Furthermore, $m_n$ is a $1$-fold cover and $p$ is an $n$ fold cover. By problem 5, since both $e$ and $f$ are finite-fold convering maps, $f = e^{-1} \circ f \circ e$ is a covering map.      

\section*{Problem 4.}
Let $p : Y \to X$ be a covering map with connected $X$. Suppose that for some $x_0 \in X$ that $p^{-1}(x_0)$ contains $k$ elements. Define,
\[U = \{x \in X \mid |p^{-1}(x)| = k\}\]
Take $x \in X$. Then $x$ has an evenly covered neighborhood $V_x$ with a homeomorphism $e : p^{-1}(V_x) \to V_x \times \Lambda$ such that the diagram commutes.

\begin{center}
\begin{tikzcd}[column sep=large]
p^{-1}(V_x) \arrow[r, "e"] \arrow[rd, "p"] & V_x \times \Lambda  \arrow[d, "\pi_1"] \\
& V_x
\end{tikzcd}
\end{center}
Now, for any $y \in V_x$ we have $e(p^{-1}(y)) = \pi_1^{-1}(y) = \{y\} \times \Lambda$ but $e$ is a bijection so $|\{y\} \times \Lambda| = |p^{-1}(y)|$. Thus $\forall y \in V_x : |p^{-1}(y)| = |\Lambda|$. In particular, every element of $V_x$ has the same covering number. Therefore, if $x \in U$ then $|p^{-1}(x)| = k$ so for any other $y \in V_x$ we have $|p^{-1}(y)| = |p^{-1}(x)| = k$ so $V_x \subset U$. Therefore, $U$ is open because it contains an open neighborhood of each point. Likewise, if $x \notin U$ then $|f^{-1}(x)| \neq k$ so for any $y \in V_x$ we have $|p^{-1}(y)| = |p^{-1}(x)| \neq k$ and thus $y \notin U$. Therefore, $V_x \subset X \setminus U$ and thus $X \setminus U$ is open because it contains an open neighborhood of each point. Therefore, $U$ is clopen but since $x_0 \in U$ we have that $U \neq \empty$. However, $X$ is connected so the only nonempty clopen set is $X$. Therefore $U = X$ and to $\forall x \in X : p^{-1}(x)$ contains $k$ elements.      

\section*{Problem 5.}

Let $f : Z \to Y$ and $g : Y \to X$ be covering maps with finite $g^{-1}(x)$ for every $x \in X$. Now, take $x \in X$ and let $U_x$ be an evenly covered neighborhood of $x$. Then the preimage $g^{-1}(U_x)$ is a union of disjoint slices, $U_i$, each homeomorphic to $U_x$ under $g$. Let $y_i$ be the preimage of $x$ in $U_i$ i.e. $y_i = g|_{U_i}^{-1}(x)$ which exists because $g|_{U_i} : U_i \to U_x$ is a bijection. There are a finite number of slices because $g^{-1}(x)$ is finite but each slice must map to $x$ because they are homeomorphic to $U_x$ under $g$. Now, each $y_i$ has an evenly covered neighborhood $V_i$ with preimage $f^{-1}(V_i)$ given by the union of disjoint slices, $V_{i}^\lambda$ with $\lambda \in \Lambda_i$. Consider the set,
\[S = \bigcap_{i = 1}^n g(V_i \cap U_i) \subset U_x\] 
Now, $V_i \cap U_i \subset U_i$ and $g$ is a homeomorphism between $U_i$ and $U_x$ so $g(V_i \cap U_i)$ is open in $U_i$ and therefore open in $X$ because $U_x$ is open in $X$. Thus, $S$ is open because it is the finite intersection of open sets. Also, $y_i \in V_i$ and $y_i \in U_i$ so $x \in g(V_i \cap U_i)$ because $g(y_i) = x$. Thus, $x \in S$. Consider $g^{-1}(S) \subset g^{-1}(U_x)$. Let $S_i = g^{-1}(S) \cap U_i = g|_{U_i}^{-1}(S)$ i.e. the component of the preimage in each slice. Now, $S \subset g(V_i \cap U_i)$ so $S_i \subset g|_{U_i}^{-1}(g(V_i \cap U_i)) = V_i \cap U_i$ because $g|_{U_i}$ is a bijection on its image. Therefore, $f^{-1}(S_i) \subset f^{-1}(V_i \cap U_i) \subset f^{-1}(V_i)$. We can decompose this preimage into each slice as $S_{i}^\lambda = f^{-1}(S_i) \cap V_{i}^\lambda = f|_{V_{i}^\lambda}^{-1}(S_i)$. Now, we need to show that $(g \circ f)^{-1}(S)$ is a disjoint union over these sets $S_i^\lambda$ and that when restricted to $S_i^\lambda$ that $g \circ f$ is a homeomorphism to $S$. First, 
\[\bigcup_{\lambda \in \Lambda_i} S_i^\lambda = \bigcup_{\lambda \in \Lambda_i} f^{-1}(S_i) \cap V_{i}^\lambda = f^{-1}(S_i) \cap f^{-1}(V_i) = f^{-1}(S_i \cap V_i) \]
because the preimage of $V_i$ under $f$ is split into slices $V_i^\lambda$. Further,   
\begin{align*}
\bigcup_{i = 1}^n \invI{f}{S_i \cap V_i} &= \invI{f}{\bigcup_{i = 1}^n S_i \cap V_i} = \invI{f}{ \bigcup_{i = 1}^n \invI{g}{S} \cap U_i \cap V_i} \\ & = \invI{f}{\invI{g}{S} \cap \bigcup_{i = 1}^n U_i \cap V_i } = f^{-1}(\invI{g}{S}) = \invI{(g \circ f)}{S}
\end{align*}
where I have used the fact that if $x \in g^{-1}(S)$ then $g(x) \in g(U_i \cap V_i)$ for each $i$ but $x \in g^{-1}(U_x)$ so $x$ is in some slice $U_{j}$ and thus $x \in U_{j} \cap V_{j}$ because on the slice $g$ is injective and therefore $x \in \bigcap\limits_{i = 1}^n (U_i \cap V_i)$ so $g^{-1}(S) \subset \bigcap\limits_{i = 1}^n (U_i \cap V_i)$. The sets $S_i^\lambda$ are clearly disjoint because if $\lambda \neq \lambda'$ then $S_i^\lambda \subset V_i^\lambda$ and $S_i^{\lambda'} \subset V_i^{\lambda'}$ which are disjoint slices. Also, if $i \neq j$ then $S_i$ and $S_j$ are disjoint because they are contained in $U_i$ and $U_j$ respectively which are disjoint slices. Therefore, $S_i^\lambda$ and $S_j^\lambda$ are disjoint because they are contained in the preimages of disjoint sets which are disjoint (since $f^{-1}(A) \cap f^{-1}(B) = f^{-1}(A \cap B) = \empty$). Finally, we need to show that $g \circ f$ is a homeomorphism restricted to $S_i^\lambda$. The map $\tilde{f} = f|_{V_i^\lambda} : V_i^\lambda \to V_i$ is a homeomorphism and thus, its restriction to $S_i^\lambda = \invI{\tilde{f}}{S_i} \subset V_i^\lambda$ is a homeomorphism to the image $S_i$. Also, $S_i = \invI{S}{g|_{U_i}} \subset U_i$ and $\tilde{g} = g|_{U_i} : U_i \to U_x$ is a homeomorphism so its restriction to $S_i$ is also a homeomorphism to the image $S$. Thus, $(g \circ f) |_{S_i^\lambda} = g |_{S_i} \circ f|_{S_i^\lambda}$ is a homeomorphism to $S$. Finally, we have that every point $x$ has an open neighborhood $S$ such that $\invI{(g \circ f)}{S}$ is the union of disjoint slices $S_i^\lambda$ on which $g \circ f$ is a homeomorphism $S$ and thus, $g \circ f$ is a covering map.   
\section*{Lemmas}


\begin{lemma} \label{patheq}
Two points being path-connected is an equivalence relation.
\end{lemma}

\begin{proof}
Take any $c \in X$ then the path $\gamma : I \to X$ given by $\gamma(t) = c$ is continuous because it is constant (Problem 1, Assignment 2) and therefore a path from $c$ to $c$. Thus, $c \sim c$. Next, suppose that $x \sim y$ then there exists a continuous map $\gamma : I \to X$ such that $\gamma(0) = x$ and $\gamma(1) = y$. Consider $\delta(t) = \gamma(1 - t)$ which is continuous because $r: t \mapsto 1 - t$ is continuous and $\delta = \gamma \circ r$. Also, $\delta(0) = \gamma(1) = y$ and $\delta(1) = \gamma(0) = x$. Thus, $y \sim x$. Finally, let $x \sim y$ and $y \sim z$ then there exist paths $\gamma_1, \gamma_2 : I \to X$ with $\gamma_1(0) = x$ and $\gamma_1(1) = y$ and $\gamma_2(0) = y$ and $\gamma_2(1) = z$. Consider the function, $\delta : I \to X$,
\[\delta(t) = 
\begin{cases}
\gamma_1(2t) & t \le \frac{1}{2} \\
\gamma_2(2t - 1) & t \ge \frac{1}{2} 
\end{cases}
\]
At $t = \frac{1}{2}$, $\gamma_1(2t) = \gamma_1(1) = y$ and $\gamma_2(2t - 1) = \gamma_2(0) = y$ so by the glueing lemma, $\delta$ is continuous. Furthermore, $\delta(0) = \gamma_1(0) = x$ and $\delta(1) = \gamma_2(1) = z$. Therefore, $\delta$ is a path from $x$ to $z$ so $x \sim z$. 
    
\end{proof}

\begin{lemma} \label{prods}
Let $p_1 : Y_1 \to X_1$ and $p_2 : Y_2 \to X_2$ be covering maps. Then, $p_1 \times p_2 : Y_1 \times Y_2 \to X_1 \times X_1$ is a covering map.
\end{lemma}

\begin{proof}
Let $p_1 : Y_1 \to X_1$ and $p_2 : Y_2 \to X_2$ be covering maps. Take a point $(x_1, x_2) \in X_1 \times X_2$. Then there are evenly covered open neighborhoods $U_1$ and $U_2$ of $x_1$ and $x_2$ under the maps $p_1$ and $p_2$ respectively. Thus, there exist homeomorphisms and discrete topological spaces such that the following top and bottom triangles in the following diagram commute,
\begin{center}
\begin{tikzcd}[column sep=large]
& & U_1 & \\
& p_1^{-1}(U_1) \arrow[r, "e_1"] \arrow[ru, "p_1"] & U_1 \times \Lambda_1  \arrow[u, "\pi_1^1"] & \\
\invI{p_1}{U_1} \times \invI{p_2}{U_2} \arrow[ru, "\pi_1^p"] \arrow[rd, "\pi_2^p"] \arrow[rr, "e_1 \times e_2", dashed] \arrow[rrr, "p_1 \times p_2", bend right = 70, dashed] & & U_1 \times \Lambda_1 \times U_2 \times \Lambda_2 \arrow[u, "\pi_1^\Lambda"] \arrow[d, "\pi_2^\Lambda"] \arrow[r, "\pi_1^1 \times \pi_1^2", dashed] & U_1 \times U_2 \arrow[luu, "\pi_1^U"] \arrow[ldd, "\pi_2^U"] \\
& p_2^{-1}(U_2) \arrow[r, "e_2"] \arrow[rd, "p_2"] & U_2 \times \Lambda_2 \arrow[d, "\pi_1^2"] &\\
& & U_2 &
\end{tikzcd}
\end{center}
The map $e_1 \times e_2$ is the unique map induced by $e_1 \circ \pi_1^p$ and $e_2 \circ \pi_2^p$. The map $p_1 \times p_2$ is induced by $e_1 \circ \pi_1^p$ and $e_2 \circ \pi_2^p$. The map $p_1 \times p_2$ is induced by $p_1 \circ \pi_1^p$ and $p_2 \circ \pi_2^p$. Finally, the map $\pi_1^1 \times \pi_1^2$ is induced by $\pi_1^1 \circ \pi_1^\Lambda$ and $\pi_1^2 \circ \pi_2^\Lambda$. However, \[\pi_1^U \circ (\pi_1^1 \times \pi_1^2) \circ (e_1 \times e_2) = \pi_1^1 \circ \pi_1^\Lambda \circ (e_1 \times e_2) = \pi_1^1 \circ e_1 \circ \pi_1^p = p_1 \circ \pi_1^p\]
Similarly,
\[\pi_2^U \circ (\pi_1^1 \times \pi_1^2) \circ (e_1 \times e_2) = \pi_1^2 \circ \pi_2^\Lambda \circ (e_1 \times e_2) = \pi_1^2 \circ e_2 \circ \pi_2^p = p_2 \circ \pi_2^p\]
Therefore, $(\pi_1^1 \times \pi_1^2) \circ (e_1 \times e_2)$ satisfies the properties of unique product map defined by $p_1 \circ \pi_1^p$ and $p_2 \circ \pi_2^p$. Thus, $(\pi_1^1 \times \pi_1^2) \circ (e_1 \times e_2) = p_1 \times p_2$ so the entire diagram commutes. Finally, $\invI{p_1}{U_1} \times \invI{p_2}{U_2} = (p_1 \times p_2)^{-1}(U_1 \times U_2)$ because \[(p_1 \times p_2)(x_1, x_2) \in U_1 \times U_2 \iff (p_1(x_1), p_2(x_2)) \in U_1 \times U_2 \iff x_1 \in \invI{p_1}{U_1} \text{ and } x_2 \in \invI{p_2}{U_2}\] and there is a natrual isomorphism between $U_1 \times \Lambda_1 \times U_2 \times \Lambda_2$ and $(U_1 \times U_2) \times (\Lambda_1 \times \Lambda_2)$ and $\Lambda_1 \times \Lambda_2$ is a discrete space because each $\{\lambda_1\} \times \{\lambda_2\} = (\lambda_1, \lambda_2)$ is open. Therefore, the product of the neighborhoods about any point is evenly covered by the product map.
\end{proof}


\section*{Addendum to Problem 3.}
For completeness, I will also exhibit the covering explicitly. Take $U = \C \sm R(e^{i\theta_0})$ where, $R(z) = \{zt \mid t \in \Rplus\}$. Now, 
\[\invI{f}{U} = \C \sm \left( \bigcup_{k = 1}^n R(e^{\frac{i}{n}(2 \pi k + \theta_0)}) \right) = \bigcup_{k = 1}^n \left\{ t e^{i \theta} \mid t \in \Rplus \text{ and } \theta \in \left(2 \pi \tfrac{k}{n} + \tfrac{\theta_0}{n}, 2 \pi \tfrac{k + 1}{n} + \tfrac{\theta_0}{n} \right) \right\} = \bigcup_{k = 1}^n U_k  \]
Because any point $r e^{\frac{i}{n}(2 \pi k + \theta_0)}$ maps to $t^n e^{2 \pi i + i \theta_0} = t^n e^{i \theta_0} \in R(e^{i \theta_0})$. These sets are disjoint by construction. For any $z = t e^{i \theta} \in U$ write $z = t e^{i (\theta - \theta_0)} e^{i \theta}$ where $\theta - \theta_0 \in (0, 2\pi)$. Now let $g_k(z) = t^{1/n} e^{\frac{i}{n}(2 \pi k + \theta)}$ where the form of the domain ensures that $g_k(z) \in U_k$. By analysis $f$ and $g$ are continuous. Now, $f \circ g_k(z) = t e^{2 \pi i k + \theta} = t e^{\theta} = z$ and if $z \in U_k$ then $z = t e^{\frac{i}{n}(2 \pi k + \theta)}$ with $\theta > \theta_0$ so $g_k \circ f(z) = g_k(t^n e^{i \theta}) = t e^{\frac{i}{n}(2 \pi k + \theta)}$ because $\theta - \theta_0 > 0$ so the angle is ihe proper form. Therefore, $f|_{U_k}$ and $g_k$ are inverse functions and therefore $f|_{U_k}$ is a bijection to $U$. Thus, $f$ restricted to $U_k$ is a continuous bijection onto $U$ with continuous inverse so we have shown that $f^{-1}(U)$ is partitioned into slices which are homeomorphic under $f$ to $U$. Since every $z \in \C \sm \{0\}$ is in $\C \sm R(i z)$ because $z$ and $iz$ are not positive real multiples of each other, we have that every point has an evenly covered neighborhood so $f$ is a covering map since $U$ is an open set.  

\end{document}
