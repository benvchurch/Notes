\documentclass[12pt]{extarticle}
\usepackage[utf8]{inputenc}
\usepackage[english]{babel}
\usepackage[utf8]{inputenc}
\usepackage[english]{babel}
\usepackage[a4paper, total={7in, 9.5in}]{geometry}
\usepackage{tikz-cd}

 
\usepackage{amsthm, amssymb, amsmath, centernot, graphicx}
\usepackage{accents}
\DeclareMathAccent{\wtilde}{\mathord}{largesymbols}{"65}
\newcommand{\orb}[1]{\mathrm{Orb}(#1)}
\newcommand{\stab}[1]{\mathrm{Stab}(#1)}
\newcommand{\rp}{\mathbb{RP}}
\newcommand{\cp}{\mathbb{CP}}

\newcommand{\notimplies}{%
  \mathrel{{\ooalign{\hidewidth$\not\phantom{=}$\hidewidth\cr$\implies$}}}}
 
\renewcommand\qedsymbol{$\square$}
\newcommand{\cont}{$\boxtimes$}
\newcommand{\divides}{\mid}
\newcommand{\ndivides}{\centernot \mid}
\newcommand{\Z}{\mathbb{Z}}
\newcommand{\N}{\mathbb{N}}
\newcommand{\C}{\mathbb{C}}
\newcommand{\Zplus}{\mathbb{Z}^{+}}
\newcommand{\Primes}{\mathbb{P}}
\newcommand{\ball}[2]{B_{#1} \! \left(#2 \right)}
\newcommand{\Q}{\mathbb{Q}}
\newcommand{\R}{\mathbb{R}}
\newcommand{\Rplus}{\mathbb{R}^+}
\newcommand{\invI}[2]{#1^{-1} \left( #2 \right)}
\newcommand{\End}[1]{\text{End}\left( A \right)}
\newcommand{\legsym}[2]{\left(\frac{#1}{#2} \right)}
\renewcommand{\mod}[3]{\: #1 \equiv #2 \: \mathrm{mod} \: #3 \:}
\newcommand{\nmod}[3]{\: #1 \centernot \equiv #2 \: mod \: #3 \:}
\newcommand{\ndiv}{\hspace{-4pt}\not \divides \hspace{2pt}}
\newcommand{\finfield}[1]{\mathbb{F}_{#1}}
\newcommand{\finunits}[1]{\mathbb{F}_{#1}^{\times}}
\newcommand{\ord}[1]{\mathrm{ord}\! \left(#1 \right)}
\newcommand{\quadfield}[1]{\Q \small(\sqrt{#1} \small)}
\newcommand{\vspan}[1]{\mathrm{span}\! \left\{#1 \right\}}
\newcommand{\galgroup}[1]{Gal \small(#1 \small)}
\newcommand{\sm}{\! \setminus \!}
\newcommand{\topo}{\mathcal{T}}
\newcommand{\base}{\mathcal{B}}
\renewcommand{\bf}[1]{\mathbf{#1}}
\renewcommand{\Im}[1]{\mathrm{Im} \: #1}
\renewcommand{\empty}{\varnothing}
\newcommand{\id}{\mathrm{id}}
\newcommand{\Hom}[2]{\mathrm{Hom}\left( #1, #2 \right)}
\newcommand{\Tor}[4]{\mathrm{Tor}^{#1}_{#2} \left( #3, #4 \right)}

\renewcommand{\theenumi}{(\alph{enumi})}

\newcommand{\atitle}[1]{\title{% 
	\large \textbf{Mathematics GU4053 Algebraic Topology
	\\ Assignment \# #1} \vspace{-2ex}}
\author{Benjamin Church }
\maketitle}

\newcommand{\hook}{\hookrightarrow}


\theoremstyle{remark}
\newtheorem*{remark}{Remark}

\theoremstyle{definition}
\newtheorem{theorem}{Theorem}[section]
\newtheorem{lemma}[theorem]{Lemma}
\newtheorem{proposition}[theorem]{Proposition}
\newtheorem{corollary}[theorem]{Corollary}
\newtheorem{example}[theorem]{Example}


\newenvironment{definition}[1][Definition:]{\begin{trivlist}
\item[\hskip \labelsep {\bfseries #1}]}{\end{trivlist}}


\begin{document}
\atitle{12}

 
\section*{Problem 1.}

Let $X$ and $Y$ be connected $n$-dimensional CW complexes and $f : X \to Y$ a map which induces isomorphisms $\pi_i(X) \to \pi_i(Y)$ for $i \le n$. Let $p_* : \tilde{X} \to X$ and $q_* : \tilde{Y} \to Y$ be covering maps of the universal covers of $X$ and $Y$. The universal covers can be given an $n$-dimensional cell complex structure.
\begin{center}
\begin{tikzcd}
\tilde{X} \arrow[d, "p"] \arrow[r, dashed, "\tilde{f}"] & \tilde{Y} \arrow[d, "q"]
\\
X \arrow[r, "f"] & Y
\end{tikzcd}
\end{center} 
Since the map $f \circ p : \tilde{X} \to Y$ is a map from a simply-connected space (which is also locally path-connected since its a CW complex) there is a lift to the covering space of $f \circ p$ to a map $\tilde{f} : \tilde{X} \to \tilde{Y}$. Since $p_*$ and $q_*$ and $f_*$ all induce isomorphisms on $\pi_i$ for $1 < i \le n$ we know that $\tilde{f}$ also induces isomorphism on $\pi_i$ for $i \le n$ since $\tilde{f}$ trivially induces isomorphisms on $\pi_0$ and $\pi_1$ because $\tilde{X}$ and $\tilde{Y}$ are simply-connected. Now, consider the long exact homotopy sequence of the pair $(M_{\tilde{f}}, \tilde{X})$,
\begin{center}
\begin{tikzcd}
\pi_i(\tilde{X}) \arrow[r, "\sim"] & \pi_i(M_{\tilde{f}}) \arrow[r] & \pi_i(M_{\tilde{f}}, \tilde{X}) \arrow[r] & \pi_{i-1}(\tilde{X}) \arrow[r, "\sim"] & \pi_{i-1}(M_{\tilde{f}})
\end{tikzcd}
\end{center}
The maps $\tilde{f}_* : \pi_i(\tilde{X}) \to \pi_i(M_{\tilde{f}}) \cong \pi_i(Y)$ are isomorphisms for each $i \le n$ so $\pi_i(M_{\tilde{f}}, X) = 0$ for each $i \le n$. Therefore, the pair $(M_{\tilde{f}}, X)$ is $n$-connected so by Hurewicz's theorem we have isomorphisms, $h_i : \pi_i(M_{\tilde{f}}, X) \to H_i(M_{\tilde{f}}, X)$ for $i \le n+1$. In particular, $H_i(M_{\tilde{f}}, X) = 0$ for $i \le n$. Furthermore, the Hurewicz map is natural so,
\begin{center}
\begin{tikzcd}
\pi_{n+1}(M_{\tilde{f}}, \tilde{X}) \arrow[r] \arrow[d, "h_{n+1}"] & \pi_n(\tilde{X}) \arrow[d, "h_n"] \arrow[r, "\sim"]  & \pi_n(M_{\tilde{f}}) \arrow[d]
\\
H_{n+1}(M_{\tilde{f}}, \tilde{X}) \arrow[r]  & H_n(\tilde{X}) \arrow[r] & H_n(M_{\tilde{f}}) 
\end{tikzcd}
\end{center}
and thus the map $\pi_{n+1}(M_{\tilde{f}}, \tilde{X}) \to \pi_n(\tilde{X})$ is the zero map. Because the Hurewicz maps $h_{n+1}$ and $h_n$ are isomorphisms, the map $H_{n+1}(M_{\tilde{f}}, \tilde{X}) \to H_n(\tilde{X})$ is also the zero map.
Now, applying the long exact sequence of homology to the pair $(M_{\tilde{f}}, \tilde{X})$,
\begin{center}
\begin{tikzcd}
H_{i+1}(M_{\tilde{f}}, \tilde{X}) \arrow[r, "zero"] &
H_i(\tilde{X}) \arrow[r, "\sim"] & H_i(M_{\tilde{f}}) \arrow[r, "zero"] & H_i(M_{\tilde{f}}, \tilde{X})
\end{tikzcd}
\end{center}
we see that the map $\tilde{f}_* : H_i(\tilde{X}) \to H_i(M_{\tilde{f}})$ is an isomorphism for $i \le n$. However, $\tilde{X}$ and $\tilde{Y}$ are $n$-dimensional CW complexes so $H_i(\tilde{X}) = H_i(\tilde{Y}) = 0$ for $i > n$. Thus, $\tilde{f}_* : H_i(\tilde{X}) \to H_i(\tilde{Y})$ is an isomorphism for all $i$. However, $\tilde{X}$ and $\tilde{Y}$ are simply-connected CW complexes so my Whitehead's theorem for homology $\tilde{f}$ is a homotopy equivalence. Consider the induced map $f_* : \pi_i(X) \to \pi_i(Y)$. Since $p_* : \pi_i(\tilde{X}) \to \pi_i(X)$ and $q_* : \pi_i(\tilde{Y}) \to \pi_i(Y)$ are isomorphisms for $i > 1$ and $\tilde{f}_* : \pi_i(\tilde{X}) \to \pi_i(\tilde{Y})$ is an isomorphism because $\tilde{f}$ is a homotopy equivalence by the commutativity of the above diagram, $f_* : \pi_i(X) \to \pi_i(Y)$ is an isomorphism for $i > 1$. However, by assumption, $f_* : \pi_1(X) \to \pi_1(Y)$ is also an isomorphism and both $X$ and $Y$ are connected so $f_* : \pi_0(X) \to \pi_0(Y)$ is trivially an isomorphism. Therefore, the map $f$ induces $f_* : \pi_i(X) \to \pi_i(X)$ isomorphisms for each $i$ and thus by Whitehead's theorem, $f$ is a homotopy equivalence.

\section*{Problem 2.}
Let $X$ be an $(n-1)$-connected CW complex with $n > 1$.
\begin{enumerate}
\item Consider the long exact homotopy sequence associated with the pair $(X, X^{n+1})$,
\begin{center}
\begin{tikzcd}
\pi_{n+1}(X^{n+1}) \arrow[r] & \pi_{n+1}(X) \arrow[r] & \pi_{n+1}(X, X^{n+1}) \arrow[r] & \pi_n(X^{n+1}) \arrow[r] & \pi_n(X) 
\end{tikzcd}
\end{center}
We know that the pair $(X, X^{n+1})$ is $(n+1)$-connected. Thus, $\pi_{n+1}(X, X^{n+1}) = 0$. Therefore, the map $\pi_{n+1}(X^{n+1}) \to \pi_{n+1}(X)$ is surjective since the sequence is exact.
\bigskip\\
Similarly, the long exact sequence of homology assoicated with the pair $(X, X^{n+1})$ gives,
\begin{center}
\begin{tikzcd}
H_{n+1}(X^{n+1}) \arrow[r] & H_{n+1}(X) \arrow[r] & H_{n+1}(X, X^{n+1}) \arrow[r] & H_n(X^{n+1}) \arrow[r] & H_n(X) 
\end{tikzcd}
\end{center} 
but the proof of cellular homology gives us that $H_k(X, X^{n+1}) = 0$ for $k \le n + 1$ and thus $H_{n+1}(X, X^{n+1}) = 0$. By exactness, the map $H_{n+1}(X^{n+1}) \to H_{n+1}(X)$ is surjective.

\item Because the Hurewicz map is natural, we have a morphism of long exact sequences for the pair $(X^{n+1}, X^n)$,
\begin{center}
\begin{tikzcd}
\pi_{n+1}(X^{n}) \arrow[r] \arrow[d, "h_{n+1}"] & \pi_{n+1}(X^{n+1}) \arrow[r] \arrow[d] & \pi_{n+1}(X^{n+1}, X^{n}) \arrow[d, "h_{n+1}'"] \arrow[r] & \pi_{n}(X^{n}) \arrow[d, "h_{n}"]
\\
H_{n+1}(X^{n}) \arrow[r] & H_{n+1}(X^{n+1}) \arrow[r] & H_{n+1}(X^{n+1}, X^{n}) \arrow[r] & H_n(X^{n})
\end{tikzcd}
\end{center}
Homology is zero for cell complexes with strictly lower dimensional cells so $H_{n+1}(X^n) = 0$. Thus, $h_{n+1}$ is surjective. Since $X$ is $(n-1)$-connected Hurewicz's theorem gives that $h_n$ is an isomorphism. Furthermore, $(X^{n+1}, X^{n})$ is $n$-connected so Hurewicz's theorem gives that $h_{n+1}'$ is an isomorphism. Therefore, by the 4-lemma, the map $\pi_{n+1}(X^{n+1}) \to H_{n+1}(X^{n+1})$ is a surjection.

\item Consider the Hurewicz map between the long exact sequences for the pair $(X, X^{n+1})$,
\begin{center}
\begin{tikzcd}
\pi_{n+1}(X^{n+1}) \arrow[r, two heads, "\iota_*"] \arrow[d, two heads, "h_{n+1}'"] & \pi_{n+1}(X) \arrow[r] \arrow[d, "h_{n+1}"] & \pi_{n+1}(X, X^{n+1}) = 0 \arrow[d]
\\
H_{n+1}(X^{n+1}) \arrow[r, two heads, "\iota_*"] & H_{n+1}(X) \arrow[r] & H_{n+1}(X, X^{n+1}) = 0 
\end{tikzcd}
\end{center}
We have shown that the maps $\iota_* : \pi_{n+1}(X^{n+1}) \to \pi_{n+1}(X)$ and $\iota_* : H_{n+1}(X^{n+1}) \to H_{n+1}(X)$ and the Hurewicz map $h_{n+1}' : \pi_{n+1}(X^{n+1}) \to H_{n+1}(X^{n+1})$ are surjective. However, the diagram commutes so,
\[ h_{n+1} \circ \iota_* = \iota_* \circ h_{n+1}' \]
 $h_{n+1}'$ and $\iota_*$ are surjective so $\iota_* \circ h_{n+1}'$ is surjective. Thus, $h_{n+1} \circ \iota_*$ is also surjective which implies that $h_{n+1} : \pi_{n+1}(X) \to H_{n+1}(X)$ must be surjective as well. 

\item We need to show that the Hurewicz map on a path-connected ($0$-connected) CW complex does not necessarily induce as surjection $h_2 : \pi_2(X) \to H_2(X)$. Consider the torus, $T^2 = S^1 \times S^1$. We know that $\pi_2(T^2) \cong \pi_2(S^1) \times \pi_2(S^1) = 0$. However, we have calculated in class that $H_2(T^2) \cong \Z$. Therefore, the map $h_2 : \pi_2(T^2) \to H_2(T^2)$ cannot be surjective. 
\end{enumerate}

\section*{Problem 3.}
Let $C$ and $D$ be chain complexes of abelian groups. Define the chain complex $C \otimes D$ with chains $(C \otimes D)_n = C_n \otimes D_n$ and a boundary map,
\[ \partial_n (x \otimes y) = (\partial_C x) \otimes y + (-1)^n x \otimes ( \partial_D y) \]
Consider the composition of boundary maps,
\begin{align*}
\partial_{n} \circ \partial_{n+1} (x \otimes y) & = \partial_{n} \left( (\partial_C x) \otimes y + (-1)^{n+1} x \otimes ( \partial_D y) \right) = \partial_{n} ((\partial_C x) \otimes y) + (-1)^{n+1} \partial_{n} (x \otimes ( \partial_D y)) 
\\
& = (\partial^2_C x) \otimes y + (-1)^{n} (\partial_C x) \otimes (\partial_D y) + (-1)^{n+1} (\partial_C x) \otimes ( \partial_D y) + (-1)^{n + n + 1} x \otimes ( \partial^2_D y)  
\\
& = (-1)^{n} \left[ (\partial_C x) \otimes (\partial_D y) - (\partial_C x) \otimes ( \partial_D y) \right]
\\
& = 0
\end{align*}
where I used the fact that the boundard maps on $C$ and $D$ statisfy $\partial_C^2 x = \partial_D^2 y = 0$. Since the boundary map $\partial_n$ is a homomorphism and it is zero on each $x \otimes y$ we have shown that $\partial_{n+1} \circ \partial_n = 0$ on $(C \otimes D)_{n-1}$.

\section*{Problem 4.}

Let $F$ be a field and $X$ a space such that $H_i(X ; F)$ has finite dimension for all $i$. Define the Poincare series,
\[ p_X(t) = \sum_{i} \left( \dim_F{H_i(X ; F)} \right) t^i \]
If $X$ and $Y$ are spaces with Poincare series $p_X$ and $p_Y$. Consider the homology,
\[ H_i(X \coprod Y ; F) = H_i(X ; F) \oplus H_i(Y ; F) \]
Therefore,
\[ \dim_{F} {H_i(X \coprod Y ; F)} = \dim_{F} {\left( H_i(X ; F) \oplus H_i(Y ; F) \right)} = \dim_{F}{ H_i(X ; F) } + \dim_{F}{ H_i(Y ; F) } \]
Thus, the Poincare series become,
\begin{align*} 
p_{X \coprod Y} (t) & = \sum_{i} \left( \dim_F{H_i(X \coprod Y ; F)} \right) t^i
\\
& = \sum_{i} \left( \dim_{F}{ H_i(X ; F) } \right) t^i + \sum_{i} \left( \dim_{F}{ H_i(Y ; F) } \right) t^i = p_X(t) + p_Y(t) 
\end{align*}
Similarly, reduced homology commutes with wedge product,
\[ \tilde{H}_i(X \vee Y ; F) = \tilde{H}_i(X ; F) \oplus \tilde{H}_i(Y ; F) \]
Therefore, 
\[ \dim_{F} {\tilde{H}_i(X \vee Y ; F)} = \dim_{F} {\left( \tilde{H}_i(X ; F) \oplus H_i(Y ; F) \right)} = \dim_{F}{ \tilde{H}_i(X ; F) } + \dim_{F}{ \tilde{H}_i(Y ; F) } \]
However,
\[ H_i(X ; F) \cong 
\begin{cases}
\tilde{H}_i(X ; F) & i > 0 
\\
\tilde{H}_i(X ; F) \oplus \Z & i = 0
\end{cases}
\]
which implies that,
\[ \dim_{F} { H_i(X ; F) } \cong 
\begin{cases}
\dim_{F} { \tilde{H}_i(X ; F) } & i > 0 
\\
\dim_{F} { \tilde{H}_i(X ; F) } + 1 & i = 0
\end{cases}
\]
Putting these facts together,
\begin{align*}
\dim_F { H_i(X \vee Y ; F) } & =
\begin{cases}
\dim_{F}{ \tilde{H}_i(X ; F) } + \dim_{F}{ \tilde{H}_i(Y ; F) } & i > 0
\\
\dim_{F}{ \tilde{H}_i(X ; F) } + \dim_{F}{ \tilde{H}_i(Y ; F) } + 1 & i = 0
\end{cases}
\\
& = 
\begin{cases}
\dim_{F}{ \tilde{H}_i(X ; F) } + \dim_{F}{ \tilde{H}_i(Y ; F) } & i > 0
\\
\dim_{F}{ \tilde{H}_i(X ; F) } + \dim_{F}{ \tilde{H}_i(Y ; F) } - 1 & i = 0
\end{cases}
\end{align*}
Thus,
\begin{align*} 
p_{X \vee Y} (t) & = \sum_{i} \left( \dim_F{H_i(X \vee Y ; F)} \right) t^i
\\
& = \sum_{i > 0} \left( \dim_{F}{ H_i(X ; F) } \right) t^i + \sum_{i > 0} \left( \dim_{F}{ H_i(Y ; F) } \right) t^i + \left( H_0(X ; F) + H_0(Y ; F) - 1 \right) t^0 
\\
& = \sum_{i} \left( \dim_{F}{ H_i(X ; F) } \right) t^i + \sum_{i} \left( \dim_{F}{ H_i(Y ; F) } \right) t^i - 1 (t^0)
\\
& = p_X(t) + p_Y(t) - 1 
\end{align*}
\bigskip\\
Now we need to consider the homology of the spaces $S^n$, $\rp^n$, $\rp^\infty$, $\cp^n$, $\cp^\infty$, and $M_g$, the orientable surface of genus $g$. 
From the universal coeficient theorem there is a short exact sequence,
\begin{center}
\begin{tikzcd}
0 \arrow[r] & H_n(X) \otimes_{\Z} F \arrow[r] & H_n(X ; F) \arrow[r] & \Tor{\Z}{1}{H_{n-1}(X)}{F} \arrow[r] & 0
\end{tikzcd}
\end{center}
However, Tor above $0$ of a field vanishes so we get an isomorphism,
\[ H_n(X ; F) \cong H_n(X) \otimes_{\Z} F \] 
However, we have calculated in class the homology with coeficients in $\Z$ for each of these spaces,
\[ H_i(S^n) = 
\begin{cases}
\Z & i = n, 0  
\\
0 & i \neq n, 0 
\end{cases}
\]
Therefore,
\[ H_i(S^n ; F) = 
\begin{cases}
F & i = n, 0  
\\
0 & i \neq n, 0 
\end{cases}
\]
and thus,
\[ p_{S^n}(t) = 1 + t^n \]
Furthermore,
\[ H_i(\rp^n) = 
\begin{cases}
\Z & i = 0
\\
\Z & i = n \text{ and } n \text{ is odd}
\\
\Z/2\Z & i \text{ odd } 0 < i < n
\\
0 & \text{else}
\end{cases}
\]
However, $(\Z / 2 \Z) \otimes_{\Z} F = 0$ so,
\[ H_i(\rp^n ; F) = 
\begin{cases}
F & i = 0
\\
F & i = n \text{ and } n \text{ is odd}
\\
0 & i \text{else}
\end{cases}
\]
and thus,
\[ p_{\rp^n} = 
\begin{cases}
1 + t^n & n \text{ is odd}
\\
1
\end{cases}\]
Furthermore, in the case of $n = \infty$ the homology is the same except with no upper bound,
\[ H_i(\rp^\infty ; F) = 
\begin{cases}
F & i = 0
\\
0 & i > 0
\end{cases}
\]
and thus $p_{\rp^{\infty}} = 1$.
Next, in class we used cellular homology to calculate,
\[ H_i(\cp^n) = 
\begin{cases}
\Z & i \text{ even } 0 \le i \le n
\\
0 & \text{else}
\end{cases}
\]
Since $\Z \otimes_{\Z} F \cong F$ we  get,
\[ p_{\cp^n}(t) = 1 + t^2 + t^4 + \cdots + t^{2n} \]
As before, the homology of the infinite-dimensional complex projective plane is the same except without an upper bound, 
\[ H_i(\cp^\infty ; F) = 
\begin{cases}
F & i \text{ even }
\\
0 & \text{else}
\end{cases}
\]
and thus, 
\[ p_{\cp^\infty}(t) = 1 + t^2 + t^4 + \cdots = \frac{1}{1 - t^2} \]
Finally, the orientable surface of genus $g$ denoted by $M_g$ has homology with coeficients in $\Z$ given by,
\[ H_i(M_g) = 
\begin{cases}
\Z & i = 0,2
\\
\Z^{2g} & i = 1
\\
0 & \text{else}
\end{cases}
\]
Since $\Z^{2g} \otimes_{\Z} F = F^{2g}$ we have,
\[ p_{M_g}(t) = 1 + 2g t + t^2 \]   

\section*{Problem 5.}

Let $F$ be a field. Then, $\mathrm{Tor}^F_1 = 0$ so the K\"{u}nneth formula givse a natural isomorphism,
\[ H_n(X \times Y ; F) \cong \bigoplus_{p + q = n} H_p(X ; F) \otimes_F H_q(Y ; F) \]
We know that the dimension of the tensor product of vectorspaces is the product of dimensions. Thus, 
\[\dim_{F} \left( H_p(X ; F) \otimes_F H_q(Y ; F) \right) = \dim_{F} \left( H_p(X ; F) \right) \cdot \dim_{F} \left( H_q(Y ; F) \right) \]
Using the definition of the Poincere sequence,
\begin{align*}
p_{X \times Y}(t) & = \sum_{i} \left( \dim_F{H_i(X \times Y ; F)} \right) t^i
\\
& = \sum_{i} \sum_{p + q = i} \left( \dim_{F} \left( H_p(X ; F) \otimes_F H_q(Y ; F) \right) \right) t^i
\\
& = \sum_{i} \sum_{p + q = i} \left[ \dim_{F} \left( H_p(X ; F) \right) \cdot \dim_{F} \left( H_q(Y ; F) \right) \right] t^i
\\
& = \left( \sum_{p} \dim_{F} \left( H_p(X ; F) \right) t^p \right) \left( \sum_{q} \dim_{F} \left( H_q(Y ; F) \right) t^q \right)
\\
& = p_X(t) \cdot p_Y(t)
\end{align*}

\end{document}
