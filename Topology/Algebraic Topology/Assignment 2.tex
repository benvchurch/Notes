\documentclass[12pt]{extarticle}
\usepackage[utf8]{inputenc}
\usepackage[english]{babel}
\usepackage[utf8]{inputenc}
\usepackage[english]{babel}
\usepackage[a4paper, total={7in, 9.5in}]{geometry}
\usepackage{tikz-cd}

 
\usepackage{amsthm, amssymb, amsmath, centernot, graphicx}
\usepackage{accents}
\DeclareMathAccent{\wtilde}{\mathord}{largesymbols}{"65}
\newcommand{\orb}[1]{\mathrm{Orb}(#1)}
\newcommand{\stab}[1]{\mathrm{Stab}(#1)}
\newcommand{\rp}{\mathbb{RP}}
\newcommand{\cp}{\mathbb{CP}}

\newcommand{\notimplies}{%
  \mathrel{{\ooalign{\hidewidth$\not\phantom{=}$\hidewidth\cr$\implies$}}}}
 
\renewcommand\qedsymbol{$\square$}
\newcommand{\cont}{$\boxtimes$}
\newcommand{\divides}{\mid}
\newcommand{\ndivides}{\centernot \mid}
\newcommand{\Z}{\mathbb{Z}}
\newcommand{\N}{\mathbb{N}}
\newcommand{\C}{\mathbb{C}}
\newcommand{\Zplus}{\mathbb{Z}^{+}}
\newcommand{\Primes}{\mathbb{P}}
\newcommand{\ball}[2]{B_{#1} \! \left(#2 \right)}
\newcommand{\Q}{\mathbb{Q}}
\newcommand{\R}{\mathbb{R}}
\newcommand{\Rplus}{\mathbb{R}^+}
\newcommand{\invI}[2]{#1^{-1} \left( #2 \right)}
\newcommand{\End}[1]{\text{End}\left( A \right)}
\newcommand{\legsym}[2]{\left(\frac{#1}{#2} \right)}
\renewcommand{\mod}[3]{\: #1 \equiv #2 \: \mathrm{mod} \: #3 \:}
\newcommand{\nmod}[3]{\: #1 \centernot \equiv #2 \: mod \: #3 \:}
\newcommand{\ndiv}{\hspace{-4pt}\not \divides \hspace{2pt}}
\newcommand{\finfield}[1]{\mathbb{F}_{#1}}
\newcommand{\finunits}[1]{\mathbb{F}_{#1}^{\times}}
\newcommand{\ord}[1]{\mathrm{ord}\! \left(#1 \right)}
\newcommand{\quadfield}[1]{\Q \small(\sqrt{#1} \small)}
\newcommand{\vspan}[1]{\mathrm{span}\! \left\{#1 \right\}}
\newcommand{\galgroup}[1]{Gal \small(#1 \small)}
\newcommand{\sm}{\! \setminus \!}
\newcommand{\topo}{\mathcal{T}}
\newcommand{\base}{\mathcal{B}}
\renewcommand{\bf}[1]{\mathbf{#1}}
\renewcommand{\Im}[1]{\mathrm{Im} \: #1}
\renewcommand{\empty}{\varnothing}
\newcommand{\id}{\mathrm{id}}
\newcommand{\Hom}[2]{\mathrm{Hom}\left( #1, #2 \right)}
\newcommand{\Tor}[4]{\mathrm{Tor}^{#1}_{#2} \left( #3, #4 \right)}

\renewcommand{\theenumi}{(\alph{enumi})}

\newcommand{\atitle}[1]{\title{% 
	\large \textbf{Mathematics GU4053 Algebraic Topology
	\\ Assignment \# #1} \vspace{-2ex}}
\author{Benjamin Church }
\maketitle}

\newcommand{\hook}{\hookrightarrow}


\theoremstyle{remark}
\newtheorem*{remark}{Remark}

\theoremstyle{definition}
\newtheorem{theorem}{Theorem}[section]
\newtheorem{lemma}[theorem]{Lemma}
\newtheorem{proposition}[theorem]{Proposition}
\newtheorem{corollary}[theorem]{Corollary}
\newtheorem{example}[theorem]{Example}


\newenvironment{definition}[1][Definition:]{\begin{trivlist}
\item[\hskip \labelsep {\bfseries #1}]}{\end{trivlist}}

\usepackage{subcaption}
\usepackage{float}
\floatplacement{figure}{H}

\begin{document}
\atitle{2}

Note. My order of path concatenation follows Hatcher,
\[\gamma * \delta(x) = \begin{cases}
\gamma(2x) & x \le \tfrac{1}{2} \\
\delta(2x - 1) & \ge \tfrac{1}{2}
\end{cases}\]
 
\section*{Problem 1.}
Consider the map $f : S^1 \times I \to S^1 \times I$ given by $f(\theta, s) = (\theta + 2 \pi s, s)$. Now, $f$ is homotopic to the identity map because the map $H : (S^1 \times I) \times I \to S^1 \times I$ given by $H(\theta, s, t) = (\theta + 2\pi st, s)$ is clearly contunuous and $H(\theta, s, 0) = (\theta, s) = \id_{S^1 \times I}(\theta, s)$ and $H(\theta, s, 1) = (\theta + 2 \pi s, s) = f(\theta, s)$. However, $H$ is not contant on $(\theta, 1)$. \bigskip \\
Now, suppose that $H : (S^1 \times I) \times I \to S^1 \times I$ is a homotopy between $\id_{S^1 \times I}$ and $f$ which is constant on both endpoint circles. Then, using the projecton $\pi_1 : S^1 \times I \to S^1$, consider the map $G : I \times I \to S^1$ given by $G(s, t) = \pi_1 \circ H(0, s, t)$. Now, $G$ is continuous because it is the composition of continuous maps. Futhermore, $G(s, 0) = \pi_1 \circ H(0, s, 0) = \pi_1(0, s) = 0$ which is a constant loop at $0$ in $S^1$. Also, $G(s, 1) = \pi_1 \circ H(0, s, 1) = \pi_1 \circ f(0, s) = 2 \pi s$ which is a loop with winding number $1$. Finally, $G(0, t) = \pi_1 \circ H(0, 0, t) = \pi_1(0, 0) = 0$ and $G(1, t) = \pi_1 \circ H(0, 1, t) = \pi_1(0, 1) = 0$ because $H$ is constant on the boundary circles. Therefore, $G$ is a path-homotopy of the constant loop at $0$ to the generator of the fundamental group of $S^1$. This is a contradiction because $\pi_1(S^1)$ is nontrivial.   

\section*{Problem 2.}
$\pi_1(S^1) \cong \Z$ and any homomorphism from $\Z$ is uniquely determined by the image of the generator $1$. Let $[\gamma]$ generate $\pi_1(S^1, x_0) \cong \Z$ where $\gamma : I \to S^1$ is given by $\gamma(t) = 2 \pi t$. An arbitrary homomorphism $\phi$ can be realized by $[\gamma] \to [\gamma]^n$. Consider the function $f : S^1 \to S^1$ given by $f(\theta) = n \theta$. Then, $f_*([\gamma]) = [f \circ \gamma]$ which is the equivalence class of the path $f \circ \gamma(t) = n \gamma(t) = 2 \pi n t = \gamma^n(t)$ because this path is a reparametrization of $\gamma$ on each interval $t \in [\tfrac{k}{n}, \tfrac{k+1}{n}]$. Therefore, $f_*([\gamma]) = [\gamma]^n$ so the map $f_* = \phi$ because they agree on the generator.

\section*{Problem 3.}   
Let $r : X \to A$ be a retraction and $\iota : A \to X$ the inclusion, then $r \circ \iota = \id_A$ so $(r \circ \iota_A)_* = r_* \circ \iota_* = (\id_A)_* = \id_{\pi_1(A)}$ so $r_*$ is a surjection and $\iota_*$ is an injection. In particular, $\pi_1(A)$ is embedded in $\pi_1(X)$.
\begin{enumerate}
\item Take $X = \R^3$ and $A \subset \R^3$ is homeomorphic to $S^1$. Then $\pi_1(X) = \{e\}$ and $\pi_1(A) \cong \Z$. Thus, $\pi_1(A)$ is not embedded in $\pi_1(X)$ so there cannot be a retraction $\R^3 \to A$. 

\item Let $X = S^1 \times D^2$ and $A = S^1 \times S^1$. Then, $\pi_1(X) \cong \pi_1(S^1) \times \pi_1(D^2) \cong \Z$ and $\pi_1(A) \cong \pi_1(S^1) \times \pi_1(S^1) \cong \Z \times \Z$. However, $\Z \times \Z$ cannot be embedded in $\Z$ because every subgroup of $\Z$ is cyclic but $\Z \times \Z$ is not. Therefore, there cannot exist a retraction $S^1 \times D^2 \to S^1 \times S^1$. 

\item Let $X = S^1 \times D^2$ and $A \cong S^1$ be the loop shown in the figure. Consider a path $\gamma$ in $X$ which traces the loop $A$. If there were a retraction $r : X \to A$ then $r \circ \gamma$ would be the generator of $\pi_1(A) \cong \Z$. However, by twisting the solid torus we can swap the upper an lower parts of the loop $A$ which sends $\gamma$ to the reversed path $\gamma^{-1}$. This twisting fixes the point where the curve loops through itself which we can choose to be the basepoint of $\gamma$. Therefore, $\gamma$ and $\gamma^{-1}$ are path-homotopic in $X$ and therefore path-homotopic in $A$ under the map $r$. This contradicts the fact that no element (asside from 0) is its own inverse in $\Z$ since $\Z$ has no torsion.  

\item Let $X = D^2 \vee D^2$ and $A = S^1 \vee S^1$ the boundary. However, $\pi_1(D^2)$ is trivial so $\pi_1(D^2) * \pi_1(D^2)$ is trivial and then by Van-Kampen, $\pi_1(D^2 \vee D^2)$ is trivial. However, $\pi_1(S^1 \vee S^1) \cong \Z * \Z$ which clearly cannot be embedded into the trivial group consdiering that it is rather infinite. Therefore, there cannot exist a retraction from $D^2 \vee D^2$ to $S^1 \vee S^1$. 

\item Let $X = D^2/ \sim$ with two points identified and let $A = S^1 \vee S^1$ the boundary. Then, $X$ is homotopy equivalent to $S^1$ because $D^2 \simeq I$. Thus, $\pi_1(X) \cong \Z$ but $\pi_1(A) \cong \Z * \Z$. However, $\Z * \Z$ cannot be embedded in $\Z$ because $\Z * \Z$ is nonabelian but $\Z$ is. Therefore, there is no retraction from $X$ to $A$. 

\item Let $X$ be the mobius band and $A \cong \S^1$ its boundary. Let $\gamma$ be a generator of $\pi_1(A)$. Then under $\iota : A \to X$ we have that $i_*([\gamma])$ is a square in $\pi_1(X)$ (see figue at the end). Write $\iota_*([\gamma]) = [\delta]^2$. Then, $r_* \circ \iota_*([\gamma]) = r_*([\delta]^2) = r_*([\delta])^2$. However, $r_* \circ \iota_* = \id_{\pi_1(A)}$ so $[\gamma] = r_*([\delta])^2$ which is false because $[\gamma]$ is a generator of $\pi_1(A) \cong \Z$.   
\end{enumerate}

\section*{Problem 4.}   

Suppose that $X$ is path-connected and locally path-connected and that $\pi_1(X)$ is finite. Then, take a map $f : X \to S^1$. The map $f_* : \pi_1(X, x_0) \to \pi_1(S^1, f(x_0))$ must be the zero map because its image would be a finite subgoup of $\Z$ which must then be trivial. Take the covering map $p : (\R, r_0) \to (S^1, f(x_0))$. Then, $f_*(\pi_1(X, x_0)) \subset p_*(\pi_1(\R, r_0))$ because $f_*(\pi_1(X, x_0)) = 0$. Furthermore, $X$ is path-connected and locally path-conected. Therefore, by the lifting criterion, there exists a unique lift $\tilde{f} : (X, x_0) \to (\R, r_0)$ such that $p \circ \tilde{f} = f$. However, $\R$ is contractable so any map $X \to \R$ is nullhomotopic. In particular, $\tilde{f}$ is homotopic to a constant map $g : X \to \R$. Therefore, $p \circ \tilde{f} \simeq p \circ g$ so $f \simeq p \circ g$ which is a constant map. Thus, $f$ is nullhomotopic. 

\section*{Problem 5.}

Let $X$ be path-connected, locally path-connected, and semi-locally simply-connected. Now, let $G = \pi_1(X, x_0)$ and take the commutator subgroup $G'$. By the Galois correspondence, there exists a covering space $p_a : \tilde{X}_a \to X$ such that $p_{a*}(\pi_1(\tilde{X}_a, \tilde{x}_a) = G'$. However, $G' \triangleleft G$ so $\tilde{X}_a$ is a normal covering space and therefore the group of deck transformations is $D_a \cong G/G' = G^{ab}$ which is an abelian group. Suppose that $p : (\tilde{X}, \tilde{x}_0) \to (X, x_0)$ is any abelian cover. Then, by definition, $p$ is a normal cover so $p_*(\pi_1(\tilde{X}, \tilde{x}_0)) = H \triangleleft G$ and thus the group of deck transformations, $D \cong G/H$. However, $D$ is abelian by definition. However, $G' \subset H \iff G/H$ is abelian. Therefore $G' \subset H$. Because the Galois correspondence is inclusion reversing, $\tilde{X}_a$ covers $\tilde{X}$. Therefore, $\tilde{X}_a$ is universal for abelian covers. The coverse is also true. Any universal abelian cover must correspond via $H = p_*(\pi_1(\tilde{X}, \tilde{x}_0))$ to the commutator subgroup $G'$ because it must cover $\tilde{X}_a$ since $\tilde{X}_a$ is an abelian covering space which implies that $H \subset G'$ but it is also abelian so $D \cong G/H$ is abelian which implies that $G' \subset H$. Thus, $H = G'$ and by the Galois correspondence (two covering spaces are isomorphic iff the pushforwards of their fundamental group agree), $\tilde{X}$ is isomorphic to $\tilde{X}_a$.  \bigskip \\
The universal abelian covers of $S^1 \vee S^1$ and $S^1 \vee S^1 \vee S^1$ must have a group of deck transformations isomorphic to the abelianizations of the fundamental groups,
\[\pi_1(S^1 \vee S^1) \cong \Z * \Z \quad \text{and} \quad \pi_1(S^1 \vee S^1 \vee S^1) \cong \Z * \Z * \Z\]
The abelianization of a free group on $n$ generators is a free abelian group on $n$ generators. Thus, the groups of deck transformations are, $D_2 \cong \Z \times \Z$ and $D_3 \cong \Z \times \Z \times \Z$. These deck transformation groups naturally correspond to integer shifts of a lattice in $\R^2$ or $\R^3$ respectively. These covering spaces can be realied as an infinite grid in $\R^2$ or $\R^3$ where moving along segments corresponds to moving around one circle in $S^1$ preserving orientation of the path. Orthogonal segments correspond to moving around the two or three distict circles. Every point on the lattice at the intersections of these segments maps to the idenified point in the wedge product. Every parallel segment maps to the same circle in the base space so integer translations are exactly the deck transformation of this cover. Also, any loop in the cover must the same number of segments in a positive coordinate direction as segments transvered in the negative coordinate direction. Therefore, it maps to a product of commutators in the base space. Thus, we can identify the fundamental group of this cover with the commutator subgroup of the base. (See figure at the end).      

\section*{Problem 6.}   

\begin{enumerate}
\item Let,

\begin{center}
\begin{tikzcd}[column sep=large]
X \arrow[r, "p"] & Y \arrow[r, "q"]&  X/G
\end{tikzcd}
\end{center}

be a composition of covering spaces. Consider, $D_{X \to Y}$, the deck transformations of the cover $p : X \to Y$. For $g \in D_{X \to Y}$ we know that $p \circ g = p$ so $q \circ p \circ g = q \circ p$ and therefore $g$ is also a deck transformation of the cover $q \circ p : X \to X/G$ so $D_{X \to Y} \subset D_{X \to X/G}$. However, $D_{X \to X/G} \cong G$ so $D_{X \to Y}$ is isomorphic (in an action preserving sense) to some subgoup $H \subset G$. Furthermore, $q \circ p$ is a normal cover i.e. for any $\tilde{x}_1, \tilde{x}_2 \in (q \circ p)^{-1}(x)$ there exists a deck transformation $g$ such the $g(\tilde{x}_1) = \tilde{x}_2$. In particular, let $\tilde{x}_1, \tilde{x}_2 \in p^{-1}(y)$ for smome $y \in Y$. Then, $q \circ p \circ g = q \circ p$ so $p \circ g$ is a lift of $q \circ p$. However, $p$ is a lift of $q \circ p$ and $p(\tilde{x}_1) = y$ and $p \circ g(\tilde{x}_1) = p(\tilde{x}_2) = y$ so the two lifts agree at a point which implies they are equal. Thus, $p \circ g = p$ so $g \in D_{X \to Y}$ which implies that $p$ is a normal cover because $D_{X \to Y}$ acts transitively on fibers of $p$. However, for any normal cover, $Y \cong X/D_{X \to Y} \cong X/H$. 

\item Consider the diagram,

\begin{center}
\begin{tikzcd}[column sep=large, row sep = normal]
X \arrow[r, "p_1"] & X/H_1 \arrow[r, "q_1"] \arrow[d, "\iota", dashed] &  X/G \arrow[d, "\id"] \\
X \arrow[r, "p_2"] & X/H_2 \arrow[r, "q_2"] &  X/G 
\end{tikzcd}
\end{center}

First, suppose that $H_1$ and $H_2$ are conjugate subgroups such that $H_1 = g^{-1} H_2 g$. Then, define $\iota(H_1 \cdot x) = H_2 \cdot (g \cdot x)$. This map is a well defined injection because,
\[H_1 \cdot x_1 = H_1 \cdot x_2 \iff (g^{-1} H_2 g) \cdot x_1 = (g^{-1} H_2 g) \cdot x_2 \iff H_2 \cdot (g \cdot x) = H_2 \cdot (g \cdot x)\]
Clearly, $\iota$ is a surjection because any element of $X/H_2$ can be written as, 
\[H_2 \cdot x = \iota(H_1 \cdot (g^{-1} \cdot x)) \in \iota(X/H_1)\]
Furthermore, $\iota \circ p_1 = p_2 \circ g$ where $g$ is the homeomorphism given by $g \in G$ acting on $X$. $p_2 \circ g$ is continuous which implies that $\iota$ is continuous because $p_1$ is a projection map. Likewise, $\iota^{-1} \circ p_2 = p_1 \circ g^{-1}$ so $\iota^{-1}$ is a continuous. Furthermore, $q_2 \circ \iota(H_1 \cdot x) = q_2(H_2 \cdot (g \circ x)) = G \cdot (g \circ x) = G \cdot x = q_1(H_1 \cdot x)$. Thus, $q_1 = \iota \circ q_2$ so the diagram commutes and $\iota$ is a covering isomorphism. \bigskip \\
Converesely, suppose that some covering isomorphism $\iota$ exists. Fix some point $x_0 \in X$ and take a path $\gamma : I \to X$ taking $x_0$ to an arbitrary point $x \in X$ (arbitrary using that $X$ is path-connected). Then, $\iota \circ p_1 \circ \gamma$ is a path in $X/H_2$. Since $p_2$ is a surjection, there exists $x_1 \in X$ such that $p_2(x_1) = \iota \circ p_1(x_0)$. Thus, $\iota \circ p_1 \circ \gamma$ lifts under $p_2$ to a unique path $\wtilde{\iota \circ p_1 \circ \gamma}$ starting at $x_1$. However, $\iota$ is a covering isomorphism so $q_2 \circ \iota = q_1$ and thus, $q_2 \circ p_2(x_1) = q_2 \circ \iota \circ p_1(x_0) = q_1 \circ p_1(x_0)$ but $q_1 \circ p_1 = q_2 \circ p_2$ which is the projection cover $X \to X/G$. Therefore, $x_0$ and $x_1$ are in the same fiber of this projection i.e. the same orbit under $G$ so $\exists g \in G : g \cdot x_0 = x_1$. Then,
\[ (q_2 \circ p_2) \circ (\wtilde{\iota \circ p_1 \circ \gamma}) = q_2 \circ (\iota \circ p_1 \circ \gamma) = q_2 \circ p_1 \circ \gamma = (q_2 \circ p_1) \circ g \circ \gamma \]
and $g \circ \gamma$ is also a path at $x_1$ so by unique lifitng under the covering map $q_2 \circ p_1$ we have $\wtilde{\iota \circ p_1 \circ \gamma} = g \circ \gamma$. Therefore, $(\wtilde{\iota \circ p_1 \circ \gamma})(1) = g \cdot x$ but $p_2 \circ (\wtilde{\iota \circ p_1 \circ \gamma}) = \iota \circ p_1 \circ \gamma$ so $\iota \circ p_1 \circ \gamma(1) = \iota \circ p_1(x) = p_2(g \cdot x) = q_2 \circ g(x)$. Since $x$ is arbitrary, $\iota \circ p_1 = p_2 \circ g$. Therefore, $\iota(H_1 \cdot x) = \iota \circ p_1 (x) = p_2(g \cdot x) = H_2 \cdot (g \cdot x)$. Since $\iota$ is a homeomorphism, 
\[H_1 \cdot x = H_1 \cdot y \iff H_2 \cdot (g \cdot x) = H_2 \cdot (g \cdot y) \iff (g^{-1} H_2 g) \cdot x = (g^{-1} H_2 g) \cdot y\]
so $y \in H_1 \cdot x \iff y \in (g^{-1} H_2 g) \cdot x$ but the action of $G$ on $X$ is free so $H_1 = g^{-1} H_2 g$.

\item I haven't figued out how to prove that $q : X/H \to X/G$ is normal iff $H \triangleleft G$. Howwever, I can prove that if this is the case that the group of deck transformations $D_{X/H \to X/G} \cong G/H$. Consider the map $\Phi : G \to D_{X/H \to X/G}$ given by $\Phi(g) = h_g$ where $h_g : X/H \to X/H$ is the map $h_g : H \cdot x \mapsto H \cdot (g \cdot x)$. This is a deck transformation because,
\[q \circ h_g(H \cdot x) = q(H \cdot (g \cdot x)) = G \cdot (g \cdot x) = G \cdot x = q(H \cdot x)\]
$\Phi$ is a homomorphism because, 
\[\Phi(g_1 g_2) = H \cdot (g_1 g_2 \cdot x) = h_{g_1}(H \cdot (g_2 \cdot x)) = h_{g_1} \circ h_{g_2} (x)\]
Now, consider $\ker{\Phi} = H$ because $H \cdot (g \cdot x) = (H g) \cdot x$ and $(Hg) \cdot x = H \cdot x$ iff $H = Hg$ i.e. $g \in H$ because the action of $G$ on $X$ is free. Finally, using an identical argument, as before with $\iota$ replaced with a deck transformation $f : X/H \to X/H$, we have that $f \circ p = p \circ g$ for some $g \in G$ which implies that $f \circ p(x) = f(H \cdot x) = p(g \cdot x) = H \cdot (g \cdot x) = h_g(H \cdot x)$ so $f = h_g$. Thus, $\Phi$ is surjective. Therefore, $D_{X/H \to X/G} \cong G/\ker{\Phi} = G/H$. \bigskip \\ 
I recreate that argument here,
\begin{center}
\begin{tikzcd}[column sep=large, row sep = normal]
X \arrow[r, "p"] & X/H \arrow[r, "q"] &  X/G 
\end{tikzcd}
\end{center}
Fix some point $x_0 \in X$ and take a path $\gamma : I \to X$ taking $x_0$ to an arbitrary point $x \in X$ (arbitrary using that $X$ is path-connected). Then, $f \circ p \circ \gamma$ is a path in $X/H$. Since $p$ is a surjection, there exists $x_1 \in X$ such that $p(x_1) = f \circ p(x_0)$. Thus, $f \circ p \circ \gamma$ lifts under $p$ to a unique path $\wtilde{f \circ p \circ \gamma}$ starting at $x_1$. However, $f$ is a deck transformation so $q \circ f = q$ and thus, $q \circ p(x_1) = q \circ f \circ p(x_0) = q \circ p(x_0)$ the projection cover $X \to X/G$. Therefore, $x_0$ and $x_1$ are in the same fiber of this projection i.e. the same orbit under $G$ so $\exists g \in G : g \cdot x_0 = x_1$. Then,
\[ (q \circ p) \circ (\wtilde{f \circ p \circ \gamma}) = q \circ (f \circ p \circ \gamma) = q \circ p \circ \gamma = (q \circ p) \circ g \circ \gamma \]
and $g \circ \gamma$ is also a path at $x_1$ so by unique lifitng under the covering map $q \circ p$ we have $\wtilde{f \circ p \circ \gamma} = g \circ \gamma$. Therefore, $(\wtilde{f \circ p \circ \gamma})(1) = g \cdot x$ but $p \circ (\wtilde{f \circ p \circ \gamma}) = f \circ p \circ \gamma$ so $f \circ p \circ \gamma(1) = f \circ p(x) = p(g \cdot x) = q \circ g(x)$. Since $x$ is arbitrary, $f \circ p = p \circ g$.



\end{enumerate}   

\end{document}
