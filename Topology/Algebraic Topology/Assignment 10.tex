\documentclass[12pt]{extarticle}
\usepackage[utf8]{inputenc}
\usepackage[english]{babel}
\usepackage[utf8]{inputenc}
\usepackage[english]{babel}
\usepackage[a4paper, total={7in, 9.5in}]{geometry}
\usepackage{tikz-cd}

 
\usepackage{amsthm, amssymb, amsmath, centernot, graphicx}
\usepackage{accents}
\DeclareMathAccent{\wtilde}{\mathord}{largesymbols}{"65}
\newcommand{\orb}[1]{\mathrm{Orb}(#1)}
\newcommand{\stab}[1]{\mathrm{Stab}(#1)}
\newcommand{\rp}{\mathbb{RP}}
\newcommand{\cp}{\mathbb{CP}}

\newcommand{\notimplies}{%
  \mathrel{{\ooalign{\hidewidth$\not\phantom{=}$\hidewidth\cr$\implies$}}}}
 
\renewcommand\qedsymbol{$\square$}
\newcommand{\cont}{$\boxtimes$}
\newcommand{\divides}{\mid}
\newcommand{\ndivides}{\centernot \mid}
\newcommand{\Z}{\mathbb{Z}}
\newcommand{\N}{\mathbb{N}}
\newcommand{\C}{\mathbb{C}}
\newcommand{\Zplus}{\mathbb{Z}^{+}}
\newcommand{\Primes}{\mathbb{P}}
\newcommand{\ball}[2]{B_{#1} \! \left(#2 \right)}
\newcommand{\Q}{\mathbb{Q}}
\newcommand{\R}{\mathbb{R}}
\newcommand{\Rplus}{\mathbb{R}^+}
\newcommand{\invI}[2]{#1^{-1} \left( #2 \right)}
\newcommand{\End}[1]{\text{End}\left( A \right)}
\newcommand{\legsym}[2]{\left(\frac{#1}{#2} \right)}
\renewcommand{\mod}[3]{\: #1 \equiv #2 \: \mathrm{mod} \: #3 \:}
\newcommand{\nmod}[3]{\: #1 \centernot \equiv #2 \: mod \: #3 \:}
\newcommand{\ndiv}{\hspace{-4pt}\not \divides \hspace{2pt}}
\newcommand{\finfield}[1]{\mathbb{F}_{#1}}
\newcommand{\finunits}[1]{\mathbb{F}_{#1}^{\times}}
\newcommand{\ord}[1]{\mathrm{ord}\! \left(#1 \right)}
\newcommand{\quadfield}[1]{\Q \small(\sqrt{#1} \small)}
\newcommand{\vspan}[1]{\mathrm{span}\! \left\{#1 \right\}}
\newcommand{\galgroup}[1]{Gal \small(#1 \small)}
\newcommand{\sm}{\! \setminus \!}
\newcommand{\topo}{\mathcal{T}}
\newcommand{\base}{\mathcal{B}}
\renewcommand{\bf}[1]{\mathbf{#1}}
\renewcommand{\Im}[1]{\mathrm{Im} \: #1}
\renewcommand{\empty}{\varnothing}
\newcommand{\id}{\mathrm{id}}
\newcommand{\Hom}[2]{\mathrm{Hom}\left( #1, #2 \right)}
\newcommand{\Tor}[4]{\mathrm{Tor}^{#1}_{#2} \left( #3, #4 \right)}

\renewcommand{\theenumi}{(\alph{enumi})}

\newcommand{\atitle}[1]{\title{% 
	\large \textbf{Mathematics GU4053 Algebraic Topology
	\\ Assignment \# #1} \vspace{-2ex}}
\author{Benjamin Church }
\maketitle}

\newcommand{\hook}{\hookrightarrow}


\theoremstyle{remark}
\newtheorem*{remark}{Remark}

\theoremstyle{definition}
\newtheorem{theorem}{Theorem}[section]
\newtheorem{lemma}[theorem]{Lemma}
\newtheorem{proposition}[theorem]{Proposition}
\newtheorem{corollary}[theorem]{Corollary}
\newtheorem{example}[theorem]{Example}


\newenvironment{definition}[1][Definition:]{\begin{trivlist}
\item[\hskip \labelsep {\bfseries #1}]}{\end{trivlist}}



\begin{document}
\atitle{10}
 
\section*{Problem 1.}

We have calculated,
\[ H_n(S^1 \times S^1) = H_n(T^2) \cong
\begin{cases}
\Z & n = 2 \\
\Z \oplus \Z & n = 1\\
\Z & n = 0 \\
0 & n > 2
\end{cases} \]
Similarly, for $n > 0$,
\[ H_n(S^1 \vee S^1 \vee S^2) = \tilde{H}_n(S^1 \vee S^1 \vee S^2) = \tilde{H}_n(S^1) \oplus \tilde{H}_n(S^1) \oplus \tilde{H}_n(S^2) \]
and since $S^1 \vee S^1 \vee S^2$ is connected, $H_0(S^1 \vee S^1 \vee S^2) \cong \Z$. Therefore,
since $H_n(S^n) \cong \Z$ and is zero otherwise (for $n > 0$) we have,
\[ H_n(S^1 \vee S^1 \vee S^2) \cong
\begin{cases}
\Z & n = 2 \\
\Z \oplus \Z & n = 1\\
\Z & n = 0 \\
0 & n > 2
\end{cases} \]
Therefore, by direct calculation,
\[ H_n(S^1 \vee S^1 \vee S^2) \cong H_n(S^1 \times S^1) \]
\bigskip\\
Furthermore, the universal cover of $S^1 \times S^1$ is $\R^2$ which is contractible. Let $\tilde{X}$ be the universal cover of $S^1 \vee S^1 \vee S^2$. We know that the covering map $p : \tilde{X} \to S^1 \vee S^1 \vee S^2$ induces an isomorphism on higher homotopy groups $p_* : \pi_2(\tilde{X}) \xrightarrow{\sim} \pi_2(S^1 \vee S^1 \vee S^2)$. However, $\pi_2(S^1 \vee S^1 \vee S^2) \neq 0$ so $\tilde{X}$ cannot be contractible because not all its homotopy groups are zero. However, since $S^1 \vee S^1 \vee S^2$ admits a CW complex structure, so does $\tilde{X}$ and thus $\tilde{X}$ is contractible iff $H_n(\tilde{X})) = 0$ for all $n$ which implies that $\tilde{X}$ does not have trivial homology since it is not contractible. 

\section*{Problem 2.}

Take any continuous map $f : S^{2n} \to S^{2n}$. The degree of $f$ cannot equal both $1$ and $(-1)^{2n+1} = -1$. Therefore, by Lemma \ref{antifixed} and Lemma \ref{fixed}, there must either be a point $x \in S^{2n}$ such that $f(x) = x$ or a point such that $f(x) = -x$. 
\bigskip\\
Take any map $f : \rp^{2n} \to \rp^{2n}$. Then, take the covering map $p : S^{2n} \to \rp^{2n}$ which projects $p(x) = p(-x) = [x]$. Therefore, we get a map $f \circ p : S^{2n} \to \rp^{2n}$ but $S^{2n}$ is simply connected (and path-connected and locally path-connected) so by the lifting criterion there exists a map $\tilde{f} : S^{2n} \to S^{2n}$ such that $p \circ \tilde{f} = f \circ p$. By the above result, there exists $x \in S^n$ such that $\tilde{f}(x) = \pm x$. Therefore, $f([x]) = p(\tilde{f}(x)) = p(\pm x) = [x]$ so $f$ has a fixed point. 
\bigskip
However, if the dimension of projective space is odd then there exist maps with no fixed points. Consider the linear map $F : \R^{2n} \to \R^{2n}$ given by the symplectic matrix,
\[ 
A = 
\begin{pmatrix}
0 & -I_{n} \\
I_{n} & 0 
\end{pmatrix}
\] 
We see that $A^2 = - I_{2n}$ so $F \circ F(x) = -x$. Suppose that $F$ had an eigenvector $x \in \R^{2n}$ with eigenvalue $\lambda \in \R$. Then we know that $F(x) = \lambda x$ so $F \circ F(x) = \lambda F(x) = \lambda^2 x$ because $F$ is linear. However, $F \circ F = - \id_{\R^{2n}}$ so $\lambda^2 x = - x$ but $x \neq 0$ since $x$ is an eigenvector. Thus, $\lambda^2 = -1$ but $\lambda \in \R$ which is impossible so $F$ has no eigenvectors. First, $F$ is injective (since it is invertible) so $F$ is a map $\R^{2n} \setminus \{0\} \to \R^{2n} \setminus \{0\}$. Since $F$ is linear, $F$ descends to the quotient $\rp^{2n - 1}$ under $x \sim \lambda x$ as a map $f : \rp^{2n - 1} \to \rp^{2n - 1}$ such that $f([x]) = [F(x)]$. If $f([x]) = [F(x)] = [x]$ then we know that $F(x) = \lambda x$ which we have shown to be impossible. Thus, $f$ has no fixed points. 

\section*{Problem 3.}

Let $f : S^n \to S^n$ have degree zero. Therefore, $\deg{f} \neq (-1)^{n+1}$ and $\deg{f} \neq 1$ so by Lemma \ref{antifixed} there must exist $x \in S^n$ such that $f(x) = x$ and by Lemma \ref{fixed} there must exist $y \in S^n$ such that $f(y) = -y$. 
\bigskip\\
Let $F$ be a nowhere vanishing continuous vector field on $D^n \subset \R^n$. Consider the continuous map $\tilde{F} : D^n \to S^{n-1}$ given by,
\[ \tilde{F}(x) = \frac{F(x)}{|F(x)|} \]
This defines a homotopy between $f = \tilde{F} |_{\partial D^n} : S^{n-1} \to S^{n-1}$ and $\tilde{F}(0)$ a constant map. Therefore, $f$ is nullhomotopic so $\deg{f} = 0$. By the above result, $\exists x,y \in \partial D^n$ such that $f(x) = x$ and $f(y) = -y$ so,
\[ F(x) = x |F(x)| \quad \text{and} \quad F(y) = - y |F(y)| \]
so $F$ points radially outwards at $x$ and radially inwards at $y$.   

\section*{Problem 4.}

We are given the exact sequence of chain complexes,
\begin{center}
\begin{tikzcd}
0 \arrow[r] & C(X) \arrow[r, "n"] & C(X) \arrow[r, "\phi"] & C(X ; \Z / n \Z ) \arrow[r] & 0  
\end{tikzcd}
\end{center}
This short exact sequence gives rise to a long exact sequence of homology,
\begin{center}
\begin{tikzcd}
\cdots \arrow[r] & H_k(X) \arrow[r, "n"] & H_k(X) \arrow[r, "\phi_*"] & H_k(X ; \Z / n \Z ) \arrow[r, "d"] & H_{k-1}(X) \arrow[r, "n"] & H_{k-1}(X) \arrow[r] & \cdots  
\end{tikzcd}
\end{center}
I can adorn this sequence by using the fact that $\phi_*$ factors as $\tilde{\phi}_* \circ \pi$ where $\tilde{\phi}_*$ is injective through the quotient by $\ker{\phi_*} = \Im{n} = n H_k(X)$ by exactness. Furthermore, again by exactness, $d$ factors as a surjective map through  $\Im{d} = \ker{n} =  T_n(H_{k-1}(X))$, the $n$-torsion group. Therefore, the following diagram commutes,
\begin{center}
\begin{tikzcd}
H_k(X) \arrow[r, "n"] \arrow[rd, "0"] & H_k(X) \arrow[r, "\phi_*"] \arrow[d, "\pi"]  & H_k(X ; \Z / n \Z ) \arrow[r, "d"] \arrow[rd, "d", two heads] & H_{k-1}(X) \arrow[r, "n"] & H_{k-1}(X) 
\\
& H_{k}(X) / n H_k(X) \arrow[ru, "\tilde{\phi}_*", hook] & & T_n(H_{k-1}(X)) \arrow[u, "\iota"] \arrow[ru, "0"]
\end{tikzcd}
\end{center}
Furthermore, $\phi_* = \tilde{\phi}_* \circ \pi$ so $\Im{\phi_*} = \Im{\tilde{\phi}_*}$ since $\pi$ is surjective. However, by exactness, $\Im{\phi_*} = \ker{d}$ and therefore the sequence,
\begin{center}
\begin{tikzcd}
0 \arrow[r] & H_{k}(X) / n H_k(X) \arrow[r, "\tilde{\phi}_*", hook] & H_k(X ; \Z / n \Z )  \arrow[r, "d", two heads] & T_n(H_{k-1}(X)) \arrow[r] & 0
\end{tikzcd}
\end{center}
is short exact. Therefore, $H_k(X ; \Z / n \Z) = 0$ if and only if $H_{k}(X) / n H_k(X) = 0$ and $T_n(H_{k-1}(X)) = 0$ if and only if multiplication by $n$ is an automorphism. The last two statements are logically equivalent since multiplication by $n$ is surjective iff $n H_k(X) = H_k$ iff $H_k(X) / n H_k(X) = 0$ and multiplication by $n$ is injective iff $\ker{n} = T_n(H_{k-1}(X)) = 0$. 
\bigskip\\ 
By Lemma \ref{module-vectorspace}, we see that the $\Z$-module $H_k(X)$ has the structure of a $\Q$-vectorspace if and only if the multiplication by $n$ is an automorphism for each $n \in \Z \setminus \{0\}$. We have shown that multiplication by $n$ is an automorphism if and only if $H_k(X ; \Z / n \Z) = 0$. In fact, we only need to check this for primes $p$ because by integer factorization if multiplication by $p$ is an automorphism for every prime then by composition any nonzero $n$ as a product of primes acts by multiplication as the composition of automorphisms and is thus an automorphism. Therefore $H_k(X)$ extends to a $\Q$-vectorspace if and only if $H_k(X ; \Z / p \Z) = 0$ for every prime $p$.   

\section*{Problem 5.}

Consider the transfer sequence associated to the covering map $p : S^\infty \to \rp^\infty$,

\begin{center}
\begin{tikzcd}
\cdots \arrow[r] & H_n(\rp^{\infty} ; \Z / 2 \Z) \arrow[r, "\tau_*"] & H_n(S^{\infty} ; \Z / 2 \Z) \arrow[r, "p_*"] \arrow[draw=none]{d}[name=Z, shape=coordinate]{} & H_n(\rp^{\infty} ; \Z / 2 \Z)  
\arrow[dll,
rounded corners, crossing over,
to path={ -- ([xshift=2ex]\tikztostart.east)
|- (Z) [near end]\tikztonodes
-| ([xshift=-2ex]\tikztotarget.west)
-- (\tikztotarget)}]
\\ 
& H_{n-1}(\rp^{\infty} ; \Z / 2 \Z) \arrow[r, "\tau_*"] & H_{n-1}(S^{\infty} ; \Z / 2 \Z) \arrow[r] & H_{n-1}(\rp^{\infty} ; \Z / 2 \Z) \arrow[r] & \cdots
\end{tikzcd}
\end{center}
However, $S^\infty$ is contractible $\tilde{H}_n(S^{\infty} ; \Z / 2 \Z) = 0$ for each $n$. Therefore, we have isomorphisms,
\begin{center}
\begin{tikzcd}
H_n(\rp^{\infty} ; \Z / 2 \Z) \arrow[r, "\sim"] & H_{n-1}(\rp^{\infty} ; \Z / 2 \Z) 
\end{tikzcd}
\end{center}
for each $n$. However, since $\rp^\infty$ is path-connected we know that $H_0(\rp^\infty; \Z / 2 \Z ) \cong \Z / 2 \Z$. Therefore, $H_n(\rp^\infty ; \Z / 2 \Z) \cong \Z / 2 \Z$ for each $n$.  
\section*{Problem 6.}

Let $f : C \to D$ be a morphism of chain complexes. Consider the complex $(C_f)_n = C_{n-1} \oplus D_n$ with the boundary map $\partial_n(x,y) = (- \partial_{C, n-1}(x), f_{n-1}(x) + \partial_{D,n}(y))$. Consider,
\begin{align*}
\partial_{n} \circ \partial_{n+1}(x, y) & = \partial_n (- \partial_{C, n}(x), f_{n}(x) + \partial_{D,n+1}(y))
\\
& = (\partial_{C, n-1} \circ \partial_{C,n}(x), -f_{n-1}(\partial_{C,n}(x)) + \partial_{C,n} \circ f_{n}(x) + \partial_{D, n} \circ \partial_{D,n+1}(y))
\\
& = (0, -f_{n-1}(\partial_{C,n}(x)) + \partial_{C,n} \circ f_{n}(x) ) = 0
\end{align*}
because $f$ is a morphism of chain complexes. Therefore, $\Im{\partial_{n+1}} \subset \ker{\partial_n}$ so $C_f$ is a complex.

\section*{Problem 7.}

Define the morphism of chain complexes $j : D \to C_f$ by $j_n(y) = (0, y)$ and a morphism of chain complexes, $d : C_f \to C[-1]$ by $d_n(x, y) = (-1)^n x $ where $(C[-1])_n = C_{n-1}$ and $\partial_{C[-1]} = - \partial_C$. These are maps of complexes because,
\[ \partial_n \circ j_n(y) = \partial_n(0, y) = (0, \partial_{D, n}(y)) = j_{n-1} \circ \partial_{D,n}(y) \]
and likewise,
\[ d_{n-1} \circ \partial_n(x,y) = d_{n-1}(-\partial_{C, n-1} (x), f_{n-1}(x) + \partial_{D, n}(y)) = (-1)^{n} \partial_{C,n-1} (x) = \partial_{C[-1], n} \circ d_{n}(x, y) \]
Clearly, $j$ is injective and $d$ is surjective. Furthermore, $\ker{d_n} = \{ (0, y) \mid y \in D_n \} = \Im{j_n}$. Therefore, 
\begin{center}
\begin{tikzcd}
0 \arrow[r] & D \arrow[r, "j"] & C_f \arrow[r, "d"] & C[-1] \arrow[r] & 0
\end{tikzcd}
\end{center}
is an exact sequence of complexes.

\section*{Problem 8.}

Let $f : C \to D$ be a morphism of complexes. Consider the long exact sequence of homology corresponding to the above exact sequence of chain complexes,

\begin{center}
\begin{tikzcd}
\cdots \arrow[r] & H_{n}(D) \arrow[r, "j_*"] & H_n(C_f) \arrow[r, "d_*"] & H_{n-1}(C) \arrow[r, "f_*"] & H_{n-1}(D) \arrow[r, "j_*"] & H_{n-1}(C_f) \arrow[r] & \cdots 
\end{tikzcd}
\end{center}
I claim that the linking map $H_{n}(C[-1]) = H_{n-1}(C) \to H_{n-1}(D)$ is equal to $f_*$. To see this, take $c \in C_{n-1}$ then $[c] \mapsto [a]$ where $a \in D_{n-1}$ is such that $c = d_{n}(x,y)$ for $(x,y) \in (C_f)_n$ and $\partial_{C_f, n} (x,y) = j_{n-1}(a)$. However, 
\[ \partial_{C_f, n} (x,y) = ( - \partial_{C,n-1} (x), f_{n-1}(x) + \partial_{D, n}(y)) = j_{n-1}(a) = (0, a) \]
Therefore, $\partial_{C,n-1}(x) = 0$ and $f_{n-1}(x) + \partial_{D,n}(y) = a$ but $(-1)^{n} x = c$ so $a = (-1)^n f_{n-1}(c) + \partial_{D,n}(y)$ so $[a] = (-1)^n [f_{n-1}(c)] = (-1)^n f_*([c])$. Therefore, up to sign, the linking map is $f_*$. From the above exact sequence, $f_* : H_n(C) \to H_n(D)$ is an isomorphism for all $n$ if and only if $H_{n+1}(C_f) = 0$ for all $n$. Therefore, $f$ is a quasi-isomorphism if and only if the complex $C_f$ is a cyclic. 

\section*{Lemmas}


\begin{lemma} \label{antifixed}
Let $f : S^n \to S^n$ have no point at which $f(x) = - x$ then $\deg{f} = 1$. 
\end{lemma}


\begin{proof}
Suppose $\forall x : f(x) \neq -x$. Thus, the line $f(x)$ to $x$ does not pass though the origin. Therefore, the map $H(x, t) = [(1 - t) f(x) + t x] / | (1 - t) f(x) + t x |$ is a homotopy between $f$ and the identity. Thus, $\deg{f} = \deg{\mathbf{1}} = 1$. 
\end{proof}

\begin{lemma} \label{fixed}
Let $f : S^n \to S^n$ have no fixed points then $\deg{f} = (-1)^{n+1}$. 
\end{lemma}

\begin{proof}
If the map $f : S^n \to S^n$ has no fixed points then the map $-f : S^n \to S^n$ satisfies Lemma \ref{antifixed} and thus $\deg{(-f)} = 1$. However, $\deg{(-f)} = \deg{(-1)} \cdot \deg{f} = (-1)^{n+1} \deg{f}$. Therefore, $\deg{f} = (-1)^{n+1}$.
\end{proof}

\begin{lemma} \label{module-vectorspace}
Let $R$ be an integral domain and $K$ its field of fractions. Then an $R$-module extends to a $K$-vectorspace if and only if for each $r \in R \setminus \{0\}$ the multiplication by $r$ map is an automorphsm. 
\end{lemma}

\begin{proof}
Suppose $M$ is an $R$-module. If $M$ is the restriction of a $K$-vectorspace then for any $r \in R \setminus \{0\} \subset K^\times$ we have $r^{-1} \in K$. Therefore, for $x \in M$ we have $r^{-1} \cdot (r \cdot x) = (r^{-1} r) \cdot x = x$ and $r \cdot (r^{-1} \cdot x) = (r r^{-1}) \cdot x = x$ so multiplication by $r$ is a bijection and thus an automorphism of the abelian group $M$ by scalar distributivity.
\bigskip\\
Convsersely, suppose that the map $f_r : M \to M$ given by $f_r(x) = r \cdot x$ is an automorphism for each $r \in R \setminus \{0\}$ then we can define a $K$-vectorspace $M$ by the action of $\frac{a}{b} \in K$ via, 
\[\frac{a}{b} \cdot x = (f_b^{-1} \circ f_a)(x) = f_b^{-1}(a \cdot x)\]
Clearly, if $r \in R \subset K$ then $r \cdot_K M = f_r(x) = r \cdot M$ so this $K$-vectorspace restricts to the given $R$-module. We need to check that this is a honest-to-god vectorspace. First, if $\frac{a}{b} = \frac{a'}{b'}$ then $ab' = a'b$ so,
\[ b \cdot \frac{a'}{b'} \cdot x = b \cdot f_{b'}^{-1}( a' \cdot x) = f_{b'}^{-1}(b a' \cdot x) = f_{b'}^{-1}(b' a \cdot x) = (f_{b'}^{-1} \circ f_{b'})( a \cdot x) = a \cdot x \implies \frac{a'}{b'} \cdot x = f_{b}^{-1}(a \cdot x) = \frac{a}{b} \cdot x\]
so the action is well-defined. Commutativity of $f_{a}$ and $f_{b}$ gives the remaining properties. Take any $p,q \in K$ then,
\[ p \cdot (q \cdot x) = (pq) \cdot x \quad (p + q) \cdot x = p \cdot x + q \cdot x \]
and we know that $f_{b}^{-1} \circ f_{a}$ is the composition of homomorphisms and thus $p \cdot (x + y) = p \cdot x + p \cdot y$. Therefore, $M$ is a $K$-vectorspace.     
\end{proof}
\end{document}
