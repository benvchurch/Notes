\documentclass[12pt]{extarticle}
\usepackage[utf8]{inputenc}
\usepackage[english]{babel}
\usepackage[a4paper, total={6in, 9in}]{geometry}
\usepackage{tikz-cd}
 
\usepackage{amsthm, amssymb, amsmath, centernot}

\newcommand{\notimplies}{%
  \mathrel{{\ooalign{\hidewidth$\not\phantom{=}$\hidewidth\cr$\implies$}}}}
 
\renewcommand\qedsymbol{$\square$}
\newcommand{\cont}{$\boxtimes$}
\newcommand{\divides}{\mid}
\newcommand{\ndivides}{\centernot \mid}
\newcommand{\Z}{\mathbb{Z}}
\newcommand{\R}{\mathbb{R}}
\newcommand{\N}{\mathbb{N}}
\newcommand{\Zplus}{\mathbb{Z}^{+}}
\newcommand{\Primes}{\mathbb{P}}
\newcommand{\colim}[1]{\mathrm{colim}(#1)}
\newcommand{\Ob}[1]{\mathrm{Ob}(#1)}
\newcommand{\cat}[1]{\mathcal{#1}}
\newcommand{\id}{\mathrm{id}}
\newcommand{\Hom}[2]{\mathrm{Hom}\left( #1, #2 \right)}
\newcommand{\catHom}[3]{\mathrm{Hom}_{#1}\left( #2, #3 \right)}
\newcommand{\Top}{\mathbf{Top}}
\newcommand{\pTop}{\mathbf{Top}_{\bullet}}
\newcommand{\Set}{\mathbf{Set}}
\newcommand{\pSet}{\mathbf{Set}_\bullet}
\newcommand{\hTop}{\mathbf{hTop}}
\newcommand{\phTop}{\mathbf{hTop}_{\bullet}}
\newcommand{\Grp}{\mathbf{Grp}}
\renewcommand{\Im}[1]{\mathrm{Im}(#1)}
\newcommand{\homspace}[2]{\left< #1, #2 \right>}
\newcommand{\rp}{\mathbb{RP}}
\newcommand{\coker}[1]{\mathrm{coker}\: #1}

\theoremstyle{definition}
\newtheorem{theorem}{Theorem}[section]
\newtheorem{lemma}[theorem]{Lemma}
\newtheorem{proposition}[theorem]{Proposition}
\newtheorem{example}[theorem]{Example}
\newtheorem{corollary}[theorem]{Corollary}
\newtheorem{remark}{Remark}

\newenvironment{definition}[1][Definition:]{\begin{trivlist}
\item[\hskip \labelsep {\bfseries #1}]}{\end{trivlist}}


\newenvironment{lproof}{\begin{proof} \renewcommand{\qedsymbol}{}}{\end{proof}}
\renewcommand{\mod}[3]{\: #1 \equiv #2 \: mod \: #3 \:}
\newcommand{\nmod}[3]{\: #1 \centernot \equiv #2 \: mod \: #3 \:}
\newcommand{\ndiv}{\hspace{-4pt}\not \divides \hspace{2pt}}
\newcommand{\gen}[1]{\langle #1 \rangle}
\newcommand{\hook}{\hookrightarrow}
\newcommand{\Tor}[4]{\mathrm{Tor}^{#1}_{#2} \left( #3, #4 \right)}
\newcommand{\Ext}[4]{\mathrm{Ext}^{#1}_{#2} \left( #3, #4 \right)}

\tikzset{
    labl/.style={anchor=south, rotate=90, inner sep=.5mm}
}

\begin{document}

\section{Category Theory}

\subsection{Limits and Colimits}
\begin{definition}
A diagram $\cat{D}$ is a small category i.e. one such that $\Ob{\cat{D}}$ is a set. 
\end{definition}
\begin{definition}
Given a diagram $\cat{D}$, a category $\cat{C}$ and a functor $F : \cat{D} \to \cat{C}$ the limit is $\lim{F} \in \Ob{\cat{C}}$ with maps $\forall d \in \Ob{\cat{D}} : \exists \pi(d) : \lim{F} \to F(d)$ such that,
\begin{center}
\begin{tikzcd}[column sep = huge, row sep = huge]
& A \arrow[ddl, bend right] \arrow[ddr, bend left] \arrow[d, dashed, "\exists ! f"] & \\
& \lim{F} \arrow[dl, swap, "\pi(d_i)"] \arrow[dr, "\pi(d_j)"] & \\
F(d_i) \arrow[rr] & & F(d_j) 
\end{tikzcd}
\end{center}
for any $A \in \Ob{\cat{C}}$ such that this diagram commutes, there exists a unique commuting map $f : A \to \lim{F}$. 
\end{definition}

\begin{definition}
Given a diagram $\cat{D}$, a category $\cat{C}$ and a functor $F : \cat{D} \to \cat{C}$ the colimit, $\colim{F} \in \Ob{\cat{C}}$ with maps $\forall d \in \Ob{\cat{D}} : \exists \iota(d) : F(d) \to \colim{F}$ such that,
\begin{center}
\begin{tikzcd}[column sep = huge, row sep = huge]
F(d_i) \arrow[rdd, bend right] \arrow[rr] \arrow[rd, swap, "\iota(d_i)"] &  & F(d_j) \arrow[ld, "\iota(d_j)"] \arrow[ldd, bend left] \\
& \colim{F} \arrow[d, dashed, "\exists ! f"] & \\
& A &
\end{tikzcd}
\end{center}
for any $A \in \Ob{\cat{C}}$ such that this diagram commutes, there exists a unique commuting map $f : \colim{F} \to A$. 
\end{definition}

\begin{definition}
A cateogroy $\cat{C}$ is complete if the limit of any diagram exists and cocomplete if the colimit of any diagram exists. 
\end{definition}

\begin{theorem}
Limits and colimits are unique up to unique isomorphism.
\end{theorem}

\begin{proof}
Let $\cat{D}$ be a diagram and $F : \cat{D} \to \cat{C}$ a functor. Suppose that $X$ and $Y$ are both limits of $F$. Then, there exist commuting maps from $X$ and from $Y$ to $F(d_i)$. Therefore, there exist unique maps $f : X \to Y$ and $g : Y \to X$ such that the diagram commutes, 
\begin{center}
\begin{tikzcd}[column sep = huge, row sep = huge]
& X \arrow[ddl, bend right] \arrow[ddr, bend left] \arrow[d, dashed, "\exists ! f", bend left] & \\
& Y \arrow[u, "\exists ! g", bend left, dashed] \arrow[dl, swap, "\pi(d_i)"] \arrow[dr, "\pi(d_j)"] & \\
F(d_i) \arrow[rr] & & F(d_j) 
\end{tikzcd} 
\end{center}
Thus, $f \circ g : Y \to Y$ commutes with the diagram but because $Y$ is a limit there is a unique map $Y \to Y$ which commutes with the diagram and $\id_Y$ satisfies this property. Thus, $f \circ g = \id_Y$. Likewise, $g \circ f : X \to X$ commutes with the diagram but $X$ is a limit of the diagram so there is a unique map $X \to X$ which commutes with the diagram, namely $\id_X$. Thus, $g \circ f = \id_X$. Therefore, $f$ and $g$ are unique isomorphisms. The case for colimits is identical. 
\end{proof}

\begin{definition}
A functor $F : \cat{C} \to \cat{C}'$ preserves limits of type $\cat{D}$ if whenever $G : \cat{D} \to \cat{C}$ is a functor from a diagram then $\lim{(F \circ G)} = F(\lim{G})$. The functor $F$ is continuous if it preserves limits of all type. Similarly, $F$ is cocontinuous if it preserves colimits i.e. $\colim{(F \circ G)} = F(\colim{G})$. 
\end{definition}

\section{Homotopy Theory}

\subsection{CW Complexes}

\begin{definition}
$D^n = \{ \mathbf{x} \in \R^n \mid |\mathbf{x}| \le 1 \}$ and $S^n = \partial D^{n + 1}$. 
\end{definition}

\begin{definition}
A CW complex is a topological space $X = \bigcup\limits_{n = 0}^{\infty} X^n$ with $X^n$ such that,
\begin{enumerate}
\item $X^0$ is a discrete set
\item For each $n > 0$, $X^n$ is formed by taking a collection $D_\alpha^n$ of $n$-cells and attaching maps $\varphi_\alpha : \partial D_{\alpha}^n \to X^{n - 1}$ and letting,
\[X^n = \left( X^{n - 1} \coprod\limits_{\alpha} D_\alpha^n \middle/  (x_\alpha \sim \varphi_\alpha(x_\alpha)) \right) \] 
\item $X = \bigcup\limits_{n = 0}^{\infty} X^n$ with $X^n$ is given the weak topology, $A \subset X$ is open if and only if $A \cap X^n$ is open in $X^n$ for every $n$.
\item If $X = X^n$ and $X \neq X^{n-1}$ then $X$ has dimension $n$. Otherwise, $X$ has infinite dimension. 
\end{enumerate}
\end{definition}

\begin{definition}
Let $X$ be a CW-complex then $(X, A)$ is a CW-pair if $A$ is a sub-complex equipped with an inclusion map $\iota : X \to A$ where a subcomplex is a subspace which is a cell-complex built from a subset of the cells of $X$. 
\end{definition}

\subsection{Product-Hom Adjunction}


\begin{definition}
Let $X$ and $Y$ be topological spaces then $\Hom{X}{Y}$ is a topological space with the compact-open topology such that if $K \subset X$ is compact and $U \subset Y$ is open then the sets $V(K, U) = \{ f : X \to Y \mid f(K) \subset Y \}$ form an open subbase for the topology on $\Hom{X}{Y}$. We denote this internal hom by $Y^X$. 
\end{definition}

\begin{lemma}
Let $Y$ be locally compact Hausdorff (LCH). There is a natural isomorphism,
\[\Hom{X \times Y}{Z} \cong \Hom{X}{Z^Y}\]
\end{lemma}

\begin{proof}
Define the map $\varphi_{X,Y,Z} : \Hom{X \times Y}{Z} \to \Hom{X}{Z^Y}$ via $\hat{f} = \varphi(f)$ defined by $x \mapsto (y \mapsto f(x, y))$ with inverse $\varphi^{-1}(\hat{f}) : (x, y) \mapsto \hat{f}(x)(y) = f(x, y)$. We need to show that $f$ is continuous iff $\hat{f}$ is continuous. 
\bigskip
\\
Suppose that $f$ is continuous. We need only check that the preimage of the subbase sets $V(K, U) \subset Z^Y$ are open. We have,
\[ \hat{f}^{-1}(V(K, U)) = \{ x \in X \mid \forall y \in K : (x, y) \in f^{-1}(U) \} \]
Take $x \in \hat{f}^{-1}(V(K, U))$ then for each $y \in K$ we know that $(x, y) \in f^{-1}(U)$ which is open so there exist open sets $U_y \subset X$ and $V_y \subset Y$ such that $(x, y) \in U_y \times V_y \subset f^{-1}(U)$. Thus, $\{ V_y \mid y \in K \}$ is an open cover of $K$ which is compact and thus has an open subcover $S \subset K$. Now define the neighborhood of $x$,
\[ U_x = \bigcap_{y \in S} U_y \]
which is open because $S$ is open. Thus, $U_x \times V_y \subset f^{-1}(U)$ for each $y$ and thus $U_x \times K \subset f^{-1}(U)$ so $x \in U_x \subset \hat{f}^{-1}(V(K, U))$. Therefore, $\hat{f}^{-1}(V(K, U))$ is open. 
\bigskip
\\
Suppose that $\hat{f}$ is continuous. Take open $U \subset Z$ and a point $(x, y) \in f^{-1}(U)$. Since $Y$ is LCH, we can find $x \in V \subset K$ where $V$ is open and $K$ is compact. Then consider,
\[ (x, y) \in \hat{f}^{-1}(V(K, U)) \times V = \{ x \in X \mid \forall y \in K : f(x, y) \subset U \} \times V \subset f^{-1}(U)  \]
However, by continuity, $\hat{f}^{-1}(V(K, U))$ is open and thus $f^{-1}(U)$ is open.
\bigskip
\\
Finally we need to check naturality. Let $X, X', Y, Y', Z, Z'$ all be LCH with maps $a : X' \to X$ and $b : Y' \to Y$ and $c : Z \to Z'$. The functor $\Hom{(-)\times(-)}{-}$ takes these maps to
\[m : \Hom{X \times Y}{Z} \to \Hom{X \times Y}{Z}\]
via $m(f) = c \circ f \circ (a \times b)$. Likewise, the functor $\Hom{-}{(-)^(-)}$ takes these maps to 
\[m' : \Hom{X}{Z^Y} \to \Hom{X'}{Z'^{Y'}}\]
via $m'(f) = (b, c)_* \circ f \circ a$ where $(b, c)_*(g) = c \circ g \circ b$. Consider, 
\[ m' \circ \varphi_{X, Y, Z}(f) = m'(\hat{f}) = (b, c)_* \circ \hat{f} \circ a \] which takes $x'$ to $c \circ (\hat{f}(a(x'))) \circ b$ which takes $y'$ to $c(f(a(x'), b(x'))$. Futhermore, $\varphi_{X', Y', Z'} \circ m(f)$ is the map taking $x'$ to the map taking $y'$ to $m(f)(x,y) = c(f(a(x'), b(y'))$. Therefore, 
\[ m' \circ \varphi_{X, Y, Z} = \varphi_{X', Y', Z'} \circ m \]
so $\varphi$ is a natural isomorphism. 
\end{proof}

\begin{proposition}
Let $Y$ be LCH then $Z^Y$ is an exponential object in $\Top$. 
\end{proposition}

\begin{proof}
Take $ev : Z^Y \times Y \to Z$ via $(g, y) \mapsto g(y)$. Suppose we have a continuous map $f : X \times Y \to Z$. By the product-hom adjunction, we get a continuous map $\hat{f} : X \to Z^Y$ such that diagram commutes,
\begin{center}
\begin{tikzcd}[column sep = large, row sep = huge]
X \times Y \arrow[rd, "f"] \arrow[d, dashed, "\hat{f} \times \id_Y"']
\\
Z^Y \times Y \arrow[r, "ev"] & Z
\end{tikzcd}
\end{center}
since $ev \circ (\hat{f} \times \id_Y)(x,y) = \hat{f}(x)(y) = f(x,y)$. Furthermore, if the above diagram commutes for a map $g \times \id_Y : X \times Y \to Z^Y \times Y$ then $g(x)(y) = f(x,y)$ so $g = \hat{f}$ meaning $\hat{f}$ is unique.  
\end{proof}

\subsection{Homotopy}

\begin{definition}
A homotopy between $f,g : X \to Y$ is a continuous map $H : I \times X \to Y$ such that $H(0, x) = f(x)$ and $H(1, y) = g(y)$.  
\end{definition}

\begin{proposition}
Let $X$ be LCH. Then a homotopy between $f$ and $g$ is naturally equivalent to a path $\gamma : I \to Y^X$ between $f$ and $g$.
\end{proposition}

\begin{proof}
A homotopy $H : I \times X \to Y$ between $f$ and $g$ is naturally equivalent to a continuous path $\hat{H} : I \to Y^X$ by product-hom adjunction such that $\hat{H}(0) = f$ and $\hat{H}(1) = g$ since $\hat{H}(0)(x) = H(0, x) = f(x)$ and $\hat{H}(1)(x) = H(1, x) = g(x)$.  
\end{proof}

\begin{proposition}
Path-connection and thus homotopy equivalence are equivalence relations.
\end{proposition}

\begin{proof}
For $x \in X$ the map $\gamma_x : I \to X$ given by $\gamma(t) = x$ is a path from $x$ to $x$. If $\gamma : I \to X$ is a path from $x$ to $y$ then $\tilde{\gamma}(t) = \gamma(1-t)$ is a path from $y$ to $x$. Finally, if $\gamma, \delta : I \to X$ are paths from $x$ to $y$ and $y$ to $z$ then
\[ \delta * \gamma(t) = 
\begin{cases}
\gamma(t) & t \le [0, \tfrac{1}{2} ]
\\
\delta(2t - 1) & t \ge [\tfrac{1}{2}, 1] 
\end{cases}
\]
is a path from $x$ to $z$ since $\gamma(1) = \delta(1) = y$ so $\delta * \gamma$ is continuous by the gluing lemma. 
\end{proof}

\begin{definition}
Define $\pi_0(X)$ to be the set of path-components of $X$. If $X$ is based at $x_0$ then $\pi_0(X)$ is a based set based at the path-component of $x_0$.
\end{definition}

\begin{proposition}
Let $X$ be LCH. The set of homotopy classes of maps $X \to Y$ is $\pi_0(\Hom{X}{Y}) = \pi_0(Y^X)$. 
\end{proposition}


\subsection{Based Homotopy}

\begin{definition}
If $(X, x_0)$ and $(Y, y_0)$ are pointed topological spaces then a homotopy of based maps $f, g : (X, x_0) \to (Y, y_0)$ is a based map $H : I \times X \to Y$ such that $H(0, x) = f(x)$ and $H(1, x) = g(x)$ and $H(t, x_0) = y_0$.
\end{definition}

\begin{definition}
Let $X$ and $Y$ be pointed topological spaces. Then $X \wedge Y =  X \times Y / X \vee Y$ defines the smash product.
\end{definition}

\begin{definition}
Let $(X, x_0)$ and $(Y, y_0)$ be pointed topological spaces then $(\catHom{\pTop}{X}{Y}, f_0)$ is a pointed topological space where $\catHom{\pTop} {X}{Y}$ is the subspace of continuous maps $f : X \to Y$ such that $f(x_0) = y_0$ and $f_0(x) = y_0$ given the compact-open topology. We may denote this internal hom by, 
\[ Y^X = \catHom{\pTop}{X}{Y}\]  
\end{definition}

\begin{theorem}
If $Y$ be LCH then there is a natural isomorphism,
\[\catHom{\pTop}{X \wedge Y}{Z} \cong \catHom{\pTop}{X}{Z^Y} \]
\end{theorem}

\begin{proof}
This is a restriction of the usual product-hom adjunction because a based map continuous map in $\catHom{\pTop}{X \wedge Y}{Z}$ is equivalent to a map $f \in \catHom{\Top}{X \times Y}{Z}$ such that $f(x_0 \times y) = z_0$ and $f(x, y_0) = z_0$. This condition is equivalent to $\hat{f}$ being a based map which sends $x$ to a based map $\hat{f}(x)$.  
\end{proof}

\begin{definition}
$\left< X, Y \right> = \pi_0(\catHom{\pTop}{X}{Y})$.
\end{definition}

\begin{proposition}
Let $X$ be LCH. Then $\left< X, Y \right>$ is the pointed set of based homotopy classes of based maps $X \to Y$.  
\end{proposition}

\begin{proof}
A homotopy $H$ between $f, g : X \to Y$ is naturally equivalent to a path 
\[\gamma_H : I \to \catHom{\Top}{X}{Y}\]
from $f$ to $g$. Furthermore, the based condition $H(t, x_0) = y_0$ implies that $\gamma_H = \hat{H} : X \to Y$ is a based map. Thus, $\gamma_H$ restricts to a path $\gamma_H : I \to \catHom{\pTop}{X}{Y}$ if and only if $H$ is a based homotopy.    
\end{proof}

\begin{remark}
We may generalize a based homotopy to fix an entire subset $A \subset X$ under a given inclusion $A \hookrightarrow Y$. Such based homotopy classes correspond to $\pi_0$ of the subspaces of maps $f : X \to Y$ fixing $A$ properly. For example, a path-homotopy is a based homotopy fixing the two end points. 
\end{remark}

\subsection{Higher Homotopy Groups}


\begin{definition}
The reduced suspension of $X$ is the space $\Sigma X = X \wedge S^1$. 
\end{definition}

\begin{proposition}
$\Sigma S^n \cong S^{n + 1}$
\end{proposition}

\begin{definition}
The loop space of $X$ is the space $\Omega X = \catHom{\pTop}{S^1}{X}$. 
\end{definition}

\begin{corollary}
\[\catHom{\pTop}{\Sigma X}{Y} \cong \catHom{\pTop}{X}{\Omega Y} \]
naturally.
\end{corollary}


\begin{corollary}
$\left< \Sigma X, Y \right> = \left< X, \Omega Y \right>$
\end{corollary}

\begin{proposition}
$\pi_1(X) = \pi_0(\Omega X) = \left<S^1, X\right> = \left< \Sigma S^0, X \right> = \left< S^0, \Omega X \right> $
\end{proposition}

\begin{definition}
Let $X$ be a pointed topological space then $\pi_n(X) = \left<S^n, X \right>$
\end{definition}

\begin{proposition}
$\pi_{n+1}(X) = \pi_n(\Omega X)$ and thus $\pi_n(X) = \pi_0(\Omega^n X)$
\end{proposition}

\begin{proof}
$\pi_{n+1}(X) = \homspace{S^{n+1}}{X} = \homspace{\Sigma S^n}{X} = \homspace{S^n}{\Omega X} = \pi_n(\Omega X)$
\end{proof}

\begin{proposition}
$\pi_n  : \pTop \to \pSet$ is a homotopy invariant functor i.e. there exists a functor $\pi_n' : \phTop \to \pSet$ such that the diagram of functors commutes,
\begin{center}
\begin{tikzcd}[column sep = large, row sep = large]
\pTop \arrow[rr, "\pi_n"] \arrow[rd] & & \pSet \\
& \phTop \arrow[ru, "\pi_n'"] &
\end{tikzcd}
\end{center}  
\end{proposition}

\begin{proof}
If $f, g : X \to Y$ are homotopic $f \simeq g$ then for any $\gamma : S^n \to X$ consider its class $[\gamma] \in \pi_n(X)$ and the pushforward $f_* [\gamma] = [f \circ \gamma]$. However, $f \circ \gamma \simeq g \circ \gamma$ so 
\[ f_* [\gamma] = [f \circ \gamma] = [g \circ \gamma] = g_* [\gamma ]\]
Thus $f_* = g_*$ so $\pi_n$ descends to the homotopy category.
\end{proof}

\begin{proposition}
\[\pi_n \left( \prod_\alpha X_\alpha \right) \cong \prod_{\alpha} \pi_n(X_\alpha)\]
\end{proposition}

\begin{proof}
This follows from the fact that $\Hom{X}{-}$ is continuous (preserves limits). Therefore,
\[ \catHom{\pTop}{S^n}{\prod_{\alpha} X_{\alpha}} \cong \prod_{\alpha} \catHom{\pTop}{S^n}{X_{\alpha}} \]
Furthermore, $\pi_0 : \Top \to \Set$ preserves products. 
\end{proof}

\begin{proposition}
The images of $\pi_1$ are groups and, for $n \ge 2$, the images of $\pi_n$ are abelian groups. 
\end{proposition}

\begin{proof}
Because $\pi_n : \phTop \to \pSet$ preserves products it takes group objects to group objects. Furthermore, for $n \ge 1$ we have $\pi_n(X) = \pi_{n-1}(\Omega X)$. However, $\Omega X$ is a group object in the homotopy category so $\pi_n(X)$ is a group object in $\pSet$ and thus a group. If $f : X \to Y$ is any morphism in the homotopy category then $\Omega f : \Omega X \to \Omega Y$ is a morphism of group objects. Thus, because $\pi_{n-1}$ preserves products $f_* : \pi_n(X) \to \pi_n(Y)$ which is $\pi_{n-1}(\Omega f) : \pi_{n-1}(\Omega X) \to \pi_{n-1}(\Omega Y)$ is a morphism of group objects. Thus $\pi_n$ is a functor, $\pi_n : \phTop \to \Grp$
\bigskip\\
Furthermore, for $n \ge 2$ we have $\pi_n(X) = \pi_{n-1}(\Omega X)$ with $\pi_{n-1} : \phTop \to \Grp$ and $\Omega X$ a group object in $\phTop$. Thus $\pi_n(X)$ is a group object in the category of groups and thus an abelian group. 
\end{proof}

\begin{remark}
We can construct this group operation explitly.
\end{remark}

\begin{proposition}
$\left< \Sigma X, Y \right>$ has a natural group structure and $\left< \Sigma^2 X, Y \right>$ is abelian.
\end{proposition}

\begin{proof}
For $f,g \in \catHom{\pTop}{X}{Y}$ define, 
\[f \cdot g(x \wedge t) = 
\begin{cases}
g(x \wedge 2t) & 0 \le t \le \tfrac{1}{2} \\
f(x \wedge (2t - 1)) & \tfrac{1}{2} \le t \le 1
\end{cases}\]
\end{proof}

\begin{corollary}
For $n \ge 1$, $\pi_n(X)$ is a group. For $n \ge 2$, the group $\pi_n(X)$ is abelian. 
\end{corollary}

\begin{proposition}
If $p : \tilde{X} \to X$ is a covering map then $p_* : \pi_n(\tilde{X}) \to \pi_n(X)$ is an isomorphism when $n \ge 2$.
\end{proposition}

\begin{proof}
This map is injective by homotopy lifting. However, since $S^n$ is simply connected, any map $S^n \to X$ can be lifted to $\tilde{X}$. Thus, $p_*$ is also surjective.
\end{proof}

\begin{corollary}
If $X$ has a contractible universal cover then $\pi_n(X) = 0$ for $n \ge 2$. 
\end{corollary}


\begin{remark}
There are no examples of finite simply-connected non-contracttible CW complexes, all of whose $\pi_n$ groups are known.
\end{remark}


\subsection{Fibrations and Cofibrations}

\begin{definition}
A map $ \iota : A \to X$ is a cofibration if for every homotopy $h : A \times I \to Y$ and a map $f : X \to Y$ such that $h(-, 0) = f \circ \iota$, there exists a homotopy $\tilde{h} : X \times I \to Y$ extending both maps.
\begin{center}
\begin{tikzcd}[column sep = large, row sep = large]
A \arrow[dd, "\iota"] \arrow[rr, "\iota_0"] & & A \times I \arrow[ld, "h"] \arrow[dd, "\iota \times \id_{I}"] \\
& Y & \\
X \arrow[ur, "f"] \arrow[rr, "\iota_0"] & & X \times I \arrow[ul, "\exists \tilde{h}", dashed]
\end{tikzcd}
\end{center}
\end{definition}

\begin{lemma}
Let $\iota : A \to X$ be an inclusion map, then $\iota$ is a cofibration iff there exists a retract $r : X \times I \to X \times \{0\} \cup A \times I$. 
\end{lemma}

\begin{proof}
Suppose that $\iota$ is a cofibration, the let $f = \id : X \times \{0\} \cup A \times I \to  X \times \{0\} \cup A \times I$ then there exists a homotopy $r : X \times I \to X \times \{0\} \cup A \times I$ which satisfies $r(x, 0) = \iota(x)$ but on $A$ this is an inclusion so $r$ is a retract. \bigskip \\
Conversely, suppose we have such a retract $r$ and $g :  X \times \{0\} \cup A \times I \to Y$ then tate $ g \circ r$ which satisfies the extension property.  
\end{proof}

\begin{corollary}
Let $(X, A)$ be a CW-pair then the inclusion map $\iota : X \to A$ is a cofibration. 
\end{corollary}

\begin{definition}
Given a map $f : X \to Y$ then the mapping cylinder of $f$ is, 
\[M_f = \left(X \times I \coprod Y \middle/ (x, 1) \sim f(x) \right)\] 
Equivalently, the mapping cylinder $M_f$ is the pushout, i.e. colimit of the diagram,
\begin{center}
\begin{tikzcd}[column sep = large, row sep = large]
Y & X \arrow[r, "\iota_0", hook] \arrow[l, "f"] & X \times I
\end{tikzcd}
\end{center}
\end{definition}

\begin{lemma}
Let $M_f$ be the mapping cylinder of $f : X \to Y$ and,
\begin{center}
\begin{tikzcd}[column sep = large, row sep = large]
X \arrow[r, hook, "j"] & M_f \arrow[r, "r"] & Y 
\end{tikzcd}
\end{center}
be the natural maps, then $j$ is a cofibration and $r$ is a homotopy equivalence.
\end{lemma}

\begin{proof}
Define the maps $j : x \mapsto (x, 0)$ and $r : (x, t) \mapsto f(x)$ and $r : y \mapsto y$. Take $\iota : Y \mapsto M_f$ be given by $\iota : y \mapsto y$. Thus, $r \circ \iota = \id_{Y}$ and define $h : M_f \times I \to M_f$ such that $h(y, s) = y$ and $h((x, t), s) = (x, (1 - s)t )$. Then, $h$ is a homotopy between $\id_{M_f}$ and $\iota \circ r$ since $h(y, 1) = y$ and $h((x, t), 1) = (x, 0)$.  
\end{proof}

\begin{definition}
The map $f : X \to Y$ is a fibration if whenever the following diagram commutes,
\begin{center}
\begin{tikzcd}[column sep = large, row sep = large]
Z \arrow[r] \arrow[d, hook, "\iota_0"] & X \arrow[d, "f"] \\
Z \times I \arrow[r] \arrow[ru, dashed, "\exists g"] & Y
\end{tikzcd}
\end{center}
there exists a (not necessarily unique) map $g: Z \times I \to X$. 
\end{definition}

\begin{definition}
A fiber bundle is a map $p : X \to Y$ such that there exists a space $F$ such that $\forall y \in Y$ there exists an open set $U$ with $y \in U \subset Y$ and $p^{-1}(U) \cong U \times F$ by a map $e$ such that the following diagram commutes,
\begin{center}
\begin{tikzcd}[column sep = large, row sep = large]
p^{-1}(U) \arrow[d, "p"] \arrow[rr, "e"] & & U \times F \arrow[lld, "\pi_1"] \\
U & &
\end{tikzcd}
\end{center}
\end{definition}

\begin{theorem}
If $f : X \to Y$ is a fiber bundle and $Y$ is paracompact then $f$ is a fibration.
\end{theorem}


\begin{definition}
For a map $f : X \to Y$, the mapping cocylinder is the space,
\[ N_f  = \{(x, \gamma) \mid \gamma(0) = f(x) \} \subset X \times \Hom{I}{Y} \]
which is the pullback (limit) of the diagram,
\begin{center}
\begin{tikzcd}[column sep = large, row sep = large]
X \arrow[r, "f"] & Y & Y^I \arrow[l, "ev_0"']
\end{tikzcd}
\end{center}  
\end{definition}

\begin{lemma}
Any map $f : X \to Y$ factors into maps $r : X \to Z$ and $j : Z \to Y$ where $r$ is a homotopy equivalence and $j$ is a fibration.
\end{lemma}

\begin{proof}
Take the maps $\nu : X \to N_f$ given by $\nu(x) = (x, e_{f(x)})$ and $\rho : N_f \to Y$ given by $\rho(x, \gamma) = \gamma(1)$. I claim that $\nu$ is a homotopy equivalence, $\rho$ is a fibration and $f = \rho \circ \nu$. 
\end{proof}

\begin{lemma}
$\iota : X \to Y$ is a cofibration if and only if for every space $Z$, map $f : Y \to Z$, and homotopy $h : X \times I \to Z$ such that the following diagram commutes,
\begin{center}
\begin{tikzcd}[column sep = large, row sep = large]
Z^I \arrow[d, "ev_0"'] & X \arrow[l, "\hat{h}"] \arrow[d, "\iota"] \\
Z & Y \arrow[l, "f"] \arrow[lu, dashed]
\end{tikzcd}
\end{center}  
there exists a map completing the commutative diagram.
\end{lemma}


\subsection{Fiber and Cofiber Sequences}

\begin{definition}
Let $f : X \to Y$ be a pointed map then the mapping cone $C_f = M_f / X$ where we have quotiented by $X$ under the natural inclusion into $M_f$.
\end{definition}

\begin{definition}
The functor $-\Sigma : \pTop \to \pTop$ is given by $-\Sigma (X) = \Sigma X$ and given $f : X \to Y$ then $- \Sigma (f) = g : \Sigma X \to \Sigma Y$ is given by $h(t \wedge x) = (1 - t) \wedge f(x)$. 
\end{definition}

\begin{definition}
Given a map $f : X \to Y$, the cofiber sequence is,
\begin{center}
\begin{tikzcd}[column sep = small, row sep = large]
X \arrow[r] & Y \arrow[r, hook] & C_f \arrow[r] & \Sigma X \arrow[r] & \Sigma Y \arrow[r] & \Sigma C_f \arrow[r] & \Sigma^2 X \arrow[r] & \Sigma^2 Y \arrow[r] & \Sigma^2 C_f \arrow[r] & \Sigma^3 X \arrow[r] & \cdots
\end{tikzcd}
\end{center}  
where $C_f \to \Sigma X$ is given by the projection map.
\end{definition}

\begin{definition}
An exact sequence of pointed sets,
\begin{center}
\begin{tikzcd}[column sep = small, row sep = large]
(X, x_0) \arrow[r, "f"] & (Y, y_0) \arrow[r, "g"] & (Z, z_0)
\end{tikzcd}
\end{center} 
where $\ker{g} = \Im{f}$ where $\ker{g} = \{y \in Y \mid g(y) = z_0\}$.
\end{definition}

\begin{theorem}
For any $Z \in \Ob{\pTop}$ the following sequence is exact,
\begin{center}
\begin{tikzcd}[column sep = small, row sep = large]
\cdots \arrow[r] & \left< \Sigma Y, Z \right> \arrow[r] & \left< \Sigma X, Z \right> \arrow[r] & \left< C_f, Z \right> \arrow[r] & \left< Y, Z \right> \arrow[r] & \left< X, Z \right>
\end{tikzcd}
\end{center}  
\end{theorem}

\begin{definition}
The functor $-\Omega : \pTop \to \pTop$ is given by $-\Omega (X) = \Omega X$ and given $f : X \to Y$ then $- \Omega (f) = g : \Omega X \to \Omega Y$ is given by $h(\gamma)(t) = f \circ \gamma(1 - t)$.  
\end{definition}

\begin{definition}
Given a map $f : X \to Y$, let $\varpi : \Omega Y \to N_f$ be the projection given by $\pi(\gamma) = (x_0, \gamma)$. Applying the functor $-\Omega$, we obtain the fiber sequence,
\begin{center}
\begin{tikzcd}[column sep = small, row sep = large]
\cdots \arrow[r] & \Omega^3 X \arrow[r] & \Omega^2 N_f \arrow[r] & \Omega^2 X \arrow[r] & \Omega^2 Y \arrow[r] & \Omega N_f \arrow[r] & \Omega X \arrow[r] & \Omega Y \arrow[r] & N_f \arrow[r] & X \arrow[r] & Y 
\end{tikzcd}
\end{center} 
\end{definition}

\begin{definition}
Let $p : (E, e) \to (B, b)$ be a fibration of based spaces. Then the fiber of $p$ is the subspace $F = p^{-1}(b)$. To denote this situaton, write,
\begin{center}
\begin{tikzcd}
F \arrow[r, "\iota", hook] & E \arrow[r, "p"] & B
\end{tikzcd}
\end{center}
\end{definition}

\begin{lemma}
Let \begin{center}
\begin{tikzcd}
F \arrow[r, "\iota", hook] & E \arrow[r, "p"] & B
\end{tikzcd}
\end{center}
be a fibration then inclusion $\phi : F \to N_p$ is a homotopy equivalence.  
\end{lemma}

\begin{proof} \label{fibrationgiveshomotopyfiber}
Let $p : (E, e_0) \to (B, b_0)$ be a pointed fibration. The fiber of $p$ is the subspace $F = p^{-1}(b_0)$. Then, define the map $\phi : F \to N_p$ by $\phi(x) = (x, e_{b_0})$ where $e_{b_0}$ is the constant loop at $b_0$. This map is well-defined because $x \in F = p^{-1}(b_0)$ so $p(x) = b_0 = e_{b_0}(0)$. Now, the projection $\pi_1 : N_p \to E$ is given by $\pi_1(x, \gamma) = x$. Therefore, $\pi_1 \circ \phi(x) = \pi_1(x, e_{b_0}) = x$ so $\pi_1 \circ \phi = \id_F$. However, $\phi \circ \pi_1(x, \gamma) = \phi(x) = (x, e_{b_0})$. Define the homotopy $H : N_p \times I \to N_p$ by $H(x, \gamma, t) = (x, \gamma_t)$ where $\gamma_t(s) = \gamma(1 - (1 - r) t)$. Thus, $\gamma_0(r) = \gamma(1) = b_0$ and $\gamma_1(r) = \gamma(r)$. Therefore, $H(x, \gamma, 0) = (x, \gamma_0) = (x, e_{b_0}) = \phi \circ \pi_1(x, \gamma)$ and $H(x, \gamma, 1) = (x, \gamma_1) = (x, \gamma)$. Thus, $H$ is a homotopy between $\phi \circ \pi_1$ and $\id_{N_p}$ so $\phi$ is a homotopy equivalence. 
\end{proof}

\begin{lemma} \label{mappingcoclyinderhasfibration}
Let $f : X \to Y$ be a map of pointed spaces and let $\pi_1 : N_f \to X$ be the projection. Then $\pi_1$ is a fibration and the inclusion $\phi : \Omega Y \to N_{\pi_1}$ given by $\phi(\gamma) = (x_0, \gamma, e_{x_0})$ is a homotopy equivalence.
\end{lemma}

\begin{proof}
Let $f : X \to Y$ be a map of pointed spaces. Consider the map $\pi_1 : N_f \to X$ given by $\pi_1(x, \gamma) = x$. Take any space $Z$ and maps $g : Z \to N_f$ and $h : Z \times I \to X$ such that the following diagram commutes,
\begin{center}
\begin{tikzcd}[column sep = huge, row sep = huge]
Z \arrow[d, hook, "\iota"] \arrow[r, "g"] \arrow[r, dashed] & N_f \arrow[d, "\pi_1"] \arrow[r, "\pi_2"] & Y^I \arrow[d, "ev_0"] \\
Z \times I \arrow[u, "\pi_Z", bend left] \arrow[r, "h"] \arrow[ru, dashed, "\tilde{h}"] & X \arrow[r, "f"] & Y
\end{tikzcd}
\end{center}
There are maps $h : Z \times I \to X$ and $\pi_2 \circ g \circ \pi_Z : Z \times I \to Y^I$. Therefore, by the universal property of the pullback, there exists a unique map $\tilde{h} : Z \times I \to N_f$ which commutes with the diagram. Therefore, $\pi_1 \circ \tilde{h} = h$. Furthermore, $\tilde{h} \circ \iota : Z \to N_f$ and $\pi_1 \circ \tilde{h} \circ \iota = h \circ \iota = \pi_1 \circ g$. Also, $\pi_2 \circ \tilde{h} \circ \iota = \pi_2 \circ g \circ \pi_Z \circ \iota = \pi_2 \circ g$. However, by the universal property of the pullback, $g$ is the unique map $Z \to N_f$ satisfying this property under the projections. Therefore, $\tilde{h} \circ \iota = g$. Thus, $\tilde{h}$ is a lift of $h$ at $g$ so $\pi_1$ is a fibration. \bigskip \\
The map $\pi_1 : N_f \to X$ is a fibration. Thus, take, $\phi : F \to N_{\pi_1}$, the natural inclusion on the fiber $F = \pi_1^{-1}(x_0)$ which is given by $\phi(x_0, \gamma) = (x_0, \gamma, e_{x_0})$ where $(x_0, \gamma) \in \pi_1^{-1}(x_0)$ so $f(x_0) = \gamma(0) = y_0$. However, $Y^I$ is the space of based loops (with $I$ based at $1$) so $\gamma(1) = y_0$. Therefore, $\gamma$ is a loop so $F \cong \Omega Y$ by $(x_0, \gamma, e_{x_0}) \mapsto \gamma$. Thus, $\phi$ can be viewed as a map $\phi : \Omega Y \to N_{\pi_1}$. However,by the lemma above, $\phi : F \to N_{\pi_1}$ is a homotopy equivalence. Therefore, $\phi : \Omega Y \to N_\pi$ is a homotopy equivalence. 
\end{proof}

\begin{theorem} \label{homexactpuppe}
For any $Z \in \Ob{\pTop}$ the following sequence is exact,
\begin{center}
\begin{tikzcd}[column sep = small, row sep = large]
\cdots \arrow[r] & \left< Z, \Omega N_f \right> \arrow[r] & \left< Z, \Omega X \right>  \arrow[r] & \left< Z, \Omega Y \right>  \arrow[r] & \left< Z, N_f \right> \arrow[r] & \left< Z, X\right> \arrow[r] & \left< Z, Y \right> 
\end{tikzcd}
\end{center}  
\end{theorem}

\begin{proof}
First, check exactness at the last segment. Consider, the pushout diagram of $N_f = X \times_Y \Hom{I}{Y}$,
 \begin{center}
\begin{tikzcd}[column sep = large, row sep = large]
Z \arrow[rrd, "\hat{h}", bend left] \arrow[rdd, "g", bend right] \arrow[rd, dashed, "F"] & & \\
& N_f \arrow[r] \arrow[d, "\pi_1"] & Y^I \arrow[d, "ev_0"] \\
& X \arrow[r, "f"] & Y  
\end{tikzcd}
\end{center} 
If $g : Z \to X$ is in the kernel of $f$ then there exists a homotopy, $h : Z \times I \to Y$ such that $h(z, 1) = y_0$ and $h(z, 0) = f \circ g(z)$. Therefore, by the adjuction relation, there exist maps $g$ and $\hat{h}$ making the square commute. Therefore, there exists a unique map $F : Z \to N_f$ makeing the diagram commute. In particular, $\pi_1 \circ F = g$ so $g$ is in the image of $\pi_1$.\bigskip \\
Conversely, if $g \in \Im{\pi_1}$ then there exists a map $F : Z \to N_f$ such that $\pi_1 \circ F = g$. Take $\hat{h} = \pi_2 \circ F$ then $ev_0 \circ \hat{h} F = ev_0 \circ \pi_2 \circ F = f \circ \pi_1 \circ F = f \circ g$. By adjunction, there is a map $h : Z \times I \to Y$. Now, $h(z,0) = g(f(x))$ and $h(z, 1) = y_0$ because $Y^I$ is the space of based paths. Thus, $\ker{f} = \Im{\pi_1}$. \bigskip \\
Now, consider the diagram, 
\begin{center}
\begin{tikzcd}[column sep = large, row sep = large]
\cdots \arrow[r] & \Omega N_f \arrow[r] & \Omega X  \arrow[rd]  \arrow[r] & \Omega Y \arrow[d, "\varphi"] \arrow[r, "\varpi"] & N_f  \arrow[r, "\pi"] & X \arrow[r] & Y \\
& & & N_\pi \arrow[ru, "\pi_1"] & & &
\end{tikzcd}
\end{center}   
where the map $\phi : \Omega Y \to N_\pi$ is given by $\phi(\gamma) = (x_0, \gamma, e_{x_0})$. By Lemma \ref{mappingcoclyinderhasfibration}, $\phi$ is a homotopy equivalence. Furthermore, the diagram commutes up to homotopy. Therefore, under the functor $\left< Z, - \right>$ this diagram commutes and $\phi$ is an isomorphism. The map $\pi(x, \gamma) = x$ so $\ker{\pi} = \{(x_0, \gamma) \mid \gamma(0) = y_0\}$. However, $\Im{\varpi} = \Im{\pi_1 \circ \varphi}$ and $\pi_1 \circ \varphi(\gamma) = \pi_1(x_0, \gamma, e_{x_0}) = (x_0, \gamma)$ where $\gamma \in \Omega Y$ so $\gamma(0) = y_0$. Thus, $\ker{\pi} = \Im{\varpi}$ so the sequence is exact at $N_f$. (COMPLETE THIS PROOF) 
\end{proof}

\begin{definition}
Given a pointed pair $(X, A, a_0)$ the space $(X, A)^I = \{ \gamma \in X^I \mid \gamma(0) \in A \}$ where by convention the based interval is $(I, 1)$ so $\gamma(1) = a_0$. 
\end{definition}

\begin{definition}
The relative homotopy group, $\pi_n(X, A) = \pi_{n - 1}((X, A)^I)$ for $n \ge 1$. 
\end{definition}

\begin{lemma}
If $(X, A, a_0)$ is a based pair then $(X, A)^I \cong N_\iota$ where $\iota : A \to X$ is the inclusion.
\end{lemma}

\begin{proof}
Let $F : (X, A)^I \to N_\iota$ be given by $F(\gamma) = (\gamma(0), \gamma)$. This is well-defined because $\iota \circ \gamma(0) = \gamma(0)$ because $\gamma(0) \in A$. Furthermore, $\pi_2 \circ F(\gamma) = \pi_2(\gamma(0), \gamma) = \gamma$ and $F \circ \pi_2(x, \gamma) = F(\gamma) = (\gamma(0), \gamma)$ but $(x, \gamma) \in N_\iota$ so $\gamma(0) = \iota(x) = x$ and thus $F \circ \pi_2(x, \gamma) = (\gamma(0), \gamma) = (x, \gamma)$.  
\end{proof}

\begin{lemma}[Five Lemma]
If the following diagram of groups commutes,
\begin{center}
\begin{tikzcd}[column sep = large, row sep = large]
A \arrow[r] \arrow[d, "\sim"] & B \arrow[r]\arrow[d, "\sim"] & C \arrow[r] \arrow[d, "k"] & D \arrow[r] \arrow[d, "\sim"] & E \arrow[d, "\sim"] \\
A' \arrow[r] & B' \arrow[r] & C' \arrow[r] & D' \arrow[r] & E' \\
\end{tikzcd}
\end{center}  
with exact rows then $k$ is an isomorphism.  
\end{lemma}

\begin{proof}
Diagram chase. 
\end{proof}

\begin{proposition}
Let $(X, A, a_0)$ be a pointed pair. Then, the sequence,
\begin{center}
\begin{tikzcd}
\cdots \arrow[r] & \pi_2(X, A) \arrow[r] & \pi_1(A) \arrow[r] & \pi_1(X) \arrow[r] & \pi_1(X, A) \arrow[r] & \pi_0(A) \arrow[r] & \pi_0(X)
\end{tikzcd}
\end{center} 
is long exact.
\end{proposition}

\begin{proof}
Consider the fiber sequence defined by the map $\iota : A \hookrightarrow X$, 
\begin{center}
\begin{tikzcd}
\cdots \arrow[r] & \Omega^2 X \arrow[r] & \Omega N_f \arrow[r] & \Omega A \arrow[r] & \Omega X \arrow[r] & N_f \arrow[r] & A \arrow[r] & X
\end{tikzcd}
\end{center}
and apply Theorem \ref{homexactpuppe} with $Z = S^0$. Then, then, since $\homspace{S^0}{\Omega^n Z} = \pi_n(Z)$ we have the long exact sequence,
\begin{center}
\begin{tikzcd}
\cdots \arrow[r] & \pi_1(N_\iota) \arrow[r] & \pi_1(A) \arrow[r] & \pi_1(X) \arrow[r] & \pi_0(N_\iota) \arrow[r] & \pi_0(A) \arrow[r] & \pi_0(X)
\end{tikzcd}
\end{center}
However, $N_\iota = (X, A)^I$ so $\pi_n(N_\iota) = \pi_n((X, A)^I) = \pi_{n+1}(X, A)$ and thus the result holds.
\end{proof}



\begin{theorem}
Let 
\begin{center}
\begin{tikzcd}
F \arrow[r, "\iota", hook] & E \arrow[r, "p"] & B
\end{tikzcd}
\end{center}
be a fibration. Then, the following sequence is long exact,
\begin{center}
\begin{tikzcd}
\cdots \arrow[r] & \pi_3(F) \arrow[r] & \pi_3(E) \arrow[r] & \pi_3(B) \arrow[r] & \pi_2(F) \arrow[r] & \pi_2(E) \arrow[draw=none]{d}[name=Z, shape=coordinate]{} \arrow[r] & \pi_2(B)
\arrow[dlllll,
rounded corners, crossing over,
to path={ -- ([xshift=2ex]\tikztostart.east)
|- (Z) [near end]\tikztonodes
-| ([xshift=-2ex]\tikztotarget.west)
-- (\tikztotarget)}]
\\ 
& \pi_1(F) \arrow[r] & \pi_1(E) \arrow[r] & \pi_1(B) \arrow[r] & \pi_0(F) \arrow[r] & \pi_0(E) \arrow[r] & \pi_0(B)
\end{tikzcd}
\end{center}
\end{theorem}

\begin{proof}
Consider the diagram,
\begin{center}
\hspace*{-1cm}\vbox{
\begin{tikzcd}
\Omega^2 F \arrow[r] \arrow[d, "(-\Omega)^2 \phi"] & \Omega^2 E \arrow[r] \arrow[d, "\id"] & \Omega N_\iota \arrow[r] \arrow[d, "-\Omega p"] & \Omega F \arrow[r] \arrow[d, "-\Omega \phi"] & \Omega E \arrow[r] \arrow[d, "\id"] & N_\iota \arrow[r] \arrow[d, "p"] & F \arrow[r, hook, "\iota"] \arrow[d, "\phi"] & E \arrow[r, "p"] \arrow[d, "\id"] & B \arrow[d, "\id"]
\\
\Omega^2 N_p \arrow[r] & \Omega^2 E \arrow[r] & \Omega^2 B \arrow[r] & \Omega N_p \arrow[r] & \Omega E \arrow[r] & \Omega B \arrow[r] & N_p \arrow[r, "\iota"] & E \arrow[r, "p"] & B 
\end{tikzcd}}
\end{center}
By Lemma \ref{fibrationgiveshomotopyfiber}, $\phi$ is a homotopy equivalence so $(-\Omega)^n \phi$ is  a homotopy equivalence. Furthermore, the squares commute up to homotopy (JUSTIFY THIS). By Theorem \ref{homexactpuppe}, applying the functor $\homspace{S^0}{-}$ which factors through $\hTop$ we get a commutative diagram with exact rows,
\begin{center}
\hspace*{-1cm}\vbox{
\begin{tikzcd}[column sep = small]
\pi_2(F) \arrow[r] \arrow[d, "(-\Omega)^2 \phi"] & \pi_2(E) \arrow[r] \arrow[d, "\id"] & \pi_1(N_\iota) \arrow[r] \arrow[d, "-\Omega p"] & \pi_1(F) \arrow[r] \arrow[d, "-\Omega \phi"] & \pi_1(E) \arrow[r] \arrow[d, "\id"] & \pi_0(N_\iota) \arrow[r] \arrow[d, "p"] & \pi_0(F) \arrow[r, "\iota"] \arrow[d, "\phi"] & \pi_0(E) \arrow[r, "p"] \arrow[d, "\id"] & \pi_0(B) \arrow[d, "\id"]
\\
\pi_2(N_p) \arrow[r] & \pi_2(E) \arrow[r] & \pi_2(B) \arrow[r] & \pi_1(N_p) \arrow[r] & \pi_1(E) \arrow[r] & \pi_1(B) \arrow[r] & \pi_0(N_p) \arrow[r, "\iota"] & \pi_0(E) \arrow[r, "p"] & \pi_0(B) 
\end{tikzcd}}
\end{center}
Because $(-\Omega)^n \phi$ is a homotopy equivalence, it is an isomorphism of homotopy groups. Therefore, by the Five Lemma, $(-\Omega)^n p$ is also an isomorphism since the diagram,
\begin{center}
\begin{tikzcd}[column sep = small]
\pi_{n+1}(F) \arrow[r] \arrow[d, "(-\Omega)^{n+1} \phi"] & \pi_{n+1}(E) \arrow[r] \arrow[d, "\id"] & \pi_{n+1}(E, F) \arrow[r] \arrow[d, "(-\Omega)^n p"] & \pi_n(F) \arrow[r] \arrow[d, "(-\Omega)^n \phi"] & \pi_n(E) \arrow[d, "\id"] 
\\
\pi_{n+1}(N_p) \arrow[r] & \pi_{n+1}(E) \arrow[r] & \pi_{n+1}(B) \arrow[r] & \pi_n(N_p) \arrow[r] & \pi_n(E) 
\end{tikzcd}
\end{center}
commutes and the downward maps (execpt the center one) are all isomorphisms and thus, $\pi_{n+1}(E, F) \cong \pi_{n+1}(B)$. Since each downward map is an isomorphism and the diagram commutes, we may weave the sequences together via the isomorphisms, 
\begin{center}
\hspace*{-1cm}\vbox{
\begin{tikzcd}[column sep = small]
\pi_2(F) \arrow[r] & \pi_2(E) \arrow[r] & \pi_2(B) \arrow[r] & \pi_1(F) \arrow[r] & \pi_1(E) \arrow[r] & \pi_1(B) \arrow[r] & \pi_0(F) \arrow[r, "\iota"] & \pi_0(E) \arrow[r, "p"] & \pi_0(B) 
\end{tikzcd}}
\end{center}
to obtain a new long exact sequence. 
\end{proof}

\begin{theorem}[Cellular Approximation]
If $f : X \to Y$ is a map of CW complexes then $f$ is homotopic to a cellular map i.e. it descends to maps on the $n$-skeletons $X^n \to Y^n$. 
\end{theorem}

\begin{corollary}
$\pi_n(S^m) = 0$ if $n < m$ 
\end{corollary}

\begin{proof}
Any map $S^n \to S^m$ is homotopic by cellular approximation to a map on the $k$-skeletons. However, the $n$-sphere is a CW complex with $k$-skeleton a single zero-cell for $k < n$ and one $n$-cell in the $n$-skeleton. Thus, if $n < m$ then the map on $n$ skeletons must send the entire $S^n$ to the $n$-skeleton on $S^m$ which is a single point. Therefore, the map is homotopic to a constant map. Thus, $\pi_n(S^m) = 0$.
\end{proof}

\begin{proposition}
$K(\Z, 1) \simeq S^1$.
\end{proposition}

\begin{definition}
Given a group $G$, an Eilenberg-MacLane space $K(G, n)$ has the property that $\pi_n(K(G, n)) \cong G$ and $\pi_m(K(G,n)) = 0$ for $m \neq n$. 
\end{definition}

\begin{theorem}
The Eilenberg-MacLane space $K(G, n)$ always exists as a CW complex if $n \le 2$ or $G$ is abelian. Furthermore, $K(G, n)$ is unique as a CW complex up to homotopy.   
\end{theorem}

\begin{proof}
Let $p : \R \to S^1$ be the covering map of the universal cover. Then, for $n \ge 2$ the map $p_* : \pi_n(\R) \to \pi_n(S^1)$ is a isomorphism. However, $\R$ is contractible so $\pi_n(\R) = 0$ and thus $\pi_n(S^1) = 0$. 
\end{proof}


\begin{example}[Hopf Fibration]
There exist nontrivial fiber bundles, \\
\begin{align*}
S^0 \hookrightarrow & S^1 \rightarrow S^1 \\
S^1 \hookrightarrow & S^3 \rightarrow S^2 \\
S^3 \hookrightarrow & S^7 \rightarrow S^4 \\
S^7 \hookrightarrow & S^{15} \rightarrow S^8 
\end{align*}
These fibrations induce long exact sequences of homotopy groups. For example, for the Hopf Fibration $S^1 \hookrightarrow S^3 \rightarrow S^2$,
\begin{center}
\begin{tikzcd}
\cdots \arrow[r] & \pi_4(S^1) \arrow[r] & \pi_4(S^3) \arrow[r] & \pi_4(S^2) \arrow[r] & \pi_3(S^1) \arrow[r] & \pi_3(S^3) \arrow[draw=none]{d}[name=Z, shape=coordinate]{} \arrow[r] & \pi_3(S^2)
\arrow[dlllll,
rounded corners, crossing over,
to path={ -- ([xshift=2ex]\tikztostart.east)
|- (Z) [near end]\tikztonodes
-| ([xshift=-2ex]\tikztotarget.west)
-- (\tikztotarget)}]
\\ 
& \pi_2(S^1) \arrow[r] & \pi_2(S^3) \arrow[r] & \pi_2(S^2) \arrow[r] & \pi_1(S^1) \arrow[r] & \pi_1(S^3) \arrow[r] & \pi_1(S^2) \arrow[r] & 0
\end{tikzcd}
\end{center}
is long exact where I have set the $\pi_0$ sets to zero because non-trivial spheres are connected. Plugging in the groups that we know,
\begin{center}
\begin{tikzcd}
\cdots \arrow[r] & 0 \arrow[r] & \pi_4(S^3) \arrow[r] & \pi_4(S^2) \arrow[r] & 0 \arrow[r] & \pi_3(S^3) \arrow[draw=none]{d}[name=Z, shape=coordinate]{} \arrow[r] & \pi_3(S^2)
\arrow[dlllll,
rounded corners, crossing over,
to path={ -- ([xshift=2ex]\tikztostart.east)
|- (Z) [near end]\tikztonodes
-| ([xshift=-2ex]\tikztotarget.west)
-- (\tikztotarget)}]
\\ 
& 0 \arrow[r] & 0 \arrow[r] & \pi_2(S^2) \arrow[r] & \Z \arrow[r] & 0 \arrow[r] & 0 \arrow[r] & 0
\end{tikzcd}
\end{center}
However, if the sequence,
\begin{center}
\begin{tikzcd}
0 \arrow[r] & A \arrow[r, "f"] & B \arrow[r] & 0
\end{tikzcd}
\end{center}
is exact then $\ker{f} = \Im{0} = 0$ and $\Im{f} = \ker{0} = B$ so $f$ is an isomorphism. Thus, $A \cong B$. Therefore, from the long exact sequence, $\pi_n(S^3) \cong \pi_n(S^2)$ for $n \ge 3$ and $\pi_2(S^2) \cong \Z$. We already know that $\pi_1(S^3) = \pi_2(S^3) = 0$. \bigskip \\
The next higher fibration allow us to calculate a few more explicit homotopy groups. The section of the long exact sequence,
\begin{center}
\begin{tikzcd}
\pi_7(S^4)\arrow[r] & \pi_6(S^3) \arrow[r] & \pi_6(S^7) \arrow[r] & \pi_6(S^4) \arrow[r] & \pi_5(S^3) \arrow[r] & \pi_5(S^7) \arrow[draw=none]{d}[name=Z, shape=coordinate]{} \arrow[r] & \pi_5(S^4)
\arrow[dlllll,
rounded corners, crossing over,
to path={ -- ([xshift=2ex]\tikztostart.east)
|- (Z) [near end]\tikztonodes
-| ([xshift=-2ex]\tikztotarget.west)
-- (\tikztotarget)}]
\\ 
& \pi_4(S^3) \arrow[r] & \pi_4(S^7) \arrow[r] & \pi_4(S^4) \arrow[r] & \pi_3(S^3) \arrow[r] & \pi_3(S^7) \arrow[r] & \pi_3(S^4) 
\end{tikzcd}
\end{center}
when plugged into gives,
\begin{center}
\begin{tikzcd}
\pi_7(S^4)\arrow[r] & \pi_6(S^3) \arrow[r] & 0 \arrow[r] & \pi_6(S^4) \arrow[r] & \pi_5(S^3) \arrow[r] & 0 \arrow[draw=none]{d}[name=Z, shape=coordinate]{} \arrow[r] & \pi_5(S^4)
\arrow[dlllll,
rounded corners, crossing over,
to path={ -- ([xshift=2ex]\tikztostart.east)
|- (Z) [near end]\tikztonodes
-| ([xshift=-2ex]\tikztotarget.west)
-- (\tikztotarget)}]
\\ 
& \pi_4(S^3) \arrow[r] & 0 \arrow[r] & \pi_4(S^4) \arrow[r] & \pi_3(S^3) \arrow[r] & 0 \arrow[r] & 0
\end{tikzcd}
\end{center}
Therefore, $\pi_4(S^4) \cong \pi_3(S^3)$ and $\pi_5(S^4) \cong \pi_4(S^3)$ and $\pi_6(S^4) \cong \pi_5(S^3)$ and $\pi_7(S^4)$ surjects onto $\pi_6(S^3)$. Using a similar argument on the fibration $S^7 \hookrightarrow S^{15} \to S^8$, we conclude that, $\pi_{n}(S^8) \cong \pi_{n - 1}(S^7)$ for $8 \le n \le 15$ and $\pi_{15}(S^8) \twoheadrightarrow \pi_{14}(S^7)$.
\end{example}

\newcommand{\cp}{\mathbb{CP}}

\begin{example}
The covering map gives a fibration, $S^1 \hookrightarrow S^{2n + 1} \rightarrow \cp^n$ for $n \ge 1$ and thus an exact sequence,
\begin{center}
\begin{tikzcd}[column sep = small]
\cdots \arrow[r] & \pi_4(S^1) \arrow[r] & \pi_4(S^{2n + 1}) \arrow[r] & \pi_4(\cp^n) \arrow[r] & \pi_3(S^1) \arrow[r] & \pi_3(S^{2n + 1}) \arrow[draw=none]{d}[name=Z, shape=coordinate]{} \arrow[r] & \pi_3(\cp^n)
\arrow[dlllll,
rounded corners, crossing over,
to path={ -- ([xshift=2ex]\tikztostart.east)
|- (Z) [near end]\tikztonodes
-| ([xshift=-2ex]\tikztotarget.west)
-- (\tikztotarget)}]
\\ 
& \pi_2(S^1) \arrow[r] & \pi_2(S^{2n + 1}) \arrow[r] & \pi_2(\cp^n) \arrow[r] & \pi_1(S^1) \arrow[r] & \pi_1(S^{2n + 1}) \arrow[r] & \pi_1(\cp^n) \arrow[r] & 0
\end{tikzcd}
\end{center}
is long exact where I have set the $\pi_0$ sets to zero because non-trivial spheres are connected. Plugging in the groups that we know,
\begin{center}
\begin{tikzcd}[column sep = small]
\cdots \arrow[r] & 0 \arrow[r] & \pi_4(S^{2n + 1}) \arrow[r] & \pi_4(\cp^n) \arrow[r] & 0 \arrow[r] & \pi_3(S^{2n + 1}) \arrow[draw=none]{d}[name=Z, shape=coordinate]{} \arrow[r] & \pi_3(\cp^n)
\arrow[dlllll,
rounded corners, crossing over,
to path={ -- ([xshift=2ex]\tikztostart.east)
|- (Z) [near end]\tikztonodes
-| ([xshift=-2ex]\tikztotarget.west)
-- (\tikztotarget)}]
\\ 
& 0 \arrow[r] & 0 \arrow[r] & \pi_2(\cp^n) \arrow[r] & \Z \arrow[r] & 0 \arrow[r] & \pi_1(\cp^n) \arrow[r] & 0
\end{tikzcd}
\end{center}
Therefore, $\pi_1(\cp^n) = 0$ and $\pi_2(\cp^n) \cong \Z$ and for $m \ge 3$, $\pi_m(S^{2n + 1}) \cong \pi_m(\cp^n)$. This reduces to the special case of the Hopf fibration for $n = 1$ in which case $\cp^1 \cong S^2$. 
This fibration extends to $n = \infty$ in which case, $S^1 \hookrightarrow S^\infty \to \cp^\infty$ is a fibration. However, $S^\infty$ is contractable so $\pi_m(\cp^\infty) \cong \pi_m(S^\infty) = 0$ for $m \ge 3$ and $\pi_1(\cp^\infty) = 0$ and $\pi_2(\cp^\infty) \cong \Z$. Therefore, $K(\Z, 2) \simeq \cp^\infty$. 
\end{example}


\begin{definition}
A map $f : X \to Y$ is an $n$-equivalence if for any choice of basepoint, $f_* : \pi_i(X, x_0) \to \pi_i(X, y_0)$ is an isomorphism for $i < n$ and a surjection for $i = n$. If $f$ is an $n$-equivalence for all $n$ then $f$ is a weak homotopy equivalence. 
\end{definition} 

\begin{theorem}[Whitehead]
An $n$-equivalence between connected CW complexes of dimension less than $n$ is a homotopy equivalnece. 
\end{theorem}

\begin{proof}
By cellular approximation, we can assume (up to homotopy) that $f$ is a cellular map. Now, we decompose,
\begin{center}
\begin{tikzcd}[row sep = large]
X \arrow[r, "\iota", hook] & M_f  \arrow[r, "r"] & Y
\end{tikzcd}
\end{center}
Because $r$ is a homotopy equivalence we have $f_* = (r \circ \iota)_* = r_* \circ \iota_*$ but $r_*$ is an isomorphism so $\iota_*$ must also be a $n$-equivalence. Now, we rename $(X, A) = (M_f, X)$ which is a CW-pair. Therefore, the following sequence is long exact,
\begin{center}
\begin{tikzcd}
\pi_{n+1}(X) \arrow[r] & \pi_{n}(X, A) \arrow[r] & \pi_{n}(A) \arrow[draw=none]{d}[name=Z, shape=coordinate]{} \arrow[r, two heads] & \pi_{n}(X) \arrow[r] & \pi_{n-1}(X, A) 
\arrow[dllll,
rounded corners, crossing over,
to path={ -- ([xshift=2ex]\tikztostart.east)
|- (Z) [near end]\tikztonodes
-| ([xshift=-2ex]\tikztotarget.west)
-- (\tikztotarget)}]
& &
\\ 
\pi_{n-1}(A) \arrow[r, "\sim"] & \pi_{n-1}(X) \arrow[r] & \pi_{n-2}(X, A) \arrow[r] & \pi_{n-2}(A) \arrow[r, "\sim"] & \pi_{n-2}(X) \arrow[r] & \cdots
\end{tikzcd}
\end{center}
Since the map into $\pi_i(X)$ is a surjection, by exactness, the map into $\pi_{n-1}(X, A)$ has total kernel i.e. is the zero map. Similarly, the map $\pi_{i-1}(A) \to \pi_{i-1}(X)$ is an injection so it has trivial kernel. Therefore, by exactness, the map $\pi_{i-1}(X, A) \to \pi_{i-1}(A)$ is the zero map. However, this sequence is exact at $\pi_{i - 1}(X, A)$ but the image of the map in is zero (since it is the zero map) but the kernel of the map out is the entire group (since it is also zero). Thus, $\pi_{i-1}(X, A) = 0$. We now need a lemma to complete the proof.  
\end{proof}

\begin{lemma}
Suppose $f : (X, A) \to (Y, B)$ is a map on a CW pair such that for any $i$ such that $X-A$ has an $i$-cell then $\pi_i(Y, B) = 0$ at any basepoint then $f$ is homotopic rel $A$ to a map $f' : X \to B$. 
\end{lemma}

\begin{proof}
(WORK IN PROGRESS)
\end{proof}

\begin{theorem}[Whitehead]
A weak homotopy equivalence of connected CW complexes is a homotopy equivalence. 
\end{theorem}

\begin{theorem}[CW approximation]
All spaces are weakly homotopy equivalent to a CW complex. Furthermore, if $X$ is $n$-simply connected then we can choose the CW complex to have a unique $0$-cell and no $q$-cells for $0 < q \le n$. 
\end{theorem}

\begin{proof}
We will first consider each path-component. Assume that $X$ is path-connected. Let $Z_1$ be a wedge of spheres $S^q$ fpr each $(q,j)$ where $j : S^q \to X$ represents and element of $\pi_q(X)$. The map $Z_1 \xrightarrow{\gamma_1} X$ is given by the $j's$ and $\gamma_1$ induces surjections on all $\pi_q(X)$ so $\gamma_1$ is a $1$-equivalence. Suppose we have constructed $\gamma_n : Z_n \to X$ such that $\gamma_n$ induces surjections of $\pi_q$ and isomorphisms on $\pi_q$ for $q \le n - 1$. Construct $Z_{n+1}$ by attaching cells as follows: for all $[f], [g] \in \pi_n(Z_n)$ such that $[f] \neq [g]$ but $[\gamma_n \circ f] = [\gamma_n \circ g]$, attach the reduced cyclinder $S^n \times I /(\{e\} \times I)$ via $f$ at one end and $g$ at the other. Choose a map $\gamma_{n + 1} : Z_{n+1} \to X$ by picking a homotopy $f$ to $g$. Cellular approximation ensures that attaching $n+1$ cells can be done without affecting the $\pi_q$ for $q < n$. Furthermore, the map $\gamma_{n+1} : Z_{n+1} \to X$ is still a surjection on $\pi_q$ for all $q$. However, no $f$ and $g$ are homotopic so $\pi_n(Z_{n+1}) \cong \pi_n(X)$. Therefore, the direct limit,
\[ \lim\limits_{\to} Z_n \simeq X\]
is weak homotopy equivalent to $X$.  
\end{proof}

\begin{definition}
A pair $(X, A)$ is $n$-connected if each path component intersects with $A$ and $\pi_n(X, A, x_0) = 0$ for all $x_0 \in A$ and $1 \le i \le n$. 
\end{definition}

\begin{corollary}
Let $(X, A)$ be an $n$-connected CW-pair then there exists a CW-pair $(Z, A)$ such that $(X, A)$ is homotopy equivalence $\mathrm{rel}\: A$ to $(X, A)$ and $Z \setminus A$ has no $i$-cells with $i \le n$.  
\end{corollary}

\begin{theorem}
There is a functor $\Gamma : \hTop \to \hTop$ taking a space to $X$ to a CW complex $\Gamma(X)$ and there exists a natural transformation $\gamma : \Gamma \to \id_{\hTop}$ such that each $\gamma_X : \Gamma(X) \to X$ is a weak homotopy equivalence. 
\end{theorem}

\begin{theorem}[Excison]
Let $X$ be a CW complex and $X = A \cup B$ subcomplexes with $C = A \cap B$ connected, nonempty and suppose that $(A, C)$ is $m$-connected and $(B, C)$ is $n$-connected for $n,m \in \Z$. Then, the inclusion,
\[ \pi_i(A, C) \to \pi_i(X, B)\]
is an isomorphism if $i < m + n$ and surjective if $i = m + n$.   
\end{theorem}

\begin{corollary}[Freudenthal Suspension Theorem]
If $X$ is an $(n-1)$-connected CW complex, then the spectrum map $\Sigma : \pi_i(X) \to \pi_{i+1}(\Sigma X)$ which takes ,
\[ f \mapsto \Sigma f = f \wedge \id : S^i \wedge S^1 \cong S^{i+1} \to X \wedge S^1 \cong \Sigma X \] 
is an isomophism for $ i < 2 n - 1$ and a surjection when $i = 2 n - 1$. 
\end{corollary}

\begin{proof}
The cones, $\Sigma X = C_+ C \cup C_{-} X$ with $C_+ X \cap C_{-} X = X$. The following diagram commutes,
\begin{center}
\begin{tikzcd}
\pi_i(X) \arrow[r, "\Sigma"] \arrow[d] & \pi_{i + 1}(\Sigma X) \arrow[d]\\
\pi_{i + 1}(C_+ X, X) \arrow[r, "\iota"] & \pi_{i + 1}(\Sigma X, C_{-} X) 
\end{tikzcd}
\end{center}
where the downward maps are segments of the long exact sequence of an inclusion which are isomorphisms because $C_{+}X$ is contractable.Using excision, the map $\iota$ is a isomorphism if $i < 2 n - 1$ and a surjection if $ i = 2n - 1$. Therefore, $\Sigma$ also has these properties.   
\end{proof}

\begin{corollary}
$\pi_i(S^n) \to \pi_{i + 1}(S^{n+1})$ is an isomorphism if $i < 2n - 1$ and a surjection if $i = 2n - 1$.
\end{corollary}

\begin{corollary}
$\pi_n(S^n) \cong \Z$
\end{corollary}
\begin{proof}
Repeatedly applying the previous result since $n < 2n - 1$ for $n > 1$. 
\begin{center}
\begin{tikzcd}
\pi_2(S^2) \arrow[r, "\sim"] & \pi_3(S^3) \arrow[r, "\sim"] & \pi_4(S^4) \arrow[r, "\sim"] & \pi_5(S^5) \arrow[r, "\sim"] & \cdots
\end{tikzcd}
\end{center}
Furthermore, since we know that $\pi_2(S^2) \cong \Z$ by the Hopf fibration, we know that $\pi_n(S^n) \cong \Z$ for all $n \ge 2$. Of course, we also know $\pi_1(S^1) \cong \Z$. 
\end{proof}

\section{Homology}


\subsection{Introduction}

Define a standard (unfilled) triangle with vertices $\alpha, \beta, \gamma$ and edges $a,b,c$. We will cook up some free abelian groups, $C_0 = \Z \alpha \oplus \Z \beta \oplus \Z \gamma$ the free group on the vertices and $C_1 = \Z a \oplus \Z b \oplus \Z c$ the free group on the edges. Now define the boundary map $\partial : C_1 \to C_0$ by $\partial b = \alpha - \gamma$ and $\partial a = \gamma - \beta$ and $\partial c = \alpha - \beta$. Then the diagram,
\begin{center}
\begin{tikzcd}
0 \arrow[r] & C_1 \arrow[r, "\partial"] & C_0 \arrow[r] & 0
\end{tikzcd}
\end{center}
is a complex meaning that the composition of any two maps is the zero map. Consider the kernel of $\partial$. Which is the set,
\[ \{ t a \oplus ub \oplus vc  \mid t(\gamma - \beta) + u(\alpha - \gamma) + v(\alpha - \beta) = 0 \} \]
which has solutions, $t = u = - v$ which is the set $\{ (1, 1, -1) \cdot t \mid t \in \Z\} \cong \Z$. We call this $H_1(C) = \ker{\partial} \cong \Z$ the first Homology group. \bigskip\\
Now consider the filled triangle labeled in the same way. Now we have a 2-cell called $A$ representing the filled triangle so $C_2 = \Z A$. Now define the boundary map $\partial_2 : C_2 \to C_1$ defined by $\partial_2 A = a + b - c$ (with some choice of orientation). Now, $H_1(C) = \ker{\partial_1}{\Im{\partial_2}} \cong (1, 1, -1) \Z / (1,1,-1)\Z = 0$.   

\subsection{Basic Definitions}

\begin{definition}
A complex is any diagram such that the composition of any two maps (if it exists) is the zero map. In particular,
\begin{center}
\begin{tikzcd}
\cdots \arrow[r, "\partial_7"] & C_6 \arrow[r, "\partial_6"] & C_5 \arrow[r, "\partial_5"] & C_4 \arrow[r, "\partial_4"] & C_3 \arrow[r, "\partial_3"] & C_2 \arrow[r, "\partial_2"] & C_1 \arrow[r, "\partial_1"] & C_0 \arrow[r, "\partial_0"] & 0 
\end{tikzcd}
\end{center}
is a complex if $\partial_{n} \circ \partial_{n + 1} = 0$ or equivalently $\Im{\partial_{n+1}} \subset \ker{\partial_n}$. \bigskip\\
We call the $C_n$ ``chains'', the $\ker{\partial_n}$ ``cycles'', and the $\Im{\partial_{n+1}}$ ``boundaries''. 
\end{definition} 

\begin{definition}
Given a complex as above, the $n^{\mathrm{th}}$ homology group is given by,
\[ H_n(C) = \ker{\partial_n} / \Im{\partial_{n+1}} \]
\end{definition}

\begin{lemma}
A sequence is exact if and only if it is a complex with trivial homology groups.
\end{lemma}

\subsection{Simplicial Homology}

\begin{definition}
The standard $n$-simplex is the subset,
\[\Delta^n = \left\{ (t_0, \cdots, t_n) \in \R^{n+1} \quad \middle| \quad \sum_{i = 0}^n t_i = 1 \quad \text{and} \quad \forall i : t_i \ge 0 \right\} \] 
We give $\Delta^n$ an orientation by ordering the vertices in the sequence defined by the order of the standard basis of $\R^{n+1}$,
\[(1,0, \cdots, 0), \quad (0,1,\cdots,0), \quad \cdots \quad (0, 0, \cdots, 1)\]
\end{definition}

\begin{definition}
An $n$-simplex is the convex hull of $n + 1$ points in $\R^m$ that do not lie in any $n$-dimensional hyperplane. 
\end{definition}

\begin{definition}
The faces of an $n$-simplex are the convex hulls of any subset with $n$ points of the simplex. There are $n+1$ faces each of which is an $n-1$-simplex. 
\end{definition}

\begin{definition}
A $\Delta$-complex $X$ is a topological space along with a collection of maps $\sigma_\alpha : \Delta^n \to X$ (where $n$ can depend on $\alpha$) subject to the constraints,
\begin{enumerate}
\item $\sigma_\alpha|_{(\Delta^n)^\circ}$ is injective and if $\alpha \neq \beta$ then $\Im{\sigma_\alpha|_{(\Delta^n)^\circ}} \cap \Im{\sigma_\beta|_{(\Delta^n)^\circ}} = \varnothing$

\item $\sigma_\alpha$ restricted to a face of $\Delta^n$ is equal to some $\sigma_\beta$ up to homoeomorphism of the domains. 

\item A set $U \subset X$ is open if and only if $\sigma_\alpha^{-1}(U)$ is open for every $\alpha$. 
\end{enumerate}
\end{definition}

\begin{definition}
Given a $\Delta$-complex $X$ define $C^\Delta_n(X)$ to be the free abelian group on all $\sigma_\alpha : \Delta^n \to X$ with $n$ fixed and define the boundary map $\partial_n  : C^\Delta_n(X) \to C^\Delta_{n-1}(X)$ by, 
\[\partial(\sigma_\alpha) = \sum_{i = 0}^n (-1)^i \sigma_\alpha|_{i^{\mathrm{th}}-\text{face}} \]

\end{definition}

\begin{lemma}
Given a $\Delta$-complex $X$ the sequence $C^\Delta(X)$ given by,
\begin{center}
\begin{tikzcd}
\cdots \arrow[r, "\partial_7"] & C^\Delta_6 \arrow[r, "\partial_6"] & C^\Delta_5 \arrow[r, "\partial_5"] & C^\Delta_4 \arrow[r, "\partial_4"] & C^\Delta_3 \arrow[r, "\partial_3"] & C^\Delta_2 \arrow[r, "\partial_2"] & C^\Delta_1 \arrow[r, "\partial_1"] & C^\Delta_0 \arrow[r, "\partial_0"] & 0 
\end{tikzcd}
\end{center}
is a complex.
\end{lemma}

\begin{proof}
\begin{align*} 
\partial_{n-1} \circ \partial_n(\sigma_\alpha) 
& = \sum_{i > j} (-1)^{j + i} (\sigma_\alpha|_{i^{\mathrm{th}}-\text{face}})|_{j^{\mathrm{th}}-\text{face}} + \sum_{i < j} (-1)^{j + i} (\sigma_\alpha|_{i^{\mathrm{th}}-\text{face}})|_{(j-1)^{\mathrm{th}}-\text{face}}
\\
& = \sum_{i > j} (-1)^{j + i} (\sigma_\alpha|_{i^{\mathrm{th}}-\text{face}})|_{j^{\mathrm{th}}-\text{face}} + \sum_{i < j} (-1)^{j + 1 + i} (\sigma_\alpha|_{i^{\mathrm{th}}-\text{face}})|_{j^{\mathrm{th}}-\text{face}}
\\
& = \sum_{i > j} (-1)^{j + i} (\sigma_\alpha|_{i^{\mathrm{th}}-\text{face}})|_{j^{\mathrm{th}}-\text{face}} - \sum_{i < j} (-1)^{j + i} (\sigma_\alpha|_{i^{\mathrm{th}}-\text{face}})|_{j^{\mathrm{th}}-\text{face}} = 0
\end{align*}
\end{proof}

\begin{definition}
Let $X$ be a $\Delta$-complex then the $n^{\mathrm{th}}$ homology group is,
\[ H^\Delta_n(X) = \ker{\partial_n}/\Im{\partial_{n + 1}} \]
which is the homology of the complex $C^\Delta(X)$. 
\end{definition}


\subsection{Singluar Homology}

\begin{definition}
A (singluar) $n$-chain on $X$ is a map $\sigma : \Delta^n \to X$. 
\end{definition}

\begin{definition}
Let $C_n(X)$ be the free abelian group of all $n$-chains. 
\end{definition}

\begin{definition}
The boundary map $\partial_n : C_n(X) \to C_{n-1}(X)$ is defined by,
\[ \partial_n(\sigma) = \sum_{i = 0}^n (-1)^i \sigma|_{i^{\mathrm{th}}-\text{face}}\]
using an identification $\Delta^n |_{\text{face}} \cong \Delta^{n-1}$ with a unique linear map preserving orientation. 
\end{definition}

\begin{definition}
The (singular) Homology groups $H_n(X)$ of $X$ are the homology groups of the chain complex,
\begin{center}
\begin{tikzcd}
\cdots \arrow[r, "\partial_7"] & C_6 \arrow[r, "\partial_6"] & C_5 \arrow[r, "\partial_5"] & C_4 \arrow[r, "\partial_4"] & C_3 \arrow[r, "\partial_3"] & C_2 \arrow[r, "\partial_2"] & C_1 \arrow[r, "\partial_1"] & C_0 \arrow[r, "\partial_0"] & 0 
\end{tikzcd}
\end{center}
\end{definition}

\begin{definition}
The category of chain complexes of abelian groups $\textbf{Ch(Ab)}$ has objects which are chain complexes of abelian groups and morphisms natural transformations between the complex diagrams. Diagramatrically, let $C$ and $D$ be chain complexes then a natural transformation $\eta : C \to D$ is a sequence of maps $\eta_n$ such that the following diagram commutes,

\begin{center}
\begin{tikzcd}
\cdots \arrow[r, "\partial_{n+3}"] & C_{n+2} \arrow[d, "\eta_{n+2}"] \arrow[r, "\partial_{n+2}"] & C_{n+1} \arrow[d, "\eta_{n+1}"]  \arrow[r, "\partial_{n+1}"] & C_{n} \arrow[d, "\eta_{n}"]  \arrow[r, "\partial_n"] & C_{n-1} \arrow[d, "\eta_{n-1}"]  \arrow[r, "\partial_{n-1}"] & C_{n-2} \arrow[d, "\eta_{n-2}"]  \arrow[r, "\partial_{n-2}"] & \cdots
\\
\cdots \arrow[r, "\partial_{n+3}"] & D_{n+2} \arrow[r, "\partial_{n+2}"] & D_{n+1} \arrow[r, "\partial_{n+1}"] & D_{n} \arrow[r, "\partial_n"] & D_{n-1} \arrow[r, "\partial_{n-1}"] & D_{n-2} \arrow[r, "\partial_{n-2}"] & \cdots
\end{tikzcd}
\end{center}
This is equivalent the condition that $\eta_{n} \circ \partial_{n + 1} = \partial_{n + 1} \circ \eta_n$ i.e. that each square commutes which is summarized by $\eta \circ \partial = \partial \circ \eta$. 
 
\end{definition}

\begin{proposition}
We have functors,
\begin{center}
\begin{tikzcd}
\Top \arrow[r, "C"] \arrow[rr, bend right, "H_n"] & \textbf{Ch(Ab)} \arrow[r, "H_n"] & \mathbf{AbGrp}
\end{tikzcd}
\end{center}
given a map $f : X \to Y$ we get a map $f_\# : C(X) \to C(Y)$ via $f_\#(\sigma) = f \circ \sigma $ extended to a linear map of free abelian groups which is a morphism of chain complexes.
\end{proposition}

\begin{proof}
\begin{align*}
f_\# \circ \partial(\sigma) & = f_\#(\partial \sigma) = f_\# \left( \sum_{i = 0}^n (-1)^i \sigma|_{i^{\mathrm{th}}-\text{face}} \right) =  \sum_{i = 0}^n (-1)^i f_\#(\sigma|_{i^{\mathrm{th}}-\text{face}}) 
\\
& = \sum_{i = 0}^n (-1)^i f \circ \sigma|_{i^{\mathrm{th}}-\text{face}} = \sum_{i = 0}^n (-1)^i (f \circ \sigma)|_{i^{\mathrm{th}}-\text{face}} 
\\
& = \sum_{i = 0}^n (-1)^i f_\#(\sigma)|_{i^{\mathrm{th}}-\text{face}} = \partial \circ f (\sigma) 
\end{align*}
Therefore $f_\# \circ \partial = \partial \circ f_\#$. 
\end{proof}

\begin{lemma}
$H_n : \textbf{Ch(Ab)} \to \mathbf{AbGrp}$ is a functor.
\end{lemma}

\begin{proof}
Given $f_\# : C \to D$ (any may between complexes). If $\alpha$ is a cycle then $\partial(\alpha) = 0$ so $\partial(f_\#(\alpha)) = f_\#(\partial(\alpha)) = f_\#(0) = 0$ so $f_\#(\alpha)$ is a cycle. Furthermore, if $\beta$ is a boundary then $\beta = \partial(\gamma)$ so $f_\#(\beta) = f_\#(\partial(\gamma)) = \partial(f_\#(\gamma))$ so $f_\#(\beta)$ is a boundary. Therefore, $f_\#$ is well-defined on homology groups by $f_\# : \alpha \Im{\partial} \mapsto f_\#(\alpha) \Im{\partial}$. 
\end{proof}

\begin{corollary}
$H_n : \Top \to \mathbf{AbGrp}$ is a functor.
\end{corollary}

\begin{proposition}
If $X$ has components $X_\alpha$ then,
\[H_n(X) = \bigoplus_{\alpha} H_n(X_\alpha) \] 
\end{proposition}

\begin{proof}
Because $\Delta^n$ is connected, any map $\sigma : \Delta^n \to X$ has a connected image as is therefore exactly some map $\sigma : \Delta^n \to X_\alpha$. Therefore we can split each chain as,
\[ C_n(X) = \bigoplus_{\alpha} C_n(X_\alpha) \]
However, $\sigma$ restricted to its faces also maps into $X_\alpha$ so the boundary map acts on each component $C_n(X_\alpha)$ seperately. Therefore since quotients and products commute, we have,
\[H_n(X) = \bigoplus_{\alpha} H_n(X_\alpha) \] 
\end{proof}

\begin{proposition}
The augmented complex,
\begin{center}
\begin{tikzcd}
\cdots \arrow[r, "\partial_6"] & C_5 \arrow[r, "\partial_5"] & C_4 \arrow[r, "\partial_4"] & C_3 \arrow[r, "\partial_3"] & C_2 \arrow[r, "\partial_2"] & C_1 \arrow[r, "\partial_1"] & C_0 \arrow[r, "\varepsilon"] & Z \arrow[r] & 0 
\end{tikzcd}
\end{center}
where,
\[ \varepsilon\left(\sum_{i = 0}^r n_i \sigma_i \right) = \sum_{i = 0}^r = n_i \] 
is a chain complex and is exact at $\Z$ if $X$ is nonempty.
\end{proposition}

\begin{proof}
Take $\varepsilon \circ \partial_1(\sigma) = \varepsilon(\sigma|_{\text{0-face}} - \sigma_{\text{1-face}}) = 0$. Thus, $\varepsilon \circ \partial_1 = 0$. Furthermore, the kernel of the zero map at $\Z$ is everything so the this sequence is a complex. \bigskip\\
Suppose that $X$ is nonempty then we have some map $\sigma : \Delta^0 \to X$. Thus, for any $n \in \Z$ the map $\varepsilon$ takes $n \sigma \mapsto n$ so $\Im{\varepsilon} = \Z = \ker{0}$ and thus the complex is exact at $\Z$. 
\end{proof}


\begin{definition}
The reduced homology of $X$, denoted $\tilde{H}_n(X)$, is the homology of the augmented complex.
\end{definition}

\begin{proposition}
For $n > 0$ we have $\tilde{H}_n(X) = H_n(X)$ and $H_0(X) = \tilde{H}_0(X) \oplus \Z$. 
\end{proposition}


\begin{proposition}
If $X$ is nonempty and path-connected then $H_0(X) \cong \Z$.
\end{proposition}

\begin{proof}
I claim that the augmented complex is exact at $C_0$. Suppose we have a cycle,
\[\varepsilon \left(\sum n_i \sigma_i \right) = \sum_{i = 0}^n n_i =  0\]
then by path-connectedness, we can pick paths $\tau_i$ from $x_i = \Im{\sigma_i}$ to $x_0 = \Im{\sigma_0}$ which are points because $\sigma_i$ is a map $\Delta^0 \to X$ which is the image of a point. Then. $I \cong \Delta^1$ so $\tau_i$ is a $1$-chain. However, $\partial(\tau_i) = \sigma_i - \sigma_0$. Consider,
\[ \partial \left(\sum_{i = 0}^r n_i \tau_i \right) = \sum_{i = 0}^r n_i \sigma_i - \sum_{i = 0}^r n_i \sigma_0 =  \sum_{i = 0}^r n_i \sigma_i\]
because the sum of $n_i$ is zero since we are concidering a chain. Therefore, we have shown that every cycle is a boundary so the complex is exact at $C_0$. Now, 
\[H_0 = C_0 / \Im{\partial_1} = C_0 / \ker{\varepsilon} \cong \Z\]
because $\varepsilon$ is a surjection $C_0 \to \Z$ since $X$ is nonemtpy so the augmented complex is exact at $\Z$.  
\end{proof}

\begin{definition}
Let $f_{\#}, g_{\#} : C \to D$ be morphisms of chain complexes. A \textit{chain homotopy} $p : f_{\#} \implies g_{\#}$ is a sequence of maps $p_n : C_n \to D_{n+1}$ such that, 
\[\partial \circ p + p \circ \partial = g_{\#} - f_{\#}\]
or more explicitly,
\[ 
\partial_{n+1} \circ p_n + p_{n-1} \circ \partial_{n} = (g_{\#})_n - (f_{\#})_n\] 
\end{definition}

\begin{lemma}
If $f_{\#}, g_{\#} : C \to D$ are chain homotopic then the induced maps on homology $f_*, g_* : H_n(C) \to H_n(D)$ are equal.
\end{lemma}

\begin{proof}
Let $p : f_{\#} \implies g_{\#}$ be a chain homotopy. It suffices to show that if $\alpha \in \ker{\partial}$ is a cycle then $(f_* - g_*)(\alpha) = 0$ which is equivalent to $(g_{\#} - f_{\#})(\alpha) \in \Im{\partial}$ is a boundary. Suppose that $\partial \alpha = 0$. Then, 
\[ (g_{\#} - f_{\#})(\alpha) = (\partial \circ p + p \circ \partial)(\alpha) = \partial(p(\alpha)) \]
and therefore $(g_{\#} - f_{\#})(\alpha)$ is a boundary. Therefore $f_* = g_*$. 
\end{proof}



\begin{theorem}
If $f, g : X \to Y$ are homotopic then the induced maps on homology $f_*, g_* : H_{n}(X) \to H_n(Y)$ are equal, $f_* = g_*$.
\end{theorem}

\begin{proof}
Let $h : X \times I \to Y$ be a homotopy between $f$ and $g$. Any map $h : X \times I \to Y$ induces, for each $n$-chain $\sigma : \Delta^n \to X$, a map $h \circ (\sigma \times \id_I) : \Delta^n \times I \to Y$. We can divide the space $\Delta^n \times I$ into $n+1$ distinct $n+1$-simplices as a $\Delta$-complex. Define $p_n : C_n(X) \to C_{n+1}(Y)$ by 
\[ p(\sigma) = \sum_{i = 0}^n (-1)^i [h \circ (\sigma \times \id_I)] |_{\Delta_i} \]  
I claim that $p$ is a chain homotopy between $f_{\#}$ and $g_{\#}$. (Hatcher p. 112). Therefore, by the above lemma, $f_* = g_*$.
\end{proof}

\begin{proposition}
We have the following commutative diagram of functors,
\begin{center}
\begin{tikzcd}
\Top \arrow[r, "C"] \arrow[d] & \mathbf{Ch(Ab)} \arrow[d] \\
\hTop \arrow[r] \arrow[d] & \mathbf{K(Ab)} \arrow[d]
\\
\mathbf{whTop} \arrow[r] & \mathbf{D(Ab)}
\end{tikzcd}
\end{center}
where $\mathbf{K(Ab)}$ is the category of chain complexes identifying chan-homotopic maps in $\mathbf{Ch(Ab)}$ and $\mathbf{D(Ab)}$ is the derived category identifing all maps which induce isomorphisms on homology and $\mathbf{whTop}$ is the weak homotopy category defined by identiying all maps which induce isomorphisms on homology and homotopy groups. 
\end{proposition}

\begin{definition}
Let $(X, A)$ be a pair of spaces then the \textit{relative $n$-chain} is the group $C_n(X, A) = C_n(X) / C_n(A)$ and the boundary map $\partial : C_n(X, A) \to C_{n-1}(X, A)$ induced by $\partial : C_n(X) \to C_n(X)$. Then, $C(X, A)$ is a complex with homology, $H_n(X, A) = H_n(C(X, A))$.  
\end{definition}

\begin{lemma}
Given a short exact sequence of chain complexes,
\begin{center}
\begin{tikzcd}
0 \arrow[r] & A \arrow[r, "i"] & B \arrow[r, "j"] & C \arrow[r] & 0
\end{tikzcd}
\end{center}
we get a long exact sequence,

\begin{center}
\begin{tikzcd}[column sep = small]
\cdots \arrow[r, "\delta"] & H_1(A) \arrow[r, "i_*"] & H_1(B) \arrow[r, "j_*"] & H_1(C) \arrow[r, "\delta"] & H_0(A) \arrow[r, "i_*"] & H_0(B) \arrow[r, "j_*"] & H_0(C) \arrow[r] & 0
\end{tikzcd}
\end{center} 

functorially.
\end{lemma}

\begin{proof}
Consider the diagram with exact rows,
\begin{center}
\begin{tikzcd}[column sep = small]
& \vdots \arrow[d, "\partial"] & \vdots \arrow[d, "\partial"] & \vdots \arrow[d, "\partial"] & 
\\
0 \arrow[r] & A_{n+1} \arrow[d, "\partial"] \arrow[r, "i"] & B_{n+1} \arrow[d, "\partial"] \arrow[r, "j"] & C_{n+1} \arrow[d, "\partial"] \arrow[r] & 0
\\
0 \arrow[r] & A_n \arrow[d, "\partial"] 
\arrow[r, "i"] & B_n \arrow[d, "\partial"] \arrow[r, "j"] & C_n \arrow[d, "\partial"] \arrow[r] & 0
\\
0 \arrow[r] & A_{n-1} \arrow[d, "\partial"] \arrow[r, "i"] & B_{n-1} \arrow[d, "\partial"] \arrow[r, "j"] & C_{n-1} \arrow[d, "\partial"] \arrow[r] & 0
\\
& \vdots & \vdots & \vdots & 
\end{tikzcd}
\end{center} 
Suppose that $c \in C_n$ is a cycle then $\exists b \in B_n$ such that $j(b) = c$ so $j(\partial b) = \partial \circ j(b) = \partial c = 0$ so by exactness, $\exists a \in A_{n-1}$ with $i(a) = \partial b$. We know that $i(\partial a) = \partial \circ i(a) = \partial^2 b = 0$ but $i$ is injective so $\partial a = 0$  Therefore $a$ is a cycle. Define $\delta([c]) = [a]$. If we pick $b'$ rather than $b$ then $b - b' \in \ker{j} = \Im{i}$ so $\exists a' \in A_n$ such that $i(a') = b - b'$. Then $i(a - \partial a') = \partial(b - i(a')) = \partial b'$ but $[a] = [a - \partial a']$. If $[c] = [c']$ are both cycles then $c = c' + \partial x$ for $x \in C_{n+1}$. Thus, $\exists y \in B_{n+1}$ such that $j(y) = x$ so $j(\partial y) = \partial \circ j(y) = \partial x$ but then $b = \partial y$ so if $i(a) = \partial^2 y = 0$ then $a = 0$ because $i$ is an injection. Therefore $\delta([\partial x]) = [0]$ so $\delta([c]) = \delta([c'])$ so $\delta$ is well-defined. It suffices to check that the sequence of homology groups is exact.
\end{proof}

\begin{theorem}
Let $(X, A)$ be a pair then there is a long exact sequence,
\begin{center}
\begin{tikzcd}[column sep = small]
\cdots \arrow[r, "\delta"] & H_1(A) \arrow[r, "\iota_*"] & H_1(X) \arrow[r, "j_*"] & H_1(X, A) \arrow[r, "\delta"] & H_0(A) \arrow[r, "\iota_*"] & H_0(X) \arrow[r, "j_*"] & H_0(X, A) \arrow[r] & 0
\end{tikzcd}
\end{center} 
\end{theorem}
\begin{proof}
Consider the diagram with exact rows,
\begin{center}
\begin{tikzcd}[column sep = small]
& \vdots \arrow[d, "\partial"] & \vdots \arrow[d, "\partial"] & \vdots \arrow[d, "\partial"] & 
\\
0 \arrow[r] & C_{n+1}(A) \arrow[d, "\partial"] \arrow[r] & C_{n+1}(X) \arrow[d, "\partial"] \arrow[r] & C_{n+1}(X, A) \arrow[d, "\partial"] \arrow[r] & 0
\\
0 \arrow[r] & C_n(A) \arrow[d, "\partial"] \arrow[r] & C_n(X) \arrow[d, "\partial"] \arrow[r] & C_n(X, A) \arrow[d, "\partial"] \arrow[r] & 0
\\
0 \arrow[r] & C_{n-1}(A) \arrow[d, "\partial"] \arrow[r] & C_{n-1}(X) \arrow[d, "\partial"] \arrow[r] & C_{n-1}(X, A) \arrow[d, "\partial"] \arrow[r] & 0
\\
& \vdots & \vdots & \vdots & 
\end{tikzcd}
\end{center} 
Therefore, we get an exact sequence of chain complexes,
\begin{center}
\begin{tikzcd}
0 \arrow[r] & C(A) \arrow[r] & C(X) \arrow[r] & C(X, A) \arrow[r] & 0
\end{tikzcd}
\end{center}
and thus a long exact sequence,
\begin{center}
\begin{tikzcd}[column sep = small]
\cdots \arrow[r, "\delta"] & H_1(A) \arrow[r, "\iota_*"] & H_1(X) \arrow[r, "j_*"] & H_1(X, A) \arrow[r, "\delta"] & H_0(A) \arrow[r, "\iota_*"] & H_0(X) \arrow[r, "j_*"] & H_0(X, A) \arrow[r] & 0
\end{tikzcd}
\end{center} 

\end{proof}


\begin{corollary}
If $x_0 \in X$ then $H_n(X, \{x_0\}) \cong \tilde{H}_n(X)$.
\end{corollary}

\begin{proof}
We have a long exact sequence of reduced homology,
\begin{center}
\begin{tikzcd}[column sep = small]
\cdots \arrow[r, "\delta"] & \tilde{H}_1(\{x_0\}) \arrow[r, "\iota_*"] & \tilde{H}_1(X) \arrow[r, "j_*"] & H_1(X, \{x_0\}) \arrow[r, "\delta"] & \tilde{H}_0(\{x_0\}) \arrow[r, "\iota_*"] & \tilde{H}_0(X) \arrow[r, "j_*"] & H_0(X, \{x_0\}) \arrow[r] & 0
\end{tikzcd}
\end{center} 
which gives isomorphisms $H_n(X, \{x_0\}) \cong \tilde{H}_n(X)$ for each $n$ since $\tilde{H}_n(\{x_0\}) = 0$.
\end{proof}

\begin{theorem}[Excision I]
Suppose that $Z \subset A \subset X$ with $\overline{Z} \subset A^\circ$ then the inclusion map $(X \setminus Z, A \setminus Z) \hook (X, A)$ induces isomorphism on all homology.
\end{theorem}

\begin{theorem}[Excision II]
Suppose that $A, B \subset X$ with $A^\circ \cup B^\circ = X$ then the inclusion $(B, A \cap B) \hook (X, A)$ induces isomorphisms on all homology. 
\end{theorem}

\begin{lemma}
Excision I and Excision II are equivalent. 
\end{lemma}

\begin{proof}
Take $B = X \setminus Z$ then $A^\circ \cup B^\circ = A^\circ \cup (X \setminus \overline{Z}) = X$ if and only if $\overline{Z} \subset A^\circ$.  
\end{proof}

\begin{lemma}
Suppose that $A, B \subset X$. If $C(A) + C(B) \hook C(X)$ induces isomorphisms on homology, then Excision II holds for $A,B, X$.
\end{lemma}

\begin{proof}
Apply the long exact sequence to the short exact sequence of complexes,

\begin{center}
\begin{tikzcd}
0 \arrow[r] & C(A) + C(B) \arrow[r, "\iota_{\#}"] & C(X) \arrow[r] & C(X) / (C(A) + C(B)) \arrow[r] & 0
\end{tikzcd}
\end{center}

By assumption, $\iota_{*}$ is an isomorphism so $H_n(C(X)/(C(A) + C(B))) = 0$. Now, consider the short exact sequence of complexes given by the third isomorphism theorem,

\begin{center}
\begin{tikzcd}
0 \arrow[r] & \frac{C(A) + C(B)}{C(A)} \arrow[r, "j_{\#}"] & \frac{C(X)}{C(A)} \arrow[r] & \frac{C(X)}{C(A) + C(B)} \arrow[r] & 0
\end{tikzcd}
\end{center}
The long exact sequence has every third term zero so $j_*$ is an isomorphism. Therefore, the following diagram of complexes commutes,

\begin{center}
\begin{tikzcd}
\frac{C(B)}{C(A \cap B)}  = \frac{C(B)}{C(A) \cap C(B)} \arrow[rr, "k_{\#}"] \arrow[rd, "\phi"] & & \frac{C(X)}{C(A)} 
\\
& \frac{C(A) + C(B)}{C(A)} \arrow[ru, "j_{\#}"] 
\end{tikzcd}
\end{center}
Where $\phi$ is an isomorphism by the second isomorphism theorem. Therefore, $k_{*}$ is an isomorphism on homology so $k_{*} : H_n(C(B)/C(A \cap B)) \to H_n(C(X)/C(A))$ is an isomorphism. Therefore, by definition,
\[ k_* : H_n(B, A \cap B) \to H_n(X, A)\]
is an isomorphism.
\end{proof}

\newcommand{\cone}{\mathrm{Cone}}

\begin{definition}
Let $K \subset \R^d$ be convex, let $p \in K$ and $\sigma : \Delta^n \to K$ then we define the cone map, $\cone_p(\sigma) : \Delta^{n+1} \to K$ is given by, 
\[\cone_p(\sigma) : (t_0, \dots, t_{n+1}) = t_0 p + (1 - t_0) \sigma\left(\tfrac{t_1}{1-t_0}, \cdots, \tfrac{t_{n+1}}{1-t_0} \right)\]
\end{definition}

\begin{lemma}
$\partial \cone_p(\sigma) = \sigma - \cone_p(\partial \sigma)$
\end{lemma}

\begin{proposition}
We can define a natural chain automorphism $S$ of the chain complex functor $C : \Top \to \mathbf{Ab(Ch)}$. Explicitly for any map $f : X \to Y$ the following diagram commutes,
\[
\begin{tikzcd}
C(X) \arrow[r, "f_{\#}"] \arrow[d, "S^X"] & C(Y) \arrow[d, "S^Y"] 
\\
C(X) \arrow[r, "f_{\#}"] & C(Y)
\end{tikzcd}
\iff 
\begin{tikzcd}
C_n(X) \arrow[r, "f_{\#}"] \arrow[d, "S^X_n"] & C_n(Y) \arrow[d, "S^Y_n"] 
\\
C_n(X) \arrow[r, "f_{\#}"] & C_n(Y)
\end{tikzcd}
\]
Furthermore, there is a natural chain homotopy $T$ between the identity natural transformaton and $S$,
\[
\begin{tikzcd}
C(X) \arrow[r, "f_{\#}"] \arrow[d, "T^X"] & C(Y) \arrow[d, "T^Y"] 
\\
C(X) \arrow[r, "f_{\#}"] & C(Y)
\end{tikzcd}
\iff 
\begin{tikzcd}
C_n(X) \arrow[r, "f_{\#}"] \arrow[d, "T^X_n"] & C_{n+1}(Y) \arrow[d, "T^Y_n"] 
\\
C_n(X) \arrow[r, "f_{\#}"] & C_{n+1}(Y)
\end{tikzcd}
\]
such that $\partial \circ T^X + T^X \circ \partial = S^{X} - \id_{C(X)}$
\end{proposition}

\begin{definition}
Using the naturality, $S^X_{n}(\sigma) = \sigma_{\#} \circ S^{\Delta^{n}}_n(\id)$.  Furthemore,
$S^{\Delta^0}(\id) = \id$ and $S^{\Delta^{n+1}}_{n+1}(\id) = \cone_b\left( S^{\Delta^{n+1}}_n(\partial \id_{\Delta^{n+1}}) \right)$
where $b$ is the baryceter of $\Delta^{n+1}$. 
\end{definition}

\begin{lemma}
$S^X : C(X) \to C(X)$ is a chain map.
\end{lemma}

\begin{proof}
For $n = 0$ we have that $S$ takes the idenity to the idenity which is clearly a chain map. Assume this holds for $n$,
\begin{align*}
\partial S_{n+1}^X(\sigma) & = \partial \circ \sigma_{\#} \circ S^{\Delta^{n+1}}_{n+1}(\id)  
\\
& = \partial \circ \sigma_{\#} \circ \cone_b\left( S^{\Delta^{n+1}}_n (\partial \id) \right) 
\\
& = \sigma_{\#} \circ \partial \circ \cone_b\left( S^{\Delta^{n+1}}_n (\partial \id) \right) 
\\
& = \sigma_{\#} \left( S^{\Delta^{n+1}}_n(\partial \id) - \cone_b \left(\partial S^{\Delta^{n+1}}_n (\partial \id) \right) \right)
\\
& = \sigma_{\#} \left( S^{\Delta^{n+1}}_n(\partial \id) - \cone_b \left( S^{\Delta^{n+1}}_n (\partial^2 \id) \right) \right)
\\
& = \sigma_{\#} \left( S^{\Delta^{n+1}}_n(\partial \id) \right) = S_{n}^{X}(\sigma \circ \partial) = S_{n}^{X}(\partial \circ \sigma)
\end{align*}
Therefore, $\partial \circ S^X = S^X \circ \partial$. 
\end{proof}

\begin{definition}
Define the chain homotopy $T : \id \implies S^X$ inducitively,
\[ T_0^{\Delta^0} : C_0(\Delta^0) \to C_1(\Delta^0)\] such that $ [\id_{\Delta^0}] = [\Delta^1 \to \Delta^0] $
And
\end{definition}

\begin{definition}
Let $\mathcal{V}$ be an open cover of $X$ then $x \in C_n(X)$ is $\mathcal{V}$-small if it lies in the inclusion of $\sum_{v \in \mathcal{V}} C_n(v)$.
\end{definition}

\begin{lemma}
If $\Delta_i$ is one of the supports of $S^{\Delta^n}_n(\id)$ then $\mathrm{diam}(\Delta_i) \le \frac{n}{n+1}$.
\end{lemma}

\begin{lemma}
Let $c \in C_n(X)$ then there exists $k \in \Zplus$ such that $[S]^k(c)$ is $\mathcal{V}$-small where $\mathcal{V}$ is any open cover of $X$. 
\end{lemma}

\begin{proof}
It suffices to show for $\sigma : \Delta^n \to X$. Let $\varepsilon$ be a Lebesgue number for the cover $\{ \sigma^{-1}(V) \mid V \in \mathcal{V} \}$ of $\Delta^n$ which is a compact metric space. Pick $k$ such that $\left(\frac{n}{n+1}\right)^k < \varepsilon$. By the above lemma, $[S]^k(\sigma)$ is contained within a single element of $\mathcal{V}$. 
\end{proof}

Now we will prove Excision.
\begin{proof}
We want to show that the inclusion $\sum_{V \in \mathcal{V}} C(V) \hook C(X)$ induces a isomorphism on homology. For each $c \in C_n(X)$ we know that $[S]^k(c) \in \sum_{V \in \mathcal{V}} C(v)$ and by the existence of a chain homotopy between $S$ and $\id$ we have that $[S]^k$ is chain homotopic to $\id$. Therefore, $c$ and $[S]^k(c)$ lie in the same homology class so the map $\sum_{V \in \mathcal{V}} C(V) \hook C(X)$ is surjective. Suppose that $c \in \sum_{V \in \mathcal{V}} C(V)$ is a boundary in $C(X)$. Suppose that $b \in C(V)$ and $\exists : c \in C(X)$ such that $\partial c = b$. Take $k$ such that $[S]^k(c)$ is $\mathcal{V}$-small. Then,
\[ \partial \circ [S]^k(c) = [S]^k(\partial c) = [S]^k(b) = b - \partial \circ T_k(b) \]
where $T_k$ is a chain homotopy between $[S]^k$ and $\id$. However, $T_k$ is a natural chain homotopy so $T_k^V$ and $T_k^X$ agree and thus $T_b(b) \in V$. Therefore, $([S]^k(c) + T_k(b) \in C(V)$ and $\partial ([S]^k(c) + T_k(b)) = b$ so $b$ is a boundary in $C(V)$. Thus, the inclusion map is an injection and thus an isomorphism on homology. 
\end{proof}

\begin{definition}
$(X,A)$ is a good pair if $A$ is closed in $X$ and there exists a neighborhood $U$ of $A$ such that $U$ deformation retracts to $A$.
\end{definition}

\begin{lemma}
$(X, A)$ is a good pair if and only if $A \hook X$ is a cofibration.
\end{lemma}

\begin{corollary}
Suppose that $A \hook X$ is a cofibrationm then the map,
\[ H_n(X, A) \xrightarrow{\sim} H_n(X/A, \{x_0\}) \cong \tilde{H}_n(X/A) \]
induced by the quotient map $X \to X/A$ is an isomorphism.
\end{corollary}

\begin{proof}
Let $U$ be an open neighborhood of $A$ and $A$ is a deformation retract of $A$.
Then, the following diagram commutes,
\begin{center}
\begin{tikzcd}
H_n(X, A) \arrow[r, "\alpha"] \arrow[d, "f_{\#}"] & H_n(X, U) \arrow[d, "f_{\#}"] & H_n(X \setminus A, U \setminus A) \arrow[l, "\gamma"] \arrow[d, "\epsilon"] 
\\
\tilde{H}_n(X/A) \cong H_n(X/A, A/A) \arrow[r, "\beta"] & H_n(X/A, U/A) & H_n(X/A \setminus A/A, U/A \setminus A/A) \arrow[l, "\delta"] \\
\end{tikzcd}
\end{center}
The map $\alpha$ is an isomorphism by the long exact sequence of $(X, U, A)$ as $H_n(U, A) = 0$ since $U$ is a deformation retract of $A$. 
\end{proof}

\begin{corollary}
Given a good pair $(X, A)$, there is a long exact sequence,
\begin{center}
\begin{tikzcd}[column sep = small]
\cdots \arrow[r] & H_{n+1}(A) \arrow[r] & H_{n+1}(X) \arrow[r] & H_{n+1}(X/A) \arrow[r] & H_{n}(A) \arrow[r] & H_n(X) \arrow[r] & H_n(X/A) \arrow[r] & \cdots
\end{tikzcd}
\end{center} 
\end{corollary}

\begin{theorem}
\[\tilde{H}_k(S^n) =
\begin{cases}
\Z & n = k \\
0 & n \neq k
\end{cases}\]
\end{theorem}

\begin{proof}
Take the pair $(D^n, \partial D^n = S^{n-1})$ then using the long exact sequence,
\begin{center}
\begin{tikzcd}[column sep = small]
\cdots \arrow[r] & \tilde{H}_{k+1}(S^{n-1}) \arrow[r] & \tilde{H}_{k+1}(D^n) \arrow[r] & \tilde{H}_{n+1}(S^n) \arrow[r] & \tilde{H}_{k}(S^{n-1}) \arrow[r] & \tilde{H}_k(D^n) \arrow[r] & \tilde{H}_k(S^n) \arrow[r] & \cdots
\end{tikzcd}
\end{center} 
We know that $\tilde{H}_k(D^n) = 0$ since $D^n$ is contractible so its homology is the same as a point. Therefore, we get exact sequences,
\begin{center}
\begin{tikzcd}[column sep = small]
0 \arrow[r] & \tilde{H}_{n+1}(S^n) \arrow[r] & \tilde{H}_{k}(S^{n-1}) \arrow[r] & 0
\end{tikzcd}
\end{center} 
for each $k$ and thus $\tilde{H}_{k+1}(S^n) \cong \tilde{H}_k(S^{n-1})$. Furthermore, we know that,
\[ \tilde{H}_k(S^0) = 
\begin{cases}
\Z & k = 0 \\
0 & k > 0 
\end{cases}\]
By induction, the result follows.
\end{proof}

\begin{theorem}
If $f : X \to Y$ is any continuous map then there is a long exact sequence,
\begin{center}
\begin{tikzcd}[column sep = small]
\cdots \arrow[r] & H_{n+1}(X) \arrow[r] & H_{n+1}(Y) \arrow[r] & H_{n+1}(C_f) \arrow[r] & H_{n}(X) \arrow[r] & H_n(Y) \arrow[r] & H_n(C_f) \arrow[r] & \cdots
\end{tikzcd}
\end{center}
\end{theorem}

\begin{proof}
The map $X \times \{0\} \to M_f$ is a cofibration. Therefore, we have a good pair $(M_f, X \times \{0\})$ so there is a long exact sequence,

\begin{center}
\begin{tikzcd}[column sep = small]
\cdots \arrow[r] & H_{n+1}(X) \arrow[r] & H_{n+1}(M_f) \arrow[r] & H_{n+1}(M_f/X) \arrow[r] & H_{n}(X) \arrow[r] & H_n(M_f) \arrow[r] & H_n(M_f/X) \arrow[r] & \cdots
\end{tikzcd}
\end{center}
However, $C_f = M_f / ( X \times \{0\} )$ and $M_f$ deformation retracts onto $Y$ so we have the required exact sequence.
\end{proof}

\begin{corollary}
Suppose that $(X_\alpha, x_\alpha)$ are pointed spaces with $(X_\alpha, \{x_\alpha\})$ a good pair. Then,
\[ \tilde{H}_n \left(\bigvee_{\alpha} X_\alpha \right) \cong \bigoplus_{\alpha} \tilde{H}_n(X_\alpha) \]
\end{corollary}

\begin{proof}
\begin{align*}
\bigoplus_{\alpha} \tilde{H}_n(X_\alpha) & \cong \bigoplus_\alpha H_n(X_\alpha, \{x_\alpha\}) \cong H_n \left( \coprod_{\alpha} X_\alpha, \coprod_{\alpha} \{x_\alpha\} \right) 
\\
& \cong \tilde{H}_n\left(  \coprod_{\alpha} X_\alpha \bigg/ \coprod_{\alpha} \{x_\alpha\}  \right) \cong \tilde{H}_n \left( \bigvee_{\alpha} X_{\alpha} \right)
\end{align*}
\end{proof}

\begin{corollary}
There is a natural isomorphism $\Sigma : \tilde{H}_n(X) \to \tilde{H}_{n+1}(\Sigma H)$
\end{corollary}

\begin{proof}
We have that $\Sigma X$ is homeomorphic to $CX/(X \times \{0\})$. Apply the long exact sequence to $(CX, X \times \{0\})$, 
\begin{center}
\begin{tikzcd}[column sep = small]
\cdots \arrow[r] & \tilde{H}_{n+1}(X) \arrow[r] & \tilde{H}_{n+1}(CX) \arrow[r] & \tilde{H}_{n+1}(\Sigma X) \arrow[r] & \tilde{H}_{n}(X) \arrow[r] & \tilde{H}_n(CX) \arrow[r] & \tilde{H}_n(\Sigma X) \arrow[r] & \cdots
\end{tikzcd}
\end{center}
However, $CX$ is contractible so we have isomorphism $\tilde{H}_{n+1}(\Sigma X) \to \tilde{H}_n(X)$. 
\end{proof}

\begin{theorem}
Let $(X, A)$ be a pair of $\Delta$-complexes. Then there are natrual isomorphisms $H_n^\Delta(X, A) \xrightarrow{\sim} H_n(X, A)$
\end{theorem}

\begin{proof}
Let $X^k$ be the $k^{\mathrm{th}}$ skeleton of $X$. 
\end{proof}


\begin{theorem}
\[ H^{\Delta}_n(X, A) \cong H_n(X, A) \]
\end{theorem}

\begin{proof}
Take $X^k$ to be a the $k$-skeleton of a $\Delta$-complex $X$. Consider the long exact sequences,
\begin{center}
\begin{tikzcd}
H^\Delta_{n+1}(X^k, X^{k-1}) \arrow[r] \arrow[d] & H_n^\Delta(X^{k-1}) \arrow[r] \arrow[d] & H_n^\Delta(X^k) \arrow[r] \arrow[d] & H_n^\Delta(X^k, X^{k-1}) \arrow[r] \arrow[d] & H_{n-1}^\Delta(X^{k-1}) \arrow[d]
\\
H_{n+1}(X^k, X^{k-1}) \arrow[r] & H_n(X^{k-1}) \arrow[r] & H_n(X^k) \arrow[r] & H_n(X^k, X^{k-1}) \arrow[r] & H_{n-1}(X^{k-1}) 
\end{tikzcd}
\end{center}
where we know,
\[ H_n^\Delta(X^k, X^{k-1}) = 
\begin{cases}
\bigoplus\limits_{k-\text{simplices of } X} \Z & n = k \quad  \\
0 & n \neq k
\end{cases}\]
Furthermore,
\[ H_n(X^k, X^{k-1}) = \tilde{H}_n(X^k/X^{k-1}) = \tilde{H}_n\left( \coprod_{\alpha} \left( \Delta^k_\alpha \middle/ \partial \Delta_{\alpha}^k \right) \right) \cong \bigoplus_{\alpha} \tilde{H}_n(\Delta^k / \partial \Delta^k) \]
However, we know that, $\Delta^k / \partial \Delta^k \cong S^k$ so,

\[ \bigoplus_{\alpha} \tilde{H}_n(\Delta^k / \partial \Delta^k) \cong \begin{cases}
\bigoplus\limits_{\alpha} \Z & n = k \\
0 & n \neq k
\end{cases} \]
 
Then we know that the map $H_{n}^\Delta(X^k, X^{k-1}) \to H_{n+1}(X^k, X^{k-1})$ is an isomorphism. Now, assume the induction hypothesis that the theorem holds for $X^{k-1}$ the $k-1$-skeleton. Then we have isomorphisms,

\begin{center}
\begin{tikzcd}
H^\Delta_{n+1}(X^k, X^{k-1}) \arrow[r] \ar[d, "\sim" labl] & H_n^\Delta(X^{k-1}) \arrow[r] \ar[d, "\sim" labl] & H_n^\Delta(X^k) \arrow[r] \arrow[d] & H_n^\Delta(X^k, X^{k-1}) \arrow[r] \ar[d, "\sim" labl] & H_{n-1}^\Delta(X^{k-1}) \ar[d, "\sim" labl]
\\
H_{n+1}(X^k, X^{k-1}) \arrow[r] & H_n(X^{k-1}) \arrow[r] & H_n(X^k) \arrow[r] & H_n(X^k, X^{k-1}) \arrow[r] & H_{n-1}(X^{k-1}) 
\end{tikzcd}
\end{center}
so by the five-lemma we know that the map $H_n^\Delta(X^k) \to H_n(X^k)$ is an isomorphism. Therefore, $H_n^\Delta(X^k) \cong H_n(X^k)$ so the result holds by induction on $k$ since the base case $H_n^\Delta(X^0) \cong H_n(X^0)$ is clear because every map $\Delta^0 \to X^0$ is a sinlge point and thus a $\Delta$-complex component. For the case of an infinite $\Delta$-complex we use the weak topology. Using the long exact sequence of a pair and the five-lemma we get the equivalence of singluar and simplicial relative homology. 
\end{proof}

\begin{theorem}[Invariance of Dimension]
Suppose that $U \subset \R^n$ and $V \subset \R^m$ and $U$ and $V$ are open such that $U \cong V$ then $n = m$.
\end{theorem}

\begin{proof}
Choose some $x \in U$ and let $f : U \to V$ be a homeomorphism. Consider the relative homology,
\[ H_i(U, U \setminus \{ x \}) \cong H_i(\R^n, \R^n \setminus \{x\}) \]
by Excision since $U$ is open. However,
\[ H_i(\R^n, \R^n \setminus \{x\}) \cong H_{i-1}(\R^n \setminus \{x\}) \cong H_{i-1}(S^{n-1}) \]
and similarly,  
\[H_i(V, V \setminus \{ f(x) \}) \cong H_i(\R^m, \R^m \setminus \{x\}) \cong H_{i-1}(\R^m \setminus \{x\}) \cong H_{i-1}(S^{m-1}) \]
Therefore, since $f$ is a homeomorphism,
\[ H_{i-1}(S^{n-1}) \cong H_{i-1}(S^{m-1}) \]
However, the homology of a sphere goes to zero exactly at the dimension so $n = m$.
\end{proof}

\begin{theorem}[Mayer-Vietoris]
Given $A, B \subset X$ with $A^\circ \cup B^\circ = X$ there exists a long exact sequence,
\begin{center}
\begin{tikzcd}
\cdots \arrow[r] & H_n(A \cap B) \arrow[r] & H_n(A) \oplus H_n(B) \arrow[r] & H_n(X) \arrow[r] & H_{n-1}(A \cap B) \arrow[r] & \cdots
\end{tikzcd}
\end{center}
\end{theorem}

\begin{proof}
Consider the sequence of chain complexes,
\begin{center}
\begin{tikzcd}
0 \arrow[r] & C(A \cap B) \arrow[r, "f"] & C(A) \oplus C(B) \arrow[r, "g"] &  C(A) + C(B) \arrow[r] & 0
\end{tikzcd}
\end{center}
where $f(x) = (x, -x)$ and $g(x,y) = x + y$. Clearly, $f$ is injective and $g$ is surjective. Furthermore, 
\[\Im{f} = \{ (x, -x) \mid x \in C(A \cap B) \}\]
and if $x + y = 0$ then since $x \in C(A)$ and $y \in C(B)$ we have that $y = -x \in C(A \cap B)$. Thus, 
\[\ker{g} = \{ (x, -x) \mid x \in C(A \cap B) \} = \Im{f}\]
Therefore, the sequence is exact.
However, by Excision, $H_n(C(A) + C(B)) = H_n(X)$ so the long exact sequence of homology given by this short exact sequence of chain complexes gives the required Meyer-Vietoris sequence.
\end{proof}

\begin{remark}
There is an equivalent Meyer-Vietoris sequence for reduced homology by augmenting the short exact sequence of complexes by,
\begin{center}
\begin{tikzcd}
0 \arrow[r] & \Z \arrow[r] & \Z \oplus \Z \arrow[r] & \Z \arrow[r] & 0
\end{tikzcd}
\end{center}
\end{remark}

\begin{remark}
If $X$ is a CW complex and $A, B$ are subcomplexes with $A \cup B = X$ then the Meyer-Vietoris sequence applies using an open $\epsilon$-neighborhood of $A$ and $B$ which deformation retract onto $A$ and $B$ respectively. 
\end{remark}

\subsection{Classical Results of Homology}

\begin{definition}
Suppose that $X$ is a fintie $\Delta$-complex then $C^\Delta_n(X)$ is a finitely generated abelian groups so $H_n(X)$ is a finitely generated abelian group. Therefore,
\[ H_n(X) \cong \Z^{r_n} \oplus T_n\] where $T_n$ is finite. We call $r_n$ the $n^{\mathrm{rm}}$ Betti number of $X$ and $T_n$ the torsion subgroup. 
\end{definition}

\begin{definition}
The Euler Characteristic $\chi(X) = \sum_{n} (-1)^n r_n$ is the alternating sum of Betti numbers. Furthermore, the genus is $g = 1 - \tfrac{1}{2} \chi(X)$.
\end{definition}

\begin{theorem}[Brouwer]
Ever map $f : D^n \to D^n$ has a fixed point.
\end{theorem}

\begin{proof}
Suppose that $f : D^n \to D^n$ has no fixed point. Then, for each $x$ take the line $f(x)$ to $x$ and take $r(x)$ the point on $\partial D^n$ where this line intersects. $r : D^n \to \partial D^n = S^{n-1}$ is a retract. Therefore, $r \circ \iota = \id_{S^{n-1}}$. Applying the functor $H_{n-1}$ we find that, $r_* \circ \iota_* = \id_{H_{n-1}(S^{n-1})}$ and therefore the map $r_* : H_{n-1}(D^n) \to H_{n-1}(S^{n-1}) \cong \Z$ is surjective. However, $H_{n-1}(D^n) = 0$ so $r$ cannot be surjective showing that $r$ cannot be well-defined. Therefore, for some point, $f(x) = x$. 
\end{proof}

\begin{definition}
Let $G$ be an abelian group and $X$ a topological space. Then the homology of $X$ with coeficients in $G$ is $H_n(X; G)$ is the homology of the complex $C_n(X; G) = \bigoplus\limits_{\sigma : \Delta^n \to X} \sigma G$ with the boundary map defined as before. 
\end{definition}

\begin{lemma}
Given a homomorphism $\phi : G_1 \to G_2$ of abelian groups there is a natural transformation of functors $H_n( - ; G_1) \implies H_n(- ; G_2)$
\end{lemma}

\begin{definition}
If $f : S^n \to S^n$ is continuous then $\deg{f} \in \Z$ is the integer such that $f_* : H_n(S^n) \to H_n(S^n)$ is the map $f_* [\alpha] = \deg{f} \cdot [\alpha]$ viewing $H_n(S^n) \cong \Z$. 
\end{definition}

\begin{proposition}
If $f : S^n \to S^n$ is a reflection then $\deg{f} = -1$. 
\end{proposition}

\begin{proof}
Write $S^n$ as the union of $\Delta^n_1$ and $\Delta^n_2$ which are exhanged under $f$. The homology group $H_n(S^n)$ is generated by $[\Delta_1^n] - [\Delta_2^n]$ so $f_*([\Delta_1^n] - [\Delta_2^n]) = [\Delta_2^n] - [\Delta_1^n] = - ([\Delta_1^n] - [\Delta_2^n]) $
so $\deg{f} = -1$ 
\end{proof}

\begin{proposition}
If $f : S^n \to S^n$ is not surjective then $\deg{f} = 0$. 
\end{proposition}

\begin{proof}
If $f : S^n \to S^n$ is not surjective then we can factor $f$ through $S^n \setminus \{x_0\}$ which is contractible so $f_*$ is the zero map. 
\end{proof}


\begin{lemma}
If $G$ is an abelian group and $f : S^n \to S^n$, then the induced map $f_* : H_n(S^n; G) \to H_n(S^n; G)$ the map $f_*(g) = \deg{f} \cdot g$. 
\end{lemma}

\begin{proof}
Let $f : \Z \to G$ be the map sending $1 \mapsto g$ so we get a natural transformation,
\begin{center}
\begin{tikzcd}
H_n(S^n ; \Z) \arrow[d, "\eta"] \arrow[r, "f_*"] & H_n(S^n ; \Z) \arrow[d, "\eta"] \\
H_n(S^n ; G) \arrow[r, "f_*"] & H_n(S^n ; G) 
\end{tikzcd}
\end{center}
Therefore, $f_*(g) = f_*(\eta(1)) = \eta(f_*(1)) = \eta( \deg{f} \cdot 1) = \deg{f} \cdot g$. 
\end{proof}

\begin{proposition}
If $f : S^n \to S^n$ is an odd map $f(-x) = - f(x)$ then $\deg{f}$ is odd. 
\end{proposition}

\begin{proof}
Consider the covering map $p : S^n \to \rp^n$, then there is an exact sequence of chain complexes,
\begin{center}
\begin{tikzcd}
0 \arrow[r] & C_n(\rp^n ; \Z / 2 \Z) \arrow[r, "\tau"] & C_n(S^n ; \Z / 2 \Z ) \arrow[r, "p_{\#}"] & C_n(\rp^n ; \Z / 2 \Z) \arrow[r] & 0 
\end{tikzcd}
\end{center}
where $\tau$ is the sum of the lifts of $\sigma$ to $S^n$. Clearly, $\tau$ is injective and $p_{\#}$ is surjective. Furthermore, the kernel of $p_{\#}$ are the chains such that $\sum_{n = 1} n \sigma_i \circ p = 0$ which are those chains which are sums of the two lifts and thus the image of $\tau$. Therefore, we get a long exact sequence of homology with coeficients in $\Z / 2 \Z$,

\begin{center}
\begin{tikzcd}
0 \arrow[r] & H_n(\rp^n) \arrow[r, "\sim"] & H_n(S^n) \arrow[r, "0"] & H_n(\rp^n)
\end{tikzcd}
\end{center}
\end{proof}


\begin{theorem}[Borsuk-Ulam]
Let $g : S^n \to \R^n$ be a continuous map then $\exists x \in S^n$ such that $g(-x) = g(x)$. 
\end{theorem}

\begin{proof}
Given $g : S^n \to \R^n$. Let $f(x) = g(x) - g(-x)$ be a map $S^n \to \R^n$. Assume that $f$ has no zeros. In this case, we can define the continuous function $\tilde{f}(x) : S^n \to S^{n-1}$ given by, $\tilde{f}(x) = \frac{f(x)}{|f(x)|}$. The restriction of this map to the equatoril $S^{n-1}$ is a map $h : S^{n-1} \to S^{n-1}$. However, $h(-x) = \frac{g(-x) - g(x)}{|g(-x) - g(x)|} = - \frac{g(x) - g(-x)}{|g(x) - g(-x)|} = - h(x)$ so $h$ is an odd function and therefore has odd degree. However, $\tilde{f}$ restricted to a hemisphere is a homotopy of $h$ to a constant map so $h$ has degree zero which is a contradiction. Therefore, $f$ has a zero so $\exists z \in S^n$ such that $g(x) - g(-x) = 0$ and thus $g(x) = g(-x)$.  
\end{proof}

\begin{corollary}
There is no subspace of $\R^n$ homeomorphic to $S^n$
\end{corollary}

\begin{proof}
Any continuous map $f : S^n \to \R^n$ satisfies $f(z) = f(-z)$ for some $z \in S^n$ so $f$ cannot be injective and therefore $f$ cannot be a homeomorphism.  
\end{proof}

\begin{corollary}[Ham Sandwich]
Let $A_1, \cdots, A_n \subset \R^n$ are bounded measurable sets. Then, there exists a hyperplane $P$ which divides each $A_i$ into two sets of equal measure.
\end{corollary}

\begin{proof}
Let $f_i : S^{n-1} \times \R \to [0, \infty)$ be the map $f_i(\hat{n},s)$ giving the measure of $A_i$ lying on the side in the direction $\hat{n}$ of the plane normal to $\hat{n}$ lying a distance $s$ from the origin. From measure theory, we know that $f_i$ is continuous. Then, $f_i$ is monotonically decreasing in $s$ and its image contains $0$. By the intermediate value theorem, $\exists a_k, b_k$ such that $[a_n, b_n]$ is all the points such that $f_n(\hat{n}, s) = m(A_k)/2$. Let $g_i : S^{n-1} \to \R$ be the map $g(\hat{n}) = \frac{b-a}{2}$. It turns out that $g_i$ is continuous. Define the map $G : S^{n-1} \to \R^n$ by $G(\hat{n}) = $
\end{proof}

\subsection{Cellular Homology}

\begin{lemma}
Let $X$ be a CW complex. Then the following hold,
\begin{enumerate}
\item $H_k (X^n, X^{n-1}) \cong
\begin{cases}
\bigoplus_{n-\text{cells}} \Z & n = k \\
0 & n \neq k 
\end{cases} $

\item $H_k(X^n) = 0$ for $k > n$ 

\item The inclusion $\iota : X^n \to X$ induces $\iota_* : H_k(X^n) \to H_k(X)$ an isomorphism if $k < n$.
\end{enumerate}
\end{lemma}

\begin{proof}
Geometrically, $X^n / X^{n-1} \cong \bigvee_{n-\text{cells}} S^n$ and therefore, because $(X^n, X^{n-1})$ is a good pair,
\[ H_k (X^n, X^{n-1}) \cong H_k(X^n / X^{n-1}) \cong H_k \left( \bigvee_{n-\text{cells}} S^n \right) \cong \bigoplus_{n-\text{cells}} H_k(S^n) \cong
\begin{cases}
\bigoplus_{n-\text{cells}} \Z & n = k \\
0 & n \neq k 
\end{cases} \]
\end{proof}


\subsection{The Hurewicz Theorem}

\begin{definition}
Fix generators $\alpha$ of $H_n(D^n, \partial D^n) \cong H_n(S^n)$ such that $\alpha_n = \Sigma \alpha_{n-1}$. For a pointed pair $(X, A, x_0)$ the Hurewicz map $h_n : \pi_n(X, A, x_0) \to H_n(X, A)$ takes $h_n([f]) = f_*(\alpha_n)$. 
\end{definition}

\begin{lemma}
If $n > 0$ then the Hurewucz map $h_n$ is a homomorphism
\end{lemma}

\begin{proof}
We need to show that $(f * g)_* = f_* * g_*$. The class $[f * g]$ is represented by the composition,
\begin{center}
\begin{tikzcd}
(D^n, \partial D^n) \arrow[r, "c"] & (D^n \vee D^n, \partial D^n \vee \partial D^n) \arrow[r, "f \vee g"] & (X \vee Y, A \vee B) \arrow[r, "\nabla"] & (X, A) 
\end{tikzcd}
\end{center} 
\end{proof}

\begin{lemma}
$h_n : \pi_n(X, A, x_0) \to H_n(X, A)$ is a natural transformation $h_n : \pi_n \implies H_n$  between the homotopy and homology functors. 
\end{lemma}

\begin{proof}
Consider the diagram,
\begin{center}
\begin{tikzcd}
\pi_n(X, A) \arrow[d, "h_n"] \arrow[r, "f_*"] & \pi_n(Y, B) \arrow[d, "h_n"]
\\
\pi_n(X, A) \arrow[r, "f_*"] & H_n(Y, B)
\end{tikzcd}
\end{center}
which commutes because $f_* \circ h_n([\gamma]) = f_*(\gamma_*(\alpha_n)) = f_* \circ \gamma_*(\alpha_n)$. Likewise, \[h_n \circ f_*([\gamma]) = h_n([f \circ \gamma]) = (f \circ \gamma)_*(\alpha_n) = f_* \circ \gamma_*(\alpha_n)\]
\end{proof}

\begin{lemma}
The Hurewicz map is compatible with suspension. That is, the diagram,
\begin{center}
\begin{tikzcd}
\pi_n(X) \arrow[d, "h_n"] \arrow[r, "\Sigma"] & \pi_{n+1}(\Sigma X) \arrow[d, "h_{n+1}"] 
\\
H_n(X) \arrow[r, "\Sigma"] & H_{n+1}(\Sigma X)
\end{tikzcd}
\end{center}
commutes. Furthermore, the Hurewicz map yields a morphism of complexes,
\begin{center}
\begin{tikzcd}
\cdots \arrow[r] & \pi_n(A) \arrow[d, "h"] \arrow[r] & \pi_n(X) \arrow[d, "h"] \arrow[r] & \pi_n(X, A) \arrow[d, "h"] \arrow[r] & \pi_{n-1}(A) \arrow[d, "h"] \arrow[r] & \cdots 
\\
\cdots \arrow[r] & H_n(A) \arrow[r] & H_n(X) \arrow[r] & H_n(X, A) \arrow[r] & H_{n-1}(A) \arrow[r] & \cdots 
\end{tikzcd}
\end{center}
\end{lemma}

\begin{theorem}[Hurewicz]
Let $X$ be $(n-1)$-connected. Then, $h_n : \pi_n(X) \to H_n(X)$ is an isomorphism if $n > 1$ and the abelianization map if $n = 1$. 
\end{theorem}

\begin{theorem}[Relative Hurewicz]
Let $(X, A)$ be $(n-1)$-connected as a pair. Then, $h_n : \pi_n(X, A) \to H_n(X, A)$ is an isomorphism if $n > 1$. 
\end{theorem}

\begin{lemma}
If $f : X \to Y$ is a weak homotopy equivalence then $f_* : H_n(X) \to H_n(Y)$ is an isomorphism.
\end{lemma} 

\begin{proof}
Assume that $n > 0$ and $X$ and $Y$ are path-connected. Replace $Y$ with $M_f$ as before and consider the pair $(M_f, X)$. Since $f$ is a weak homotopy equivalence then $f_* : \pi_n(X) \xrightarrow{\sim} \pi_n(Y)$ is an isomorphism so by the long exact sequence of a pair, $\pi_n(M_f, X) = 0$ for each $n$. Using the long exact homology sequence of a pair, it suffices to prove that $H_n(M_f, X) = 0$. \bigskip\\
Let $\alpha \in H_n(M_f, X)$ be $\alpha = \sum n_i \sigma_i$ for $\sigma_i : \Delta^n \to M_f$. We can construct a $\Delta$-complex $K$ by taking disjoin $\Delta^n$ and gluging along a face whenever $\sigma_i |_{\text{face}} = \sigma_{i'} |_{\text{face}}$. In particular, there is an induced map $\sigma : K \to M_f$ such that $\sigma(L) = X$ if $L$ is the boundary of $K$. Thus, $\sigma : (K, L) \to (M_f, X)$ and $\bar{\alpha} \in H_n(K,L)$ such that $\sigma_*(\bar{\alpha}) = \alpha$. By the compression lemma, $\sigma$ is homotopic to some $\sigma' : (K, L) \to (X, X)$ and thus $\alpha = \sigma_*(\alpha) = \sigma_*(\alpha) = 0$.  
\end{proof}

Now we give a proof of Hurewicz's Theorem.

\begin{proof}
By CW approximation and the previous lemma, we can assume $X$ is a CW complex with $X^0 = \ast$ and $X$ has no $q$-cells for $0 < q < n$ since $X$ is $(n-1)$-connected. Also, $H_n(X^{n+1}) \xrightarrow{\sim} H_n(X)$ and $\pi_n(X^{n+1}) \xrightarrow{\sim} \pi_n(X)$ so assume that $X = X^{n+1}$. (WORK IN PROGRESS) 
\end{proof}

\begin{corollary}
Let $X$ be $(n-1)$-connected then $X$ is $k$-connected for each $k < n$. Therefore, by the Hurewicz theorem, $H_k(X) \cong \pi_k(X) = 0$. 
\end{corollary}

\begin{theorem}[Whitehead on Homology]
A map $f : X \to Y$ of simply-connected CW complexes that induces isomorphisms on all homology groups is a homotopy equivalence.  
\end{theorem}

\begin{proof}
Replace $Y$ with the mapping cyclinder $M_f$ and consider the map,
\begin{center}
\begin{tikzcd}
X \arrow[r, "\iota"] & M_f \arrow[r, "r"] & Y
\end{tikzcd}
\end{center}
We know that $\pi_1(M_f, X) = 0$ because $X$ and $Y$ are simply connected and using the long exact sequence. Therefore, $(M_f, X)$ is $1$-connected so by Hurewicz, $h_2 : \pi_2(M_f, X) \xrightarrow{\sim} H_2(M_f, X)$ is an isomorphism. However, we can calculate $H_2(M_f, X) = 0$ using the long exact sequence,
\begin{center}
\begin{tikzcd}
H_2(X) \arrow[r, "\sim"] & H_2(M_f) \arrow[r] & H_2(M_f, X) \arrow[r] & H_1(X) \arrow[r, "\sim"] & H_1(M_f)
\end{tikzcd}
\end{center}
where these maps are isomorphisms by assumption. Therefore, $H_2(M_f, X) = 0$ so $\pi_2(M_f, X) = 0$ so the pair is $2$-connected and thus by Hurewicz, $h_3 : \pi_3(M_f, X) \xrightarrow{\sim} H_3(M_f, X)$ is an isomorphism. By induction, we have that $\pi_n(M_f, X) = 0$ for each $n$. Therefore, by the long exact sequence $f_* : \pi_n(X) \xrightarrow{\sim} \pi_n(M_f) \cong \pi_n(Y)$ is an isomorphism for each $n$ so by Whitehead's theorem $f$ is a homotopy equivalence.
\end{proof}

 

\subsection{K\"{u}nneth Theorem}

\subsubsection{Derived Functors}

\begin{definition}
Given an $R$-module $M$ a free resolution of $M$ is an exact sequence,
\begin{center}
\begin{tikzcd}
\cdots \arrow[r] & F_2 \arrow[r] & F_1 \arrow[r] & F_0 \arrow[r] & M \arrow[r] & 0
\end{tikzcd}
\end{center}
where each $F_i$ is free. In particular, if $R$ is a field then we can take $F_0 = M$ and $F_i = 0$ for $i > 0$. If $R$ is a PID then any $R$-submodule of a free module is free. 
\end{definition}

\begin{definition}
Let $\mathcal{F}$ be a right-exact functor. Take a free resolution of $M$,
\begin{center}
\begin{tikzcd}
\cdots \arrow[r] & F_2 \arrow[r] & F_1 \arrow[r] & F_0 \arrow[r] & M \arrow[r] & 0
\end{tikzcd}
\end{center}
Then the left-dervied functors $L_i\mathcal{F}(M)$ of $F$ are the homology of the complex,
\begin{center}
\begin{tikzcd}
\cdots \arrow[r] & \mathcal{F}(F_2) \arrow[r] & \mathcal{F}(F_1) \arrow[r] & \mathcal{F}(F_0) \arrow[r] & 0
\end{tikzcd}
\end{center}
\end{definition}

\begin{lemma}
Given a right-exact functor $\mathcal{F}$, we have $L_0 \mathcal{F} = \mathcal{F}$.
\end{lemma}

\begin{proof}
$L_0 \mathcal{F}(M) = \mathcal{F}(F_0) / \Im{\partial_1}$. However, 
by right-exactness,
\begin{center}
\begin{tikzcd}
\mathcal{F}(F_1) \arrow[r] & \mathcal{F}(F_0) \arrow[r] & \mathcal{F}(M) \arrow[r] & 0
\end{tikzcd}
\end{center}
is exact. Therefore, $\Im{\partial_1} = \ker{\partial_0}$. However, $\mathcal{F}(M) \cong \mathcal{F}(F_0)/ \ker{\partial_0} = \mathcal{F}(F_0)/\Im{\partial_1} = L_0 \mathcal{F}(M)$. 
\end{proof}

\begin{lemma}
The functor $(-) \otimes_R N$ is right-exact where $R$ is a PID.
\end{lemma}

\begin{proof}
Let
\begin{center}
\begin{tikzcd}
K \arrow[r, "i"] & L \arrow[r, "j"] & M \arrow[r] & 0 
\end{tikzcd}
\end{center}
be exact. Consider the sequence,
\begin{center}
\begin{tikzcd}
K \otimes N \arrow[r, "i \otimes \id_N"] & L \otimes N \arrow[r, "j \otimes \id_N"] & M \otimes N \arrow[r] & 0 
\end{tikzcd}
\end{center}
Construct a map $\phi : M \times N \to L \otimes N / (i \otimes \id_N)(K \otimes M)$ by $\phi(m,n) = \ell \otimes n$ where $j(\ell) = m$ where I have used the fact that $j$ is surjective. If $\ell, \ell' \in L$ where $j(\ell) = j(\ell')$ then,
\[ \ell \otimes n - \ell' \otimes n = (\ell - \ell') \otimes n \]
However, $\ell - \ell' \in \ker{j} = \Im{i}$ so take $k \in K$ such that $i(k) = \ell - \ell'$. Thus,
\[  \ell \otimes n - \ell' \otimes n = i(k) \otimes n = (i \otimes \id_N)(k \otimes n) = 0 \]
in the quotient. By the universal property of the tensor product, there exists a linear map,
\[ \tilde{\phi} : M \otimes N \to  L \otimes N / (i \otimes \id_N)(K \otimes M) \]
Furthermore, $\tilde{\phi}$ is the inverse map to $\j \otimes \id_N$ on the quotient. Therefore, $\ker{j \otimes \id_N}$ is exactly $\Im{i \otimes \id}$. 
\end{proof}

\begin{definition}
Given the above, define, $\Tor{R}{n}{-}{N}$ the $n^\mathrm{th}$ left-derived functor of $(-) \otimes_R N$.
\end{definition}

\begin{proposition}
Properties of the $\mathrm{Tor}$ functor,
\begin{enumerate}
\item $\Tor{R}{n}{\bigoplus_{i} M_i}{N} \cong \bigoplus_{i} \Tor{R}{n}{M_i}{N}$

\item If $M$ or $N$ is free then $\Tor{R}{n}{M}{N} = 0$ for $n > 0$.

\item If $r \in R$ is not a zero divisor, then,
\[ \Tor{R}{1}{R/(r)}{N} \cong \{n \in N \mid rn = 0 \} \]
the $r$-torsion of $N$ and,
\[ \Tor{R}{n}{R/(r)}{N} = 0 \]
for $n > 1$.
\end{enumerate}
\end{proposition}

\begin{proposition}
$\mathrm{Tor}$ is symmetric, $\Tor{R}{n}{M}{N} \cong \Tor{R}{n}{N}{M}$.
\end{proposition}

\begin{proposition}
Given a short exact sequence of $R$-modules,
\begin{center}
\begin{tikzcd}
0 \arrow[r] & K \arrow[r] & L \arrow[r] & M \arrow[r] & 0
\end{tikzcd}
\end{center}
then we get a long exact sequence,
\begin{center}
\begin{tikzcd}[column sep = small]
\cdots \arrow[r] & \Tor{R}{1}{K}{N} \arrow[r] & \Tor{R}{1}{L}{N} \arrow[r] & \Tor{R}{1}{M}{N} \arrow[r] & K \otimes N \arrow[r] & L \otimes N \arrow[r] & M \otimes N \arrow[r] & 0
\end{tikzcd}
\end{center}
\end{proposition}



\subsubsection{Tensor Products of Chain Complexes}

\begin{definition}
Let $A$ and $B$ be graded $R$-modules. Define the tensor product as a graded $R$-module as,
\[ (A \otimes B)^n = \bigoplus_{p+q = n} A^p \otimes B^q \]
\end{definition}

\begin{definition}
Let $C$ and $D$ be chain complexes of $R$-modules. Define the chain complex $C \otimes D$ by $(C \otimes D)_n = (C_n \otimes D_n)$ with the boundary map,
\[ \partial_r (x \otimes y) = (\partial x) \otimes y + (-1)^r x \otimes (\partial y) \] 
\end{definition}

\begin{theorem}
Let $C$ and $D$ be chain complexes of $R$-modules where $R$ is a PID. Then there are natual exact sequences,
\begin{center}
\begin{tikzcd}[column sep = small]
0 \arrow[r] & (H_\ast(C) \otimes H_\ast(D))^n \arrow[r] & H_n(C \otimes D) \arrow[r] & \bigoplus\limits_{p + q = n - 1} \mathrm{Tor}^R_1 (H_p(C), H_q(D)) \arrow[r] & 0
\end{tikzcd}
\end{center}
\end{theorem}

\begin{proof}
If the complex $C$ has zero boundary maps then we have seen that,
\[ \bigoplus\limits_{p+q = n} H_p(C) \otimes H_q(D) \xrightarrow{\sim} H_n(C \otimes D) \]
Let $Z_i \subset C_i$ be the cycles and $B_i \subset C_i$ the boundaries. We can equip them with a complex structure with zero boundary maps. The following diagram commutes 
\begin{center}
\begin{tikzcd}
& \vdots \arrow[d] &  \vdots \arrow[d] &  \vdots \arrow[d] &
\\
0 \arrow[r]  & Z_{i+1} \arrow[d, "0"] \arrow[r] & C_{i+1} \arrow[d, "\partial"] \arrow[r, "\partial"] & B_{i} \arrow[d, "0"] \arrow[r] & 0
\\
0 \arrow[r] & Z_i \arrow[d, "0"] \arrow[r] & C_i \arrow[d, "\partial"] \arrow[r, "\partial"] & B_{i-1} \arrow[d, "0"] \arrow[r] & 0
\\
0 \arrow[r] & Z_{i-1} \arrow[d] \arrow[r] & C_{i-1} \arrow[d, "\partial"] \arrow[r, "\partial"] & B_{i-2} \arrow[d] \arrow[r] & 0
\\
& \vdots & \vdots & \vdots 
\end{tikzcd}
\end{center}
which gives a short exact sequence of complexes,
\begin{center}
\begin{tikzcd}
0 \arrow[r] & Z \arrow[r] & C \arrow[r] & B[-1] \arrow[r] & 0
\end{tikzcd}
\end{center}
Since $R$ is a PID we know that any submodule of a free module is free. Thus, $Z$ and $B$ are free. Therefore, $(-) \otimes D$ is an exact functor so we get the exact sequence,
\begin{center}
\begin{tikzcd}
0 \arrow[r] & Z \otimes D \arrow[r] & C \otimes D \arrow[r] & B[-1] \otimes D \arrow[r] & 0
\end{tikzcd}
\end{center}
This short exact sequence gives rise to a long exact sequence of homology,
\begin{center}
\begin{tikzcd}[column sep = small]
H_{n+1}(B[-1] \otimes D) \arrow[r, "i_{n+1}"] & H_n(Z \otimes D) \arrow[r] & H_n(C \otimes D) \arrow[r] & H_n(B[-1] \otimes D) \arrow[r, "i_n"] & H_{n-1}(Z \otimes D)
\end{tikzcd}
\end{center}
By the special case, $H_{n+1}(B[-1] \otimes D) \cong H_{n}(B \otimes D) \cong \bigoplus\limits_{p+q = n} H_{p}(B) \otimes H_{q}(D)$ and likewise $H_n(Z \otimes D) \cong \bigoplus\limits_{p+q = n} H_p(Z) \otimes H_q(D)$. Therefore, we get, 
\begin{center}
\begin{tikzcd}[column sep = small]
\bigoplus\limits_{p+q = n} H_{p}(B) \otimes H_{q}(D) \arrow[r, "i_{n+1}"] & \bigoplus\limits_{p+q = n} H_n(Z) \otimes H_n(D) \arrow[r, "j"] & H_n(C \otimes D) \arrow[r] & H_{n-1}(B \otimes D)
\end{tikzcd}
\end{center}
The map $i_{n+1}$ is induced by the inclusion $B_p \hookrightarrow Z_p$. Consider the exact sequence,
\begin{center}
\begin{tikzcd}
0 \arrow[r] & \coker{i_{n+1}} \arrow[r] & H_n(C \otimes D) \arrow[r] & \ker{i_{n+1}} \arrow[r] & 0
\end{tikzcd}
\end{center}
From the above long exact sequence we see that the cokernel of this map equals,
\begin{align*}
\coker{i_{n+1}} & = H_n(Z \otimes D) / i_{n+1}\left( \bigoplus\limits_{p+q = n} H_p(Z) \otimes H_q(D) \right) \cong  \bigoplus\limits_{p+q = n} \left[ \left( Z_p / i(B_p) \right) \otimes H_q(D) \right] 
\\
& \cong \bigoplus\limits_{p+q = n} H_p(C) \otimes H_q(D)
\end{align*}
Note that the sequence,
\begin{center}
\begin{tikzcd}
0 \arrow[r] & B_p \arrow[r] & Z_p \arrow[r] & H_p(C) \arrow[r] & 0
\end{tikzcd}
\end{center}
is exact by definition. Therefore, applying the functor $\Tor{R}{n}{-}{H_q(D)}$ gives a long exact sequence,
\begin{center}
\begin{tikzcd}[column sep = small]
\Tor{R}{1}{Z_p}{ H_q(D)} \arrow[r] & \Tor{R}{1}{H_p(C)}{H_q(D)} \arrow[r] & B_p \otimes H_q(D) \arrow[r] & Z_p \otimes H_q(D) \arrow[r] & H_p(C) \otimes H_q(D)
\end{tikzcd}
\end{center}
However, $Z_p$ is free so $\Tor{R}{1}{Z_p}{H_p(D)} = 0$. Furthermore, since the chain maps are zero, $Z_p = H_p(Z)$ and $B_p = H_p(B)$. Thus, taking the direct sum over such sequences, we get an exact sequence,
\begin{center}
\begin{tikzcd}[column sep = small]
0\arrow[r] & \bigoplus\limits_{p+q = n} \Tor{R}{1}{H_p(C)}{H_q(D)} \arrow[r] & \bigoplus\limits_{p+q = n} H_p(B) \otimes H_q(D) \arrow[r, "i_{n+1}"] & \bigoplus\limits_{p+q = n} H_p(Z) \otimes H_q(D) 
\end{tikzcd}
\end{center} 
Therefore, $\ker{i+1} = \bigoplus\limits_{p+q = n} \Tor{R}{1}{H_p(C)}{H_q(D)}$. 
Plugging these into the cokernel-kernel short exact sequence, we arrive at,
\begin{center}
\begin{tikzcd}
0 \arrow[r] & \bigoplus\limits_{p+q = n} H_p(C) \otimes H_q(D) \arrow[r] & H_n(C \otimes D) \arrow[r] & \bigoplus\limits_{p+q = n} \Tor{R}{1}{H_p(C)}{H_q(D)} \arrow[r] & 0
\end{tikzcd}
\end{center}

\end{proof}

\begin{corollary}[Universal Coefficient Theorem]
Let $R = \Z$, let $G$ be a $\Z$-module i.e. any abelian group then there exists a split short exact sequnce,
\begin{center}
\begin{tikzcd}
0 \arrow[r] & H_n(X) \otimes_{\Z} G \arrow[r] & H_n(X ; G) \arrow[r] & \Tor{\Z}{1}{H_{n-1}(X)}{G} \arrow[r] & 0
\end{tikzcd}
\end{center}
\end{corollary}

\begin{proposition}
If $X$ and $Y$ are CW complexes then, 
\[C^{\text{CW}}(X \times Y) \cong C^{\text{CW}}(X) \otimes C^{\text{CW}}(Y)\]  
\end{proposition}

\begin{proof}
Let $X \times Y$ has a CW structure,
\[(X \times Y)^n = \bigcup_{p + q = n} X^p \times Y^q \]
and the $n$-cells are labelled by pairs $(i, j)$ where $i$ is a $p$-cell of $X$ and $j$ is a $q$-cell of $Y$. Define a map, 
\[\kappa : C^{\text{CW}}(X) \otimes C^{\text{CW}}(Y) \to C^{\text{CW}}(X \times Y) \]
by the formula $\kappa([i] \otimes [j]) = (-1)^{pq} [(i,j)]$. I claim that $\kappa_n$ is an isomorphism of gradded $R$-modules which commutes with the differentials and thus an isomorphism of chain complexes.  
\end{proof}

\begin{theorem}[K\"{u}nneth]
Let $X$ and $Y$ be spaces and let $R$ be a PID. Then there exist natural short exact sequences,
\begin{center}
\begin{tikzcd}[column sep = small]
0 \arrow[r] & (H_\ast(X) \otimes H_\ast(Y))^n \arrow[r] & H_n(X \times Y) \arrow[r] & \bigoplus\limits_{p + q = n - 1} \mathrm{Tor}^R_1 (H_p(X), H_q(Y)) \arrow[r] & 0
\end{tikzcd}
\end{center}
which splits unnaturally. 
\end{theorem}


\begin{proof}
By CW approximation, we can assume that $X$ and $Y$ are CW complexes since weak homotopy equivalence preserves homology. By the above proposition, $H_n(X \times Y) = H_n(C(X) \otimes C(Y))$. Therefore, the above theorem we recover the required exact sequence. 
\end{proof}

\begin{remark}
If $R$ is a field then $\mathrm{Tor}^R_1$ vanishes.
\end{remark}

\section{Cohomology}

\begin{definition}[Singular Cohomlogy]
Define the chain complex $C^{\ast}(X ; G) = \Hom{C_{\ast}}{G}$ with chain maps induced via hom on the complex $C_{\ast}$ the standard singular chains, that is, $\delta_n : C^{n}(X ; G) \to C^{n+1}(X ; G)$ which acts as $\delta(f) = f \circ \partial_{C_{\ast}}$. This is known as a cochain complex. Then, the (singular) cohomology of $X$ are the groups $H^n(X ; G) = \ker{\delta_{n}} / \Im{\delta_{n-1}}$.  
\end{definition}

\begin{theorem}
There exists a short exact sequence,
\begin{center}
\begin{tikzcd}
0 \arrow[r] & \Ext{1}{G}{H_{n-1}(X)}{G} \arrow[r] & H^n(X ; G) \arrow[r] & \Hom{H_n(X)}{G} \arrow[r] & 0
\end{tikzcd}
\end{center}
\end{theorem}

\begin{definition}
Let $\phi \in C^k(X ; R)$ and $\phi \in C^{\ell}(X ; R)$ where $R$ is a ring then we can define the cup product which gives a function $(\phi \smile \psi) : C_{k + \ell} \to R$ and thus an element $(\phi \smile \psi) \in C^{k + \ell}$ which acts as,
\[ (\phi \smile \psi)([v_0, \cdots, v_{k + \ell}]) = \phi([v_0, v_{k}]) \cdot \psi([v_k, \cdots, v_{k+\ell}]) \] 
\end{definition}

\begin{lemma}
The cup product interacts with the boundary map via,
\[\delta(\psi \smile \psi) = (\delta \psi) \smile \psi + (-1)^k \psi \smile (\delta \psi) \]
\end{lemma}

\begin{proposition}
Cup products descend to a product structure on cohomology.
\[\smile \: : H^k(X ; R) \times H^{\ell}(X ; R) \to H^{k + \ell}(X ; R) \]
\end{proposition}

\begin{proposition}
Properties of the cup product,
\begin{enumerate}
\item $\smile$ is associative.
\item $\smile$ distributes over addition.
\end{enumerate}
\end{proposition}

\begin{definition}
The graded cohomology ring is $H^{\ast}(X ; R) = \bigoplus\limits_{i} H^i(X ; R)$ which is a graded ring under $(+, \smile)$. If $R$ is commutative then $\phi \smile \psi = (-1)^{k \ell} \psi \smile \phi$. 
\end{definition}

\begin{example}
Let $X = \rp^2$ with the ring $R = \Z / 2 \Z$. Direct calculation gives \[ H^i(X) = 
\begin{cases}
\Z / 2 \Z & i = 0, 1, 2
\\
0 & i > 2
\end{cases}\]
I claim that $H^{\ast}(X) \cong (\Z / 2 \Z) [\alpha] / (\alpha^3)$.
\end{example}

\end{document}

