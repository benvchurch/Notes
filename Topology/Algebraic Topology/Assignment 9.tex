\documentclass[12pt]{extarticle}
\usepackage[utf8]{inputenc}
\usepackage[english]{babel}
\usepackage[utf8]{inputenc}
\usepackage[english]{babel}
\usepackage[a4paper, total={7in, 9.5in}]{geometry}
\usepackage{tikz-cd}

 
\usepackage{amsthm, amssymb, amsmath, centernot, graphicx}
\usepackage{accents}
\DeclareMathAccent{\wtilde}{\mathord}{largesymbols}{"65}
\newcommand{\orb}[1]{\mathrm{Orb}(#1)}
\newcommand{\stab}[1]{\mathrm{Stab}(#1)}
\newcommand{\rp}{\mathbb{RP}}
\newcommand{\cp}{\mathbb{CP}}

\newcommand{\notimplies}{%
  \mathrel{{\ooalign{\hidewidth$\not\phantom{=}$\hidewidth\cr$\implies$}}}}
 
\renewcommand\qedsymbol{$\square$}
\newcommand{\cont}{$\boxtimes$}
\newcommand{\divides}{\mid}
\newcommand{\ndivides}{\centernot \mid}
\newcommand{\Z}{\mathbb{Z}}
\newcommand{\N}{\mathbb{N}}
\newcommand{\C}{\mathbb{C}}
\newcommand{\Zplus}{\mathbb{Z}^{+}}
\newcommand{\Primes}{\mathbb{P}}
\newcommand{\ball}[2]{B_{#1} \! \left(#2 \right)}
\newcommand{\Q}{\mathbb{Q}}
\newcommand{\R}{\mathbb{R}}
\newcommand{\Rplus}{\mathbb{R}^+}
\newcommand{\invI}[2]{#1^{-1} \left( #2 \right)}
\newcommand{\End}[1]{\text{End}\left( A \right)}
\newcommand{\legsym}[2]{\left(\frac{#1}{#2} \right)}
\renewcommand{\mod}[3]{\: #1 \equiv #2 \: \mathrm{mod} \: #3 \:}
\newcommand{\nmod}[3]{\: #1 \centernot \equiv #2 \: mod \: #3 \:}
\newcommand{\ndiv}{\hspace{-4pt}\not \divides \hspace{2pt}}
\newcommand{\finfield}[1]{\mathbb{F}_{#1}}
\newcommand{\finunits}[1]{\mathbb{F}_{#1}^{\times}}
\newcommand{\ord}[1]{\mathrm{ord}\! \left(#1 \right)}
\newcommand{\quadfield}[1]{\Q \small(\sqrt{#1} \small)}
\newcommand{\vspan}[1]{\mathrm{span}\! \left\{#1 \right\}}
\newcommand{\galgroup}[1]{Gal \small(#1 \small)}
\newcommand{\sm}{\! \setminus \!}
\newcommand{\topo}{\mathcal{T}}
\newcommand{\base}{\mathcal{B}}
\renewcommand{\bf}[1]{\mathbf{#1}}
\renewcommand{\Im}[1]{\mathrm{Im} \: #1}
\renewcommand{\empty}{\varnothing}
\newcommand{\id}{\mathrm{id}}
\newcommand{\Hom}[2]{\mathrm{Hom}\left( #1, #2 \right)}
\newcommand{\Tor}[4]{\mathrm{Tor}^{#1}_{#2} \left( #3, #4 \right)}

\renewcommand{\theenumi}{(\alph{enumi})}

\newcommand{\atitle}[1]{\title{% 
	\large \textbf{Mathematics GU4053 Algebraic Topology
	\\ Assignment \# #1} \vspace{-2ex}}
\author{Benjamin Church }
\maketitle}

\newcommand{\hook}{\hookrightarrow}


\theoremstyle{remark}
\newtheorem*{remark}{Remark}

\theoremstyle{definition}
\newtheorem{theorem}{Theorem}[section]
\newtheorem{lemma}[theorem]{Lemma}
\newtheorem{proposition}[theorem]{Proposition}
\newtheorem{corollary}[theorem]{Corollary}
\newtheorem{example}[theorem]{Example}


\newenvironment{definition}[1][Definition:]{\begin{trivlist}
\item[\hskip \labelsep {\bfseries #1}]}{\end{trivlist}}

\tikzset{
    labl/.style={anchor=south, rotate=90, inner sep=.5mm}
}


\begin{document}
\atitle{9}

 
\section*{Problem 1.}

\begin{enumerate}
\item Let $X$ be a path-connected space and $A$ a finite set of points of $X$. Consider the long exact sequence of relative homology generated by the pair $(X, A)$,
\begin{center}
\begin{tikzcd}[column sep = small]
\cdots \arrow[r, "\delta"] & \tilde{H}_1(A) \arrow[r, "\iota_*"] & \tilde{H}_1(X) \arrow[r, "j_*"] & H_1(X, A) \arrow[r, "\delta"] & \tilde{H}_0(A) \arrow[r, "\iota_*"] & \tilde{H}_0(X) \arrow[r, "j_*"] & H_0(X, A) \arrow[r] & 0
\end{tikzcd}
\end{center} 
Because $X$ is path-connected, we know that $\tilde{H}_0(X) = 0$ so the exactness at,
\begin{center}
\begin{tikzcd}
0 \arrow[r] & H_0(X, A) \arrow[r] & 0
\end{tikzcd}
\end{center}
implies that $H_0(X, A) = 0$. Furthermore, for $n > 1$ we know that $\tilde{H}_n(A) = 0$ since $A$ is a collection of points. Therefore, the long exact sequence gives rise to the short exact sequences,
\begin{center}
\begin{tikzcd}
0 \arrow[r] & \tilde{H}_n(A) \arrow[r] & \tilde{H}_n(X, A) \arrow[r] & 0
\end{tikzcd}
\end{center}
which implies that $H_n(X, A) \cong \tilde{H}_n(X)$. Finally, consider the case $n = 1$, 
\begin{center}
\begin{tikzcd}[column sep = small]
0 \arrow[r, "\iota_*"] & \tilde{H}_1(X) \arrow[r, "j_*"] & H_1(X, A) \arrow[r, "\delta"] & \tilde{H}_0(A) \arrow[r, "\iota_*"] \arrow[l, bend left, "f"] & 0
\end{tikzcd}
\end{center}
We will construct a map $f : \tilde{H}_0(A) \to H_1(X, A)$ such that $\delta \circ f = \id_{\tilde{H}_0(A)}$. The relative homology groups is constructed as,
\[ \tilde{H}_0(A) = \ker{\epsilon} / \Im{\partial_1} \]
However. $A$ is a discrete set so any map $\sigma : \Delta^1 \to A$ is constant and therefore, $\partial_1 \sigma = 0$ so $\partial_1 = 0$. Furthermore,
\[ \epsilon \left( \sum_{a \in A} n_a \, [a] \right) = \sum_{a \in A} n_a \]
so the kernel is the set generated by elements $[a_i] - [a_0]$. Thus, we can construct the map $f$ by its action on these generators,
\[ f([a_i] - [a_0]) = \sigma_i \]
where $\sigma_i$ is some choice of path from $a_0$ to $a_i$ which exists due to path-connectedness. This is a well-defined homomorphism $\tilde{H}_0(A) \to H_1(X, A)$ because $\sigma_i$ has boundary in $C_0(A)$ so it is an element of the relative homology. Furthermore,
\[ \delta \circ f([a_i] - [a_0]) = \delta(\sigma_i) = [a_i] - [a_0] \]
and therefore, extending $f$ to a homomorphism, we see that $\delta \circ f = \id_{\tilde{H}_0(A)}$. Therefore, the sequence splits so,
\[ H_1(X, A) \cong \tilde{H}_1(X) \oplus \tilde{H}_0(A) \cong \tilde{H}_1(X) \oplus \Z^{k-1} \]
where $|A| = k$ since $H_0(A) \cong \Z^k$, the number of path components, and relative homology reduces this factor by $1$. 
In summary,

\[ H_n(X, A) \cong
\begin{cases}
\tilde{H}_n(X) & n \neq 1 \\
\tilde{H}_1(X) \oplus \Z^{k-1} & n = 1
\end{cases} \]

Explicitly, for the case $X = S^2$ we know that,

\[ \tilde{H}_n(X) \cong
\begin{cases}
\Z & n = 2 \\
0 & n \neq 2
\end{cases} \]

so we can compute,

\[ H_n(S^2, A) \cong
\begin{cases}
\Z & n = 2 \\
\Z^{k-1} & n = 1 \\
0 & n \neq 1, 2
\end{cases} \]

Likewise, for the case $X = T^2 = S^1 \times S^1$ we know that,

\[ \tilde{H}_n(X) 
\cong
\begin{cases}
\Z    &  n = 2 \\
\Z \oplus \Z &  n = 1 \\
0 & n \neq 1, 2
\end{cases} \]

so we can compute,

\[ H_n(T^2, A) = \begin{cases}
\Z & n = 2 \\
\Z^{k+1} & n = 1 \\
0 & n \neq 1, 2
\end{cases} \]
  

\item Both $(X, A)$ and $(X, B)$ are good pairs. Therefore,  

\[ H_n(X, A) \cong \tilde{H}_n(X/A) \]
However, $X/A$ is the wedge of two tori. Therefore,
\[ H_n(X, A) \cong \tilde{H}_n(X/A) = \tilde{H}_n \left( T^2 \vee T^2 \right) \cong \tilde{H}_n(T^2) \oplus \tilde{H}_n(T^2) = 
\begin{cases}
\Z \oplus \Z & n = 2 \\
\Z \oplus \Z \oplus \Z \oplus \Z & n = 1 \\
0 & n \neq 1,2 
\end{cases} \]
Furthermore, $X/B$ is homotopic to the wedge of a torus and a circle. Thus, again using the fact that $H_n(X, B) \cong \tilde{H}_n(X/B)$,
\[ H_n(X, B) \cong \tilde{H}_n(X/B) = \tilde{H}_n \left( T^2 \vee S^1 \right) \cong \tilde{H}_n(T^2) \oplus \tilde{H}_n(S^1) = 
\begin{cases}
\Z & n = 2 \\
\Z \oplus \Z \oplus \Z & n = 1 \\
0 & n \neq 1,2 
\end{cases} \]
\end{enumerate}

\section*{Problem 2.}

Consider the subspace $\Q \subset \R$. The pair $(\R, \Q)$ gives rise to the long exact sequence,

\begin{center}
\begin{tikzcd}[column sep = small]
\cdots \arrow[r, "\delta"] & H_{1}(\Q) \arrow[r, "\iota_*"] & H_{1}(\R) \arrow[r, "j_*"] & H_{1}(\R, \Q) \arrow[r, "\delta"] & H_0(\Q) \arrow[r, "\iota_*"] & H_0(\R) \arrow[r, "j_*"] & H_0(\R, \Q) \arrow[r] & 0
\end{tikzcd}
\end{center} 
However, $H_1(\R) = 0$ and $H_0(\R) \cong \Z$ because $\R$ is contractible. Furthermore,
\[ H_0(\Q) = \ker{\partial_0} / \Im{\partial_1} = C_0(\Q) / \Im{\partial_1} \]
However, if $\sigma : \Delta^1 \to \Q$ is continuous then $\Im{\sigma}$ is connected and thus $\Im{\sigma} = \{x_0\}$ so $\sigma$ is constant. Thus, $\partial_1 \sigma = 0$ so $\partial_1 = 0$. Therefore, $H_0(\Q) = C_0(\Q) = \Z^{\Q}$. Therefore, we have the exact sequence,
\begin{center}
\begin{tikzcd}
0 \arrow[r] & H_1(\R, \Q) \arrow[r, "\delta"] & \Z^{\Q} \arrow[r, "i_*"] & \Z 
\end{tikzcd}
\end{center}
The map $i_{\#} : C_0(\Q) \to C_0(\R)$ acts as the inclusion on generators. Therefore, $i_{*} : H_0(\Q) \to H_0(\R)$ takes generators to generators. However, $H_0(\R) \cong \Z$ so there is a single generator. Therefore, 
\[ i_* \left( \sum_{q \in \Q} n_q \: [q] \right) = \sum_{q \in \Q} n \]
where $n_q = 0$ for all but finitely many values. Thus,
\[ \ker{i_*} = \left\{ \sum_{q \in \Q} n_q \: [q] \quad \middle| \quad \sum_{q \in \Q} n_q = 0  \right\} \]
From the exact sequence, we see that $\Im{\delta} = \ker{i_*}$ and $\ker{\delta} = 0$ so $\Im{\delta} \cong H_1(\R, \Q)$.
Therefore,
\[ H_1(\R, \Q) \cong \Im{\delta} = \ker{i_*} = \left\{ \sum_{q \in \Q} n_q \: [q] \quad \middle| \quad \sum_{q \in \Q} n_q = 0  \right\}  \subset \bigoplus_{q \in \Q} \Z \]
We can give an explicity basis,
\[ \{ ([q] - [0]) \mid q \in \Q \sm \{0\} \} \]
Because given an element,
\[ \sum_{q \in \Q} n_q \: [q] \quad \text{such that} \quad \quad \sum_{q \in \Q} n_q = 0\]
then we can write,
\[ \sum_{q \in \Q} n_q \: [q] = \sum_{q \in \Q} n_q \: ([q] - [0]) +  \sum_{q \in \Q} n_q \: [0] = \sum_{q \in \Q} n_q \: ([q] - [0]) \]
Clearly, any linear combination of these basis elements is in the kernel of $i_*$. 

\section*{Problem 3.}

We know that the suspension is a union of cones $SX = C_{+}X \cup C_{-} X$ whose intersection is $X$. Take $A = C_{+} X$ and $B = C_{-} X$. Since $C_{+} X$ is contractible, by Lemma \ref{contractible_pair} we know that $\tilde{H}_n(S X) \cong \tilde{H}_n(S X, C_{+} X)$. However, by Excision, we know that $\tilde{H}_n(B, A \cap B) \cong \tilde{H}_n(X, A)$ and therefore, 
\[\tilde{H}_n(C_{-} X, X) \cong \tilde{H}_n(S X, C_{+}) \cong \tilde{H}_n(S X)\]
Furthermore, consider the pair $(C_{-} X, X)$. Since $C_{-} X$ is contractible, by Lemma \ref{contractible_base}, we know that, \[\tilde{H}_{n+1}(C_{-} X, X) \cong \tilde{H}_{n}(X)\]
Putting these results together, we find that,
\[ \tilde{H}_n(X) \cong \tilde{H}_{n+1}(S X) \]
\bigskip\\
Now consider the problem when $Y$ is the union of $k$ cones of $X$, \[Y = \bigcup_{i = 1}^k C_{i} X\]
which all intersect at the base to form $X \subset Y$. I claim that,
\[ \tilde{H}_{n+1}(Y) \cong \bigoplus_{i = 1}^{k-1} \tilde{H}_{n+1}(S_i X) \cong \bigoplus_{i = 1}^{k-1} \tilde{H}_k(X) \]
By Excision,
\[ \tilde{H}_{n+1}(Y, C_k X) \cong \tilde{H}_{n+1}\left( \bigcup_{i = 1}^{k-1} C_i X, X \right) \]
However, the relative homology in the last line is of a good pair so,
\[ \tilde{H}_{n+1}\left( \bigcup_{i = 1}^{k-1} C_i X, X \right)  \cong \tilde{H}_{n+1}\left( \left[ \bigcup_{i = 1}^{k-1} C_i X \right] / X \right) \cong \tilde{H}_{n+1} \left( \bigvee_{i = 1}^{k - 1} S_i X \right) = \bigoplus_{i = 1}^{k - 1} \tilde{H}_{n+1}(S_i X) \]
However, by Lemma \ref{contractible_pair}, since $C_k X$ is contractible, we know that $\tilde{H}_{n+1}(Y, C_k X) \cong \tilde{H}_{n+1}(Y)$. Furthermore, using our previous result that $\tilde{H}_{n+1}(S X) \cong \tilde{H}_n(X)$ we get that,
\[ \tilde{H}_{n+1}(Y) \cong \tilde{H}_{n+1}(Y, C_k X) \cong \bigoplus_{i = 1}^{k - 1} \tilde{H}_{n+1}(S_i X) \cong \bigoplus_{i = 1}^{k - 1} \tilde{H}_{n+1}(X) \]
proving the claim.

\section*{Problem 4.}

\begin{enumerate}
\item
Suppose we have a morphism of pairs $f : (X, A) \to (Y, B)$ such that $f : X \to Y$ and $f : A \to B$ are homotopy equivalences. The long exact sequence of pairs is natural. Therefore, given a map of pairs $f : (X, A) \to (Y, B)$ we get a morphism of long exact sequences $f_{\#}$ such that the following diagram commutes,
\begin{center}
\begin{tikzcd}[column sep = small]
\cdots \arrow[r] & H_{n+1}(A) \arrow[d, "f_{*}"] \arrow[r] & H_{n+1}(X) \arrow[d, "f_{*}"] \arrow[r] & H_{n+1}(X, A) \arrow[d, "f_{*}"] \arrow[r] & H_n(A) \arrow[d, "f_{*}"] \arrow[r] & H_n(X) \arrow[d, "f_{*}"] \arrow[r] & H_{n}(X, A) \arrow[d, "f_{*}"] \arrow[r] & \cdots
\\
\cdots \arrow[r] & H_{n+1}(B) \arrow[r] & H_{n+1}(Y) \arrow[r] & H_{n+1}(Y, B) \arrow[r] & H_n(B) \arrow[r] & H_n(Y) \arrow[r] & H_{n}(Y, A)  \arrow[r] & \cdots
\end{tikzcd}
\end{center}
For the current situation, because $f : X \to Y$ and $f : A \to B$ are homotopy equivalences we know that $f_{*} : H_{n}(X) \to H_{n}(Y)$ and $f_{*} : H_{n}(A) \to H_{n}(B)$ are isomorphisms. Consider the section of the long exact sequence, 
\begin{center}
\begin{tikzcd}
H_{n+1}(A) \ar[d, "\sim" labl, "f_{*}"] \arrow[r] & H_{n+1}(X) \ar[d, "\sim" labl, "f_{*}"] \arrow[r] & H_{n+1}(X, A) \arrow[d, "f_{*}"] \arrow[r] & H_n(A) \ar[d, "\sim" labl, "f_{*}"] \arrow[r] & H_n(X) \ar[d, "\sim" labl, "f_{*}"]
\\
H_{n+1}(B) \arrow[r] & H_{n+1}(Y) \arrow[r] & H_{n+1}(Y, B) \arrow[r] & H_n(B) \arrow[r] & H_n(Y)
\end{tikzcd}
\end{center}
Therefore, by the five-lemma, we know that $f_{*} : H_{n+1}(X, A) \to H_{n+1}(Y, B)$ is an isomorphism for each $n$. This argument also holds for $n = 0$ because the right half of the diagram is just zeros which still satisfies the isomorphism conditions. 

\item
Suppose that $f : (D^n, S^{n-1}) \hook (D^n, D^n \sm \{ 0 \} )$ is a homotopy equivalence of pairs. Then, by Lemma \ref{hom_pairs} we  know that $f : (D^n, S^{n-1}) \to (D^n, D^n)$ is a homotopy equivalence of pairs. However, since $D^n$ is contractible, by Lemma \ref{contractible_base} we know that $\tilde{H}_k(D^n, S^{n-1}) \cong \tilde{H}_{k-1}(S^{n-1})$ and $\tilde{H}_k(D^n, D^n) \cong \tilde{H}_{k-1}(D^n)$. However, $\tilde{H}_{k-1}(D^n) = 0$ for all $k$ since $D^n$ is contractible but $\tilde{H}_{n-1}(S^{n-1}) \cong \Z$ is nontrivial. Therefore, $f : (D^n, S^{n-1}) \to (D^n, D^n)$ cannot be a homotopy equivalence and thus $f : (D^n, S^{n-1}) \hook (D^n, D^n \sm \{ 0 \} )$ cannot be a homotopy equivalence.
\end{enumerate}


\section*{Problem 5.}

We define the homotopy category of chain complexes, $\mathbf{K(Ab)}$ as the category with objects as chain complexes of abelian groups and morphisms which are chain homotopy classes of morphisms of chain complexes. To show that this is well-defined, we need to show that chain homotopy is an equivalence relation and that chain homotopy respects composition.
\bigskip\\
First, if $f : C \to D$ is a morphism of chain complexes then $p_n = 0 : C_n \to D_{n+1}$ is a chain homotopy from $f$ to $f$ since,
\[ \partial_{n+1} \circ p_n + p_{n-1} \circ \partial_{n} = 0 = f_n - f_n\]
Therefore $f \simeq f$ so chain homotopy is reflexive. 
Furthermore, if $f, g : C \to D$ are chain homotopic morphisms of chain complexes such that $f \sim g$ and thus there exists a chain homotopy, $p_n : C_{n} \to D_{n+1}$ such that,
\[ \partial_{n+1} \circ p_n + p_{n-1} \circ \partial_{n} = g_n - f_n\] 
Then consider the map $(-p_n) : C_n \to D_n$ such that,
\[ \partial_{n+1} \circ (-p_n) + (-p_{n-1}) \circ \partial_{n} = -(\partial_{n+1} \circ p_n + p_{n-1} \circ \partial_{n}) = f_n - g_n\]
so $g \simeq f$. Therefore, chain homotopy is symmetric. Finally,
suppose that $f,g,h : C \to D$ are morphisms of chain complexes such that $f \simeq g$ and $g \simeq h$. Then, we have chain homotopies, $p_n : C_n \to D_{n+1}$ and $q_n : C_n \to D_{n+1}$ such that,
\[ \partial_{n+1} \circ p_n + p_{n-1} \circ \partial_n = g_n - f_n \]
and likewise,
\[ \partial_{n+1} \circ q_n + q_{n-1} \circ \partial_n = h_n - g_n\]
Then, consider the map $p_n + q_n : C_n \to D_{n+1}$. Using the above relations,
\begin{align*}
\partial_{n+1} \circ (p_n + q_n) + (p_{n-1} + q_{n-1}) \circ \partial_n  & = \partial_{n+1} \circ p_n + p_{n-1} \circ \partial_n + \partial_{n+1} \circ q_n + q_{n-1} \circ \partial_n 
\\
& = (g_n - f_n) + (h_n - g_n) = h_n - f_n
\end{align*}
Therefore, $f \simeq h$ since $p + q$ is a chain homotopy between them. Therefore, chain homotopy is an equivalence relation. We much further check that chain homotopy respects composition. Suppose that,
$f, f' : C \to D$ are chain homotopy morphisms of chain complexes and $g, g' : D \to E$ are also chain homotopic morphisms of chain complexes. Then, there exist chain homotopies, $p_n : C_n \to D_{n+1}$ and $q_n : D_n \to E_{n+1}$ such that,
\[ \partial_{n+1} \circ p_n + p_{n-1} \circ \partial_n = f'_n - f_n \]
and likewise,
\[ \partial_{n+1} \circ q_n + q_{n-1} \circ \partial_n = g'_n - g_n \] 
Using the fact that the maps $f, f', g, g'$ are all chain maps, we can simplify,
\begin{align*}
g'_n \circ f'_n - g_n \circ f_n & = g'_n \circ f'_n - g'_n \circ f_n + g'_n \circ f_n - g_n \circ f_n = g'_n \circ (f'_n - f_n) + (g'_n - g_n) \circ f_n 
\\
& = g'_n \circ (\partial_{n+1} \circ p_n + p_{n-1} \circ \partial_n) + (\partial_{n+1} \circ q_n + q_{n-1} \circ \partial_n) \circ f_n
\\
& = \partial_{n+1} \circ g'_{n+1} \circ p_n + \partial_{n+1} \circ q_n \circ f_n + g'_n \circ p_{n-1} \circ \partial_n + q_{n-1} \circ f_{n-1} \circ \partial_{n}
\\
& = \partial_{n+1} \circ (g'_{n+1} \circ p_n + q_n \circ f_n) + (g'_n \circ p_{n-1} + q_{n-1} \circ f_{n-1}) \circ \partial_n 
\end{align*}
Which shows that $r_n = g'_{n+1} \circ p_n + q_n \circ f_n : C_n \to E_{n+1}$ is a chain homotopy between $g_n \circ f_n$ and $g'_n \circ f'_n$. Therefore, $g_n \circ f_n \simeq g'_n \circ f'_n$ so chain homotopy respects composition. Therefore, the composition in the category $\mathbf{K(Ab)}$ is well defined since if $[f] =[f']$ and $[g] = [g']$ then,
$[g] \circ [f] = [g \circ f]$ and $[g'] \circ [f'] = [g' \circ f']$ but since $f \simeq f'$ and $g \simeq g'$ we know that $g \circ f \simeq g' \circ f'$ and thus, $[g \circ f] = [g' \circ f']$. So finally,
\[ [g] \circ [f] = [g'] \circ [f'] \]
so composition does not depend on representative. 

\section*{Problem 6.}
Suppose $C$ is a contractible complex i.e. such that the identity map is chain homotopic to the zero map through a chain homotopy, $p : C_n \to C_{n+1}$ such that $\partial_{n + 1} \circ p_{n} + p_{n-1} \circ \partial_n = \id_n$. Then, take any cycle $a \in C_n$ such that $\partial_n a = 0$. Using the above result, 
\[ \partial_{n + 1} \circ p_{n}(a) + p_{n-1} \circ \partial_n( a) = a \implies \partial_{n+1} (p_n(a)) = a \]
so $a \in \Im{\partial_{n+1}}$ is a boundary. Therefore, the complex is exact and therefore has trivial homology which, by definition, means that the complex is acyclic.
\bigskip\\
However, consider the sequence,
\begin{center}
\begin{tikzcd}
0 \arrow[r] & 2 \Z \arrow[r, "\iota"] & \Z \arrow[r, "\pi"] & \Z / 2 \Z \arrow[r] & 0 
\end{tikzcd}
\end{center}
which is exact with the inclusion  and quotient maps. Since this sequence is exact, it is a complex with trivial homology and thus acyclic. However, this complex is not contractible. To see this, suppose there were a chain homotopy $p$ between the identity and the zero map,
\begin{center}
\begin{tikzcd}
0 \arrow[r] \arrow[d] & 2 \Z \arrow[r, "\iota"] \arrow[d] \arrow[dl] & \Z \arrow[r, "\pi"] \arrow[d] \arrow[dl, "p_1"] & \Z / 2 \Z \arrow[r] \arrow[d] \arrow[dl, "p_2"] & 0 \arrow[d] \arrow[dl]  \\
0 \arrow[r] & 2 \Z \arrow[r, "\iota"] & \Z \arrow[r, "\pi"] & \Z / 2 \Z \arrow[r] & 0 
\end{tikzcd}
\end{center}
For this sequence of maps to give a chain homotopy, we need to have,
\[ \iota \circ p_1 + p_2 \circ \pi = \id_{\Z} \]
However, the map $p_2 : \Z / 2 \Z \to \Z$ must be trivial because $\Im{p_2}$ is a torsion group but $\Z$ has trivial torsion. Therefore, $p_2 = 0$ so we must have,
\[ \iota \circ p_1 = \id_{\Z} \]
which is clearly impossible because $\Im{\iota} = 2 \Z \subsetneq \Z$. 

\section{Lemmas}

\begin{lemma} \label{contractible_pair}
Let $(X, A)$ be a pair such that $A$ is contractible then $\tilde{H}_n(X, A) \cong \tilde{H}_n(X)$.
\end{lemma}

\begin{proof}
Consider the long exact sequence,

\begin{center}
\begin{tikzcd}[column sep = small]
\cdots \arrow[r, "\delta"] & \tilde{H}_{n+1}(A) \arrow[r, "\iota_*"] & \tilde{H}_{n+1}(X) \arrow[r, "j_*"] & \tilde{H}_{n+1}(X, A) \arrow[r, "\delta"] & H_n(A) \arrow[r, "\iota_*"] & \tilde{H}_n(X) \arrow[r, "j_*"] & \tilde{H}_n(X, A) \arrow[r] & \cdots
\end{tikzcd}
\end{center} 
However, since $A$ is contractible, we know that it has isomorphic homology to a point and thus $\tilde{H}_n(A) = 0$. Therefore, the long exact sequence gives short exact sequences,
\begin{center}
\begin{tikzcd}
0 \arrow[r] & \tilde{H}_{n}(X) \arrow[r] &  \tilde{H}_{n}(X, A) \arrow[r] & 0
\end{tikzcd}
\end{center} 
and therefore $\tilde{H}_n(X) \cong \tilde{H}_{n}(X, A)$ for each $n$.
\end{proof}

\begin{lemma} \label{contractible_base}
Let $(X, A)$ be a pair such that $X$ is contractible then $\tilde{H}_{n+1}(X, A) \cong \tilde{H}_{n}(A)$.
\end{lemma}

\begin{proof}
Consider the long exact sequence,

\begin{center}
\begin{tikzcd}[column sep = small]
\cdots \arrow[r, "\delta"] & \tilde{H}_{n+1}(A) \arrow[r, "\iota_*"] & \tilde{H}_{n+1}(X) \arrow[r, "j_*"] & \tilde{H}_{n+1}(X, A) \arrow[r, "\delta"] & \tilde{H}_n(A) \arrow[r, "\iota_*"] & \tilde{H}_n(X) \arrow[r, "j_*"] & \tilde{H}_n(X, A) \arrow[r] & \cdots
\end{tikzcd}
\end{center} 
However, since $X$ is contractible, we know that it has isomorphic homology to a point and thus $\tilde{H}_n(X) = 0$. Therefore, the long exact sequence gives short exact sequences,
\begin{center}
\begin{tikzcd}
0 \arrow[r] & \tilde{H}_{n+1}(X, A) \arrow[r] &  \tilde{H}_{n}(A) \arrow[r] & 0
\end{tikzcd}
\end{center} 
and therefore $\tilde{H}_{n+1}(X, A) \cong \tilde{H}_{n}(A)$ for each $n$.
\end{proof}

\begin{lemma} \label{hom_pairs}
If $f : (X, A) \to (Y, B)$ is a homotopy equivalence of pairs then $f : (X, \overline{A}) \to (Y, \overline{B})$ is a homotopy equivalence of pairs.
\end{lemma}

\begin{proof}
Let $H : X \times I \to Y$ be a homotopy such that $H(A \times \{ t \} ) \subset B$. Then, because $H$ is continuous, $H( \overline{A \times \{t\}} ) \subset \overline{H(A \times \{t\})} \subset \overline{B}$. Therefore, $H$ is a homotopy of pairs $(X, \overline{A})$ to $(Y, \overline{B})$.  
\end{proof}

\end{document}
