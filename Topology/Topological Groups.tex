\documentclass[12pt]{extarticle}
\usepackage{import}
\import{./}{TopologyCommands}

\newcommand{\SU}[1]{\mathrm{SU}(#1)}

\begin{document}

\section{Topological Groups}

\begin{definition}
A topological group is a group object in $\Top$. 
\end{definition}

\begin{theorem}
Let $X$ be a topological group then $\pi_1(X)$ is an abelian group.
\end{theorem}

\begin{proof}
The functor $\pi_1 : \pTop \to \Grp$ preserves products and thus preserves group objects. Thus $\pi_1(X)$ is a group object in $\Grp$ which is an abelian group. 
\end{proof}


\begin{proposition} \label{discrete_in_center}
Let $G$ be a connected topological group and $K \subset G$ a discrete normal subgroup. Then $K \subset Z(G)$. 
\end{proposition}

\begin{proof}
Consider the continuous map $G \times K \to K$ given by $(g, k) \mapsto g k g^{-1}$ which is well-defined by normality $K \triangleleft G$. For each fixed $k \in K$ consider the map $G \to K$ via $g \mapsto g k g^{-1}$. Since $G$ is connected its image is also connected in $K$ and thus is a point since $K$ is discrete. However, $1 \mapsto k$ meaning that $g k g^{-1} = k$ for all $g \in G$ and each fixed $k \in K$. Thus $K \subset Z(G)$. 
\end{proof}

\begin{proposition}
Let $H \triangleleft G$ be topological groups then the quotient $\pi : G \to G / H$ is an open homeomorphism.
\end{proposition}

\begin{proof}
A set $U \subset G / H$ is open iff $\pi^{-1}(U)$ is open. Furthermore, for any open $U \subset G$ consider,
\[ \pi^{-1}(\pi(U)) = H \cdot U = \bigcup_{h \in H} h \cdot U \]
which is a union of open sets and thus open since $h \cdot$ is a homeomorphism and thus $h \cdot U$ is open.
\end{proof}

\begin{proposition}
Let $H \subset G$ be topological groups. If $H$ is open then $H$ is closed. If $H$ is closed of finite index then $H$ is open.
\end{proposition}

\begin{proof}
Because the cosets form a disjoint cover, we may write,
\[ G \setminus H = \bigcup_{g H \in (G / H) \setminus H} g H \]
If $H$ is open then $g H$ is open because multiplication by $g$ is open (it is a homeomorphism) so $G \setminus H$ is a union of open sets and thus open i.e. $H$ is closed. If $H$ is closed then $g H$ is closed since $g$ is a closed map and if furthermore $[G : H]$ is finite then $G \setminus H$ is also closed since it is a finite union of closed sets and thus $H$ is open. 
\end{proof}

\begin{proposition}
Let $G$ be a compact topological group and $H \subset G$ an open subgroup. Then $G / H$ is finite.
\end{proposition}

\begin{proof}
The open sets $\{ g H \mid g \in G \}$ form a cover of $G$ which has a finite subcover because $G$ is compact. However, cosets are equivalence classes and thus disjoint so there must be a finite number of cosets. Thus $[G : H]$ is finite so $G / H$ is finite.  
\end{proof}

\begin{proposition}
A topological group $G$ is Hausdorff iff $1 \in G$ is a closed point.
\end{proposition}

\begin{proof}
If $G$ is Hausdorff then $G$ is $T_1$ so, in particular, $1 \in G$ is closed. Conversely, assume that $1 \in G$ is closed. Consider the continuous map $G \times G \to G$ given by $(x, y) \mapsto x y^{-1}$. The preimage of $\{ 1 \} \subset G$ under this map is the diagonal $\Delta \subset G \times G$ which is then closed. Therefore $G$ is Haudorff. 
\end{proof}

\begin{proposition}
Let $H \triangleleft G$ be topological groups then,
\[ G / H \text{ is Hausdorff} \iff H \subset G \text{ is closed} \] 
\end{proposition}

\begin{proof}
$G / H$ is Hausdorff $\iff 1 \in G / H$ is a closed point $\iff H \subset G$ is closed. 
\end{proof}

\begin{proposition}
Let $H \triangleleft G$ be topological groups then,
\[ G / H \text{ is discrete} \iff H \subset G \text{ is open} \] 
\end{proposition}

\begin{proof}
$G / H$ is discrete $\iff 1 \in G / H$ is an open point $\iff H \subset G$ is open. 
\end{proof}

\begin{proposition}
Let $H \subset G$ be a subgroup of a topological group then $\overline{H} \subset G$ is a closed subgroup. Furthermore if $H \triangleleft G$ is normal then $\overline{H} \triangleleft G$ is normal.
\end{proposition}

\begin{proof}
Let $a,b \in \overline{H}$ and consider the continuous map $f : G \times G \to G$ given by $f(x,y) = x y^{-1}$. Let $U$ be any neighborhood of $ab^{-1}$ then $f^{-1}(U)$ is open with $(a, b) \in f^{-1}(U)$ so there exist open sets $A, B \subset G$ such that $(a, b) \in A \times B \subset f^{-1}(U)$. However, since $a, b \in \overline{H}$ and $A,B$ are neighborhoods of $a,b$ repectivly, since $a$ and $b$ are closure points of $H$ then $\exists x \in A \cap H$ and $y \in B \cap H$. Thus $(x, y) \in f^{-1}(U)$ so $xy^{-1} \in H \cap U$ since $H$ is a subgroup. Since $U$ is arbitrary containing $a b^{-1}$ then $a b^{-1} \in \overline{H}$. Therefore $\overline{H}$ is a subgroup. 
\bigskip\\
Now suppose that $H \triangleleft G$ is normal. Fix $g \in G$ and consider the continous homomorphism $f_g : G \to G$ given by $f_g(x) = g x g^{-1}$. Because $f_g$ is continuous $f_g(\overline{H}) \subset \overline{f(H)}$. However, since $H$ is normal $f_g(H) = H$ and $f_g(\overline{H}) = g \overline{H} g^{-1}$ so we find $g \overline{H} g^{-1}) \subset \overline{H}$ for each $g \in G$ so $\overline{H} \triangleleft G$ is normal. 
\end{proof}



\begin{theorem} \label{quotient_condition}
If $f : G \to H$ is an open continuous surjective homomorphism of topological groups then $G / \ker{f} \cong H$ naturally.
\end{theorem}

\begin{proof}
When $G / \ker{f}$ is given the quotient topology, the canoncial map on the quotient, $G / \ker{f} \to H$, is a continuous bijective homomorphism. However, generically it may not be a homeomorphism. However, if $f : G \to H$ is open then consider,
\begin{center}
\begin{tikzcd}
G \arrow[rr, "f"] \arrow[rd, "\pi"'] &  & H 
\\
& G / \ker{f} \arrow[ru, "\tilde{f}"', dashed]
\end{tikzcd}
\end{center}
If $U \subset G / \ker{f}$ is open then $U = \pi(\pi^{-1}(U))$ since it is surjective so $\tilde{f}(U) = f(\pi^{-1}(U))$ which is open. Thus $\tilde{f}$ is a open continuous bijective and thus a homeomorphism since $\tilde{f}$ is also a homomorphism it is an isomorphism in the category of topological groups.
\end{proof}

\section{Connected Components}

\begin{proposition}
The connected components of $X$ are closed and connected. Furthermore, if there are finitely many components then they are open.
\end{proposition}

\begin{proof}
Connectedness is obvious from their maximality in the poset of connected sets and so is closure since if $Y$ is connected then $\overline{Y}$ is also connected so by maximality $Y = \overline{Y}$. 
\bigskip\\
Now, if there are finitely many connected components then the complement of one is a finite union of closed sets (the other components) and thus closed so it is open.
\end{proof}

\begin{remark}
Finiteness is necessary to ensure that the connected components are open. Accordingly, the space $\Q$ (with the Euclidean topology) has connected connected components $\{ q \}$ for $q \in \Q$ which are open but not closed. 
\end{remark}


\begin{proposition}
Let $G$ be a topological group. Then $G_0$ the connected component of $e$ is a topological subgroup. 
\end{proposition}

\begin{proof}
The map $\ell_g : G \to G$ by left multiplication is continuous. Thus, if $g \in G_0$ then consider $\ell_{g}(G_0)$ which is connected and contains $g$ so it is contained in $G_0$. Furthermore, the inversion map $i : G \to G$ is continuous so $i(G_0)$ is connected and contains $e$ so $i(G_0) \subset G_0$. Therefore $G_0$ is a subgroup. 
\end{proof}

\begin{proposition}
Let $G$ be a topological group then there is an exact sequence of topological groups,
\begin{center}
\begin{tikzcd}
1 \arrow[r] & G_0 \arrow[r] & G \arrow[r] & \pi_0(G) \arrow[r] & 1
\end{tikzcd}
\end{center}
\end{proposition}

\begin{proof}
First, note that $\pi_0 : \mathbf{Top} \to \mathbf{Set}$ is a functor respecting products and thus preserves group objects. The map $G \to \pi_0(G)$ given by sending $g$ to $[g]$ the unique connected component containing it. This map is a continuous group homomorphism when $\pi_0(G)$ is given the quotient topology. 
Now $g \in \ker{(G \to \pi_0(G))}$ iff $[g] = [e]$ iff $g \in G_0 = [e]$ so the sequence is exact. In particular, $G_0 \triangleleft G$ is normal.
\end{proof}

\begin{proposition}
Let $G$ be a topological group and $G_0 \subset G$ the connected component of the identity. Then $G_0$ is a subgroup of $G$ and the cosets correspond to the connected components via an isomorphism $G / G_0 \cong \pi_0(G)$.
\end{proposition}

\begin{proof}
Take $g \in G_0$ and consider the map $G_0 \to G$ given by $x \mapsto gx^{-1}$. Since this map is continuous and $G_0$ is connected its image is connected. However, its image contains $g$ since $1 \in G_0$ meaning that the image must lie in the connected component of $g$ which is $G_0$ since connected components partition $G$. Thus for $x,y \in G_0$ we have $xy^{-1} \in G_0$ so $G_0$ is a subgroup. 
\bigskip\\
Furthermore, the set $\pi_0(G)$ is naturally a group. This is because $G$ is a group object in $\Top$ and $\pi_0$ preserves products so $\pi_0(G)$ is a group object in $\Set$. Explicitly, multiplication is given by $[x] \cdot [y] = [x \cdot y]$ where $[x]$ is the connected component of $x \in G$. Consider the map $G \to \pi_0(G)$ via $x \mapsto [x]$. Clearly this is surjective with kernel $G_0$ so $G / G_0 \cong \pi_0(G)$.
\end{proof}

\begin{lemma}
The connected components of any manifold are open. 
\end{lemma}

\begin{proof}
Let $C \subset M$ be a connected component. Then for any $x \in C$ there is a chart $(U, \varphi)$ containing $x$. Then $\varphi(U)$ is open in $\R^n$ which is locally connected so there exists an open connected set $V$ containing $\varphi(x)$ which implies that $\tilde{V} = \varphi^{-1}(V)$ is an open connected neighborhood of $x$ so $\varphi$ is a homeomorphism. Thus, by maximality, $x\in \tilde{V} \subset C$ so $C$ is open.
\end{proof}

\begin{proposition}
Every compact Lie group is a finite extenson of a connected group. 
\end{proposition}

\begin{proof}
Let $G$ be a Lie compact group. Then $G_0$ is open since $G$ is a manifold. Therefore, $\pi_0(G) = G / G_0$ is finite since the cosets form a disjoint open cover. Then the sequence,
\begin{center}
\begin{tikzcd}
1 \arrow[r] & G_0 \arrow[r] & G \arrow[r] & \pi_0(G) \arrow[r] & 1
\end{tikzcd}
\end{center}
makes $G$ a finite extension of $G_0$. 
\end{proof}

\begin{remark}
The requirement that $G$ be a manifold is necessarly. For example $\Z_p$ is a compact topological group but it is totally disconnected and points are not open and it is infinite. 
\end{remark}

\subsection{Covering Groups}


\section{Manifolds with any Finite Fundamental Group}

\begin{remark}
For loops $\gamma_1, \gamma_2 : I \to X$ we will use the notation $\gamma_1 * \gamma_2$ to denote the loop, \[h(t) = \begin{cases} \gamma_1(2t) & t \le \frac{1}{2} \\ \gamma_2(2t - 1) & t \ge \frac{1}{2} \end{cases}\]
\end{remark}

\begin{definition}
An action of of a group $G$ on a topological space $X$ is a homomorphism $A : G \to \mathrm{Homeo}(X)$. Equivalently, one may define a map $\varphi : G \times X \to X$ and let $\varphi_g(x) = \varphi(g, x)$ such that $\varphi_e = \mathrm{id}_X$ and $\varphi_{gh} = \varphi_g \circ \varphi_h$ and $\varphi_g$ is a continuous map. Because $\varphi_{g^{-1}}$ is also continuous and $\varphi_g \circ \varphi_{g^{-1}} = \varphi_{g^{-1}} \circ \varphi_{g} = \varphi_e = \mathrm{id}_e$ then each map is a homeomorphism of $X$ to itself so $g \mapsto \varphi_g$ is a homomorphism from $G$ to $\mathrm{Homeo(X)}$.  
\end{definition}

\begin{definition}
Let $G$ be a group acting on a topological space $X$ then $X/G$ is the quotient space under the equivalence relation $x \sim y \iff \exists g \in G : g \cdot x = y$. 
\end{definition}

\begin{remark}
For $x \in X$, let $[x]_G$ denote the equivalence class under a group action and for $\gamma : I \to X$ let $[\gamma]$ denote the equivalence class under path-homotopy. 
\end{remark}

\begin{definition}
A group $G$ acts freely on a set $X$ if every stabilizer is trivial. Equivalently, if for some $x \in G$ we have $g \cdot x = h \cdot x$ then $(h^{-1}g) \cdot x = x$ so $g = h$. 
\end{definition}

\begin{definition}
A group action on $X$ is properly discontinuous if for any $x \in X$ there exists an open neighborhood $x \in U$ such that $(g \cdot U) \cap U = \empty$ for each $g \neq e$. 
\end{definition}

\begin{lemma}
Let the action of $G$ on $X$ be properly discontinuous, then $X$ is a covering space of $X/G$ with the covering map $\pi : X \to X/G$.
\end{lemma}

\begin{proof}
Take an open set $U \subset X$ and consider $\pi(U)$. Then, $\pi(x) \in \pi(U)$ if and only if $\exists y \in U$ such that $x \sim y \iff \exists g \in G : x = g \cdot y \iff x \in g \cdot U$. Therefore,
\[\pi^{-1}(\pi(U))  = \bigcup_{g \in G} g \cdot U\] which is open because each $g$ acts as an open map (in fact a homeomorphism). By the definition of $X/G$ then $\pi(U)$ is open so $\pi$ is an open map. Take a point $x_0 \in X$ and because the action is properly discontinuous, there exists an open $x_0 \in U$ such that $(g \cdot U) \cap U = \empty$ for each $g \neq e$. Consider $V = \pi(U) \subset X/G$ which is open. Since for $g \neq h$, we have $(h^{-1} g) \cdot U \cap U = \empty$ then $(g \cdot U) \cap (h \cdot U) = \empty$ so the slices are disjoint. Finally, take $x, y \in g \cdot U$ then if $\pi(x) = \pi(y)$ we have $[x] = [y]$ so $x = h \cdot y$ for some $g \in G$. But since $y \in g \cdot U$ then $x \in hg \cdot U$ and $x \in g \cdot U$ so $h = e$ and thus $x = y$ because for $g \neq e$ the sets $hg \cdot U$ and $g \cdot U$ are disjoint since $hg \neq g$. Therefore, $\pi|_{g \cdot U}$ is injective but it is trivialy surjective onto $V = \pi(U) = \pi(g \cdot U)$. Furthermore, $\pi$ is an open continuous map and thus a homeomorphism when restricted to $U$. Therefore, $V$ is an openly covered neighborhood of $[x]_G$ so $\pi$ is a covering map of $X/G$.  
\end{proof}

\newpage

\begin{theorem}
Let $X$ be a simply connected and a let the action of $G$ on $X$ be free and properly discontinuous. Then $\pi_1(X/G, [x_0]_G) \cong G$.
\end{theorem} 

\begin{proof}
Fix $x_0 \in X$, then take $g \in G$ and let $\gamma_g : I \to X$ be a path from $x_0$ to $g \cdot x_0$. Such a path exists because $X$ is path-connected. Take the projection map $\pi : X \to X/G$ given by $\pi(x) = [x]_G$. These paths project to loops in the quotient space, $\eta_g = \pi \circ \gamma_g$ which is a loop because $\eta_g(0) = \pi(x_0) = [x_0]$ and $\eta_g(1) = \pi(g \cdot x_0) = [g \cdot x_0] = [x_0]$ and action by $g$ is a continuous map. \\\\
Define the map $\phi : G \to \pi_1(X/G, [x_0]_G)$ given by $\phi : g \mapsto [\pi \circ \gamma_g]$. Take $g, h \in G$ and consider the path $\delta = \gamma_g * (g \cdot \gamma_h)$  where $(h \cdot \gamma_g)(t) = h \cdot \gamma_g(t)$ with endpoints:
\[\gamma_g * (g \cdot \gamma_h)(0) = \gamma_g(0) = x_0 \text{ and } \gamma_g * (g \cdot \gamma_h)(1) = (g \cdot \gamma_h)(1) = g \cdot (h \cdot x_0) = (gh) \cdot x_0\]
Therefore, because $X$ is simply connected, $\delta \sim \gamma_{gh}$ and thus, 
\[\pi \circ \delta = (\pi \circ \gamma_g) * (\pi \circ (g \cdot \gamma_h)) \sim \pi \circ \gamma_{gh} = \eta_{gh}\]
However, $\pi \circ \gamma_g = \eta_g$ and $\pi \circ (g \cdot \gamma_h)(t) = \pi(g \cdot \gamma_h(t)) = [g \cdot \gamma_h(t)]_G = [\gamma_h(t)]_G = \eta_h(t)$ because the orbits are equivalence classses. Thus, $\pi \circ (g \cdot \gamma_h) = \eta_h$ so $\eta_g * \eta_h \sim \eta_{gh}$. 
Therefore, $\phi(gh) = [\eta_{gh}] = [\eta_g * \eta_h] = [\eta_g] [\eta_h] = \phi(g) \phi(h)$ so $\phi$ is a homomorphism. It remains to show that $\phi$ is a bijection. \bigskip \\
$X$ is the universal cover of $X/G$ so any path $\delta : I \to X/G$ can be lifted to a a unique path $\gamma : I \to X$ up to a choice of initial point. Thus, if $\delta$ is a loop at $[x_0]_G$ then there exists a unique path $\gamma : I \to X$ such that $\pi \circ \gamma = \delta$ and $\gamma(0) = x_0$. However, $\pi \circ \gamma(1) = \delta(1) = [x_0]_G$ so $[\gamma(1)]_G = [x_0]_G$ thus $\exists g \in G : \gamma(1) = g \cdot x_0$. Because $X$ is simply connected, $\gamma \sim \gamma_g$ since they share endpoints. Finlly, $\phi(g) = [\pi \circ \gamma_g] = [\pi \circ \gamma] = [\delta]$ so the map $\phi$ is surjective. Finally, take $g, h \in G$ and suppose that $\phi(g) = \phi(h)$ then $\pi \circ \gamma_g \sim \pi \circ \gamma_h$. By the homotopy lifting lemma, these loops lift to unique path-homotopic paths in $X$ with initial point $x_0$. However, $\gamma_g$ and $\gamma_h$ already satisfy the projection property and therefore must be the unique lifts so $\gamma_g \sim \gamma_h$. In particular, $\gamma_g(1) = \gamma_h(1)$ because they are path homotopic so $g \cdot x_0 = h \cdot x_0$ but because $G$ acts freely on $X$ this implies that $g = h$. Therefore, $\phi$ is a bijection.      
\end{proof}

\begin{lemma}
A free action of a finite group on a Hausdorff space is properly discontinuous.  
\end{lemma}

\begin{proof}
Take $x \in X$ and, because the action is free, for each $g \neq e$ we have $g \cdot x \neq x$ so because $X$ is Hausdorff, there exist open sets $U_g$ and $V_g$ such that $x \in U_g$ and $g \cdot x \in U_g$ and $U_g \cap V_g$. Now, let,
\[U = \bigcap_{g \in G \setminus \{e\}} (U_g \cap g^{-1} \cdot V_g)\]
which is open because the intersection is finite. Also, for each $g$, $x \in U_g$ and $g \cdot x \in V_g$ so $x \in g^{-1} \cdot V_g$. Thus, $x \in U$. Now, take any $g \neq e$. We have $U \subset U_g$ and $U \subset g^{-1} \cdot V_g$ so $g \cdot U \subset V_g$. However, $U_g$ and $V_g$ are disjoint so $U$ and $g \cdot U$ are disjoint.  
\end{proof}


\begin{lemma}
Any quotient of a compact connected space is compact and connected.
\end{lemma}

\begin{proof}
Let $X$ be compact and connected. Then, $\pi : X \to X / \sim$ is continuous and surjective. Therefore, $X / \sim$ is the image of a compact and conected set and is thus compact and connected.
\end{proof}


\begin{theorem}
For any finite group $G$, there exists a compact connected manifold with fundamental group $G$.  
\end{theorem}

\begin{proof}
By considering $n \times n$ permutation matrices in $\SU{n}$ we get an embedding of $S_n$ inside $\SU{n}$. However, by Cayley's theorem, any group with order $n$ can be embedded as a subgroup of $S_n$. Therefore, we have the embeddings,
\begin{center}
\begin{tikzcd}[column sep=large]
G \arrow[r, hook] & S_n \arrow[r, hook] & \SU{n}
\end{tikzcd}
\end{center}
Let $\Gamma \subset \SU{n}$ be the embedded copy of $G$. $\Gamma$ acts on $\SU{n}$ by left multiplication which is a topological action because $\SU{n}$ is a topological group and thus has continuous multiplcation. Futhermore, if $g \cdot h = h$ then $gh = h$ so $g = e$. Thus, $\Gamma$ acts freely on $\SU{n}$. However, $\SU{n}$ is a simply connected compact manifold. In particular, $\SU{n}$ is a Hausdorff space and $\Gamma$ is finite so the action is properly discontinuous. Therefore, since $\SU{n}$ is simply connected, $\pi_1(\SU{n}/\Gamma, [x_0]) \cong \Gamma \cong G$. Furthermore $\SU{n}/\Gamma$ is a compact connected space because it is a quotient of $\SU{n}$ which is compact and connected. Finally, we must show that $\SU{n}/\Gamma$ is a manifold. (WIP)    
\end{proof}

\begin{theorem}
For any finite cyclic group $G$, there exists a compact connected 3-manifold with fundamental group $G$.  
\end{theorem}

\begin{proof}
Consider the matrix,
\[
M = \begin{pmatrix}
e^{\frac{2 \pi i}{n}} & 0 \\
0 & e^{-\frac{2 \pi i}{n}}
\end{pmatrix} \in \SU{2}\]
Let $\Gamma = \left< M \right> \subset \SU{2}$. Since $M$ has order $n$, $\Gamma \cong C_n$. By an identical argument to above, $\pi_1(\SU{2}/\Gamma, [x_0]) \cong \Gamma \cong C_n$ and $\SU{2}/\Gamma$ is compact and connected. It remains to show that $\SU{2}/\Gamma$ is a 3-manifold. (WIP)
\end{proof}

\section{Lie Groups}

\newcommand{\g}{\mathfrak{g}}
\newcommand{\h}{\mathfrak{h}}

\begin{proposition}
Let $f : G \to H$ be a morphism of lie groups with $f_* : \g \to \h$ surjective and $H$ connected. Then $f$ is surjective.
\end{proposition}

\begin{proof}
Since $f_* : \g \to \h$ is surjective then $\d{f} : T_g G \to T_{f(g)} H$ is surjective so $f$ is a submersion and thus open. Then $f(G) \subset H$ is an open subgroup and thus closed. However, $H$ is connected and $f(G)$ is nonempty clopen so $f(G) = H$. 
\end{proof}

\begin{lemma}
Let $f : M \to N$ be a local diffeomorphism. Then the fibres $f^{-1}(y)$ are discrete.
\end{lemma}

\begin{proof}
Let $x \in f^{-1}(y)$ then there exists a neighborhood $U \subset M$ on which $f|_U : U \to f(U)$ is a diffeomorphism and thus $f|_U$ is injective so $U \cap f^{-1}(y) = \{ x \}$ which implies that $f^{-1}(y)$ is a discrete set.
\end{proof}

\begin{proposition}
Let $f : G \to H$ be a morphism of connected lie groups such that the lie algebra map $f_* \g \to \h$ is an isomorphism. Then $\Gamma = \ker{f}$ is a discrete subgroup $\Gamma \subset Z(G)$ and $f$ induces an isomorphism $f : G / \Gamma \xrightarrow{\sim} H$. 
\end{proposition}

\begin{proof}
Since $f_* : \g \to \h$ is an isomorphism we know that $\d{f} : T_g G \to T_{f(g)} H$ is an isomorphism and thus $f$ is a local diffeomorphism (by the inverse function theorem). Thus, by the lemma, $\Gamma = \ker{f} = f^{-1}(0)$ is discrete and also normal (since it is a kernel) so by \ref{discrete_in_center} we have $\Gamma \subset Z(G)$. Furthermore, local diffeomorphisms are open maps and, by above, $f$ is surjective so by \ref{quotient_condition} the induced map $\tilde{f} : G / \Gamma \xrightarrow{\sim} H$ is a homeomorphism. It suffices to show that $\tilde{f}$ is, in fact, a diffeomorphism (DO THIS).
\end{proof}

\end{document}