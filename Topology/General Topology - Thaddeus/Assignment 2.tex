\documentclass[12pt]{extarticle}
\usepackage{import}
\import{./}{Includes}

\begin{document}
\atitle{2}
 
\section*{Problem 1.}
Let $f : X \rightarrow Y$ be constant. Then $\exists c \in Y : \forall x \in X : f(x) = c$. Then for any $U \subset Y$, $\invI{f}{U} = \begin{cases} \emptyset & c \notin U \\ X & c \in U \end{cases}$. Since both $\emptyset$ and $X$ are open in any topology on $X$, for any choice of $U \subset Y$, $\invI{f}{U}$ is open in $X$. Thus, $f$ is continuous regardless of the topologies on $X$ and $Y$. \\ \\
Let $f : X \rightarrow Y$ be continuous for every topology on $X$ and on $Y$. In particular, let $X$ have the indiscrete topology and $Y$ have the discrete topology. Since every set is open in $Y$, for some $x_0 \in X$ take $U = \{f(x_0)\}$ which is open in $Y$. Then $\invI{f}{U} \neq \emptyset$ is open by continuity. However, $X$ has the indiscrete topology so the only non-empty open set is $X$. Thus $\invI{f}{U} = X$ so $\forall x \in X : f(x) \in \{ f(x_0) \}$ therefore $f(x) = f(x_0)$. 

\section*{Problem 2.}
Let the cofinite topology on $X$ be: \[\topo = \{ U \subset X \mid \text{ either } U  = \emptyset \text{ or } X \sm U \text{ is finite} \}\]
Clearly, $\emptyset, X \in \topo$. Now suppose that $\Lambda$ is an index set s.t. $V_\lambda \in \topo$. \\ Consider,  \[X \sm \bigcup\limits_{\lambda \in \Lambda} V_\lambda = \bigcap\limits_{\lambda \in \Lambda} X \sm V_\lambda \] which is finite because each $X \sm V_\lambda$ is finite and subsets of finite sets are finite. Therefore, \[X \sm \bigcup\limits_{\lambda \in \Lambda} V_\lambda \in \topo\]
Let $\Lambda$ be finite and consider \[X \sm \bigcap\limits_{\lambda \in \Lambda} V_\lambda = \bigcup\limits_{\lambda \in \Lambda} X \sm V_\lambda \]
which is finite because each $X \sm V_\lambda$ is finite and finite unions of finite sets are finite. \\ Thus if $\Lambda$ is finite, \[X \sm \bigcap\limits_{\lambda \in \Lambda} V_\lambda \in \topo\]
Therefore, $\left(X, \topo\right)$ is a topological space. 

\section*{Problem 3.} 
Let $\left(X, \topo\right)$ be a topological space with the cofinite topology. In the cofinite topology, \\ $U \subset X$ is closed $\iff$ $X \sm U$ is open $\iff$ $X \sm \left( X \sm U \right) = U$ is finite or $U = X$. Therefore, by the closed set formulation of continuity and since $\invI{f}{X} = X$ for any function: \\ \\  $f : X \rightarrow X$ is continuous $\iff$ ($U$ is finite $\implies \invI{f}{U}$ is finite or equal to $X$)  \\ \\ 
Thus, if $f$ is continuous, then since $\forall x\in X : \{x\}$ is finite then $\invI{f}{\{x\}}$ is finite or equals $X$. If for any $x_0 \in X$, $\invI{f}{\{x_0\}} = X$ then $f$ is constant because $\forall x \in X : f(x) \in \{ x_0 \}$ therfore $f(x) = x_0$. Otherwise, $\invI{f}{\{x\}}$ is finite for every $x \in X$. \\ \\
Conversely, if $f$ is constant then it is continuous on every topology. Otherwise, let $\invI{f}{\{ x \}}$ be finite for every $x \in X$. If $U = X$ then $\invI{f}{U} = X$. If $U \subset X$ is finite then \[U = \{x_1, \dots , x_n \} = \bigcup_{i = 1}^{n} \{ x_i \} \] thus, \[\invI{f}{U} = \bigcup_{i = 1}^{n} \invI{f}{\{ x_i \}} \] is finite because it is a finite union of finite sets. Therefore, if $U$ is closed under $\topo$ then $\invI{f}{U}$ is closed under $\topo$ thus $f$ is continuous. 

\section*{Problem 4.} 
Let $i : \R \rightarrow \R$ take $i : x \mapsto x$. Then $\invI{i}{U} = U$ because $x \in U \iff i(x) \in U$. Therefore, $i$ is continuous iff $U \in \topo_{codom} \implies \invI{i}{U} = U \in \topo_{dom}$ i.e. iff $\topo_{codom} \subset \topo_{dom}$. In particular, if $\topo_{codom} = \topo_{dom}$ then $i$ is continuous. \\ \\
Any $f : X \rightarrow Y$ is continuous when $Y$ has the indiscrete toplogy because both $\invI{f}{\emptyset} = \emptyset$ and $\invI{f}{Y} = X$ are always open in $X$.
Conversely, the indescrete topology is the smallest possible topology on a given set so if the domain has the indiscrete topology and the codomain has any other topology then $\topo_{codom} \supsetneq \topo_{dom}$ so $i$ is not continuous. \\ \\ 
Also, any $f : X \rightarrow Y$ is continuous when $X$ has the discrete toplogy because every set in $X$ and thus every preimage is automatically open in $X$.
Conversely, the descrete topology is the largest possible topology on a given set so if the codomain has the discrete topology and the domain has any other topology then $\topo_{codom} \supsetneq \topo_{dom}$ so $i$ is not continuous. \\ \\ 
There remain two possiblities not covered by the cases above which are: both topologies are equal, one is the discrete topology, or one is the indiscrete topology. These are: the standard topology mapping to the cofinite and the cofinite topology mapping to the standard topology. If $U \in \topo_{cofin.}$ then $\R \sm U$ is finite so $\R \sm U$ is closed in $\topo_{stand.}$. However, most open sets in $\topo_{stand.}$ do not have finite complement so $\topo_{cofin.} \subsetneq \topo_{stand.}$. Therefore, $i$ is continuous from standard to cofinite but not from cofinite to standard.   \\ \\
With respect to various toplogies, $i$ is (C. = continuous, N.C. = not continuous): 

\begin{center}
\begin{tabular}{ c | c c c c}
 from $\setminus$ to & stand. & disc. & indisc. & cofin. \\
\hline
 stand. & C. & N.C. & C. & C. \\ 
 disc. & C. & C. & C. & C. \\  
 indisc. & N.C. & N.C. & C. & N.C. \\
 cofin. & N.C & N.C. & C. & C. 
\end{tabular}
\end{center}

\section*{Problem 5.}

Let $S \in \topo \iff \forall x \in S : \exists x \in [a, b) \subset S$. Then vacuously, $\emptyset \in \topo$ and \\ $\forall x \in \R : x \in [x - 1, x + 1) \subset \R$ so  $\R \in \topo$. Now take an index set $\Lambda$ s.t. $V_\lambda$ is open. \\ Then if \[x \in \bigcup\limits_{\lambda \in \Lambda} V_\lambda \] then $\exists V_\lambda$ s.t. $x \in V_\lambda$ so because $V_\lambda$ is open, \[x \in [a, b) \subset V_\lambda \subset \bigcup\limits_{\lambda \in \Lambda} V_\lambda \]
Thus, $\bigcup\limits_{\lambda \in \Lambda} V_\lambda$ is open in $\topo$. Now if $\Lambda$ is a finite index set, take any \[x \in \bigcap\limits_{\lambda \in \Lambda} V_\lambda \]
Then for each $\lambda \in \Lambda$, $x \in [a_\lambda, b_\lambda ) \subset V_\lambda$ and since $\Lambda$ is finite then $\max\limits_{\lambda \in \Lambda}{a_\lambda} \le x < \min\limits_{\lambda \in \Lambda}{b_\lambda}$  so $x \in [\max\limits_{\lambda \in \Lambda}{a_\lambda}, \min\limits_{\lambda \in \Lambda}{b_\lambda}) \subset [a_\lambda, b_\lambda) \subset V_\lambda$. Therefore, \[ x \in [\max\limits_{\lambda \in \Lambda}{a_\lambda}, \min\limits_{\lambda \in \Lambda}{b_\lambda}) \subset \bigcap\limits_{\lambda \in \Lambda} V_\lambda \] Thus, $\bigcap\limits_{\lambda \in \Lambda} V_\lambda$ is open in $\topo$.
Therefore, $\left(X, \topo \right)$ is a topological space.

\section*{Problem 6.}

Take the subset topology $\topo_\Z$ on $\Z \subset \R$ under the standard (Euclidean) topology. For any $n \in \Z$, the interval $\ball{\frac{1}{2}}{n}$ contains only one integer namely $n$. Thus, $\Z \cap \ball{\frac{1}{2}}{n} = \{ n \}$. \\ \\

Let $S \subset \Z$ be any set of integers. Then \[S = \bigcup\limits_{n \in S} \{ n \} = \bigcup\limits_{n \in S} \Z \cap  \ball{\frac{1}{2}}{n} = \Z \cap \left(\bigcup\limits_{n \in S}\ball{\frac{1}{2}}{n} \right) \]
But each $\ball{\frac{1}{2}}{n}$ is open in $\R$ so therefore, $\bigcup\limits_{n \in S}\ball{\frac{1}{2}}{n}$ is open in $R$. Thus, $S$ is open in the subspace topology on $\Z$. $\topo_\Z$ is the discrete topology because every set is open under $\topo_\Z$.

\section*{Problem 7.}

Let $\pi : \R^2 \rightarrow \R$ be the projection $\pi : (x,y) \mapsto x$. Let \[S = \{ (x,0) \in \R^2 \mid x \in \R \}\]
Now let $f : \R \rightarrow S$ be given by $f : x \mapsto (x,0)$ is the inverse of $\pi |_S$ because $f \circ \pi |_S(x,0) = f(x) = (x, 0)$ and $\pi |_S \circ f(x) = \pi |_S(x, 0) = x$.  Therefore, $\pi |_S$ is a bijection. Since $\pi : \R^2 \rightarrow \R$ is linear, it is continuous by Lemma \ref{linearlem} and thus, by Lemma \ref{contrestrict}, $\pi |_S$ is continuous. Also, $f : \R \rightarrow S$ is linear and thus, by Lemma \ref{linearlem}, continuous. So $\pi |_S$ is a continuous bijection with continuous inverse and thus a homeomorphism. Therefore $\R$ with the standard topology and $S$ with the subspace topology are homeomorphic. 

\section*{Lemmas}

\begin{lemma} \label{linearlem}
If $f : \R^n \rightarrow \R^m$ is linear then $f$ is uniformly continuous which makes $f$ a continuous map with respect the standard topologies of $R^n$ and $R^m$.
\end{lemma}

\begin{proof}
If $f : \R^n \rightarrow \R^m$ is linear then $g(\mathbf{x}) = \begin{cases} |f(\mathbf{x})|/|\mathbf{x}| & \mathbf{x} \neq \vec{0} \\
0 & \mathbf{x} = \vec{0} \end{cases}$ \quad is bounded \\ (proven in Honors Math). Thus $\exists M \in \Rplus : \forall \mathbf{v} \in \R^n : |f(\mathbf{v})| < M |\mathbf{v}|$ so $f$ is Lipschitz. \\ \\
Given $\epsilon > 0$ take $\delta = \frac{1}{M} \epsilon$.  \\
If $|\mathbf{x} - \mathbf{y}| < \delta$ then $|f(\mathbf{x}) - f(\mathbf{y})| = |f(\mathbf{x} - \mathbf{y})| < M |\mathbf{x} - \mathbf{y}| < M \delta = \epsilon$ \\ \\
Therefore, $|\mathbf{x} - \mathbf{y}| < \delta \implies |f(\mathbf{x}) - f(\mathbf{y})| < \epsilon$
\end{proof}

\begin{lemma} \label{contrestrict}
If $f : X \rightarrow Y$ is continuous and $S \subset X$ then $f |_{S} : S \rightarrow Y$ is continuous under the subspace topology on $S$.
\end{lemma}
\begin{proof}
Let $U$ be open in $Y$ then $x \in \invI{f|_S}{U} \iff f(x) \in U \text{ and } x \in S \iff x \in \invI{f}{U} \cap S$. But $\invI{f}{U}$ is open in $X$ so $\invI{f|_S}{U} = \invI{f}{U} \cap S$ is open in $S$ under the subsapce topology. 
\end{proof}



\end{document}