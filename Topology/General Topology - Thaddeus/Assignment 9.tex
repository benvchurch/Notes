\documentclass[12pt]{extarticle}
\usepackage{import}
\import{./}{Includes}

\newcommand{\const}[2]{\left< #1 \right>_{#2}}

\usepackage[utf8]{inputenc}
\usepackage[english]{babel}
\usepackage[a4paper, total={7in, 9.5in}]{geometry}
\usepackage{tikz-cd}

 
\usepackage{amsthm, amssymb, amsmath, centernot, graphicx}
\usepackage{accents}
\DeclareMathAccent{\wtilde}{\mathord}{largesymbols}{"65}
\newcommand{\orb}[1]{\mathrm{Orb}(#1)}
\newcommand{\stab}[1]{\mathrm{Stab}(#1)}
\newcommand{\rp}{\mathbb{RP}}
\newcommand{\cp}{\mathbb{CP}}

\newcommand{\notimplies}{%
  \mathrel{{\ooalign{\hidewidth$\not\phantom{=}$\hidewidth\cr$\implies$}}}}
 
\renewcommand\qedsymbol{$\square$}
\newcommand{\cont}{$\boxtimes$}
\newcommand{\divides}{\mid}
\newcommand{\ndivides}{\centernot \mid}
\newcommand{\Z}{\mathbb{Z}}
\newcommand{\N}{\mathbb{N}}
\newcommand{\C}{\mathbb{C}}
\newcommand{\Zplus}{\mathbb{Z}^{+}}
\newcommand{\Primes}{\mathbb{P}}
\newcommand{\ball}[2]{B_{#1} \! \left(#2 \right)}
\newcommand{\Q}{\mathbb{Q}}
\newcommand{\R}{\mathbb{R}}
\newcommand{\Rplus}{\mathbb{R}^+}
\newcommand{\invI}[2]{#1^{-1} \left( #2 \right)}
\newcommand{\End}[1]{\text{End}\left( A \right)}
\newcommand{\legsym}[2]{\left(\frac{#1}{#2} \right)}
\renewcommand{\mod}[3]{\: #1 \equiv #2 \: \mathrm{mod} \: #3 \:}
\newcommand{\nmod}[3]{\: #1 \centernot \equiv #2 \: mod \: #3 \:}
\newcommand{\ndiv}{\hspace{-4pt}\not \divides \hspace{2pt}}
\newcommand{\finfield}[1]{\mathbb{F}_{#1}}
\newcommand{\finunits}[1]{\mathbb{F}_{#1}^{\times}}
\newcommand{\ord}[1]{\mathrm{ord}\! \left(#1 \right)}
\newcommand{\quadfield}[1]{\Q \small(\sqrt{#1} \small)}
\newcommand{\vspan}[1]{\mathrm{span}\! \left\{#1 \right\}}
\newcommand{\galgroup}[1]{Gal \small(#1 \small)}
\newcommand{\sm}{\! \setminus \!}
\newcommand{\topo}{\mathcal{T}}
\newcommand{\base}{\mathcal{B}}
\renewcommand{\bf}[1]{\mathbf{#1}}
\renewcommand{\Im}[1]{\mathrm{Im} \: #1}
\renewcommand{\empty}{\varnothing}
\newcommand{\id}{\mathrm{id}}
\newcommand{\Hom}[2]{\mathrm{Hom}\left( #1, #2 \right)}
\newcommand{\Tor}[4]{\mathrm{Tor}^{#1}_{#2} \left( #3, #4 \right)}

\renewcommand{\theenumi}{(\alph{enumi})}

\newcommand{\atitle}[1]{\title{% 
	\large \textbf{Mathematics GU4053 Algebraic Topology
	\\ Assignment \# #1} \vspace{-2ex}}
\author{Benjamin Church }
\maketitle}

\newcommand{\hook}{\hookrightarrow}


\theoremstyle{remark}
\newtheorem*{remark}{Remark}

\theoremstyle{definition}
\newtheorem{theorem}{Theorem}[section]
\newtheorem{lemma}[theorem]{Lemma}
\newtheorem{proposition}[theorem]{Proposition}
\newtheorem{corollary}[theorem]{Corollary}
\newtheorem{example}[theorem]{Example}


\newenvironment{definition}[1][Definition:]{\begin{trivlist}
\item[\hskip \labelsep {\bfseries #1}]}{\end{trivlist}}



\begin{document}
\atitle{9}
 
\section*{Problem 1.}

Let $f_0, f_1 : X \to Y$ be homotopic and $g_0, g_1 : Y \to Z$ be homotopic. Take homotopies for each pair of homotopic functions: $F : X \times I \to Y$ which satisfies $F(x, 0) = f_0(x)$ and $F(x, 1) = f_1(x)$ and is continuous and $G : X \times I \to Y$ which satisfies $G(y, 0) = g_0(y)$ and $G(y, 1) = g_1(y)$ and is continuous. Consider the map $H : X \times I \to Z$ given by $H(x, t) = G(F(x,t),t)$. Firstly, $H(x, 0) = G(F(x, 0), 0) = g_0(f_0(x)) = (g_0 \circ f_0)(x)$. Similarly, $H(x, 1) = G(F(x, 1), 1) = g_1(f_1(x)) = (g_1 \circ f_1)(x)$. To show that $H$ is a homotopy between $g_0 \circ f_0$ and $g_1 \circ f_1$ we must show that $H$ is continuous. The map, 
\[f = (F, \pi_2) : X \times I \to Y \times I\] 
is continuous because both $F : X \times I \to Y$ and $\pi_2 : X \times I \to I$ are continuous so $G \circ (F, \pi_2)$ is continuous. Also, $G \circ (F, \pi_2) (x,t) = G(F(x,t), \pi_2(x,t)) = G(F(x,t), t) = H(x,t)$ so $H$ is continuous. The argument is summarized in the commutative diagram:

\begin{center}
\begin{tikzcd}[column sep=large]
& & Y \\
X \times I \arrow[rru, "F"] \arrow[rrd, "\pi_2"] \arrow[rr, "f", dashed] \arrow[rrr, bend right = 50, dashed, "H" description] & & Y \times I \arrow[u, "\pi^Y_1"] \arrow[d, "\pi^Y_2"]  \arrow[r, "G"] & Z \\
& & I
\end{tikzcd}
\end{center}
Therefore, $H$ is a homotopy between $g_0 \circ f_0$ and $g_1 \circ f_1$.
 
\section*{Problem 2.}    

A note on notation: For $y_0 \in Y$, I will use $\const{y_0}{X} : X \to Y$ to denote the contant map $\const{y_0}{X} : x \mapsto y_0$.    
\begin{enumerate}
\item Let $L \subset \R$ be a nonempty interval. Take $x_0 \in L$ and $\id_L : L \to L$. Now, define $G : \R^2 \to \R$ by $G(x, y) = x_0 y + x (1 - y)$ which is continuous by analysis. Now, if $x \in L$ and $t \in [0,1]$ then $x \le G(x,y) \le x_0$ or $x_0 \le G(x,y) \le x$ so by the interval property $G(x,y) \in L$. Thus, $F : L \times I \to L$ given by $F = G|_{L \times I}$ is a well defined continuous map. Also, $F(x, 0) = x = \id_L(x)$ and $F(x, 1) = x_0$ a constant map. Thus, $\id_L \sim \left<x_0\right>$ where $\left<x_0\right>$ represents the constant map $x \to x_0$ so $L$ is contractable.  

\item Let $X$ be contractable then there is a homotopy $F : X \times I \to X$ such that $F(x, 0) = x$ and $F(x, 1) = x_0$ for some $x_0 \in X$. Take any $x_1, x_2 \in X$. Define $\gamma : I \to X$ by,
\[ \gamma(t) =
\begin{cases}
F(x_1, 2t) & t \le \frac{1}{2} \\
F(x_2, 2 - 2t) & t \ge \frac{1}{2}
\end{cases} \]  
Because at $t = \frac{1}{2}$ we have $F(x_1, 2t) = F(x_1, 1) = x_0$ and $F(x_2, 2 - 2t) = F(x_2, 1) = x_0$ the path $\gamma$ is continuous by the glueing lemma. Also, $\gamma(0) = F(x_1, 0) = x_1$ and $\gamma(1) = F(x_2, 2 - 2) = F(x_2, 0) = x_2$ so $\gamma$ is a path from $x_1$ to $x_2$ and thus $X$ is path connected. \bigskip \\
Now, take any loop $\gamma$ at $x_0$. Now, define,
\[ G(x,t) = 
\begin{cases}
F(x_0, 2xt) & x \le \frac{1}{2} \\
F(\gamma(4x - 2), t) & \frac{1}{2} \le x \le \frac{3}{4} \\
F(x_0, (4 - 4x)t) & x \ge \frac{3}{4}
\end{cases} \] 

At $x = \frac{1}{2}$, $F(x_0, 2xt) = F(x_0, t)$ and $F(\gamma(4x - 2), t) = F(\gamma(0), t) = F(x_0, t)$. Similarly, at $x = \frac{3}{4}$, $F(\gamma(4x - 2), t) = F(\gamma(1), t) = F(x_0, t)$ and $F(x_0, (4 - 4x)t) = F(x_0, t)$. Therefore, by the glueing lemma, $G$ is continuous. Also, let $\delta(x) = F(x_0, x)$ which is a loop at $x_0$ because $\delta(0) = F(x_0, 0) = x_0$ and $F(x_0, 1) = x_0$. 
\begin{align*}
G(x, 0) &= 
\begin{cases}
F(x_0, 0) = x_0 & x \le \frac{1}{2} \\
F(\gamma(4x - 2), 0) = \gamma(4x - 2) & \frac{1}{2} \le x \le \frac{3}{4} \\
F(x_0, 0) = x_0 & x \ge \frac{3}{4}
\end{cases} \\ 
& = (e_{x_0} * (\gamma * e_{x_0}))(x) \\
G(x, 1) &= 
\begin{cases}
F(x_0, 2x) = \delta(2x) & x \le \frac{1}{2} \\
F(\gamma(4x - 2), 1) = x_0 & \frac{1}{2} \le x \le \frac{3}{4} \\
F(x_0, (4 - 4x)) = \delta(4 - 4x) & x \ge \frac{3}{4}
\end{cases} \\ 
& = (\delta * (e_{x_0} * \delta^{-1}))(x) \\
G(0, t) &= 
\begin{cases}
F(x_0, 0) & 0 \le \frac{1}{2} \\
F(\gamma(-2), t) & \frac{1}{2} \le 0 \le \frac{3}{4} \\
F(x_0, (-4x)t) & 0 \ge \frac{3}{4}
\end{cases} \\
& = F(x_0, 0) = x_0 \\
G(1, t) &= 
\begin{cases}
F(x_0, 2t) & 1 \le \frac{1}{2} \\
F(\gamma(2), t) & \frac{1}{2} \le 1 \le \frac{3}{4} \\
F(x_0, 0) & 1 \ge \frac{3}{4}
\end{cases} \\
& = F(x_0, 0) = x_0 \\
\end{align*}

Therefore, $e_{x_0} * (\gamma * e_{x_0})$ and $\delta * (e_{x_0} * \delta^{-1})$ are path-homotopic so $[e_{x_0} * (\gamma * e_{x_0})] = [\delta * (e_{x_0} * \delta^{-1})]$ However, $[e_{x_0} * (\gamma * e_{x_0})] = [e_{x_0}] * [\gamma] * [e_{x_0}] = [\gamma]$ because $[e_{x_0}]$ is the identity of $\pi_1(X, x_0)$. Furthermore, $[\delta * (e_{x_0} * \delta^{-1})] = [\delta] * [e_{x_0}] * [\delta]^{-1} = [\delta] * [\delta]^{-1} = [e_{x_0}]$ because the reversed path generates the inverse homotopy class. Thus, $[\gamma] = [e_{x_0}]$ but $\gamma$ was arbitrary so every element of $\pi_1(X, x_0)$ is the identity. Now, for any other base point $x \in X$ we know that $\pi_1(X, x) \cong \pi_1(X, x_0)$ with isomorphism induced by conjugation with a path from $x_0$ to $x$. Therefore, $\pi_1(X, x) \cong \pi_1(X, x_0) \cong \{e\}$. 

\item Let $f_0, f_1 : X \to Y$ be continuous and let $Y$ contractable. Then there exists a homotopy $G : Y \times I \to I$ such that $G(y, 0) = y$ and $G(y, 1) = y_0$ for some $y_0 \in Y$. Define, $F : X \times I \to Y$ by,
\[ F(x, t) = 
\begin{cases}
G(f_0(x), 2t) & t \le \frac{1}{2} \\
G(f_1(x), 2 -2t) & t \ge \frac{1}{2}
\end{cases}\]
First, $G(f_0(x), 2t)$ and $G(f_1(x), 2 - 2t)$ are continuous by composition of continuous functions. Now, because on the closed set $X \times \{\frac{1}{2}\}$, we have $G(f_0(x), 2t) = G(f_0(x), 1) = y_0$ and $G(f_1(x), 2 - 2t) = G(f_1(x), 1) = y_0$ then $F$ is continuous by the glueing lemma. Also, $F(x, 0) = G(f_0(x), 0) = f_0(x)$ and $F(x, 1) = G(f_1(x), 0) = f_1(x)$ so $F$ is a homotopy from $f_0$ to $f_1$ so $f_0 \sim f_1$.  \bigskip \\

An alternative proof goes as follows. Take continuous $f_0, f_1 : X \to Y$. Because $Y$ is contractable, $\id_Y \sim \const{y_0}{Y}$ where $y_0$ is some fixed point $y_0 \in Y$. Now, $f_0 \sim f_0$ so by the problem 1, we have $f_0 = \id_Y \circ f_0 \sim \const{y_0}{Y} \circ f_0 = \const{y_0}{X}$ by Lemma \ref{constmaps}. Similarly, $f_1 \sim f_1$ so $f_1 = \id_Y \circ f_1 \sim \const{y_0}{Y} \circ f_1 = \const{y_0}{X}$. Thus, $f_0 \sim \const{y_0}{X}$ and $f_1 \sim \const{y_0}{X}$ so $f_0 \sim f_1$ by transitivity. 

\item Let $g_0, g_1 : X \to Y$ be continuous, let $X$ be contractable, and let $Y$ be path-connected. Because $X$ is contractable, there exists a point $x_0 \in X$ such that $\id_X \sim \const{x_0}{X}$. Then, because $g_0 \sim g_0$ we know that $g_0 = g_0 \circ \id_X \sim g_0 \circ \const{x_0}{X} = \const{g_0(x_0)}{X}$ by Lemma \ref{constmaps}. Similarly, $g_1 = g_1 \circ \id_X \sim g_1 \circ \const{x_0}{X} = \const{g_1(x_0)}{X}$. However, because $Y$ is path-connected, by Lemma \ref{pathconnectedhomotopicconsts}, all constant functions are homotopic so $\const{g_0(x_0)}{X} \sim \const{g_1(x_0)}{X}$. Thus, by transitivity, $g_0 \sim \const{g_0(x_0)}{X} \sim \const{g_1(x_0)}{X} \sim g_1$ so $g_0$ and $g_1$ are homotopic.  
\end{enumerate}  

\section*{Problem 3.} 

\begin{enumerate}
\item Take $S \subset \R^2$ to be the axes, $S = (\R \times \{0\}) \cup (\{0\} \times \R)$. This set is star-convex because any point $P \in S$ lies either on the $x$-axis or the $y$-axis. Either way, the segment $\overline{PO} \subset S$, where $O = (0,0)$ is the origin, because it is a subset of the corresponding axis. However, take $P = (1, 0)$ and $Q = (0, 1)$. Both $P, Q \in S$ but $\overline{PQ} \notin S$ because $(\tfrac{1}{2}, \tfrac{1}{2}) \notin S$ so $S$ is nonconvex.  

\item Let $T \subset \R^2$ be the graph of a parabola, $T = \{ (t, t^2) \mid t \in \R \}$. Then $T$ is not star-convex because it contains no nontrivial line segements. However, $T$ is contractable. Consider the map $G : \R^3 \to \R^2$ given by $G(x, y, z) = (x (1 - z), y (1 - z)^2)$ which is continuous by analysis. Also if $x^2 = y$ then $(x(1-z))^2 = x^2 (1 - z)^2 = y (1 - z)^2$ so $\Im{G|_{T \times I}} \subset T$. Therefore, the map $F : T \times I \to T$ given by $F((x,x^2), t) = G(x, x^2, t)$ is continuous. Also, \[F((x, x^2), 0) = G(x, x^2, 0) = (x, x^2) \quad \quad  F((x, x^2), 1)  = (x(1-1), x^2(1-1)) = (0, 0)\] so $F$ is a homotopoy from $\id_T$ to the constant map from $T$ to $(0,0)$. 

\item Let $S \subset \R^n$ be star-convex. Therefore, $\exists \bf{x} \in S$ such that $\forall \bf{y} \in S$ the segment $\overline{\bf{xy}} \subset S$. Consider the function   $G : \R^{n} \times \R \to \R^{n}$ given by $G(\bf{y}, t) = (1 - t) \bf{y} + t \bf{x}$. By analysis, $G$ is continuous. Also, for $t \in [0,1]$ we have $G(\bf{y}, t) \in \overline{\bf{xy}}$ so $G(\bf{y}, t) \in S$. Thus, $\Im{G|_{S \times I}} \subset S$ so the funtion $F : S \times I \to S$ given by $F(\bf{y}, t) = G(\bf{y}, t)$ is continuous and well defined. Also, $F(\bf{y}, 0) = \bf{y} = \id_S(\bf{y})$ and $F(\bf{y}, 1) = \bf{x}$ which is a contant function. Thus, $\id_S$ is homotopic to the constant function mapping to $\bf{x}$. Therefore, $S$ is contractable.     
\end{enumerate}

\section*{Problem 4.}
For $S \subset X$ let $f : X \to S$ be a retraction. Take $x_0 \in S$ and any loop $\gamma : I \to S$ at $x_0$. Now we can lift the loop $\gamma$ into the ambient space $X$ simply by defining $\tilde{\gamma} : I \to X$ by $\tilde{\gamma}(t) = \gamma(t)$. Consider the homomorphism induced by the retraction, $f_* : \pi_1(X, x_0) \to \pi_1(S, f(x_0))$. However, $x_0 \in S$ and $f|_S = \id_S$ so $f(x_0) = x_0$. Thus, $f_* : \pi_1(X, x_0) \to \pi_1(S, x_0)$. Now, consider $f_*([\tilde{\gamma}]) = [f \circ \tilde{\gamma}]$ then we have, $f \circ \tilde{\gamma} : I \to S$ and $f(\tilde{\gamma}(t)) = f(\gamma(t)) = \gamma(t)$ because $\gamma(t) \in S$ and $f|_S = \id_S$. Thus, $f_*([\tilde{\gamma}]) = [\gamma]$. However, $\gamma$ was an aribitrary loop at $x_0$ so the function $f_*$ is surjective because the equivalence class of any loop is in the image. 
\section*{Problem 5.}

The projections $\pi_1 : X \times Y \to X$ and $\pi_2 : X \times Y \to Y$ are continuous and thus induce homomorphisms $f_1 : \pi_1(X \times Y, x \times y) \to \pi_1(X, x)$ and $f_2 : \pi_1(X \times Y, x \times y) \to \pi_1(Y, y)$ because $\pi_1(x \times y) = x$ and $\pi_2(x \times y) = y$. Using Lemma \ref{comphomos}, define the homomorphism, \[F : \pi_1(X \times Y, x \times y) \to \pi_1(X, x) \times \pi_2(Y, y)\]
by $F = (f_1, f_2)$. It remains to show that $F$ is a bijection. Let \[G : \pi_1(X, x) \times \pi_2(Y, y) \to \pi_1(X \times Y, x \times y)\] be given by $G([\gamma], [\delta]) = [\Gamma]$ where $\Gamma = (\gamma, \delta) : I \to X \times Y$. Lemma \ref{homotopicproduct} shows that this function maps loops to loops with the correct base points and is well defined on path-homotopy equivalence classes. Now, 
\[G \circ F([\Gamma]) = G([\pi_1 \circ \Gamma], [\pi_2 \circ \Gamma]) = [(\pi_1 \circ \Gamma, \pi_2 \circ \Gamma)] = [\Gamma]\]
In the last line, I used the fact that, for any function $\gamma : I \to X \times Y$, the map $(\pi_1 \circ \gamma, \pi_2 \circ \gamma) = \gamma$.
Also,
\[F \circ G([\gamma], [\delta]) = F([\Gamma]) = ([\pi_1 \circ \Gamma], [\pi_2 \circ \Gamma]) = ([\gamma], [\delta])\]
where I have used the fact that $\Gamma = (\gamma, \delta)$ so $\pi_1 \circ \Gamma = \gamma$ and $\pi_2 \circ \Gamma = \delta$. Therefore, $G$ is the inverse function of $F$ so $F$ must be a bijection. Therefore, $F$ is an isomorphism. 
    

\section*{Problem 6.}   
\begin{enumerate} 
\item
Let $G$ be a topological group with a multiplcation function $M : G \times G \to G$ which takes $M(x,y) = x \cdot y$. Let $\gamma_1, \gamma_2 : I \to G$ be continuous loops based at $e$. Then, let $\gamma_1 \diamond \gamma_2 : I \to G$ be given by $(\gamma_1 \diamond \gamma_2)(t) = \gamma_1(t) \cdot \gamma_2(t)$. This is a loop at $e$ because $(\gamma_1 \diamond \gamma_2)(0) = \gamma_1(0) \cdot \gamma_2(0) = e \cdot e = e$ and
$(\gamma_1 \diamond \gamma_2)(1) = \gamma_1(1) \cdot \gamma_2(1) = e \cdot e = e$. This function is also continuous because, $f = (\gamma_1, \gamma_2)$ is continuous thus $\gamma_1 \diamond \gamma_2 = f \circ M$ is continuous by composition of continuous functions.
 
\begin{center}
\begin{tikzcd}[column sep=large]
& & G \\
I \arrow[rru, "\gamma_1"] \arrow[rrd, "\gamma_2"] \arrow[rr, "f", dashed] \arrow[rrrr, bend right = 50, dashed, "\gamma_1 \diamond \gamma_2" description] & & G \times G \arrow[u, "\pi_1"] \arrow[d, "\pi_2"]  \arrow[rr, "M"] & & G \\
& & G
\end{tikzcd}
\end{center}   

Now, 
\[(\gamma_1 * \gamma_2)(t) = 
\begin{cases}
\gamma_2(2t) & t \le \frac{1}{2} \\
\gamma_1(2t - 1) & t \ge \frac{1}{2}
\end{cases}\]
Let $f : [0, \tfrac{1}{2}] \times I \to I^2$ given by, 
\[f(x, t) = (tx, (2-t)x)\]
which is continuous and let $g : [\tfrac{1}{2}, 1] \times I \to I^2$ be given by 
\[g(x,y) = \big((2-t)x + t - 1, 1 + t(x - 1)\big)\]
which is also continuous. Also, at $x = \frac{1}{2}$, 
\begin{align*}
f(x, t) & = \big( \tfrac{1}{2} t, \tfrac{1}{2}(2-t) \big) \\
g(x, t) & = \big((2 - t) \tfrac{1}{2} + t - 1, 1 + t(\tfrac{1}{2} - 1)\big) = \big(\tfrac{1}{2} t, \tfrac{1}{2}(2-t)\big)
\end{align*}
so by the glueing lemma, the function $R : I^2 \to I^2$ given by,
\[ R(x,t) = 
\begin{cases}
f(x,t) & x \le \frac{1}{2} \\
g(x,t) & x \le \frac{1}{2}
\end{cases}\]
is continuous. Define, $F = M \circ (\gamma_1 \times \gamma_2) \circ R$ which is a well defined continuous map detailed in the commutative diagram below.

\begin{center}
\begin{tikzcd}[column sep=large]
& & I \arrow[r, "\gamma_1"] & G \\
I \times I \arrow[rrrrr, bend right = 30, dashed, "F" description] \arrow[r, "R"] & I \arrow[ru, "\pi_1"] \arrow[rd, "\pi_2"] \times I \arrow[rrr, "\gamma_1 \times \gamma_2", dashed]  & & & G \times G \arrow[lu, "\pi_1"] \arrow[ld, "\pi_2"]  \arrow[r, "M"] & G \\
& & I \arrow[r, "\gamma_2"] & G
\end{tikzcd}
\end{center}   

Thus, the function $F : I^2 \to G$ is given by, 
\[F(x,t) = 
\begin{cases}
\gamma_1(tx) \cdot \gamma_2((2-t)x) & x \le \frac{1}{2} \\
\gamma_1((2-t)x + t - 1) \cdot \gamma_2(1 + t(x - 1)) & x \ge \frac{1}{2}
\end{cases}\]

Finally, using the fact that $\gamma_1(0) = \gamma_1(1) = \gamma_2(0) = \gamma_2(1) = e$ so products with these elements do nothing. 
\begin{align*}
F(x,0) &= 
\begin{cases}
\gamma_1(0) \cdot \gamma_2(2x) & x \le \frac{1}{2} \\
\gamma_1(2x - 1) \cdot \gamma_2(1) & x \ge \frac{1}{2}
\end{cases} = 
\begin{cases}
\gamma_2(2x) & x \le \frac{1}{2} \\
\gamma_1(2x - 1) & x \ge \frac{1}{2}
\end{cases}
= (\gamma_1 * \gamma_2)(x)
\\
F(x,1) &= 
\begin{cases}
\gamma_1(x) \cdot \gamma_2(x) & x \le \frac{1}{2} \\
\gamma_1(x) \cdot \gamma_2(x) & x \ge \frac{1}{2}
\end{cases}
= (\gamma_1 \diamond \gamma_2)(x)
\\
F(0,t) &= 
\begin{cases}
\gamma_1(0) \cdot \gamma_2(0) & 0 \le \frac{1}{2} \\
\gamma_1(t - 1) \cdot \gamma_2(1 - t) & 0 \ge \frac{1}{2}
\end{cases}
= \gamma_1(0) \cdot \gamma_2(0) = e
\\
F(1,t) &= 
\begin{cases}
\gamma_1(t) \cdot \gamma_2(2-t) & 1 \le \frac{1}{2} \\
\gamma_1(1) \cdot \gamma_2(1) & 1 \ge \frac{1}{2}
\end{cases}
= \gamma_1(1) \cdot \gamma_2(1) = e
\end{align*}
Therefore, $F$ is a path-homotopy from $\gamma_1 * \gamma_2$ to $\gamma_1 \diamond \gamma_2$. Therefore, 
\[[\gamma_1] \diamond [\gamma_2] = [\gamma_1 \diamond \gamma_2] = [\gamma_1 * \gamma_2] = [\gamma_1] * [\gamma_2]\]
Because $*$ is well-defined on equivalence classes we have that $\diamond$ is also a well-defined operation on equivalence classes and gives the same group structure. 

\item Let $\gamma_1, \gamma_2 : I \to G$ be loops at $e$. In an analogous fashion to part (a) but with the components of the output flipped, define $f : [0, \tfrac{1}{2}] \times I \to I^2$ given by, 
\[f(x, t) = ((2-t)x, tx)\]
which is continuous and let $g : [\tfrac{1}{2}, 1] \times I \to I^2$ be given by 
\[g(x,y) = \big(1 + t(x - 1), (2-t)x + t - 1 \big)\]
which is also continuous. Also, at $x = \frac{1}{2}$, 
\begin{align*}
f(x, t) & = \big(\tfrac{1}{2}(2-t), \tfrac{1}{2} t \big) \\
g(x, t) & = \big(1 + t(\tfrac{1}{2} - 1), (2 - t) \tfrac{1}{2} + t - 1 \big) = \big( \tfrac{1}{2}(2-t), \tfrac{1}{2} t \big)
\end{align*}
so by the glueing lemma, the function $R : I^2 \to I^2$ given by,
\[ R(x,t) = 
\begin{cases}
f(x,t) & x \le \frac{1}{2} \\
g(x,t) & x \le \frac{1}{2}
\end{cases}\]
is continuous. Define, $F = M \circ (\gamma_1 \times \gamma_2) \circ G$ which is a well defined continuous map detailed in the commutative diagram below.

\begin{center}
\begin{tikzcd}[column sep=large]
& & I \arrow[r, "\gamma_1"] & G \\
I \times I \arrow[rrrrr, bend right = 30, dashed, "F" description] \arrow[r, "R"] & I \arrow[ru, "\pi_1"] \arrow[rd, "\pi_2"] \times I \arrow[rrr, "\gamma_1 \times \gamma_2", dashed]  & & & G \times G \arrow[lu, "\pi_1"] \arrow[ld, "\pi_2"]  \arrow[r, "M"] & G \\
& & I \arrow[r, "\gamma_2"] & G
\end{tikzcd}
\end{center}   

Thus, the function $F : I^2 \to G$ is given by, 
\[F(x,t) = 
\begin{cases}
\gamma_1((2-t)x) \cdot \gamma_2(tx) & x \le \frac{1}{2} \\
\gamma_1(1 + t(x - 1)) \cdot \gamma_2((2-t)x + t - 1) & x \ge \frac{1}{2}
\end{cases}\]

Finally, using the fact that $\gamma_1(0) = \gamma_1(1) = \gamma_2(0) = \gamma_2(1) = e$ so products with these elements do nothing. 
\begin{align*}
F(x,0) &= 
\begin{cases}
\gamma_1(2x) \cdot \gamma_2(0) & x \le \frac{1}{2} \\
\gamma_1(1) \cdot \gamma_2(2x - 1) & x \ge \frac{1}{2}
\end{cases} = 
\begin{cases}
\gamma_1(2x) & x \le \frac{1}{2} \\
\gamma_2(2x - 1) & x \ge \frac{1}{2}
\end{cases}
= (\gamma_2 * \gamma_1)(x)
\\
F(x,1) &= 
\begin{cases}
\gamma_1(x) \cdot \gamma_2(x) & x \le \frac{1}{2} \\
\gamma_1(x) \cdot \gamma_2(x) & x \ge \frac{1}{2}
\end{cases}
= (\gamma_1 \diamond \gamma_2)(x)
\\
F(0,t) &= 
\begin{cases}
\gamma_1(0) \cdot \gamma_2(0) & 0 \le \frac{1}{2} \\
\gamma_1(1 - t) \cdot \gamma_2(t - 1) & 0 \ge \frac{1}{2}
\end{cases}
= \gamma_1(0) \cdot \gamma_2(0) = e
\\
F(1,t) &= 
\begin{cases}
\gamma_1(2 - t) \cdot \gamma_2(t) & 1 \le \frac{1}{2} \\
\gamma_1(1) \cdot \gamma_2(1) & 1 \ge \frac{1}{2}
\end{cases}
= \gamma_1(1) \cdot \gamma_2(1) = e
\end{align*}
Therefore, $F$ is a path-homotopy from $\gamma_2 * \gamma_1$ to $\gamma_1 \diamond \gamma_2$.

\item From the previous parts, $\gamma_1 \diamond \gamma_2 \sim \gamma_2 * \gamma_1$ and also, $\gamma_1 \diamond \gamma_2 \sim \gamma_1 * \gamma_2$ therefore, $\gamma_1 * \gamma_2 \sim \gamma_2 * \gamma_1$ by transitivity. Therefore, $[\gamma_1] * [\gamma_2] = [\gamma_1 * \gamma_2] = [\gamma_2 * \gamma_1] = [\gamma_2] * [\gamma_1]$ so the fundamental group $\pi_1(G, e)$ is abelian.     

\end{enumerate}  
            
\section*{Lemmas}

\begin{lemma} \label{constmaps}
Let $g : X \to Y$ be any function, with $x_0 \in X$ and $y_0 \in Y$ then $\const{y_0}{Y} \circ g = \const{y_0}{X}$ and $g \circ \const{x_0}{X} = \const{g(x_0)}{X}$
\end{lemma}

\begin{proof}
For all $x \in X$ we have $(\const{y_0}{Y} \circ g)(x) = \const{y_0}{Y}(g(x)) = y_0$ thus $ \const{y_0}{Y} \circ g = \const{y_0}{Y}$. Also, for any $x \in X$ we have $(g \circ \const{x_0}{X})(x) = g(x_0)$ thus $g \circ \const{x_0}{X} = \const{g(x_0)}{X}$.
\end{proof}


\begin{lemma} \label{pathconnectedhomotopicconsts}
If $Y$ is path connected then any two constant functions from $X$ to $Y$ are homotopic. 
\end{lemma}

\begin{proof}
If $X$ is empty than all functions from $X$ are homotopic. Let $X$ be nonempty, let $g_0, g_1 : X \to Y$ be constant then $g_0(X) = \{y_0\}$ and $g_1(X) = \{y_1\}$. Since $Y$ is path connected, there exists a path $\gamma : I \to Y$ from $g_0(x_0)$ to $g_1(x_0)$. Define the function $G : X \times I \to Y$ by, $G = \gamma \circ \pi_2$ which is continuous as a composition of contiuous maps. Then, $G(x,0) = \gamma(0) = y_0 = g_0(x)$ and $G(x, 1) = \gamma(0) = y_1 = g_1(x)$. Thus, $G$ is a homotopy from $g_0$ to $g_1$. 
\end{proof}

\begin{lemma} \label{comphomos}
Let $G$, $H_1$, and $H_2$ be groups with homomorphisms $f_1 : G \to H_1$ and $f_2 : G \to H_2$ then there is a unique homomorphism $F : G \to H_1 \times H_2$ given by $F = (f_1, f_2)$ such that $\pi_1 \circ F = f_1$ and $\pi_2 \circ F = f_2$. In other words, the product $H_1 \times H_2$ satisfies the following universal property:  

\begin{center}
\begin{tikzcd}[column sep=large]
& & H_1 \\
G \arrow[rr, "F", dashed] \arrow[rru, "f_1"] \arrow[rrd, "f_2"] & & H_1 \times H_2 \arrow[u, "\pi_1"] \arrow[d, "\pi_2"] \\
& & H_2
\end{tikzcd}
\end{center}
  
\end{lemma}

\begin{proof}
For $g,h \in G$, we have, \[F(gh) = (f_1(gh), f_2(gh)) = (f_1(g) f_1(h), f_2(g) f_2(h)) = (f_1(g), f_2(g)) * (f_1(h), f_2(h)) = F(g) * F(h)\] 
Thus, $F$ is a homomorphism. Let $K : G \to H_1 \times H_2$ be any homomorphism satisfying $\pi_1 \circ K = f_1$ and $\pi_2 \circ K = f_2$ then for any $g \in G$ we have $K(g) \in H_1 \times H_2$ so $G(g) = (h_1, h_2)$ for $h_1 \in H_1$ and $h_2 \in H_2$ and $\pi_1 \circ K(g) = h_1 = f_1(g)$ and $\pi_2 \circ K(g) = h_2 = f_2(g)$ so $K(g) = (f_1(g), f_2(g)) = F(g)$.  
\end{proof}

\begin{lemma} \label{homotopicproduct}
Let $\gamma_0, \gamma_1 : I \to X$ be path-homotopic loops at $x_0$ and let $\delta_0, \delta_1 : I \to Y$ be path-homotopic loops at $y_0$ then $\Gamma_0 = (\gamma_0, \delta_0) : I \to X \times Y$ and $\Gamma_1 = (\gamma_1, \delta_1) : I \to X \times Y$ are path-homotopic loops at $(x_0, y_0)$. 
  
\end{lemma}

\begin{proof}
Becuase $\gamma_0$ and $\delta_0$ are continuous, $\Gamma_0$ is also continuous. $\Gamma_0$ is a loop at $(x_0, y_0)$ because $\Gamma_0(0) = (\gamma_0(0), \delta_0(0)) = (x_0,y_0)$ and $\Gamma_0(1) = (\gamma_0(1), \delta_0(1)) = (x_0,y_0)$. An identical argument shows that $\Gamma_1$ is a loop at $(x_0, y_0)$. Take path-homotopies $F : I^2 \to X$ and $G : I^2 \to Y$ for $\gamma_0 \sim \gamma_1$ and $\delta_0 \sim \delta_1$ respectively. Now, consider $H = (F, G) : I^2 \to X \times Y$ which is continuous because $F$ and $G$ are continuous. Also,
\begin{align*}
H(0, t) & = (F(0,t), G(0,t)) = (x_0, y_0) \\ 
H(1, t) & = (F(1,t), G(1,t)) = (x_0, y_0) \\
H(x, 0) & = (F(x,0), G(x,0)) = (\gamma_0(x), \delta_0(x)) = \Gamma_0(x) \\ 
H(x, 1) & = (F(x,1), G(x,1)) = (\gamma_1(x), \delta_1(x)) = \Gamma_1(x)
\end{align*}
Therefore, $H$ is a path-homotopy between $\Gamma_0$ and $\Gamma_1$.   
\end{proof}



\end{document}
