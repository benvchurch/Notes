\documentclass[12pt]{extarticle}
\usepackage{import}
\import{./}{Includes}


\begin{document}
\atitle{6}
 
\section*{Problem 1.}
\begin{enumerate}
\item Let $\mathcal{S} \subset \topo$ be topologies on a set $X$. Suppose that $X$ is compact  under $\topo$. Now let \[\{U_\lambda \in \mathcal{S} \mid \lambda \in \Lambda \}\] be an open cover of $X$ consisting of sets of $\mathcal{S}$. Then, each $U_\lambda \in \mathcal{S} \subset \topo$ so this is an open cover of $X$ under the topology $\topo$. Therefore, by compactness, there exists an open subcover consisting of $\{U_\lambda \mid \lambda \in \Lambda\}$ which are open sets in $\mathcal{S}$ by definition. Thus, $X$ is compact under $\mathcal{S}$. So compactness under $\topo$ implies compactness under $\mathcal{S}$. However, the converse is false because the indiscrete topology is a subset of any topology and under the indiscrete topology any set is compact. However, we can choose $X$ to be noncompact under the origional topology.

\item I believe that I misinterpreted this question. I intially thought that $\mathcal{S}$ and $\topo$ were \textit{any} two topologies not the two related ones above. However, after working on that problem for many hours I realized that, in this case, the statment is false. To not feel that I wasted by life, I will both solve the indended problem and give a counterexample I constructed to the more general case. \\

First, the real question. Let $(X, \mathcal{S})$ and $(X, \topo)$ be compact Hausdorff spaces with $\mathcal{S} \subset \topo$. Consider the identity map $\iota : (X, \topo) \to (X, \mathcal{S})$ which takes $\iota : x \mapsto x$. This function is continuous because if $U \in \mathcal{S}$ then $\invI{i}{U} = U \in \mathcal{S} \subset \topo$. Thus, $\iota$ is a continuous map from a compact space to a Hausdorff space and therefore, for closed $C \subset X$ under $\topo$, $\iota(C) = C$ is closed under $\mathcal{S}$. Thus, if $U \in \topo$ then $X \sm U$ is closed in $\topo$ and then by above also in $\mathcal{S}$ so $X \sm (X \sm U) = U$ is open in $\mathcal{S}$. Therefore, $\topo \subset \mathcal{S}$ so $\mathcal{S} = \topo$. This can be stated more elegantly as: continuous bijections from compact to Hausdorff are homeomorphisms and thus $\iota$ is a homeomorphism so $\mathcal{S} = \topo$.   \\

Now for a counterexample. To get inequivalent Hausdorff spaces even without considering compactness, we must consider an infinite set because all Hausdorff spaces are $T_1$ and thus discrete on a finite set. Let $X = [0, 1]$ and $\mathcal{S}$ be the standard topology on $X$ which we know is compact and metrizable so Hausdorff. Now let, \[\topo = \{ U \subset X \mid 1 \notin U \text{ or } (1 \in U \text{ and } X \sm U \text{ is finite}) \}\]  This can be summed up as the discrete topology on $[0,1)$ plus the indiscrete topology restricted to each open set containing $1$. We have to show three things: $\topo$ is a topology, $(X, \topo)$ is Hausdorff, and $(X, \topo)$ is compact. First, $\empty, X \in \topo$ because $1 \notin \empty$ and $1 \in X$ and $X \sm X$ is finite. Now, take a collection of open sets $\{U_\lambda \in \topo \mid \lambda \in \Lambda \}$. If $1$ is not an element of any $U_\lambda$ then their union still does not contain $1$ so it is open. If for some $\lambda_0$, $1 \in U_{\lambda_0}$ then the union contains $U_{\lambda_0}$ so it contains $1$ and has a complement smaller than $U_{\lambda_0}$ which is already finite so the union is also open. Suppose the collection is finite. If some $U_\lambda$ does not contain $1$ then the intersection also does not and thus is open. Else, $1$ is in every $U_\lambda$ so each $X \sm U_\lambda$ is finite. However, \[X \setminus \bigcap_{\lambda \in \Lambda} U_\lambda = \bigcup_{\lambda \in \Lambda} X \sm U_\lambda \]
which is a finite union of finite sets and therefore, finite. Thus, the intersection contains $1$ and has finite complement and so is open. Next, we must show that $(X, \topo)$ is Hausdorff. Let $x \neq y$, so WLOG $x \neq 1$. Then, take $U = \{x \}$ and $V = X \sm \{x \}$. $U$ an $V$ are disjoint and $x \in U$ and $y \in V$ because $x \neq y$. However, $1 \notin U$ so $U$ is open and $X \sm V = \{x \}$ is finite with $1 \in V$ so $V$ is open. Therefore, $(X, \topo)$ is Hausdorff. Finally, take any open cover of $X$, $\{U_\lambda \in \topo \mid \lambda \in \Lambda \}$. Now, $1 \in X$ so, because this collection is a cover, there is some $\lambda_0 \in \Lambda$ s.t. $1 \in U_{\lambda_0}$ and thus, $X \sm U_{\lambda_0}$ is finite because $U_{\lambda_0}$ is open. For each $x \in X \sm U_{\lambda_0}$, $\exists \lambda_x \in \Lambda$ s.t. $x \in U_{\lambda_x}$. Thus, \[X = U_{\lambda_0} \cup \bigcup\limits_{x \in X \setminus U_{\lambda_0}} U_{\lambda_x}\]  
which is a finite union so we have found a finite subcover. Therefore, $(X, \topo)$ is compact and Hausdoff but $\{0\}$ is open with respect to $\topo$ so this is not the same as the standard topology. Note, this consitruction is equivalent to the one-point compactification of $[0,1)$ with the discrete topology. 
\end{enumerate}

\section*{Problem 2.}
\begin{enumerate}
\item Let $\topo$ be the cofinite topology on $X$. If $X = \empty$ then it is trivially compact (there is only one open set). Take any open cover of $X$, $\{U_\lambda \in \topo \mid \lambda \in \Lambda \}$. Now, $\exists x_0 \in X$ so, because this collection is a cover, there is some $\lambda_0 \in \Lambda$ s.t. $x_0 \in U_{\lambda_0}$ and thus, $X \sm U_{\lambda_0}$ is finite because $U_{\lambda_0}$ is open and non-empty. For each $x \in X \sm U_{\lambda_0}$, since the collection is a cover, $\exists \lambda_x \in \Lambda : x \in U_{\lambda_x}$ and thus, \[X = U_{\lambda_0} \cup \bigcup\limits_{x \in X \setminus U_{\lambda_0}} U_{\lambda_x}\] which is a finite union so we have found a finite subcover. Therefore, $(X, \topo)$ is compact.

\item Let $\R$ have the cocountable topology $\topo$. Now consider the collection open sets \[U_z = \R \setminus (\Z \setminus \{z\})\] which are indexed by $\Z$. Each of these sets is indeed open because $\Z \setminus \{z\}$ is countable so $\R \setminus (\Z \setminus \{z\})$ is cocountably open. Now, $U_z = (\R \setminus \Z) \cup \{z\}$ so \[ \bigcup\limits_{z \in \Z} U_z =  (\R \setminus \Z) \cup \bigcup\limits_{z \in \Z} \{z\} = (\R \setminus \Z) \cup \Z = \R\]
and thus $\{U_z \mid z \in \Z\}$ is a open cover of $\R$. However, take any finite subcover indexed by a finite set $S \subset \Z$. Then let $m = \max{S} + 1$ which exists because $S$ is finite. Thus, $m \notin S$ so $U_m \notin \{U_z \mid z \in S\}$ but $m$ is an integer so $m \in U_z$ only if $z = m$ so $m \notin \bigcup\limits_{z \in S} U_z$. Therefore, no finite subset of $\{U_z \mid z \in \Z\}$ can cover $\R$ so $\R$ is noncompact.   
\end{enumerate}

\section*{Problem 3.}
Let $\{A_n \mid n \in \N\}$ be a collection of compact, connected subsets of $X$ which is a Hausdorff space such that $A_n \supset A_{n+1}$. Now, consider the set \[A = \bigcap_{n \in \N} A_n\]
\begin{enumerate}
\item If some $A_k$ were empty then $A \subset A_k = \empty$ so $A$ is empty. Suppose that $A$ is empty. Define $K_n = X \setminus A_n$. Since $X$ is Hausdorff and $A_n$ is compact, $A_n$ must be closed so $K_n$ is open. However, \[\bigcup_{n \in \N} K_n = X \setminus \bigcap_{n \in \N} A_n = X \setminus A = X \supset A_0\]
because $A = \empty$ by hypothesis. Therefore, $\{K_n \mid n \in \N \}$ is an open cover of $A_0$ but $A_0$ is compact so there is a finite subcover indexed by $S \subset \N$ such that \[\bigcup_{n \in S} K_n = X \setminus \bigcap_{n \in S} A_n \supset A_0\] 
therefore, $A_0 \cap \bigcap\limits_{n \in S} A_n = \empty$ because $x \in A_0 \implies x \in X \setminus \bigcap\limits_{n \in S} A_n \implies x \notin \bigcap\limits_{n \in S} A_n$. However, $S$ is finite so $m = \max{S}$ exists. For any $n \le m$ we have $A_n \supset A_m$ which follows by induction from the property $A_{n} \supset A_{n+1}$. Thus, $A_m \subset A_0 \cap \bigcap\limits_{n \in S} A_n = \empty$ because $A_m$ is contained in every set in the intersection. Thus, $A_m = \empty$ and therefore, $A$ is empty iff some $A_m$ is empty.  

\item Each $A_n$ is closed because it is a compact subset of a Hausdorff space and therefore, $A$ is closed because it is the arbitrary intersection of closed sets. However, $A \subset A_0$ and $A_0$ is compact so $A$ being closed must also be compact. 

\item Suppose that $A$ is disconnected, then $A = C \cup D$ with $C$ and $D$ nonempty disjoint clopen (in $A$) sets. Since $A$ is compact and $C, D \subset A$ are closed, then $C$ and $D$ are compact. However, $X$ is a Hausdorff space so there are disjoint open sets $U, V$ such that $C \subset U$ and $D \subset V$. Now if $A_n \subset U \cup V$ then because $U$ and $V$ are open and disjoint, $A_n$ would be disconnected as long as $A_n \cap U \neq \empty$ which holds because $C \subset A \subset A_n$ is not empty and $C \subset U$. Likewise, $A_n \cap D \neq \empty$. Thus, $A_n \setminus (U \cup V) \neq \empty$ and the set is compact because $A_n \setminus (U \cup V)$ is closed in $A_n$ (since $U \cup V$ is open) which is compact. Furthermore, $A_{n} \supset A_{n+1} \implies A_n \setminus (U \cup V) \supset A_{n+1} \setminus (U \cup V)$ so by part (a), \[ \bigcap_{n \in \N} \left(A_n \setminus (U \cup V) \right) = \left( \bigcap_{n \in \N} A_n \right) \setminus (U \cup V) = A \setminus (U \cup V)\]
is non empty. However, $A = C \cup D \subset U \cup V$ which is a contradiction. Thus, $A$ is connected.
\end{enumerate}

\end{document}
