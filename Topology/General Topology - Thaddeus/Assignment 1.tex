\documentclass[12pt]{extarticle}
\usepackage{import}
\import{./}{Includes}

\begin{document}
\atitle{1}
 
\section*{Problem 1.}
 
\begin{enumerate}
\item $(-\infty ,  a) \cup (b, \infty)$ is open in $\R$: \\
Let $ x \in (-\infty, a)$ then $x < a$ so take $\delta = a - x$ so that whenever $|y - x| < \delta$, $y < \delta + x = a$ then $y \in (- \infty, a)$. Therefore, $\ball{\delta}{x} \subset (-\infty, a)$ so $(-\infty, a)$ is open. \\

Similarly, let $ x \in (b, \infty)$ then $b < x$ so take $\delta = x - b$ so that whenever $|y - x| < \delta$ then $y > x - \delta = b$ so $y \in (b, \infty)$. Therefore, $\ball{\delta}{x} \subset (b, \infty)$ so $(b, \infty)$ is open. So as a union of open sets, $(-\infty ,  a) \cup (b, \infty)$ is open. \\

For $a < b$, $S = \mathbb{R} \setminus ((-\infty ,  a) \cup (b, \infty))$ is not open in $\R$: \\ Take $a \in S$ (since $a \nless a$ and $a < b$) then suppose that $\exists \delta \in \Rplus : \ball{\delta}{a} \subset S$ then let $x = a - \frac{1}{2}\delta < a$ thus $x \in (-\infty, a)$ so $x \notin S$  a contradiction because $|x - a| < \delta$ so $x \in \ball{\delta}{a} \subset S$. 
\item $\Z$ is not open in $\R$: \\
Take $0 \in \Z$ then suppose that $\exists \delta \in \Rplus : \ball{\delta}{0} \subset \Z$ but since $\ball{\delta}{0}$ is an interval, $\exists x \in \ball{\delta}{0} \setminus \Q$ thus $x \notin \Q \supset \Z$  a contradiction because $x \in \ball{\delta}{0} \subset \Z$.
\\

$\R \setminus \Z$ is open in $\R$: \\
Since for any $x \in \R$, $\exists n \in \Z : n \leq x < n + 1$ we have $\R \setminus \Z = \bigcup\limits_{n \in \Z} (n, n + 1)$ but each $(n, n + 1)$ is open so the union is open.

\item $\Q$ is not open in $\R$: \\
Take $q \in \Q$ and suppose $\exists \delta \in \Rplus : \ball{\delta}{q} \subset \Q$ then since $\ball{\delta}{q}$ is an interval, $\exists x \in \ball{\delta}{q} \setminus \Q$ so $x \notin \Q$ which is a contradiction because $x \in \ball{\delta}{q} \subset \Q$.  \\

$\R \setminus \Q$ is not open in $\R$: \\
Take $r \in \R \setminus \Q$ and suppose $\exists \delta \in \Rplus : \ball{\delta}{q} \subset \R \setminus \Q$ then since $\ball{\delta}{q}$ is an interval, $\exists x \in \ball{\delta}{q} \cap \Q$ so $x \in \Q$ which is a contradiction because $x \in \ball{\delta}{q} \subset \R \setminus \Q$ so $x \in \Q$. \\

\item $S = \{1/n \mid n \in \Zplus\}$ is not open in $\R$: \\
Take $x = 1 \in S$ and suppose that $\exists \delta \in \Rplus : \ball{\delta}{1} \subset S$ then take $y = 1 + \frac{1}{2} \delta$ then $y > \sup(S) = 1$ so $y \notin S$ but $|y - x| < \delta$ so $y \in \ball{\delta}{1} \subset S$ which is a contradiction. \\

$\R \setminus S$ is not open in $\R$: \\
For all $n \in \Zplus$, $1/n \neq 0$ so $0 \in \R \setminus \S$ so suppose $\exists \delta \in \Rplus : \ball{\delta}{0} \subset \R \setminus S$. But by the unboundedness of $\Z$ there exists $k \in \Zplus$ s.t. $0 < 1/k < \delta$ and $1/k \in S$ but then $1/k \in \ball{\delta}{0} \subset \R \setminus S$ which is a contradiction. \\

\end{enumerate}

\section*{Problem 2.}

\begin{enumerate}
\item $f(x) = |x|$ is continuous: \\
given $ \epsilon > 0$ take $\delta  = \epsilon$. Whenever $|x - y| < \delta$ then $|f(x) - f(y)| = \left||x| - |y|\right| \leq |x - y| < \delta = \epsilon$ thus $|x - y| < \delta  \implies |f(x) - f(y)| < \epsilon$.\\

\item $g(x) = \begin{cases} 
      0 & x \in \Q \\
      1 & x \notin \Q
   \end{cases}$ is not continuous: \\
   
$U = (-\frac{1}{2}, \frac{1}{2}) \subset \R$ is open in $\R$ but $\invI{g}{U} = \Q$ since $g(\Q) = \{0\} \subset U$ and if $x \notin \Q$ then $g(x) = 1 \notin U$. But $\Q$ is not open in $\R$ so $g$ cannot be continuous. 
\end{enumerate}

\section*{Problem 3.}
$f : \R \rightarrow \R$ is continuous iff $\invI{f}{V}$ is closed for any closed $V \subset \R$ 
\begin{proof}
By Lemma \ref{complimentlem}, $\invI{f}{\R \setminus V} = \R \setminus \invI{f}{V}$. \\

Now suppose that $f$ is continuous. Then let $V \subset \R$ be closed so $\R \setminus V$ is open. By continuity, $\invI{f}{\R \setminus V} = \R \setminus \invI{f}{V}$ is open and therefore, $\invI{f}{V}$ is closed. \\

Suppose that $\invI{f}{V}$ is closed for any closed $V \subset \R$ \\
Let $V \subset \R$ be open. Then $\R \setminus V$ is closed, since $V = \R \setminus (\R \setminus V)$ is open, so $\invI{f}{\R \setminus V} = \R \setminus \invI{f}{V}$ is closed. Therefore, $\R \setminus (\R \setminus \invI{f}{V}) = \invI{f}{V}$ is open. Thus, $V \subset \R$ is open $\implies \invI{f}{V}$ is open so $f$ is continuous. 
\end{proof}

\section*{Problem 4.} 

False. Let $f(x) = 0$ then $\invI{f}{V} = \begin{cases} 
	\varnothing &  0 \notin V \\
	\R   &  0 \in V
	\end{cases}$ which is always open in $\R$ so $f$ is continuous. However, $\R$ is open but $f(\R) = \{0\}$ is not open. 
	
\section*{Problem 5.}

\begin{enumerate}
\item Let $U \subset \R^m$ and $V \subset \R^n$ be open. Take $\mathbf{x} \in U \times V$ then $\mathbf{x} = (\mathbf{x_1}, \mathbf{x_2})$ where $\mathbf{x_1} \in U$ and $\mathbf{x_2} \in V$.\\

Now since $U$ and $V$ are open, $\exists \delta_1,\delta_2 \in \Rplus : \ball{\delta_1}{\mathbf{x_1}} \subset U$ and $\ball{\delta_2}{\mathbf{x_2}} \subset V$. \\

Take $\delta = \text{min} \{ \delta_1, \delta_2 \}$ so that for $\mathbf{y} = (\mathbf{y_1}, \mathbf{y_2}) \in \R^{m+n}$ if $| \mathbf{y} - \mathbf{x} | < \delta$ then $|\mathbf{y_1} - \mathbf{x_1}|^2 + |\mathbf{y_2} - \mathbf{x_2}|^2 \le \delta^2$ therefore, $|\mathbf{y_1} - \mathbf{x_1}| < \delta \le \delta_1$ and $|\mathbf{y_2} - \mathbf{x_2}| < \delta < \delta_2$ so $\mathbf{y_1} \in \ball{\delta_1}{\mathbf{x_1}} \subset U$ and $\mathbf{y_2} \in \ball{\delta_2}{\mathbf{x_2}} \subset V$ so $\mathbf{y} \in U \times V$. \\
Therefore, $\ball{\delta}{\mathbf{y}} \subset U \times V$ so $U \times V$ is open.

\item No. Take $m = n = 1$ and $S  = \{ (x, y) \mid x,y \in \R \text{ and } x \neq y \} \subset \R^2$. \\ 

Now take $f : \R^2 \rightarrow \R$ given by $f(x,y) = x-y$ so $f$ is linear so, by Lemma \ref{linearlem}, $f$ is continuous. Since $\R \setminus \{ 0 \} = (-\infty, 0) \cup (0, \infty)$ is open, $\invI{f}{\R \setminus \{ 0 \}} = S$ is open because \[\invI{f}{\{ 0 \}} = \{ (x,y) \mid x,y \in \R \text{ and } x = y \}\]

However, suppose $S = U \times V$ with $U, V \subset \R$ then since $(1, 0), (0, 1) \in S$ we have $0 \in U$ and $0 \in V$ so $(0, 0) \in U \times V = S$ which is a contradiction.
\end{enumerate}

\section*{Problem 6.} Let $L \subset \R^2$ be a line given by $L = \{(x, y) \mid x,y \in \R \text{ and } ax + by = c \}$. Let $f : \R^2 \rightarrow \R$ be given by $f(x,y) = ax + by$ is linear and thus continuous (by Lemma \ref{linearlem}). Since $\R \setminus \{ c \} = (-\infty, c) \cup (c, \infty)$ is open, $\invI{f}{\R \setminus \{ c \}} = \R \setminus L$ is open because $\invI{f}{\{ c \}} = \{ (x,y) \mid x,y \in \R \text{ and } ax + by = c \}$.  \\  
Now let $\{L_1, \dots , L_n \}$ be a finite collection of lines and $S = \bigcup\limits_{i = 1}^{n} L_i$. Then by DeMorgan, \[\R \setminus S = \bigcap\limits_{i = 1}^{n} \R \setminus L_i\] but each $\R \setminus L_i$ is open so $\R \setminus S$ is open as a finite intersection of open sets. 


\section*{Lemmas}

\begin{lemma} \label{complimentlem}
For $f : X \rightarrow Y$ and $V \subset Y$, $\invI{f}{Y \setminus V} = X \setminus \invI{f}{V}$
\end{lemma}

\begin{proof}
Let $x \in \invI{f}{Y \setminus V}$ then $f(x) \in Y \setminus V$  so $f(x) \notin V$ thus $x \notin \invI{f}{V}$ so $x \in X \setminus \invI{f}{V}$ since $\invI{f}{Y \setminus V} \subset X$. \\ \\
Also if $x \in X \setminus \invI{f}{V}$ then $f(x) \notin V$ but $f(x) \in Y$ (because $\text{Im}(f) \subset Y$) so $f(x) \in Y \setminus V$ so $f(x) \in Y \setminus V$ therefore, $x \in \invI{f}{Y \setminus V}$. \\ \\
Thus, $\invI{f}{Y \setminus V} = X \setminus \invI{f}{V}$.
\end{proof}

\begin{lemma} \label{linearlem}
if $f : \R^n \rightarrow \R^m$ is linear then $f$ is uniformly continuous
\end{lemma}

\begin{proof}
If $f : \R^n \rightarrow \R^m$ is linear then $g(\mathbf{x}) = \begin{cases} |f(\mathbf{x})|/|\mathbf{x}| & \mathbf{x} \neq \vec{0} \\
0 & \mathbf{x} = \vec{0} \end{cases}$ \quad is bounded \\ (proven in Honors Math). Thus $\exists M \in \Rplus : \forall \mathbf{v} \in \R^n : |f(\mathbf{v})| < M |\mathbf{v}|$ so $f$ is Lipschitz. \\ \\
Given $\epsilon > 0$ take $\delta = \frac{1}{M} \epsilon$.  \\
If $|\mathbf{x} - \mathbf{y}| < \delta$ then $|f(\mathbf{x}) - f(\mathbf{y})| = |f(\mathbf{x} - \mathbf{y})| < M |\mathbf{x} - \mathbf{y}| < M \delta = \epsilon$ \\ \\
Therefore, $|\mathbf{x} - \mathbf{y}| < \delta \implies |f(\mathbf{x}) - f(\mathbf{y})| < \epsilon$
\end{proof}

\end{document}
