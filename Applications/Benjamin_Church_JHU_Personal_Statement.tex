\documentclass[11pt]{article}
\usepackage{import}
\import{"../Algebraic Geometry/"}{AlgGeoCommands}

\newcommand{\Loc}[1]{\mathfrak{Loc}\left( #1 \right)}
\newcommand{\AbGrp}{\mathbf{AbGrp}}

\newtheorem*{defnn}{Definition}
\newtheorem*{conj}{Conjecture}

\usepackage{hyperref}
\usepackage{fancyhdr}

\geometry{margin=0.7in}

\pagestyle{fancy}
\fancyhead[LH]{\textbf{Benjamin V. Church}}
\fancyhead[RH]{\textbf{Personal Statement: JHU}}
\setlength{\headheight}{15pt}
\setlength{\headsep}{0.2in}

\usepackage[backend=bibtex, citestyle=apa, style=phys]{biblatex}
\addbibresource{bibliography.bib}

\begin{document}
I aspire to earn a doctorate and pursue a career in academia because I am passionate about both research and teaching. My entire life, I have had the good fortune to be taught and guided by people who encouraged my curiosity and supported my studies, turning my interests into a dedication to research. In doing so, these people illustrated to me the importance of committed and passionate teachers at every level of education. Knowing the debt I owe to my mentors and the certain advantages I have had in access to these people and institutions, I have been involved in promoting equitable higher education through student groups, a commitment I intend to continue through graduate school. Additionally, given the profound impact that inspirational teachers have had on my own life and educational pursuits, it is my goal to someday be that sort of teacher for students of my own. 
\par
The internet, especially MIT OpenCourseWare, was an extremely influential resource in my adolescence. Early in high school, I devoured Walter Lewin’s lectures on physics and Gilbert Strang's linear algebra. Soon, the world became my laboratory; I began carrying around a polarizer film in my pocket wherever I went in order to investigate atmospheric optical phenomena. By this point, I was convinced I would become a physicist. Encouraged by a high school science teacher, I signed up to take the US Physics Olympiad qualifying exam. Although I qualified for the final round, my school had no logistics in place to proctor the exam so instead I convinced my English teacher to stay late after school as a proctor. I won a gold medal but narrowly missed the cutoff for the US team. Although I also participated in math competitions, the problems did not capture my imagination the way physics did. That changed when I enrolled in a summer number theory course. I was awestruck; the subtle and surprising proofs immediately captivated me, offering the same thrill of discovery that had drawn me to physics. Moreover, these beautiful and intricate patterns were drawn out of the most familiar and mundane objects: the integers. As with physics, I became enthralled by uncovering the wonders that lay hiding in plain sight.
\par
When I entered college, I dove head-first into both mathematics and physics coursework as well as a variety of research projects. It was again a course in number theory, this time taught by Prof.\ Michael Harris, which drew me deeper into pure mathematics. Bit by bit, I began to glimpse the startling depth and overwhelming beauty that springs forth from even basic mathematical objects and their interrelations, solidifying my curiosity into a focused passion. After Prof.\ Harris’s course, I began to seriously consider a career in mathematics and therefore I decided to apply to mathematics REUs. These research opportunities, as well as my undergraduate thesis work with Prof.\ Aise Johan de Jong, showed me that an academic career in mathematics is truly what I want to spend my life doing.
\par
As an undergraduate, I became heavily involved with the Columbia Undergraduate Mathematics Society (UMS) and Columbia Society of Physics Students (SPS), volunteering to give talks for undergraduates and give demonstrations and outreach for local middle school groups. When I served on the UMS board, in collaboration with the Columbia Association for Women in Mathematics (AWM), I gave help sessions and wrote example-based materials aimed at students who were new to proof-based mathematics courses. The transition from high school math to college math is challenging for most students and often accentuates inequalities. Recognizing that disparities in higher education are widened by disparities in access to advanced preparation at the high school level, we aimed to mitigate the attrition of underrepresented groups in mathematics by helping incoming students bridge the gap between high school and college. We offered help sessions, problem-based written materials, and guided sessions which exposed students to proof techniques and examples, and we advertised available resources for students, especially for women, offered by AWM and UMS. Furthermore, as president of SPS, I helped organize a mentorship program in collaboration with Columbia Society Women in Physics (now Columbia Spectra). The program connected new students interested in physics with peer mentors who could speak to student-specific issues more directly than faculty advisors. In particular, we aimed to pair students from underrepresented populations (per the student's preferences) with seniors from similar backgrounds who might be better able to address, from personal experience, issues specific to these groups. We hoped that one-on-one mentorship would ameliorate the discouraging effects that under-representation has on incoming students.
\par 
I am excited to begin graduate school so that I can delve deeper into mathematical research and so that I can gain the necessary skills and credentials in order to pass my love of mathematics on to future generations of students. 
\end{document}