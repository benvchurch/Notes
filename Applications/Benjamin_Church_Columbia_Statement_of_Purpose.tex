\documentclass[11pt]{article}
\usepackage{import}
\import{"../Algebraic Geometry/"}{AlgGeoCommands}

\newcommand{\Loc}[1]{\mathfrak{Loc}\left( #1 \right)}
\newcommand{\AbGrp}{\mathbf{AbGrp}}

\newtheorem*{defnn}{Definition}
\newtheorem*{conj}{Conjecture}

\usepackage{hyperref}
\usepackage{fancyhdr}
\geometry{margin=0.8in}
\pagestyle{fancy}
\fancyhead[LH]{\textbf{Benjamin V. Church}}
\fancyhead[RH]{\textbf{Statement of Purpose: Columbia}}
\setlength{\headheight}{15pt}
\setlength{\headsep}{0.2in}


\usepackage[backend=bibtex, citestyle=apa, style=phys]{biblatex}
\addbibresource{bibliography.bib}


\begin{document}
Mathematical research is what I want to spend my life doing. Although I cast a wide net of scientific interests and research projects, including published work in astrophysics and bioinformatics, my interests have firmly coalesced around algebraic and arithmetic geometry. Having completed upper-level mathematics coursework and dipped my toes into mathematical research, I aspire to earn a doctorate at Columbia to build a career in academia. At Columbia, I would be honored to work with innovative Professors in my areas of interest, particularly Aise Johan de Jong and Michael Harris, professors who advised me extensively during my undergraduate studies and with whom I already have a good working relationship.
\par
Working closely with professors through independent study was an invaluable opportunity to explore topics outside the standard curriculum and prepare for graduate-level study. Reading courses on elliptic curves with Prof.\ David Hansen in spring 2018 and on modular forms and Galois representations with Prof.\ Chao Li in fall 2018 cemented my interest in the interplay between number theory and algebraic curves. In my senior year, advised by Prof.\ Michael Harris, I studied Deligne's proof that Hodge cycles on abelian varieties are absolutely Hodge. Concurrently, I attended Prof.\ Johan de Jong's weekly seminar on Weil cohomology theories and algebraic de Rham cohomology. These experiences motivated me to read further on comparison theorems, Hodge theory, and motives. Deligne's proof also introduced me to Shimura varieties encoding moduli of abelian varieties, sparking my interest in arithmetic geometry and leading me to pursue further independent study with Prof.\ Harris on Shimura varieties.
\par
I participated in the 2018 Columbia math REU studying the zeta functions of surfaces over finite fields. Under the supervision of Professors Daniel Litt and Alex Perry, we aimed to generalize Shioda’s classification of supersingular Fermat varieties \footfullcite{shioda_on_fermat} to weighted-projective diagonal hypersurfaces. We implemented an efficient algorithm for determining supersingularity using Stickelberger's theorem and Jacobi sums\footfullcite{weil_counting}. Using a computer search, I identified patterns in certain new examples of supersingular surfaces. From this observation, I proved the existence of an infinite family of supersingular surfaces such that the minimal covering Fermat surface fails to be supersingular. This project solidified my love of algebraic geometry, especially geometry in positive characteristic and its relations to arithmetic. It also introduced me to the Weil conjectures which inspired me to study scheme theory and \etale cohomology, devoting myself to EGA, Hartshorne exercises, and Milne's \etale cohomology, in order to understand the proofs of Grothendieck and Deligne. 
\par
After building a solid foundation in toric geometry at the 2019 Paris Diderot University REU, I decided to write my senior thesis under Prof.\ de Jong on embedding curves in toric surfaces. Using a corollary of Harris and Mumford's result\footfullcite{harris1982kodaira} on the Kodaira dimension of $\overline{\mathcal{M}}_g$, I gave a proof that very general curves cannot be embedded in any toric surface and I studied obstructions to these embeddings intersecting the toric divisor transversally. This project aimed to investigate the regularity conditions introduced in a paper of Dokchitser \footfullcite{models_of_curves} which provides an algorithm to construct the minimal r.n.c.\ models of certain curves over DVRs using toric embeddings. I constructed a degeneration of a genus 5 curve which cannot result from Dokchitser's method showing that no affine equation for this curve can satisfy required regularity conditions. This work was an invaluable learning experience about how research is conducted in mathematics. Thanks to Prof.\ de Jong’s excellent mentorship and our working relationship, the process was also personally rewarding and enjoyable; it fully convinced me that mathematical research is what I want to spend my life doing. 
\par 
Although I had planned to spend a gap year working and traveling overseas, this became impossible due to the Covid-19 pandemic. Instead, immediately following the completion of my thesis, I began a project with Prof.\ de Jong determining the unirationality of various supersingular surfaces which I had discovered during the 2018 REU. These surfaces may be described as cyclic covers of $\mathbb{P}^2$ and also as cyclic quotients of products of easily understood curves. I am working on methods to find nonfree rational curves on such surfaces which are applicable in positive characteristic. Additionally, Prof.\ de Jong guided my reading of papers on N\'{e}ron models, algebraic connections and Atiyah classes, Frobenius descent and $p$-curvature, and unirational $3$-folds. I attended Prof.\ Litt's course on \etale cohomology and Prof.\ Max Lieblich's joint course with Prof.\ de Jong on resolution of singularities. Currently, I am attending Prof.\ Jarod Alper’s seminar and his course on moduli and stacks.
\par
I have been passionate about teaching since high school when I organized a student-run seminar promoting interest in pure mathematics. I have volunteered to teach over two dozen classes at Columbia and MIT Splash, a six-week HSSP course introducing high school students to elliptic curves, and a dozen talks at Columbia's math and physics clubs aimed at undergraduates. In collaboration with the Columbia Association for Women in Mathematics, I helped create introductory talks, materials, and help sessions aimed at supporting freshmen who were new to college-level mathematics courses. I was specifically selected by Prof.\ Brian Cole to teach weekly recitations for his accelerated physics course.
\par
The Columbia mathematics department's expertise in algebraic and arithmetic geometry would foster my current interests, and its breadth would inspire me to explore new research areas. Prof.\ de Jong is a leading expert in several areas of algebraic geometry that interest me, including stacks, moduli problems, and resolution of singularities. Although I am open to many research directions, I am particularly interested in Prof.\ de Jong’s papers studying the space of rational curves on cubic 4-folds. Another possible research direction would investigate the structure of unirational varieties in positive characteristic and questions related to my summer project advised by Prof.\ de Jong. Prof.\ Harris has made inspiring contributions to the Langlands program and the proof of the Sato-Tate conjecture. Primarily, I am interested in his work on arithmetic geometry including the study of motivic L-functions, Shimura varieties, and automorphic representations, particularly his paper showing that certain local systems on smooth proper curves are potentially automorphic and his ongoing project investigating Deligne’s conjecture for automorphic motives over CM-fields. I strongly believe that Columbia is an ideal place for my intellectual development and mathematical studies, and I sincerely hope to be given the opportunity to learn from and contribute to this vibrant community.
\end{document}