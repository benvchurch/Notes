\documentclass[10pt]{amsart}
\usepackage{import}
\import{"../Algebraic Geometry/"}{AlgGeoCommands}

\newcommand{\Loc}[1]{\mathfrak{Loc}\left( #1 \right)}
\newcommand{\AbGrp}{\mathbf{AbGrp}}

\newtheorem*{defnn}{Definition}
\newtheorem*{conj}{Conjecture}

\usepackage{hyperref}
\newcommand{\M}{\mathcal{M}}

\usepackage{fancyhdr}

\pagestyle{fancy}
\fancyhead[LH]{\textbf{Benjamin Church}}
\fancyhead[RH]{Research Plan}
\setlength{\headheight}{15pt}
\setlength{\headsep}{0.2in}

\begin{document}

Mathematical research is what I want to spend my life doing. Although my specific interests have encompassed a wide array of topics in the sciences including mathematical modeling, physics, chemistry, and pure mathematics, I knew from a young age that I wanted to be a researcher and an academic. In high school, a summer course in number theory opened my eyes to the intricate, subtle, and surprising patterns in what seem to be the most familiar and mundane objects: the integers. I began to glimpse the startling depth and overwhelming beauty within mathematics. My college career solidified this amorphous pull into a strengthened and focused passon. Having completed a broad selection of mathematics coursework, including at the graduate level, and dipped my toes into the world mathematical research, I am more excited than ever to pursue these mysteries in a doctoral program at Oxford where I hope to collaborate with innovative researchers and prepare to enter academia.

\par

My research interests have been predominantly shaped by undergraduate research and independent study directly with professors. Although my wide breadth of mathematics coursework provided me an essential foundation for my later studies, I found these experiences outside the classroom to be the most formative and rewarding. 

Working closely with professors through independent study was an invaluable opportunity to explore interesting topics, gain vital insight into the field, and prepare for graduate-level work. In spring 2018, I studied elliptic curves with Professor David Hansen initially from the perspective of number theory as I was concurrently taking graduate-level coursework in class field theory. However, this independent study was my first exposure to algebraic geometry and I became so enamored with the subject that I pivoted the course towards algebraic curves. Since then, algebraic geometry continues to by my primary interest. That said, I am fascinated also by the interplay between number theory, complex geometry of elliptic curves, and geometry in positive characteristic. To deepen my understanding of these interelations, I learned about modular forms and Galois representations through an independent study with Professor Chao Li covering Diamond and Shurman's book. (MAYBE DELETE) [That year, I also did a reading course with Professor Brian Cole on advanced topics in electricity and magnetism using Jackson. This independent study expanded upon the basic topics covered in the undergraduate curriculum and bridged the gap between the classical theory and the results of quantum electrodynamics I was simultaneously learning in quantum field theory.] 
\par
My senior year, I focused primarily on independent study and my undergraduate thesis work. I worked through Deligne's proof that Hodge cycles on abelian varieties are absolutely Hodge in my independent study with Professor Michael Harris. he methods used in this proof brought together nearly all my mathematical studies from algebraic geometry, number theory and Galois theory, representation theory, and algebraic topology, linking together disparate pieces to form an incredibly beautiful, powerful, and surprising argument. 

Comparison between cohomology theories played a major role in this study. This, coupled with Prof. Johan de Jong's seminar on Weil cohomology theories and algebraic de Rham cohomology which I was concurrently attending, inspired me to read further on this cohomology zoo. During this independent study, I read sections of Grothendieck's ``Tohoku'' paper to understand Delinge's use of hypercohomology, universal $\delta$-functors, and the \etale site and its cohomology. I also read Milne's notes on motives to understand the motivic perspective on absolute Hodge cycles presented in Milne's treatment of Deligne's proof. An important ingredient of Deligne's proof involved constructing a family of abelian varieties over a moduli space such that a distinguished fiber is of CM-type. This moduli space turns out to be a Shimura variety. The following semester, I continued independent study with Prof. Harris on Shimura varieties which solidifed my interest in arithmetic geometry. 

(Research HERE)

Although I have worked on a variety of research projects reflecting my broad interests in the sciences, mathematical research experience has shaped my other main research interest.

I participated in the 2018 Columbia math REU studying the zeta functions of diagonal weighted-projective surfaces over finite fields using computational methods. We aimed to generalize Shioda’s classification of supersingular Fermat varieties to weighted-projective diagonal hypersurfaces. However, the naive generalization given by applying Shioda’s classification to the minimal covering Fermat surface provided a sufficient but not necessary condition for supersingularity. We approached the problem from an arithmetic perspective by using a result of Weil to compute the zeta functions of these diagonal hypersurfaces in terms of Gaussian sums. We then applied Stickelberger's theorem to determine the factorization of ideals generated by Gaussian sums and thus determine the roots and poles of the zeta function corresponding to eigenvalues of the Frobenius action on the variety’s l-adic cohomology reducing the problem to a numerical condition on the exponents and characteristic. Using a computer search, I was able to identify patterns in certain new examples of supersingular surfaces. From this observation, I proved the existence of an infinite family of supersingular weighted-projective surfaces such that the minimal Fermat surface parametrizing them fails to be supersingular. We then identified other infinite families with this same property.  
\par
The summer of 2019, I had the wonderful opportunity to study toric geometry and inequalities in convex geometry in Paris through a joint program between Columbia and Paris Diderot University. Under the direction of Prof Huayi Chen, my group studied the relationship between intersection pairings of big nef divisors and inequalities in convex geometry and the relationships between these inequalities and constructions on toric varieties. Associated to such divisors are compact convex sets called Okunkov bodies whose volume reflects the intersection pairing and asymptotic number of sections. Variants of the Brunn-Minkowski, Alexandrov-Finchel, and isoperemetric inequalities applied to these Okunkov bodies can be strengthened by introducing probabilistic techniques [1]. Specifically, an important term arising in these inequalities is the correlation between convex bodies which has a form similar to a Kantorovich optimal transport problem. We established an upper bound on the correlation between special cases of convex pairs using the Brenier map.  
\par
The Paris Diderot program gave me a solid background in toric geometry which prepared me for my thesis topic. Under Prof Johan de Jong, I studied the problem of embedding smooth curves in toric surfaces. Using a result of Harris and Mumford [2], I showed that very general curves cannot be embedded in any toric surface and I gave examples showing restrictions to these embeddings being Cartier. This project was inspired by a paper of Dokchitser [3] which provides an algorithm to construct, for a given curve, the minimal regular normal crossings model defined over a DVR. Dokchitser's method requires embedding the curve in a toric surface and constructs the model through gluing semi-toric surfaces associated to subdivisions of the Newton polygon. I constructed a degeneration of a genus 5 curve with a nontrivial Galois action on the components of its special fiber and showed that such a regular normal crossings model cannot result from Dokchitser's method. For my thesis, I was awarded the John Dash van Buren Jr. Prize in Mathematics. This work was an invaluable learning experience about how research is conducted in mathematics: what techniques to try, how to manage frustration, and how problems evolve. It clinched my decision to pursue graduate studies in mathematics. This problem was extremely rewarding and, largely due to Prof de Jong’s excellent mentorship and our working relationship, I am fully convinced that mathematical research is what I want to spend my life doing.  

(FINAL GOALS!!)

At Oxford, I hope to ...



My interests have coalesced around algebraic geometry, specifically birational geometry in positive characteristic. Currently, I am particularly interested in the Shioda conjecture and supersingular algebraic surfaces. Immediately following the completion of my senior thesis, I began working on a related problem (outlined in my research statement) under the direction of Prof de Jong. I intend to pursue this question further during my graduate studies as well as incorporating and exploring the deep interplay between algebraic geometry and branching out into related areas in number theory and complex geometry. I also hope to attend a graduate program with a strong mathematical physics group because I continue to be extremely interested in the ways algebraic geometry finds applications in modern physics. I hope that in my graduate studies, I will be able to unify these two interests towards solving research problems on the boundary between mathematics and physics. My ultimate goal is to become a professor. This goal goes beyond having a research position; I see teaching as a vital aspect of my career both in order to repay the excellent instruction I was given and because I believe teaching others to be an invaluable part of my own intellectual development.


\end{document}