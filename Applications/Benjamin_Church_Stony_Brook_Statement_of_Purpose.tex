\documentclass[11pt]{article}
\usepackage{import}
\import{"../Algebraic Geometry/"}{AlgGeoCommands}

\newcommand{\Loc}[1]{\mathfrak{Loc}\left( #1 \right)}
\newcommand{\AbGrp}{\mathbf{AbGrp}}

\newtheorem*{defnn}{Definition}
\newtheorem*{conj}{Conjecture}

\usepackage{hyperref}
\usepackage{fancyhdr}

\geometry{margin=0.7in}

\pagestyle{fancy}
\fancyhead[LH]{\textbf{Benjamin V. Church}}
\fancyhead[RH]{\textbf{Statement of Purpose: Stony Brook}}
\setlength{\headheight}{15pt}
\setlength{\headsep}{0.2in}

\usepackage[backend=bibtex, citestyle=apa, style=phys]{biblatex}
\addbibresource{bibliography.bib}

\begin{document}
Mathematical research is what I want to spend my life doing. I am captivated by the mysterious depth and startling beauty that springs forth from even basic mathematical objects and their interrelations. Although I cast a wide net of scientific interests and research projects, including published work in astrophysics and bioinformatics, my research interests have firmly coalesced around algebraic and arithmetic geometry. Having completed upper-level mathematics coursework and dipped my toes into mathematical research, I aspire to earn a doctorate at Stony Brook University to pursue mathematical mysteries and build a career in academia. At Stony Brook, I would be honored to work with innovative Professors in my areas of interest, particularly Jason Starr, Radu Laza, and Samuel Grushevsky.
\par
Working closely with professors through independent study was an invaluable opportunity to explore topics outside the standard curriculum and prepare for graduate-level study. 
Reading courses on elliptic curves with Prof.\ David Hansen in spring 2018 and on modular forms and Galois representations with Prof.\ Chao Li in fall 2018 cemented my interest in the interplay between number theory and algebraic curves. In my senior year, advised by Prof.\ Michael Harris, I studied Deligne's proof that Hodge cycles on abelian varieties are absolutely Hodge. I found Deligne's argument extremely compelling because it brought together many aspects of my mathematical studies from algebraic geometry to representation theory, linking together disparate pieces to form a marvelous and surprising whole. Concurrently, I attended Prof.\ Johan de Jong's weekly seminar on Weil cohomology theories and algebraic de Rham cohomology. Together, these experiences motivated me to read further on comparison theorems and Hodge theory. I also read Milne's notes on motives to understand the motivic perspective on absolute Hodge cycles. An important ingredient in Deligne's proof involves constructing a family of abelian varieties such that a distinguished fiber is of CM-type. The base turns out to be a Shimura variety encoding moduli of abelian varieties, sparking my interest in arithmetic geometry and leading me to pursue further independent study with Prof.\ Harris on Shimura varieties.
\par
I participated in the 2018 Columbia math REU studying the zeta functions of surfaces over finite fields. Under the supervision of Professors Daniel Litt and Alex Perry, we aimed to generalize Shioda’s classification of supersingular Fermat varieties \footfullcite{shioda_on_fermat} to weighted-projective diagonal hypersurfaces. We computed the roots and poles of the zeta functions in terms of Gaussian sums using Weil's formula\footfullcite{weil_counting} and then applied Stickelberger's theorem to efficiently determine supersingularity. Using a computer search, I was able to identify patterns in certain new examples of supersingular surfaces. From this observation, I proved the existence of an infinite family of supersingular surfaces such that the minimal covering Fermat surface fails to be supersingular. My team and I then identified other infinite families with these properties. This project solidified my love of algebraic geometry, especially geometry in positive characteristic and its relations to arithmetic. It also introduced me to the Weil conjectures which inspired me to study scheme theory and \etale cohomology, devoting myself to EGA, Hartshorne exercises, and Milne's \etale cohomology, in order to understand the proofs of Grothendieck and Deligne. The introduction of \etale cohomology to explain, geometrically, properties of the numbers of solutions to polynomials over finite fields remains my absolute favorite piece of math. 
\par
In summer 2019, I had the wonderful opportunity to study toric geometry and inequalities in convex geometry in Paris through a joint REU program between Columbia and Paris Diderot University. Led by Prof.\ Huayi Chen, my group studied intersection pairings of divisors and their relationship to mixed volumes of the associated Okunkov bodies. Variants of the Brunn-Minkowski inequality for divisors can be strengthened by introducing probabilistic techniques \footfullcite{probabiliste}.
An important term arising in these inequalities is the correlation index between two convex bodies which takes the form of a modified Monge-Kantorovich optimal transport problem. Based on the Monge formulation, we applied the Brenier map and Knothe map whose Jacobians, and thus the Radon-Nikodym derivative of their induced measures, are understood thereby giving upper bounds on the correlation index purely in terms of the volumes of the convex bodies. Although our results turned out to be known in the literature, the experience exposed me to new methods in geometry and improved my flexibility in approaching research.
\par
The Paris Diderot program gave me a solid background in toric geometry which prepared me for my undergraduate thesis topic. Under Prof.\ de Jong, I studied the problem of embedding smooth curves in toric surfaces. Using a result of Harris and Mumford \footfullcite{harris1982kodaira}, I gave a proof that very general curves cannot be embedded in any toric surface and I studied obstructions to these embeddings having transverse intersection with the toric divisor. This project aimed to investigate the regularity conditions introduced in a paper of Dokchitser \footfullcite{models_of_curves} which provides an algorithm to construct the minimal r.n.c.\ models of certain curves over DVRs using toric embeddings. I constructed a degeneration of a genus 5 curve with a nontrivial Galois action on the components of its special fiber and showed that such an r.n.c.\ model cannot result from Dokchitser's method. Thus, no affine equation for this curve can satisfy required regularity conditions on its intersections with the toric divisors. This work was an invaluable learning experience about how research is conducted in mathematics: what techniques to try, how to manage frustration, and how problems evolve. Thanks to Prof.\ de Jong’s excellent mentorship and our working relationship, the process was also personally rewarding and enjoyable; it fully convinced me that mathematical research is what I want to spend my life doing. 
\par 
Although I had planned to spend a gap year working and traveling overseas, this became impossible due to the Covid-19 pandemic. Instead, immediately following the completion of my thesis, I began a project with Prof.\ de Jong determining the unirationality of various supersingular surfaces which I had discovered during the 2018 REU. These surfaces may be described as cyclic covers of $\mathbb{P}^2$ and also as cyclic quotients of products of easily understood curves. I am working on methods to find nonfree rational curves on such surfaces which are applicable in positive characteristic. Additionally, Prof.\ de Jong guided my reading of papers on N\'{e}ron models, algebraic connections and Atiyah classes, Frobenius descent and $p$-curvature, and unirational $3$-folds. Currently, I am attending Prof.\ Max Lieblich's seminar and his joint course with Prof.\ de Jong on resolution of singularities as well as Prof.\ Litt's course on \etale cohomology.
\par
I have been passionate about teaching since high school when I organized a student-run seminar promoting interest in pure mathematics. I have volunteered to teach over two dozen classes at Columbia and MIT Splash aimed at curious high school students, a six-week HSSP course introducing high school students to elliptic curves, and a dozen talks at Columbia's math and physics clubs aimed at undergraduates. Last year, in collaboration with the Columbia Association for Women in Mathematics, I helped create introductory talks, materials, and help sessions aimed at supporting freshmen who were new to college-level mathematics courses. I was specifically selected by Prof.\ Brian Cole to teach weekly recitations for his accelerated physics course, an unusual honor for an undergraduate. Teaching was one of the highlights of my college career as I got the pleasure of leading the next cohort of eager students and seeing the fascination I have for the subject mirrored in them.
\par
Given the outstanding scholarship of Stony Brook’s algebraic geometry group, as well as the size and breadth of the department, there are many professors at Stony Brook with whom I would be interested in working. I am particularly interested in Prof. Starr’s work studying the cohomology of varieties over function fields, rational points on rationally connected varieties, and the structure of Fano manifolds. I am also interested in Prof. Laza’s work studying the birational structure of moduli spaces and his papers relating the limiting behavior of Hodge structures of a degeneration to the mixed Hodge structure of its singular fiber. Prof. Grushevsky’s work on compactifications of the moduli space of principally polarized abelian varieties and on extending the map taking a curve to its jacobian at the level of moduli spaces to the compactifications particularly interests me. My passion for mathematics has grown into dedication which motivated me to pursue intense coursework, research projects, and tenacious self-study. This dedication has prepared me with the technical background and perseverance necessary to thrive in the rigorous Ph.D. program at Stony Brook. I strongly believe that Stony Brook is an ideal place for my intellectual development and mathematical studies, and I sincerely hope to be given the opportunity to learn from and contribute to this vibrant community. Thank you for considering my application.
\end{document}