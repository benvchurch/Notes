\documentclass[11pt]{article}
\usepackage{import}
\import{"../Algebraic Geometry/"}{AlgGeoCommands}

\newcommand{\Loc}[1]{\mathfrak{Loc}\left( #1 \right)}
\newcommand{\AbGrp}{\mathbf{AbGrp}}

\newtheorem*{defnn}{Definition}
\newtheorem*{conj}{Conjecture}

\usepackage{hyperref}
\usepackage{fancyhdr}
\geometry{margin=0.8in}
\pagestyle{fancy}
\fancyhead[LH]{\textbf{Benjamin V. Church}}
\fancyhead[RH]{\textbf{Statement of Purpose: NYU Courant}}
\setlength{\headheight}{15pt}
\setlength{\headsep}{0.2in}


\usepackage[backend=bibtex, citestyle=apa, style=phys]{biblatex}
\addbibresource{bibliography.bib}
\usepackage{setspace}
\doublespacing
\begin{document}
Mathematical research is what I want to spend my life doing. Although I cast a wide net of scientific interests and research projects, including published work in astrophysics and bioinformatics, my interests have firmly coalesced around algebraic and arithmetic geometry. Having completed upper-level mathematics coursework and dipped my toes into mathematical research, I aspire to earn a doctorate at NYU Courant to build a career in academia. At NYU, I would be honored to work with innovative Professors in my areas of interest, particularly Fedor Bogomolov, Yuri Tschinkel, and Alena Pirutka.
\par
Working closely with professors through independent study was an invaluable opportunity to explore topics outside the standard curriculum and prepare for graduate-level study. Reading courses on elliptic curves with Prof.\ David Hansen in spring 2018 and on modular forms and Galois representations with Prof.\ Chao Li in fall 2018 cemented my interest in the interplay between number theory and algebraic curves. In my senior year, advised by Prof.\ Michael Harris, I studied Deligne's proof that Hodge cycles on abelian varieties are absolutely Hodge. Concurrently, I attended Prof.\ Johan de Jong's weekly seminar on Weil cohomology theories and algebraic de Rham cohomology. These experiences motivated me to read further on comparison theorems, Hodge theory, and motives. Deligne's proof also introduced me to Shimura varieties encoding moduli of abelian varieties, sparking my interest in arithmetic geometry and leading me to pursue further independent study with Prof.\ Harris on Shimura varieties.
\par
I participated in the 2018 Columbia math REU studying the zeta functions of surfaces over finite fields. Under the supervision of Professors Daniel Litt and Alex Perry, we aimed to generalize Shioda’s classification of supersingular Fermat varieties \footfullcite{shioda_on_fermat} to weighted-projective diagonal hypersurfaces. We implemented an efficient algorithm for determining supersingularity using Stickelberger's theorem and Jacobi sums\footfullcite{weil_counting}. Using a computer search, I identified patterns in certain new examples of supersingular surfaces. From this observation, I proved the existence of an infinite family of supersingular surfaces such that the minimal covering Fermat surface fails to be supersingular. This project solidified my love of algebraic geometry, especially geometry in positive characteristic and its relations to arithmetic. It also introduced me to the Weil conjectures which inspired me to study scheme theory and \etale cohomology, devoting myself to EGA, Hartshorne exercises, and Milne's \etale cohomology, in order to understand the proofs of Grothendieck and Deligne. 
\par
After building a solid foundation in toric geometry at the 2019 Paris Diderot University REU, I decided to write my senior thesis under Prof.\ de Jong on embedding curves in toric surfaces. Using a corollary of Harris and Mumford's result\footfullcite{harris1982kodaira} on the Kodaira dimension of $\overline{\mathcal{M}}_g$, I gave a proof that very general curves cannot be embedded in any toric surface and I studied obstructions to these embeddings intersecting the toric divisor transversally. This project aimed to investigate the regularity conditions introduced in a paper of Dokchitser \footfullcite{models_of_curves} which provides an algorithm to construct the minimal r.n.c.\ models of certain curves over DVRs using toric embeddings. I constructed a degeneration of a genus 5 curve which cannot result from Dokchitser's method showing that no affine equation for this curve can satisfy required regularity conditions. This work was an invaluable learning experience about how research is conducted in mathematics. Thanks to Prof.\ de Jong’s excellent mentorship and our working relationship, the process was also personally rewarding and enjoyable; it fully convinced me that mathematical research is what I want to spend my life doing. 
\par
Upon completing my thesis, I began a project with Prof.\ de Jong determining the unirationality of various supersingular surfaces which I had discovered during the 2018 REU.  I am working on methods to find nonfree rational curves on such surfaces applicable in positive characteristic. Additionally, Prof.\ de Jong guided my reading of papers on N\'{e}ron models, Atiyah classes, Frobenius descent and $p$-curvature, and unirational $3$-folds. Currently, I am attending Prof.\ Max Lieblich's seminar and his course on resolution of singularities as well as Prof.\ Litt's course on \etale cohomology.
\par
The NYU mathematics department's expertise in algebraic and arithmetic geometry would foster my current interests, and its breadth would inspire me to explore new research areas. I am particularly inspired by Prof.\ Bogomolov’s far-reaching contributions to algebraic geometry. Through my study of unirational surfaces, I have gained a deep appreciation for Prof.\ Bogomolov’s techniques for studying rational curves and his results extending Mori-Mukai theory showing, for example, that there exist infinitely many rational curves on certain K3 surfaces. Prof.\ Tschinkel’s research on rational points of algebraic varieties and equivariant birational geometry, particularly his introduction of obstructions to the rationality of geometrically rational 3-folds using cycle class maps, are areas I am particularly interested in pursuing. The research of Prof.\ Pirutka on stable rationality is another area of interest. I am particularly intrigued by her papers showing that generic cubic bundles are not stably rational. My passion for mathematics has motivated me to pursue intense coursework, research projects, and tenacious self-study, preparing me with the technical background and perseverance necessary to thrive in the rigorous Ph.D. program at NYU. I strongly believe that NYU is an ideal place for my intellectual development and mathematical studies, and I sincerely hope to be given the opportunity to learn from and contribute to this vibrant community.
\end{document}