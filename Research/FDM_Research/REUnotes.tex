\documentclass{article}
\usepackage[utf8]{inputenc}

\usepackage{amsthm, amssymb, amsmath, centernot}

\newcommand{\notimplies}{%
  \mathrel{{\ooalign{\hidewidth$\not\phantom{=}$\hidewidth\cr$\implies$}}}}

\renewcommand\qedsymbol{$\square$}
\newcommand{\cont}{$\boxtimes$}
\newcommand{\divides}{\mid}
\newcommand{\ndivides}{\centernot \mid}
\newcommand{\Z}{\mathbb{Z}}
\newcommand{\N}{\mathbb{N}}
\newcommand{\C}{\mathbb{C}}
\newcommand{\Zplus}{\mathbb{Z}^{+}}
\newcommand{\Primes}{\mathbb{P}}
\newcommand{\ball}[2]{B_{#1} \! \left(#2 \right)}
\newcommand{\Q}{\mathbb{Q}}
\newcommand{\R}{\mathbb{R}}
\newcommand{\Rplus}{\mathbb{R}^+}
\newcommand{\invI}[2]{#1^{-1} \left( #2 \right)}
\newcommand{\End}[1]{\text{End}\left( A \right)}
\newcommand{\legsym}[2]{\left(\frac{#1}{#2} \right)}
\renewcommand{\mod}[3]{\: #1 \equiv #2 \: \mathrm{mod} \: #3 \:}
\newcommand{\nmod}[3]{\: #1 \centernot \equiv #2 \: mod \: #3 \:}
\newcommand{\ndiv}{\hspace{-4pt}\not \divides \hspace{2pt}}
\newcommand{\finfield}[1]{\mathbb{F}_{#1}}
\newcommand{\finunits}[1]{\mathbb{F}_{#1}^{\times}}
\newcommand{\ord}[1]{\mathrm{ord}\! \left(#1 \right)}
\newcommand{\quadfield}[1]{\Q \small(\sqrt{#1} \small)}
\newcommand{\vspan}[1]{\mathrm{span}\! \left\{#1 \right\}}
\newcommand{\galgroup}[1]{Gal \small(#1 \small)}
\newcommand{\sm}{\! \setminus \!}
\newcommand{\topo}{\mathcal{T}}
\newcommand{\base}{\mathcal{B}}
\renewcommand{\bf}[1]{\mathbf{#1}}
\renewcommand{\Im}[1]{\mathrm{Im} \: #1}
\renewcommand{\empty}{\varnothing}


\newenvironment{definition}[1][Definition:]{\begin{trivlist}
\item[\hskip \labelsep {\bfseries #1}]}{\end{trivlist}}

\theoremstyle{theorem}
\newtheorem{theorem}{Theorem}[section]
\newtheorem{lemma}[theorem]{Lemma}
\newtheorem{corollary}[theorem]{Corollary}
\newtheorem{conjecture}[theorem]{Conjecture}

\theoremstyle{definition}
\newtheorem*{problem}{Problem}

\theoremstyle{definition}
\newtheorem*{proposition}{Proposition}

\theoremstyle{remark}
\newtheorem*{fact}{Fact}

\theoremstyle{definition}
\newtheorem{example}{Example}[section]

\theoremstyle{remark}
\newtheorem{remark}{Remark}[subsection]

\begin{document}
\author{Benjamin Church}
\title{\Huge REU Notes}

\maketitle
\tableofcontents
\newpage

\section{Introduction}

Suppose $f \in \Z[X_0, \cdots, X_n]$ is an integral polynomial in $n$-variables. We want to find relations between the complex solutions to $f = 0$ and the solutions of $f$ over $\Z, \Q, \finfield{p^k}$. We want to find out how many solutions to the system, 
\begin{align*}
f_1(x_0, \cdots, x_n) & = 0
\\
\vdots & 
\\
f_r(x_0, \cdots, x_n) & = 0 
\end{align*}
over $\finfield{q}[x_0, \cdots, x_n]$. How many solutions does this system have over $\finfield{q^k}$. 
\begin{definition}
$X(\finfield{q^k})$ is the set of solutions to $\forall i : f_i = 0$ in $(\finfield{q^k})^{n+1}$.
\end{definition}
Weil defined the zeta function,
\begin{definition}
\[\zeta_X(t) = \exp\left[ \sum_{r = 1}^\infty \frac{\#|X(\finfield{q^r})|}{r} t^r \right] \]
\end{definition}
and made the following conjectures which were later proven,
\begin{theorem}
The following hold about $\zeta_X$ for a variety $X$, 
\begin{enumerate}
\item $ \zeta_X(t)$ is a rational function.
\item If $f_i$ are homogenous polynomials define the projective solutions, 
\[ X'(\finfield{q^k}) = \frac{X(\finfield{q^k}) \sm \{ \vec{0} \}}{ \finfield{q^k}^\times} \]
and $\zeta_{X'}(t)$ satisfied a function equation if $X'$ is smooth. 
\item All roots and poles $\zeta_X$ in $\C$ are $q$-Weil numbers e.g. if $\alpha$ is a root or pole then $|\alpha| = q^{i/2}$ for some $i \in \Z$. 
\item If $X$ is smooth and projective then 
\[ \zeta_X(t) = \prod_{i = 0}^d p^i(t)^{(-1)^{i+1}} \]
where $d = \dim{X}$ and $p_i \in \Z[t]$. For almost all primes $\deg{p_i} = b_i(X(\C))$ and the roots of $p_i$ have absolute value $q^{i/2}$ where $b_i$ is the $i^{\mathrm{th}}$ Betti number.
\end{enumerate}

\begin{definition}
Suppose $f(x_0, x_n)$ is a homogeneous polynomial in $\finfield{q}[x_0, \cdots, x_n]$ then,
\[ X(\finfield{q^k} = \frac{ \{\vec{x} \in (\finfield{q^k})^{n+1} \mid f(\vec{x}) = 0\} \sm \{\vec{0}\}}{\finfield{q^k}^\times} \]
\end{definition}

\begin{definition}
A variety $X$ is supersigular if $\zeta_X$ has the property that every root and pole is of the form $\omega q^{i/2}$ where $\omega$ is a root of unity.
\end{definition}


There are may conjectures about supersigular varieties.

\begin{definition}
A $2$-dimensional hypersurface is unirational if there exist $x_i(t_1, t_2) \in \overline{\finfield{q}}(t_1, t_2)$ such that $f(x_0(t_1, t_2), \cdots, x_3(t_1, t_2)) = 0$. 
\end{definition}

\begin{conjecture}[Shioda]
A supersingular $2$-dimensional hypersurface is unirational 
\end{conjecture}

\begin{conjecture}[Tate]
Supersingular surjaces have a lot of curves in them!
\end{conjecture}

\end{theorem} 

\end{document}
