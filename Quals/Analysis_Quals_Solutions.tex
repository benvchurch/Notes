\documentclass[12pt]{article}
\usepackage{import}
\import{"../Algebraic Geometry/"}{AlgGeoCommands}

\newcommand{\Loc}[1]{\mathfrak{Loc}\left( #1 \right)}
\newcommand{\AbGrp}{\mathbf{AbGrp}}

\renewcommand{\K}{\mathbb{K}}

\newcommand{\inner}[2]{\left< #1, #2 \right>}

\newcommand{\B}{\mathcal{B}}
\newcommand{\R}{\mathbb{R}}
\renewcommand{\C}{\mathbb{C}}
\renewcommand{\T}{\mathbb{T}}

\newcommand\eqae{\mathrel{\stackrel{\makebox[0pt]{\mbox{\normalfont\tiny a.e.}}}{=}}}
\renewcommand{\F}{\mathcal{F}}

\renewcommand{\D}{\mathcal{D}}

\begin{document}

\tableofcontents

\section{Notes for Meeting}

\begin{enumerate}
\item For Volterra operator why do we need $p > 1$ what about $p = 1$?.
\item Argument about if $B$ is metrizable then $X$ with weak topology should be metrizable?
\item Ask about Holder norm!
\end{enumerate}


\section{Spring 2009 Part II}

\subsection{1}

\subsubsection{a}

\begin{exercise}
Let $f \in C(\T)$ and let $M_f$ be the multiplication operator on $L^p(\T)$ for $p \in [1, \infty)$. Show that $\sigma(M_f) = f(\T)$.
\end{exercise}

Consider $\psi = (M_f - \lambda I) \phi$. Since $\psi(x) = (f(x) - \lambda) \phi(x)$ we see that,
\[ \phi(x) = \frac{\psi(x)}{f(x) - \lambda} \]
If $\lambda \notin f(\T)$ because $\T$ is compact $f(\T)$ is closed so then $(f(x) - \lambda)^{-1}$ is bounded meaning that $\phi \in L^p(\T)$ if $\psi \in L^p(\T)$ so $(M_f - \lambda I)$ is invertible showing that $\lambda \notin \sigma(M_f)$. Thus, $\sigma(M_f) \subset f(\T)$. However, if $\lambda \in f(\T)$ then for any $n \in \Z^{+}$ there is some $\delta_n > 0$ for which $|x - x_0| < \delta_n \implies |f(x) - \lambda| < \frac{1}{n}$ and therefore consider $\psi_n = \delta_n^{-\frac{1}{p}} \chi_{B_{\delta_n(x_0)}}$ so $|| \phi_n || = 1$ but $|\phi_n(x)| \ge n \delta_n^{- \frac{1}{p}}$ for $x \in B_\delta(x)$ and thus,
\[ ||\phi_n || =  \int_\T | \phi_n(x) |^p \, \d{x} \le \int_{B_\delta(x_0)} \delta_n^{-1} n^{p} \, \d{x} = n^p \to \infty \]
Therefore, $(M_f - \lambda I)^{-1}$ would be unbounded if it existed in general and thus $(M_f - \lambda I)$ cannot be bijective by the bounded inverse theorem so $\lambda \in \sigma(M_f)$. Therefore $\sigma(M_f) = f(\T)$.

\subsubsection{b}

\begin{exercise}
Let $X$ be a Banach space $D \subset X$ a dense subspace and $A_n \in \L(X)$ and $A \in A$. Show that $A_n \to A$ in the strong operator topology if and only if $|| A_n ||$ is bounded uniformly in $n$ and $A_n x \to A x$ for each $x \in D$.
\end{exercise}

Clearly if $A_n \to A$ in the strong operator topology then $A_n x \to A x$ for each $x \in D$ by definition. Furthermore, since $A_n x$ is a convergent sequence it is bounded. Thus $\{ A_n \}$ is pointwise bounded so by the uniform boundedness prinicipal it is uniformly bounded i.e. $|| A_n || < M$.
\bigskip\\
Conversely, suppose that $A_n x \to A x$ for each $x \in D$ and $|| A_n || < M$. Then for any $x \in X$ choose a sequence $x_k \in D$ such that $x_k \to x$ in $X$. Now consider,
\[ || A_n x - A x || \le || A_n x - A_n x_k || + || A_n x_k - A x_k || + || A x_k - A x|| \]
Furthermore,
\[ || A_n x - A_n x_k|| = || A_n (x - x_k) || \le || A_n || \cdot || x - x_k || \]
and also,
\[ || A x_k - A x || = || A (x_k - x) || \le || A || \cdot || x - x_k || \]
Therefore,
\[ || A_n x - A x|| \le (|| A_n || + || A ||) || x - x_k || + || A_n x_k - A x_k || \]
For each $\epsilon > 0$ we can choose $k$ such that,
\[ || x - x_k || < \frac{\epsilon}{2(|| A || + M)} \]
and therefore,
\[ || A_n x - A x|| \le || A_n x_k - A x_k || + \tfrac{\epsilon}{2} \]
and then we can choose $N$ such that for each $n > N$ we have,
\[ || A_n x_k - A x_k || < \tfrac{\epsilon}{2} \]
such that for $n > N$ we have,
\[ || A_n x - A x || \le \epsilon \]
proving that $A_n x \to A x$ and thus $A_n \to A$ in the strong operator topology.

\subsection{2}

\subsubsection{a}

Let $Y$ be a normed complex nector space and $\lambda : Y \to \C$ linear but not continuous. Consider $N = \ker{\lambda} \subset Y$. Choose any $z \in Y$ such that $\lambda(z) \neq 0$ (exists because the zero map is continuous) then we can decompose $Y = z \C \oplus N$ as a vector space. Now $\overline{N}$ is a linear space by continuity of the addition and scaling operations. If $y \in \overline{N} \setminus N$ then we can write $y = \alpha z + n$ for some $n \in N$ with $\alpha \neq 0$ so $z = \alpha^{-1}(y - n)$ and thus $z \in \overline{N}$ so $\overline{N} = Y$. Therefore, either $N$ is closed or $N$ is dense. Thus it suffices to show that $N$ is not closed. Suppose $N$ were closed then we know that $\pi : Y \to Y / N \cong \C z$ is continuous and $\lambda$ factors through $Y/N \to \C$ meaning that $\lambda$ is continuous because any linear map between finite dimensional spaces is continuous contradicting our assumption. Thus $N$ is dense.

\subsubsection{b}

Let $X = L^2([0,1])$ and $Y = C^1([0,1])$ with the $L^2$-norm. Note that $X$ is Banach but $Y$ is only normed. Consider the map $\lambda : Y \to \C$ given by $\lambda(\phi) = \phi'(\tfrac{1}{4}) - \phi(\tfrac{1}{2})$. I claim that $\lambda$ is not continuous. Indeed, we can find smooth functions with integral $1$ such that $\phi'(\tfrac{1}{4}) = 0$ but $\phi(\tfrac{1}{2})$ is arbitrarily large (use bump functions centered around $\tfrac{1}{2}$). Thus $\lambda$ is not bounded and so not continuous. Then $N = \ker{\lambda}$ is dense in $C^1([0,1])$ and therefore also dense in $L^2([0,1])$. By a Gram-Schmidt process and Zorn's lemma, then $N$ contains an orthonormal basis see \chref{https://math.stackexchange.com/questions/201119/a-complete-orthonormal-system-contained-in-a-dense-sub-space}{this answer} for details.

\subsection{3}

Let $K \in L^p([0,1] \times [0,1])$ with $1 < p < \infty$ let $q$ be the dual exponent. Let $A : L^q([0,1]) \to L^p([0,1])$ be the operator,
\[ (A f)(x) = \int K(x, y) \, f(y) \, \d{y} \]

\subsubsection{a}

To show that $A f$ exists almost everywhere and that $A \in \L(L^q([0, 1]), L^p([0,1]))$ it suffices to show that $A$ is bounded in $L^p$ norm (because then $A f$ must be finite almost everywhere). Now consider,
\[ || A f ||_p^p = \int \left| \int K(x, y) f(y) \, \d{y} \right|^p \, \d{x} \]
By the H\"{o}lder inequality,
\[ \left| \int K(x, y) f(y) \, \d{y} \right| \le \int | K(x,y) f(y) | \, \d{y} \le || K(x, -) ||_p \cdot || f ||_q \]
where,
\[ || K(x, -) ||_p = \left( \int |K(x,y)|^p \, \d{y} \right)^{\frac{1}{p}} \]
Therefore, 
\[ || A f ||_p^p \le || f ||_q^p \int \left(  \int |K(x,y)|^p \, \d{y} \right) \d{x} = || f ||_q^p \int_{[0,1] \times [0,1]} |K(x,y)|^p \, \d{(x,y)} = || f ||_q^p \cdot || K ||_p^p \]
and therefore,
\[ || A f || \le || K ||_p \cdot || f ||_q \]
proving that $A$ is a bounded operator with $|| A || \le || K ||_p$.

\subsubsection{b}

Suppose that for every $f \in L^q([0,1])$ we have $(A f)(x) = 0$ almost everywhere. This means that $A f = 0$ in $L^q([0,1])$ so $A$ is the zero operator. Now, for any function $g \in L^q([0,1]) = (L^p([0, 1]))^*$ we know that,
\[ \int g(x) (A f)(x) \, \d{x} = 0 \]
and therefore,
\[ \int_{[0, 1] \times [0,1]} K(x,y) g(x) f(y) \, \d{(x, y)} = 0 \]
by Fubini. However, I claim that the span of functions of the form $g(x) f(y)$ for $f,g \in L^q([0,1])$ are dense in $L^q([0,1] \times [0,1])$. This follows from the density of continuous functions and Stone-Weierstrass applied to finite sums of these products. Therefore, $K = 0$ in $L^p([0,1] \times [0,1])$ (meaning $K(x,y) = 0$ almost everywhere) because it is killed by a dense subset of $L^q([0,1] \times [0,1]) = (L^p([0,1] \times [0,1])^*$.

\subsection{4}

Let $\S'(\R^n)$ denote the space of tempered distributions, and $H^s(\R^n)$ be the subspace of $\S'(\R^n)$ of distributions $f$ whose Fourier transforms are represented by a function $\hat{f}(\xi)$ such that,
\[ || f ||_{H^s}^2 = \int (1 + | \xi|^2)^s | \hat{f}(\xi)|^2 \, \d{\xi} < \infty \]

\subsubsection{a HOW??}

We know that the Fourier transform gives an isometry $L^2(\R^n) \to L^2(\R^n)$ and therefore if $\hat{f}$ is represented by an $L^2$ function then so is $f$ because,
\[ \inner{f}{\varphi} = \inner{\hat{f}}{\hat{\varphi}} = \int \overline{\hat{f}} \hat{\varphi} = \int \overline{ \F^{-1} \hat{f}} \varphi \] 
by Parseval. Therefore, for any $s \ge 0$ we actually have,
\[ || \hat{f} ||_{L^2} = \left( \int | \hat{f}(\xi) |^2 \, \d{\xi} \right)^{\frac{1}{2}} \le \left( \int (1 + |\xi|^2)^s | \hat{f}(\xi) |^2 \, \d{\xi} \right)^{\frac{1}{2}} = || f ||_{H^s} \]
and therefore $\hat{f} \in L^2(\R^n)$ for all $f \in H^s(\R^n)$ proving that $f \in L^2(\R^n)$. Now consider the case $p = 2$. Then by Parseval,
\[ || f ||_{L^2} = \left( \int | \hat{f}(\xi) |^2 \, \d{\xi} \right)^{\frac{1}{2}} \le \left( \int (1 + | \xi |^2)^s | \hat{f}(\xi) |^2 \, \d{\xi} \right)^{\frac{1}{2}} = || f ||_{H^s} \]
Now we consider the case $p = \infty$. Let $s > \frac{n}{2}$ then by H\"{o}lder's inequality,
\begin{align*}
|f(x)| & = \left| \int \hat{f}(\xi) e^{2 \pi i \xi x} \, \d{\xi} \right| \le \int | \hat{f}(\xi) | \, \d{\xi} =  \int \frac{(1 + |\xi|^2)^{\frac{s}{2}}}{(1 + | \xi|^2)^{\frac{s}{2}}} | \hat{f}(\xi) | \, \d{\xi} 
\\
& \le \left( \int (1 + |\xi|^2)^{-s} \, \d{\xi} \right)^{\frac{1}{2}} \cdot \left( \int (1 + |\xi|^2)^{s} | \hat{f}(\xi)|^2 \, \d{\xi} \right)^{\frac{1}{2}} = C(s) || f ||_{H^s} 
\end{align*}
where,
\[ C(n, s) = \left( \int (1 + |\xi|^2)^{-s} \, \d{\xi} \right)^{\frac{1}{2}} \]
is finite because $s > \frac{n}{2}$. Therefore,
\[ || f ||_{\infty} = \sup_{x \in \R^n} | f(x) | \le C(n, s) || f ||_{H^s} \]
proving the claim.
\bigskip\\
Now we consider $2 < p < \infty$ and take $s > \frac{n}{2} - \frac{n}{p}$. Let $1 < q < 2$ be the conjugate exponent to $p$. Then we consider,
\[ \frac{1}{q} = \frac{1}{2} + \frac{1}{r} \]
and therefore,
\[ || \hat{f} ||_q \le \left( \int (1 + |\xi|^2)^s | \hat{f}(\xi) |^2 \, \d{\xi} \right)^{\frac{1}{2}} \cdot \left( \int (1 + | \xi |^2 )^{-rs} \, \d{\xi} \right)^{\frac{1}{r}} \]
However,
\[ r = (\tfrac{1}{2} - \tfrac{1}{p})^{-1} \]
and therefore if $s > \frac{n}{2} - \frac{n}{p}$ then $rs > n$ and therefore,
\[ C(n,p,s) = \left( \int_{\R^n} (1 + | \xi |^2 )^{-rs} \, \d{\xi} \right)^{\frac{1}{r}} \]
is finite meaning that,
\[ || \hat{f} ||_q \le C(n,p,s) || f ||_{H^s} \]
Furthermore,
\[ || f ||_p = || \F^{-1} \hat{f} ||_p \le || \F^{-1} ||_{L^q \to L^p} \cdot || \hat{f} ||_q \]
where by Riesz-Thorin,
\[ || \F^{-1} ||_{L^q \to L^p} \le || \F^{-1} ||_{L^1 \to L^\infty}^{1 - \theta} \cdot || \F^{-1} ||_{L^2 \to L^2}^\theta  \]
where,
\[ \frac{1}{q} = \frac{1 - \theta}{1} + \frac{\theta}{2} \iff \frac{1}{p} = \frac{\theta}{2} \] 
However, $|| \F^{-1} ||_{L^1 \to L^\infty} = 1$ and $|| \F^{-1} ||_{L^2 \to L^2} = 1$ and therefore,
\[ || f ||_p \le C(n,p,s) || f ||_{H^s} \]
proving the claim.

\subsubsection{b WHAT THE HELL??}

\subsection{5}

Let $X$ be a Banach space over $\C$ and $M,N \subset X$ closed subspaces.

\subsubsection{a}


Consider the map $M \oplus N \to X$ via $(m,n) \mapsto m + n$. Now we pass to the quotient \[ \phi : (M \oplus N)/(M \cap N) \to X \]
to get an injection. Then the image, namely $M + N$, is closed iff $\phi$ is bounded below. I claim that $\phi$ being bounded below is equivalent to there existing $C > 0$ such that for all $x \in M + N$ there is $m \in M$ and $n \in N$ such that $|| m || + || n || \le C || x ||$. Indeed, given this condition, $|| \phi [m,n] || = || x ||$ and 
\begin{align*}
|| [m, n] || & = \inf_{y \in M \cap N} || (m+y,n-y) || = \inf_{y \in M \cap N} || (m + y, n - y) || = \inf_{y \in M \cap N} || m  + y || + || n - y|| 
\\
& \le || m || + || n || \le C || x || = C || \phi [m, n] || 
\end{align*} 
proving that $\phi$ is bounded below. Conversely, if $\phi$ is bounded below then there is some $C > 0$ such that for all $[m,n] \in (M \oplus N)/(M \cap N)$,
\[ || [m,n] || = \inf_{y \in M \cap N} || m + y|| + || n - y || \le C || x || \]
Therefore, we can choose a pair $(m', n')$ such that,
\[ || m' || + || n' || \le C || x || + \epsilon || x || \]
for each fixed choice of $\epsilon > 0$ and $|| x ||$. Then choose $\epsilon = 1$ so we get,
\[ || m' || + || n' || \le (C + 1) || x || \]
proving the claim.

\subsubsection{b}

Suppose that $\ell_M : M \to \C$ and $\ell_N : N \to \C$ are continuous linear functionals and $\ell_M |_{M \cap N} = \ell_N |_{M \cap N}$. Furthermore, suppose that $M + N$ is closed. This means that the injective map $\phi : (M \oplus N)/(M \cap N) \to X$ is an isomorphism onto $M + N$ by the bounded inverse theorem (we need $M + N$ be closed or else it may not be Banach which is required for the theorem). Consider the continuous linear functional $\ell : (M \oplus N)/(M \cap N) \to \C$ defined by $\ell([m,n]) = \ell_M(m) + \ell_N(m)$ which is well-defined because $\ell_M |_{M \cap N} = \ell_N |_{M \cap N}$. Therefore, $\ell \circ \phi^{-1}$ defines a continuous linear functional $M + N \to \C$ which restricts to $\ell_M$ and $\ell_N$ on $M$ and $N$. Therefore, by Hahn-Banach we can extend this to a continuous linear functional $X \to \C$ with the same property.


\subsubsection{c}

Let $X = \ell^2$ and let $M$ be the closed span of $\{ e_{2n} \}$ and $N$ be the closed span on $\{ e_{2n} + \frac{1}{n+1} e_{2n+1} \}$. Then $M + N$ contains each $e_k$ and thus is dense. However, I claim that,
\[ z = \sum \frac{e_{2n + 1}}{n+1} \]
cannot be in $M + N$ so it is not closed. Indeed, suppose we can write,
\[ z = \sum a_n e_{2n} + \sum b_n e_{2n} + \sum \frac{b_n}{n+1} e_{2n+1} \]
Then,
\[ \inner{z}{e_{2n+1}} = \frac{1}{n+1} \implies b_n = 1 \]
giving a contradiction because,
\[ \sum e_{2n} + \sum \frac{1}{n+1} e_{2n+1} \] 
is not a well-defined element of $N$ (it does not converge) even though morally we should be able to take $a_n = -1$ and cancel off the infinite sum but we can only do this at finite stages and then take the limit (i.e. inside $X$) we cannot do it in $M + N$ where we have to take limits beforing adding. 

\section{Fall 2009 Part I}

\subsection{1}

\subsubsection{a}

Let $X, Y$ be compact Hausforff. Consider the subalgebra $A \subset C(X \times Y)$ consisting of sums of functions of the form $u(x,y) = \phi(x) \psi(y)$ for $\phi \in C(X)$ and $\psi \in C(Y)$. By Stone-Weierstrass, to show that $A$ is dense it suffices to show that $A$ separates points and vanishes nowhere. Because $A$ contains the constant function, vanishing nowhere is obvious. Now, given $(x_1, y_1), (x_2, y_2) \in X \times Y$ we can choose continuous functions by Urysohn's lemma $\phi \in C(X)$ and $\psi \in C(Y)$ such that $\phi(x_1) = 0$ and $\phi(x_2) \neq 0$ and $\psi(y_1) = 0$ and $\psi(y_2) \neq 0$. Then clearly $\phi(x) \psi(y)$ separates $(x_1, y_1)$ and $(x_2, y_2)$.

\subsubsection{b}

Let $X$ be a Banach space and $\{ x_i \}$ a sequence in $X$ such that,
\[ \bigcup_{n = 1}^\infty X_n = X \]
where $X_n = \vspan{x_1, \dots, x_n}$. Then $X_n$ is closed because it is the intersection of all $\ker{\ell}$ for $\ell \in X^*$ such that $\ell(x_n) = 0$. Furthermore, if $X_n$ contains an open set then because it is a linear subspace it contains some open neighborhood of the origin. However, any linear subspace containing a ball around the origin equals $X$ by scaling. Therefore either $X = X_n$ or $X_n^\circ = \empty$. In the latter case, by the Baire category theorem, the union of closed sets with empty interior has empty interior contradicting the fact that the union is all of $X$. Thus, $X = X_n$ for some $n$ so $\dim{X}$ is finite.


\subsection{2}

\begin{exercise}
Let $X$ be a Banach space and $S = \{ x \in X \mid || x || = 1 \}$.
\begin{enumerate}
\item Suppose that $y_n \in S$ for all $n$ and $y_n \to y$ for $y \in X$ weakly. Show that $||y || \le 1$
\item Suppose that $X$ is a separable infinite dimensional Hilbert space, $y \in X$ and $||y|| \le 1$. Show that there exists a sequence $\{ y_n \}$ with $y_n \in S$ for all $n$ such that $y_n \to y$ weakly.
\end{enumerate}
\end{exercise}

\subsubsection{a}

Suppose that $|| y || > 1$. Then there exists a $\ell \in X^*$ such that $\ell(y) > 1$ and $|| \ell || = 1$. In particular,
\[ || \ell y_n || \le || \ell || \cdot || y_n || \le 1 \]
becuse $y_n \in S$. Therefore, since we have weak convergence,
\[ \ell(y) = \lim_{n \to \infty} \ell(y_n) \le 1 \]
giving a contradiction. Therefore $|| y || \le 1$.

\subsubsection{b}

Let $X$ be a separable Hilbert space. Then we can find a countable orthonormal basis (in the infinite sum sense) $\{ e_n \}$. Then we can write,
\[ y = \sum_{i = 1}^\infty c_i e_i \]
with,
\[ || y ||^2 = \sum_{i = 1}^\infty |c_i|^2 \le 1 \]
Consider,
\[ y_n = \sum_{i = 1}^n c_i e_i + \sqrt{1 - (|c_1|^2 + \cdots + |c_n|^2)} \cdot e_{n+1} \]
which makes sense because all finite sums of $|c_i|^2$ are less than $1$. Furthermore, $y_n \in S$ by construction. Now I claim that $y_n \to y$ weakly. Any $\ell \in X^*$ has a unique representation as $\ell = \inner{x}{-}$ for $x \in X$ by Riesz. Therefore, consider
\[ \inner{x}{y_n} = \sum_{i = 1}^n c_i \inner{x}{e_i} + \sqrt{1 - (|c_1|^2 + \cdots + |c_n|^2)} \cdot \inner{x}{e_{n+1}} \]
and thus,
\[ \inner{x}{y_n} - \inner{x}{y} = \sqrt{1 - (|c_1|^2 + \cdots + |c_n|^2)} \cdot \inner{x}{e_{n+1}} - \sum_{i = n}^\infty c_i \inner{x}{e_i} \]
However, by H\"{o}lder,
\[ |\inner{x}{y} | \le \sum_{n = 1}^\infty |c_i \inner{x}{y_i}| \le || x || \cdot || y || < \infty \]
we see that,
\[ \lim_{n \to \infty} \sum_{i = n}^\infty | c_i \inner{x}{y_i} | = 0 \]
and therefore,
\[ |\inner{x}{y_n} - \inner{x}{y}| \le \sum_{i = n}^\infty |c_i| \cdot |\inner{x}{e_i}| + \sqrt{1 - (|c_1|^2 + \cdots + |c_n|^2)}  \cdot | \inner{x}{e_{n+1}}| \to 0 \]
since $\sqrt{1 - (|c_1|^2 + \cdots + |c_n|^2)}$ is bounded by $1$. Thus $y_n \to y$ weakly. 

\subsection{3 (DO THIS!!!!!!)}


\subsection{4}

Let $X, Y$ be Banach spaces, $T \in \L(X, Y)$ compact. 

\subsubsection{a}

Let $S \in \L(X, X)$ and $R \in \L(Y, Y)$. First, $TS$ is compact because for any bounded set $B \subset X$ then $S(B)$ is bounded because $S$ is bounded so $T(S(B))$ is precompact so $TS$ is compact. Furthermore, consider $RT(B) = R(T(B))$ where $T(B)$ is precompact. We need to show that $R(T(B))$ is precompact. Because $R$ is continuous $R(\overline{T(B)})$ is compact so in particular it is closed. Therefore, $R(T(B)) \subset R(\overline{T(B)})$ implies that $\overline{R(T(B))} \subset R(\overline{T(B)})$ and a closed subspace of a compact space is compact so $R(T(B))$ is precompact and thus $RT$ is compact.

\subsubsection{b}

Let $B \subset Y^*$ be bounded by $M$. Then consider $T^*(B)$ which we need to show is precompact. Let,
\[ S = \{ x \in X \mid || x || \le 1 \} \]
then $Z = \overline{T(S)}$ is a compact Hausdorff space. It suffices to show that any sequence $\ell_n \in T^*(B)$ has a Cauchy subsequence because then such a sequence will converge in $X^*$ and thus in $\overline{T^*(B)}$. Then there are $\psi_n \in B$ such that $\ell_n = \psi_n \circ T$. Consider,
\[ || \ell_n - \ell_m || = \sup_{|| x || = 1} | \psi_n(T(x)) - \psi_m(T(x)) | \le \sup_{y \in T(S)} | \psi_n(y) - \psi_m(y) | = || \psi_n - \psi_m ||_{\infty} \]
where the last is the uniform norm as a function $C(Z, \R)$. Note that,
\[ | \psi_n(x) - \psi_n(y) | \le || \psi_n || \cdot || x - y || \le M || x - y || \] 
Therefore, if $|| x - y || < \epsilon/M$ then,
\[ | \psi_n(x) - \psi_n(y) | < \epsilon \]
for each $n$ so $\{ \psi_n \} \in C(Z, \R)$ is equicontinuous and pointwise (in fact uniformly by $M$) bounded and thus is precompact so there is a Cauchy subsequence. However, since $|| \ell_n - \ell_m || \le || \psi_n - \psi_m ||_{\infty}$ this implies that $\{ \ell_n \}$ has a Cauchy subsequence so $T^*(B)$ is precompact and thus $T^*$ is compact.


\subsection{5}

We say that a Banach space $B$ is \textit{uniformly convex} if for every $\epsilon \in (0,1)$ there exists $\eta < 1$ such that if $x,y \in B$ with $||x|| = ||y|| = 1$ and $|| x - y || > 2 \epsilon$ then $|| \tfrac{1}{2}(x + y) || < \eta$. Let $B$ be a uniformly convex Banach space.

\subsubsection{a}

Let $x_n \in B$ for $n \in \Z^{+}$ and $x_n \to x_0$ in the weak topology and $||x_n|| \to ||x_0||$. 
\bigskip\\
First, if $\{ x_n \}$ is Cauchy then $x_n \to x$ converges in norm. We need to show that $x = x_0$. Suppose that $x \neq x_0$ then there would exist a linear functional $\ell$ such that $\ell(x) \neq \ell(x_0)$ but,
\[ \ell(x) = \lim_{n \to \infty} \ell(x_n) = \ell(x_0) \]
because of weak convergence. Basically, norm limits imply weak limits and weak limits are unique so it suffices to show that $\{ x_n \}$ has \textit{some} norm limit.
\bigskip\\
Either $x_0 = 0$ in which case $|| x_n || \to ||x_0|| = 0$ already implies convergence in norm or after throwing away finitely many terms we may assume that $x_n \neq 0$ and let $x'_n = x_n / || x_n ||$ then,
\[ \lim_{n \to \infty} \ell(x'_n) = \lim_{n \to \infty} \frac{\ell(x_n)}{|| x_n||} = \frac{\ell(x_0)}{||x_0||} \]
because both limits exist. Therefore, $x'_n \to x'_0$ converges weakly and $|| x'_n || \to 1$ trivially because each $|| x'_n || = 1$. 
\bigskip\\
Now, assume that $\{ x_n' \}$ is not Cauchy. Then for each $\epsilon > 0$ and $N$ there is $n,m > N$ such that,
\[ || x_n' - x_m' || \ge 2 \epsilon \]
so there exists a sequence $y_i = \tfrac{1}{2}(x_{n_j} + x_{m_j})$ of such entries with $n_j \to \infty$ and $m_j > n_j$. By uniform convexity, there exists some $\epsilon < 1$ such that,
\[ || y_i || < \eta \]
However, for any $\ell$ we have,
\[ \lim_{n \to \infty} | \ell(y_i)| = | \ell(x_0) | \]
by weak convergence. By Hahn-Banach there exists some $\ell \in B^*$ with $\ell(x_0) = || x_0 || = 1$ and $|| \ell || = 1$. Thus,
\[ | \ell(y_i) | \le || \ell || \cdot || y_i || < \eta \]
and therefore,
\[ \lim_{n \to \infty} | \ell(y_i) | < \eta \]
but $| \ell(x_0) | = 1$ giving a contradiction. Therefore, $\{ x_n' \}$ must be Cauchy and thus $x_n = x_n' \cdot || x_n ||$ is also Cauchy because we multiplied by a convergent sequence.

\subsubsection{b (FIND EXAMPLE)}



\section{Fall 2009 Part II}


\subsection{1}

\subsubsection{a}

Let $X$ be a finite dimensional real vector space with the induced topology from $\R$ using that $X \cong \R^n$ and let $p, q : X \to \R$ be two norms on $X$. We need to show that there exist constants $c_1, c_2 > 0$ such that $c_1 p(x) \le q(x) \le c_2 p(x)$ for all $x \in X$. Take a basis $e_1, \dots, e_n$ then choose,
\[ c_1 = \inf_{p(x) = 1} q(x) \quad \text{and} \quad c_2 = \sup_{q(x) = 1} p(x) \]
We need to show that these are positive reals.
\bigskip\\
I claim that any norm on $X$ is continuous in the induced topology. It suffices to show that $q^{-1}(B_\epsilon(0))$ is open. Let $c_i = q(e_i)$. If $q(v) < \epsilon$ then let $r_i = (\epsilon - q(v))/(n c_i)$ and if $|a_i - v_i| < r_i$ then,
\begin{align*}
q(a_1 e_1 + \cdots + a_n e_n) & = q(v + (a_1 - v_i) e_1 + \cdots + (a_n - v_i) e_n) 
\\
& \le q(v) + | a_1 - v_1| q(c_1) + \cdots + |a_n - v_n| q(c_n) \le q(v) + r_1 c_1 + \cdots + r_n c_n < \epsilon
\end{align*}
Therefore, 
\[ v \in B_{r_1}(v_1) \times \cdots \times B_{r_n}(v_n) \subset q^{-1}(B_\epsilon(0)) \]
so $q^{-1}(B_\epsilon(0))$ is open. Furthermore, $q$ generates the induced topology on $X$. To see this, consider a standard open $B_{r_1}(0) \times \cdots \times B_{r_n}(0)$. Then let $\epsilon = \min\{ q(e_i) r_i \}$ then if $q(v) < \epsilon$ we have
\bigskip\\
Therefore, $q(x) = 1$ is compact 

(INDUCED TOPOLOGY FROM NORM DOES IT REQUIRE TRANSLATIONS??)

\subsubsection{b (PROVE THIS)}

Let $H$ be a complex Hilbert space and $Y \subset H$ a linear subspace. Let $f \in Y^*$. By Hahn-Banach, there exists a continuous linear extension to $f' \in \overline{Y}^*$ where $\overline{Y}$ is the closure and $|| f' || = || f ||$. Because $f'$ is continuous, it is uniquely determined by its values $f$ on the dense subset $Y \subset \overline{Y}$. Now, because $\overline{Y}$ is a closed subspace, it has an orthognal complement $\overline{Y}^\perp$ and $H = \overline{Y} \oplus \overline{Y}^\perp$ is a orthogonal decomposition. Therefore, we can take $\ell \in H^*$ to be $f'$ on $\overline{Y}$ and zero on $\overline{Y}^\perp$ then clearly $|| \ell || = || f ||$. Furthermore, any $\ell' \in H^*$ has,
\[ || \ell' ||^2 = || \ell'|_{\overline{Y}} ||^2 + || \ell'|_{\overline{Y}^\perp}^2 \]
where $\ell'|_{\overline{Y}} = f'$ so if $|| \ell' || = || f ||$ then $|| \ell'|_{\overline{Y}} || = 0$ which implies that $\ell' |_{\overline{Y}} = 0$ and therefore $\ell' = \ell$ because they are equal on $\overline{Y}$ and $\overline{Y}^\perp$.

\subsection{2 (AHH NO IDEA!!)}



\subsection{3}

For $f \in C([0,1])$ let $T f$ be the function,
\[ (Tf)(x) = \int_0^x f(y) \, \d{y} \]
This is continuous and differentiable. Then we get a well-defined map,
\[ T : C([0,1]) \to C([0,1]) \]
Let $|| f ||_{\infty} = 1$ then consider,
\[ | (Tf)(x)| = \left| \int_0^x f(y) \, \d{y} \right| \le \int_0^x | f(y) | \, \d{y} \le x \]
Therefore, $|| T f ||_{\infty} \le 1$ so we have,
\[ || T || \le 1 \]
I claim that $|| T || = 1$. Indeed, consider $f = 1$ then we have $T f = \id$ and $|| \id ||_{\infty} = 1$ so we see that $|| T || = 1$.
\bigskip\\
I claim that the spectral radius of $T$ is zero. Indeed, we apply the following formula,
\[ r(T) = \lim_{k \to \infty} || T^k ||^{\frac{1}{k}} \]
However, I claim that $|| T^k || \le \frac{1}{k!}$. Indeed, we explicitly have if $|| f ||_{\infty} = 1$ then by definition $|f(x)| \le 1$ which implies that $| (T^k f)(x) | \le \frac{x^k}{k!}$ and thus $ || T^k f ||_{\infty} \le \frac{1}{k!}$. Then,
\[ r(T) = \lim_{k \to \infty} || T^k ||^{\frac{1}{k}} \le \lim_{k \to \infty} (k!)^{-\frac{1}{k}} = 0 \]


\subsection{4}

Let $X$ be a locally convex vector space with topology $\tau$ generated by a family $\{ \rho_\alpha \mid \alpha \in A \}$ of seminorms.

\subsubsection{a}

Let $|| \bullet ||$ be a continuous seminorm on $X$. Then consider, $U = \{ x \in X \mid || x || < 1 \}$ which is open. A neighborhood basis of the origin is given by the finite intersections of sets of the form $\rho_\alpha^{-1}(B_\epsilon(0))$. Therefore, there exist $\alpha_1, \dots, \alpha_n \in A$ such that,
\[ \bigcap_{i = 1}^n \rho_\alpha^{-1}(B_{\epsilon_i}(0)) \subset U \]
Let $C = 2 \max\{ \epsilon_i^{-1} \}$. Then for any $x \in X$, there is some $\lambda \in \R^+$ such that for each $i$,
\[ \rho_{\alpha_i}(\lambda x) = \lambda \rho_{\alpha_i}(x) < \epsilon_i \]
and furthermore, for some $i$ we have $\rho_{\alpha_i}(\lambda x) = \tfrac{1}{2} \epsilon_i$ unless all $\rho_{\alpha_i}(x) = 0$ in which case $\lambda x \in U$ for all $\lambda$ meaning that $|| x || $ so the inequality works. Thus, $\lambda x \in U$ so $\lambda || x || < 1$ and furthermore,
\[ C (\rho_{\alpha_1}(\lambda x) + \cdots + \rho_{\alpha_n}(\lambda x)) \ge 2 \epsilon_i^{-1} \rho_{\alpha_i}(\lambda x) = 1 \] 
which implies that,
\[ || x || \le \lambda^{-1} C (\rho_{\alpha_1}(\lambda x) + \cdots + \rho_{\alpha_n}(\lambda x)) = C (\rho_{\alpha_1}(x) + \cdots + \rho_{\alpha_n}(x)) \]

\subsubsection{b (WHY DID THEY USE $\T$ HERE NOT $[0,1]$?}

Let $X = C^{\infty}(\T)$ whose topology is generated by the seminorms $\{ \rho_k = || \bullet ||_{C^k} \mid k \in \N \}$. We need to show that this topology is not generated by any norm.
\bigskip\\
Suppose that $|| \bullet ||$ is a continuous norm on $X$. Then by the previous problem, there is some $C > 0$ such that,
\[ || f || \le C \left( || f ||_{C^{k_1}} + \cdots + || f ||_{C^{k_n}} \right) \]
for every $f \in C^{\infty}(\T)$. Let $m = \max \{ k_n \}$ then,
\[ || f || \le n C || f ||_{C^m} \]
However, there are sequences $f_n \to f$ converging in the $C^m$ sense that do not converge in $C^\infty$. For example, consider,
\[ f_n(x) = (2 \pi n)^{-(m+1)} \sin{(2 \pi n x)} \]
Then clearly, $f_n \to 0$ in the uniform norm. Furthermore, 
\[ f_n^{(k)}(x) = (2 \pi n)^{k - m-1} \sin{(2 \pi n x)} \]
Therefore, $|| f_n^{(k)} ||_{\infty} \to 0$ for all $k \le m$ and therefore $|| f_n ||_{C^m} \to 0$. However, $|| f_n ||_{C^{m+1}} = 1$ for all $n$ and therefore the sequence does not converge in $C^\infty$. Thus, $|| \bullet ||$ cannot generate the topology on $X$.

\subsection{5}

Let $f : \R \to \C$ be a Schwartz function such that,
\[ \int_\R |f(x)|^2 \, \d{x} = 1 \]
Then let $\hat{f}(y)$ be the Fourier transform of $f$. 
\bigskip\\
We set up some notation. From the inclusion $\S(\R) \subset L^2(\R)$ we get an inner product $\inner{-}{-}$ on $\S(\R)$ (although this does not make $\S(\R)$ a Hilbert space because the topology is finer than the topology of convergence in $\inner{-}{-}$). Now let $T = - i \deriv{}{x}$ is an operator $T : \S(\R) \to \S(\R)$. 
\bigskip\\
Given two operators $A,B : \S(\R) \to \S(\R)$ and $x \in \S(\R)$ with $\inner{x}{x} = 1$ define,
\[ \psi = (A - \inner{x}{A x})x \quad \text{and} \quad \phi = (B - \inner{x}{Bx}) x \]
Then by Cauchy-Schwartz,
\[ |\inner{\psi}{\phi}| \le || \psi || \cdot || \phi || \]
Then, if $A$ is self-adjoint then $\inner{x}{Ax}$ is real and,
\[ \inner{\psi}{\phi} = \inner{x}{(A - \inner{x}{Ax})(B - \inner{x}{Bx})x} = \tfrac{1}{2} \inner{x}{[A,B] x} + \tfrac{1}{2} \inner{x}{\{A, B \} x} - \inner{x}{Ax} \inner{x}{Bx}   \]
Furthermore, if $B$ is self-adjoint then $[A, B]$ is anti-self-adjoint and $\{ A , B \}$ is self-adjoint so the first term is imaginary and the second two terms are real. Therefore,
\[ |\inner{\psi}{\phi}| \ge \tfrac{1}{2} | \inner{x}{[A, B] x}] \]
We are going to apply this to the operators $A = T$ and $B$ being the multiplication by $x$ operators. Clearly $B$ is self-adjoint. Furthermore using integration by parts (which we may apply to Schwartz functions), $T$ is self-adjoint. Furthermore, 
\[ [A, B] f = - i \pderiv{}{x} (x f) + i x \pderiv{}{x} f = - i f \]
Therefore, $[A, B] = - i I$ and thus for $\inner{f}{f} = 1$ we see that, $\inner{f}{[A, B] f} = 1$ and thus,
\[ || \psi ||_2^2 \cdot || \phi ||_2^2 \ge | \inner{\psi}{\phi} |^2 = \tfrac{1}{4} \]
Let $x_0 = \inner{x}{B x}$ and $y_0 = \inner{x}{A x}$ then we find,
\[ || \phi ||_2^2 = \int_\R (x - x_0)^2 | f(x) |^2 \, \d{x} \]
furthermore,
\[ || \psi ||_2^2 = \int_\R | - f'(x) - y_0 f(x) |^2 \, \d{x} \]
By the Plancherel formula,
\[ \int_\R |g(x)|^2 \, \d{x} = \int_\R |\hat{f}(y)|^2 \, \d{y} \]
Therefore, using that $\F(f')(y) = iy \F(f)(y)$ we see that,
\[ || \psi ||_2^2 = \int_\R | - i f'(x) - y_0 f(x) |^2 \, \d{x} = \int_\R |-i (i y) \hat{f} - y_0 \hat{f} |^2 \, \d{y} = \int_\R (y - y_0)^2 | \hat{f}(y) |^2 \, \d{y} \]
Therefore, we find that,
\[ \left( \int_\R (x - x_0)^2 |f(x)|^2 \, \d{x} \right) \cdot \left( \int_\R (y - y_0)^2 |\hat{f}(y)|^2 \, \d{y} \right) \ge \frac{1}{4} \]
However, we were asked to prove this for any $x_0, y_0$. However, because $x_0$ and $y_0$ are the averages of the functions $f$ and $\hat{f}$ we see that replacing them with any other value only increases the expectation of $(x - x_0)^2$ and $(y - y_0)^2$.

\subsubsection{b}

The equality in Cauchy-Schwartz holds exactly when $\psi$ and $\phi$ are linearly dependent i.e. there is some $\lambda \in \C$ such that $\psi = i \lambda \phi$ (since neither can be zero else the inequality of (a) is not possible). However, we also threw out a term,
\[ \mathrm{Re}(\inner{\psi}{\phi}) = \inner{x}{\{ A, B \} y} - \inner{x}{A x} \inner{x}{B x}  \]
so we require that $\inner{\psi}{\phi}$ be purely imaginary i.e. $\lambda$ is real. Then,
\[ - i \deriv{f}{x} - y_0 f = i \lambda (x - x_0) f \]
Therefore,
\[ \deriv{f}{x} = i y_0 f - \lambda (x - x_0) f \]
which implies that,
\[ f(x) = C \exp{\left( i y_0 x - \tfrac{\lambda}{2}(x - x_0)^2 \right)} \]
Furthermore, $|| f ||_2 = 1$ which implies that $\lambda > 0$ proving the claim up to a normalization condition. 

\section{Fall 2010 Part I}

\subsection{1}

\begin{exercise}
Let $X, Y$ be Banach spaces and $Y \subset X$ with continuous inclusion map $\iota : Y \to X$. Suppose that $T_n \in \L(X)$ and for each $x \in X$ and $n \in \N$ we have $T_n x \in Y$ and there exists $C_x$ such that $|| T_n x ||_Y \le C_x$. Show that for all $n$ we have $T_n \in \L(X, Y)$ and there is some $C$ such that $|| T_n ||_{\L(X, Y)} \le C$
\end{exercise}

Because $\iota : Y \to X$ is continuous then there is $C > 0$ such that for all $y \in Y$,
\[ || y ||_X \le C_\iota || y ||_Y \]
Therefore,
\[ || T_n x ||_X \le C_\iota || T_n x ||_Y \le C_\iota C_x \]
Therefore, $\{ T_n \} \in \L(X, X)$ is pointwise bounded so by Banach-Steinhaus it is uniformly bounded i.e. there is some $C > 0$ such that $|| T_n ||_{\L(X)} \le C$ for all $n$. Furthermore, consider the graph of each operator $T : X \to Y$. Given a sequence $x_n \to x$ such that $T x_n \to y$. Because $\iota T : X \to X$ is continuous we know that $\iota T x_n \to \iota T x$ in $X$. Because $\iota$ is continuous we also know that $\iota T x_n \to \iota y$ and thus $\iota T x = \iota y$ so $T x = y$ because $\iota$ is injective. Therefore $T : X \to Y$ has a closed graph and is thus continuous. Thus each $T_n \in \L(X, Y)$. Furthermore, 
\[ || T_n x || < C_x \]
so $\{ T_n \} \subset \L(X, Y)$ is pointwise bounded and thus by Banach-Steinhaus is uniformly bounded so there is some $C > 0$ such that,
\[ || T_n ||_{\L(X, Y)} < C \]

\subsection{2}

\begin{exercise}
Consider the spaces $L^p([0,1])$ for $p \in [1, \infty)$. For which $p$ is the unit ball,
\[ B = \{ f \in L^p([0,1]) \mid || f ||_{L^p} \le 1 \} \]
weakly sequentially compact?
\end{exercise}

We use the following facts:

\begin{enumerate}
\item if $\Omega$ is a separable measure space then $L^p(\Omega)$ is separable for $p \in [1, \infty)$ 
\item for any measure space $\Omega$ the Banach space $L^p(\Omega)$ is reflexive iff $p \in (1, \infty)$
\item if $X$ is a reflexive Banach space then the unit ball $B$ is compact in the weak topology by Banach-Alaoglu
\item if $X$ is a Banach space with $X^*$ separable (in the operator norm topology) then the unit ball $B$ is metrizable.
\end{enumerate}

Therefore, for $p \in (1, \infty)$ the space $L^p([0,1])$ is separable reflexive and thus has a separable dual $L^q([0,1])$. Therefore, the unit ball $B$ is compact in the weak topology by Banach-Alaoglu and is metrizable which implies that it is sequentially compact. 
\bigskip\\
Consider the sequence,
\[ f_n = n \chi_{[0, \frac{1}{n}]}  \]
and suppose that $\{ f_{n_j} \}$ is a weakly convergent subsequence. Let $f_{n_j} \to f$ weakly for $f \in L^1([0,1)$. Then for any interval $I \subset [0,1]$ if $0 \notin I$ then by weak convergence,
\[ \int_I f \, \d{x} = \lim_{j \to \infty} \int_I f_{n_j} \, \d{x}  = 0 \]
and therefore we must have $f \eqae 0$. Therefore,
\[ \int_0^1 f \, \d{x} = \lim_{j \to \infty} \int_0^1 f_{n_j} \, \d{x} = 1 \]
giving a contradiciton. Therefore, for $p = 1$ the unit ball is not sequentially compact.

\subsection{3}

\subsubsection{a}

Let $f$ be a measureable real-valued function on a finite measure space $(X, \F, \mu)$. Define,
\[ m_n(f) = \mu( \{ x \in X \mid 2^n \le |f(x)| < 2^{n+1} \} ) \]
By definition,
\[ || f ||_{L^p}^p = \int_X |f(x)|^p \, \d{\mu} = \int_0^\infty \mu( \{ x \in X \mid |f(x)|^p > t \} ) \, \d{t} \]
Let $t = u^p$ so then,
\begin{align*}
|| f ||_{L^p}^p & = p \int_0^\infty \mu( \{ x \in X \mid |f(x)| > u \} ) \, u^{p-1} \d{u}
\\
& = p \sum_{n = -\infty}^\infty \int_{2^n}^{2^{n+1}} \mu( \{ x \in X \mid |f(x)| > u \} ) \, u^{p-1} \d{u}
\end{align*}
Therefore,
\[ \sum_{n = -\infty}^\infty \mu( \{ x \in X \mid |f(x)| \ge 2^{n+1} \} ) p \int_{2^n}^{2^{n+1}} u^{p-1} \d{u} \le || f ||_{L^p}^p \le \sum_{n = -\infty}^\infty \mu( \{ x \in X \mid |f(x)| \ge 2^{n} \} ) p \int_{2^n}^{2^{n+1}} u^{p-1} \d{u} \]
computing the integrals gives,
\[ \sum_{n = -\infty}^\infty 2^{np} (2^p - 1) \mu( \{ x \in X \mid |f(x)| \ge 2^{n+1} \} ) \le || f ||_{L^p}^p \le \sum_{n = -\infty}^\infty 2^{np} (2^p - 1) \mu( \{ x \in X \mid |f(x)| \ge 2^{n} \} ) \]
Furthermore,
\[ \mu(\{ x \in X \mid |f(x)| \ge 2^n\}) = \sum_{k = n}^\infty \mu(\{ x \in X \mid 2^{k+1} > |f(x)| \ge 2^k \}) = \sum_{k = n}^\infty m_k(f) \]
so therefore we find,
\[ \sum_{n = -\infty}^\infty \sum_{k = n}^\infty 2^{np} (2^p - 1) m_{k+1}(f) \le || f ||_{L^p}^p \le \sum_{n = -\infty}^\infty \sum_{k = n}^\infty 2^{np} (2^p - 1) m_k(f) \]
Furthermore, by Fubini or something,
\[ \sum_{n = -\infty}^\infty \sum_{k = n}^\infty a_{n,k} = \sum_{k = -\infty}^\infty \sum_{n = -\infty}^k a_{n,k} \]
and furthermore,
\[ \sum_{n = -\infty}^k 2^{np} (2^p - 1)  = (2^p - 1) \left[ \frac{1}{1 - 2^{-p}} + \frac{2^{(k+1)p} - 2^p}{2^p - 1} \right] = (2^p - 1) \frac{2^{(k+1)p}}{2^p - 1} = 2^{(k+1)p} \]
Therefore we find that,
\[ \sum_{k = -\infty}^\infty 2^{(k+1)p} m_{k+1}(f) \le || f ||_p^p \le \sum_{k = -\infty}^\infty 2^{(k+1)p} m_k(f) \]

\subsubsection{b}

\begin{exercise}
Let $(X, \F, \mu)$ be a $\sigma$-finite measure space, $K$ a measurable function on $X \times Y$ such that,
\[ \int_X |K(x, y)| \, \d{y} \le C \quad \text{ and } \quad \int_X |K(x, y)| \, \d{x} \le C \]
almost everywhere. Show that the integral operator $A : L^2(X) \to L^2(X)$ defined by,
\[ (A f)(x) = \int_X K(x,y) f(y) \, \d{y} \]
is well-defined and bounded, and its norm is bounded by $C$.
\end{exercise}

To show that $A f \in L^2(X)$ and to bound $|| A ||$ it suffices to compute,
\begin{align*}
|| A f ||_{L^2}^2 = \int_X | (A f)(x) |^2 \, \d{x} & = \int_X \left| \int_X K(x, y) f(y) \, \d{y} \right| \left| \int_X K(x, y') f(y') \, \d{y'} \right| \, \d{x}
\\
& \le \int_X \left( \int_X | K(x, y) f(y) | \, \d{y} \right) \left( \int_X | K(x, y') f(y')| \, \d{y'} \right) \d{x}
\\
& = \int_X \left( \int_X \left( \int_X | K(x, y) f(y) K(x, y') f(y')| \, \d{y} \right) \d{y'} \right) \d{x}
\end{align*}
Using Fubini repeatedly,
\begin{align*}
|| A f ||_{L^2}^2 & \le \int_{X \times X \times X} | K(x, y) K(x, y') f(y) f(y')| \, \d{\mu} = \int_{X \times X} \left( \int_X |K(x, y) K(x, y') \, \d{x} \right) |f(y) f(y')| \, \d{\mu} 
\end{align*}
Now define,
\[ P(y,y') = \int_X |K(x, y) K(x, y')| \, \d{x} \]
Notice that, again by Fubini,
\begin{align*}
\int_X P(y, y') \, \d{y} & = \int_{X} \left( \int_X |K(x, y) K(x, y')| \, \d{x} \right) \, \d{y} 
\\
& = \int_X \left( \int_X |K(x, y)|  \, \d{y} \right) |K(x, y')| \, \d{x}
\\
& \le C \int_X | K(x, y') | \, \d{x}  \le C^2
\end{align*}
almost everywhere in $y'$. Similarly,
\[ \int_X P(y, y') \, \d{y'} \le C^2 \]
almost everywhere in $y$. Now we see that,
\[ || A f ||_{L^2}^2 \le \int_{X \times X} P(y, y') |f(y) f(y')| \, \d{\mu} \]
Let $F(y, y') = |f(y)| \sqrt{G(y, y')}$ and $G(y, y') = |f(y')| \sqrt{G(y, y')}$ so thus,
\[ || A f ||_{L^2}^2 \le \int_{X \times X} F(y, y') G(y, y') \, \d{\mu} = || F G ||_{L^1} \le || F ||_{L^2} \cdot || G ||_{L^2} \]
by H\"{o}lder. Furthermore, by Fubini,
\begin{align*}
|| F ||_{L^2}^2 & = \int_{X \times X} |F(y, y')|^2 \, \d{\mu} = \int_{X \times X} P(y, y') |f(y)|^2 \, \d{\mu} 
\\
& = \int_X \left( \int_X P(y, y') \, \d{y'} \right) |f(y)|^2 \, \d{y} \le \int_X C^2 |f(y)|^2 \, \d{y} = C^2 || f ||_{L^2}^2 
\end{align*}
Likewise,
\begin{align*}
|| G ||_{L^2}^2 & = \int_{X \times X} |G(y, y')|^2 \, \d{\mu} = \int_{X \times X} P(y, y') |f(y')|^2 \, \d{\mu} 
\\
& = \int_X \left( \int_X P(y, y') \, \d{y} \right) |f(y')|^2 \, \d{y'} \le \int_X C^2 |f(y;)|^2 \, \d{y'} = C^2 || f ||_{L^2}^2 
\end{align*}
Returning to our previous inequality,
\[ || A f ||_{L^2}^2 \le || F ||_{L^2} \cdot || G ||_{L^2} \le C^2 || f ||_{L^2}^2 \]
and therefore,
\[ || A f ||_{L^2} \le C || f ||_{L^2} \]
proving that $A f \in L^2(X)$ and that $A$ is bounded with $|| A || \le C$.

\subsection{4}

Let $X$ be a Banach space and $X^*$ the continuous dual space. Let $\tau = \sigma(X, X^*)$ denote the weak topology on $X$ with respect to $X^*$.

\subsubsection{a}

Suppose that $(X, \tau)$ is first-countable. To show the property, it suffices to find a countable list $\{ A_n \}_I$ of closed (in the weak and thus also the norm topology) linear subspaces of $X$ with finite codimension which is cofinal with respect to the kernels $\ker{\ell}$ for $\ell \in X^*$. In that case, for each $A_n$ we choose a basis of linear functionals $X / A_n \to \C$ which is finite because $\dim{X / A_n} < \infty$ by assumption. The union of these gives a countable set $\{ f_i \}$ of linear functionals. Then if $\ell : X \to \C$ is any element of $X^*$ we see that $\ker{\ell}$ is a closed codimension $1$ linear subspace and thus there is some $A_n \subset \ker{\ell}$ so $\ell$ desends to $\tilde{\ell} : X / A_n \to \C$ and therefore $\ell$ is a finite sum of $\{ f_i \}$ since this list includes a basis of $(X / A_n)^*$.
\bigskip\\
Now we prove that such a list $\{ A_n \}$ exists. Indeed, we are given a countable neighborhood basis $\{U_i \}$ of the origin in the weak topology. Thus, each $U_k$ contains some a finite intersection of open sets of the form $\ell^{-1}(B_\epsilon(0))$ for some $\epsilon > 0$ and $\ell \in X^*$. Furthermore, since,
\[ \bigcap_{i = 1}^n \ell_i^{-1}(B_{\epsilon_i}(0)) \subset U_k \]
we know that,
\[ A_k = \bigcap_{i = 1}^n \ker{\ell_i} \subset U_k \]
which has codimension at most $n$ since each $\ker{\ell_i}$ has codimension $1$. Now I claim that $\{ A_k \}$ is such a family. Countability is obvious as is the fact that each $A_k$ is a closed linear subspace of finite codimension. Furthermore, for any $\ell \in X^*$ we know that $\ell^{-1}(B_\epsilon(0))$ is a neighborhood of the origin so it contains some $U_k$ and thus also $A_k$. It suffices to show that $A_k \subset \ker{\ell}$. Indeed, for any $x \in A_k$ if $\ell(x) \neq 0$ then because $A_k$ is a linear space we can scale $x$ such that $\ell(x) > \epsilon$ but we know that $A_k \subset U_k \subset \ell^{-1}(B_\epsilon(0))$ giving a contradiction. Therefore, $A_k \subset \ker{\ell}$ so we conclude.

\subsubsection{b}

Now suppose that $(X, \tau)$ is metrizable. By considering balls of radius $\frac{1}{n}$ around the origin we get a countable neighborhood basis so $(X, \tau)$ is first countable. Thus we now get a list $\{ f_i \}$ of $f_i \in X^*$ such that,
\[ X^* = \vspan{f_i} \]
Let $F_n = \vspan{f_1, \dots, f_n}$. Then,
\[ X^* = \bigcup_{n = 1}^\infty F_n \]
However, each $X_n$ is closed and either $X_n = X$ or $X_n$ has empty interior because a linear space containing any open set contains an open ball around the origin and thus contains $X^*$ by scaling. However, the Baire category theorem implies that $X^*$ is a Baire space so the union of closed sets with empty interior has empty interior. Therefore, since the union covers $X^*$ we must have $X^* = X_n$ for some $n$ meaning that $X^*$ is finite dimensional contradicting our assumption that $\dim{X} = \infty$.

\subsection{5}

\begin{exercise}
Let $A : \ell^2(\Z) \to \ell^2(\Z)$ be defined by,
\[ (A x)_k = x_{k - 1} - 2 x_k + x_{k + 1} \]
\begin{enumerate}
\item Show that $A$ is a bounded symmetric operator.
\item Let $T : \ell^2(\Z) \to L^2([-\pi, \pi])$ be defined by,
\[ (T x)(t) = \frac{1}{\sqrt{2 \pi}} \sum_{k \in \Z} x_k e^{i k t} \]
Show that $T A T^{-1} : L^2([-\pi, \pi]) \to L^2([-\pi, \pi])$ is the muliplication by $\mu(t)$ operator where $\mu \in L^2([-\pi, \pi])$ is some function
\item determine the spectrum of $A$
\item Find the eigenvalues of $A$.
\end{enumerate}
\end{exercise}

\subsubsection{a}

Consider,
\[ || A x || = \sum_{k \in \Z} |x_{k - 1} - 2 x_k + x_{k + 1}|^2 \le \sum_{k \in \Z} \left( | x_{k - 1} |^2 + 4 | x_k |^2 + | x_{k + 1}|^2 \right) = 6 \sum_{k \in \Z} | x_k |^2 = 6 || x || \]
and therefore $A$ is bounded with $|| A || \le 6$. Furthermore, consider,
\begin{align*}
\inner{x}{A y} & = \sum_{k \in \Z} x_k (y_{k - 1} - 2 y_k + y_{k + 1}) = \sum_{k \in \Z} x_k y_{k - 1} - 2 \sum_{k \in \Z} x_k y_k + \sum_{k \in \Z} x_k y_{k + 1} 
\\
& = \sum_{k \in \Z} x_{k+1} y_k - 2 \sum_{k \in \Z} x_k y_k + \sum_{k \in \Z} x_{k - 1} y_k
\\
& = \sum_{k \in \Z} (x_{k - 1} - 2 x_{k} + x_{k+1}) y_k 
\\
& = \inner{A x}{y}
\end{align*}
Therefore $A$ is symmetric (even self-adjoint).

\subsubsection{b}

First, notice that $T$ transforms the shift operator $(S x)_k = x_{k+1}$ into the multiplication by $e^{i x}$ operator. Indeed,
\[ T (S x) = \sum_{k \in \Z} x_k e^{ikx} = \sum_{k \in \Z} x_{k+1} e^{i (k+1) x} = e^{ix} \sum_{k \in \Z} (S x)_{k} e^{ik x} \]
Therefore, $T S T^{-1} = M_{e^{ix}}$. Now, notice that $A = S^{-1} - 2 I + S$ and therefore,
\[ T A T^{-1} = M_{e^{-i x}} - 2 I + M_{e^{i x}} = M_{\mu(x)} \]
where,
\[ \mu(x) = e^{-ix} + e^{i x} - 2 = 2(\cos{x} - 1) \]

\subsubsection{c}

Because $T$ is an isomorphism, $\sigma(A) = \sigma(T A T^{-1}) = \sigma(M_\mu)$ so we consider the spectrum of $M_\mu$. Notice that,
\[ (M_\mu - \lambda I) = M_{\mu - \lambda} \]
and therefore, we need to examine the multiplication by $2(\cos{x} - 1) - \lambda$ operator. Clearly, this operator is injective on $L^2([-\pi, \pi])$ because the function $2( \cos{x} - 1) - \lambda$ is almost everywhere invertible.
\bigskip\\
Therefore, we just need to check when this operator is surjective. If, $(M_\mu - \lambda I) u = f$ then,
\[ u = \frac{f}{2(\cos{x} - 1) - \lambda} \in L^2([-\pi, \pi]) \]
However, if $f = 1$ this cannot happen for $\lambda \in [-4,0]$ because then $[2( \cos{x} - 1) - \lambda]^{-1}$ is not in $L^2([-\pi, \pi])$ since the integral blows up near zeros. However, for $\lambda \notin [-4, 0]$ the function \[ [2 (\cos{x} - 1) - \lambda] \]
 has no zeros and therefore is $L^\infty$ because it is continuous. Since we are on a finite measure space, multiplication by an $L^\infty$ function preserves being in $L^p$ for any $p$. Thus, $\sigma(A) = [-4, 0]$.

\subsubsection{d}

We have seen that $(M_\mu - \lambda I)$ is always injective and therefore $M_\mu$ and thus $A$ has no eigenvalues.


\section{Fall 2010 Part II}

\subsection{4}

Let $H$ be a Hilbert space and $T \in \L(H)$ a bounded operator. Let $T^*$ be the adjoint. 

\subsubsection{a}

Clearly $\ker{T} \subset H$ is a closed subspace. Therefore, by the orthogonal decomposition theorem, $H = \ker{T} \oplus (\ker{T})^\perp$. Now, if $T^* x \in \im{T^*}$ then for any $y \in \ker{T}$ we have $\inner{T^* x}{y} = \inner{x}{T y} = 0$ and thus $T^* x \in (\ker{T})^\perp$. Since $(\ker{T})^\perp$ is closed because,
\[ (\ker{T})^\perp = \bigcap_{x \in \ker{T}} \ker{\inner{x}{-}} \]
is closed since $\inner{x}{-}$ is continuous. Therefore, 
\[ \overline{\im{T^*}} \subset (\ker{T})^\perp \]
To show the other inclusion, I will use the fact that for closed subspaces $V \subset H$ we have $(V^\perp)^\perp = V$. Now, if $x \in (\overline{\im{T^*}})^\perp$ then we know that for any $T^* y$ we have $\inner{T x}{y} = \inner{x}{T^*y} = 0$. In particular, take $y = T x$ then,
\[ || T x ||^2 = \inner{T x}{T x} = \inner{x}{T^* T x} = 0 \]
because $T^* T x \im \im{T^*}$. Therefore, $T x = 0$ so $x \in \ker{T}$. Thus,
\[ (\ker{T})^\perp \subset \overline{\im{T^*}} \]

\subsubsection{b}

For any sequence $\{ x_n \}$ such that $T x_n \to y$ then we know that $\{ T x_n \}$ is a Cauchy sequence. However,
\[ || x_n - x_m || \le C || T (x_n - x_m) || = C || T x_n - T x_m || \]
so $\{ x_n \}$ is Cauchy. Thus, $x_n \to x$ because $H$ is complete. Therefore, because $T$ is continuous, $T x_n \to T x$ and thus $T x = y$ so $y \in \im{T}$. Therefore, $\im{T}$ is closed.

\subsubsection{c}

Let $T T^* = T^* T = I$ and let $| \lambda | \neq 1$. Then consider $T - \lambda I \in \L(H)$. First, $T - \lambda I$ is injective. Otherwise, $T v = \lambda v$ and then,
\[ \inner{T v}{T v} = \inner{v}{T^* T v} = \inner{v}{v} \]
but $\inner{T v}{T v} = |\lambda|^2 \inner{v}{v}$ and $|\lambda| \neq 1$ so $\inner{v}{v} = 0$ and thus $v = 0$. 
\bigskip\\
Let $T_\lambda = T - \lambda I$. The same argument shows that $T_\lambda^* = T^* - \bar{\lambda} I$ is also injective. We need to show that $T_\lambda$ is surjective. Since $\ker{T^*_\lambda} \oplus \overline{\im{T_\lambda}} = H$ and $\ker{T^*_\lambda} = 0$ we see that $T_\lambda$ has dense image. Furthermore, 
\[ || (T - \lambda I) x ||  \ge | || T x || - || \lambda x || | \]
Also, $|| T x || = \inner{T x}{T x} = \inner{x}{T^* T x} = \inner{x}{x} = || x ||$ so we have,
\[ || (T - \lambda I) x || \ge | 1 - |\lambda| | \cdot || x || \]
and therefore $T_\lambda$ has closed image so $T_\lambda$ is surjective. By the bounded inverse theorem, $T_\lambda$ is invertible. Furthermore, 
\[ || (T - \lambda I)^{-1} || = \sup_{|| x || = 1} || (T - \lambda I)^{-1} x || \le | 1 - | \lambda | |^{-1} || (T - \lambda I) (T - \lambda I)^{-1} x || = | 1 - | \lambda | |^{-1} \cdot || x || \le | 1 - | \lambda | |^{-1} \]

\section{Spring 2010 Part I}

\subsection{1}

Let $X$ be a Banach space.

\subsubsection{a}

Suppose that the weak topology and weak-$*$ topologies on $X^*$ agree. It suffices to show that the canonical map $\ev : X \to X^{**}$ is surjective. For any $\psi \in X^{**}$ by definition, $\psi^{-1}(B_\epsilon(0)) \subset X^*$ is open in the weak and thus also the weak-$*$ topology. Therefore, it contains the intersection of $\ker{\ev_{x_i}}$ for finitely many $x_i \in X$ so we see that,
\[ \bigcap_{i = 1}^n \ker{\ev_{x_i}} \subset \ker{\psi} \]
Therefore by a lemma in Rudin there exist $\alpha_1, \dots, \alpha_n \in \C$ such that,
\[ \psi = \alpha_1 \ev_{x_1} + \cdots + \alpha_n \ev_{x_n} = \ev_{\alpha_1 x_1 + \cdots + \alpha_n x_n} \]
so $\psi$ is in the image of $X \to X^{**}$. 

\subsubsection{b}

Let $A$ be a closed (in the norm topology) linear subspace of $X$. Then I claim that,
\[ A = \bigcap_{\ell \in X^* \text{ s.t. } \ell|_A = 0} \ker{\ell} \]
Clearly $A \subset \ker{\ell}$ for each $\ell \in X^*$ such that $\ell|_A = 0$. To show the opposite inclusion suppose that $x \in X$ is in the intersection but $x \notin A$. Since $A$ is closed, there exists a linear functional $\ell \in X^*$ with $\ell(x) \neq 0$ and $\ell|_A = 0$ using Hahn-Banach contradicting the fact that $x \in \ker{\ell}$ by definition. Thus, $x \in A$. Therefore, $A$ is closed in the weak topology.

\subsection{1}

Suppose that $X$ is a Banach space.

\subsubsection{a}

\begin{exercise}
Suppose that the weak and weak-$*$ topologies on $X^*$ are the same. Show that $X$ is reflexive.
\end{exercise}

For any $\psi \in X^{**}$ we know that $\psi^{-1}(B_\epsilon(0))$ is open in the weak topology on $X^*$ and thus in the weak-$*$ topology. Therefore, there exist $x_1, \dots, x_n \in X$ such that,
\[ \ker{\psi} \subset \psi^{-1}(B_\epsilon(0)) \subset \bigcap_{i = 1}^n \ev_{x_i}^{-1}(B_\epsilon(0)) \]
Furthermore, if $\ell \in \ker{\psi}$ but $\ev_{x_i}(\ell) \neq 0$ then $\lambda \ell \in \ker{\psi}$ meaning that $\lambda \ev_{x_i}(\ell)$ can be arbitrarily large contradicting the inclusion. Therefore, 
\[ \ker{\psi} \subset \bigcap_{n = 1}^\infty \ker{\ev_{x_i}} \]
By a Lemma in Rudin, therefore,
\[ \psi = \alpha_1 \ev_{x_1} + \cdots + \alpha_n \ev_{x_n} = \ev_{x} \]
where $x = \alpha_1 x_1 + \cdots + \alpha_n x_n$
so the canonical map $X \to X^{**}$ is surjective proving that $X$ is reflexive.

\subsubsection{b}

\begin{exercise}
Show that every closed subspace of $X$ (in the norm topology) is weakly closed.
\end{exercise}

Let $V \subset X$ be closed in the norm topology. I claim that,
\[ V = \bigcap_{\substack{ \ell \in X^* \\
\ell|_V  = 0 }} \ker{\ell} \]
Clearly $V \subset \ker{\ell}$ for each $\ell \in X^*$ with $\ell|_V = 0$. Conversely, by Hahn-Banach for any $x \notin V$ there is some $\ell \in X^*$ with $\ell|_V = 0$ such that $x \notin \ker{\ell}$ because $V$ is closed in the norm topology. Therefore, $V$ is the intersection of weakly closed sets and thus is weakly closed.

\subsection{2 (WHY NOT p = 1?)}

\begin{exercise}
Suppose that $K$ is a bounded measurable function on $\{ (x,y) : 0 \le y \le x \le 1 \} \subset [0,1]^2$. Let $T$ be the Volterra operator,
\[ (T f)(x) = \int_0^x K(x, y) f(y) \, \d{y} \]
for $f \in L^p([0,1])$ and $1 < p < \infty$. Show that $T$ is well-defined, and $T : L^p([0,1]) \to L^p([0,1])$ is bounded. Show moreover that $\sigma(T) = \{ 0 \}$.
\end{exercise}

Because $[0,1]$ is a finite measure space $L^p([0,1]) \subset L^1([0,1])$ and therefore integrating against the bounded function $K$ (let $K$ be bounded by $M$) is well-defined. We need to show that $T f \in L^p([0, 1])$. Consider,
\[ \int_0^1 |(T f)(x)|^p \, \d{x} \le \int_0^1 \left( \int_0^x |K(x,y) f(y)|^p \, \d{y} \right) \, \d{x} \]
Furthermore we can compute by Fubini's theorem,
\begin{align*}
\int_0^1 \left( \int_0^x |K(x,y) f(y)|^p \, \d{y} \right) \, \d{x} & = \int_0^1 \left( \int_0^1 \chi_{[0, x]}(y) |K(x,y) f(y)|^p \, \d{y} \right) \, \d{x}
\\
& = \int_0^1 \left( \int_0^1 \chi_{(y, 1]}(x) |K(x,y)|^p \, \d{x} \right) |f(y)|^p \, \d{y} 
\\
& \le \int_0^1 (1 - y) M^p \, |f(y)|^p \, \d{y} \le M^p || f ||_p^p < \infty
\end{align*}
proving that $T f \in L^p([0,1])$. In fact, we have shown more, we have shown that,
\[ || T f ||_p \le M || f ||_p \]
proving that $|| T || \le M$ and thus $T$ is bounded. In fact, $| (T f)(x) | \le M || f ||_p$ so $T f$ is a bounded function although it may not be continuous if $f$ is not bounded. Therefore, $T$ is not surjective so $0 \in \sigma(T)$. We need to show there is no other element of its spectrum. 
\bigskip\\
Notice that if $f$ is bounded by $N$ then we have $| (T f)(x) | \le M N x$. Likewise if $|f(x)| \le MN \frac{x^n}{n!}$ then,
\[ | (T f)(x) | \le MN \frac{x^{n+1}}{(n+1)!} \]
Therefore by induction, $| (T^{n+1} f)(x) | \le M || f ||_p \cdot \frac{x^n}{n!}$ meaning that,
\[ || T^{n+1} f ||_p \le \frac{M || f ||_p}{(n p + 1)^{\frac{1}{p}} n!} \]
and therefore,
\[ || T^{n+1} || \le \frac{M}{(n p + 1)^{\frac{1}{p}} n!} \]
Therefore the spectral radius is,
\[ r(T) = \lim_{n \to \infty} || T^n ||^{\frac{1}{n}} \le \lim_{n \to \infty} \frac{M^{\frac{1}{n}}}{(n p + 1)^{\frac{1}{np}}} \cdot \frac{1}{(n!)^{\frac{1}{n}}} = 0 \]
because $(n!)^{\frac{1}{n}} \to \infty$ but $n^{\frac{1}{n}} \to 1$ and $M^{\frac{1}{n}} \to 1$. Therefore we can only have $\sigma(T) = \{ 0 \}$ or else $r(T)$ would be positive.

\subsection{3}

Let $\S(\R^n)$ and $\S'(\R^n)$ denote the Schwartz functions and respectively the tempered distributions on $\R^n$.

\subsubsection{a}

Suppose that $u \in S'(\R^n)$ and $x_j u = 0$ for $j = 1, \dots, n$. Let $\varphi$ be a bump function around zero with $\varphi(0) = 1$. Then let $c = \inner{u}{\varphi}$. Consider $\tilde{u} = u - c \delta_0$. Now for any Schwartz function $f$ we can write,
\[ f = f(0) \varphi + x_1 f_1 + \cdots + x_n f_n \]
for Schwartz functions $f_1, \dots, f_n \in \S(\R^n)$. To prove this, notice that $g = f - f(0)$ is a Schwartz function such that $g(0) = 0$. Then by the Hadamard lemma,
\[ g(x) = x_1 f_1 + \cdots + x_n f_n \]
Now we see that,
\[ \inner{\tilde{u}}{f} = \inner{\tilde{u}}{f(0) \varphi} + \sum_{j = 1}^n \inner{\tilde{u}}{x_j f_j} = \sum_{j = 1}^n \inner{x_j \tilde{u}}{f_j} = 0 \]
because $x_j \tilde{u} = x_j u - x_j c \delta_0 = 0$ by assumption. Therefore, $\tilde{u} = 0$.

\subsubsection{b}

Suppose that $f \in \S'(\R)$ and $\phi_0 \in \S(\R)$ with $\int_\R \psi_0 \, \d{x} \neq 0$ and $a \in \R$. Without loss of generality we may assume that $\int_\R \psi_0 \, \d{x} = 1$.
\bigskip\\
We want $u$ such that $u' = f$ meaning,
\[ \inner{u'}{\varphi} = - \inner{u}{\varphi'} = \inner{f}{\varphi} \]
Therefore, for any $\varphi \in \S(\R)$ with an anti-derivative $\eta \in \S(\R)$ (which is unique because the difference of any two anti-derivatives is a constant but the only constant Schwartz function is zero) and define,
\[ \inner{u}{\varphi} = \inner{u}{\eta'} = - \inner{f}{\eta} \]
Now I claim that a Schwartz function $\varphi$ has a Schwartz anti-derivative if and only if $\int_\R \varphi \, \d{x} = 0$. If $\eta' = \varphi$ then,
\[ \int_\R \varphi \, \d{x} = \lim_{x \to \infty} \eta(x) - \eta(-x) = 0 \]
because $\eta$ is rapidly decreasing. Now if,
\[ \int_\R \varphi \, \d{x} = 0 \]
define,
\[ \eta(x) = \int_{-\infty}^x \varphi \, \d{x} \]
Then since $\eta' = \varphi \in \S'(\R)$ we know that,
\[ \sup_{x \in \R} | x^n \eta^{(k)} | < \infty \]
for $k \ge 1$ so we only need to consider the case $k = 0$. However,
\[ \sup_{x \in \R} | x^{n} \varphi | = M_n < \infty \]
Therefore, 
\[ |\varphi(x)| \le \min \{ M_0, M_n |x^{-n}| \} \]
\begin{align*}
| x^n \eta(x)| & \le |x^n| \int_{-\infty}^{x} |\varphi| \, \d{x} \le |x^n| \int_{-\infty}^{x} \min \{ M_0, M_{n+1} |x^{-(n+1)}| \} \, \d{x} = \frac{M_{n+1}}{n+1}
\end{align*}
for $x < - \left( \frac{M_{n+1}}{M_0} \right)^{\frac{1}{n+1}}$ clearly it is bounded for $|x| \le \left( \frac{M_{n+1}}{M_0} \right)^{\frac{1}{n+1}}$ because it is continuous. Finally, notice that,
\[ \eta(x) = \int_{-\infty}^x \varphi \, \d{x} = \int_{-\infty}^\infty \varphi \, \d{x} - \int_x^{\infty} \varphi \, \d{x} = - \int_x^\infty \varphi \, \d{x} \]
and therefore,
\[ |x^n \eta(x)| \le |x^n| \int_x^\infty | \varphi| \, \d{x} \le |x^n| \int_x^\infty \min \{ M_0, M_{n+1} |x^{-(n+1)}| \} \, \d{x} \le \frac{M_{n+1}}{n+1} = \frac{M_{n+1}}{n+1} \]
for $x > \left( \frac{M_{n+1}}{M_0} \right)^{\frac{1}{n+1}}$. Therefore, $\eta \in \S'(\R)$.
\bigskip\\
Now, for any $\varphi$ compute,
\[ c = \int_\R \varphi \, \d{x} \]
and let $\tilde{\varphi} = \varphi - c \psi_0$ and then,
\[ \int_\R \tilde{\varphi} \, \d{x} = \int_\R \varphi \, \d{x} - c \int \psi_0 \, \d{x} = c - c = 0 \]
Therefore, $\tilde{\varphi}$ has an anti-derivative $\eta$ so we define,
\[ \inner{u}{\varphi} = - \inner{f}{\eta} + c a \]
so that clearly $\inner{u}{\psi_0} = a$. 
\bigskip\\
We need to show that $u$ is continuous (SHOW CONTINUITY!)
\bigskip\\
Finally, suppose that $\tilde{u} \in \S'(\R)$ also satisfies $\tilde{u}' = f$ and $\inner{\tilde{u}}{\psi_0} = a$. Then writing $\varphi = \eta' + c \psi_0$ we see that,
\[ \inner{\tilde{u}}{\varphi} = \inner{\tilde{u}}{\eta'} + c \inner{\tilde{u}}{\psi_0} = - \inner{f}{\eta} + c a \]
and therefore $\tilde{u} = u$.

\subsection{5 (DO THIS!!!)}

\section{Spring 2010 Part II}

\subsection{1}

\subsubsection{a}

By the Lebesgue differentiation theorem, almost everywhere we have,
\[ \lim_{\delta \to 0} \frac{1}{\mu(B_\delta(x))} \left| \int_{B_\delta(x)} f' \, \d{\mu} \right| = f'(x) \]
For any $\epsilon > 0$ there is some $\delta > 0$ such that,
\[ |h| \le \delta \implies \left| \frac{f(x + h) - f(x)}{h} - f'(x) \right| < \epsilon \]
Furthermore, $f$ is increasing so $f(x+h) - f(x)$ for $h > 0$ and thus also $f'(x) \ge 0$. Therefore, for any $h \le \delta$,
\[ f'(x) h \le f(x + h) - f(x) + \epsilon h \]
and likewise,
\[ f'(x) (h + h') \le f(x + h) - f(x - h') + \epsilon (h + h') \] 
Furthermore, there is some $\delta > 0$ such that for any $r \le \delta$,
\[ \int_{B_r(x)} f' \, \d{\mu} \le f'(x) 2r + 2 \epsilon r \]
Take $\delta_x$ such that both hold and $\delta_x < 1$. Now let $E \subset [0,1]$ be the set of points where $f'$ exists that are also Lebesgue points for the above integral. Then we know $\mu(E) = 1$ and obviously $E \subset \bigcup_{x \in E} B_{\delta_x}(x)$ which means that,
\[ \mu \left( [0, 1] \cap \bigcup_{x \in E} B_{\delta_x}(x) \right) = 1 \]
Also, for any $\eta$, there exists an open $U$ with $\mu(U) < \eta$ and $[1, 0] \setminus E \subset U$ and thus $\{ B_{\delta_x}(x) \}_{x \in \E}$ and $U$ forms an open cover of $[0,1]$ so by compactness there is a finite subcover $\{ B_{\delta_i}(x_i) \}$ and $U$. Therefore, by shifting and shrinking we get intervals $I_i = (x_i, x_{i+1})$ with $I_i \subset B_{\delta_j}(x_j)$ for some $j$ and $x_1 < x_2 < \dots < x_{n+1}$ (then it is true that we possibly are not including finitely many midpoints but this does not change the integral). Then, choose $\eta$ such that for $\mu(A) < \eta$ we have,
\[ \left| \int_A f' \, \d{\mu} \right| < \epsilon \]
and therefore,
\begin{align*}
\int_0^1 f' \, \d{\mu} & = \int_U f' \, \d{\mu} + \sum_{i = 1}^n \int_{I_i} f' \, \d{\mu} \le \epsilon + \sum_{i = 1}^n \left[ f'(x_i) (x_{i+1} - x_i) + \epsilon (x_{i + 1} - x_i) \right]
\\
& = \sum_{i = 1}^n \left[ f(x_{i+1}) - f(x_i) +  \epsilon (x_{i + 1} - x_i) + \epsilon (x_{i + 1} - x_i) \right] \le \eta + f(1) - f(0) + 3 \epsilon 
\end{align*}
by telescoping and the fact that $0 < x_1 < x_{n+1} < 1$. Since $\epsilon$ is arbitrary we see that,
\[ \int_0^1 f' \, \d{\mu} \le f(1) - f(0) \]

\subsubsection{b (DO THIS HOW TO SHOW CONTINUOUS ON SCHWARTZ SPACE}

Let $\phi \in \S(\R)$ and $\epsilon > 0$ and $k \in \Z$. Define $u_\epsilon : \S(\R) \to \C$ via,
\[ u_\epsilon(\phi) = \int_\R (x + i \epsilon)^{-k} \phi(x) \, \d{x} \]

\subsection{2}

\begin{exercise}
Let $(X, \T)$ be a compact Hausdorff space and $f_j \in C(X, \R)$ be a sequence separating points of $X$. Show that $(X, \T)$ is metrizable.
\end{exercise}

Because $X$ is compact the image of $f_j$ is bounded. By rescaling and shifting if need be we can assume that $\im{f_j} \subset [0,1]$. Then, consider the distance function,
\[ d(x,y) = \sum_{j = 1}^\infty 2^{-j} |f_j(x) - f_j(y)| \]
Because $|f_j(x) - f_j(y)| \le 1$ we see that $d(x, y)$ always converges absolutely and $d(x,y) \le 1$. Furthermore, if $x \neq y$ there there is some $j$ for which $f_j(x) \neq f_j(y)$ and thus $d(x, y) \neq 0$. To show that $d$ is a metric it suffices to show that $d$ satisfies the triangle inequality. Consider,
\begin{align*}
d(x, z) & = \sum_{j = 1}^\infty 2^{-j} |f_j(x) - f_j(z)| = \sum_{j = 1}^\infty 2^{-j} |f_j(x) - f_j(y) + f_j(y) - f_j(z)| 
\\
& \le \sum_{j = 1}^\infty 2^{-j} |f_j(x) - f_j(y)| + \sum_{j = 1}^\infty 2^{-j} |f_j(y) - f_j(z)| = d(x,y) + d(y,z) 
\end{align*}
Therefore, $d$ is a metric so we just need to show that $d$ induces the correct topology on $X$. By the Weirstrass $M$-test, and the continuity of the functions $f_j$ we see that $d(x,y)$ is continuous in each argument. Therefore, $B_r(x) = f^{-1}(B_\epsilon(0))$ where $f(y) = d(x,y)$ is open. Thus it just suffices to show that these open sets for a base for the topology. Let $U$ be open and $x \in U$. Suppose that for every $\epsilon > 0$ we have $B_\epsilon(x) \not\subset U$. Then there exists a sequence $x_n \in B_{\frac{1}{n}}(x) \setminus U$. By compactness, $\{ x_n \}$ has a limit point $y$. Now I claim that $y = x$. Otherwise, let $r = d(x,y)$ and for $n$ such that $\frac{1}{n} < \frac{1}{2} r$ let $\delta = \min \{ d(y, x_1), \dots, d(y, x_n), \frac{1}{2} r \}$ unless $y = x_n$ for some $n$ in which case throw it out of the list so $\delta > 0$. Then $B_\delta(y)$ is open but I claim that at most one $x_n \in B_\delta(y)$. Indeed, by construction at most one value $x_1, \dots, x_n$ can be in $B_\delta(y)$. For $k > n$ we know $x_k \in B_{\frac{1}{n}}(x)$ and thus $d(x_k, x) < \frac{1}{n} < \frac{1}{2} r$ and $\delta < \frac{1}{2} r$ so $x_k \notin B_\delta(y)$ because then $d(x, y) < d(x, x_k) + d(x_k, y) < r$ which is impossible. Therefore, $x = y$ which implies that $x$ is a limit point of $U^C$ which is closed but $x \in U$ giving a contradiction. Therefore for some $\epsilon > 0$ we have $B_\epsilon(x) \subset U$.
\bigskip\\
Maybe a better way to do this is as follows. Consider $\iota : (X, \T) \to (X, d)$ which is continuous because $d$ is continuous and therefore opens of the metric topology are open in $\T$. Furthermore it is injective. Now because $(X, \T)$ is compact and $(X, d)$ is Hausdorff we see that $\iota$ is closed and therefore is a homeomorphism onto its image. Furthermore, subspaces of metric spaces are metric spaces proving the claim.

\section{Fall 2011 Part I}

\subsection{1}

\subsubsection{a}

\begin{exercise}
Let $(X, \F, \mu)$ be a finite measure space and $f : X \to \R$ a measurable function. Prove that,
\[ || f ||_{\infty} = \lim_{p \to \infty} || f ||_p \]
\end{exercise}
First, notice that,
\[ || f ||_p^p = \int_X |f|^p \, \d{\mu} \le || f ||_{\infty}^p \cdot \mu(X) \]
and therefore,
\[ || f ||_p \le || f ||_\infty \cdot \mu(X)^{\frac{1}{p}} \]
and thus
\[ \lim_{p \to \infty} || f ||_p \le || f ||_{\infty} \]
Furthermore, let $|| f ||_{\infty} = c$ then we know,
\[ || f ||_p^p = \int_0^{c^p} \mu(\{ x \in X \mid | f(x) |^p > t \}) \, \d{t} = \int_0^{c^p} \mu(\{ x \in X \mid |f(x)| > t^{\frac{1}{p}} \}) \, \d{t} \]
Let $u = \frac{t^{\frac{1}{p}}}{c}$ then
\[ \d{u} = \frac{\d{t}}{t^{\frac{p-1}{p}} cp} = \frac{\d{t}}{p c^p u^{p-1}} \]
and therefore,
\[ || f ||^p_p = p c^p \int_0^1 \mu(\{ x \in X \mid |f(x)| > cu \}) \, u^{p - 1} \d{u}  \]
Notice that, because the measure is decreasing,
\[ \int_0^k \mu(\{ x \in X \mid |f(x)| > cu \}) \, u^{p - 1} \d{u} \ge \mu(\{ x \in X \mid |f(x)| > ck \}) \cdot \int_0^k u^{p-1} \d{u} = \mu(\{ x \in X \mid |f(x)| > ck \}) \cdot \frac{k^p}{p} \]
Therefore,
\[ \frac{|| f ||_p}{|| f ||_{\infty}} \ge k \cdot \mu(\{ x \in X \mid |f(x)| > ck \})^{\frac{1}{p}} \]
Because $ck < || f ||_{\infty}$ we know that $\mu(\{ x \in X \mid |f(x)| > ck \}) > 0$. Therefore, as $p \to \infty$ we see that (because this is a fixed positive number) that,
\[ \lim_{p \to \infty} || f ||_p \ge k \cdot || f ||_{\infty} \]
Furthermore, because $k \in (0, 1)$ is arbitrary this proves that,
\[ \lim_{p \to \infty} || f ||_p \ge || f ||_{\infty} \]
proving the result.

\subsubsection{b}

\begin{exercise}
Consider $S^1 \subset \C$ and $\phi \in C(S^1)$ and let,
\[ \Omega_\delta = \{ z \in \C \mid 1 - \delta < | z | < 1 \} \]
Suppose there exists $\delta \in (0,1)$ and $f \in C(\overline{\Omega_\delta})$ such that $f|_{\Omega_\delta}$ is holomorphic and $f |_{S^1} = \phi$. Now show that the Fourier coefficients,
\[ (\F \phi)_n =  \int_{0}^{1} e^{- 2\pi i n \theta} \, \phi(\theta) \, \d{\theta} \]
satisfy for all $k > 0$ there exists $C_k > 0$ such that $| (\F \phi)_n | \le C_k |n|^{-k}$ for all $n \le -1$.
\end{exercise}

It suffices to prove the conclusion for $k \in \Z^{+}$. Consider the loop $\gamma_{s, n}$ defined by $\gamma_{s, n}(t) = s e^{2 \pi i n t}$. Then, the integals,
\[ a_{n, s} = \frac{1}{2 \pi i} \oint_{\gamma_s} z^{-(n+1)} f(z) \, \d{z} = \int_0^{1} s^{-n} e^{-2 \pi i n \theta} f(\gamma_s(\theta)) \, \d{\theta}  \]
for $s \in (1 - \delta, 1)$ do not depend on $s$. Furthermore, because $f$ is continuous on $\overline{\Omega_\delta}$ I claim that, 
\[ (\F \phi)_n = \lim_{s \to 1} a_{n,s} \]
Indeed, consider,
\[ |(\F \phi)_n - a_{n, s}| \le \int_0^1 | s^{-n} f(\gamma_s(\theta)) - \phi(\theta) | \, \d{t} \le \int_0^1 \left[ |1 - s^{-n}| \cdot |f(\gamma_s(\theta)| + |f(\gamma_s(\theta)) - \phi(\theta) | \right] \, \d{t} \]
However, because $\overline{\Omega_\delta}$ is compact we know that $f$ is  uniformly continuous so for each $\epsilon > 0$ there is a $\delta' > 0$ such that when $| z - z'| < \delta'$ that $|f(z) - f(z')| < \frac{\epsilon}{2}$. In particular, 
\[ | f(\gamma_s(\theta)) - \phi(\theta)| = | f(s e^{2 \pi i t}) - f(e^{2 \pi i t}) | < \tfrac{\epsilon}{2} \]
when $| s e^{2 \pi i t} - e^{2 \pi i t} | = (1 - s) < \delta'$. Therefore, choose $\delta''$ such that $(1 - \delta'')^{-n} - 1 < \frac{\epsilon}{2 M}$ where $M = \sup\limits_{z \in \overline{\Omega_\delta}} | f(z) |$ which exists because $f$ is continuous on a compact set. Thus if $|1 - s| < \min{(\delta', \delta'')}$,
\[ |(\F \phi)_n - a_{n, s}| \le \int_0^1 \left[ | 1 - s^{-n}| |f(\gamma_s(\theta)| + |f(\gamma_s(\theta)) - \phi(\theta) | \right] \, \d{t} < \tfrac{\epsilon}{2 M} \cdot M + \tfrac{\epsilon}{2} = \epsilon \]
meaning that,
\[ \lim_{s \to 1} a_{n,s} = (\F \phi)_n \]
Then, by the integral theorem, $a_{n,s}$ is actually constant in $s$. Therefore,
\[ |(\F \phi)_n| = |a_{n, s}| = s^{-n} \left| \int_0^1 e^{-2\pi i n \theta} f(\gamma_{s}(\theta)) \, \d{\theta} \right| \le s^{-n} M \]
Therefore, for $n \le -1$ since $0 < s < 1$ we have that $| (\F \phi)_n | \to 0$ exponentially fast. (WTF IS THIS RIGHT???)
\bigskip\\
where $g_s(\theta) = f(\gamma_s(\theta))$ is smooth (even real analytic) and therefore, for any $k > 0$ we know that $|n|^k \cdot | (\F g_s)_n | \to 0$ as $n \to \infty$. In fact, we can do a bit better. Using integration by parts,
\[ (\F g_s)_n = \int_0^1 e^{-2\pi i n \theta} g_s(\theta) \, \d{\theta} = (2 \pi i n)^{-k} \int_0^1 e^{- 2 \pi i n \theta} g_s^{(k)} \, \d{\theta} \]
Therefore,

\subsection{2}

Let $f : \N \to \C$ be a bounded function and $M_f : \ell^p \to \ell^p$ be the bounded linear operator $M_f(a)$ is the sequence $\{ f(n) a_n \}$.

\subsubsection{a}

I claim that $\sigma(M_f) = \overline{\im{f}}$. Indeed, if $z \in \im{f}$ then $z = f(i)$ for some $i$ and then $e_i = f(i) e_i$ so $z$ is an eigenvalue meaning that $\im{f} \subset \sigma(M_f)$. However, the spectrum is closed so $\overline{\im{f}} \subset \sigma(M_f)$.
\bigskip\\
Conversely, suppose that $z \notin \overline{\im{f}}$. Then there is some $\epsilon > 0$ such that $B_\epsilon(z)$ contains no element of $\im{f}$. Now I claim that $M_f - z$ is bijective and thus invertible by the bounded inverse theorem. If $(M_f - z) a = 0$ then $f(n) a_n - z a_n = 0$ so either $a_n = 0$ or $f(n) = z$ but we assumed that $z \notin \overline{\im{z}}$ so $a_n = 0$. Thus $M_f - z$ is injective. Now, for surjectivity we try to solve $(M_f - z) a = b$ for arbitrary $b \in \ell^p$. We can set,
\[ a_n = \frac{b_n}{f(n) - z} \]
and we need to check that $a \in \ell^p$. Indeed, $|f(n) - z| \ge \epsilon$ so,
\[ \left| \frac{b_n}{f(n) - z} \right|^p \le \frac{|b_n|^p}{\epsilon^p} \]
which is summable because $b \in \ell^p$. Therefore, $(M_f - z)$ is surjective. Therefore, $z \notin \sigma(M_f)$ proving the claim. 

\subsubsection{b}

I claim that $M_f$ is compact iff $\lim_{n \to \infty} f(n) = 0$. Indeed, suppose that $f(n) \to 0$. Then consider the operators $M_k$ which send $a \in \ell^p$ to
\[ (M_k a)_n = \begin{cases}
f(n) a_n & n < k
\\
0 & n \ge k
\end{cases} \]
Then consider,
\[ || M_f - M_k || = \sup_{|| a || = 1} || M_f a - M_k a || \]
However, 
\begin{align*}
| (M_f a - M_k a)_n | & = 
\begin{cases}
f(n) a_n & n \ge k 
\\
0 & n < k
\end{cases} 
\\
& \le \left( \sup_{n \ge k} |f(n)| \right) |a_n| 
\end{align*}
and therefore,
\[ || M_f a - M_k a || \le \left( \sup_{n > k} |f(n)| \right) || a || \le \left( \sup_{n > k} |f(n)| \right) \]
because $|| a || \le 1$. Therefore,
\[ \lim_{k \to \infty} || M_f - M_k || \le \left( \limsup_{n \to \infty} f(n) \right) = 0 \]
Thus $M_k \to M_f$ but each $M_k$ is a compact operator because it has finite image and thus $M_f$ is a compact operator.
\bigskip\\
Conversely, suppose that $M_f$ is compact. In particular, let $B = \{ a \in \ell^p \mid || a || \le 1 \}$ which is bounded by definition. Then $M_f(B)$ is precompact and thus totally bounded so for each $\epsilon > 0$ there exists a finite list $y_1, \dots, y_n$ such that,
\[ M_f(B) \subset \bigcup_{i = 1}^n B_\epsilon(y_i) \]
In particular, $M_f e_k \in M_f(B)$ because $|| e_k || = 1$. Therefore, for all $k \ge 0$,
\[ \min_{i} || (y_i)_k - M_f e_k || < \epsilon \]
However,
\[ || y_i - M_f e_k || = |y_i - f(k)| \]
because $M_f e_k$ only has support at $k$. Furthermore, 
\[ \lim_{k \to \infty} (y_i)_k = 0 \]
because they are summable and therefore, we find that $f(k) < 2 \epsilon$ for sufficiently large $k$ (large enough $k$ such that $|(y_i)_k| < \epsilon$ for all $i$). Since $\epsilon$ was arbitrary we find that,
\[ \lim_{n \to \infty} f(n) = 0 \]

\subsubsection{c}

Yes it is possible for $\sigma(M_f)$ to contain non-eigenvalues. For example, let $p = 1$ and,
\[ f(n) = \frac{1}{n+1} \]
Then we have shown that $0 \in \sigma(M_f)$ because $0 \in \overline{ \{\tfrac{1}{n+1} \mid n \in \N \}}$. However, I claim that $M_f$ is injective so $0$ is not an eigenvalue. Indeed, if $M_f a = 0$ then for each $n \in \N$ we have,
\[ (M_f a) = f(n) a_n = \frac{a_n}{n+1} = 0 \]
and therefore $a_n = 0$ so $a = 0$. 

\subsection{3}

Let $X, Y$ be Hilbert spaces and $T \in \L(X, Y)$.

\subsubsection{a}

Suppose that $T$ is bijective. Then I claim that $T$ is invertible with $T^{-1} \in \L(X, Y)$. Indeed, it suffices to show that $T^{-1}$ is bounded. By the open mapping theorem, $T$ is open and therefore $T^{-1}$ is continuous and thus bounded.

\subsubsection{b}

Consider $M_f : \ell^2 \to \ell^2$ from the previous problem where $f(n) = \frac{1}{n+1}$. We showed that $M_f$ is injective but not surjective. Furthermore, $S$ is uniquely defined on the image of $M_f$. Clearly, there we must have $(S a)_n = (n+1) a_n$. Now I need to show there is no such $S \in \L(X, Y)$. Indeed, $\ev_n : a \mapsto a_n$ is a bounded linear functional and thus continuous. So if $S$ were continuous for any convergent sequence $a_k \to a$ with $a_k \in \im{M_f}$ we would have,
\[ (S a)_n = \ev_n(S \lim_{k \to \infty} a_k) = \lim_{k \to \infty} (S a_k)_n = \lim_{k \to \infty} (n + 1) (a_k)_n = (n + 1) a_n \]
Therefore, $S$ is defined the same way everywhere because $\im{M_f}$ is dense. However, this is impossible because if $a_n = \frac{1}{n+1}$ then $S a$ is not summable.

\subsubsection{c}

Suppose that $T \in \L(X, Y)$ is surjective. Then by the open mapping theorem $T$ is open. Therefore, $T : X / \ker{T} \to Y$ is an isomorphism. Furthermore, there is an orthogonal decomposition $X \cong \ker{T} \oplus X / \ker{T}$ so we can clearly choose a continuous section $S : Y \iso X / \ker{T} \to X$. (CHECK THIS!!)

\subsection{4}

Let $X$ be a separable reflexive Banach space.

\subsubsection{a}

I claim that the weak topology on $X$ coincides with the weak-$*$ topoolgy on $X^{**}$ under the canonical map $X \to X^{**}$. Indeed, both topologies are the coarsest such that $x \mapsto \ell(x)$ for each $\ell \in X^*$ is continuous because the weak-$*$ topology is the coarsest topology such that $\psi \mapsto \psi(\ell)$ is continuous for each $\ell \in X^*$ but $\psi = \ev_x$ for some $x \in X$ because $X$ is reflexive. Since $X \to X^{**}$ is an isometry, by Banach-Alaoglu 
\[ B = \{ x \in X \mid || x || \le 1 \} \cong \{ \psi \in X^{**} \mid || \psi || \le 1 \} \]
is compact in the weak-$*$ topology on $X^{**}$ and thus in the weak topology on $X$.

\subsubsection{b}

I claim that,
\[ S = \{ x \in X \mid || x || = 1 \} \]
is compact if and only if $\dim{X} < \infty$. If $X$ is finite dimensional then clearly $S$ is compact in the norm topology for any Banach space and thus in the weak topology. Conversely, assume that $S$ is compact in the weak topology. Since $X^*$ separates points, $\sigma(X, X^*)$ is Hausdorff. Therefore, $S \subset X$ is closed in the weak topology. However, if $\dim{X} = \infty$ then we can show that the weak closure of $S$ contains $B$ and thus $S$ cannot be weakly closed. 
\bigskip\\
Indeed, for any $x_0 \in B$ and weakly open set $U$ with $x_0 \in U$ we can shrink $U$ such that,
\[ U = x_0  + \bigcap_{i = 1}^n f_i^{-1}(B_\epsilon(0)) \]
for $f_i \in X^*$. Because $\dim{X} = \infty$ the map $X \to \C^n$ defined by $x \mapsto (f_i(x))$ cannot be injective. Therefore, there is some $y$ such that $f_i(y) = 0$ for all $i$ and thus $t y \in U$ for all scalars $t$. Now consider $g(t) = || x_0 + t y ||$. Because $g(0) = || x_0 || \le 1$ and $g(t) \to \infty$ as $t \to \infty$ and $g$ is continuous we see that there is some $t$ for which $g(1) = 1$ and thus $x_0 + t y \in S$. However, $x_0 + t y \in U$ so we see that $x_0$ is a weak closure point of $S$.
\bigskip\\
Thus we see that if $\dim{X} = \infty$ then $S$ is not weakly compact.

\subsubsection{c (HOW TO DO THIS!!!)}

Consider,
\[ B = \{ x \in X \mid || x || \le 1 \} \cong \{ \psi \in X^{**} \mid || \psi || \le 1 \} \]
and suppose that $x_n \in B$ is some sequence. We need to find a weakly convergent subsequence.
\bigskip\\
Theorem: if $X^*$ is separable then $B$ is metrizable because it is compact in the weak topology.


\subsection{5}


\begin{exercise}
Let $\SL{2}{\R}$ act on $\R^2$. Find every Borel measure finite on compact sets invariant under the $\SL{2}{\R}$ action.
\end{exercise}

It is clear that $\delta_0$ and the Lebesgue measure $\mu_{\L}$ both satisfy these conditions. I claim these are the only such measures up to linear combinations.
\bigskip\\
Since Baire measures are $\sigma$-finite (CHECK THIS) we can apply Lebesgue decomposition with respect to $\mu_{\L}$ then $\nu = \nu_s + \nu_c$ where $\nu_c \ll \mu_{\L}$. Then there is some measurable function $f$ such that,
\[ \nu_c(A) = \int_A f \, \d{\mu_{\L}} \]
However, since both sides are $\SL{2}{\R}$ invariant we must have $f$ is $\SL{2}{\R}$ invariant a.e. but $\SL{2}{\R}$ acts transitively except on $0$ meaning that $f$ is a.e. constant so $\mu_c$ is up to a constant equal to $\mu_{\L}$. Next, $\mu_s \perp \mu_{\L}$ so we can decompose $\R^2 = A \cup B$ with $A, B$ disjoint such that $\mu_{\L}$ is zero on all measurable subsets of $A$ and $\nu_s$ is zero on all measurable subsets of $B$. Therefore, $\nu_s$ is suppored on a Lebesgue measure zero subset, However, all $\SL{2}{\R}$-translates of $S \subset A$ have the same $\nu_s$-measure. Either $\nu_s$ is a multiple of $\delta_0$ or there exists a subset $S \subset A$ not containing $0$ such that $\nu_s(S) > 0$. But then for all $p \in \SL{2}{\R}$,
\[ \nu_s(p \cdot S) = \nu_s((p \cdot S) \cap A) + \nu_s((p \cdot S) \cap B) \]
is positive so we must have $\nu_s((p \cdot S) \cap A) > 0$ for all $S$ and $A$. 
(HOW TO FINISH!!)


\section{Fall 2011 Part II}

\subsection{1}

\begin{exercise}
Let $A \subset \R$ be Borel, $T \subset \R$ dense and $\tau_t(A) \setminus A$ has Lebesgue measure zero for each $t \in T$ where $\tau_t : \R \to \R$ is translation $x \mapsto x + t$. Prove that either $A$ or $\R \setminus A$ has Lebesgue measure zero. 
\end{exercise}

The function $f : \R \to L^1(\R)$ defined by $t \mapsto \chi_{\tau_t(A)}$ is continuous. Then,
\[ g(t) = \mu(\tau_t(A) \setminus A) = \int_\R \chi_{\tau_t(A)} \chi_{A^C} \, \d{\mu} = \int_\R f(t) \chi_{A^C} \, \d{\mu} \]
is continuous and is zero on the dense set $T$. Thus $g$ is identically zero. Now by Fubini,
\begin{align*}
\int_\R g(t) \, \d{t} & = \int_{\R \times \R} \chi_{\tau_t(A)} \chi_{A^C} \, \d{\mu} = \int_{\R} \int_{\R} \chi_{\tau_t(A)}(x) \chi_{A^C}(x) \, \d{(x, t)}
\\
&  = \int_\R \chi_{A^C}(x) \left( \int_\R \chi_{\tau_t(A)}(x) \, \d{t} \right) \, \d{x} = \mu(A) \cdot \mu(A^C)
\end{align*}
proving that either $\mu(A) = 0$ or $\mu(A^C) = 0$.

\section{Spring 2011 Part I}

\subsection{1}

\begin{exercise}

Consider $\ell^2$ and let $T : \ell^2 \to \ell^2$ given by $(T a)_n = \frac{1}{n} a_n$. Show that $T$ is compact and find its spectrum and eigenvalues.
\end{exercise}

Consider the operator,
\[ T_k : \ell^2 \to \ell^2 \]
defined by,
\[ (T_k a)_n 
= \begin{cases}
\tfrac{1}{n} a_n & n < k 
\\
0 & n \le n
\end{cases} \]
Then we see that,
\[ || (T - T_k) a || = \sum_{n = k}^{\infty} \frac{|a_n|^2}{n^2} \le \frac{1}{k^2} \sum_{n = k}^\infty |a_n|^2 || a || \cdot \frac{1}{k^2} \]
Therefore,
\[ || T - T_k || \le \frac{1}{k^2} \]
which implies that $T_k \to T$ in the operator norm. Futhermore, $T_k$ has constant rank so $T$ is a limit of compact operators and thus compact.
\bigskip\\
I claim that $\sigma(T) = \overline{\{ \frac{1}{n} \mid n \in \Z^{+} \}}$. Indeed, we see that $T e_n = \frac{1}{n} e_n$ and therefore $\frac{1}{n}$ is an eigenvalue. However, the spectrum is always closed so,
\[ \overline{\{ \tfrac{1}{n} \mid n \in \Z^{+} \}} \subset \sigma(T) \]
Furthermore let $\lambda \notin \overline{\{ \frac{1}{n} \mid n \in \Z^{+} \}}$. Then for any $b \in \ell^2$ we can take $a$ such that,
\[ a_n = \frac{b_n}{\frac{1}{n} - \lambda} \]
such that $(T - \lambda I) a = b$ and this is clearly the unique solution. Thus to show that $(T - \lambda I)$ is invertible it suffices to show that $a \in \ell^2$. Indeed,
\[ \sum_{n = 0}^\infty | a_n |^2 = \sum_{n = 0}^\infty \frac{|b_n|^2}{(\frac{1}{n} - \lambda)^2} < \infty \]
because $(\frac{1}{n} - \lambda)^{-2}$ is bounded because $\lambda$ is not a limit point of the sequence $\frac{1}{n}$ so there is some $\epsilon > 0$ such that no $\frac{1}{n}$ is in $B_\epsilon(\lambda)$ and thus $|\frac{1}{n} - \lambda|^{-2} < \epsilon^{-2}$. Therefore multiplication by this sequence makes a convergent series still converge.
\bigskip\\
I claim that $T$ is injective and thus the set of eigenvalues is $\{ \frac{1}{n} \mid n \in \N \}$. Indeed, if $T a = 0$ then $\frac{1}{n} a_n = 0$ so $a_n = 0$ for all $n$ so $a = 0$. 

\subsection{2}

\subsubsection{a}

\newcommand{\diam}{\mathrm{diam}}

Let $X = [0,1]$ and let $f : [0,1] \to \R$ be a function. Then consider,
\[ C(f) = \{ x \in X \mid f \text{ is continuous at } x \} = \bigcap_{n = 1}^\infty \{ x \in X \mid \exists \delta > 0 : f(B_\delta(x)) \subset B_{\frac{1}{n}}(f(x)) \} \]
Furthermore, the sets,
\[ U_n = \{ x \in X \mid \exists \delta > 0 : \diam{f(B_{\delta}(x))} < \frac{1}{n} \} \]
are open because if $x \in U_n$ then there is some $\delta > 0$ such that $\diam{f(B_\delta(x))} < \frac{1}{n}$ and then $B_{\frac{\delta}{2}}(x) \subset U_n$ because if $|x' - x| < \frac{\delta}{2}$ then for any $|x'' - x'| < \frac{\delta}{2}$ we see that $|x'' - x| < \delta$ so $f(B_{\frac{\delta}{2}}(x')) \subset f(B_\delta(x))$ so it has diameter less that $\frac{1}{n}$ and thus $B_{\frac{\delta}{2}}(x) \subset U_n$ so $U_n$ is open. Therefore, $C(f)$ is $G_\delta$.

\subsubsection{b}

Consider the function,
\[ f(x) = 
\begin{cases}
\frac{1}{q} & x = \frac{p}{q} \text{ in lowest terms}
\\
0 & x \notin \Q 
\end{cases} \]
Then I claim that $C(f) = [0,1] \setminus \Q$. Indeed, if $x \in \frac{p}{q}$ then any neighborhood about $x$ contains some irrational so the minimum spread of $f$ is $\frac{1}{q}$ and thus $C(f) \subset [0, 1] \setminus \Q$. Furthermore, if $x \in [0,1] \setminus \Q$ then for any $\epsilon > 0$ I can choose $\delta > 0$ such that $(x - \delta, x + \delta)$ does not contain any $\frac{p}{q}$ with $p < 2\epsilon^{-1}$. This is possible because there are finitely many such numbers. Therefore, it is clear that if $|x' - x| < \delta$ then $|f(x') - f(x)| \le \frac{2}{p} < \epsilon$ and thus $x \in C(f)$.

\subsubsection{c}

I claim that $\Q$ is not a $G_\delta$ set of $[0, 1]$. Indeed, by the Baire category theorem $\R$ is a Baire space. If $\Q$ were a $G_\delta$ set then because,
\[ \R \setminus \Q = \bigcap_{q \in \Q} \R \setminus \{ q \} \]
is a $G_\delta$ set then $\Q \cap (\R \setminus \Q) = \varnothing$ must be the countable intersection of dense open sets (because $\Q$ is dense). However, in a Baire space, a countable intersection of dense open sets is dense. Thus $\Q$ is not $G_\delta$ so it cannot be the set of continuity points of any function.

\subsection{3 (CHECK THIS!!)}

Let $e_j$ be the image of the $j$-th unit vector of $\R^n$ in $\T^n$ and let $\D'(\T^n)$ be the set of distributions on the torus equipped with the weak-$*$ topology.

\subsubsection{a}

For $y \in \T^n$ define $L_y u$ via $\inner{L_y u}{f} = \inner{u}{L_{-y} f}$. This is continuous in the weak-$*$ topology because for any $f \in \D(\T^n)$ the map $u \mapsto \inner{L_y u}{f} = \inner{u}{L_{-y} f}$ is contunuous because $L_{-y} : \D(\T^n) \to \D(\T^n)$ is continuous. I claim this extends $(L_y f)(x) = f(x + y)$ on $C(\T^n) \subset D'(\T^n)$. Indeed, for any $u \in C(\T^n)$ and the map $\iota : C(\T^n) \to \D'(\T^n)$ consider,
\[ \inner{L_y \iota u}{f} = \inner{u}{L_{-y} f} = \int_{\T^n} u(x) f(x - y) \, \d{y} = \int_{\T^n} u(x + y) f(x) \, \d{y} = \inner{\iota (L_y u) }{f} \]
Now I claim that this extension is unique. Indeed, $C(\T^n) \subset \D'(\T^n)$ is dense in the weak-$*$ topology which is a Hausdorff space so operators defined on $C(\T^n)$ are uniquely determined.

\subsubsection{b}

Let $u \in \D'(\T^n)$. For $h > 0$ consider $u_h = h^{-1}(L_{h e_j} u - u)$. We need to show that $u_h \to \partial_j u$ in the weak-$*$ topology as $h \to 0$. Convergence in the weak-$*$ is equivalent to for each $f \in \D'(\T^n)$ convergence of,
\[ \inner{u_h}{f} \to \inner{\partial_j u}{f} \]
Furthermore,
\[ \lim_{h \to 0} \inner{u_h}{f} = \lim_{h \to 0} \frac{\inner{u}{L_{-h e_j} f} - \inner{u}{f}}{h} \]
and likewise,
\[ \inner{\partial_j u}{f} = - \inner{u}{\partial_j f} \]
Therefore, because $\inner{u}{-}$ is continuous on $\D(\T^n)$ it suffices to show that,
\[ \lim_{h \to 0} \frac{L_{-he_j} f - f}{h} = - \partial_j f \]
in the canonical LF-topology on $\D(\T^n)$. By definition ($f$ is smooth) we have pointwise convergence everywhere. (HOW TO FINISH THIS!!!)

\subsubsection{c}

Let $\phi \in C^\infty(\T^{n+m})$ let $\phi_x(y) = \phi(y, x)$ for $(x, y) \in \T^n \times \T^m$ then $\phi_x \in C^\infty(\T^n)$. Let $y \in \D'(\T^n)$ and $\phi \in C6\infty(\T^{n+m})$ then $f : \T^m \to \C$ defined by $f(x) = u(\phi_x)$ is in $C^\infty(\T^m)$.
\bigskip\\
First suppose that $u \in C(\T^n)$ then,
\[ f(x) = u(\phi_x) = \int_{\T^n} u(y) \phi(y, x) \, \d{y} \]
Then because everything in sight is continuous and $u(y) \phi(y, x)$ is differntiable in $x$ whose derivtive $u(y) \partial_x \phi(y, x)$ is continuous. Thus, 
\[ f'(x) = \int_{\T^n} u(y) \partial_x \phi(y, x) \, \d{y} \]
exists. Since $\phi(y, x)$ is smooth, $\partial_x^n \phi(y, x)$ is continuous in $(x, y)$ and thus we can repeat to find that $f$ is smooth. 
\bigskip\\
Now in general, (HOW TO DO THIS???)


Maybe,
\[ \partial_x f(x) = \lim_{h \to 0} h^{-1} \inner{u}{\phi_{x + h} - \phi_x} \]
However, $\lim_{h \to 0} h^{-1} (\phi_{x + h} - \phi_x) = \partial_x \phi$ in the LF-topology because we have uniform convergence of the derivatives (SHOW THIS!!!). Therefore,
\[ \partial_x f(x) = \inner{u}{\partial_x \phi_x} \]
exists. Replacing $\phi$ by $\partial_x \phi$ we repeat the same argument to show that $f$ is infinitely differentiable.

\subsection{4}

\begin{exercise}
Let $E \subset \R$ and consider the sets,
\[ E + E = \{ x + y \in \R \mid x, y \in E \} \quad \text{and} \quad E - E = \{ x - y \in \R \mid x, y \in E \} \]
Then if $E$ is measurable with $\mu(E) > 0$ prove that $E + E$ and $E - E$ contain an open interval.
\end{exercise}

Notice that we are free to shift and rescale $E$ without changing the conclusion. If $\mu(E) > 0$ then almost every point of $E$ is a Lebesgue point so there must be at least one. By shifting we can assume that $0 \in E$ is a Lebesgue point. Therefore, for any $\epsilon > 0$ there exists a $\delta > 0$ such that $|2\delta - \mu(E \cap B_\delta(0))| \le 2\epsilon \delta$. Then let $E_\delta = E \cap B_\delta(0)$. 
\bigskip\\
Notice that $x \notin E_\delta + E_\delta$ iff $(x - E_\delta) \cap E_\delta = \empty$ in which case $\mu((x - E_\delta) \cup E_\delta) = 2 \mu(E_\delta)$. However,
\[ (x - E_\delta) \cup E_\delta \subset (x - B_\delta(0)) \cup B_\delta \subset 
\begin{cases}
(-\delta, \delta + x) & x \ge 0
\\
(- \delta - |x|, \delta) & x < 0
\end{cases} \]
and thus,
\[ \mu((x - E_\delta) \cup E_\delta) \le 2 \delta + |x| \]
However,
\[ 2 \mu(E_\delta) \ge 4 \delta (1 - \epsilon) \]
Therefore, if $|x| < 2 \delta (1 - 2 \epsilon)$ (which we can make positive by ensuring that $\epsilon < \tfrac{1}{2}$ then,
\[ \mu((x - E_\delta) \cup E_\delta) \le 2 \delta + |x| < 4 \delta (1 - \epsilon) \le 2 \mu(E_\delta) \]
and therefore $E_\delta$ and $(x - E_\delta)$ are not disjoint meaning that $x \in E + E$ so $E + E$ contains the interval $(-2 \delta(1 - 2 \epsilon), 2 \delta(1 - 2 \epsilon))$. A similar argument works for $E - E$.

\subsection{5}

Let $H^s(\T^n) \subset L^2(\T^n)$ be the subset of $f \in L^2(\T^n)$ such that the Sobolev norm,
\[ || f ||_{H^s} = \sum_{k \in \Z^n} (1 + |k|^2)^s |\hat{f}(k)|^2 < \infty \]
is finite. Suppose that $P : H^s(\T^n) \to H^m(\T^n)$ is a continuous linear map satisfying,
\[ || u ||_{H^s} \le C \left( || P u ||_{H^m} + || u ||_{H^r} \right) \]
for some $s > r \ge 0$ and $m \ge 0$.
\bigskip\\
First I claim that the inclusion $\iota : H^s(\T^n) \embed H^r(\T^n)$ is compact. Indeed, consider the finite rank operators,
\[ \iota_N : H^s(\T^n) \to H^r(\T^n) \]
defined by,
\[ (\iota_N f)(x) = \sum_{|k| < N} e^{i k \cdot x} \hat{f}(k) \]
Then consider,
\begin{align*}
|| \iota_N - \iota || & = \sup_{|| f || = 1} || \iota_N f - \iota f || = \sup_{|| f ||_{H^s} = 1} \sum_{|k| \ge N} (1 + |k|^2)^r |\hat{f}(k)|^2 
\\
& = \sup_{|| f ||_{H^s} = 1} \sum_{|k| \ge N} (1 + |k|^2)^{-(s - r)} (1 + |k|^2)^{s} |\hat{f}(k)|^2 
\\
& \le (1 + N^2)^{-(s - r)} \sup_{|| f ||_{H^s} = 1}  \sum_{|k| \ge N} (1 + |k|^2)^s |\hat{f}(k)|^2 
\\
& \le (1 + N^2)^{-(s - r)} \sum_{k \in \Z^n} (1 + |k|^2)^s |\hat{f}(k)|^2 
\\
& = (1 + N^2)^{-(s - r)} \sup_{|| f ||_{H^s} = 1} || f ||_{H^s} \le (1 + N^2)^{-(s - r)}
\end{align*}
Thus, since $s > r$ we see that $|| \iota_N - \iota || \to 0$ as $N \to \infty$ so $\iota$ is compact.
\bigskip\\
Let $X = \ker{P}$ then $X \subset H^s(\T^n)$ is closed and we see that $\iota : (X, || \bullet||_{H^s}) \to (X, || \bullet ||_{H^r})$ is continuous (because $|| u ||_{H^r} \le || u ||_{H^s}$ by definition) and bounded below because,
\[ || u ||_{H^s} \le C || u ||_{H^r} \]
Therefore the image of $\iota$ is closed so it is an isomorphism onto its image. However, $\iota$ is compact which implies that $\id : X \to X$ is compact and thus $\dim{X} < \infty$ because the unit ball of a Banach space is only compact when $X$ is finite dimensional.
\bigskip\\
Because $X \subset H^s(\T^n)$ is finite dimensional (and thus closed) it is complemented so write $X \oplus V = H^s(\T^n)$ for $V \subset H^s(\T^n)$ closed. Then $P|_V : V \to H^m(\T^n)$ is injective and it suffices to show that $P|_V$ has closed image or equivalently that $P|_V$ is bounded below. For any $\epsilon > 0$ we can find $N$ such that $|| \iota_N - \iota || < \epsilon$. Then for $u \in \ker{\iota_N}$ we see that $|| \iota u ||_{H^r} \le \epsilon || u ||_{H^s}$ so,
\[ || u ||_{H^s} \le C || P u ||_{H^r} + C \epsilon || u ||_{H^s} \]
and therefore choosing $\epsilon < C^{-1}$ we see that,
\[ || u ||_{H^s} \le \frac{C}{1 - \epsilon C} || P u ||_{H^r} \]
so if $K = \ker{\iota_N} \cap V$ then $P|_K$ is bounded below and thus has closed image. However, $\iota{\iota_N}$ has finite codimension since $\iota_N$ is of finite rank. Therefore, $K$ is complemented in $V$ by a finite dimensional subspace $L$ so $V = K \oplus L$ and $P(V) = P(K) \oplus P(L)$ because $P|_V$ is injective but $P(L)$ is finite dimensional and $P(K)$ is closed so $P(V)$ is also closed.

\section{Spring 2011 Part II}

\subsection{1 (Seems too easy!)}

\begin{exercise}
A projection operator $P \in \L(X)$ on a Hilbert space $X$ is a bounded operator such that $P^2 = P$. Show that $x - P(x) \in P(X)^\perp$ for all $x \in X$ if and only if $P$ is self-adjoint.
\end{exercise}

If $P$ is self-adjoint then $\inner{x - P(x)}{Py} = \inner{x}{Py} - \inner{x}{P^2 y} = \inner{x}{Py} - \inner{x}{P y} = 0$ so $P$ is orthogonal. Now suppose that $P$ is orthogonal. We need to show that $\inner{x}{P y} = \inner{P x}{y}$ for all $x, y \in H$. By orthogonality, we know this holds when $y \in P(X)$ since,
\[ \inner{x - Px}{Pz} = 0 \iff \inner{P x}{Pz} = \inner{x}{P z} = \inner{x}{P^2 z} \]
Now, $P z = y$ implies that $P z = P^2 z = P y = y$ so $P y = y \iff y \in P(X)$. Therefore $P(X) = \ker{(P - I)}$ is closed.
Therefore since $\ker{P} = \ker{P^*} = P(X)^\perp$ and $P(X)$ is closed we can decompose $X = \ker{P} \oplus P(X)$ and so we can write $x = x_0 + x_1$ and $y = y_0 + y_1$ and compute,
\[ \inner{P x}{y} = \inner{Px}{y_0} + \inner{P x}{y_1} = \inner{x}{P y_1} = \inner{x}{P y} \]
because $y_0 \in \ker{P} = P(X)^\perp$ and $y_1 \in P(X)$.

\subsection{2 (SEEMS TOO EASY??)}

\begin{exercise}
Let $X$ be a $\C$-vectorspace and $\F$ a vector subspace of $\Hom{\C}{\F}{\C}$ and equip $X$ with the $\F$-weak topology. Show that the only continuous linear maps $X \to \C$ are those in $\F$.
\end{exercise}

Let $\varphi : X \to \C$ be a continuous linear functional. Then $\varphi^{-1}(B_\epsilon(0))$ is open in $X$. Therefore, there exist $f_1, \dots, f_n \in \F$ such that,
\[ \bigcap_{i = 1}^n f_i^{-1}(B_\epsilon(0)) \subset \varphi^{-1}(B_\epsilon(0)) \]
Therefore,
\[ \bigcap_{i = 1}^n \ker{f_i} \subset \ker{\varphi} \]
because if $f_i(x) = 0$ but $\varphi(x) \neq 0$ then because the kernels are linear spaces we can make $\varphi(\lambda x)$ arbitrarily large contradicing the containment so $x \in \ker{\varphi}$. However, by a lemma of Rudin this implies that there exist $\alpha_1, \dots, \alpha_n \in \C$ such that,
\[ \varphi = \alpha_1 f_1 + \cdots + \alpha_n f_n \]
and therefore $\varphi \in \F$.

\subsection{3}

\begin{exercise}
Let $X, Y$ be reflexive Banach spaces, $A_n \in \L(X, Y)$ for $n \in \N$. Suppose that for all $x \in X$ and all $\ell \in Y^*$ that,
\[ \lim_{n \to \infty} \ell(A_n x) \]
exists. Show that there exists $A \in \L(X, Y)$ such that $A_n \to A$ in the weak operator topology. Give an example where $A_n$ does not converge in the strong operator topology.
\end{exercise}


Consider the sequence of continuous linear functionals $\ev_{A_n x} \in \L(Y^*, \C)$ sending $\ell \to \ell(A_n x)$. These are pointwise bounded because,
\[ \lim_{n \to \infty} \ell(A_n x) \]
exists and thus,
\[ || \ev_{A_n x} (\ell) || \le M_{x, \ell} = \sup_{n} |\ell(A_n x)| \] 
Therefore, by Banach-Steinhaus, $\{ \ev_{A_n x} \}$ is uniformly bounded meaning there is some $M_x > 0$ such that $|| \ev_{A_n x} || \le M_x$. Furthermore,
\[ || \ev_{A_n x} || = \sup_{|| \ell || = 1} || \ell(A_n x) || = || A_n x || \]
Therefore, $|| A_n x || \le M_x$ so $\{ A_n \} \subset \L(X, Y)$ is a pointwise bounded sequence and thus again by Banach-Steinhaus $\{ A_n \}$ is uniformly bounded meaning $|| A_n || \le M$ for all $n \in \N$. 
\bigskip\\
Consider the linear map $\psi_x : Y^* \to \C$ sending $\ell \mapsto \lim\limits_{n \to \infty} \ell(A_n x)$. I claim that $\psi$ is continuous. It suffices to show that $\psi$ is bounded. However,
\[ | \psi(\ell) | = \lim_{n \to \infty} | \ell(A_n x) | \le \lim_{n \to \infty} || \ell || \cdot || A_n || \cdot || x || \le || \ell || \cdot M \cdot || x || \]
Therefore, for each $x \in X$ we see that $\psi_x$ is continuous. Now because $Y$ is reflexive there exists some $y \in Y$ such that $\psi_x = \ev_{y}$ and therefore,
\[ \lim_{n \to \infty} \ell(A_n x) = \psi(\ell) = \ell(y) \]
Thus $A_n x \to y$ weakly. Define $A$ via $A x = y$ which makes sense because weak convergence is unique. This is clearly linear so it suffices to show that $A$ is bounded. Let $|| x || \le 1$ then there exists some $\ell$ such that $|| \ell || = 1$ and $\ell(A x) = || A x ||$ by Hahn-Banach. Consider,
\[ || A x ||  = \ell(A x) = \ell(y) = \lim_{n \to \infty} \ell(A_n x) \le \lim_{n \to \infty} | \ell(A_n x) | \le \lim_{n \to \infty} || A_n x || \le \lim_{n \to \infty} || A_n || \le M \]
Therefore $|| A || \le M$ so $A$ is bounded. Furthermore, by definition $A_n x \to A x$ weakly for each $x \in X$. Therefore, $A_n \to A$ in the weak operator topology.

\subsection{4 (ASK ABOUT THIS!!)}

Let $C^\alpha([0, 1])$ denote the space of H\"{o}lder continuous functions on $[0, 1]$ i.e. such that the norm,
\[ || f ||_{\alpha} = \sup_x | f(x) | + \sup_{x \neq y} \frac{|f(x) - f(y)|}{|x - y|} \]
is finite. Then consider the normed space $(C^\alpha([0,1]), || \bullet ||_\alpha)$.

\subsubsection{a}

\begin{exercise}
Suppose that $0 < \alpha < 1$, and $X \subset C^\alpha([0, 1])$ is a subspace closed in the topology on $C([0, 1])$. Show that $X$ is finite dimensional. 
\end{exercise}

Consider the map $\iota : (X, || \bullet ||_\alpha) \to (X, || \bullet ||_\infty)$. Then since by definition $|| f ||_{\infty} \le || f ||_{\alpha}$ we see that $\iota$ is continuous. Furthermore, $X \subset C([0,1])$ is closed so $(X, || \bullet ||_\infty)$ is a Banach space. Likewise, because $\iota : C^\alpha([0,1]) \to C([0,1])$ is continuous the preimage $(X, || \bullet ||_\alpha) \subset C^\alpha([0,1])$ of the closed subset $(X, || \bullet ||_{\infty}) \subset C([0,1])$ is closed showing that $(X, || \bullet ||_{C^\alpha})$ is Banach since it is a closed subset of $C^\alpha([0,1])$.
\bigskip\\
Consider the inverse $\iota^{-1} : (X, || \bullet ||_\infty) \to (X, || \bullet||_\alpha)$. We show that $\iota^{-1}$ has closed graph as follows. If $f_n \to f$ in uniform norm and $f_n \to f'$ in $\alpha$-norm then $f_n \to f'$ in uniform norm so $f = f'$ because such limits are unique. Therefore by the closed graph theorem $\iota^{-1}$ is continuous and thus bounded. Since this is a bijective continuous map of Banach spaces we could also use the bounded inverse theorem. Either way we conclude that there exists some $M > 0$ such that $|| f ||_\alpha \le M || f ||_{\infty}$ 
\bigskip\\
However, I claim that $\iota$ is compact. Indeed, if $B \subset X$ is H\"{o}lder bounded by $C$ then consider $\iota(B)$. We know that,
\[ | f(x) - f(y) | \le \left( \sup_{x \neq y} \frac{|f(x) - f(y)|}{|x - y|^\alpha} \cdot | x - y |^\alpha  \right) \le || f ||_{C^\alpha} | x - y|^\alpha \le C | x - y|^\alpha \] 
and thus for and $\epsilon > 0$ choose $\delta = (\epsilon C^{-1})^{\frac{1}{\alpha}}$ then for $|x - y | < \delta$ and any $f \in \iota(B)$ we have,
\[ |f(x) - f(y)| \le M | x - y|^\alpha < C \delta^\alpha = \epsilon \]
so $\iota(B)$ is uniformly equicontinuous. Furthermore $\iota(B)$ is clearly bounded by $C$ so by Ascoli's theorem $\iota(B)$ is precompact. However, $\iota$ is an isomorphism so $B$ is precompact proving that $\dim{X} < \infty$ since by Riesz's lemma an infinite dimensional normed space cannot have a compact unit ball. 

\subsubsection{b (DO THIS!!)}

\subsection{5 (HOW DOES THIS WORK??)}

We say that a sequence $\{ a_n \}$ in $[0, 1]$ is equidistributed in $[0,1]$ if for all intervals $[c,d]$ the proportion of elements $a_n \in [c,d]$ for $n \le N$ converges to $d - c$ as $N \to \infty$.

\subsubsection{a (FINISH THIS!)}

Consider the measures,
\[ \mu_N = \frac{1}{N} \sum_{n = 1}^N \delta_{a_n} \]
This converges weakly to the Lebesgue measure $\mu$ if and only if for any continuous function $f$,
\[ \lim_{N \to \infty} \int f \, \d{\mu_N} = \int f \, \d{\mu} \]
This is equivalent to,
\[ \lim_{N \to \infty} \frac{1}{N} \sum_{n = 1}^N f(a_n) = \int f \, \d{\mu} \]
This is basically exactly the Riemann integrability of continuous functions. 


\subsubsection{b}

Consider the sequence $a_n = n \alpha \mod 1$ for $n \ge 1$. If $\alpha$ is rational then $a_n$ only takes on finitely many values and therefore clearly cannot be equidistributed (consider any interval not containing on the finitely many values). 



\section{Spring 2012 Part I}

\subsubsection{5}

\begin{exercise}

Let $X, Y$ be reflexive separable Banach spaces, $X^*$ and $Y^*$ the duals $P \in \L(X, Y)$ and suppose that the adjoint $P^* \in \L(Y^*, X^*)$ satisfies the following property. There is a Banach space $Z$ and a compact map $\iota : Y^* \to Z$ and $C > 0$ such that for all $\phi \in Y^*$,
\[ || \phi ||_{Y^*} \le C (|| P^* \phi ||_{X^*} +  || \iota \phi ||_Z ) \]
Show the following.
\begin{enumerate}
\item $\ker{P^*}$ is finite dimensional
\item if $V \subset Y^*$ is a closed subspace with $V \oplus \ker{P^*} = Y^*$ then there is some $C' > 0$ such that for $\phi \in V$ we have $|| \phi ||_{Y^*} \le C' || P^* \phi ||_{X^*}$ 
\item for $f \in Y$ such that $\ell(f) = 0$ for all $\ell \in \ker{P^*}$ there is $u \in X$ such that $P u = f$.
\end{enumerate}
\end{exercise}

\subsubsection{a}

Restricting to $K = \ker{P^*}$ we see that,
\[ || \phi ||_{K} \le C || \iota \phi ||_Z \]
Therefore, $\iota$ is bounded below and thus is closed. Furthermore, if $\iota \phi = 0$ then $\phi = 0$ by the inequality so $\phi$ is bijective onto its closed image. Thus, $\iota$ is an isomorphism which implies that $\id_K$ is compact and therefore $\dim{K} < \infty$.

\subsubsection{b}

Let $V \subset Y^*$ be a closed subspace such that $V \oplus \ker{P^*} = Y^*$. Suppose to the contrary that,
\[ \inf_{B \cap V} || P^* \phi ||_{X^*} = 0 \]
where,
\[ B = \{ x \in Y^* \mid || x ||_{Y^*} \le 1 \} \]
Then there exists a sequence $\phi_n \in V$ with $|| \phi_n ||_{Y^*} = 1$ and $|| P^* \phi ||_{X^*} < \frac{1}{n}$. By Banach-Alaoglu, $B$ is weakly sequentially compact (because $Y$ is reflexive, the weak and weak-$*$ topologies are the same so $B$ is weakly compact and furthermore $Y$ is separable so $Y^{**} = Y$ is separable and thus $B \subset Y^*$ with the weak topology is metrizable so weak compactness implies weak sequential compactness) and therefore $\{ \phi_n \}$ has a weakly convergent subsequence $\phi_{n_j} \to \phi$ in $B$. Therefore $P^* \phi_{n_j} \to P^* \phi$ weakly and thus,
\[ || P^* \phi || \le \liminf_{j \to \infty} || P^* \phi_{n_j} || \le \liminf_{j \to \infty} \tfrac{1}{n_j} = 0 \]
Therefore $\phi \in \ker{P^*}$ but $V$ is closed and thus weakly closed so $\phi \in V$ and therefore $\phi = 0$. Furthermore, because $\iota$ is compact, $\iota \phi_{n_j} \to \iota \phi = 0$ converges in norm. Therefore,
\[ \limsup_{n \to \infty} || \phi_n ||_{Y^*} \le C \left( \limsup_{n \to \infty}  || P^* \phi_n ||_{Y^*} + \limsup_{n \to \infty} || \iota \phi_n ||_Z \right) = 0 \]
because $\iota \phi_n \to 0$ in norm and $|| P^* \phi_n || < \frac{1}{n}$ by construction. However, for all $n$ we know that $|| \phi_n ||_{Y^*} = 1$ by construction giving a contradiction. Therefore, 
\[ \inf_{B \cap V} || P^* \phi ||_{X^*} = \epsilon > 0 \]
and thus $C' = \epsilon^{-1}$ is such a constant so that for all $\phi \in V$,
\[ || \phi || \le C' || P^* \phi ||_{X^*} \]
\bigskip\\
WHAT FOLLOWS IS A SOLUTION WHICH IS ONLY CORRECT WHEN $Z$ IS HILBERT
\bigskip\\
Let $V \subset Y^*$ be a closed subspace such that $V \oplus \ker{P^*} = Y^{*}$. We can approximate $\iota$ by a sequence $\iota_n : Y^* \to Z$ of finite rank operators. Then for any $\epsilon > 0$ there exists $N$ such that for $n > N$ we have $|| \iota - \iota_n || < \epsilon$. Therefore, for $\phi \in \ker{\iota_n}$ we have,
\[ || \iota \phi ||_Z = || (\iota - \iota_n) \phi ||_Z + || \iota_n \phi ||_Z = || (\iota - \iota_n) \phi ||_Z \le \epsilon || \phi ||_{Y^*} \]
therefore,
\[ || \phi ||_{Y^*} \le C || P^* \phi ||_{X*} + C \epsilon || \phi ||_{Y^*} \]
In particular, taking $\epsilon < C^{-1}$ we find that,
\[ || \phi ||_{Y^*} \le \frac{C}{1 - \epsilon C} || P^* \phi ||_{X^*} \]
Since $\ker{\iota_n}$ has finite codimension, we decompose $V = K \oplus L$ where $K = V \cap \ker{\iota_n}$ and $L$ is finite dimensional.
Now $P^*|_K$ is bounded below and therefore has closed image because if $P^*(\phi_n)$ is a convergent sequence with $\phi_n \in K$ then the above inequality implies that $\phi_n$ is Cauchy and thus convergent so the limit is in $P^*(K)$. Furthermore, since $P^*(L)$ is finite dimensional, it is closed. Now,
\[ \im{P^*} = P^*(V) = P^*(K) + P^*(L) \]
is the sum of a closed space and a finite dimensional space and therefore is closed. Indeed, since $P^*(L)$ is finite dimensional, there is a closed complmement and therefore a continuous projection map $\pi : X^* \to P^*(L)$ meaning there is some $C_1 > 0$ such that $|| P^*(\phi) ||_{X^*} \le C_1 || \psi ||_{X^*}$ whenver $\psi = \psi' + P^*(\phi)$ for $\psi'$ in the complement therefore if $\psi_n \in \im{P^*}$ is a Cauchy sequence then $\psi_n = \psi_n' + P^*(\phi_n)$ and by the inequality $\phi_n$ is Cauchy and thus convergent in the closed subspace $L$. Therefore, $\psi_n'$ is the sum of Cauchy sequences and thus Cauchy and thus convergent in the closed space $P^*(K)$. Therefore, $\im{P^*}$ is closed.
\bigskip\\
Now, $P^*|_V$ is injective with closed image so I claim it is bounded below. To see this, notice that $\im{P^*}$ is Banach and $P^*|_V$ is bijective onto its image so its inverse is bounded giving a constant $C'$ such that,
\[ || \phi ||_{Y^*} = || (P^*|_V)^{-1} (P^* \phi) ||_{Y^*} \le ||(P^*|_V)^{-1}|| \cdot || P^* \phi ||_{X^*} \le C' || P^* \phi || \]

\begin{rmk}
The solution set is wrong. The mistake is when $\phi_n \to \phi$ weakly with $|| \phi_n || = 1$ and thus $|| \phi || \le 1$ is assumed to have $\phi \neq 0$ so that $P^* \phi \neq 0$ (since $\phi \in V$). This does not work suppose that $X = Y = \ell^2$ which are self-dual and take $(P a)_n = \frac{1}{n} a_n$ then $\phi_n = e_n$ is a sequence with $|| \phi_n || = 1$ but weakly convergent to zero. This explains why here we do have $|| P \phi_n || \to 0$ although $|| \phi_n || = 1$ for all $n$ contrary to what the solutions say.
\end{rmk}

\subsubsection{c}

By the closed range theorem, $\im{P}$ is closed if and only if $\im{P^*}$ is closed so $\im{P}$ is closed.
\bigskip\\
Let $f \in Y$ be such that $\ell(f) = 0$ for all $\ell \in \ker{P^*}$. Suppose that $f \notin \im{P}$ then by Hahn-Banach there exists $\ell$ such that $\ell|_{\im{P}} = 0$ and $\ell(f) \neq 0$. However, if $\ell|_{\im{P}} = 0$ then $P^*(\ell)(x) = \ell(P(x)) = 0$ and thus $\ell \in \ker{P^*}$ so $\ell(f) = 0$ by hypothesis giving a contradiction. Thus, $f \in \im{P}$.


\section{Fall 2012 Part II}

\subsection{1}

\subsubsection{a}

\begin{exercise}
Show that the spectrum of a bounded linear operator $A$ on a Banach space $X$ is non-empty.
\end{exercise}

Let $T : X \to X$ be a bounded linear operator. Suppose towards a contradiction that $\sigma(T) = \empty$. Then, $\rho(T) = \C$ so the resolvent function, 
\[ \rho : \C \to \L(X, X) \]
given by $\lambda \mapsto (T - \lambda I)^{-1}$ is analytic everywhere. Then, for any $\ell \in \L(X, X)^*$ we find that $\ell \circ \rho : \C \to \C$ is an entire function because $\ell$ is continuous and linear and therefore commutes with Taylor series. Furthermore, for $| \lambda | > || T ||$ there is a convergent series representation,
\[ \rho(\lambda) = (T - \lambda I)^{-1} = - \frac{1}{\lambda} \sum_{n = 0}^\infty \left( \frac{T}{\lambda} \right)^n \]
Then we see,
\[ \ell(\rho(\lambda)) = - \frac{1}{\lambda} \sum_{n = 0}^\infty \frac{\ell(T^n)}{\lambda^n} \]
but $| \ell(T^n) | \le || \ell || \cdot || T^n || \le || \ell || \cdot || T ||^n$ and therefore for $| \lambda | > || T ||$ we see that the series is bounded by $(1 - || T || / | \lambda |)^{-1}$ which goes to zero as $\lambda \to \infty$. However, the function $\ell \circ \rho : \C \to \C$ is entire meaning that is it continuous on the compact set $\overline{B_{2 || T ||}(0)}$ and therefore it is bounded everywhere. Thus, by Liouville, $\ell \circ \rho$ is constant for every $\ell$. Since $\L(X, X)$ is Banach its dual space separates points implying that $\lambda \mapsto (T - \lambda I)^{-1}$ is constant which is clearly impossible. Therefore we cannot have $\sigma(T)$ empty.

\subsubsection{b (DO THIS!!!)}

\section{Fall 2013 Part II}

\subsection{3}

Let $\S(\R)$ be the set of Schwartz functions and $\S'(\R)$ the tempered distributions. For $u \in \S'(\R)$ let $Du$ be the distributional derivative of $u$ and let $M u$ be the dsitribution $x u$ meaning $(M u) \phi = u(x \phi)$ for $\phi \in \S(\R)$. Let.
\[ \H = \{ u \in L^2(\R) \mid D u \in L^2, Mu \in L^2 \} \]
equiped with the norm,
\[ || u ||_{\H}^2 = || u ||_{L^2}^2 + || D u ||_{L^2}^2 + || M u ||^2_{L^2} \]
It is clear that $|| u ||_{\H}$ is induced by an inner product and furthermore $|| u ||_{L^2} \le || u ||_{\H}$ so Cauchy sequences in $\H$ are $L^2$-Cauchy and thus convergent in $L^2$. Then let $v_i = u_i - u$ and $v_i \to v$ in $L^2$ with $D u_i$ and $M u_i$ Cauchy in $L^2$. Therefore, $D u_i$ and $M u_i$ converge in $L^2$ and convergence in $L^2$ implies convergence in the weak topology as distributions so $D u_i \to $


\section{Spring 2013 Part I}

\subsection{1}

\subsubsection{a}

Let $X$ be a Banach space such that $X^* \to X^{***}$ is an isomorphism. Then the inclusion $X \to X^{**}$ is an isometry and therefore closed. Then an application of Hahn-Banach proves that if $Y \subset X$ is a closed subspace of a reflexive normed space then $Y$ is reflexive. Furthermore, $X^{**}$ is reflexive because $X^{*}$ is so we conclude that $X$ is also reflexive. 

\subsubsection{b}

Let $x \in \R^n$ and $\epsilon > 0$. Consider the bump function,
\[ f(x) = \begin{cases}
e^{\frac{1}{x - 1}} & x < 1
\\
0 & x \ge 1
\end{cases} \]
Then let $\varphi(\vec{x}) = f(|x-x_0|^2/\epsilon^2)$ it is easy to show that $f$ is smooth and thus $\varphi$ is also smooth. Furthermore, if $|x - x_0| \ge \epsilon$ then $\varphi(\vec{x}) = 0$ by construction.

\subsection{2}

\subsubsection{a}

Suppose that $f, g$ are positive measurable functions on $[0, 1]$ and $f(x) g(x) \ge 1$ for $x \in [0, 1]$. Then let $\tilde{f} = \sqrt{f}$ and $\tilde{g} = \sqrt{g}$ exist and are measurable and satisfy $\tilde{f} \tilde{g} \ge 1$. Therefore, by Cauchy-Schwartz,
\[ \left( \int |\tilde{f}|^2 \d{x} \right) \cdot \left( \int |\tilde{g}|^2 \d{x} \right) \ge \int |\tilde{f} \tilde{g}| \d{x} \ge 1 \]
However, $|\tilde{f}|^2 = f$ and likewise $|\tilde{g}|^2 = g$ proving the result.

\subsubsection{b (CANT DO IT!!!)}

Suppose that $(X, \F, \mu)$ is a $\sigma$-finite measure space, $K$ is a measurable function on $X \times X$ and,
\[ \int | K(x,y) | \d{y} \le C \quad \text{and} \quad \int |K(x,y)| \d{x} \le C \]
$\mu$-almost everywhere. Show that the integral operator $A : L^2(X) \to L^2(X)$ defined by,
\[ (Af)(x) = \int K(x,y) f(y) \d{y} \]
is well-defined and bonuded and its norm is bounded by $C$.
\bigskip\\
Notice that, by Fubini's theorem,
\begin{align*}
\int \left| (Af)(x) \right|^2 \d{x} & \le \int \left( \int |K(x,y) f(y)| \d{y} \right)^2 \d{x} 
\\
& = \int \left( \int \int |K(x, y) K(x, y') f(y) f(y') | \d{y} \d{y'} \right) \d{x}
\end{align*}
However, the first is uniformly bounded

\subsection{3 (WHAT THE FUCK IS THIS!!)}

\subsection{4}

Let $X = S^1$.
By density, it suffices to prove that there exist $f \in C(X)$ and $c \in \C$ such that $\inner{u}{e_n} = \inner{f^{(k)} + c}{e_n}$ where $e_n(t) = e^{2 \pi i n t}$. Clearly, 
\[ \inner{f^{(k)} + c}{e_n} = (-1)^k \int_{0}^1 f(t) (2 \pi i n)^k e^{2 \pi i n t} \d{t} + \delta_{n,0} c = (-1)^k (2 \pi i n)^k b_n + \delta_{n, 0} c \]
where $b_n = \inner{f}{e_n}$ is the $n$-th Fourier coefficient of $f$. However, since $u$ is a distribution, the coefficients
\[ u_n = \inner{u}{e_n} \]
must have polynomial growth say of order $k$ (REFERENCE!!!). Therefore, setting,
\[ b_n = \begin{cases}
(-2 \pi i n)^{-(k+2)} u_n + \delta & n \neq 0
\\
0 & n = 0
\end{cases} \]
we immediately see that,
\[ \sum_{n \in \Z} |b_n| < \infty \]
becuase the $|b_n| \sim n^{-2}$ since $u_n$ has gowth bounded by a polynomial of degree $k$. Thus, there exists a unique continuous function $f$ whose whose Fourier coefficients are $b_n$ (REFERENCE!!). Thus we see that $u = f^{(k+2)} + u_0$. 

\subsection{5}

For each of the following maps $f : \R \to X$ where $X$ is a topological vector space, prove or disprove that the map is continuous or differentiable.

\subsubsection{a}

Let $X = L^2(\R)$ and $f_t(x) = \chi_{[t, t+1]}(x)$. Consider, for $t, t' \in \R$ WLOG $t < t'$,
\begin{align*}
|| f_{t'} - f_t ||_2^2 & = \int_{-\infty}^{\infty} (\chi_{[t', t'+1]} - \chi_{[t, t + 1]})^2 \d{x} 
\\
& = 
\begin{cases}
2 & t + 1 \le t'
\\
2 (t' - t) & t' < t + 1
\end{cases} 
\end{align*}
therefore $f$ is continuous because $|| f_{t'} - f_t ||_2 < \epsilon$ when $| t' - t | < \tfrac{1}{2} \epsilon^2$. However, I claim that $f$ is nowhere differentiable. Indeed, if the limit,
\[ \lim_{h \to 0} \frac{f_{t + h} - f_t}{h} \]
exists then so much the limit of the norms,
\[ \lim_{h \to 0} \frac{|| f_{t + h} - f_t ||_2}{h} \]
However, for $0 < h < 1$ we have,
\[ \frac{|| f_{t + h} - f_t ||_2}{h} = \sqrt{2}{h} \to \infty \]
is not bounded and thus has no limit as $h \to 0$.

\subsubsection{b}

Let $X = L^2(\R)$ and,
\[ f_t(x) = 
\begin{cases}
\sin{(x - t)} & t \le x \le t + \pi
\\
0 & \text{else} 
\end{cases} \]
Then, I claim that $f$ is differentiable whose derivative is,
\[ g_t(x) = \begin{cases}
-\cos{(x - t)} & t \le x \le t + \pi
\\
0 & \text{else} 
\end{cases} \]
for $t \in \R$ we only care about small $h$ so let $h < \pi$ then,
\begin{align*}
|| h^{-1}(f_{t+h} - f_t) - g_t ||_2^2 & = \int_{-\infty}^{\infty} [h^{-1}(f_{t+h}(x) - f_{t}(x)) - g_t]^2 \d{x} 
\\
& = \int_{0}^{\pi} \left( \frac{\sin{(x' - h)} - \sin{(x')}}{h} + \cos{(x')} \right)^2 \d{x'} + \int_0^h \left( \frac{\sin{(h - x')}}{h} \right)^2 \d{x'} 
\end{align*}
However, for small $h$ and $0 \le x \le h$ we have $\sin{(h - x')} \le h$ and therefore, the second integrand is bounded by $1$ so the second integral is bounded by $h$. Furthermore, \[ \sin{(x' - h)} = \sin{(x')} \cos{(h)} - \cos{(x')} \sin{(h)} \]
and therefore,
\[ \frac{\sin{(x' - h)} - \sin{(x')}}{h} + \cos{(x')} = \frac{\cos{h} - 1}{h} \sin{x'} + \left( 1 - \frac{\sin{h}}{h} \right) \cos{x'} \]
Therefore, these functions uniformly converge to zero as $h \to 0$ and thus because the integral is over the compact space $[0, \pi]$,
\begin{align*}
|| h^{-1}(f_{t+h} - f_t) - g_t ||_2 \to 0 
\end{align*}
explicitly because the integrand is bounded by $\epsilon$ so the integral is bounded by $\pi \epsilon^2 \to 0$.

\subsubsection{c}

Let $X = \mathcal{S}'(\R)$ be the space of tempered distributions with the weak-$*$ topology and let $f_t = \delta_t$. A function to $X$ with the weak-$*$ topology is continuous iff $t \mapsto \inner{f_t}{g}$ is continuous for all $g \in \mathcal{S}(\R)$. Thus, consider,
\[ t \mapsto \inner{f_t}{g} = g(t) \]
but by definition, all functions is the Schwartz space are continuous so $f$ is continuous.
\bigskip\\
Now we consider the limit,
\[ \lim_{h \to 0} \frac{\delta_{t + h} - \delta_t}{h} \]
In the weak-$*$ topology convergence is equivalent to pointwise congergence in the sense that $\{ f_n \}$ converges iff $\{ f_n(g) \}$ converges for each $g \in \mathcal{S}(\R)$. Then consider,
\[ \lim_{h \to 0} \frac{\delta_{t + h}(g) - \delta_t(g)}{h} = \lim_{h \to 0} \frac{g(t + h) - g(t)}{h} \]
exists because by definition any Schwartz function is differentiable. Therefore, $f$ is differentiable.

\section{Spring 2013 Part II}

\subsection{1 (NO IDEA!!)}

Suppose that $F_n$ and $F$ are increasing functions on $[a,b]$ such that for all $x \in [a,b]$,
\[ F(x) = \sum_{n = 1}^\infty F_n(x) \]

\subsection{2}

Suppose that $1 < p < \infty$ and $f_n, f \in L^p([0,1])$ such that $||f_n ||_{L^p} \le 1$ and $f_n \to f$ almost everywhere. Show that $f_n \to f$ weakly and $|| f ||_{L^p} \le 1$.
\bigskip\\
Convergence in the weak topology is equivalent to convergence of $\ell(f_n) \to \ell(f)$ for each $\ell \in L^p([0,1])^*$. However, for $1 < p < \infty$ we know that $L^p([0, 1])^* = L^q([0, 1])$ with $\frac{1}{p} + \frac{1}{q} = 1$. Therefore, it suffices to check that,
\[ \lim_{n \to \infty} \int_0^1 f_n g \, \d{t} = \int_0^1 fg \, \d{t} \]
for all $g \in L^q([0, 1])$.
\bigskip\\
We apply Egorov's theorem, for any $\epsilon > 0$ there exists a measurable $E \subset [0,1]$ such that $f_n$ converges uniformly on $F = E^C$ and $\mu(E) < \epsilon$. Then we have,
\begin{align*}
\left| \int_0^1 f_n g \, \d{t} - \int_0^1 fg \, \d{t} \right| \le \int_E |f_n - f| |g| \, \d{t} + \int_F |f_n - f| |g| \, \d{t} 
\end{align*}
However, for sufficiently large $n$ we know $|f_n - f| < \epsilon$ on $F$ by uniform convergence so,
\[ \int_E |f_n - f| |g| \, \d{t} + \int_F |f_n - f| |g| \, \d{t} \le || f_n - f ||_p \left( \int_E | g |^q \right)^{\frac{1}{q}} +  \epsilon \cdot || g ||_1 \]
by H\"{o}lder. However, $||f_n - f ||_p \le || f_n ||_p + || f ||_p$ is bounded uniformly in $n$ and as $\mu(E) \to 0$ we know that,
\[ \int_E | g |^q \, \d{\mu} \to 0 \]
because $g \in L^q([0,1])$ so therefore,
\[ \left| \int_0^1 f_n g \, \d{t} - \int_0^1 fg \, \d{t} \right| \to 0 \]
Furthermore, setting $g = \mathrm{sign}(f) \cdot |f|^{p - 1}$ we know that,
\[ \int_0^1 |f|^p = \lim_{n \to \infty} \int_0^1 f_n g \, \d{t} \le \lim_{n \to \infty} || f_n ||_{p} || g ||_q \]
However, $q(p-1) = p$ and therefore,
\[ || g ||_q = \left( \int_0^1 |f|^{q(p-1)} \, \d{t} \right)^{\frac{1}{q}} = \left( \int_0^1 |f|^p \, \d{t} \right)^{q} = || f ||_p^{\frac{p}{q}} \]
which means that,
\[ || f ||_p^p \le \lim_{n \to \infty} || f_n ||_p \cdot || f ||_p^{\frac{p}{q}} = || f ||_p^{\frac{p}{q}} \]
However, $p > p/q$ so this is only possible if $|| f ||_p \le 1$.

\subsection{3}

Let $X$ be a separable Hilbert space. 

\subsubsection{a}

Let $T \in \L(X)$ be compact and $T = T^*$. 

\subsubsection{b}

\subsection{4}

Let $X$ be an uncountable set with the discrete topology and $\hat{X}$ the one point compactification. Let $C(\hat{X})$ be the Banach space of real-valued continuous functions on $\hat{X}$. 
\bigskip\\
Notice that $X$ is Hausdorff and locally compact so $\hat{X}$ is Hausdorff and compact. 

\subsubsection{a}

Remark: I use the word countable to include finite (some people might prefer I say ``at most countable'' but I don't want to keep writing that).
\bigskip\\
Consider the $\sigma$-algebra $\F$ generated by compact $G_\delta$-sets and the $\sigma$-algebra Borel sets. The closed subsets of $\hat{X}$ are exactly the finite sets and the sets that contain $\infty$. Therefore, the $\sigma$-algebra of Borel sets contains every set because any subset of $X$ is open in $\hat{X}$ and its union with $\infty$ is closed in $X$. 
\bigskip\\
Now consider $\F$. The compact sets of $\hat{X}$ are the finite ones and those that contain $\infty$. The $G_\delta$ sets are countable intersections of open sets and thus are either arbitrary subsets of $X$ or contain $\infty$ and thus must be an intersection of open sets containing $\infty$ which are cofinite. Therefore $G_\delta$ sets are exactly the arbitrary subsets of $X$ or the cocountable subsets of $\hat{X}$. A $G_\delta$ set is compact if it is finite or contains infinity. Therefore, $\F$ is generated by the finite sets and the cocountable ones containing $\infty$. Thus, $\F$ is exactly the countable sets not containing $\infty$ and the cocountable sets containing $\infty$.

\subsubsection{b} 

A continuous function $f : \hat{X} \to \R$ must have closed and bounded image because $\hat{X}$ is compact. Let $f(\hat{X}) \subset [a,b]$ with $a,b \in f(\hat{X})$. Consider $c = f(\infty) \in [a,b]$. Then, $f^{-1}((c - \epsilon, c + \epsilon))$ is open but contains $\infty$ and thus must be cofinite. Therefore, the continuous functions $f : \hat{X} \to \R$ are exactly those satisfying the property that for any $\epsilon > 0$ there are only finitely many $x \in \hat{X}$ with $|f(x) - f(\infty)| \ge \epsilon$.
\bigskip\\
Let $\B$ be the $\sigma$-algebra of sets such that either $A$ is countable or $A^C$ is countable (they can't both be countable because $X$ is uncountable). Then $\F \subset \B$ because the Baire $\sigma$-algebra contains exactly the countable sets not containing $\infty$ and the cocountable sets containing $\infty$.
\bigskip\\
Now, let $\mu_1 = \delta_\infty$ be the indicator measure of $\infty$ i.e.
\[ \mu_1(A) = 
\begin{cases}
1 & \infty \in A
\\
0 & \infty \notin A 
\end{cases} \]
and $\mu_2$ the cocountable measure,
\[ \mu_2(A) = 
\begin{cases}
1 & A^C \text{ is countable }
\\
0 & A \text{ is countable} 
\end{cases} \]
It is obvious that these are finite measures and that they are equal on $\F$ but not equal on $\B$ because $\mu_1(\{ \infty \}) = 1$ but $\mu_2(\{ \infty \}) = 0$.
\bigskip\\
Now, for any continuous function $f : \hat{X} \to \R$, let $f(\infty) = c$. Then we know that all but finitely many values in the image are $\epsilon$-near $c$ to by adding a constant we can assume $f > 0$ (adding a constant just shifts the integrals because the measures are finite). Now,
\[ \int_{\hat{X}} f \, \d{\mu_i} = \int_0^\infty \mu_i(\{ x \in \hat{X} \mid f(x) > t \}) \, \d{t} \]
Let $f^*_i(t) = \mu_i(\{ x \in \hat{X} \mid f(x) > t \})$.
However, if $t > c$ then $f^{-1}((t, \infty))$ is finite any does not include $\infty$ so $f^*_i(t) = 0$. However, for $t < c$ we know $f^{-1}((t, \infty))$ is open and contains $\infty$ and thus is cofinite so therefore $f^*_i(t) = 1$ and we find,
\[ \int_{\hat{X}} f \, \d{\mu_i} = c \]
and in particular,
\[ \int_{\hat{X}} f \, \d{\mu_1} = \int_{\hat{X}} f \, \d{\mu_2} \]
This does not contradict the representation theorem because $\mu_2$ cannot be extended to a Borel measure and furthermore is not inner regular because it vanishes on all compact subsets of a set not containing $\infty$ (which are just the finite sets) so it is not a Radon measure.


\section{Spring 2014 Part I}

\subsection{1}

\begin{exercise}
For $\alpha \in (0,1)$ consider the space $C^\alpha([0,1])$ of continuous functions with,
\[ || f ||_{C^\alpha} = \sup_x | f(x) | + \sup_{x \neq y} \frac{|f(x) - f(y)|}{|x - y|^\alpha} \]
finite equiped with the $|| \bullet ||_{C^\alpha}$ norm.
\begin{enumerate}
\item Show that the unit ball of $C^\alpha([0,1])$ has compact closure in $C([0,1])$.
\item Show that $C^\alpha([0,1])$ is of first category in $C([0,1])$.
\end{enumerate}
\end{exercise}

\subsubsection{a}

We apply the Ascoli theorem to the space $X = B = \{ f \in C^\alpha([0,1]) \mid || f ||_{C^\alpha} = 1 \}$ inside $C([0,1])$. To show that $X$ is compact it suffices to prove that $X$ is equicontinuous and bounded. Boundeness is obvious because $|| f ||_{\infty} \le || f ||_{C^\alpha} \le 1$ for all $f \in B$. Now we consider,
\[ | f(x) - f(y) | \le | x - y |^\alpha \cdot \sup_{x \neq y} \frac{|f(x) - f(y)|}{|x - y|^\alpha} \le | x  - y|^\alpha || f ||_{C^\alpha} \le | x - y |^\alpha \]
because $|| f ||_{C^\alpha} = 1$ and therefore for all $\epsilon > 0$ choose $\delta = \epsilon^{\frac{1}{\alpha}}$ and therefore if $| x - y | < \delta$ we see that,
\[ | f(x) - f(y) | \le | x - y |^\alpha < \delta^\alpha = \epsilon \]
for all $f \in B$ proving that $X$ is equicontinuous. Therefore, by Ascoli's theorem, $B$ is precompact.

\subsubsection{b}

We apply the following lemma.

\begin{lemma}
Let $X$ be an infinite dimensional normed space and $V \subset X$ a linear subspace and $B_V \subset V$ a convex subset containing a nonzero element of each ray. If $B_V$ is precompact then $V$ is of first category in $X$.
\end{lemma}

\begin{proof}
We see that,
\[ V = \bigcup_{n = 1}^\infty n B_V \]
because for each $v \in V$ there is some $n$ such that $v \in n B_V$ because $\vspan{v}$ intersects  $B_V$ at some nonzero element $v' \in B_V$ and then $v = t v$ so choose $n \in \Z$ with $n \ge t$ and because $B_V$ is convex $v \in B_V$.
Therefore it suffices to prove that $n B_V$ is nowhere dense for each $n$. Because multiplication by $n$ is a homeomorphism it suffices to show that $B_V$ is nowhere dense. Indeed, $\overline{B_V}$ is compact. So if $x \in \overline{B_V}^\circ$ then there is a ball $B_\epsilon(x) \subset \overline{B_V}^\circ$ and thus $\overline{B_\epsilon(x)} \subset \overline{B_V}$ so $\overline{B_\epsilon(x)}$ is a closed subset of a compact set and thus compact. However, since $X$ is infinite dimensional, by Riesz lemma, the closed unit ball, which is homeomorphic to $\overline{B_\epsilon(x)}$, is not compact. Thus, $\overline{B_V}^\circ = \empty$ and therefore $B_V$ is nowhere dense proving that $V$ is of first category.
\end{proof}

Then (b) is a direct consequence of (a) if we let $B \subset C^\alpha([0,1]) \subset C([0,1])$ be the unit ball with respect to the $|| \bullet ||_{C^\alpha}$ norm.



\section{Spring 2015 Part I}

\subsection{1}

Suppose that $A, B \subset \R / \Z$ are measurable sets with $\mu(A), \mu(B) > 0$.

\subsubsection{a}

Since $A, B$ have positive measure, they have at least one Lebesgue point say $x \in A$ and $y \in B$. Choose $z = y - x$ such that $A' = A + z$ and $B$ both have $y$ as a Lebesgue point. Then for any $\epsilon > 0$ there is some $\delta > 0$ such that,
\[ \mu(A' \cap B_\delta(y)) \ge 2 \delta (1 - \epsilon) \]
and likewise,
\[ \mu(B \cap B_\delta(y)) \ge 2 \delta(1 - \epsilon) \]
However, suppose that,
\[ \mu(A' \cap B) = 0 \]
then we see that,
\begin{align*}
\mu((A' \cup B) \cap B_\delta(y)) & = \mu(A' \cap B_\delta(y)) + \mu(B' \cap B_\delta(y)) - \mu(A' \cap B \cap B_\delta(y)) 
\\
& = \mu(A' \cap B_\delta(y)) + \mu(B' \cap B_\delta(y)) \ge 4 \delta( 1 - \epsilon) > 2 \delta 
\end{align*}
whenever $\epsilon < \tfrac{1}{2}$ which is a contradiction because,
\[ \mu((A' \cup B) \cap B_\delta(y)) \le \mu(B_\delta(y)) = 2 \delta \]
Therefore, we must have,
\[ \mu(A' \cap B) > 0 \]
proving the claim.

\subsubsection{b}

Consider the map $f : \R / \Z \to L^1(\R / \Z)$ via $f(y) = \chi_{A + y}$. This is continuous since the map $\R / \Z \to \L(L^1(\R / \Z))$ given by sending $y \mapsto L_y$ to the shift operator $L_y$ is continuous in the strong operator topology. 
\bigskip\\
Therefore,
\[ g(y) = \mu((A + y) \cap B) = \int_{\R / \Z} \chi_{A + y} \chi_B \, \d{\mu} \]
is a continuous function of $y$. Furthermore,
\begin{align*}
\int_{\R / \Z} g(y) \, \d{y} & = \int_{\R / \Z} \mu((A + y) \cap B) \, \d{y} = \int_{\R/\Z} \left( \int_{\R / \Z} \chi_{A + y}(x) \, \d{y} \right) \chi_B(x) \, \d{x}
\\
& = \int_{\R / \Z} \left( \int_{\R / \Z} \chi_{A - x}(y) \, \d{y} \right) \chi_B(x) \, \d{x} = \mu(A) \int_{\R/\Z} \chi_B(x) \, \d{x} = \mu(A) \mu(B)
\end{align*}
Furthermore, since $g$ is continuous, we know that $g \in L^\infty$. Because $\mu(\R / \Z) = 1$, by H\"{o}lder,
\[ || g ||_1 \le || g ||_{\infty} \]
Therefore,
\[ || g ||_{\infty} \ge || g ||_1 = \int_{\R / \Z} g(y) \, \d{y} = \mu(A) \mu(B) \]
Since $g$ is continuous it achieves it supremum and therefore, there is some $y \in \R / \Z$ such that,
\[ \mu((A + y) \cap B) = g(y) = || g ||_{\infty} \ge \mu(A) \mu(B) \]

\subsection{2}

Let $X, Y$ be Banach spaces.

\subsubsection{a}

Suppose that $T_n \in \L(X, Y)$ are compact and $T_n \to T$ in norm for some $T \in \L(X, Y)$. We need to show that $T(B)$ is totally bounded because a set is precompact in a complete metric space if and only if it is totally bounded. For any $\epsilon > 0$ there is some $T_n$ such that $|| T - T_n || < \frac{1}{2} \epsilon$ and $T_n(B)$ is precompact so there is a finite cover $\{ B_{\frac{1}{2} \epsilon}(y_i) \}$ of $T_n(B)$ by $\frac{1}{2} \epsilon$-balls. Now, I claim that $\{ B_\epsilon(y_i) \}$ is a finite cover of $T(B)$ by $\epsilon$-balls. Indeed, for any $x \in B$ we know that $|| Tx - T_n x|| \le || T - T_n || \cdot || x || < \frac{1}{2} \epsilon$ and furthermore, there is some $y_i$ such that $T_n x \in B_{\tfrac{1}{2} \epsilon}(y_i)$ because $T_n x \in T_n(B)$ and thus,
\[ || T x - y_i || \le || T x - T_n x || + || T_n x - y_i || < \tfrac{1}{2} \epsilon + \tfrac{1}{2} \epsilon = \epsilon \]
Therefore, $T x \in B_\epsilon(y_i)$ proving the claim. Therefore $T(B)$ is precompact so $T$ is compact.

\subsubsection{b}

Let $X, Y$ be separable Hilbert spaces and $T \in \L(X, Y)$ a compact operator. Because $X$ and $Y$ are separable they have orthogonal Schauder bases $\{ e_i \}$ and $\{ f_i \}$ giving isomorphisms $X \cong \ell^2$ and $Y \cong \ell^2$. Let $P_n : X \to X$ be the projection onto $\vspan{e_1, \dots, e_n}$ and consider the finite rank operators,
\[ T_n x = T P_n = T \left( \sum_{i = 1}^n \inner{e_i}{x} e_i \right) = \sum_{i = 1}^n \inner{e_i}{x} T e_i \]
whose image lies inside $\vspan{T e_1, \dots, T e_n}$.  Furthermore, $T_n \in \L(X, Y)$ because $T = T P_n$ and $P_n$ is continuous because $\inner{e_i}{-}$ is continuous and $t \mapsto t e_i$ is continuous so $P_n$ is the sum of finitely many continuous functions. It is clear that $T_n x \to T x$ for each $x$ (i.e. $T_n \to T$ pointwise) because for any $x \in X$,
\[ x = \sum_{i = 1}^\infty \inner{e_i}{x} e_i = \lim_{n \to \infty} \sum_{i = 1}^n \inner{e_i}{x} e_i \]
and thus $P_n \to I$ pointwise and thus $T_n = T P_n \to T$ poinwise. Now I claim that $T_n \to T$ in norm which will require the hypothesis that $T$ is compact.
\bigskip\\
Notice that,
\[ || T - T_n || = \sup_{|| x || = 1} || T x - T_n x || \le \sup_{|| x || = 1} \sum_{i = n}^\infty | \inner{e_i}{x} |^2 \cdot || T e_i ||^2 \]
I claim that $T e_i \to 0$ in norm. Indeed, because $\{ e_i \}$ is orthogonal, $e_i \to 0$ weakly since for any $x \in X$,
\[ \inner{e_i}{x} \to 0 \text{ because } \sum_{i = 1}^\infty | \inner{e_i}{x} |^2 = || x || < \infty \]
and therefore, because bounded operators are continuous in the weak topology $T e_i \to 0$ weakly. Alternatively, for any $y \in Y$,
\[ \lim_{i \to \infty} \inner{T e_i}{y} = \lim_{i \to \infty} \inner{e_i}{T^* y} = 0 \]
so $T e_i \to 0$ weakly. If $T e_i$ does not converge to $0$ in norm, there is some $\epsilon > 0$ such that $|| T e_i || \ge \epsilon$ infinitely often and therefore by compactness of $T(B)$ there is a convergent subsequence $T e_{n_j}$ with $|| T e_{n_j} || \ge \epsilon$. Let $T e_{n_j} \to y$ in norm then consider,
\[ || y || = \lim_{j \to \infty} || T e_{n_j} || \ge \epsilon \]
and thus,
\[ \lim_{j \to \infty} \inner{T e_{n_j}}{y} = \inner{y}{y} = || y ||^2 \ge \epsilon^2 \]
contradicting the fact that $T e_{i} \to 0$ weakly because subsequences of weakly convergent subsequences weakly converge to the same limit. Thus $T e_i \to 0$ in norm. 
\bigskip\\
Finally,
\[ || T - T_n || \le \sup_{|| x || = 1} \sum_{i = n}^\infty | \inner{e_i}{x} |^2 \cdot || T e_i ||^2 \le \sum_{i = n}^\infty | \inner{e_i}{x} |^2 \cdot \left( \sup_{k \ge n} || T e_k || \right) \le \sup_{|| x || = 1} || x || \sup_{k \ge n} || T e_k || \le \sup_{k \ge n} || T e_k || \]
Therefore,
\[ \lim_{n \to \infty} || T - T_n || \le \limsup_{n \to \infty} || T e_n || = 0 \]
because,
\[ \lim_{n \to \infty} || T e_n || = 0 \]
Therefore $T \to T_n$ in norm.
\bigskip\\
CAN WE DO IT THE OTHER WAY???
Consider the finite rank operators,
\[ T_n x = \sum_{i = 1}^n \inner{f_i}{Tx} f_i \]
whose image lies inside the span of $\{ f_1, \dots, f_n \}$. Furthermore, $T_n \in \L(X, Y)$ because it is the sum of continuous operators ($x \mapsto T x$ is continuous $y \mapsto \inner{f_i}{y}$ is continuous and $t \mapsto t f_i$ is continuous). Now I claim that $T_n \to T$ in norm. 
\bigskip\\
Indeed, $T_n x \to T x$ for each $x$ (i.e. $T_n \to T$ pointwise) because for any $y \in Y$ we have,
\[ y = \sum_{i = 1}^\infty \inner{f_i}{y} f_i = \lim_{n \to \infty} \sum_{i = 1}^n \inner{f_i}{y} f_i \]
However, to show that $T_n \to T$ in norm we need to use the compactness of $T$.
\bigskip\\
Notice that for any $x \in B$,
\[ |\inner{f_i}{T x}| = |\inner{T^* f_i}{x}| \le || T^* f_i || \cdot || x || \le || T^* f_i || \] 
and $T^*$ is also compact so I claim that $T^* f_i \to 0$ in norm. Indeed because $\{ f_i \}$ is orthogonal, $f_i \to 0$ weakly and bounded operators are continuous for the weak topology so $T f_i \to 0$ weakly. If $T^* f_i$ does not converge to zero in norm then there is a subsequence with $|| T^* f_{n_j} || \ge \epsilon$ for some $\epsilon > 0$ but $T(B)$ is precompact so this subsequence has a convergent subsequence $\{ T^* f_{n'_j} \}$. Let $T^* f_{n'_j} \to x$ in norm and therefore,
\[ || x || = \lim_{j \to \infty} || T^* f_{n'_j} || \ge \epsilon \]
Thus,
\[ \lim_{j \to \infty} \inner{T^* f_{n'_j}}{x} = \inner{x}{x} \ge \epsilon \]
which contradicts $T^* f_i \to 0$ weakly because limits of subsequences are unique. Therefore $T^* f_i \to 0$ in norm.
\bigskip\\
Therefore,
\[ || T - T_n || = \sup_{|| x || = 1} || T x - T_n x|| = \sup_{|| x || = 1} \sum_{i = n}^\infty | \inner{f_i}{T x} |^2 \]

\subsection{3}

Let $\S(\R)$ be the space of Schwartz functions on $\R$. 

\subsubsection{a}

Consider,
\[ \Phi(x) = \sum_{n \in \Z} \phi(x + 2 \pi n) \]
which exists because $\phi$ decays superpolynomially. Furthermore, notice that $\Phi$ is $2 \pi$-periodic and is continuous because for $x \in [0, 2\pi]$ the sum converges uniformly by the $M$-test because $\phi(x + 2 \pi n)$ is bounded by $M/(2 \pi n)^2$ for some $M$.
\bigskip\\
Therefore, $\Phi$ has a Fourier series representation so it suffices to compute the Fourier coefficients,
\begin{align*}
(\F \Phi)_k & = \int_0^{2 \pi} \Phi(x) e^{- i k x} \, \d{x} = \int_0^{2 \pi} \sum_{n \in \Z} \phi(x + 2 \pi i) e^{-i k x} \, \d{x} = \sum_{n \in \Z} \int_0^{2\pi} \phi(x + 2 \pi n) e^{- i k x} \, \d{x}
\\
& = \sum_{n \in \Z} \int_{2 \pi n}^{2 \pi (n+1)} \phi(x) e^{- i k (x - 2 \pi n)} \, \d{x} = \sum_{n \in \Z} \int_{2 \pi n}^{2 \pi (n+1)} \phi(x) e^{- i k x} \, \d{x} = \int_{-\infty}^\infty \phi(x) e^{- i k x} \, \d{x} = (\F \phi)(k)
\end{align*}
Therefore, by Fourier inversion,
\[ \Phi(x) = \frac{1}{2 \pi} \sum_{k \in \Z} (\F \Phi)_k e^{i k x} \]
and therefore unpacking,
\[ 2 \pi \sum_{n \in \Z} \phi(x + 2 \pi n) = \sum_{k \in \Z} (\F \phi)(k) e^{i k x} \]

\subsubsection{b}

Consider the function $\phi(x) = \exp{\left( - t x^2 / 2 \right)}$ which is a Schwartz function. Furthermore, it is easy to compute via contour integration,
\begin{align*}
(\F \phi)(\xi) & = \int_{-\infty}^{\infty} \phi(x) e^{- i \xi x} \, \d{x} = \int_{-\infty}^\infty \exp{\left( - i \xi x - t x^2 / 2 \right) } \, \d{x} 
\\
& = \int_{-\infty}^\infty \exp{\left( - t (x + i \xi / t)^2 / 2 - \xi^2 / (2 t) \right) } \, \d{x} 
\end{align*}
Now this function is entire in $x$ and goes to zero exponentially for large real $x$. Therefore we can shift our contour up to $x' = x + i \xi / t$ without changing the integral  to get,
\[ (\F \phi)(\xi) = \exp{\left( - \xi^2 /(2 t) \right)} \int_{- \infty}^{\infty} \exp{\left( - t x'^2 / 2 \right)} \d{x'} = \exp{\left( - \xi^2 /(2 t) \right)} \cdot \sqrt{\frac{2 \pi}{t}} \]
Therefore, plugging into Poisson summation with $x = \pi$,
\[ \sum_{n \in \Z} \exp{ \left( - t (\pi + 2 \pi n)^2 / 2 \right)} = \frac{1}{\sqrt{2 \pi t}} \sum_{k \in \Z} \exp{ \left( - k^2 / (2 t) \right)} e^{i k \pi} = \frac{1}{\sqrt{2 \pi t}} \sum_{k \in \Z} (-1)^k \exp{ \left( - k^2 / (2 t) \right)} \]

\subsection{4}

Let $\S(\R^n)$ denote the Schwartz functions and $\S'(\R^n)$ the dual space of tempered distributions. For $u \in \S'(\R^n)$ let $D_j u$ denote the distributional derivative of $u$ in the $j$ coordinate. Let $H = \{ u \in L^2(\R^2) \mid D_2 u \in L^2(\R^2) \}$ equipped with the norm,
\[ || u ||_H^2 = || u ||_{L^2}^2 + || D_2  u ||_{L^2}^2 \]

\subsubsection{a}

For any two $u, v$ consider,
\[ \inner{u}{v}_H = \inner{u}{v}_{L^2} + \inner{D_2 u}{D_2 v}_{L_2} \]
Since $u,v \in L^2$ and $D_2 u, D_2 v \in L^2$ this is clearly a well-defined bracket. Furthermore it is clearly bilinear and,
\[ \inner{u}{u}_H = \inner{u}{u}_{L^2} + \inner{D_2 u}{D_2 u}_H = || u ||_{L^2} + || D_2 u ||_{L^2} \]
so it is an inner product giving the norm on $H$ and thus $H$ is a pre-Hilbert space. To show that $H$ is a Hilbert space, it suffices to prove that $H$ is complete. 
\bigskip\\
If $f_n \in H$ is Cauchy then $|| f_n - f_m ||_H \le || f_n - f_m ||_{L^2}$ and thus $f_n \in L^2(\R^2)$ is Cauchy so $f_n \to f$ to some $f \in L^2$. Furthermore, for any $\phi \in \S(\R^2)$ because strong convergence implies weak convergence,
\[ \lim_{n \to \infty} \inner{f_n}{D_2 \phi} \, \d{x} = \lim_{n \to \infty} \int f_n \, D_2 \phi \, \d{x} = \inner{f}{D_2 \phi} = \int f D_2 \phi \, \d{x} \]
However, $D_2 f_n \in L^2(\R^2)$ and therefore,
\[ \lim_{n \to \infty} \inner{f_n}{D_2 \phi} = - \lim_{n \to \infty} \inner{D_2 f_n}{\phi} = - \lim_{n \to \infty} \int D_2 f_n \, \phi \, \d{x} \]
By the Riez representation theorem, there is a function $g \in L^2(\R^2)$ such that $D_2 f_n \to g$ weakly. Furthermore, $|| D_2 f_n ||$ is bounded so this implies that $D_2 f_n \to g$ strongly. Furthermore, by above,
\[ \int f D_2 \phi \, \d{x} = - \lim_{n \to \infty} \int D_2 f_n \, \phi \, \d{x} = - \int g \phi \, \d{x} \]
and therefore $g = D_2 f$ so $f \in H$ and $f_n \to f$ in $H$ because $f_n \to f$ in $L^2(\R^2)$ and $D_2 f_n \to D_2 f$ in $L^2(\R^2)$.
\bigskip\\
Furthermore, consider $\S(\R^2) \embed H \embed L^2(\R^2)$. Because $\S(\R^2)$ is dense in $L^2(\R^2)$ and $H \embed L^2(\R^2)$ is closed we see that $\S(\R^2) \embed H$ is dense. (DOES THIS ACTUALLY WORK??)
\bigskip\\
DOES THIS ARGUMENT WORK?
It suffices to show that the continuous inclusion $H \to L^2(\R^2)$ (it is bounded because $|| u ||_{L^2} \le || u ||_H$ by definition) is closed. Indeed, for any $\phi \in \S(\R^2)$ we know,
\[ \inner{D_2 u}{\phi}_{\S} = - \int u D_2 \phi \, \d{x} = \int D_2 u \phi \, \d{x} = \inner{D_2 u}{\phi}_{L^2} \]
and therefore because $\S(\R^2) \subset L^2(\R)$ is dense,
\[ || D_2 u ||_{L^2} = \sup_{|| \phi ||_{L^2} = 1} | \inner{D_2 u}{\phi}_{L^2} | = \sup_{|| \phi ||_{L^2} = 1} | \inner{u}{D_2 \phi}_{L^2} | \]
However, if $|| \phi ||_{L^2} = 1$ then,
\[ || D_2 \phi ||_{L^2}^2 = \inner{D_2 \phi}{D_2 \phi}_{L^2} = - \inner{\phi}{D_2^2 \phi}_{L^2} \le || \phi ||_{L^2} \cdot || D_2^2 \phi ||_{L^2} \]

\subsubsection{b}

Consider the restriction map $R : \S(\R^2) \to \S(\R)$ via $(R \phi)(x) = \phi(x, 0)$. Because $\S(\R^2) \subset H$ is dense we know that any extension is uniquely defined if it exists. 
\bigskip\\
For any $u \in \S(\R^2)$ using the Fourier representation,
\[ u(x_1, 0) = \int_{\R^2} e^{i \xi x} (\F \phi)(\xi) \, \d{\xi} = \int_{\R} e^{i \xi_1 x_1} \int_\R (\F u)(\xi_1, \xi_2) \, \d{\xi_2} \d{\xi_1} \]
Therefore we use this representation which makes sense for any $L^2(\R^2)$ function for which the integral converges. Therefore we need to show that the integral converges for $u \in H$. Notice that,
\[ || u ||_{H} = || u ||_{L^2} + || D_2 u ||_{L^2} = \int_{\R} (| u(x) |^2 + |D_2 u(x)|^2) \, \d{x} \]
Then by Plancherel's theorem,
\[ || u ||_{H}^2 = \int_{\R} (1 + |\xi_2|^2) | (\F u)(\xi) |^2 \, \d{\xi} \]
because the Fourier transform of $D_2 u$ is $i \xi_2 \F u$. We now apply the finiteness of $|| U ||_H$ as follows to show that $R u$ is $L^2$. First, by Cauchy Schwartz,
\begin{align*}
| [\F (R u)](\xi_1)| & = \left| \int_{\R} \frac{(1 + |\xi_2|^2)}{(1 + |\xi_2|^2)} (\F u)(\xi_1, \xi_2)  \, \d{\xi_2} \right|
\\
& \le \int_\R \frac{(1 + |\xi_2|^2)^{\frac{1}{2}}}{(1 + |\xi_2|^2)^{\frac{1}{2}}} | (\F u)(\xi_1, \xi_2) | \, \d{\xi_2} \le \left( \int_\R (1 + |\xi_2|^2) | (\F u)(\xi_1, \xi_2) |^2 \, \d{\xi_2} \right)^{\frac{1}{2}} \left( \int_{\R} \frac{1}{1 + |\xi_2]^2}  \, \d{\xi_2} \right)^{\frac{1}{2}}
\end{align*} 
Then the following integral is finite,
\[ C = \left( \int_{\R} \frac{1}{1 + |\xi_2]^2}  \, \d{\xi_2} \right)^{\frac{1}{2}} \]
showing that,
\[ | [\F (R u)](\xi_1)| \le C \left( \int_\R (1 + |\xi_2|^2) | (\F u)(\xi_1, \xi_2) |^2 \, \d{\xi_2} \right)^{\frac{1}{2}} \]
Therefore, by Plancherel's theorem,
\begin{align*}
|| R u ||_{L^2}^2 = \int_{\R} | [\F (R u)](\xi_1)|^2 \, \d{\xi_1} \le C \int_{\R^2} \int_\R (1 + |\xi_2|^2) | (\F u)(\xi_1, \xi_2) |^2 \, \d{\xi_2} \, \d{\xi_1} = C^2 || u ||_H^2
\end{align*}
proving that,
\[ || R u ||_{L^2} \le C || u ||_H \]
so $R$ is a bounded and thus continuous operator $H \to L^2$.

\subsection{5} 

Let $X$ be a Banach space and $B = \{ x \in X \mid || x ||_X \le 1 \}$ be the unit ball in $X$. 

\subsubsection{a}

Let $Z \subset X$ be finite dimensional. Then choose a basis $\{ e_1, \dots, e_n \}$ of $Z$. By Hahn-Banach we can choose $\ell_i \in X^*$ sushc htat $\ell_i(e_j) = \delta_{ij}$. Then consider, \[ W = \bigcap_{i = 1}^n \ker{\ell_i} \]
is closed. Furthermore, clearly $W \cap Z = \{ 0 \}$ since if $z \in Z$ then $z = a_1 e_1 + \cdots + a_n e_n$ so $\ell_i(z) = a_i = 0$ implies that $z = 0$. Likewise, for any $x \in X$ consider,
\[ a_i = \ell_i(x) \]
then take,
\[ y = x - (a_1 e_1 + \cdots + a_n e_n) \]
which satisfies $\ell_i(y) = \ell_i(x) - a_i = 0$ and thus $y \in Z$ so we see that algebraically,
\[ X = Z \oplus W \]
with $Z$ and $W$ closed. Furthermore, the maps, 
\[ x \mapsto \ell_1(x) e_1 + \cdots + \ell_n(x) e_n \quad \text{and} \quad x \mapsto x - \ell_1(x) e_1 + \cdots + \ell_n(x) e_n \]
are clearly continuous and are the projectons onto $Z$ and $W$ proving that $X = Z \oplus W$ in the category of Banach spaces.

\subsubsection{b}

Suppose that $X$ is infinite dimensional. Choose $x_0 \in B$ and then $V_0 = \vspan{x_0}$. Now recursively, by Riesz' lemma, we can choose $x_{n+1} \in B$ such that $|| x_{n + 1} - u || \ge \tfrac{1}{2}$ for all $u \in V_n$ then let $V_{n+1} = \vspan{x_0, \dots, x_n}$. Now, notice that $\{ x_n \} \subset B$ is a sequence such that,
\[ || x_{m} - x_n || \ge \tfrac{1}{2} \]
for all $n,m$ and thus $\{ x_n \}$ cannot have a Cauchy and thus convergent subsequence. Therefore $B$ is not sequentially compact but $X$ is a metric space so thus $B$ cannot be compact.

\subsubsection{c FUCK YOU GOD}

Let $X, Y$ be Banach spaces, $X \subset Y$ with the inclusion map $\iota : X \to Y$ continuous and compact. Let $T \in \L(X, Y)$, and suppose that for all $x \in X$ we know,
\[ || x ||_X \le C \left( || T x ||_Y  + || x ||_Y \right) \]
Let $K = \ker{T}$. Then for any $x \in K$,
\[ || x ||_X \le C || x ||_Y \]
and therefore $\iota|_K : K \to Y$ has closed image and is injective. Therefore, by the bounded inverse theorem ($\im{\iota|_K} \subset Y$ is closed and thus a Banach space) $\iota|_K$ is invertible onto its image and therefore $\id_K$ is compact since $\iota|_K$ is compact so the unit ball of $K$ is compact and thus $\dim{K} < \infty$.
\bigskip\\
Because $\ker{T}$ is finite dimensional, it is complemented $X = V \oplus \ker{T}$. To show that $\im{T}$ is closed it suffices to show that $T |_V$ is bounded below. Suppose to the contrary that,
\[ \inf_{B \cap V} || T x || = 0 \]
then there would be a sequence $x_n \in B \cap V$ such that $|| T x_n || < \frac{1}{n}$. Because this is bounded and $\iota$ is compact $\{ \iota x_n \}$ has a convergent subsequence in $Y$ 


\section{Spring 2015 Part II}

\subsection{1}

\begin{exercise}
Suppose that $f \in L^1([0,1])$. Prove that there are nondecreasing sequences of continuous functions $\{ \varphi_k \}$ and $\{ \psi_k \}$ on $[0,1]$ such that for almost every $x \in [0,1]$ both $\varphi_k(x)$ and $\psi_k(x)$ are bounded sequences and moreover,
\[ f(x) = \lim_{k \to \infty} \varphi_k(x) - \lim_{k \to \infty} \psi_k(x) \]
\end{exercise}

First we consider $f \in L^1([0,1])$ nonegative. By Lusin's theorem, for each $\epsilon > 0$ there exists a continuous function $f_\epsilon$ such that $f_\epsilon = f$ on $E_\epsilon$ where $\mu(E_\epsilon) > 1 - \epsilon$ and,
\[ \sup_{x \in [0,1]} f_\epsilon(x) \le \sup_{x \in [0,1]} f(x) \]


Consider $f_n = \min\{f, n \}$. 

\subsection{2}

Appears also as Spring 2011 Part II Q2. 

\begin{exercise}
Let $X$ be a $\C$-vectorspace and $\F$ a vector subspace of $\Hom{\C}{\F}{\C}$ and equip $X$ with the $\F$-weak topology. Show that the only continuous linear maps $X \to \C$ are those in $\F$.
\end{exercise}

Let $\varphi : X \to \C$ be a linear continuous map. Then $\varphi(B_\epsilon^{-1}(0))$ is open so it must contain some open of the form,
\[ U = \bigcap_{i = 1}^n f_i^{-1}(B_\epsilon(0)) \]
where $f_i \in \F$. However, this implies that,
\[ \bigcap_{i = 1}^n \ker{f_i} \subset \ker{\varphi} \]
and then by a lemma in Rudin this means that,
\[ \varphi = \alpha_1 f_1 + \cdots + \alpha_n f_n \]
for some constants. Therefore $\varphi \in \F$.

\subsection{3}

\begin{exercise}
Consider the partial sums,
\[ (S_N f)(x) = \sum_{|n| \le N} c_n e^{inx} \]
of the Fourier series of continuous functions $f \in C(\R)$ where $c_n = (\F f)_n$. Recall that $S_N f$ is given by the convolution of the Direhlet kernel $D_n$ with $f$.
\begin{enumerate}
\item Show that $|| D_n ||_{L^1} \to \infty$ as $N \to \infty$ 
\item Show that there exists a continuous function $f \in C(\T)$ such that the Fourier series of $f$ does not converge uniformly to $f$ i.e. $S_N f$ does not converge uniformly to $f$ as $N \to \infty$.
\end{enumerate}
\end{exercise}

\subsubsection{a}

Recall that,
\[ D_n(x) = \frac{\sin{(n + \tfrac{1}{2}) x}}{2 \pi \sin{(\tfrac{1}{2} x)}} \]
Then,
\[ || D_n ||_{L^1} = \int_{-\infty}^\infty \left| \frac{\sin{(n + \tfrac{1}{2}) x}}{2 \pi \sin{(\tfrac{1}{2} x)}} \right| \, \d{x} = (n + \tfrac{1}{2})^{-1} \int_{-\infty}^{\infty} \left| \frac{\sin{u}}{2 \pi \sin{\left( (2 n + 1)^{-1} u \right)}} \right| \, \d{u}  \]
Notice that if $|u| < 1$ then $|\sin{u}| \ge \tfrac{1}{2} |x|$ and therefore,
\[ || D_n ||_{L^1} \ge  (n + \tfrac{1}{2})^{-1} \int_{-(2 n + 1)}^{(2 n + 1)} \left| \frac{\sin{u}}{2 \pi \sin{\left( (2 n + 1)^{-1} u \right)}} \right| \, \d{u} \ge \int_{-(2 n + 1)}^{(2n + 1)} \left| \frac{\sin{u}}{u} \right| \, \d{u}  \]
However, it is well known that $\frac{\sin{u}}{u}$ is not $L^1$ (basically this follows from the divergence of the integral of $u^{-1}$) and therefore as $n \to \infty$,
\[ || D_n ||_{L^1} \to \infty \]

\subsubsection{b (HOW TO DO THIS!!)}

Suppose that for every continuous $f \in C(\T)$ we have $S_N f \to f$ uniformly.


\subsection{4}

Suppose that $H$ is a Hilbert space and $U \in \L(H)$ is unitary meaning $U U^* = I = U^* U$.

\subsubsection{a}

Notice that $\ker{(I - U^*)} \oplus \overline{\im{(I - U)}}$ because this holds for any bounded operator. Then we need to show that $\ker{(I - U^*)} = \ker{(I - U)}$. Indeed,
\[ U x = x \iff U^* U x = U^* x \iff U^* x = x \]

\subsubsection{b}

Let $P$ be an orthogonal projection to $\ker{(I - U)}$. Let,
\[ S_n = \frac{1}{n} \sum_{j = 0}^{n-1} U^j \]
Suppose that $S_n x \to y$ converges to a limit. Then, 
I claim that $(I - U) S_n \to 0$ in operator norm. Indeed, $S_n (I - U) = (I - U) S_n = \frac{1}{n} (I - U^n)$ so,
\[ ||(I - U) S_n|| \le \tfrac{1}{n} (|| I || + || U^n ||) \to 0 \]
because $U^n$ is unitary so $|| U^n || = 1$. Therefore, if $S_n x \to y$ converges then $y \in \ker{(I - U)}$.
\bigskip\\
Furthermore, I claim that $x - y \in (\ker{(I - U)})^\perp$. Indeed, for $z \in \ker{(I - U)}$ then,
\[ \inner{y}{z} = \lim_{n \to \infty} \inner{S_n y}{z} = \lim_{n \to \infty} \frac{1}{n} \sum_{j = 0}^{n-1} \inner{U^j x}{z} = \lim_{n \to \infty} \frac{1}{n} \sum_{j = 0}^{n-1} \inner{x}{z} = \inner{x}{z} \]
because $\inner{U^j x}{z} = \inner{x}{(U^*)^j z}$ but $(U^*)^j z = z$ because $U z = z$ and thus $U^* z = z$ and the same for every power. Therefore, if $S_n x \to y$ converges then $y = P x$ because $P$ is determined by $P x \in \ker{(I - U)}$ and $x - P x \in (\ker{(I - U)})^\perp$. Therefore, we just need to show that $S_n x$ converges.
\bigskip\\
For $x \in \ker{(I - U)}$ it is clear that $S_n x = x$ for all $n$ so there is nothing to prove. For $x \in \im{(I - U)}$ write $x = (i - U) z$ and then,
\[ S_n x = S_n (I - U) z = \tfrac{1}{n} (z - U^n z) \]
and therefore,
\[ || S_n x || \le \tfrac{2}{n} || z || \]
because $||U^n || = 1$. Therefore, for fixed $z$, the sequence $S_n x \to 0$ as $n \to \infty$. Now let $x \in \overline{(I - U)}$ and $x_k \to x$ be a sequence converging to $x$ with $x_k = (I - U) z_k$. Consider,
\[ || S_n x - S_n x_k || \le || S_n || \cdot || x - x_k || \le || x - x_k || \]
because,
\[ || S_n || \le \frac{1}{n} \sum_{j = 0}^{n-1} || U^j || = 1 \]
Taking the limit on both sides,
\[ \limsup_{n \to \infty} || S_n x - S_n x_k || \le || x - x_k || \]
Then,
\[ \limsup_{n \to \infty} || S_n x || \le \limsup_{n \to \infty} || S_n x - S_n x_k || + \limsup_{n \to \infty} || S_n x_k || \le || x - x_k || \]
However, as $k \to \infty$ we have $|| x - x_k || \to 0$ and therefore $S_n x \to 0$ in norm. Now we can write any $x \in H$ as $x = x_1 + x_2$ for $x_1 \in \ker{(I - U)}$ and $x_2 \in \overline{\im{(I - U)}}$. We know that $S_n x_1 \to x_1$ and $S_n x_2 \to 0$ converge and therefore $S_n x \to x_1 = P x$ converges. Thus $S_n \to P$ in the strong operator topology.

\subsubsection{c}

Consider $H = \ell^2$ and the unitary operator $U : H \to H$ defined by $(U x)_n = e^{\pi i / n} x_n$. Then $I - U$ is the multiplication by $f(n) = 1 - e^{\pi i / n}$ function and since $f(n) \to 0$ as $n \to \infty$ we see that $I - U$ is actually compact. Now, $I - U$ is injective because $f(n) \neq 0$ for all $n$ but we know that $I - U$ is not surjective but it does have dense image because $H = \ker{(I - U)} \oplus \overline{\im{(I - U)}}$. Therefore $P = 0$. I claim that $S_n$ does not converge in norm.
\bigskip\\
Indeed, we compute
\[ || S_n - P || = || S_n || = \sup_{|| x || = 1} || S_n x || \]
Now,
\[ || S_n x ||^2 = \frac{1}{n^2} \sum_{k = 1}^\infty \left| \frac{1 - e^{\pi i k / n}}{1 - e^{\pi i/n}} \right|^2 |x_k|^2 \]
For any $n$ we can choose $k = n$ and then 
\[ \left| \frac{1 - e^{\pi i k / n}}{1 - e^{\pi i/n}} \right|^2 = \frac{4}{| 1 -e^{\pi i /n} |^2} \to \frac{n^2}{\pi^2} \]
for large $n$. Thus, for $x = e_n$,
\[ || S_n e_n || \to \pi \]
as $n \to \infty$ and therefore,
\[ \liminf_{n \to \infty} \sup_{|| x || = 1} || S_n x || \ge \pi \]
proving that $S_n$ cannot converge in the operator norm.


\subsection{5}

\begin{exercise}
Consider the integral,
\[ I(r) = \int_\R e^{i r  \sin{\theta}} \varphi(\theta) \d{\theta} \]
for $\varphi \in C^{\infty}(\T)$. 
Show that for some $C > 0$,
\[ I(r) \le C r^{-\frac{1}{2}} \]
for all $r \ge 1$.
\end{exercise}

Suppose first that $\varphi$ is supported away from $0$ and $\pi$. Then, there must be some sufficently small $\epsilon > 0$ such that $\varphi(\theta) \neq 0$. Therefore, 
\[ \frac{\varphi(x)}{\sin{x}} \]
is bounded because $\varphi(x)$ is zero in a neighborhood on the zeroes of $\sin{x}$. Then,
\[ |I(r)| =  - \int_\T e^{i r \cos{\theta}} \frac{\varphi(\theta)}{\sin{x}} \, \d{(\cos{\theta})}  \]
Then integrating by parts,
\[ I(r) = \int_{\T} \frac{1}{i r} \left( \frac{\varphi(\theta)}{\sin{x}} \right)' e^{ir \cos{\theta}} \d{\cos{\theta}} \]
However, $\varphi$ and $\varphi'$ are uniformly bounded so we see that,
\[ | I(r) | < C / r \]
Because,
\[ \frac{\varphi(\theta)}{\sin{x}} \]
is a smooth function. 
\bigskip\\
Now don't assume any condition on the support of $\varphi$. Let $\chi$ be a sum of bump functions of radius $\epsilon > 0$ about $0$ and about $\pi$. Then $\varphi = (1 - \chi) \varphi + \chi \varphi$ so $\varphi_1 = (1 - \chi) \varphi$ is supported away from $0$ and $\pi$ and $\varphi_2 = \chi \varphi$ has support contained in $B_\epsilon(0) \cup B_{\epsilon}(\pi)$. Now,
\begin{align*}
I(r) & = \int_{\T} e^{i r \cos{\theta}} \varphi(\theta) \, \d{\theta} = \int_{\T} e^{i r \cos{\theta}} \varphi_1(\theta) \, \d{\theta} + \int_{\T} e^{i r \cos{\theta}} \varphi_2(\theta) \, \d{\theta}
\\
& = \int_{\T} e^{i r \cos{\theta}} \varphi_1(\theta) \, \d{\theta} + \int_{-\epsilon}^{\epsilon} e^{i r \cos{\theta}} \varphi_2(\theta) \, \d{\theta} + \int_{\pi -\epsilon}^{\pi + \epsilon} e^{i r \cos{\theta}} \varphi_2(\theta) \, \d{\theta}
\end{align*}
However, by the previous work, since $\phi_1$ is supported away from $\{ 0, \pi \}$,
\[ \left| \int_{\T} e^{i r \cos{\theta}} \varphi_1(\theta) \, \d{\theta} \right| \le \frac{C}{r} \]
For sufficiently small $\epsilon > 0$ we know that $|\cos{\theta} - \pm (1 - \frac{1}{2} \theta^2)| < q \epsilon^3$ for all $\theta \in B_\epsilon(0) \cup B_{\epsilon}(\pi)$. Therefore,
\[ \left| \int_{-\epsilon}^{\epsilon} e^{i r \cos{\theta}} \varphi_2(\theta) \, \d{\theta} - \int_{-\epsilon}^{\epsilon} e^{i r (1 - \theta^2/2)} \varphi_2(\theta) \, \d{\theta} \right| < C' \epsilon^3 \]
Furthermore,
\[ \int_{-\epsilon}^{\epsilon} e^{i r (1 - \theta^2/2)} \varphi_2(\theta) \, \d{\theta} = e^{i r} \int_{-\epsilon}^{\epsilon} e^{- i r \theta^2/2} \varphi_2(\theta) \d{\theta} = \frac{e^{i r}}{\sqrt{r}} \int_{-\epsilon^2/r}^{\epsilon^2/r} e^{-i s / 2} \varphi_2(s) \, \tfrac{1}{2} s^{-\frac{1}{2}} \d{s} \]
Therefore as $r \to \infty$ we can take $\epsilon = \sqrt{r}$ which becomes sufficiently small and since the integral then does not depend on $\epsilon$ or $r$ we find,
\[ \left| \int_{-\epsilon}^{\epsilon} e^{i r (1 - \theta^2/2)} \varphi_2(\theta) \, \d{\theta} \right| \le C'' r^{-\frac{1}{2}} \]
The integral aroung $\pi$ gives the same contribution. Therefore putting everything together,
\[ |I(r)| \le \tilde{C} r^{-\frac{1}{2}} \]
because $r^{-\frac{1}{2}}$ dominates $r^{-1}$

\section{Fall 2015 Part I}

\subsection{1}

\begin{exercise}
Let $A$ be a Borel set in $\R$ wuth the property that if $x \in A$ and $y - x \in T = \{ \frac{a}{10^n} \mid n \in \N, a \in \Z \}$ (i.e $T$ consists of all finite decimal expansions) then $y \in A$. Prove that either $\mu(A) = 0$ or $\mu(\R \setminus A) = 0$ where $\mu$ is the Lebesgue measure.
\end{exercise}

Notice that $B = \R \setminus A$ also has the property that if $x \in B$ and $y - x \in T$ then $y \in B$ because if $y \in A$ then $x - y \in T$ so $x \in A$ contradicting $x \in B$. Therefore it suffices to show that if a Borel set has this property and $\mu(A) > 0$ then every point is a Lebesgue point. In that case if $\mu(A) > 0$ and $\mu(B) > 0$ then $A$ and $B$ would share Lebesgue points which is impossible.
\bigskip\\
Suppose that $\mu(A) > 0$ then it has some Lebesgue point $x \in A$. Now, for any $y \in \R$ I claim that $y$ is a Lebesgue point. For any $\epsilon > 0$ there is some $\delta > 0$ such that for all $r < \delta$,
\[ \left| 1 - \frac{\mu(A \cap B_r(x))}{\mu(B_r(x))} \right| < \frac{\epsilon}{2} \]
Because $T \subset \R$ is dense, there is $t \in T$ with $|y - x - t| < q$ and then,
\[ \mu(A \cap B_r(y)) = \mu(A + t \cap B_r(y + t)) \]
but $A + t = A$ and $B_{r-q}(x) \subset B_{r}(y + t) \subset B_{r + q}(x)$. Therefore,
\[ (r - q) (1 - \tfrac{1}{2} \epsilon)  \le \mu(A \cap B_{r - q}(x)) \le \mu(A \cap B_r(y + t)) \le \mu(A \cap B_{r + q}(x)) \le (r + q)(1 + \tfrac{1}{2} \epsilon) \]
and therefore,
\[ \left| 1 - \frac{\mu(A \cap B_r(y))}{\mu(B_r(y))} \right| \le \frac{2q}{r} + \epsilon   \]
whenever $0 < r - q < r < r + q < \delta$. For any $0 < r < \delta$ we can choose $q < \min \{ \tfrac{1}{2} r \epsilon, r, r - \delta \}$ such that $0 < r - q < r < r + q < \delta$ and thus,
\[ \left| 1 - \frac{\mu(A \cap B_r(y))}{\mu(B_r(y))} \right| < \epsilon + \epsilon \]
proving that $y$ is a Lebesgue point of $A$. Therefore $A$ and $B$ cannot both have a Lebesgue point otherwise some $x \in A \cap B$ would be a Lebesgue point of both in which case,
\[ \mu(A \cap B_\delta(x)) > 2 \delta (1 - \epsilon) \]
and likewise,
\[ \mu(B \cap B_\delta(x)) > 2 \delta ( 1 - \epsilon) \]
but then because $A$ and $B$ are disjoint and $A \cup B = \R$ we have,
\[ \mu(B_\delta(x)) = \mu(A \cap B_\delta(x)) + \mu(B \cap B_\delta(x)) > 4 \delta (1 - \epsilon) > 2 \delta \]
giving a contradiction. Thus either $\mu(A) = 0$ or $\mu(B) = 0$.

\subsection{2}

Let $T$ be a non-zero compact operator on a Hilbert space $H$.

\subsubsection{a}

Consider the Volterra operator $T : L^2([0,1]) \to L^2([0,1])$ defined by,
\[ (T f)(x) = \int_0^x f(x) \, \d{x} \]
Then, it is clear that,
\[ | (T f)(x) | = \left| \int_0^x f(x) \, \d{x} \right| \le \int_0^x | f(x) | \, \d{x} \le || f ||_1 \le || f ||_2 \]
where $|| f ||_1 \le || f ||_2$ by Holder since $|| 1 ||_1 = 1$. Thus $T f$ is bounded. Then,
\[ | (T^{n+1} f)(x) | \le \frac{x^n}{n!} || f ||_2 \]
and therefore,
\[ || T^{n+1} f||_2 \le || (T^{n+1} f) ||_{\infty} \le \frac{|| f ||_2}{n!} \]
This implies that $|| T^{n+1} || \le \frac{1}{n!}$ and thus,
\[ r(T) = \lim_{n \to \infty} || T^n ||^{\frac{1}{k}} = 0 \]
because $\frac{1}{(n!)^{\frac{1}{n}}} \to 0$. Therefore, $\sigma(T) = \{ 0 \}$ because the spectrum is nonempty. Furthermore, $T$ is compact by general facts about integral operators. (FIND BETTER EXAMPLE!!)

\subsubsection{b}

Suppose that $T$ is self-adjoint. Then consider,
\[ || T ||^2 = \sup_{|| x || = 1} \inner{T x}{T x} = \sup_{|| x || = 1} \inner{T^2 x}{x} \le \sup_{|| x || = 1} || T^2 x || \le || T^2 || \]
However, $|| T^2 || \le || T ||^2$ always so $|| T^2 || = || T ||^2$. The same argument shows that for any even power $|| T^n || = || T ||^n$ and therefore,
\[ r(T) = \lim_{n \to \infty} || T^n ||^{\frac{1}{n}} = || T || \]
Because $T \neq 0$ we have $|| T || \neq 0$ and thus the spectrum of $T$ must not be equal to $\{ 0 \}$.

\subsection{3}

\begin{exercise}
Show that there exists a compactly supported $C^{\infty}$ function $\phi$ on $\R$ with $\phi \ge 0$ and $\phi(0) > 0$ and $(\F \phi)(\xi) \ge 0$.
\end{exercise}

Notice that $\F \phi$ is real iff $\phi$ is even. It suffices to show that there exists a compactly supported smooth function $\psi$ with $\psi \ge 0$ and $\psi(0) > 0$ and $\F \psi$ real. Because then we can take $\phi = \psi * \psi$ the convolved square. It is clear that $\phi \ge 0$ and $\phi(0) > 0$ and $\phi$ is compactly supported because $\psi$ satisfies these properties. Furthermore, $(\F \phi) = (\F \phi)^2 \ge 0$ because $(\F \phi)$ is real. Therefore, consider the bump function,
\[ \psi(x) = 
\begin{cases}
1 & x \in [-1,1]
\\
1 - e^{-\frac{1}{(|x| - 1)^2}} & |x| > 1
\end{cases} \]
which is smooth and obviously $\psi \ge 0$ and $\psi(0) = 1$ and $\psi$ is compactly supported. Furthermore, $\psi$ is compactly supported and smooth and therefore rapidly decreasing so $\psi \in \S(\R)$. Furthermore, $\psi$ is symmetric so $\F \psi$ is real and thus we may take $\phi = \psi * \psi$.


\subsection{4 (ACTUALLY DO THIS)}

\subsubsection{a}

Consider,
\begin{align*}
\frac{\partial^2 f}{\partial y \partial x} & = \lim\limits_{h_y \to 0} \frac{\lim\limits_{h_x \to 0} \frac{f(x + h_x, y + h_y) - f(x, y + h_y)}{h_x} - \lim\limits_{h_x \to 0} \frac{f(x + h_x, y) - f(x,y)}{h_x}}{h_y} 
\\
& = \lim_{h_y \to 0} \lim_{h_x \to 0} \frac{f(x + h_x, y + h_y) - f(x, y+ h_y) - f(x + h_x, y) + f(x, y)}{h_x h_y}
\end{align*}
Thus it suffices to show that we can swap the limits. We know that all the limits exist and furthermore that the second derivatives are continuous. This implies that the first derivates are bounded on some neighborhood of $(x,y)$. This implies that convergence of the limit,
\[  \lim_{h_x \to 0} \frac{f(x + h_x, y + h_y) - f(x, y+ h_y) - f(x + h_x, y) + f(x, y)}{h_x h_y} \]
can be made uniform in $h_y$. (SHOW THIS!!)


\subsection{5 (DO THIS!!)}


\section{Fall 2015 Part II}

\subsection{1}

\begin{exercise}
Let $f \in L^1([0,1])$ and suppose that $\int f \varphi^{(n)} \, \d{x} = 0$ for every $\varphi \in C^{\infty}_c((0,1))$ where $\varphi^{(n)}$ is the $n$-th derivative. Show that $f$ is a polynomial of degree at most $n - 1$.
\end{exercise}

For any $f \in L^1([0,1])$ we can approximate $f$ by compactly supported smooth functions as follows. Let $\psi_0$ be a smooth bump function such that $\int \psi_0 \, \d{x} = 1$. Then define $\psi_\epsilon = \epsilon^{-1} \psi_0(x / \epsilon)$ such that its area remains $1$. Now we regulate $f_\epsilon = f * \psi_\epsilon$ gives a family of smooth functions with $f_\epsilon \to f$ in $L^1$. Because the convolution is defined on $\R$ we extend our functions by zero to the entire real line (notice we can extend $\varphi$ by zero because its compact support must be properly contained in $(0,1)$ and thus it must be zero on some neighborhood of $0$ and $1$). Since we have convergence in $L^1$, this shows that,
\[ \lim_{\epsilon \to 0} \int_\R f_\epsilon \varphi^{(n)} \, \d{x} = \int_\R f \varphi^{(n)} \, \d{x} = 0 \]
for all $\varphi \in C^\infty_c((0,1))$. However, since $f_\epsilon$ is smooth, 
\[ \int_\R f_\epsilon \varphi^{(n)} \, \d{x} = (-1)^n \int_\R f_\epsilon^{(n)} \varphi \, \d{x} \]
where the boundary term vanishes because these functions are zero outside $[0,1]$. Furthermore,
\begin{align*}
\int_\R f_\epsilon \varphi^{(n)} \, \d{x} &= \int_\R \left( \int_\R f(y) \psi_\epsilon(x - y) \, \d{y} \right) \varphi^{(n)}(x) \, \d{x} = \int_\R f(y) \left( \int_\R \psi_\epsilon(x - y) \varphi^{(n)}(x) \, \d{x} \right) \, \d{y}
\\
& = \int_\R f(y) \left( \int_\R \psi_\epsilon(x') \varphi^{(n)}(x'+y) \, \d{x'} \right) \, \d{y} = \int_\R f(y) \, \partial_y^n \left( \int_\R \psi_\epsilon(x') \varphi(x' + y) \, \d{x'} \right) \, \d{y}
\\
& = \int_\R f(y) (\psi_\epsilon *' \varphi)^{(n)}(y) \, \d{y} = \int_0^1 f (\varphi_\epsilon *' \varphi)^{(n)} \, \d{y} = 0
\end{align*}
where I can pull out the derivative because $\psi_\epsilon(x') \varphi(x' + y)$ is smooth and $*'$ refers to the convolution with reversed signs as seen above and $\psi_\epsilon *' \varphi \in C^\infty_c((0,1))$ for small enough $\epsilon$ such that $B_\epsilon(\supp{}{\varphi}) \subset (0,1)$ which is possible because $\supp{}{\varphi}$ is is compact and thus closed in $\R$ and therefore can be inflated by some $\epsilon > 0$ to still remain in $(0,1)$. Therefore, the last equality holds by assumption on $f$. Therefore we find,
\[ \int_\R f_\epsilon^{(n)} \varphi \, \d{x} = 0 \]
for all $\epsilon > 0$ and all test functions $\varphi \in C^{\infty}_c((0,1))$. Because $f_\epsilon$ is smooth, this implies that $f_\epsilon$ is a polynomial of degree at most $n-1$ on $[0,1]$ (it is smooth and has zero $n$-th derivative). Then since $f_\epsilon \to f$ converges in $L^1$ but $f_\epsilon$ belongs to the finite dimensional (and thus closed) subspace of polynomials of degree at most $n-1$ we must have that $f$ also belongs to this subspace of $L^1([0,1])$ and therefore $f$ is a polynomial of degree at most $n-1$.


\subsection{2}

\begin{exercise}
Let $H \subset \ell^2$ be the subspace of sequences $\{ x_n \}$ such that,
\[ \sum_{n = 1}^\infty n^2 | x_n |^2 < \infty \]
Show that $H$ is of first category in $\ell^2$.
\end{exercise}

Notice that $H = \im{T}$ where $T : \ell^2 \to \ell^2$ is the operator $(T x)_n = \frac{1}{n} x_n$. Now, because $\frac{1}{n} \to 0$ we know that $T$ is compact (it is the limit of finite rank operators) and thus $T(B)$ is precompact. However, we see that,
\[ \im{T} = \bigcup_{n = 1}^\infty n T(B) \]
so it suffices to show that $T(B)$ is nowhere dense. This is because $\overline{T(B)}$ is compact and thus cannot contain any ball $D$ because then $\overline{D}$ would be closed in the compact $\overline{T(B)}$ and thus compact but $\ell^2$ is infinite dimensional so its closed unit ball (homeomorphic to $\overline{D}$) cannot be compact. Thus , $\overline{T(B)}$ has empty interior so $T(B)$ is nowhere dense. 
\bigskip\\
Remark we can apply the general lemma about meager subspaces.

\subsection{3}

\subsubsection{a}

Consider the sequence of operators,
\[ (A_n x)_k = 
\begin{cases}
x_k & k \ge n
\\
0 & k < n
\end{cases} \]
Then it is clear that $A_n \to 0$ in the strong operator topology because,
\[ || A_n x || = \sum_{i = n}^\infty | x_i |^2 \]
which goes to zero as $x \to \infty$ because,
\[ || x || = \sum_{i = n}^\infty | x_i |^2 < \infty \]
However, $A_n e_n = e_n$ and thus $|| A_n || \ge 1$ so $A_n$ cannot converge to zero in norm.
\bigskip\\
Consider the sequence of operators $A_n \in \L(\ell^2)$ given by powers of the right-shift operator $A_n = R^n$ where,
\[ (R x)_k = 
\begin{cases}
x_{k-1} & k \ge 1
\\
0 & k = 0
\end{cases} \]
Then I claim that $A_n \to 0$ in the weak operator topology. Indeed, $A_n e_i = e_{i + n}$ and therefore,
\[ A_n x = \sum_{i = 0}^\infty x_i e_{i + n} \]
furthermore, for any $y \in \ell^2$ write,
\[ y = \sum_{i = 0}^\infty y_i e_i \]
and therefore,
\[ \inner{y}{A_n x} = \sum_{i = 0}^\infty \overline{y_{i + n}} x_i = \inner{L^n y}{x} \]
where $L$ is the left-shift operator $(L y)_k = y_{k + n}$. Furthermore, I claim that $L^n \to 0$ in the strong operator topology (but like above not in norm) because,
\[ || L^n y || = \sum_{i = 0}^\infty | y_{i + n} |^2 = \sum_{i = n}^\infty | y_i |^2 \]
which tends to zero because,
\[ || y || = \sum_{i = 0}^\infty | y_i |^2 < \infty \] 
however, $L^n e_n = e_0$ and thus $|| L^n || \ge 1$ so $L^n$ does not converge in norm. We saw that,
\[ \lim_{n \to \infty} \inner{y}{R^n x} = \lim_{n \to \infty} \inner{L^n y}{x} = 0 \]
because $L^n y \to 0$ in norm. Therefore, $R^n \to 0$ in the weak operator topology. However, $R^n e_0 = e_n$ and $|| R^n e_0 || = || e_n || = 1$ and therefore $R^n e_0$ does not converge to zero in norm showing that $R^n$ cannot converge in the strong operator topology.

\subsubsection{b}

Consider the map $a : A \to A^*$. To show that $a$ is continuous in the norm topology it suffices to show that $a$ is bounded. Indeed,
\[ || A^* ||^2 = \sup_{|| x || = 1} || A^* x ||^2 = \sup_{|| x || = 1} \inner{A^* x}{A^* x} = \sup_{|| x || = 1} \inner{x}{A A^* x} \le \sup_{|| x || = 1} || x ||^2 \cdot || A || \cdot || A^* || = || A || \cdot || A^* || \]
and therefore $|| A^* || \le || A ||$ showing that $|| a || \le 1$ so $a$ is continuous in the norm topology.
\bigskip\\
Furthermore, to show that $a$ is continuous in the weak operator topology it suffices to show that for all $x, y \in H$ the linear map $A \mapsto \inner{x}{A^* y}$ is continuous in the weak operator topology on $\L(H)$ but $\inner{x}{A^* y} = \inner{A x}{y} = \overline{\inner{y}{A x}}$ and $A \mapsto \inner{y}{A x}$ is continuous in the weak operator topology by definition. 
\bigskip\\
Finally, we need to show that $a$ is not continuous in the strong operator topology. This means that for some $x \in H$ the map $A \to A^* x$ is not continuous as $\L(H) \to H$ with the strong operator topology on $\L(H)$. Indeed we already showed that $L^n \to 0$ strongly but its adjoint sequence $R^n$ does not converge strongly although it does weakly. Therefore, $a$ cannot be continuous in the strong operator topology.

\subsection{4}


\section{Spring 2016 Part I}

\subsection{1}

\subsubsection{a}

If $\mu(X) < \infty$ then,
\[ L^\infty(X, \mu) \subset L^2(X, \mu) \subset L^1(X, \mu) \]
This is because, by H\"{o}lder, if,
\[ \frac{1}{p} + \frac{1}{q} = \frac{1}{r} \]
then
\[ || f ||_r \le || f ||_p || 1 ||_q = || f ||_p \cdot \mu(X)^{\frac{1}{r}} \]
and therefore, if $r \le p$ then $|| f ||_r$ is finite when $|| f ||_p$ is finite proving the inclusions.
\bigskip\\
Now suppose that $\mu(X) = \infty$ but $\mu$ is $\sigma$-finite we want to know if,
\[ L^1(X, \mu) \subset L^2(X, \mu) \subset L^\infty(X, \mu) \]
is possible.
If we let $X = \N$ with the counting measure (which is $\sigma$-finite) then $L^p(X, \mu) = \ell^p$. Furthermore, we know that,
\[ \ell^1 \subset \ell^2 \subset \ell^\infty \]
because, if a sequence is summable then it is square summable since,
\[ \sum_{i = 0}^\infty | x_i | < \infty \implies \lim_{i \to \infty} x_i = 0 \implies \sum_{i = 0}^\infty = \sum_{i = 0}^N |x_i|^2 + \sum_{i = N+1}^\infty | x_i |^2 \le \sum_{i = 0}^N |x_i|^2 +  \sum_{i = 0}^\infty | x_i | < \infty \]
where $N$ is the point at which $| x_i | < 1$. Furthermore, square summable sequences are convergent to zero and therefore bounded so $\ell^2 \subset \ell^\infty$.

\subsubsection{b}

Let $E_k$ be measurable sets with,
\[ \sum_{k = 1}^\infty \mu(E_k) < \infty \]
Consider, 
\[ E = \limsup_{k \to \infty} E_k = \bigcap_{k = 1}^\infty \bigcup_{n \ge k} E_n = \{ x \in X \mid x \in E_k \text{ infinitely often } \} \]
from the union intersection definition this is clearly measurable since the $E_k$ are measurable. Notice that,
\[ \mu \left( \bigcup_{n \ge k} E_n \right) \le \sum_{n \ge k} \mu(E_n) < \infty \]
Furthermore, the measure is continuous with respect to a decreasing sequence of finite measure sets and thus,
\[ \mu(E) = \lim_{k \to \infty} \mu\left( \bigcup_{n \ge k} E_n \right) \le  \lim_{k \to \infty} \sum_{n \ge k} \mu(E_n) \]
However since,
\[ \sum_{k = 1}^\infty \mu(E_k) < \infty \]
we see that the partial sums tend to zero so,
\[ \mu(E) \le \lim_{k \to \infty} \sum_{n \ge k} \mu(E_n) = 0 \]

\subsection{2 (OKAY DO THIS ONE)}

\begin{exercise}
Show that any weakly convergent $\ell^1$ sequence is convergent in norm.
\end{exercise}

Let $x_n = \{ x_{n,k} \} \in \ell^1$ be a weakly convergent sequence. Without loss of generality we may assume that $x_n \to 0$ weakly. Assume that $x_n$ does not converge in norm. Then for any $\epsilon > 0$ we may pass to a subsequence to assume that $ || x_n || \ge \epsilon$ for all $n$. 
\bigskip\\
Furthermore, because $x \mapsto x_k$ is continuous we see that,
\[ \lim_{n \to \infty} x_{n,k} = 0 \]
for all $k$. For each $n$ we can choose some $r_n$ such that,
\[ \sum_{i = r_n}^\infty | x_{n,i} | < \epsilon \]
which is possible because $|| x_{n} ||$ is finite. Furthermore since,
\[ \lim_{n \to \infty} x_{n,k} = 0 \]
we also see that for any finite $N$,
\[ \lim_{n \to \infty} \sum_{k = 1}^N x_{n,k} = 0 \]
therefore we can choose some $N_n$


\subsection{3}

Let $u \in \S(\R^n)$ then by Fuorier inversion,
\begin{align*}
u(x) = \int_{\R^n} (\F u)(\xi) e^{i \xi x} \, \d{\xi} 
\end{align*}
and therefore,
\[ u(x',0) = \int_{\R^n} (\F u)(\xi) e^{i \xi \cdot (x', 0)} \, \d{\xi} = \int_{\R^{n-1}} e^{i \xi' \cdot x'} \int_{\R} (\F u)(\xi', \xi_n) \, \d{\xi_n} \d{\xi'} \] 
This formula makes sense for a general $u \in L^2(\R^n)$ as long as we can show convergence. By uniqueness of the Fourier representation for $L^2$ functions we see that,
\[ (\F u')(\xi') = \int_{\R} (\F u)(\xi', \xi_n) \, \d{\xi_n} \]
To show that this exists, take $u \in H^s(\R^n)$ and consider,
\begin{align*}
|(\F u')(\xi')| & = \left| \int_{\R} \frac{(1 + |\xi|^2)^{\frac{s}{2}}(1 + |\xi'|^2)^{\frac{s - \frac{1}{2}}{2}}}{(1 + |\xi|^2)^{\frac{s}{2}}(1 + |\xi'|^2)^{\frac{s - \frac{1}{2}}{2}}} (\F u)(\xi', \xi_n) \, \d{\xi_n} \right| 
\\
& \le \left( \int_\R \frac{(1 + |\xi|^2)^s}{(1 + |\xi'|^2)^{s - \frac{1}{2}}} | (\F u)(\xi', \xi_n) |^2 \, \d{\xi_n} \right)^{\frac{1}{2}} \left( \int_\R \frac{(1 + |\xi'|^2)^{s - \frac{1}{2}}}{(1 + |\xi|^2)^{s}} \, \d{\xi_n} \right)^{\frac{1}{2}}
\end{align*}
Furthermore,
\[ (1 + |\xi'|^2)^{s - \frac{1}{2}} (1 + |\xi|^2)^{-s} = (1 + |\xi'|^2)^{s - \frac{1}{2}} (1 + |\xi'|^2)^{-s} \left( 1 + \frac{|\xi_n|^2}{1 + |\xi'|^2} \right)^{-s} = \left( 1 + \frac{|\xi_n|^2}{1 + |\xi'|^2} \right)^{-s} \frac{1}{\sqrt{1 + |\xi'|^2}} \]
Therefore let $t = \xi_n / \sqrt{1 + |\xi'|^2}$ and we see that,
\[ \int_\R \frac{(1 + |\xi'|^2)^{s - \frac{1}{2}}}{(1 + |\xi|^2)^{s}} \, \d{\xi_n}  = \int_\R (1 + t^2)^{-s} \, \d{t} = C_s \]
which is finite for $s > \frac{1}{2}$. Therefore,
\[ |(\F u')(\xi')| \le C_s^{\frac{1}{2}} \left( \int_\R \frac{(1 + |\xi|^2)^s}{(1 + |\xi'|^2)^{s - \frac{1}{2}}} | (\F u)(\xi', \xi_n) |^2 \, \d{\xi_n} \right)^{\frac{1}{2}}  \]
which implies that,
\begin{align*}
\int_{\R^{n-1}} (1 + |\xi'|^2)^{s- \frac{1}{2}} | (\F u')(\xi')|^2 & \le C_s \int_{\R^{n-1}} \int_\R (1 + |\xi|^2)^s |(\F u)(\xi',\xi_n)|^2 \, \d{\xi'} \d{\xi_n}
\\
& = C_s \int_{\R^n} (1 + |\xi|^2)^s |(\F u)(\xi)|^2 \, \d{\xi} = C_s || u ||_{H^s} < \infty
\end{align*}
Therefore, $R : H^s(\R^n) \to H^{s- \frac{1}{2}}(\R^{n-1})$ seding $u \mapsto u'$ is well-defined everywhere. Since $\S(\R^n)$ is dense in $L^2(\R^n)$ it is also dense in $H^s(\R^n)$ proving uniqueness of $R$.

\subsection{4}

Let $\T^n$ be the $n$-torus and let $\D'(\T^n)$ be the space of distributions on the torus.

\subsubsection{a}

Let $\phi \in C^{\infty}(\T^{n+m})$ and $\phi_x(y) = \phi(y,x)$ so $\phi_x \in \C^{\infty}(\T^n)$. Consider $u \in \D'(\T^n)$ then define $f : \T^n \to \C$ by $f(x) = u(\phi_x)$. 
\bigskip\\
Notice that,
\[ \partial_i f(x) = \lim_{h \to 0} \frac{u(\phi_{x + h e_i}) - u(\phi_x)}{h} = u \left( \frac{\phi_{x + h e_i} - \phi_x}{h} \right) = u \left( \lim_{h \to 0} \frac{\phi_{x + h e_i} - \phi_x}{h} \right) = u(\partial_i \phi_x) \]
by continuity. Therefore, because $\phi$ is smooth then the function $\partial_i \phi_x \in C^{\infty}(\T^n)$ we see that $f$ is differentiable. Then replacing $\phi$ by $\partial_i \phi$ we can repeat the same argument to show that every derivative exists. Thus $f \in C^\infty(\T^n)$.

\subsubsection{b}

The map $\phi : \T^m \to C^{\infty}(\T^n)$ is continuous in the uniform seminorms because $\phi \in C^{\infty}(\T^{n+m})$ is smooth and thus has bounded derivatives. By the Stone Weierstrass theorem, functions of the form $\phi_n = \eta_1(x) \psi_1(y) + \cdots + \eta_n(x) \psi_k(y)$ are dense in $C^{\infty}{\T^n}$ and therefore we can use them to approximate the integral. Furthermore,
\[ \int_{\T^m} u([\phi_n]_x) \, \d{x} = \int_{\T^m} u \left( \sum_{i = 1}^k \eta_i(x) \psi_i \right) \, \d{x} = \sum_{i = 1}^k u(\psi_i) \int_{\T^m} \eta_i(x)  \, \d{x}  \]
and likewise,
\[ u \left( \int_{\T^m} [\phi_n]_x \, \d{x} \right) = u \left( \int_{\T^m} \sum_{i = 1}^k \eta_i(x) \psi_i \, \d{x} \right) = u \left( \sum_{i = 1}^k \psi_i \int_{\T^m} \eta_i(x) \, \d{x} \right) = \sum_{i = 1}^k u(\psi_i) \int_{\T^m} \eta_i(x) \, \d{x} \]
Therefore,
\[ u \left( \int_{\T^m} [\phi_n]_x \, \d{x} \right) = \int_{\T^m} u([\phi_n]_x) \, \d{x} \]
Then in $C^{\infty}(\T^n)$ we get a convergent sequence $\phi_n \to \phi$ meaning that,
\begin{align*}
u \left( \int_{\T^m} \phi_x \, \d{x} \right) & = u \left( \int_{\T^m} \lim_{n \to \infty} [\phi_n]_x \, \d{x} \right) = \lim_{n \to \infty} u \left( \int_{\T^m} [\phi_n]_x \, \d{x} \right) = \lim_{n \to \infty} \int_{\T^m} u([\phi_n]_x) \, \d{x}
\\
& = \int_{\T^m} u( \lim_{n \to \infty} [\phi_n]_x) \, \d{x} = \int_{\T^m} u(\phi_x) \, \d{x}
\end{align*}
because the convergence is uniform and $u$ is continuous.

\subsection{5}

%% SEEN BEFORE Fall 2010 Part I Q3

\begin{exercise}
Suppose that $w$ is a measurable function on $\R^n$ strictly positive almost everywhere and that $K$ is a measurable function on $\R^{2n}$ such that,
\[ \int_{\R^n} |K(x,y)| w(y) \, \d{y} \le A w(x) \quad \text{and} \quad \int_{\R^n} |K(x,y)| w(x) \, \d{x} \le A w(y) \]
for almost all $x$ and almost all $y$ respectively. Prove that the integral operator,
\[ (T f)(x) = \int_{\R^n} K(x, y) f(y) \, \d{y} \]
is bounded on $L^2(\R^n)$ with $|| T || \le A$. 
\end{exercise}

Suppose that $f \in L^2(\R^n)$ and consider,
\begin{align*}
|| T f ||_2^2 & = \int_{\R^n} |(T f)(x)|^2 \, \d{x} = \int_{\R^n} \left| \int_{\R^n} K(x, y) f(y) \, \d{y} \right|^2 \, \d{x}
\\
& \le \int_{\R^n} \int_{\R^n} \int_{\R^n} |K(x,y) K(x, y') f(y) f(y')| \, \d{y} \, \d{y'} \, \d{x} 
\end{align*}
Let,
\[ Q(y, y') = \int_{\R^n} |K(x,y) K(x,y')| \, \d{x} \]
Then by Fubini,
\[ \int_{\R^n} Q(y, y') w(y) \, \d{y} = \int_{\R^n} |K(x,y') | \int_{\R^n} | K(x,y) | w(y) \, \d{y} \, \d{x} \le \int_{\R^n} |K(x, y')| A w(x) \, \d{x} \le A^2 w(y') \]
and likewise,
\[ \int_{\R^n} Q(y, y') w(y') \, \d{y'} \le A^2 w(y) \]
Also by Fubini,
\[ || T f ||_2^2 \le \int_{\R^n \times \R^n} Q(y, y') f(y) f(y') \, \d{(y, y')} \]
Almost everywhere $w(x) > 0$ and thus we can add a factor of $\frac{w(x)}{w(x)}$ without changing the integral because this exists and equals 1 almost everywhere. Therefore let,
\[ F(y, y') = \sqrt{ \frac{Q(y, y') w(y') }{w(y)}} |f(y)| \quad \text{and} \quad G(y,y') = \sqrt{\frac{Q(y, y') w(y)}{w(y')}} |f(y')| \]
so that,
\[ F(y, y') G(y, y') = Q(y,y') |f(y) f(y')| \]
Then applying Cauchy-Schwartz,
\[ || T f ||_2^2 = || F G ||_{1} \le || F ||_2 \cdot || G ||_2 \]
and expanding,
\begin{align*}
|| F ||_2^2 & = \int_{\R^n \times \R^n} \frac{Q(y, y') w(y') }{w(y)} |f(y)|^2 \, \d{(y, y')} = \int_{\R^n} \frac{| f(y) |^2}{w(y)} \int_{\R^n} Q(y, y') w(y') \, \d{y'} \, \d{y}
\\
& \le \int_{\R^n} \frac{|f(y)|^2}{w(y)} A^2 w(y) \, \d{y} = \int_{\R^n} A^2 | f(y) |^2 \, \d{y} = A^2 || f ||_2^2 
\end{align*}
thus $|| F ||\_2 \le A || f ||_2$ and likewise,
\begin{align*}
|| G ||_2^2 & = \int_{\R^n \times \R^n} \frac{Q(y, y') w(y) }{w(y')} |f(y')|^2 \, \d{(y, y')} = \int_{\R^n} \frac{| f(y') |^2}{w(y')} \int_{\R^n} Q(y, y') w(y) \, \d{y} \, \d{y'}
\\
& \le \int_{\R^n} \frac{|f(y')|^2}{w(y')} A^2 w(y') \, \d{y'} = \int_{\R^n} A^2 | f(y') |^2 = A^2 || f ||_2^2 
\end{align*}
showing that $||G ||_2 \le A || f ||_2$. Putting everything together,
\[ || T f ||_2^2 \le || F ||_2 \cdot || G ||_2 \le A^2 || f ||_2^2 \]
and therefore $|| T f ||_2 \le A || f ||_2$ showing that $T$ is bounded and that $|| T || \le A$.

\section{Spring 2016 Part II}

\subsection{1}

\subsubsection{a}

\begin{exercise}
Suppose that $Y$ is a normed complex vector space and $\lambda : Y \to \C$ is a linear but not continuous functional. Show that $N = \ker{\lambda}$ is dense in $Y$
\end{exercise}

I claim that $\lambda$ is continuous if and only if $\ker{\lambda}$ is closed and if $\lambda \neq 0$ then $\ker{\lambda}$ is closed if and only if it is not dense.
\bigskip\\
Indeed, if $\lambda$ is continuous then $\ker{\lambda}$ is closed and if $\ker{\lambda}$ is closed then $\lambda$ factors as $X \to X / \ker{\lambda} \to \C$ but $X / \ker{\lambda}$ is finite dimensional so since $X / \ker{\lambda} \to \C$ is linear it is continuous and thus $\lambda$ is continuous ($\pi : X \to X / \ker{\lambda}$ is continuous because $\ker{\lambda}$ is closed). 
\bigskip\\
Therefore, let $\lambda \neq 0$. it suffices to show that $\ker{\lambda}$ is closed if and only if it is not dense. Clearly if $\ker{\lambda}$ is closed then it is not dense because $\lambda \neq 0$ so $\ker{\lambda} \subsetneq X$. Furthermore, if $\ker{\lambda}$ is dense then take any $z \in \partial \ker{\lambda}$ then $\lambda(z) \neq 0$ and therefore for any $x \in X$ we can write $x = a z + (x - az)$ where $\lambda(x) = a \lambda(z)$ and then $\lambda(x - az) =  0$ so $x \in \overline{\ker{\lambda}}$ because the closure is also a linear space and thus $\ker{\lambda}$ is dense.
\bigskip\\
Finally, if $\lambda = 0$ then $\ker{\lambda} = Y$ is dense but $\lambda$ is continuous. 

\subsubsection{b}

Let $\lambda : \ell^1 \to \C$ be defined by,
\[ x \mapsto \sum_{k = 1}^\infty x_k  \]
However, give $\ell^1$ the subspace topology from the inclusion $\ell^1 \subset \ell^2$. Then I claim that $\lambda$ is not continuous. Indeed, it is not bounded because for any $\epsilon > 0$ we can take a sequence $x_k \sim n^{-(1 + \epsilon)}$ which is in $\ell^1$ with bounded $\ell^2$-norm but $\ell(x) \to 0$ as $\epsilon \to 0$ showng that $\lambda$ is unbounded. Therefore, $N = \ker{\lambda}$ is dense in $\ell^1$ with the $\ell^2$-topology and $\ell^1$ is dense in $\ell^2$. Thus $N$ is dense in $\ell^2$. Furthermore, because $\ell^2$ is separable, any dense subspace contains an orthonormal basis by the Grahm-Schmit process and thus there exists a orthonormal basis inside $N$.

\subsection{2}

\begin{exercise}
Suppose that $1 < p < \infty$ and $f, f_n \in L^p([0,1])$ and $n \in \N$. Let $|| f_n || \le 1$ for all $n$ and $f_n \to f$ almost everywhere. Show that $f_n \to f$ weakly and $|| f ||_{L^p} \le 1$. 
\end{exercise}

Let,
\[ N = \{ x \in X \mid \lim_{n \to \infty} f_n(x) \neq f(x) \} \]
Then by Ergorov's theorem for any $\delta > 0$ there exists some $E$ such that $\mu(E) < \delta$ and $f_n \to f$ uniformly on $X \setminus (E \cup N)$. Now for any $g \in L^q([0,1]) = (L^p([0,1]))^*$ consider,
\[ \int_X |f_n - f| g \, \d{\mu} = \int_{N} |f_n - f| g \, \d{\mu} + \int_{E} |f_n - f| g \, \d{\mu} + \int_{X \setminus (E \cup N)} |f_n - f| g \, \d{\mu} \] 
For $n > N$ we have $|f_n - f| < \epsilon$ and because the measure space is finite $g \in L^1([0,1])$ and thus,
\[ \left| \int_{X \setminus (E \cup N)} |f_n - f| g \, \d{\mu} \right| < \epsilon || g ||_1 \]
Furthermore, because $\mu(N) = 0$,
\[ \int_{N} |f_n - f| g \, \d{\mu} = 0 \]
Also, because $|g|^q \in L^1([0,1])$ for any $\epsilon > 0$ there is a $\delta > 0$ such that whenever $\mu(E) < \delta$,
\[ \int_E |g|^q \, \d{\mu} < \epsilon^q \]
and therefore,
\[ \left| \int_{E} |f_n - f| g \, \d{\mu} \right| \le (|| f_n ||_{L^p} + || f ||_{L^p}) \cdot \left( \int_{E} |g|^q \, \d{\mu} \right)^{\frac{1}{q}} < (|| f_n ||_{L^p} + || f ||_{L^p}) \cdot \epsilon \]
but we have $|| f_n ||_{L^1} \le 1$ so the integral is bounded uniformly in $n$. Therefore,
\[ \left| \int_X f_n g \, \d{\mu} - \int_X f g \, \d{\mu} \right| \le \int_X |f_n - f| g \, \d{\mu} < \epsilon (2 + || f ||_{L^1}) \cdot || g ||_{L^1} \]
for $n > N$ proving that,
\[ \lim_{n \to \infty} \int_X f_n g \, \d{\mu} = \int fg \, \d{\mu} \]
and therefore $f_n \to f$ weakly. Furthermore, by the lower semi-continuity of weak limits,
\[ || f ||_{L^p} \le \liminf_{n \to \infty} || f_n ||_{L^p} = 1 \]

\subsection{3}

\subsubsection{a}

Let $X = Y = \ell^2$ and $A \in \L(X, Y)$ be given by $(T x)_n = \frac{1}{n} x_n$. Then $T$ is not surjective but $\ker{T} = \{ 0 \}$ and $T^* = T$ so by the decomposition $X = \ker{T} \oplus \overline{\im{T}}$ we see that $T$ has dense image and thus $\im{T}$ cannot be closed.

\subsubsection{b}

Let $X, Y$ be Hilbert spaces, $A \in \L(X, Y)$ and $\im{A}$ is closed. Decompose $X = \ker{A} \oplus \overline{\im{A^*}}$ and therefore restricting to $\overline{\im{A^*}} \subset X$ we may assume that $A$ is injective and $\im{A^*}$ is dense. Furthermore, decompose $Y = \ker{A^*} \oplus \overline{\im{A}}$ but $\im{A}$ is closed so $Y = \ker{A^*} \oplus \overline{\im{A}}$. Therefore, restricting gives $\tilde{A} : \overline{\im{A^*}} \to \im{A}$ and both are closed in Hilbert spaces and are therefore Hilbert spaces. Since $\tilde{A}$ is bijective, by the bounded inverse theorem it is invertible. 
\bigskip\\
Therefore we may assume that $A$ is invertible. Then it is immediate that $A^* (A^{-1})^* = I$ and $(A^{-1})^* A^* = I$ so $A^*$ is invertible as well which implies that $A^*$ is surjective and thus $\im{A^*} = \overline{\im{A^*}}$ so $A^*$ has closed image.


\subsubsection{c}

If there exists $C > 0$ such that for all $y \in Y$,
\[ || y ||_Y \le C || A^* y ||_X \]
then we see that $A^*$ is injective and $\im{A^*}$ is closed. Thus, $\ker{A^*} = 0$ implies that $\overline{\im{A}} = Y$ so $\im{A}$ is dense. Furthermore, $\im{A^*}$ is closed so by the previous problem $\im{A}$ is closed and thus $\im{A} = Y$ proving that $A$ is surjective.
\bigskip\\
However, if $A^*$ is just injective then the conclusion that $A$ is surjective is false. Take the example given in (a) where $A = A^*$ and $A$ is injective but not surjectibe.

\subsection{4}

\subsubsection{a}
We can assume that,
\[ \int \psi_0 \, \d{x} = 1 \]
Let $f \in \R'(\R)$ and $\psi_0 \in \S(\R)$ with,
\[ \int_\R \psi_0(x) \, \d{x} \neq 0 \]
and $a \in \R$. For any Schwartz function $\phi \in \S(\R)$ which admits an antiderivative $\eta \in \S(\R)$ (which is unique because nonzero constant functions are not Schwartz) we can define,
\[ \inner{u}{\phi} = - \inner{f}{\eta} + a \inner{f}{\psi} \]
I claim that a Schwartz function $\phi$ has an antiderivative if and only if $\int \phi \d{x} = 0$. Indeed, if $\phi = \eta'$ then,
\[ \int \phi \, \d{x} = \lim_{x \to \infty} [\eta(x) - \eta(-x)] = 0 \]
because $\eta$ is rapidly decreasing. Now suppose that,
\[ \int \phi \, \d{x} = 0 \]
Consider,
\[ \eta(x) = \int_{-\infty}^x \phi \, \d{x} \]
We need to show that $\eta \in \S(\R)$ by showing that,
\[ || \eta ||_{n,m} = \sup_x | x^n \eta^{(m)} | < \infty \]
Because $\eta' = \phi \in \S(\R)$ we already know that $|| \eta ||_{n, m+1} = || \phi ||_{n,m}$ is finite so we just need to consider $|| \eta ||_{n,0}$. Because $\phi$ is rapidly decreasing we know,
\[ |\phi(x)| \le \min \{ M_0, M_{n+1} / x^{n+1} \} \]
Then for $x < - \left( \frac{M_{n+1}}{M_0} \right)^{\frac{1}{n+1}}$ we have,
\[ | x^n \eta(x) | \le |x^n| \left| \int_{-\infty}^x \frac{M_{n+1}}{x^{n+1}} \, \d{x} \right| = \frac{M_{n+1}}{n+1} \]
is bounded. Furthermore, for $x > \left( \frac{M_{n+1}}{M_0} \right)^{\frac{1}{n+1}}$ we have,
\[ \eta(x) = \int_{-\infty}^x \phi(x) \, \d{x} = \int_{-\infty}^{\infty}  \phi(x) \, \d{x} - \int_x^{\infty} \phi(x) \, \d{x} = - \int_x^{\infty} \phi(x) \, \d{x} \]
and thus by the exact same argument,
\[ | x^n \eta(x) | \le \frac{M_{n+1}}{n+1} \]
so we have shown that $x^n \eta$ is bounded on the complement of a compact and it is clearly bounded on the compact by continuity so $|| \eta ||_{n,0}$ is finite. 
\bigskip\\
Therefore, given any $\phi \in \S(\R)$ take,
\[  c = \int \phi \, \d{x} \]
then $\phi - c \psi_0$ has zero integral so it has an antiderivative $\eta \in \S(\R)$ then define,
\[ \inner{u}{\phi} = - \inner{f}{\eta} + a c \]
This is clearly linear and continuous so it defines a $u \in \S'(\R)$. Furthermore, 
\[ \inner{u'}{\phi} = - \inner{u}{\phi'} = \inner{f}{\phi} \]
because $\phi'$ has zero integral and $\phi$ is its antiderivative by construction. Thus $u' = f$.
\bigskip\\
Finally, if $u$ and $\tilde{u}$ both satisfy the given properties then for any $\phi$ we write $\phi = \eta' + c \psi_0$ and then,
\[ \inner{\tilde{u}}{\phi} = \inner{\tilde{u}}{\eta'} + c \inner{\tilde{u}}{\psi_0} = - \inner{f}{\eta} + c a \]
because $\inner{\tilde{u}'}{\eta} = - \inner{\tilde{u}}{\eta'}$ by definition and $\tilde{u}' = f$ and $\inner{u}{\psi_0} = a$ by hypothesis. Since we defined $u$ by this equation, $\tilde{u} = u$ proving uniqueness.

\subsubsection{b (FUCK OFFF!)}

For $\epsilon > 0$ and $k \in \Z^{+}$ define $u_{\pm, \epsilon}^{(k)} : \S(\R) \to \C$ by,
\[ u_{\pm, \epsilon}^{(k)}(\phi) = \int_\R (x \pm i \epsilon)^{-k} \phi(x) \, \d{x} \]
for all $\phi \in \S(\R)$. It is obvious that $u_{\pm, \epsilon}^{(k)}$ is linear and it is continuous because,
\[ | (x \pm i \epsilon)^{-k}| \ge \epsilon^{-k} \]
and therefore,
\[ | u_{\pm, \epsilon}^{(k)}(\phi) = \left| \int_\R (x \pm i \epsilon)^{-k} \phi(x) \, \d{x} \right| \le || (x + i \epsilon)^{-k} ||_{\infty} || \phi ||_{1} \]
Therefore, if $\phi_n \to \phi$ in $\S(\R)$ then $\phi_n \to \phi$ in $L^1$ (by uniform convergence) and thus we see that $u_{\pm, \epsilon}^{(k)}(\phi_n) \to u_{\pm, \epsilon}^{(k)}(\phi)$ because,
\[ | u_{\pm, \epsilon}^{(k)}(\phi) - u_{\pm, \epsilon}^{(k)}(\phi_n)| = |u_{\pm,\epsilon}(\phi - \phi_n)| \le || (x + i \epsilon)^{-k} ||_{\infty} \cdot || \phi - \phi_n ||_{1} \xrightarrow{n \to \infty} 0 \]
We define,
\[ u_{\pm}(\phi) = \lim_{\epsilon \to 0} u_{\pm, \epsilon}^{(k)}(\phi) \]
we need to show that this exists and defines a tempered distribution. Notice that, $(x \pm i \epsilon)^{-k}$ is the $(k-1)$-th derivative of,
\[ \frac{(-1)^{k-1}}{(k-1)!} (x \pm i \epsilon)^{-1} \]
Therefore, we first investigate the case $k = 1$ and then take derivatives. Consider,
\[ u_{\pm, \epsilon}^{(1)}(\phi) = \int_\R (x \pm i \epsilon)^{-1} \phi(x) \, \d{x} = \int_{\R} \frac{x^2}{x^2 + \epsilon^2} \frac{\phi(x)}{x} \, \d{x} - i \int_\R \frac{\epsilon}{x^2 + \epsilon^2} \phi(x) \, \d{x} \]
Therefore,
\[ \lim_{\epsilon \to 0} u_{\pm, \epsilon}^{(1)}(\phi) = \mathcal{PV} \int_\R \frac{\phi(x)}{x} \, \d{x} \mp i \pi \phi(0) \]
and thus $u_{\pm, \epsilon}^{(1)} \to u_{\pm}^{(k)}$ where $u_{\pm} = \mathcal{PV}[x^{-1}] \mp i \pi \delta_0$. Now, I claim that,
\[ u_{\pm, \epsilon}^{(k)} \to u_{\pm}^{(k)} \]
where the notation is consistent such that $u_{\pm}^{(k)}$ is the $k-1$-th derivative of $u_{\pm}^{(1)}$ (up to some numerical factors). Indeed, more generally, if $u_n \to u$ is a convergent sequence of distributions then I claim that $u_n' \to u'$ also converges. Indeed, for any $\phi \in \S(\R)$ we have,
\[ u'(\phi) = - u(\phi') = - \lim_{n \to \infty} u_n(\phi') = \lim_{n \to \infty} u_n'(\phi) \]
Therefore we find that the limit of $u_{\pm, \epsilon}^{(k)}$ exists and equals,
\[ \frac{(-1)^{k-1}}{(k - 1)!} \mathcal{PV}[x^{-1}]^{(k-1)} \mp \frac{(-1)^{k-1}}{(k-1)!} \delta_0^{(k-1)} \]
Therefore,
\[ u_{+}^{(k)} - u_{-}^{(k)} = - \frac{2 (-1)^{k-1}}{(k-1)!} \delta_0^{(k-1)} \]


\subsection{5}

Let $\phi : \R \to \C$ be $2\pi$-periodic and H\"{o}lder continuous with exponent $\alpha$. Let,
\[ c_n = \frac{1}{2 \pi} \int_{-\pi}^\pi \phi(x) e^{-inx} \, \d{x} \]
be its Fourier coefficients.

\subsubsection{a}

By the H\"{o}lder condition, there is a constant $C$ such that,
\[ |\phi(x + h) - \phi(x)| \le C | h |^\alpha \]
and therefore,
\[ \frac{1}{4 \pi} \int_{-\pi}^\pi | \phi(x + h) - \phi(x) |^2 \d{x} \le \tfrac{1}{2} C^2 | h |^{2 \alpha} \]
Furthermore, consider the Fourier coefficients of the function $\psi_h(x) = \phi(x + h) - \phi(x)$ which is also $2\pi$-periodic. Explicitly,
\begin{align*}
c'_n & = \frac{1}{2 \pi} \int^{\pi}_{-\pi} (\phi(x + h) - \phi(x)) e^{-inx} \, \d{x}
\\
& = \frac{1}{2 \pi} \int_{-\pi}^{\pi} \phi(x) e^{-in(x - h)} \, \d{x} - \int_{-\pi}^{\pi} \phi(x) \, \d{x} 
\\
& = (e^{i n h} - 1) c_n  
\end{align*}
and therefore,
\[ |c'_n|^2 = |e^{i n h} - 1|^2 \cdot |c_n|^2 = 2 (1 - \cos{(nh)}) | c_n |^2 \]
Because $\psi_h \in C(\T) \subset L^2(\T)$ we can apply Planachel's formula to conclude that,
\[ \sum_{n \in \Z} |c_n'|^2 = \frac{1}{4 \pi} \int_{-\pi}^{\pi} | \psi_h(x)|^2 \, \d{x} \]
and therefore, diving by two we find,
\[ \sum_{n \in \Z} (1 - \cos{(n h)}) |c_n|^2 = \frac{1}{4 \pi} \int_{-\pi}^\pi | \phi(x + h) - \phi(x) |^2 \le \tfrac{1}{4} C^2 | h |^{2 \alpha} \]

\subsubsection{b (NO IDEA HOW TO DO THIS)}

Now suppose that $\alpha > \tfrac{1}{2}$ then we find,
\[ \left| \sum_{n \in \Z} \left( \frac{1 - \cos{(n h)}}{h} \right) |c_n|^2 \right| \le C |h|^{2 \alpha - 1} \]
Now for a fixed $m$ we let $h = \frac{\pi}{m}$ and thus $\cos{(nh)} = - 1$ for $n = \pm m$ and therefore,
\[ 2 \left( \frac{m}{\pi} \right) |c_n|^2 \le \sum_{n \in \Z} \left( \frac{1 - \cos{(nh)}}{h} \right) |c_n|^2 \le C \left( \frac{\pi}{m} \right)^{2 \alpha-1} \]
which shows that,
\[ \lim_{|m| \to \infty} m |c_m|^2 = 0 \]
or more strongly that $|c_m| \le C' |m|^{-\alpha}$. However, such sequences are not automatically summable so we need to do more work. First the same argument for $n = (2 k + 1) m$ and thus $\cos{(nh)} = -1$ for $h = \frac{\pi}{m}$ gives,
\[ \sup_{k \in \Z} | c_{(2 k + 1)m}| \le \tfrac{1}{2} C \left( \frac{\pi}{m} \right)^\alpha \]
Furthermore, for any $n = (2 k + 1) m$ we have $\cos{(nh)} = -1$ if $h = \frac{\pi}{m}$. Therefore,
\[ \sum_{k \in \Z} 2 | c_{(2 k + 1)m}|^2 \le \sum_{n \in \Z} (1 - \cos{(nh)}) |c_n|^2 \le C \left( \frac{\pi}{m} \right)^{2 \alpha} \]
and therefore,


 Now,
\[ \sum_{n \in \Z} |c_n| \le \sum_{m = 1}^\infty \sum_{k \in \Z} |c_{(2k + 1)m}| \]
Furthermore, 

\section{Fall 2016 Part I}

\subsection{1}

\begin{exercise}
Let $X$ be a Banach space and $A \in \L(X)$ a bounded linear operation. Show that $\sigma(A)$ is closed and bounded.
\end{exercise}

We use the fact that if $|| S || < 1$ then the series,
\[ (I - S)^{-1} = \sum_{i = 0}^\infty S^i \]
converges. This immediately gives two series representations of the resolvent. First, for $\lambda_0 \in \rho(A)$ let $C = || (A - \lambda_0 I)^{-1}$ and $\delta = C^{-1}$. Then suppose that $| \lambda - \lambda_0 | < \delta$. We have,
\[ (A - \lambda I) = (A - \lambda_0 I) - (\lambda - \lambda_0) I = (A - \lambda_0 I) [I - (\lambda - \lambda_0)(A - \lambda_0 I)^{-1} ] \]
therefore let,
\[ S = (\lambda - \lambda_0) (A - \lambda_0 I)^{-1} \]
because then,
\[ || S || = | \lambda - \lambda_0 | C < 1 \]
and therefore we see that $I - S$ is invertible meaning that $A - \lambda I$ is invertible as the product of invertible operators. Thus $B_\delta(\lambda_0) \subset \rho(A)$ so $\rho(A)$ is open and thus $\sigma(A)$ is closed. Furthermore, if $| \lambda | > || A ||$ then let $S = \frac{A}{\lambda}$ in which case,
\[ (A - \lambda I)^{-1} = - \lambda^{-1} (I - S)^{-1} = - \frac{1}{\lambda} \sum_{i = 0}^\infty \left( \frac{A}{\lambda} \right)^i \]
exists and therefore $\lambda \in \rho(T)$ proving that $\sigma(T) \subset \overline{B_{|| A||}(0)}$ and is thus bounded.

\subsection{2}

\begin{exercise}
Let $\S(\R)$ be the space of Schwartz functions and $\S'(\R)$ the tempered distributions. Show that there exists no $u \in \S'(\R)$ such that for all $\phi \in \S(\R)$ with $\supp{\phi} \subset (0, \infty)$ we have,
\[ u(\phi) = \int e^{\frac{1}{x}} \phi(x) \, \d{x} \]
\end{exercise}

If $u \in \S'(\R)$ then $f \mapsto |u(f)|$ is a continuous seminorm on $\S(\R)$ and therefore because $\S(\R)$ is generated by a seminorm topology with seminorms,
\[ || f ||_{n,m} = \sup | x^n \partial_m f(x) | \]
we see that there must exist some $N$ and $C > 0$ such that,
\[ |u(f)| \le C \sum_{i,j = 0}^{N} \sup |x^n \partial_m f(x) | \]
Now we consider a bump function $\phi$ of area one with compact support $[a,b] \subset (0, \infty)$. Consider $\phi_t(x) = \phi(tx)$ which is supported on $[a/t, b/t] \subset (0, \infty)$. Therefore,
\[ \sup_x |x^n \partial_m \phi_t(x)| = \sup_y |(y/t)^n t^m \partial_m \phi(y) | = t^{m-n} \sup_y |y^n \partial_m \phi(y) | \]
and thus,
\[ |u(f)| \le C \sum_{i,j = 0}^N t^{m-n} || f ||_{n,m} \le \tilde{C} t^N \]
for sufficiently large $t$. Furthermore,
\[ |u(\phi_t)| = \left| \int e^{\frac{1}{x}} \phi(tx) \, \d{x} \right| = \frac{1}{t} \left| \int e^{\frac{t}{y}} \phi(y) \, \d{y} \right| \] 
However, $\phi(y) = 1$ on $[c,d] \subset [a,b]$ and therefore,
\[ | \phi(\phi_t) | \ge \frac{d - c}{t} \cdot e^{\frac{t}{d}} \]
This grows exponentially in $t$ and therefore cannot be bounded by a polynomial giving the required contradiction. 

\subsection{3}

Let $\mu$ be a nonnegative Borel measure on $\R^n$ such that $\mu(A) < \infty$ for each bounded Borel subset $A \subset \R^n$.

\subsubsection{a}

Fix $x \in \R^n$ and consider the sequence of sets,
\[ A_n = B_{\rho + \frac{1}{n}}(x) \]
which have,
\[ \bigcap_{n = 1}^\infty A_n = \overline{B}_\rho(x) \]
Because these are bounded, $\mu(A_n) < \infty$ and thus the measure is continuous downwards,
\[ \mu(\overline{B}_\rho(x)) = \lim_{n \to \infty} \mu(A_n) = \limsup_{n \to \infty} \mu(A_n) \]
Furthermore, for $|x - y| < \frac{1}{n}$ we see that $\overline{B}_\rho(y) \subset A_n$ because if $|y - z| \le \rho$ then $|x - z| \le |x - y| + |y - z| < \rho + \frac{1}{n}$. Therefore,
\[ \mu(\overline{B}_\rho(y)) \le \mu(A_n) \]
Therefore,
\[ \limsup_{y \to x} \theta(y) \le \limsup_{n \to \infty} \mu(A_n) = \theta(x) \]
Explicilty, for any $\epsilon > 0$ there is some $n$ such that $\mu(A_n) - \epsilon \le \theta(x)$ and thus for $|x - y| < \frac{1}{n}$ we have,
\[ \theta(y) = \mu(\overline{B}_\rho(y)) \le \mu(A_n) \le \theta(x) + \epsilon \]
proving that,
\[ \limsup_{y \to x} \theta(y) \le \theta(x) \]

\subsubsection{b}

Let $\mu = \delta_0$ where,
\[ \delta_0(A) =
\begin{cases}
1 & 0 \in A 
\\
0 & 0 \notin A
\end{cases}\]
Then $\theta(x) = \mu(\overline{B}_\rho(x)) = \chi_{B_\rho(0)}$ is not continuous because it jumps at the boundary.

\subsection{4}

Consider the Hilbert space $H = L^2(\R, e^{-x^2} \d{x})$ and let $\phi_n(x) = x^n$ for $n \ge 0$ so $\phi_n \in H$.

\subsubsection{a}

Let $e_\xi(x) = e^{i \xi x}$ for $x \in \R$. Consider,
\[ s_k(\xi) = \sum_{n = 0}^k \frac{(i \xi)^n}{n!} \phi_n \]
Consider,
\begin{align*}
|| e_\xi - s_k ||^2_H = \int \left| \sum_{n = 0}^k \frac{(i \xi x)^n}{n!}  - e^{i \xi x} \right|^2 e^{-x^2} \, \d{x} 
\end{align*}
Now by the remainder form of Taylor's theorem,
\[ \sum_{n = 0}^N \frac{(i \xi x)^n}{n!}  - e^{i \xi x} = \frac{x^{k+1}}{(k+1)!} (i \xi)^{k+1} e^{i \xi y} \]
for some $y \in [0, x]$. and therefore,
\[ \left| \sum_{n = 0}^k \frac{(i \xi x)^n}{n!}  - e^{i \xi x} \right| \le \left( \frac{|\xi x|^{k+1}}{(k+1)!} \right)^2 \]
Therefore,
\[ || e_\xi - s_k ||^2 \le \frac{|\xi|^{k+1}}{(k + 1)!} \int x^{2(k+1)} e^{-x^2} \, \d{x} = \left( \frac{|\xi|^{k+1}}{(k + 1)!} \right)^2 || \phi_{k+1} ||^2 \]
However,
\[ ||\phi_{n+1}||^2_H = \int_{-\infty}^{\infty} x^{2(n+1)} e^{-x^2} \, \d{x} = \int_0^{\infty} u^{n + \tfrac{1}{2}} e^{-u} \, \d{u} = \Gamma(n + \tfrac{3}{2}) = n! \cdot \Gamma(\tfrac{1}{2}) \]
therefore,
\[ || e_\xi - s_k ||^2 \le \frac{|\xi|^{k + 1} \Gamma(\tfrac{1}{2})}{(k + 1) \cdot (k + 1)!} \xrightarrow{k \to \infty} 0 \]
and therefore, $s_k(\xi) \to e_\xi$ in the norm topology for each $\xi \in \R$.

\subsubsection{b}

Suppose that $\phi \in H$ with $\inner{\phi_n}{\phi} = 0$ for all $n$. Then we see that,
\[ \inner{e^{i \xi x}}{\phi} = \lim_{k \to \infty} \inner{s_k(\xi)}{\phi} = \lim_{k \to \infty} \sum_{n = 0}^k \frac{(i \xi)^n}{n!} \inner{\phi_n}{\phi} = 0 \]
Therefore,
\[ \inner{e^{i \xi x}}{\phi} = \int e^{-i \xi x} \phi(x) e^{-x^2} \, \d{x} = [\F (\phi e^{-x^2})](\xi) \]
and therefore $\F (\phi(x) e^{-x^2}) = 0$. However, $\phi  e^{-x^2} \in L^2(\R)$ and therefore if its Fouier transform is zero then it is zero (in $L^2(\R)$ i.e. $f \eqae 0$) because the Fourier transform is an isometry on $L^2$ which implies that $\phi = 0$ in $H$.

\subsubsection{c}

Let $V$ be the span of $\{ \phi_n \}$ in $H$. I claim that $V$ is dense in $H$. Indeed, we have shown that if $\inner{\phi_n}{\phi} = 0$ for all $n$ then $\phi = 0$ and therefore this shows that $V^\perp = \{ 0 \}$. However, $V^\perp = \overline{V}^\perp$ and because $\overline{V}$ is a closed linear subspace we have $H = \overline{V} \oplus \overline{V}^\perp = \overline{V}$ proving that $V$ is dense.
\bigskip\\
Therefore, by applying Grahm-Schmit to the linear space $V$ we obtain an orthonormal basis of $H$. Indeed, let $\psi_0 = \phi_0 || \phi_0 ||^{-1}$ and then define recursively,
\[ \psi_{n+1} = \frac{\phi_{n+1} - \sum\limits_{k = 0}^n \inner{\psi_k}{\phi_{n+1}} \psi_k}{|| \phi_{n+1} - \sum\limits_{k = 0}^n \inner{\psi_k}{\phi_{n+1}} \psi_k ||} \]
giving an orthonormal basis such that $\vspan{\phi_0, \phi_1, \dots, \phi_n} = \vspan{ \psi_0, \psi_1, \dots, \psi_n}$ by construction. The only way this can go wrong is if $\phi_{n+1}$ is in the span of $\phi_0, \dots, \phi_n$ in which case we might get zero in the formula causing it to not be well-defined. However, clearly $\phi_{n+1}$ which is a polynomial of degree $n+1$ cannot be in the span of polynomials of degree at most $n$ (because it is contained in the kernel of differentiation $n$ times while $\phi_{n+1}$ is not). 


\subsection{5 (WHY THE FUCK??)}

\begin{exercise}
Let $\T$ be the unit circle and for $m > 0$ let $H^m(\T)$ be the Sobolev space of $L^2(\T)$-functions $f$ satisfying,
\[ || f ||_{H^m} = \sum_{n \in \Z} | \hat{f}(n) |^2 ( 1 + n^2)^m < \infty \]
equiped with the $|| \bullet ||_{H^m}$ norm. Suppose that $A : H^m(\T) \to L^2(\R)$ is a continuous linear map and $B \in \L(L^2(\T), H^m(\T)$ such that $AB - I$ and $BA -  I$ are compact operators on $L^2(\T)$ and $H^m(\T)$ respectively. Suppose also that,
\[ \inner{A f}{\iota g}_{L^2} = \inner{\iota f}{Ag}_{L^2} \]
for $f, g \in H^m(\T)$ and $\iota : H^m(\T) \to L^2(\R)$ the inclusion. Show that $A - \lambda \iota$ is invertible when $\lambda \neq \R$.
\end{exercise}

Let $\imag{\lambda} \neq 0$ and consider,
\begin{align*}
\imag{\inner{(A - \lambda \iota) f}{\iota f}_{L^2}} & = \tfrac{1}{2} \left[ \inner{(A - \lambda \iota) f}{\iota f}_{L^2} - \inner{\iota f}{(A - \lambda \iota) f}_{L^2} \right]
\\
&  = \tfrac{1}{2} \left[ \lambda \inner{\iota f}{\iota f}_{L^2} - \bar{\lambda} \inner{\iota f}{\iota f}_{L^2} \right] = \imag{\lambda} || \iota f ||_{L^2}^2
\end{align*}
Therefore, we find,
\[ \imag{\lambda} || \iota f ||_{L^2}^2 \le \imag{\inner{(A - \lambda \iota) f}{\iota f}_{L^2}} \le | \inner{(A - \lambda \iota) f}{\iota f}_{L^2} | \le || (A - \lambda \iota) f ||_{L^2} \cdot || \iota f ||_{L^2} \]
therefore,
\[ || \iota f ||_{L^2} \le \imag{\lambda}^{-1} || (A - \lambda \iota) ||_{L^2} \]
In particular, $A - \lambda \iota$ is injective because $\iota$ is injective. Furthermore,

\section{Fall 2016 Part II}

\subsection{1}

\begin{exercise}
Show that if $\mu$ is a $\sigma$-finite measure on a measurable space $(X, \F)$ then there exists a \textit{finite} measure $\nu$ on $(X, \F)$ with $\nu \ll \mu$ and $\mu \ll \nu$.
\end{exercise}

Choose a countable family $A_n \in \F$ with $\mu(A_n) < \infty$ such that,
\[ X = \bigcup_{n = 1}^\infty A_n \]
We can assume that the $\{ A_n \}$ are disjoint by redefining,
\[ A_n' = A_n \setminus \bigcup_{i = 1}^n A_i \]
Then let, 
\[ f = \sum_{i = 1}^\infty \frac{1}{2^n (1 + \mu(A_n))} \chi_{A_n} \]
which converges everywhere uniformly by the $M$-test. I claim that $f \in L^1(X)$. Indeed,
\[ \int_X f \, \d{\mu} = \lim_{n \to \infty} \sum_{i = 1}^n \int_{A_i} f \, \d{\mu} \]
Furthermore, on $A_n$ we have that each partial sum is bounded by $1$ which is integrable since $\mu(A_n) < \infty$ so by the dominated convergence theorem,
\[  \int_{A_n} f \, \d{\mu} = \sum_{i = 1}^\infty \frac{1}{2^n 2^n (1 + \mu(A_n))} \int_{A_n} \chi_{A_n} \, \d{\mu} = \frac{1}{2^n} \cdot \frac{\mu(A_n)}{1 + \mu(A_n)} \]
because $A_i$ and $A_n$ are disjoint unless $i = n$. Then define,
\[ \nu(A) = \int_A f \, \d{\mu} = \sum_{i = 1}^\infty \frac{1}{2^n (1 + \mu(A_n))} \mu(A_n \cap A) \]
which is a well-defined measure because $f \in L^1(X)$. Suppose that $\mu(A) = 0$. Then $\mu(A_n \cap A) \le \mu(A) = 0$ so $\mu(A_n \cap A) = 0$ for all $n$ since $A_n \cap A \in \F$. Therefore, $\nu(A) = 0$. Likewise, if $\nu(A) = 0$ then it implies that $\mu(A_n \cap A) = 0$ because otherwise,
\[ \nu(A) \ge \frac{1}{2^n (1 + \mu(A_n))} \mu(A_n \cap A) > 0 \]
Therefore, 
\[ \mu(A) = \mu \left( \bigcup_{i = 1}^n A_n \cap A \right) = \sum_{i = 1}^\infty \mu(A_n \cap A) = 0 \]
because $\mu(A_n \cap A) = 0$. Therefore $\mu \ll \nu$ and $\nu \ll \mu$.

\subsection{2}

\begin{exercise}
Let $K \in L^2(\R^{2n})$ and define,
\[ (T f)(x) = \int_{\R^n} K(x, y) f(y) \, \d{y} \]
\end{exercise}

It is clear that $T$ is linear. Furthermore,
\begin{align*}
|| T f ||_2^2 = \int_{\R^n} \left| \int_{\R^n} K(x,y) f(y) \, \d{y} \right|^2 \, \d{x} 
\end{align*}
Furthermore, by Cauchy-Schwartz,
\[ \left| \int_{\R^n} K(x,y) f(y) \, \d{y} \right|^2 \le \left( \int_{\R^n} |K(x,y) \, \d{y} \right) \left( \int_{\R^n} f(y) \, \d{y} \right) \]
Therefore,
\[ || T f ||_2^2 \le \int_{\R^n} \left( \int_{\R^n} |K(x,y)|^2 \, \d{y} \right) || f ||_2^2 \, \d{x} = || K ||^2_2 \cdot || f ||^2_2 \]
showing that,
\[ || T f ||_2 \le || K ||_2 \cdot || f ||_2 \]
and therefore $|| T || \le || K ||_2$ so, in particular, $T \in \L(L^2(\R^{n}))$.
\bigskip\\
Let $A \subset C_c(\R^{2n})$ be the ring of functions generated by $\phi(x) \psi(y)$ for $\phi \in C_c(\R^n)$ and $\psi \in C_c(\R^n)$. This clearly separates points and vanishes nowhere so by Stone-Weierstrass $A$ is dense in $C_0(\R^{2n})$ and therefore also dense in $L^2(\R^{2n})$. Therefore, we can approximate $K$ by,
\[ K_n = \phi_1 \psi_1 + \cdots + \phi_n \psi_n \]
and,
\[ || K - K_n ||_2 < \frac{1}{n} \] 
Because $T - T_n$ is an integral operator with kernel $K - K_n$,
\[ || T - T_n || \le || K - K_n ||_2 < \frac{1}{n} \]
and therefore $T_n \to T$ in operator norm. However,
\[ (T_n f)(x) = \int_{\R^n} K_n(x, y) f(y) \, \d{y} = \sum_{i = 1}^n \int_{\R^n} \phi_i(x) \psi_i(y) f(y) \, \d{y} = \sum_{i = 1}^n \left( \int_{\R^n} \psi_i(y) f(y) \, \d{y} \right) \phi_i(x) \] 
and thus,
\[ T_n f = \sum_{i = 1}^n \inner{\bar{\psi}}{f} phi_i \]
showing that $\im{T_n} \subset \vspan{\phi_1, \dots, \phi_n}$ and therefore $T_n$ has finite rank. Therefore, $T$ is the norm limit of finite rank operators and thus is compact.

\subsection{3}

\begin{exercise}
Let $\S(\R)$ be the space of Schwartz functions and $\S'(\R)$ the tempered distributions. Let $\phi_n \in \S(\R)$ and suppose that $u \in \S'(\R)$ and that $\phi_n \to u$ in $\S'(\R)$.
\begin{enumerate}
\item Suppose there exists $C > 0$ such that $|| \phi_n ||_{L^2} < C$ for all $n$. Show that there exists $\phi \in L^2$ such that,
\[ u(\psi) = \int \psi \phi \, \d{x} \]
and $\phi_n \to \phi$ weakly in $L^2$.
\item Show that the analogous statement is false if $L^2$ is replace by $L^1$. Namely show that there exists $\phi_n$ and $u$ with $|| \phi_n ||_{L^1} < C$ such that $u$ is not given by any $L^1$ function $\phi$.
\end{enumerate}
\end{exercise}

\subsubsection{a}

For each $\psi \in \S(\R)$ we know that,
\[ u(\psi) = \lim_{n \to \infty} \int \phi_n \psi \, \d{x} \]
which is a continuous linear functional on $\S(\R)$. By the Riesz representation theorem, it suffices to extend $u$ to a continuous linear functional on $L^2(\R)$. We attempt to do so by defining,
\[ u(f) = \lim_{n \to \infty} \int \phi_n f \, \d{x} \]
and we need to show that this limit exists. For any $f \in L^2(\R)$ we can find a sequence $f_n \in \S(\R)$ such that $f_n \to f$ in $L^2$-norm. Then,
\[ \lim_{n \to \infty} \int \phi_n f \, \d{x} = \lim_{n \to \infty} \lim_{m \to \infty} \int \phi_n f_m \, \d{x} \]
where we know that,
\[ u(f_m) = \lim_{n \to \infty} \int \phi_n f_m \, \d{x} \]
exists because $f_m \in \S(\R)$. I will define,
\[ a = \lim_{m \to \infty} u(f_m) \]
and show that the limit exists by showing that $a = u(f)$ is this limit. Consider,
\[ \left| a - \int \phi_n f \, \d{x} \right| \le |a - u(f_m)| + \left| u(f_m) - \int \phi_n f_m \, \d{x} \right| + \left| \int \phi_n f_m \, \d{x} - \int \phi_n f \, \d{x} \right| \]
First notice that,
\[ \left| \int \phi_n f_m \, \d{x} - \int \phi_n f \, \d{x} \right| \le \int \phi_n | f - f_m | \, \d{x} \le || \phi_n ||_2 \cdot || f - f_m ||_2 \le C || f - f_m || \]
and therefore this term is bounded uniformly in $n$. Then for sufficiently large $m$ we have $| a - u(f_m) | < \epsilon / 3$ and $|| f - f_m || < \epsilon / (3 C)$ so we can choose $n$ large enough such that,
\[ \left| u(f_m) - \int \phi_n f_m \, \d{x} \right| < \frac{\epsilon}{3} \]
which is possible because,
\[ u(f_m) = \lim_{n \to \infty} \int \phi_n f_m \, \d{x} \]
exists. Then,
\[ \left| \int \phi_n f_m \, \d{x} - \int \phi_n f \, \d{x} \right| < \epsilon \]
showing that,
\[ a = \lim_{n \to \infty} \phi_n f \, \d{x} \]
exists so the limit exists and equals the limit of $u(f_m) \to u(f)$ so we hope that $u$ is continuous. Indeed,
\[ || u || = \sup_{|| f || = 1} |u(f)| \le \sup_{|| f || = 1} \lim_{n \to \infty} \left| \int \phi_n f \, \d{x} \right| \le \sup_{|| f|| = 1} || \phi_n || \cdot || f || < C \]
and therefore $|| u ||$ is bounded and thus continuous (it is clearly linear). Therefore, by the Riesz representation theorem there is some $\phi \in L^2(\R)$ such that,
\[ u(f) = \int \phi f \, \d{x} \]
Furthermore, by definition $\phi_n \to \phi$ weakly in $L^2$ because,
\[ \inner{\phi}{f} = \int \phi f \, \d{x} = \lim_{n \to \infty} \int \phi_n f \, \d{x} = \lim_{n \to \infty} \inner{\phi_n}{f} \]

\subsubsection{b}

Consider,
\[ \phi_n = n \chi_{[-\frac{1}{n}, \frac{1}{n}]} \]
which has $|| \phi_n ||_{L^1} = 1$. Let $u = 2 \delta_0$ so,
\[ u(\psi) = 2 \psi(0) \]
which exists because $\psi \in \S(\R)$ is continuous. Furthermore, because $\psi$ is continuous,
\[ \lim_{n \to \infty} \inner{\phi_n}{\psi} = \lim_{n \to \infty} \int \phi_n \psi \, \d{x} = \lim_{n \to \infty} n \int_{-\frac{1}{n}}^{\frac{1}{n}} \psi \, \d{x} = 2 \psi(0) = u(\psi) \]
meaning that $\phi_n \to u$ in the weak topology on $\S'(\R)$. However, $u$ cannot be given by an $L^1$ function.

\subsection{4 (FUCK GOD)}

\begin{exercise}
Suppose that $X \subset Z$ and $Y \subset V$ are all Banach spaces with continuous inclusions. Suppose that $P : Z \to V$ is a bounded operator with the property that for $u \in Z$ and $P u \in Y$ implies that $u \in X$. Show that there is $C > 0$ such that for all $u \in Z$ satisfying $P u \in Y$ one has,
\[ || u ||_X \le C ( || P u ||_Y + || u ||_Z ) \]
\end{exercise}

Assume otherwise. Then there is a sequence $u_n \in Z$ with $P u_n \in Y$ and $|| u_n ||_Z \to 0$ and $|| P u_n ||_Y \to 0$ and $|| u_n ||_X = 1$.

Let $K = P^{-1}(Y) \subset X$ with the $Z$ topology. Then consider the graph of $P : K \to Y$ inside $K \times Y$. I claim that this graph is closed. If $x_n \to x$ is a sequence converging in $K$ and $P x_n \to y$ in $Y$ we need to show that $P x = y$. Because $\iota P : K \to V$ is continuous $\iota P x_n \to \iota P x$ in $V$ but the inclusion is also continuous so $\iota P x_n \to \iota y$ and therefore $P x = y$ because the inclusion map $\iota : Y \to V$ is continuous. Therefore, this graph is closed. 



 Now consider $Q : (K, || \bullet ||_X) \to K \times Y$ sending $u \mapsto (u, P(u))$ but notice that the domain $K$ is given the $X$ not the $Z$ topology while the codomain $K$ is given the $Z$ topology. Clearly $Q$ is injective and the image of $Q$ is $\Gamma(P)$ which is closed in $K \times Y$. 


First, $\ker{P} \subset X$ because $P u = 0 \in Y$ if $u \in \ker{P}$ and $\ker{P}$ is closed in the $Z$ topology. Consider the continuous bijective map $\mathcal{I} : (\ker{P}, || \bullet ||_X) \to (\ker{P}, || \bullet ||_Z)$. Because $\ker{P}$ is closed in the $Z$-topology it is also closed in the $X$-topology (the $X$ topology is finer because if $\iota :  X \to Z$ is the inclusion then $\iota^{-1}(C)$ is closed for any closed $C \subset Z$). Thus $\mathcal{I}$ is a bijective continuous map of Banach spaces and therefore invertible by the bounded inverse theorem. Thus there exists some $C_1 > 0$ such that,
\[ || x ||_X \le C_1 || x ||_Z \]
for all $x \in \ker{P}$. 

Therefore, consider $\tilde{P}  : X / \ker{P} \to V$ which is a continuous linear operator between Banach spaces and is injective. 



\subsection{5 (HOW THE HELL DOES THIS WORK??)}

Let $S^1 = \R / \Z$. 

\subsubsection{a}

\begin{exercise}
Suppose that for a given integrable function $\phi \in L^1(S^1)$ (specify a function in its $L^1$-class) and $\alpha \in \R \setminus \Z$ such that $\phi(x) = \phi(x + \alpha)$ for almost all $x \in S^1$. Show that there is a constant function $c$ such that $\phi \eqae c$.
\end{exercise}

Let, 
\[ c = \int_{S^1} \phi \, \d{x} \]
then by taking $\tilde{\phi} = \phi - c$ we may assume that $\phi$ has zero integral.
\bigskip\\
Now by Lusin's theorem, for any $\delta > 0$ there exists a measurable $E \subset S^1$ such that $\mu(E) < \delta$ and a continuous function $f_\delta$ such that $f_\delta = \phi$ on $S^1 \setminus E$ and $|| f ||$. Now, let $N = \{ x \in S^1 \mid \phi(x + \alpha) \neq \phi(x) \}$ which has $\mu(N) = 0$ by hypothesis. Let,
\[ \tilde{E} = E \cup \bigcup_{n = 0}^\infty (N - n \alpha) \]
such that if $x \notin \tilde{E}$ then by induction $\phi(x + n \alpha) = \phi(x)$ for $n \in \N$ indeed the case $n = 0$ is obvious and $x \notin N - n \alpha$ so 
\[ \phi(x + (n+1)\alpha) = \phi(x + n \alpha + \alpha) = \phi(x + n \alpha) = \phi(x) \]
since $x + n\alpha \notin N$ and the induction hypothesis. Furthermore, since $\mu(N) = 0$ we have $\mu(\tilde{E}) = \mu(E)$. Therefore, for all $x \in S^1 \setminus \tilde{E}$ we have $f(x) = \phi(x)$ and $\phi(x + n \alpha) = \phi(x)$.
\bigskip\\
Let $D = \{ n \alpha \}$. I claim that for each $x_0 \in S^1 \setminus \tilde{E}$ and for almost all $x \in S^1$ the set $(D + x) \cap S^1 \setminus \tilde{E}$ is dense at $x_0$. If $(D + x) \cap S^1 \setminus \tilde{E}$ is not dense at $x_0$ then there is some $n$ such that if $y \in D + x \cap B_{\frac{1}{n}}(x_0)$ then $y \in \tilde{E}$



\section{Spring 2019 Part I}

\subsection{1 (LOOK UP ABSOLUTE CONTINUITY)}

\subsection{2 (DO THIS MORE PRECISELY)}

\begin{exercise}
Let $(X, \F, \mu)$ be a probability space and put gaussian random varibles $f_n : X \to \R$ on $X$. Explicitly this means that the PDF of each $f_n$ is.
\[ p(y) = \frac{1}{\sqrt{2 \pi}} e^{-t^2/2} \]
and therefore,
\[ \mu(\{x \mid f_n(x) < y \}) = \int_{-\infty}^y e^{-t^2/2} \frac{\d{t}}{\sqrt{2 \pi}} \]
Show that for almost all $ \in X$ there exist only finitely many $n \in \N$ such that $|f_n(x)| > 2019 \sqrt{\log{n}}$.
\end{exercise}

Let $E_n = \{ x \in X \mid |f_n(x)| > 2019 \sqrt{\log{n}} \}$ and then the set in question is,
\[ E = \{ x \in X \mid |f_n(x)| > 2019 \sqrt{\log{n}} \text{ infinitely often } \} = \limsup_{n \to \infty} E_n = \bigcap_{n = 1}^\infty \bigcup_{k \ge n} E_k \]
Then by the Borel-Cantelli lemma, if
\[ \sum_{n = 1}^\infty E_n < \infty \]
then,
\[ \mu(E) = \mu \left( \limsup_{n \to \infty} E_n \right) = 0 \]
which is what we need to show. Therefore, it suffices to consider,
\[ \sum_{n = 1}^\infty \mu(E_n) \]
Notice that,
\[ E_n = \{ x \in X \mid f_n(x) > 2019 \sqrt{\log{n}} \} \cup \{ x \in X \mid f_n(x) < - 2019 \sqrt{\log{n}} \} \]
Therefore,
\[ \mu(E_n) = \int_{-\infty}^{-2019 \sqrt{\log{n}}} e^{-t^2/2} \frac{\d{t}}{\sqrt{2 \pi}} + \int_{2019 \sqrt{\log{n}}}^{\infty} e^{-t^2/2} \frac{\d{t}}{\sqrt{2 \pi}} = 2 \int_{2019 \sqrt{\log{n}}}^{\infty} e^{-t^2/2} \frac{\d{t}}{\sqrt{2 \pi}} \]
For large $n$ this decays more rapidly than $e^{-(2019\sqrt{\log})^2/2} = n^{-\frac{2019^2}{2}}$
which is summable and therefore $\mu(E_n)$ is summable proving the claim.

\subsection{3}

\subsection{4}

\subsubsection{a}

Suppose that $A$ is a Hilbert-Schmidt on a separable Hilbert space $H$. Then consider the operator,
\[ A_n e_k = 
\begin{cases}
A e_k & k \le n
\\
0 & k > n
\end{cases} \]
Then consider,
\[ || A - A_n || = \sup_{|| x || = 1} || (A - A_n) x || \]
However,
\[ || (A - A_n) x || = \sum_{k > n} \inner{e_k}{x} A e_k \]

\subsection{5}

\subsubsection{a}

Let $s > \frac{1}{2}$ and $f \in H^s(\R)$ then consider,
\[ || f ||_{\infty} = \sup_{x \in \R} |f(x)| = \sup_{x \in \R} \left| \int_\R \hat{f}(\xi) e^{i \xi x} \, \d{\xi} \right| \]
However,
\begin{align*}
\left| \int_\R \hat{f}(\xi) e^{i \xi x} \, \d{\xi} \right| \le \int_\R | \hat{f}(\xi) | \, \d{\xi} = \int_{\R} \frac{(1 + |\xi|^2)^{\frac{s}{2}}}{(1 + |\xi|^2)^{\frac{s}{2}}} | \hat{f}(\xi) | \, \d{\xi} \le \left( \int_{\R} (1 + | \xi |^2)^s | \hat{f}(\xi) |^2 \, \d{\xi} \right)^{\frac{1}{2}} \left( \int_{\R} (1 + |\xi|^2)^{-s} \, \d{\xi} \right)^{\frac{1}{2}}
\end{align*}
Therefore,
\[ || f ||_{\infty} \le || f ||_{H^s} C_s \]
where,
\[ C_s = \left( \int_{\R} (1 + |\xi|^2)^{-s} \, \d{\xi} \right)^{\frac{1}{2}} \]
is finite because $s > \frac{1}{2}$. 

\subsubsection{b (DO THIS ONE!!!)}

Let $u,v \in H^1(\R)$

\section{Spring 2018 Part I}


\subsection{2}

Let $S \subset C([0,1])$ be closed and suppose that if $f \in S$ then $f$ is continuous differentiable. Thus $S \subset C^1([0,1])$ consider the map $\iota : C^1([0,1]) \to C([0,1])$ which is countinuous because,
\[ || f ||_{C^1} = || f ||_{C^0} + || f' ||_{C^0} \]
and therefore,
\[ || f ||_{C^0} \le || f ||_{C^1} \]
Thus, $S \subset C^1([0,1])$ is closed since it is the preimage. Consider $B(S)$ the unit ball of $S$ in the $C^1([0,1])$ norm. To show that $B(S)$ is compact by Arzela-Ascoli it suffices to show that it is bounded and pointwise equicontinuous.  Boundedness is automatic since $f(x) \le || f ||_{C^0} \le || f ||_{C^1} \le 1$ for all $f$ and for all $x$. Furthermore, 
\[ | f(x) - f(y) | \le || f' ||_{C^0} | x - y | \le | x - y | \]
since $|| f' ||_{C^0} \le || f ||_{C^1} \le 1$ and therefore for $| x - y | < \epsilon$ we have $|f(x) - f(y)| < \epsilon$ for all $f$ and all $x, y \in [0,1]$ so actually $S$ is uniformly equicontinuous. Therefore $\iota(B(S))$ is precompact so $\iota$ is compact. Furthermore $(S, || \bullet ||_{C_1})$ and $(S, || \bullet ||_{C^0})$ are both Banach spaces and $\iota$ is bijective so by the bounded inverse theorem we see that $\iota$ is an isomorphism and therefore the unit ball of $(S, || \bullet ||_{C^1})$ is precompact in the $C^1$ topoology and thus $S$ is finite dimensional because $(S, || \bullet ||_{C^1})$ is a Banach space.

\subsection{3}

\subsubsection{a}

Standard argument for Volterra operator bounded. Use Stone-Weierstrass to show that product functions are dense in $C_0(\R^2)$ which is dense in $L^2(\R^2)$ to show that we can approximate in $L^2$-norm $K$ by finite sums of product functions and these define finite rank operators converging in operator norm so $T$ is compact.

\subsubsection{b}

This is a general fact about bounded operators $T : X \to X$. Consider the operator $I + \alpha T$ and we want to see when $I + \alpha T$ is bijective i.e. for all $g \in X$ there exists a unique solution $f \in X$ to $(I + \alpha T) f = g$. By bounded inverse this is equivalent to asking that $I + \alpha T$ be an isomorphism. For $\alpha = 0$ this is obvious. For $\alpha \neq 0$,
\[ I + \alpha T = \alpha (T - \lambda I) \]
where $\lambda = (-\alpha)^{-1}$ and thus this is invertible iff $\lambda \in \sigma(T)$. Since $T$ is bounded we know that $\sigma(T)$ is bounded by $|| T ||$ and thus if $| \alpha | < || T ||^{-1}$ we win. 

\subsubsection{c}

Suppose that $\inner{f}{Tf} \ge 0$ for all $h \in L^2_\R(\R)$.Obviously for $\alpha = 0$ that $T_\alpha = I + \alpha T$ is invertable. Since $T$ is compact its spectrum away from $0$ consists entirely of eigenvalues which are discrete. Therefore, we need to show there are no eigenvalues on the negative real axis. Indeed, if $T f = \lambda f$ for $\lambda \in \R$ then,
\[ \inner{f}{T f} = \lambda \inner{f}{f} \ge 0 \]
but $\inner{f}{f} = || f ||^2 > 0$ so $\lambda \ge 0$ and therefore we win.

\subsection{4 DO THIS!!}

Let $L^2 = L^2((0, \infty), x^{-1} \d{x})$. 

\subsection{5}

Let $X$ be a Banach space of $\C$ and $M, N \subset X$ closed subspaces of $X$. 

\subsubsection{a}

If $M + N \subset X$ is closed then consider the map $M \times N \to M + N$ via $(m,n) \mapsto m + n$ is continuous because,
\[ || m + n || \le || m || + || n || \]
Then $(M \times N)/(M \cap N) \to M + N$ is a bijection and both are Banach spaces so it is an isomorphism. Therefore the map is bounded below so there is some $C$ such that,
\[ || [m, n] || = \inf_{y \in M \cap N} \left( || m + y || + || n - y|| \right) \le C || m + n || \]
and thus for each $x \in M + N$ there exists $(m,n) \in M + N$ such that,
\[ || m || + || n || \le 2 \cdot || [m, n] || \le 2 C || x || \]
proving the claim. Conversely, if such a constant exists then the map,
\[ (M \times N)/(M \cap N) \to X \]
is bounded below and thus its image $M + N$ is closed. 

\subsubsection{b}


Let $\ell_M : M \to \C$ and $\ell_N : N \to \C$ are continuous and linear and $\ell_M |_{M \cap N} = \ell_N |_{M \cap N}$. Suppose that $M + N$ is closed. From the isomorphism $(M \times N)/(M \cap N) \iso M + N$ we can extend to $\ell : M + N \to \C$ since $\ell$ is well-defined on $(M \times N)/(M \cap N)$ because $\ell_M(x) + \ell_N(-x) = 0$ if $x \in M \cap N$. Then by Hahn-Banach we can extend $\ell$ to the entire space. 

\subsubsection{c}

\section{Spring 2018 Part II}

\subsection{1}

Let $f$ be a non-negative Lebesgue measurable function on $[0,1]$ such that $f > 0$ a.e. and consider $|| f ||_{L^1} > 0$ (it may be infinite) because if $|| f ||_{L^1} = 0$ then $f = 0$ a.e. Suppose that the conclusion is false. Then there is $\epsilon > 0$ such that for every $\delta > 0$ there is $E_\delta \subset [0,1]$ with $\mu(E_\delta) \ge \epsilon$ such that,
\[ \nu(E_\delta) = \int_{E_\delta} f(x) \, \d{x} < \delta \]
where we define a measure,
\[ \nu(A) = \int_A f(x) \, \d{\mu} \]
Consider the sequence $E_n = E_{\frac{1}{2^n}}$. Then consider,
\[ E = \limsup_{n \to \infty} E_n = \bigcap_{n = 1}^\infty \bigcup_{k = n}^\infty E_n \]
Since $\nu(E_n) < \frac{1}{2^n}$ the sequence is summable and therefore by Borel-Cantelli, $\nu(E) = 0$. However, because $\mu$ is finite,
\[ \mu(E) = \lim_{n \to \infty} \mu \left( \bigcup_{k = n}^\infty E_n \right) \ge \lim_{n \to \infty} \mu(E_n) \ge \epsilon \]
Therefore we see that $\mu \not\ll \nu$ however because $f > 0$ a.e. we see that $\frac{1}{f}$ exists and is measurable a.e. and thus,
\[ \mu(A) = \int_{A} \frac{1}{f(x)} \, \d{\nu} \]
proving that $\mu \ll \nu$ giving a contradiction.

\subsection{2}

Let $X$ be a non-zero Banach space and $T \in \L(X)$.

\subsubsection{a}

Suppose that $\lambda_n \in \rho(T)$ and $\lambda_n \to \lambda$ with $\lambda \in \sigma(T)$. For any point $\eta \in \rho(T)$ we know that,
\[ d(\eta, \sigma(T)) \ge || (T - \eta I)^{-1} ||^{-1} \]
and therefore if $\lambda_n \to \lambda$ we see that $d(\lambda_n, \sigma(T)) \to 0$ which implies that,
\[ \lim_{n \to \infty} || (T - \lambda_n I)^{-1} ||^{-1} = 0 \]
proving the claim.

\subsubsection{b}

Suppose that for each $\lambda \in \sigma(T)$ the operator $T - \lambda I$ is bounded below. Notice that if $T - \lambda I$ is invertible then,
\[ || x || = || (T - \lambda I)^{-1} (T - \lambda I) x || \le || (T - \lambda I)^{-1} || \cdot || (T - \lambda I) x || \]
and therefore, $(T - \lambda I)$ is bounded below with constant $(T - \lambda I)^{-1}$. Because $\sigma(T)$ is nonempty and compact it has a nonempty boundary so choose $\lambda \in \partial \sigma(T)$ then I claim that $(T - \lambda I)$ is not bounded below. Choose a sequence $\lambda_n \in \rho(T)$ such that $\lambda_n \to \lambda$. Then by (a) there is a subsequence such that,
\[ || (T - \lambda_n I)^{-1} || > n \]
and therefore there exists a sequence $x_n \in X$ with $|| x_n || = 1$ such that $|| (T - \lambda _n I)^{-1} x_n || > n$ and let $y_n = [(T - \lambda_n I)^{-1} x_n]/||(T - \lambda_n I)^{-1} x_n||$. Therefore, $|| y_n || = 1$ and,
\[ || (T - \lambda_n I) y_n || = \frac{|| x_n ||}{|| (T - \lambda_n I)^{-1}||} < \frac{1}{n} \]
Now,
\[ || (T - \lambda I) y_n || \le || (T - \lambda_n I) y_n || + || (\lambda - \lambda_n) y_n || < \frac{1}{n} + | \lambda - \lambda_n | \]
and therefore,
\[ \lim_{n \to \infty} || (T - \lambda I) y_n || = 0 \]
while $|| y_n || = 1$ proving that $T - \lambda I$ is not bounded below.

\subsection{3 (HOW TO DO THIS)}

\subsection{4}

\subsubsection{a}

Let $X$ be a Banach space and $E \subset X^*$ a weak-$*$ closed subspace such that,
\[ \bigcap_{\lambda \in E} \ker{\lambda} = (0) \]
Consider $\ell \in X^*$. Then for any finite dimensional subspace $V \subset X$ we can choose $\ell_V \in E$ such that $\ell|_V = \ell_V$. Indeed, for any nonzero $v \in X$ there must be some $\ell \in E$ with $\ell(v) \neq 0$ and thus $E \onto V^*$ since $V$ is finite dimensional so choosing a basis $e_i \in V$ I can find $\ell_i \in E$ with $\ell_i(e_j) = \delta_{ij}$. Then $\ell_V \to \ell$ pointwise so by weak-$*$ closure $\ell \in E$ so $E = X^*$.

\subsubsection{b}

Let $X, Y$ be Banach spaces and $T \in \L(X, Y)$ and $T^* \in \L(Y^*, X^*)$ is adjoint. If $T$ is injective then the weak-$*$ closure of $\im{T^*}$ is $(\ker{T})^\perp = X^*$ so it is dense. It $\im{T^*}$ is weak-$*$ dense then $(\ker{T})^\perp = X^*$ and thus because $\ker{T}$ is closed $\ker{T} = (\ker{T})^{\perp \perp} = (X^*)^\perp = (0)$ because if $\ell(x) = 0$ for all $\ell \in X^*$ then $x = 0$ by Hahn-Banach. 


\subsection{5 HOW TO DO THIS!!}

\section{Spring 2019 Part II}

\subsection{4}

Define $X_j$ as the completion of $\S(\R^2)$ with respect to the norm,
\[ || u ||_j^2 = || u ||_{L^2}^2  +|| \partial_j u ||^2_{L^2} \]
Clearly the map $\S(\R^2) \to L^2(\R^2)$ is continuous where $\S(\R^2)$ is given the $j$ norm. Furthermore since the inclusion is bounded it sends Cauchy sequences to Cauchy sequences and since $L^2(\R^2)$ is complete we get a continuous $X_j \to L^2(\R^2)$. We need to show this is injective. Consider $V_j \subset L^2$ the subspace of $f \in L^2$ with $\partial_j f \in L^2$ given the norm,
\[ || f ||_{V_j}^2 = || f ||_{L^2} + || \partial_j f ||_{L^2} \]
Then $\S(\R^2) \to L^2(\R^2)$ lands in $V_j$ and $\S(\R^2) \to V_j$ is an isometry for the $j$ norm. Furthermore, $V_j$ is complete and $\S(\R^2)$ is dense because after Fourier transforming $V_j = L^2(\mu (1 + x_j^2)) \subset L^2$. Therefore $\S(\R^2) \to V_j$ extends to an isomorphism $X_j \to V_j$ and thus an injective continuous map $X_j \to L^2(\R^2)$. It is obvious by definition that $H^1(\R^2) = V_1 \cap V_2$ and these are the images of $X_1$ and $X_2$.

\section{Fall 2019 Part I}

\subsection{1}

Let $H$ be a Hilbert space. Suppose that $T$ is invertible. Then, $||T^{-1}||$ is bounded so we have,
\[ || x || \le || T^{-1} (T x) || \le || T^{-1} || \cdot || T x || \]
and $T^{-1} \neq 0$ so $|| T^{-1} || > 0$ proving that $T$ is bounded. Furthermore, $T^*$ is also invertible so the same argument shows that $T^*$ is bounded below.
\bigskip\\
Now suppose that $T$ and $T^*$ are bounded below. By the bounded inverse theorem (a corollary of the open mapping theorem) it sufices to show that $T$ is bijective then $T^{-1}$ is also bounded. 
\bigskip\\
First, suppose that $T x = 0$ then $|| T x || = 0$ but by hypothesis, 
\[ c || x || \le || T x || \]
for some $c > 0$ so $|| x || = 0$ and thus $x = 0$ meaning that $T$ is injective. 
\bigskip\\
The same argument shows that $T^*$ is injective so both $T T^*$ and $T^* T$ are injective. However, $\overline{\im{T}} = (\ker{T^*})^\perp = H$ because $\ker{T^*} = (0)$. Therefore, $T$ and (for the same reason) $T^*$ have dense image. Therefore, it suffices to show that $T$ has closed image. Indeed, if $T x_n \to y$ is a convergence sequence then $T x_n$ is Cauchy so,
\[ c || x_n - x_m || \le || T x_n - T x_m || < \epsilon \]
for $n, m > N$ meaning that $x_n$ is a Cauchy sequence so $x_n \to x$ in $H$. Because $T$ is bounded it is continuous and thus,
\[ Tx = T \left( \lim_{n \to \infty} x_n \right) = \lim_{n \to \infty} T x_n = y \]
so $y \in \im{T}$. Therefore $\im{T}$ is closed so $\im{T} = H$ and therefore $T$ is invertible.

\subsection{2}

\begin{exercise}
Let $X \subset \R$ be a Borel set and $\mu$ the Lebesgue measure. Suppose there exists $1 \le p < q < + \infty$ such that $L^q(X, \mu) \subset L^p(X, \mu)$. Prove that $\mu(X) < + \infty$.
\end{exercise}

Let $\iota : L^q(X, \mu) \to L^p(X, \mu)$ be the inclusion. I claim that $\iota$ is bounded. By the closed graph theorem, it suffices to show that $\Gamma(\iota)$ is closed. Explicitly, this means if $f_n \to f$ in $L^q(X, \mu)$ and $f_n \to f'$ in $L^p(X, \mu)$ then $f = f'$ in $L^p(X, \mu)$ (i.e. $f \eqae f'$). Indeed, $f_n$ has a subsequence $f_{n_j}$ such that $f_{n_j} \to f$ pointwise a.e. and therefore $f_{n_j} \to f'$ in $L^p$-norm. Consider $g_j = |f_{n_j} - f'|^p$ then by Fatou's lemma,
\[ \int_X \liminf_{j \to \infty} g_j \, \d{\mu} \le \liminf_{j \to \infty} \int_X g_j \, \d{\mu} = 0 \]
because $f_{n_j} \to f'$ in $L^p$-norm. However, 
\[ \liminf_{j \to \infty} g_j = \liminf_{j \to \infty} |f_{n_j} - f'|^p \eqae |f - f'|^p \]
and therefore,
\[ \int_X |f - f'|^p \, \d{\mu} = 0 \]
which implies that $f \eqae f'$.
\bigskip\\
Now, assume that $\mu(X) = \infty$. Then by inner regularity of $\mu$ there exist compact $K \subset X$ with arbitrarily large measure. Then consider the constant function $f \in C(K)$ such that $f(x) = 1$ then,
\[ ||\iota || \ge \frac{|| f ||_p}{|| f ||_q} = \frac{\mu(K)^{\frac{1}{p}}}{\mu(K)^{\frac{1}{q}}} = \mu(K)^{\frac{1}{r}} \]
where,
\[ \frac{1}{r} = \frac{1}{p} - \frac{1}{q} > 0 \]
Since $\mu(K) \to \infty$ we see that $\iota$ cannot be bounded contradicting the above argument. Therefore, $\mu(X) < \infty$. 

\subsection{3}

Let $X$ be a reflexive Banach space and $V \subset X$ closed. Consider the canonical map $\iota : V^{**} \to X^{**}$. For any $\psi \in V^{**}$ we need to show there is some $v \in V$ such that $\psi = \ev_v$. Because $X$ is reflexive, there is some $x \in X$ such that $\iota(\psi) = \ev_x$ meaning that for any $\ell \in X^*$ we have $\psi(\ell|_V) = \ell(x)$. First, I claim that $x \in V$. Otherwise, by Hahn-Banach (using that $V$ is clsoed), I can choose $\ell \in X^*$ with $\ell|_V = 0$ but $\ell(x) \neq 0$ contradicting $\psi(\ell|_V) = \ell(x)$. Thus $x \in V$. Now I claim that $\psi = \ev_x$. Indeed, for any $\ell \in V^*$ there exists an extension to $\ell' \in X^*$ by Hahn-Banach such that $\ell'|_V = \ell$. Then,
\[ \psi(\ell) = \psi(\ell'|_V) = \iota(\psi)(\ell') = \ev_x(\ell') = \ev_x(\ell) \]
where $\ev_x(\ell') = \ev_x(\ell)$ because $x \in V$. Therefore, $V$ is reflexive.

\subsection{4}

\begin{exercise}
Let $f \in \D'(\R)$ where $\D'(\R)$ is the space of distributions on $\R$.
\begin{enumerate}
\item Show that there exists $u \in \D'(\R)$ such that $u' = f$.
\item Show that if $v' = f$ as well, then $u - v$ is a constant distribution.
\end{enumerate}
\end{exercise}

\subsubsection{a}

I claim that a test function $\varphi \in C^\infty_c(\R)$ admits an antiderivative $\eta \in C^\infty_c(\R)$ (meaning that $\eta' = \varphi$) if and only if,
\[ \int_\R \varphi \, \d{x} = 0 \]
Indeed, if $\eta' = \varphi$ then,
\[ \int_\R \varphi \, \d{x} = \lim_{x \to \infty} \eta(x) - \eta(-x) = 0 \]
because $\eta$ has compact support. Alternatively, define,
\[ \eta(x) = \int_{-\infty}^x \varphi(x) \, \d{x} \]
which is clearly smooth. It suffices to show that $\eta$ has compact support. Let $K = \supp{}{\varphi}$. For $x < \inf \supp{}{\varphi}$ clearly $\eta(x) = 0$. Likewise, for $x > \sup \supp{}{\varphi}$,
\[ \eta(x) = \int_{-\infty}^x \varphi(x) \, \d{x} = \int_{-\infty}^{\infty} \varphi(x) \, \d{x} + \int_{x}^\infty \varphi(x) \, \d{x} = 0 \]
by hypothesis and the fact that $\varphi|_{[x, \infty)} = 0$. Therefore, since $\supp{}{\varphi}$ is bounded $\supp{}{\eta}$ is also bounded and therefore compact.
\bigskip\\
Now choose a fixed bump function $\psi_0 \in C^\infty_c(\R)$ with,
\[ \int_\R \psi_0 \, \d{x} = 1 \]
Then for any $\varphi \in C_c^\infty(\R)$ we can write,
\[ \varphi = \eta' + c \psi_0 \]
where,
\[ c = \int_\R \varphi(x) \, \d{x} \]
because then $\varphi - c \psi_0$ has zero integral and thus admits an antiderivative $\eta \in C^\infty_c(\R)$. Then define,
\[ \inner{u}{\varphi} = - \inner{f}{\eta} \]
This is clearly linear and continuous (limits in $\C_c^\infty$ are stronger than uniform so they commute with integrals) so it defines a distrbution $u \in \D'(\R)$. Furthermore $u' = f$ because,
\[ \inner{u'}{\varphi} = - \inner{u}{\varphi'} = \inner{f}{\varphi} \]
because $\varpi \in C^\infty_c$ is the unique compactly supported anti-derivative of $\varphi'$. 

\subsubsection{b}

Suppose that $u' = v' = f$ for two distributions $u,v \in \D'(\R)$. Then for any $\varphi \in C^\infty_c(\R)$ write it as $\varphi = \eta' + c \psi_0$ for a unique antiderivative $\eta \in C^\infty_c(\R)$. Therefore,
\[ \inner{u - v}{\varphi} = \inner{u - v}{\eta'} + c \inner{u - v}{\psi_0} = - \inner{u' - v'}{\eta} + c \inner{u - v}{\psi_0} = c \inner{u - v}{\psi_0} \]
because $u' = v' = f$. Therefore,
\[ \inner{u - v}{\varphi} = \inner{u - v}{\psi_0} + \int_\R \varphi \, \d{x} \]
and thus $u - v = \inner{u - v}{\psi_0} 1$ where $1$ is the constant distribution,
\[ \inner{1}{\varphi} = \int_\R \varphi \, \d{x} \]

\subsection{5 (DO THIS IN PARTICULAR!!)}

\subsubsection{a}

\begin{exercise}
Show that there is no continuous map,
\[ P : L^2(\R^n) \times L^2(\R^n) \to L^2(\R^n) \]
such that $P(u,v) = uv$ for $u, v \in C^\infty_c(\R^n)$. (I think they meant $C^\infty_0(\R^n)$ becasuse they say compactly supported smooth function though if they did mean $C^\infty_0(\R^n)$ that is fine either are dense).
\end{exercise}

Let $\psi_n$ be a bump function such that $\psi_n(x) = 1$ when $\frac{1}{n} < |x| < 1$.
Consider the sequence,
\[ f_k(x) = |x|^{\frac{n-1}{2}-\frac{1}{4}} \psi_{n}(x) \]
Then I claim that $f_n \to f$ where 
\[ f(x) = |x|^{\frac{n-1}{2}-\frac{1}{4}} \psi \]
 is in $L^2$ where $\tilde{\phi}$ is the pointwise (and uniform) limit of $\psi_n$ which is a bump function such that $\psi(x) = 1$ for $|x| < 1$. Therefore, $f_k \to f$ converges in $L^2$. However,
\[ P(f_k, f_k) = f_k^2 \]
does not converge in $L^2$. To see this, consider,
\[ || f_k^2 ||^2_2 = \int_{\R^n} |f_k(x)|^2 \, \d{x} \ge C_n \int_{\frac{1}{k}}^1 r^{n-1} r^{-2(n-1) - 1} \d{r} = C_n \int_{\frac{1}{k}}^1 r^{-n} \, \d{r} = \frac{C_n}{n+1} \left[ k^n - 1 \right] \xrightarrow{k \to \infty} \infty \] 
where $C_n$ is the ``surface area'' of the unit sphere in $\R^n$.

\subsubsection{b (HOW TO GET THE CORRECT BOUND)}

We know that $\F (uv) = (\F u) * (\F v)$ for any $u,v \in C^\infty_c(\R^n)$. Therefore, it suffices to show that for $u,v \in H^s(\R^n)$ with $s > \frac{n}{2}$ that the convolution  $(\F u) * (\F v)$ exists and thus defines the required map through the Fourier isomorphism on $L^2(\R^n)$ functions. Indeed consider,
\begin{align*}
\int_{\R^n} ( 1 + |\xi|^2)^s | (\hat{u} * \hat{v}) (\xi) |^2 \, \d{\xi} &=  \int_{\R^n} (1 + |\xi|^2)^s \left| \int_\R \hat{u}(\xi') \hat{v}(\xi - \xi') \, \d{\xi'} \right|^2 \, \d{\xi} 
\\
& = \int_{\R^n} (1 + |\xi|^2)^{-s} \left| \int_{\R^n} (1 + |\xi|^2)^s \hat{u}(\xi') \hat{v}(\xi - \xi') \, \d{\xi'} \right|^2 \, \d{\xi}  
\end{align*}
Therefore, consider,
\begin{align*}
\left| \int_{\R^n} (1 + |\xi|^2)^s \hat{u}(\xi') \hat{v}(\xi - \xi') \, \d{\xi'} \right| & \le \int_{\R^n} (1 + |\xi|^2)^s |\hat{u}(\xi') \hat{v}(\xi - \xi')| \, \d{\xi'} 
\\
& \le C \left( \int_{\R^n} (1 + |\xi'|^2)^s |\hat{u}(\xi') \hat{v}(\xi - \xi')| \, \d{\xi'} + \int_{\R^n} (1 + |\xi - \xi'|^2)^s |\hat{u}(\xi') \hat{v}(\xi - \xi')| \, \d{\xi'} \right)
\end{align*}
However,
\begin{align*}
\int_{\R^n} (1 + |\xi'|^2)^s |\hat{u}(\xi') \hat{v}(\xi - \xi')| \, \d{\xi'} & \le \left( \int_{\R^n} (1 + |\xi'|^2)^{2s} |\hat{u}(\xi')|^2 \, \d{\xi'} \right)^{\frac{1}{2}} \cdot \left( \int_{\R^n} |\hat{v}(\xi - \xi')|^2 \, \d{\xi'} \right)^{\frac{1}{2}}
\\
& = || u ||_{H^{2s}} || v ||_{L^2} 
\end{align*}
and likewise,
\begin{align*}
\int_{\R^n} (1 + |\xi - \xi'|^2)^{s}  |\hat{u}(\xi') \hat{v}(\xi - \xi')| \, \d{\xi'} & \le \left( \int_{\R^n} |\hat{u}(\xi')|^2 \, \d{\xi'} \right)^{\frac{1}{2}} \cdot \left( \int_{\R^n} (1 + |\xi - \xi'|^2)^{2s}  |\hat{v}(\xi - \xi')|^2 \, \d{\xi'} \right)^{\frac{1}{2}} 
\\
& = || u ||_{L^2} || v ||_{H^{2s}} 
\end{align*}
Therefore,
\begin{align*}
\int_{\R^n} ( 1 + |\xi|^2)^s | (\hat{u} * \hat{v}) (\xi) |^2 \, \d{\xi} & \le C^2 (|| u ||_{H^{2s}} || v ||_{L^2} + || u ||_{L^2} || v ||_{H^{2s}})^2  \int_{\R^n} (1 + |\xi|^2)^{-s} \, \d{\xi} 
\end{align*}
is finite because $s > \frac{n}{2}$.

THIS ONLY WORKS IF $u,v \in H^{2s}(\R^n)$

\section{Fall 2019 Part II}

\subsubsection{1}

\subsubsection{a}

\begin{exercise}
Let $X,Y$ be compact Hausdorff spaces. Let $D$ be the linear span of functions of the form $u(x,y) = \phi(x) \psi(y)$ for $\phi \in C(X)$ and $\psi \in C(Y)$. Show that $D \subset C(X \times Y)$ is dense.
\end{exercise}

By Stone-Weierstrass, it suffices to show that $D$ separates points and vanishes nowhere. Let $(x_1, y_1), (x_2, y_2) \in X \times Y$ be two points. By Urysohn's lemma, because $X, Y$ are compact Hausdorff and thus regular, there exist $\phi \in C(X)$ such that $\phi(x_1) = 0$ and $\phi(x_2) = 1$ and $\psi \in C(Y)$ such that $\psi(y_1) = 0$ and $\psi(y_2) = 1$. Then consider $u(x,y) = \phi(x) \psi(y) \in D$. We have $u(x_1, y_1) = 0$ but $u(x_1, y_2) = 1$ so $D$ separates points. Futhermore from the constant functions it is obvious that $D$ is nowhere vanishing. Therefore, $D$ is dense by the Stone-Weierstrass theorem.

\subsubsection{b}

Let $X$ be a separable Hilbert space and $K : X \to X$ a compact operator. For each $\epsilon > 0$ there is a finite cover $\{ B_\epsilon(y_i) \}$ of $K(B)$ by $\epsilon$-balls because $K(B)$ is precompact where $B = \{ x \in X \mid || x || = 1 \}$. Let $V_\epsilon = \vspan{y_1, \dots, y_n}$ and let $P_\epsilon : X \to V$ be an orthogonal projection. Then let $K_\epsilon = P_\epsilon K$. Consider,
\[ || K_\epsilon - K || = \sup_{|| x || \le 1} || (P_\epsilon K - K) x || \]
For any $x \in B$ there is some $y_i$ such that $|| K x - y_i || < \epsilon$. Therefore,
\[ || (P_\epsilon K - K) x || \le || P_\epsilon K x - y_i || + || K x - y_i || \le \epsilon + \epsilon \]
where,
\[ || K x - y_i ||^2 = || P_\epsilon K x - y_i ||^2 + || (K x)^\perp - y_i ||^2 \]
because $K x = P_\epsilon K x + (K x)^\perp$ is an orthogonal decomposition and $y_i \in V$ showing that,
\[ || P_\epsilon K x - y_i || \le || K x - y_i || < \epsilon \]
justifying the above equation. Therefore,
\[ || P_\epsilon K - K || < 2 \epsilon \]
and $P_\epsilon K$ has image contained in $V$ which is finite dimensional by construction. Therefore we can take $K_n = P_{\frac{1}{2n}} K$ to get a sequence $K_n \in \L(X)$ such that,
\[ || K_n - K || < \frac{1}{n} \]
and therefore $K_n \to K$ in norm.

\subsection{3 (DON"T KNOW HOW)}

\begin{exercise}
Let $(X, \mu)$ be a finite measure space i.e. $\mu(X) < \infty$. Suppose that $f_i \to f$ in measure and $|| f_i ||_{L^p} < M$ for all $i$ and some $p > 0$. Then $f_i \to f$ in $L^1$.
\end{exercise}

The convergence $f_i \to f$ in measure implies that there is a subsequence $\{ f_{n_j} \}$ converging pointwise almost everywhere to $f$. 
\bigskip\\
I claim that poinwise convergence + bounded $L^p$-norm implies weak $L^p$ convergence.
\bigskip\\


 Then weak $L^p$ convergence implies $L^1$ convergence of the subsequence. 

Then we have to show the entire sequence converges in $L^1$.

\subsection{4 (DON"T KNOW HOW)}

\begin{exercise}
Let $f : [0, \infty) \to [0, \infty)$ be continuous and for any $x \in [0, \infty)$ the sequence $a_n(a) = f(n x)$ converges to zero. Show that,
\[ \lim_{x \to \infty} f(x) = 0 \]
\end{exercise}

Consider the sequence of functions $f_n(x) = f(nx)$. The condition shows that $f_n \to 0$ pointwise. By Egorov's theorem, on each compact inteval $K_r = [0,r]$ and for $\epsilon > 0$ there is a subset $E_{r, \epsilon} \subset K_r$ such that $f_n \to 0$ uniformly on $K_r \setminus E_{r,\epsilon}$ and $\mu(E_{r, \epsilon}) < \epsilon$. Therefore, for any $\eta > 0$ there exists some $N_r$ such that for $n > N_r$,
\[ || f_n ||_{K_r \setminus E_{r, \epsilon}} < \eta \]
Furthermore, because $K_r$ is compact $f_n$ is $L^\infty$ on $K_r$ (it is continuous and thus bounded). Then consider,
\[ \int_{K_r} |f_n| \, \d{\mu} = \int_{E_{r, \epsilon}} | f_n | \, \d{\mu} + \int_{K_r \setminus E_{r, \epsilon}} |f_n| \, \d{\mu} < \int_{E_{r, \epsilon}} | f_n | \, \d{\mu} + r \eta \]

\end{document}
