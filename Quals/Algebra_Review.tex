\documentclass[12pt]{article}
\usepackage{import}
\import{"../Algebraic Geometry/"}{AlgGeoCommands}

\newcommand{\Loc}[1]{\mathfrak{Loc}\left( #1 \right)}
\newcommand{\AbGrp}{\mathbf{AbGrp}}

\renewcommand{\K}{\mathbb{K}}

\newcommand{\inner}[2]{\left< #1, #2 \right>}

\newcommand{\B}{\mathcal{B}}
\newcommand{\R}{\mathbb{R}}

\newcommand\eqae{\mathrel{\stackrel{\makebox[0pt]{\mbox{\normalfont\tiny a.e.}}}{=}}}
\renewcommand{\F}{\mathcal{F}}
\renewcommand{\K}{\mathcal{K}}

\begin{document}

\tableofcontents



\section{Galois Theory}

\begin{prop}
Let $E$ be the splitting field of a $f \in K[x]$. Then,
\[ | \Aut{E/K} | \le [E : K ] \]
with equality if and only if $f$ is separable.
\end{prop}

\begin{proof}
Dummit and Foote p.561.
\end{proof}

\begin{lemma}[Independence of Characters]
Let $\sigma_1, \dots, \sigma_n : G \to E^\times$ be distinct linear characters. Then in $E[G]$ the elements $\sigma_1, \dots, \sigma_n$ are linearly independent.
\end{lemma}

\begin{proof}
We proceed by induction on $n$. For the case $n = 1$ this is obvious because a character $G \to E^\times$ is nonzero as a map $G \to E$. 
\bigskip\\
Now suppose that,
\[ a_1 \sigma_1 + \cdots + a_n \sigma_n = 0 \]
Now, this must hold for both $x \in G$ and $g x \in G$ so,
\[ a_1 \sigma_1(x) + \cdots + a_n \sigma_n(x) = 0 \]
and likewise,
\[ a_1 \sigma_1(gx) + \cdots + a_n \sigma_n(gx) = 0 \]
but $\sigma_i(gx) = \sigma_i(g) \sigma_i(x)$. Multiplying the first equation by $\sigma_n(g)$ and subtracting we find,
\[ a_1 [\sigma_n(g) - \sigma_1(g)] \sigma_n(x) + \cdots + a_{n-1} [\sigma_n(g) - \sigma_{n-1}(g)] \sigma_n(x) = 0 \]
Therefore by the independence of $\sigma_1, \dots, \sigma_{n-1}$ by assumption, we see that,
\[ a_1 [\sigma_n(g) - \sigma_1(g)] = 0 \]
Therefore either $a_1 = 0$ or $\sigma_1 = \sigma_n$ for all $g$. Since we assumed the characters are distinct this shows that $a_1 = 0$ reducing to the $n-1$ case so $a_i = 0$ for all $i$ by the induction hypothesis. Thus $\sigma_1, \dots, \sigma_n$ are independent.
\end{proof}

\begin{cor}
Distinct field embeddings $\sigma_1, \dots, \sigma_n : K \embed L$ are independent.
\end{cor}

\begin{proof}
Indeed, these are independent as characters $K^\times \to L^\times$ inside the $L$-vectorspace of maps $K^\times \to L$. Therefore, they must be independent as maps $K \to L$.
\end{proof}

\begin{cor}
Let $x_1, \dots, x_n \in E$ be a basis for $E / K$ and $n = [E : K]$. Let $G \subset \Aut{E/K}$ then the vectors $v_\sigma \in E^n$ defined by $(v_\sigma)_i = \sigma(x_i)$ are independent over $E$.
\end{cor}

\begin{proof}
Suppose that,
\[ \sum_{\sigma \in G} \alpha_\sigma v_\sigma = 0 \]
for $\alpha_\sigma \in E$. Then for each $i = 1,\dots,n$ we have,
\[ \sum_{\sigma \in G} \alpha_\sigma \sigma(x_i) = \sum_{\sigma \in G} \alpha_\sigma (v_\sigma)_i = 0 \]
Furthermore, we can write any $x \in E$ as,
\[ x = \beta_1 x_1 + \cdots + \beta_n x_n \]
for $\beta_i \in K$. Since $\sigma$ is a $K$-algebra map, multiplying the $i^{\text{th}}$ equation by $\beta_i$ and adding them gives,
\[ \sum_{i = 1}^n \beta_i \sum_{\sigma \in G} \alpha_\sigma \sigma(x_i) = \sum_{\sigma \in G} \alpha_\sigma \sum_{i = 1}^n \beta_i \sigma(x_i) = \sum_{\sigma \in G} \alpha_\sigma \sigma(\beta_1 x_1 + \cdots + \beta_n x_n) = \sum_{\sigma \in G} \alpha_\sigma \sigma(x) \]
and thus,
\[ \sum_{\sigma \in G} \alpha_\sigma \sigma(x) = 0 \]
Since $x \in E$ is arbitrary, we see that,
\[ \sum_{\sigma \in G} \alpha_\sigma \sigma = 0 \]
showing that $\alpha_\sigma = 0$ for all $\sigma \in G$ by the independence of the characters thus proving that the $v_\sigma \in E^n$ are independent. 
\end{proof}

\begin{cor}
If $G \subset \Aut{E/K}$ then $|G| \le [E : K]$.
\end{cor}

\begin{prop}
Let $E/K$ be a field extension and $G \subset \Aut{E/K}$. Then,
\[ |G| = [E : K] \iff K = E^G \]
\end{prop}

\begin{proof}
Suppose that $|G| = [E : K]$. Take $F = E^G$ giving a tower $K \subset F \subset E$. We know that $[E : K] = [E : F][F : K] = |G|$. However, $G \subset \Aut{E/F}$ because each automorphism fixes $F$ by definition. Thus $|G| \le [E : F]$ meaning that,
\[ |G| \le [E : F] \le [E : K] = |G| \]
proving that $[E : F] = [E : K]$ so $F = K$.
\bigskip\\
Now suppose that $K = E^G$. See Dummit and Foote p.571.
\end{proof}

\begin{rmk}
The proof shows that in general,
\[ [E : K] = |G| \cdot [E^G : K] \]
\end{rmk}

\begin{defn}
We say that $E / K$ is \textit{Galois} if $K = E^{\Aut{E/K}}$ and write $\Gal{E/K} := \Aut{E/K}$.
\end{defn}

\begin{cor}
We see that $E / K$ is Galois if and only if $|\Aut{E/K}| = [E : K]$.
\end{cor}

\subsection{The Galois Correspondence}

\begin{prop}
Let $E/K$ be a finite extension and $G \subset \Aut{E/K}$. Let $F = E^G$ then $E/F$ is Galois and $G = \Aut{E/F}$.
\end{prop}

\begin{proof}
By definition, $G \subset \Aut{E/F}$. Since $F = E^G$ we have $|G| = [E : F]$ and therefore,
\[ |G| \le |\Aut{E/F}| \le [E : F] = |G| \]
proving that $|G| = |\Aut{E/F}| = [E : F]$ and thus $G = \Aut{E/F}$ and that $E/F$ is Galois (note we actually automatically get that $E/F$ is Galois because $F = E^G = E^{\Aut{E/F}}$ using that $G = \Aut{E/F}$).
\end{proof}

\begin{prop}[Galois Connection]
Let $E/K$ be a finite extension and $G = \Aut{E/K}$.
\[ \{ \text{subgroups } H \subset G \} \substack{ \xrightarrow{\quad H \mapsto E^H \quad}  \\ \xleftarrow[F \mapsto \Aut{E/F}]{} } \{ \text{intermediate extensions } K \subset F \subset E \} \]
satisfy the following properties,
\begin{enumerate}
\item $H \mapsto E^H \mapsto \Aut{E/E^H} = H$ meaning that 
\end{enumerate}
\end{prop}


\subsubsection{Field Norm and Trace}

\renewcommand{\Tr}{\mathrm{Tr}}
\newcommand{\Nm}{\mathrm{N}}

\begin{defn}
Let $L/K$ be a finite extension of fields. Then we define the relative trace, 
\[ \Tr_{L/K} : L \embed \mathrm{End}_K(L) \xrightarrow{\mathrm{tr}} K \]
and relative norm,
\[ \Nm_{L/K} : L \embed \mathrm{End}_K(L) \xrightarrow{\det} K \]
and the relative characteristic polynomial,
\[ \ch_{L/K} : L \embed \mathrm{End}_k(L) \xrightarrow{\text{char poly}} K[x] \]
\end{defn}

\begin{rmk}
By Cayley-Hamilton, if $p = \ch_{L/K}(\alpha)$ then $p(\alpha) = 0$. Therefore $m_\alpha \divides \ch_{L/K}$ where $m_\alpha$ is the minimal polynomial of $\alpha$ over $K$.
\end{rmk}

\begin{lemma}
Suppose that $L/K$ is separable. Then for any $\alpha \in L$,
\[ \ch_{L/K}(\alpha) = \prod_{\sigma : L \embed \overline{K}} (x - \sigma(\alpha)) = m_\alpha^{\frac{[L : K]}{\deg{\alpha}}} \]
where the sum is taken over $K$-linear embeddings of $L$ into $\overline{K}$.
\end{lemma}

\begin{proof}
Consider $L / K(\alpha) / K$. Then choosing a $K(\alpha)$-basis of $L$ decompses $L$ into isomorphic $\alpha$-invariant $K$-subspaces of which there are $e = [L : K(\alpha)] = \frac{[L : K]}{\deg{\alpha}}$. Therefore, $\ch_{L/K}(\alpha) = \ch_{K(\alpha)/K}(\alpha)^e$. Furthermore, $\m_\alpha \divides \ch_{K(\alpha)/K}(\alpha)$ and they both have degree $\deg{\alpha}$ and are monic so $\m_\alpha = \ch_{K(\alpha)/K}$. 
\bigskip\\
Now let $E / L / K$ be the Galois closure. Then $\Hom{K}{L}{K^\sep} = \Hom{K}{L}{E}$ are given by cosets of $H = \Gal{E/L} \subset \Gal{E/K}$. Thus,
\[ \prod_{\sigma \in \Hom{K}{L}{E}} (x - \sigma(\alpha)) = \prod_{\sigma H \in G/H}  (x - \sigma(\alpha)) \]
which makes sense because any $\tau \in \sigma H$ is $\tau = \sigma \gamma$ for $\gamma \in H = \Gal{E/L}$ fixes $L$ by definition so $\tau(\alpha) = \sigma(\gamma(\alpha)) = \sigma(\alpha)$. Now let $H' = \Gal{E/K(\alpha)} \supset H$. Then,
\[ \prod_{\sigma H \in G/H}  (x - \sigma(\alpha)) = \prod_{\sigma \in G/H'} \prod_{\tau \in \sigma H'/H} (x - \tau(\alpha)) = \prod_{\sigma \in G/H} (x - \sigma(\alpha))^{[L : K(\alpha)]} \]
where $|H'/H| = [L : K(\alpha)]$ because $\tau \in \sigma H'$ is $\tau = \sigma \gamma$ for $\gamma \in H' = \Gal{E/K(\alpha)}$ fixes $\alpha$ by definition so $\tau(\alpha) = \sigma(\gamma(\alpha)) = \sigma(\alpha)$. Therefore, 
\[ \prod_{\sigma \in \Hom{K}{L}{E}} (x - \sigma(\alpha)) = \left( \prod_{G/H'} (x - \sigma(\alpha)) \right)^{[L : K(\alpha)]} \]
Now I claim that,
\[ f(x) = \prod_{\sigma \in G/H'} (x - \sigma(\alpha)) \]
is the minimal polynomial of $\alpha$. Consider $\tau \in G$ then,
\[ \tau(f(x)) = \prod_{\sigma \in G/H'}(x - \tau(\sigma(\alpha))) = \prod_{\sigma' \in G/H'}(x - \sigma'(\alpha)) = f(x) \]
so $f \in K[x]$ and clearly $f(\alpha) = 0$ (because $(x - \alpha)$ for $\sigma = \id$ is a factor) so $\m_\alpha \divides f$ in $K[x]$. However, $m_\alpha(\sigma(\alpha)) = \sigma(m_\alpha(\alpha)) = 0$ since $m_\alpha \in K[x]$ so each $\sigma(\alpha)$ is a root of $m_\alpha$. Furthermore, the $\sigma(\alpha)$ appearing in $f$ are \textit{distinct} because if $\sigma(\alpha) = \sigma'(\alpha)$ then $\sigma^{-1} \sigma'(\alpha) = \alpha$ so $\sigma^{-1} \sigma' \in \Gal{E/K(\alpha)}$ and thus $\sigma H' = \sigma' H'$. Therefore, $f \divides m_\alpha$ in $E[x]$ because each linear factor divides $m_\alpha$ since each $\sigma(\alpha)$ is a root of $m_\alpha$. Therefore $f = m_\alpha$ and we conclude.
\end{proof}


\begin{cor}
Let $m_\alpha = x^n + a_{1} x^{n-1} + \cdots + a_n$. Then,
\[ \Tr_{L/K}(\alpha) = \sum_{\sigma : L \embed \overline{K}} \sigma(\alpha) = (-1)^{[L:K]} a_1^{\frac{[L : K]}{\deg{\alpha}}} \quad \text{ and } \quad \Nm_{L/K}(\alpha) = \prod_{\sigma : L \embed \overline{K}} \sigma(\alpha) = a_n^{\frac{[L:K]}{\deg{\alpha}}}\]
\end{cor}


\begin{lemma}
Let $L/K$ be a finite extension of fields. Let $V$ be a finite dimensional $L$-vectorspace and $\varphi : L \to V$ an $L$-linear map. Then,
\[ \Tr_K(\varphi) = \Tr_{L/K}(\Tr_L(\varphi)) \]
and likewise,
\[ \mathrm{det}_K(\varphi) = \Nm_{L/K}(\mathrm{det}_L(\varphi)) \]
\end{lemma}

\begin{proof}
Choosing bases this becomes a direct computation (see \href{https://stacks.math.columbia.edu/tag/0BII}{Tag 0BIE}).
\end{proof}

\begin{cor}
Given a tower of finite field extensions $F / L / K$,
\[ \Tr_{F/K} = \Tr_{L/K} \circ \Tr_{F/L} \quad \text{and} \quad \Nm_{F/K} = \Nm_{L/K} \circ \Nm_{F/L} \] 
\end{cor}

\subsection{The Discriminant}

\begin{lemma}
Given a bilinear form $B : V \times V \to K$ if we choose any basis $e_1, \dots, e_n \in V$ then,
\[ \Delta(B) = \det{B(e_i, e_j)} \in K / (K^\times)^2 \]
is independent of the choice of basis.
\end{lemma}

\begin{proof}
Let $M_{ij} = B(e_i, e_j)$ and $M'_{ij} = B(e'_i, e'_j)$.
There is a change of basis matrix,
\[ e_j' = \sum_{k} C_{kj} e_k \]
and therefore,
\[ M'_{ij} = \sum_{k,\ell} C_{k i} B(e_k, e_\ell) C_{\ell j} =  (C^\top M C)_{ij} \]
Thus,
\[ \Delta'(B) = \det{M'} = \det{(C^\top M C)} = (\det{C})^2 \det{M} = (\det{C})^2 \Delta(B) \]
so in $K / (K^\times)^2$ we have $\Delta'(B) = \Delta(B)$.
\end{proof}

\begin{lemma}
The quadratic form $B$ is degenerate iff $\Delta(B) = 0$.
\end{lemma}

\begin{proof}
If $B$ is degenerate then there exists $v \in V$ such that $B(v, -) = 0$ and then extending to a basis of $V$ we see immediately that $\Delta(B) = 0$. Conversely, if $\Delta(B) = 0$ then for some basis $e_1, \dots, e_n \in V$ the columns $B(e_i, e_j)$ are dependent meaning that there exist $v_1, \dots, v_n$ such that,
\[ \sum_{j} B(e_i, e_j) v_j = 0 \]
for all $i$ and thus setting $v = v_1 e_1 + \cdots + v_n e_n$ we see that $B(e_i, v) = 0$ for all $e_i$ and thus since the $e_i$ span $V$ we find that $B(-, v) = 0$ so $B$ is degenerate.
\end{proof}

\begin{lemma}
Let $L/K$ be a finite separable extension and $e_1, \dots, e_n \in L$ a $K$-basis of $L$. Then,
\[ \det{(\Tr_{L/K}(e_i e_j))} = \det{(\sigma_i(e_j))}^2 \]
running over $\sigma_j \in \Hom{K}{L}{K^\sep}$ of which there are $[L : K]$ because $L/K$ is separable.
\end{lemma}

\begin{proof}
Let $M_{ij} = \sigma_i(e_j)$ then,
\[ A_{ij} = \Tr_{L/K}(e_i e_j) = \sum_{k} \sigma_k(e_i) \sigma_k(e_j) = \sum_k M_{ki} M_{kj} = (M^\top M)_{ij} \]
Therefore,
\[ \det{A} = \det{(M^\top M)} = (\det{M})^2 \]
proving the proposition.
\end{proof}

\begin{lemma}
Let $L/K$ be a finite extension of fields. Then the following are equivalent,
\begin{enumerate}
\item $L/K$ is separable
\item $\Tr_{L/K}(xy)$ is not identically zero
\item the bilinear form $B_{L/K}(x,y) = \Tr_{L/K}{(xy)}$ is nondegenerate
\item $\Delta_{L/K} = \Delta(B_{L/K}) \neq 0$.
\end{enumerate}
\end{lemma}

\begin{proof}
If $\Tr_{L/K}(\gamma) \neq 0$ then for any $\alpha \in L$ we have $B_{L/K}(\alpha, \gamma/\alpha) = \Tr_{L/K}(\gamma) \neq 0$ so $B_{L/K}$ is nondegenerate. Clearly (c) $\implies$ (b) so we see that (b) $\iff$ (c). Furthermore, (c) $\iff$ (d) by a previous lemma.
\bigskip\\
Now suppose that $L/K$ is inseparable. Then there exists an intermediate extension $L/F/K$ such that $F/K$ is separable and $L/F$ is purely inseparable. Then there exists some $\alpha \in L$ such that $\alpha^p \in F$ but $\alpha \notin F$. Then we have a tower $L / F(\alpha)/F/K$ which implies that,
\[ \Tr_{L/K} = \Tr_{F/K} \circ \Tr_{F(\alpha)/F} \circ \Tr_{L/F(\alpha)} \]
Therefore, it suffices to show that $\Tr_{F(\alpha)/F} = 0$. Indeed, $[F(\alpha) : F] = p$ so $\Tr_{F(\alpha)/F}(1) = p = 0$ in $F$. Furthermore, the minimal polynomial of $\alpha^i$ for $0 < i < p$ is $x^p - \alpha^{ip}$ and thus $\Tr_{F(\alpha)/F}(\alpha^i) = 0$ showing that $\Tr_{F(\alpha)/F} = 0$ by linearity.
\bigskip\\
Finally, suppose that $L/K$ is separable. Then by the previous result, it suffices to show that $\det{(\sigma_i(e_j))} \neq 0$. Suppose that there exist $v_1, \dots, v_n \in K$ such that,
\[ \sum_i v_i \sigma_i(e_j) = 0 \]
for all $j$ and therefore because $\{ e_j \}$ span $L$ we have,
\[ \sum_i v_i \sigma_i = 0 \]
so by independence of characters $v_i = 0$. Thus the square matrix $\sigma_i(e_j)$ has independent rows and thus $\det{(\sigma_i(e_j))} \neq 0$.
\end{proof}


\section{Galois Groups of Cubics}


\section{Structure Theorem of Modules Over a PID}

\begin{rmk}
In this section let $R$ be a PID.
\end{rmk}

\begin{prop}
Any submodule $M \subset R^n$ is free of rank at most $n$.
\end{prop}

\begin{proof}
We proove this by induction on $n$. The case $n = 1$ is the definition of a PID since any submodule of $R$ is an ideal. Now consider a submodule $M \subset R^n$ and its image $N \subset R^{n-1}$ under the projection and kernel $K \subset R$ giving,
\begin{center}
\begin{tikzcd}
0 \arrow[r] & R \arrow[r] & R^{n} \arrow[r] & R^{n-1} \arrow[r] & 0
\\
0 \arrow[r] & K \arrow[u, hook] \arrow[r] & M \arrow[u, hook] \arrow[r] & N \arrow[u, hook] \arrow[r] & 0
\end{tikzcd}
\end{center}
by the case $n = 1$ we see that $N$ is free of rank at most $1$ and $N$ is free of rank at most $n - 1$ by the induction hypothesis. Since $N$ is projective, the sequence splits giving $M \cong K \oplus N$ which is thus free of rank at most $n$ proving the claim.
\end{proof}

\begin{rmk}
The rank inequality is a general fact about modules over a domain $A$. If $M \subset N$ then $\mathrm{rank}(M) \le \mathrm{rank}(N)$ because if $K = \Frac{A}$ then,
\[ M \otimes_A K \embed M \otimes_A N \]
since $K$ is flat over $A$. Therefore, 
\[ \rank_{A}(M) = \dim_{K} M \le \dim_K N = \rank_{A}(N) \]
Here, rank means ``rank at the generic point'' which agrees with the notion of rank for free modules.
\end{rmk}

\begin{lemma}
Let $A$ be a domain. Let $M$ be a finite $A$-module. Then $M$ is torsion-free if and only if $M$ is contained in a finite free module.
\end{lemma}

\begin{proof}
If $M$ is a submodule of $R^{n}$ then clearly $M$ is torsion-free. Assume that $M$ is torsion-free. Let $K = \Frac{A}$. Because $M$ is torsion-free, the map $M \embed M \otimes_A K$ is injective and $M \otimes_A K$ is a finite-dimensional $K$-vectorspace. Choose generators $x_1, \dots, x_n$ of $M$. By clearing denominators, choose a basis $e_1, \dots, e_r \in M \otimes_A K$ such that each $x_i$ is in the $A$-span of $e_1, \dots, e_r$. Then,
\[ M \subset A e_1 \oplus \cdots \oplus A e_r \subset M \otimes_A K \]
and the module $A e_1 \oplus \cdots \oplus A e_r \cong A^n$ is an internal direct sum (i.e. is free) by the $K$-independence (and thus $R$-independence) of $e_1, \dots, e_r$.  
\end{proof}

\begin{prop}
A finite $R$-module is torsion-free if and only if it is free.
\end{prop}

\begin{proof}
Clearly free modules are torsion-free so assume that $M$ is finite and torsion-free. By the previous lemma, there is an embedding $M \embed R^n$ and thus by the previous result $M$ is free as the submodule of a free module.
\end{proof}

\subsection{Interlude on Torsion-Freeness}

\begin{lemma}
Let $A$ be a domain. Any flat $A$-module is torsion free.
\end{lemma}

\begin{proof}
Let $M$ be a flat $A$-module. Since $A$ is a domain for any nonzero $x \in A$ the map $A \xrightarrow{x} A$ is injective. Since $M$ is flat we see that $M \xrightarrow{x} M$ is injective so $M$ has no $x$-torsion and thus $M$ is torsion-free.
\end{proof}

\begin{lemma}
If $A$ is a valuation ring then $M$ is flat if and only if $M$ is torsion-free.
\end{lemma}

\begin{proof}
See \href{https://stacks.math.columbia.edu/tag/0539}{Tag 0539}.
\end{proof}

\begin{prop}
Let $A$ be a Dedekind domain. 
\begin{enumerate}
\item An $A$-module is flat if and only if it is torsion-free
\item A finite torsion-free $A$-module is finite locally free.
\end{enumerate}
\end{prop}

\begin{proof}
We know that flat implies torsion-free. Suppose that $M$ is torsion-free. Then for each maximal ideal $\m \subset A$ we know that $M_\m$ is a torsion-free $A_\m$-module but $A_\m$ is a DVR and hence a valuation ring so $M_\m$ is flat. Thus $M$ is flat because exactness can be checked on maximal ideals.
\bigskip\\
The second follows from the fact that finite flat modules are finitely locally free (see \href{https://stacks.math.columbia.edu/tag/00NX}{Tag 00NX}).
\end{proof}

\subsection{The Structure Theorem}

\begin{rmk}
Again let $R$ be a PID and let $M$ be a finite $R$-module. Then consider the torsion submodule $T(M) \subset M$. We get an exact sequence,
\begin{center}
\begin{tikzcd}
0 \arrow[r] & T(M) \arrow[r] & M \arrow[r] & M / T(M) \arrow[r] & 0
\end{tikzcd}
\end{center}
where $M/T(M)$ is finite and torsion-free and thus free by our previous work. Thus $M/T(M) \cong R^n$ is projective so the sequence splits showing that,
\[ M \cong R^n \oplus T(M) \]
where $n = \rank_A(M)$ (immediate from tensoring the above sequence by $K$). Therefore, it suffices to classify the structure of torsion modules.
\end{rmk}

\begin{defn}
For each prime element $p \in R$ consider the $p$-torsion subgroup,
\[ M_p = \{ m \in T(M) \mid \exists n : p^n m = 0 \} \]
\end{defn}

\begin{prop}
For any finite $R$-module $M$,
\[ T(M) = \bigoplus_{p} M_p \]
where only finitely many $M_p$ are nonzero.
\end{prop}

\begin{proof}
First suppose that $r \in M_p \cap M_q$ for distinct prime elements $p$ and $q$. Then because nonzero prime ideals are maximal (since being a prime element implies irreducible) and thus $(p) + (q) = R$ since $q \notin (p)$ this is a strictly larger ideal. Therefore, if $p^n m = 0$ and $q^n m = 0$ (take $n$ to be sufficiently large for both) then $R = (p^n, q^n) \subset \Ann{A}{m}$ (if $1 \in (p, q)$ then $1 \in (p,q)^{2n} \subset (p^n, q^n)$) so $1 \in \Ann{A}{m}$ and thus $m = 0$.
\bigskip\\
Now, since $\Ann{A}{m} \subset R$ is an ideal we have $\Ann{A}{m} = (r)$. Because $m \in T(M)$ the annihilator is nontrivial so $r \neq 0$ and if $r \in R^\times$ then $1 \in \Ann{A}{m}$ meaning that $m = 0$ which is in $M_p$ for each $p$. Otherwise $\Ann{A}{m} = (r)$ is a nontrivial ideal. We apply the fact that $R$ is a UFD to write,
\[ r = p_1^{e_1} \cdots p_r^{e_r} \]
in terms of prime elements $p_i$. If $r = 1$ then we are done because $r = p_1^{e_1}$ and thus $p_1^{e_1} m = 0$ so $m \in M_{p_1}$. Otherwise, $(p_1, \dots, p_r) = R$ and thus taking sufficiently large $n$,
\[ R = (p_2^{e_2} \cdots p_r^{e_r}, \cdots, p_1, \dots, p_r)^n \subset (p_1^{e_1} \cdots p_{r-1}^{e_{r-1}}) \]
and thus we can write,
\[ 1 = \alpha_1 p_2^{e_2} \cdots p_r^{e_r} + \cdots + \alpha_r p_1^{e_1} \cdots p_{r-1}^{e_{r-1}} \]
meaning that,
\[ m = \alpha_1 p_2^{e_2} \cdots p_r^{e_r} m + \cdots + \alpha_r p_1^{e_1} \cdots p_{r-1}^{e_{r-1}} m  \]
where the $i^{\mathrm{th}}$-term is clearly killed by $p_i^{e_i}$ and thus is in $M_{p_i}$ proving that the $M_{p_i}$ span $T(M)$. 
\bigskip\\
Finally, the finiteness statement follows immediately from the fact that $M$ is finitely generated and that $M_p \cap M_q = (0)$ if $p \neq q$ are distinct primes.
\end{proof}

\begin{lemma}
Let $A$ be an Artin local ring with principal maximal ideal $\m = (\varpi)$. Then for any finite $A$-module $M$ there is a decomposition,
\[ M \cong \bigoplus_{i = 1}^n R / (\varpi^{a_i}) \]
where the numbers $a_1 \le a_2 \le \cdots \le a_n$ are uniquely determined by $M$.
\end{lemma}

\begin{proof}
Notice that every ideal is of the form $(\varpi^k)$ for some $k$. Indeed, for any proper nonzero ideal $\a \subset A$ because $\m$ is the unique maximal ideal, $\a \subset \m$ but because $\m^N = (0)$ for sufficiently large $N$ there is a maximal power $k$ such that $\a \subset \m^k$. Choose $y \in \a \setminus \m^{k+1}$. Thus $y = u \varpi^k$ but $y \notin \m^{k+1}$ so we must have $u \notin \m$ and thus $u$ is a unit. Thus $\m^k = (\varpi^k) = (y) \subset \a \subset \m^k$ so $\a = (\varpi^r)$.
\bigskip\\
Let $\kappa = A / \m$ be the residue field then we proceed by induction on,
\[ n = \dim_{\kappa} (M \otimes_A \kappa) = \dim_{\kappa} M/ \varpi M \]
Since $A$ is local, by Nakayama's lemma, $M$ can be generated by $n$ elements. Thus if $n = 1$ then $M = A/(\varpi^{a_1})$ because the kernel of $A \onto M$ is some ideal and thus of the form $(\varpi^{a_1})$. 
\bigskip\\
Now consider $\Ann{A}{M} = (\varpi^k)$ then $M$ is an $A' = A / (\varpi^k)$-module and there is some element $m \in M$ such that $m$ is not killed by any smaller power of $\varpi$ (else then $(\varpi^{k-1}) \subset \Ann{A}{M}$) and thus $\Ann{A'}{m} = (0)$ because it does not contain any $(\varpi^{i})$ for $i < k$. Therefore $A' \embed M$ sending $1 \mapsto m$ is injective so we get an exact sequence,
\begin{center}
\begin{tikzcd}
0 \arrow[r] & A \arrow[r, "1 \mapsto m"] & M \arrow[r] & K \arrow[r] & 0 
\end{tikzcd}
\end{center}
of $A'$-modules. However $A'$ is an injective module over itself (use Baer's criterion DO THIS!!) and thus the sequence of $A'$-modules is split. Therefore we get an exact sequence,
\begin{center}
\begin{tikzcd}
0 \arrow[r] & \kappa \arrow[r] & M \otimes_A \kappa \arrow[r] & K \otimes_A \kappa \arrow[r] & 0
\end{tikzcd}
\end{center}
and thus $\dim_\kappa (K \otimes_{A'} \kappa) = \dim_\kappa (K \otimes_A \kappa) = n - 1$ so by induction it is of the required form. Therefore, by the splitting,
\[ M \cong A' \oplus K \cong A' \oplus \bigoplus_{i = 1}^{n-1} A' / (\varpi^{a_i}) = A / (\varpi^k) \oplus \bigoplus_{i = 1}^{n-1} A / (\varpi^{a_i}) \]
with $a_1 \le \cdots \le a_{n-1} \le a_n$ where we set $a_n = k$.
\bigskip\\
For uniqueness, we use the fact that the clearly intrinsic decreasing sequence,
\[ b_i = \dim_{\kappa} \varpi^i M / \varpi^{i+1} M = \# \{ j \mid a_j \ge i \} \]
uniquely characterizes the sequence $a_1 \le \cdots \le a_n$ (including the number $n = b_0$).
\end{proof}

\begin{prop}
Let $M$ be a finie $R$-module and $p \in R$ a prime element. Then,
\[ M_p \cong \bigoplus_{i = 1}^n R / (p^{a_i}) \]
where the numbers $a_1 \le a_2 \le \cdots \le a_n$ are uniquely determined by $M$. 
\end{prop}


\begin{proof}
Because $M$ is finitely generated $M_p \subset M$ is finitely generated ($R$ is Noetherian) so there is some maximum power $n$ such that $p^k$ kills the generators and thus all of $M$. Therefore, $M_p$ is a $A = R/(p^k)$-module. Then, $A$ is an Artin local ring with maximal ideal $(p)$ and $M_p$ is a finite $A$-module. Therefore, the theorem follows directly from the previous lemma since $A / (p^{a_i}) = R / (p^{a_i})$ for $a_i \le k$.
\end{proof}

\begin{theorem}[Structure Theorem]
Let $R$ be a PID and $M$ be a finite $R$-module. Then,
\[ M \cong R^r \oplus \bigoplus_{p} \bigoplus_{i = 1}^{n_p} R / (p^{a_{p,i}}) \]
where the numbers $r, n_p, a_{p,i}$ are unque and may be computed as follows,
\[ r = \dim_K (M \otimes_R K) \quad n_p = \dim_{R/(p)} M_p/p M_p \quad b_{p,i} = \dim_{R/(p)} p^i M_p / p^{i+1} M_p \]
where $K = \Frac{R}$ and $M_p$ is the $p$-torsion submodule and the $b_{p,i}$ determine the $a_{p,i}$ as above.
\end{theorem}


\subsection{Smith Normal Form}

\begin{prop}[Smith Normal Form]

\end{prop}




\section{Nakayama's Lemma}

\begin{prop}
Let $A$ be a commutative ring and $M$ a finitely generated $A$-module with an ideal $I \subset A$. If $I \cdot M = M$ then there exists $r \in I$ such that $(r - 1) M = 0$.
\end{prop}

\begin{proof}
Let $\pi : A^n \onto M$ be a generating set. Let $\varphi : M \to M$ be an $A$-linear map with $\varphi(M) \subset I M$. Then consider the diagram,
\begin{center}
\begin{tikzcd}
A^n \arrow[d, two heads, "\pi"] \arrow[r, "\tilde{\varphi}", dashed] & I \cdot A^n \arrow[r, hook] \arrow[d, two heads, "\pi"] & A^n \arrow[d, "\pi"]
\\
M \arrow[r, "\varphi"] & IM \arrow[r, hook] & M
\end{tikzcd}
\end{center}
where there is a lift $\tilde{\varphi} : A^n \to I \cdot A^n \subset A^n$ over the surjection $I \cdot A^n \to I \cdot M$ since $A^n$ is free. Thus $\tilde{\varphi}$ is given by a matrix with coefficients $a_{ij} \in I$. Then its characteristic polynomial,
\[ x^n + a_1 x^{n-1} + \cdots + a_n \]
has $a_i \in I^i$ because $a_i$ is a polynomial in $a_{ij}$ of degree $i$. By Cayley-Hamilton,
\[ \tilde{\varphi}^n + a_1 \tilde{\varphi}^{n-1} + \cdots + a_n = 0 \]
Thus,
\[ \pi \circ (\tilde{\varphi}^n + a_1 \tilde{\varphi}^{n-1} + \cdots + a_n) = 0 \]
but $\pi \circ \tilde{\varphi} = \varphi \circ \pi$ and thus,
\[ (\varphi^n + a_1 \varphi^{n-1} + \cdots + a_n) \circ \pi = 0 \]
but $\pi$ is surjective so we see that,
\[ \varphi^n + a_1 \varphi^{n-1} + \cdots + a_n = 0 \]
In particular, if $M = I M$ then we can let $\varphi = \id$ so for any $m \in M$ we conclude that,
\[ m + a_1 m + \cdots + a_n m = 0 \]
hence $(1 + a_1 + \cdots + a_{n}) \cdot m = 0$ so taking $r = -(a_1 + \cdots + a_n) \in I$ we conclude. 
\end{proof}


\begin{prop}
Let $R$ be a (possibly noncommutative) ring and $M$ a finitely generated left $R$-module and $I \subset R$ a left-ideal. Then if $I \cdot M = M$ then there exists some $r \in I$ such that $(r - 1) M = 0$.
\end{prop}

\begin{proof}

\end{proof}


\section{Groups of Lie Type}

\section{Products of Ideals}

\begin{lemma}
Let $I, J \subset R$ be ideals. Then,
\[ V(IJ) = V(I \cap J) = V(I) \cup V(J) \]
\end{lemma}

\begin{proof}
If $I \subset \p$ then $\p \supset I \cap J \subset IJ$ so it is clear that,
\[ V(I) \cup V(J) \subset V(I \cap J) \subset V(IJ) \]
Thus suppose that $\p \supset IJ$ but $\p \notin V(I) \cup V(J)$. Then there is $x \in I$ and $y \in J$ such that $x, y \notin \p$ so that $\p \not\supset I$ and $\p \not \supset J$. Then $x y \in IJ \subset \p$ so $x y \in \p$ contradicting the primality of $\p$ and proving the claim.
\end{proof}

\begin{prop}
Let $R$ be a comutative ring and $I, J \subset R$ are ideals.
If any of the following are true,
\begin{enumerate}
\item $I + J = R$
\item $\nilrad{R / IJ} = (0)$
\end{enumerate}
then $I \cap J = IJ$.
\end{prop}

\begin{proof}
If $I + J = R$ then for any $r \in I \cap J$ consider $1 = x + y$ with $x \in I$ and $y \in J$ and $r = r x + ry \in IJ$ so $I \cap J \subset IJ \subset I \cap J$ proving equality. 
\bigskip\\
Now suppose that $\nilrad{R / IJ} = (0)$. Consider the ideal $(I \cap J)/IJ \subset R / IJ$. I claim that it is contained in the nilradical. Indeed, for any prime $\p$ of $R / IJ$, that is a prime of $R$ above $IJ$ because $V(IJ) = V(I \cap J)$ and thus $(I \cap J)/IJ \subset \nilrad{R / IJ}$ so $I \cap J = IJ$.
\end{proof}

\section{Induced Representations}

\subsection{Restriction}

\begin{rmk}
There is a functor $\Rep_R : \mathbf{Grp}^\op \to \mathbf{Cat}$ sending $G \mapsto \mathrm{Rep}_R(G)$ taking $\phi : G \to H$ to the functor $\Res{}{\phi}{-} : \Rep_R(H) \to \Rep_R(G)$ via $\rho_W \mapsto \rho_W \circ \phi$ and $(T : W \to W') \mapsto (T : W \to W')$ which still commutes with $\rho_W \circ \phi$ by definition.
\bigskip\\
This restriction functor is just restriction of modules from the ring map $R[G] \to R[H]$.
\bigskip\\
Therefore we get a map $\Aut{G}^\op \to \Aut{\Rep_R(G)}$ and thus a natural right action (which we turn into a left action via $\Aut{G} \to \Aut{G}^\op$ sending $g \mapsto g^{-1}$) on $G$-representations. 
\end{rmk}

\begin{prop}
If $\phi : G \to H$ is surjective then $\Rep_R(H) \to \Rep_R(G)$ preserves irreducibles.
\end{prop}

\begin{proof}
If $W$ is an irreducible $H$-rep then if $V \subset \Res{}{\phi}{W}$ is a $G$-invariant subspace then $\rho_W(\phi(g)) \cdot V = V$ and thus $\rho_W(h) \cdot V = V$ so $V$ is $H$-invariant because $\phi$ is surjective.
\end{proof}

\subsubsection{The Case of a Normal Subgroup}

\begin{rmk}
For the special case of a normal subgroup $H \subset G$ we denote the conjugation action $c : G \to \Aut{H}$ and then applying the above construction we find the following.
\end{rmk}

\begin{defn}
Let $H \subset G$ be a normal subgroup and $W$ an $H$-representation. Then for $g \in G / H$ we define $g * W$ to be the $H$-representation given by $\rho_W \circ c_g^{-1}$ 
\end{defn}

\begin{rmk}
Notice that if $g' = g h$ then $\rho_W \circ c_{g'}^{-1} = \rho_W \circ c_h^{-1} \circ c_g^{-1}$ but $\rho_W \circ c_h^{-1} \cong \rho_W$ so we get $g * W \cong g' * W$ as required. This is a manifestation of the fact that $\Rep_R : \mathbf{Grp}^\op \to \mathbf{Cat}$ is really a $2$-functor sending the natural transformation (isomorphism) $\eta : \phi \to \phi'$ (which just says that $\phi' = c_h \circ \phi$ for some $h = \eta_* \in H$) to the natural isomorphism $\Res{}{\eta}{V} : \Res{}{\phi}{V} \to \Res{}{\phi'}{V}$ given by $v \mapsto h \cdot v$ because then,
\[ h \cdot (g \cdot_{\phi} v) = h \cdot (\phi(g) \cdot v) = (h \phi(g) h^{-1}) \cdot (h \cdot v) = g \cdot_{\phi'} (h \cdot v) \]
\end{rmk}

\begin{prop}
If $H \subset G$ is normal and $V$ is a $G$-representation then $g * \Res{G}{H}{V} \cong \Res{G}{H}{V}$.
\end{prop}

\begin{proof}
Consider the map $\eta : V \to V$ by sending $\eta: v \mapsto g \cdot v$. I claim this is an isomorphism $\eta : g * \Res{G}{H}{V} \to \Res{G}{H}{V}$. Indeed it is clearly bijective and linear. Now,
\[ (g * \rho)(h) \cdot v = g^{-1} h g \cdot v \mapsto g \cdot (g^{-1} h g) \cdot v = hg \cdot v = h \cdot (g \cdot v) = \rho(h) \cdot v \]
so $\eta \circ (g * \rho)(h) = \rho(h) \circ \eta$. 
\end{proof}

\begin{prop}
Let $H \subset G$ be normal and $V$ a $G$-representation. Then $G / H$ acts on the $H$-subrepresentations $W \subset \Res{G}{H}{V}$ via $W \mapsto g \cdot W$ where $g \cdot W \cong g * W$ as $H$-representations.
\end{prop}

\begin{proof}
We need to show that $g \cdot W$ is a well-defined subrepresentation. First, for $v \in W$,
\[ h \cdot (g \cdot v) = hg \cdot v = g(g^{-1} h g) \cdot v = g \cdot ((g^{-1} h g) \cdot v) \]
proving that $g \cdot W$ is indeed $H$-invariant since $g^{-1} h g \in H$ so $g^{-1} h g \cdot v \in W$ and also that $g * W \cong g \cdot W$ via $v \mapsto g \cdot v$ by the same argument above. Furthermore, if $g' = gh$ then $g' \cdot W = g \cdot (h \cdot W) = g \cdot W$ because $W$ is $H$-invariant. 
\end{proof}

\begin{rmk}
It is clear that the $G$-invariant subspaces of $V$ are exactly the fixed points under the $G/H$-action.
\end{rmk}

\subsection{Induction and Coinduction}

\begin{prop}
Let $H \subset G$ then $R[G]$ is a free $R[H]$-module.
\end{prop}

\begin{proof}
Consider, 
\[ R[G] \cong \bigoplus_{g \in H  G} g R[H] \]
as \textit{right} $R[H]$-modules (we can make them left modules by $R[H]^\op \cong R[H]$) via sending $g \cdot h \mapsto gh$. This is clearly surjective because $gh$ covers each coset. Furthermore, this is injective because if,
\[ \sum_{g \in G/H} g \left( \sum_{h \in H} \alpha_{g,h} h \right) = \sum_{g \in G / H} \sum_{h \in H} \alpha_{g,h} gh = 0 \]
but there is an bijection $G / H \times H \to G$ via $(g,h) \mapsto gh$ then $\alpha_{g,h} = 0$. Finally, this map is $R[H]$-linear because $g \cdot h h' \mapsto gh h' = (gh) \cdot h'$.
\end{proof}

\begin{prop}
If $H \subset G$ is normal then for any $H$-representation $W$,
\[ \Res{G}{H}{\Ind{G}{H}{W}} \cong \bigoplus_{g \in G / H} g * W \]
\end{prop}

\begin{prop}
If $H \subset G$ is normal then for any $G$-representation $V$,
\[ \Ind{G}{H}{\Res{G}{H}{V}} \cong R[G/H] \otimes_R V \]
as $R[G]$-modules.
\end{prop}

\begin{proof}
Consider the map, $\Ind{G}{H}{\Res{G}{H}{V}} \cong R[G] \otimes_{R[H]} V \to R[G/H] \otimes_R V$ defined by,
\[ g \otimes v \mapsto [g] \otimes g \cdot v \]
This is well-defined because,
\[ gh \otimes v \mapsto [gh] \otimes gh \cdot v \quad \text{ and } \quad g \otimes (h \cdot v) \mapsto [g] \otimes g h \cdot v = [gh] \otimes gh \cdot v \]
This is clearly surjective and both sides are free $R$-modules of equal rank so it is an isomorphism.
\end{proof}

(DEFINITION OF INDUCTION AND COINDUCTION)
(WHEN ARE THEY EQUAL)
(EXPLICIT DESCRIPTIONS)
(CHARACTER FORMULAE)
(FORMULA FOR IND(RES))
(NON-NORMAL CASE?)

\section{Noetherian Normalization}

\begin{theorem}
Let $A$ be a finitely generated $K$-algebra domain. Then there are algebraically independent $x_1, \dots, x_d \in A$ where $d = \dim{A}$ such that,
\[ K[x_1, \dots, x_d] \subset A \]
is a finite extension of domains.
\end{theorem}

\begin{proof}
We proceed by induction on the number of generators of $A$ as a $K$-algebra. If $n = 0$ then $A = K$ and we are done. Now we apply an induction hypothesis and assume that $A$ is generated by $n$ elements $y_1, \dots, y_n$ over $K$. If these are algebraically independent then we are done. Otherwise there is some relation $f \in K[x_1, \dots, x_n]$ such that,
\[ f(y_1, \dots, y_n) = 0 \]
in $A$. Let $z_i = y_i - y_n^{r^i}$ for $i < n$. Then obviously,
\[ f(z_1 + y_n^r, \dots, z_{n-1} + y_n^{r^{n-1}}, y_n) = 0 \]
The monomials in this expansion are of the form,
\[ \alpha \left( \prod_{i = 1}^{n-1} (z_i + y_n^{r^i})^{a_i} \right) y_n^{a_n} = \alpha y^{a_n + a_1 r + \cdots a_{n-1} r^{n-1}}_n + \cdots \]
However the exponent of $y_n$ encodes a unique base $r$ number if we choose $r$ larger than every exponent in $f$. Therefore, there is only one term of $f$ that can contribute to this largest $y_n$ exponent term (each monomial has a different $y_n$ exponent). Dividing by $\alpha$ we get a monic polynomial $f' \in K[z_1, \dots, z_{n-1}][x]$ such that $f'(y_n) = 0$ and thus $y_n$ is integral over $K[z_1, \dots, z_{n-1}]$. By using the induction hypothesis, there exist algebraically independent $x_1, \dots, x_d \in K[z_1, \dots, z_{n-1}]$ (the dimensions are the same because the extension is integral) such that,
\[ K[x_1, \dots, x_d] \subset K[z_1, \dots, z_{n-1}] \subset A \]
is a sequence of integral extensions proving the claim for $A$ and thus for all $A$ by induction on the number of generators. 
\end{proof}

\section{Going Up and Going Down}

\begin{lemma}
Let $A \subset B$ be an integral extension of domains. Then $A$ is a field iff $B$ is a field.
\end{lemma}


\begin{proof}
Let $A$ be a field. Let $b \in B$ be nonzero then $b$ is integral over $A$ so,
\[ b^n + a_{n-1} b^{n-1} + \cdots + a_0 = 0 \]
By diving though by $b$ we may assume that $a_0 \neq 0$ and thus $a_0 \in A$ is invertible so,
\[ b^{-1} = (-a_0)^{-1} (b^{n-1} + a_{n-1} b^{n-2} + \cdots + a_1) \in B \]
proving that $B$ is a field. If $B$ is a field then for any nonzero $a \in A$ we have $a^{-1} \in B$ is integral over $A$ so,
\[ a^{-n} + c_{n-1} a^{-n+1} + \cdots + c_0 = 0 \]
and therefore,
\[ a^{-1} = -(c_{n-1} + \cdots + a_0 a^{n-1}) \in A \]
so $A$ is a field.
\end{proof}

\begin{rmk}
Notice that if $B$ is a domain then any subring $A \subset B$ is automatically a domain. 
\end{rmk}

\begin{lemma}
Let $f : A \to B$ be an integral map of rings and $\p \subset B$ a prime. Then $f^{-1}(\p)$ is maximal if and only if $\p$ is maximal.
\end{lemma}

\begin{proof}
Indeed, consider $A / f^{-1}(\p) \subset B / \p$ which is an integral extension of domains. Thus $\p$ is maximal iff $B / \p$ is a field iff $A / f^{-1}(\p)$ is a field iff $f^{-1}(\p)$ is maximal.
\end{proof}

\begin{prop}[Lying Over]
Let $f : A \embed B$ be an integral extension of rings. Then the continuous map $f^* : \Spec{B} \to \Spec{A}$ is surjective.
\end{prop}

\begin{proof}
Let $\p \subset A$ be a prime and $B_\p = S^{-1} B$ for $S = A \setminus \p$. Consider the diagram,
\begin{center}
\begin{tikzcd}
A \arrow[d] \arrow[r, hook] & B \arrow[d]
\\
A_\p \arrow[r, hook] & B_\p
\end{tikzcd}
\end{center}
where the bottom extension is integral and injective because localization is exact. Since $A_\p$ is a nonzero ring so is $B_\p$ because $A_\p \embed B_\p$. Therefore, there exists a maximal ideal $\m \subset B_\p$. By the previous lemma, $\m$ pulls back to a maximal ideal in $A_\p$ which must be $\p A_\p$ since $A_\p$ is local and thus under $A \to A_\p \to B_\p$ we see that $\m \mapsto \p$. Hence by commutativity of the above square, the preimage of $\m$ in $B$ is a prime ideal lying over $\p$.
\end{proof}

\begin{cor}[Going Up]
If $f : A \to B$ is an integral map of rings then $f$ satisfies going up and $f^*(V(I)) = V(f^{-1}(I))$ which means that $f^* : \Spec{B} \to \Spec{A}$ is a closed map.
\end{cor}

\begin{proof}
Let $I \subset B$ be an ideal. The map $A / f^{-1}(I) \embed B / I$ is an integral extension of rings so $\Spec{B/I} \to \Spec{A / f^{-1}(I)}$ is surjective proving that $f^* V(I) = V(f^{-1}(I))$. Indeed, if $\q \in V(I)$ then $f^{-1}(\q) \supset f^{-1}(I)$ so $f^*(V(I)) \subset V(f^{-1}(I))$ and the surjectivity proves that $f^*(V(I)) = V(f^{-1}(I))$. In particular, if $I = \q$ is prime then we recover going up. Namely if $\p = f^{-1}(\q)$ and $\p' \supset \p$ then there exists $\q' \supset \q$ such that $\q' \mapsto \p'$. 
\end{proof}

\begin{prop}[Incomparablility]
If $A \to B$ is an integral map and $\p \subset \p'$ are primes of $B$ above $\q \subset A$ then $\p = \p'$.
\end{prop}

\begin{proof}
Since $A / \q \embed B / \p$ is an integral extension of domains then $(A/\q)_\q \embed (B / \p)_\q$ is an integral extension of domains with $(A / \q)_\q$ a field so $(B / \p)_\q$ is a field. Therefore $\p' = \p$ since there is a unique prime prime ideal in a field and $\Spec{(B/\p)_\q} \to \Spec{B}$ is injective.
\end{proof}

\begin{cor}
If $f : A \embed B$ is an integral extension of rings then $\dim{A} = \dim{B}$.
\end{cor}

\begin{proof}
Lying over + going up imply $\dim{A} \le \dim{B}$ and incomparability implies $\dim{B} \le \dim{A}$. 
\end{proof}

\begin{prop}[Going Down] 
If $f : A \embed B$ is an integral extension of domains and $A$ is integrally closed (i.e. $A$ is a normal domain) then
\begin{enumerate}
\item $f$ satisfies going down
\item if the extension of fraction fields $L / K$ is normal and $B$ is the integral closure of $A$ in $L$ then the fibers of $\Spec{B} \to \Spec{A}$ are acted on transitively by $G = \Gal{L/K}$.
\end{enumerate}
\end{prop}

(DO THIS PROPERLY!!!!!)

\begin{proof}
Let $K'/K$ be Galois and $B$ integrally closed. For each prime $\q \subset B$ I claim that the fibers of $\Spec{B'} \to \Spec{B}$ are finite (THIS HOLDS IF NOETHERIAN).
\bigskip\\
Let $\p_1, \dots, \p_n$ be the primes above $\p_1$ ordered such that $\p_1 \not\supset \p_j$ for $j > 1$ i.e. $\p_1$ is minimal (there are no relations by part (a) so there is actually no requirement on the order). Then by prime avoidance, there is some,
\[ x \in \p_1 \setminus \bigcup_{i = 2}^n \p_n \]
otherwise $\p_1$ would lie above some $\p_j$ for $j > 1$.
Now consider,
\[ y = \prod_{\sigma \in G} \sigma(x) \]
Then $y \in (K')^G = K$. Therefore,
\[ y \in \p_1 \cap K = \p_1 \cap B' \cap K = \p_1 \cap B = \q \]
because $B' \cap K = B$ since $B$ is integrally closed in $K$. Therefore, $y \in \p_i$ for each $i$ meaning that for each $i$ there is some $\sigma(x) \in \p_i$ and thus $x \in \sigma^{-1}(\p_i)$. However, $\sigma^{-1}(\p_i) = \p_j$ for some $j$ since it is a prime lying above $\q$. However, $x \in \p_j$ and thus $\p_j = \p_1$. Therefore $\p_i = \sigma(\p_1)$ so the Galois group acts transitively.
\bigskip\\
Now consider part 6. We may assume that $L / K$ is finite since we can always write $L$ as a union of finite extensions. Suppose we have prime ideals $\P$ and $\P'$ of $B$ both above $\p$. Assume that $\sigma_i(\P) \neq \P'$ for all $i$ running over the finite group $\Aut{L/K}$. By 2, $\P' \not\subset \sigma_i(\P)$ so there exists $x \in \P'$ such that $x \notin \sigma_i(\P)$. Take,
\[ y = \prod_{i = 1}^n \sigma_i(x) \]
and thus $\sigma(y) = y$ which implies that $y^{p^n} \in K$ for $\mathrm{char} \: K = p$. Since $x$ is integral over $A$ we know that $y^{p^n}$ is integral over $A$. But $A$ is integrally closed so $y^{p^n} \in A \cap \P' = \P$ then $y \in \p \subset \P$ which is a prime ideal so $\sigma_i(x) \in \P$ for some $i$ and thus $x \in \sigma_i^{-1}(\P)$ a contradiction. 
\bigskip\\
For part 5. we have integral domains $A \subset B$. Let $K = \Frac{A}$ and $L = \Frac{B}$ and let $L_1$ be the normal closure of $K$. Take $B_1$ to be the integral closure of $A$ inside $L_1$. Suppose we have a prime $\p \subset \p'$ in $A$ and $\P'$ above $\p'$. Furthermore, we can find $\P_1 \subset \P_1'$ in $B_1$ above $\p \subset \p'$ by surjectivity of the spec map and the going up property and also $\P_1''$ in $B_1$ above $\P'$ in $B$. Now $\P_1''$ and $\P_1'$ both lie above the same prime of $A$ so there is an automorphism $\sigma \in \Aut{L_1 / K}$ such that $\P_1'' = \sigma(\P_1')$. Thus,
\[ \sigma(\P_1) \subset \sigma(\P_1') = \P_1'' \]
Define $\P = \sigma(\P_1) \cap B \subset \sigma(\P_1') = \P_1''$. Thus, $\P \subset \P_1'' \cap B = \P'$. Finally,
\[ \P \cap A = \sigma(\P_1) \cap B \cap A = \sigma(\P_1) \cap A = \sigma(\P_! \cap A) = \sigma(\p) = \p \]
which satisfies the going down property.    

\end{proof}

\begin{example}
Let $C = \Spec{R}$ with $R = k[x,y]/(y^2 - x^2(x + 1))$ be the nodal cubic curve and $\wt{C} = \Spec{k[t]}$ its normalization where $\wt{C} \to C$ is given by $x \mapsto t^2-1$ and $y \mapsto t(t^2-1)$. This is dominant so $R \subset k[t]$. Then consider the map $\A^2 = \wt{C} \times \A^1 \to C \times \A^1$ given by,
\[ A = R[z] = k[x,y,z]/(y^2 - x^2(x + 1)) \embed k[t,z] = B \]
This is an integral extension of domains because $R \embed k[t]$ is finite (also $t^2 = x + 1$) and therefore satisfies lying over, incomparability, and going up. However, I claim it does not satisfy going down (and indeed $A$ is not normal). Visualize this map as the plane mapping down to the plane with the lines $t = 1$ and $t = -1$ glued together. Consider the diagonal line $L$ cut out by $\q = (t - z) \subset B$. Then its image $\bar{L}$ in $A$ is a line cut out by the ideal $\p' = (x - z^2 + 1, y - z(z^2 - 1))$ wrapping around and intersecting the singular line twice. Therefore the preimage of $\bar{L}$ is $L \cup (-1,1) \cup (1,-1)$. The point $\p = (x,y,z-1)$ is on the image of this line so $\p' \subset \p$ and is mapped to by the point $\mathfrak{P} = (t+1, z-1)$ (this is $(-1,1)$ in the plane). However, I claim that there is no prime $\mathfrak{P}' \subset \mathfrak{P}$ with $\mathfrak{P}' \mapsto \p'$. Indeed, the only height $1$ prime (there is a unique height zero prime $(0)$ and height $2$ primes are maximal and thus map to height $2$ primes) mapping to $\p'$ is $\q$ because the map is generically injective over $\bar{L}$ (injective exacly away from the points $(x,y,z-1)$ and $(x,y,z+1)$). 
\bigskip\\
More geometrically, this means that $f : \Spec{B} \to \Spec{A}$ is not open (going down implies ``stability under generalization'' which for finite type maps is equivalent to $f$ being open). Indeed, let $U = L^C$ be the complement of the line. Then $f(U) = \bar{L}^C \cup \{(0,0,1), (0,0,-1)\}$ is not open.
\end{example}

\section{Flatness}

\begin{defn}
A module $M$ over a ring $A$ is \textit{faithfully flat} if any sequence of $A$-modules,
\begin{center}
\begin{tikzcd}
N_1 \arrow[r, "f"] & N_2 \arrow[r, "g"] & N_3 
\end{tikzcd}
\end{center}
is exact if and only if the sequence,
\begin{center}
\begin{tikzcd}
N_1 \ot_A M \arrow[r, "f \ot \id_M"] & N_s \ot_A M \arrow[r, "g \ot \id_M"] & M_3 \ot_A M
\end{tikzcd}
\end{center}
is also exact. 
\end{defn}

\begin{rmk}
The ``only if'' direction immediately implies that $M$ is flat over $A$ so faithful flatness says additionally that tensoring cannot ``make a sequence exact''. 
\end{rmk}

\begin{lemma}
Let $M$ be a flat $A$-module. Then the following are equivalent,
\begin{enumerate}
\item $M$ is faithfully flat
\item for any $A$-module $N$ if $M \ot_A N = 0$ then $N = 0$
\item $\m M \neq M$ for every maximal ideal $\m \subset A$.
\end{enumerate}
\end{lemma}

\begin{proof}
We first show the equivalent of (a) and (b). Assuming (a) if $M \ot_A N = 0$ then the sequence,
\begin{center}
\begin{tikzcd}
0 \arrow[r] & N \arrow[r] & 0
\end{tikzcd}
\end{center}
becomes exact after tensoring and therefore it was already exact so $N = 0$ proving (b). Conversely, suppose that,
\begin{center}
\begin{tikzcd}
N_1 \ot_A M \arrow[r, "f \ot \id_M"] & N_s \ot_A M \arrow[r, "g \ot \id_M"] & M_3 \ot_A M
\end{tikzcd}
\end{center}
is exact. Then $(g \circ f) \ot_A M = 0$ so $\im{(g \circ f)} \ot_A M = \im{((g \circ f) \ot \id_M)} = 0$ by flatness so by assumption $\im{(g \circ f)} = 0$ and thus $g \circ f = 0$. Furthermore, by flatness
\[ (\ker{g} / \im{f}) \ot_A M = \ker{(g \ot_A \id_M)} / \im{(f \ot_A \id_M)} = 0 \] 
and thus $\ker{g} = \im{f}$ so the original sequence is exact proving (a). 
\bigskip\\
Now we show that (b) and (c) are equivalent. Assuming (b) let $\m \subset A$ be a maximal ideal. Since $A / \m_A \neq 0$ we have $M \ot_A A / \m_A \neq 0$ by (b) so $\m M \neq M$ proving (c). Conversely, suppose that $M \ot_A N = 0$ with $N \neq 0$. Then there is some nonzero $x \in N$ and we have $M \ot_A Ax \embed M \ot_A N = 0$ so $M \ot_A Ax = 0$. Let $I = \Ann{A}{x}$ then $A / I \iso A x$ so $M \ot_A A/I = 0$. Since $x \neq 0$ the ideal $I \subset A$ does not contain $1$ so we can choose a maximal ideal $\m \supset I$. Then $A/I \onto A / \m$ so $M \ot_A A/I \onto M \ot A / \m$ but $M \ot_A A/I = 0$ so $M \ot_A A / \m = 0$ showing that $\m M = M$.
\end{proof}

\begin{prop}
Let $\varphi : A \to B$ be flat local map of local rings and $M$ a nonzero finite $B$-module. Then $M$ is flat over $A$ if and only if $M$ is faithfully flat over $A$. 
\end{prop}

\begin{proof}
Faithfully flat modules are flat so it suffices to show that if $M$ is $A$-flat it is faithfully flat over $A$. Because $\m_A \subset A$ is the unique maximal ideal it suffices to show that $\m_A M \neq M$. Suppose that $\m_A M = M$ then $M \ot_A A / \m_A = 0$. Then there is a surjection, $B / \m_A B \onto B / \m_B$. Therefore, there is a surjection, $M \ot_B B / \m_A B \onto M \ot_B B / \m_B$. However, 
\[ M \ot_B B / \m_A B = M \ot_B (B \ot_A A / \m_A) = M \ot_A A / \m_A = 0 \]
and hence $M \ot_B B / \m_B = 0$ meaning $\m_B M = M$. Since $M$ is a finite $B$-module by Nakayama $M = 0$ giving a contradiction. This conclusion holds without $A$-flatness of $M$ but then if $M$ is $A$-flat the property $\m_A M \neq M$ implies that $M$ is faithfully flat over $A$.
\end{proof}

\begin{cor}
Let $\varphi : A \to B$ be a flat local map of local rings. Then $\varphi$ is faithfully flat.
\end{cor}

\begin{proof}
This is immediate from the previous proposition but we can also prove it directly as follows.
We want to show that for any $A$-module $N$ we have $B \otimes_A N = 0$ implies that $N = 0$. First we reduce to the case that $N$ is finitely generated.
If $N$ is not finitely generated then for every $N' \subset N$ finitely generated consider $B \otimes_A N' \subset B \otimes_A N$ (because $B$ is flat it is still injective) but $B \otimes_A N = 0$ so $B \otimes_A N' = 0$. Therefore, if we can prove the claim for finitely generated $N'$ then we would conclude that $N' = 0$ proving that $N = 0$ because for each $x \in N$ the submodule $A x \subset N$ is zero. 
\bigskip\\
Thus we may assume that $N$ is finitely generated. Consider the injection of fields $A / \m_A \embed B / \m_B$. Since $A/ \m_A$-module $N \otimes_A A / \m_A$ is a flat $A / \m_A$-module since $A / \m_A$ is a field there is an injection,
\[ N \otimes_A A / \m_A \embed (N \otimes_A A / \m_A) \ot_{A / \m_A} B / \m_B = N \ot_A B / \m_B = (N \otimes_A B) \ot_B B / \m_B \]
Since $N \otimes_A B = 0$ we see that $N \otimes_A A / \m_A = 0$. Therefore $N = \m_A N$ and $N$ is finitely generated so by Nakayama we see that $N = 0$ proving the claim.
\end{proof}

Indeed, $\varphi$ is faithfully flat. If $M$ is an $A$-module such that $M \otimes_A B = 0$ then for every finitely generated submodule $M' \subset M$ we have $M' \otimes_A B \subset M \otimes_A B = 0$ (injective by flatness). Consider the injection of fields $\kappa_A \embed \kappa_B$. Since $M ' \otimes_A \kappa_A$ is a flat $\kappa_A$-module ($\kappa_A$ is a field) we get an injection,
\[ M' \otimes_A \kappa_A \embed M' \otimes_A \kappa_B = (M' \otimes_A B) \otimes_B \kappa_B = 0 \]
and therefore $M' \otimes_A \kappa_A = 0$ and thus $M' = 0$ by Nakayama. Therefore $M = 0$ so $\varphi$ is fathfully flat. 

\begin{prop}
Let $\varphi : A \to B$ be flat. Then the following are equivalent,
\begin{enumerate}
\item $\varphi$ is faithfully flat
\item $\varphi^* : \Spec{B} \to \Spec{A}$ is surjective
\item $\mSpec{A} \subset \im{\varphi}$ meaning every maximal ideal is in the image.
\end{enumerate}
\end{prop}

\begin{proof}
Suppose that $\varphi$ is faithfully flat. For any $\p \in \Spec{A}$ we know that $A_\p / \p A_\p \neq 0$ so $B \ot_A A_\p / \p A_\p \neq 0$ by faithful flatness and therefore $\Spec{B \ot_A A_\p / \p A_\p}$ is nonempty proving that the fiber over $\p$ is nonempty so $\Spec{B} \to \Spec{A}$ is surjective. Thus (a) implies (b). It is clear that (b) implies (c). Now suppose that $\mSpec{A} \subset \im{\varphi}$. Since $B$ is a flat $A$-module to show that $B$ is faithfully flat it suffices to show that $\m B \neq B$ for all maximal ideals $\m \subset A$. For each maximal $\m \subset A$ there is some $\p \subset B$ so that $\varphi^{-1}(\p) = \m$ and thus $B / \m B \onto B / \p$ is nonzero so $\m B \neq B$ (the fiber $\Spec{B \ot_A A / \m}$ is nonempty so $B / \m B = B \ot_A A / \m \neq 0$). 
\end{proof}

\begin{prop}[Going Down]
Any flat ring map $\varphi : A \to B$ satisfies going down.
\end{prop}

\begin{proof}
Going down is equivalent to surjectivity of $\Spec{B_\p} \to \Spec{A_{\varphi^{-1}(\p)}}$ for each prime $\p \subset B$ which follows because $A_{\varphi^{-1}(\p)} \to B_\p$ is a flat local map and hence faithfully flat. 
\end{proof}

\subsection{Vector Bundles}

\begin{rmk}
The following has nice results to vector bundles which are explored in my vector bundles notes.
\end{rmk}

\begin{prop}
Let $\varphi : A \to B$ be a flat local map of local rings. Let $M$ be a finitely presented $B$-module which is flat over $A$. Suppose that $M / \m_A M$ is a free $B / \m_A B$-module. Then $M$ is a free $M$-module. 
\end{prop}

\begin{proof}
Choose an isomorphism,
\[ (B / \m_A B)^n \iso M / \m_A M \]
and choose a lift to a map $B^n \to M$ inducing a sequence,
\begin{center}
\begin{tikzcd}
0 \arrow[r] & K \arrow[r] & B^n \arrow[r] & M \arrow[r] & C \arrow[]r & 0
\end{tikzcd}
\end{center}
Since $M$ is finitely-presented, $K$ and $C$ are finite $B$-modules. From the exact sequence, $C / \m_A C = 0$ and thus,
\[ C / \m_A C \onto C / \m_B C \]
proves that $C = \m_B C$ and thus by Nakayama's lemma $C = 0$. Therefore, we have a short exact sequence,
\begin{center}
\begin{tikzcd}
0 \arrow[r] & K \arrow[r] & B^n \arrow[r] & M \arrow[r] & 9
\end{tikzcd}
\end{center}
Since $M$ is flat over $A$ this sequences remains exact after applying $- \ot_A (A/\m_A)$ and thus $K / \m_A K = 0$ and hence $K / \m_B K = 0$. Since $K$ is a finite $B$-module, by Nakayama, we see that $K = 0$ and hence $B^n \iso M$. 
\end{proof}

\begin{cor}
Let $f : X \to Y$ be a flat map of schemes and $\F$ a coherent $\struct{X}$-module flat over $Y$. Suppose that $\F|_{X_y}$ is a vector bundle on $X_y$ for some $y$. Then there is an open neighborhood $U \subset X$ of $X_y$ such that $\F|_U$ is a vector bundle.
\end{cor}

\begin{proof}
Since $\F$ is coherent, it suffices to show that $\F_x$ is a free $\stalk{X}{x}$-module for each $x \in X_y$ which follows immediately from the previous result.
\end{proof}

\begin{example}
Consider $X = \A^3 \sm \{(0,0,0)\} \to \A^1 = \Spec{k[z]}$ and $\F = \wt{(x,y)}$. This sheaf is obviously flat but its fiber over $z = 0$ is a vector bundle since it is $\struct{X}$ away from $x = y = 0$. However, it is not a vector bundle on any other fiber. 
\end{example}


\begin{cor}
Let $f : X \to Y$ be a flat and proper map of schemes and $\F$ a coherent $\struct{X}$-module flat over $Y$. Suppose that $\F|_{X_{y_0}}$ is a vector bundle on $X_{y_0}$ for some $y_0 \in Y$. Then there is an open $y_0 \in V \subset Y$ such that $\F|_{X_V}$ is a vector bundle. In particular for all $y \in V$ we have that $\F|_{X_{y}}$ is a vector bundle.
\end{cor}

\begin{proof}
Using the previous result, it suffices to show that the set,
\[ V = \{ y \in Y \mid \F|_{X_y} \text{ is a vector bundle} \} \]
is poen. For any $y \in V$ there is an open neighborhood $X_y \subset U \subset X$ so that $\F|_U$ is a vector bundle and thus $y \in f(U^C)^C \subset V$ is open because $f$ is closed.
\end{proof}

\begin{example}
Let $\pi_1 : X = \P^1 \times \P^1 \to \P^1 = S$ be the projection. Let $x = X$ be a point and $\I \subset \struct{X}$ the ideal sheaf of $x = (0,0) \in X$. For each fiber $X_t$ with $t \neq 0$ we have $\I|_{X_t} = \struct{X_t}$ is a vector bundle. However, $\I$ is not a vector bundle so we cannot have $\I|_{X_0}$ be a vector bundle by the above result. I claim that $\I$ is $\pi_1$-flat. This is clear on $X \sm \{ x \}$ so I we consider the local structure around $x$. On a dense open we have the following algebra problem,
\[ A = k[x]_{(x)} \to k[x,y]_{(x,y)} = B \quad \text{ with the ideal } \quad I = \m_B = (x,y) \subset k[x,y]_{(x,y)} \]
I claim that $I$ is flat over $A$. There is an exact sequence,
\begin{center}
\begin{tikzcd}
0 \arrow[r] & B \arrow[r, "(y \; -x)"] & B^2 \arrow[r, "(x \; y)"] & I \arrow[r] & 0
\end{tikzcd}
\end{center}
Then applying \href{https://stacks.math.columbia.edu/tag/00MK}{Tag 00MK} we just need to show that $B/\m_A B \to (B / \m_A B)^2$ is injective which is true because $y$ is a non zero-divisor on $B / \m_A B$. Thus $I$ is $A$-flat. Furthermore, there is an exact sequence,
\begin{center}
\begin{tikzcd}
0 \arrow[r] & (B / \m_A B) \arrow[r, "(y \; 0)"] & (B / \m_A B)^2 \arrow[r, "(0 \; y)"] & I / \m_A I \arrow[r] & 0
\end{tikzcd}
\end{center}
Therefore, we get the local structure,
\[ I / \m_A I \cong k \oplus k[y]_{(y)} \]
but its image in $B / \m_A B$ is just $(y)$ which is locally free. This we see that $\I|_{X_0} \cong \struct{X_0}(-1) \oplus \iota_* k$ which has degree zero as it must because $\I|_{X_t} \cong \struct{X_t}$ for $t \neq 0$ and degree is constant in flat families.
\end{example}

\begin{example}
Consider a degeneration, 
\[ f : X = \Proj{k[t][X,Y,Z]/(XY - t Z^2)} \to \Spec{k[t]} = S \]
with $X$ smooth and $f$ flat and proper but $f$ has a singular fiber over $t = 0$. Then there is a sequence,
\begin{center}
\begin{tikzcd}
0 \arrow[r] & f^* \Omega^1_S \arrow[r] & \Omega_X \arrow[r] & \Omega_{X/S} \arrow[r] & 0
\end{tikzcd}
\end{center}
Now $\Omega_{X/S} |_{X_t} = \Omega_{X_t}$ is a vector bundle for the smooth fibers ($t \neq 0$). However, $\Omega_{X/S}|_{X_0} = \Omega_{X_0}$ is not a vector bundle since $X_0$ is singular. I claim that $\Omega_{X/S}$ is flat over $S$. We consider the local structure, on the chart $D_+(Z)$. Let $A = k[t]$ and $B = k[t][x,y]/(xy - t)$ then the above exact sequence becomes,
\begin{center}
\begin{tikzcd}
0 \arrow[r] & B \d{t} \arrow[r, "x \d{y} + y \d{x}"] & B \d{x} \oplus B \d{y} \arrow[r] & \Omega_{D_+(Z)/S} \arrow[r] & 0
\end{tikzcd}
\end{center}
Therefore,
\[ M = \Omega_{D_+(Z)/S} = (B \d{x} \oplus B \d{y})/(x \d{y} + x \d{y}) \]
Thus the rank jumps at $\m = (x,y)$. However, I claim that $M$ is flat over $A$. Applying \href{https://stacks.math.columbia.edu/tag/00MK}{Tag 00MK} we just need to show that,
\[ (B / t B)_{\m} \d{t} \to (B / t B)_\m \d{x} \oplus  (B / t B)_\m \d{y}  \]
is injective. Indeed, if $f \d{t} \mapsto 0$ then $fx = 0$ and $fy = 0$ in $(B / t B)_\m = (k[x,y]/(xy))_{\m}$. Then $f \in \Ann{}{x} \cap \Ann{}{y} = (y) \cap (x) = (xy)$ so $f = 0$ in $(B / t B)_\m$. Thus the map is injective.
\end{example}

\begin{rmk}
We saw in the first example that a smooth proper map can have a flat ideal sheaf fail to be a vector bundle. However, this does not happen if the closed subscheme is flat over the base.
\end{rmk}

\begin{prop}
Let $f : X \to Y$ be a smooth proper map of schemes and $Z \subset X$ a closed subscheme flat over $Y$. Then the locus,
\[ V = \{ y \in Y \mid Z_y \subset X_y \text{ is Cartier} \} \]
is clopen.
\end{prop}

\begin{proof}
Consider the ideal sheaf sequence,
\begin{center}
\begin{tikzcd}
0 \arrow[r] & \I \arrow[r] & \struct{X} \arrow[r] & \iota_* \struct{Z} \arrow[r] & 0
\end{tikzcd}
\end{center}
Because $Z \to Y$ is flat, $\I|_{X_y}$ is the ideal sheaf of $Z_y \subset X_y$. By the previous result, the locus where $\I|_{X_y}$ is a vector bundle (and hence a line bundle since it embedds in $\struct{X}$) is open. Thus we just need to prove closedness. It suffices to show that $V$ is stable under specialization. 
(REDUCE TO THE DVR CASE, 1 NOETHERIAN, 2 BLOW UP, 3 NORMALIZE)
Thus we can assume that $Y = \Spec{R}$ where $R$ is a DVR and $D_K \subset X_K$ is a Cartier divisor. We we need to show that $D_0 \subset X_0$ is Cartier. For each $x \in X_0$ let $A = \stalk{X}{x}$ and we have the following: a flat ring map $R \to A$ with $A$ regular, an ideal $I \subset A$ with $R \to A/I$ flat such that $I \ot_R K \subset A \ot_R K$ is principal. Since $R \to A/I$ is flat $A/I$ can only have associated  points in the generic fiber thus $A/I$ is unmixed since in the generic fiber $I$ is principal and $A$ is regular so $I$ has no embedded primes by the unmixedness theorem. Consider the primary decomposition, 
\[ I = Q_1 \cap \cdots \cap Q_r \]
where $Q_i$ is $\p_i$-primary where $\height{\p_i} = 1$ by unmixedness. Since $A$ is a UDF we have $\p_i = (p_i)$ are principal. Therefore, 
\end{proof}

\begin{rmk}
The following example shows that smoothness really is necessary. 
\end{rmk}

\begin{example}
Consider,
\[ f : X = \Proj{k[t][X,Y,Z]/(X^3-Y^2Z)} \to S = \Spec{k[t]} \]
and the divisor
\[ D = \Proj{k[t][X,Y,Z]/(X^3 - Y^2 Z, X - t^2 Z, Y - t^3 Z)} \]
which is the image of a section of $f$ and hence flat. For $t \neq 0$ we have $D_t \subset X_t$ a Cartier divisor but $D_0 \subset X_0$ is not a Cartier divisor. 
\end{example}

\section{Dedekind Domains}

\begin{defn}
A \textit{Dedekind Domain} is a Noetherian integrally closed domain $A$ with $\dim{A} = 1$.
\end{defn}

\subsection{Fractional Ideals}

\begin{defn}
Let $A$ be a domain and $K = \Frac{A}$. A \textit{fractional ideal} is a nonzero $A$-submodule $J \subset K$ such that for some nonzero $d \in A$ we have $d J \subset A$.
\end{defn}

\begin{rmk}
For the remainder of the section, $A$ is a domain.
\end{rmk}

\begin{prop}
If $A$ is Noetherian, then every fractional ideal is finitely generated.
\end{prop}

\begin{proof}
Since $d J \subset A$ is an ideal it is finitely generated and since $A$ is a domain $d : J \to d J$ is an isomorphism. 
\end{proof}

\begin{defn}
A fractional ideal $J$ is \textit{invertible} if there is a fractional ideal $J'$ such that $J' J = A$.
\end{defn}

\begin{rmk}
If $J$ is principal meaning $J = r A$ for nonzero $r \in K$ then $J$ is invertible with inverse $J^{-1} = r^{-1} A$.
\end{rmk}

\begin{prop}
If $J \subset K$ is a fractional ideal of $A$ then,
\[ J^{-1} = \{ x \in K \mid x J \subset A \} \]
is also a fractional ideal. 
\end{prop}

\begin{proof}
Indeed, choose $d \in A$ such that $d J \subset A$ and choose nonzero $x \in d J \subset A$. Then by definition $J^{-1} x \subset A$ and $d \in J^{-1}$ is nonzero proving that $J^{-1}$ is a fractional ideal. 
\end{proof}

\begin{lemma}
Let $A$ be a Noetherian ring and $I \subset A$ an ideal. Then there is a finite list of prime ideals $\p_1, \dots, \p_n$ such that,
\[ \p_1 \dots \p_n \subset I \]
\end{lemma}

\begin{proof}
Indeed, since $A$ is Noetherian, there are finitely many minimal primes $\p_1, \dots, \p_r$ over $I$. Since $\p_1 \cdots \p_r \subset \sqrt{I}$ and all the idealls are finitely generated, there is some $n$ such that,
\[ (\p_1 \cdots \p_r)^n \subset I \]
\end{proof}

\begin{prop}
If $A$ is Noetherian and $I \subset A$ is a nonzero ideal then $I^{-1} \supsetneq A$.
\end{prop}

\begin{proof}
Choose a nonzero $x \in I$ and consider a minimal list of primes such that,
\[ \p_1 \cdots \p_r \subset (x) \]
so $I \subset \p_i$ for some $i$ WLOG $i = r$. Therefore,
\[ x^{-1} \p_1 \cdots \p_{r-1} I \subset x^{-1} \p_1 \cdots\p_{r-1} \p_r \subset A \]
so if we choose nonzero $x_i \in \p_i$ then $x^{-1} x_1 \cdots x_{r-1} \in I^{-1}$. If $x^{-1} x_1 \cdots x_{r-1} \in A$ then $x_1 \cdots x_{r-1} \subset (x)$ for all choices of $x_i \in \p_i$ meaning $\p_1 \cdots \p_{r-1} \subset (x)$ contradicting minimality. Therefore, we have an element of $I^{-1} \sm A$. 
\end{proof}

\begin{rmk}
Although $J^{-1}$ is defined in general, it will only satisfy $J^{-1} J = A$ when $J$ is invertible. Indeed often $J^{-1} J = A$ even though $J^{-1} J \subsetneq A$. For example, let $A = k[x,y]/(y^2 - x^3)$ and consider $J = (x,y)$. Then $J^{-1} = A[\frac{y}{x}]$ because if $f \in K$ satisfies $f x \in A$ and $f y \in A$ then $f = \frac{a}{x} = \frac{a'}{y}$ so $a y = a' x$ then $\bar{a} y = 0$ in $k[y]/(y^2)$ so $a \in (y)$. However, $J J^{-1} = J$ since $\frac{y}{x} (x,y) = (y, x^2)$.
\end{rmk}

\begin{prop}
If $J$ is invertible then its inverse is unique and equals,
\[ J^{-1} = \{ x \in K \mid x J \subset A \} \]
\end{prop}

\begin{proof}
Fractional ideals form a commutative monoid under multiplication so inverses are unique. Suppose that $J' J = A$. Since $J^{-1} J \subset A$ we see that $J^{-1} = J^{-1} J J' \subset J'$. Furthermore, by definition $J' \subset J^{-1}$ since $J' J \subset A$. 
\end{proof}

\begin{cor}
A fractional ideal $J$ is invertible iff $J^{-1} J = A$.
\end{cor}

\renewcommand{\Cl}[2][]{\mathrm{Cl}_{#1} \left( #2 \right)}

\begin{defn}
The ideal class group $\Cl[\text{ideal}]{A}$ is the group of invertible fractional ideals.
\end{defn}

\begin{rmk}
This is really not the correct definition of the class group (hence the subscript) in general. We want $\Cl{A} = 0$ iff $A$ is a UFD which will be true for the Weil class group. However, in the case of Dedekind domains all the definitions agree. 
\end{rmk}

\subsection{The Picard Group}

\begin{prop}
A fractional ideal $J$ is invertible iff it is invertible as an $A$-module. 
\end{prop}

\begin{cor}

\end{cor}

\subsection{The Weil Class Group}

\begin{defn}
DO THIS
\end{defn}

\begin{prop}
$\Cl{A} = 0$ if and only if $A$ is a UFD.
\end{prop}

\begin{prop}
There is a natural map $\Cl[\text{ideal}]{A} \to \Cl{A}$ which is an isomorphism if and only if $A$ is locally factorial. 
\end{prop}

\subsection{Fractional Ideals In Dedekind Domains}

\begin{defn}
An $A$-module $M$ is \textit{faithful} if $aM = 0$ implies $a = 0$. 
\end{defn}

\begin{lemma}
Let $A \to B$ be a ring map and $b \in B$. Then the following are equivalence,
\begin{enumerate}
\item $b$ is integral over $A$
\item $A[b]$ is a finite $A$-module
\item there exists a faithful $A[b]$-module $M$ which is finite as an $A$-module.
\end{enumerate}
\end{lemma}

\begin{proof}
If $b$ is integral over $A$ then it satisfies some,
\[ b^n + a_{n-1} b^{n-1} + \cdots + a_{0} = 0 \]
proving that $1, b, \dots, b^{n-1}$ is an $A$-generating set of $A[b]$ over $A$. Now suppose that $A[b]$ is a finite $A$-module then (c) follows trivially taking $M = A[b]$ since if $a A[b] = 0$ then $a \cdot 1 = 0$ so $a = 0$. Thus it suffices to show that $(c) \implies (a)$. 
\bigskip\\
Let $M$ be a faithful $A[x]$-module finite over $A$. Let $\pi : A^n \onto M$ be a generating set. Then multiplication by $b$ produces a diagram,
\begin{center}
\begin{tikzcd}
A^n \arrow[r, dashed, "\varphi"] \arrow[d, "\pi", two heads] & A^n \arrow[d, two heads, "\pi"]
\\
M \arrow[r, "(-)\cdot b"] & M
\end{tikzcd}
\end{center}
Let $p \in A[x]$ be the characteristic polynomial of $\varphi$ which is monic. By Cayley-Hamilton, $p(\varphi) = 0$ and thus,
\[ \pi \circ p(\varphi) = (- \cdot p(b)) \circ \pi = 0 \]
but $\pi$ is surjective so $p(b) M = 0$ and thus $p(b) = 0$ proving that $b$ is integral over $A$.
\end{proof}

\begin{prop}
Let $A$ be a Dedekind domain. Then every nonzero fractional ideal $J$ of $A$ is invertible.
\end{prop}

\begin{proof}
First suppose that $J = \p$ is a nonzero (hence maximal) prime. We have already shown that $\p^{-1}$ is a fractional ideal and $\p^{-1} \neq A$. Now $\p^{-1} \p \subset A$ so because $\p$ is maximal either $\p^{-1} \p = A$ or $\p^{-1} \p = \p$. Choose $x \in \p^{-1} \sm A$ if $\p^{-1} \p = \p$ then $x \p \subset \p$ meaning $\p$ is an $A[x]$-module. However, $\p$ is a finite $A$-module by Noetherianity and is faithful as an $A[x]$-module since $\p$ is nonzero and $A[x] \subset K$ is a domain. Hence $x$ is integral over $A$ by the lemma so $x \in A$ giving a contradiction. Thus $x \p = A$ so $\p^{-1} \p = A$ and $A$ is invertible. Now for any fractional ideal $J$ choose $d \in A$ such that $I = d J$ is a nonzero ideal. Then there exist primes such that,
\[ \q \p_1 \cdots \p_r \subset I \subset \q \]
and applying $\q^{-1}$ we get,
\[ \p_1 \cdots \p_{r} \subset \q^{-1} I \subset A \]
giving a new ideal $I' = \q^{-1} I$. Either $I' = A$ or $I'$ is a proper ideal so $I \subset \p_i$ for some $i$. Inducting we see that $I$ is invertible hence $d^{-1} I^{-1} J = I^{-1} I = A$ so $J$ is invertible. 
\end{proof}

\begin{rmk}
This proof is similar to \href{https://math.stackexchange.com/questions/188585/localizations-of-dedekind-domains-are-discrete-valuation-rings}{this} one on mathoverflow.
\end{rmk}

\begin{cor}
Let $A$ be a Dedekind domain. Then the natural maps,
\[ \Cl{A} \leftarrow \Cl[\text{ideal}]{A} \to \Pic{A} \]
are isomorphisms.
\end{cor}

\begin{thm}
Let $A$ be a Dedekind domain. Then every ideal $I \subset A$ has a unique factorization,
\[ I = \p_1^{e_1} \cdots \p_r^{e_r} \]
into prime ideals.
\end{thm}

\begin{proof}
From the proof that $I$ is invertible we saw that $\p_1^{-1} \cdots \p_r^{-1} I = A$ for some sublist of primes whose product is contained in $I$. Therefore by inversion,
\[ \p_1 \cdots \p_r = I \]
where there may be repeats. Uniqueness follows from if $\p$ contains $I$ then $\p$ must lie above some $\p_i$ so $\p = \p_i$ by maximality. Then applying inverses we conclude that any two such multisets of primes are equal. 
\end{proof}

\begin{proof}

\end{proof}

\subsection{DVRs}

\begin{defn}
A \textit{Discrete Valuation Ring} (DVR) is a local PID with exactly two prime ideal (i.e. not a field).
\end{defn}

\begin{rmk}
For any PID, $\dim{A} = 1$ because if $\p_1 \subsetneq \p_2$ for two primes then write $\p_1 = (p_1)$ and $\p_2 = (p_2)$ so $p_1 = r p_2$ so $r p_2 \in \p_1$ so either $r \in \p_1$ or $p_2 \in \p_1$. Since $\p_1 \neq \p_2$ we know $p_2 \notin \p_1$ hence $r \in \p_1$ so $p_1 = rs p_1$ and thus since $A$ is a domain $rs = 1$ or $p_1 = 0$. In the first case $r \in A^\times$ so $\p_1 = \p_2$ giving a contradiction so $\p_1 = (0)$. Therefore if $A$ is a local PID either $A$ is a field or $A$ is a DVR.
\end{rmk}

\begin{rmk}
Let $\m$ be the unique maximal ideal. Then $\m = (\varpi)$ for some $\varpi \in R$ which we call a \textit{uniformizer}. 
\end{rmk}

\begin{prop}
Let $R$ be a DVR then $R$ is a valuation ring in $K = \Frac{R}$.
\end{prop}

\begin{proof}
For each $x \in K$ we need to show that either $x$ or $x^{-1}$ is in $R$. Suppose not then write $x = \frac{a}{b}$ with $a,b \in R$ and neither is a unit else either $x$ or $x^{-1}$ would lie in $R$. Thus $a,b \in \m$ so write $a = a_1 \varpi$ and $b = b_2 \varpi$ so,
\[ \frac{a}{b} = \frac{a_1}{b_1} \]
This gives a contradiction by descent. Indeed, we get that $r_1, r_2 \in \m$ so iterating the proof we we get a sequence of increasing ideals,
\[ (a) \subset (a_1) \subset (a_2) \subset \cdots \]
which must stabilize (PIDs are noetherian since in particular every ideal is finitely generated). Thus we must have $a_i = a_{i+1}$ for some $i$ but $a_i = \varpi a_{i+1}$ so $a_i = 0$ since $\varphi \neq 1$. Therefore we conclude. 
\end{proof}

\begin{prop}
Let $A$ be a Dedekind domain and $\p \subset A$ a nonzero prime. Then $A_\p$ is a DVR.
\end{prop}



\begin{proof}
Since $\dim{A_\p} = 1$ and $A$ is a local domain we see that $A_\p$ has exactly two prime ideals. Also $A_\p$ is Noetherian, integrally closed, and dimension $1$ so it suffices to show that a Dedekind domain $A$ with exactly two prime ideals is a PID. Let $I \subset A$ be a nonzero ideal. By Dedekind prime factorization $I = \m^e$ since there is exactly one nonzero ideal. Thus it suffices to prove that $\m$ is principal. Choose $x \in \m$ so that $e$ where $(x) = \m^e$ is minimal. Then every $x \in \m$ is contained in $\m^e$ so $\m \subset \m^e$ so by Nakayama\footnote{Indeed, $\m$ is maximal so $\m^e = \m$. If $e > 1$ then $\m^e \subset \m^2 \subset \m$ so $\m^2 = \m$ but $\Jac{A} = \m$ and $\m$ is finitely generated by Noetherianity so by Nakayama $\m = 0$ which is false by assumption. Note that Noetherianity is necessary. Otherwise we could have $k[x, x^{\frac{1}{2}}, x^{\frac{1}{4}}, \dots]$ and $\m = (x, x^{\frac{1}{2}}, x^{\frac{1}{4}}, \dots)$ satisfies $\m^2 = \m$.} $e = 1$ so $(x) = \m$ proving the claim.
\end{proof}


\end{document}