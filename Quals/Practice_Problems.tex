\documentclass[12pt]{article}
\usepackage{import}
\import{"../Algebraic Geometry/"}{AlgGeoCommands}

\newcommand{\Loc}[1]{\mathfrak{Loc}\left( #1 \right)}
\newcommand{\AbGrp}{\mathbf{AbGrp}}

\renewcommand{\K}{\mathbb{K}}

\newcommand{\inner}[2]{\left< #1, #2 \right>}

\newcommand{\B}{\mathcal{B}}


\begin{document}
\section{Group Theory}

\subsection{Semi-Direct Products}

\begin{prop}
If $N, M \subset G$ are normal subgroups such that $N \cap M = \{ e \}$ and $NM = G$ then $G \cong N \times M$.
\end{prop}

\begin{proof}
Consider the map $\varphi : N \times M \to G$ via $\varphi(n,m) = nm$. First, we need to show that this is a homomorphism. It suffices to show that if $n \in N$ and $m \in M$ then $nm = mn$. Indeed, $nmn^{-1}m^{-1} \in N \cap M$ because both are normal (so $n m n^{-1} \in M$ and thus so is $n m n^{-1} m^{-1}$ and ditto for $N$). However, $N \cap M = \{ e \}$ thus $nm = mn$. Because $NM = G$ the map $\varphi$ is surjective. Finally, if $nm = e$ then $n = m^{-1}$ so $n \in N \cap M$ and thus $n = m = e$ so $\ker{\varphi} = \{ e \}$ and thus $\varphi$ is an isomorphim.
\end{proof}

\begin{rmk}
The semidirect product $N \rtimes_{\varphi} H$ for the action $\varphi : H \to \Aut{N}$ is defined by,
\[ (n,h) \cdot (n', h') = (n \varphi(h) \cdot n', h h') \] Then notice that $(n, h)^{-1} = (\varphi(h^{-1}) \cdot n^{-1}, h^{-1})$ because,
\[ (n, h) \cdot (\varphi(h^{-1}) \cdot n^{-1}, h^{-1}) = (n \varphi(h) \varphi(h^{-1}) n^{-1}, h h^{-1}) = (e,e) \]
and likewise,
\[ (\varphi(h^{-1}) \cdot n^{-1}, h^{-1}) \cdot (n, h) = (\varphi(h^{-1}) n^{-1} \varphi(h^{-1}) \cdot n, h^{-1} h) = (\varphi(h^{-1}) (n^{-1} n), e) = (e, e) \]
Then notice,
\[ (e, h) \cdot (n, e) \cdot (e, h^{-1}) = (\varphi(h) \cdot n, h) \cdot (e, h^{-1}) = (\varphi(h) \cdot n, e) \]
so we say that $H \acts N$ through $\varphi$ via conjugation inside $G = N \rtimes H$. 
\end{rmk}

\begin{prop}
Fix two groups $N$ and $H$. The isomorphism type of $N \rtimes_\varphi H$ only depends on the class of $\varphi : H \to \Aut{N}$ up to precomposition with automorphisms of $H$ and with conjugation by automorphisms of $N$.
\end{prop}

\begin{proof}
Suppose that $\varphi, \psi : H \to \Aut{N}$ are homomorphisms such that,
\[ \psi(h) = \gamma \circ \varphi(\eta^{-1}(h)) \circ \gamma^{-1} \]
where $\eta \in \Aut{H}$ and $\gamma \in \Aut{N}$. Then consider the bijection $f_{\gamma, \eta} : N \rtimes_\varphi H \to N \rtimes_\psi H$ via $(n, h) \mapsto (\gamma(n), \eta(h))$. We claim this is an isomorphism if and only if $\varphi$ and $\psi$ satisfy the above relation. Indeed,
\begin{align*}
f_{\gamma, \eta}((n, h) \cdot_\varphi (n', h')) & = f_{\gamma, \eta}((n \varphi(h) \cdot n', h h')) = (\gamma(n) \gamma \circ \varphi(h) \cdot n', \eta(h h'))
\\
(\gamma(n), \eta(h)) \cdot_{\psi} (\gamma(n'), \eta(h')) & = (\gamma(n) \psi(\eta(h)) \cdot \gamma(n'), \eta(h h'))
\end{align*}
Therefore, we get equality if and only if,
\[ \psi(\eta(h)) \circ \gamma = \gamma \circ \varphi(h) \]
which is equivalent to the stated relation.
\end{proof}

\begin{rmk}
The converse is, in general, false meaning there exist groups $H, N$ and homomorphisms $\varphi, \psi : H \to \Aut{N}$ which do not lie in the same class such that $N \rtimes_\varphi H \iso N \rtimes_\psi H$ anways. See \chref{https://math.stackexchange.com/questions/527800/when-are-two-semidirect-products-isomorphic}{this} answer. However, the converse does hold in some special cases where automorphisms of the semidirect product are easier to understand. 
\end{rmk}
 
\begin{prop}
Suppose that all maps $N \to H$ and $H \to N$ are trivial. Then,
\[ \frac{\{ \varphi : H \to \Aut{N} \}}{\text{conjugation and composition}} \to \frac{\{ N \rtimes_\varphi H \}}{\text{isomorphism}} \]
is a bijection. 
\end{prop}

\begin{proof}
Consider an isomorphism $f : N \rtimes_\varphi H \iso N \rtimes_\psi H$. Then, consider,
\[ N \embed N \rtimes_{\varphi} H \xrightarrow{f} N \rtimes_{\psi} H \onto H \]
By assumption this map is trivial so $N \subset \ker{\pi_H \circ f}$ where $\pi_H : N \rtimes_\psi H \onto H$ is the canonical projection. Therefore, $f$ fits into a diagram,
\begin{center}
\begin{tikzcd}
1 \arrow[r] & N \arrow[d, "\gamma"] \arrow[r] & N \rtimes_\varphi H \arrow[d, "f"] \arrow[r] & H \arrow[d, "\eta"] \arrow[r] & 1
\\
1 \arrow[r] & N \arrow[r] & N \rtimes_\psi H \arrow[r] & H \arrow[r] & 1
\end{tikzcd}
\end{center}
This forces the map $f$ to be ``upper triangular'' but we want it to be ``diagonal. Consider the map,
\[ H \xrightarrow{\eta^{-1}} H \to N \rtimes_{\varphi} H \xrightarrow{f} N \rtimes_\psi H \]
This is a section of $N \rtimes_\psi H \to H$ and hence it differs from the standard section by a map $H \to N$ which must be trivial by assumption. Thus $f$ is ``diagonal'' meaning it also commutes with the sections and thus $f((n, h)) = (\gamma(n), \eta(h))$. Then we can apply the proof of the previous proposition to conclude.
\end{proof}

\begin{rmk}
Let $H$ be any group then $\varphi : H \to \Aut{H}$ sending $h \mapsto \varphi_h$ where $\varphi_h$ is the inner automorphism $\varphi_h : x \mapsto h x h^{-1}$ is a crossed module. Indeed,
\[ \varphi(\psi \cdot h) = \psi \circ \varphi_h \circ \psi^{-1} \]
because,
\[ \varphi(\psi \cdot h)(x) = (\psi \cdot h) x (\psi \cdot h^{-1}) = \psi(h \psi^{-1}(x) h^{-1}) = (\psi \circ \varphi_h \circ \psi^{-1})(x) \]
and furthermore,
\[ \varphi(h) \cdot h' = h h' h^{-1} \]
by definition.
\end{rmk}

\begin{prop}
If $N \subset G$ is normal and $H \subset G$ is any subgroup such that $N \cap H = \{ e \}$ and $NH = G$ then $G \cong N \rtimes_\varphi H$ for some action $\varphi : H \to \Aut{N}$.
\end{prop}

\begin{proof}
Consider the action $H \acts N$ via conjugation $h \cdot n = h n h^{-1}$ giving $\varphi : H \to \Aut{N}$. Then consider the map,
\[ f : N \rtimes_\varphi H \to G \quad \text{via} \quad (n, h) \mapsto n h \]
Then consider,
\[ f((n, h) \cdot (n', h')) = f((n \varphi(h) \cdot n', h h')) = n (\varphi(h) \cdot n') h h' = n (h n' h^{-1}) h h' = (n h) (n' h') = f((n,h)) f((n', h')) \]
Furthermore, $f$ is injective since if $f(n,h) = e$ then $nh = e$ so $n, h \in N \cap H = \{ e \}$ so $n = h = e$. Finally, $NH = G$ so $f$ is surjective. 
\end{proof}

\begin{rmk}
Notice that this condition is exactly the condition that there is a split short exact sequence,
\begin{center}
\begin{tikzcd}
1 \arrow[r] & N \arrow[r] & G \arrow[r] & H \arrow[r] \arrow[l, bend right] & 1
\end{tikzcd}
\end{center}
Furthermore, note that being split on the left is stronger. A diagram,
\begin{center}
\begin{tikzcd}
1 \arrow[r] & N \arrow[r] & G \arrow[r] \arrow[l, bend right] & H \arrow[r] & 1 
\end{tikzcd}
\end{center}
gives a map $G \to N \times H$ via the two projections and hence an isomorphism of diagrams,
\begin{center}
\begin{tikzcd}
1 \arrow[r] & N \arrow[d, equals] \arrow[r] & G \arrow[d] \arrow[r] & H \arrow[d, equals] \arrow[r] & 1
\\
1 \arrow[r] & N \arrow[r] & N \times H \arrow[r] & H \arrow[r] & 1
\end{tikzcd}
\end{center}
One way to understand the difference is that the second scenario will give a splitting on the right along with specifiying that $H = \ker{(G \to N)}$ and thus $H$ is normal which implies that the semi-direct product is actually direct.  
\end{rmk}


\subsection{The Isomorphism Theorem}

\begin{thm}[Second Isomorphism Theorem]
Let $N \subset G$ be a normal subgroup and $H \subset G$ any subgroup. Then $N \cap H$ is normal in $H$ and $HN \subset G$ is a subgroup and $H/H \cap N \cong HN/N$.
\end{thm}

\begin{proof}
First, suppose that $h \in H$ and $n \in H \cap N$. Then consider $h n h^{-1}$. Because $N \subset G$ is normal then $h n h^{-1} \in N$ but $n \in H$ so $h n h^{-1} \in H$ and thus $h n h^{-1} \in H \cap N$ so $H \cap N$ is normal in $H$. 
Consider the map $\varphi : H \to HN / N$ via $h \mapsto [h \cdot 1]$. Consider $hn, h'n' \in HN$ then notice that $hn \cdot h'n' = h n h' n' = h h' n'' n' \in HN$ for $n'' = h'^{-1} n h' \in N$ because $N$ is normal. Furthermore, $(h n)^{-1} = n^{-1} h^{-1} = h^{-1} n' \in HN$ where $n' = h n^{-1} h^{-1} \in N$ because $N$ is normal. Thus $NH \subset G$ is a subgroup. Now, for any $hn \in HN$ clearly, $[h \cdot 1] = [hn]$ in $HN / N$ so $\varphi$ is surjective. Furthermore, clearly $\ker{\varphi} = H \cap N$ so the result follows.
\end{proof}


\begin{cor}
If $|N| \cdot |H| = |G|$ then $NH = G \iff N \cap H = \{ e \}$.
\end{cor}

\begin{proof}
By the second isomorphism theorem,
\[ |H N| \cdot |H \cap N| = |H| \cdot |N| = |G| \]
\end{proof}

\subsection{Groups of Order $pq$}

Let $G$ be a group of order $n = pq$ for distinct primes $p,q$. Let $P, Q \subset G$ be the Sylow $p$ and $q$ subgroups. From the Sylow theorems,
\[ n_P = p k_P + 1 \divides q \quad \text{and} \quad n_Q = q k_Q + 1 \divides p \]
Without loss of generality, let $p < q$ then we must have $n_Q = 1$ so $Q \subset G$ is normal.
By the second isomorphism theorem,
\[ PQ/Q \cong P / P \cap Q \]
However, $P \cap Q$ is a subgroup of both $P$ and $Q$ which must be trivial by Lagrange since they have coprime orders. Thus, $P \cong PQ / Q$ so $|PQ| = |P| |Q| = p q$ and thus $PQ = G$. Therefore, we conclude that,
\[ G \cong Q \rtimes P \]
for some action $P \to \Aut{Q}$. Furthermore, since $P$ and $Q$ have prime orders they must be cyclic. Thus $P \to \Aut{Q}$ is a map $C_p \to \Aut{C_q} \cong C_{q-1}$. Such a map is given by sending a generator $x$ to $y^k$ for some generator $y \in C_{q-1}$ where $q - 1 \divides p k$.

\subsection{Exercises}

\begin{exercise}
Let $G$ be a finite group and $N \subset G$ normal such that $|N|$ and $[G : N]$ are coprime. Then prove that $H \subset G$ is the unique subgroup of order $|N|$.
\end{exercise}

Suppose that $H \subset G$ is a subgroup of order $|N|$. By the second isomorphism theorem,
\[ HN/N \cong H/H \cap N \]
Write $n = |G|$ as $n = ab$ where $a = |N|$ and $b = [G : N]$. Then, $HN/N$ must divide $b$ because $|HN|$ divides $ab$ but is divisible by $a$ (it contains $N$) so $|HN/N| = |NH|/a$ divides $b$. However, $H / H \cap N$ divides $a$ so both sides must be $1$ since $a$ and $b$ are coprime. Thus $H \cap N = N$ so $H = N$ since they have the same number of elements.

\begin{exercise}
Let $G$ be a group of order $30 = 2 \cdot 3 \cdot 5$ then,
\begin{enumerate}
\item show that $G$ has a subgroup of order $15$
\item show that every group of order $15$ is cyclic
\item show that $G$ is a semi-direct product $C_{15} \rtimes C_2$
\item exhibit three nonisomorphic (with proof) groups of order $30$.
\end{enumerate}
\end{exercise} 

Let $P, Q$ be the Sylow $3$ and $5$ subgroups. By the Sylow theorems,
\[ n_P = 3 k_P + 1 \divides 2 \cdot 5 \quad \text{and} \quad n_Q = 5 k_Q + 1 \divides 2 \cdot 3 \]
So $n_P = 1$ or $n_P = 10$ and $n_Q = 1$ or $n_Q = 6$. If niether one is normal then $n_P = 10$ and $n_Q = 6$ which would mean there are $10 \cdot 2$ elements of order $3$ (these groups are prime order so they must be disjoint except at $e$) and $6 \cdot 4$ elements of order $4$ but $10 \cdot 2 + 6 \cdot 4 + 1 = 45$ which is bigger than $30$ so one must be normal. Let $N$ be the normal one and $H$ the other subgroup. Then $N \cap H = \{ e \}$ because they have coprime order so by the second isomorphism theorem,
\[ NH / N \cong H / H \cap N = H \]
meaning that $|NH| = |N| \cdot |H| = 3 \cdot 5 = 15$ so $NH$ is a subgroup of order $15$.
\bigskip\\
This follows from Sylow arguments. Groups of order $15$ are type $pq$ which are all semi-direct products $C_q \rtimes C_p$ if $p < q$ and there are no nontrivial maps $C_3 \to \Aut{C_5} = C_4$ so this semi-direct product is direct. Thus $C_3 \times C_5 = C_{15}$ is the only group of order $15$.
\bigskip\\
Let $R$ be the Sylow $2$ subgroup and let $H$ be the cyclic subgroup of order $15$. Because $[G : H] = 2$ we know $H$ is normal so by the second isomorphism theorem,
\[ R H / H \cong R / R \cap H \]
but $R \cap H = \{ e \}$ because they have coprime orders so $|RH| = |H| \cdot |R| = 30$ and thus $RH = G$. Therefore, $G \cong H \rtimes R$ but we know that $H \cong C_{15}$ and $R \cong C_2$ so we find $G \cong C_{15} \rtimes C_2$.
\bigskip\\
Semi-direct products $C_{15} \rtimes C_2$ are (almost) classified by conjugation types of homomorphisms
\[ C_2 \to \Aut{C_{15}} \cong C_{2} \times C_4 \]
Since $C_{15}$ is abelian there are no inner automorphisms. Consider three maps, the trivial group $\varphi_0$ the map $\varphi_1 : C_2 \embed C_2 \times C_4$ into the first factor and the map $\varphi_2 : C_2 \embed C_2 \times C_4$ sending $C_2 \to C_4$ the unique subgroup of order $2$. Then let $G_i = C_{15} \rtimes_{\varphi_i} C_2$. Clearly $G_0$ is abelian but $G_1$ and $G_2$ are not so it suffices to show that $G_1$ and $G_2$ are not isomorphic. Just write down the table. I don't want to but we could also just consider $D_{15}$ and $C_5 \times S_3$. Indeed $Z(D_{15})$ is trivial but $Z(C_5 \times S_3)$ is not. 

\begin{exercise}
Let $G$ be a group of order $105 = 3 \cdot 5 \cdot 7$ and let $P,Q,R$ be the corresponding Sylow subgroups. Prove that,
\begin{enumerate}
\item one of $Q$ or $R$ is normal in $G$
\item $G$ has a cyclic subgroup of order $35$
\item both $Q$ and $R$ are normal in $G$
\item if $P$ is normal in $G$ then $G$ is cyclic 
\end{enumerate}
\end{exercise} 

By the Sylow theorems,
\[ n_Q = 5 k_Q + 1 \divides 3 \cdot 7 \quad \text{and} \quad n_R = 7 k_P + 1 \divides 3 \cdot 5 \]
then either $n_Q = 1$ or $n_Q = 21$ and $n_R = 1$ or $n_R = 15$. Because these subgroups have prime order the conjugates but be disjoint (except for $e$). Each $n_Q$ contains $4$ elements of order $5$ and thus if $n_Q = 21$ there are $4 \cdot 21$ elements of order $5$ and if $n_R = 15$ there must be $6 \cdot 15$ elements of order $7$. However, $4 \cdot 21 + 6 \cdot 15 + 1 = 175$ greater than the total number of elements so either $n_Q = 1$ or $n_R = 1$ proving that either $P$ or $R$ is normal.
\bigskip\\
Any group of order $35$ is cyclic by a Sylow argument, notice there are only trivial maps $C_5 \to \Aut{C_7} \cong C_6$. Therefore it suffices to find a subgroup of $G$ of order $35$. Consider $QR$. Since one is normal, call it $N$ and the other $H$, by the second isomorphism theorem,
\[ HN / N \cong H / H \cap N \]
but $H$ and $N$ have coprime orders so $H \cap N = \{ e \}$. Therefore $|HN| = |H| \cdot |N|$ so $|QR| = |Q| \cdot |R| = 5 \cdot 7 = 35$ so the subgroup $QR$ is a subgroup of order $35$.
\bigskip\\
Clearly $Q,R \subset QR$ and since $QR$ is cyclic any subgroup is also cyclic proving that both $Q$ and $R$ are cyclic.
\bigskip\\
Let $H$ be the cyclic subgroup of order $35$. Suppose that $P$ is normal in $G$. Then by the second isomorphism theorem,
\[ HP/P \cong H / H \cap P \]
However, $P$ and $H$ have coprime order so $H \cap P = \{ e \}$ and thus we find that $|HP| = |H| \cdot |P| = |G|$ so $HP = G$. Therefore, $G$ is a semi-direct product of $P$ and $H$ but $\Aut{P} \cong C_2$ and there is no nontrivial map $H \to C_2$ because $|H|$ is odd. Thus $G = P \times H$ and since $P$ and $H$ are cyclic of coprime orders we have that $G$ is also cyclic by the Chinese remainder theorem.

\begin{exercise}
Let $F$ be a field and $E / F$ an extension. Let $\alpha \in E$ be algebraic of odd degree over $F$. Then prove that,
\begin{enumerate}
\item $F(\alpha) = F(\alpha^2)$
\item the element $\alpha^n \in E$ has odd degree over $F$
\end{enumerate}
\end{exercise}

Assume that $\alpha \notin F(\alpha^2)$ then $1, \alpha$ is clearly a basis of $F(\alpha)$ over $F(\alpha^2)$ so $[F(\alpha) : F(\alpha^2)] = 2$ but $[F(\alpha) : F] = [F(\alpha) : F(\alpha^2)] [F(\alpha^2) : F]$ is even giving a contradiction so $\alpha \in F(\alpha^2)$. Furthermore, 
\[ [F(\alpha) : F] = [F(\alpha) : F(\alpha^n)] [F(\alpha^n) : F] \]
and thus $[F(\alpha^n) : F]$ is odd so the degree of $\alpha^n$ is odd.

\section{Analysis}

\begin{exercise}
Let $f : D^\circ \to \mathfrak{h}$ be a holomorphic function from the open unit disk to the upper half plane. Assume that $f(0) = i n$ then find a sharp bound on $|f'(0)|$.
\end{exercise}

Notice that $g = e^{if/s} : D^\circ \to \mathbb{C}$ is bounded by $1$. Then by Cauchy,
\[ g'(0) = \frac{1}{2\pi i} \oint_\gamma \frac{g(z)}{z^2} \d{z} = \int_0^1 \frac{g(r e^{2\pi i t})}{r e^{2 \pi i t}} \d{t} \]
Therefore,
\[ |g'(0)| \le \frac{1}{r} \]
so we can take the limit $r \to 1$ and get $|g'(0)| \le 1$. Then,
\[ g'(0) = \frac{i f'(0)}{s} e^{i f(0)/s} \]
so we find that,
\[ |f'(0)| \le |g'(0)| \, s \, e^{n/s} \] 
Now we minimize over $s$. Consider $m(s) = s \, e^{2/s}$ then $m'(s) = (1 - n / s) e^{n/s}$ so the minimum occurs at $s = n$ and thus we find that,
\[ |f'(0)| \le n e \]
Actually though, we can do better by using a better transform. Consider,
\[ g : \mathfrak{h} \to D^\circ \quad \text{via} \quad g(z) = \frac{z - i s}{z + i s} \]
Notice that,
\[ \left| \frac{z - i s}{z + i s} \right|^2 = \frac{x^2 + (y - s)^2}{x^2 + (y + s)^2} \le 1 \]
because $y > 0$ and $s > 0$.
Then $g \circ f$ is a self-map of the disk and thus is bounded by $1$.  Therefore, by the Cauchy integral formula,
\[ |(g \circ f)'(0)| \le 1 \]
however,
\[ (g \circ f)'(z) = \frac{2 i s f'(z)}{(f(z) + i s)^2} \]
Therefore, 
\[ |f'(0)| = \frac{1}{2s} \cdot (n + s)^2 \cdot |(g \circ f)'(0)| \] 
Now we minimize with respect to $s$. Consider $m(s) = \frac{(n + s)^2}{2 s}$ then
\[ m'(s) = \frac{n + s}{s} - \frac{(n + s)^2}{2s^2} = \frac{n + s}{2 s^2} \cdot \left( 2 s - (n + s) \right)  \]
and therefore $s = n$ so we find that,
\[ |f'(0)| \le 2 n \]
To show that this bound is sharp, consider,
\[ f(z) = i n \cdot \frac{1 + z}{1 - z} \]
It is easy to show that $\Im{f(z)} > 0$ and $f(0) = i n$. Furthermore,
\[ f'(0) = 2 i n \]

\newcommand{\R}{\mathbb{R}}

\begin{exercise}
Suppose we have Lebesgue integrable functions $f,g : \R \to \R$ then show that,
\[ \lim_{n \to \infty} || f + g_n ||_{1} = || f ||_{1} + || g ||_{1} \]
where $g_n(x) = g(x - n)$.
\end{exercise}

Consider the functions,
\[ F(x) = \int_{-\infty}^x |f(t)| \d{t} \quad \text{and} \quad G(x) = \int^{\infty}_x |g(t)| \d{t} \]
Then for any $\epsilon > 0$ we can find $x_1$ and $x_2$ such that $F(x_1) > ||f||_{1} - \epsilon$ and $G(x_2) > ||g||_{1} - \epsilon$ because the limits of each are $||f||_{1}$ and $||g||_{1}$ respectively. Notice that $F$ is increasing and $G$ is decreasing. Then choose $n$ large enough such that $x_1 < x_2 + n$. Then consider,
\[ ||f + g_n||_{1} = \int_{-\infty}^{\infty} |f(t) + g(t - n)|| \d{t} = \int_{-\infty}^{x_1} |f(t) + g(t - n)| \d{t} + \int_{x_1}^{x_2 + n} | f(t) + g(t - n)| \d{t} + \int_{x_2 + n}^\infty |f(t) + g(t - n) | \d{t} \]
Each term is nonnegative so we can throw away the middle term and use,
\[ || f + g_n ||_{1} \ge = \int_{-\infty}^{x_1} |f(t) + g(t - n)| \d{t}  + \int_{x_2 + n}^\infty |f(t) + g(t - n) | \d{t} \ge F(x_1) + G(x_2) > || f ||_{1} + || g ||_{1} - 2 \epsilon \]
proving that the limit converges,
\[ \lim_{n \to \infty}  ||f  + g_n ||_{1} = || f ||_{1} + || g_n ||_{1} \] since of course $|| f + g_n ||_{1} \le || f ||_{1} + || g ||_{1}$ using that $|| g_n ||_{1} = || g ||_{1}$.


\begin{exercise}
Suppose that $f_n \to f$ almost everywhere and $\int |f_n| \to \int |f|$. Then prove that $\int f_n \to f$.
\end{exercise}

Consider $g_n = |f_n| - |f_n - f|$ which are measurable and notice that,
\[ |g_n| = ||f_n| - |f_n - f|| \le |f_n - (f_n - f)| = |f| \]
and therefore are uniformly bounded by the integrable function $|f|$. Therefore by the dominated convergence theorem we find that,
\[ \int g_n \to \int |f| \]
since $g_n \to |f|$ almost eveywhere. However, 
\[ \int |f_n - f| = \int |f_n| - g_n = \int |f_n| - \int g_n \to \int |f| - \int |f| = 0 \]
because we assumed that $\int |f_n| \to \int |f|$. Therefore, 
\[ \lim_{n \to \infty} \left| \int f - \int f_n \right| \le \lim_{n \to \infty} \int |f_n - f| = 0 \]
meaning that $\int f_n \to \int f$.

\begin{exercise}
Let $f : \R \to \R$ be a continous function which is zero outside of a finite interval. Then show that,
\[ g(z) = \int_{-\infty}^{\infty} f(t) e^{-i z t} \d{t} \]
is entire.
\end{exercise}

Because $f$ is continuous and supported on a compact set it is bounded, say by $M$. We need to consider,
\[ g'(z) = \lim_{h \to 0} \frac{g(z + h) - g(z)}{h} = \lim_{h \to 0} \int_{-\infty}^{\infty} f(t) e^{- i z t} \left( \frac{e^{-i h t} - 1}{h} \right) \d{t} \]
Now, for all $z$ and $t$ the following series converges absolutly,
\[ \left( \frac{e^{-i h t} - 1}{h} \right) \d{t} = -i t \sum_{n = 9}^{\infty} \frac{(-ih t)^n}{(n+1)!} \]
On any compact interval for $t$ this power series also converges uniformly by the $M$-test. Therefore, because $f$ is supported on such a compact interval as is bounded, by the $M$-test,
\[ \sum_{n = 0}^\infty f(t) e^{- i z t} \frac{(- i h t)^n}{(n + 1)!} \]
is also uniformly convergent on that interval and each term is a continuous function of $t$ with compact support and thus integrable meaning that,
\[ g'(z) = \lim_{h \to 0} \sum_{n = 0}^{\infty} \left( \int_{-\infty}^{\infty} f(t) e^{- i z t} \frac{(- i t)^{n+1}}{(n + 1)!} \d{t} \right) h^n \]
Since we know this power series converges for any fixed $h$ this implies that its radius of convergence must be infinite (it would have been easier to just use the series expansion for $e^{-i z t}$ whoops) and in particular it is a continuous function everywhere the the limit exists,
\[ g'(z) = \int_{-\infty}^{\infty} f(t) (-i t) e^{-i z t} \d{t} \]

\begin{exercise}
Is every complete bounded metric space compact?
\end{exercise}

No, for example take the closed unit ball in $\ell_2$. Explicitly, $B = \{ (a_n) \mid \sum_{i = 1}^\infty a_i^2 = 1 \}$. Since $B$ is a closed subset of a complete metric space is it is complete and is bounded by construction. However, $B$ is not compact. To see this, consider the open cover $\{ U_i \}$ where $U_i$ is the open subset where $a_i \neq 0$. Then for any finite subset the union is contained in $\bigcup_{i = 1}^k U_i$ which does not contain,
\[ a_i = 
\begin{cases}
0 & i \le k 
\\
\tfrac{1}{2^{i - k}}
\end{cases} \]
and thus there is no finite subcover so $B$ is not compact.

\begin{exercise}
Let $(X, \F, \mu)$ be a finite measure space. Let $\{ f_n \} \subset \L^1(X, \mu)$ be a sequence of functions and $f \in \L^1(X, \mu)$ such that $f_n \to f$ pointwise a.e. Prove theat for every $\epsilon > 0$ there exists $M > 0$ and a set $E \subset X$, such that $\mu(E) \le \epsilon$ and $|f_n(x)| \le M$ for all $x \in X \setminus E$ and all $n \in \N$.
\end{exercise}

Let $N \subset X$ be the set on which $f_n(x)$ does not converge to $f(x)$. Then by assumption $\mu(N) = 0$. Now, $f_n \to f$ pointwise on $X \setminus N$ so by Egorov's theorem, for any $\epsilon > 0$ there exists some measurable $E \subset X \setminus N$ such that $f_n \to f$ uniformly on $X \setminus (E \cup N)$ and $\mu(E) < \tfrac{1}{2} \epsilon$. 
By unifom convergence, on $X \setminus (N \cup E)$ we have that for $n > N$,
\[ | f_n(x) - f(x) | \le 1 \] 
Furthermore, $f_n, f \in L^1(X, \mu)$ meaning that,
\[ \int_0^{\infty} \mu(\{ x \in X \mid |f(x)| > t \}) \, \d{t} < \infty \]
so the integrands must tend to zero. Therefore, there is some $M > 0$ such that the sets,
\[ E_i = \{ x \in X \mid |f_i(x)| > (M - 1) \} \quad \text{and} \quad E' = \{ x \in X \mid |f(x)| > (M - 1) \} \]
for $i = 1,\dots, N$ have measure less than $\tfrac{1}{2 (N+1)} \epsilon$.  
Therefore on $X \setminus (N \cup E_1 \cup \cdots \cup E_N \cup E')$ we have,
\[ | f_n(x) | \le M \]
because if $n \le N$ then this follows since $x \notin E_n$ and if $n > N$ then,
\[ | f_n(x) | \le |f_n(x) - f(x)| + |f(x)| \le 1 + (M - 1) \]
because $x \notin (E \cup N)$ and $x \notin E'$.

\end{document}
