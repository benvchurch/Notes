\documentclass[12pt]{article}
\usepackage[english]{babel}
\usepackage[utf8]{inputenc}
\usepackage[english]{babel}
\usepackage[a4paper, total={7.25in, 9.5in}]{geometry}
\usepackage{tikz-feynman}
\tikzfeynmanset{compat=1.0.0} 
\usepackage{subcaption}
\usepackage{float}
\floatplacement{figure}{H}
\usepackage{simpler-wick}
\usepackage{mathrsfs}  
\usepackage{dsfont}
\usepackage{relsize}
\DeclareMathAlphabet{\mathdutchcal}{U}{dutchcal}{m}{n}


\newcommand{\field}{\hat{\Phi}}
\newcommand{\dfield}{\hat{\Phi}^\dagger}
 
\usepackage{amsthm, amssymb, amsmath, centernot}
\usepackage{slashed}
\newcommand{\notimplies}{%
  \mathrel{{\ooalign{\hidewidth$\not\phantom{=}$\hidewidth\cr$\implies$}}}}
 
\renewcommand\qedsymbol{$\square$}
\newcommand{\cont}{$\boxtimes$}
\newcommand{\divides}{\mid}
\newcommand{\ndivides}{\centernot \mid}

\newcommand{\Integers}{\mathbb{Z}}
\newcommand{\Natural}{\mathbb{N}}
\newcommand{\Complex}{\mathbb{C}}
\newcommand{\Zplus}{\mathbb{Z}^{+}}
\newcommand{\Primes}{\mathbb{P}}
\newcommand{\Q}{\mathbb{Q}}
\newcommand{\R}{\mathbb{R}}
\newcommand{\ball}[2]{B_{#1} \! \left(#2 \right)}
\newcommand{\Rplus}{\mathbb{R}^+}
\renewcommand{\Re}[1]{\mathrm{Re}\left[ #1 \right]}
\renewcommand{\Im}[1]{\mathrm{Im}\left[ #1 \right]}
\newcommand{\Op}{\mathcal{O}}

\newcommand{\invI}[2]{#1^{-1} \left( #2 \right)}
\newcommand{\End}[1]{\text{End}\left( A \right)}
\newcommand{\legsym}[2]{\left(\frac{#1}{#2} \right)}
\renewcommand{\mod}[3]{\: #1 \equiv #2 \: \mathrm{mod} \: #3 \:}
\newcommand{\nmod}[3]{\: #1 \centernot \equiv #2 \: mod \: #3 \:}
\newcommand{\ndiv}{\hspace{-4pt}\not \divides \hspace{2pt}}
\newcommand{\finfield}[1]{\mathbb{F}_{#1}}
\newcommand{\finunits}[1]{\mathbb{F}_{#1}^{\times}}
\newcommand{\ord}[1]{\mathrm{ord}\! \left(#1 \right)}
\newcommand{\quadfield}[1]{\Q \small(\sqrt{#1} \small)}
\newcommand{\vspan}[1]{\mathrm{span}\! \left\{#1 \right\}}
\newcommand{\galgroup}[1]{Gal \small(#1 \small)}
\newcommand{\bra}[1]{\left| #1 \right>}
\newcommand{\Oa}{O_\alpha}
\newcommand{\Od}{O_\alpha^{\dagger}}
\newcommand{\Oap}{O_{\alpha '}}
\newcommand{\Odp}{O_{\alpha '}^{\dagger}}
\newcommand{\im}[1]{\mathrm{im} \: #1}
\renewcommand{\ker}[1]{\mathrm{ker} \: #1}
\newcommand{\ket}[1]{\left| #1 \right>}
\renewcommand{\bra}[1]{\left< #1 \right|}
\newcommand{\inner}[2]{\left< #1 | #2 \right>}
\newcommand{\expect}[2]{\left< #1 \right| #2 \left| #1 \right>}
\renewcommand{\d}[1]{ \mathrm{d}#1 \:}
\newcommand{\dn}[2]{ \mathrm{d}^{#1} #2 \:}
\newcommand{\deriv}[2]{\frac{\d{#1}}{\d{#2}}}
\newcommand{\nderiv}[3]{\frac{\dn{#1}{#2}}{\d{#3^{#1}}}}
\newcommand{\pderiv}[2]{\frac{\partial{#1}}{\partial{#2}}}
\newcommand{\fderiv}[2]{\frac{\delta #1}{\delta #2}}
\newcommand{\parsq}[2]{\frac{\partial^2{#1}}{\partial{#2}^2}}
\newcommand{\topo}{\mathcal{T}}
\newcommand{\base}{\mathcal{B}}
\renewcommand{\bf}[1]{\mathbf{#1}}
\renewcommand{\a}{\hat{a}}
\newcommand{\adag}{\hat{a}^\dagger}
\renewcommand{\b}{\hat{b}}
\newcommand{\bdag}{\hat{b}^\dagger}
\renewcommand{\c}{\hat{c}}
\newcommand{\cdag}{\hat{c}^\dagger}
\newcommand{\hamilt}{\hat{H}}
\renewcommand{\L}{\hat{L}}
\newcommand{\Lz}{\hat{L}_z}
\newcommand{\Lsquared}{\hat{L}^2}
\renewcommand{\S}{\hat{S}}
\renewcommand{\empty}{\varnothing}
\newcommand{\J}{\hat{J}}
\newcommand{\lagrange}{\mathcal{L}}
\newcommand{\dfourx}{\mathrm{d}^4x}
\newcommand{\meson}{\phi}
\newcommand{\dpsi}{\psi^\dagger}
\newcommand{\ipic}{\mathrm{int}}
\newcommand{\tr}[1]{\mathrm{tr} \left( #1 \right)}
\newcommand{\C}{\mathbb{C}}
\newcommand{\CP}[1]{\mathbb{CP}^{#1}}
\newcommand{\Vol}[1]{\mathrm{Vol}\left(#1\right)}

\newcommand{\Tr}[1]{\mathrm{Tr}\left( #1 \right)}
\newcommand{\Charge}{\hat{\mathbf{C}}}
\newcommand{\Parity}{\hat{\mathbf{P}}}
\newcommand{\Time}{\hat{\mathbf{T}}}
\newcommand{\Torder}[1]{\mathbf{T}\left[ #1 \right]}
\newcommand{\Norder}[1]{\mathbf{N}\left[ #1 \right]}
\newcommand{\Znorm}{\mathcal{Z}}
\newcommand{\EV}[1]{\left< #1 \right>}
\newcommand{\interact}{\mathrm{int}}
\newcommand{\covD}{\mathcal{D}}
\newcommand{\conj}[1]{\overline{#1}}

\newcommand{\SO}[2]{\mathrm{SO}(#1, #2)}
\newcommand{\SU}[2]{\mathrm{SU}(#1, #2)}

\newcommand{\anticom}[2]{\left\{ #1 , #2 \right\}}


\newcommand{\pathd}[1]{\! \mathdutchcal{D} #1 \:}

\renewcommand{\theenumi}{(\alph{enumi})}


\renewcommand{\theenumi}{(\alph{enumi})}

\newcommand{\atitle}[1]{\title{% 
	\large \textbf{Physics GR8040 General Relativity
	\\ Assignment \# #1} \vspace{-2ex}}
\author{Benjamin Church }
\maketitle}

\theoremstyle{definition}
\newtheorem{theorem}{Theorem}[section]
\newtheorem{definition}{definition}[section]
\newtheorem{lemma}[theorem]{Lemma}
\newtheorem{proposition}[theorem]{Proposition}
\newtheorem{corollary}[theorem]{Corollary}
\newtheorem{example}[theorem]{Example}
\newtheorem{remark}[theorem]{Remark}
\begin{document}


\atitle{6}

\section*{1.}

If we assume that the star collapses in free fall then we may apply the results for a beacon under free fall in Schwarzschild geometry (Birkhoff's theorem implies that the geometry outside the surface is Schwarzschild) from the previous problem set. We found that if an object free falling from radius $r_*$ to $r$ emits a photon at radius $r$ with energy $E$ then that photon is measured by a static observer at $\infty$ with energy,
\[  E_{\infty} = E \left(1 - \frac{r_s}{r} \right) \left(1 - \frac{r_s}{r_*} \right)^{-\frac{1}{2}} \left( 1  + \sqrt{\frac{r_s}{r} \cdot \frac{r_* - r}{r_* - r_s}} \right)^{-1} \]
Since the energy of all emitted photons is scaled by the same factor, a black body spectrum with temperature $T_0$ is observed to be a black body spectrum with temperature,
\[ T = T_0 \left(1 - \frac{r_s}{r} \right) \left(1 - \frac{r_s}{r_*} \right)^{-\frac{1}{2}} \left( 1  + \sqrt{\frac{r_s}{r} \cdot \frac{r_* - r}{r_* - r_s}} \right)^{-1} \]
In the limit $r \to r_s$ consider $r = r_s (1 + \epsilon)$ in which case we have,
\[ T = T_0 \left( 1 - \frac{r_s}{r_*} \right)^{-\frac{1}{2}} \left(  \tfrac{1}{2} \epsilon  + O(\epsilon^2) \right) \]
Furthermore, under free fall from $r_*$ we have,
\[ \frac{\dot{r}}{c} = -\left(1 - \frac{r_s}{r} \right) \sqrt{\frac{r_s}{r}\cdot \frac{r_* - r}{r_* - r_s}} \]
Therefore,
\[ t = \frac{1}{c} \int_{r}^{r_*} \frac{\d{r}}{\left(1 - \frac{r_s}{r} \right) \sqrt{\frac{r_s}{r}\cdot \frac{r_* - r}{r_* - r_s}} } \]
We cannot do this integral exactly. However, we can investigate it's asymptotics. Near the horizon,
\[ \dot{r} = - c \epsilon + O(\epsilon^2) \] 
and thus,
\[ \dot{\epsilon} = - \frac{c}{r_s} \epsilon + O(\epsilon^2) \]
meaning that,
\[ \epsilon = \epsilon_0 e^{-t (c/r_s)} \]
Furthermore, we must consider the light travel time from the emission radius $r$ to the observer radius $r_0$. For radially outgoing light-like geodesics we have,
\[ \frac{\dot{r}}{c} = \left( 1 - \frac{r_s}{r} \right) \]
Therefore,
\[ t_{\text{lt}} = \frac{1}{c} \int_{r}^{r_o} \frac{\d{r}}{1 - \frac{r_s}{r}} = \frac{r_o - r}{c} + \frac{r_s}{c} \log{\left( \frac{r_o - r_s}{r - r_s} \right)} \]
We write this as,
\[ t_{\text{lt}} = \frac{r_o - r_s}{c} - \frac{r_s \epsilon}{c} + \frac{r_s}{c} \log{\left( \frac{r_o}{r_s} \right)} - \frac{r_s}{c} \log{\epsilon} \]
Therefore, in the limit of small epsilon we find,
\[  t_{\text{lt}} = C - \frac{r_s}{c} \log{\epsilon} \]
Thus the time at which a photon emitted from the surface when it is at radius $r_s(1 + \epsilon)$ is observed is,
\[ t_{\text{obs}} = t + t_{lt} = \frac{r_o - r_s}{c} + \frac{r_s}{c} \log{\epsilon_0} - \frac{2 r_s}{c} \log{\epsilon} = \frac{r_s}{c} K - \frac{2 r_s}{c} \log{\epsilon} \]
This implies that, asymptotically, 
\[ \frac{T}{T_0} \propto e^{- t_{\text{obs}} / (2 r_s / c)} \]
so the apparent temperature decreases exponentially with time constant,
\[ \tau = \frac{2 r_s}{c} = \frac{4 M G}{c^3} \]

\section*{2.}

Consider the Kerr metric,
\begin{align*}
\d{s^2} & = - \left( 1 - \frac{r r_g}{\rho^2} \right) c^2 \d{t^2} - \frac{a r_g r}{\rho^2} \sin^2{\theta} (2 c \d{t} \d{\phi}) + \frac{\rho^2}{\Delta} \d{r^2} + \rho^2 \d{\theta^2}
\\
& \quad \quad + \frac{\sin^2{\theta}}{\rho^2} \left[ \left( r^2 + a^2 \right)^2 - a^2 \Delta \sin^2{\theta} \right] \d{\phi^2} 
\end{align*}
where,
\[ a = \frac{J}{Mc} \quad \quad \rho^2 = r^2 + a^2 \cos^2{\theta} \quad \quad \Delta = r^2 + a^2 - r_g r \quad \quad r_g = \frac{2MG}{c^2} \]
Now restrict to the equatorial plane $\theta = \pi / 2$ which reduces the metric to the form,
\begin{align*}
\d{s^2} & = - \left( 1 - \frac{r_g}{r} \right) c^2 \d{t^2} - \frac{a r_g}{r}  (2  c \d{t} \d{\phi}) + \frac{r^2}{r^2 - r_g r + a^2} \d{r^2} 
 +  \left[r^2 + a^2 + a^2 r_g / r \right] \d{\phi^2} 
\end{align*}
This metric has a azimuthal Killing vector field due to the rotational symmetry along the rotation axis of the black hole. This gives rise to a conserved angular momentum,
\begin{align*}
\ell & = u_\phi = g_{\phi \alpha} u^\alpha = g_{\phi t} u^t + g_{\phi \phi} u^\phi
\\
& = - \frac{a r_s}{r} u^t + (r^2 + a^2 + a^2 r_g / r) u^\phi 
\end{align*} 
Furthermore, there is a time-like killing vector field because the metric is time-invariant giving rise to a conserved energy,
\begin{align*}
- \epsilon & = u_t = g_{t \alpha} u^\alpha = g_{tt} u^t + g_{t \phi} u^\phi
\\
& = - \left( 1 - \frac{r_g}{r} \right) u^t - \frac{ar_s}{r} u^\phi
\end{align*}
We may now solve for the four velocities in terms of these conserved quantities,
\begin{align*}
-\left( g_{tt} g_{\phi \phi} - g_{\phi t} g_{t \phi} \right) u^t & = \left( g_{t \phi} \ell + g_{\phi \phi} \epsilon \right)
\\
\left( g_{tt} g_{\phi \phi} - g_{\phi t} g_{t \phi} \right) u^\phi  & = \left( g_{t t} \ell + g_{t \phi} \epsilon \right)
\end{align*}
Now let,
\[ D =  - \left( g_{tt} g_{\phi \phi} - g_{\phi t} g_{t \phi} \right) =  \left( 1 - \frac{r_g}{r} \right) [r^2 + a^2 + a^2 r_g / r ] + \frac{a^2 r_g^2}{r^2} = r^2 \left( 1 - \frac{r_g}{r} \right) + a^2 \]
Next, we use the normalization of the four-velocity of a particle $u_\alpha u^\alpha = - c^2$,
\begin{align*}
g_{tt} (u^t)^2 + 2 g_{t \phi} u^t u^\phi + g_{rr} (u^r)^2 + g_{\phi \phi} (u^\phi)^2 & = - c^2 
\\
g_{tt} (g_{t \phi} \ell + g_{\phi \phi} \epsilon)^2 - 2 g_{t \phi} (g_{t \phi} \ell + g_{\phi \phi} \epsilon)(g_{tt} \ell + g_{t \phi} \epsilon) + D^2 g_{rr} (u^r)^2 + g_{\phi \phi} (g_{tt} \ell + g_{t \phi} \epsilon )^2 & = - D^2 c^2
\end{align*}
Now consider a particle on a parabolic trajectory from infinity with zero angular momentum. At large radii its constants of motion become,
\[ \epsilon = u^t = c \quad \quad \quad \ell = r^2 u^\phi = 0 \]
Then, plugging in, we find that,
\[ \left( \frac{u^r}{c} \right)^2 = [D^2 g_{rr} ]^{-1} \left[ - g_{tt} g_{\phi \phi}^2 + g_{t \phi}^2 g_{\phi \phi} \right] - g_{rr}^{-1} \]
Which simplifies to,
\[ \left( \frac{u^r}{c} \right)^2 = \frac{(r^2 + a^2) r_g}{r^3} \] 
Plugging in the expression for the metric in the above formulae we find the components of the four velocity,
\begin{align*}
u^t & = c \frac{r^2 + a^2(1 + r_g / r)}{a^2 + r(r - r_g)}
\\
u^r & = -c \sqrt{ \frac{(r^2 + a^2)r_g}{r^3}}
\\
u^\theta & = 0
\\
u^\phi & = c \frac{a r_g}{r (a^2 + r(r - r_g))}
\end{align*}
To find how many turns the particle makes around the black hole we consider,
\begin{align*}
\deriv{\phi}{r} = \frac{u^\phi}{u^r} = - \frac{1}{a^2 + r(r - r_g)} \sqrt{\frac{a^2 r r_g}{r^2 + a^2}} 
\end{align*}
Thus,
\[ \Delta \phi = \int_{r_H}^\infty \deriv{\phi}{r} \d{r} =  \int_{\infty}^{r_H} \frac{\d{r}}{a^2 + r(r - r_g)} \sqrt{\frac{a^2 r r_g}{r^2 + a^2}}  \]
where $r_H = r_g$ is the radius of the Horizon (at $\theta = \pi / 2$) which is the largest solution to,
\[ r^2 - r_g r + a^2 = 0 \]
where $g_{rr}$ blows up. This occurs at,
\[ r_H = \tfrac{1}{2} (r_g + \sqrt{r_g^2 - 4 a^2}) \]
Near $r_H$ the integrand goes as $1/(r - r_H)$ and thus the integral diverges. Therefore the particle makes an infinite number of turns around the black hole as it approaches the horizon. 

\section*{3.}

Consider the Friedmann metric, 
\[ \d{s^2} = - c^2 \d{t^2} + a^2 \left[ \frac{\d{r^2}}{1 - k r^2} + r^2 \d{\Omega^2} \right] \]
which has Christoffel symbols,
\begin{align*}
\Gamma^r_{rr} & = \frac{k r}{1 -k r^2} 
\\
 \Gamma^t_{rr} & = \frac{a \dot{a}}{1 -k r^2} \quad \quad \Gamma^t_{\theta \theta} = \frac{k r}{1 -k r^2}  \quad \quad \Gamma^t_{\phi \phi} = a \dot{a} r^2 \sin^2{\theta} 
\\
\Gamma^r_{tr} & = \Gamma^r_{rt} = \Gamma^\theta_{t \theta} = \Gamma^\theta_{\theta t} = \Gamma^\phi_{t \phi} = \Gamma^\phi_{\phi t} = \frac{\dot{a}}{a}
\\
\Gamma^r_{\theta \theta} & = - r (1 - k r^2) \quad \quad \Gamma^r_{\phi \phi} = - r(1 - k r^2) \sin^2{\theta} 
\\
\Gamma^\theta_{r \theta} & = \Gamma^\theta_{\theta r} = \Gamma^\phi_{r \phi} = \Gamma^\phi_{\phi r} = \frac{1}{r}  
\\
\Gamma^\theta_{\phi \phi} & = - \sin{\theta} \cos{\theta} \quad \quad \Gamma^\phi_{\theta \phi} = \Gamma^{\phi}_{\phi \theta} = \cot{\theta} 
\end{align*}
Consider the Riemann tensor,
\[ R^\rho_{\: \sigma \mu \nu} = \partial_\mu \Gamma^{\rho}_{\nu \sigma} - \partial_\nu \Gamma^\rho_{\mu \sigma} + \Gamma^\rho_{\mu \lambda} \Gamma^\lambda_{\nu \sigma} - \Gamma^\rho_{\nu \lambda} \Gamma^\lambda_{\mu \sigma} \]
Then we compute (note the index $\alpha$ is here not summed over),
\[ R^\alpha_\alpha = g^{\alpha \beta} \left[ \partial_\mu \Gamma^{\mu}_{\beta \alpha} - \partial_\beta \Gamma^\mu_{\mu \alpha} + \Gamma^\mu_{\mu \lambda} \Gamma^\lambda_{\beta \alpha} - \Gamma^\mu_{\beta \lambda} \Gamma^\lambda_{\mu \alpha} \right] \]
Now consider,
\begin{align*}
R^r_r & = g^{rr} \left[ \partial_\mu \Gamma^{\mu}_{rr} - \partial_r \Gamma^\mu_{\mu r} + \Gamma^\mu_{\mu \lambda} \Gamma^\lambda_{rr} - \Gamma^\mu_{r \lambda} \Gamma^\lambda_{\mu r} \right] 
\\
& = g^{rr} \left[ \partial_t \Gamma^{t}_{rr} + \partial_r \Gamma^r_{rr} - \partial_r [\Gamma^r_{r r} + \Gamma^\theta_{\theta r} + \Gamma^\phi_{\phi r}]  + [\Gamma^r_{rt} + \Gamma^\theta_{\theta t} + \Gamma^\phi_{\phi t}]  \Gamma^t_{rr} + [\Gamma^r_{rr} + \Gamma^\theta_{\theta r} + \Gamma^\phi_{\phi r}]  \Gamma^r_{rr}  \right] 
\\
& - g^{rr} \left[ \Gamma^r_{r r} \Gamma^r_{r r} + \Gamma^\theta_{r \theta} \Gamma^\theta_{\theta r} + \Gamma^\phi_{r \phi} \Gamma^\phi_{\phi r} - 2 \Gamma^t_{rr} \Gamma^r_{tr} \right]
\\
& = \frac{1}{a^2} \left[ \dot{a}^2 + a \ddot{a} + \frac{2}{r^2(1 - k r^2)} + 3 \dot{a}^2 + 2k  - \frac{2}{r^2(1 - k r^2)} - 2 \dot{a}^2 \right]
\\
& = \frac{a \ddot{a} + 2 \dot{a}^2 + 2k}{a^2} 
\end{align*}
I claim that the symmetry of the space-like hypersurface of fixed $t$ implies that $R^r_r = R^\theta_\theta = R^\phi_\phi$. We may choose local coordinates which diagonalize the Ricci tensor and then the symmetry of the spacetime ensures that these eigenvalues are equal. However, a $(1,1)$-tensor transforms like a matrix representing a linear transformation under a change of basis. In particular, a multiple of the identity matrix is sent to a multiple of the identity matrix. 

\section*{4.}
Consider a Friedmann model,
\[ \left( \frac{\dot{a}}{a} \right)^2 = \frac{8 \pi G}{3} \rho - \frac{kc^2}{a^2}  \quad \quad \quad \frac{\ddot{a}}{a} = - \frac{4 \pi G}{3} (\rho + 3 c^{-2} p) \]
where we write the Hubble parameter $H = \frac{\dot{a}}{a}$ and the deceleration parameter $q = - \frac{a \ddot{a}}{\dot{a}^2}$. Therefore, via the Friedmann equations,
\[ \left( \frac{\dot{a}}{a} \right)^2 q = - \frac{\ddot{a}}{a} \]
Therefore,
\[ H^2 q = \frac{4 \pi G}{3} (\rho + 3 c^{-2} p) \]
However,
\[ \frac{8 \pi G}{3} \rho = H^2 + \frac{kc^2}{a^2} \]
and thus,
\[ 2 H^2 q = H^2 + \frac{kc^2}{a^2} + \frac{8 \pi G}{c^2} p \]
Rearranging, we find that,
\[ \frac{8 \pi G}{c^2} p = H^2 (2 q - 1) - \frac{kc^2}{a^2} \]


\subsection*{(a)}

In the case of dust, $p = 0$, this equation gives,
\[ H^2 (2q - 1) = \frac{kc^2}{a^2} \]
Therefore, from the first Friedmann equation, we have,
\[ H^2 = \frac{8 \pi G}{3} \rho - H^2 (2 q - 1) \]
meaning that,
\[ \rho = 2q \frac{3 H^2}{8 \pi G} = 2 q \rho_{\text{crit}} \]
where,
\[ \rho_{\text{crit}} = \frac{3 H^2}{8 \pi G} \]

\subsection*{(b)}

In the case of radiation, $p = \rho c^2 / 3$, this equation gives,
\[ \frac{8 \pi G}{3} \rho = H^2 (2q - 1) - \frac{kc^2}{a^2} \]
Therefore, from the first Friedmann equation, we have,
\[ H^2 = \frac{16 \pi G}{3} \rho - H^2 (2 q - 1) \]
meaning that,
\[ \rho = 2q \frac{3 H^2}{16 \pi G} = q \rho_{\text{crit}} \]


\section*{5.}

In a dust-dominated universe (with $p = 0$) the first Friedmann equation gives,
\[ \left( \frac{\dot{a}}{a} \right)^2 = \frac{8 \pi G}{3} \rho  - \frac{kc^2}{a^2}  \]
Furthermore, the equation of state is $w = 0$ and thus, from the continuity equation $\nabla_\mu T^{\mu \nu} = 0$ we have,
\[ \rho = \rho_0 \left( \frac{a_0}{a} \right)^3 \]
Using the above expression with $p = 0$ we find at $t_0$,
\[ H^2_0 (2 q_0 - 1) = \frac{k c^2}{a_0^2} \]
Therefore,
\[ \left( \frac{\dot{a}}{a} \right)^2 = H_0^2 \Omega \left( \frac{a_0}{a} \right)^3 - H_0^2( 2 q_0 - 1) \left( \frac{a_0}{a} \right)^2  \]
where,
\[ \Omega = \frac{8 \pi G \rho_0}{3 H_0^2} \]
Furthermore,
\[ H^2_0 = H_0^2 \Omega - \frac{kc^2}{a_0^2} \]
and therefore,
\[ \frac{k c^2}{a_0^2} = H_0^2(\Omega - 1) \]
which implies that,
\[ \Omega = 2 q_0 \]
Finally,
\[ \left( \frac{\dot{a}}{a} \right)^2 = H_0^2 (2 q_0) \left( \frac{a_0}{a} \right)^3 - H_0^2( 2 q_0 - 1) \left( \frac{a_0}{a} \right)^2 \]
We assume a closed universe i.e. $k = 1$ meaning that,
\[ a_0^2 H_0^2 (2 q_0 - 1) = c^2 \]
so we may simplify the equation to, 
\[ \left( \frac{\dot{a}}{a} \right)^2 = H_0^2 (2 q_0) \left( \frac{a_0}{a} \right)^3 - \left( \frac{c}{a} \right)^2 \]
Consider the variable $\d{\eta} = c \d{t} / a$ which gives,
\[ a' = \deriv{a}{\eta} = \frac{a \dot{a}}{c} \]
Therefore,
\begin{align*}
\eta = \int_0^a \deriv{\eta}{a} \d{a} = \int_0^a \frac{c}{a \dot{a}} \d{a} =  \int_0^a \frac{c \d{a}}{\sqrt{2 q_0 a_0^3 H_0^2 a - c^2 a^2}} 
\end{align*}
Now we define the variable, $a = 2 q_0 a_0^3 H_0^2 u / c^2$ and in which the integral becomes,
\[ \eta = \int_0^{u} \frac{\d{u}}{\sqrt{u - u^2}} = \int_0^u \frac{\d{u}}{\sqrt{\tfrac{1}{4} - (\tfrac{1}{2} - u)^2}} = \int^{u-1/2}_{-1/2} \frac{\d{u}}{\sqrt{\tfrac{1}{4} - u^2}} = \int^{2u - 1}_{-1} \frac{\d{u}}{\sqrt{1 - u^2}} = \cos^{-1}(2u - 1) - \pi \]
Plugging in we find,
\[ 2 u - 1 = - \cos{\eta} \]
and therefore, 
\[ a = \frac{q_0 a_0^3 H_0^2}{c^2} (1 - \cos{\eta}) \]
Furthermore, $\d{t} = a \d{\eta} / c$ and thus,
\[ t = \int_0^\eta a \d{\eta} / c = \int_0^\eta \frac{2 q_0 a_0^3 H_0^2}{c^3} (1 - \cos{\eta}) \d{\eta} = \frac{q_0 a_0^3 H_0^2}{c^3} (\eta - \sin{\eta}) \]
At the current time $a = a_0$ and thus,
\[ \eta_0 = \cos^{-1}(1 - 2 u_0) \]
where $u_0 = c^2 / (2 q_0 a_0^2 H_0^2)$. Therefore,
\[ t_0 =  \frac{q_0 a_0^3 H_0^2}{c^3} \left[ \cos^{-1}(1 - 2 u_0) - \sqrt{1 - (1 - 2 u_0)^2} \right] \]
Finally, we know that,
\[ 2 q_0 - 1 = \frac{c^2}{a_0^2 H_0^2} \]
meaning that,
\[ u_0 = \frac{2 q_0 - 1}{2 q_0} \]
which means that,
\[ 1 - 2 u_0 = \frac{1}{q_0} - 1 \]
and thus, finally,
\[ t_0 = \frac{q_0}{H_0(2q_0 - 1)^{\frac{3}{2}}} \left[ \cos^{-1}(1/q_0 - 1) - 1/q_0 \sqrt{2 q_0 - 1} \right] \]

\section*{6.}

Consider a massive particle in a Friedmann universe which moves radially out from the origin in the metric,
\[ \d{s^2} = - c^2 \d{t^2} + a(t)^2 \left( \frac{\d{r^2}}{1 - k r^2} + r^2 \d{\Omega^2} \right) \]
We have the geodesic equation,
\[ \deriv{u^\mu}{\tau} + \Gamma^\mu_{\alpha \beta} u^\alpha u^\beta = 0 \]
and the normalization condition,
\[ u^\alpha u_\alpha = - c^2 \]
on the four-velocity $u^\alpha$ of the particle. In the current case we have restricted to radial motion i.e. $u^\alpha = (u^t, u^r, 0, 0)$. First consider the radial component,
\[ \deriv{u^r}{\tau} + \Gamma^r_{tt} u^t u^t + 2\Gamma^r_{tr} u^t u^r + \Gamma^r_{rr} u^r u^r = 0 \]
In a Friedmann universe we have computed,
\begin{align*}
\Gamma^r_{tt} & = 0
\\
\Gamma^r_{tr} & = \frac{\dot{a}}{a}
\\
\Gamma^r_{rr} & = \tfrac{1}{2} \partial_r \log{g_{rr}} = \frac{k r}{1 - kr^2} 
\end{align*}
Therefore,
\[ \deriv{u^r}{\tau} + 2 \frac{\dot{a}}{a} u^t u^r + [ \tfrac{1}{2} \partial_r \log{g_{rr}} ] (u^r)^2 = 0 \]
Now consider,
\[ \deriv{\sqrt{g_{rr}} u^r}{\tau} = \left[ \deriv{}{\tau} \tfrac{1}{2} \log{g_{rr}} \right] \sqrt{g_{rr}} u^r + \sqrt{g_{rr}} \deriv{u^r}{\tau} \]
however,
\[ \deriv{}{\tau} \log{g_{rr}} = u^t \partial_t \log{g_{rr}} + u^r \partial_r \log{g_{rr}} = 2 u^t \frac{\dot{a}}{a} + u^r \partial_r \log{g_{rr}} \] 
and thus,
\[ \deriv{\sqrt{g_{rr}} u^r}{\tau} = \sqrt{g_{rr}} \left[ \deriv{u^r}{\tau} +  \frac{\dot{a}}{a} u^r u^t + [\tfrac{1}{2} \partial_r \log{g_{rr}} ] (u^r)^2 \right] = - \sqrt{g_{rr}} \frac{\dot{a}}{a} u^r u^t \]
Now let,
\[ p = \sqrt{g_{rr}} u^r \]
this is the radial momentum of the particle measured by a local comoving observer. The above equation then becomes,
\[ \deriv{p}{\tau} = - \frac{\dot{a}}{a} p u^t \]
Furthermore, $u^t \d{\tau} = \d{t}$ and thus,
\[ \frac{\dot{p}}{p} + \frac{\dot{a}}{a} = 0 \]
or equivalently,
\[ \deriv{}{t} \left[ \log{p} + \log{a} \right] = 0 \]
Thus,
\[ p \propto a^{-1} \]  
Furthermore, we may consider the energy directly, 
\[ \deriv{u^t}{\tau} + \Gamma^t_{tt} u^t u^t + 2\Gamma^t_{tr} u^t u^r + \Gamma^t_{rr} u^r u^r = 0 \]
\begin{align*}
\Gamma^r_{tt} & = 0
\\
\Gamma^r_{tr} & = 0
\\
\Gamma^r_{rr} & = \frac{\dot{a}}{a} g_{rr} 
\end{align*}
and thus,
\[ \deriv{u^t}{\tau} + \frac{\dot{a}}{a} g_{rr} (u^r)^2 = 0 \]
However, the normalization condition gives,
\[ g_{tt} (u^t)^2 + g_{rr} (u^r)^2 = -c^2 \]
and thus,
\[ \deriv{u^t}{\tau} = \frac{\dot{a}}{a} (c^2 - (u^t)^2) \]
Recalling that $\d{t} = u^t \d{\tau}$ and for a comoving observer,
\[ E = - m u^\alpha (u_{\text{obs}})_\alpha = - g_{tt} m u^t (u_{\text{obs}})^t = m c u^t \]
we find,
\[ \frac{\dot{E}}{E} + \frac{\dot{a}}{a} \left( 1 - \frac{m^2 c^4}{E^2} \right) = 0 \]
In the relativistic limit $E \gg m c^2$ we recover $E \propto a^{-1}$.

\section*{7.}

Consider the metric,
\[ \d{s^2} = - c^2 \d{t^2} + e^{2 H t} ( \d{r^2} + r^2 \d{\Omega^2} ) \]
In this spacetime consider a light beam emitted radially from the origin at time $t_0$ (recall that such a spacetime is actually spatially symmetric so this is no restriction). Since the beam follows a null geodesic,
\[ \dot{r} = c e^{-H t} \]
Therefore, the comoving coordinate evolves with cosmic time as,
\[ r(t) = \int_{t_0}^t c e^{-H t} \d{t} = \frac{c}{H} e^{- H t_0} (1 - e^{-2 H (t - t_0)}) \]
In the limit $t \to \infty$, note that there is a maximum (comoving) distance traversed i.e.
\[ r(\infty) = \frac{c}{H} e^{-H t_0} \]
The picture is entirely symmetric in the sense that an observer at the origin will receive at future infinity a light beam emitted from a (comoving) distance of $r(\infty)$ at cosmic time $t_0$ meaning that no light (and thus nothing) emitted further than $r(\infty)$ can ever be received. Therefore, $r(\infty)$ is the (comoving) radius of a spherical horizon surrounding the observer at the origin separating it from a causally disconnected region. It appears as if the radius of this horizon depends on time but this is not actually the case. We calculated the above in comoving coordinates and thus the physical radius of the spherical horizon with comoving radius $r(\infty)$ at time $t_0$ is,
\[ s_\infty = e^{H t_0} r(\infty) = \frac{c}{H} \]

\section*{8.}

Consider the metric,
\[ \d{s^2} = - c^2 \d{t^2} + [1 + A \cos{(\omega(t - z/c))}] \d{x^2} + [1 - A \cos{(\omega (t - z/c))}] \d{y^2} + \d{z^2} \]
Now I claim that a test particle at rest with respect to these coordinates remains at rest for all time. To show this, consider an initial four velocity $u^\alpha = (1, 0, 0, 0)$ in which case the geodesic equation,
\[ \deriv{u^\mu}{\tau} + \Gamma^\mu_{\alpha \beta} u^\alpha u^\beta = 0 \]
becomes,
\[ \deriv{u^\alpha}{\tau} + \Gamma^\mu_{00} (u^0)^2 = 0 \]
Furthermore,
\[ \Gamma^\mu_{00} = \tfrac{1}{2} \eta^{\mu \mu} ( \partial_0 h_{\mu 0} + \partial_0 h_{0 \mu} - \partial_\mu h_{00} ) = 0 \]
since $h_{0 \mu} = 0$. Therefore at the initial time we find,
\[ \deriv{u^\alpha}{\tau} = 0 \]
and thus $u^\alpha = (1, 0, 0, 0)$ is a solution to the geodesic equation. Therefore, we may assume that the mirrors remain fixed in the given coordinate system. Let the laser have wold line $x^\alpha = (ct, 0, 0, 0)$ and the mirror, $x^\alpha = (ct, L, 0, 0)$. We need to orient the beam line along the $x$-axis or equivalently the $y$-axis for maximum response since the wave has $+$ polarization. 
\bigskip\\
Now we need to solve the geodesic equation for photons traveling between the laser and the mirror and back. Consider a photon with initial four momentum $p^\alpha = p (1 + \tfrac{1}{2} A \cos{(\omega t_0)}, 1, 0, 0)$ (such that $p^\alpha p_\alpha = 0$ to first-order in $A$) emitted upwards from the laser at time $t_0$. Then the geodesic equation gives,
\[ \deriv{u^\mu}{\lambda} + \Gamma^\mu_{00} (u^0)^2 + 2 \Gamma^\mu_{01} u^0 u^1 + \Gamma^\mu_{11} (u^1)^2 = 0 \]
We have shown that $\Gamma^\mu_{00} = 0$ and we can easily calculate,
\begin{align*}
\Gamma^\mu_{01} & = - \tfrac{1}{2} (A/c) \omega \sin{(\omega (t - z/c))} (0, 1, 0, 0)
\\
\Gamma^\mu_{11} & =  - \tfrac{1}{2} (A/c) \omega \sin{(\omega (t - z/c))} (1, 0, 0, 1)
\end{align*}
We see that the photon picks up momentum in the $z$-direction. If we assume that the amplitude $A$ is small, we can ignore the second order effects of this momentum on the geodesic equations. Therefore we have,
\begin{align*}
\deriv{u^0}{\lambda} & = \tfrac{1}{2} (A/c) \omega \sin{(\omega (t - z/c))} (u^1)^2
\\
\deriv{u^1}{\lambda} & = (A/c) \omega \sin{(\omega (t - z/c))} u^0 u^1
\\
\deriv{u^3}{\lambda} & = \tfrac{1}{2} (A/c) \omega \sin{(\omega (t - z/c))} (u^1)^2
\end{align*}
We will solve the first-order effects of these differential equations via perturbation theory. Choosing to set $u^0 = 1$ initially, at zeroth-order, we have $x_{(0)}^\mu = (\lambda (1 + \tfrac{1}{2} A \cos{(\omega t_0)}) + c t_0, \lambda , 0, 0)$ meaning that it reaches its destination at $\lambda = L$ and thus $c \Delta t = L (1 + \tfrac{1}{2} A \cos{(\omega t_0)})$. Now, we find the first-order deflection from this path via integrating the equations of motion,
\begin{align*}
x_{(1)}^0(\lambda) & = \int_0^\lambda \int_0^\lambda \deriv{u^0}{\lambda} \d{\lambda} \d{\lambda} = \tfrac{1}{2} A \omega/c \int_0^\lambda \int_0^\lambda \sin{(\omega (\lambda / c + t_0))} \d{\lambda} \d{\lambda}
\\
& = \tfrac{1}{2} A c/\omega \left[(\omega / c) \lambda \cos{(\omega t_0)} + \sin{(\omega t_0)}  - \sin{(\omega (\lambda / c + t_0))} \right]
\\
x_{(1)}^1(\lambda) & = \int_0^\lambda \int_0^\lambda \deriv{u^0}{\lambda} \d{\lambda} \d{\lambda} = A \omega/c \int_0^\lambda \int_0^\lambda \sin{(\omega (\lambda / c + t_0))} \d{\lambda} \d{\lambda}
\\
& =  A c/\omega  \left[(\omega / c) \lambda \cos{(\omega t_0)} + \sin{(\omega t_0)}  - \sin{(\omega (\lambda / c + t_0))} \right]
\\
x_{(1)}^3(\lambda) & = \int_0^\lambda \int_0^\lambda \deriv{u^3}{\lambda} \d{\lambda} \d{\lambda} = \tfrac{1}{2} A \omega/c \int_0^\lambda \int_0^\lambda \sin{(\omega (\lambda / c + t_0))} \d{\lambda} \d{\lambda}
\\
& = \tfrac{1}{2} A c/\omega  \left[(\omega / c) \lambda \cos{(\omega t_0)} + \sin{(\omega t_0)}  - \sin{(\omega (\lambda / c + t_0))} \right]
\end{align*}
We need to solve for $x^1(\lambda) = L$ and then find the corresponding value of $x^0(\lambda)$. We have,
\[ \lambda +  A c/\omega \left[(\omega / c) \lambda \cos{(\omega t_0)} + \sin{(\omega t_0)}  - \sin{(\omega (\lambda / c + t_0))} \right] = L \]
and thus,
\begin{align*}
x^0(\lambda) & = \lambda (1 + \tfrac{1}{2} A \cos{(\omega t_0)}) +  \tfrac{1}{2}  A c/\omega  \left[ (  \omega / c) \lambda \cos{(\omega t_0)} + \sin{(\omega t_0)}  - \sin{(\omega (\lambda / c + t_0))} \right]
\\
& = L + \tfrac{1}{2} \lambda  A \cos{(\omega t_0)} - \tfrac{1}{2}  A c/\omega  \left[ (  \omega / c) \lambda \cos{(\omega t_0)} + \sin{(\omega t_0)}  - \sin{(\omega (\lambda / c + t_0))} \right]
\end{align*}
Now to zeroth-order $\lambda = L$ so when we plug in it is not necessary to use a higher-order correction since this would give an effect quadratic in $A$. Thus,
\[ t_{\text{travel}} = L/c (1 + \tfrac{1}{2} A \cos{(\omega t_0)}) - \tfrac{1}{2} A/\omega  \left[ (\omega L / c) \cos{(\omega t_0)} + \sin{(\omega t_0)} - \sin{(\omega(L/c + t_0))} \right] \]
Suppose that $\omega L / c \ll 1$ then we can expand this expression in a nice form. Consider,
\begin{align*}
\sin{(\omega(L / c + t_0))} & = \sin{(\omega L /c)} \cos{(\omega t_0)} + \cos{(\omega L / c)} \sin{(\omega t_0)}
\\
& =  (\omega L / c) \cos{(\omega t)} + \left[ 1 - \tfrac{1}{2}  (\omega L / c)^2 \right] \sin{(\omega t_0)} 
\end{align*}
Therefore,
\[ t_{\text{travel}} = L/c (1 + \tfrac{1}{2} A \cos{(\omega t_0)}) - \tfrac{1}{4} A \omega (L / c)^2  \sin{(\omega t_0)}  \]
and thus,
\[ \frac{\Delta t_{\text{travel}}}{t_{\text{avg}}} = \tfrac{1}{2} A \cos{(\omega t_0)} - \tfrac{1}{4} A (L \omega / c) \sin{(\omega t)} \]
The return trip is symmetric assuming that the travel time is short compared to the period of the gravitational wave and thus,
\[ \frac{\Delta t}{t_{\text{avg}}} = A \cos{(\omega t_0)} - \tfrac{1}{2} A (L \omega / c) \sin{(\omega t_0)} \]
Therefore, the maximum deviation from the mean time $L / c$ is,
\[ \Delta t = L / c \sqrt{ A^2 + \tfrac{1}{4} A^2 (L \omega / c)^2 } = A (L / c) (1 + \tfrac{1}{8} (L \omega / c)^2) \]
This gravitational wave detector is only sensitive to frequencies much less than the light-clock frequency i.e. $\omega \ll c / L$. 
\bigskip\\
We can also compute this result another way. From the null geodesic condition $g_{\mu \nu} \d{x^\mu} \d{x^\nu} = 0$ we find,
\[ c \d{t} = \frac{1}{g_{00}} \left[ - g_{0i} \d{x^i} - \sqrt{ (g_{0i} g_{0j} - g_{ij} g_{00}) \d{x^i} \d{x^j} } \right] \]
For a geodesic traveling along the $x$-axis we have,
\[ \d{t} = \frac{1}{g_{00}} \left[ - g_{01}  - \sqrt{g_{01}^2 - g_{11} g_{00}} \right] \d{x} \]
In this particular case,
\[ c \d{t} = \sqrt{-\frac{g_{11}}{g_{00}}} = [ 1 + \tfrac{1}{2} A \cos{(\omega t_0)} ] \d{x} \]
Assuming that the phase of the wave does not change significantly over the course of the geodesic we have,
\[ t_{\text{travel}} = \frac{1}{c} \int_0^L [ 1 + \tfrac{1}{2} A \cos{(\omega t_0)} ] \d{x} = L/c [ 1 + \tfrac{1}{2} A \cos{(\omega t_0)} ] \]
reproducing the zeroth-order result above. 

\section*{9.}

Consider an ellipsoid with semi-axes $p$, $q$, and $s$ of mass $M$. This object has a mass inertia tensor,
\[ I^{ij} = \int \rho \: r^i r^j \: \d{V} = \frac{M}{5} 
\begin{pmatrix}
p^2 & 0 & 0
\\
0 & q^2 & 0
\\
0 & 0 & s^2
\end{pmatrix} \]
in body centered coordinates. If we rotate this object about the $z$-axis by an angle $\theta$ via the rotation matrix,
\[ R = \begin{pmatrix}
\cos{\theta} & - \sin{\theta} & 0
\\
\sin{\theta} & \cos{\theta} & 0 
\\
0 & 0 & 1
\end{pmatrix}
\] 
Then we have,
\[ I(\theta) = R I R^\top = \frac{M}{5}
\begin{pmatrix}
p^2 \cos^2{\theta} + q^2 \sin^2{\theta} & (p^2 - q^2) \cos{\theta} \sin{\theta} & 0
\\
(p^2 - q^2) \cos{\theta} \sin{\theta} & q^2 \cos^2{\theta} + q^2 \sin^2{\theta} & 0 
\\
0 & 0 & s^2
\end{pmatrix} \]
The ellipsoid rotates with angular velocity $\omega$ and thus the mass inertia has time dependence,
\[ \dot{I}(t) = \frac{M \omega}{5} (p^2 - q^2)
\begin{pmatrix}
 \sin{(2 \omega t)} & \cos{(2 \omega t)} & 0
\\
\cos{(2 \omega t)} & -\sin{(2 \omega t)} & 0
\\
0 & 0 & 0
\end{pmatrix} \]
We require the second derivative of this tensor,
\[ \ddot{I}(t) = \frac{2 M}{5} \omega^2 (p^2 - q^2) 
\begin{pmatrix}
\cos{(2 \omega t)} & -\sin{(2 \omega t)} & 0
\\
-\sin{(2 \omega t)} & -\cos{(2 \omega t)} & 0
\\
0 & 0 & 0
\end{pmatrix} \]
Now consider an observer with directional unit vector from the ellipsoid $\hat{n}$. Then, in the TT-gauge, the gravitational waves measured by the observer take the form,
\[ \bar{h}_{ij}(t) = \frac{2 G}{r c^4} \ddot{I}_{ij}^{TT}(t - r / c) \]
where,
\[ I^{TT} = P I P^\top - \tfrac{1}{2} P (\tr{I} - \hat{n} \cdot I \cdot \hat{n}) \]
and $P_{ij} = \delta_{ij} - n_i n_j$ is the projection matrix perpendicular to $\hat{n}$. If we set the observer along the $x$-axis at a distance $d$ from the source then we find,
\[ P = \begin{pmatrix}
0 & 0 & 0
\\
0 & 1 & 0 
\\
0 & 0 & 1
\end{pmatrix} \]
and thus,
\[ \ddot{I}^{TT}_{ij}(t) = \frac{M}{5} \omega^2 (p^2 - q^2) 
\begin{pmatrix}
0 & 0 & 0
\\
0 & \sin{(2 \omega t)} & 0
\\
0 & 0 & - \sin{(2 \omega t)}
\end{pmatrix} \]
Thus, finally,
\[ \bar{h}_{ij}(t) = \frac{2 MG}{5 c^4} \cdot \frac{\omega^2 (p^2 - q^2)}{d}  \begin{pmatrix}
0 & 0 & 0
\\
0 & \sin{(2 \omega (t - r/c))} & 0
\\
0 & 0 & - \sin{(2 \omega (t - r/c))}
\end{pmatrix} \]
so the amplitudes of the two gravitational wave polarizations are,
\begin{align*}
h_{+} & = \frac{2 MG}{5 c^4} \cdot \frac{\omega^2 (p^2 - q^2)}{d} 
\\
h_{\times} & = 0
\end{align*}
\end{document}