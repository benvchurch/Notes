\documentclass[12pt]{article}
\usepackage[english]{babel}
\usepackage[utf8]{inputenc}
\usepackage[english]{babel}
\usepackage[a4paper, total={7.25in, 9.5in}]{geometry}
\usepackage{tikz-feynman}
\tikzfeynmanset{compat=1.0.0} 
\usepackage{subcaption}
\usepackage{float}
\floatplacement{figure}{H}
\usepackage{simpler-wick}
\usepackage{mathrsfs}  
\usepackage{dsfont}
\usepackage{relsize}
\DeclareMathAlphabet{\mathdutchcal}{U}{dutchcal}{m}{n}


\newcommand{\field}{\hat{\Phi}}
\newcommand{\dfield}{\hat{\Phi}^\dagger}
 
\usepackage{amsthm, amssymb, amsmath, centernot}
\usepackage{slashed}
\newcommand{\notimplies}{%
  \mathrel{{\ooalign{\hidewidth$\not\phantom{=}$\hidewidth\cr$\implies$}}}}
 
\renewcommand\qedsymbol{$\square$}
\newcommand{\cont}{$\boxtimes$}
\newcommand{\divides}{\mid}
\newcommand{\ndivides}{\centernot \mid}

\newcommand{\Integers}{\mathbb{Z}}
\newcommand{\Natural}{\mathbb{N}}
\newcommand{\Complex}{\mathbb{C}}
\newcommand{\Zplus}{\mathbb{Z}^{+}}
\newcommand{\Primes}{\mathbb{P}}
\newcommand{\Q}{\mathbb{Q}}
\newcommand{\R}{\mathbb{R}}
\newcommand{\ball}[2]{B_{#1} \! \left(#2 \right)}
\newcommand{\Rplus}{\mathbb{R}^+}
\renewcommand{\Re}[1]{\mathrm{Re}\left[ #1 \right]}
\renewcommand{\Im}[1]{\mathrm{Im}\left[ #1 \right]}
\newcommand{\Op}{\mathcal{O}}

\newcommand{\invI}[2]{#1^{-1} \left( #2 \right)}
\newcommand{\End}[1]{\text{End}\left( A \right)}
\newcommand{\legsym}[2]{\left(\frac{#1}{#2} \right)}
\renewcommand{\mod}[3]{\: #1 \equiv #2 \: \mathrm{mod} \: #3 \:}
\newcommand{\nmod}[3]{\: #1 \centernot \equiv #2 \: mod \: #3 \:}
\newcommand{\ndiv}{\hspace{-4pt}\not \divides \hspace{2pt}}
\newcommand{\finfield}[1]{\mathbb{F}_{#1}}
\newcommand{\finunits}[1]{\mathbb{F}_{#1}^{\times}}
\newcommand{\ord}[1]{\mathrm{ord}\! \left(#1 \right)}
\newcommand{\quadfield}[1]{\Q \small(\sqrt{#1} \small)}
\newcommand{\vspan}[1]{\mathrm{span}\! \left\{#1 \right\}}
\newcommand{\galgroup}[1]{Gal \small(#1 \small)}
\newcommand{\bra}[1]{\left| #1 \right>}
\newcommand{\Oa}{O_\alpha}
\newcommand{\Od}{O_\alpha^{\dagger}}
\newcommand{\Oap}{O_{\alpha '}}
\newcommand{\Odp}{O_{\alpha '}^{\dagger}}
\newcommand{\im}[1]{\mathrm{im} \: #1}
\renewcommand{\ker}[1]{\mathrm{ker} \: #1}
\newcommand{\ket}[1]{\left| #1 \right>}
\renewcommand{\bra}[1]{\left< #1 \right|}
\newcommand{\inner}[2]{\left< #1 | #2 \right>}
\newcommand{\expect}[2]{\left< #1 \right| #2 \left| #1 \right>}
\renewcommand{\d}[1]{ \mathrm{d}#1 \:}
\newcommand{\dn}[2]{ \mathrm{d}^{#1} #2 \:}
\newcommand{\deriv}[2]{\frac{\d{#1}}{\d{#2}}}
\newcommand{\nderiv}[3]{\frac{\dn{#1}{#2}}{\d{#3^{#1}}}}
\newcommand{\pderiv}[2]{\frac{\partial{#1}}{\partial{#2}}}
\newcommand{\fderiv}[2]{\frac{\delta #1}{\delta #2}}
\newcommand{\parsq}[2]{\frac{\partial^2{#1}}{\partial{#2}^2}}
\newcommand{\topo}{\mathcal{T}}
\newcommand{\base}{\mathcal{B}}
\renewcommand{\bf}[1]{\mathbf{#1}}
\renewcommand{\a}{\hat{a}}
\newcommand{\adag}{\hat{a}^\dagger}
\renewcommand{\b}{\hat{b}}
\newcommand{\bdag}{\hat{b}^\dagger}
\renewcommand{\c}{\hat{c}}
\newcommand{\cdag}{\hat{c}^\dagger}
\newcommand{\hamilt}{\hat{H}}
\renewcommand{\L}{\hat{L}}
\newcommand{\Lz}{\hat{L}_z}
\newcommand{\Lsquared}{\hat{L}^2}
\renewcommand{\S}{\hat{S}}
\renewcommand{\empty}{\varnothing}
\newcommand{\J}{\hat{J}}
\newcommand{\lagrange}{\mathcal{L}}
\newcommand{\dfourx}{\mathrm{d}^4x}
\newcommand{\meson}{\phi}
\newcommand{\dpsi}{\psi^\dagger}
\newcommand{\ipic}{\mathrm{int}}
\newcommand{\tr}[1]{\mathrm{tr} \left( #1 \right)}
\newcommand{\C}{\mathbb{C}}
\newcommand{\CP}[1]{\mathbb{CP}^{#1}}
\newcommand{\Vol}[1]{\mathrm{Vol}\left(#1\right)}

\newcommand{\Tr}[1]{\mathrm{Tr}\left( #1 \right)}
\newcommand{\Charge}{\hat{\mathbf{C}}}
\newcommand{\Parity}{\hat{\mathbf{P}}}
\newcommand{\Time}{\hat{\mathbf{T}}}
\newcommand{\Torder}[1]{\mathbf{T}\left[ #1 \right]}
\newcommand{\Norder}[1]{\mathbf{N}\left[ #1 \right]}
\newcommand{\Znorm}{\mathcal{Z}}
\newcommand{\EV}[1]{\left< #1 \right>}
\newcommand{\interact}{\mathrm{int}}
\newcommand{\covD}{\mathcal{D}}
\newcommand{\conj}[1]{\overline{#1}}

\newcommand{\SO}[2]{\mathrm{SO}(#1, #2)}
\newcommand{\SU}[2]{\mathrm{SU}(#1, #2)}

\newcommand{\anticom}[2]{\left\{ #1 , #2 \right\}}


\newcommand{\pathd}[1]{\! \mathdutchcal{D} #1 \:}

\renewcommand{\theenumi}{(\alph{enumi})}


\renewcommand{\theenumi}{(\alph{enumi})}

\newcommand{\atitle}[1]{\title{% 
	\large \textbf{Physics GR8040 General Relativity
	\\ Assignment \# #1} \vspace{-2ex}}
\author{Benjamin Church }
\maketitle}

\theoremstyle{definition}
\newtheorem{theorem}{Theorem}[section]
\newtheorem{definition}{definition}[section]
\newtheorem{lemma}[theorem]{Lemma}
\newtheorem{proposition}[theorem]{Proposition}
\newtheorem{corollary}[theorem]{Corollary}
\newtheorem{example}[theorem]{Example}
\newtheorem{remark}[theorem]{Remark}
\begin{document}

\atitle{1}

\section*{1.}

\subsection*{(a)}

Let $a^i_j$ and $b^i_j$ be $(1,1)$-tensors. Define the coefficients $c^i_j = a^i_j + b^i_j$ in all bases. We need to show that such an object has the tensor property. Under a transformation,
\[ c^{i'}_{j'} = a^{i'}_{j'} + b^{i'}_{j'} = \Lambda^i_{i'} \Lambda^{j'}_j a^i_j + \Lambda^i_{i'} \Lambda^{j'}_j b^i_j = \Lambda^i_{i'} \Lambda^{j'}_j (a^i_j + b^i_j) = \Lambda^i_{i'} \Lambda^{j'}_j c^i_j \]
Therefore $c$ transforms as a $(1,1)$-tensor.

\subsection*{(b)}

Let $a_{ij}$ and $b^k$ be tensors and define the coefficients $c^k_{ij} = a_{ij} b^k$ in all bases. We need to show that $c^k_{ij}$ has the tensor property. Under a change of basis,
\[ c^{k'}_{i'j'} = a_{i'j'} b^{k'} = \Lambda_{i'}^i \Lambda_{j'}^j a_{ij} \Lambda^{k'}_k b^k = \Lambda_{i'}^i \Lambda_{j'}^j  \Lambda^{k'} c^k_{ij} \]
so $c$ transforms as a $(1,2)$-tensor.

\subsection*{(c)}

Let $a^i_{jk}$ be a $(1, 2)$-tensor and define the coefficients $c_k = a^i_{ik}$ in all bases. We need to show that $c_k$ has the tensor property. Under a change of basis,
\[ c_{k'} = a^{i'}_{i'k'} = \Lambda^{i'}_i \Lambda_{i'}^j \Lambda_{k'}^k a^i_{jk} = \delta^j_i \Lambda_{k'}^k a^{i}_{jk} = \Lambda_{k'}^k a^i_{ik} = \Lambda_{k'}^k c_k \]
so $c$ transforms as a $(0,1)$-tensor.

\section*{2.}

Let $\phi : V \times V \to \R$ be a bilinear function on a real vectorspace $\R$. Let $V$ have a basis $\bf{e}_i$. The map $\phi$ defines components $\phi_{ij} = \phi(\bf{e}_i, \bf{e}_j)$. Consider these components under a change of basis $\bf{e}_{i'} = \Lambda^i_{i'} \bf{e}_i$. Then, using bilinearity, we have,
\[ \phi_{i'j'} = \phi(\bf{e}_{i'}, \bf{e}_{j'}) = \phi(\Lambda^i_{i'} \bf{e}_i, \Lambda^j_{j'} \bf{e}_j) = \Lambda^i_{i'} \Lambda^{j}_{j'} \phi(\bf{e}_i, \bf{e}_j) = \Lambda^i_{i'} \Lambda^{j}_{j'} \phi_{ij} \]
Thus, $\phi_{ij}$ transform as a $(0,2)$-tensor. 

\section*{3.}

Suppose that a $(0, m)$-tensor $a_{i_1, \cdots, i_m}$ is symmetric in some basis. Consider this object transformed to another basis which we can express in terms of the original basis as,
\[ a_{i_1', \cdots, i_m'} = \Lambda^{i_1}_{i_1'} \cdots \Lambda^{i_m}_{i_m'} a_{i_1, \dots, i_m} \]
Therefore, swapping index $i_a$ and $i_b$ with $a < b$ we find,
\[ a_{i_1', \dots, i_b', \dots, i_a', \dots i_m'} = \Lambda^{i_1}_{i_1'} \dots \Lambda^{i_a}_{i_b'} \Lambda^{i_b}_{i_a'} \cdots \Lambda^{i_m}_{i_m'} a_{i_1, \dots, i_a, \dots, i_b, \cdots i_m} = \Lambda^{i_1}_{i_1'} \dots \Lambda^{i_a}_{i_b'} \cdots \Lambda^{i_b}_{i_a'} \dots \Lambda^{i_m}_{i_m'} a_{i_1, \dots, i_b, \dots, i_a, \dots i_m} \]
where I used the symmetry of $a$ to swap $i_a$ and $i_b$. Now renaming $i_a \mapsto i_b$ and $i_b \mapsto i_a$ we find,
\[ a_{i_1', \dots, i_b', \dots, i_a', \dots i_m'} = \Lambda^{i_1}_{i_1'} \dots \Lambda^{i_a}_{i_b'} \Lambda^{i_b}_{i_a'} \cdots \Lambda^{i_m}_{i_m'} a_{i_1, \dots, i_a, \dots, i_b, \cdots i_m} = \Lambda^{i_1}_{i_1'} \dots \Lambda^{i_b}_{i_b'} \cdots \Lambda^{i_a}_{i_a'} \dots \Lambda^{i_m}_{i_m'} a_{i_1, \dots, i_a, \dots, i_b, \dots i_m} = a_{i_1', \dots, i_a', \dots, i_b', \dots i_m'} \]
and thus the symmetry is preserved. 

\section*{4.}

Let $a_{ij}$ be a $(0,2)$-tensor. Define $b_{ij} = \tfrac{1}{2}(a_{ij} + a_{ji})$ and $c_{ij} = \tfrac{1}{2} (a_{ij} - a_{ji})$. By the first problem $b_{ij}$ and $c_{ij}$ are tensors. Furthermore, 
\begin{align*}
b_{ji} & = \tfrac{1}{2}(a_{ji} + a_{ij}) = \tfrac{1}{2}(a_{ij} + a_{ji}) = b_{ij}
\\
c_{ji} & = \tfrac{1}{2}(a_{ji} - a_{ij}) = - \tfrac{1}{2}(a_{ij} - a_{ji}) = - c_{ji}
\end{align*} 
so $b_{ij}$ is symmetric and $c_{ij}$ is antisymmetry. Finally, it is clear that $a_{ij} = b_{ij} + c_{ij}$. 

\section*{5.} 

\subsection*{(a)}

The object $\delta_{ij}$ satisfies the tensor property with respect to a transformation $\Lambda$ exactly when,
\[ \delta_{i'j'} = \Lambda^{i}_{i'} \Lambda^{j}_{j'} \delta_{ij} = \Lambda^{i}_{i'} \Lambda^{i}_{j'} \]
which is neatly summarized by the equivalent condition on $\Lambda$ as a matrix that $\Lambda^{\top} \Lambda = I$. Therefore $\delta_{ij}$ is a tensor with respect to exactly the orthogonal transformations. In particular, reflections are orthogonal because they preserve dot products so reflections also preserve $\delta_{ij}$. 

\subsection*{(b)}

The easiest way to see the failure of $\epsilon_{ijk}$ to be a tensor is to consider the cross product of $(1,0)$-tensors $a$ and $b$ i.e. $c_i = \epsilon_{ijk} a^j b^k$. Under a full parity inversion, (which, in 3D, is a rotation plus a single reflection), we find $c_{i'} = \epsilon_{ijk} a^{j'} b^{k'} =   \epsilon_{ijk} (-a^j)(-b^k) = \epsilon_{ijk} a^j b^k = c_i$. However, if $c_i$ were a type $(0,1)$-tensor then it would transform as $c_i \mapsto - c_i$ under a parity inversion. Since it does not, we know that $\epsilon_{ijk}$ cannot satisfy the tensor property for such transformations otherwise its contraction with other tensors must also be tensorial. 

\section*{6.}

A linear map $\bf{B} : V \to V$ can be expressed in a basis $\bf{e}_i$ via the rule $\bf{B}(\bf{e}_j) = B^i_j \bf{e}_i$. Therefore, $B^i_j = \bf{e}^i(\bf{B}(\bf{e}_j))$. Consider these coefficients under a transformation,
\[ B^{i'}_{j'} = \bf{e}^{i'}(\bf{B}(\bf{e}_{j'})) = \Lambda^{i'}_i \bf{e}^{i}(\bf{B}(\Lambda^j_{j'} \bf{e}_j)) = \Lambda^{i'}_i \Lambda^j_{j'} \bf{e}^{i}(\bf{B}(\bf{e}_j)) \Lambda^{i'}_i \Lambda^j_{j'} B^i_j \]
using linearity. Therefore, $B^i_j$ transforms as a $(1, 1)$-tensor. 

\section*{7.}  

Consider the metric $(0,2)$- tensor $g_{ij}$. We define its inverse by the equation $g_{ij} g^{jk} = \delta_i^k$. Under a transformation, the transformed version of $g^{jk}$ must still satisfy the transformed equation,
\[ g_{i'j'} g^{j' k'} = \delta_{i'}^{k'} \implies  \Lambda^i_{i'} \Lambda^j_{j'} g_{ij} g^{j' k'} = \delta_{i'}^{k'} \] 
This equation is most easily manipulated in matrix form,
\[ \Lambda^{\top} g \Lambda g'^{-1} = I \]
which gives,
\[ g'^{-1} =  \Lambda^{-1} g^{-1} (\Lambda^{-1})^{\top} \]
Rewriting this in components,
\[ g^{i'j'} = \Lambda^{i'}_i \Lambda^{j'}_j g^{ij} \]
which shows that $g^{ij}$ transforms as a $(2,0)$-tensor. 

\section*{8.}

Consider the tensor $\epsilon^{ijk} \epsilon_{jkm}$. When $i \neq m$ then there do not exist values for $j$ and $k$ such that both $ijk$ and $jkm$ are permutations of $123$ since all four of $ijkm$ must be different but take on at most three values. Thus, for $i \neq m$ we have $\epsilon^{ikj} \epsilon_{jkm} = 1$. Furthermore, when $i = m$, then the only nonzero terms come from $i, j, k$ all different. Given a fixed $i$ there are two such terms. Then, $\epsilon^{ijk} = \epsilon_{jki} = \epsilon_{jkm}$ so these two terms are both $+1$. Therefore,
\[ \epsilon^{ijk} \epsilon_{jkm} = 2 \]
In summary,
\[ \epsilon^{ijk} \epsilon_{jkm} = 2 \delta^i_m \]  

\section*{9.}

Assume the metric signature $\eta = \mathrm{diag}(-, +, +, +)$. We have the tensors in a given basis,
\[ 
X^\mu_{\:\: \nu} = 
\begin{pmatrix}
-2 & 0 & 1 & - 1
\\
1 & 0 & 3 & 2 
\\
1 & 1 & 0 & 0
\\
2 & 1 & 1 & -2 
\end{pmatrix}
\implies X_{\mu}^{\:\: \nu} = \eta_{\mu \alpha} \eta^{\nu \beta} X^{\alpha}_{\:\: \beta} = 
\begin{pmatrix}
-2 & 0 & -1 & 1
\\
-1 & 0 & 3 & 2 
\\
-1 & 1 & 0 & 0
\\
-2 & 1 & 1 & -2 
\end{pmatrix}
\]
Furthermore,
\[ X_{[\mu, \nu]} = 
\frac{1}{2} \begin{pmatrix}
-2 & -1 & 0 & -3
\\
-1 & 0 & 4 & 3 
\\
0 & 4 & 0 & 0
\\
-3 & 3 & 1 & -4 
\end{pmatrix}
\quad \quad 
X^{(\mu, \nu)} = 
\frac{1}{2} \begin{pmatrix}
0 & -1 & -2 & -1
\\
1 & 0 & 2 & 1 
\\
2 & -2 & 0 & -1
\\
1 & -1 & 1 & 0 
\end{pmatrix} \]
and likewise,
\[ X^{\lambda}_{\lambda} = -4 \]
Next consider the vector,
\[ V^\mu = (-1, 2, 0, -2) \]
which gives,
\[ V^\mu V_\mu = 7 \quad \quad V_\mu X^{\mu \nu} = (4, -2, 5, 7) \]
\section*{10.}

\subsection*{(a)}

Suppose that a particle moves with constant acceleration $a = \deriv{v}{t}$ with respect to an inertial frame $(t, x, y, x)$. Then the proper time satisfies,
\[ \deriv{\tau}{t} = \sqrt{1 - \frac{v^2}{c^2}} = \sqrt{1 - \frac{a^2 t^2}{c^2}} \]
Integrating this equation gives,
\[ \tau = \frac{c}{2a} \left( \frac{at}{c} \sqrt{1 - \frac{a^2 t^2}{c^2}} +  \sin^{-1}{\left( \frac{at}{c} \right)} \right) \] 

\subsection*{(b)}

Now suppose that the particle experiences constant \textit{proper} acceleration i.e. $\deriv{u^\alpha}{\tau}$ is constant in the rest frame of the particle. In this frame, 
\[ \deriv{u^\alpha}{\tau} = (0, a, 0, 0) \]
where $a$ is the coordinate acceleration in the rest frame (normalized by factors of $c$ in the following) and thus also the proper acceleration. A Lorentz transformation to the frame $(t, x, y, z)$ then gives,
\[ \deriv{u^\alpha}{\tau} = (\gamma \beta a, \gamma a, 0, 0)  \] 
However, in terms of coordinate variables,
\[ u^{\alpha} = (\gamma, \gamma \beta, 0, 0) \]
and thus,
\[ \deriv{u^{\alpha}}{\tau} = \gamma (\dot{\gamma}, \dot{\gamma} \beta + \gamma \dot{\beta}, 0, 0) = (\gamma \dot{\gamma}, \gamma \dot{\gamma} \beta + \gamma^2 \dot{\beta}, 0, 0) \]
where,
\[ \dot{\gamma} = \deriv{}{t} \frac{1}{\sqrt{1 - \beta^2}} = \frac{\beta \dot{\beta}}{(1 - \beta^2)^{\frac{3}{2}}} = \gamma^3 \beta \dot{\beta} \]
Comparing the two expressions for the four-acceleration we find that,
\[ \dot{\beta} = \gamma^{-3} a \]
Thus,
\[ \deriv{v}{t} = \left(1 - \frac{v^2}{c^2} \right)^{\frac{3}{2}} a \]
which implies that,
\[ v(t) = \frac{at}{\sqrt{1 + \left( \frac{at}{c} \right)^2}}  \]
Finally, the proper time is given by,
\[ \deriv{\tau}{t} = \sqrt{1 - \beta^2} = \frac{1}{\sqrt{1 + \left( \frac{at}{c} \right)^2}} \]
which implies that,
\[ \tau = \sinh^{-1}\left( \frac{at}{c} \right) \]

\subsection*{(c)}

Finally, suppose that $a^{\alpha}$ is constant in the fixed frame $(t, x, y, z)$. Then we have,
\[ \deriv{u^{\alpha}}{\tau} = (0, a, 0, 0) \]
in this frame at all times. However, as before,
\[ \deriv{u^\alpha}{\tau} = (\gamma^4 \beta \dot{\beta}, \gamma^4 \dot{\beta}, 0, 0) \]
This is inconsistent unless the time-component of $a^\alpha$ is allowed to vary. Making this assumption, we find,
\[ u^x(\tau) = a \tau \]
Therefore,
\[ \gamma \beta = a \tau \implies \gamma^2 - 1 = (a \tau)^2 \]
which implies that  
\[ \gamma = \sqrt{1 + (a \tau)^2} \]
and thus,
\[ \deriv{t}{\tau} = \gamma = \sqrt{1 + (a \tau)^2}\]
Integrating this we find,
\[ t = \int_0^t \d{\tau} \sqrt{1 + (a \tau)^2} = \tfrac{\tau}{2} \sqrt{1 + (a \tau)^2} + \tfrac{1}{2 a} \sinh^{-1}(a \tau) \]
This cannot be inverted explicitly. 

\section*{11.} 

Consider a source emitting uniformly at a constant definite frequency $\nu$ with luminosity $L$ in its rest frame.   

\subsection*{(a)}

In the rest frame of the particle $S$, the derivative of energy momentum vector has the form,
\[ \deriv{p^\alpha}{\tau} = (-L, 0, 0, 0) \]
since proper time coincides with coordinate time and the radiation carries away zero total momentum. Therefore, boosting to the laboratory frame $S'$ we find,
\[ \deriv{p'^\alpha}{\tau} = (-\gamma L, -\gamma \beta L, 0, 0) \]
Thus, in this frame,
\[ \deriv{E'}{\tau} = - \gamma L \]
implying that,
\[ L' = - \deriv{E'}{t'} = L \]
recovering the well-known fact that radiated power is a Lorentz invariant. 

\subsection*{(b)}

Consider a photon emitted in the frame $S$ with angular variables $(\theta, \phi)$ with the pole $\theta = 0$ oriented along the direction of motion. This photon will have a wavevector (ignoring factors of $c$ for now),
\[ k^\alpha = (\nu, \nu \cos{\theta}, \nu \sin{\theta} \cos{\phi}, \nu \sin{\theta} \sin{\phi} ) \]
Under a Lorentz transformation, this wavevector becomes,
\[ k^\alpha = (\gamma \nu (1 + \beta \cos{\theta}), \gamma \nu (\cos{\theta} + \beta), \nu \sin{\theta} \cos{\phi}, \nu \sin{\theta} \sin{\phi} ) \]
which means that in $S'$ the photon has frequency $\nu' = \gamma (1 + \beta \cos{\theta}) \nu$ and angular coordinates $(\theta', \phi')$ with,
\[ \cos{\theta'} = \frac{\cos{\theta} + \beta}{\beta \cos{\theta} + 1} \quad \quad \sin{\theta'} = \frac{\sin{\theta}}{\gamma(\beta \cos{\theta} + 1)} \]
and $\phi' = \phi$. The number of photons is conserved so the number of photons $N(\theta)$ in a solid angle from polar angle $0$ to $\theta$ must satisfy $N'(\theta') = N(\theta)$. Thus, the angular distribution of photons in $S'$ is computed as,
\[ \deriv{N'}{\Omega'} = \frac{1}{2 \pi} \deriv{N'(\theta')}{\cos(\theta')} = \frac{1}{2 \pi} \deriv{\cos{\theta}}{\cos{\theta'}} \deriv{N(\theta)}{\cos{\theta}} = \deriv{\cos{\theta}}{\cos{\theta'}} \frac{L t}{4 \pi h \nu} \]
because the angular distribution of photons emitted in a time $t$ is,
\[ \deriv{N}{\Omega} = \frac{L t}{4 \pi h \nu} \]
is uniform in the rest frame $S$. An identical argument running the opposite direction will show that,
\[ \cos{\theta} = \frac{\cos{\theta'} - \beta}{1 - \beta \cos{\theta'})} \] 
Therefore,
\begin{align*}
\deriv{\cos{\theta}}{\cos{\theta'}} = \frac{1}{1 - \beta \cos{\theta'}} + \frac{\beta(\cos{\theta'} - \beta)}{(1 - \beta \cos{\theta'})^2} = \frac{1}{\gamma^2(1 - \beta \cos{\theta'})^2}  
\end{align*}
Therefore, the angular distribution of photons in the lab frame $S'$ is given by,
\[ \deriv{N'}{\Omega'} = \frac{Lt}{4 \pi h \nu} \frac{1}{\gamma^2 (1 - \beta \cos{\theta'})^2} \]
We can likewise compute the angular distribution of power by computing the energy flow in photons though a given solid angle. This is simply,
\[ \deriv{L'}{\Omega'} = \deriv{N'}{\Omega' \d{t'}} h \nu'(\theta') = \frac{L}{4 \pi} \deriv{t}{t'} \frac{\nu'}{\nu} \frac{1}{\gamma^2 (1 - \beta \cos{\theta'})^2} \]
Now,
\[ \frac{\nu'}{\nu} = \gamma(1 + \beta \cos{\theta}) = \gamma \left( 1 + \beta \frac{\cos{\theta'} - \beta}{1 - \beta \cos{\theta'}} \right) = \frac{\gamma(1 - \beta^2)}{1 - \beta \cos{\theta'}} = \frac{1}{\gamma(1 - \beta \cos{\theta'})} \]
Furthermore, for the time interval $t$ in the rest frame during which the photons are emitted, we have $t' = \gamma t$ in the frame $S'$.
Thus,
\[ \deriv{L'}{\Omega'} = \frac{L}{4 \pi} \frac{1}{\gamma^4 (1 - \beta \cos{\theta'})^3} \]
A good check of our work,
\[ L' = \int \deriv{L'}{\Omega'} \d{\Omega'} = \frac{L}{4 \pi \gamma^4} \int_{-1}^{1} \frac{2 \pi\d{(\cos{\theta}})}{(1 - \beta \cos{\theta'})^3} = \frac{L}{4 \pi \gamma^4} (4 \pi \gamma^4) = L \]
which shows that the total radiated power $L'$ in the frame $S'$ satisfied $L' = L$ so the total radiated power is indeed Lorentz invariant. 

\subsection*{(c)}   

We have computed the frequency $\nu'$ of photons in the frame $S'$ at a given angular position $(\theta', \phi')$ to be,
\[ \frac{\nu'}{\nu} = \frac{1}{\gamma (1 - \beta \cos{\theta'})} \]  
Therefore, averaging over the angular distribution, we find,
\[ \EV{\nu'} = \frac{1}{N} \int \deriv{N'}{\Omega'} \nu' \d{\Omega} = \frac{\nu}{4 \pi} \int \frac{1}{\gamma^3 (1 - \beta \cos{\theta'})^3} = \gamma \nu \]
so the average frequency is enhanced over the frequency of the emitted radiation in the source frame. Thus, the mean photon energy in $S'$ is $\EV{h\nu'} = \gamma h \nu$. This is clear from the invariance of the radiated power $L$. The number of photons emitted during some interval is fixed and the energy emitted is larger by $\gamma$ in the frame $S'$ since it is a constant power over a time interval longer by a factor of $\gamma$. Therefore, the energy of each photon must also be larger by a factor of $\gamma$.  
\end{document}