\documentclass[12pt]{article}
\usepackage[english]{babel}
\usepackage[utf8]{inputenc}
\usepackage[english]{babel}
\usepackage[a4paper, total={7.25in, 9.5in}]{geometry}
\usepackage{tikz-feynman}
\tikzfeynmanset{compat=1.0.0} 
\usepackage{subcaption}
\usepackage{float}
\floatplacement{figure}{H}
\usepackage{simpler-wick}
\usepackage{mathrsfs}  
\usepackage{dsfont}
\usepackage{relsize}
\DeclareMathAlphabet{\mathdutchcal}{U}{dutchcal}{m}{n}


\newcommand{\field}{\hat{\Phi}}
\newcommand{\dfield}{\hat{\Phi}^\dagger}
 
\usepackage{amsthm, amssymb, amsmath, centernot}
\usepackage{slashed}
\newcommand{\notimplies}{%
  \mathrel{{\ooalign{\hidewidth$\not\phantom{=}$\hidewidth\cr$\implies$}}}}
 
\renewcommand\qedsymbol{$\square$}
\newcommand{\cont}{$\boxtimes$}
\newcommand{\divides}{\mid}
\newcommand{\ndivides}{\centernot \mid}

\newcommand{\Integers}{\mathbb{Z}}
\newcommand{\Natural}{\mathbb{N}}
\newcommand{\Complex}{\mathbb{C}}
\newcommand{\Zplus}{\mathbb{Z}^{+}}
\newcommand{\Primes}{\mathbb{P}}
\newcommand{\Q}{\mathbb{Q}}
\newcommand{\R}{\mathbb{R}}
\newcommand{\ball}[2]{B_{#1} \! \left(#2 \right)}
\newcommand{\Rplus}{\mathbb{R}^+}
\renewcommand{\Re}[1]{\mathrm{Re}\left[ #1 \right]}
\renewcommand{\Im}[1]{\mathrm{Im}\left[ #1 \right]}
\newcommand{\Op}{\mathcal{O}}

\newcommand{\invI}[2]{#1^{-1} \left( #2 \right)}
\newcommand{\End}[1]{\text{End}\left( A \right)}
\newcommand{\legsym}[2]{\left(\frac{#1}{#2} \right)}
\renewcommand{\mod}[3]{\: #1 \equiv #2 \: \mathrm{mod} \: #3 \:}
\newcommand{\nmod}[3]{\: #1 \centernot \equiv #2 \: mod \: #3 \:}
\newcommand{\ndiv}{\hspace{-4pt}\not \divides \hspace{2pt}}
\newcommand{\finfield}[1]{\mathbb{F}_{#1}}
\newcommand{\finunits}[1]{\mathbb{F}_{#1}^{\times}}
\newcommand{\ord}[1]{\mathrm{ord}\! \left(#1 \right)}
\newcommand{\quadfield}[1]{\Q \small(\sqrt{#1} \small)}
\newcommand{\vspan}[1]{\mathrm{span}\! \left\{#1 \right\}}
\newcommand{\galgroup}[1]{Gal \small(#1 \small)}
\newcommand{\bra}[1]{\left| #1 \right>}
\newcommand{\Oa}{O_\alpha}
\newcommand{\Od}{O_\alpha^{\dagger}}
\newcommand{\Oap}{O_{\alpha '}}
\newcommand{\Odp}{O_{\alpha '}^{\dagger}}
\newcommand{\im}[1]{\mathrm{im} \: #1}
\renewcommand{\ker}[1]{\mathrm{ker} \: #1}
\newcommand{\ket}[1]{\left| #1 \right>}
\renewcommand{\bra}[1]{\left< #1 \right|}
\newcommand{\inner}[2]{\left< #1 | #2 \right>}
\newcommand{\expect}[2]{\left< #1 \right| #2 \left| #1 \right>}
\renewcommand{\d}[1]{ \mathrm{d}#1 \:}
\newcommand{\dn}[2]{ \mathrm{d}^{#1} #2 \:}
\newcommand{\deriv}[2]{\frac{\d{#1}}{\d{#2}}}
\newcommand{\nderiv}[3]{\frac{\dn{#1}{#2}}{\d{#3^{#1}}}}
\newcommand{\pderiv}[2]{\frac{\partial{#1}}{\partial{#2}}}
\newcommand{\fderiv}[2]{\frac{\delta #1}{\delta #2}}
\newcommand{\parsq}[2]{\frac{\partial^2{#1}}{\partial{#2}^2}}
\newcommand{\topo}{\mathcal{T}}
\newcommand{\base}{\mathcal{B}}
\renewcommand{\bf}[1]{\mathbf{#1}}
\renewcommand{\a}{\hat{a}}
\newcommand{\adag}{\hat{a}^\dagger}
\renewcommand{\b}{\hat{b}}
\newcommand{\bdag}{\hat{b}^\dagger}
\renewcommand{\c}{\hat{c}}
\newcommand{\cdag}{\hat{c}^\dagger}
\newcommand{\hamilt}{\hat{H}}
\renewcommand{\L}{\hat{L}}
\newcommand{\Lz}{\hat{L}_z}
\newcommand{\Lsquared}{\hat{L}^2}
\renewcommand{\S}{\hat{S}}
\renewcommand{\empty}{\varnothing}
\newcommand{\J}{\hat{J}}
\newcommand{\lagrange}{\mathcal{L}}
\newcommand{\dfourx}{\mathrm{d}^4x}
\newcommand{\meson}{\phi}
\newcommand{\dpsi}{\psi^\dagger}
\newcommand{\ipic}{\mathrm{int}}
\newcommand{\tr}[1]{\mathrm{tr} \left( #1 \right)}
\newcommand{\C}{\mathbb{C}}
\newcommand{\CP}[1]{\mathbb{CP}^{#1}}
\newcommand{\Vol}[1]{\mathrm{Vol}\left(#1\right)}

\newcommand{\Tr}[1]{\mathrm{Tr}\left( #1 \right)}
\newcommand{\Charge}{\hat{\mathbf{C}}}
\newcommand{\Parity}{\hat{\mathbf{P}}}
\newcommand{\Time}{\hat{\mathbf{T}}}
\newcommand{\Torder}[1]{\mathbf{T}\left[ #1 \right]}
\newcommand{\Norder}[1]{\mathbf{N}\left[ #1 \right]}
\newcommand{\Znorm}{\mathcal{Z}}
\newcommand{\EV}[1]{\left< #1 \right>}
\newcommand{\interact}{\mathrm{int}}
\newcommand{\covD}{\mathcal{D}}
\newcommand{\conj}[1]{\overline{#1}}

\newcommand{\SO}[2]{\mathrm{SO}(#1, #2)}
\newcommand{\SU}[2]{\mathrm{SU}(#1, #2)}

\newcommand{\anticom}[2]{\left\{ #1 , #2 \right\}}


\newcommand{\pathd}[1]{\! \mathdutchcal{D} #1 \:}

\renewcommand{\theenumi}{(\alph{enumi})}


\renewcommand{\theenumi}{(\alph{enumi})}

\newcommand{\atitle}[1]{\title{% 
	\large \textbf{Physics GR8040 General Relativity
	\\ Assignment \# #1} \vspace{-2ex}}
\author{Benjamin Church }
\maketitle}

\theoremstyle{definition}
\newtheorem{theorem}{Theorem}[section]
\newtheorem{definition}{definition}[section]
\newtheorem{lemma}[theorem]{Lemma}
\newtheorem{proposition}[theorem]{Proposition}
\newtheorem{corollary}[theorem]{Corollary}
\newtheorem{example}[theorem]{Example}
\newtheorem{remark}[theorem]{Remark}
\begin{document}

\atitle{3}

\section*{1.}

Consider a sphere of radius $a$ in 3D Euclidean space with coordinates $(\theta, \phi)$ in which the metric is
\[ g = \begin{pmatrix}
a^2 & 0 
\\
0 & a^2 \sin^2{\theta} 
\end{pmatrix} \]


\subsection*{(a)}

First embed the sphere $S^n$ of radius $a$ isometrically in $\R^{n+1}$. The tangent space of $S^n$ at a given point is canonically identified with hyperplane in $\R^{n+1}$ tangent to it. Pick a point $\vec{x} \in S^n$ and a tangent vector $\vec{v} \in T_{\vec{x}} S^n \subset \R^{n+1}$. I claim that the the curve, $\gamma(t) = \vec{x} \cos{t} + c \vec{v} \sin{t}$ lies on the sphere and is a geodesic where $c$ is a normalizing factor,
\[ c = \frac{a}{|\vec{v}|} \]
First, using the fact that $\vec{v} \in T_{\vec{x}} S^n$ is tangent to the sphere at $\vec{x}$ and thus perpendicular to $\vec{x}$, 
\[ (\vec{x} \cos{t} + c^2 \vec{v} \sin{t})^2 = \vec{x}^{\, 2} \cos^2{t} + 2 \vec{x} \cdot \vec{v} \cos{t} \sin{t} + c^2 \vec{v}^{\, 2} \sin^2{t} = a^2 \cos^2{t} + a^2 \sin^2{t} = a^2 \]
Thus $\gamma(t) \in S^n$. Furthermore,
\[ \nderiv{2}{}{t} \gamma(t) = - \vec{x} \cos{t} - c \vec{v} \sin{t} = - \gamma(t) \]
Therefore, $\ddot{\gamma}(t)$ is parallel to $\gamma(t)$ so its projection in $T_{\gamma(t)} S^n$ is zero since it is parallel to the normal of $S^n$ at the point $\gamma(t)$. I claim that this implies that $\gamma$ is a geodesic. For fixed $t$, we have shown that $\ddot{\gamma}(t)$ projects to zero in the tangent space which is equivalent to the derivative vanishing on $S^n$ at $\gamma(t)$ defined by the exponential map. However, in these coordinates the Christoffel symbols vanish at $\gamma(t)$ so $\gamma(t)$ satisfies the geodesic equation at $t$. Finally, by the existence and uniqueness theorem for ODEs, a geodesic is uniquely characterized by an initial point and a tangent vector at that point. Therefore, up to scaling the tangent vector i.e. parameterizing the curve, we have found all geodesics.

\subsection*{(b)}

Pick the north pole $\vec{n}$ on the sphere and consider a geodesic circle of radius $r$. By the above argument, the geodesics through the north pole have constant $\phi$ coordinate. We can compute the geodesic distance in terms of coordinates via,
\[ r = \int_0^{\theta} \sqrt{a^2 \d{\theta^2} + a^2 \sin^2{\theta} \: \d{\phi^2}} = \int_0^{\theta} a \d{\theta} = a \theta \]
Furthermore, fixing $r$ we take the geodesic circle to be all points with geodesic distance $r$ from $\vec{n}$ i.e. the points $(r/a, \phi)$ for $\phi \in [0, 2 \pi)$. The circumference of this circle is given by integrating the arc-length as $\phi$ varies,
\[ C = \int_0^{2 \pi} \sqrt{a^2 \d{\theta^2} + a^2 \sin^2{\theta} \: \d{\phi^2}} = \int_0^{2 \pi} a \sin{\theta} \: \d{\phi} = 2 \pi a \sin{\theta} = 2 \pi r \left( \frac{\sin{\theta}}{\theta} \right) \]
Therefore, the ratio of the circumference to the radius of this circle is,
\[ \frac{C}{2 \pi r} = 2 \pi \left( \frac{\sin{\theta}}{\theta} \right) \approx 2 \pi \left( 1 - \tfrac{1}{6} \theta^2 \right) =  2 \pi \left( 1 - \tfrac{1}{6} \left( \tfrac{r}{a} \right)^2 \right)  \]
which shows that,
\[ \frac{C}{2 \pi r} = \frac{\sin{\theta}}{\theta} = 1 - \frac{1}{6} \left( \frac{r}{a} \right)^2 + O((r/a)^4) \]
Thus, this ratio is decreased at second order in the size of the circle. Furthermore, integrating the area element over this circle we find its area is,
\[ A = \int_0^{2 \pi} \int_0^\theta \sqrt{g} \: \d{\theta} \d{\phi} = \int_0^{2 \pi} \int_0^\theta  a^2 \sin{\theta} \d{\theta} \d{\phi} = 2 \pi a^2 \int_0^\theta \sin{\theta} = 2 \pi a^2 (1 - \cos{\theta}) \]
Expanding this area,
\[ \frac{A}{\pi r^2} = \frac{2(1 - \cos{\theta})}{\theta^2} = 1 - \frac{1}{12} \left( \frac{r}{a} \right)^2 + O((r/a)^4) \]
Therefore, this area is decreased from $1$ at second order in the size of the circle. 

\section*{2.}

The Riemann Tensor defined as, 
\[R^\mu_{\alpha \nu \beta} = \partial_{\nu} \Gamma^\mu_{\alpha \beta} - \partial_{\beta} \Gamma^\mu_{\nu \alpha} + \Gamma^{\delta}_{\alpha \beta} \Gamma^{\mu}_{\nu \delta} - \Gamma^\delta_{\nu \alpha} \Gamma^\mu_{\beta \delta} \]
In a locally inertial frame centered at $x_0$ we can make the Christoffel symbols vanish at $x_0^\mu$. Therefore, using the fact that,
\[ \Gamma^\mu_{\alpha\beta} = \tfrac{1}{2}g^{\mu\nu}\left(\partial_\alpha g_{\beta\nu}+\partial_\beta g_{\alpha\nu} - \partial_\nu g_{\alpha\beta}\right) \] 
we find, 
\begin{align*}
R^\mu_{\alpha\beta\gamma}&\left(x^\mu_0\right) = \partial_\beta\Gamma^\mu_{\alpha\gamma} - \partial_\gamma\Gamma^\mu_{\alpha\beta}
\end{align*}
Index lowering (and allowing the metric to pass through the derivative because, at $x_0$, the Christoffel symbols which would normally correct for the derivative of the metric vanish) we find,
\begin{align*}
R_{\mu\alpha\beta\gamma} &(x^\mu_0) = \tfrac{1}{2}\left(\partial_\beta \partial_\alpha g_{\gamma\mu} + \partial_\beta \partial \gamma g_{\alpha\mu} - \partial_\beta \partial_\mu g_{\alpha\gamma} - \partial_\gamma\partial_\alpha g_{\beta\mu} - \partial_\gamma\partial_\beta g_{\alpha\mu} + \partial_\gamma\partial_\mu g_{\alpha\beta}\right)
\\
& = \tfrac{1}{2}\left(\partial_\beta\partial_\alpha g_{\gamma\mu} + \partial_\gamma\partial_\mu g_{\alpha\beta} - \partial_\beta\partial_\mu g_{\alpha\gamma} - \partial_\gamma\partial_\alpha g_{\beta\mu}\right)
\end{align*}
This expression will allow us to easily read off the symmetries of the curvature tensor. First label,
\[R_{\alpha\beta\gamma\delta} (x^\mu_0) =\frac{1}{2}\left(\underbrace{\partial_\gamma\partial_\beta g_{\alpha\delta}}_{(1)} + \underbrace{\partial_\alpha\partial_\delta g_{\beta\gamma}}_{(2)}- \underbrace{ \partial_\gamma\partial_\alpha g_{\beta\gamma} }_{(3)}-\underbrace{ \partial_\delta\partial_\beta g_{\alpha\gamma}}_{(4)}\right)\]
We note that there are specific symmetries in exchanging indices. In other words, in swapping the Greek indices we find that terms switch roles:
\[\alpha \leftrightarrow \beta \implies \text{ Term $(1)$} \leftrightarrow \text{ Term $(3)$} \text{ and } \text{ Term $(2)$} \leftrightarrow \text{ Term $(4)$}\]
\[ \gamma\leftrightarrow \delta \implies \text{ Term $(1)$} \leftrightarrow \text{ Term $(4)$} \text{ and } \text{ Term $(2)$} \leftrightarrow \text{ Term $(3)$}\]
Therefore, we get the property that, 
\[R_{\alpha\beta\gamma\delta} = -R_{\beta\alpha \gamma\delta} =-R_{\alpha\beta\delta\gamma} = R_{\beta \alpha \delta \gamma}\]
Similarly, 
\[\ \alpha\leftrightarrow \gamma \quad \text{and} \quad \beta\leftrightarrow\delta \implies \text{ Term $(1)$} \leftrightarrow \text{ Term $(2)$} \text{ and } \text{ Term $(3)$} \leftrightarrow \text{ Term $(4)$}\]
which implies that, 
\[R_{\alpha\beta\gamma\delta} = R_{\gamma\delta\alpha\beta} = R_{\delta\gamma\beta \alpha} = -R_{\delta\gamma\alpha\beta} = - R_{\gamma\delta\beta\alpha}\]
Finally, at $x_0$, consider,
\begin{align*}
R_{\alpha \beta \gamma \delta} & = \tfrac{1}{2}\left( \partial_\gamma\partial_\beta g_{\alpha\delta} + \partial_\alpha\partial_\delta g_{\beta\gamma} - \partial_\gamma\partial_\alpha g_{\beta\gamma} - \partial_\delta\partial_\beta g_{\alpha\gamma} \right)
\\
R_{\alpha \gamma \delta \beta} & = \tfrac{1}{2}\left( \partial_\delta \partial_\gamma g_{\alpha\beta} + \partial_\alpha\partial_\beta g_{\gamma\delta} - \partial_\delta\partial_\alpha g_{\gamma\delta} - \partial_\beta\partial_\gamma g_{\alpha\delta} \right)
\\
R_{\alpha \delta \beta \gamma} & = \tfrac{1}{2}\left( \partial_\beta\partial_\delta g_{\alpha\gamma} + \partial_\alpha\partial_\gamma g_{\delta\beta} - \partial_\beta\partial_\alpha g_{\delta\beta} - \partial_\gamma\partial_\delta g_{\alpha\beta} \right) 
\end{align*}
Let $(a, b)$ refer to the $b^{\text{th}}$ term of the $a^{\text{th}}$ line. Now,
\begin{enumerate}
\item (1,1) and (2,4) cancel
\item (1,2) and (2,3) cancel
\item (1,3) and (3,2) cancel
\item (1,4) and (3,1) cancel
\item (2,1) and (3,4) cancel
\item (2,2) and (3,3) cancel
\end{enumerate}
Therefore, all terms cancel to zero leaving,
\[ R_{\alpha \beta \gamma \delta} + R_{\alpha \gamma \delta \beta} + R_{\alpha \delta \beta \gamma} = 0 \]
Since the symmetries we derived are covariant, this proves them in all reference frames not only locally inertial ones.
\section*{3.}

\subsection*{(a)}

Consider a 2D Riemannian manifold with Riemann tensor $R_{\alpha \beta \gamma \delta}$. 
By the symmetries of $R_{\alpha \beta \gamma \delta}$ proven above we know that $R_{\alpha \beta \gamma \delta}$ vanishes whenever $\alpha = \beta$ or $\gamma = \delta$. Therefore, we need to only consider,
\begin{align*}
R_{1212} = - R_{2112} = - R_{1221} = R_{2121}
\end{align*}
and all other components vanish. Therefore there is only one independent. Furthermore, in this case,
\begin{align*}
R & = R^{ij}_{ij} = g^{i \alpha} g^{j \beta} R_{\alpha \beta i j} 
\\
& = g^{11} g^{22} R_{1212} + g^{12} g^{21} R_{1221} + g^{21} g^{12} R_{2112} + g^{22} g^{11} R_{2121} 
\\
& = 2 R_{1212} \left( g^{11} g^{22} - g^{12} g^{21} \right) = 2 R_{1212} \deg{g^{-1}} 
\end{align*}
Since the metric is non-degenerate $\det{g} \neq 0$ and therefore the entire Riemann tensor can be reconstructed from the curvature scalar,
\[ R = 2 R_{1212} \det{g^{-1}} \]

\subsection*{(b)}

For a 3D Riemannian manifold there is more than one independent component of the Riemann tensor. Using the symmetries, we can classify,
\begin{align*}
R_{1212} & = - R_{2112} = - R_{1221} = R_{2121}
\\
R_{1313} & = - R_{3113} = - R_{1331} = R_{3131}
\\
R_{2323} & = - R_{3223} = - R_{2332} = R_{3232}
\\
R_{1213} & = - R_{2113} = - R_{1231} = R_{1312} = - R_{3112} = - R_{1321} = R_{3121}
\\
R_{1223} & = - R_{2123} = - R_{1232} = R_{2312} = - R_{3212} = -R_{2331} = R_{3221}
\\
R_{1332} & = - R_{3132} = - R_{1323} = R_{3213} = - R_{3231} = -R_{3231} = R_{2331}
\end{align*}
to get six independent terms. These terms can be fully reconstructed from the Ricci tensor which also has six independent terms. 

\section*{4.}

\subsection*{(a)}

Let $T^{\alpha \beta}$ be a symmetric tensor. Then,
\begin{align*}
\nabla_\alpha T^{\mu}_{\nu} & = \partial_\alpha T^{\mu}_{\nu} + \Gamma_{\alpha \beta}^\mu T^{\beta \nu} - \Gamma^\beta_{\alpha \nu} T^{\mu}_{\beta} 
\end{align*}
Now tracing over $\alpha$ and $\mu$ we find,
\begin{align*}
\nabla_\mu T^{\mu}_{\nu} = \partial_\mu T^{\mu \nu} + \Gamma^\mu_{\mu \beta} T^{\beta}_{\nu} - \Gamma^\beta_{\mu \nu} T^{\mu}_{\beta} 
\end{align*}
Now using the identity,
\[ \frac{1}{\sqrt{g}} \partial_\mu \sqrt{g} = \Gamma^\alpha_{\alpha \mu} \]
we find that,
\begin{align*}
\nabla_\mu T^{\mu}_{\nu}  & = \partial_\mu T^{\mu}_{\nu} + \Gamma^\mu_{\mu \beta} T^{\beta}_{\nu} - \Gamma^\beta_{\mu \nu} T^{\mu}_{\beta} 
\\
& = \partial_\mu T^{\mu}_{\nu}  + \frac{1}{\sqrt{g}} \pderiv{\sqrt{g}}{x^\beta} T^{\beta}_{\nu} - g_{\alpha \beta} \Gamma^\beta_{\mu \nu} T^{\mu \alpha} 
\\
& = \frac{1}{\sqrt{g}} \pderiv{\sqrt{g} T^{\mu}_{\nu}}{x^\mu} - g_{\alpha \beta} \Gamma^\beta_{\mu \nu} T^{\mu \alpha} 
\end{align*}
Furthermore,
\[ g_{\alpha \beta} \Gamma^\beta_{\mu \nu} = \tfrac{1}{2} \left( \partial_\mu g_{\alpha \nu} + \partial_\nu g_{\mu \alpha} - \partial_\alpha g_{\mu \nu} \right) \]
and thus, using the symmetry of $T^{\mu \alpha}$,
\begin{align*}
g_{\alpha \beta} \Gamma^\beta_{\mu \nu} T^{\mu \alpha} & = \tfrac{1}{2} \left( \partial_\mu g_{\alpha \nu} T^{\mu \alpha} + \partial_\nu g_{\mu \alpha} T^{\mu \alpha} - \partial_\alpha g_{\mu \nu} T^{\mu \alpha} \right)
\\
& = \tfrac{1}{2} \left( \partial_\mu g_{\alpha \nu} T^{\mu \alpha} + \partial_\nu g_{\mu \alpha} T^{\mu \alpha} - \partial_\alpha g_{\mu \nu} T^{\alpha \mu} \right)
\\
\\
& = \tfrac{1}{2}  (\partial_\nu g_{\mu \alpha}) T^{\mu \alpha}
\end{align*}
Therefore, finally,
\[ \nabla_\mu T^{\mu}_{\nu} = \frac{1}{\sqrt{g}} \pderiv{\sqrt{g} T^{\mu}_{\nu}}{x^\mu} - \frac{1}{2} (\partial_\nu g_{\alpha \beta}) T^{\alpha \beta} \]

\subsection*{(b)}

Let $F^{\alpha \beta}$ be an antisymmetric tensor.  Then we can compute,
\begin{align*}
\nabla_\mu F^{\alpha \beta} = \partial_\mu F^{\alpha \beta} + \Gamma^\alpha_{\mu \nu} F^{\nu \beta} + \Gamma^\beta_{\mu \nu} F^{\alpha
 \nu} 
\end{align*}
Therefore taking the trace,
\begin{align*}
\nabla_\alpha F^{\alpha \beta} = \partial_\alpha F^{\alpha \beta} + \Gamma^\alpha_{\alpha \nu} F^{\nu \beta} + \Gamma^{\beta}_{\alpha \nu} F^{\alpha \nu}
\end{align*}
However, $F^{\alpha \nu}$ is symmetric an $\Gamma^\beta_{\alpha \nu}$ is antisymmetric in $\alpha \iff \nu$ so the term $\Gamma^{\beta}_{\alpha \nu} F^{\alpha \nu}$ vanishes. Thus we are left with,
\[ \nabla_\alpha F^{\alpha \beta} = \partial_\alpha F^{\alpha \beta} + \Gamma^\alpha_{\alpha \nu} F^{\nu \beta} \]
Now using the identity,
\[ \frac{1}{\sqrt{g}} \partial_\mu \sqrt{g} = \Gamma^\alpha_{\alpha \mu} \]
we find that,
\begin{align*}
\nabla_\alpha F^{\alpha \beta} & = \partial_\alpha F^{\alpha \beta} + \Gamma^\alpha_{\alpha \nu} F^{\nu \beta} 
\\
& = \partial_\alpha F^{\alpha \beta} + \frac{1}{\sqrt{g}} \pderiv{\sqrt{g}}{x^\nu} F^{\nu \beta} = \frac{1}{\sqrt{g}} \pderiv{\sqrt{g} F^{\alpha \beta}}{x^\alpha} 
\end{align*}

\section*{5.}


Consider the 3D sphere $S^3$ in coordinates $(\psi, \theta, \phi)$ with canonical metric
\[ \d{s^2} = \d{\psi^2} + \sin^2{\psi} [ \d{\theta^2} + \sin^2{\theta} \d{\phi^2} ] \]
Using the formula in terms of metric derivatives the Christoffel symbols are,
\begin{align*}
\Gamma^\psi_{\alpha \beta} & = 
\begin{pmatrix}
0 & 0 & 0
\\
0 & - \cos{\psi} \sin{\psi} & 0 
\\
0 & 0 & - \cos{\psi} \sin{\psi} \sin^2 {\theta} 
\end{pmatrix}
\\
\Gamma^\theta_{\alpha \beta} & = 
\begin{pmatrix}
0 & \cot{\psi} & 0
\\
\cot{\psi} & 0 & 0 
\\
0 & 0 & - \cos{\theta} \sin{\theta} 
\end{pmatrix}
\\
\Gamma^{\phi}_{\alpha \beta} & =  
\begin{pmatrix}
0 & 0 & \cot{\psi}
\\
0 & 0 & \cot{\theta}
\\
\cot{\psi} & \cot{\theta} & 0
\end{pmatrix} 
\end{align*}
Using these explicit forms, we can compute the six independent components of the Riemann tensor,
\begin{align*}
R_{1212} & = \sin^2{\psi}
\\
R_{1313} & = \sin^2{\psi} \sin^2{\theta}
\\
R_{2323} & = \sin^4{\psi} \sin^2{\theta}
\\
R_{1213} & = 0
\\
R_{1223} & = 0
\\
R_{1332} & = 0
\end{align*}
\section*{6.}

\subsection*{(a)}

Let $K$ and $L$ be Killing vector fields i.e.
\[ \nabla_\mu K_\nu + \nabla_\nu K_\mu = 0 \quad \quad \nabla_\mu L_\nu + \nabla_\mu L_\nu = 0 \]
For any constants $\alpha, \beta$ then clearly,
\[ \nabla_\mu (\alpha K + \beta L)_\nu + \nabla_\nu (\alpha K + \beta L)_\mu = \alpha (\nabla_\mu K_\nu + \nabla_\nu K_\mu) + \beta (\nabla_\mu L_\nu + \nabla_\nu L_\mu) = 0 \]
so $\alpha K + \beta L$ is a Killing vector field. 

\subsection*{(b)}

Let $K$ and $L$ be Killing vector fields. Now consider the vector field,
\[ [K, L]^\mu = K^\alpha \partial_\alpha L^\mu - L^\alpha \partial_\alpha K^\mu \] 
Our first task will be to write this expression in manifestly covariant notation. Consider,
\begin{align*}
K^\alpha \nabla_\alpha L^\mu - L^\alpha \partial_\alpha K^\mu & = K^\alpha \partial_\alpha L^\mu - L^\alpha \partial_\alpha K^\mu + K^\alpha \Gamma^\mu_{\alpha \beta} L^\beta - L^\alpha \Gamma^\mu_{\alpha \beta} K^\beta
\\
& = [K, L]^\mu + \Gamma^\mu_{\alpha \beta} (K^\alpha L^\beta - L^\alpha K^\beta) = [K, L]^\mu
\end{align*}
where the second terms vanishes by the symmetry of the Christoffel symbols. Therefore, by the fact that $\nabla_\mu g_{\alpha \beta} = 0$ we may trivially lower our indices to find that,
\[ [K, L]_\mu = K^\alpha \nabla_\alpha L_\mu - L^\alpha \nabla_\alpha K_\mu \]
Now consider,
\begin{align*}
\nabla_\mu [K, L]_\nu + \nabla_\nu [K, L]_\mu & = \nabla_\mu (K^\alpha \nabla_\alpha L_\nu - L^\alpha \nabla_\alpha K_\nu) + \nabla_\nu (K^\alpha \nabla_\alpha L_\mu - L^\alpha \nabla_\alpha K_\mu) 
\\
& = (\nabla_\mu K^\alpha \nabla_\alpha L_\nu - \nabla_\nu L^\alpha \nabla_\alpha K_\mu) + (\nabla_\nu K^\alpha \nabla_\alpha L_\mu - \nabla_\mu L^\alpha \nabla_\alpha K_\nu)
\\
& \quad + K^\alpha \nabla_\mu \nabla_\alpha L_\nu - L^\alpha \nabla_\mu \nabla_\alpha K_\nu + K^\alpha \nabla_\nu \nabla_\alpha L_\mu - L^\alpha \nabla_\nu \nabla_\alpha K_\mu
\end{align*}
Now using the Killing equation we can swap $\nabla_\mu K_\alpha = - \nabla_\alpha K_\mu$ and $\nabla_\alpha L_\nu = - \nabla_\nu L_\alpha$ so we find,
\begin{align*}
\nabla_\mu [K, L]_\nu + \nabla_\nu [K, L]_\mu & = 
(\nabla_\alpha K_\mu \nabla_\nu L^\alpha - \nabla_\nu L^\alpha \nabla_\alpha K_\mu) + (\nabla_\alpha K_\nu \nabla_\mu L^\alpha - \nabla_\mu L^\alpha \nabla_\alpha K_\nu)
\\
& \quad - K^\alpha \nabla_\mu \nabla_\nu L_\alpha + L^\alpha \nabla_\mu \nabla_\nu K_\alpha - K^\alpha \nabla_\nu \nabla_\mu L_\alpha + L^\alpha \nabla_\nu \nabla_\mu K_\alpha
\\
& = L^\alpha \{ \nabla_\mu, \nabla_\nu \} K_\alpha - K^\alpha \{ \nabla_\mu, \nabla_\nu \} L_\alpha 
\end{align*}
However, by the following problem $\nabla_\mu \nabla_\nu K_\alpha$ is antisymmetric in $\mu$ and $\nu$ when acting on a Killing field. Thus each anticommutator gives zero so,
\[ \nabla_\mu [K, L]_\nu + \nabla_\nu [K, L]_\mu = 0 \]
showing that the commutator of Killing fields is a Killing field.

\section*{7.}

Let $K$ be a Killing field. Then consider,
\begin{align*}
\nabla_\mu \nabla_\sigma K^\rho = g^{\rho \gamma} \nabla_\mu \nabla_\sigma K_\gamma 
\end{align*}
By the Killing equation,
\[ \nabla_\sigma K_\gamma + \nabla_\gamma K_\sigma = 0 \]
Now differentiating,
\[ \nabla_\mu \nabla_\sigma K_\gamma + \nabla_\mu \nabla_\gamma K_\sigma = 0 \]
Re-indexing this equation we find,
\begin{align*}
\nabla_\mu \nabla_\sigma K_\gamma + \nabla_\mu \nabla_\gamma K_\sigma & = 0
\\
\nabla_\sigma \nabla_\gamma K_\mu + \nabla_\sigma \nabla_\mu K_\gamma & = 0
\\
\nabla_\gamma \nabla_\mu K_\sigma + \nabla_\gamma \nabla_\sigma K_\mu & = 0
\end{align*} 
Adding the second and subtracting the third, we find, after some rearrangement,
\begin{align*}
\nabla_\mu \nabla_\sigma K_\gamma + \nabla_\sigma \nabla_\mu K_\gamma + \nabla_\sigma \nabla_\gamma K_\mu - \nabla_\gamma \nabla_\sigma K_\mu - \nabla_\gamma \nabla_\mu K_\sigma + \nabla_\mu \nabla_\gamma K_\sigma & = 0
\end{align*}
which we rewrite as,
\[ \nabla_\mu \nabla_\sigma K_\gamma + \nabla_\sigma \nabla_\mu K_\gamma + [ \nabla_\sigma, \nabla_\gamma] K_\mu + [\nabla_\mu, \nabla_\gamma] K_\sigma = 0 \]
Now introducing the Riemann tensor we find,
\[ 
\nabla_\mu \nabla_\sigma K_\gamma + \nabla_\sigma \nabla_\mu K_\gamma +  R_{\mu \rho \sigma \gamma} K^\rho + R_{\sigma \rho \mu \gamma} K^\rho  = 0 \]
and therefore, by subtracting the above equation,
\[ \nabla_\mu \nabla_\sigma K_\gamma = \nabla_\sigma \nabla_\mu K_\gamma + R_{\gamma \rho \mu \sigma} K^\rho = - \nabla_\mu \nabla_\sigma K_\gamma + (R_{\gamma \rho \mu \sigma} - R_{\mu \rho \sigma \gamma} - R_{\sigma \rho \mu \gamma} ) K^\rho \]
However, by the Bianchi identity any Riemann tensor index symmetries,
\[ R_{\mu \rho \sigma \gamma} + R_{\mu \sigma \gamma \rho} + R_{\mu \gamma \rho \sigma} = 0 \implies R_{\gamma \rho \mu \sigma} - R_{\mu \rho \sigma \gamma} - R_{\sigma \rho \mu \gamma} = - 2 R_{\mu \rho \sigma \gamma} \]
which implies that,
\[ \nabla_\mu \nabla_\sigma K_\gamma = - \nabla_\mu \nabla_\sigma K_\gamma - 2 R_{\mu \rho \sigma \gamma} K^\rho \]
Finally this gives,
\[ \nabla_\mu \nabla_\sigma K_\gamma = R_{\gamma \sigma \mu \rho} K^\rho \]
which is equivalent via index raising to,
\[ \nabla_\mu \nabla_\sigma K^\nu = R^\nu_{\: \sigma \mu \rho} K^\rho \]

\section*{8.}

Consider the Newtonian limit in which we take $v^i \ll c$ and $g_{\alpha \beta} = \eta_{\alpha \beta} + h_{\alpha \beta}$ with $h_{\alpha \beta}$ small and stationary $\partial_0 g_{\alpha \beta} = 0$. Consider the Christoffel symbols,
\[ \Gamma^\mu_{\alpha \beta} = \tfrac{1}{2} g^{\mu \nu} \left( \partial_\alpha g_{\nu \beta} + \partial_\beta g_{\alpha \nu} - \partial_\nu g_{\alpha \beta} \right) \]
Using the fact that $\eta_{\alpha \beta}$ is constant this becomes,
\[ \Gamma^\mu_{\alpha \beta} = \tfrac{1}{2} \eta^{\alpha \beta} \left( \partial_\alpha h_{\nu \beta} + \partial_\beta h_{\alpha \nu} - \partial_\nu h_{\alpha \beta} \right) \]
where I have dropped higher-order terms in $h$. Now consider the Riemann tensor,
\[ R^\mu_{\nu \alpha \beta} = \partial_\alpha \Gamma^\mu_{\beta \nu} - \partial_\beta \Gamma^\mu_{\alpha \nu} + \Gamma^\mu_{\alpha \lambda} \Gamma^\lambda_{\beta \nu} - \Gamma^\mu_{\beta \lambda} \Gamma^\lambda_{\alpha \nu} \]
Since $\Gamma^\mu_{\alpha \beta}$ is already first-order in $h_{\alpha\beta}$ then only the first two terms of the Riemann tensor contribute to first-order in $h_{\alpha \beta}$ so we find,
\begin{align*}
R_{\mu \nu \alpha \beta} & = \tfrac{1}{2} \partial_\alpha (\partial_\beta h_{\mu \nu} + \partial_\nu h_{\beta \mu} - \partial_\mu h_{\beta \nu} ) - \tfrac{1}{2} \partial_\beta (\partial_\alpha h_{\mu \nu} + \partial_\nu h_{\alpha \mu} - \partial_\mu h_{\alpha \nu} ) 
\\
& = \tfrac{1}{2} \partial_\alpha \partial_\nu (h_{\beta \mu} - h_{\alpha \mu} ) - \tfrac{1}{2} \partial_\mu (\partial_\alpha h_{\beta \nu} - \partial_\beta h_{\alpha \nu} ) 
\end{align*}
Now consider,
\begin{align*}
R_{i0j0} = \tfrac{1}{2} \partial_j \partial_0 (h_{0i} - h_{ji}) - \tfrac{1}{2} \partial_i (\partial_j h_{00} - \partial_0 h_{j0}) = - \tfrac{1}{2} \partial_i \partial_j h_{00} 
\end{align*}
because we assume that the metric is stationary so $\partial_0 h_{\alpha \beta} = 0$. Now recall that we define the Newtonian potential via,
\[ h_{00} = - \frac{2 \Phi}{c^2} \]
Therefore,
\[ R_{i0j0} = \frac{1}{c^2} \frac{\partial^2 \Phi}{\partial x^j \partial x^j} \]
Now consider the equation for Geodesic deviation,
\[ A_\alpha = R_{\alpha \sigma \mu \beta} u^\sigma u^\mu S^\beta \]
In the regime $v^i \ll c$ we have $\gamma \approx 1$ and thus $u^0 \approx 0$ and $u^i \approx v^i \ll c$. Therefore, taking the spatial components, in the Newtonian limit the Geodesic deviation reduces to,
\[ A^i = R_{i00j} u^0 u^0 S^j = - R_{i0j0} c^2 S^j =  - \frac{\partial^2 \Phi}{\partial x^j \partial x^j} S^j \]
Furthermore,
\begin{align*}
A^i = T^\alpha \nabla_\alpha (T^\beta \nabla_\beta S^i) & = T^\alpha \nabla_\alpha \left( \pderiv{S^i}{t} + \Gamma^i_{\alpha \beta} T^\alpha S^\beta \right)
\\
& = \parsq{S^i}{t} + \pderiv{}{t} \left( \Gamma^i_{\alpha \beta} T^\alpha S^\beta \right) + \Gamma^i_{\gamma \delta} T^\gamma \left( \pderiv{S^\delta}{t} + \Gamma^\delta_{\alpha \beta} T^\alpha S^\beta \right) 
\end{align*}
However, $\Gamma^i_{\alpha \beta}$ is suppressed by a factor of $c^{-2}$ and each Christoffel symbol is only paired with one $T^\alpha$ of which $T^0 = u^0 = c$ is the dominant term. Therefore, to leading order in $c$ we have,
\[ A^i = \parsq{S^i}{t} \]
Putting everything together, we find,
\[ \parsq{S^i}{t} = - \frac{\partial^2 \Phi}{\partial x^j \partial x^j} S^j \]

\section*{9. }

Consider Minkowski space in cylindrical coordinates $(t, r, \phi, z)$ with metric,
\[ \d{s^2} = - c^2 \d{t^2} + \d{r^2} + r^2 \d{\phi^2} + \d{z^2} \]
Under the coordinate transformation $\phi = \tilde{\phi} + \Omega t$ we get co-rotating coordinates which have the metric,
\begin{align*}
\d{s^2} & = - c^2 \d{t^2} + \d{r^2} + r^2 (\d{\tilde{\phi}} + \Omega \d{t})^2 + \d{z^2}
\\
& \text{thus}
\\
\d{s^2} & = - c^2(1 - r^2 \Omega^2 c^{-2}) \d{t^2} + 2 r \Omega \: \d{\tilde{\phi}} \d{t} + \d{r^2} + r^2 \d{\tilde{\phi}^2} + \d{z^2}
\end{align*} 
This metric is stationary with respect to $t$ i.e. $\partial_t g_{\alpha \beta} = 0$ so we have a time-like killing field 
\[ K^\mu = (1, 0, 0, 0) \]
In the original non-rotating coordinates, this vector-field has components,
\[ K^\mu_{\text{inertial}} = (1, 0, \Omega, 0) \]
which is a helical Killing field combining the time-translation invariance and $z$-rotational invariance of Minkowski space. Now we can compute the metric of the corresponding 3D space, 
\[ \gamma_{ij} = - \frac{g_{0i} g_{0j}}{g_{00}} + g_{ij} = 
\begin{pmatrix}
1 & 0 & 0 
\\
0 & \frac{r^2}{1 - r^2 \Omega^2 / c^2} & 0 
\\
0 & 0 & 1
\end{pmatrix} \]
Therefore, the measured lengths are,
\[ \d{\ell^2} = \d{r^2} + \frac{r^2 \d{\tilde{\phi}^2}}{1 - \left( \frac{r^2 \Omega^2}{c^2} \right)^2} + \d{z^2} \]
Therefore, a circle of radius $r$ with origin $r = 0$ in the plane $ = 0$ has circumference,
\[ C = \int_0^{2 \pi} \sqrt{\d{r^2} + \frac{r^2 \d{\tilde{\phi}^2}}{1 - \left( \frac{r^2 \Omega^2}{c^2} \right)^2} + \d{z^2}} = \int_0^{2\pi} \frac{r \d{\phi}}{\sqrt{1 - \left( \frac{r^2 \Omega^2}{c^2} \right)^2}} = \frac{2 \pi r}{\sqrt{1 - \left( \frac{r^2 \Omega^2}{c^2} \right)^2}} \]
therefore we have a ratio,
\[ \frac{2 \pi r}{C} = \sqrt{1 - \left( \frac{r^2 \Omega^2}{c^2} \right)^2} \]
Which returns to the flat (non-relativistic) value of $1$ in the limit $\Omega r \ll c$ of slow rotations.  


\end{document}