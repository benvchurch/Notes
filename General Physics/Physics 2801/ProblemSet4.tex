\documentclass[11pt]{article}
\usepackage{geometry} % see geometry.pdf on how to lay out the page. There's lots.
\geometry{a4paper} % or letter or a5paper or ... etc
% \geometry{landscape} % rotated page geometry
\usepackage{amsmath}
\usepackage{graphicx}
\usepackage{breqn}
\usepackage{xfrac} %for the \sfrac command

\setcounter{MaxMatrixCols}{10}

\title{Phys 2801 Problem Set 4 Solutions}
\author{Laura Havener and Michael Clark \\ Edited: Benjamin Church}
\date{September 2017}

\begin{document}

\maketitle

\section*{Problem 1 Kleppner and Kolenkow 3.4}

The first realization to make is that the projectiles take the same amount of time to reach the peak of their trajectory as they do to fall to earth again. This reduces it to a one dimensional conservation of momentum problem. We will call the total mass $M$, so that the smaller projectile has a mass of $M/4$ and the larger one has mass $3M/4$. To land back at the launching station, the velocity of the smaller piece must be the negative of the velocity $V$ of the pair just before the explosion. We can then apply conservation of momentum to find $V'$, the velocity of the larger piece.
$$p_{tot} = MV = -\frac{M}{4}V + \frac{3M}{4}V'$$
Solving for $V'$ gives $V' = \frac{5}{3}V$, meaning the large fragment will travel $\frac{5}{3}$ as far in the second half of its flight. Thus the total distance $X$  travelled by the larger piece is
$$X = \frac{8}{3}L$$


\section*{Problem 2. Kleppner and Kolenkow 3.7}

This problem is split into two parts. First, block 2 accelerates in the $x$ direction until it gets to the equilibrium position. At this point there will no longer be a force holding $m_1$ against the stop, and the system will be carried forward by the momentum gained by $m_2$. After this time there will be oscillation between the two masses but the center of mass will move at constant velocity since there are no external forces on the system along the $x$ axis.\\
\indent
If $x_2$ is the coordinate of $m_2$ from its equilibrium position, then Hooke's law gives us $m_2 \ddot{x}_2 = -k x_2$. We have seen that the solution to this differential equation is $x_2 = A\cos{\omega t} + B\sin{\omega t}$ where $\omega \equiv \sqrt{\frac{k}{m_2}}$. Since $x_2(0) = -l/2$ and $\dot{x}_2(0) = 0$, we can set $A$ and $B$ to get
$$x_2(t) = -\frac{l}{2}\cos{\omega t}$$
until $m_2$ reaches the equilibrium position, when $x_2 = 0$. This occurs when $\cos{\omega t} = 0$, i.e. when $t = \frac{\pi}{2\omega}$. The total  momentum at this point is $m_2 \dot{x}_2 = \frac{1}{2}m_2 l \omega$.\\
\indent
The position of the center of mass relative to the wall is

$$x_{cm} = \frac{m_1 x_1 + m_2 (l + x_2)}{m_1 + m_2}$$

Note that the position of the 2nd mass is at $l + x_2$, since $x_2$ indicates the position of the 2nd block relative to the equilibrium position, which is $l$ to the right of the wall. Thus

$$x_{cm}(t) = \frac{m_2 l}{m_1+m_2} \left(1 - \frac{1}{2}\cos{\omega t}\right) \indent \textrm{for} \:t \leq \frac{\pi}{2\omega}$$

Once $t \geq \frac{\pi}{2\omega}$, the system will be moving with the momentum we found earlier, $\frac{1}{2} m_2 l \omega$, so its velocity $\dot{x}_{cm} = \frac{ m_2 l \omega}{2(m_1+m_2)}$. Now we can write $x_{cm}$ in the form of $x(t) = x_0 + v_0 (t - t_0)$:

\begin{eqnarray*}
x_{cm}(t) &=& \frac{m_2 l}{m_1 + m_2} + \frac{m_2 l \omega}{2(m_1+m_2)}\left(t - \frac{\pi}{2 \omega} \right)\\
&=& \frac{m_2 l}{m_1+m_2}\left(1 + \frac{1}{2}\omega t - \frac{\pi}{4} \right) \indent \textrm{for} \: t \geq \frac{\pi}{2\omega}
\end{eqnarray*}

\section*{Problem 3. Kleppner and Kolenkow 3.12}

We will use $M(t)$ and $v(t)$ to refer to the car's mass and velocity respectively as functions of time, $u = 5 \, \sfrac{\textrm{m}}{\textrm{s}}$ as the relative velocity of the sand to the car, and $\mu \equiv \frac{dm}{dt} = 10 \, \sfrac{\textrm{kg}}{\textrm{s}}$ as the rate of flow of mass. Since the rate of flow of mass is constant, $M(t) = M_0 + \mu t$, where $M_0$ is given to be $2000 \, \textrm{kg}$. At time $t$, a small mass of sand $\Delta m$ with velocity $v + u$ collides with the car of mass $M(T)$ and velocity $v(t)$. A small time $\Delta t$ later, the combinded system has momentum $M(t + \Delta t) v(t + \Delta t)$. Keeping only 1st order terms and taking the $\Delta t \rightarrow 0$ limit when appropriate, we have by conservation of momentum

\begin{eqnarray*}
M(t+\Delta t)v(t+\Delta t) &=& M(t)v(t) + \Delta m (v(t) + u)\\
(M(t) + \Delta m)(v(t) + \Delta v) &=& M(t)v(t) + v(t) \Delta m + u \Delta m\\
M(t) \Delta v &=& u \Delta m\\
M(t) \frac{\Delta v}{\Delta t} &=& u \frac{\Delta m}{\Delta t}\\
\frac{dv}{dt} &=& \frac{u \mu}{M(t)}\\
\int_0^{v(t)} dv' &=& u \mu \int_0^t \frac{dt'}{M_0 + \mu t'}\\
v(t) &=& u \ln{\left(1 + \frac{\mu t}{M_0}\right)}
\end{eqnarray*}

Along the way, we used the fact that $\frac{\Delta m}{\Delta t} \rightarrow \frac{dm}{dt} = \mu$ and $M(t) = M_0 + \mu t$. You probably did not have to be this rigorous (with all the $\Delta m$, $\Delta v$ business) to solve this problem, but I wanted to illustrate a general method that can be applied to trickier problems. Plugging in numerically, we find that

$$v(100 \, \textrm{s}) = (5 \, \sfrac{\textrm{m}}{\textrm{s}}) \ln{\left(1 + \frac{1}{2} \right)} \approx 2.0 \, \sfrac{\textrm{m}}{\textrm{s}}$$

\section*{Problem 4. Kleppner and Kolenkow 3.13}

First we will evaluate the average force the tow must apply over a single skier's trip to the top. The skier has mass $m = 70 \, \textrm{kg}$ and the slope is at an angle $\theta = 20^\circ$. The tow is active over a distance $L = 100 \, \textrm{m}$ and pulls at a constant velocity of $v = 1.5 \, \sfrac{\textrm{m}}{\textrm{s}}$. The temptation is to say there is only a force $mg \sin \theta$ needed to maintain constant velocity to the top. However, the wording of the problem implies that the skiier must be accelerated \emph{from rest} but leaves the top of the tow with momentum $mv$. To clarify the notation: in this problem, ``$\Delta$'' before a quantity does not imply that it will be taken to an infinitesimal limit - merely that it represents a net change in that quantity. We have in general

$$\Delta \mathbf{p} = \int_{\mathbf{p}_1}^{\mathbf{p}_2} d\mathbf{p} = \int_{t_1}^{t_2} \frac{d\mathbf{p}}{dt} dt = \int_{t_1}^{t_2} \mathbf{F}_{\textrm{net}} dt = \bar{\mathbf{F}}_{\textrm{net}} \Delta t$$
where $\Delta t = t_2 - t_1$ is the time period we are averaging over and the bar over $\bar{\mathbf{F}}_{\textrm{net}}$ indicates that it is the average net force. Working along an axis parallel to and pointing up the slope, we apply this to our problem and find

$$\bar{F}_{\textrm{net}} = \bar{F}_{1} - mg \sin{\theta}$$
where $\bar{F}_{1}$ is the average force applied by the tow for a single skier. We do not need to average the gravitational component because it is constant. Now, since $\Delta p = mv$, we have

$$mv = \Delta p_ = \bar{F}_{\textrm{net}} \Delta t = (\bar{F}_{1} - mg \sin{\theta}) \Delta t$$
so therefore
$$\bar{F}_{1} = mg \sin{\theta} + \frac{mv}{\Delta t}$$

Note that the time over which the skier accelerates to velocity $v$ does not play a role in our answer, because we are merely interested in the \emph{average} force over the entire trip to the top. The time taken to reach the top, $\Delta t$, is given by $L/v$.\\
\indent
Finally, we can find the average number of skiers $N$ on the tow at any one time by dividing the trip time $\Delta t$ by the average time it takes a new skier to latch on, $\tau = 5 \, \textrm{s}$. This gives $N = \frac{\Delta t}{\tau} = \frac{L}{v\tau}$, and since the total average force applied by the tow $\bar{F}_{\textrm{tow}} = N\bar{F}_{1}$, we obtain

$$\bar{F}_{\textrm{tow}} = \frac{L}{v\tau}\left(\frac{mv}{\Delta t} + mg \sin{\theta} \right) = \frac{m}{\tau} \left( v + \frac{gL}{v} \sin{\theta}\right)$$
After mashing our fingers against our calculators (or more probably, typing into Wolfram Alpha), we arrive at our final answer:
$$\bar{F}_{\textrm{tow}} = 3150 \, \textrm{N}$$

\section*{Problem 5. Kleppner and Kolenkow 3.14}

Since the problem does not specify, we will assume that the flatcar begins at rest. If it were not at rest, we could always transform to the inertial frame in which it is stationary, and if we cared, add its initial velocity to our answers.

\subsection*{a}

Since the passengers all jump off simultaneously, we need only apply one conservation of momentum argument. Assume that the flatcar of mass $M$ gains velocity $\Delta V$ and the people gain velocity $\Delta v$. By conservation of momentum,
\[M\Delta V + N m \Delta v = 0\]
Now for the slightly tricky part. $\Delta v \neq u$ as one might assume. Actually, $\Delta V - \Delta v = u$ because it is the relative velocity. 
Therefore,
\[M \Delta V + N m (\Delta V - u) = 0 \]
so \[\Delta V = u \frac{N m}{M + Nm} \]
\subsection*{b}

This problem has no more physics content than the first, it is just repeated multiple times. After the $n^{\text{th}}$ person has jumped off let the flatcar have velocity $V_n$ with $V_0 = 0$. Now, let $\Delta V_n = V_n - V_{n-1}$. Using the conservation of momentum as the $n^{\text{th}}$ person jumps off the car, \[(M + (N - n) m) V_n + m v = (M + (N - n + 1) m) V_{n-1}\]   
Therefore, 
 \[(M + (N - n) m) \Delta V_n + m (v - V_{n-1}) = 0\] 
As before, $V_{n} - v = u$ so 
 \[(M + (N - n) m) \Delta V_n + m (V_{n} - V_{n-1} - u) = 0\]
Combining terms,  
 \[(M + (N - n + 1) m) \Delta V_n = m u \]
and therefore,
\[\Delta V_n = u \frac{m}{M + (N - n + 1)m}\]
Now to find $V_N$ we simply sum the changes in velocity. 
\[V_N = \sum_{n = 0}^{N} \Delta V_n = \sum_{n = 0}^{N} \frac{mu}{M + (N - n + 1)m} \]
\subsection*{c}
We can rewrite the velocity found in case 2 in a suggestive way,
\[V_N =  \sum_{n = 0}^{N} \frac{mu}{M + (N - n + 1)m} = u \frac{Nm}{M + Nm} \frac{1}{N} \sum_{n = 0}^{N} \frac{M + N m}{M + (N - n + 1)m}\]
which we recognize as equal to the velocity in the case that everyon jumps off at the same time multiplied by the average of $\frac{M + N m}{M + (N - n + 1)m}$ from $n = 0$ to $N$. However, the numerator is always greater than the denominator so the term has an average larger than $1$. Therefore, the second case gives a greater final velocity to the flatcar. 

This is a very subtle problem and there are many different ways to interperet this result. I will give the best one I can think of. Suppose that we have a fixed amount of mass to kick out the back of our vehicle and we want to maximize the final velocity we get. Thus, since the total mass is fixed, we want to maximize the momentum kick we from get each individual mass per unit mass that is expelled. For a single lump, using the formul from part (a),
\[\Delta P_M  = M \Delta V = \frac{Mm}{M + m} u\] which we notice is the reduced mass. We are interested in \[\frac{\Delta P_M}{m} = \frac{M}{M + m} u\] This expression is maximized as $m \to 0$. Therefore, to make each kick as efficient as possible at delivering momentum to the vehicle, we want each kick to be as small as possible. Therefore, breakig up a large kick into small kicks will give a greater final velocty. In effect, we want as much of the relative velocity to go into the momentum of the expelled mass.  

\section*{Problem 6. Kleppner and Kolenkow 3.15}

\subsection*{a}
If $x(t)$ denotes the length of rope hanging through the hole at time $t$, let us call $d(t) = l - x(t)$ the length of rope lying on the table. Denote the mass of a length $x$ of rope by $m(x) = \frac{x}{l}M$ where $M$ and $l$ are the mass and length of the entire rope. The rope hanging through the hole experiences a downwards gravitational pull of $F = m(x(t))g$ and an upwards tension force $F_T$. The rope above the table is pulled through the hole by $F_T$. We then have
$$F_{\textrm{x,net}} = m(x)\ddot{x} = m(x)g - F_T$$
for the rope beneath the table and
$$F_{\textrm{d,net}} = m(d)\ddot{d} = -m(d)\ddot{x} = -F_T$$
Combining these two yields
\begin{eqnarray*}
m(x)\ddot{x} &=& m(x)g - m(d)\ddot{x}\\
(m(x) + m(d))\ddot{x} &=& \frac{x}{l}Mg\\
\ddot{x} &=& \gamma^2 x
\end{eqnarray*}
with $\gamma^2 \equiv \frac{g}{l}$. We know this differential equation has exponential solutions \\
$$x(t) = A e^{\gamma t} + B e^{-\gamma t}$$
where the constants $A$ and $B$ are arbitray without defining initial conditions. \\
b. Once we do apply the initial conditions, $x(0) = l_0$ and $\dot{x}(0) = 0$, we get two equations for $A$ and $B$ \\
\begin{eqnarray*}
A + B &=& l_0\\
\gamma A - \gamma B &=& 0
\end{eqnarray*} \\
and we find that 
$$A = B = \frac{l_0}{2}$$
Plugging back into our expression for $x(t)$ we can write
$$x(t) = \frac{l_0}{2}e^{\gamma t} + \frac{l_0}{2}e^{-\gamma t} = l_0 \cosh{\gamma t}$$

\section*{Problem 7. Kleppner and Kolenkow 3.18}

This problem is analyzing a raindrop that is gathering mass as it falls through a rain cloud. We have how the mass changes with velocity and what the initial mass is. We want to find the velocity of the drop with respect to time and then determine the terminal velocity of the drop. We should start newton's 2nd law for varying mass (change in momentum). \\
\begin{eqnarray*}
dp &=& Fdt \\
p(t) &=& mv \\
p(t+dt) &=& (m+dm)(v+dv) \\
dp &=& mv+dmv+mdv+dmdv-mv = mdv+vdm \\
 mdv+vdm &=& \sum{F}dt = mgdt \\
m\frac{dv}{dt}+v\frac{dm}{dt} &=& mg \\
\frac{dm}{dt} &=& kmv \\
kmv^{2}+\frac{dv}{dt}m &=& mg \\
\frac{dv}{dt} &=& g-kv^{2} \\
\frac{dv}{dt} &=& -k(v^{2}-g/k) \\
\frac{dv}{v^{2}-g/k} &=& -kdt \\
\int_{0}^{v}\frac{1}{(v')^{2}-g/k}\,dv' &=& -\int_{0}^{t}k\,dt' \\
\frac{1}{v^{2}-g/k} &=& \frac{A}{v-\sqrt{g/k}} +\frac{B}{v+\sqrt{g/k}} \\
0 &=& A+B \\
1 &=& \sqrt{g/k}(A-B) \\
B &=& -\sqrt{k/g} \\
\int_{0}^{v}(\frac{\sqrt{k/g}}{v'-\sqrt{g/k}}-\frac{\sqrt{k/g}}{v'+\sqrt{g/k}})\,dv' &=& -kt \\
u1 &=& v-\sqrt{g/k} \\
du1 &=& dv \\
u2 &=& v+\sqrt{g/k} \\
du2 &=& dv \\
\sqrt{k/g}\int_{0}^{v}(\frac{1}{u1}-\frac{1}{u2})\,du &=& -kt \\
(ln(v'-\sqrt{g/k})-ln(v'+\sqrt{g/k}))|_{0}^{v} &=& -kt\sqrt{g/k} \\
ln(\frac{v-\sqrt{g/k}}{-\sqrt{g/k}})-ln(\frac{v+\sqrt{g/k}}{\sqrt{g/k}}) &=& -\sqrt{kg}t \\
ln(\frac{\sqrt{g/k}-v}{\sqrt{g/k}+v}) &=& -\sqrt{kg}t \\
\frac{\sqrt{g/k}-v}{\sqrt{g/k}+v} &=& e^{-\sqrt{kg}t} \\
(\sqrt{g/k}-v)e^{-\sqrt{kg}t} &=& (\sqrt{g/k}+v)e^{-\sqrt{kg}t} \\
-v(1+e^{-\sqrt{kg}t}) &=& \sqrt{g/k}(e^{-\sqrt{kg}t}-1) \\
v(t)  &=& \sqrt{g/k}\frac{1-e^{-\sqrt{kg}t}}{1+e^{-\sqrt{kg}t}} = \sqrt{g/k}tanh(\sqrt{kg}t) \\
v(t->\infty) &=& \sqrt{g/k} \\
v_{t} &=& \sqrt{g/k} 
\end{eqnarray*} \\
The terminal velocity is in the limit as time approaches infinity. Thus, the velocity reaching the value found above eventually and keeps falling at that speed, similar to what happens on a falling object under air resistance. \\ \\

\section*{Problem 8. Kleppner and Kolenkow 3.20}

This problem is analyzing the motion of a rocket undergoing a gravitational force and air resistance. The equation for the motion of a rocket was derived in the book so lets use that. We are given the rate that mass is being excelled ($\gamma$) and the velocity of the exhaust. \\
\begin{eqnarray*}
P(t) &=& (m(t)+dm)v(t) \\
P(t+dt) &=& m(v+dv)+dm(v-u)\\
dP &=& P(t+dt)-P(t) = mdv-udm = Fdt = (-mg-mbv)dt \\
m\frac{dv}{dt}-u\frac{dm}{dt} &=& -mg-mbv \\
\frac{dm}{dt} &=& \gamma{m} \\
m\frac{dv}{dt} -mu\gamma &=& -mg -mbv \\
\frac{dv}{dt} &=& -g -bv+u\gamma \\
\frac{dv}{dt} &=& -b(v+g/b-u\gamma/b) \\
\frac{dv}{v+g/b-u\gamma/b} &=& -bdt \\
\int_{0}^{v}\frac{1}{v'+g/b-u\gamma/b}\,dv' &=& -\int_{0}^{t}b\,dt' \\
u &=& v'+g/b-u\gamma/b \\
du &=& dv' \\
\int_{0}^{v}\frac{1}{u}\,du &=& -bt \\
 ln(v'+g/b-u\gamma/b)|_{0}^{v} &=& -bt \\
ln(\frac{v+g/b-u\gamma/b}{g/b-u\gamma/b}) &=& -bt \\
\frac{v+g/b-u\gamma/b}{g/b-u\gamma/b} &=& e^{-bt} \\ 
v+g/b-u\gamma/b &=& (g/b-u\gamma/b)e^{-bt} \\
v(t) &=& (u\gamma/b-g/b)(1-e^{-bt}) 
\end{eqnarray*} \\
Now check the hint giving is the book for the terminal velocity. \\
\begin{eqnarray*} 
v_{t} &=& v(t->\infty) = u\gamma/b-g/b
\end{eqnarray*} \\

\section*{Problem 9. Center of Mass: Cone}

We want to calculate the center of mass of a cone with radius $R_{c}$ and length R. The equation for the center of mass is given below. \\ 
\begin{eqnarray*} 
\vec{r}_{CM} &=& \frac{\int{\rho}\vec{r}\,dV}{\int{\rho}\,dV} 
\end{eqnarray*} \\
The symmetry of the cone about the y and z axis tells us that $y_{CM}=0$ and $z_{CM}=0$. Therefore, we only need to determine the $x_{CM}$. The differential volume can be written as $dV=A(x)dx$ with A(x) as the area of the face of a slice of the cone. This will depend on the variable y since that is in the radial direction of the cone, but we can use proportions on the cone to determine how y depends on x. \\
\begin{eqnarray*} 
x_{CM} &=& \frac{\int{\rho}xA(x)\,dx}{\int{\rho}A(x)\,dx} \\
A(x) &=& \pi{y^{2}} \\
\frac{R_{c}}{R} &=& \frac{y}{x} \\
y &=& \frac{R_{c}}{R}x 
\end{eqnarray*} \\
Now we are at the point where we can perform the integral. Start with the integral on the denominator since it is a little easier. The integrands will be x from 0 to $R$. \\
\begin{eqnarray*} 
\int_{0}^{R}\rho{}A(x')\,dx' &=& \int_{0}^{R}\rho\pi\frac{R_{c}^{2}}{R^{2}}(x')^{2}\,dx' \\
&=& \frac{R_{c}^{2}}{R^{2}}\int_{0}^{R}\rho\pi(x')^{2}\,dx' \\
&=& \rho\pi\frac{R_{c}^{2}}{R^{2}}\frac{1}{3}(x')^{3}|_{0}^{R} \\
&=&\rho\pi\frac{R_{c}^{2}}{3R^{2}}R^{3} = \rho\pi\frac{R_{c}^{2}R}{3} 
\end{eqnarray*} \\
Now do the integral in the numerator. \\ 
\begin{eqnarray*}
\int_{0}^{R}\rho{x}A(x')\,dx' &=& \int_{0}^{R_{c}}\rho\pi\frac{R_{c}^{2}}{R^{2}}(x')^{3}\,dx' \\
&=& \rho\pi\frac{R_{c}^{2}}{R^{2}}\frac{1}{4}(x')^{4}|_{0}^{R} \\
&=&  \rho\pi\frac{R_{c}^{2}}{R^{2}}\frac{1}{4}R^{4} = \rho\pi\frac{R_{c}^{2}R^{2}}{4}
\end{eqnarray*} \\
Now lets divide them and find the position of the center of mass of the cone in the x direction. \\
\begin{eqnarray*}
x_{CM} &=& \frac{\rho\pi\frac{R_{c}^{2}R^{2}}{4}}{\rho\pi\frac{R_{c}^{2}R}{3}} \\
x_{CM} &=&\frac{3}{4}R \\
\vec{r}_{CM} &=& (\frac{3}{4}R, 0, 0) 
\end{eqnarray*} \\

\end{document}

