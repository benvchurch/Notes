\documentclass[11pt]{amsart}
\usepackage{geometry} % see geometry.pdf on how to lay out the page. There's lots.
\geometry{a4paper} % or letter or a5paper or ... etc
% \geometry{landscape} % rotated page geometry
\usepackage{amsmath}
\usepackage{graphicx}
\usepackage{breqn}

\setcounter{MaxMatrixCols}{10}

\flushbottom
\chardef\atcode=\catcode`\@
\makeatletter
\@addtoreset{figure}{section}
\@addtoreset{table}{section}
\renewcommand{\figurename}{Figure}
\renewcommand{\tablename}{Table}
\setcounter{topnumber}{3}               % orig: 2
\setcounter{totalnumber}{4}             % orig: 3
\renewcommand{\textfraction}{0}         
\renewcommand{\bottomfraction}{0.65}    
\renewcommand{\topfraction}{0.75}       
\renewcommand{\floatpagefraction}{0.75} 
\catcode`\@=\atcode 
\newcommand{\grad}{$^\circ$}
\newcommand{\gradm}{^\circ}
\newcommand{\bqn}{ \begin{eqnarray} }
\newcommand{\eqn}{ \end{eqnarray} }
\newcommand{\beq}{ \begin{equation} }
\newcommand{\eeq}{ \end{equation} }
\setlength{\baselineskip}{2.1ex}
\renewcommand{\baselinestretch}{1.06}
\setlength{\parskip}{1.5ex plus 0.8ex minus 0.6ex}
\setlength{\evensidemargin}{0.3cm}
\setlength{\oddsidemargin}{-0.3cm}
\setlength{\topmargin}{-1cm}
\setlength{\textwidth}{17 cm}
\setlength{\textheight}{26cm}
\newcommand{\mat}[1]{\mbox{$\underline{\underline{#1}}$}}
\newcommand{\etal}{\mbox{\sl et al.}}
\renewcommand{\refname}{}
\newcommand{\vol}[1]{{\bf{#1}}}
\newcommand{\dg}{$^\circ\;$}
\def\D{\displaystyle}
\newcommand{\lapprox}{\ensuremath{<\atop{\mbox{\raisebox{0.5ex}{$\sim$}}}}}
\parindent 0cm
% Macros for Scientific Word 2.5 documents saved with the LaTeX filter.
%Copyright (C) 1994-95 TCI Software Research, Inc.
\typeout{TCILATEX Macros for Scientific Word 2.5 <22 Dec 95>.}
\typeout{NOTICE:  This macro file is NOT proprietary and may be 
freely copied and distributed.}
%
\makeatletter
%
%%%%%%%%%%%%%%%%%%%%%%
% macros for time
\newcount\@hour\newcount\@minute\chardef\@x10\chardef\@xv60
\def\tcitime{
\def\@time{%
  \@minute\time\@hour\@minute\divide\@hour\@xv
  \ifnum\@hour<\@x 0\fi\the\@hour:%
  \multiply\@hour\@xv\advance\@minute-\@hour
  \ifnum\@minute<\@x 0\fi\the\@minute
  }}%

%%%%%%%%%%%%%%%%%%%%%%
% macro for hyperref
\@ifundefined{hyperref}{\def\hyperref#1#2#3#4{#2\ref{#4}#3}}{}

% macro for external program call
\@ifundefined{qExtProgCall}{\def\qExtProgCall#1#2#3#4#5#6{\relax}}{}
%%%%%%%%%%%%%%%%%%%%%%
%
% macros for graphics
%
\def\FILENAME#1{#1}%
%
\def\QCTOpt[#1]#2{%
  \def\QCTOptB{#1}
  \def\QCTOptA{#2}
}
\def\QCTNOpt#1{%
  \def\QCTOptA{#1}
  \let\QCTOptB\empty
}
\def\Qct{%
  \@ifnextchar[{%
    \QCTOpt}{\QCTNOpt}
}
\def\QCBOpt[#1]#2{%
  \def\QCBOptB{#1}
  \def\QCBOptA{#2}
}
\def\QCBNOpt#1{%
  \def\QCBOptA{#1}
  \let\QCBOptB\empty
}
\def\Qcb{%
  \@ifnextchar[{%
    \QCBOpt}{\QCBNOpt}
}
\def\PrepCapArgs{%
  \ifx\QCBOptA\empty
    \ifx\QCTOptA\empty
      {}%
    \else
      \ifx\QCTOptB\empty
        {\QCTOptA}%
      \else
        [\QCTOptB]{\QCTOptA}%
      \fi
    \fi
  \else
    \ifx\QCBOptA\empty
      {}%
    \else
      \ifx\QCBOptB\empty
        {\QCBOptA}%
      \else
        [\QCBOptB]{\QCBOptA}%
      \fi
    \fi
  \fi
}
\newcount\GRAPHICSTYPE
%\GRAPHICSTYPE 0 is for TurboTeX
%\GRAPHICSTYPE 1 is for DVIWindo (PostScript)
%%%(removed)%\GRAPHICSTYPE 2 is for psfig (PostScript)
\GRAPHICSTYPE=\z@
\def\GRAPHICSPS#1{%
 \ifcase\GRAPHICSTYPE%\GRAPHICSTYPE=0
   \special{ps: #1}%
 \or%\GRAPHICSTYPE=1
   \special{language "PS", include "#1"}%
%%%\or%\GRAPHICSTYPE=2
%%%  #1%
 \fi
}%
%
\def\GRAPHICSHP#1{\special{include #1}}%
%
% \graffile{ body }                                  %#1
%          { contentswidth (scalar)  }               %#2
%          { contentsheight (scalar) }               %#3
%          { vertical shift when in-line (scalar) }  %#4
\def\graffile#1#2#3#4{%
%%% \ifnum\GRAPHICSTYPE=\tw@
%%%  %Following if using psfig
%%%  \@ifundefined{psfig}{\input psfig.tex}{}%
%%%  \psfig{file=#1, height=#3, width=#2}%
%%% \else
  %Following for all others
  % JCS - added BOXTHEFRAME, see below
    \leavevmode
    \raise -#4 \BOXTHEFRAME{%
        \hbox to #2{\raise #3\hbox to #2{\null #1\hfil}}}%
}%
%
% A box for drafts
\def\draftbox#1#2#3#4{%
 \leavevmode\raise -#4 \hbox{%
  \frame{\rlap{\protect\tiny #1}\hbox to #2%
   {\vrule height#3 width\z@ depth\z@\hfil}%
  }%
 }%
}%
%
\newcount\draft
\draft=\z@
\let\nographics=\draft
\newif\ifwasdraft
\wasdraftfalse

%  \GRAPHIC{ body }                                  %#1
%          { draft name }                            %#2
%          { contentswidth (scalar)  }               %#3
%          { contentsheight (scalar) }               %#4
%          { vertical shift when in-line (scalar) }  %#5
\def\GRAPHIC#1#2#3#4#5{%
 \ifnum\draft=\@ne\draftbox{#2}{#3}{#4}{#5}%
  \else\graffile{#1}{#3}{#4}{#5}%
  \fi
 }%
%
\def\addtoLaTeXparams#1{%
    \edef\LaTeXparams{\LaTeXparams #1}}%
%
% JCS -  added a switch BoxFrame that can 
% be set by including X in the frame params.
% If set a box is drawn around the frame.

\newif\ifBoxFrame \BoxFramefalse
\newif\ifOverFrame \OverFramefalse
\newif\ifUnderFrame \UnderFramefalse

\def\BOXTHEFRAME#1{%
   \hbox{%
      \ifBoxFrame
         \frame{#1}%
      \else
         {#1}%
      \fi
   }%
}


\def\doFRAMEparams#1{\BoxFramefalse\OverFramefalse\UnderFramefalse\readFRAMEparams#1\end}%
\def\readFRAMEparams#1{%
 \ifx#1\end%
  \let\next=\relax
  \else
  \ifx#1i\dispkind=\z@\fi
  \ifx#1d\dispkind=\@ne\fi
  \ifx#1f\dispkind=\tw@\fi
  \ifx#1t\addtoLaTeXparams{t}\fi
  \ifx#1b\addtoLaTeXparams{b}\fi
  \ifx#1p\addtoLaTeXparams{p}\fi
  \ifx#1h\addtoLaTeXparams{h}\fi
  \ifx#1X\BoxFrametrue\fi
  \ifx#1O\OverFrametrue\fi
  \ifx#1U\UnderFrametrue\fi
  \ifx#1w
    \ifnum\draft=1\wasdrafttrue\else\wasdraftfalse\fi
    \draft=\@ne
  \fi
  \let\next=\readFRAMEparams
  \fi
 \next
 }%
%
%Macro for In-line graphics object
%   \IFRAME{ contentswidth (scalar)  }               %#1
%          { contentsheight (scalar) }               %#2
%          { vertical shift when in-line (scalar) }  %#3
%          { draft name }                            %#4
%          { body }                                  %#5
%          { caption}                                %#6


\def\IFRAME#1#2#3#4#5#6{%
      \bgroup
      \let\QCTOptA\empty
      \let\QCTOptB\empty
      \let\QCBOptA\empty
      \let\QCBOptB\empty
      #6%
      \parindent=0pt%
      \leftskip=0pt
      \rightskip=0pt
      \setbox0 = \hbox{\QCBOptA}%
      \@tempdima = #1\relax
      \ifOverFrame
          % Do this later
          \typeout{This is not implemented yet}%
          \show\HELP
      \else
         \ifdim\wd0>\@tempdima
            \advance\@tempdima by \@tempdima
            \ifdim\wd0 >\@tempdima
               \textwidth=\@tempdima
               \setbox1 =\vbox{%
                  \noindent\hbox to \@tempdima{\hfill\GRAPHIC{#5}{#4}{#1}{#2}{#3}\hfill}\\%
                  \noindent\hbox to \@tempdima{\parbox[b]{\@tempdima}{\QCBOptA}}%
               }%
               \wd1=\@tempdima
            \else
               \textwidth=\wd0
               \setbox1 =\vbox{%
                 \noindent\hbox to \wd0{\hfill\GRAPHIC{#5}{#4}{#1}{#2}{#3}\hfill}\\%
                 \noindent\hbox{\QCBOptA}%
               }%
               \wd1=\wd0
            \fi
         \else
            %\show\BBB
            \ifdim\wd0>0pt
              \hsize=\@tempdima
              \setbox1 =\vbox{%
                \unskip\GRAPHIC{#5}{#4}{#1}{#2}{0pt}%
                \break
                \unskip\hbox to \@tempdima{\hfill \QCBOptA\hfill}%
              }%
              \wd1=\@tempdima
           \else
              \hsize=\@tempdima
              \setbox1 =\vbox{%
                \unskip\GRAPHIC{#5}{#4}{#1}{#2}{0pt}%
              }%
              \wd1=\@tempdima
           \fi
         \fi
         \@tempdimb=\ht1
         \advance\@tempdimb by \dp1
         \advance\@tempdimb by -#2%
         \advance\@tempdimb by #3%
         \leavevmode
         \raise -\@tempdimb \hbox{\box1}%
      \fi
      \egroup%
}%
%
%Macro for Display graphics object
%   \DFRAME{ contentswidth (scalar)  }               %#1
%          { contentsheight (scalar) }               %#2
%          { draft label }                           %#3
%          { name }                                  %#4
%          { caption}                                %#5
\def\DFRAME#1#2#3#4#5{%
 \begin{center}
     \let\QCTOptA\empty
     \let\QCTOptB\empty
     \let\QCBOptA\empty
     \let\QCBOptB\empty
     \ifOverFrame 
        #5\QCTOptA\par
     \fi
     \GRAPHIC{#4}{#3}{#1}{#2}{\z@}
     \ifUnderFrame 
        \nobreak\par #5\QCBOptA
     \fi
 \end{center}%
 }%
%
%Macro for Floating graphic object
%   \FFRAME{ framedata f|i tbph x F|T }              %#1
%          { contentswidth (scalar)  }               %#2
%          { contentsheight (scalar) }               %#3
%          { caption }                               %#4
%          { label }                                 %#5
%          { draft name }                            %#6
%          { body }                                  %#7
\def\FFRAME#1#2#3#4#5#6#7{%
 \begin{figure}[#1]%
  \let\QCTOptA\empty
  \let\QCTOptB\empty
  \let\QCBOptA\empty
  \let\QCBOptB\empty
  \ifOverFrame
    #4
    \ifx\QCTOptA\empty
    \else
      \ifx\QCTOptB\empty
        \caption{\QCTOptA}%
      \else
        \caption[\QCTOptB]{\QCTOptA}%
      \fi
    \fi
    \ifUnderFrame\else
      \label{#5}%
    \fi
  \else
    \UnderFrametrue%
  \fi
  \begin{center}\GRAPHIC{#7}{#6}{#2}{#3}{\z@}\end{center}%
  \ifUnderFrame
    #4
    \ifx\QCBOptA\empty
      \caption{}%
    \else
      \ifx\QCBOptB\empty
        \caption{\QCBOptA}%
      \else
        \caption[\QCBOptB]{\QCBOptA}%
      \fi
    \fi
    \label{#5}%
  \fi
  \end{figure}%
 }%
%
%
%    \FRAME{ framedata f|i tbph x F|T }              %#1
%          { contentswidth (scalar)  }               %#2
%          { contentsheight (scalar) }               %#3
%          { vertical shift when in-line (scalar) }  %#4
%          { caption }                               %#5
%          { label }                                 %#6
%          { name }                                  %#7
%          { body }                                  %#8
%
%    framedata is a string which can contain the following
%    characters: idftbphxFT
%    Their meaning is as follows:
%             i, d or f : in-line, display, or floating
%             t,b,p,h   : LaTeX floating placement options
%             x         : fit contents box to contents
%             F or T    : Figure or Table. 
%                         Later this can expand
%                         to a more general float class.
%
%
\newcount\dispkind%

\def\makeactives{
  \catcode`\"=\active
  \catcode`\;=\active
  \catcode`\:=\active
  \catcode`\'=\active
  \catcode`\~=\active
}
\bgroup
   \makeactives
   \gdef\activesoff{%
      \def"{\string"}
      \def;{\string;}
      \def:{\string:}
      \def'{\string'}
      \def~{\string~}
      %\bbl@deactivate{"}%
      %\bbl@deactivate{;}%
      %\bbl@deactivate{:}%
      %\bbl@deactivate{'}%
    }
\egroup

\def\FRAME#1#2#3#4#5#6#7#8{%
 \bgroup
 \@ifundefined{bbl@deactivate}{}{\activesoff}
 \ifnum\draft=\@ne
   \wasdrafttrue
 \else
   \wasdraftfalse%
 \fi
 \def\LaTeXparams{}%
 \dispkind=\z@
 \def\LaTeXparams{}%
 \doFRAMEparams{#1}%
 \ifnum\dispkind=\z@\IFRAME{#2}{#3}{#4}{#7}{#8}{#5}\else
  \ifnum\dispkind=\@ne\DFRAME{#2}{#3}{#7}{#8}{#5}\else
   \ifnum\dispkind=\tw@
    \edef\@tempa{\noexpand\FFRAME{\LaTeXparams}}%
    \@tempa{#2}{#3}{#5}{#6}{#7}{#8}%
    \fi
   \fi
  \fi
  \ifwasdraft\draft=1\else\draft=0\fi{}%
  \egroup
 }%
%
% This macro added to let SW gobble a parameter that
% should not be passed on and expanded. 

\def\TEXUX#1{"texux"}

%
% Macros for text attributes:
%
\def\BF#1{{\bf {#1}}}%
\def\NEG#1{\leavevmode\hbox{\rlap{\thinspace/}{$#1$}}}%
%
%%%%%%%%%%%%%%%%%%%%%%%%%%%%%%%%%%%%%%%%%%%%%%%%%%%%%%%%%%%%%%%%%%%%%%%%
%
%
% macros for user - defined functions
\def\func#1{\mathop{\rm #1}}%
\def\limfunc#1{\mathop{\rm #1}}%

%
% miscellaneous 
%\long\def\QQQ#1#2{}%
\long\def\QQQ#1#2{%
     \long\expandafter\def\csname#1\endcsname{#2}}%
%\def\QTP#1{}% JCS - this was changed becuase style editor will define QTP
\@ifundefined{QTP}{\def\QTP#1{}}{}
\@ifundefined{QEXCLUDE}{\def\QEXCLUDE#1{}}{}
%\@ifundefined{Qcb}{\def\Qcb#1{#1}}{}
%\@ifundefined{Qct}{\def\Qct#1{#1}}{}
\@ifundefined{Qlb}{\def\Qlb#1{#1}}{}
\@ifundefined{Qlt}{\def\Qlt#1{#1}}{}
\def\QWE{}%
\long\def\QQA#1#2{}%
%\def\QTR#1#2{{\em #2}}% Always \em!!!
%\def\QTR#1#2{\mbox{\begin{#1}#2\end{#1}}}%cb%%%
\def\QTR#1#2{{\csname#1\endcsname #2}}%(gp) Is this the best?
\long\def\TeXButton#1#2{#2}%
\long\def\QSubDoc#1#2{#2}%
\def\EXPAND#1[#2]#3{}%
\def\NOEXPAND#1[#2]#3{}%
\def\PROTECTED{}%
\def\LaTeXparent#1{}%
\def\ChildStyles#1{}%
\def\ChildDefaults#1{}%
\def\QTagDef#1#2#3{}%
%
% Macros for style editor docs
\@ifundefined{StyleEditBeginDoc}{\def\StyleEditBeginDoc{\relax}}{}
%
% Macros for footnotes
\def\QQfnmark#1{\footnotemark}
\def\QQfntext#1#2{\addtocounter{footnote}{#1}\footnotetext{#2}}
%
% Macros for indexing.
\def\MAKEINDEX{\makeatletter\input gnuindex.sty\makeatother\makeindex}%	
\@ifundefined{INDEX}{\def\INDEX#1#2{}{}}{}%
\@ifundefined{SUBINDEX}{\def\SUBINDEX#1#2#3{}{}{}}{}%
\@ifundefined{initial}%  
   {\def\initial#1{\bigbreak{\raggedright\large\bf #1}\kern 2\p@\penalty3000}}%
   {}%
\@ifundefined{entry}{\def\entry#1#2{\item {#1}, #2}}{}%
\@ifundefined{primary}{\def\primary#1{\item {#1}}}{}%
\@ifundefined{secondary}{\def\secondary#1#2{\subitem {#1}, #2}}{}%
%
%
\@ifundefined{ZZZ}{}{\MAKEINDEX\makeatletter}%
%
% Attempts to avoid problems with other styles
\@ifundefined{abstract}{%
 \def\abstract{%
  \if@twocolumn
   \section*{Abstract (Not appropriate in this style!)}%
   \else \small 
   \begin{center}{\bf Abstract\vspace{-.5em}\vspace{\z@}}\end{center}%
   \quotation 
   \fi
  }%
 }{%
 }%
\@ifundefined{endabstract}{\def\endabstract
  {\if@twocolumn\else\endquotation\fi}}{}%
\@ifundefined{maketitle}{\def\maketitle#1{}}{}%
\@ifundefined{affiliation}{\def\affiliation#1{}}{}%
\@ifundefined{proof}{\def\proof{\noindent{\bfseries Proof. }}}{}%
\@ifundefined{endproof}{\def\endproof{\mbox{\ \rule{.1in}{.1in}}}}{}%
\@ifundefined{newfield}{\def\newfield#1#2{}}{}%
\@ifundefined{chapter}{\def\chapter#1{\par(Chapter head:)#1\par }%
 \newcount\c@chapter}{}%
\@ifundefined{part}{\def\part#1{\par(Part head:)#1\par }}{}%
\@ifundefined{section}{\def\section#1{\par(Section head:)#1\par }}{}%
\@ifundefined{subsection}{\def\subsection#1%
 {\par(Subsection head:)#1\par }}{}%
\@ifundefined{subsubsection}{\def\subsubsection#1%
 {\par(Subsubsection head:)#1\par }}{}%
\@ifundefined{paragraph}{\def\paragraph#1%
 {\par(Subsubsubsection head:)#1\par }}{}%
\@ifundefined{subparagraph}{\def\subparagraph#1%
 {\par(Subsubsubsubsection head:)#1\par }}{}%
%%%%%%%%%%%%%%%%%%%%%%%%%%%%%%%%%%%%%%%%%%%%%%%%%%%%%%%%%%%%%%%%%%%%%%%%
% These symbols are not recognized by LaTeX
\@ifundefined{therefore}{\def\therefore{}}{}%
\@ifundefined{backepsilon}{\def\backepsilon{}}{}%
\@ifundefined{yen}{\def\yen{\hbox{\rm\rlap=Y}}}{}%
\@ifundefined{registered}{%
   \def\registered{\relax\ifmmode{}\r@gistered
                    \else$\m@th\r@gistered$\fi}%
 \def\r@gistered{^{\ooalign
  {\hfil\raise.07ex\hbox{$\scriptstyle\rm\text{R}$}\hfil\crcr
  \mathhexbox20D}}}}{}%
\@ifundefined{Eth}{\def\Eth{}}{}%
\@ifundefined{eth}{\def\eth{}}{}%
\@ifundefined{Thorn}{\def\Thorn{}}{}%
\@ifundefined{thorn}{\def\thorn{}}{}%
% A macro to allow any symbol that requires math to appear in text
\def\TEXTsymbol#1{\mbox{$#1$}}%
\@ifundefined{degree}{\def\degree{{}^{\circ}}}{}%
%
% macros for T3TeX files
\newdimen\theight
\def\Column{%
 \vadjust{\setbox\z@=\hbox{\scriptsize\quad\quad tcol}%
  \theight=\ht\z@\advance\theight by \dp\z@\advance\theight by \lineskip
  \kern -\theight \vbox to \theight{%
   \rightline{\rlap{\box\z@}}%
   \vss
   }%
  }%
 }%
%
\def\qed{%
 \ifhmode\unskip\nobreak\fi\ifmmode\ifinner\else\hskip5\p@\fi\fi
 \hbox{\hskip5\p@\vrule width4\p@ height6\p@ depth1.5\p@\hskip\p@}%
 }%
%
\def\cents{\hbox{\rm\rlap/c}}%
\def\miss{\hbox{\vrule height2\p@ width 2\p@ depth\z@}}%
%\def\miss{\hbox{.}}%        %another possibility 
%
\def\vvert{\Vert}%           %always translated to \left| or \right|
%
\def\tcol#1{{\baselineskip=6\p@ \vcenter{#1}} \Column}  %
%
\def\dB{\hbox{{}}}%                 %dummy entry in column 
\def\mB#1{\hbox{$#1$}}%             %column entry
\def\nB#1{\hbox{#1}}%               %column entry (not math)
%
%\newcount\notenumber
%\def\clearnotenumber{\notenumber=0}
%\def\note{\global\advance\notenumber by 1
% \footnote{$^{\the\notenumber}$}}
%\def\note{\global\advance\notenumber by 1
\def\note{$^{\dag}}%
%
%

\def\newfmtname{LaTeX2e}
\def\chkcompat{%
   \if@compatibility
   \else
     \usepackage{latexsym}
   \fi
}

\ifx\fmtname\newfmtname
  \DeclareOldFontCommand{\rm}{\normalfont\rmfamily}{\mathrm}
  \DeclareOldFontCommand{\sf}{\normalfont\sffamily}{\mathsf}
  \DeclareOldFontCommand{\tt}{\normalfont\ttfamily}{\mathtt}
  \DeclareOldFontCommand{\bf}{\normalfont\bfseries}{\mathbf}
  \DeclareOldFontCommand{\it}{\normalfont\itshape}{\mathit}
  \DeclareOldFontCommand{\sl}{\normalfont\slshape}{\@nomath\sl}
  \DeclareOldFontCommand{\sc}{\normalfont\scshape}{\@nomath\sc}
  \chkcompat
\fi

%
% Greek bold macros
% Redefine all of the math symbols 
% which might be bolded	 - there are 
% probably others to add to this list

\def\alpha{\Greekmath 010B }%
\def\beta{\Greekmath 010C }%
\def\gamma{\Greekmath 010D }%
\def\delta{\Greekmath 010E }%
\def\epsilon{\Greekmath 010F }%
\def\zeta{\Greekmath 0110 }%
\def\eta{\Greekmath 0111 }%
\def\theta{\Greekmath 0112 }%
\def\iota{\Greekmath 0113 }%
\def\kappa{\Greekmath 0114 }%
\def\lambda{\Greekmath 0115 }%
\def\mu{\Greekmath 0116 }%
\def\nu{\Greekmath 0117 }%
\def\xi{\Greekmath 0118 }%
\def\pi{\Greekmath 0119 }%
\def\rho{\Greekmath 011A }%
\def\sigma{\Greekmath 011B }%
\def\tau{\Greekmath 011C }%
\def\upsilon{\Greekmath 011D }%
\def\phi{\Greekmath 011E }%
\def\chi{\Greekmath 011F }%
\def\psi{\Greekmath 0120 }%
\def\omega{\Greekmath 0121 }%
\def\varepsilon{\Greekmath 0122 }%
\def\vartheta{\Greekmath 0123 }%
\def\varpi{\Greekmath 0124 }%
\def\varrho{\Greekmath 0125 }%
\def\varsigma{\Greekmath 0126 }%
\def\varphi{\Greekmath 0127 }%

\def\nabla{\Greekmath 0272 }
\def\FindBoldGroup{%
   {\setbox0=\hbox{$\mathbf{x\global\edef\theboldgroup{\the\mathgroup}}$}}%
}

\def\Greekmath#1#2#3#4{%
    \if@compatibility
        \ifnum\mathgroup=\symbold
           \mathchoice{\mbox{\boldmath$\displaystyle\mathchar"#1#2#3#4$}}%
                      {\mbox{\boldmath$\textstyle\mathchar"#1#2#3#4$}}%
                      {\mbox{\boldmath$\scriptstyle\mathchar"#1#2#3#4$}}%
                      {\mbox{\boldmath$\scriptscriptstyle\mathchar"#1#2#3#4$}}%
        \else
           \mathchar"#1#2#3#4% 
        \fi 
    \else 
        \FindBoldGroup
        \ifnum\mathgroup=\theboldgroup % For 2e
           \mathchoice{\mbox{\boldmath$\displaystyle\mathchar"#1#2#3#4$}}%
                      {\mbox{\boldmath$\textstyle\mathchar"#1#2#3#4$}}%
                      {\mbox{\boldmath$\scriptstyle\mathchar"#1#2#3#4$}}%
                      {\mbox{\boldmath$\scriptscriptstyle\mathchar"#1#2#3#4$}}%
        \else
           \mathchar"#1#2#3#4% 
        \fi     	    
	  \fi}

\newif\ifGreekBold  \GreekBoldfalse
\let\SAVEPBF=\pbf
\def\pbf{\GreekBoldtrue\SAVEPBF}%
%

\@ifundefined{theorem}{\newtheorem{theorem}{Theorem}}{}
\@ifundefined{lemma}{\newtheorem{lemma}[theorem]{Lemma}}{}
\@ifundefined{corollary}{\newtheorem{corollary}[theorem]{Corollary}}{}
\@ifundefined{conjecture}{\newtheorem{conjecture}[theorem]{Conjecture}}{}
\@ifundefined{proposition}{\newtheorem{proposition}[theorem]{Proposition}}{}
\@ifundefined{axiom}{\newtheorem{axiom}{Axiom}}{}
\@ifundefined{remark}{\newtheorem{remark}{Remark}}{}
\@ifundefined{example}{\newtheorem{example}{Example}}{}
\@ifundefined{exercise}{\newtheorem{exercise}{Exercise}}{}
\@ifundefined{definition}{\newtheorem{definition}{Definition}}{}


\@ifundefined{mathletters}{%
  %\def\theequation{\arabic{equation}}
  \newcounter{equationnumber}  
  \def\mathletters{%
     \addtocounter{equation}{1}
     \edef\@currentlabel{\theequation}%
     \setcounter{equationnumber}{\c@equation}
     \setcounter{equation}{0}%
     \edef\theequation{\@currentlabel\noexpand\alph{equation}}%
  }
  \def\endmathletters{%
     \setcounter{equation}{\value{equationnumber}}%
  }
}{}

%Logos
\@ifundefined{BibTeX}{%
    \def\BibTeX{{\rm B\kern-.05em{\sc i\kern-.025em b}\kern-.08em
                 T\kern-.1667em\lower.7ex\hbox{E}\kern-.125emX}}}{}%
\@ifundefined{AmS}%
    {\def\AmS{{\protect\usefont{OMS}{cmsy}{m}{n}%
                A\kern-.1667em\lower.5ex\hbox{M}\kern-.125emS}}}{}%
\@ifundefined{AmSTeX}{\def\AmSTeX{\protect\AmS-\protect\TeX\@}}{}%
%

%%%%%%%%%%%%%%%%%%%%%%%%%%%%%%%%%%%%%%%%%%%%%%%%%%%%%%%%%%%%%%%%%%%%%%%
% NOTE: The rest of this file is read only if amstex has not been
% loaded.  This section is used to define amstex constructs in the
% event they have not been defined.
%
%
\ifx\ds@amstex\relax
   \message{amstex already loaded}\makeatother\endinput% 2.09 compatability
\else
   \@ifpackageloaded{amstex}%
      {\message{amstex already loaded}\makeatother\endinput}
      {}
   \@ifpackageloaded{amsgen}%
      {\message{amsgen already loaded}\makeatother\endinput}
      {}
\fi
%%%%%%%%%%%%%%%%%%%%%%%%%%%%%%%%%%%%%%%%%%%%%%%%%%%%%%%%%%%%%%%%%%%%%%%%
%%
%
%
%  Macros to define some AMS LaTeX constructs when 
%  AMS LaTeX has not been loaded
% 
% These macros are copied from the AMS-TeX package for doing
% multiple integrals.
%
\let\DOTSI\relax
\def\RIfM@{\relax\ifmmode}%
\def\FN@{\futurelet\next}%
\newcount\intno@
\def\iint{\DOTSI\intno@\tw@\FN@\ints@}%
\def\iiint{\DOTSI\intno@\thr@@\FN@\ints@}%
\def\iiiint{\DOTSI\intno@4 \FN@\ints@}%
\def\idotsint{\DOTSI\intno@\z@\FN@\ints@}%
\def\ints@{\findlimits@\ints@@}%
\newif\iflimtoken@
\newif\iflimits@
\def\findlimits@{\limtoken@true\ifx\next\limits\limits@true
 \else\ifx\next\nolimits\limits@false\else
 \limtoken@false\ifx\ilimits@\nolimits\limits@false\else
 \ifinner\limits@false\else\limits@true\fi\fi\fi\fi}%
\def\multint@{\int\ifnum\intno@=\z@\intdots@                          %1
 \else\intkern@\fi                                                    %2
 \ifnum\intno@>\tw@\int\intkern@\fi                                   %3
 \ifnum\intno@>\thr@@\int\intkern@\fi                                 %4
 \int}%                                                               %5
\def\multintlimits@{\intop\ifnum\intno@=\z@\intdots@\else\intkern@\fi
 \ifnum\intno@>\tw@\intop\intkern@\fi
 \ifnum\intno@>\thr@@\intop\intkern@\fi\intop}%
\def\intic@{%
    \mathchoice{\hskip.5em}{\hskip.4em}{\hskip.4em}{\hskip.4em}}%
\def\negintic@{\mathchoice
 {\hskip-.5em}{\hskip-.4em}{\hskip-.4em}{\hskip-.4em}}%
\def\ints@@{\iflimtoken@                                              %1
 \def\ints@@@{\iflimits@\negintic@
   \mathop{\intic@\multintlimits@}\limits                             %2
  \else\multint@\nolimits\fi                                          %3
  \eat@}%                                                             %4
 \else                                                                %5
 \def\ints@@@{\iflimits@\negintic@
  \mathop{\intic@\multintlimits@}\limits\else
  \multint@\nolimits\fi}\fi\ints@@@}%
\def\intkern@{\mathchoice{\!\!\!}{\!\!}{\!\!}{\!\!}}%
\def\plaincdots@{\mathinner{\cdotp\cdotp\cdotp}}%
\def\intdots@{\mathchoice{\plaincdots@}%
 {{\cdotp}\mkern1.5mu{\cdotp}\mkern1.5mu{\cdotp}}%
 {{\cdotp}\mkern1mu{\cdotp}\mkern1mu{\cdotp}}%
 {{\cdotp}\mkern1mu{\cdotp}\mkern1mu{\cdotp}}}%
%
%
%  These macros are for doing the AMS \text{} construct
%
\def\RIfM@{\relax\protect\ifmmode}
\def\text{\RIfM@\expandafter\text@\else\expandafter\mbox\fi}
\let\nfss@text\text
\def\text@#1{\mathchoice
   {\textdef@\displaystyle\f@size{#1}}%
   {\textdef@\textstyle\tf@size{\firstchoice@false #1}}%
   {\textdef@\textstyle\sf@size{\firstchoice@false #1}}%
   {\textdef@\textstyle \ssf@size{\firstchoice@false #1}}%
   \glb@settings}

\def\textdef@#1#2#3{\hbox{{%
                    \everymath{#1}%
                    \let\f@size#2\selectfont
                    #3}}}
\newif\iffirstchoice@
\firstchoice@true
%
%    Old Scheme for \text
%
%\def\rmfam{\z@}%
%\newif\iffirstchoice@
%\firstchoice@true
%\def\textfonti{\the\textfont\@ne}%
%\def\textfontii{\the\textfont\tw@}%
%\def\text{\RIfM@\expandafter\text@\else\expandafter\text@@\fi}%
%\def\text@@#1{\leavevmode\hbox{#1}}%
%\def\text@#1{\mathchoice
% {\hbox{\everymath{\displaystyle}\def\textfonti{\the\textfont\@ne}%
%  \def\textfontii{\the\textfont\tw@}\textdef@@ T#1}}%
% {\hbox{\firstchoice@false
%  \everymath{\textstyle}\def\textfonti{\the\textfont\@ne}%
%  \def\textfontii{\the\textfont\tw@}\textdef@@ T#1}}%
% {\hbox{\firstchoice@false
%  \everymath{\scriptstyle}\def\textfonti{\the\scriptfont\@ne}%
%  \def\textfontii{\the\scriptfont\tw@}\textdef@@ S\rm#1}}%
% {\hbox{\firstchoice@false
%  \everymath{\scriptscriptstyle}\def\textfonti
%  {\the\scriptscriptfont\@ne}%
%  \def\textfontii{\the\scriptscriptfont\tw@}\textdef@@ s\rm#1}}}%
%\def\textdef@@#1{\textdef@#1\rm\textdef@#1\bf\textdef@#1\sl
%    \textdef@#1\it}%
%\def\DN@{\def\next@}%
%\def\eat@#1{}%
%\def\textdef@#1#2{%
% \DN@{\csname\expandafter\eat@\string#2fam\endcsname}%
% \if S#1\edef#2{\the\scriptfont\next@\relax}%
% \else\if s#1\edef#2{\the\scriptscriptfont\next@\relax}%
% \else\edef#2{\the\textfont\next@\relax}\fi\fi}%
%
%
%These are the AMS constructs for multiline limits.
%
\def\Let@{\relax\iffalse{\fi\let\\=\cr\iffalse}\fi}%
\def\vspace@{\def\vspace##1{\crcr\noalign{\vskip##1\relax}}}%
\def\multilimits@{\bgroup\vspace@\Let@
 \baselineskip\fontdimen10 \scriptfont\tw@
 \advance\baselineskip\fontdimen12 \scriptfont\tw@
 \lineskip\thr@@\fontdimen8 \scriptfont\thr@@
 \lineskiplimit\lineskip
 \vbox\bgroup\ialign\bgroup\hfil$\m@th\scriptstyle{##}$\hfil\crcr}%
\def\Sb{_\multilimits@}%
\def\endSb{\crcr\egroup\egroup\egroup}%
\def\Sp{^\multilimits@}%
\let\endSp\endSb
%
%
%These are AMS constructs for horizontal arrows
%
\newdimen\ex@
\ex@.2326ex
\def\rightarrowfill@#1{$#1\m@th\mathord-\mkern-6mu\cleaders
 \hbox{$#1\mkern-2mu\mathord-\mkern-2mu$}\hfill
 \mkern-6mu\mathord\rightarrow$}%
\def\leftarrowfill@#1{$#1\m@th\mathord\leftarrow\mkern-6mu\cleaders
 \hbox{$#1\mkern-2mu\mathord-\mkern-2mu$}\hfill\mkern-6mu\mathord-$}%
\def\leftrightarrowfill@#1{$#1\m@th\mathord\leftarrow
\mkern-6mu\cleaders
 \hbox{$#1\mkern-2mu\mathord-\mkern-2mu$}\hfill
 \mkern-6mu\mathord\rightarrow$}%
\def\overrightarrow{\mathpalette\overrightarrow@}%
\def\overrightarrow@#1#2{\vbox{\ialign{##\crcr\rightarrowfill@#1\crcr
 \noalign{\kern-\ex@\nointerlineskip}$\m@th\hfil#1#2\hfil$\crcr}}}%
\let\overarrow\overrightarrow
\def\overleftarrow{\mathpalette\overleftarrow@}%
\def\overleftarrow@#1#2{\vbox{\ialign{##\crcr\leftarrowfill@#1\crcr
 \noalign{\kern-\ex@\nointerlineskip}$\m@th\hfil#1#2\hfil$\crcr}}}%
\def\overleftrightarrow{\mathpalette\overleftrightarrow@}%
\def\overleftrightarrow@#1#2{\vbox{\ialign{##\crcr
   \leftrightarrowfill@#1\crcr
 \noalign{\kern-\ex@\nointerlineskip}$\m@th\hfil#1#2\hfil$\crcr}}}%
\def\underrightarrow{\mathpalette\underrightarrow@}%
\def\underrightarrow@#1#2{\vtop{\ialign{##\crcr$\m@th\hfil#1#2\hfil
  $\crcr\noalign{\nointerlineskip}\rightarrowfill@#1\crcr}}}%
\let\underarrow\underrightarrow
\def\underleftarrow{\mathpalette\underleftarrow@}%
\def\underleftarrow@#1#2{\vtop{\ialign{##\crcr$\m@th\hfil#1#2\hfil
  $\crcr\noalign{\nointerlineskip}\leftarrowfill@#1\crcr}}}%
\def\underleftrightarrow{\mathpalette\underleftrightarrow@}%
\def\underleftrightarrow@#1#2{\vtop{\ialign{##\crcr$\m@th
  \hfil#1#2\hfil$\crcr
 \noalign{\nointerlineskip}\leftrightarrowfill@#1\crcr}}}%
%%%%%%%%%%%%%%%%%%%%%

% 94.0815 by Jon:

\def\qopnamewl@#1{\mathop{\operator@font#1}\nlimits@}
\let\nlimits@\displaylimits
\def\setboxz@h{\setbox\z@\hbox}


\def\varlim@#1#2{\mathop{\vtop{\ialign{##\crcr
 \hfil$#1\m@th\operator@font lim$\hfil\crcr
 \noalign{\nointerlineskip}#2#1\crcr
 \noalign{\nointerlineskip\kern-\ex@}\crcr}}}}

 \def\rightarrowfill@#1{\m@th\setboxz@h{$#1-$}\ht\z@\z@
  $#1\copy\z@\mkern-6mu\cleaders
  \hbox{$#1\mkern-2mu\box\z@\mkern-2mu$}\hfill
  \mkern-6mu\mathord\rightarrow$}
\def\leftarrowfill@#1{\m@th\setboxz@h{$#1-$}\ht\z@\z@
  $#1\mathord\leftarrow\mkern-6mu\cleaders
  \hbox{$#1\mkern-2mu\copy\z@\mkern-2mu$}\hfill
  \mkern-6mu\box\z@$}


\def\projlim{\qopnamewl@{proj\,lim}}
\def\injlim{\qopnamewl@{inj\,lim}}
\def\varinjlim{\mathpalette\varlim@\rightarrowfill@}
\def\varprojlim{\mathpalette\varlim@\leftarrowfill@}
\def\varliminf{\mathpalette\varliminf@{}}
\def\varliminf@#1{\mathop{\underline{\vrule\@depth.2\ex@\@width\z@
   \hbox{$#1\m@th\operator@font lim$}}}}
\def\varlimsup{\mathpalette\varlimsup@{}}
\def\varlimsup@#1{\mathop{\overline
  {\hbox{$#1\m@th\operator@font lim$}}}}

%
%%%%%%%%%%%%%%%%%%%%%%%%%%%%%%%%%%%%%%%%%%%%%%%%%%%%%%%%%%%%%%%%%%%%%
%
\def\tfrac#1#2{{\textstyle {#1 \over #2}}}%
\def\dfrac#1#2{{\displaystyle {#1 \over #2}}}%
\def\binom#1#2{{#1 \choose #2}}%
\def\tbinom#1#2{{\textstyle {#1 \choose #2}}}%
\def\dbinom#1#2{{\displaystyle {#1 \choose #2}}}%
\def\QATOP#1#2{{#1 \atop #2}}%
\def\QTATOP#1#2{{\textstyle {#1 \atop #2}}}%
\def\QDATOP#1#2{{\displaystyle {#1 \atop #2}}}%
\def\QABOVE#1#2#3{{#2 \above#1 #3}}%
\def\QTABOVE#1#2#3{{\textstyle {#2 \above#1 #3}}}%
\def\QDABOVE#1#2#3{{\displaystyle {#2 \above#1 #3}}}%
\def\QOVERD#1#2#3#4{{#3 \overwithdelims#1#2 #4}}%
\def\QTOVERD#1#2#3#4{{\textstyle {#3 \overwithdelims#1#2 #4}}}%
\def\QDOVERD#1#2#3#4{{\displaystyle {#3 \overwithdelims#1#2 #4}}}%
\def\QATOPD#1#2#3#4{{#3 \atopwithdelims#1#2 #4}}%
\def\QTATOPD#1#2#3#4{{\textstyle {#3 \atopwithdelims#1#2 #4}}}%
\def\QDATOPD#1#2#3#4{{\displaystyle {#3 \atopwithdelims#1#2 #4}}}%
\def\QABOVED#1#2#3#4#5{{#4 \abovewithdelims#1#2#3 #5}}%
\def\QTABOVED#1#2#3#4#5{{\textstyle 
   {#4 \abovewithdelims#1#2#3 #5}}}%
\def\QDABOVED#1#2#3#4#5{{\displaystyle 
   {#4 \abovewithdelims#1#2#3 #5}}}%
%
% Macros for text size operators:

%JCS - added braces and \mathop around \displaystyle\int, etc.
%
\def\tint{\mathop{\textstyle \int}}%
\def\tiint{\mathop{\textstyle \iint }}%
\def\tiiint{\mathop{\textstyle \iiint }}%
\def\tiiiint{\mathop{\textstyle \iiiint }}%
\def\tidotsint{\mathop{\textstyle \idotsint }}%
\def\toint{\mathop{\textstyle \oint}}%
\def\tsum{\mathop{\textstyle \sum }}%
\def\tprod{\mathop{\textstyle \prod }}%
\def\tbigcap{\mathop{\textstyle \bigcap }}%
\def\tbigwedge{\mathop{\textstyle \bigwedge }}%
\def\tbigoplus{\mathop{\textstyle \bigoplus }}%
\def\tbigodot{\mathop{\textstyle \bigodot }}%
\def\tbigsqcup{\mathop{\textstyle \bigsqcup }}%
\def\tcoprod{\mathop{\textstyle \coprod }}%
\def\tbigcup{\mathop{\textstyle \bigcup }}%
\def\tbigvee{\mathop{\textstyle \bigvee }}%
\def\tbigotimes{\mathop{\textstyle \bigotimes }}%
\def\tbiguplus{\mathop{\textstyle \biguplus }}%
%
%
%Macros for display size operators:
%

\def\dint{\mathop{\displaystyle \int}}%
\def\diint{\mathop{\displaystyle \iint }}%
\def\diiint{\mathop{\displaystyle \iiint }}%
\def\diiiint{\mathop{\displaystyle \iiiint }}%
\def\didotsint{\mathop{\displaystyle \idotsint }}%
\def\doint{\mathop{\displaystyle \oint}}%
\def\dsum{\mathop{\displaystyle \sum }}%
\def\dprod{\mathop{\displaystyle \prod }}%
\def\dbigcap{\mathop{\displaystyle \bigcap }}%
\def\dbigwedge{\mathop{\displaystyle \bigwedge }}%
\def\dbigoplus{\mathop{\displaystyle \bigoplus }}%
\def\dbigodot{\mathop{\displaystyle \bigodot }}%
\def\dbigsqcup{\mathop{\displaystyle \bigsqcup }}%
\def\dcoprod{\mathop{\displaystyle \coprod }}%
\def\dbigcup{\mathop{\displaystyle \bigcup }}%
\def\dbigvee{\mathop{\displaystyle \bigvee }}%
\def\dbigotimes{\mathop{\displaystyle \bigotimes }}%
\def\dbiguplus{\mathop{\displaystyle \biguplus }}%
%
%Companion to stackrel
\def\stackunder#1#2{\mathrel{\mathop{#2}\limits_{#1}}}%
%
%
% These are AMS environments that will be defined to
% be verbatims if amstex has not actually been 
% loaded
%
%
\begingroup \catcode `|=0 \catcode `[= 1
\catcode`]=2 \catcode `\{=12 \catcode `\}=12
\catcode`\\=12 
|gdef|@alignverbatim#1\end{align}[#1|end[align]]
|gdef|@salignverbatim#1\end{align*}[#1|end[align*]]

|gdef|@alignatverbatim#1\end{alignat}[#1|end[alignat]]
|gdef|@salignatverbatim#1\end{alignat*}[#1|end[alignat*]]

|gdef|@xalignatverbatim#1\end{xalignat}[#1|end[xalignat]]
|gdef|@sxalignatverbatim#1\end{xalignat*}[#1|end[xalignat*]]

|gdef|@gatherverbatim#1\end{gather}[#1|end[gather]]
|gdef|@sgatherverbatim#1\end{gather*}[#1|end[gather*]]

|gdef|@gatherverbatim#1\end{gather}[#1|end[gather]]
|gdef|@sgatherverbatim#1\end{gather*}[#1|end[gather*]]


|gdef|@multilineverbatim#1\end{multiline}[#1|end[multiline]]
|gdef|@smultilineverbatim#1\end{multiline*}[#1|end[multiline*]]

|gdef|@arraxverbatim#1\end{arrax}[#1|end[arrax]]
|gdef|@sarraxverbatim#1\end{arrax*}[#1|end[arrax*]]

|gdef|@tabulaxverbatim#1\end{tabulax}[#1|end[tabulax]]
|gdef|@stabulaxverbatim#1\end{tabulax*}[#1|end[tabulax*]]


|endgroup
  

  
\def\align{\@verbatim \frenchspacing\@vobeyspaces \@alignverbatim
You are using the "align" environment in a style in which it is not defined.}
\let\endalign=\endtrivlist
 
\@namedef{align*}{\@verbatim\@salignverbatim
You are using the "align*" environment in a style in which it is not defined.}
\expandafter\let\csname endalign*\endcsname =\endtrivlist




\def\alignat{\@verbatim \frenchspacing\@vobeyspaces \@alignatverbatim
You are using the "alignat" environment in a style in which it is not defined.}
\let\endalignat=\endtrivlist
 
\@namedef{alignat*}{\@verbatim\@salignatverbatim
You are using the "alignat*" environment in a style in which it is not defined.}
\expandafter\let\csname endalignat*\endcsname =\endtrivlist




\def\xalignat{\@verbatim \frenchspacing\@vobeyspaces \@xalignatverbatim
You are using the "xalignat" environment in a style in which it is not defined.}
\let\endxalignat=\endtrivlist
 
\@namedef{xalignat*}{\@verbatim\@sxalignatverbatim
You are using the "xalignat*" environment in a style in which it is not defined.}
\expandafter\let\csname endxalignat*\endcsname =\endtrivlist




\def\gather{\@verbatim \frenchspacing\@vobeyspaces \@gatherverbatim
You are using the "gather" environment in a style in which it is not defined.}
\let\endgather=\endtrivlist
 
\@namedef{gather*}{\@verbatim\@sgatherverbatim
You are using the "gather*" environment in a style in which it is not defined.}
\expandafter\let\csname endgather*\endcsname =\endtrivlist


\def\multiline{\@verbatim \frenchspacing\@vobeyspaces \@multilineverbatim
You are using the "multiline" environment in a style in which it is not defined.}
\let\endmultiline=\endtrivlist
 
\@namedef{multiline*}{\@verbatim\@smultilineverbatim
You are using the "multiline*" environment in a style in which it is not defined.}
\expandafter\let\csname endmultiline*\endcsname =\endtrivlist


\def\arrax{\@verbatim \frenchspacing\@vobeyspaces \@arraxverbatim
You are using a type of "array" construct that is only allowed in AmS-LaTeX.}
\let\endarrax=\endtrivlist

\def\tabulax{\@verbatim \frenchspacing\@vobeyspaces \@tabulaxverbatim
You are using a type of "tabular" construct that is only allowed in AmS-LaTeX.}
\let\endtabulax=\endtrivlist

 
\@namedef{arrax*}{\@verbatim\@sarraxverbatim
You are using a type of "array*" construct that is only allowed in AmS-LaTeX.}
\expandafter\let\csname endarrax*\endcsname =\endtrivlist

\@namedef{tabulax*}{\@verbatim\@stabulaxverbatim
You are using a type of "tabular*" construct that is only allowed in AmS-LaTeX.}
\expandafter\let\csname endtabulax*\endcsname =\endtrivlist

% macro to simulate ams tag construct


% This macro is a fix to eqnarray
\def\@@eqncr{\let\@tempa\relax
    \ifcase\@eqcnt \def\@tempa{& & &}\or \def\@tempa{& &}%
      \else \def\@tempa{&}\fi
     \@tempa
     \if@eqnsw
        \iftag@
           \@taggnum
        \else
           \@eqnnum\stepcounter{equation}%
        \fi
     \fi
     \global\tag@false
     \global\@eqnswtrue
     \global\@eqcnt\z@\cr}


% This macro is a fix to the equation environment
 \def\endequation{%
     \ifmmode\ifinner % FLEQN hack
      \iftag@
        \addtocounter{equation}{-1} % undo the increment made in the begin part
        $\hfil
           \displaywidth\linewidth\@taggnum\egroup \endtrivlist
        \global\tag@false
        \global\@ignoretrue   
      \else
        $\hfil
           \displaywidth\linewidth\@eqnnum\egroup \endtrivlist
        \global\tag@false
        \global\@ignoretrue 
      \fi
     \else   
      \iftag@
        \addtocounter{equation}{-1} % undo the increment made in the begin part
        \eqno \hbox{\@taggnum}
        \global\tag@false%
        $$\global\@ignoretrue
      \else
        \eqno \hbox{\@eqnnum}% $$ BRACE MATCHING HACK
        $$\global\@ignoretrue
      \fi
     \fi\fi
 } 

 \newif\iftag@ \tag@false
 
 \def\tag{\@ifnextchar*{\@tagstar}{\@tag}}
 \def\@tag#1{%
     \global\tag@true
     \global\def\@taggnum{(#1)}}
 \def\@tagstar*#1{%
     \global\tag@true
     \global\def\@taggnum{#1}%  
}

% Do not add anything to the end of this file.  
% The last section of the file is loaded only if 
% amstex has not been.



\makeatother
\endinput


% See the ``Article customise'' template for come common customisations

\title{Physics C2801 Fall 2013 Problem Set 9}
\author{Laura Havener}
\date{Nov 27} % delete this line to display the current date

%%% BEGIN DOCUMENT
\begin{document}


\maketitle

Problem 1. Lorentz transformations, Lorentz boost factors, non-relativistic limit \\
a. Find the lorentz boost factors for these values. \\ 
\begin{eqnarray*}
\beta_{B} &=& 0.1 \\
\gamma_{B} &=& \frac{1}{\sqrt{1-\beta_{B}^{2}}} = 1.00504 \\
\beta_{B} &=& 0.2 \\
\gamma_{B} &=& 1.02062 \\
\beta_{B} &=& 0.4 \\
\gamma_{B} &=& 1.09109 \\
\beta_{B} &=& 0.6 \\
\gamma_{B} &=& 1.25 \\
\beta_{B} &=& 0.8 \\
\gamma_{B} &=& 1.66667 \\
n &=& 1 \\
1-\beta_{B} &=& 1x10^{-n} = 0.1 \\
n &=& 2 \\
1-\beta_{B} &=&  0.01 \\
n &=& 3 \\
1-\beta_{B} &=&  0.001 \\
n &=& 4 \\
1-\beta_{B} &=&  0.0001 \\
n &=& 6 \\
1-\beta_{B} &=&  0.000001 \\
n &=& 8 \\
1-\beta_{B} &=&  0.00000001 \\
n &=& 10 \\
1-\beta_{B} &=&  0.0000000001 \\
\end{eqnarray*} \\
b. The lorentz transformations are the following. \\
\begin{eqnarray*} 
t' &=& \gamma(t-\frac{vx}{c^{2}}) \\
x' &=& \gamma(x-vt) \\
y' &=& y \\
z' &=& z \\
\gamma &=& \frac{1}{\sqrt{1-\frac{v^{2}}{c^{2}}}} \\
\end{eqnarray*} \\
Now make the approximation that $\frac{v}{c}<<1$. \\
\begin{eqnarray*} 
\gamma &=& (1-\frac{v^{2}}{c^{2}})^{-1/2} \approx 1+\frac{v^{2}}{2c^{2}} \\
t' &\approx& (1+\frac{v^{2}}{2c^{2}})(t-\frac{vx}{c^{2}}) = t+t\frac{v^{2}}{2c^{2}}-\frac{vx}{c^{2}}-\frac{v^{3}x}{2c^{4}} \\
&\approx& t-\frac{vx}{c^{2}} \\
x' &\approx& (1+\frac{v^{2}}{2c^{2}})(x-vt) \approx x-vt 
\end{eqnarray*} \\
The transformation in the time still has an extra factor, but if you think of how big $c^{2}$ will be and how small the x and v will be in the scales that we use Newtonian Mechanics, this term can be neglected in this limit. Thus, $t'=t$. \\
c. Now keep the second order terms and evaluate them for $v/c = 0.0000259$. \\
\begin{eqnarray*} 
t' &\approx& t -\frac{vx}{c^{2}}+t\frac{v^{2}}{2c^{2}} \\
x' &\approx& x-vt+\frac{xv^{2}}{2c^{2}}-\frac{tv^{3}}{2c^{2}} \\
t' &\approx& t-(9*10^{-14})x+(3.35*10^{-10})t \\
x' &\approx& x-(1285)t+(3.35*10^{-10})x-(4.3*10^{-7})t
\end{eqnarray*} \\
You can see that these terms are negligible in comparison with the first order terms. \\
d. Calculate the time intervals in proper units: \\
\begin{eqnarray*}
\Delta{t} &=& c\Delta{t} = (3*10^{8}\frac{m}{s})(1*10^{-n}s) \\
n &=& 0 \\
\Delta{t} &=& 3*10^{8}m \\
n &=& 3 \\
\Delta{t} &=& 3*10^{5}m \\
n &=& 6 \\
\Delta{t} &=& 3*10^{2}m \\
n &=& 9 \\
\Delta{t} &=& 3*10^{-1}m \\
n &=& 15 \\
\Delta{t} &=& 3*10^{-7}m \\
n &=& 21 \\
\Delta{t} &=& 3*10^{-13}m 
\end{eqnarray*} \\
The following physical time intervals correspond to the natural time intervals given. \\
\begin{eqnarray*} 
d &=& 1*10^{-10}m \\
\Delta{t} &=& 0.3*10^{-18}s \\
d &=& 1*10^{-14}m \\
\Delta{t} &=& 0.3*10^{-22}s \\
d &=& 1.5*10^{11}m \\
\Delta{t} &=& 0.5*10^{3}s \\
d &=& 3*10^{16}m \\
\Delta{t} &=& 1*10^{8}s \\
d &=& 3*10^{20}m \\
\Delta{t} &=& 1*10^{12}s
\end{eqnarray*} \\
e. We need to figure out what value of x will make the difference 1 percent between t' when the x term is included and when the x term isn't included. Find $\Delta{t}=(t'-t)/t=0.01$. \\
\begin{eqnarray*}
\Delta{t} &=& \frac{vx}{tc^{2}} = 0.01 \\
x &=& (1s)(0.01)(c^{2})/v = (0.01s)c/\beta \\
&=& 3.03*10^{8}m 
\end{eqnarray*} \\
f. Use the equation for lorentz transformations. The change in position in the particles rest frame will be 0. \\
\begin{eqnarray*} 
x' &=& \gamma(x-\beta{t}) \\
t &=& \gamma(\tau-\beta{x'}) \\
\tau &=& t/\gamma \\
x' &=& 0 \\
0 &=& \gamma(x-\beta{\gamma\tau}) \\
x &=& \gamma\beta\tau 
\end{eqnarray*} \\
g. The muon's lifetime in natural units will be $6.6*10^{2}$m. Use the equation from part f to calculate the distance that the muon would travel after it decayed. \\
\begin{eqnarray*} 
x &=& \gamma\beta\tau = \frac{1}{\sqrt{1-0.999^{2}}}(0.999)(6.6*10^{2}) \\
&=& 14700m 
\end{eqnarray*} \\

Problem 2. Relativistic transformation of velocities \\ \\
a. We want to calculate the transformation of velocities parallel to the boost direction with velocity  $\vec{\beta}=\beta\hat{i}=\frac{v}{c}\hat{i}$ and boost $\beta_{B}$.\\
\begin{eqnarray*}
\beta_{x}' &=& \frac{dx'}{dt'} \\
dx' &=& \gamma(dx-\beta_{B}dt) \\
dt' &=& \gamma(dt-\beta_{B}dx/c) \\
\frac{dx'}{dt'} &=& \frac{dx-\beta_{B}dt}{dt-\beta_{B}dx/c} \\
&=& \frac{\frac{dx}{dt}-\beta_{B}}{1-\beta_{B}\frac{dx}{dt}/c} \\
\beta &=& \frac{dx}{dt} \\
\beta_{x}' &=& \frac{\beta-\beta_{B}}{1-\beta_{B}\beta} 
\end{eqnarray*} \\
b. Now consider the case where $\beta<<1$ and $\beta_{B}<<1$. \\
\begin{eqnarray*}
\beta_{x}' &\approx& (\beta-\beta_{B})(1+\beta_{B}\beta) \\
&\approx& \beta-\beta_{B} \\
v_{x}' &\approx& v-v_{B} 
\end{eqnarray*} \\
c. Look at $\beta=\pm{1}$: \\
\begin{eqnarray*}
\beta &=& +1 \\
\beta' &=& \frac{1-\beta_{B}}{1-\beta_{B}} = 1 = c\\
\beta &=& -1 \\
\beta' &=&c\frac{1+\beta_{B}}{1+\beta_{B}} = 1 = c 
\end{eqnarray*} \\
d. Now look at the case $|\beta|<1$ and show there is always a frame where $\beta'=0$. If this is true then $\beta-\beta_{B}$ must be equal to 0. Therefore, $\beta=\beta_{B}$. \\
e. For the same case, $|\beta_{B}|<1$, show that $|\beta_{B}'|<1$. \\
\begin{eqnarray*} 
\beta' &=& \frac{\beta-\beta_{B}}{1-\beta\beta_{B}} \\
\beta'(1-\beta\beta_{B}) &=& \beta-\beta_{B} \\
\beta &=& \frac{\beta'-\beta_{B}}{1-\beta_{B}\beta'} < 1 \\
\beta'-\beta_{B} &<& 1-\beta_{B}\beta' \\
\beta' &<& \frac{1-\beta_{B}}{1-\beta_{B}} \\
\beta' &<& 1 
\end{eqnarray*} \\
f. We want to find $\gamma'$ for a lorentz boosted velocity. \\
\begin{eqnarray*} 
\beta' &=& \frac{\beta-\beta_{B}}{1-\beta_{B}\beta} \\
\gamma' &=& \frac{1}{\sqrt{1-(\frac{\beta-\beta_{B}}{1-\beta_{B}\beta})^{2}}} \\
&=& \frac{(1-\beta\beta_{B})^{2}}{\sqrt{(1-\beta\beta_{B})^{2}-(\beta-\beta_{B})^{2}}} \\
&=& \frac{1-\beta\beta_{B}}{\sqrt{1-2\beta\beta_{B}+\beta^{2}\beta_{B}^{2}-\beta^{2}-\beta_{B}^{2}+2\beta\beta_{B}}} \\
&=& \frac{1-\beta\beta_{B}}{\sqrt{1-\beta^{2}-\beta_{B}^{2}(1-\beta^{2})}} \\
\gamma' &=& \frac{1-\beta\beta_{B}}{\sqrt{(1-\beta^{2})(1-\beta_{B}^{2})}} 
\end{eqnarray*} \\
g. Now find the tranformation for the velocity in y and z. \\
\begin{eqnarray*} 
dy' &=& dy \\
dt' &=& \gamma(dt-\beta_{B}{dx}) \\
\beta_{y}' &=& \frac{dy}{\gamma(dt-\beta{dx})} = \frac{\frac{dy}{dt}}{\gamma(1-\beta_{B}\frac{dx}{dt})} \\
&=& \frac{\beta_{y}}{\gamma(1-\beta_{B}\beta_{x})} \\
\beta_{z}' &=& \frac{\beta_{z}}{\gamma(1-\beta_{B}\beta_{x})} 
\end{eqnarray*} \\

Problem 3. Kleppner and Kolenkow 12.4 \\ \\
a. We want to find how the angle tranforms between the S' and S frame. The speed of light is frame independent. Therefore, the x component of the velocity in the S' frame is $u_{x}'=c\cos{\theta_{0}}$ and the x component of velocity in the S frame is $u_{x}=c\cos{\theta}$. Then we can use the velocity transformations to determine how the angles tansform. \\
\begin{eqnarray*}
u_{x} &=& \frac{u_{x}'+v}{1+u_{x}'v/c^{2}} \\
c\cos{\theta} &=& \frac{c\cos{\theta_{0}}+v}{1+v\cos{\theta_{0}}/c} \\
\cos{\theta} &=& \frac{\cos{\theta_{0}}+v/c}{1+\cos{\theta_{0}}v/c} 
\end{eqnarray*} \\
b. Now we want to find the speed of a source that has half of its radiation in a cone subtending $\theta =10^{-3}$ radians. This is the angle that an observer in the S frame sees the cone at since the source is at rest in the S' frame which is moving at a velocity v relative to S. In the rest frame 50 percent of the radiation subtends a cone starting at 90 degrees ($\theta_{0}=\pi/2$) since it radiates equally in all directions. Thus we can use the transformation of angles equation from part a for the angles at the cone boundaries to see what the boost velocity needs to be. \\
\begin{eqnarray*}
cos(10^{-3}) &=& \frac{cos(\pi/2)+v/c}{1+cos(\pi/2)v/c} = \frac{0+v/c}{1+0} \\
v &=& c\cos{10^{-3}} \approx c(1-\frac{1}{2}(10^{-3})^{2}) \\
&=& c(1-5*10^{-7}) 
\end{eqnarray*} \\

Problem 4. Kleppner and Kolenkow 12.6 \\ \\
The observer in S' will see the rod length constracted. S' is moving at a speed $v'$ with respect to the S frame, thus the $\gamma$ factor will actually be $\gamma'$ which takes into account the boost. \\
\begin{eqnarray*}
l &=& l_{0}/\gamma' \\
\gamma' &=& \frac{1}{\sqrt{1-(\beta')^{2}}} \\
v' &=& \frac{u-v}{1-uv/c^{2}} \\
\gamma' &=& \frac{1}{\sqrt{1-(1/c^{2})\frac{(v-u)^{2}}{(1-uv/c^{2})^{2}}}} = \frac{c(1-uv/c^{2})}{c^{2}(1-uv/c^{2})^{2}-(u-v)^{2}} \\
&=& \frac{c-uv/c}{c^{2}-2uv+u^{2}v^{2}/c^{2}-u^{2}-v^{2}+2uv} = \frac{c-uv/c}{c^{2}+u^{2}v^{2}/c^{2}-u^{2}-v^{2}} = \frac{c^{2}-uv}{c^{2}(c^{2}-u^{2})-v^{2}(c^{2}-u^{2})} \\
\gamma' &=& \frac{c^{2}-uv}{(c^{2}-v^{2})(c^{2}-u^{2})} \\
l &=& l_{0}\frac{(c^{2}-v^{2})(c^{2}-u^{2})}{c^{2}-uv} 
\end{eqnarray*} \\ 

Problem 5. Kleppner and Kolenkow 12.10 \\ \\
To resolve this paradox, let's consider the order of events in each of the frames. The events are: when the front of the pole reaches the front of the barn (A), when the front of the pole reaches the back of the barn (B), and when the back of the pole reaches the front of the barn (C). Let's look at the frame of the observer first. Define $t_{A}=0$ and $x_{A}=0$ to be event A. \\
\begin{eqnarray*} 
x_{B} &=& \frac{3}{4}l_{0} \\
t_{B} &=& x_{B}/v = \frac{\frac{3}{4}l_{0}}{\frac{\sqrt{3}}{2}c} = \frac{3l_{0}}{2\sqrt{3}c}= \frac{\sqrt{3}l_{0}}{2c}  \\
x_{C} &=& 0 \\
x'_{C} &=& -l_{0} \\
x'_{C} &=& \gamma(x_{C}-t_{C}v) \\
t_{C} &=& \frac{l_{0}}{\gamma{v}} \\
\gamma &=& \frac{1}{\sqrt{1-3/4}} = 2 \\
t_{C} &=& \frac{l_{0}}{\sqrt{3}c} 
\end{eqnarray*} \\
Thus, the events occured in the order A, C, B. \\
Now look at the frame of the pole vaulter. Define $t'_{A}=0$ and $x'_{A}=0$. \\
\begin{eqnarray*} 
t'_{B} &=& \gamma(t_{B}-vx_{B}) = 2(\frac{\sqrt{3}l_{0}}{2c}-\frac{3\sqrt{3}l_{0}}{8c}) = \frac{\sqrt{3}l_{0}}{4c} \\
t'_{C} &=& \gamma(t_{C}-vx_{C}) = 2(t_{C}) =\frac{2l_{0}}{\sqrt{3}c} 
\end{eqnarray*} \\
Thus the events occured in the order A, B, C. \\
Therefore, both are correct in their respective frames because the events happened in different orders. \\

Problem 6. Kleppner and Kolenkow 12.11 \\ \\
Derive the expression for the acceleration transformation from the S to the S' frame, considering the case that $u_{x}'=u_{x}=0$ initially. \\
\begin{eqnarray*}
a_{x} &=& \frac{du_{x}'}{dt'} \\
dt' &=& \gamma(dt-(v/c^{2})dx) = \gamma(1-\frac{vu_{x}}{c^{2}}) \\
du_{x}' &=& \frac{du_{x}}{1-\frac{vu_{x}}{c^{2}}}+\frac{v}{c^{2}}du_{x}\frac{u_{x}-v}{(1-\frac{vu_{x}}{c^{2}})^{2}} \\
&=& du_{x}\frac{(1-vu_{x}/c^{2}+vu_{x}/c^{2}-v^{2}/c^{2})}{(1-vu_{x}/c^{2})^{2}} \\
a_{x} &=& \frac{du_{x}(1-v^{2}/c^{2})}{dt\gamma(1-\frac{vu_{x}}{c^{2}})^{3}} \\
a_{0} &=& \frac{du_{x}}{dt} \\
u_{x} &=& 0 \\
a_{x} &=& \frac{a_{0}}{\gamma}(1-v^{2}/c^{2}) = \frac{a_{0}}{\gamma^{3}} 
\end{eqnarray*} \\

Problem 7. Kleppner and Kolenkow 12.12 \\ \\
a. Find the velocity after a time t for an observer in the S frame. \\
\begin{eqnarray*} 
a_{x} &=& \frac{dv}{dt} =\frac{a_{0}}{\gamma^{3}} = a_{0}(1-\frac{v^{2}}{c^{2}})^{3/2} \\
\frac{dv}{(1-v^{2}/c^{2})^{3/2}} &=& a_{0}dt \\
\int_{0}^{v}\frac{1}{(1-(v')^{2}/c^{2})^{3/2}}\,dv' &=& \int_{0}^{t}a_{0}\,dt' = a_{0}t \\
v' &=& csin(\theta) \\
dv' &=& ccos(\theta)d\theta \\
\int_{0}^{v}\frac{c}{cos^{3}(\theta)}cos(\theta)\,d\theta &=& a_{0}t \\
\int_{0}^{v}csec^{2}(\theta)\,d\theta &=& a_{0}t \\
ctan(\theta)|_{0}^{v} &=& a_{0}t \\
\frac{cv'}{\sqrt{c^{2}-(v')^{2}}}|_{0}^{v} &=& a_{0}t \\
\frac{v}{\sqrt{1-v^{2}/c^{2}}} &=& a_{0}t \\
v\gamma&=& a_{0}t \\
v &=& \frac{a_{0}t}{\gamma} \\
v &=& a_{0}t\sqrt{1-v^{2}/c^{2}} \\
v^{2} &=& a_{0}^{2}t^{2}(1-v^{2}/c^{2}) \\
v^{2}(1+\frac{a_{0}^{2}t^{2}}{c^{2}}) &=& a_{0}^{2}t^{2} \\
v &=& \frac{a_{0}t}{\sqrt{1+\frac{a_{0}^{2}t^{2}}{c^{2}}}} 
\end{eqnarray*} \\
b. Now let's look at the velocity for 3 different cases. \\
\begin{eqnarray*}
v_{0} &=& a_{0}t \\
v_{0} &=& 10^{-3} \\
v_{0} &<<& c \\
v &\approx& v_{0}(1-\frac{1}{2}v_{0}^{2}/c^{2}) \\
&\approx& v_{0}(1-5*10^{-7}) \\
v_{0} &=& c \\
v &=& \frac{c}{\sqrt{1+c^{2}/c^{2}}} = \frac{c}{\sqrt{2}} \\
v_{0} &>>& c \\
v &=& \frac{v_{0}}{\sqrt{1+v_{0}^{2}/c^{2}}} = \frac{cv_{0}}{\sqrt{c^{2}+v_{0}}} \\
&=& \frac{c}{\sqrt{c^{2}/v_{0}^{2}+1}} \approx c(1-\frac{1}{2}c^{2}/v_{0}^{2}) \\
v &\approx& c(1-5*10^{-7}) 
\end{eqnarray*} \\

Problem 8. Lorentz transformations with matrices	 \\ \\
a. Look at $L_{x}(\beta_{B}=0)$. \\
\begin{eqnarray*}
\gamma_{B}(\beta_{B}=0) &=& \frac{1}{\sqrt{1-0}} = 1 \\
L_{x}(0) &=& \begin{bmatrix}
	1 & 0 & 0 & 0 \\
	0 & 1 & 0 & 0 \\
	0 & 0 & 1 & 0 \\
	0 & 0 & 0 & 1 
	\end{bmatrix}  =1
\end{eqnarray*} \\
b. Show the matrix for the lorentz transformation is the same as doing the lorentz transformations using the general equations. \\
\begin{eqnarray*} 
\begin{bmatrix}
	t' \\
	x' \\
	y' \\
	z' 
	\end{bmatrix} &=& \begin{bmatrix}
	\gamma_{B} & -\beta_{B}\gamma_{B} & 0 & 0 \\
	-\beta_{B}\gamma_{B} & \gamma_{B} & 0 & 0 \\
	0 & 0 & 1 & 0 \\
	0 & 0 & 0 & 1 
	\end{bmatrix}\begin{bmatrix}
	t \\
	x \\
	y \\
	z 
	\end{bmatrix} \\
&=& \begin{bmatrix}
	\gamma_{B}(t-\beta_{B}x) \\
	\gamma_{B}(-\beta_{B}x+t) \\
	y \\
	z 
	\end{bmatrix}
\end{eqnarray*} \\
c. Evaluate $L_{x}(-\beta_{B})L_{x}(\beta_{B})$ in order to show that $L_{x}(\beta_{B})$ is the inverse of $L_{x}(\beta_{B})$. \\
\begin{eqnarray*} 
L_{x}(-\beta_{B})L_{x}(\beta_{B}) &=& \begin{bmatrix}
	\gamma_{B} -\beta_{B}\gamma_{B} & 0 & 0 \\
	-\beta_{B}\gamma_{B} & \gamma_{B} & 0 & 0 \\
	0 & 0 & 1 & 0 \\
	0 & 0 & 0 & 1 
	\end{bmatrix}\begin{bmatrix}
	\gamma_{B} & \beta_{B}\gamma_{B} & 0 & 0 \\
	\beta_{B}\gamma_{B} & \gamma_{B} & 0 & 0 \\
	0 & 0 & 1 & 0 \\
	0 & 0 & 0 & 1
	\end{bmatrix} \\
	&=& \begin{bmatrix}
	\gamma^{2}(1-\beta_{B}^{2}) & \beta_{B}\gamma_{B}^{2}-\beta_{B}\gamma_{B}^{2} & 0 & 0 \\
	-\beta_{B}\gamma_{B}^{2}+\beta_{B}\gamma_{B}^{2} & \gamma^{2}(1-\beta_{B}^{2}) & 0 & 0 \\
	0 & 0 & 1 & 0 \\
	0 & 0 & 0 & 1 
	\end{bmatrix} = \begin{bmatrix}
	\frac{\gamma^{2}}{\gamma^{2}} & 0 & 0 & 0 \\
	0 & \frac{\gamma^{2}}{\gamma^{2}} & 0 & 0 \\
	0 & 0 & 1 & 0 \\
	0 & 0 & 0 & 1 
	\end{bmatrix} = 1
\end{eqnarray*} \\
Now show that this is the reverse transformation matrix. \\
\begin{eqnarray*}
L_{x}(-\beta_{B})X' &=& L_{x}(-\beta)L_{x}(\beta)X \\
L_{x}(-\beta_{B})X'  &=& X \\
X &=& L_{x}(-\beta)X' 
\end{eqnarray*} \\
d. Show that two seccessive boosts in the same direction can be represented as one boost $\beta_{B}''=\frac{\beta_{B}+\beta_{B}'}{1+\beta_{B}\beta_{B}'}$. \\
\begin{eqnarray*}
L_{x}(\beta_{B}')L_{x}(\beta_{B}') &=& \begin{bmatrix}
	\gamma_{B}' & -\beta_{B}'\gamma_{B}' & 0 & 0 \\
	-\beta_{B}'\gamma_{B}' & \gamma_{B}' & 0 & 0 \\
	0 & 0 & 1 & 0 \\
	0 & 0 & 0 & 1 
	\end{bmatrix}\begin{bmatrix}
	\gamma_{B} & -\beta_{B}\gamma_{B} & 0 & 0 \\
	-\beta_{B}\gamma_{B} & \gamma_{B} & 0 & 0 \\
	0 & 0 & 1 & 0 \\
	0 & 0 & 0 & 1 
	\end{bmatrix} \\
&=& \begin{bmatrix}
	\gamma_{B}'\gamma_{B}(1+\beta_{B}'\beta_{B}) & -\gamma_{B}'\gamma_{B}(\beta_{B}+\beta_{B}') & 0 & 0 \\
	 -\gamma_{B}'\gamma_{B}(\beta_{B}+\beta_{B}') & \gamma_{B}'\gamma_{B}(1+\beta_{B}'\beta_{B}) & 0 & 0 \\
	0 & 0 & 1 & 0 \\
	0 & 0 & 0 & 1 
	\end{bmatrix} 
\end{eqnarray*} \\
Say that $\gamma_{B}''=\gamma_{B}\gamma_{B}'(1+\beta_{B}'\beta_{B})$. \\
\begin{eqnarray*} 
\gamma_{B}''\beta_{B}'' &=& \gamma_{B}\gamma_{B}'(\beta_{B}+\beta_{B}') \\
L_{x}(\beta_{B}')L_{x}(\beta_{B}) &=& L_{x}(\beta_{B}'') = \begin{bmatrix}
	\gamma_{B}'' & -\beta_{B}''\gamma_{B}'' & 0 & 0 \\
	-\beta_{B}''\gamma_{B}'' & \gamma_{B}'' & 0 & 0 \\
	0 & 0 & 1 & 0 \\
	0 & 0 & 0 & 1
	\end{bmatrix} 
\end{eqnarray*} \\
e. She that length is preserved in lorentz transformations when you use the metric g. \\
\begin{eqnarray*} 
U'\cdot{V'} &=& L_{x}^{T}U\cdot{L_{x}V} = L_{x}^{T}UgL_{x}V \\
&=& \begin{bmatrix}
	\gamma_{B}U_{0}-\beta_{B}\gamma_{B}U_{1} & -\beta_{B}\gamma_{B}U_{0}+\gamma_{B}U_{1} & U_{2} & U_{3} 
\end{bmatrix}\begin{bmatrix}
	1 & 0 & 0 & 0 \\
	0 & -1 & 0 & 0 \\
	0 & 0 & -1 & 0 \\
	0 & 0 & 0 & -1 
	\end{bmatrix}\begin{bmatrix}
	\gamma_{B}V_{0}-\beta_{B}\gamma_{B}V_{1} \\
	-\beta_{B}\gamma_{B}V_{0}+\gamma_{B}V_{1} \\
	V_{2} \\
	V_{3}
	\end{bmatrix} \\
&=& \gamma_{B}^{2}U_{0}V_{0}+\beta_{B}^{2}\gamma_{B}^{2}U_{1}V_{1}-\gamma_{B}^{2}\beta_{B}(U_{0}V_{1}+U_{1}V_{0})-\beta_{B}^{2}\gamma_{B}^{2}U_{0}V_{0} \\
& & -\gamma_{B}^{2}U_{1}V_{1}+\gamma_{B}^{2}\beta_{B}(U_{0}V_{1}+U_{1}V_{0})-U_{2}V_{2}-U_{3}V_{3} \\
&=& U_{0}V_{0}(\gamma_{B}^{2}(1-\beta_{B}^{2})-U_{1}V_{1}\gamma_{B}^{2}(1-\beta_{B}^{2})-U_{2}V_{2}-U_{3}V_{3} \\
&=& U_{0}V_{0}-U_{1}V_{1}-U_{2}V_{2}-U_{3}V_{3} = U\cdot{V} 
\end{eqnarray*} \\
f. Show that $g'=L_{x}^{T}gL_{x}=g$. \\
\begin{eqnarray*} 
L_{x}^{T} &=& L_{x} \\
g' &=& L_{x}gL_{x} = \begin{bmatrix}
	\gamma_{B} & -\beta_{B}\gamma_{B} & 0 & 0 \\
	-\beta_{B}\gamma_{B} & \gamma_{B} & 0 & 0 \\
	0 & 0 & 1 & 0 \\
	0 & 0 & 0 & 1 
	\end{bmatrix}\begin{bmatrix}
	1 & 0 & 0 & 0 \\
	0 & -1 & 0 & 0 \\
	0 & 0 & -1 & 0 \\
	0 & 0 & 0 & -1 
	\end{bmatrix}\begin{bmatrix}
	\gamma_{B} & -\beta_{B}\gamma_{B} & 0 & 0 \\
	-\beta_{B}\gamma_{B} & \gamma_{B} & 0 & 0 \\
	0 & 0 & 1 & 0 \\
	0 & 0 & 0 & 1 
	\end{bmatrix} \\
&=&  \begin{bmatrix}
	\gamma_{B} & \beta_{B}\gamma_{B} & 0 & 0 \\
	-\beta_{B}\gamma_{B} & -\gamma_{B} & 0 & 0 \\
	0 & 0 & -1 & 0 \\
	0 & 0 & 0 & -1 
	\end{bmatrix}\begin{bmatrix}
	\gamma_{B} & -\beta_{B}\gamma_{B} & 0 & 0 \\
	-\beta_{B}\gamma_{B} & \gamma_{B} & 0 & 0 \\
	0 & 0 & 1 & 0 \\
	0 & 0 & 0 & 1 
	\end{bmatrix} \\
&=& \begin{bmatrix}
	\gamma_{B}^{2}(1-\beta_{B}^{2}) & 0 & 0 & 0 \\
	0 & -\gamma_{B}^{2}(1-\beta_{B}^{2}) & 0 & 0 \\
	0 & 0 & -1 & 0 \\
	0 & 0 & 0 & -1 
	\end{bmatrix} = \begin{bmatrix}
	1 & 0 & 0 & 0 \\
	0 & -1 & 0 & 0 \\
	0 & 0 & -1 & 0 \\
	0 & 0 & 0 & -1
	\end{bmatrix} = g 
\end{eqnarray*} \\

\end{document}