\documentclass[11pt]{amsart}
\usepackage{geometry} % see geometry.pdf on how to lay out the page. There's lots.
\geometry{a4paper} % or letter or a5paper or ... etc
% \geometry{landscape} % rotated page geometry
\usepackage{amsmath}
\usepackage{graphicx}
\usepackage{breqn}

\setcounter{MaxMatrixCols}{10}

\flushbottom
\chardef\atcode=\catcode`\@
\makeatletter
\@addtoreset{figure}{section}
\@addtoreset{table}{section}
\renewcommand{\figurename}{Figure}
\renewcommand{\tablename}{Table}
\setcounter{topnumber}{3}               % orig: 2
\setcounter{totalnumber}{4}             % orig: 3
\renewcommand{\textfraction}{0}         
\renewcommand{\bottomfraction}{0.65}    
\renewcommand{\topfraction}{0.75}       
\renewcommand{\floatpagefraction}{0.75} 
\catcode`\@=\atcode 
\newcommand{\grad}{$^\circ$}
\newcommand{\gradm}{^\circ}
\newcommand{\bqn}{ \begin{eqnarray} }
\newcommand{\eqn}{ \end{eqnarray} }
\newcommand{\beq}{ \begin{equation} }
\newcommand{\eeq}{ \end{equation} }
\setlength{\baselineskip}{2.1ex}
\renewcommand{\baselinestretch}{1.06}
\setlength{\parskip}{1.5ex plus 0.8ex minus 0.6ex}
\setlength{\evensidemargin}{0.3cm}
\setlength{\oddsidemargin}{-0.3cm}
\setlength{\topmargin}{-1cm}
\setlength{\textwidth}{17 cm}
\setlength{\textheight}{26cm}
\newcommand{\mat}[1]{\mbox{$\underline{\underline{#1}}$}}
\newcommand{\etal}{\mbox{\sl et al.}}
\renewcommand{\refname}{}
\newcommand{\vol}[1]{{\bf{#1}}}
\newcommand{\dg}{$^\circ\;$}
\def\D{\displaystyle}
\newcommand{\lapprox}{\ensuremath{<\atop{\mbox{\raisebox{0.5ex}{$\sim$}}}}}
\parindent 0cm
% Macros for Scientific Word 2.5 documents saved with the LaTeX filter.
%Copyright (C) 1994-95 TCI Software Research, Inc.
\typeout{TCILATEX Macros for Scientific Word 2.5 <22 Dec 95>.}
\typeout{NOTICE:  This macro file is NOT proprietary and may be 
freely copied and distributed.}
%
\makeatletter
%
%%%%%%%%%%%%%%%%%%%%%%
% macros for time
\newcount\@hour\newcount\@minute\chardef\@x10\chardef\@xv60
\def\tcitime{
\def\@time{%
  \@minute\time\@hour\@minute\divide\@hour\@xv
  \ifnum\@hour<\@x 0\fi\the\@hour:%
  \multiply\@hour\@xv\advance\@minute-\@hour
  \ifnum\@minute<\@x 0\fi\the\@minute
  }}%

%%%%%%%%%%%%%%%%%%%%%%
% macro for hyperref
\@ifundefined{hyperref}{\def\hyperref#1#2#3#4{#2\ref{#4}#3}}{}

% macro for external program call
\@ifundefined{qExtProgCall}{\def\qExtProgCall#1#2#3#4#5#6{\relax}}{}
%%%%%%%%%%%%%%%%%%%%%%
%
% macros for graphics
%
\def\FILENAME#1{#1}%
%
\def\QCTOpt[#1]#2{%
  \def\QCTOptB{#1}
  \def\QCTOptA{#2}
}
\def\QCTNOpt#1{%
  \def\QCTOptA{#1}
  \let\QCTOptB\empty
}
\def\Qct{%
  \@ifnextchar[{%
    \QCTOpt}{\QCTNOpt}
}
\def\QCBOpt[#1]#2{%
  \def\QCBOptB{#1}
  \def\QCBOptA{#2}
}
\def\QCBNOpt#1{%
  \def\QCBOptA{#1}
  \let\QCBOptB\empty
}
\def\Qcb{%
  \@ifnextchar[{%
    \QCBOpt}{\QCBNOpt}
}
\def\PrepCapArgs{%
  \ifx\QCBOptA\empty
    \ifx\QCTOptA\empty
      {}%
    \else
      \ifx\QCTOptB\empty
        {\QCTOptA}%
      \else
        [\QCTOptB]{\QCTOptA}%
      \fi
    \fi
  \else
    \ifx\QCBOptA\empty
      {}%
    \else
      \ifx\QCBOptB\empty
        {\QCBOptA}%
      \else
        [\QCBOptB]{\QCBOptA}%
      \fi
    \fi
  \fi
}
\newcount\GRAPHICSTYPE
%\GRAPHICSTYPE 0 is for TurboTeX
%\GRAPHICSTYPE 1 is for DVIWindo (PostScript)
%%%(removed)%\GRAPHICSTYPE 2 is for psfig (PostScript)
\GRAPHICSTYPE=\z@
\def\GRAPHICSPS#1{%
 \ifcase\GRAPHICSTYPE%\GRAPHICSTYPE=0
   \special{ps: #1}%
 \or%\GRAPHICSTYPE=1
   \special{language "PS", include "#1"}%
%%%\or%\GRAPHICSTYPE=2
%%%  #1%
 \fi
}%
%
\def\GRAPHICSHP#1{\special{include #1}}%
%
% \graffile{ body }                                  %#1
%          { contentswidth (scalar)  }               %#2
%          { contentsheight (scalar) }               %#3
%          { vertical shift when in-line (scalar) }  %#4
\def\graffile#1#2#3#4{%
%%% \ifnum\GRAPHICSTYPE=\tw@
%%%  %Following if using psfig
%%%  \@ifundefined{psfig}{\input psfig.tex}{}%
%%%  \psfig{file=#1, height=#3, width=#2}%
%%% \else
  %Following for all others
  % JCS - added BOXTHEFRAME, see below
    \leavevmode
    \raise -#4 \BOXTHEFRAME{%
        \hbox to #2{\raise #3\hbox to #2{\null #1\hfil}}}%
}%
%
% A box for drafts
\def\draftbox#1#2#3#4{%
 \leavevmode\raise -#4 \hbox{%
  \frame{\rlap{\protect\tiny #1}\hbox to #2%
   {\vrule height#3 width\z@ depth\z@\hfil}%
  }%
 }%
}%
%
\newcount\draft
\draft=\z@
\let\nographics=\draft
\newif\ifwasdraft
\wasdraftfalse

%  \GRAPHIC{ body }                                  %#1
%          { draft name }                            %#2
%          { contentswidth (scalar)  }               %#3
%          { contentsheight (scalar) }               %#4
%          { vertical shift when in-line (scalar) }  %#5
\def\GRAPHIC#1#2#3#4#5{%
 \ifnum\draft=\@ne\draftbox{#2}{#3}{#4}{#5}%
  \else\graffile{#1}{#3}{#4}{#5}%
  \fi
 }%
%
\def\addtoLaTeXparams#1{%
    \edef\LaTeXparams{\LaTeXparams #1}}%
%
% JCS -  added a switch BoxFrame that can 
% be set by including X in the frame params.
% If set a box is drawn around the frame.

\newif\ifBoxFrame \BoxFramefalse
\newif\ifOverFrame \OverFramefalse
\newif\ifUnderFrame \UnderFramefalse

\def\BOXTHEFRAME#1{%
   \hbox{%
      \ifBoxFrame
         \frame{#1}%
      \else
         {#1}%
      \fi
   }%
}


\def\doFRAMEparams#1{\BoxFramefalse\OverFramefalse\UnderFramefalse\readFRAMEparams#1\end}%
\def\readFRAMEparams#1{%
 \ifx#1\end%
  \let\next=\relax
  \else
  \ifx#1i\dispkind=\z@\fi
  \ifx#1d\dispkind=\@ne\fi
  \ifx#1f\dispkind=\tw@\fi
  \ifx#1t\addtoLaTeXparams{t}\fi
  \ifx#1b\addtoLaTeXparams{b}\fi
  \ifx#1p\addtoLaTeXparams{p}\fi
  \ifx#1h\addtoLaTeXparams{h}\fi
  \ifx#1X\BoxFrametrue\fi
  \ifx#1O\OverFrametrue\fi
  \ifx#1U\UnderFrametrue\fi
  \ifx#1w
    \ifnum\draft=1\wasdrafttrue\else\wasdraftfalse\fi
    \draft=\@ne
  \fi
  \let\next=\readFRAMEparams
  \fi
 \next
 }%
%
%Macro for In-line graphics object
%   \IFRAME{ contentswidth (scalar)  }               %#1
%          { contentsheight (scalar) }               %#2
%          { vertical shift when in-line (scalar) }  %#3
%          { draft name }                            %#4
%          { body }                                  %#5
%          { caption}                                %#6


\def\IFRAME#1#2#3#4#5#6{%
      \bgroup
      \let\QCTOptA\empty
      \let\QCTOptB\empty
      \let\QCBOptA\empty
      \let\QCBOptB\empty
      #6%
      \parindent=0pt%
      \leftskip=0pt
      \rightskip=0pt
      \setbox0 = \hbox{\QCBOptA}%
      \@tempdima = #1\relax
      \ifOverFrame
          % Do this later
          \typeout{This is not implemented yet}%
          \show\HELP
      \else
         \ifdim\wd0>\@tempdima
            \advance\@tempdima by \@tempdima
            \ifdim\wd0 >\@tempdima
               \textwidth=\@tempdima
               \setbox1 =\vbox{%
                  \noindent\hbox to \@tempdima{\hfill\GRAPHIC{#5}{#4}{#1}{#2}{#3}\hfill}\\%
                  \noindent\hbox to \@tempdima{\parbox[b]{\@tempdima}{\QCBOptA}}%
               }%
               \wd1=\@tempdima
            \else
               \textwidth=\wd0
               \setbox1 =\vbox{%
                 \noindent\hbox to \wd0{\hfill\GRAPHIC{#5}{#4}{#1}{#2}{#3}\hfill}\\%
                 \noindent\hbox{\QCBOptA}%
               }%
               \wd1=\wd0
            \fi
         \else
            %\show\BBB
            \ifdim\wd0>0pt
              \hsize=\@tempdima
              \setbox1 =\vbox{%
                \unskip\GRAPHIC{#5}{#4}{#1}{#2}{0pt}%
                \break
                \unskip\hbox to \@tempdima{\hfill \QCBOptA\hfill}%
              }%
              \wd1=\@tempdima
           \else
              \hsize=\@tempdima
              \setbox1 =\vbox{%
                \unskip\GRAPHIC{#5}{#4}{#1}{#2}{0pt}%
              }%
              \wd1=\@tempdima
           \fi
         \fi
         \@tempdimb=\ht1
         \advance\@tempdimb by \dp1
         \advance\@tempdimb by -#2%
         \advance\@tempdimb by #3%
         \leavevmode
         \raise -\@tempdimb \hbox{\box1}%
      \fi
      \egroup%
}%
%
%Macro for Display graphics object
%   \DFRAME{ contentswidth (scalar)  }               %#1
%          { contentsheight (scalar) }               %#2
%          { draft label }                           %#3
%          { name }                                  %#4
%          { caption}                                %#5
\def\DFRAME#1#2#3#4#5{%
 \begin{center}
     \let\QCTOptA\empty
     \let\QCTOptB\empty
     \let\QCBOptA\empty
     \let\QCBOptB\empty
     \ifOverFrame 
        #5\QCTOptA\par
     \fi
     \GRAPHIC{#4}{#3}{#1}{#2}{\z@}
     \ifUnderFrame 
        \nobreak\par #5\QCBOptA
     \fi
 \end{center}%
 }%
%
%Macro for Floating graphic object
%   \FFRAME{ framedata f|i tbph x F|T }              %#1
%          { contentswidth (scalar)  }               %#2
%          { contentsheight (scalar) }               %#3
%          { caption }                               %#4
%          { label }                                 %#5
%          { draft name }                            %#6
%          { body }                                  %#7
\def\FFRAME#1#2#3#4#5#6#7{%
 \begin{figure}[#1]%
  \let\QCTOptA\empty
  \let\QCTOptB\empty
  \let\QCBOptA\empty
  \let\QCBOptB\empty
  \ifOverFrame
    #4
    \ifx\QCTOptA\empty
    \else
      \ifx\QCTOptB\empty
        \caption{\QCTOptA}%
      \else
        \caption[\QCTOptB]{\QCTOptA}%
      \fi
    \fi
    \ifUnderFrame\else
      \label{#5}%
    \fi
  \else
    \UnderFrametrue%
  \fi
  \begin{center}\GRAPHIC{#7}{#6}{#2}{#3}{\z@}\end{center}%
  \ifUnderFrame
    #4
    \ifx\QCBOptA\empty
      \caption{}%
    \else
      \ifx\QCBOptB\empty
        \caption{\QCBOptA}%
      \else
        \caption[\QCBOptB]{\QCBOptA}%
      \fi
    \fi
    \label{#5}%
  \fi
  \end{figure}%
 }%
%
%
%    \FRAME{ framedata f|i tbph x F|T }              %#1
%          { contentswidth (scalar)  }               %#2
%          { contentsheight (scalar) }               %#3
%          { vertical shift when in-line (scalar) }  %#4
%          { caption }                               %#5
%          { label }                                 %#6
%          { name }                                  %#7
%          { body }                                  %#8
%
%    framedata is a string which can contain the following
%    characters: idftbphxFT
%    Their meaning is as follows:
%             i, d or f : in-line, display, or floating
%             t,b,p,h   : LaTeX floating placement options
%             x         : fit contents box to contents
%             F or T    : Figure or Table. 
%                         Later this can expand
%                         to a more general float class.
%
%
\newcount\dispkind%

\def\makeactives{
  \catcode`\"=\active
  \catcode`\;=\active
  \catcode`\:=\active
  \catcode`\'=\active
  \catcode`\~=\active
}
\bgroup
   \makeactives
   \gdef\activesoff{%
      \def"{\string"}
      \def;{\string;}
      \def:{\string:}
      \def'{\string'}
      \def~{\string~}
      %\bbl@deactivate{"}%
      %\bbl@deactivate{;}%
      %\bbl@deactivate{:}%
      %\bbl@deactivate{'}%
    }
\egroup

\def\FRAME#1#2#3#4#5#6#7#8{%
 \bgroup
 \@ifundefined{bbl@deactivate}{}{\activesoff}
 \ifnum\draft=\@ne
   \wasdrafttrue
 \else
   \wasdraftfalse%
 \fi
 \def\LaTeXparams{}%
 \dispkind=\z@
 \def\LaTeXparams{}%
 \doFRAMEparams{#1}%
 \ifnum\dispkind=\z@\IFRAME{#2}{#3}{#4}{#7}{#8}{#5}\else
  \ifnum\dispkind=\@ne\DFRAME{#2}{#3}{#7}{#8}{#5}\else
   \ifnum\dispkind=\tw@
    \edef\@tempa{\noexpand\FFRAME{\LaTeXparams}}%
    \@tempa{#2}{#3}{#5}{#6}{#7}{#8}%
    \fi
   \fi
  \fi
  \ifwasdraft\draft=1\else\draft=0\fi{}%
  \egroup
 }%
%
% This macro added to let SW gobble a parameter that
% should not be passed on and expanded. 

\def\TEXUX#1{"texux"}

%
% Macros for text attributes:
%
\def\BF#1{{\bf {#1}}}%
\def\NEG#1{\leavevmode\hbox{\rlap{\thinspace/}{$#1$}}}%
%
%%%%%%%%%%%%%%%%%%%%%%%%%%%%%%%%%%%%%%%%%%%%%%%%%%%%%%%%%%%%%%%%%%%%%%%%
%
%
% macros for user - defined functions
\def\func#1{\mathop{\rm #1}}%
\def\limfunc#1{\mathop{\rm #1}}%

%
% miscellaneous 
%\long\def\QQQ#1#2{}%
\long\def\QQQ#1#2{%
     \long\expandafter\def\csname#1\endcsname{#2}}%
%\def\QTP#1{}% JCS - this was changed becuase style editor will define QTP
\@ifundefined{QTP}{\def\QTP#1{}}{}
\@ifundefined{QEXCLUDE}{\def\QEXCLUDE#1{}}{}
%\@ifundefined{Qcb}{\def\Qcb#1{#1}}{}
%\@ifundefined{Qct}{\def\Qct#1{#1}}{}
\@ifundefined{Qlb}{\def\Qlb#1{#1}}{}
\@ifundefined{Qlt}{\def\Qlt#1{#1}}{}
\def\QWE{}%
\long\def\QQA#1#2{}%
%\def\QTR#1#2{{\em #2}}% Always \em!!!
%\def\QTR#1#2{\mbox{\begin{#1}#2\end{#1}}}%cb%%%
\def\QTR#1#2{{\csname#1\endcsname #2}}%(gp) Is this the best?
\long\def\TeXButton#1#2{#2}%
\long\def\QSubDoc#1#2{#2}%
\def\EXPAND#1[#2]#3{}%
\def\NOEXPAND#1[#2]#3{}%
\def\PROTECTED{}%
\def\LaTeXparent#1{}%
\def\ChildStyles#1{}%
\def\ChildDefaults#1{}%
\def\QTagDef#1#2#3{}%
%
% Macros for style editor docs
\@ifundefined{StyleEditBeginDoc}{\def\StyleEditBeginDoc{\relax}}{}
%
% Macros for footnotes
\def\QQfnmark#1{\footnotemark}
\def\QQfntext#1#2{\addtocounter{footnote}{#1}\footnotetext{#2}}
%
% Macros for indexing.
\def\MAKEINDEX{\makeatletter\input gnuindex.sty\makeatother\makeindex}%	
\@ifundefined{INDEX}{\def\INDEX#1#2{}{}}{}%
\@ifundefined{SUBINDEX}{\def\SUBINDEX#1#2#3{}{}{}}{}%
\@ifundefined{initial}%  
   {\def\initial#1{\bigbreak{\raggedright\large\bf #1}\kern 2\p@\penalty3000}}%
   {}%
\@ifundefined{entry}{\def\entry#1#2{\item {#1}, #2}}{}%
\@ifundefined{primary}{\def\primary#1{\item {#1}}}{}%
\@ifundefined{secondary}{\def\secondary#1#2{\subitem {#1}, #2}}{}%
%
%
\@ifundefined{ZZZ}{}{\MAKEINDEX\makeatletter}%
%
% Attempts to avoid problems with other styles
\@ifundefined{abstract}{%
 \def\abstract{%
  \if@twocolumn
   \section*{Abstract (Not appropriate in this style!)}%
   \else \small 
   \begin{center}{\bf Abstract\vspace{-.5em}\vspace{\z@}}\end{center}%
   \quotation 
   \fi
  }%
 }{%
 }%
\@ifundefined{endabstract}{\def\endabstract
  {\if@twocolumn\else\endquotation\fi}}{}%
\@ifundefined{maketitle}{\def\maketitle#1{}}{}%
\@ifundefined{affiliation}{\def\affiliation#1{}}{}%
\@ifundefined{proof}{\def\proof{\noindent{\bfseries Proof. }}}{}%
\@ifundefined{endproof}{\def\endproof{\mbox{\ \rule{.1in}{.1in}}}}{}%
\@ifundefined{newfield}{\def\newfield#1#2{}}{}%
\@ifundefined{chapter}{\def\chapter#1{\par(Chapter head:)#1\par }%
 \newcount\c@chapter}{}%
\@ifundefined{part}{\def\part#1{\par(Part head:)#1\par }}{}%
\@ifundefined{section}{\def\section#1{\par(Section head:)#1\par }}{}%
\@ifundefined{subsection}{\def\subsection#1%
 {\par(Subsection head:)#1\par }}{}%
\@ifundefined{subsubsection}{\def\subsubsection#1%
 {\par(Subsubsection head:)#1\par }}{}%
\@ifundefined{paragraph}{\def\paragraph#1%
 {\par(Subsubsubsection head:)#1\par }}{}%
\@ifundefined{subparagraph}{\def\subparagraph#1%
 {\par(Subsubsubsubsection head:)#1\par }}{}%
%%%%%%%%%%%%%%%%%%%%%%%%%%%%%%%%%%%%%%%%%%%%%%%%%%%%%%%%%%%%%%%%%%%%%%%%
% These symbols are not recognized by LaTeX
\@ifundefined{therefore}{\def\therefore{}}{}%
\@ifundefined{backepsilon}{\def\backepsilon{}}{}%
\@ifundefined{yen}{\def\yen{\hbox{\rm\rlap=Y}}}{}%
\@ifundefined{registered}{%
   \def\registered{\relax\ifmmode{}\r@gistered
                    \else$\m@th\r@gistered$\fi}%
 \def\r@gistered{^{\ooalign
  {\hfil\raise.07ex\hbox{$\scriptstyle\rm\text{R}$}\hfil\crcr
  \mathhexbox20D}}}}{}%
\@ifundefined{Eth}{\def\Eth{}}{}%
\@ifundefined{eth}{\def\eth{}}{}%
\@ifundefined{Thorn}{\def\Thorn{}}{}%
\@ifundefined{thorn}{\def\thorn{}}{}%
% A macro to allow any symbol that requires math to appear in text
\def\TEXTsymbol#1{\mbox{$#1$}}%
\@ifundefined{degree}{\def\degree{{}^{\circ}}}{}%
%
% macros for T3TeX files
\newdimen\theight
\def\Column{%
 \vadjust{\setbox\z@=\hbox{\scriptsize\quad\quad tcol}%
  \theight=\ht\z@\advance\theight by \dp\z@\advance\theight by \lineskip
  \kern -\theight \vbox to \theight{%
   \rightline{\rlap{\box\z@}}%
   \vss
   }%
  }%
 }%
%
\def\qed{%
 \ifhmode\unskip\nobreak\fi\ifmmode\ifinner\else\hskip5\p@\fi\fi
 \hbox{\hskip5\p@\vrule width4\p@ height6\p@ depth1.5\p@\hskip\p@}%
 }%
%
\def\cents{\hbox{\rm\rlap/c}}%
\def\miss{\hbox{\vrule height2\p@ width 2\p@ depth\z@}}%
%\def\miss{\hbox{.}}%        %another possibility 
%
\def\vvert{\Vert}%           %always translated to \left| or \right|
%
\def\tcol#1{{\baselineskip=6\p@ \vcenter{#1}} \Column}  %
%
\def\dB{\hbox{{}}}%                 %dummy entry in column 
\def\mB#1{\hbox{$#1$}}%             %column entry
\def\nB#1{\hbox{#1}}%               %column entry (not math)
%
%\newcount\notenumber
%\def\clearnotenumber{\notenumber=0}
%\def\note{\global\advance\notenumber by 1
% \footnote{$^{\the\notenumber}$}}
%\def\note{\global\advance\notenumber by 1
\def\note{$^{\dag}}%
%
%

\def\newfmtname{LaTeX2e}
\def\chkcompat{%
   \if@compatibility
   \else
     \usepackage{latexsym}
   \fi
}

\ifx\fmtname\newfmtname
  \DeclareOldFontCommand{\rm}{\normalfont\rmfamily}{\mathrm}
  \DeclareOldFontCommand{\sf}{\normalfont\sffamily}{\mathsf}
  \DeclareOldFontCommand{\tt}{\normalfont\ttfamily}{\mathtt}
  \DeclareOldFontCommand{\bf}{\normalfont\bfseries}{\mathbf}
  \DeclareOldFontCommand{\it}{\normalfont\itshape}{\mathit}
  \DeclareOldFontCommand{\sl}{\normalfont\slshape}{\@nomath\sl}
  \DeclareOldFontCommand{\sc}{\normalfont\scshape}{\@nomath\sc}
  \chkcompat
\fi

%
% Greek bold macros
% Redefine all of the math symbols 
% which might be bolded	 - there are 
% probably others to add to this list

\def\alpha{\Greekmath 010B }%
\def\beta{\Greekmath 010C }%
\def\gamma{\Greekmath 010D }%
\def\delta{\Greekmath 010E }%
\def\epsilon{\Greekmath 010F }%
\def\zeta{\Greekmath 0110 }%
\def\eta{\Greekmath 0111 }%
\def\theta{\Greekmath 0112 }%
\def\iota{\Greekmath 0113 }%
\def\kappa{\Greekmath 0114 }%
\def\lambda{\Greekmath 0115 }%
\def\mu{\Greekmath 0116 }%
\def\nu{\Greekmath 0117 }%
\def\xi{\Greekmath 0118 }%
\def\pi{\Greekmath 0119 }%
\def\rho{\Greekmath 011A }%
\def\sigma{\Greekmath 011B }%
\def\tau{\Greekmath 011C }%
\def\upsilon{\Greekmath 011D }%
\def\phi{\Greekmath 011E }%
\def\chi{\Greekmath 011F }%
\def\psi{\Greekmath 0120 }%
\def\omega{\Greekmath 0121 }%
\def\varepsilon{\Greekmath 0122 }%
\def\vartheta{\Greekmath 0123 }%
\def\varpi{\Greekmath 0124 }%
\def\varrho{\Greekmath 0125 }%
\def\varsigma{\Greekmath 0126 }%
\def\varphi{\Greekmath 0127 }%

\def\nabla{\Greekmath 0272 }
\def\FindBoldGroup{%
   {\setbox0=\hbox{$\mathbf{x\global\edef\theboldgroup{\the\mathgroup}}$}}%
}

\def\Greekmath#1#2#3#4{%
    \if@compatibility
        \ifnum\mathgroup=\symbold
           \mathchoice{\mbox{\boldmath$\displaystyle\mathchar"#1#2#3#4$}}%
                      {\mbox{\boldmath$\textstyle\mathchar"#1#2#3#4$}}%
                      {\mbox{\boldmath$\scriptstyle\mathchar"#1#2#3#4$}}%
                      {\mbox{\boldmath$\scriptscriptstyle\mathchar"#1#2#3#4$}}%
        \else
           \mathchar"#1#2#3#4% 
        \fi 
    \else 
        \FindBoldGroup
        \ifnum\mathgroup=\theboldgroup % For 2e
           \mathchoice{\mbox{\boldmath$\displaystyle\mathchar"#1#2#3#4$}}%
                      {\mbox{\boldmath$\textstyle\mathchar"#1#2#3#4$}}%
                      {\mbox{\boldmath$\scriptstyle\mathchar"#1#2#3#4$}}%
                      {\mbox{\boldmath$\scriptscriptstyle\mathchar"#1#2#3#4$}}%
        \else
           \mathchar"#1#2#3#4% 
        \fi     	    
	  \fi}

\newif\ifGreekBold  \GreekBoldfalse
\let\SAVEPBF=\pbf
\def\pbf{\GreekBoldtrue\SAVEPBF}%
%

\@ifundefined{theorem}{\newtheorem{theorem}{Theorem}}{}
\@ifundefined{lemma}{\newtheorem{lemma}[theorem]{Lemma}}{}
\@ifundefined{corollary}{\newtheorem{corollary}[theorem]{Corollary}}{}
\@ifundefined{conjecture}{\newtheorem{conjecture}[theorem]{Conjecture}}{}
\@ifundefined{proposition}{\newtheorem{proposition}[theorem]{Proposition}}{}
\@ifundefined{axiom}{\newtheorem{axiom}{Axiom}}{}
\@ifundefined{remark}{\newtheorem{remark}{Remark}}{}
\@ifundefined{example}{\newtheorem{example}{Example}}{}
\@ifundefined{exercise}{\newtheorem{exercise}{Exercise}}{}
\@ifundefined{definition}{\newtheorem{definition}{Definition}}{}


\@ifundefined{mathletters}{%
  %\def\theequation{\arabic{equation}}
  \newcounter{equationnumber}  
  \def\mathletters{%
     \addtocounter{equation}{1}
     \edef\@currentlabel{\theequation}%
     \setcounter{equationnumber}{\c@equation}
     \setcounter{equation}{0}%
     \edef\theequation{\@currentlabel\noexpand\alph{equation}}%
  }
  \def\endmathletters{%
     \setcounter{equation}{\value{equationnumber}}%
  }
}{}

%Logos
\@ifundefined{BibTeX}{%
    \def\BibTeX{{\rm B\kern-.05em{\sc i\kern-.025em b}\kern-.08em
                 T\kern-.1667em\lower.7ex\hbox{E}\kern-.125emX}}}{}%
\@ifundefined{AmS}%
    {\def\AmS{{\protect\usefont{OMS}{cmsy}{m}{n}%
                A\kern-.1667em\lower.5ex\hbox{M}\kern-.125emS}}}{}%
\@ifundefined{AmSTeX}{\def\AmSTeX{\protect\AmS-\protect\TeX\@}}{}%
%

%%%%%%%%%%%%%%%%%%%%%%%%%%%%%%%%%%%%%%%%%%%%%%%%%%%%%%%%%%%%%%%%%%%%%%%
% NOTE: The rest of this file is read only if amstex has not been
% loaded.  This section is used to define amstex constructs in the
% event they have not been defined.
%
%
\ifx\ds@amstex\relax
   \message{amstex already loaded}\makeatother\endinput% 2.09 compatability
\else
   \@ifpackageloaded{amstex}%
      {\message{amstex already loaded}\makeatother\endinput}
      {}
   \@ifpackageloaded{amsgen}%
      {\message{amsgen already loaded}\makeatother\endinput}
      {}
\fi
%%%%%%%%%%%%%%%%%%%%%%%%%%%%%%%%%%%%%%%%%%%%%%%%%%%%%%%%%%%%%%%%%%%%%%%%
%%
%
%
%  Macros to define some AMS LaTeX constructs when 
%  AMS LaTeX has not been loaded
% 
% These macros are copied from the AMS-TeX package for doing
% multiple integrals.
%
\let\DOTSI\relax
\def\RIfM@{\relax\ifmmode}%
\def\FN@{\futurelet\next}%
\newcount\intno@
\def\iint{\DOTSI\intno@\tw@\FN@\ints@}%
\def\iiint{\DOTSI\intno@\thr@@\FN@\ints@}%
\def\iiiint{\DOTSI\intno@4 \FN@\ints@}%
\def\idotsint{\DOTSI\intno@\z@\FN@\ints@}%
\def\ints@{\findlimits@\ints@@}%
\newif\iflimtoken@
\newif\iflimits@
\def\findlimits@{\limtoken@true\ifx\next\limits\limits@true
 \else\ifx\next\nolimits\limits@false\else
 \limtoken@false\ifx\ilimits@\nolimits\limits@false\else
 \ifinner\limits@false\else\limits@true\fi\fi\fi\fi}%
\def\multint@{\int\ifnum\intno@=\z@\intdots@                          %1
 \else\intkern@\fi                                                    %2
 \ifnum\intno@>\tw@\int\intkern@\fi                                   %3
 \ifnum\intno@>\thr@@\int\intkern@\fi                                 %4
 \int}%                                                               %5
\def\multintlimits@{\intop\ifnum\intno@=\z@\intdots@\else\intkern@\fi
 \ifnum\intno@>\tw@\intop\intkern@\fi
 \ifnum\intno@>\thr@@\intop\intkern@\fi\intop}%
\def\intic@{%
    \mathchoice{\hskip.5em}{\hskip.4em}{\hskip.4em}{\hskip.4em}}%
\def\negintic@{\mathchoice
 {\hskip-.5em}{\hskip-.4em}{\hskip-.4em}{\hskip-.4em}}%
\def\ints@@{\iflimtoken@                                              %1
 \def\ints@@@{\iflimits@\negintic@
   \mathop{\intic@\multintlimits@}\limits                             %2
  \else\multint@\nolimits\fi                                          %3
  \eat@}%                                                             %4
 \else                                                                %5
 \def\ints@@@{\iflimits@\negintic@
  \mathop{\intic@\multintlimits@}\limits\else
  \multint@\nolimits\fi}\fi\ints@@@}%
\def\intkern@{\mathchoice{\!\!\!}{\!\!}{\!\!}{\!\!}}%
\def\plaincdots@{\mathinner{\cdotp\cdotp\cdotp}}%
\def\intdots@{\mathchoice{\plaincdots@}%
 {{\cdotp}\mkern1.5mu{\cdotp}\mkern1.5mu{\cdotp}}%
 {{\cdotp}\mkern1mu{\cdotp}\mkern1mu{\cdotp}}%
 {{\cdotp}\mkern1mu{\cdotp}\mkern1mu{\cdotp}}}%
%
%
%  These macros are for doing the AMS \text{} construct
%
\def\RIfM@{\relax\protect\ifmmode}
\def\text{\RIfM@\expandafter\text@\else\expandafter\mbox\fi}
\let\nfss@text\text
\def\text@#1{\mathchoice
   {\textdef@\displaystyle\f@size{#1}}%
   {\textdef@\textstyle\tf@size{\firstchoice@false #1}}%
   {\textdef@\textstyle\sf@size{\firstchoice@false #1}}%
   {\textdef@\textstyle \ssf@size{\firstchoice@false #1}}%
   \glb@settings}

\def\textdef@#1#2#3{\hbox{{%
                    \everymath{#1}%
                    \let\f@size#2\selectfont
                    #3}}}
\newif\iffirstchoice@
\firstchoice@true
%
%    Old Scheme for \text
%
%\def\rmfam{\z@}%
%\newif\iffirstchoice@
%\firstchoice@true
%\def\textfonti{\the\textfont\@ne}%
%\def\textfontii{\the\textfont\tw@}%
%\def\text{\RIfM@\expandafter\text@\else\expandafter\text@@\fi}%
%\def\text@@#1{\leavevmode\hbox{#1}}%
%\def\text@#1{\mathchoice
% {\hbox{\everymath{\displaystyle}\def\textfonti{\the\textfont\@ne}%
%  \def\textfontii{\the\textfont\tw@}\textdef@@ T#1}}%
% {\hbox{\firstchoice@false
%  \everymath{\textstyle}\def\textfonti{\the\textfont\@ne}%
%  \def\textfontii{\the\textfont\tw@}\textdef@@ T#1}}%
% {\hbox{\firstchoice@false
%  \everymath{\scriptstyle}\def\textfonti{\the\scriptfont\@ne}%
%  \def\textfontii{\the\scriptfont\tw@}\textdef@@ S\rm#1}}%
% {\hbox{\firstchoice@false
%  \everymath{\scriptscriptstyle}\def\textfonti
%  {\the\scriptscriptfont\@ne}%
%  \def\textfontii{\the\scriptscriptfont\tw@}\textdef@@ s\rm#1}}}%
%\def\textdef@@#1{\textdef@#1\rm\textdef@#1\bf\textdef@#1\sl
%    \textdef@#1\it}%
%\def\DN@{\def\next@}%
%\def\eat@#1{}%
%\def\textdef@#1#2{%
% \DN@{\csname\expandafter\eat@\string#2fam\endcsname}%
% \if S#1\edef#2{\the\scriptfont\next@\relax}%
% \else\if s#1\edef#2{\the\scriptscriptfont\next@\relax}%
% \else\edef#2{\the\textfont\next@\relax}\fi\fi}%
%
%
%These are the AMS constructs for multiline limits.
%
\def\Let@{\relax\iffalse{\fi\let\\=\cr\iffalse}\fi}%
\def\vspace@{\def\vspace##1{\crcr\noalign{\vskip##1\relax}}}%
\def\multilimits@{\bgroup\vspace@\Let@
 \baselineskip\fontdimen10 \scriptfont\tw@
 \advance\baselineskip\fontdimen12 \scriptfont\tw@
 \lineskip\thr@@\fontdimen8 \scriptfont\thr@@
 \lineskiplimit\lineskip
 \vbox\bgroup\ialign\bgroup\hfil$\m@th\scriptstyle{##}$\hfil\crcr}%
\def\Sb{_\multilimits@}%
\def\endSb{\crcr\egroup\egroup\egroup}%
\def\Sp{^\multilimits@}%
\let\endSp\endSb
%
%
%These are AMS constructs for horizontal arrows
%
\newdimen\ex@
\ex@.2326ex
\def\rightarrowfill@#1{$#1\m@th\mathord-\mkern-6mu\cleaders
 \hbox{$#1\mkern-2mu\mathord-\mkern-2mu$}\hfill
 \mkern-6mu\mathord\rightarrow$}%
\def\leftarrowfill@#1{$#1\m@th\mathord\leftarrow\mkern-6mu\cleaders
 \hbox{$#1\mkern-2mu\mathord-\mkern-2mu$}\hfill\mkern-6mu\mathord-$}%
\def\leftrightarrowfill@#1{$#1\m@th\mathord\leftarrow
\mkern-6mu\cleaders
 \hbox{$#1\mkern-2mu\mathord-\mkern-2mu$}\hfill
 \mkern-6mu\mathord\rightarrow$}%
\def\overrightarrow{\mathpalette\overrightarrow@}%
\def\overrightarrow@#1#2{\vbox{\ialign{##\crcr\rightarrowfill@#1\crcr
 \noalign{\kern-\ex@\nointerlineskip}$\m@th\hfil#1#2\hfil$\crcr}}}%
\let\overarrow\overrightarrow
\def\overleftarrow{\mathpalette\overleftarrow@}%
\def\overleftarrow@#1#2{\vbox{\ialign{##\crcr\leftarrowfill@#1\crcr
 \noalign{\kern-\ex@\nointerlineskip}$\m@th\hfil#1#2\hfil$\crcr}}}%
\def\overleftrightarrow{\mathpalette\overleftrightarrow@}%
\def\overleftrightarrow@#1#2{\vbox{\ialign{##\crcr
   \leftrightarrowfill@#1\crcr
 \noalign{\kern-\ex@\nointerlineskip}$\m@th\hfil#1#2\hfil$\crcr}}}%
\def\underrightarrow{\mathpalette\underrightarrow@}%
\def\underrightarrow@#1#2{\vtop{\ialign{##\crcr$\m@th\hfil#1#2\hfil
  $\crcr\noalign{\nointerlineskip}\rightarrowfill@#1\crcr}}}%
\let\underarrow\underrightarrow
\def\underleftarrow{\mathpalette\underleftarrow@}%
\def\underleftarrow@#1#2{\vtop{\ialign{##\crcr$\m@th\hfil#1#2\hfil
  $\crcr\noalign{\nointerlineskip}\leftarrowfill@#1\crcr}}}%
\def\underleftrightarrow{\mathpalette\underleftrightarrow@}%
\def\underleftrightarrow@#1#2{\vtop{\ialign{##\crcr$\m@th
  \hfil#1#2\hfil$\crcr
 \noalign{\nointerlineskip}\leftrightarrowfill@#1\crcr}}}%
%%%%%%%%%%%%%%%%%%%%%

% 94.0815 by Jon:

\def\qopnamewl@#1{\mathop{\operator@font#1}\nlimits@}
\let\nlimits@\displaylimits
\def\setboxz@h{\setbox\z@\hbox}


\def\varlim@#1#2{\mathop{\vtop{\ialign{##\crcr
 \hfil$#1\m@th\operator@font lim$\hfil\crcr
 \noalign{\nointerlineskip}#2#1\crcr
 \noalign{\nointerlineskip\kern-\ex@}\crcr}}}}

 \def\rightarrowfill@#1{\m@th\setboxz@h{$#1-$}\ht\z@\z@
  $#1\copy\z@\mkern-6mu\cleaders
  \hbox{$#1\mkern-2mu\box\z@\mkern-2mu$}\hfill
  \mkern-6mu\mathord\rightarrow$}
\def\leftarrowfill@#1{\m@th\setboxz@h{$#1-$}\ht\z@\z@
  $#1\mathord\leftarrow\mkern-6mu\cleaders
  \hbox{$#1\mkern-2mu\copy\z@\mkern-2mu$}\hfill
  \mkern-6mu\box\z@$}


\def\projlim{\qopnamewl@{proj\,lim}}
\def\injlim{\qopnamewl@{inj\,lim}}
\def\varinjlim{\mathpalette\varlim@\rightarrowfill@}
\def\varprojlim{\mathpalette\varlim@\leftarrowfill@}
\def\varliminf{\mathpalette\varliminf@{}}
\def\varliminf@#1{\mathop{\underline{\vrule\@depth.2\ex@\@width\z@
   \hbox{$#1\m@th\operator@font lim$}}}}
\def\varlimsup{\mathpalette\varlimsup@{}}
\def\varlimsup@#1{\mathop{\overline
  {\hbox{$#1\m@th\operator@font lim$}}}}

%
%%%%%%%%%%%%%%%%%%%%%%%%%%%%%%%%%%%%%%%%%%%%%%%%%%%%%%%%%%%%%%%%%%%%%
%
\def\tfrac#1#2{{\textstyle {#1 \over #2}}}%
\def\dfrac#1#2{{\displaystyle {#1 \over #2}}}%
\def\binom#1#2{{#1 \choose #2}}%
\def\tbinom#1#2{{\textstyle {#1 \choose #2}}}%
\def\dbinom#1#2{{\displaystyle {#1 \choose #2}}}%
\def\QATOP#1#2{{#1 \atop #2}}%
\def\QTATOP#1#2{{\textstyle {#1 \atop #2}}}%
\def\QDATOP#1#2{{\displaystyle {#1 \atop #2}}}%
\def\QABOVE#1#2#3{{#2 \above#1 #3}}%
\def\QTABOVE#1#2#3{{\textstyle {#2 \above#1 #3}}}%
\def\QDABOVE#1#2#3{{\displaystyle {#2 \above#1 #3}}}%
\def\QOVERD#1#2#3#4{{#3 \overwithdelims#1#2 #4}}%
\def\QTOVERD#1#2#3#4{{\textstyle {#3 \overwithdelims#1#2 #4}}}%
\def\QDOVERD#1#2#3#4{{\displaystyle {#3 \overwithdelims#1#2 #4}}}%
\def\QATOPD#1#2#3#4{{#3 \atopwithdelims#1#2 #4}}%
\def\QTATOPD#1#2#3#4{{\textstyle {#3 \atopwithdelims#1#2 #4}}}%
\def\QDATOPD#1#2#3#4{{\displaystyle {#3 \atopwithdelims#1#2 #4}}}%
\def\QABOVED#1#2#3#4#5{{#4 \abovewithdelims#1#2#3 #5}}%
\def\QTABOVED#1#2#3#4#5{{\textstyle 
   {#4 \abovewithdelims#1#2#3 #5}}}%
\def\QDABOVED#1#2#3#4#5{{\displaystyle 
   {#4 \abovewithdelims#1#2#3 #5}}}%
%
% Macros for text size operators:

%JCS - added braces and \mathop around \displaystyle\int, etc.
%
\def\tint{\mathop{\textstyle \int}}%
\def\tiint{\mathop{\textstyle \iint }}%
\def\tiiint{\mathop{\textstyle \iiint }}%
\def\tiiiint{\mathop{\textstyle \iiiint }}%
\def\tidotsint{\mathop{\textstyle \idotsint }}%
\def\toint{\mathop{\textstyle \oint}}%
\def\tsum{\mathop{\textstyle \sum }}%
\def\tprod{\mathop{\textstyle \prod }}%
\def\tbigcap{\mathop{\textstyle \bigcap }}%
\def\tbigwedge{\mathop{\textstyle \bigwedge }}%
\def\tbigoplus{\mathop{\textstyle \bigoplus }}%
\def\tbigodot{\mathop{\textstyle \bigodot }}%
\def\tbigsqcup{\mathop{\textstyle \bigsqcup }}%
\def\tcoprod{\mathop{\textstyle \coprod }}%
\def\tbigcup{\mathop{\textstyle \bigcup }}%
\def\tbigvee{\mathop{\textstyle \bigvee }}%
\def\tbigotimes{\mathop{\textstyle \bigotimes }}%
\def\tbiguplus{\mathop{\textstyle \biguplus }}%
%
%
%Macros for display size operators:
%

\def\dint{\mathop{\displaystyle \int}}%
\def\diint{\mathop{\displaystyle \iint }}%
\def\diiint{\mathop{\displaystyle \iiint }}%
\def\diiiint{\mathop{\displaystyle \iiiint }}%
\def\didotsint{\mathop{\displaystyle \idotsint }}%
\def\doint{\mathop{\displaystyle \oint}}%
\def\dsum{\mathop{\displaystyle \sum }}%
\def\dprod{\mathop{\displaystyle \prod }}%
\def\dbigcap{\mathop{\displaystyle \bigcap }}%
\def\dbigwedge{\mathop{\displaystyle \bigwedge }}%
\def\dbigoplus{\mathop{\displaystyle \bigoplus }}%
\def\dbigodot{\mathop{\displaystyle \bigodot }}%
\def\dbigsqcup{\mathop{\displaystyle \bigsqcup }}%
\def\dcoprod{\mathop{\displaystyle \coprod }}%
\def\dbigcup{\mathop{\displaystyle \bigcup }}%
\def\dbigvee{\mathop{\displaystyle \bigvee }}%
\def\dbigotimes{\mathop{\displaystyle \bigotimes }}%
\def\dbiguplus{\mathop{\displaystyle \biguplus }}%
%
%Companion to stackrel
\def\stackunder#1#2{\mathrel{\mathop{#2}\limits_{#1}}}%
%
%
% These are AMS environments that will be defined to
% be verbatims if amstex has not actually been 
% loaded
%
%
\begingroup \catcode `|=0 \catcode `[= 1
\catcode`]=2 \catcode `\{=12 \catcode `\}=12
\catcode`\\=12 
|gdef|@alignverbatim#1\end{align}[#1|end[align]]
|gdef|@salignverbatim#1\end{align*}[#1|end[align*]]

|gdef|@alignatverbatim#1\end{alignat}[#1|end[alignat]]
|gdef|@salignatverbatim#1\end{alignat*}[#1|end[alignat*]]

|gdef|@xalignatverbatim#1\end{xalignat}[#1|end[xalignat]]
|gdef|@sxalignatverbatim#1\end{xalignat*}[#1|end[xalignat*]]

|gdef|@gatherverbatim#1\end{gather}[#1|end[gather]]
|gdef|@sgatherverbatim#1\end{gather*}[#1|end[gather*]]

|gdef|@gatherverbatim#1\end{gather}[#1|end[gather]]
|gdef|@sgatherverbatim#1\end{gather*}[#1|end[gather*]]


|gdef|@multilineverbatim#1\end{multiline}[#1|end[multiline]]
|gdef|@smultilineverbatim#1\end{multiline*}[#1|end[multiline*]]

|gdef|@arraxverbatim#1\end{arrax}[#1|end[arrax]]
|gdef|@sarraxverbatim#1\end{arrax*}[#1|end[arrax*]]

|gdef|@tabulaxverbatim#1\end{tabulax}[#1|end[tabulax]]
|gdef|@stabulaxverbatim#1\end{tabulax*}[#1|end[tabulax*]]


|endgroup
  

  
\def\align{\@verbatim \frenchspacing\@vobeyspaces \@alignverbatim
You are using the "align" environment in a style in which it is not defined.}
\let\endalign=\endtrivlist
 
\@namedef{align*}{\@verbatim\@salignverbatim
You are using the "align*" environment in a style in which it is not defined.}
\expandafter\let\csname endalign*\endcsname =\endtrivlist




\def\alignat{\@verbatim \frenchspacing\@vobeyspaces \@alignatverbatim
You are using the "alignat" environment in a style in which it is not defined.}
\let\endalignat=\endtrivlist
 
\@namedef{alignat*}{\@verbatim\@salignatverbatim
You are using the "alignat*" environment in a style in which it is not defined.}
\expandafter\let\csname endalignat*\endcsname =\endtrivlist




\def\xalignat{\@verbatim \frenchspacing\@vobeyspaces \@xalignatverbatim
You are using the "xalignat" environment in a style in which it is not defined.}
\let\endxalignat=\endtrivlist
 
\@namedef{xalignat*}{\@verbatim\@sxalignatverbatim
You are using the "xalignat*" environment in a style in which it is not defined.}
\expandafter\let\csname endxalignat*\endcsname =\endtrivlist




\def\gather{\@verbatim \frenchspacing\@vobeyspaces \@gatherverbatim
You are using the "gather" environment in a style in which it is not defined.}
\let\endgather=\endtrivlist
 
\@namedef{gather*}{\@verbatim\@sgatherverbatim
You are using the "gather*" environment in a style in which it is not defined.}
\expandafter\let\csname endgather*\endcsname =\endtrivlist


\def\multiline{\@verbatim \frenchspacing\@vobeyspaces \@multilineverbatim
You are using the "multiline" environment in a style in which it is not defined.}
\let\endmultiline=\endtrivlist
 
\@namedef{multiline*}{\@verbatim\@smultilineverbatim
You are using the "multiline*" environment in a style in which it is not defined.}
\expandafter\let\csname endmultiline*\endcsname =\endtrivlist


\def\arrax{\@verbatim \frenchspacing\@vobeyspaces \@arraxverbatim
You are using a type of "array" construct that is only allowed in AmS-LaTeX.}
\let\endarrax=\endtrivlist

\def\tabulax{\@verbatim \frenchspacing\@vobeyspaces \@tabulaxverbatim
You are using a type of "tabular" construct that is only allowed in AmS-LaTeX.}
\let\endtabulax=\endtrivlist

 
\@namedef{arrax*}{\@verbatim\@sarraxverbatim
You are using a type of "array*" construct that is only allowed in AmS-LaTeX.}
\expandafter\let\csname endarrax*\endcsname =\endtrivlist

\@namedef{tabulax*}{\@verbatim\@stabulaxverbatim
You are using a type of "tabular*" construct that is only allowed in AmS-LaTeX.}
\expandafter\let\csname endtabulax*\endcsname =\endtrivlist

% macro to simulate ams tag construct


% This macro is a fix to eqnarray
\def\@@eqncr{\let\@tempa\relax
    \ifcase\@eqcnt \def\@tempa{& & &}\or \def\@tempa{& &}%
      \else \def\@tempa{&}\fi
     \@tempa
     \if@eqnsw
        \iftag@
           \@taggnum
        \else
           \@eqnnum\stepcounter{equation}%
        \fi
     \fi
     \global\tag@false
     \global\@eqnswtrue
     \global\@eqcnt\z@\cr}


% This macro is a fix to the equation environment
 \def\endequation{%
     \ifmmode\ifinner % FLEQN hack
      \iftag@
        \addtocounter{equation}{-1} % undo the increment made in the begin part
        $\hfil
           \displaywidth\linewidth\@taggnum\egroup \endtrivlist
        \global\tag@false
        \global\@ignoretrue   
      \else
        $\hfil
           \displaywidth\linewidth\@eqnnum\egroup \endtrivlist
        \global\tag@false
        \global\@ignoretrue 
      \fi
     \else   
      \iftag@
        \addtocounter{equation}{-1} % undo the increment made in the begin part
        \eqno \hbox{\@taggnum}
        \global\tag@false%
        $$\global\@ignoretrue
      \else
        \eqno \hbox{\@eqnnum}% $$ BRACE MATCHING HACK
        $$\global\@ignoretrue
      \fi
     \fi\fi
 } 

 \newif\iftag@ \tag@false
 
 \def\tag{\@ifnextchar*{\@tagstar}{\@tag}}
 \def\@tag#1{%
     \global\tag@true
     \global\def\@taggnum{(#1)}}
 \def\@tagstar*#1{%
     \global\tag@true
     \global\def\@taggnum{#1}%  
}

% Do not add anything to the end of this file.  
% The last section of the file is loaded only if 
% amstex has not been.



\makeatother
\endinput


% See the ``Article customise'' template for come common customisations

\title{Physics C2801 Fall 2013 Problem Set 10}
\author{Laura Havener}
\date{Nov 20} % delete this line to display the current date

%%% BEGIN DOCUMENT
\begin{document}


\maketitle

Problem 1.  Hyperbolic functions in Lorentz transformations and hyperbolic identities \\ \\
a. Show the indentities: \\
\begin{eqnarray*}
\cosh{y}^{2}-\sinh{y}^{2} &=& (\frac{e^{x}+e^{-x}}{2})^{2}- (\frac{e^{x}-e^{-x}}{2})^{2} \\
&=& \frac{e^{2x}+e^{-2x}+2}{4}-\frac{e^{2x}+e^{-2x}-2}{4} = 1/2+1/2 = 1 \\
\cosh{y_{1}+y_{2}} &=& \frac{e^{y_{1}+y_{2}}+e^{-y_{1}-y_{2}}}{2} =  \frac{e^{y_{1}+y_{2}}+e^{-y_{1}-y_{2}}}{2}+\frac{e^{y_{1}}e^{-y_{2}}}{4}-\frac{e^{y_{1}}e^{-y_{2}}}{4} +\frac{e^{-y_{1}}e^{y_{2}}}{4} -\frac{e^{-y_{1}}e^{y_{2}}}{4}\\
&=& \frac{e^{y_{1}}}{4}(e^{y_{2}}+e^{-y_{2}})+\frac{e^{-y_{1}}}{4}((e^{y_{2}}+e^{-y_{2}})+\frac{e^{y_{1}}}{4}(e^{y_{2}}-e^{-y_{2}})-\frac{e^{-y_{1}}}{4}((e^{y_{2}}-e^{-y_{2}})\\
&=& \frac{e^{y_{1}}+e^{-y_{1}}}{2}\frac{e^{y_{2}}+e^{-y_{2}}}{2}+ \frac{e^{y_{1}}-e^{-y_{1}}}{2}\frac{e^{y_{2}}-e^{-y_{2}}}{2} = \cosh{y_{1}}\cosh{y_{2}}+\sinh{y_{1}}\sinh{y_{2}} \\
\sinh{y_{1}+y_{2}} &=& \frac{e^{y_{1}+y_{2}}-e^{-y_{1}-y_{2}}}{2} =  \frac{e^{y_{1}+y_{2}}-e^{-y_{1}-y_{2}}}{2}+\frac{e^{y_{1}}e^{-y_{2}}}{4}-\frac{e^{y_{1}}e^{-y_{2}}}{4} +\frac{e^{-y_{1}}e^{y_{2}}}{4} -\frac{e^{-y_{1}}e^{y_{2}}}{4}\\
&=& \frac{e^{y_{1}}}{4}(e^{y_{2}}+e^{-y_{2}})-\frac{e^{-y_{1}}}{4}((e^{y_{2}}+e^{-y_{2}})+\frac{e^{y_{1}}}{4}(e^{y_{2}}-e^{-y_{2}})+\frac{e^{-y_{1}}}{4}((e^{y_{2}}-e^{-y_{2}})\\
&=& \frac{e^{y_{1}}+e^{-y_{1}}}{2}\frac{e^{y_{2}}-e^{-y_{2}}}{2}+ \frac{e^{y_{1}}-e^{-y_{1}}}{2}\frac{e^{y_{2}}+e^{-y_{2}}}{2} = \cosh{y_{1}}\sinh{y_{2}}+\sinh{y_{1}}\cosh{y_{2}} \\
\tanh{y_{1}+y_{2}} &=& \frac{\sin{y_{1}+y_{2}}}{\cosh{y_{1}+y_{2}}} = \frac{e^{y_{1}+y_{2}}+e^{-y_{1}-y_{2}}}{e^{y_{1}+y_{2}}-e^{-y_{1}-y_{2}}} 
\end{eqnarray*} \\
b. Write the boost rapidity in terms of the boost velocity. \\
\begin{eqnarray*}
\beta_{B} &=& \tanh{y_{B}} = \frac{e^{y_{B}}-e^{-y_{B}}}{e^{y_{B}}+e^{-y_{B}}} \\
&=& \frac{e^{2y_{B}}-1}{e^{2y_{B}}+1} \\
\beta_{B}(1+e^{2y_{B}}) &=& e^{2y_{B}}-1 \\
e^{2y_{B}}(\beta_{B}-1) &=& -(1+\beta_{B}) \\
e^{2y_{B}} &=& \frac{1+\beta_{B}}{1-\beta_{B}} \\
y_{B} &=& \frac{1}{2}\ln{\frac{1+\beta_{B}}{1-\beta_{B}}} 
\end{eqnarray*} \\
c. Show that 2 successive boosts, $y_{B1}$ and $y_{B2}$, is the same as one boost $y_{B12}=y_{B1}+y_{B2}$. First write this transformation in terms of a matrix. \\
\begin{eqnarray*} 
\begin{bmatrix}
t' \\
x' 
\end{bmatrix} &=& \begin{bmatrix}
	\cosh{y_{B}} & -\sinh{y_{B}} \\
	-\sinh{y_{B}} & \cosh{y_{B}} 
	\end{bmatrix}\begin{bmatrix}
	t \\
	x 
	\end{bmatrix} 
\end{eqnarray*} \\
Now we can multiply the 2 boost matrixes by each other to determine the boost matrix for both boosts. \\
\begin{eqnarray*}
\begin{bmatrix}
	\cosh{y_{B2}} & -\sinh{y_{B2}} \\
	-\sinh{y_{B2}} & \cosh{y_{B2}} 
	\end{bmatrix}\begin{bmatrix}
	\cosh{y_{B1}} & -\sinh{y_{B1}} \\
	-\sinh{y_{B1}} & \cosh{y_{B1}} 
	\end{bmatrix} = \\
 \begin{bmatrix}
	\cosh{y_{B1}}\cosh{y_{B2}}+\sinh{y_{B1}}\sinh{y_{b2}} & -\cosh{y_{B1}}\sinh{y_{B2}}-\cosh{y_{B2}}\sin{y_{B1}} \\
	-\cosh{y_{B1}}\sinh{y_{B2}}-\cosh{y_{B2}}\sin{y_{B1}} & \cosh{y_{B1}}\cosh{y_{B2}}+\sinh{y_{B1}}\sinh{y_{b2}}
	\end{bmatrix} \\
= \begin{bmatrix}
	\cosh{y_{B1}+y_{B2}} & -\sinh{y_{B1}+y_{B2}} \\
	-\sinh{y_{B1}+y_{B2}} & \cosh{y_{B1}+y_{B2}} 
	\end{bmatrix}
	\end{eqnarray*} 
d. Show how the rapidity transforms. \\
\begin{eqnarray*}
y' &=& y-y_{B} = \frac{1}{2}\ln{\frac{1+\beta}{1-\beta}}-\frac{1}{2}\ln{\frac{1+\beta_{B}}{1-\beta_{B}}} \\
&=& \frac{1}{2}\ln{\frac{(1+\beta)(1-\beta_{B})}{(1-\beta)(1+\beta_{B})}} \\
&=& \frac{1}{2}\ln{\frac{1+\beta-\beta_{B}-\beta\beta_{B}}{1-\beta+\beta_{B}-\beta\beta_{B}}} 
\end{eqnarray*} \\
From pset 9, the boost velocity for 2 boosts in opposite directions is: $\beta'=\frac{\beta-\beta_{B}}{1-\beta\beta_{B}}$. \\
\begin{eqnarray*} 
y' &=& \frac{1}{2}\ln{\frac{1+\beta'}{1-\beta'}} = \frac{1}{2}\ln{\frac{1+\frac{\beta-\beta_{B}}{1-\beta\beta_{B}}}{1-\frac{\beta-\beta_{B}}{1-\beta\beta_{B}}}} \\
&=& \frac{1}{2}\ln{\frac{1-\beta_{B}\beta+\beta-\beta_{B}}{1-\beta\beta_{B}-\beta+\beta_{B}}}  
\end{eqnarray*} \\
The 2 forms match so the transformations of rapidity hold. \\


Problem 2. Boosts along general direction \\ \\
a. Using the decomposition show that the Lorentz transformation for a general boost is given by the equations in part a of the problem. \\
\begin{eqnarray*}
r_{h} &=& \vec{r}\cdot{\hat{\beta_{B}}} \\
r_{v} &=& \vec{r}-(\vec{r}\cdot{\hat{\beta_{B}}}) \hat{\beta_{B}} \\
t' &=& \gamma_{B}t-\gamma(\vec{\beta_{B}}\cdot{\vec{r}}) \\
r'_{h} &=& -\gamma_{B}\vec{\beta_{B}}t+\gamma_{B}\vec{r}\cdot{\vec{\beta_{B}}} \\
r'_{v} &=& \vec{r}-(\vec{r}\cdot{\hat{\beta_{B}}} )\hat{\beta_{B}} 
\end{eqnarray*} \\
Then when you add the horizontal and vectical components together into one equation, you obtain the results given. \\
b. Suppose that $\vec{\beta_{B}}=\beta_{B}\hat{x}$ and find the lorentz transformations. \\
\begin{eqnarray*}
t' &=& \gamma{t}-\gamma(\beta\hat{x}\cdot{\vec{r}})\\
&=& \gamma{t}-\gamma\beta{x} \\
\vec{r}' &=& \vec{r}+(\gamma-1)\frac{\beta\hat{x}(\beta\hat{x}\cdot{\vec{r})}}{\beta^{2}}-\gamma\beta\hat{x}t \\
&=& \vec{r}+(\gamma-1)\frac{\beta^{2}\hat{x}x}{\beta^{2}}-\gamma\beta\hat{x}t \\
&=& \vec{r}+(\gamma-1)x\hat{x}-t\gamma\beta\hat{x} \\
x' &=& x+(\gamma-1)x-t\gamma\beta = \gamma{x}-t\gamma\beta \\
y' &=& y \\
z' &=& z 
\end{eqnarray*} \\
c. Show that $(t')^{2}-\vec{r'}\cdot{\vec{r'}}=t^{2}-\vec{r}\cdot{\vec{r}}$: \\
\begin{eqnarray*}
(t')^{2} &=& \gamma^{2}t^{2}+\gamma^{2}(\vec{\beta}\cdot{\vec{r}})^{2}-2\gamma^{2}t\vec{\beta}\cdot{\vec{r}} \\
\vec{r'}\cdot{\vec{r'}} &=& \vec{r}\cdot{\vec{r}}+(\gamma-1)^{2}(\vec{\beta}\cdot{\vec{r}})^{2}/\vec{\beta^{2}}+2(\gamma-1)(\vec{\beta}\cdot{\vec{r}})^{2}/\vec{\beta}^{2}+\gamma^{2}\vec{\beta}^{2}t^{2}-2t\gamma(\vec{\beta}\cdot{\vec{r}})-2t\gamma(\gamma-1)(\vec{\beta}\cdot{\vec{r}}) \\
(t')^{2} -\vec{r'}\cdot{\vec{r'}} &=& \gamma^{2}t^{2}(1-\beta^{2})+(\gamma^{2}\vec{\beta^{2}}-(\gamma^{2}-2\gamma+1)(\vec{\beta}\cdot{\vec{r}})^{2}/\vec{\beta}^{2}-2(\gamma-1)(\vec{\beta}\cdot{\vec{r}})^{2}/\vec{\beta}^{2}-\vec{r}\cdot{\vec{r}} \\
&=&  t^{2}-\vec{r}\cdot{\vec{r}} +(-1+2\gamma-1)(\vec{\beta}\cdot{\vec{r}})^{2}/\vec{\beta}^{2}-2(\gamma-1)(\vec{\beta}\cdot{\vec{r}})^{2}/\vec{\beta}^{2} = \vec{r}\cdot{\vec{r}}-t^{2} 
\end{eqnarray*} \\
d. Now we want to write out the boost where $\vec{\beta}\cdot{\hat{k}}=0$. \\
\begin{eqnarray*} 
t' &=& \gamma{t}-\gamma(\beta_{x}x+\beta_{y}y) \\
\vec{r}' &=& \vec{r}+(\gamma-1)(\beta_{x}\hat{x}+\beta_{y}\hat{y})(\beta_{x}x+\beta_{y}y)/\vec{\beta^{2}}-\gamma(\beta_{x}\hat{x}+\beta_{y}\hat{y})t \\
x' &=& x+(\gamma-1)\beta_{x}(\beta_{x}x+\beta_{y}y)/\vec{\beta^{2}}-\gamma\beta_{x}t \\
y' &=& y+(\gamma-1)\beta_{y}(\beta_{x}x+\beta_{y}y)/\vec{\beta^{2}}-\gamma\beta_{y}t \\
z' &=& z \\
\begin{bmatrix}
	t' \\
	x' \\
	y' \\
	z' 
	\end{bmatrix} &=& \begin{bmatrix}
	\gamma & -\gamma\beta_{x} & -\gamma\beta_{y} & 0 \\
	-\gamma\beta_{x} & 1+(\gamma-1)\beta_{x}^{2}/\vec{\beta^{2}} & (\gamma-1)\beta_{x}\beta_{y}/\vec{\beta^{2}} & 0 \\
	-\gamma\beta_{y} &  (\gamma-1)\beta_{x}\beta_{y}/\vec{\beta^{2}} & 1+(\gamma-1)\beta_{y}^{2}/\vec{\beta^{2}} & 0 \\
	0 & 0 & 0 & 1 
	\end{bmatrix}\begin{bmatrix}
	t \\
	x \\
	y \\
	z 
	\end{bmatrix}	
\end{eqnarray*} \\
e. Write out the transformation matrix for these 2 boosts. The boost happens just in x first so make sure to multiply this matrix by the lorentz vector first. \\
\begin{eqnarray*} 
\beta_{x}^{2}, \beta_{y}^{2}, \beta_{x}\beta_{y} &\approx& 0 \\
L''&=& L(\vec{\beta}')L(\beta) = \begin{bmatrix}
	\gamma' & -\gamma'\beta'_{x} & -\gamma'\beta'_{y} & 0 \\
	-\gamma'\beta'_{x} & 1 & 0 & 0 \\
	-\gamma'\beta'_{y} & 0 & 1 & 0 \\
	0 & 0 & 0 & 1 
	\end{bmatrix}\begin{bmatrix}
	\gamma & -\beta\gamma & 0 & 0 \\
	-\beta\gamma & \gamma & 0 & 0 \\
	0 & 0 & 1 & 0 \\
	0 & 0 & 0 & 1 
	\end{bmatrix} \\
&=& \begin{bmatrix}
	\gamma'\gamma(1+\beta'_{x}\beta) &-\gamma'\gamma(\beta+\beta'_{x}) & -\gamma'\beta'_{y} & 0 \\
	-\gamma'\gamma\beta'_{x}-\gamma\beta & \gamma'\gamma\beta'_{x}\beta+\gamma & 0 & 0 \\
	-\gamma'\gamma\beta'_{y} & \gamma'\gamma\beta_{y}'\beta & 1 & 0 \\
	0 & 0 & 0 & 1 
	\end{bmatrix}
\end{eqnarray*}
f. Let's evaluate the relative velocity for both boosts together by using $X=(t, 0, 0, 0)$. \\
\begin{eqnarray*}
\begin{bmatrix}
	\gamma'\gamma(1+\beta'_{x}\beta) &-\gamma'\gamma(\beta+\beta'_{x}) & -\gamma'\beta'_{y} & 0 \\
	-\gamma'\gamma\beta'_{x}-\gamma\beta & \gamma'\gamma\beta'_{x}\beta+\gamma & 0 & 0 \\
	-\gamma'\gamma\beta'_{y} & \gamma'\gamma\beta_{y}'\beta & 1 & 0 \\
	0 & 0 & 0 & 1 
	\end{bmatrix}\begin{bmatrix}
	t \\
	0 \\
	0 \\
	0 
	\end{bmatrix} &=& \begin{bmatrix}
	\gamma'\gamma(1+\beta'_{x}\beta)t \\
	(-\gamma'\gamma\beta'_{x}-\gamma\beta)t \\
	-\gamma'\gamma\beta'_{y}t \\
	0 
	\end{bmatrix}
	\end{eqnarray*} \\
To find $\beta''$ divide the x and y components by the time component of the lorentz vector. \\
\begin{eqnarray*}
\beta''_{x} &=& \frac{(-\gamma'\gamma\beta'_{x}-\gamma\beta)}{\gamma'\gamma(1+\beta'_{x}\beta)} = -\frac{\gamma'\beta'_{x}+\beta}{\gamma'(1+\beta'_{x}\beta)} \\
\beta''_{y} &=& -\frac{\gamma'\gamma\beta'_{y}}{\gamma'\gamma(1+\beta'_{x}\beta)} = -\frac{\beta'_{y}}{1+\beta'_{x}\beta} 
\end{eqnarray*} \\
g. Now we want to boost back to the original frame using the velocity we just found. This can be done by multiplying the lorentz boost matrix from part f by the matrix found in part e. First let's make some approximations to simplify the results. Since we are going backwards from the original boost, all the velocities switch directions. \\
\begin{eqnarray*}
\gamma' &=& \frac{1}{\sqrt{1-(\beta'_{x})^{2}-(\beta')_{y}^{2}}} \approx 1+\frac{1}{2}(\beta_{x}^{2}+\beta_{y}^{2}) \approx 1 \\
\beta''_{x} &=& \frac{\beta'_{x}+\beta}{1+\beta'_{x}\beta} \\
\beta''_{y} &=& \frac{\beta'_{y}}{1+\beta'_{x}\beta} \\
\gamma'' &=& \frac{1}{1-(\beta'')_{x}^{2}-(\beta'')_{y}^{2}} \\
(\beta'')^{2} &=& \frac{(\beta')_{x}^{2}+\beta^{2}+2\beta'_{x}\beta+\beta'_{y}}{(1+\beta\beta'_{x})^{2}} \approx \frac{\beta^{2}+2\beta'_{x}\beta}{(1+\beta'_{x}\beta)^{2}} \\
\gamma'' &\approx& \sqrt{\frac{1}{1-\frac{\beta^{2}+2\beta'_{x}\beta}{(1+\beta'_{x}\beta)^{2}}}} = \frac{1+\beta'_{x}\beta}{\sqrt{1+2\beta'_{x}\beta+(\beta')_{x}^{2}\beta^{2}-\beta^{2}-2\beta'_{x}\beta}} \\
&\approx& \frac{1+\beta'_{x}\beta}{\sqrt{1-\beta^{2}}}  = \gamma(1+\beta'_{x}\beta) \\
L(-\beta'') &=&  \begin{bmatrix}
	\gamma'' & \gamma''\beta''_{x} & \gamma''\beta''_{y} & 0 \\
	\gamma''\beta''_{x} & 1+(\gamma''-1)(\beta'')_{x}^{2}/\vec{(\beta'')^{2}} & (\gamma''-1)\beta''_{x}\beta''_{y}/\vec{(\beta'')^{2}} & 0 \\
	\gamma''\beta''_{y} &  (\gamma''-1)\beta''_{x}\beta''_{y}/\vec{(\beta'')^{2}} & 1+(\gamma''-1)(\beta'')_{y}^{2}/\vec{(\beta'')^{2}} & 0 \\
	0 & 0 & 0 & 1 
	\end{bmatrix} \\
&=& \begin{bmatrix}
	\gamma(1+\beta'_{x}\beta) & \gamma(\beta'_{x}+\beta) & \gamma\beta'_{y} & 0 \\
	\gamma(\beta'_{x}+\beta) & \gamma(1+\beta'_{x}\beta) & \frac{(\gamma-1)\beta'_{y}}{\beta+2\beta'_{x}} & 0 & 0 \\
	 \gamma\beta'_{y} &  \frac{(\gamma-1)\beta'_{y}}{\beta+2\beta'_{x}} & 1 & 0 \\
0 & 0 & 0 & 1 
\end{bmatrix}  \\
(\gamma-1)\beta'_{y}(\beta+2\beta'_{x}) &=& (\gamma-1)\frac{\beta'_{y}}{\beta}(1-2\beta'_{x}) \approx \frac{\beta'_{y}}{\beta}(\gamma-1) \\
L(-\beta'')L'' &=& \begin{bmatrix}
	\gamma(1+\beta'_{x}\beta) & \gamma(\beta'_{x}+\beta) & \gamma\beta'_{y} & 0 \\
	\gamma(\beta'_{x}+\beta) & \gamma(1+\beta'_{x}\beta) & \frac{(\gamma-1)\beta'_{y}}{\beta} & 0  \\
	 \gamma\beta'_{y} & \frac{(\gamma-1)\beta'_{y}}{\beta} & 1 & 0 \\
0 & 0 & 0 & 1 
\end{bmatrix}\begin{bmatrix}
	\gamma(1+\beta'_{x}\beta) &-\gamma(\beta+\beta'_{x}) & -\beta'_{y} & 0 \\
	-\gamma\beta'_{x}-\gamma\beta & \gamma\beta'_{x}\beta+\gamma & 0 & 0  \\
	-\gamma\beta'_{y} & \gamma\beta_{y}'\beta & 1 & 0 \\
	0 & 0 & 0 & 1 
	\end{bmatrix} \\
&=& \begin{bmatrix}
	1 & 0 & 0 & 0 \\
	0 & 1 & \frac{\beta'_{y}(1-\gamma)}{\gamma\beta} & 0 \\
	0 & -\frac{\beta'_{y}(1-\gamma)}{\gamma\beta} & 1 & 0 \\
	0 & 0 & 0 & 1
	\end{bmatrix} 
\end{eqnarray*} \\

Problem 3. Four-velocity, four-acceleration, four-force	 \\ \\
a. Transform the acceleration. \\
\begin{eqnarray*}
\bold{a} &=& \frac{d\bold{U}}{d\tau} = \gamma(\frac{d\bold{U}}{dt}) \\
&=& \gamma(\frac{d}{dt}(\gamma, \gamma\beta_{x}, \gamma\beta_{y}, \gamma\beta_{z}) \\
&=& \gamma(\dot{\gamma}, \gamma\dot{\beta_{x}}+\beta_{x}\dot{\gamma}, \gamma\dot{\beta_{y}}+\beta_{y}\dot{\gamma}, \gamma\dot{\beta_{z}}+\beta_{z}\dot{\gamma}) \\
&=& \gamma(\dot{\gamma}, \gamma\dot{\beta_{x}}+\beta_{x}\dot{\gamma}, \gamma\dot{\beta_{y}}, \gamma\dot{\beta_{z}}) \\
\dot{\gamma} &=& \gamma^{3}\beta\dot{\vec{\beta}} \\
\vec{a_{p}} &=& \gamma(\gamma^{3}\beta_{x}\dot{\beta_{x}}, \gamma\dot{\beta_{x}}+\gamma^{3}\beta_{x}\beta_{x}\dot{\beta_{x}}, \gamma\dot{\beta_{y}}, \gamma\dot{\beta_{z}}) \\
a_{px} &=& \gamma(\gamma\dot{\beta_{x}}+\gamma^{3}\beta_{x}^{2}\dot{\beta_{x}}) \\
\dot{\beta_{x}} &=& \frac{a_{px}}{\gamma^{2}(1+\gamma^{2}\beta_{x}^{2})} \\
&=& \frac{a_{px}}{\gamma^{3}(1-\beta_{x}^{2}+\beta_{x}^{2})} = \frac{a_{px}}{\gamma^{3}} \\
a_{py} &=& \gamma^{2}\dot{\beta_{y}} \\
\dot{\beta_{y}} &=& \frac{a_{py}}{\gamma^{2}} \\
\dot{\beta_{z}} &=& \frac{a_{pz}}{\gamma^{2}} 
\end{eqnarray*} \\
b. Derive the four force in a similar way: \\
\begin{eqnarray*}
\bold{F} &=& \frac{d\bold{p}}{d\tau} = \gamma(\frac{d\bold{F}}{dt}) \\
&=& \gamma\frac{d}{dt}(\gamma{m}, \vec{p}) \\
&=& \gamma(\dot{\gamma}m, \vec{F}) \\
\bold{F_{x}} &=& \gamma\vec{F_{x}} 
\end{eqnarray*} \\
c. Find the Newtonian force given the proper force. \\
\begin{eqnarray*}
\bold{F} &=& L(-\beta_{x}) = \begin{bmatrix}
	\gamma & \beta_{x}\gamma & 0 & 0 \\
	\beta_{x}\gamma & \gamma & 0 & 0 \\
	0 & 0 & 1 & 0 \\
	0 & 0 & 0 & 1
	\end{bmatrix}\begin{bmatrix}
	0 \\
	F_{px} \\
	F_{py} \\
	F_{px} 
	\end{bmatrix} \\
&=& \begin{bmatrix}
	\beta_{x}\gamma{F_{px}} \\
	\gamma{F_{px}} \\
	F_{py} \\
	F_{pz} 
	\end{bmatrix} \\
\bold{F}_{x} &=& \gamma\vec{F}_{x} \\
F_{x} &=& F_{px} \\
F_{y} &=& \frac{F_{py}}{\gamma} \\
F_{z} &=& \frac{F_{pz}}{\gamma} 
\end{eqnarray*} \\

Problem 4. Kinematics of protons in the Large Hadron Collider \\ \\
a. We want to find the momentum of the protons using the energy and mass. Use the relativistic equation relating energy, mass and momentum. \\
\begin{eqnarray*}
E^{2} &=& p^{2}+m^{2} \\
p &=& \sqrt{E^{2}-m^{2}} = \sqrt{(4*10^{12}eV)^{2}-(0.94*10^{9}eV)^{2}} \\
p &=& \sqrt{16*10^{24}-(0.94^{2})*10^{18}}eV \\
&=&  4(1-2.76*10^{-8})TeV/c 
\end{eqnarray*} \\
b. Now find velocity, Lorentz factor and rapidities of the protons. \\
\begin{eqnarray*}
p &=& m\gamma\beta \\
p &=& \frac{m\beta}{1-\beta^{2}} \\
p^{2}(1-\beta^{2}) &=& m^{2}\beta^{2} \\
\beta &=& \sqrt{\frac{p^{2}}{m^{2}+p^{2}}}= \frac{p}{E} \\
\beta &\approx& 1(1-2.76*10^{-8}) \\
v &\approx& c \\
\gamma &=& \frac{p}{m\beta} = \frac{4*10^{12}eV}{0.94*10^{9}eV} = \frac{4}{0.94}10^{3} \\
&=& 4.3*10^{3}  \\
y &=& cosh^{-1}(\gamma) = 9.06
\end{eqnarray*} \\
c. The total energy of the proton-proton collisons in the lab, or center of mass frame, is just twice the energy of each proton so 8 TeV. \\
d. The rapidity boost will be the same a before since the boost velocity is the same as the velocity of the particles in the center of mass frame. \\
e. Use the invarient mass to clculate the energy and momentum of the proton moving with respect to the other protons rest frame. We can compare the invarient mass in the lab frame to that in the rest frame. \\
\begin{eqnarray*}
ds^{2} &=& 2m^{2}+2E_{1}E_{2}-2p_{1}p_{2}cos(\theta) \\
\theta &=& 180 \\
p_{2} &=& 0 \\
E_{2} &=& m \\
ds^{2} &=& 2m^{2}+2Em \\
&=& E_{com}^{2} \\
2m^{2}+2mE &=& E_{com}^{2} \\
E &=& \frac{E_{com}^{2}-2m^{2}}{2m} \\
&=& E_{com}^{2}/2m-m = 3.404*10^{16} eV = 34040 TeV  \\
p &=& \sqrt{(E)^{2}-m^{2}} = 34040 TeV
\end{eqnarray*} \\
f. Find the velocity, lorentz factor, and rapidity of the protons from part d. \\
\begin{eqnarray*}
\gamma &=& \frac{p}{m} = \frac{34040TeV}{0.94GeV} = 3.62*10^{7} \\
y &=& cosh^{-1}(\gamma) = 18.1 \\
\beta &\approx& 1-3.8*10^{-16} 
\end{eqnarray*} \\
g. To calculate the total energy, we can just add the energy of each of the protons. \\
\begin{eqnarray*}
E_{tot} &=& E_{1}+E_{2} = 34040TeV+0.94GeV = 34040TeV
\end{eqnarray*} \\
h. To find the force needed to keep the protons in the LHC right we need to calculate the force in the lab frame. $F=\frac{dp}{dt}$, where $p=m\gamma\beta$. \\
\begin{eqnarray*} 
F &=& m\gamma\dot{\beta} \\
\dot{\beta} &=& \frac{\beta^{2}}{R} \\
F &=& \frac{\gamma{m}\beta^{2}}{R}  = \frac{4300*0.94GeV}{27000/2\pi} = 9.41*10^{8}eV/m  = 1.504*10^{-10} N
\end{eqnarray*} \\

Problem 5. Cosmic ray muons \\ \\
a. To find the momentum of the muons use the relativistic energy equation. \\
\begin{eqnarray*}
p &=& \sqrt{E^{2}-m^{2}} = \sqrt{36*10^{18}eV-(0.106)^{2}10^{18}} = \sqrt{36-.106^{2}}10^{9}eV \\
&=& 5.999*10^{9}eV 
\end{eqnarray*} \\
b. Find the magnitude of the velocity from the formula found in problem 4. \\
\begin{eqnarray*}
\beta &=& \frac{p}{E} = \frac{5.999}{6} = 0.99984 \\
\gamma &=& \frac{p}{m\beta} = 56.6 \\
y &=& \frac{1}{2}\ln{\frac{1+\beta}{1-\beta}} = \frac{1}{2}\ln{\frac{1.9998}{0.0002}} \\
&=& 4.61 
\end{eqnarray*} \\
c. To calculate the energy and momentum of the electron and neutrino in the rest frame use the invarient mass. $P_{\mu}=P_{e}+P_{\nu}$. \\
\begin{eqnarray*} 
(P_{\mu}-P_{e})\cdot{(P_{\mu}-P_{e})} &=& P_{\nu}\cdot{P_{\nu}} \\
m_{mu}^{2}+m_{e}^{2}-2P_{\mu}\cdot{P_{e}} &=& m_{\nu}^{2}  = 0 \\
p_{\mu} &=& 0 \\
E_{\mu} &=& m_{\mu} \\
m_{\mu}^{2}+m_{e}^{2}-2E_{e}m_{\mu} &=& 0 \\
E_{e} &=& \frac{m_{\mu}^{2}+m_{e}^{2}}{2m_{\mu}} = \frac{106^{2}+0.511^{2}}{2(106)}MeV \\
&=& 53 MeV \\
p_{e} &=& \sqrt{E_{e}^{2}-m_{e}^{2}} = 53 MeV \\
(P_{\mu}-P_{\nu})\cdot{(P_{\mu}-P_{e})} &=& P_{e}\cdot{P_{e}} \\
m_{\mu}^{2}-2E_{\nu}m_{\mu} &=& m_{e}^{2} \\
E_{\nu} &=& \frac{m_{\mu}^{2}-m_{e}^{2}}{2m_{\mu}}MeV = 53 MeV \\
p_{\nu} &=& -p_{e} = -53 MeV
\end{eqnarray*} \\
d. In this case the angle between the electron and muon in the lab frame is $\theta=0$. To analyze ths look at the invarient mass in the lab frame. Also, use energy and momentum conservaton in the lab frame: $E_{\mu}=E_{e}+E_{\nu}=$6GeV and $p_{\mu}=p_{e}+p_{\nu}=$5.999GeV). \\
\begin{eqnarray*}
P_{\mu} &=& P_{e}+P_{\nu} \\
P_{\mu}-P_{\nu} &=& P_{e} \\
m_{\mu}^{2}-2P_{\mu}\dot{P_{\nu}} &=& m_{e}^{2}  \\
m_{e}^{2} &=& m_{\mu}^{2}-2E_{\mu}E_{\nu}+2p_{\mu}p_{\nu}cos(\theta) \\
m_{e}^{2} &=& m_{\mu}^{2}-2E_{\mu}E_{\nu}-2p_{\mu}p_{\nu} \\
p_{\mu} &=& m_{\mu}\gamma\beta \\
E_{\mu} &=& \sqrt{m_{\mu}^{2}\gamma^{2}\beta^{2}+m_{\mu}^{2}} \\
&=& m_{\mu}\sqrt{\frac{\beta^{2}+1-\beta^{2}}{1-\beta^{2}}} \\
&=& m_{\mu}\gamma \\
m_{e}^{2} &=& m_{\mu}^{2}-2m_{\mu}\gamma{E_{\nu}}-2m_{\mu}\gamma\beta{p_{\nu}}  \\
p_{\nu} &=& \sqrt{E_{\nu}^{2}-m_{\nu}^{2}} = E_{\nu} \\
-2m_{\mu}\gamma\beta{E_{\nu}} &=& 2m_{\mu}\gamma{E_{\nu}}-m_{\mu}^{2}+m_{e}^{2} \\
2E_{\nu}m_{\mu}\gamma(1+\beta) &=& m_{\mu}^{2}-m_{e}^{2} \\
E_{\nu} &=& \frac{m_{\mu}^{2}-m_{e}^{2}}{2m_{\mu}}\sqrt{\frac{1-\beta}{1+\beta}}=0.474MeV \\
E_{e} &=& E_{\mu}-E_{\nu} = 5.9995TeV \\
p_{\nu} &=& E_{\nu} =0.474MeV \\
p_{e} &=& \sqrt{E_{e}^{2}-m_{e}^{2}} = 5.9995TeV 
\end{eqnarray*} \\
e. To find the momentum and energy of the decay particles in the moving frame when the electron is emitted at 90 degrees from the muon in the rest frame start with the energy and momentum in the rest frame. We know that the magnitude of the energy and momentum of the electron and neutrino will be the same as it was when it was emitted parallel to the muon using the invarient mass before and after the collision. The momentum is now directed in the y direction instead of the x direction. Therefore, we can Lorentz boost the four-momenta to the moving frame. \\
\begin{eqnarray*}
P_{e}' &=& \begin{bmatrix}
	\gamma & \beta\gamma & 0 & 0 \\
	\beta\gamma & \gamma & 0 & 0 \\
	0 & 0 & 1 & 0 \\
	0 & 0 & 0 & 1 
	\end{bmatrix}\begin{bmatrix}
	E_{e}=53MeV \\
	0 \\
	p_{e}=53MeV \\
	0 
	\end{bmatrix} \\
&=& \begin{bmatrix}
	\gamma{53MeV} \\
	\beta\gamma{53MeV} \\
	53MeV \\
	0 
	\end{bmatrix} = \begin{bmatrix}
	2.9998GeV \\
	2.9993GeV \\
	53MeV \\
	0 
	\end{bmatrix} \\
P_{\nu}' &=& \begin{bmatrix}
	\gamma & \beta\gamma & 0 & 0 \\
	\beta\gamma & \gamma & 0 & 0 \\
	0 & 0 & 1 & 0 \\
	0 & 0 & 0 & 1 
	\end{bmatrix}\begin{bmatrix}
	E_{\nu}=53MeV \\
	0 \\
	p_{\nu}=-53MeV \\
	0 
	\end{bmatrix} \\
&=& \begin{bmatrix}
	\gamma{53MeV} \\
	\beta\gamma{53MeV} \\
	-53MeV \\
	0 
	\end{bmatrix} = \begin{bmatrix}
	2.9998GeV \\
	2.9993GeV \\
	-53MeV \\
	0 
	\end{bmatrix} \\
	\end{eqnarray*} 

Problem 6. Kleppner and Kolenkow, problem 13.2 \\
a. We need to find the value of $\beta^{2}$ when the relative kinetic energy differs from the classical one by 0.1. \\
\begin{eqnarray*}
\frac{K_{rel}-K_{cl}}{K_{cl}} &=& \frac{K_{rel}}{K_{cl}}-1 = 0.1 \\
\frac{K_{rel}}{K_{cl}} &=& 1.1 \\\
K_{rel} &=& \frac{m_{0}c^{2}}{\sqrt{1-\frac{v^{2}}{c^{2}}}}-m_{0}c^{2} \\
1.1 &=& \frac{2c^{2}}{v^{2}\sqrt{1-\frac{v^{2}}{c^{2}}}}-\frac{2c^{2}}{v^{2}} \\
&\approx& \frac{2c^{2}}{v^{2}}(1+\frac{v^{2}}{2c^{2}}+\frac{3v^{4}}{8c^{4}})-\frac{2c^{2}}{v^{2}} \\
&=& 1+\frac{3v^{2}}{4c^{2}} \\
0.1 &=& \frac{3v^{2}}{4c^{2}} \\
\beta^{2} &=& 0.4/3 \approx 0.133
\end{eqnarray*} \\
b. Calculate the relativistic kinetic energy using the $\beta^{2}$ for an electron and a proton. \\
\begin{eqnarray*}
m_{0}c^{2} &=& 0.51MeV \\
K_{rel} &=& \frac{m_{0}c^{2}}{\sqrt{1-\frac{v^{2}}{c^{2}}}}-m_{0}c^{2} = \frac{0.51MeV}{\sqrt{1-0.133}}-0.51MeV \\
&=& 0.54MeV-0.51MeV = 0.03MeV \\
m_{0}c^{2} &=& 960MeV \\
K_{rel} &=& 72MeV 
\end{eqnarray*} \\

Problem 7. Kleppner and Kolenkow, problem 13.4 \\
This problem can be solved in 2 ways: using the invarient mass and using Lorentz transformations for energy and momentum. Let's use the invarient mass method first by comparing the center of mass frame (primed frame) to the frame where one partice is at rest (unprimed frame). \\
\begin{eqnarray*} 
ds^{2} &=& (P_{1}+P_{2})\cdot{(P_{1}+P_{2})} =  (P_{1}'+P_{2}')\cdot{(P_{1}'+P_{2}')} \\
2m_{0}^{2}+2P_{1}\cdot{P_{2}} &=& 2m_{0}^{2}+2P_{1}'\cdot{P_{2}'} \\
E_{1}E_{2}-p_{1}p_{2}cos(\theta) &=& E_{1}'E_{2}'-p_{1}'p_{2}'cos(\theta') \\
p_{2} &=& 0 \\
E_{2} &=& m_{0} \\
\theta' &=& \pi \\
E_{1}' &=& E_{2}' = E' \\
p_{1}' &=& p_{2}' = p' = m_{0}v \\
Em_{0} &=& (E')^{2}+(p')^{2} = ((p')^{2}+m_{0}^{2})+(p')^{2} = 2(p')^{2}+m_{0}^{2} \\
p' &=& m_{0}\gamma\beta \\
E &=& 2m_{0}\gamma^{2}\beta^{2}+m_{0}  = \frac{2m_{0}\beta^{2}}{1-\beta^{2}}+m_{0} \\
&=& \frac{2m_{0}\beta^{2}+m_{0}-m_{0}\beta^{2}}{1-\beta^{2}} = m_{0}\frac{1+\beta^{2}}{1-\beta^{2}} 
\end{eqnarray*} \\
Check the hint $\beta^{2}=1/2$. \\
\begin{eqnarray*}
E &=& m_{0}\frac{1+1/2}{1-1/2} = 3m_{0}c^{2} 
\end{eqnarray*} \\
Now use Lorentz transformations for one of the particles from the center of mass frame to moving frame. The four-momentum in the center of mass frame is: \\
\begin{eqnarray*} 
P' &=& \begin{bmatrix}
	\gamma{m_{0}} \\
	\gamma\beta{m_{0}} \\
	0 \\
	0 
	\end{bmatrix}
\end{eqnarray*} \\
Then we are boosting the particle in the $-\beta$ direction so use the Lorentz boost matrix. \\
\begin{eqnarray*}
P &=& L_{x}(-\beta)P' =\begin{bmatrix}
	\gamma & \beta\gamma & 0 & 0 \\
	\beta\gamma & \gamma & 0 & 0 \\
	0 & 0 & 1 & 0 \\
	0 & 0 & 0 & 1 
	\end{bmatrix}\begin{bmatrix}
	\gamma{m_{0}} \\
	\gamma\beta{m_{0}} \\
	0 \\
	0 
	\end{bmatrix} \\
P &=& \begin{bmatrix}
	\gamma^{2}m_{0}+\gamma^{2}\beta^{2}m_{0} \\
	\beta\gamma^{2}m_{0}+\gamma^{2}\beta{m_{0}} \\
	0 \\
	0 
	\end{bmatrix} = \begin{bmatrix}
	m_{0}\frac{1+\beta^{2}}{1-\beta^{2}} \\
	\frac{2m_{0}\beta^{2}}{1-\beta^{2}} \\
	0 \\
	0 
	\end{bmatrix}
	\end{eqnarray*} \\
The first term in P is the transformed energy, which is equilvalent to what we found using the invarient mass. \\

Problem 8. Kleppner and Kolenkow, problem 13.8 \\
a. To find the energy of the scattered photon, use four-momentum conservation before and after the collision. $P_{0\gamma}+P_{0e}=P_{f\gamma}+P_{fe}$. Then we can using the invarient mass to determine the energy of the scattered photon. Since we don't care about the energy of the scattered electron, rewrite the four-momentum equation to be $P_{fe}=P_{0\gamma}+P_{0e}-P_{f\gamma}$. Now lets find the invarient mass. \\
\begin{eqnarray*}
ds^{2} &=& P_{fe}\cdot{P_{fe}} = (P_{0\gamma}+P_{0e}-P_{f\gamma})\cdot{(P_{0\gamma}+P_{0e}-P_{f\gamma})} \\
m_{e}^{2} &=& m_{\gamma}^{2}+m_{e}^{2}+m_{\gamma}^{2}+2P_{0e}\cdot{P_{0\gamma}}-2P_{0\gamma}\cdot{P_{f\gamma}} -2P_{0e}\cdot{P_{f\gamma}} \\
m_{\gamma} &=& 0 \\
E_{0\gamma} &=& E_{0} = p_{0\gamma} \\
p_{0e} &=& \gamma{m_{e}}\beta \\
E_{0e} &=& \sqrt{\gamma^{2}\beta^{2}m_{e}^{2}+m_{e}^{2}} = m_{e}\sqrt{\frac{\beta^{2}+1-\beta^{2}}{1-\beta^{2}}} = m_{e}\gamma \\
0 &=& 2(E_{0}m_{e}-E_{0}p_{0e}cos(180))-2(E_{0}E_{f\gamma}-E_{0}E_{f\gamma}cos(90))-2(E_{0e}E_{f\gamma}-2p_{0e}E_{f\gamma}cos(90)) \\
0 &=& 2E_{0}E_{0e}+2E_{0}p_{0e}-2E_{0}E_{f\gamma}-2E_{f\gamma}E_{0e} \\
E_{f\gamma} &=& \frac{E_{0}E_{0e}+E_{0}p_{0e}}{E_{0}+E_{0e}} \\
&=& \frac{E_{0}(m_{0}\gamma+m_{0}\gamma\beta)}{E_{0}+m_{0}\gamma} \\
&=& \frac{E_{0}(1+\beta)}{\frac{E_{0}}{m_{0}\gamma}+1} \\
E_{i} &=& m_{0}\gamma \\
E_{fe} &=& \frac{E_{0}(1+\beta)}{1+\frac{E_{0}}{E_{i}}} 
\end{eqnarray*} \\
b. To solve for the broadening in the wavelength we need to find $\lambda-\lambda_{0}$ so substitute in $E=\frac{hc}{\lambda}$. \\
\begin{eqnarray*}
\frac{hc}{\lambda} &=& \frac{\frac{hc}{\lambda_{0}}(1+\beta)}{1+\frac{hc}{\lambda_{0}E_{i}}} \\
\lambda &=& \lambda_{0}\frac{1+\frac{hc}{\lambda_{0}E_{i}}}{1+\beta} \\
&=& \frac{\lambda_{0}}{1+\beta}+\frac{hc}{E_{i}(1+\beta)} \\
\lambda -\lambda_{0} &=& \frac{hc}{E_{i}(1+\beta)}+ \frac{\lambda_{0}}{1+\beta}-\lambda_{0} \\
&=&  \frac{hc}{E_{i}(1+\beta)}-\frac{\lambda_{0}\beta}{1+\beta} \\
\frac{hc}{E_{i}} &=& \frac{h\sqrt{1-\beta^{2}}}{m_{0}c} = (2.426*10^{-12}m)\sqrt{1-36*10^{-6}}=2.426*10^{-12}m \\
\lambda -\lambda_{0} &=& (2.426-0.4266)*10^{-12}/(1+\beta) = 1.987*10^{-12}m 
\end{eqnarray*}
\end{document}