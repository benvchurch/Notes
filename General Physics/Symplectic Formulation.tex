\documentclass[12pt]{extarticle}
\usepackage{import}
\import{../General/}{General_Includes}
\renewcommand{\L}{\mathcal{L}}
\newcommand{\R}{\mathbb{R}}
\newcommand{\dR}{\mathrm{dR}}
\newcommand{\Ad}{\mathrm{Ad}}


\begin{document}

\author{Benjamin Church}
\title{\Huge Classical Mechanics from the Symplectic Viewpoint}

\maketitle
\tableofcontents
\newpage

\section{Introduction}

\subsection{Invitation}

Consider a Hamiltonian system on $\R^n$ giving a phase space $\R^{2n}$ with coordinates, $q^1, \dots, q^n, p_1, \dots, p_n$ and a Hamiltonian $H : \R^{2n} \to \R$. In these coordinates, Hamilton's equations of motion are,
\[ \dot{q}^i = \pderiv{H}{p_i} \quad \quad \dot{p}_i = - \pderiv{H}{q^i} \]
Part of the power of the Hamiltonian framework is the greater freedom to reparametrize the problem beyond a simple coordinate change of phase space in the Lagrangian framework. Such reparametrizations are given by so called \textit{canonical} transformations which are reparametrizations that preserve the ``form'' of Hamilton's equations. The desire to formalize this notion leads us to symplectic geomery.
\bigskip\\
The first step will be to put Hamilton's equation is a coordinate independent form in which canonical transformations will be ellucidated. Notice that the 1-form,
\[ \d{H} = \sum_{i = 1}^n \left( \pderiv{H}{q^i} \d{q_i} + \pderiv{H}{p_i} \d{p_i} \right) \]
and the vector field,
\[ X_H = \deriv{}{t} (q,p)(t) = \sum_{i = 1}^n \left( \dot{q}^i \pderiv{}{q^i} + \dot{p}_i \pderiv{}{p_i} \right) \]
are related by Hamilton's equations. To relate 1-forms and vector fields we need a 2-form,
\[ \omega = \sum_{i = 1}^n \d{q^i} \wedge \d{p_i} \]
which we call the symplectic form. Notice that,
\[ \omega(X_H, -) = \sum_{i = 1}^n \left( \dot{q}_i \d{p_i} - \dot{p}_i \d{q^i} \right) \]
and therefore Hamilton's equations may be rewritten as,
\[ \iota_{X_H} \omega = \d{H} \]


\subsection{More General Configuration Spaces}

There is no reason to restrict ourselves to Euclidean configuration space. In fact, a natural symplectic form
\[ \omega = \sum_{i = 1}^n \d{q^i} \wedge \d{p_i} \]
arises on the phase space $T^* Q$ of any configuration manifold $Q$. To see how this happens, we first construct the tautological $1$-form $\theta$ on $T^* Q$. Let $X = T^* Q$ and $\pi : X \to Q$ be the fiber bundle projection. Then $\d{\pi} : TX \to TQ$ is induced. A point $x \in X$ corresponds to some point $q \in Q$ and map $\varphi : T_q Q \to \R$. Then define,
\[ \theta_x = \varphi \circ \d{\pi_x} \]
Thus $\theta_x : T_x X \to \R$ is linear giving a section $\theta : X \to T^* X$.
\bigskip\\
If we choose a chart $(U, \psi)$ of $Q$ with local coordinate functions $q^1, \dots, q^n$ (where $q^i = x^i \circ \psi$ for $\psi : U \to \R^n$ and $x^i : \R^n \to \R$ are the standard coordinates) then there is an induced chart $(\tilde{U}, \tilde{\psi})$ of $X$ defined as $\tilde{U} = \pi^{-1}(U)$ with $\tilde{\psi} : \tilde{U} \to \R^n \times \R^n$ via,
\[ \tilde{\psi}(q, p_i \d{q^i}) = (\psi(q), p_1, \dots, p_n) \]
where $\d{q^i}$ are derivates of the coordinate functions $q^i : U \to \R$. Notice that,
\[ \d{q^i} = \d{(x^i \circ \psi)} = \d{x_i} \circ \d{\psi} = \psi^* \d{x^i} \]
Then let $p_i : \tilde{U} \to \R$ be the coordinate functions of the second projection $\tilde{U} \to \R^n \times \R^n \to \R^n$. Abusing notation, we write $q^i$ for the pull back of $q^i$ to $Q$, explicitly $q^i : \tilde{U} \xrightarrow{\pi} U \xrightarrow{q^i} \R$. Now we compute $\theta_x$ on the local vector fields $\pderiv{}{q^i}$ and $\pderiv{}{p_i}$. For the point $x = (q, \varphi)$ we have,
\[ \theta_x \left( \pderiv{}{q^i} \right) = \varphi \left( \pderiv{}{q^i} \right) = p_i \]
where $\varphi = p_i \d{q^i}$ since $\d{\pi}\left( \pderiv{}{q^i} \right) = \pderiv{}{q^i}$ using that the first $q^i$ is really $q^i \circ \pi$. Furthermore, clearly $\d{\pi} \left( \pderiv{}{p_i} \right) = 0$ since on the chart side $\tilde{U} \to U$ corresponds to $\R^n \times \R^n \to \R^n$ via the first projection. Thus,
\[ \theta_x \left( \pderiv{}{p_i} \right) = 0 \]
Since these vector fields form a local frame of $TX$ we find that,
\[ \theta = \sum_{i = 1}^n p_i \d{q^i} \]
Therefore, the symplectic 2-form $\omega = -\d{\theta}$ is given in local coordinates as,
\[ \omega = - \d{\theta} = \sum_{i = 1}^n \d{q^i} \wedge \d{p_i} \]
Therefore out ``natural'' symplectic form for doing Hamiltonian mechanics actually arises quite canonically on the cotangent space of any manifold or as the physicists would say: on the phase space induced by any configuration space.

\section{Symplectic Geometry}

\begin{defn}
Let $V$ be a finite $k$-vectorspace and $\omega \in \bigwedge^2 V^*$ a $2$-form. We say that $\omega$ is \textit{nondegenerate} if for all nonzero $v \in V$ the map $\omega(v, -) \in V^*$ is nonzero. Equivalently, $\omega$ is nondegenerate exactly when the map $V \to V^*$ defined by $v \mapsto \omega(v, -)$ is an isomorphism. 
\end{defn}

\begin{lemma}
If $\omega$ is a nondegenerate $2$-form on $V$ then $\dim{V} = 2n$ is even.
\end{lemma}

\begin{proof}
Choose a basis $e_1, \dots, e_k$ of $V$. Then we have a matrix $M_{ij} = \omega(e_i, e_j)$ which is antisymmetric. Then $\omega$ is nondegenerate implies that $\det{M} \neq 0$. However, $M^\top = - M$ so we must have,
\[ \det{M} = \det{(-M)} = (-1)^{\dim{V}} \det{M} \]
Thus $\dim{V} = 2n$ is even.
\end{proof}

\begin{defn}
Let $M$ be a smooth $2n$-manifold. A \textit{symplectic form} $\omega$ on $M$ is a closed nondegenerate $2$-form. We say that the pair $(M, \omega)$ is a \textit{symplectic manifold}. A \textit{symplectomorphism} $f : (M, \omega_M) \to (N, \omega_N)$ is a smooth map $f : M \to N$ such that $f^* \omega_N = \omega_M$. 
\end{defn}

\begin{rmk}
Consider a vector field $X$ on $M$. Such a vector field defines a flow $\phi_t : M \to M$. We consider when this flow preserves the symplectic structure. This occurs when $\phi_t$ is a symplectomorphism i.e. when $\phi_t^* \omega = \omega$. Now, recall that, the Lie derivative is defined via,
\[ \L_X \omega = \deriv{}{t} \bigg|_{t = 0} \bigg( \phi^*_t \omega \bigg) \]
Therefore $\phi_t : M \to M$ is symplectic iff $\L_X \omega = 0$.
\end{rmk}

\begin{defn}
We say a vector field $X$ on $M$ is \textit{symplectic} if $\L_X \omega = 0$. 
\end{defn}

\begin{defn}
We say a vector field $X$ on $M$ is \textit{Hamiltonian} if there exists a smooth function $H : M \to \R$ such that $\iota_X \omega = \d{H}$. 
\end{defn}

\begin{lemma}
Hamiltonain vector fields are symplectic.
\end{lemma}

\begin{proof}
Let $X$ be Hamiltonian such that $\iota_X \omega = \d{H}$. Then, we use Cartan's magic formula,
\[ \L_X \omega = \d (\iota_X \omega) + \iota_X \d \omega \]
Applying $\iota_X \omega = \d{H}$ and using $\d{\omega} = 0$ we find,
\[ \L_X \omega = \d(\d{H}) = 0 \]
\end{proof}

\section{Symptectic Geometry}

\begin{definition}
A \textit{symplectic form} on $M$ is a closed non-degenerate $2$-form $\omega$. We say that $(M, \omega)$ is a \textit{symplectic manifold}. A \textit{symplectomorphism} $f : (M, \omega_M) \to (N, \omega_N)$ is a smooth map $f : M \to N$ such that $f^* \omega_N = \omega_M$. 
\end{definition}

\begin{lemma}
Symplectic forms can only exist on even-dimensional manifolds. 
\end{lemma}

\begin{proof}
Locally, a symplectic form $\omega$ is a nondegenerate anti-symmetric bilinear form $S : T_p M \times T_p M \to \R$. So we have $S^\top = - S$ and $\det{S} \neq 0$. However, \[ \det{S} = \det{S^\top} = \det{(- S)} = (-1)^n \det{S} \]
since $\det{S} \neq 0$ we must have $(-1)^n = 1$ i.e. $n$ is even.
\end{proof}

\begin{definition}
We say that a vector field $X$ on $(M, \omega)$ is symplectic if $\L_X \omega = 0$.
\end{definition}

\begin{remark}
We see that the condition $\L_X \omega = 0$ that a vector field be symplectic is equivalent to the condition that its flows $\phi_t :  M \to M$ be symplectomorphisms since,
\[ \L_X \omega = \deriv{}{t} ((\phi_t)^* \omega) = 0 \]
Thus, symplectic vector fields are fields whose flows preserve the symplectic structure.
\end{remark}

\begin{lemma}
Let $(M, \omega)$ be symplectic. A vector field $X$ is symplectic iff $\iota_X \omega$ is closed.
\end{lemma}

\begin{proof}
From Cartan,
\[ \L_X \omega = \iota_X \d{\omega} + \d \iota_X \omega = \d \iota_X \omega \]
because $\d{\omega} = 0$. Therefore,
\[ \L_X \omega = 0 \iff \d{(\iota_X \omega)} = 0 \]
\end{proof}

\begin{definition}
We say that a vector field $X$ on $(M, \omega)$ is Hamiltonian if the form $\iota_X \omega \in \Omega^1(M)$ is exact i.e. there exists a smooth function $H : M \to \R$ such that,
\[ \iota_X \omega = \d{H} \]
\end{definition}

\begin{remark}
Note that since $\omega$ is non-degenerate, the map $\omega : TM \to \Omega^1(M)$ via $X \mapsto \iota_X \omega$ is an isomorphism and thus we can consider $\omega^{-1} : \Omega^1(M) \to TM$. Then the above condition is that,
\[ X = \omega^{-1}(\d{H}) \] 
\end{remark}

\begin{lemma}
Hamiltonian vector fields are symplectic.
\end{lemma}

\begin{proof}
Let $X$ be Hamiltonian then $\iota_X \omega$ is exact and thus closed so $X$ is symplectic. Explicitly,
\[ \L_X \omega = \iota_X \d \omega + \d \iota_X \omega \]
Since $\omega$ is a symplectic form $\d{\omega} = 0$ and since $X$ is Hamiltonainm $\iota_X \omega$ is exact and thus closed so $\d \iota_X \omega = 0$. Therefore,
\[ \L_X \omega = 0 \]
so $X$ is symplectic. 
\end{proof}

\newcommand{\sym}{\mathfrak{sym}}

\newcommand{\ham}{\mathfrak{ham}}

\begin{lemma}
Symplectic and Hamiltonian vector fields form Lie subalgebras. Furthermore, 
\[ [ \sym, \sym ] \subset \ham \]
where we explicitly see that if $X, Y$ are symplectic then $[X, Y]$ is Hamiltonian with Hamiltonian function $\iota_X \iota_Y \omega = \omega(Y, X)$ meaning that,
\[ \iota_{[X, Y]} \omega = \d{(\iota_X \iota_Y \omega)} = \d(\omega(Y, X)) \] 
\end{lemma}

\begin{proof}
We know that,
\[ \L_{[X,Y]} \omega = \L_X \L_Y \omega - \L_Y \L_X \omega \]
so if $X,Y$ are symplectic then so is $[X, Y]$. Furthermore, 
\[ \iota_{[X, Y]} \omega = \L_X \iota_Y \omega - \iota_Y \L_X \omega \]
However, $\L_X \omega = 0$ since $X$ is symplectic. Furthermore, by Cartan's formula,
\[ \iota_{[X, Y]} \omega = \L_X \iota_Y \omega = \iota_X (\d \iota_Y \omega) + \d( \iota_X \iota_Y \omega) \]
However, since $Y$ is symplectic, $\iota_Y \omega$ is closed and thus,
\[ \iota_{[X, Y]} \omega = \d (\iota_X \iota_Y \omega) = \d{(\omega(Y, X))} \]
which is exact so $[X, Y]$ is Hamiltonian. 
\end{proof}

\begin{remark}
We have $\L_X \d{\omega} = \d{(\L_X \omega)}$ because $\d$ is a natural transformation in the sense that $f^* \d = \d f^*$ for any smooth map and, in particular, for the flow of $X$. 
\end{remark}

\begin{prop}
Let $(M, \omega)$ be a symplectic manifold. Then,
\[ H^1_{\dR}(M) \cong \sym / \ham \]
\end{prop}

\begin{proof}
Obvious from the correspondences between $\sym$ and closed forms and $\ham$ and exact forms.
\end{proof}

\begin{definition}
Let $f, g : M \to \R$ be functions and let $X_f = \omega^{-1}(\d{f})$ and $X_g = \omega^{-1}(\d{g})$ be the associated Hamiltonian vector fields. Then we define the \textit{Poisson bracket} via,
\[ \{ f, g \} = \omega(X_f, X_g) \] 
\end{definition}

\begin{remark}
From the definitions of $X_f$ and $X_g$,
\begin{align*}
\{ f, g \} & = \omega(X_f, X_g) = \d{f}(X_g) = X_g(f) = \L_{X_g} f
\\
& = - \omega(X_g, X_f) = -\d{g}(X_f) = - X_f(g) = - \L_{X_f} g
\end{align*}
So $\{ f, g \}$ represents the flow of $f$ along the vector field generated by $g$. 
\end{remark}

\begin{lemma}
$[X_f, X_g] = - X_{\{ f, g \}}$ 
\end{lemma}

\begin{proof}
We have shown that if $X$ and $Y$ are symplectic then,
\[ \iota_{[X,Y]} \omega = \d{(\omega(Y, X))} \]
Therefore,
\[ X_{\omega(Y, X)} = \omega^{-1}(\d{(\omega(Y, X))}) = [X, Y] \]
Now applying this to $X_f$ and $X_g$ we find,
\[ [X_f, X_g] = \omega^{-1}(\d{(\omega(X_g, X_f)}) = - \omega^{-1}(\d \{f, g \}) = - X_{\{ f, g \}} \]
\end{proof}

\begin{proposition}
The Poisson bracket on smooth functions forms a Lie algebra.
\end{proposition}

\newcommand{\ad}{\mathrm{ad}}

\begin{proof}
Clearly the Poisson bracket is bilinear. Furthermore, it is antisymmetric because,
\[ \{ f ,g \} = \omega(X_f, X_g) = - \omega(X_g, X_f) = - \{ g, f \} \]
The Jacobi identity is equivalent to the fact that $\ad : \mathfrak{g} \to \mathfrak{gl}(\mathfrak{g})$ via $\xi \mapsto [\xi, -]$ is a Lie algebra homomorphism. 
\bigskip\\
In the current case, $\ad_f(g) = \{ f, g \} = -X_f(g)$ so $\ad_f = -X_f$ as a derivation. Then we know that,
\[ [ \ad_f, \ad_g] = [-X_f, -X_g] = - X_{\{f, g\}} = \ad_{\{f, g \}} \]
since the commutator of vector fields is their comutator as differential operators. 
\end{proof}

\begin{proposition}
The map $f \mapsto -X_f = -\omega^{-1}(\d{f})$ is a homomorphism of Lie algebras $\varphi : C^{\infty}(M) \to \ham$ from smooth functions to Hamiltonian vector fields.
\end{proposition}

\begin{proof}
Immediate from $-X_{\{f, g\}} = [X_f, X_g] = [- X_f, - X_g]$. 
\end{proof}

\newcommand{\X}{\mathfrak{X}}

\begin{rmk}
Unfortunately the physicists convention for Hamilton's equations plus the definition of the Poisson bracket (mathematicians might have defined the Poisson bracket with a minus sign to agree with the convention of Lie brackets $[X, Y] = \L_X Y$ where as $\{f, g \} = \d{f}(X_g) = \L_{X_g} f$ explaining the sign difference between the two brackets) do not permit the map $f \mapsto X_f$ to be a Lie algebra homomorphism. One might attempt to remedy this by replacing $X_f$ by $-X_f$ however this messes up the form of Hamilton's equation unless simultaneously $\omega$ is replaced by $-\omega$ which then messes up the sign of $\{ -, - \}$. Thus the only true remedy is reversing either the Poisson bracket or the Lie bracket. However, $\X(M)$ is sometimes given the opposite Lie algebra structue, remedying our conundrum, because this is the induced Lie bracket on $\X(M) = \mathrm{Lie}(\mathrm{Diff}(M))$.
\end{rmk}

\section{Hamiltonian Actions}

\newcommand{\g}{\mathfrak{g}}
\newcommand{\inner}[2]{\left< #1 , #2 \right>}

(THERE IS A PROBLEM HERE WITH LEFT VS RIGHT INVT VECTOR FIELDS $\X(M)$ NEEDS OPPOSITE LIE BRACKET) 

\begin{lemma}
Let $\rho : G \times M \to M$ be a smooth action of a Lie group on a smooth manifold. Then there is a Lie algebra map $\rho : \g \to \X(M)$ given by $\rho(\xi)_m = \d{\rho}_{(e, m)}(\xi, 0)$.
\end{lemma}

\begin{proof}
We need to show that $\rho([\xi,\eta]) = [\rho(\xi), \rho(\eta)]$. Let $X_\xi$ denote the left-invariant vector field on $G$ with $X_\xi(e) = \xi$. Then I claim that $\rho_*(X_\xi) = \rho(\xi)$. To see this, note that,
\[ \d{\rho}_{(g,m)}(X_\xi(g)) = \d{\rho}_{(g,m)}(\d{L_g}(\xi)) = \d{(\rho(g,-) \circ \rho)}_{(e,m)}(\xi) = \d{\rho(g,-)}(\rho(\xi)_m) \]
(DO THISS!)
\end{proof}

\begin{defn}
A Lie group action $G \acts M$ on a symplectic manifold $(M, \omega)$ is \textit{symplectic} if $G$ acts through symplectomorphisms i.e. for each $g \in G$ the map $g : M \to M$ satisfies $g^* \omega = \omega$.
\end{defn}

\begin{rmk}
In this case, for each $\xi \in \g$ the vector field $\rho(\xi)$ is symplectic.
\end{rmk}

\begin{rmk}
We want a Hamiltonian action to be one that acts through Hamiltonian vector fields meaning $\rho(\xi) \in \ham$ for each $\xi \in \ham$. This means we know that $\iota_{\rho(\xi)} \omega$ is exact so $\iota_{\rho(\xi)} \omega = \d{H_\xi}$ for some choice of function $H_\xi : M \to \R$. However, we want to package the functions $H_\xi$ together so they vary in a coherent way. This is formalized as follows.
\end{rmk}

\begin{defn}
Given a symplectic action $G \acts M$, a \text{moment map} is a smooth map $\mu : M \to \g^*$ such that,
\begin{enumerate}
\item $\d{\inner{\mu}{\xi}} = \iota_{\rho(\xi)} \omega$

\item $\mu : M \to \g^*$ is $G$-equivariant where $G \acts \g^*$ via the coadjoint action.
\end{enumerate}
\end{defn}

\begin{defn}
A \textit{Hamiltonian action} $G \acts M$ is a symplectic action along with a choice of moment map $\mu : M \to \g^*$.
\end{defn}

\begin{example}
The translation action $\R^2 \acts \R^2$ clearly acts through Hamiltonian vector fields however is not Hamiltonian. To so see this, suppose there is a moment map $\mu : \R^2 \to \R^2$ which is equivariant but $\R^2$ acts on the first copy by translation and on the section trivially so $\mu$ must be constant contradicting the first property. 
\end{example}

\begin{lemma}
If $G$ is connected, a moment map $\mu : M \to \g^*$ is equivalent to a comoment map, a morphism of Lie algebras $\tilde{\mu} : \g \to C^\infty(M)$ such that $\d{\tilde{\mu}(\xi)} = \iota_{\rho(\xi)} \omega$.
\end{lemma}

\begin{proof}
Consider the natural correspondence between smooth functions $\mu : M \to \g^*$ and linear maps $\tilde{\mu} : \g \to C^\infty(M)$. Indeed, we define $\tilde{\mu}(\xi) = \inner{\mu(-)}{\xi}$ and $\mu(x) = \tilde{\mu}(-)(x)$. It is clear that $\mu : M \to \g^*$ is $G$-equivariant iff $\tilde{\mu} : \g \to C^\infty(M)$ is $G$-equivariant where $G \acts C^\infty(M)$ via $(g \cdot f)(x) = f(g^{-1} \cdot x)$. Indeed,
\[ 
\tilde{\mu}(\Ad_g \cdot \xi) = \inner{\mu(-)}{\mathrm{Ad}_g \cdot \xi} = \inner{\Ad_{g^{-1}} \cdot \mu(-)}{\xi} = \inner{\mu(g^{-1} \cdot - )}{\xi} = g \cdot \inner{\mu(-)}{\xi} = g \cdot \tilde{\mu}(\xi) \]
and likewise,
\[ \mu(g \cdot x) = \tilde{\mu}(-)(g \cdot x) = (g^{-1} \cdot \tilde{\mu}(-))(x) = \tilde{\mu}(\mathrm{Ad}_{g^{-1}} -)(x) = \Ad_g^* \cdot \tilde{\mu}(-)(x) = \Ad_g^* \cdot \mu(x) \]
Therefore, it suffices to show that $G$-equivariance of $\tilde{\mu}$ corresponds to $\tilde{\mu}$ being a map of Lie algebras. If $\tilde{\mu}$ is $G$-equivariant then differentiating $\tilde{\mu}(\Ad_g \cdot \eta) = g \cdot \tilde{\mu}(\eta)$ we see that,
\[ \tilde{\mu}([\xi, \eta]) = -\rho(\xi) (\tilde{\mu}(\eta)) = -\omega^{-1}(\d{\tilde{\mu}(\xi)})(\tilde{\mu}(\eta)) = - X_{\tilde{\mu}(\xi)}(\tilde{\mu}(\eta)) = \{ \tilde{\mu}(\xi), \tilde{\mu}(\eta) \}  \]
Alternatively, if $\tilde{\mu}$ is a map of Lie algebras we need to integrate to find the $G$-action. Explicitly, we have shown that the derivative of,
\[ \tilde{\mu}(\Ad_g \cdot \xi) - g \cdot \tilde{\mu}(\xi) \]
is zero at $g = e$ and thus at every point by noticing                                          
\end{proof}

\begin{lemma}
Let $G \acts M$ be a Hamiltonian action with moment map $\mu : M \to \g^*$. Then the derivative $\d{\mu} : TM \to \g^*$ is given by $X \mapsto \omega(\rho(-), X)$.
\end{lemma}

\begin{proof}
We know that $\d{\inner{\mu}{\xi}} = \iota_{\rho(\xi)} \omega$. Thus for $X \in \Gamma(M, TM)$ viewing $\xi \in \g$ as a function on $\g^*$,
\[ \d{\mu}(X)(\xi) = X(\xi \circ \mu) = X(\inner{\mu}{\xi}) = \d{\inner{\mu}{\xi}}(X) = \omega(\rho(\xi), X) \]
\end{proof}


\begin{defn}
Let $G \acts M$ be a symplectic action. Then consider the pullback of Lie algebras,
\begin{center}
\begin{tikzcd}
\tilde{\g} \arrow[dr, phantom, "\usebox\pullback" , very near start, color=black] \arrow[d] \arrow[r] & C^{\infty}(M) \arrow[d, "\varphi"]
\\
\g \arrow[r, "\rho"'] & \sym
\end{tikzcd}
\end{center}
Explicitly,
\[ \tilde{\g} = \{(\xi, f) \in \g \oplus C^{\infty}(M) \mid \rho(\xi) = - X_f \} \]
\end{defn}

\begin{rmk}
I claim that the map $\varphi : C^\infty(M) \to \sym$ is $G$-equivariant. Consider $\varphi(g \cdot f) = - X_{g \cdot f}$. First,
\[ \d{(g \cdot f)} = \d{f} \circ \d{g^{-1}} \]
However, because the action is symplectic,
\[ \omega(\d{g}(X_f), Y) = \omega(X_f, \d{g^{-1}}(Y)) \]
and therefore $\d{(g \cdot f)}(Y) = \d{f} \circ \d{g^{-1}}(Y) = \omega(\d{g}(X_f), Y)$ which shows that,
\[ X_{g \cdot f} = \d{g}(X_f) \]
Therefore, the above diagram is in the category of $G$-equivariant Lie algebras. Explicilty,
\[ [\Ad(g) \cdot \xi_1, \Ad(g) \cdot \xi_2] = \Ad(g) \cdot [\xi_1, \xi_2] \]
and likewise,
\[ \{ g \cdot f_1, g \cdot f_2 \} = \omega(X_{g \cdot f_1}, X_{g \cdot f_2}) = \omega(\d{g}(X_{f_1}), \d{g}(X_{f_2})) = \omega(X_{f_1}, X_{f_2}) \circ g^{-1} \]
meaning that $\{ g \cdot f_1, g \cdot f_2 \} = g \cdot \{ f_1, f_2 \}$.
\end{rmk}

\begin{prop}
Let $G \acts M$ be a symplectic action such that $\rho(\xi) \in \ham$. Then there is a central extension of Lie algebras,
\begin{center}
\begin{tikzcd}
0 \arrow[r] & \R^{\pi_0(M)} \arrow[r] & \tilde{\g} \arrow[r] & \g \arrow[r] & 0
\end{tikzcd}
\end{center}
\end{prop}

\begin{proof}
It is clear that $\tilde{\g} \to \g$ is surjective because $C^\infty(M) \onto \ham$ is surjective and $\rho : \g \to \sym$ lands inside $\ham$. Then consider,
\[ \ker{(\tilde{\g} \to \g)} = \{ f \in C^\infty(M) \mid X_f = 0 \} \]
However, $\omega(X_f, -) = \d{f}$ and thus $\d{f} = 0$ so $f$ is locally constant. Furthermore, for any element $(0, f) \in \ker{(\tilde{\g} \to \g)}$ we know $X_f = 0$ so $\{f, g \} = \omega(X_f, X_g) = 0$ so the extension is central.
\end{proof}

\begin{prop}
Let $G \acts M$ be a symplectic action such that $\rho(\xi) \in \ham$. Then moment maps $\mu : M \to \g^*$ correspond to splitngs of,
\begin{center}
\begin{tikzcd}
0 \arrow[r] & \R^{\pi_0(M)} \arrow[r] & \tilde{\g} \arrow[r] & \g \arrow[r] & 0
\end{tikzcd}
\end{center}
as $G$-representations.
\end{prop}

\begin{proof}
There is a canonical map $\tilde{\mu} : M \to \tilde{\g}^*$ defined by $\inner{\tilde{\mu}(x)}{(\xi, f)} = -f(x)$ which is $G$-equivariant because,
\begin{align*}
\inner{\tilde{\mu}(g \cdot x)}{(\xi, f)} & = -f(g \cdot x) = \inner{\tilde{\mu}(x)}{(\Ad(g^{-1}) \cdot \xi, g^{-1} \cdot f)} 
\end{align*}
and thus $\tilde{\mu}(g \cdot x) = g \cdot \tilde{\mu}(x)$. Therefore,
\[ \d{\inner{\tilde{\mu}}{(\xi, f)}} = -\d{f} = -\iota_{X_f} \omega \]
Then suppose that $s : \g \to \tilde{\g}$ is a section. Then consider $\mu = s^* \circ \tilde{\mu}$. Then,
\[ \d{\inner{\mu}{\xi}} = \d{\inner{s^* \circ \tilde{\mu}}{\xi}} = \d{\inner{\tilde{\mu}}{s(\xi)}} = \iota_{\rho(\xi)} \omega \]
because $s(\xi) = (\xi, f)$ for some $f$ such that $X_f = - \rho(\xi)$. Therefore, since $s^* \circ \tilde{\mu}$ is $G$-equivariant, $s^* \circ \tilde{\mu}$ is a moment map. Conversely, given a moment map $\mu : M \to \g^*$ then $q : \xi \mapsto -\inner{\mu}{\xi}$ gives a $G$-equivariant map $\g \to C^{\infty}(M)$ such that the diagram of $G$-representations,
\begin{center}
\begin{tikzcd}
\g \arrow[d, equals] \arrow[r] & C^{\infty}(M) \arrow[d, "\varphi"]
\\
\g \arrow[r, "\rho"] & \sym
\end{tikzcd}
\end{center}
commutes, and therefore we get a $G$-section $s : \g \to \tilde{\g}$ such that $s^* \circ \tilde{\mu} = \mu$ because,
\[ \inner{s^* \circ \tilde{\mu}}{\xi} = \inner{\tilde{\mu}}{(\xi, q(\xi)} = q(\xi) = \inner{\mu}{\xi} \]
Finally, given a $G$-section $s : \g \to \tilde{\g}$ or equivalently a $G$-map $q : \g \to C^{\infty}(M)$ then for the moment map $\mu = s^* \circ \tilde{\mu}$ consider $q'(\xi) = - \inner{s^* \circ \tilde{\mu}}{\xi} = - \inner{\tilde{\mu}}{(\xi, q(\xi))} = q(\xi)$ and thus our proceedure produces the section $s$ so this is a bijective correspondence.
\end{proof}

\begin{rmk}
Therefore, if $G$ is compact, then the category of $G$-representations is semi-simple and thus all exact sequences split. Thus, every symplectic action $G \acts M$ such that $\rho(\xi) \in \ham$ for all $g \in G$ then $G \acts M$ is Hamiltonian.
\end{rmk}

\begin{cor}
If one exists, the space of moment maps is isomorphic to $\Hom{G}{\g}{\R^{\pi_0(M)}}$ which are $\pi_0(M)$ choices of $G$-invariant elements of $\g^*$ representing an additive constant shift for $\mu : M \to \g^*$ on each connected component of $M$.
\end{cor}

\begin{proof}
This follows directly from the correspondence between moment maps and splittings of,
\begin{center}
\begin{tikzcd}
0 \arrow[r] & \R^{\pi_0(M)} \arrow[r] & \tilde{\g} \arrow[r] & \g \arrow[r] & 0
\end{tikzcd}
\end{center}
as $G$-representations which form a $\Hom{G}{\g}{\R^{\pi_0(M)}}$-torsor.
\end{proof}

\begin{prop}
If $G$ is reductive then any symplectic action $G \acts M$ with $\rho(\xi) \in \ham$ is hamiltonian. Additionally, if $Z(G)$ is trivial then the moment map is unique.
\end{prop}

\begin{proof}
The category of representations of a reductive group is semi-simple. Therefore, exact sequences of $G$-representations always split. 
\end{proof}

\begin{cor}
Let $T^n = (S^1)^n$ be the torus group and $(M, \omega)$ a simply-connected symplectic manifold. Then any symplectic action $T^n \acts M$ is hamiltonian and the space of moment maps is affine space over $\g \cong \R^n$.
\end{cor}

\begin{proof}

\end{proof}

\begin{example}
In the above corollary, $M$ being simply-connected is necessary. For example consider $M = S^1 \times S^1$ with the symplectic structure $\omega = \d{x} \wedge \d{y}$ where $x$ and $y$ are the coordinates on the two factors. Let $S^1 \acts M$ via left translation on the first factor. Since $\omega$ is constant (this makes sense since the tangent bundle is trivial), this is clearly a symplectic action. However, translation is not a Hamiltonian vector field because $\omega(\pderiv{}{x}, -) = \d{y}$ which is not closed since it has a nonvanishing integral along the curve $\{ * \} \times S^1 \subset S^1 \times S^1$. Therefore, this action cannot be Hamiltonian.
\end{example}

\begin{prop}
Let $G$ be a Lie group acting smoothly on a manifold $G \acts Q$. Then there is an induced action $G \acts T^* Q$ which is automati  cally Hamiltonian for the standard symplectic structure on $T^* Q$.
\end{prop}

\begin{proof}
The action is defined as $g \cdot (q, p) = (g \cdot q, (\d{g^{-1}})^* p)$. Notice that $\pi : T^* Q \to Q$ is by definition $G$-equivariant. The tautological $1$-form $\theta$ has the defining property that for any $1$-form $\beta : Q \to T^* Q$ we have $\beta^* \theta = \beta$. Then consider the form $\tilde{g}^* \theta$ for $\tilde{g} : T^* Q \to T^* Q$. We have,
\[ \beta^* \tilde{g}^* \theta  = (\tilde{g} \circ \beta)^* \theta = (\beta \circ g)^* \theta = g^* \beta^* \theta = g^* \beta \] 
Then for $\xi \in \g$ the vector field $\rho(\xi)$ is 

We define $\mu = \iota_{\rho(-)} \theta$.

(FINISH THIS!!)
\end{proof}

(FROM HERE ON NOT CORRECT!!!)

\begin{prop}
Any $G$-equivariant section $\g \to \tilde{\g}$ is automatically a Lie algebra map. If $G$ is connected, Lie algebra sections and $G$-equivariant sections coincide.
\end{prop}

\begin{proof}
DO THIS!!!
\end{proof}

\begin{cor}
If $G$ is semi-simple then there exists a unique moment map for any symplectic action $G \acts M$ with $\rho(\xi) \in \ham$.
\end{cor}

\begin{proof}
Uniqueness follows from the fact that $\g$ is semi-simple and thus has a trivial center. However, a $G$-equivariant map $\g \to \R$ must be a Lie algebra map because differentiating the action of $G$ gives 
Since $G$ is connected, it suffices to show that 
\end{proof}

\newcommand{\h}{\mathfrak{h}}

\begin{defn}
Let $G$ be a Lie group and $\g = \mathrm{Lie}(G)$. We say a $G$-action $\rho : G \to \Aut{\h}$ and a linear map $\varphi : \g \to \h$ are compatible if $\rho_*(\xi) = [\varphi(\xi), -]$ for all $\xi \in \g$.
\end{defn}

\begin{lemma}
Given a compatible action an linear map $\varphi : \g \to \h$ the map $\varphi$ is automatically a Lie algebra map and, if $G$ is connected, a $G$-equivariant map.
\end{lemma}

\begin{proof}

\end{proof}

\section{Connections on Principal Bundles}

\begin{defn}
Let $\pi : P \to X$ be a principal $G$-bundle. Then conisder the exact sequence of vector bundles on $P$,
\begin{center}
\begin{tikzcd}
0 \arrow[r] & \ker{\d{\pi}} \arrow[r] & TP \arrow[r, "\d{\pi}"] & \pi^* TX \arrow[r] & 0
\end{tikzcd}
\end{center}
A \textit{connection} on $P$ is a $G$-invariant splitting. Explicitly, a bundle map $\delta : \pi^* TX \to TP$ such that,
\begin{enumerate}
\item $\d{\pi} \circ \delta = \id_{\pi^* TX}$
\item for each $g \in G$ the diagram,
\begin{center}
\begin{tikzcd}
\pi^* TX \arrow[r, "\delta"] \arrow[d, equals] & TP \arrow[d, "\d{\ell_g}"]
\\
\ell_g^* \pi^* TX \arrow[r, "\ell_g^* \delta"] & \ell_g^* TP
\end{tikzcd}
\end{center} 
commutes where $\ell_g : P \to P$ is the left action by $g \in G$. Note that $\pi \circ \ell_g = \pi$ so there is a natural isomorphism $\pi^* = \ell_g^* \pi^*$.
\end{enumerate}
\end{defn}

\begin{rmk}
The equivariant condition is equivalent to $\delta$ being a morphism of descent data for the covering $\pi : P \to X$ or equivalently a morphism of $G$-equivariant bundles.
\end{rmk}

\begin{rmk}
Such a splitting is equivalent to the choice of a $G$-equivariant complement to the vertical space $V = \ker{\d{\pi}}$. Explicitly this is a subbundle $H \subset TP$ such that $TP = H \oplus V$ and $\d{\ell_g} : TP \to \ell_g^* TP$ takes $H$ to $\ell_g^* H$. This is a $G$-invariant Ehresmann connection on $P$.
\end{rmk}

\begin{lemma}
A connection on $\pi : P \to X$ is equivalent to the choice of a $\g$-valued $1$-form $\theta \in \Gamma(P, T^*P \otimes \g)$ such that,
\begin{enumerate}
\item (FINISH!!)
\end{enumerate}
\end{lemma}

\begin{proof}
A connection is a right splitting of the sequence,
\begin{center}
\begin{tikzcd}
0 \arrow[r] & V \arrow[r] & TP \arrow[r, "\d{\pi}"] & \pi^* TX \arrow[r] & 0
\end{tikzcd}
\end{center}
which is equivalent to a choice of right splitting $\theta : TP \to V$ such that $\theta|_V = \id_{V}$. However, since $P$ is a principal $G$-bundle, $V$ is the trivial bundle $P \times \g$ because $\xi \mapsto \rho(\xi)$ for the action $G \acts P$ is an isomorphism of vector bundles $P \times \g \to V$. Therefore, $\theta : TP \to V \cong P \times \g$ is equivalent to a form $\theta \in \Gamma(P, T^*P \otimes \g)$.
(PROVE PROPERTIES!!!)
\end{proof}

(DO THIS Adjoint bundle AP!!)

\newcommand{\A}{\mathcal{A}}

\begin{defn}
The curvature of a connection $\delta : \pi^* TX \to TP$ is a $\A_P$-valued $2$-form $F \in \Gamma(P, \wedge^2 T^* X \otimes \A_P)$ on $X$ defined by,
\[ F(X, Y) = [\delta(X), \delta(Y)] - \delta([X, Y]) \]
\end{defn}

%% (EXAMPLE OF S^1 BUNDLES OVER S^2 <-> Complex line bundles on S^2) there are Z of these.

%% (WHY IS SYMPLECTIC TORIC -> TORIC VARIETY?) 

%% (what are other complex structures on K3 surface).

\section{Quaternionic Manifolds}

\renewcommand{\H}{\mathbb{H}}
\renewcommand{\End}[2]{\mathrm{End}_{#1}\left(#2\right)}
\renewcommand{\Aut}[2]{\mathrm{Aut}_{#1}\left(#2\right)}
\newcommand{\GL}[1]{\mathrm{GL}\left(#1 \right)}

\subsection{First Attempts}

\subsection{Definition via $G$-Structues}

\begin{rmk}
We have the following setup. Let $V = \H^n$ be a $\R$-vector space and \textit{left} $\H$-module. The $\H$-module structue is equivalent to a map $\H \to \End{\R}{V}$ whose (faithful) image is a subalgebra $H \subset \End{\R}{V}$ isomorphic to $\H$. The group $\H^\times \times \GL{n, \H}$ acts on $V$ via $(q, A) \cdot v = q \cdot v \cdot A^{-1}$ (which is well-defined because right and left actions commute). Notice that $\GL{n, \H}$ acts via $\H$-linear maps while $\H^\times$ does not because $\H^\times$ is not abelian and acts on the left. Therefore, we get a map,
\[ \H^\times \times \GL{n, \H} \to \GL{4n, \R} = \Aut{\R}{V} \]  We denote its image by $G_{\H}$. Clearly, $G_{\H}$ is the product of $H^\times$ and $\GL{n, \H}$ inside $\GL{4n, \R}$,
\[ G_{\H} = \H^\times \cdot \GL{n, \H} \subset \GL{4n, \R} \]
Furthermore, because $\H^\times \cap \GL{n, \H} = \R^\times$ inside $\GL{4n, \R}$, there is an isomorphism,
\[ G_{\H} \cong (\H^\times \times \GL{n, \H})/\R^\times \]
Notice that, as it must given the embedding into $\GL{4n, \R}$, that $G_{\H}$ acts on $V$ because,
\[ (\lambda q, \lambda A) \cdot v = (\lambda q) \cdot v \cdot (\lambda^{-1} A^{-1}) = q \cdot v \cdot A^{-1} =  (q, A) \cdot v  \]
for all $\lambda \in \R^\times$.
\end{rmk}

\begin{lemma}
$\Aut{}{\H} = \mathrm{Inn}(\H) \cong \mathrm{SO}(3)$.
\end{lemma}

\begin{proof}
For any unit imaginary quaternions $v, u, w \in S^2 \subset \Im{\H}$ we know that $vu = - v \cdot u + v \times u$. Since any $\varphi \in \Aut{}{\H}$ must preserve scalars we see that $\varphi(v) \cdot \varphi(u) = v \cdot u$. Furthermore, it preserves the scalar part of $v(uw) = - v \cdot (u \times w)$ meaning that $\Aut{}{\H}$ preserves the metric and orientation form on $\R^3$ and fixes zero giving a map $\Aut{}{\H} \to \mathrm{SO}(3)$. Furthermore, because automorphisms fix the scalar part and respect scaling, such a transformation of the imaginary sphere determines the automorphism so $\Aut{}{\H} \iso \mathrm{SO}(3)$. Furthermore, we know that all rotations of the imaginary sphere are realized through inner automorphisms.
\end{proof}

\begin{lemma}
Since $V$ is a $G_{\H}$-representation, we get a $G_{\H}$-action on $\End{\R}{V}$. Then $H$ is invariant under $G_{\H}$ and $G_{\H}$ is exactly the stabilizer of $H$ under the inclusion $G_{\H} \subset \GL{4n, \R}$,
\[ G_{\H} = \mathrm{Stab}(H) = \{ \varphi \in \Aut{\R}{V} \mid \varphi \cdot H = H \} \]
Furthermore, the subgroup $\GL{n, \H} \subset \GL{4n, \R}$ is the \textit{pointwise stabilizer},
\[ \GL{n, \H} = \mathrm{Stab}(\{ H \}) = \{ \varphi \in \Aut{\R}{V} \mid \forall h \in H : \varphi \cdot h = h \} \]
\end{lemma}

\begin{proof}
By definition, $\varphi \cdot h = \varphi \circ h \circ \varphi^{-1}$ meaning that,
\[ \varphi \in \mathrm{Stab}(\{ H \}) \iff \forall h  \in H : \varphi \cdot h = h \iff \forall h \in H : \varphi \circ h = h \circ \varphi \]
and thus $\mathrm{Stab}(\{ H \})$ is the group of $H$-linear automorphisms of $V$ which is exactly $\GL{n, \H}$ acting on the right.
\bigskip\\
Now we consider the case that $\varphi \cdot h \in H$. Since $\Aut{\R}{V}$ acts on $\End{\R}{V}$ by algebra automorphism we know that $h \mapsto \varphi \cdot h = \varphi \circ h \circ \varphi^{-1}$ is an algebra automorphism. Since all automorphisms of $\H$ are inner, there exists some $q \in H^\times$ such that,
\[ \varphi \circ h \circ \varphi^{-1} = q^{-1} \circ h \circ q \]
Therefore, $\varphi' = q \circ \varphi$ is $\H$-linear so $\varphi' \in \GL{n, \H}$ and thus $\varphi \in q \circ \GL{n, \H} \subset G_{\H}$. Therefore, we conclude that $\mathrm{Stab}(H) = G_{\H}$.
\end{proof}

\begin{prop}
Let $V$ be a $4n$ dimensional $\R$-vectorspace. Then,
\begin{enumerate}
\item the data of a $G_{\H}$-torsor of isomorphisms $V \to \H^n$ equivalent to a subalgebra $H \subset \End{\R}{V}$ isomorphic to $\H$
\item the data of a $\GL{n, \H}$-torsor of isomorphisms $V \to \H^n$ is equivalent to the data of a subalgebra $H \subset \End{\R}{V}$ and an algebra isomorphism $\varphi : \H \to H$.
\end{enumerate}
\end{prop}

\begin{proof}
Given a $G_{\H}$ (or $\GL{n, \H}$) torsor of isomorphism $V \to \H^n$ choose one such isomorphism $\psi : V \iso \H^n$. Then $H = \psi^{-1} \circ \H \circ \psi \subset \End{\R}{V}$ is a subalgebra isomorphic to $\H$ via
\[ \varphi : q \mapsto \psi^{-1} \circ (q \cdot -) \circ \psi \]
Furthermore, any other isomorphism $\psi' : V \iso \H^n$ is of the form $\psi' = g \circ \psi$. Then, 
\[ \psi'^{-1} \circ \H \circ \psi = \psi^{-1} \circ (g^{-1} \circ \H \circ g) \circ \psi = \psi^{-1} \circ \H \circ \psi \]
because $G_{\H}$ satabilizes $\H \subset \Aut{\R}{\H^n}$ so $H \subset \End{\R}{V}$ is well-defined. Furthermore, if we have a $\GL{n, \H}$-torsor, then $\varphi' : q \mapsto \psi^{-1} \circ (g^{-1} \circ (q \cdot -) \circ g) \circ \psi = \psi^{-1} \circ (q \cdot -) \circ \psi$ because $g$ stabilizes $\H \subset \Aut{\R}{\H^n}$ pointwise.
\bigskip\\
Conversely, given a subalgebra $H \subset \End{\R}{V}$ and an algebra isomorphism $\varphi : \H \iso H$ we choose an isomorphism $\psi : V \to \H^n$. Then define, $S \subset \mathrm{Iso}(V, \H^n)$ as the set of isomorphisms $\psi'$ such that $H = \psi'^{-1} \circ \H \circ \psi'$. Then $\psi' \circ \psi^{-1}$ preserves $\H \subset \Aut{\R}{\H^n}$ so $S$ is a $G_{\H}$-torsor. Furthermore, let $S' \subset \mathrm{Iso}(V, \H^n)$ be the subset such that $\psi'^{-1} \circ (q \cdot -) \circ \psi' = \varphi(q)$ for all $q \in \H$. Then clearly $\psi' \circ \psi^{-1}$ preserves $\H \subset \Aut{\R}{\H^n}$ pointwise and thus $S'$ forms a $\GL{n, \H}$-torsor. These constructions are inverse to eachother.
\end{proof}

\begin{theorem}
Let $M$ be a smooth manifold. Then,
\begin{enumerate}
\item a $G_{\H}$-structure on $M$ is equivalent to an algebra subbundle $H \subset \End{}{TM}$ with $H_x \iso \H$
\item a $\GL{n,\H}$-structure on $M$ is equivalent to an algebra subbundle $H \subset \End{}{TM}$ with a global trivialization $\varphi : \H \times M \iso H$. This is equivalent to a choice of $I, J, K \in \End{}{TM}$ satisfying the quaterion algebra relations: $I^2 = J^2 = K^2 = -\id$ and $IJK = -\id$.
\end{enumerate}
\end{theorem}

\begin{proof}
DO THIS!!!
\end{proof}

\begin{rmk}
A $G_{\H}$-structure does not in general admit globally defined almost complex structures $I, J, K \in \End{}{TM}$ satisfying $IJK = - \id$. However, such always exist locally (although such choices are not canonical given the data in contrast to the global $I,J,K$ from a $\GL{n, \H}$-structure).
\end{rmk}

\subsection{Integrability Conditions}

\subsection{Special Holonomy}

\section{Some Real Algebras}

\subsection{Algebra Basics}

\begin{rmk}
Rings are assumed to be unital but need not be commutative. Homomorphisms of rings must preserve the unit.
\end{rmk}

\begin{defn}
An \textit{algebra} over a commutative ring $R$ is a $R$-module $A$ equiped with an $R$-bilinear map $B : A \times A \to A$ or equivalently an $R$-linear structure map $B : A \otimes_R A \to A$. A homomorphism of $R$-algebras $f : A \to A'$ is an $R$-linear map such that $f(B(x,y)) = B'(f(x), f(y))$.
\end{defn}

\begin{rmk}
We conventionally write $xy$ or $x \cdot y$ for $B(x,y)$.
\end{rmk}


\begin{defn}
Let $A$ be an $R$-algebra. Then we say that $A$ is:
\begin{enumerate}
\item \textit{unital} if there exists an element $1_A \in A$ such that $1_A \cdot x = x \cdot 1_A = x$ for all $x \in A$
\item \textit{associative} if for all $x,y,z \in A$ we have $(xy)z = x(yz)$
\item \textit{division} if for all $a, b \in A$ with $a \neq 0$ the equations $ax = b$ and $xa = b$ have unique solutions
\item \textit{zero-divisor free} if for all $a, b \in A$ such that $ab = 0$ either $a = 0$ or $b = 0$.
\end{enumerate}
\end{defn}

\begin{prop}
A unital algebra has a unique unit. 
\end{prop}

\begin{proof}
Suppose that $1_A, 1_A' \in A$ are both units. Then $1_A = 1_A \cdot 1_A' = 1_A'$ by the unit properties of $1_A$ and $1_A'$.
\end{proof}


\subsection{Division Algebras}

\begin{prop}
Let $R = K$ be a field. Then a finite dimensional $K$-algebra is zero-divisor free iff it is a divison algebra.
\end{prop}

\begin{proof}
For any nonzero $a \in A$. The maps $B(a, -)$ and $B(-, a)$ are endomorphisms of finite dimensional $K$-vectorspaces and thus are injective iff bijective. Injectivity is equivalent to $ab = 0$ implies $b = 0$ and $ba = 0$ implies $b = 0$ which is equivalent to being zero-divisor free. Bijectivity is equivalent to $A$ being a division ring.
\end{proof}

\begin{prop}
If $K$ is algebraically closed, then $K$ is the only finite dimensional unital division algebra over $K$.
\end{prop}

\begin{proof}
Let $A$ be a finite dimensional unital division algebra over $K$.
Since $K$ is algebraically closed, for each $a \in A$ the map $\ell_a : A \to A$ has an eigenvector, that is a nonzero $v \in A$ and $\lambda \in K$ such that $a v = \lambda v$ and thus $(a - \lambda \cdot 1_A) \cdot v = 0$. However, $v \neq 0$ so because $A$ is a finite dimensional division algebra it is zero divisor free meaning that $a = \lambda \cdot 1_A$. Since $a \in A$ is arbitrary, we see that $K \to A$ is an isomorphism.
\end{proof}

\newcommand{\C}{\mathbb{C}}

\begin{rmk}
The three hypotheses: finite dimensional, unital, division are each necessary. For example, over $K = \C$ we have $A = \C(x)$ which is a unital division algebra but not finite dimensional, $A = HMMMMMMMMMMMMMM$, and $A = \mathrm{M}_2(\C)$ which is finite dimensional and unital but not division because of nilpotent matrices.
\end{rmk}


\subsection{Properties of Subalgebras}

\begin{prop}
Let $A$ be an $R$-algebra. Then the center $Z(A) = \{ x \in A \mid \forall a \in A : a x = x a \}$ is a submodule. If $A$ is associative then $Z(A)$ is a subalgebra. 
\end{prop}

\begin{proof}
Clearly, if $x, y \in Z(A)$ then $r(x + y) = r x + ry = xr + yr = (x + y)r$ so $x + y \in Z(A)$. Furthermore, for all $r \in R$ we know that $a(rx) = r \cdot (ax) = r \cdot (xa) = (r x) \cdot a$ so $r x \in Z(A)$. Thus $Z(A)$ is a submodule. Similarly, if $A$ is associative, 
\[ a (xy) = (ax)y = (xa)y = x(ay) = x(ya) = (xy)a \]
and thus $xy \in Z(A)$ so $Z(A)$ is a subalgebra.
\end{proof}

\begin{rmk}
The center of a unital associative algebra is a ring, thus motivating the following result.
\end{rmk}

\begin{prop}
Unital associative $R$-algebras are equivalent to rings $A$ with the additional data of a homomorphism of (commutative) rings $\varphi : R \to Z(A)$.
\end{prop}

\begin{proof}
Let $A$ be a unital associative $R$-algebra. Then $(A, +, \cdot)$ defines exactly the structure of a ring. Furthermore, there is a map $\varphi : R \to Z(A)$ given by $r \mapsto r \cdot 1_A$. To see why this lands in the center, notice that for all $x \in A$ we have 
\[ x (r \cdot 1_A) = r \cdot (x \cdot 1_A) = r \cdot x = r \cdot (1_A \cdot x) = (r \cdot 1_A) \cdot x \]
because the product is $R$-bilinear. Conversely, given a ring $A$ and a map $\varphi : R \to Z(A)$ then $A$ becomes an $R$-module via $r \cdot x = \varphi(r)x$. Furthermore, because $\varphi(r) \in Z(A)$ the product become bilinear since $x (r \cdot y) = x \varphi(r) y = \varphi(r) xy = r \cdot (xy)$ (linearity in the first factor and distributative laws follow directly from associativity). 
\end{proof}

\begin{cor}
Rings are exactly unital associative algebras over $\Z$.
\end{cor}

\begin{defn}
Let $A$ be an $R$-algebra. A \textit{left (resp.\ right) ideal} is an $R$-submodule $I \subset A$ such that $A \cdot I \subset I$ (resp.\ $I \cdot A \subset I$). A \textit{two-sided ideal} or simply an \textit{ideal} is both a left and a right ideal.
\end{defn}

\begin{rmk}
Any left/right/two-sided ideal $I \subset A$ is a subalgebra of $A$.
\end{rmk}

\begin{rmk}
If $A$ is a ring, the algebra structure makes a right ideal into a right $A$-module, a left ideal into a left $A$-module, and an ideal into an $A$-bimodule.
\end{rmk}

\begin{defn}
Let $A$ be a unital associative $R$-algebra. Then we say that $A$ is:
\begin{enumerate}
\item \textit{central} if $Z(A) = R \cdot 1_A$
\item \textit{simple} if $A$ has no nontrivial ideals.
\end{enumerate}
\end{defn}

\begin{prop}
Let $A$ be a simple ring (or unital associative $R$-algebra). Then $K = Z(A)$ is a field and $A$ naturally has the structure of a central unital associative $K$-algebra.
\end{prop}

\begin{proof}
First, note that a simple commutative ring is a field (because then $(0)$ is maximal). I claim that if $A$ is a simple ring then $Z(A)$ is simple. Suppose that $I \subset Z(A)$ is an ideal. Then $x A$ is a two-sided ideal because $x \in Z(A)$. If $x \neq 0$ then $x A = A$ so $x a = 1$ for some $a \in A$. Forthermore, $ab = ab (xa) = ax ba = ba$ for all $b \in A$ so $a \in Z(A)$. Thus $1_A = xa \in I$ so $I = Z(A)$. Therefore, $Z(A)$ is a field and the identity map $K \to Z(A)$ makes $A$ a unital associative $K$-algebra such that $Z(A) = K$ so $A$ is central. 
\end{proof}


\subsection{Central Simple Algebras}

\begin{defn}
Let $K$ be a field. A \textit{Brauer algebra} over $K$ is a finite dimensional unital associative central simple algebra over $K$.
\end{defn}

\begin{prop}
Every Brauer algebra is isomorphic to a matrix algebra over a divison algebra over $K$.
\end{prop}

\begin{proof}
DO!!
\end{proof}

(DOOO THIS SECTION!!!)


\subsection{Normed Algebras}

\begin{defn}
Let $V$ be a $K$-vectorspace. A \textit{quadratic form} on $K$ is a map $q : V \to K$ so that,
\begin{enumerate}
\item $q(\lambda \cdot v) = \lambda^2 q(v)$ for each $\lambda \in K$ and $v \in V$
\item $B(v, w) = q(v + w) - q(v) - q(w)$ is a bilinear form $B : V \times V \to K$.
\end{enumerate}
\end{defn}

NONDENGENERATE

\begin{defn}
A \textit{composition algebra} over $K$ is a finite dimensional $K$-algebra equiped with a nondegenerate quadratic form $N : A \to K$ such that $N(xy) = N(x) N(y)$ for all $x,y \in A$.
\end{defn}

\begin{theorem}

\end{theorem}


\begin{rmk}
We want to define an algebra structure on $\R^n$. In analogy with the quaternions, we use a vectorspace splitting $\R^n = \R \cdot 1 \oplus \R^{n-1}$. We write elements as $x = a + \vec{v}$ with $a \in \R$ and define multiplication as follows,
\[ x y = (a + \vec{v}) (b + \vec{u}) = ab - \vec{v} \cdot \vec{u} + a \vec{u} + b \vec{v} + \vec{v} \times \vec{u} \]
where $\vec{v} \times \vec{u}$ is a bilinear ``cross product'' $V \times V \to V$. Clearly, this is a bilinear mulitiplication map. Furthermore, we have an involution $x \mapsto x^*$ via $a + \vec{v} \mapsto a - \vec{v}$. Then we want $(xy)^* = y^* x^*$ which is equivalent to $\vec{v} \times \vec{u} = - \vec{u} \times \vec{v}$. We want to define a norm $N(x) = x x^*$. Notice that $x x^* = a^2 + \vec{v} \cdot \vec{v}$ is nongegenerate. To have multiplicativity of $N$ we must have,
\[ (xy) (xy)^* = (xy) (y^* x^*) = x (y y^*) y \]
Writing this out,
\begin{align*}
(xy) (y^* x^*) & = (ab - \vec{v} \cdot \vec{u} + a \vec{u} + b \vec{v} + \vec{v} \times \vec{u}) (ab - \vec{v} \cdot \vec{u} - a \vec{u} - b \vec{v} - \vec{v} \times \vec{u})
\\
& = (ab - \vec{v} \cdot \vec{u})^2 + (a^2 ||\vec{u}||^2 + b^2 ||\vec{v}||^2 + ||\vec{v} \times \vec{u}||^2 + 2 ab \, \vec{v} \cdot \vec{u} + 2 a \, \vec{u} \cdot (\vec{v} \times \vec{u}) + 2 b \, \vec{v} \cdot (\vec{v} \times \vec{u}))
\\
& = (a^2 + ||\vec{v}||^2)(b^2 + ||\vec{u}^2||) + ||\vec{v} \times \vec{u}||^2 + (\vec{v} \cdot \vec{u})^2 - ||\vec{v}||^2 ||\vec{u}||^2 + 2 (a \vec{u} + b \vec{v}) \cdot (\vec{v} \times \vec{u})
\\
x (y y^*) x^* & = (a^2 + ||\vec{v}||^2)(b + ||\vec{u}||^2)
\end{align*}
Therefore, for these to agree we must have,
\[ ||\vec{v} \times \vec{u}||^2 + (\vec{v} \cdot \vec{u})^2 - ||\vec{v}||^2 ||\vec{u}||^2 + 2 (a \vec{u} + b \vec{v}) \cdot (\vec{v} \times \vec{u}) = 0 \]
Taking $a = b = 0$ we see that,
\[ || \vec{v} \times \vec{u} ||^2 = ||\vec{v}||^2 ||\vec{u}||^2 - (\vec{v} \cdot \vec{u})^2 \]
Taking $a = 0$ we see that $\vec{u} \cdot (\vec{v} \times \vec{u}) = 0$ and likewise for $b = 0$ we see that $\vec{v} \cdot (\vec{v} \times \vec{u}) = 0$. This justifies the following definition.
\end{rmk}

\begin{defn}
A \textit{cross product} on an inner product space is a bilinear map $\times : V \times V \to V$ such that,
\begin{enumerate}
\item $\vec{v} \cdot (\vec{v} \times \vec{u}) = 0$ and $(\vec{v} \times \vec{u}) \cdot \vec{u} = 0$
\item $|| \vec{v} \times \vec{u} ||^2 = || \vec{v} ||^2 || \vec{u} ||^2 - (\vec{v} \cdot \vec{u})^2$.
\end{enumerate}
\end{defn}


\end{document}


