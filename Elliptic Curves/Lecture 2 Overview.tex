\documentclass{article}
\usepackage[utf8]{inputenc}

\title{Elliptic Curves, Complex Tori, Modular Forms, and $\ell$-adic Galois Representations}
\author{Benjamin Church}

\usepackage[english]{babel}
\usepackage[a4paper, total={6in, 9in}]{geometry}
\usepackage{tikz-cd}
\usepackage{graphicx}
\usepackage{float}
 
\usepackage{amsthm, amssymb, amsmath, centernot}

\newcommand{\into}{\hookrightarrow}
\newcommand{\Gal}[1]{\mathrm{Gal}\left( #1 \right)}
\newcommand{\Aut}[1]{\mathrm{Aut} \small(#1 \small)}
\newcommand{\SL}[0]{\mathrm{SL}}
\newcommand{\End}[0]{\mathrm{End}}
\newcommand{\GL}[2]{\mathrm{GL}_{#1}\left(#2\right)}
\newcommand{\SO}[0]{\mathrm{SO}}
\newcommand{\Rept}[0]{\mathrm{Re}}
\newcommand{\Impt}[0]{\mathrm{Im}}
\newcommand{\dv}[0]{\mathrm{div}}
\newcommand{\Dv}[0]{\mathrm{Div}}
\newcommand{\Og}[0]{\mathit{\Omega}}
\newcommand{\og}[0]{\mathit{\omega}}

\newcommand{\notimplies}{%
  \mathrel{{\ooalign{\hidewidth$\not\phantom{=}$\hidewidth\cr$\implies$}}}}
 
\renewcommand\qedsymbol{$\square$}
\newcommand{\cont}{$\boxtimes$}
\newcommand{\divides}{\mid}
\newcommand{\ndivides}{\centernot \mid}
\newcommand{\Z}{\mathbb{Z}}
\newcommand{\R}{\mathbb{R}}
\newcommand{\Q}{\mathbb{Q}}
\newcommand{\N}{\mathbb{N}}
\newcommand{\C}{\mathbb{C}}
\newcommand{\Zplus}{\mathbb{Z}^{+}}
\newcommand{\Primes}{\mathbb{P}}
\newcommand{\colim}[1]{\mathrm{colim}(#1)}
\newcommand{\Ob}[1]{\mathrm{Ob}(#1)}
\newcommand{\cat}[1]{\mathcal{#1}}
\newcommand{\id}{\mathrm{id}}
\newcommand{\Hom}[2]{\mathrm{Hom}\left( #1, #2 \right)}
\newcommand{\catHom}[3]{\mathrm{Hom}_{#1}\left( #2, #3 \right)}
\newcommand{\Top}{\mathbf{Top}}
\newcommand{\pTop}{\mathbf{Top}_{\bullet}}
\newcommand{\Set}{\mathbf{Set}}
\newcommand{\pSet}{\mathbf{Set}_\bullet}
\newcommand{\hTop}{\mathbf{hTop}}
\newcommand{\phTop}{\mathbf{hTop}_{\bullet}}
\renewcommand{\Im}[1]{\mathrm{Im}(#1)}
\newcommand{\homspace}[2]{\left< #1, #2 \right>}
\newcommand{\rp}{\mathbb{RP}}
\newcommand{\coker}[1]{\mathrm{coker}\: #1}

\renewcommand{\d}[1]{\: \mathrm{d}#1 \:}
\newcommand{\dn}[2]{\: \mathrm{d}^{#1} #2 \:}
\newcommand{\deriv}[2]{\frac{\d{#1}}{\d{#2}}}
\newcommand{\nderiv}[3]{\frac{\dn{#1}{#2}}{\d{#3^2}}}
\newcommand{\pderiv}[2]{\frac{\partial{#1}}{\partial{#2}}}
\newcommand{\parsq}[2]{\frac{\partial^2{#1}}{\partial{#2}^2}}
\newcommand{\Imt}[0]{\mathrm{Im}}

\theoremstyle{definition}
\newtheorem{theorem}{Theorem}[section]
\newtheorem{lemma}[theorem]{Lemma}
\newtheorem{proposition}[theorem]{Proposition}
\newtheorem{example}[theorem]{Example}
\newtheorem{corollary}[theorem]{Corollary}
\newtheorem{remark}{Remark}[section]
\newtheorem{problem}{Problem}

\newenvironment{definition}[1][Definition:]{\begin{trivlist}
\item[\hskip \labelsep {\bfseries #1}]}{\end{trivlist}}


\newenvironment{lproof}{\begin{proof} \renewcommand{\qedsymbol}{}}{\end{proof}}
\renewcommand{\mod}[3]{\: #1 \equiv #2 \: mod \: #3 \:}
\newcommand{\nmod}[3]{\: #1 \centernot \equiv #2 \: mod \: #3 \:}
\newcommand{\ndiv}{\hspace{-4pt}\not \divides \hspace{2pt}}
\newcommand{\gen}[1]{\langle #1 \rangle}
\newcommand{\hook}{\hookrightarrow}
\newcommand{\Tor}[4]{\mathrm{Tor}^{#1}_{#2} \left( #3, #4 \right)}
\newcommand{\Ext}[4]{\mathrm{Ext}^{#1}_{#2} \left( #3, #4 \right)}

\tikzset{
    labl/.style={anchor=south, rotate=90, inner sep=.5mm}
}

\renewcommand{\bf}[1]{\mathbf{#1}}
\newcommand{\Class}[2]{\mathcal{C}^{#1} \left( #2 \right)}
\newcommand{\Res}[2]{\mathrm{Res}_{#1} \: #2}
\newcommand{\F}{\mathcal{F}}
\newcommand{\G}{\mathcal{G}}
\renewcommand{\O}{\mathcal{O}}
\newcommand{\iO}{\mathcal{O}}
\newcommand{\finfield}[1]{\mathbb{F}_{#1}}

\renewcommand{\S}[1]{\mathcal{S}_{#1}}
\newcommand{\M}[1]{\mathcal{M}_{#1}}
\newcommand{\E}[1]{\mathcal{E}_{#1}}

\renewcommand{\P}[2]{\mathbb{P}^{#1} \left( #2 \right)}

\newcommand{\h}{\mathfrak{h}}
\newcommand{\MG}{\SL_2(\Z)}
\newcommand{\res}{\mathrm{res}}

\begin{document}

\title{Lecture 2}

\maketitle

\section{The Cauchy Integral Formula}

Recall that if $f(z)$ has a single simple pole at $z_0$ then for any path containing $z_0$ we get,
\[ \oint_{\gamma} f(z) \d{z} = 2 \pi i \res_{z_0}(f) \]
In particular, suppose that $f$ is holomorphic everywhere inside the disk bounded by $\gamma$. Then the function,
\[ g(z) = \frac{f(z)}{z - z_0} \]
has a simple pole at $z_0$. Therefore, by the residue theorem,
\[ \oint_{\gamma} \frac{f(z)}{z - z_0} \d{z} = 2 \pi i \res_{z_0}(g) = 2 \pi i f(z_0) \]
This gives us the following incredible formula.

\begin{theorem}[Cauchy]
Let $\gamma : [0,1] \to \C$ be a closed curve bounding a disk $D \subset \C$. Suppose that $f : \C \to \C$ is holomorphic on all of $D$ then for any $w \in D^\circ$ (in the interior of $D$),
\[  f(w) = \frac{1}{2 \pi i} \oint_{\gamma} \frac{f(z)}{z - w} \d{z} \]
\end{theorem}

\noindent\\
This amazing theorem tells us that we can compute the values of a holomorphic function everywhere on a disk just given its values on the boundary.

\section{Power Series Expansion}

The Cauchy integral formula has another possibly even more amazing implication: that all holomorphic functions are analytic i.e. can be expressed as a convergent power series. The argument goes as follows. Suppose that $f : \C \to \C$ is holomorphic. Then for any point $w \in \C$ we can choose a fixed circle $\gamma$ centered at $w_0$ and enclosing $w$. Then by Cauchy's formula,
\[ f(w) = \frac{1}{2 \pi i} \oint_\gamma \frac{f(z)}{z - w} \d{z} \]
Now the trick! Notice that the right hand side only depends on $w$ through the function,
\[ \frac{1}{z - w}  \]
which we already know has a convergent Laurent series as follows,
\[ \frac{1}{z - w} = \frac{1}{(z - w_0) + (w_0 - w)} = \frac{1}{z - w_0} \cdot \frac{1}{1 - \frac{w - w_0}{z - w_0}} = \frac{1}{z - w_0} \left( 1 + \left( \frac{w - w_0}{z - w_0} \right) + \left( \frac{w - w_0}{z - w_0} \right)^2 + \cdots \right) \]
Why does this converge? Well because $w$ is inside the circle centered at $w_0$ (say of radius $r$) so $|w - w_0| < r$ and $z$ is on the boundary of this circle so $|z - w_0| = r$. Therefore,
\[ \left| \frac{w - w_0}{z - w_0} \right| < 1 \]
and thus we know this power series converges. Now here comes the analysis black magic. Because convergent power series converge uniformly, we can exchange sums and integrals to get,
\begin{align*}
f(w) & = \frac{1}{2 \pi i} \oint_\gamma \frac{f(z)}{z - w} \d{z} = \frac{1}{2 \pi i} \oint_{\gamma} \frac{f(z)}{z - w_0} \left( 1 + \left( \frac{w - w_0}{z - w_0} \right) + \left(\frac{w - w_0}{z - w_0} \right)^2 + \cdots \right)  \d{z} 
\\
& = \sum_{n = 0}^{\infty} \frac{1}{2 \pi i} \oint_{\gamma} \frac{f(z)}{z - w_0} \cdot \left( \frac{w - w_0}{z - w_0} \right)^{n} \d{z}
\\
& = \sum_{n = 0}^\infty (w - w_0)^n \left[ \frac{1}{2 \pi i} \oint_{\gamma} \frac{f(z)}{(z - w_0)^{n+1}} \d{z} \right] 
\end{align*}
Bam a convergent power series expression for $f$ centered at $w_0$. Even better we get a formula for the coefficients,
\[ f(w) = \sum_{n = 0}^\infty a_n (w - w_0)^n \quad \text{where} \quad a_n = \frac{1}{2 \pi i} \oint_{\gamma} \frac{f(z)}{(z - w_0)^{n+1}} \d{z} \]
In particular, these coefficients again only depend on the values of $f$ on the boundary of the disk. Amazing! 
Since any power series expansion for $f$ must be the taylor series,
\[ f(w) = \sum_{n = 0}^\infty a_n (w - w_0)^n = \sum_{n = 0}^\infty \frac{f^{(n)}(w_0)}{n!} (w - w_0)^n \]
Therefore, we can identify the coefficients $a_n$ from our integral formulae with the derivates of $f$ at $w_0$,
\[ a_n = \frac{f^{(n)}(w_0)}{n!} \]
Therefore, we conclude,
\[ f^{(n)}(w_0) = \frac{n!}{2 \pi i} \oint_\gamma \frac{f(z)}{(z - w_0)^{n+1}} \d{z} \]
This generalizes the Cauchy integral formula which the above reduces to in the case $n = 0$.

\section{Elliptic Functions}

On the real line, analytic periodic functions like sinusoids have wonderful properties are are intimately related to the points quadratic curves i.e. circles. We might naturally ask: ``what is the analogue of a periodic function on the complex plane?'' The following is one such notion.

\begin{definition}
A function $f : \C \to \C$ is \textit{doubly periodic} or \textit{elliptic} if there exits two independent (not real multiples of each other) complex numbers $\omega_1$ and $\omega_2$ such that,
\[ f(z + \omega_1) = f(z) \quad \text{and} \quad f(z + \omega_2) = f(z) \]
We define the lattice of periods,
\[ \Lambda = \{ n \omega_1 + m \omega_2 \mid n,m \in \Z \} \]
so that $f(z + \omega) = f(z)$ for all $\omega \in \Lambda$ so $f$ factors as a function on the quotient $f : \C / \Lambda \to \C$. 
\end{definition}

\begin{remark}
The space $\C / \Lambda$ is topologically a torus. It is equivalent to the ``fundamental parallelogram''
\[ \{ \alpha \omega_1 + \beta \omega_2 \mid \alpha, \beta \in [0, 1] \} \]
with opposite edges identified. This is the standard construction of a torus. However, different lattices $\Lambda$ will put a different complex structure on $\C / \Lambda$ which we therefore call a complex torus. 
\end{remark}
\noindent
We would want to consider elliptic homomorphic functions. However, a classical theorem in complex analysis poses a difficulty.

\begin{theorem}[Liouville]
Every bounded entire\footnote{holomorphic on the entire complex plane} function is constant.
\end{theorem}

\begin{proof}
Let $f : \C \to \C$ be entire and bounded everywhere by $M$. Take $w \in \C$ and let $C$ be a circle arround $z$ with radius $R$. Then applying the Cauchy integral formula,
\[ f'(w) = \frac{1}{2 \pi i} \oint_C \frac{f(z)}{(z - w)^2} \d{z} = \frac{1}{2 \pi} \int_0^{2 \pi} \frac{f(w + R e^{i \theta})}{R^2 e^{2 i \theta}} R \d{\theta} \]
Therefore,
\[ |f'(w)| = \frac{1}{2 \pi} \left| \oint_C \frac{f(z)}{(z - w)^2} \d{z} \right| \le \frac{1}{2 \pi} \int_0^{2 \pi} \frac{|f(w + R e^{i \theta})|}{R^2} R \d{\theta} \le \frac{1}{2 \pi} \int_0^{2 \pi} \frac{M}{R} \d{\theta} = \frac{M}{R} \]
which goes to zero in the limit $R \to \infty$. Since $R$ is arbitrarily large, $f'(w) = 0$ so $f$ is constant since it has zero derivative everywhere. 
\end{proof}

\begin{corollary}
There do not exist nonconstant doubly periodic holomorphic functions.
\end{corollary}

\begin{proof}
Suppose that $f : \C \to \C$ is holomorphic and doubly periodic with periods $\omega_1$ and $\omega_2$. Consider $f$ restricted to the so-called ``fundamental domain''
\[ D = \{ \alpha \omega_1 + \beta \omega_2 \mid \alpha, \beta \in [0, 1] \} \]
By using the periodicity, the behavior of $f$ everywhere is determined by its values on $D$. 
Since $D$ is compact\footnote{closed and bounded for subsets of Euclidean space} and $f$ is continuous (since it is holomorphic) it must be bounded\footnote{A continuous function on a compact set cannot diverge approaching a point because the set is closed nor diverge off to infinity because the set is bounded.} on $D$. Therefore, $f$ is entire and bounded and thus, by Liouville's theorem, constant. 
\end{proof}
Our plan has been thwarted as soon as it was devised. Since we cannot find interesting holomorphic examples of elliptic functions we now ask for the next best thing.
We want to consider elliptic meromorphic functions. That is, meromorphic functions $f : \C / \Lambda \to \C$. It turns out that now were are in luck. A canonical example of a meromorphic function is $z^{-2}$, think about how we might modify such a function such that it is doubly periodic.
\end{document}
