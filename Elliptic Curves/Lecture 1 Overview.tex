\documentclass{article}
\usepackage[utf8]{inputenc}

\title{Elliptic Curves, Complex Tori, Modular Forms, and $\ell$-adic Galois Representations}
\author{Benjamin Church}

\usepackage[english]{babel}
\usepackage[a4paper, total={6in, 9in}]{geometry}
\usepackage{tikz-cd}
\usepackage{graphicx}
\usepackage{float}
 
\usepackage{amsthm, amssymb, amsmath, centernot}

\newcommand{\into}{\hookrightarrow}
\newcommand{\Gal}[1]{\mathrm{Gal}\left( #1 \right)}
\newcommand{\Aut}[1]{\mathrm{Aut} \small(#1 \small)}
\newcommand{\SL}[0]{\mathrm{SL}}
\newcommand{\End}[0]{\mathrm{End}}
\newcommand{\GL}[2]{\mathrm{GL}_{#1}\left(#2\right)}
\newcommand{\SO}[0]{\mathrm{SO}}
\newcommand{\Rept}[0]{\mathrm{Re}}
\newcommand{\Impt}[0]{\mathrm{Im}}
\newcommand{\dv}[0]{\mathrm{div}}
\newcommand{\Dv}[0]{\mathrm{Div}}
\newcommand{\Og}[0]{\mathit{\Omega}}
\newcommand{\og}[0]{\mathit{\omega}}

\newcommand{\notimplies}{%
  \mathrel{{\ooalign{\hidewidth$\not\phantom{=}$\hidewidth\cr$\implies$}}}}
 
\renewcommand\qedsymbol{$\square$}
\newcommand{\cont}{$\boxtimes$}
\newcommand{\divides}{\mid}
\newcommand{\ndivides}{\centernot \mid}
\newcommand{\Z}{\mathbb{Z}}
\newcommand{\R}{\mathbb{R}}
\newcommand{\Q}{\mathbb{Q}}
\newcommand{\N}{\mathbb{N}}
\newcommand{\C}{\mathbb{C}}
\newcommand{\Zplus}{\mathbb{Z}^{+}}
\newcommand{\Primes}{\mathbb{P}}
\newcommand{\colim}[1]{\mathrm{colim}(#1)}
\newcommand{\Ob}[1]{\mathrm{Ob}(#1)}
\newcommand{\cat}[1]{\mathcal{#1}}
\newcommand{\id}{\mathrm{id}}
\newcommand{\Hom}[2]{\mathrm{Hom}\left( #1, #2 \right)}
\newcommand{\catHom}[3]{\mathrm{Hom}_{#1}\left( #2, #3 \right)}
\newcommand{\Top}{\mathbf{Top}}
\newcommand{\pTop}{\mathbf{Top}_{\bullet}}
\newcommand{\Set}{\mathbf{Set}}
\newcommand{\pSet}{\mathbf{Set}_\bullet}
\newcommand{\hTop}{\mathbf{hTop}}
\newcommand{\phTop}{\mathbf{hTop}_{\bullet}}
\renewcommand{\Im}[1]{\mathrm{Im}(#1)}
\newcommand{\homspace}[2]{\left< #1, #2 \right>}
\newcommand{\rp}{\mathbb{RP}}
\newcommand{\coker}[1]{\mathrm{coker}\: #1}

\renewcommand{\d}[1]{\: \mathrm{d}#1 \:}
\newcommand{\dn}[2]{\: \mathrm{d}^{#1} #2 \:}
\newcommand{\deriv}[2]{\frac{\d{#1}}{\d{#2}}}
\newcommand{\nderiv}[3]{\frac{\dn{#1}{#2}}{\d{#3^2}}}
\newcommand{\pderiv}[2]{\frac{\partial{#1}}{\partial{#2}}}
\newcommand{\parsq}[2]{\frac{\partial^2{#1}}{\partial{#2}^2}}
\newcommand{\Imt}[0]{\mathrm{Im}}

\theoremstyle{definition}
\newtheorem{theorem}{Theorem}[section]
\newtheorem{lemma}[theorem]{Lemma}
\newtheorem{proposition}[theorem]{Proposition}
\newtheorem{example}[theorem]{Example}
\newtheorem{corollary}[theorem]{Corollary}
\newtheorem{remark}{Remark}[section]
\newtheorem{problem}{Problem}

\newenvironment{definition}[1][Definition:]{\begin{trivlist}
\item[\hskip \labelsep {\bfseries #1}]}{\end{trivlist}}


\newenvironment{lproof}{\begin{proof} \renewcommand{\qedsymbol}{}}{\end{proof}}
\renewcommand{\mod}[3]{\: #1 \equiv #2 \: mod \: #3 \:}
\newcommand{\nmod}[3]{\: #1 \centernot \equiv #2 \: mod \: #3 \:}
\newcommand{\ndiv}{\hspace{-4pt}\not \divides \hspace{2pt}}
\newcommand{\gen}[1]{\langle #1 \rangle}
\newcommand{\hook}{\hookrightarrow}
\newcommand{\Tor}[4]{\mathrm{Tor}^{#1}_{#2} \left( #3, #4 \right)}
\newcommand{\Ext}[4]{\mathrm{Ext}^{#1}_{#2} \left( #3, #4 \right)}

\tikzset{
    labl/.style={anchor=south, rotate=90, inner sep=.5mm}
}

\renewcommand{\bf}[1]{\mathbf{#1}}
\newcommand{\Class}[2]{\mathcal{C}^{#1} \left( #2 \right)}
\newcommand{\Res}[2]{\mathrm{Res}_{#1} \: #2}
\newcommand{\F}{\mathcal{F}}
\newcommand{\G}{\mathcal{G}}
\renewcommand{\O}{\mathcal{O}}
\newcommand{\iO}{\mathcal{O}}
\newcommand{\finfield}[1]{\mathbb{F}_{#1}}

\renewcommand{\S}[1]{\mathcal{S}_{#1}}
\newcommand{\M}[1]{\mathcal{M}_{#1}}
\newcommand{\E}[1]{\mathcal{E}_{#1}}

\renewcommand{\P}[2]{\mathbb{P}^{#1} \left( #2 \right)}

\newcommand{\h}{\mathfrak{h}}
\newcommand{\MG}{\SL_2(\Z)}

\begin{document}

\title{Lecture 1}

\maketitle

\section{Holomorphic Functions}

\begin{definition}
We say that a complex function $f : \C \to \C$ is \textit{holomorphic} at $z \in \C$ if the limit,
\[ f'(z) = \lim_{w \to 0} \frac{f(z + w) - f(z)}{w} \]
exists in which case we call its value $f'(z)$ the \textit{complex derivative} of $f$ at $z$.
\end{definition}

\begin{remark}
Notice that the limit here means that $|h| \to 0$ so $h$ can go to zero from any direction.
\end{remark}

\begin{example}
Consider $f = z^2$. Then,
\[ f'(z) = \lim_{w \to 0} \frac{(z + w)^2 - z^2}{w} = \lim_{w \to 0} \frac{z^2 + 2 zw + w^2 - z^2}{w} = \lim_{w \to 0} \frac{2zw + w^2}{w} = \lim_{w \to 0} (2 z + w) = 2 z \]
as expected.
\end{example}

\begin{example}
Consider $f(z) = \overline{z}$. Then,
\[ f'(z) = \lim_{w \to 0} \frac{\overline{z + w} - \overline{z}}{w} =  \lim_{w \to 0}  \frac{\overline{z} + \overline{w} - \overline{z}}{w} =  \lim_{w \to 0} \frac{\overline{w}}{w} \]
However, suppose we send $w$ to zero along the real axis i.e. $w = t$ for $t \in \R$ and take,
\[ f'(z) = \lim_{t \to 0} \frac{\overline{t}}{t} = \lim_{t \to 0} \frac{t}{t} = 1 \]
However, if we send $w$ to zero along the imaginary axis i.e. $w = it$ $t \in \R$ and take,
\[ f'(z) = \lim_{t \to 0} \frac{\overline{i t}}{t} = \lim_{t \to 0} \frac{-it}{it} = -1 \]
Oh no. These do not agree so the limit cannot exist. Therefore $f(z) = \overline{z}$ is not holomorphic anywhere.
\end{example}

\begin{theorem}[Cauchy]
Let $\gamma : [0, 1] \to \C$ be a closed curve ($\gamma(0) = \gamma(1)$) in the complex plane and $f : \C \to \C$ be holomorphic everywhere on the region bounded by $\gamma$. Then,
\[ \oint_\gamma f(z) \d{z} = 0 \]
\end{theorem}

\begin{proof}
Look up the proof of Green's theorem and the Cauchy-Riemann equations. It is a good exercise to try and prove Cauchy's theorem from these facts.
\end{proof}

\begin{remark}
The integral,
\[ \oint_\gamma f(z) \d{z}  \]
can be defined as follows. Parametrize the loop as $\gamma(t)$ for $t \in [0, 1]$ and take ``by the chain rule'',
\[ \oint \gamma f(z) \d{z} = \int_0^1 f(\gamma(t)) \gamma'(t) \d{t} \]
this may serve as a definition of the loop integral.
\end{remark}

\begin{example}
Let's take $f(z) = z^2$ and consider a loop tracing out a circle of radius $r$ around the origin. Explicitly,
\[ \gamma(t) = r e^{2 \pi i t} \]
Then we can compute,
\[ \oint_\gamma f(z) \d{z} = \int_0^1 (r e^{2 \pi i})^2 (r e^{2 \pi i t}) \cdot (2 \pi i) \d{t} = (2 \pi i) r^3  \int_0^1 e^{3 \cdot (2 \pi i t)} \d{t} = 0 \]
Think about why this integral is zero!  
\end{example}

\section{Meromorphic Functions}

\begin{example}
Consider the function $f(z) = \frac{1}{z}$. It is not difficult to show that $f$ is holomorphic everywhere except at $z = 0$ where it blows up. We say $f$ has a pole at $z = 0$. Let's compute the loop integral for the same circular path $\gamma$,
\[ \oint_{\gamma} f(z) \d{z} = \int_0^1 \frac{1}{r e^{2 \pi i t}} (r e^{2 \pi i t} (2 \pi i) \d{t} = \int_0^1 (2 \pi i) \d{t} = 2 \pi i \]
Interesting! We might hypothesize that each pole in the interior of $\gamma$ contributes a factor of $2 \pi i$ to the loop integral. Indeed this is true if we include the ``residue'' at the pole.
\end{example}

\begin{definition}
We say a function $f : \C \to \C$ has a \textit{pole} of order $n$ at $z_0$ if close to $z_0$ we can write $f = (z - z_0)^{-n} u(z)$ where $u(z)$ is some nonvanishing holomorphic function near $z_0$.
\end{definition}

\begin{remark}
We say the pole is \textit{simple} if its order is $1$. For example,
\[ f(z) = \frac{1}{z} \]
has a simple pole at $z = 0$.
\end{remark}

\newcommand{\res}{\mathrm{res}}

\begin{definition}
Let $f$ have a simple pole at $z_0$. Then the residue of $f$ at $z_0$ is,
\[ \res_{z_0}(f) = \lim_{z \to z_0} (z - z_0) f(z) \]
Since, by definition, closed to $z_0$ we can write,
\[ f(z) = (z - z_0)^{-1} u(z) \]
and $u$ is holomorphic (hence continuous in the complex plane) we see that,
\[ \res_{z_0}(f) = \lim_{z \to z_0} (z - z_0) (z - z_0) u(z) = \lim_{z \to z_0} u(z) = u(z_0) \]
\end{definition}

\begin{definition}
We say a function $f : \Omega \to \C$ for $\Omega \subset \C$ is \textit{meromorphic} if there is a set of isolated poles $P \subset \Omega$ such that $f$ is holomorphic on $\Omega \setminus P$ and $f$ has a pole at each point $p \in P$.
\end{definition}

\begin{remark}
It is equivalent to say that a meromorphic function $f$ is a ratio of two holomorphic functions $g,h$ i.e.
\[ f(z) = \frac{g(z)}{h(z)} \]
Think about how you would prove this?
\end{remark}

\begin{theorem}[Residue]
Let $\gamma : [0,1] \to \C$ be a closed curve bouding a region $D \subset \C$ and let $f$ be meromorphic on $D$. Then,
\[ \oint_{\gamma} f(z) \d{z} = 2 \pi i \sum_{p \in D} \res_p(f) \]
\end{theorem}

\end{document}
