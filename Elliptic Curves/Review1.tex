\documentclass{article}
\usepackage[utf8]{inputenc}

\title{Elliptic Curves, Complex Tori, Modular Forms, and $\ell$-adic Galois Representations}
\author{Benjamin Church}

\usepackage[english]{babel}
\usepackage[a4paper, total={6in, 9in}]{geometry}
\usepackage{tikz-cd}
\usepackage{graphicx}
\usepackage{float}
 
\usepackage{amsthm, amssymb, amsmath, centernot}

\newcommand{\into}{\hookrightarrow}
\newcommand{\Gal}[1]{\mathrm{Gal}\left( #1 \right)}
\newcommand{\Aut}[1]{\mathrm{Aut} \small(#1 \small)}
\newcommand{\SL}[0]{\mathrm{SL}}
\newcommand{\End}[0]{\mathrm{End}}
\newcommand{\GL}[2]{\mathrm{GL}_{#1}\left(#2\right)}
\newcommand{\SO}[0]{\mathrm{SO}}
\newcommand{\Rept}[0]{\mathrm{Re}}
\newcommand{\Impt}[0]{\mathrm{Im}}
\newcommand{\dv}[0]{\mathrm{div}}
\newcommand{\Dv}[0]{\mathrm{Div}}
\newcommand{\Og}[0]{\mathit{\Omega}}
\newcommand{\og}[0]{\mathit{\omega}}

\newcommand{\notimplies}{%
  \mathrel{{\ooalign{\hidewidth$\not\phantom{=}$\hidewidth\cr$\implies$}}}}
 
\renewcommand\qedsymbol{$\square$}
\newcommand{\cont}{$\boxtimes$}
\newcommand{\divides}{\mid}
\newcommand{\ndivides}{\centernot \mid}
\newcommand{\Z}{\mathbb{Z}}
\newcommand{\R}{\mathbb{R}}
\newcommand{\Q}{\mathbb{Q}}
\newcommand{\N}{\mathbb{N}}
\newcommand{\C}{\mathbb{C}}
\newcommand{\Zplus}{\mathbb{Z}^{+}}
\newcommand{\Primes}{\mathbb{P}}
\newcommand{\colim}[1]{\mathrm{colim}(#1)}
\newcommand{\Ob}[1]{\mathrm{Ob}(#1)}
\newcommand{\cat}[1]{\mathcal{#1}}
\newcommand{\id}{\mathrm{id}}
\newcommand{\Hom}[2]{\mathrm{Hom}\left( #1, #2 \right)}
\newcommand{\catHom}[3]{\mathrm{Hom}_{#1}\left( #2, #3 \right)}
\newcommand{\Top}{\mathbf{Top}}
\newcommand{\pTop}{\mathbf{Top}_{\bullet}}
\newcommand{\Set}{\mathbf{Set}}
\newcommand{\pSet}{\mathbf{Set}_\bullet}
\newcommand{\hTop}{\mathbf{hTop}}
\newcommand{\phTop}{\mathbf{hTop}_{\bullet}}
\renewcommand{\Im}[1]{\mathrm{Im}(#1)}
\newcommand{\homspace}[2]{\left< #1, #2 \right>}
\newcommand{\rp}{\mathbb{RP}}
\newcommand{\coker}[1]{\mathrm{coker}\: #1}

\renewcommand{\d}[1]{\: \mathrm{d}#1 \:}
\newcommand{\dn}[2]{\: \mathrm{d}^{#1} #2 \:}
\newcommand{\deriv}[2]{\frac{\d{#1}}{\d{#2}}}
\newcommand{\nderiv}[3]{\frac{\dn{#1}{#2}}{\d{#3^2}}}
\newcommand{\pderiv}[2]{\frac{\partial{#1}}{\partial{#2}}}
\newcommand{\parsq}[2]{\frac{\partial^2{#1}}{\partial{#2}^2}}
\newcommand{\Imt}[0]{\mathrm{Im}}

\theoremstyle{definition}
\newtheorem{theorem}{Theorem}[section]
\newtheorem{lemma}[theorem]{Lemma}
\newtheorem{proposition}[theorem]{Proposition}
\newtheorem{example}[theorem]{Example}
\newtheorem{corollary}[theorem]{Corollary}
\newtheorem{remark}{Remark}[section]
\newtheorem{problem}{Problem}

\newenvironment{definition}[1][Definition:]{\begin{trivlist}
\item[\hskip \labelsep {\bfseries #1}]}{\end{trivlist}}


\newenvironment{lproof}{\begin{proof} \renewcommand{\qedsymbol}{}}{\end{proof}}
\renewcommand{\mod}[3]{\: #1 \equiv #2 \: mod \: #3 \:}
\newcommand{\nmod}[3]{\: #1 \centernot \equiv #2 \: mod \: #3 \:}
\newcommand{\ndiv}{\hspace{-4pt}\not \divides \hspace{2pt}}
\newcommand{\gen}[1]{\langle #1 \rangle}
\newcommand{\hook}{\hookrightarrow}
\newcommand{\Tor}[4]{\mathrm{Tor}^{#1}_{#2} \left( #3, #4 \right)}
\newcommand{\Ext}[4]{\mathrm{Ext}^{#1}_{#2} \left( #3, #4 \right)}

\tikzset{
    labl/.style={anchor=south, rotate=90, inner sep=.5mm}
}

\renewcommand{\bf}[1]{\mathbf{#1}}
\newcommand{\Class}[2]{\mathcal{C}^{#1} \left( #2 \right)}
\newcommand{\Res}[2]{\mathrm{Res}_{#1} \: #2}
\newcommand{\F}{\mathcal{F}}
\newcommand{\G}{\mathcal{G}}
\renewcommand{\O}{\mathcal{O}}
\newcommand{\iO}{\mathcal{O}}
\newcommand{\finfield}[1]{\mathbb{F}_{#1}}

\renewcommand{\S}[1]{\mathcal{S}_{#1}}
\newcommand{\M}[1]{\mathcal{M}_{#1}}
\newcommand{\E}[1]{\mathcal{E}_{#1}}

\renewcommand{\P}[2]{\mathbb{P}^{#1} \left( #2 \right)}

\newcommand{\h}{\mathfrak{h}}
\newcommand{\MG}{\SL_2(\Z)}

\begin{document}

\title{Diagnostic Problems}

\section{Calculus}


\subsection{}

Compute the following derivatives:

\begin{enumerate}
\item[(a.)] $\deriv{}{x} x^2$
\item[(b.)] $\deriv{}{x} (x^3 + a x + b)$
\item[(c.)] $\deriv{}{x} \tan{x}$
\end{enumerate}

\subsection{}

Compute the following definite and indefinite integrals:

\begin{enumerate}
\item[(a.)] \[ \int (x^3 + x + 1) \d{x} \]
\item[(b.)] \[ \int_0^{\pi/2} \sin{x} \d{x} \]
\item[(c.)] \[ \int_2^{\infty} \frac{3 x^2 - 1}{x(x^2 - 1)} \d{x}  \]
\item[(d.)] \[ \int_1^{\infty} \frac{3 x^2 - 1}{\sqrt{x(x^2 - 1)}} \d{x}  \]
\item[(e.)] \[ \int \frac{1}{x^2 - 1} \d{x} \]
\end{enumerate}

\subsection{}

Demonstrate uning limits why
\[ \deriv{}{x} e^x = e^x \]
use your favorite definition of $e^x$ and justify all limits used. 

\subsection{}

Do the following series converge or diverge?

\begin{enumerate}
\item[(a.)] \[ \sum_{n = 1}^\infty \frac{1}{n} \]
\item[(b.)] \[ \sum_{n = 1}^\infty \frac{1}{n^2} \]
\item[(c.)] \[ \sum_{n = 1, m = 1}^\infty \frac{1}{n^2 + m^2} \]
\item[(d.)] \[ \prod_{p \text{ primes}} \frac{1}{1 - p^{-2}} \]
\end{enumerate}

\section{Polynomials}

\subsection{}

Find all complex roots of the following polynomials

\begin{enumerate}
\item[(a.)] $x^3 - 6 x^2 + 11 x - 6$

\item[(b.)] $x^6 - 1$

\item[(c.)] $x^4 - 2 x^2 + 6$
\end{enumerate}

\subsection{}

Find as many integer solutions to 
\[ y^2 = x^3 + x + 1 \]
as you can.

\subsection{}

How many integer solutions does $x^2 - 2 y^2 = 1$ have?


\section{Complex Numbers}

\subsection{}

How many real roots does $x^3 + x + 1$ have? How many complex roots does $x^3 + x + 1$ have? What is the sum of the roots? What is the product?

\subsection{}

Find and justify reasonable values for the following expressions (here $i = \sqrt{-1}$)

\begin{enumerate}
\item[(a.)] $i^i$

\item[(b.)] $\log{(-1)}$

\item[(c.)] $\log{i}$
\end{enumerate}

\section{Algebra}


\begin{definition}
A group $G$ is a set with a binary operation $\circ$ which satisfies,
\begin{enumerate}
\item associativity, $x \circ (y \circ z) = (x \circ y) \circ z$
\item there exists an identity $e \in G$ such that $e \circ g = g \circ e = g$ for any $g \in G$
\item for each $g \in G$ there exists an inverse $g^{-1} \in G$ such that $g \circ g^{-1} = g^{-1} \circ g = e$
\end{enumerate} 
\end{definition}

\subsection{}

Which of the following pairs of a set and a binary operation are groups? If not, why?

\begin{enumerate}
\item[(a.)] the natural numbers $\{0, 1, 2, \dots \}$ with addition $+$
\item[(b.)] the integers $\Z$ with addition $+$
\item[(c.)] the integers $\Z$ with multiplication $\cdot$
\item[(d.)] the rational numbers $\Q$ with addition $+$
\item[(e.)] the rational numbers $\Q$ with multiplication
\item[(f.)] the nonzero rational numbers $\Q^\times$ with addition $+$
\item[(g.)] the nonzero rational numbers $\Q^\times$ with multiplication $\cdot$
\item[(h.)] the permutations of the numbers $\{1, \dots, n\}$ with composition of functions
\item[(i.)] pairs of integers $(a,b)$ with $a,b \in \Z$ and addition $(a,b) + (c,d) = (a+c, b+d)$.
\end{enumerate}

\begin{definition}
A group $G$ is \textit{finitely generated} if there exists a finite set $S \subset G$ such that every element in $g \in G$ can be expressed as a finite combination of elements of $S$ (and the inverses of elements in $S$) i.e. $g = s_1 \circ \cdots \circ s_n$ for $s_1, \dots, s_n \in S \cup S^{-1}$ where $S^{-1} = \{ s^{-1} \mid s \in S \}$.
\end{definition}

\subsection{}

Of the groups you identified above, which are finitely generated? If yes, provide a finite generating set. If no, justifty why no finite generating set exists.

\end{document}
