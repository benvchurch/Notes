\documentclass{article}
\usepackage[utf8]{inputenc}

\title{Elliptic Curves, Complex Tori, Modular Forms, and $\ell$-adic Galois Representations}
\author{Benjamin Church}

\usepackage[english]{babel}
\usepackage[a4paper, total={6in, 9in}]{geometry}
\usepackage{tikz-cd}
\usepackage{graphicx}
\usepackage{float}
 
\usepackage{amsthm, amssymb, amsmath, centernot}

\newcommand{\into}{\hookrightarrow}
\newcommand{\Gal}[1]{\mathrm{Gal}\left( #1 \right)}
\newcommand{\Aut}[1]{\mathrm{Aut} \small(#1 \small)}
\newcommand{\SL}[0]{\mathrm{SL}}
\newcommand{\End}[0]{\mathrm{End}}
\newcommand{\GL}[2]{\mathrm{GL}_{#1}\left(#2\right)}
\newcommand{\SO}[0]{\mathrm{SO}}
\newcommand{\Rept}[0]{\mathrm{Re}}
\newcommand{\Impt}[0]{\mathrm{Im}}
\newcommand{\dv}[0]{\mathrm{div}}
\newcommand{\Dv}[0]{\mathrm{Div}}
\newcommand{\Og}[0]{\mathit{\Omega}}
\newcommand{\og}[0]{\mathit{\omega}}

\newcommand{\notimplies}{%
  \mathrel{{\ooalign{\hidewidth$\not\phantom{=}$\hidewidth\cr$\implies$}}}}
 
\renewcommand\qedsymbol{$\square$}
\newcommand{\cont}{$\boxtimes$}
\newcommand{\divides}{\mid}
\newcommand{\ndivides}{\centernot \mid}
\newcommand{\Z}{\mathbb{Z}}
\newcommand{\R}{\mathbb{R}}
\newcommand{\Q}{\mathbb{Q}}
\newcommand{\N}{\mathbb{N}}
\newcommand{\C}{\mathbb{C}}
\newcommand{\Zplus}{\mathbb{Z}^{+}}
\newcommand{\Primes}{\mathbb{P}}
\newcommand{\colim}[1]{\mathrm{colim}(#1)}
\newcommand{\Ob}[1]{\mathrm{Ob}(#1)}
\newcommand{\cat}[1]{\mathcal{#1}}
\newcommand{\id}{\mathrm{id}}
\newcommand{\Hom}[2]{\mathrm{Hom}\left( #1, #2 \right)}
\newcommand{\catHom}[3]{\mathrm{Hom}_{#1}\left( #2, #3 \right)}
\newcommand{\Top}{\mathbf{Top}}
\newcommand{\pTop}{\mathbf{Top}_{\bullet}}
\newcommand{\Set}{\mathbf{Set}}
\newcommand{\pSet}{\mathbf{Set}_\bullet}
\newcommand{\hTop}{\mathbf{hTop}}
\newcommand{\phTop}{\mathbf{hTop}_{\bullet}}
\renewcommand{\Im}[1]{\mathrm{Im}(#1)}
\newcommand{\homspace}[2]{\left< #1, #2 \right>}
\newcommand{\rp}{\mathbb{RP}}
\newcommand{\coker}[1]{\mathrm{coker}\: #1}

\renewcommand{\d}[1]{\: \mathrm{d}#1 \:}
\newcommand{\dn}[2]{\: \mathrm{d}^{#1} #2 \:}
\newcommand{\deriv}[2]{\frac{\d{#1}}{\d{#2}}}
\newcommand{\nderiv}[3]{\frac{\dn{#1}{#2}}{\d{#3^2}}}
\newcommand{\pderiv}[2]{\frac{\partial{#1}}{\partial{#2}}}
\newcommand{\parsq}[2]{\frac{\partial^2{#1}}{\partial{#2}^2}}
\newcommand{\Imt}[0]{\mathrm{Im}}

\theoremstyle{definition}
\newtheorem{theorem}{Theorem}[section]
\newtheorem{lemma}[theorem]{Lemma}
\newtheorem{proposition}[theorem]{Proposition}
\newtheorem{example}[theorem]{Example}
\newtheorem{corollary}[theorem]{Corollary}
\newtheorem{remark}{Remark}[section]
\newtheorem{problem}{Problem}
\newtheorem{exercise}{Exercise}

\newenvironment{definition}[1][Definition:]{\begin{trivlist}
\item[\hskip \labelsep {\bfseries #1}]}{\end{trivlist}}


\newenvironment{lproof}{\begin{proof} \renewcommand{\qedsymbol}{}}{\end{proof}}
\renewcommand{\mod}[3]{\: #1 \equiv #2 \: mod \: #3 \:}
\newcommand{\nmod}[3]{\: #1 \centernot \equiv #2 \: mod \: #3 \:}
\newcommand{\ndiv}{\hspace{-4pt}\not \divides \hspace{2pt}}
\newcommand{\gen}[1]{\langle #1 \rangle}
\newcommand{\hook}{\hookrightarrow}
\newcommand{\Tor}[4]{\mathrm{Tor}^{#1}_{#2} \left( #3, #4 \right)}
\newcommand{\Ext}[4]{\mathrm{Ext}^{#1}_{#2} \left( #3, #4 \right)}

\tikzset{
    labl/.style={anchor=south, rotate=90, inner sep=.5mm}
}

\renewcommand{\bf}[1]{\mathbf{#1}}
\newcommand{\Class}[2]{\mathcal{C}^{#1} \left( #2 \right)}
\newcommand{\Res}[2]{\mathrm{Res}_{#1} \: #2}
\newcommand{\F}{\mathcal{F}}
\newcommand{\G}{\mathcal{G}}
\renewcommand{\O}{\mathcal{O}}
\newcommand{\iO}{\mathcal{O}}
\newcommand{\finfield}[1]{\mathbb{F}_{#1}}

\renewcommand{\S}[1]{\mathcal{S}_{#1}}
\newcommand{\M}[1]{\mathcal{M}_{#1}}
\newcommand{\E}[1]{\mathcal{E}_{#1}}

\renewcommand{\P}[2]{\mathbb{P}^{#1} \left( #2 \right)}

\newcommand{\h}{\mathfrak{h}}
\newcommand{\MG}{\SL_2(\Z)}
\newcommand{\res}{\mathrm{res}}

\begin{document}

\title{Lecture 3}

\maketitle

\section{Meromorphic Functions: a reprise}

There was some confusion about meromorphic functions. Here I hope to clear up some of the
confusion. The definition we have previously was:

\begin{definition}
We say a function $f : \Omega \to \C$ for $\Omega \subset \C$ is meromorphic if there is a set of isolated
poles $P \subset \Omega$ such that $f$ is holomorphic on $\Omega \setminus P$ and $f$ has a pole at each point $p \in P$.
\end{definition}
\noindent
where having a pole means the quite specific statement defined as follows.

\begin{definition}
We say that a function $f : \C \to \C$ has a \textit{pole} of order $n$ at $z_0$ if near $z_0$ we can write,
\[ f(z) = (z - z_0)^{-n} u(z) \]
where $u(z)$ is a nonvanishing holomorphic function.
\end{definition}

\begin{definition}
We say a function $f : \Omega \to \C$ for $\Omega \subset \C$ is \textit{meromorphic} if there is a set of isolated poles $P \subset \Omega$ such that $f$ is holomorphic on $\Omega \setminus P$ and $f$ has a pole at each point $p \in P$.
\end{definition}
\noindent
We see that a function with a jump discontinuity cannot be meromorphic because, in that case, near
the jump, the function would not blow up like $\sim \frac{1}{z^n}$. We can give an alternative definition which may be more intuitive.

\begin{definition}
A function $f$ is \textit{meromorphic} if it is a ratio of two holomorphic functions $g,h$ i.e.
\[ f(z) = \frac{g(z)}{h(z)} \]
where $h$ is not identically zero i.e. $h \neq 0$ ($h$ is not the constant zero function).
\end{definition}
\noindent
Perhaps a more sophisticated definition of a meromorphic functions involves the Riemann sphere
$\hat{\C} = \C \cup \{ \infty \}$ which is the complex plane closed into a sphere with the point at infinity. We will think about this space much more when we talk about projective space. Then we could make the
following definition.

\begin{definition}
A \textit{meromorphic} function on $\C$ is a holomorphic function $f : \C \to \hat{\C}$.
\end{definition}
\noindent
This corresponds to our previous notion that whenever $f$ is undefined (i.e. is sent to $\infty$) then $f$
must “blow up” nearby. This corresponds to the fact that for a continuous function $f : \C \to \hat{\C}$ to hit $\infty$ it must, nearby, pass through arbitrarily large values in $\C \subset \hat{\C}$. Thus we can’t have a function which is nice and bounded and suddently jumps up to $\infty$ since this corresponds to a discontinuity
viewed as a map $\C \to \hat{\C}$. Now let’s try to prove that these definitions agree.

\begin{theorem}
Let $f : \C \setminus P \to \C$ be a holomorphic function where $P \subset \C$ is an isolated set of points. Then the following are equivalent:
\begin{enumerate}
\item near each $p \in P$ we can write $f(z) = (z - p)^{-n} u(z)$ where $u(z)$ is holomorphic and nonvanishing near $p$.
\item $f = \frac{g}{h}$ where $g,h : \C \to \C$ are entire functions and $h$ is nonvanishing outside of $P$.
\item there exists a holomorphic map $\tilde{f} : \C \to \hat{\C}$ extending $f$ i.e. $\tilde{f}$ restricts to $f$ when $\tilde{f}(z) \neq \infty$, 
\[ \tilde{f} |_{\C \setminus P} = f \text{ viewed as a map } \C \setminus P \to \C = \hat{\C} \setminus \{ \infty \} \subset \hat{\C} \]
\end{enumerate}
\end{theorem}

\section{The Weirestrass $\wp$ Function}

Our plan to find a holomorphic doubly periodic (elliptic) function has been thwarted as soon as it
was devised. Since we cannot find interesting holomorphic examples of elliptic functions we now ask
for the next best thing. We want to consider elliptic meromorphic functions. That is, meromorphic
functions $f : \C/\Lambda \to \C$. It turns out that now were are in luck. Notice the following fact. If
$f : \C \to \C$ is any function then,
\[ g(z) = \sum_{\omega \in \Lambda} f(z + \omega) \]
is invariant under shifts by any $\omega \in \Lambda$ since this simply reorders the summation. However, in general there is no reason such a function should converge. In fact. using Liouville’s theorem, we know that if $f$ is any holomorphic function that $g$ cannot be a convergent holomorphic function. Thus we
should try with a simple meromorphic $f$ instead. In order for the sum to converge, we need that $f$
goes to zero sufficiently quickly (faster that $1/z$ because harmonic series do not converge). Therefore,
we might try the meromorphic function is $z^{-2}$ and try,
\[ f(z) = \sum_{\omega \in \Lambda} \frac{1}{(z + \omega)^2} \]
which is doubly periodic because if $\omega \in \Lambda$ then,
\[ f(z + \omega) = \sum_{\omega' \in \Lambda} \frac{1}{(z + \omega + \omega')^2} = \sum_{\omega'' \in \Lambda} \frac{1}{(z + \omega'')^2} = f(z) \]
However, this definition has one major flaw. That sum still does not converge! This is because,
\[ f(z) = \sum_{n,m \in \Z} \frac{1}{(z + n \omega_1 + m \omega_2)^2} \sim \sum_{r \in \Z^{+}} \frac{2 \pi r}{(z + r)^2} \sim 2 \pi \sum_{r \in \Z^{+}} \frac{r}{r^2} \to \infty \]
where I have reordered the sum to sum over circles with radius $r$ in the $\Z^2$ plane each circle having approximately $2 \pi r$ lattice points. To fix this, we use a clever subtraction to remove the divergent part of the sum and arrive at the following definition by Weierstrass.
\end{document}
