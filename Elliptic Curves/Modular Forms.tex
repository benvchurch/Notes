\documentclass{article}
\usepackage[utf8]{inputenc}

\title{Modular Forms}
\author{Benjamin Church}

\usepackage[english]{babel}
\usepackage[a4paper, total={6in, 9in}]{geometry}
\usepackage{tikz-cd}
\usepackage{graphicx}
\usepackage{float}
 
\usepackage{amsthm, amssymb, amsmath, centernot}

\newcommand{\into}{\hookrightarrow}
\newcommand{\Gal}[1]{\mathrm{Gal}\left( #1 \right)}
\newcommand{\Aut}[1]{\mathrm{Aut} \small(#1 \small)}
\newcommand{\SL}[2]{\mathrm{SL}_{#1}(#2)}
\newcommand{\End}[0]{\mathrm{End}}
\newcommand{\GL}[2]{\mathrm{GL}_{#1}\left(#2\right)}
\newcommand{\SO}[0]{\mathrm{SO}}
\newcommand{\Rept}[0]{\mathrm{Re}}
\newcommand{\Impt}[0]{\mathrm{Im}}
\newcommand{\dv}[0]{\mathrm{div}}
\newcommand{\Dv}[0]{\mathrm{Div}}
\newcommand{\Og}[0]{\mathit{\Omega}}
\newcommand{\og}[0]{\mathit{\omega}}

\newcommand{\notimplies}{%
  \mathrel{{\ooalign{\hidewidth$\not\phantom{=}$\hidewidth\cr$\implies$}}}}
 
\renewcommand\qedsymbol{$\square$}
\newcommand{\cont}{$\boxtimes$}
\newcommand{\divides}{\mid}
\newcommand{\ndivides}{\centernot \mid}
\newcommand{\Z}{\mathbb{Z}}
\newcommand{\R}{\mathbb{R}}
\newcommand{\Q}{\mathbb{Q}}
\newcommand{\N}{\mathbb{N}}
\newcommand{\C}{\mathbb{C}}
\newcommand{\Zplus}{\mathbb{Z}^{+}}
\newcommand{\Primes}{\mathbb{P}}
\newcommand{\colim}[1]{\mathrm{colim}(#1)}
\newcommand{\Ob}[1]{\mathrm{Ob}(#1)}
\newcommand{\cat}[1]{\mathcal{#1}}
\newcommand{\id}{\mathrm{id}}
\newcommand{\Hom}[2]{\mathrm{Hom}\left( #1, #2 \right)}
\newcommand{\catHom}[3]{\mathrm{Hom}_{#1}\left( #2, #3 \right)}
\newcommand{\Top}{\mathbf{Top}}
\newcommand{\pTop}{\mathbf{Top}_{\bullet}}
\newcommand{\Set}{\mathbf{Set}}
\newcommand{\pSet}{\mathbf{Set}_\bullet}
\newcommand{\hTop}{\mathbf{hTop}}
\newcommand{\phTop}{\mathbf{hTop}_{\bullet}}
\renewcommand{\Im}[1]{\mathrm{Im}(#1)}
\newcommand{\homspace}[2]{\left< #1, #2 \right>}
\newcommand{\rp}{\mathbb{RP}}
\newcommand{\coker}[1]{\mathrm{coker}\: #1}

\renewcommand{\d}[1]{\: \mathrm{d}#1 \:}
\newcommand{\dn}[2]{\: \mathrm{d}^{#1} #2 \:}
\newcommand{\deriv}[2]{\frac{\d{#1}}{\d{#2}}}
\newcommand{\nderiv}[3]{\frac{\dn{#1}{#2}}{\d{#3^2}}}
\newcommand{\pderiv}[2]{\frac{\partial{#1}}{\partial{#2}}}
\newcommand{\parsq}[2]{\frac{\partial^2{#1}}{\partial{#2}^2}}
\newcommand{\Imt}[0]{\mathrm{Im}}

\theoremstyle{definition}
\newtheorem{theorem}{Theorem}[section]
\newtheorem{lemma}[theorem]{Lemma}
\newtheorem{proposition}[theorem]{Proposition}
\newtheorem{example}[theorem]{Example}
\newtheorem{corollary}[theorem]{Corollary}
\newtheorem{remark}{Remark}[section]
\newtheorem{problem}{Problem}

\newenvironment{definition}[1][Definition:]{\begin{trivlist}
\item[\hskip \labelsep {\bfseries #1}]}{\end{trivlist}}


\newenvironment{lproof}{\begin{proof} \renewcommand{\qedsymbol}{}}{\end{proof}}
\renewcommand{\mod}[3]{\: #1 \equiv #2 \: mod \: #3 \:}
\newcommand{\nmod}[3]{\: #1 \centernot \equiv #2 \: mod \: #3 \:}
\newcommand{\ndiv}{\hspace{-4pt}\not \divides \hspace{2pt}}
\newcommand{\gen}[1]{\langle #1 \rangle}
\newcommand{\hook}{\hookrightarrow}
\newcommand{\Tor}[4]{\mathrm{Tor}^{#1}_{#2} \left( #3, #4 \right)}
\newcommand{\Ext}[4]{\mathrm{Ext}^{#1}_{#2} \left( #3, #4 \right)}

\tikzset{
    labl/.style={anchor=south, rotate=90, inner sep=.5mm}
}

\renewcommand{\bf}[1]{\mathbf{#1}}
\newcommand{\Class}[2]{\mathcal{C}^{#1} \left( #2 \right)}
\newcommand{\Res}[2]{\mathrm{Res}_{#1} \: #2}
\newcommand{\F}{\mathcal{F}}
\newcommand{\G}{\mathcal{G}}
\renewcommand{\O}{\mathcal{O}}
\newcommand{\iO}{\mathcal{O}}
\newcommand{\finfield}[1]{\mathbb{F}_{#1}}

\renewcommand{\P}[2]{\mathbb{P}^{#1} \left( #2 \right)}
\newcommand{\ord}{\mathrm{ord}}

\newcommand{\A}[1]{\mathcal{A}_{#1}}
\renewcommand{\S}[1]{\mathcal{S}_{#1}}
\newcommand{\M}[1]{\mathcal{M}_{#1}}
\newcommand{\E}[1]{\mathcal{E}_{#1}}
\newcommand{\half}{\mathcal{H}}

\begin{document}

\maketitle

\section{Exercises Chapter 1}

\section{Exercises Chapter 2}


\section{Exercises Chapter 3}

\subsubsection*{3.1.4}

\subsubsection*{3.2.2}
Consider the weight $0$ automorphic form $j : \half \to \hat{C}$ defined by,
\[ j(\tau) = 1728 \frac{g_2(\tau)^3}{\Delta(\tau)} \]
where $g_2(\tau) = 60 G_4(\tau)$ and $g_3(\tau) = 140 G_6(\tau)$ are Eisenstein series and $\Delta(\tau) = g_2(\tau)^3 - 27 g_3(\tau)^2$. Now consider the expansion,  

\subsubsection*{3.2.4}

Let $f \in \A{k}(\Gamma)$ be nonzero for some congruence subgroup $\Gamma$. Take any $g \in \A{k}(\Gamma)$ then the ratio $g / f$ is the ratio of meromophic weakly-modular functions of weight $k$ and thus is meromorphic of weight $0$. Therefore, $g / f$ is $\Gamma$-invariant. Therefore, $g / f$ descends to a meromorphic function on the quotient $Y(\Gamma)$. Furthermore, by definition, $f$ and $g$ are meromorphic at the cusps of $Y(\Gamma)$ implying that $g / f$ extends to a well-defined meromorphic function on $X(\Gamma)$. Thus, $g \in \C(X(\Gamma)) \cdot f$. Therefore,
\[ \A{k}(\Gamma) \cong \C(X(\Gamma)) \cdot f \]

\subsubsection*{3.3.6}

Let $X(\Gamma)$ be a modular curve and consider holomorphic differentials on $\Omega^1_{\text{hol}}(X(\Gamma))$. For any $\omega \in \Omega^1_{\text{hol}}(X(\Gamma))$ the pullback $\pi^*(\omega) = f(\tau) \d{\tau}$ on $\half$ is given by a holomorphic function $f(\tau)$ which is holomorphic at the cups of $X(\Gamma)$. For any $\gamma \in \Gamma$ clearly $\pi \circ \gamma = \pi$ which implies that $\gamma^* \circ \pi^* = \pi^*$ and thus $f(\tau) \d{\tau} = \pi^*(\omega)$ must be $\gamma^*$ invariant. Explicitly,
\[ \gamma^*(f(\tau) \d{\tau}) = f(\gamma \cdot \tau) \gamma'(\tau) \d{\tau} = j(\gamma, \tau)^{-2} f(\gamma \cdot \tau) \d{\tau} \]
Therefore if $\gamma^*(f(\tau) \d{\tau}) = f(\tau) \d{\tau}$ then $f$ must transform as $f(\gamma \cdot \tau) = j(\gamma \cdot \tau)^2 f(\tau)$ so $f$ is weakly-modular of weight $2$. Furthermore, we have shown that the order of vanishing of a differential $\omega \in \Omega^{\otimes n}(X(\Gamma))$ satisfies $\ord_{s}(\pi^*(\omega)) = \ord_{\pi(s)}(f) + n$ at each cusp. Therefore, in the case $n = 1$ and $\omega$ is holomorphic, the order of vanishing of $f$ is positive at each cusp. Therefore, $f$ is a cusp form. Clearly any weight $2$ cusp form satisfies above criteria for the differential form $f(\tau) \d{\tau}$ to decend to a holomorphic $1$-form in $\Omega^1_{\text{hol}}(X(\Gamma))$. Therefore,
\[ \S{2}(\Gamma) \cong \Omega^1_{\text{hol}}(X(\Gamma)) \]

\subsubsection*{3.5.2}

\subsubsection*{3.5.3}

\subsubsection*{3.5.5}

\subsubsection*{3.9.5}



\section{Eisenstein Series}

\subsection{Eisenstein of Weight One}

\begin{definition}
For a given lattice $\Lambda \subset \C$ we may define the Weierstrass $\sigma$-function,
\[ \sigma_\Lambda(z) = z \prod_{\omega \in \Lambda^\times} \left( 1 - \frac{z}{\omega} \right) e^{\frac{z}{\omega} = \tfrac{1}{2}\left( \frac{z}{\omega} \right)^2} \]
which is a logarithmic antidrivative of the Weierstrass $\zeta$-function,
\[ \frac{\sigma'_{\Lambda}(z)}{\sigma_\Lambda(z)} = \zeta_\Lambda(z) \]
where the Weierstrass $\zeta$-function is,
\[ \zeta_\Lambda(z) = \frac{1}{z} + \sum_{\omega \in \Lambda^\times} \left[ \frac{1}{z - \omega} + \frac{1}{\omega} + \frac{z}{\omega^2}  \right] \]
which is not doubly periodic on the lattice $\Lambda$ but is the antiderivative of a doubly periodic function, namely the Weierstrass $\wp$-function,
\[ \zeta_\Lambda'(z) = - \wp_\Lambda(z) \]
\end{definition}

\begin{proposition}[Transformation Properties]
If $\Lambda = \omega_1 \Z \oplus \omega_2 \Z$ then,
\begin{align*}
\eta_1(\Lambda) & = \zeta_\Lambda(z + \omega_1) - \zeta_\Lambda(z)
\\
\eta_2(\Lambda) & = \zeta_\Lambda(z + \omega_2) - \zeta_\Lambda(z)
\end{align*}
are constants of the lattice because they have zero derivative everywhere due to the periodicity of the derivative $\zeta'_\Lambda(z) = \wp_\Lambda(z)$. Given this we find the transformation rule,
\[ \zeta_\Lambda(z + n_1 \omega_1 + n_2 \omega_2) = \zeta_\Lambda(z) + n_2 \eta_1(\Lambda) + n_2 \eta_2(\Lambda) \]
If we impose the ratio $\tau = \omega_1 / \omega_2 \in \half$ then we find the Legendre condition,
\[ \eta_2(\Lambda) \omega_1 - \eta_1(\Lambda) \omega_2 = 2 \pi i \]  
\end{proposition}

\begin{proof}
Consider the integral arround the fundamental domain,
\[ \int_{D} \zeta_\Lambda(z) \d{z} = 2 \pi i \]
where $\zeta_\Lambda$ has a unique pole with residue $1$ at $z = 0$ inside the domain. Label the paths along the periods $\gamma_1$ and $\gamma_2$. If $\omega_1 / \omega_2 \in \half$ then traversing the domain counterclockwise means traversing $\gamma_2$ first. Thus, 
\begin{align*}
\int_D \zeta_\Lambda(z) \d{z} & = \int_{\gamma_2} \zeta_\Lambda(z) \d{z} + \int_{\gamma_1} \zeta_\Lambda(z + \omega_2) \d{z} - \int_{\gamma_2} \zeta_\Lambda(z + \omega_1) - \int_{\gamma_1} \zeta_\Lambda(z) \d{z} 
\\
& = \int_{\gamma_2} \left[ \zeta_\Lambda(z) - \zeta_\Lambda(z + \omega_1) \right] + \int_{\gamma_1} \left[ \zeta_\Lambda(z + \omega_2) - \zeta_\Lambda(z) \right] 
\\
& = - \int_{\gamma_2} \eta_1(\Lambda) \d{z} + \int_{\gamma_1} \eta_2(\Lambda) \d{z} = \eta_2(\Lambda) \omega_1 - \eta_2(\Lambda) \omega_2 
\end{align*}
proving that,
\[ \eta_2(\Lambda) \omega_1 - \eta_1(\Lambda) \omega_2 = 2 \pi i \]  
\end{proof}


For any vector $\bar{v} \in (\Z / N \Z)^\times$ of order $N$, the function of modular points,
\[ F_1^{\bar{v}}(\C / \Lambda, (\tfrac{\omega_1}{N}, \tfrac{\omega_2}{N})) = \zeta_\Lambda \left( \frac{c_v \omega_1 + d_v \omega_2}{N} \right) - \frac{c_v \eta_1(\Lambda) + d_v \eta_2(\Lambda)}{N} \]
is well-defined and degree-$1$ homogeneous with respect to acting with $\Gamma(N)$. The corresponding function,
\[ g_1^{\bar{v}}(\tau) = \frac{1}{N} \zeta_{\Lambda_\tau} \left( \frac{c_v \omega_1 + d_v \omega_2}{N} \right) - \frac{c_v \eta_1(\Lambda_\tau) + d_v \eta_2(\Lambda_\tau)}{N^2} \] 
is weakly-modular of weight $1$ for $\Gamma(N)$. Now splitting up the sum in $\zeta_\Lambda$ over zero, positive, and negative $c$ we arrive at a formula,
\[ g^{\bar{v}}(\tau) = G_1^{\bar{v}} - \frac{C_1}{N} \left( \frac{c_v}{N} - \frac{1}{2} \right) \]
restricting to the representative $0 \le c_v < N$ where,
\[ G_1^{\bar{v}} = \delta(\bar{c}_v) \zeta^{\bar{d_v}}(1) + \frac{C_1}{N} \sum_{n = 1}^\infty \sigma_0^{\bar{v}}(n) q_N^n \]
with $q_N = e^{2 \pi i  \tau / N}$. This function is the analouge of the Eisenstein series for low $k$. 

\begin{definition}
Let $\psi$ and $\varphi$ be Dirichlet characters modulo $u$ and $v$ where $uv = N$ and $\varphi$ primitive and $(\varphi \psi)(-1) = -1$. Define the sums,
\[ G_1^{\varphi, \psi}(\tau) = \sum_{c = 0}^{u - 1} \sum_{d = 0}^{v - 1} \sum_{e = 0}^{u - 1} \psi(c) \bar{\varphi}(d) g_1^{\overline{(cv, d + ev)}} (\tau) \]
and
\[ E_1^{\varphi, \psi}(\tau) = \delta(\varphi) L(0, \psi) + \delta(\psi) L(0, \varphi) + 2 \sum_{n = 1}^\infty \sigma_0^{\psi, \varphi}(n) q^n \]
\end{definition}

\begin{theorem}
\[ G_1^{\psi, \varphi} \in \M{1}(N, \psi \varphi) \quad \quad G_1^{\psi, \varphi}(\tau) = \frac{C_1 g(\bar{\varphi})}{v} E_1^{\psi, \varphi}(\tau) \]
\end{theorem}

\begin{theorem}
Let $N \in \Zplus$. The set,
\[ \{ E_1^{\psi, \varphi, t} \mid (\{ \psi, \varphi \}, t) \in A_{N, 1} \} \]
is a basis of $\E{1}(\Gamma_1(N))$
where $A_{N,1}$ is the set of $(\{ \psi, \varphi \}, t)$ such that $t \in \Zplus$ with $tuv \divides N$ and $\psi$, $\varphi$ are primitive Dirichlet characters modulo $u$ and $v$ satisfying the parity condition $(\psi \varphi)(-1) = -1$. For any character $\chi$ modulo $N$, the set,
\[ \{ E_1^{\psi, \varphi, t} \mid (\{ \psi, \varphi \}, t) \in A_{N, 1} \text{ and } \psi \varphi = \chi \} \]
is a basis of $\E{1}(N, \chi)$. 
\end{theorem}

\subsection{The Fourier Transform and Mellin Transform}

\renewcommand{\L}{\mathcal{L}}
\newcommand{\inner}[2]{\left< #1, #2 \right>}

\begin{definition}
For any $f \in \L^1(\R^\ell)$ function, define the Fourier transform $\hat{f} : \R^\ell \to \C$ via,
\[ \hat{f}(x) = \int_{y \in \R^{\ell}} f(y) e^{-2\pi i \inner{y}{x}} \d{y} \]
\end{definition}

\begin{remark}
If $f$ is additionally $\L^2(\R^\ell)$ then $\hat{f} \in \L^2(\R^\ell)$.
\end{remark}

\newcommand{\vt}{\vartheta}

\begin{definition}
Define the $\theta$-function, $\vt : \half \times \Zplus \to \C$
\[ \vt(\tau, \ell) = \sum_{n \in \Z^{\ell}} e^{\pi i |n|^2 \tau} \]
This function converges uniformly and absolutly on compact subsets of $\half$ via the M-test so $\vt$ is holomorphic. 
\end{definition}

\begin{remark}
Let $\tau = i t$ with $t \in \R^+$ then take the $\L^1$ function,
\[ f(x) = e^{- \pi |x|^2} \]
Therefore, we may write,
\[ \vt(it, \ell) = \sum_{n \in \Z^\ell} f(n \sqrt{t}) \]
\end{remark}

\begin{proposition}
Given $f(x) = e^{- \pi |x|^2}$ then the Fourier transform satisfies $\hat{f} = f$.
\end{proposition}

\begin{proof}
For the case $\ell = 1$ we have,
\[ \hat{f}_1(x) = \int_{-\infty}^{\infty} e^{-\pi (y^2 + 2 \pi i y x - x^2)} e^{-\pi x^2} \d{y} = e^{-\pi x^2} \int_{-\infty}^{\infty} e^{- \pi (y + i x)^2} \d{y} = e^{- \pi x^2} \]
via contour integration. Next, for $\ell > 1$ and any vector $x = (x_1, \dots, x_\ell)$ we have,
\[ f(x) = e^{-\pi |x|^2} = f_1(x_1) \dots f_\ell(x_\ell) \]
which are independent and thus,
\[ \hat{f} = \hat{f}_1 \cdots \hat{f}_\ell = f_1 \cdots f_\ell = f \]
\end{proof}

\begin{lemma}
For any $h \in \L^1(\R^\ell)$ let $\tilde{h}(x) = h(xr)$ then,
\[ \hat{\tilde{h}}(x) = r^{-\ell} \hat{g}(x/r) \]
\end{lemma}

\begin{proof}
Consider,
\begin{align*}
\hat{\tilde{h}} & = \int_{\R^{\ell}} \tilde{h}(y) e^{- 2\pi i \inner{x}{y}} \d{y} = \int_{\R^{\ell}} h(yr) e^{- 2\pi i \inner{x/r}{yr}} \d{y} 
\\
& r^{-\ell} \int_{\R^{\ell}} h(yr) e^{- 2\pi i \inner{x/r}{yr}} \d{(r y)} = r^{-\ell} \hat{h}(x/r) 
\end{align*}
\end{proof}

\begin{corollary}
The Fourier transform of $f(x \sqrt{t})$ is $t^{-\frac{\ell}{2}} f(x t^{-\frac{1}{2}})$.
\end{corollary}

\begin{theorem}[Poisson]
For any $h \in \L^1(\R^\ell)$ such that the sum,
\[ \sum_{d \in \Z^\ell} h(x + d) \] 
converges absolutly and uniformly on all compact subsets and is infinitely differentiable as a function of $x$. Then we have,
\[ \sum_{d \in \Z^\ell} h(x + d) = \sum_{m \in \Z^{\ell}} \hat{h}(m) e^{2 \pi i \inner{m}{x}} \]
\end{theorem}

\begin{proof}

\end{proof}

\begin{corollary}
Via Poisson summation with $x = 0$,
\[ \sum_{n \in \Z^\ell} f(n t^{\frac{1}{2}}) = t^{-\frac{\ell}{2}} \sum_{n \in \Z^\ell} f(n t^{-\frac{1}{2}}) \] 
which implies that,
\[ \vt(i/t, \ell) = t^{\frac{\ell}{2}} \vt(it \ell) \]
for $t > 0$. Therefore,
\[ \vt(-1/\tau, \ell) - (- \tau)^{\frac{\ell}{2}} \vt(\tau, \ell) \quad \quad \tau \in \half \]
is holomorphic and vanishes on $i \R^+$ which is not discreete and thus the function must vanish everywhere so 
\[ \vt(-1/\tau, \ell) = (- \tau)^{\frac{\ell}{2}} \vt(\tau, \ell) \]
\end{corollary}

\begin{remark}
Denote $\vt(\tau) = \vt(\tau, 1)$.
\end{remark}

\newcommand{\Rplus}{\R^+}
\renewcommand{\Re}[1]{\mathrm{Re}\left( #1 \right)}

\begin{definition}
Given a function $f : \Rplus \to \C$ the Mellin Transform of $f$ is,
\[ g(s) = \int_{0}^\infty f(t) t^s \frac{\d{t}}{t} \]
\end{definition}


\begin{theorem}

\end{theorem}

\begin{proof}
Consider the Mellin transformation of,
\[ \tfrac{1}{2} (\vt(it) - 1) = \sum_{n = 1}^\infty e^{-\pi n^2 t} \]
which is,
\[ g(s) = \int_0^\infty \sum_{n =1}^\infty e^{-\pi n^2 t} t^s \frac{\d{t}}{t}  = \frac{1}{2} \int_0^\infty (\vt(it) - 1) \frac{\d{t}}{t} \]
Sice $\vt(it)$ converges to $1$ in the limit $t \to \infty$ then the transformation law gives $\vt(it) \sim t^{-\frac{1}{2}}$ for $t \to 0$. Therefore, the integral converges for $\Re(s) > \frac{1}{2}$. Since $\vt(it) - 1 \to 0$ rapidly, the right integral converges for all $s$ and so we can pass the sum throught the integral,
\[ g(s) = \sum_{n = 1}^\infty \int_0^\infty e^{- \pi n^2 t} t^s \frac{\d{t}}{t} = \sum_{n = 1}^\infty (\pi n^2)^{-s} \int _0^\infty e^{-t} t^s \frac{\d{t}}{t} = \pi^{-s} \Gamma(s) \zeta(2 s) \]
Therefore for $\Re{s} > 1$,
\[ g(s/2) = \xi(s) = \pi^{-s/2} \Gamma(s/2) \zeta(s) \]
Furthermore, using the tranformation law,
\begin{align*}
\frac{1}{2} \int_0^1 (\vt(it) - 1) r^{s/2} \frac{\d{t}}{t} & = \frac{1}{2} \int_0^1 \vt(it) t^{s/2} \frac{d{t}}{t} - \frac{1}{s}
\\
& = \frac{1}{2} \int_1^\infty \vt(i / t) t^{-s / 2} \frac{\d{t}}{t} - \frac{1}{s}
\\
& = \frac{1}{2} \int_1^\infty \vt(i t) t^{(1 - s)/2} \frac{\d{t}}{t} - \frac{1}{s}
\\
& = \frac{1}{2} \int_1^\infty (\vt(i t) - 1) t^{(1 - s)/2} \frac{\d{t}}{t} - \frac{1}{s} - \frac{1}{1 - s}
\end{align*}
Therefore, 
\begin{align*}
\xi(s) & = \frac{1}{2} \int_0^\infty (\vt(it) - 1) t^{s/2} \frac{\d{t}}{t} = \frac{1}{2} \int_0^1 (\vt(it) - 1) t^{s/2} \frac{\d{t}}{t} + \frac{1}{2} \int_1^\infty (\vt(it) - 1) t^{s/2} \frac{\d{t}}{t}
\\
& = \frac{1}{2} \int_1^\infty (\vt(it) - 1) t^{(1 - s)/2} \frac{\d{t}}{t} + \frac{1}{2} \int_1^\infty (\vt(it) - 1) t^{s/2} \frac{\d{t}}{t} - \frac{1}{s} - \frac{1}{1 - s}
\\
& = \frac{1}{2} \int_1^\infty (\vt(it) - 1) (t^{(1 - s)/2} + t^{s / 2}) \frac{\d{t}}{t} - \frac{1}{s} - \frac{1}{1 - s} 
\end{align*}
The righthand side is holomorphic everywhere away from $s = 0$ and $s = 1$ and clearly invariant under $s \mapsto 1 - s$. Therefore, $\xi$ has a meromorphic continuation to the entire complex plane with simple poles at $s = 0, 1$ which satisfies,
\[ \xi(1 - s) = \xi(s) \]
\end{proof}

\subsection{Nonholomorphic Eisenstein Series}

\begin{definition}
Let $\bar{v}$ be a bector in $(\Z / N \Z)^2$ of order $N$, $k$ be any integer and $\epsilon_N = \frac{1}{2}$ if $N = 1,2$ and $\epsilon_N = 1$ otherwise,
\[ E^{\bar{v}}_k(\tau, s) = \epsilon_N \sum'_{\substack{(c,d) \equiv_N v \\ \gcd(c,d) = 1}} \frac{\Im{\tau}^s}{(c \tau + d)^k |c \tau + d|^{2 s}} \]
This converges absolutly on the right half plane $\{s \mid \Re(k + 2 s) > 2 \}$ and uniformly on compact subsets so it is analytic there. 
\end{definition}

\begin{remark}
We may rewrite this formula in instrinsic form. Let,
\[ \delta = \begin{pmatrix}
a & B 
\\
c_v & d_v 
\end{pmatrix} \in \SL{2}{\Z} \]
where $(c_v, d_v) \in \Z^2$ is a lift of $\bar{v}$. Furthermore, recall the positive parabolic subgroup,
\[ P_+ = \left\{ \begin{pmatrix}
1 & n
\\
0 & 1 
\end{pmatrix}
\; \middle| \; n \in \Z \right\} \]
We first extend the weigh$-k$ operator to two varable functions via,
\[ (f[\gamma]_k)(t,s) = j(\gamma, \tau)^{-k} f(\gamma(\tau), s) \quad \quad \gamma \in \SL{2}{\Z} \]
Therefore,
\[ E_k^{\bar{v}}(\tau, s) = \epsilon_N \sum_{\gamma (P_+ \cap \Gamma(N)) \setminus \Gamma(N) \delta} \Im{\tau}^s [\gamma]_k \]
Thus, $(E^{\bar{v}}_k [\gamma]_k)(\tau, s) = E^{\bar{v \gamma}}_k(\tau, s)$ for all $\gamma \in  \SL{2}{\Z}$, so in particular, 
\[ E^{\bar{v}}_k [\gamma]_k = E^{\bar{v}}_k \quad \quad \gamma \in \Gamma(N) \]
\end{remark}

\begin{remark}
As before, it is convient to work without normalization,
\[ G^{\bar{V}}_k(\tau, s) = \sum'_{(c,d) \equiv_N v} \frac{\Im{\tau}^s}{(c \tau + d)^k |c \tau + d|^{2 s}} \]
\end{remark}

\begin{remark}
We show that $G^{\bar{v}}_k$ has a meromorphic continuation to the entire $s$-plane on which the transformation law is still required to hold because it holds on a non-discreete subset. We will then consider $G^{\bar{v}}_k(\tau, 0)$ which can give Eisenstein series for all $k$ even when the origional sum does not converge.
\end{remark}

\begin{definition}
Specialize to $\R^2$ and define, $\vt : \GL{2}{\R} \to \R$ via,
\[ \vt(\gamma) = \sum_{n \in \Z^2} e^{-\pi |n \gamma|^2} \]
\end{definition}

\begin{lemma}
For $\gamma \in \SL{2}{\R}$, the Fourier transform of $f(x \gamma r)$ is $r^{-2} \hat{f}(x \gamma^{- \top} / r)$.
\end{lemma}

\begin{corollary}
In particular, recal $f(x) = e^{- \pi |x|^2}$ then via Poisson summation,
\[ r \sum_{n \in \Z^2} f(n \gamma r) =r^{-1} \sum_{n \in \Z^2} f(n \gamma^{- \top} / r) \]
\end{corollary}

\begin{proposition}
Define the matrix,
\[ S = \begin{pmatrix}
0 & - 1
\\
1 & 0
\end{pmatrix} \]
Then $S \gamma^{- \top} = \gamma S$ for all $\gamma \in \SL{2}{\R}$ and $|x S| = |x|$ for all $x \in \R^2$.
\end{proposition}

\begin{corollary}
Therefore, $f(n S \gamma^{- \top } / r) = f(n \gamma S / r) = e^{- \pi |n \gamma S|^2 / r^2} = e^{ - \pi |n \gamma|^2 / r^2} = f(n \gamma / r)$
\end{corollary}

\begin{proposition}
\[ r \vt(\gamma r) = r^{-1} \vt(\gamma / r) \quad \quad \gamma \in \SL{2}{\R} \quad r > 0 \]
\end{proposition}
 
\begin{proof}
By Poission summation,
\[ r \sum_{n \in \Z^2} f(n \gamma r) =r^{-1} \sum_{n \in \Z^2} f(n \gamma^{- \top} / r) \]
Furthermore, $nS$ runs over all $n \in \Z^2$ and thus,
\[ r \sum_{n \in \Z^2} f(n \gamma r) = r^{-1} \sum_{n \in \Z^2} f(n S \gamma^{- \top} / r) = \sum_{n \in \Z^2} f(n \gamma / r) \]
\end{proof}

\begin{remark}
As before, the Mellin transformation of $vt(\gamma t^{\frac{1}{2}})  - 1)$ is,
\[ g(s, \gamma) = \int_0^\infty \sum_{n \in \Z^2}' e^{- \pi |n \gamma|^2 t } t^s \frac{\d{t}}{t} = \int_0^\infty (\vt(\gamma t^{\frac{1}{2}}) - 1) t^s \frac{\d{t}}{t} \] 
converges at the right and also at the left for $\Re{s} > 1$ so we can change variables,
\[ g(s, \gamma) = \sum_{n \in \Z^2}' (\pi |n ga
|^2)^{-s} \int_0^\infty e^{-t} t^s \frac{\d{t}}{t} = \pi^{-s} \Gamma(s) \sum_{n \in \Z^2}' |n \gamma|^{-2 s} \]
for $\Re{s} > 1$. For any $\tau = x + i y \in \half$ define,
\[ \gamma_\tau = \frac{1}{\sqrt{y}} \begin{pmatrix}
y & x 
\\
0 & 1 
\end{pmatrix} 
\in \SL{2}{\R} \]
For $n = (c,d) \in \Z^2$ then,
\[ | n \gamma_\tau |^2 = \frac{|c \tau + d|^2}{y} \]
and thus,
\[ g(s, \gamma_\tau) = \pi^{-s} \Gamma(s) \sum'_{(c,d)} \frac{y^s}{|c \tau + d|^{2s}} = \pi^{-s} \Gamma(s) G_0(\tau, s) \quad \quad \Re{s} > 1 \]
where $G_0(\tau, s)$ is the weight $k = 0$ and level $N = 1$ series.
\end{remark}

\begin{proposition}
\[ g(s, \gamma) = \int_1^\infty (\vt(\gamma t^{\frac{1}{2}}) - 1) (t^s + t^{1 - s}) \frac{\d{t}}{t} - \frac{1}{s} - \frac{1}{1 - s} \]
and therefore $g(s, \gamma)$ is meromorphic with simple poles at $s = 0$ and $s = 1$ and otherwise holomorphic on the $s$-plane and it is invariant under $s \mapsto 1 - s$. For $\gamma = \gamma_\tau$, the function $\pi^{-s} \Gamma(s) G_0(\tau, s)$ is inariant under $s \mapsto 1 - s$ so it satisfies the functional equation,
\[ \pi^{-(1 - s)} \Gamma(1 - s) G_0(\tau, 1 - s) = \pi^{-s} \Gamma(s) G_0(\tau, s) \]
\end{proposition}

\begin{proof}
Use the transformation law. The calculation is idential to before.
\end{proof}

\subsubsection{Extending the Argument to Higher Weights and Levels}

\begin{definition}
Let $N \ge 1$ and $G = (\Z / N \Z)^2$. Consider a function $a : G \to \C$. Define the Fourier transform as,
\[ \hat{a}(\bar{v}) = \frac{1}{N} \sum_{\bar{w} \in G} a(\bar{W}) \zeta_N^{- \inner{w}{vS}} \]
and then the inverse Fourier transform gives,
\[ a(\bar{u}) = \frac{1}{N} \sum_{\bar{v} \in G} \hat{a}(\bar{v}) \zeta_N^{\inner{u}{vS}} \]
\end{definition}

\begin{definition}
Let $k \in \Zplus$ and $h_k$ be the harmonic polynomial,
\[ h_k(c,d) = (-i)^k (c + id)^k \quad \quad (c,d) \in \R^2 \]
For each $\bar{v} \in \in G$ we may define a function,
\[ \vt^{\bar{v}}_k(\gamma) = \sum_{n \in \Z^2} h_k((v/N + n) \gamma) e^{-\pi |(v /N + n) \gamma)|^2} \quad \quad \gamma \in \GL{2}{\R} \]
If we define the Schwarts function,
\[ f_k(x) = h_k(x) f(x) \]
then we may write,
\[ \vt^{\bar{v}}_k(\gamma) = \sum_{n \in \Z^2} f_k((v / N + n) \gamma) \]
\end{definition}

\begin{proposition}
For any function $a : G \to \C$ the associated $\theta$-function,
\[ \Theta^a_k(\gamma) = \sum_{\bar{v} \in G} (a(\bar{v}) + (-1)^k \hat{a}(-\bar{v})) \vt^{\bar{v}}_k(\gamma N^{\frac{1}{2}}) \quad \quad \gamma \in \GL{2}{\R} \]
satisfies the transfromation law,
\[ r \Theta^a_k(\gamma r) = r^{-1} \Theta^a_k(\gamma r^{-1}) \quad \quad \gamma \in \SL{2}{\R} \quad r > 0 \]
\end{proposition}

\begin{proof}
The same Poisson summation trick with the matrix $S$ plus a good deal of reagrangement gives this result.
\end{proof}

\begin{theorem}
For any function $a : G \to \C$ let,
\[ G_k^a(\tau, s) = \sum_{\bar{v}} (a(\bar{v}) + (-1)^k \hat{a}(-\bar{v})) G^{\bar{v}}_k(\tau, s) \quad \quad \Re{k/2 + s} > 1 \]
Then for any integer $k$ and $\tau \in half$,
\[ (\pi / N)^{-s} \Gamma(|k|/2 + s) G^a_k(\tau, s - k / 2) \quad \quad \Re{s} > 1 \]
has a continuation to the full $s$-plane invariant under $s \mapsto 1 - s$ and analytic for $k \neq 0$ with simple poles at $s = 0,1$ for $k = 0$. 
\end{theorem}

\begin{proof}
Consider the Mellin transform,
\[ g^a_k(s, \gamma) = \int_0^\infty \Theta^a_k(\gamma t^{\frac{1}{2}}) t^s \frac{\d{t}}{t} \]
Using the covergence to pass sums through integrals and $h_k(xr) = h_k(x) r^k$ for $r \in \R$ gives,
\begin{align*}
g^a_k(s, \gamma) = \pi^{-k/2 - s} \Gamma(k/2 + s) N^s \sum_{\bar{v} \in G} a(\bar{v}) + (-1)^k \hat{a}(-\bar{v})) \sum_{n \equiv_N v}' h_k(n \gamma) |n \gamma|^{-k - 2 s} 
\end{align*}
For $\gamma = \gamma_\tau$ we have,
\[ h_k(n \gamma) = \frac{1}{y^k/2} \overline{(c \tau + d)}^k \quad \text{and} \quad |n \gamma|^{-k - 2s} = y^{k/2 + s} / |c \tau + d|^{k + 2s} \]
Therefore,
\[  h_k(n \gamma) |n \gamma|^{-k - 2s} = \frac{y^s}{(c \tau + d)^k |c \tau + d|^{2 (s - k /2)}} \]
Thus, for $\Re{s} > 1$,
\[ g^a_k(s, \gamma_\tau) = \pi^{-k/2 - s} \Gamma(k/2 + s) N^s y^{k/2} G^a_k(\tau, s - k /2) \]
As before, using the above transformation property,
\[ g^{\bar{v}}_k(s, \gamma) = \int_1^\infty \Theta^ak(\gamma t^{\frac{1}{2}}) (t^s + t^{1 - s}) \frac{\d{t}}{t} \]
Clearly this function is entire in $s$ and ivaraint under $s \mapsto 1 - s$. 
\end{proof}

\section{Modular Forms Via $\theta$-Functions}

\newcommand{\e}[1]{\mathbf{e}\left( #1 \right)}

\begin{remark}
Introduce the notation $\e{z} = e^{2 \pi i z}$. Let $A = \Z[\zeta_3]$ and $\alpha = i \sqrt{3}$ and $B = \alpha^{-1} A$. Thus,
\[ A \subset B \subset \tfrac{1}{3} A \]
Note that for $x = x_1 + x_2 \zeta_3 \in \R[\zeta_3]$,
\[ |x|^2 = x_1^2 - x_1 x_2 + x_2^2 \]
\end{remark}

\newcommand{\tr}[1]{\mathrm{tr}\left( #1 \right)}

\begin{lemma}
Let $\tr{z} = z + z^*$ then for $x, y \in \C$,
\[ |x + y|^2 = |x|^2 + \tr{x y^*} + |y|^2 \]
\end{lemma}

\begin{definition}
Take $\bar{u} \in \frac{1}{3} A / NA$ for $N \in \Zplus$ and define,
\[ \theta^{bar{u}}(\tau, N) = \sum_{n \in A} \e{N \cdot | u /N + n|^2 \tau} \quad \quad \tau \in \half \]
which is holomorphic.
\end{definition}

\begin{lemma}
Let $N \in \Zplus$ then,
\begin{flalign*}
&& \theta^{\bar{u}}(\tau + 1, N) & = \e{\frac{|u|^2}{N}} \theta^{\bar{u}}(\tau, N) && \bar{u} \in B / NA
\\
&& \theta^{\bar{U}}(\tau, N) & = \sum_{\substack{\bar{v} \in B / d NA // \bar{v} \equiv_{NA} \bar{u}}} \theta^{\bar{v}}(d \tau, d N) && \bar{u} \in B / NA 
\\
&& \theta^{\bar{U}}(\tau, N) & = \frac{-i \tau}{N \sqrt{3}} \sum_{\bar{w} \in B / NA} \e{- \frac{\tr{vw^*}}{N}} \theta^{\bar{w}}(\tau, N) && \bar{v} \in B / NA 
\end{flalign*}
\end{lemma}

\begin{proof}
First, for $u \in B$ and $n \in A$,
\[ N  \, | \tfrac{u}{N} + n |^2 \equiv_\Z \tfrac{|u|^2}{N} \]
because $\tr{u n^*} \in Z$. 
\end{proof}

\begin{definition}
Consider the group,
\[ \Gamma_0(3N, N) = \left\{ 
\begin{pmatrix}
a & b 
\\
c & d 
\end{pmatrix} 
\in \SL{2}{\Z} \; \middle| \; b \equiv_N 0 \quad c \equiv_{3N} 0 \right\} \] 
\end{definition}

\begin{theorem}
Let $N \in \Zplus$ then,
\[ (\theta^{\bar{u}} [\gamma]_1)(\tau, N) = \left( \frac{d}{3} \right) \theta^{\bar{a u}}(\tau, N) \quad \quad \bar{u} \in  A / NA \quad \quad \gamma = 
\begin{pmatrix}
a & b 
\\
c & d
\end{pmatrix}
\in \Gamma_0(3N, N) \]
\end{theorem}

\begin{proof}
Since $\theta^{\bar{-au}} = \theta^{\bar{au}}$ we may assume that $d > 0$. Write,
\[ \frac{a \tau + b}{c \tau + d} = \frac{1}{d} \left( \frac{1}{d / \tau + c} + b \right) \]
Apply the lemma to find,
\[ \theta^{\bar{u}}(\gamma(\tau), N) = \sum_{\substack{\bar{v} \in B / d N A \\ \bar{v} \equiv_{NA} \bar{u}}} \e{\frac{b |v|^2}{d N}} \theta^{\bar{v}}\left( - \frac{1}{-d / \tau - c}, d N \right) \]
The final statment of the lemma gives,
\begin{align*}
\theta^{\bar{u}}(\gamma(\tau), N) & = \frac{i (d / \tau + c)}{d N \sqrt{3}} \sum_{\substack{\bar{v}, \bar{w} \in B / d NA \\ \bar{v} \equiv_{NA} \bar{u}}} \e{ \frac{b |v|^2 - \tr{vw^*}}{dN}} \theta^{\bar{w}}(-d / \tau - c, d N) 
\\
& = \frac{i (c \tau + d)}{d N \sqrt{3} \tau} \sum_{\substack{\bar{v}, \bar{w} \in B / d NA \\ \bar{v} \equiv_{NA} \bar{u}}} \e{ \frac{b |v|^2 - \tr{vw^*} - c |w|^2}{dN}} \theta^{\bar{w}}(-d / \tau - c, d N)
\end{align*}
Note that $c w \in NA$ for $w \in B$ since $c \equiv_{3N} 0$. Therefore,
\begin{align*}
\sum_{\substack{\bar{v}, \bar{w} \in B / d NA \\ \bar{v} \equiv_{NA} \bar{u}}} \e{ \frac{b |v|^2 - \tr{vw^*} - c |w|^2}{dN}} & = \sum_{\substack{\bar{v}, \bar{w} \in B / d NA \\ \bar{v} \equiv_{NA} \bar{u}}} \e{ \frac{b |v - cw|^2 - \tr{(v - cw)w^*} - c |w|^2}{dN}} 
\\
& = \e{-\frac{a \tr{v w^*}}{d N}} \sum_{\substack{\bar{v}, \bar{w} \in B / d NA \\ \bar{v} \equiv_{NA} \bar{u}}}  \e{\frac{b |v|^2}{dN}}
\end{align*}
since, using $ad - bc = 1$,
\begin{align*}
b |v - c w|^2 - \tr{(v - cw) w^*} - c|w|^2 - b |v|^2 & = - bc \, \tr{v w^*} - \tr{(v - cw) w^*} + c(bc - 1) |w|^2
\\
& = - ad \, \tr{v w^*} + 2 c |w|^2 + c(b c - 1) |w|^2
\\
& = - ad \, \tr{v w^*} + c(b c + 1) |w|^2
\\
& = - ad \, \tr{v w^*} + adc |w|^2
\end{align*}
and $adc \equiv_{dN} 0$ since $c \equiv_N 0$. Since $b \equiv_N 0$ the summand depends only on $v$ modulo $d A$, and since $(d, N) = 1$ and $\bar{u} \in A / NA$ the sum can be rewitten as,
\[ \sum_{\bar{v} \in A / d A} \e{\frac{b |v|^2}{dN}} = \left( \frac{d}{3} \right) d \]
Applying the transformation formula,
\begin{align*}
\theta^{\bar{u}}(\gamma(\tau), N) & = \frac{i (c \tau + d)}{N \sqrt{3} \tau} \left( \frac{d}{3} \right) \sum_{\bar{w} \in B / d N A} \e{- \frac{\tr{a u w^*}}{N}} \theta^{\bar{w}}(d(-1/\tau), d N)
\\
& = \frac{i (c \tau + d)}{N \sqrt{3} \tau} \left( \frac{d}{3} \right) \sum_{\bar{v} \in B / N A} \e{- \frac{\tr{a u v^*}}{N}} \theta^{\bar{w}}(-1/\tau, N)
\\
& = \frac{i (c \tau + d)}{N \sqrt{3} \tau} \left( \frac{d}{3} \right) \sum_{\bar{w} \in B / d N A} \e{- \frac{\tr{a u w^*}}{N}} \theta^{\bar{w}}(d(-1/\tau), d N)
\\
& = \frac{c \tau + d}{3 N^2} \left( \frac{d}{3} \right) \sum_{\bar{v}, \bar{w} \in B / NA} \e{- \frac{\tr{v w^* a u v^*}}{N}} \theta^{\bar{W}}(\tau, N)
\end{align*}
Furthermore, the inner sum is,
\[ \sum_{\bar{v} \in B / NA} \e{ - \frac{-\tr{v(w^* + a u^*)}}{N}} = \begin{cases}
3 N^2 & \bar{w} = - \bar{a u}
\\
0 & \text{else}
\end{cases} \]
\end{proof}

\begin{remark}
These functions have an interesting transfromation property yet are not quite modular forms in the sense we like since $\Gamma_0(3N, N)$ is not a usual congruence subgroup. We fix this via conjugation with,
\[ \delta = 
\begin{pmatrix}
N & 0 
\\
0 & 1 
\end{pmatrix} \]
then,
\[ \delta \Gamma_0(3 N^2) \delta^{-1} = \Gamma_0(3N, N) \]
and conjugation preserves the diagonal. Then for any $\gamma \in \Gamma_0(3 N^2)$ let $\gamma' = \delta \gamma \delta^{-1} \in \Gamma_0(3N, N)$ we have,
\[ (\theta^{\bar{u}} [\delta \gamma]_1)(\tau) = (\theta^{\bar{u}}[\gamma' \delta]_1)(\tau) = \left( \frac{d}{3} \right) (\theta^{\bar{au}} [\delta]_1)(\tau) \]
where $d$ is the lower right entry of both $\gamma$ and $\gamma'$ since the diagonal is preserved under conjugation. 
\end{remark}

\begin{theorem}
Let $N \in \Zplus$ and let $\chi : (A / NA)^\times \to \C^\times$ be a character. Extend $\chi$ to $A$ multiplicativly. Then define,
\[ \theta_\chi(\tau) = \frac{1}{6} \sum_{\bar{u} \in A / NA} \chi(u) \theta^{\bar{u}}(N \tau, N) \]
Then,
\[ \theta_\chi[\gamma]_1 = \chi(d) \left( \frac{d}{3} \right) \theta_\chi \]
and therefore $\theta_\chi \in \M{1}(3N^2, \psi)$ where,
\[ \psi(d) = \chi(d) \left( \frac{d}{3} \right) \]
\end{theorem}

\begin{remark}
Note,
\[ \theta_\chi(\tau) = \frac{1}{6} \sum_{n \in A} \chi(n) \e{|n|^2 \tau} = \sum_{m = 0}^\infty a_m(\theta_\chi) e^{2 \pi i \tau} \]
where,
\[ a_m(\theta_\chi) = \frac{1}{6} \sum_{\substack{n \in A \\ |n|^2 = m}} \chi(n) \]
This shows that $a_n(\theta) \sim n$. 
\end{remark}

\begin{theorem}[Cubic Reciprocity]
Let $d \in \Zplus$ be cubefree and let,
\[ N = 3 \prod_{p \divides d} p \]
then there exists a character,
\[ \chi : (A / NA)^\times \to \left< \zeta_3 \right> \]
such that $\chi$ extended to $A$ is trivial on $A^\times$ and primes $p \ndivides N$ while on $\pi \in A$ such that $\pi \bar{\pi}$ is a prime $p \ndivides A$ it is tivial iff $d$ is a cube modulo $p$. 
\bigskip\\
Furthermore, for each rational prime $p \equiv_3 1$ there exists some $\pi = a + b \zeta_3 \in A$ such that,
\[ \{ n \in A \mid |n|^2 = p \} = A^\times \pi \cup A^\times \bar{\pi} \]
\end{theorem}

\begin{corollary}
Taking the above character $\chi$, and construction $\theta_\chi \in \M{1}{3 N^2, \psi}$ where $\psi$ is the quadratic characer which condunctor $3$ then the previous remark shows that the Fourier coefficients of primee index are,
\[ a_p(\theta_\chi) = 
\begin{cases}
2 & p \equiv_3 1 \text{ and d is a nonzro cube modulo p} 
\\
- 1 & p \equiv_3 1 \text{ and d is not a cub modulo p} 
\\
0 & p \equiv_3 2 \text{ or } p \divides 3 d 
\end{cases} \]
\end{corollary}

\section{Jacobian Varieties}
\newcommand{\Jac}[1]{\mathrm{Jac}(#1)}

\subsection{Introduction}

Let $X$ be a compact Riemann surfact of genus $g \ge 1$. Fix a basepoint $x_0 \in X$ and consider the space $H^0(X, \Omega)$ of holomorphic $1$-forms $\omega$ (i.e. sectons of the canonical bundle $\Omega$). Define a map $I : X \to H^0(X, \Omega)^\vee$ via,
\[ I(z) = \left( \omega \to \int_{x_0}^x \omega \right) \]
However, this map is not well-defined up to the choice of paths. Choose a basis $\chi_i$ of $2g$ homology cycles of $X$ and quotient the codomain by the subgroup $H$ generated by,
\[ I_\chi = \left( \omega \to \int_\chi \omega \right) \]
We know that any two paths $\gamma_1$ and $\gamma_2$ with equal end-points are homologous up to such cycles in the sense that,
\[ \int_{\gamma_1} \omega - \int_{\gamma_2} \omega = \sum_{i = 1}^{2g} C_i \int_{\chi_i} \omega \]
which implies that,
\[ \left( \omega \to \int_{\gamma_1} \omega \right) - \left( \omega \to \int_{\gamma_2} \omega \right) \in H \]
so the map $I : X \to H^0(X, \Omega)^\vee / H$ is well-defined. Denote $\Jac{X} = H^0(X, \Omega)^\vee / H$ which naturally has the structure of an abelian variety.
\bigskip\\
By Riemann-Roch $H^0(X, \Omega)$ is a $g$-dimensional complex vectorspace. And therefore, we may choose a basis and view $I$ as a map $I : X \to \C^g / \Lambda_g$ where $\Lambda_g$ is the lattice $\Z^{2g}$ generated by the period matrix. Thus $\Jac{X} = \C^g / \Lambda_g$.

\subsection{The Case of Elliptic Curves}

Let $X$ be a complex elliptic curve $\C / \Lambda$ so $g = 1$. For any point $z \in \C$ consider a family of integrals,
\[ \left\{ \int_0^{z + \lambda} \d{\zeta} \: \middle| \lambda \in \Lambda \right\} = z + \Lambda \] 
ech such integral can be viewed as a path between lattice points $0$ to $\lambda$ and differ by the integration over any path between lattice points. Thus there is an isomorphism of algebraic groups,
\[ X \cong \left\{ \text{path integrals } \int_{0 + \Lambda}^{z + \Lambda} \d{\zeta} \right\} \Bigg/ \left\{ \text{loop integrals } \int_\alpha \d{\zeta} \right\} \]
where addition is given by,
\[ \int_{0 + \Lambda}^{z_1 + \Lambda} \d{\zeta} + \int_{0 + \Lambda}^{z_2 + \Lambda} \d{\zeta} = \int_{0 + \Lambda}^{z_1 + \Lambda} \d{\zeta} + \int_{z_1 + \Lambda}^{z_1 + z_2  + \Lambda} \d{\zeta} = \int_{0 + \Lambda}^{z_1 + z_2 + \Lambda} \]

\subsection{Defining the Jacobian}

\newcommand{\Homover}[3]{\mathrm{Hom}_{#1}(#2, #3)}

Now let $X$ be a compact Riemann surface of genus $g \ge 1$ with homology cycles $A_1, \dots, A_g$ and $B_1, \dots B_g$ along longitudinal and latitudinal loops of the handles respectivly. Since all holomorphic $1$-forms on a Riemann surface are closed, homologous paths give the same integrals. Therefore, for any loops $\alpha_1, \dots, \alpha_N$ and $\ell_1, \dots, \ell_N \in \Z$ there is a unique list of integers $(m,n) \in \Z^{2g}$ such that,
\[ \sum_{i = 1}^N \ell_i \int_{\alpha_i} \omega = \sum_{i = 1}^g m_i \int_{A_i} \omega + \sum_{i = 1}^g n_i \int_{B_i} \omega \quad \quad \quad \omega \in \Omega = \Omega^1_{\text{hol}}(X) \]
Therefore, we may present the first homology group of $X$ as,
\[ H_1(X, \Z) = \Z \int_{A_1} \oplus \cdots \oplus \Z \int_{A_g} \oplus \cdots \oplus \Z \int_{B_g} \cong \Z^{2g} \]
where we view this homology as dual to de Rham cohomology i.e. the homology group is maps taking holomorphic differentials on $X$ to complex numbers which send exact forms to zero (i.e. are integrals along closed loops which thus have no boundary) modulo maps factoring through exterior derivative followed by area integration (which in this case are all exactly zero since holomorphic $1$-forms are closed).
\bigskip\\
Therefore we may view the homology as a subspace of the dual space of holomorphic $1$-forms,
\[ H_1(X, \Z) \subset \Omega^1_{\text{hol}}(X)^\wedge = \Homover{\C}{\Omega^1_{\text{hol}}(X)}{\C} \]
where,
\[ \Omega^1_{\text{hol}}(X)^\wedge = \R \int_{A_1} \oplus \cdots \oplus \R \int_{A_g} \oplus \cdots \oplus \R \int_{B_g} \cong \R^{2g} \]
Furthermore, Riemann-Roch gives that $\dim_{\C}(\Omega^1_{\text{hol}}(X)) = g$ and thus $H_1(X, \Z)$ is a lattice in the dual space which is complex analytically some lattice $\Lambda_g \subset \C^g$. This motivates the following definition.
\begin{definition}
The \textit{Jacobian Variety} of $X$ is the quotient abelian variety,
\[  \Jac{X} = \Omega^1_{\text{hol}}(X)^\wedge / H_1(X, \Z) = \C^g / \Lambda_g \]
where $\Lambda_g$ is the lattice in $\C^g$ defined by the periods.
\end{definition}

\newcommand{\Div}{\mathrm{Div}}
\newcommand{\Pic}{\mathrm{Pic}}
\renewcommand{\div}{\mathrm{div}}

\begin{definition}
Recall,
\[ \Div(X) = \bigoplus_{x \in X} \Z = \left\{ f : X \to \Z \mid f \text{ has finite support} \right\} = \left\{ \sum_{x \in X} n_x [x] \: \middle| \: n_x \in \Z \text{ and } n_x \neq 0 \text{ for finitely many } x \right\} \]
Since the sums are finite there is a map $\deg : \Div{X} \to \Z$ given by the identity on each factor of the coproduct,
\[ \deg{\left( \sum_{x \in X} n_x [x] \right)} = \sum_{x \in X} n_x \]
Now define,
\[ \Div^0(X) = \ker{\deg} \]
Recall that each function $f \in \C(X)$ in the function field has an associated divisor,
\[ \div{(f)} = \sum_{x \in X} \ord_x(f) \: [x] \]
Then the principal divisors are defined as the divisors of all rational finctions,
\[ \Div^{\ell}(X) = \{ \div{(f)} \mid f \in \C(X) \} \]
Finally, we define the picard group,
\[ \Pic(X) = \Div(X) / \Div^\ell(X) \]
Now $\deg{\div{(f)}} = 0$ meaning that $\deg : \Div^\ell(X) \to \Z$ is the zero map. Thus $\deg : \Div(X) \to \Z$ factors through the quotient $\Pic(X)$. Its kernel is $\Pic^0(X)$ or equivalently,
\[ \Pic^0(X) = \Div^0(X) / \Div^{\ell}(X) \]
\end{definition}

\begin{theorem}
Let $X$ have positive genus and choose a base point $x_0 \in X$. Then the map $X \to \Pic^0(X)$ given by,
\[ x \mapsto [x] - [x_0] \]
is an embedding. 
\end{theorem}

\begin{proof}
If $g > 0$ we must show that there do not exits rational functions $f \in \C(X)$ such that $\div{(f)} = [x] - [y]$ for two points $x \neq y$. 
\end{proof}

\begin{theorem}[Abel]
The map $\Div^0(X) \to \Jac{X}$ given by,
\[ \sum_{x \in X} n_x [x] \mapsto \sum_{x \in X} n_x \int_{x_0}^x \]
descends to an isomorphism $\Pic^0(X) \to \Jac{X}$. 
\end{theorem}

\begin{corollary}
If $X$ has positive genus, the map $X \to \Pic^0(X) \to \Jac{X}$ is an embedding via,
\[ x \mapsto \int_{x_0}^x \]
\end{corollary}

\begin{corollary}
\[ \Omega^1_{\text{hol}}(X)^\wedge = \left\{ \sum_{\gamma} n_\gamma \int_\gamma \: \middle| \: n_\gamma \in \Z \text{ and } \sum_{\gamma} n_\gamma = 0 \right\} \]
\end{corollary}

\begin{definition}
The Jacobian of the modular curve $X_0(N)$ is denoted $J_0(N) = \Jac{X_0(N)}$.
\end{definition}

\begin{theorem}[Modularity]
Let $E$ be a complex elliptic curve with $j(E) \in \Q$. Then for some $N \in \Z$ there exists a surjective holomorphic homomorphism of complex torii,
\[ J_0(N) \to E \]
\end{theorem}

\subsection{Maps Between Jacobians}

\begin{remark}
In this section, let $h : X \to Y$ be a nonconstant holomorphic map of compact Riemann surfaces.
\end{remark}

\begin{definition}
The pullback $h^* : \C(Y) \to \C(X)$ is simply the map $h^*9g) = g \circ h$. Then the order of vanishing satisfies,
\[ v_x(h^*g) = e_x v_{h(x)}(g) \]
where $e_x$ is the ramification index of $h$. Thus if $g$ is holomorphic then $h^* g$ is also holomorphic. We may extend the pullback to holomorphic differentials
\[ h^* : \Omega^1_{\text{hol}}(Y) \to \Omega^1_{\text{hol}}(X) \]
via $(h^* \omega)(v) = \omega(\d{h}(v))$. This pullback dualizes to a map,
\[ h_* : \Omega^1_{\text{hol}}(X)^\wedge \to \Omega^1_{\text{hol}}(Y)^\wedge \]
\end{definition}

\begin{lemma}
For any path $\gamma$ in $X$ and holomorphic $1$-form $\omega \in \Omega^1_{\text{hol}}(Y)$ on $Y$ we have,
\[ \int_\gamma h^* \omega = \int_{h \circ \lambda} \omega \]
\end{lemma}

\begin{corollary}
If $\alpha$ is a loop in $X$ and $\varphi \in \Omega^1_{\text{hol}}(X)^\wedge$ is $\int_\alpha$ then we can compute $h_* \varphi \in \Omega^1_{\text{hol}}(Y)^\wedge$ as,
\[ h_* \varphi = \int_\alpha h^* = \int_{h(\alpha)} \]
Since $h \circ \alpha$ is a loop in $Y$ this shows that $h_*$ takes homology to homology i.e. $h_* : H_1(X, \Z) \to H_1(Y, \Z)$. Therefore, $h_*$ descends to a map of the Jacobians,
\[ h_* : \Jac{X} \to \Jac{Y} \] 
\end{corollary}

\begin{definition}
The forward map on Jacobians is,
\[ h_J : \Jac{X} \to \Jac{Y} \quad \quad \quad h_J [\varphi] = [h_* \varphi] = [ \varphi \circ h^* ] \]
In particular,
\[ h_J \left( \sum_{x \in X} n_x \int_{x_0}^x \right) = \sum_{x \in X} n_x \int_{h(x_0)}^{h(x)} \]
\end{definition}

\begin{example}
Let $X = \C / \Lambda_X$ and $Y = \C / \Lambda_Y$ be elliptic curves. Then any holomorphic $h : X \to Y$ is some $z \mapsto mz + b$. Thus, the forward map on Jacobians becomes,
\[ h_J \left( \sum_{x \in X} n_x \int_{0 + \Lambda}^{x + \Lambda} \right) = \sum_{x \in X} n_x \int_{b + \Lambda}^{m x + b + \Lambda} = \sum_{x \in X} n_x \int_{0 + \Lambda}^{m x + \Lambda} \]
Therefore any nonconstant holomorphic map is sent to an isogeny $X \to Y$ viewed as their associated Jacobians. In particular if $h$ is an isogeny i.e. $b = 0$ then $h_J = h$. 
\end{example}

\newcommand{\Nm}{\mathrm{Nm}}

\begin{definition}
There is a norm map associated to $h : X \to Y$ given by,
\[ \Nm_h : \C(X) \to \C(Y) \quad \quad [\Nm_h(f)](y) = \prod_{h(x) = y} f(x)^{e_x} \]
\end{definition}

\begin{lemma}
For any $f \in \C(X)^\times$ then,
\[ v_y(\Nm_h(f)) = \sum_{h(x) = y} v_x(f) \]
and thus,
\[ \div{\Nm_h(f)} = \sum_{y \in Y} \left( \sum_{h(x) = y} v_x(f) \right) [y] = \sum_{x \in X} v_x(f) [h(x)] \]
\end{lemma}

\begin{definition}
At the level of principal divisors, the norm maps,
\[ \div{(f)} = \sum_{x \in X} v_x(f) [x] \mapsto \div{\Nm_h(f)} =  \sum_{x \in X} v_x(f) [h(x)] \]
We extend this to a homomorphism of divisors via,
\[ h_D : \Div(X) \to \Div(Y) \quad  \quad h_D\left( \sum_{x \in X} n_x \: [x] \right) = \sum_{x \in X} n_x \: h(x) \]
Clearly, $\deg_Y \circ h_D = \deg_X$ because,
\[ \deg_Y{h_D\left( \sum_{x \in X} n_x [x] \right)} = \deg_Y{\left( \sum_{x \in X} n_x [h(x)] \right)} = \sum_{x \in X} n_x = \deg_X{\left( \sum_{x \in X} n_x [x] \right)} \]
and by definition, $h_D(\div{(f)}) = \div{\Nm_h(f)}$. Thus this map descends to,
\[ h_D : \Div^0(X) \to \Div^0(Y) \]
and then to,
\[ h_P : \Pic^0(X) \to \Pic^0(Y) \]
via,
\[ h_P \left[ \sum_{x \in X} n_x [x] \right] = \left[ \sum_{x \in X} n_x [h(x)] \right] \]
\end{definition}

\begin{theorem}
The induced maps $h_P : \Pic^0(X) \to \Pic^0(Y)$ and $h_J : \Jac{X} \to \Jac{Y}$ are equivalent under the isomorphism $\Pic^0(X) \to \Jac{X}$ of Abel's theorem. That is, the following diagram commutes,
\begin{center}
\begin{tikzcd}
\Pic^0(X) \arrow[r, "h_P"] \arrow[d] & \Pic^0(Y) \arrow[d]
\\
\Jac{X} \arrow[r, "h_J"] & \Jac{Y} 
\end{tikzcd}
\end{center}
\end{theorem}

\begin{lemma}
Let $\mathcal{E}(X) = \{ x \in X \mid e_x > 1 \}$ be the set of exceptional points of $X$ under $h$. Then consider $Y' = Y \setminus h(X)$ and $X' = h^{-1}(Y)$. Since we have removed all ramified points then $h$ is a $d$-sheeted covering map where $d = \deg{h}$. 
\end{lemma}

\renewcommand{\tr}{\mathrm{tr}}

\begin{definition}
The trace map on holomorphic differentials,
\[ \tr_h : \Omega^1_{\text{hol}}(X) \to \Omega^1_{\text{hol}}(Y) \]
is defined as follows. Since $h : X' \to Y'$ is a covering map, about any $y \in Y'$ we can find $h_i : U_i \to \tilde{U}$ open with $y \in \tilde{U}$ such that $h_i$ is an isomorphism. Locally,
\[ (\tr_h \omega)|_{\tilde{U}} = \sum_{ i = 1 }^d (h_i^{-1})^*(\omega|_{U_i}) \]
If $\delta$ is a path in $Y'$ lifting to $\gamma$ in $X'$ and $\delta(0) = y$ and $\gamma(0) \in U_i$ then $h_i^{-1}$ has an analytic continuation along $\delta$ giving a corresponding local inverse about $\delta(1)$ sending it to $\gamma(1)$. Performing this continuation arround a loop will permute the $d$ different local inverses which preserves the trace. Thus, such an analytic continuation constructs a global trace. The trace extends holomorphically to $Y$. (PROVE THIS)
\bigskip\\
Dualizing, we get a map,
\[ \tr_h^\wedge : \Omega^1_{\text{hol}}(Y)^\wedge \to \Omega^1_{\text{hol}}(X)^\wedge \quad \quad \quad \tr_h^\wedge \psi = \psi \circ \tr_h \]
\end{definition}

\begin{lemma}
For any path $\delta$ in $Y'$ we have,
\[ \int_\delta (h^{-1})^* \omega = \int_{h^{-1} \circ \delta} \omega \quad \quad \quad \omega \in \Omega^1_{\text{hol}}(X) \]
where $h^{-1}$ is some local inverse of $h$ at $\delta(0)$ analytically extended along $\delta$ making $h^{-1} \circ \delta$ a lift of $\delta$. Therefore,
\[ \int_\delta \tr_h \omega = \sum_{h \circ \gamma = \delta} \int_\gamma \omega \quad \quad \quad \omega \in \Omega^1_{\text{hol}}(X)\]
In particular, if $\beta$ is a loop in $Y'$ and $\psi \in \Omega^1_{\text{hol}}(Y)^\wedge$ is $\int_\beta$ then,
\[ \tr_h^\wedge \psi = \int_\beta \tr_h = \sum_{h \circ \gamma = \delta} \int_\gamma \]
Since $\beta$ lifts to a concatenation of loops in $X$ then $\tr_h^\wedge$ takes homology to homology and thus descends to the Jacobian.
\end{lemma}

\begin{definition}
The reverse map of Jacobians is the holomorphic homomorphism induced by the holomorphic trace,
\[ h^J : \Jac{X} \to \Jac{Y} \quad \quad \quad h^J[\psi] = [\psi \circ \tr_h] \]
In particular, applying Abel's theorem to write elements of the Jacobian as sums of integratons,
\[ h^J \left( \sum_{y \in Y} n_y \int_{y_0}^y \right) = \sum_{y \in Y} n_y \sum_{h(x) = y} e_x \int_{x_0}^x \]
\end{definition}

\begin{example}
Suppose $X = \C / \Lambda$ and $Y = \C / \Lambda'$ are elliptic curves and $h : X \to Y$ is a nonconstant holomorphic map which must be of the form $h(z + \Lambda) = mz + b + \Lambda'$. This map is unramified as has degree $\deg{(h)} = [m^{-1} \Lambda' : \Lambda]$. For $w \in \C$ then,
\[ h^{-1}(w + \Lambda') = \{ m^{-1}(w - b) + t + \Lambda \mid t \in m^{-1} \Lambda' / \Lambda \} \]
Then,
\[ h^j \left( \int_{0 + \Lambda'}^{w + \Lambda'} \right) = \sum_{t} \int_{- m^{-1} b + \Lambda}^{m^{-1}(w - b) + t + \Lambda} = \sum_t \int_{0 + \Lambda}^{m^{-1} w + t + \Lambda} = \int_{0 + \Lambda}^{\sum_t m^{-1} w + t + \Lambda} \]
Under the isomorphisms $X \xrightarrow{\sim} \Jac{X}$ this is the map,
\[ h^J(w + \Lambda') = \sum_{t} m^{-1} w + t + \Lambda \]
If we then apply the forward map, we find,
\[ (h_J \circ h^J)(w + \Lambda') = [m^{-1} \Lambda^{-1} : \Lambda] (w + \Lambda') \]
meaning that $h_J \circ h^J = [\deg{(h)}]$ so $h^J$ is the dual isogeny to $h_J$. 
\end{example}

\begin{proposition}
$h_J \circ h^J = [\deg{(h)}]$
\end{proposition}

\begin{proof}
For any nonconstant holomorphic map $h : X \to Y$,
\[ (\tr_h \circ h^*)(\omega) = \deg{(h)} \omega \quad \quad \quad \omega \in \Omega^1_{\text{hol}}(Y) \]
because there are $d$ lifts summed over in the trace.
Therefore, dualizing gives,
\[ h_* \circ \tr_h^\wedge = \deg{(h)} \id \]
and therefore,
\[ h_J \circ h^J = [\deg{(h)}] \]
\end{proof}

\begin{proposition}
Let $h : X \to Y$ be a nonconstant holomorphic map of Riemann surfaces. Let $h_* = (h_*)^\wedge|_{H_1(X, \Z)}$ be the induced forward homomorphism of homology,
\[ h_* : H_1(X, \Z) \to H_1(Y, \Z) \quad \quad \quad h_* \left( \sum_\alpha n_\alpha \int_\alpha \right) = \sum_\alpha n_\alpha \int_{h \circ \alpha} \]
Then $h_*(H_1(X, \Z))$ is a subgroup of finite index in $H_1(Y, \Z)$. If $V$ is a subspace of $\Omega^1_{\text{hol}}(Y)$ then the restriction $h_*(H_1(X,\Z))|_V$ is a subgroup of finite index in $H_1(Y, \Z)|_B$. 
\end{proposition}

\begin{proof}
We have $\Im{h_*} \supset \Im{h_* \circ \tr_h^\wedge}$ and thus,
\[ h_*(H_1(X, \Z)) \supset ((h^*)^\wedge \circ \tr_h^\wedge)(H_1(Y, \Z)) = (tr_h \circ h^*)^\wedge(H_1(Y, \Z)) = \deg{(h)} H_1(Y, \Z) \]
Therefore, $h_*(H_1(X, \Z))$ contains a subgroup of index $d = \deg{(h)}$ in $H_1(Y, \Z)$ and thus is finite index. Therefore, $H_1(Y, \Z) / h_*(H_1(X, \Z))$ is a quotent of \[ H_1(Y, \Z) / \deg{(h)} H_1(Y, \Z) = (\Z / d \Z)^{2g} \]
If $V$ is a subspace of $\Omega^1_{\text{hol}}(Y)$ then the map $H_1(Y, \Z) \to H_1(Y, \Z)|_V$ is a surjection. This implies that,
\[ [ H_1(Y, \Z)|_V :  h_*(H_1(X, \Z))|_V ] \le   [H_1(Y,\Z) : h_*(H_1(X, \Z))] \]
\end{proof}

\begin{remark}
The pullback $h^* : \C(Y) \to \C(X)$ transfers function backwards and acts on divisors via,
\[ \div(h^* g) = \sum_{x \in X} e_x v_{h(x)}(g) [x] = \sum_{y \in X} v_y(g) \sum_{h(x) = y} e_x [x] \]
We can extend this map to a reverse map on divisors.
\end{remark}

\begin{definition}
Define the reverse map via,
\[ h^D : \Div(Y) \to \Div(X) \quad \quad \quad h^D \left( \sum_{y \in Y} n_y [y] \right) =\sum_{y \in Y} n_y \sum_{h(x) = y} e_x [x] \]
We have $\deg_X \circ h^D = \deg{(h)} \deg_Y$ because,
\[ \deg_X{h^D \left( \sum_{y \in Y} n_y [y] \right)} = \deg_X{\left( \sum_{y \in Y} n_y \sum_{h(x) = y} e_x [x] \right)} = \sum_{y \in X} n_y \sum_{h(x) = y} e_x = \deg{(h)} \sum_{y \in Y} n_y = \deg{(h)} \deg_Y{\left( \sum_{y \in Y} n_y [y] \right)} \]
Furthermore, by definition, $h^D(\div{(g)}) = \div{(h^* g)}$. Therefore, $h^D$ descends to give maps,
\[ h^D : \Div^0(Y) \to \Div^0(X) \]
and
\[ h^P : \Pic^0(Y) \to \Pic^0(X) \]
via,
\[ h^P\left[ \sum_{y \in Y} n_y [y] \right] = \left[ \sum_{y \in Y} n_y \sum_{h(x) = y} e_x [x] \right] \]
\end{definition}

\begin{remark}
As with the forward maps, the reverse maps we have defined are compatible with isomorphisms $X \xrightarrow{\sim} \Jac{X}$ of Abel's theorem.
\end{remark}

\begin{theorem}
The induced maps $h^P : \Pic^0(P) \to \Pic^0(X)$ and $h^J : \Jac{Y} \to \Jac{X}$ are equivalent under the isomorphisms $\Pic^0(X) \to \Jac{X}$ of Abel's theorem. That is, the following diagram commutes,
\begin{center}
\begin{tikzcd}
\Pic^0(Y) \arrow[r, "h^P"] \arrow[d] & \Pic^0(X) \arrow[d]
\\
\Jac{Y} \arrow[r, "h^J"] & \Jac{X}
\end{tikzcd}
\end{center}
\end{theorem}

\begin{proposition}
$h_D \circ h^D = [\deg{(h)}]$ and therefore $h_J \circ h^J = [\deg{(h)}]$ formally from the above commutativity,
\end{proposition}

\begin{proof}
Consider,
\begin{align*}
h_D \circ h^D \left( \sum_{y \in Y} n_y [y] \right) & = h_D \left( \sum_{y \in Y} n_y \sum_{h(x) = y} e_x [x] \right) = \sum_{y \in Y} n_y \sum_{h(x) = y} e_x [h(x)]
\\
& = \left( \sum_{h(x) = y} e_x \right) \sum_{y \in Y} n_y [h(x)] = \deg{(h)} \sum_{y \in Y} n_y [y]
\end{align*}
\end{proof}

\begin{remark}
Recall the earlier version of the Modularity theorem: let $E$ be an elliptic curve such that $j(E) \in \Q$ then there is a holomorphic surjective map $X_0(N) \to E$ for some $N$.
\end{remark}

\begin{theorem}
The two versions of the Modularity Theorem are equivalent.
\end{theorem}

\begin{proof}
Suppose there exists a holomorphic surjection $h : X_0(N) \to E$ then consider the induced map on Jacobian varieties, $h_J : \Jac{X_0(N)} \to \Jac{E}$. Because $E$ is an elliptic curve we have naturally $\Jac{E} \cong E$ and $h_J$ is surjective because $h_J \circ h^J = [\deg{(h)}]$ which is surjective. Thus $h_J : J_0(N) \to E$ gives the necessary surjective holomorphic homomorphism.
\bigskip\\
Conversely, if $J_0(N) \to E$ is a holomorphic surjective map then consider the embedding $X_0(N) \to J_0(N)$ and the composition $X_0(N) \to J_0(N) \to E$. This map is surjective because it is a nonconstant holomorphic map between compact complex manifolds. Thus the composition gives the necessary holomorphic surjection.
\end{proof}

\subsection{Modular Jacobians and Hecke Operators}

\begin{remark}
In this section we aim to extend the Hecke operators to acting on modular Jacobians. 
\end{remark}

\begin{remark}
Recall that for any congruence subgroup $\Gamma$ there is an isomorphism,
\[ \omega : \S{2}(\Gamma) \to \Omega^1_{\text{hol}}(X(\Gamma)) \]
This map dualizes to give an isomorphism,
\[ \S{2}(\gamma)^\wedge = \omega^\wedge(\Omega^1_{\text{hol}}(X(\Gamma))^\wedge) \]
which immiedatly gives the following theorem.
\end{remark}

\begin{theorem}
Let $\Gamma$ be a congruence subgroup. Then there is a natural identification,
\[ J(\Gamma) = \Jac{X(\Gamma)} = \S{2}(\gamma)^\wedge / \omega^\wedge(H_1(X, \Z)) \]
\end{theorem}

\begin{lemma}
Let $h : X \to Y$ be the map $h(\Gamma_X  \tau) = \Gamma_Y \alpha(\tau)$. 
Pullback is performed on the weight-2 cusp forms via the weight-2 operator via the following commutative diagram,
\begin{center}
\begin{tikzcd}
\S{2}(Y) \arrow[r, "\alpha_2"] \arrow[d,"\omega_Y"] & \S{2}(X) \arrow[d, "\omega_X"]
\\
\Omega^1_{\text{hol}}(Y) \arrow[r, "h^*"] & \Omega^1_{\text{hol}}(X)
\end{tikzcd}
\end{center}
and thus the map $h_* : \S{2}(X) \to \S{2}(Y)$ acts via $h_* \varphi = \varphi \circ [\alpha]_2$. 
\end{lemma}

\end{document}
