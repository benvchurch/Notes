\documentclass[12pt]{extarticle}
\usepackage[utf8]{inputenc}
\usepackage[english]{babel}
\usepackage[a4paper, total={6in, 9in}]{geometry}
\usepackage{tikz-cd}
 
\usepackage{amsthm, amssymb, amsmath, centernot}
\usepackage{mathrsfs} 

\newcommand{\notimplies}{%
  \mathrel{{\ooalign{\hidewidth$\not\phantom{=}$\hidewidth\cr$\implies$}}}}
 
\renewcommand\qedsymbol{$\square$}
\newcommand{\cont}{$\boxtimes$}
\newcommand{\divides}{\mid}
\newcommand{\ndivides}{\centernot \mid}
\newcommand{\Z}{\mathbb{Z}}
\newcommand{\R}{\mathbb{R}}
\newcommand{\C}{\mathbb{C}}
\newcommand{\N}{\mathbb{N}}
\newcommand{\Zplus}{\mathbb{Z}^{+}}
\newcommand{\Primes}{\mathbb{P}}
\newcommand{\colim}[1]{\mathrm{colim}(#1)}
\newcommand{\Ob}[1]{\mathrm{Ob}(#1)}
\newcommand{\cat}[1]{\mathcal{#1}}
\newcommand{\id}{\mathrm{id}}
\newcommand{\Hom}[2]{\mathrm{Hom}\left( #1, #2 \right)}
\newcommand{\catHom}[3]{\mathrm{Hom}_{#1}\left( #2, #3 \right)}
\newcommand{\Top}{\mathbf{Top}}
\newcommand{\pTop}{\mathbf{Top}_{\bullet}}
\newcommand{\Set}{\mathbf{Set}}
\newcommand{\pSet}{\mathbf{Set}_\bullet}
\newcommand{\hTop}{\mathbf{hTop}}
\newcommand{\phTop}{\mathbf{hTop}_{\bullet}}
\renewcommand{\Im}[1]{\mathrm{Im}(#1)}
\newcommand{\homspace}[2]{\left< #1, #2 \right>}
\newcommand{\rp}{\mathbb{RP}}
\newcommand{\coker}[1]{\mathrm{coker}\: #1}

\theoremstyle{definition}
\newtheorem{theorem}{Theorem}[section]
\newtheorem{lemma}[theorem]{Lemma}
\newtheorem{proposition}[theorem]{Proposition}
\newtheorem{example}[theorem]{Example}
\newtheorem{corollary}[theorem]{Corollary}
\newtheorem{remark}{Remark}

\newenvironment{definition}[1][Definition:]{\begin{trivlist}
\item[\hskip \labelsep {\bfseries #1}]}{\end{trivlist}}


\newenvironment{lproof}{\begin{proof} \renewcommand{\qedsymbol}{}}{\end{proof}}
\renewcommand{\mod}[3]{\: #1 \equiv #2 \: mod \: #3 \:}
\newcommand{\nmod}[3]{\: #1 \centernot \equiv #2 \: mod \: #3 \:}
\newcommand{\ndiv}{\hspace{-4pt}\not \divides \hspace{2pt}}
\newcommand{\gen}[1]{\langle #1 \rangle}
\newcommand{\hook}{\hookrightarrow}
\newcommand{\Tor}[4]{\mathrm{Tor}^{#1}_{#2} \left( #3, #4 \right)}
\newcommand{\Ext}[4]{\mathrm{Ext}^{#1}_{#2} \left( #3, #4 \right)}

\tikzset{
    labl/.style={anchor=south, rotate=90, inner sep=.5mm}
}

\renewcommand{\bf}[1]{\mathbf{#1}}
\newcommand{\res}{\mathrm{res}}
\newcommand{\F}{\mathcal{F}}
\newcommand{\G}{\mathcal{G}}
\renewcommand{\O}{\mathcal{O}}
\renewcommand{\d}[1]{\mathrm{d} #1}
\newcommand{\deriv}[2]{\frac{\d{#1}}{\d{#2}}}
\newcommand{\Aut}[1]{\mathrm{Aut}\left( #1 \right)}
\newcommand{\End}[1]{\mathrm{End}\left( #1 \right)}


\begin{document}

\section{Introduction}

\section{Topological Quantum Field Theory}

\begin{definition}
A topological quantum field theory $Q$ is a tensorial functor $Q$ from the category of cobordisms between closed manifolds to an algebraic category such as the category of complete normed vector spaces. 
\end{definition} 

\begin{definition}
A cobordism between oriented closed smooth $n-1$-manifolds $M, N$ is an $n$-manifold $S$ such that $\partial S = M \sqcup N$.   
\end{definition}

\begin{remark}
Cobordisms form a symmetric monoidal category with tensor $M \otimes N = M \sqcup N$ and unit given by the empty manifold.
\end{remark}

\begin{example}
For TQFT in $n = 1$ is determined by its values on points (with $\pm$ orientation) because it is a tensor functor. We assign
\end{example}

\newcommand{\Cob}{\mathbf{Cob}}
\renewcommand{\Vec}{\mathbf{Vec}}

\begin{example}
For TQFT in $n = 2$ we must consider a moniodal functor $F : \Cob_2 \to \Vec_k$. First, we get a vectorspace $A = F(S^1)$ and then by the moniodal structure,
\[ F(S^1 \sqcup \cdot \sqcup S^1) = F(S^1)^{\otimes 1} = A^{\otimes n} \]
Furthermor, $F(\varnothing) = k$ and thus the cap cobordism gives $\iota : k \to A$ and the cap cobordism gives a map $\varepsilon : A \to k$. Furthermore the pants coborhism gives a map $m : A^{\otimes 2} \to A$ and the coprants coborhism gives a map $\Delta : A \to A^{\otimes 2}$ (giving the structure of a Hopf algebra). Composing two pairs of pants and using the isomorphism given by moving the bottom leg shows that $m \circ (\id \otimes m) = m \circ (m \otimes \id)$ as maps $A^{\otimes 3} \to A$ giving the associative laws. Furthermore, the cup composed with pants (which is the same as a tube i.e. the indentity gives) $m \circ (\iota \circ \id) = m \circ (\id \circ \iota) = \id$ as maps $A \to A$. Furthermore the map $s$ which takes $x \otimes y \mapsto y \otimes x$ given by the exchange coborhism gives the following $m \circ s = m$ so $m$ is symmetric. Thus $m$ is associative and commutative and unital. 
\bigskip\\
Furthermore since composing tubes gives the identity the form $\varepsilon \circ m : A^{\otimes 2} \to A$ must be a symmetric nondegenerate form on $A$. Also since pants can be turned to copants by tubes which swap input and output correspodning to $m^* : A^* \to (A^*)^{\otimes 2}$ followed by the isomorphism $\varepsilon : A \to A^*$.
\end{example}

\begin{definition}
A finite dimensional $k$-algebra $A$ is Frobenius if it has a nondegenerate trace $\varepsilon : A \to k$. This is equivalent to an isomorphism $A \to A^*$ as left $A$-modules. 
\end{definition}

\begin{remark}
\[ \text{Frobenius Alg} \supset \text{symmetric algebras} \supset \text{commutative Frobenius Alg} \]
\end{remark}

\begin{theorem}
There is a one-to-one correspondence between 2D TQFTs over $k$ and commutative Frobenious algebras over $k$ both up to isomorphism. 
\end{theorem}

\begin{example}
Let $M$ be a $2m$-dimensional manifold and $A = H^*(M, k)$ be its cohomology ring. Then $A$ is a supercommutative algebra and we can take,
\[ H^{\text{even}}(M, k) = \bigoplus_{i = 0}^m H^{2i}(M, k) \]
which is a commutative Frobenius algebra with $\varepsilon : H^{2m}(M, k) \xrightarrow{\sim} k$ generated by the class $[M]$.  
\end{example}

\end{document}