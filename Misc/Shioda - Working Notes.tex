\documentclass[12pt]{article}
\usepackage{hyperref}
\hypersetup{
    colorlinks=true,
    linkcolor=blue,
    filecolor=magenta,      
    urlcolor=blue,
}

\usepackage{import}
\import{"../Algebraic Geometry/"}{AlgGeoCommands}

\newcommand{\Loc}[1]{\mathfrak{Loc}\left( #1 \right)}
\newcommand{\AbGrp}{\mathbf{AbGrp}}

\renewcommand{\tr}{\operatorname{tr}}

\newcommand{\LL}{\mathbb{L}}
\newcommand{\ob}{\mathrm{ob}}
\newcommand{\cM}{\mathcal{M}}
\newcommand{\cT}{\mathcal{T}}
\newcommand{\vir}{\mathrm{vir}}
\newcommand{\cN}{\mathcal{N}}
\newcommand{\cO}{\mathcal{O}}
\newcommand{\V}{\mathbb{V}}

\DeclareMathOperator{\Exc}{\mathrm{Exc}}


\begin{document}


\section{Semple Jets}


\begin{defn}
A \textit{directed variety} $(X, \E)$ is a pair of a variety $X$ with a subbundle $\E \subset \T_X$. A morphism of directed varities $f : (X, \E) \to (Y, \E')$ is a morphism $f : X \to Y$ such that under $f_* \T_X \to \T_Y$ we have $f_* \E \to \E'$.
\end{defn}

\begin{rmk}
Demailly's philosophy is that it is usefull to study this ``relative notion'' even for the absolute case $\E = \T_X$ since it has better functoriality properties.
\end{rmk}

\begin{rmk}
Here our convention is that $\P(\E) := \rProj{X}{\Sym{}{\E^\vee}}$ so that $\cO(-1)$ is the universal subbundle. Hence $\cO(1)$ on $\P(\T_X)$ is what I usually call $\cO(1)$ on $\P(\Omega_X)$.
\end{rmk}

\begin{defn}
To a directed pair $(X, \E)$ we introduce the \textit{projectivization} to produce a new pair $\P(X, \E) := (\wt{X}, \wt{\E})$ where $\wt{X} := \P(\E)$ and $\wt{\E}$ is defined via the diagram,
\begin{center}
\begin{tikzcd}[row sep = small, column sep = large]
0 \arrow[r] & \T_{\wt{X}/X} \arrow[dd] \arrow[r] & \wt{\E} \arrow[dd] \pullback \arrow[r] & \cO(-1) \arrow[d, hook] \arrow[r] & 0
\\
& & & \pi^* \E \arrow[d, hook]
\\
0 \arrow[r] & \T_{\wt{X}/X} \arrow[r] & \T_{\wt{X}} \arrow[r] & \pi^* \T_{X} \arrow[r] & 0
\end{tikzcd}
\end{center}
Then we have,
\[ \dim{\wt{X}} = \dim{X} + \rank{\E} - 1 \quad \quad \rank{\wt{\E}} = \rank{\E} \] 
\end{defn}

\begin{rmk}
Note that the Euler exact sequence takes the form,
\begin{center}
\begin{tikzcd}
0 \arrow[r] & \cO \arrow[r] & \pi^* \E \ot \cO(1) \arrow[r] & \T_{\wt{X}/X} \arrow[r] & 0
\end{tikzcd}
\end{center}
\end{rmk}

\begin{prop}
Given a morphism of directed varities $f : (X, \E) \to (Y, \F)$ we get a rational map $\wt{f} : (\wt{X}, \wt{\E}) \rat (\wt{Y}, \wt{\F})$ such that the diagram,
\begin{center}
\begin{tikzcd}
(\wt{X}, \wt{\E}) \arrow[d,"\pi"] \arrow[r, "\wt{f}", dashed] & (\wt{Y}, \wt{\F}) \arrow[d, "\pi"]
\\
(X, \E) \arrow[r, "f"] & (Y, \F)
\end{tikzcd}
\end{center} 
commutes in the category of directed manifolds (with rational maps). Moreover, if $f$ is ``immersive along $\E$'', meaning $f_{\#} : \E \to f^* \F$ is injective, then $\wt{f}$ is a morphism.
\end{prop}

\begin{defn}
Let $(X, V)$ be a directed manifold. The \textit{projectivized Semple $k$-jet bundle} $P_k V = X_k$ is defined iteratively via,
\[ (X_0, V_0) := (X, V) \quad \quad (X_{k+1}, V_{k+1}) := (\wt{X_k}, \wt{V_k}) \]
and we have,
\[ \dim{P_k V} = \dim{X} + k (\rank{V} - 1) \quad \quad \rank{V_k} = \rank{V} \]
\end{defn}

\begin{rmk}
We can alternatively think of the Semple construction in the dual sense,
\begin{center}
\begin{tikzcd}[row sep = small, column sep = large]
0 \arrow[r] & \pi^* \Omega_X \arrow[r] \arrow[d, two heads] & \Omega_{\wt{X}} \arrow[dd] \arrow[r] & \Omega_{\wt{X}/X} \arrow[r] \arrow[dd, equals] & 0
\\
& \pi^* \E^\vee \arrow[d, two heads]
\\
0 \arrow[r] & \struct{\wt{X}}(1) \arrow[r] & \wt{\E}^\vee \arrow[r] & \Omega_{\wt{X}/X} \arrow[r] & 0
\end{tikzcd}
\end{center}
This will be our standard perspective although we retain the dual notation to remain in agreement wit hthe complex geometry literature. Now the Euler sequence
\begin{center}
\begin{tikzcd}
0 \arrow[r] & \Omega_{\wt{X}/X} \arrow[r] & \pi^* \E^\vee \ot \struct{\wt{X}}(-1) \arrow[r] & \struct{\wt{X}} \arrow[r] & 0
\end{tikzcd}
\end{center}
gives $\pi_* \nSym{d}{\Omega_{\wt{X}/X}} = 0$ and $R^1 \pi_* \Omega_{\wt{X}/X} = \struct{X}$. Furthermore, applying Sym to the botom row gives,
\begin{center}
\begin{tikzcd}
0 \arrow[r] & \nSym{d-1}{\wt{\E}^\vee} \ot \cO_{\wt{X}}(1) \arrow[r] & \nSym{d}{\wt{\E}^\vee} \arrow[r] & \nSym{d}{\Omega_{\wt{X}/X}} \arrow[r] & 0
\end{tikzcd}
\end{center}
so applying $\pi_*$ gives,
\[ \pi_* \nSym{d}{\wt{\E}^\vee} = \pi_* [ \nSym{d-1}{\wt{\E}} \ot \cO_{\wt{X}}(1)] \]
\end{rmk}

\begin{example}
For the directed manifold $(X, \T_X)$ we set $P_k = X_k$ and set $\cP^{k,d} = \pi_{k*} \cO_{P_k}(d)$. Notice that there are exact sequence,

{\color{red} DO THIS}
\end{example}

The semple tower is defined so that the following holds. Suppose that $f : C \to X$ is an immersed curve such that $\d{f} : \T_C \to f^* \T_X$ factors through $f^* \E \subset f^* \T_X$. Since $\d{f}$ is a subbundle this gives a subbundle $\T_X \embed \pi^* \E$ and hence a lift $f' : C \to \wt{X}$ such $\d{f} : \T_C \to f^* \E \to f^* \T_X$ is $f'^* [\struct{\wt{X}}(-1) \to \pi^* \E \to \pi^* \T_X]$. Therefore, consider $\d{f'} : \T_C \to f'^* \T_{\wt{X}}$. Since this map lifts $\d{f}$ we see that $\d{f'} : \T_X \to f'^* \wt{\E}$.

Hence, if we start with an immersed curve $f : C \to X$ then there are lifts $f_k : C \to P_k$ for all $k$.

\subsection{Arc Spaces and Hasse-Schmidt Derivations}

\begin{defn}
Let $X$ be an $S$-scheme. Then $\ell^{\text{th}}$-order \textit{arc} of $X$ is a $S$-morphism $\Delta^\ell_S \to X$ where 
\[ \Delta^{\ell}_S = \rSpec{S}{\struct{S}[t]/(t^{\ell+1})} = S \times_{\ZZ} \Spec{\ZZ[t]/(t^{\ell + 1})} \]
If it exists, the $\ell^{\text{th}}$-order arc space is $J_\ell(X) = \Hom{S}{\Delta^\ell_S}{X}$ which represents the functor,
\[ T \mapsto \Hom{T}{\Delta_\ell \times_k T}{X_T} \]
\end{defn}

When $X$ is a $k$-scheme we let $S = \Spec{k}$ and let $\Delta^\ell = \Spec{k[t]/(t^{\ell + 1})}$ without adornment.

\newcommand{\HSDer}[4]{\mathrm{Der}^{#2}_{#1}\left(#3, #4\right)}
\newcommand{\HS}{\mathrm{HS}}

\begin{defn}
Let $R$ be a ring and $A, B$ be $R$-algebras. Then the group of $m^{\text{th}}$-order \textit{Hasse-Schmidt derivations} $\HSDer{R}{m}{A}{B}$ is the group of sequences $(D_0, D_1, \dots, D_m)$ of $R$-linear maps $D_i : A \to B$ such that,
\[ D_k(xy) = \sum_{p + q = k} D_p(x) D_q(y) \]
for all $k \le m$ and $x,y \in A$.
\end{defn}

\begin{prop}
For any $R$-algebra $A$ we have,
\[ \Hom{R}{\Delta^m_R}{A} = \Hom{R}{A}{R[t]/(t^{m+1})} = \HSDer{R}{m}{A}{R} \]
\end{prop}

\begin{proof}
The correspondence sends $\varphi : A \to R[t]/(t^{m+1})$ writen as,
\[ \varphi(x) = \sum_{i = 0}^m \varphi_i(x) t^i \]
to the HS derivation $(\varphi_0, \varphi_1, \dots, \varphi_m)$.
\end{proof}

\begin{prop}
Let $A$ be an $R$-algebra. Then there exists an $A$-algebra $\HS_{A/R}^m$ equipped with a universal HS-derivation $D : A \to \HS_{A/R}^m$ representing $\HSDer{R}{m}{A}{-}$ menaing,
\[ \Hom{A}{\HS_{A/R}^m}{B} = \HSDer{R}{m}{A}{B} \]
functorially in $R$-algebras $B$. Furthermore, this has an explicit presentation,
\[ \HS_{A/R}^m = A [ \d_i{x} ]_{x \in A, 0 \le i \le m} / \left< \d_i(x+y) = \d_i{x} + \d_i{y} \, \d_i{r} = 0 \, \d_i(xy) = \sum_{p + q = i} \d_p(x) \d_q(y) \right>_{r \in R} \]
Clearly, $\HS_{A/R}^m$ is graded by $A$-modules $\HS_{A/R}^{m,d}$ where we put $\d_i{x}$ in degree $i$ and the degree $k$ part consists of sums of monomials of total degree $k$.
\end{prop}

\begin{rmk}
The map $D_0 : A \to \HS^m_{A/R}$ makes $\HS^m_{A/R}$ into an $A$-algebra. Furthermore, if $B$ is an $A$-algebra then $\Hom{A}{\HS^m_{A/R}}{B} \subset \Hom{R}{\HS^m_{A/R}}{B}$ is identified with the sub $\HSDer{R}{m}{A}{B}_0 \subset \HSDer{R}{m}{A}{B}$ of HS-derivations $\varphi$ with $\varphi_0 : A \to B$ equal to the structure map. It is clear that representing $\HSDer{R}{m,B}{A}{-}_0$ on the category of $A$-algebras uniquely determines $\HS^m_{A/R}$ with its $A$-algebra structure and universal HS-derivation whose zeroth term agrees with the structure map.   
\end{rmk}

\begin{prop}
Let $f : X \to S$ be an $S$-scheme. Then these glue together to give a sheaf $\HS_{X/S}^m$ representing,
\[ \Hom{f^{-1} \struct{S}}{\HS_{X/S}^m}{\cA} = \HSDer{f^{-1} \struct{S}}{m}{\struct{X}}{\cA} \]
where $\cA$ is any sheaf of $\struct{X}$-algebras.
\end{prop}

\begin{lemma}
If $A \to B$ is a map of $R$-algebras then there is an exact sequence,
\begin{center}
\begin{tikzcd}
\HS_{A/R}^m \ot_A B \arrow[r] & \HS_{B/R}^m \arrow[r] & \HS^m_{B/A} \arrow[r] & 0
\end{tikzcd}
\end{center}
\end{lemma}

\begin{proof}
Surjectivity is immediate from the presentation. Thus we need to show that the kernel is generated by $\HS_{A/R}^m$. To show this, it suffices to show that,
\[ 0 \to \Hom{B}{\HS_{B/A}^m}{C} \to \Hom{B}{\HS_{B/R}^m}{C} \to \Hom{B}{\HS_{A/R}^m \ot_A B}{C} \] 
is exact for any $C$. But this is exactly,
\[ 0 \to \HSDer{A}{m}{B}{C}_0 \to \HSDer{R}{m}{B}{C}_0 \to \HSDer{R}{m}{A}{C}_0 \]
and the kernel is exactly those HS-derivations which vanish on the image of $A$ and hence correspond exactly to $A$-linear derivations by definition.
\end{proof}

\begin{lemma}
If $A \to B$ is an \etale map of $R$-algebras then $\HS_{A/R} \ot_A B \to \HS_{B/R}$ is an isomorphism.
\end{lemma}

\begin{proof}
By localizing we can assume that $A \to B$ is standard \etale meaning $B = A[x]_g/(f(x))$ where $f'(x)$ is a unit. From the exact sequence, it suffices to show injectivity and $\HS_{B/A}^m = 0$. Indeed, $f(x) = 0$ so $\d_i{(f(x))} = 0$ but $\d_1(f(x)) = f'(x) \d{x}$ so $\d_1{x} = 0$ since $f'(x)$ is a unit. Now assume that $\d_i(x) = 0$ for $i < k$ we will show that $\d_{k}(x) = 0$. First compute,
\[ \d_k(x^n) = n x^{n-1} \d_k(x) \]
because $\d_k(x^m) = \d_k(x^{m-1}) x + x^{m-1} \d_k(x)$ since the intermediate terms are zero so the claim is true by induction. Therefore, we see that $\d_k(f(x)) = f'(x) \d_k(x)$ but $f'(x)$ is a unit and thus $\d_k{x} = 0$ so we win. Now to show injectivity we need to show that if $C$ is a $B$-algebra then the map
\[ \HSDer{R}{m}{B}{C}_0 \to \HSDer{R}{m}{A}{C}_0 \]
is surjective. Given $\varphi : A \to C$ it suffices to specify $\varphi'(x)$ such that it becomes a HS-derivation. Because $f'(x) \d_k(x) = p$ for $p$ a polynomial $\d_i(x)$ for $i < k$ and $\d_i(a)$ for $a \in A$ we can specify $\varphi'_k(x) = - \varphi'_{<k}(p) \cdot \varphi_0(f'(x))^{-1}$ where $\varphi_0 : B \to C$ is the struture map and $f'(x)$ is a unit so this makes sense. Then it is elementary to check this defines a HS-derivation.	 
\end{proof}

\begin{prop}
If $X/S$ is locally of finite type then $\HS_{X/S}^m$ is graded by coherent $\struct{X}$-algebra. It is graded by vector bundles if $X/S$ is smooth.
\end{prop}

\begin{proof}
This immediately reduces to the corresponding property for $\HS_{A/R}$. If $R[x_1, \dots, x_n] \onto A$ then we claim that the natural map $\HS_{R[x_1, \dots, x_n]/R} \to \HS_{A/R}$ is surjective then the finite generation is obvious from examining the structure of the Hasse-Schmidt algebra of a polynomial ring. For smoothness we use the \etale-local structure to reduce to the polynomial ring. Furthermore,
\[ \HS_{R[x_1, \dots, x_n]/R}^{m,d} = \bigoplus R \, \d_{i_1}(x_{j_1}) \cdots \d_{i_r}(x_{j_r}) \]
where we sum over all monomials $\d_{i_1}(x_{j_1}) \cdots \d_{i_r}(x_{j_r})$ such that $i_1 + \cdots + i_r = k$ and $i_{\ell} \le m$. 
\end{proof}

\begin{example}
$\HS^0_{A/R} = A$ and $\HS^1_{A/R} = \Sym{R}{A}$.
\end{example}

{\color{red} DO I NEED SMOOTHNESS FOR THE FILTRATION??}

\begin{prop}
There are exact sequences,
\end{prop}


\subsection{Jets a la Jason Starr}

\newcommand{\pr}{\mathrm{pr}}

\begin{theorem}[FGA IV.3 p.267]
Let $p : Y \to X$ be flat and projective and $q : Z \to Y$ finitely-presented quasi-projective morphism then the functor,
\[ T \to \{ (f : T \to X, g : T \times_X Y \to Z) \mid q \circ g = \pr_2 \} \]
(i.e. to each $X$-scheme $T$ a $Y$-morphism $T \times_X Y \to Z$)
is representable by a universal pair,
\[ (r : \Pi_{Z/Y/X} \to X, \, s : \Pi_{Z/Y/X} \times_X Y \to Z) \]
\end{theorem}

\begin{rmk}
In the case $Z = W \times_X Y$ we just get the Hom scheme $\Hom{X}{Y}{W}$. Furthermore, if $q : Z \to Y$ is a bundle then this represents the functor of sections of $q$ because  the functor can be identified with, an $X$-scheme $T \to X$ and a morphism $g : T \times_X Y \to T \times_X Z$ such that,
\begin{center}
\begin{tikzcd}
& T \times_X Z \arrow[dd, "\id \times q"]
\\
T \times_X Y \arrow[ru, "g"] \arrow[rd, equals] & 
\\
& T \times_X Y
\end{tikzcd}
\end{center}  
\end{rmk}

Let $S$ be a scheme and $f : X \to S$ be a smooth separated morphism and let $\Delta_{X/S} : X \to X \times_S X$ be the relative diagonal which is a closed embedding defined by an ideal sheaf $\I$. Let $\Delta_e : X_e \embed X \times_S X$ be the closed embedding corresponding to $\I^{e+1}$. The associated projections $\pr_i : X_e \to X$ are finite flat (hence proper). 

\begin{defn}
Let $\pi : Z \to X$ be finitely presented and quasi-projective then so is the base change,
\[ B \times_{X, \pr_1} (X \times_X X) \to X \times_S X \]
thus the pullback $\pi_e : B_e \to X_e$ over $\Delta_e$ is also finitely presented and quasi-projective. Then the ``relative jets'' parameter space is the universal pair,
\[ (r : \Pi_{B_e / X_e / X} \to X, \, s : \Pi_{B_e/X_e/X} \times_{X, \pr_2} X_e \to B_e) \]
representing the functor, defined via $\pr_2 : X_e \to X$,
\[ f : T \to S, \,\, g : T \times_X X_e \to B_e \quad \text{such that} \quad \pi_e \circ g = \pr_2 \]  
\end{defn}

\begin{rmk}
We think of $\pi : B \to X$ as a bundle and $J^e(\pi) := \Pi_{B_e/X_e/X}$ is then the bundle of jets of sections of $\pi$. Note, a map $T \times_{X, \pr_2} X_e \to B_e$ over $X_e$ is the same as a map $T \times_{X, \pr_2} X_e \to B$ over $X$ (where we view the $X$-structure of $T \times_{X, \pr_2} X_e$ through $\pr_1$ on $X_e$) since $B_e = B \times_{X, \pr_1} X_e$. Consider the case, $B = Z \times_S X$ where $Z$ is an $S$-scheme. This case $\Pi_{B_e/X_e/X}$ is the space of jets of morphisms $f : X \to Z$. Indeed, in this case, $B_e = Z \times_S X_e$ and hence a $X_e$-morphism $g : T \times_X X_e \to B_e$ is just as $S$-morphism $T \times_X X_e \to Z$.
\end{rmk}

\begin{rmk}
Associated to the space of jets $\Pi = J^e(\pi)$ and a point $x : S \to X$ we get the space of jets at the point is $\Pi_x := \Pi \times_X S$.
\end{rmk}

\begin{example}
For $X = \A^1_S$ then $X_e = X \times \Delta^e$ where,
\[ \Delta^e = \Spec{\ZZ[t]/(t^{e+1})} \]
and we take $B = \A^1_S \times_S Z$ then we get,
\[ \Pi := \Pi_{B_e/X_e/X} = \A^1_S \times J_e(Z) \]
since it represents, as an $\A^1_S$-scheme, morphisms $T \times \Delta^e \to Z$ over $S$. Therefore, $\Pi_0 = J_e(Z)$ is the arc scheme in the usual sense.
\end{example}

\newcommand{\VV}{\mathbb{V}}

\begin{example}
Conversely, suppose that $Z = \A^1_S$ so we consider jets of maps $X \to \A^1_S$. Then,
\[ \{ T \to \Pi \} = \{ (T \to X, T \times_X X_e \to \A^1_S) \} = \{ (f : T \to X, s \in \Gamma(T \times_X X_e)) \} \] 
However, $\pi_2 : X_e \to X$ is affine corresponding to the algebra $\pr_{2*} (\struct{X \times_S X} / \I^{e+1}) = J^e(X)$ and hence $\Gamma(T \times_X X_e) = \Gamma(T, f^* J^e(X))$ since cohomology along an affine map commutes with base change. Therefore,
\[ \Gamma(T \times_X X_e) = \Gamma(T, f^* J^e(X)) = \Hom{\struct{T}\text{-alg}}{f^* \Sym{\bullet}{J^e(X)^\vee}}{\struct{T}} = \Hom{X}{T}{\VV_X(J^e(X))} \]
\end{example}

\begin{rmk}
To any section $s : X \to B$ of $\pi : B \to X$ we get a corresponding $X$-point of $J^e(\pi)$ (i.e. a section of the bundle of jets corresponding to the $e^{\text{th}}$-jet of $s$). Indeed, consider $X_e \to B_e = B \times_{X, \pr_1} X_e$ defined by $(s \circ \pr_1, \id)$ which is an $X_e$-morphism. However, to a $T$-point $s : T \to B$   of $B$ (which we can think of a section of $B \times_X T \to T$) we cannot associate a $T$-point of $J^e(\pi)$ meaning a morphism $T \times_{X, \pr_2} X_e \to B_e$ over $X_e$ because to write $s \times \id$ we need that the projections to $X$ commute with $\id : X \to X$ which they do not since these are $\pi_1, \pi_2 : X_e \to X$. This shows that the map $\{ \text{sections of } \pi : B \to X \} \to \{ \text{sections of } \pi : \Pi \to X \}$ is nonlinear. For the case $T = X$ the fact we use is that $X \times_{X, \pr_1} X_e \cong X \times_{X, \pr_2} X_e$ over $X_e$. In general, an isomorphism $T \times_{X, \pr_1} X_e \cong T \times_{X, \pi_2} X_e$ over $X_e$ is a sort of higher-order connection on $T$ over $X$.
\end{rmk}

\begin{rmk}
Notice that in the definition of $B_e$ we use $\pi_1$ while in the definition of the functor we form $T \times_X X_e$ through $\pi_2$. This is essential to get the jets of nontrivial bundles correct. It is analogous to how in the definition: $J^e(\E) := \pr_{2*} \pr_1^* \E$ for the projections $\pr_i : X_e \to X$ it is essential we use the two different projections. This means that the diagram,
\begin{center}
\begin{tikzcd}
B_e \arrow[d] \arrow[r, "\pi_e"] & X_e \arrow[d, "\pr_1"]
\\
B \arrow[r, "\pi"] & X
\end{tikzcd}
\end{center}
commutes for $\pr_1$ but \textit{not} for $\pr_2$ while we use $\pr_2$ for the construction of $T \times_X X_e$. 
\end{rmk}

\begin{example}
Let $\pi : B \to X$ be a vector bundle $\VV_X(\E) \to X$. A morphism $T \times_{X, \pr_2} X_e \to B$ over $X$ (through $\pr_1 : X_e \to X$) given $f : T \to X$ corresponds to a morphism of algebras,
\[ \pr_1^* \Sym{\bullet}{\E^\vee} \to \struct{T \times_{X, \pr_2} X_e} \]
and hence a section,
\[ s \in \Gamma(T \times_{X, \pr_2} X_e, \pr_1^* \E) = \Gamma(T, f^* \pr_{2*} \pr_1^* \E) = \Hom{X}{T}{\VV_X(J^e(\E))} \]
where we used that $\pr_2 : X_e \to X$ is affine so pushforward commutes with any base change. 
\end{example}

HOW TO MAKE THE ARCS TANGENT TO SOMETHING?

THE DERIVATIVE OPERATOR ON GG-JETS

REPARAMETRIZATION OF ARCS

\subsection{Semple Jets are Invariant Hasse-Schmidt Jets}

\newcommand{\aff}{\mathrm{aff}}

Construction: given a vector bundle $\E$ on $X$ note that $\struct{\V(\E)}$ is canonically identified with the graded ring
\[ \nSym{\bullet}{\E^\vee} = \bigoplus_{n \ge 0} \struct{\P(\E)}(n) \]
via the $\Gm$-equivariant rational map
\begin{center}
\begin{tikzcd}
\V(\E) \arrow[rd] \arrow[rr, dashed] & & \P(\E) \arrow[ld]
\\
& X
\end{tikzcd}
\end{center}
whose indeterminancy locus is in codimension $\rank{\E}$ and therefore functions extend over all of $\V(\E)$ by Harthogs' theorem (note that the case $\rank{\E} = 1$ is trivial for other reasons). Suppose we have a pair $(X, \E)$ where $\E$ is a vector bundle equipped with a map $\E \to \T_X$ (not assumed to be injective) and we construct the Semple tower $(X_k, \E_k)$. We can interpret this construction in terms of ``physical'' vector bundles as well. On $\V(\E)$ there is a map $\struct{\V(\E)}(-1) \to \pi^* \E$ of $\Gm$-equivariant coherent sheaves on $\V(\E)$ (or equivalently of graded $\cA_\E := \nSym{\bullet}{\E^\vee}$-modules where the $(-1)$ corresponds to the grading) given by the canonical cocontraction map
\[ \nSym{n-1}{\E^\vee} \to \nSym{n}{\E^\vee} \ot \E \] 
\[ s_1 \cdots s_{n-1} \mapsto \sum_{i = 1}^r s_1 \cdots s_{n-1} e_i \ot e^i \]
where $e_i$ is a local basis of $\E^\vee$ and $e^i$ is the dual basis. Therefore, setting $\wt{X}^{\aff} = \V(\E)$ we can create a diagram
\begin{center}
\begin{tikzcd}[row sep = small]
0 \arrow[r] & \T_{\wt{X}^\aff /X} \arrow[dd, equals] \arrow[r] & \wt{\E}^\aff \arrow[dd] \pullback \arrow[r] & \struct{\wt{X}^{\aff}}(-1) \arrow[d] \arrow[r] & 0
\\
& & & \pi^* \E \arrow[d]
\\
0 \arrow[r] & \T_{\wt{X}^{\text{aff}}/X} \arrow[r] & \T_{\wt{X}^\aff} \arrow[r] & \pi^* \T_{X} \arrow[r] & 0
\end{tikzcd}
\end{center}
the only difference to the projective case being that the downward maps are now not injective over the zero section. We now iterate this construction to produce a tower of directed affine bundles along with $\Gm$-equivariant maps to the ordinary Semple tower,
\begin{center}
\begin{tikzcd}
\vdots \arrow[d] & \vdots \arrow[d]
\\
(X^{\aff}_2, \E_2^\aff) \arrow[d] \arrow[r, "\varphi_2", dashed] & (X_2, \E_2) \arrow[d]
\\
(X^{\aff}_1, \E_1^\aff) \arrow[d] \arrow[r, "\varphi_1", dashed] & (X_1, \E_1) \arrow[d]
\\
(X, \E) \arrow[r, equals] & (X, \E)
\end{tikzcd}
\end{center}
Now the claim is that the $\Gm$-equivariant maps induce canonical injections of $\struct{X}$-algebras
\[ \cP^{k, \bullet} = \bigoplus_{d \ge 0} \pi_{k*} \struct{X_k}(d)\embed \pi_{k*} \struct{X_k^{\aff}}  \]
Indeed, consider the diagram
\begin{center}
\begin{tikzcd}[column sep={4em,between origins},row sep=1em]
& 0 \arrow[rr] & & \varphi_k^* \T_{X_k / X_{k-1}} \arrow[rr] \arrow[dd, equals] & & \varphi_k^* \E_k \arrow[rr] \arrow[dd] & & \varphi_k^* \struct{X_k}(-1) \arrow[rr] \arrow[dd] & & 0
\\
0 \arrow[rr] & & \T_{X_k^\aff / X_{k-1}^\aff}|_{U_k} \arrow[rr, crossing over] \arrow[ru] & & \E^\aff_k |_{U_k} \arrow[ru] \arrow[ru] \arrow[rr, crossing over] & & \struct{X_k^\aff}(-1)|_{U_k} \arrow[ru, equals] \arrow[rr, crossing over] & & 0
\\
& 0 \arrow[rr] & & \varphi^*_k \T_{X_k/X_{k-1}} \arrow[rr] & & \varphi^*_k \T_{X_k} \arrow[rr] & & \varphi^*_k \pi^* \T_{X_{k-1}} \arrow[rr] & & 0
\\
0 \arrow[rr] & & \T_{X_k^\aff / X_{k-1}^\aff} |_{U_k} \arrow[rr] \arrow[from=uu, equals, crossing over] \arrow[ur] & & \T_{X_k^\aff} |_{U_k} \arrow[from=uu, crossing over] \arrow[ur] \arrow[rr] & & \pi^* \T_{X_{k-1}^\aff}|_{U_k} \arrow[from=uu, crossing over] \arrow[rr] \arrow[ur] & & 0
\end{tikzcd}
\end{center}
Thus, given the map $\varphi_k : (X_k^\aff, \E_k^\aff) \rat (X_k, \E_k)$ we can build $\varphi_{k+1} : (X_{k+1}^\aff, \E_{k+1}^\aff) \rat (X_{k+1}, \E_{k+1})$. 
\bigskip\\
Indeed, given a $\Gm$-equivariant rational map $f : X \rat Y$ and $\Gm$-equivariant vector bundles $\E_X$ and $\E_Y$ and a $\Gm$-equivariant morphism of vector bundles $\varphi : \E_X|_U \embed f^* \E_Y$ then we produce a $\Gm$-equivariant rational map $f' : \V(\E_X) \rat \P(\E_Y)$ which is defined on $U' = \pi^{-1}(U) \sm V(\varphi)$ where $V(\varphi)$ is the locus
\[ V(\varphi) = \{ x \in U \mid v \in \ker{\varphi_x} \} \] 
The map $f' : U' \to \P(\E_Y)$ is defined by 
\[ \struct{\V(\E_X)}(-1)|_{U'} \to \pi^* \E_X|_{U'} \xrightarrow{\pi^* \varphi} f^* \E_Y \]
which is a subbundle over $U'$ because over $U'$ the composite is fiberwise injective.
\bigskip\\
In the case of the Semple tower, $\E_0^{\aff} = \E_0$ with rank $r$ and then $\rank{\E_k} = 1 + \rank{\T_{X_k/X_{k-1}}} = \rank{\E_{k-1}}$ and $\rank{\E_k^{\aff}} = 1 + \rank{\T_{X_k^\aff / X_{k-1}^\aff}} = 1 + \rank{\E_{k-1}^{\aff}}$ so $\rank{\E_k} = k + r$. Now $U_1 = X_1 \sm V(0)$ has codimension $r$. Furthermore, $\varphi_k \E_k^{\aff}|_{U_k} \onto \varphi^*_k \E_k$ is surjective with kernel of rank $k$ inside $\E_k^{\aff}$ which has rank $r + k$ so $V(\varphi_k)$ has codimension $r$. Therefore, we can build the morphisms in the Semple tower and each $\varphi_k$ is naturally defined away from codimension $r$. Since $r \ge 2$ sections extend and therefore there is an injective pullback map,
\[ \cP^{k, \bullet} = \bigoplus_{d \ge 0} \pi_{k*} \struct{X_k}(d)\embed \pi_{k*} \struct{X_k^{\aff}}  \]

\begin{defn}
Consider the projectivized Semple tower $(X_m, \E_m)$ where $\E_0 = \T_X$ then the \textit{projectivied Semple $m$-jet space} is defined as $P_k \E = X$ and the \textit{projectivied Semple $m$-jet bundle} is defined as $\cP^{m,d}_X = \pi_{m*} \struct{X_m}(d)$. Likewise, consider the affine Semple tower $(X_m^\aff, \E_m^\aff)$ where $\E_0 = \T_X$. Then the \textit{affine Semple $m$-jet space} is defined $J_m X = X_m^{\aff}$ and the \textit{affine Semple $m$-jet bundle} is $\E^{m,d} = [\pi_{m*} \struct{X^\aff_m}]_d$ where we take the degree $d$ part induced by the $\Gm$-action.
\end{defn}

\begin{prop}
Let $\cP^{m,d}_X = \pi_{m*} \struct{X_m}(d)$ where $(X_m, \E_m)$ is the projectivied Semple $m$-jet bundle $P_k \E = X_k$ with $\E_0 = \T_X$. Then there is a canonical doubly graded injection,
\[ \cP^{m,d} \embed \HS^{m,d}_{X} \] 
of $\struct{X}$-algebras.
\end{prop}

\begin{proof}
To illustrate, for $m = 0$ we set,
\[ \cP^{0,d} = \HS^{0,d}_{X} = \begin{cases}
\struct{X} & d = 0
\\
0 & d > 0
\end{cases} \]
Now for $m = 1$ there are canonical isomorphisms,
\[ \cP^{1,d} = \nSym{d}{\Omega_X} = \HS^{1,d}_{X} \]
To prove the claim, it suffices for each quasi-coherent $\struct{X}$-algebra $\cA$ to produce a functorial degree-preserving surjection
\[ \Hom{\struct{X}}{\HS_{X/S}^m}{\cA} \onto \Hom{\struct{X}}{\cP^{m,\bullet}}{\cA} \]
Note that
\[ \Hom{\struct{X}}{\HS_{X/S}^m}{\cA} = \Hom{S}{\Delta_{\cA}^m}{X}_0 \]
where $\Delta^m_{\cA} = \rSpec{X}{\cA[t]/(t^{m+1})}$ and the zero denotes that we are only considering maps compatible with the structure map $\rSpec{X}{\cA} \to X$. Given $f : \Delta^m_{\cA} \to X$ there is a differential
\[ \d{f} : f^* \Omega_X \to \Omega_{\Delta^m_{\cA}/\cA} = [\struct{\Delta_{\cA}^m} \d{t}]/((m+1) t^m \d{t}) \to \struct{\Delta^{m-1}_{\cA}} \]
where the last map takes $\d{t} \mapsto 1$ which is well-defined since $t^m \d{t} \mapsto t^m = 0$. This produces a morphism $f' : \Delta^{m-1}_{\cA} \to \V(\T_X) = \wt{X}^{\aff}$ lifting $f$. Note that if $\d{f}$ factors through $f^* \Omega_X \to f^* \E^\vee$ then the induced map $f'$ satisfies
\[ \d{f'} : f'^* \Omega_{\wt{X}^{\aff}} \to \Omega_{\Delta^{m-1}_{\cA}/\cA} \]
factors through $f'^* \Omega_{\wt{X}^{\aff}} \to \wt{\E}^\vee$ because, by definition, the following diagram commutes
\begin{center}
\begin{tikzcd}
f^* \Omega_X \arrow[r] \arrow[d] & f'^* \struct{\wt{X}^\aff}(1) \arrow[d] \arrow[rdd, bend left]
\\
f'^* \Omega_{\wt{X}^{\aff}} \arrow[rrd, bend right] \arrow[r, "\d{f'}"] & \wt{\E}^\aff \arrow[rd, dashed]
\\
& & \struct{\Delta^{m-1}_{\cA}}
\end{tikzcd}
\end{center}
Iterating this process produces a map $\rSpec{X}{\cA} \to X_m^{\aff}$ lifting $\rSpec{X}{\cA} \to X$. The pullback map of sections then gives the required map of algebras 
\[ \cP^{m, \bullet} \embed \pi_{k*} \struct{X_m^{\aff}} \to \cA \]
It suffices to prove that the obtained map
\[ \Hom{\struct{X}}{\HS_{X/S}^m}{\cA} \onto \Hom{\struct{X}}{\cP^{m,\bullet}}{\cA} \]
is surjective and graded. It is graded because everything so constructed is $\Gm$-equivariant for the obvious $\Gm$-action on $\Delta_{\cA}^m$ which corresponds to the grading on $\HS_{X/S}^m$. To check surjectivity, since $X$ is smooth, using the \etale-local structure, we reduce to cheking this property for $\A^n_S$. In this case we can directly compute. There is a presentation
\[ \HS_{X/S}^m = \struct{S}[\d_i(x_j)]_{\substack{0 \le i \le m \\ 0 \le j \le n}} \]
We now consider the map $\varphi_{ij} : \HS^m_{X/S} \to \struct{S}$ sending $\d_i(x_j) \mapsto 1$ and all other to zero. This corresponds to the Hasse-Schmidt differential $(D_0, \dots, D_m)$ where $D_j(x_i) = 1$ and $D_{j'}(x_i') = 0$ for all other $i' \neq i$ and $j' \neq j$. Now we consider the lift of the map
\[ \varphi_{ij} : \Delta^m_{S} \to X \]
to the Semple tower. We construct
\[ X_1 = \rSpec{S}{\struct{S}[x_1, \dots, x_n][\d{x_1}, \dots, \d{x_n}]} \]
and then $\pi^* \Omega_X \to \struct{\V(\T_X)}(1)$ is given by $\d{x_1} \mapsto$ 
\[ X_2 = \rSpec{S}{\struct{S}[x_1, \dots, x_n][\d{x_1}, \dots, \d{x_n}][\d_2{x_1}, \dots, \d_2{x_n}][s]} \]
\end{proof}


\section{Foliation-obstruction setup}

\newcommand{\Ram}{\mathrm{Ram}}

Let $X$ be a smooth proper scheme over a noetherian scheme $S$. Let $s : \Spec{k} \embed S$ be a closed point and $C / k$ a smooth proper curve. Consider a separable map $f : C \to X$ (eg an immersion). Consider $P := \P(\Omega_{X/S}^1)$ then by base change $\P(\Omega_{X_s//k}) = P \times_S k$ to which by separability there is a lift $f' : C \to \P(\Omega_{X_s/k})$.

\begin{proof}
Let $L := \struct{P_s}(1)$ be the tautological bundle on $P$, and $\chi_C = 2 - 2 g(C)$ the geometric Euler characteristic of the curve. Then,
\[ L \cdot_{f'} C = - \chi_C - \Ram_f \le - \chi_C \]
\end{proof}

\begin{proof}
This is immediate by definition: the map $\pi^* \Omega_X \to L$ pulls back to $f^* \Omega_X \onto \L \subset \Omega_C$ hence 
\[ L \cdot_{f'} C = \deg{\L} = \deg{\Omega_C} - \Ram_f \] 
\end{proof}

Now suppose that $f : C \to X$ is invariant by a foliation $\F$ by curves. Outside of the singular locus $Z$ the foliation defines a section of $\P(\Omega_{X/S}) \to X$ whose closure over $Z$ is a blowup  $\pi : \wt{X} \to X$ with center supported in $Z$. Therefore, there is an exceptional divisor $E$ and $L|_{\wt{X}} = \pi^* K_{\F}(-E)$ and provided that $f$ doesn't factor through $Z$, the intersection $E \cdot_{f'} C$ can be identified with the Segre class $s_Z(f)$ so that,
\[ K_{\F} \cdot_{f} C = - \chi_C - \Ram_f + s_Z(f) \]

The main result we need is in controlling this Segre class $s_Z(f)$.


The issue with $J^k_X$ is that if $X$ has no symmetric forms then it has no $J^k_X$ sections. The same is not true for $E^GG$ or $P$ the Green-Griffiths or Semple jets.
\bigskip\\
Desiderata: logarithmic GG or Semple jets and an extension result for them. 

\section{Strategy of RR14}

Let $\X \to X$ be the coarse space of a smooth stack such that $X$ has only ADE singularities. 

\subsection{Orbifold Riemann-Roch}

By the standard computations,
\[ \chi(\X, P^{2,m}_\X) = \frac{m^3}{6} (13 c_1^2 -  9 c_2) + O(m^2) \]
{\color{red} DO THIS!! GET NUMERICS CORRECT}

\subsection{Vanishing Theorems}

We need Bogomolov's vanishing theorem for smooth stacks. If $\X$ is general type then,
\[ H^0(\X, S^m \T_{\X} \ot K_{\X}^p) = 0 \]
whenever $m > 2 p$. {\color{red} RR claim this is because of an orbifold Kahler-Einstein metric and Bochner identities? Does Bogomolov's Original Proof Work ALSO?}

{\color{red} WE REALLY NEED THE FOLLOWING THEOREM OF BOGOMOLOV}

\begin{theorem}[Bogomolov]
Let $X$ be a surface of general type and $i_1 + \cdots + i_k > 2q$ integers then,
\[ H^2(X, S^{i_1} \Omega_X \ot \cdots \ot S^{i_k} \Omega_X \ot K_X^{\ot -q}) = 0 \]
\end{theorem}

\begin{prop}
Let $X$ be a surface of general type then $H^2(\X, E^{k,m}_{\X} \ot K_{\X}^{\ot -q}) = 0$ for $m > 2k$ and $\floor{m/k} > 2 q$ with $k \ge 1$.
\end{prop}


\begin{proof}
We use induction and the fundamental filtration,
\[ E^{k-1,m} = F^0 E^{k,m} \subsetneq F^1 E^{k,m} \subsetneq \cdots \subsetneq F^{\floor{k/m}} E^{k,m} = E^{k,m} \]  
where the quotients are given by,
\[ F^p / F^{p-1} \cong S^p \Omega_X \ot E^{k-1, m-kp} \]
Therefore, $E^{k,m}$ has a composition series by terms,
\[ S^{i_1} \Omega_X \ot \cdots \ot S^{i_k} \Omega_X \]
where each $(i_1, \dots, i_k)$ such that $i_1 + 2 i_2 + \cdots + k i_k = m$ appears exactly once. Therefore, because $\dim{X} = 2$ by Bogomolov's vanishing condition we conclude. 
\end{proof}

\subsection{Log Jets and Extension (PROBLEM)}

We need to construct a log jet bundle $P_X^{k,m}(\log{D})$ and we need the following result,
\[ H^0(Y \sm E, P_X^{k,m}) = H^0(Y, P_X^{k,m}(\log{D})) \]
This is the jet version of the extension theorem of Miyaoka.

\subsection{The Proof (PROBLEM)}

Let $g_k(X)$ be the leading term in the asymtotic Riemann-Roch computation of $P^{k, m}_X$. {\color{red} For $k = 2$ IT IS SOMETHING LIKE $g_k(X) = \frac{1}{12}(13 c_1^2 - 9 c_2)$ OR SOMETHING}

\begin{theorem}
Suppose that $g_k(Y) + g_k(\X) > 0$ then,
\[ h^0(Y, P^{k,m}_X) \ge \frac{g_k(Y) + g_k(\X)}{2} m^3 + O(m^2) \]
in particular there are enough $k$-jets.
\end{theorem}

\begin{proof}
Consider the exact sequences,
\[ 0 \to P^{k,m}_Y \to P^{k,m}_Y(\log{D}) \to Q_m \to 0 \]
where $Q_m$ is supported on the exceptional divisor $E$. Since the singularities are ADE, the minimal resolution is crepant so there exists a neighborhood $U$ of $E$ such that the canonical bundle is tirival. Therefore, we get the sequence
\[ 0 \to P^{k,m}_Y \ot K_Y^{\ot (1 - m)} \to P^{k,m}_Y(\log{D}) \ot K_Y^{\ot (1 - m)} \to Q_m \to 0 \]
The proof will distinguish two cases according to the value of $\limsup \frac{h^0(Q_m)}{m^3}$.
Let us first suppose that,
\[ \limsup \frac{h^0(Q_m)}{m^3} \le \frac{g_k(\X) - g_k(Y)}{2} \]
As expalined above, Bogomolov's vanishing theorem implies that,
\[ \limsup \frac{1}{m^3} h^0(\X, P^{k,m}_{\X}) \ge g_k(\X) \]
Then the exact sequence implies that,
\begin{align*}
\limsup \frac{1}{m^3} h^0(Y, P^{k,m}_Y) & \ge \limsup \frac{1}{m^3} h^0(Y, P^{k,m}_Y(\log{E}) - \limsup \frac{h^0(Q_m)}{m^3} \ge g_k(\X) - \frac{g_k(\X) - g_k(Y)}{2} 
\\
&= \frac{g_k(\X) + g_k(Y)}{2}
\end{align*}
by the extension property. Alternatively, if
\[ \limsup \frac{h^0(Q_m)}{m^3} > \frac{g_k(\X) - g_k(Y)}{2} \]
then the extension property and the triviality of $K_Y$ near $E$ gives,
\begin{align*}
h^0(Y, P^{k,m}_Y(\log{E}) \ot K_Y^{\ot (1-m)}) = h^0(Y \sm E, P^{k,m}_Y \ot K_Y^{\ot (1 - m)}) = h^0(\X, P^{k,m}_{\X} \ot K_{\X}^{\ot (1 - m)})
\end{align*}
{\color{red} IN THIS CASE WE NEED AN UPPER BOUND ON $h^0(Q_m)$ IN TERMS OF $h^1(Y, P^{k,m}_Y)$ }
PROBLEM SERRE DUALITY DOESNT SEND $P^{k,m}$ TO ITSELF TIMES THE CANONICAL DOES IT
{\color{red} THIS FAILS BADLY DONT SEE HOW TO CONCLUDE WITHOUT THIS TRICK}
\end{proof}

\section{Strategy of Bruin}

Let $X$ be an ADE surface and $\tau : Y \to X$ its minimal resolution. Let $\cA = E^{k,m}$ consider $\tau_* \cA$ and its reflexive hull $\hat{\cA} = (\tau_* \cA)^{\vee \vee}$. The Leray spectral sequence,
\[ E_2^{p,q} = H^p(X, R^q \tau_* \cA) \implies H^{p+q}(Y, \cA) \]
gives rise to the 6-term exact sequence,
\begin{center}
\begin{tikzcd}[column sep = tiny]
0 \arrow[r] & H^1(X, \tau_* \cA) \arrow[r] & H^1(Y, \cA) \arrow[r] & H^0(X, R^1 \tau_* \cA) \arrow[r] & H^2(X, \tau_* \cA) \arrow[r] & \ker{(H^2(Y, \cA) \to H^0(X, R^2 \tau_* \cA))} \arrow[r] & H^1(X, R^1 \tau_* \cA) 
\end{tikzcd}
\end{center}
The sheaf $R^1 \tau_* \cA$ has 0-dimensional support on the singular locus $S \subset X$ and thus,
\[ H^1(X, R^1 \tau_* \cA) = H^2(X, R^1 \tau_* \cA) = 0 \]
Furthermore, $\tau$ has $1$-dimensional fibers only and hence $R^2 \tau_* \cA = 0$. Therefore, the exact sequence becomes,
\begin{center}
\begin{tikzcd}
0 \arrow[r] & H^1(X, \tau_* \cA) \arrow[r] & H^1(Y, \cA) \arrow[r] & H^0(X, R^1 \tau_* \cA) \arrow[r] & H^2(X, \hat{\cA}) \arrow[r] & H^2(Y, \cA) \arrow[r] & 0
\end{tikzcd}
\end{center}
where we use the fact that $\hat{\cA} / \tau_* \cA$ and $\ker{(\tau_* \cA \to \hat{\cA})}$ are both supported on $S$ which is 0-dimensional so $H^2(X, \tau_* \cA) = H^2(X, \hat{\cA})$. Therefore, via the vanishing result $H^2(X, \hat{\cA}) = 0$ we get.
\[ h^1(Y, \cA) = h^1(X, \tau_* \cA) + h^0(X, R^1 \tau_* \cA) \]

{\color{red} DOES THE REFLEXIVE VANISHING RESULT HOLD FOR JETS?}

Note that we use that $\hat{\cA}$ is exactly reflexive symmetric forms on $X$. This is obvious because we can compute everything on $X \sm S$ which is isomorphic to $Y \sm E$ so $\tau_* \cA|_U = S^m \Omega_X |_U$ and therefore $\hat{\cA} = j_* (\tau_* \cA|_U) = (S^m \Omega_X)^{\vee \vee}$. The same argument should work for the jets since it works for any canonical object. 

\subsection{Bogomolov's Vanishing Lemma for Reflexive Forms}

\newcommand{\reg}{\mathrm{reg}}

\begin{prop}
Let $X$ be a surface with nodal singularities whose resolution $Y$ is a surface of general type. Then for $m > 0$,
\[ H^2(X, \hat{S}^m \Omega_X) = 0 \]
\end{prop}

\begin{proof}
From Serre duality for reflexive sheaves on normal surfaces:
\[ h^0(X, \hat{S}^m \Omega) = h^0(X, ((\hat{S}^m \Omega_X)^\vee \ot \omega_X)^{\vee \vee}) \]
The normality propery plus the reflexive nature on $X_{\reg}$ give,
\[ h^0(X, ((\hat{S}^m \Omega_X)^\vee \ot \omega_X)^{\vee \vee}) = h^0(X_{\reg}, (S^m \Omega_X)^\vee \ot \omega_X) = 0 \]
But on $X_{\reg}$ we have the quality $\Omega_X^\vee = \Omega_X \ot \omega_X$ and hence,
\[ h^2(X, \hat{S}^m \Omega_X) = h^0(X_{\reg}, S^m \Omega_X \ot \omega_X^{1-m}) \]
Suppose there is a $m > 2$ such that this is positive. Let $\ell = qm$ and,
\[ s \in H^0(X_{\reg}, S^m \Omega_X \ot \omega_X^{(1 - m)}) \]
a nonzero section and $w \in H^0(X_{\reg}, \omega_X^{(m-2)q})$ then
\[ s^{2q} \ot w \in H^0(X_{\reg}, S^{2 \ell} \Omega_X \ot \omega_X^{-\ell}) \]
Since $m > 2$ this implies that if $q \gg 0$ then $S^{2 \ell} \Omega_X \ot \omega_X^{-\ell}$ has a nontrivial section vanishing at some point of $X_{\reg}$. 
\bigskip\\
The existence of a nontrivial section of $S^{2 \ell} \Omega_X \ot \omega_X^{-\ell}$ vanishing at some point of $X_{\reg}$ implies instability properties of $\Omega_X$ on $X_{\reg}$. Applying the stability theory of Bogomolov-Mumford it follows that there is a line bundle $L \subset \Omega_X$ on $X_{\reg}$ such that the line bundle $(L^2 \ot \omega_X^{-1})^k$ has nontrivial sections for some $k$. The bigness of $K_X$ on $X_{\reg}$ then implies that $L$ is big over $X_{\reg}$. Now we apply the following.
\end{proof}

\begin{lemma}
Let $X$ be a projective surface with nodal singularities. Then there is a smooth cover of $X$ meaning a smooth projective surface $S$ and a finite map $f : S \to X$. In particular, the pre-image of the singular points consists of a finite set of smooth points. 
\end{lemma}


Completing the proof: there is a smoothing cover $f : S \to X$. Denote by $T$ the preimage on $S$ of the set of nodal points of $X$. Let $\d{f} : f^* \Omega_X^1 \to \Omega_S^1$ be the differential. Consider the rank one subsheaf of $\Omega_{X}|_{S\sm T}$ on $S \sm T$ which is the image of the sheaf $\d{f}(L)$. Since $T$ has codimension $2$ on $S$, this extends to a rank one coherent subsheaf of $\Omega_S^1$ and let $L' \subset \Omega_S^1$ be its saturation. A saturated subsheaf of a reflexive sheaf is reflexive and since it is rank $1$ we see that $L'$ is a line bundle.


\begin{lemma}
Let $\F \subset \G$ be a saturated subsheaf and $\G$ reflexive. Then $\F$ is reflexive. In fact $\F$ is reflexive if and only if it is a saturated subsheaf of a vector bundle. 
\end{lemma}

\begin{proof}
Consider the sequence,
\[ 0 \to \F \to \G \to \H \to 0 \]
where $\H$ is torsion-free. Then applying the dual we find,
\[ 0 \to \H^\vee \to \G^\vee \to \F^\vee \to \shExt{1}{}{\H}{\struct{X}} \] 
but the last term is torsion so we get
\[ 0 \to \F^{\vee \vee} \to \G^{\vee \vee} \to \H^{\vee \vee} \to \shExt{1}{}{\F^\vee}{\struct{X}} \]
but now consider the diagram,
\begin{center}
\begin{tikzcd}
0 \arrow[r] & \F^{\vee \vee} \arrow[r] & \G^{\vee \vee} \arrow[r] & \H^{\vee \vee} \arrow[r] & \shExt{1}{}{\F^\vee}{\struct{X}}
\\
0 \arrow[r] & \F \arrow[r] \arrow[u] \arrow[r] & \G \arrow[r] \arrow[u] \arrow[r] & \H \arrow[u] 
\end{tikzcd}
\end{center}
but $\G \to \G^{\vee \vee}$ is an isomorphism and $\H \to \H^{\vee \vee}$ is injective so $\F \to \F^{\vee \vee}$ is an isomorphism via a diagram chase. 
\end{proof}

Now we finish up: we conclude by pulling back that $L'$ is big. However, it is a theorem of Bogomolov that if $L' \subset \Omega_Y$ for a smooth surface $Y$ then $L'$ is not big giving a contradiction. 
\bigskip\\
The following seems like an easier proof: the orbifold cover $\pi : \X \to X$ is \etale away from the singular points. Then we consider $\pi^* \pi_* S^k \Omega_{\X} \to S^k \Omega_{\X}$. We want to show that,
\[ H^2(X, \hat{S}^k \Omega_X) = 0 \]
However, I claim that $\pi_* S^k \Omega_{\X} = \hat{S}^k \Omega_X$. Indeed, if $\E$ is a vector bundle on $\X$ then $\pi_* \E = j_* \E|_U$ where $U$ is the open over which $X$ is smooth and hence isomorphic to $\X$. Indeed, it suffices to show that for $j' : U \to \X$ we have $\E \to j'_* \E|_U$ is an isomorphism. But this can be checked \etale locally on schematic covers where it is clear because $\X$ is smooth and $U$ is big. Therefore,
\[ H^2(X, \hat{S}^k \Omega_X) = H^2(X, \pi_* S^k \Omega_\X) \]  
But $\pi : \X \to X$ is a good moduli space morphism and hence $\pi_*$ is exact on quasi-coherent sheaves. Since $\X$ and $X$ have affine diagonal\footnote{Indeed, locally this map is modeled on $[\A^n / G] \to \A^n / G$ for a finite group quotient. To show that $[\A^n / G]$ has affine diagonal notice that the diagonal is covered by $G \times \A^n \to \A^n \times \A^n$ given by projection and action which is affine since it is a map of affine schemes (here I need $G$ to be a finite or more generally affine group scheme)} this implies that $R^i \pi_* = 0$ for $i > 0$ so by the Leray spectral sequence,
\[ H^2(X, \hat{S}^k \Omega_X) = H^2(\X, S^k \Omega_{\X}) = 0 \]
by Bogomolov's vanishing result for stacks. This proof has the advantage of working directly for products of symmetric powers and hence for $k$-jets. 

\section{Some Questions}

Look at this paper \chref{https://arxiv.org/pdf/1603.02225.pdf}{by Ya Deng}

Questions:
\begin{enumerate}
\item why is ``every entire curve is algebraically degenerate'' enough to imply the GGL conjecture? Why couldn't for each one the locus it maps into be different so that the union is Zariski dense. 
\item Do the conjectures hold for $2$-jets i.e. for resolving foliations on $3$-folds? McQuillan seems to think the answer is yes. 
\end{enumerate}

\section{Note on Horikawa Surfaces}

\chref{https://arxiv.org/pdf/1201.5822.pdf}{In this paper} they prove that certain Horikawa surfaces are hyperbolic just using big orbifold cotangent bundle. Maybe this could give some example?

Specifically look at Proposition 56.


\section{Demailly - Goul}

\begin{defn}
A Riemann surfaces is \textit{parabolic} if it does not admit a positive non constant superharmonic function (equivalently it does not admit a Green function). 
\end{defn}

\begin{rmk}
This is the same as the $1$-dimensional case of Bost's \textit{Liouville property} that every bounded above plurisubharmonic function is constant.
\end{rmk}

\begin{theorem}[Liouville]
Let $M$ be a compact complex manifold. Then any plurisubharmonic function on $M$ is constant. 
\end{theorem}

\begin{theorem}
If a connected complex manifold $M$ satisfies the Liouville property and $F \subset M$ is a closed analytic subset of positive codimension then $M \sm F$ satisfies the Liouville property.
\end{theorem}

{\color{red} IS THIS TRUE?}

\begin{lemma}
Let $\Omega \subset \CC$ be a domain. Let $f : \Omega \sm \{ 0 \} \to \RR$ be a bounded below superharmonic function. Then $f$ extends to $f : X \to \RR$.
\end{lemma}

\begin{proof}
\chref{https://mathoverflow.net/questions/461150/a-compact-riemann-surface-with-a-finite-set-of-points-removed-is-parabolic}{See this answer}.
\end{proof}

Therefore, an open algebraic curve is parabolic if and only if its compactification is parabolic.

\begin{lemma}
Let $X$ be a compact Riemann surface minus finitely many points, then $X$ is parabolic.
\end{lemma}

\begin{proof}
This follows from the extension theorem and Liouville for subharmonic functions.
\end{proof}


\subsection{Notation}

\begin{enumerate}
\item $D_2 := \P(T_{X_1/X}) \subset X_2 = \P(V_1)$ is the zero divisor of the map,
\[ \struct{X_2}(-1) \to \pi_2^* \struct{X_1}(-1) \]
induced by the canonical maps $\struct{X_2}(-1) \embed \pi_2^* V_1$ and $V_1 \to \struct{X_1}(-1)$.
\end{enumerate}

\subsection{Multifoliations}

\begin{defn}
An \textit{algebraic multi-foliation} on $X$ is a rank $1$ subsheaf $\F \subset S^m \Omega_X$ locally generated by a jet differentials of order $1$.
\end{defn}

\begin{theorem}
Let $X$ be a nonsingular surface of general type and let $\theta_k$ be the $k$-jet threshold. Assume that either $\theta_1 < 0$ or that the following three conditions are satisfies,
\begin{enumerate}
\item $\theta_1 \ge 0$ and $\theta_2 < 0$
\item $\Pic{X} = \Z$
\item $\frac{c_1^2}{c_2} > \frac{9}{13 + 12 \theta_2}$
\end{enumerate}
then every nonconstant holomorphic map $f : \CC \to X$ is the leaf of an algebraic multi-foliation on $X$.
\end{theorem}

\begin{prop}
Let $X$ be a minimal surface of general type, equipped with an algebraic multi-foliation $\F \subset S^m \Omega_X$. Assume that
\[ m (c_1^2 - c_2) + c_1 \cdot c_1(\F) > 0 \]
Then there is a curve $\Gamma$ in $X$ such that all parabolic leaves of $\F$ are contained in $\Gamma$.
\end{prop}

\begin{proof}
Note that any rank $1$ torsion-free {\color{red} reflexive} sheaf on a surface is locally free. The inclusion $\F \embed S^m \Omega_X$ viewed as a section of $S^m \Omega_X \ot \F^{\vee}$ defines a section of $\struct{X_1}(m) \ot \pi^* \F^{-1}$ whose zero divisor $Z \subset X_1$ is precisely the divisor associated with the foliaiton. Therefore $Z = m u_1 - \pi^* c_1(\F)$ in $\Pic{X_1}$ and our calculuations imply that $\struct{X_1}(1)|_Z$ is big as soon as,
\[ (u_1|_Z)^2 = u_1^2 \cdot Z = m (c_1^2 - c_2) + c_1 \cdot c_1(\F) > 0 \quad (u_1|_Z) \cdot (-c_1) = m c_1^2 + c_1 \cdot c_1(\F) > 0 \]
However, $X$ is minimal so $c_2 \ge 0$ hence the first inequality implies the second proving the claim.
\end{proof}

\begin{prop}
Let $X$ be a surface of general type equipped with a multifoliation $\F \subset S^m \Omega_X$ and let $\sigma \in H^0(X_1, \struct{X_1}(m) \ot \pi^*_{1,0} \F^\vee)$ be the associated canonical section. Let $G_1$ be the divisorial part of the subscheme defined by $\ker{\d{\sigma}|_{V_1|Z}}$ then if,
\[ m^2 (4 c_1^2 - 3 c_2) + m(5 c_1^2 - 3 c_2) + (8m + 4)c_1 \cdot \F + 3 \F^2 - (3 u_1 - c_1) \cdot G_1  > 0\]
all parabolic leaves of $\F$ are algebraically degenerate.  
\end{prop}

{\color{red} why does algebraically degneracy of each leaf imply it for the union?}

\section{Proof}

\begin{theorem}[Dem95]
Let $X$ be an algebraic surface of general type and $A$ an ample line bundle. Then,
\[ h^0(X, P^{2,m} \ot A^{-1}) \ge \frac{m^4}{648} (13 c_1^2 - 9 c_2) + O(m^3) \]
\end{theorem}

Consider the Semple tower $X_2 \to X_1 \to X$ which are iterated $\P^1$-bundles. Thus,
\[ \Pic{X_2} = \Pic{X} \oplus \Z u_1 \oplus \Z u_2 \]
so every line bundle is of the form,
\[ \pi_{2,1}^* \struct{X_1}(a_1) \ot \struct{X_2}(a_2) \ot \pi_{2,0}^* \L \]
where we write,
\[ u_1 = \pi_{2,1}^* \struct{X_1}(1) \quad u_2 = \struct{X_2}(1) \]
The canonical exact sequence,
\begin{center}
\begin{tikzcd}
0 \arrow[r] & T_{X_1/X} \arrow[r] & V_1 \arrow[r] & \struct{X_1}(-1) \arrow[r] & 0
\end{tikzcd}
\end{center}
which serves as the definition of $V_1$ and the canonical injection,
\[ \struct{X_2}(-1) \embed \pi_2^* V_1 \]
yield the morphism
\[ \struct{X_2}(-1) \embed \pi_2^* \struct{X_1}(-1) \]
which admits the hyperplane section $D_2 := \P(T_{X_1/X}) \subset X_2 = \P(V_1)$ as its zero divisor. Therefore,
\[ \struct{X_2}(-1) = \pi_2^* \struct{X_1}(-1) \ot \struct{}(-D_2) \]

\begin{lemma}
With respect to the projection $\pi_{2,0} : X_2 \to X$ the weighted line bundle $\struct{X_2}(a_1, a_2)$ is
\begin{enumerate}
\item relatively effective iff $a_1 + a-2 \ge 0$ and $a_2 \ge 0$
\item relativiely big iff $a_1 + a_2 > 0$ and $a_2 > 0$
\item relatively nef iff $a_1 \ge 2 a_2 \ge 0$
\item relatively ample iff $a_1 > 2 a_2 > 0$.
\end{enumerate}
Moreover, the following properties hold,
\begin{enumerate}
\item for $m = a_1 + a_2 \ge 0$ there is an injection
\[ (\pi_{2,0})_* \struct{X_2}(a_1, a_2) \embed P^{2,m} \]
which is an isomorphism for $a_1 - 2 a_2 \le 0$
\item let $Z \subset X_2$ be an irreducible divisor such that $Z \neq D_2$ then in $\Pic{X_2}$ 
\[ Z \sim a_1 u_1 + a_2 u_2 + \pi_{2,0}^* L \]
where $a_1 \ge 2 a_2 \ge 0$
\item Let $F \in \Pic{X}$ be any divisor class. In $H^*(X_2) = H^*(X)[u_1, u_2]$ the following hold,
\[ u_1^4 = 0 \quad u_1^3 u_2 = c_1^2 - c_2 \quad u_1^2 u_2^2 = c_2 \quad u_1 u_2^3 = c_1^2 - 3 c_2 \quad u_2^4 = 5 c_2 - c_1^2 \]
and 
\[ u_1^3 \cdot F = 0 \quad u_1^2 u_2 \cdot F = - c_1 \cdot F \quad u_1 u_2^2 \cdot F = 0 \quad u_2^3 \cdot F = 0 \]
\end{enumerate}
\end{lemma}

\begin{proof}
{\color{red} DO THIS}
\end{proof}

Let $B_2$ be the base locus of $\struct{X_2}(m)$ on $X_2$.

\begin{prop}
Let $X$ be a minimal surface of general type. If $c_1^2 - 9/7 c_2 > 0$ then the restriction of $\struct{X_2}(1)$ to every irreducible $3$-dimensional component $Z$ of $B_2 \subset X_2$ which projects onto $X_1$ and differs from $D_2$ is big.
\end{prop}

\begin{proof}
Because $Z$ is effective we can write,
\[ Z = a_1 u_1 + a_2 u_2 - \pi_{2,0}^* F \quad (a_1, a_2) \in \Z^2 \quad a_1 \ge 2 a_2 > 0 \]
where $F$ is some divisor in $X$. Our strategy is to show that $\struct{X_2}(2,1)|_Z$ is big. By Lemma 3.3(e) we find,
\[ (2 u_1 + u_2)^3 \cdot Z = (_1 + z_2) (13 c_!^2 - 9 c_2) + 12 c_1 \cdot F \]
The multiplication morphism by the canonical section of $\struct{}(Z)$ defines a sheaf injection
\[ \struct{}(\pi_{2,0}^* F) \embed \struct{X_2}(a_1, a_2) \]
\end{proof}

\subsection{2-jets}

Let $Z \subset X_1$ be the divisor associated with a given foliation $\F$ and $\sigma \in H^0(X_1, \struct{X_1}(m) \ot \pi^* \F^{-1})$ the corresponding section. Let $\T_Z$ be the tangent sheaf of $Z$ defined by the exact sequence,
\begin{center}
\begin{tikzcd}
0 \arrow[r] & \T_Z \arrow[r] & \T_{X_1}|_Z \arrow[r, "\d{\sigma}"] & \struct{X_1}(m)|_Z \ot \pi^* \F^{-1}|_Z \arrow[r] & 0
\end{tikzcd}
\end{center}
If we define $S = \T_Z \cap \struct{}(V_1)$ then there is an exact sequence
\begin{center}
\begin{tikzcd}
0 \arrow[r] & S \arrow[r] & V_1|_Z \arrow[r, "\d{\sigma}"] & \struct{X_1}(m)|_Z \ot \pi^* \F|_Z^{-1}
\end{tikzcd}
\end{center}
and $S$ is an invertible sheaf so there is a dual sequence
\begin{center}
\begin{tikzcd}
0 \arrow[r] & \struct{X_1}(-m)|_Z \ot \pi^* \F|_Z \arrow[r] & V^*_1 |_Z \arrow[r] & S^* 
\end{tikzcd}
\end{center}
We can then lift $Z$ into a surface $\wt{Z} \subset X_2$ in such a way that the projection map $\pi_{2,1} : \wt{Z} \to Z$ is a modification, at a generic point $x \in Z$, the point of $\wt{Z}$ lying above $x$ is taken to be $(x, [S_x]) \in X_2$ (meaning we take the induced morphism from the generic quotient map $V_1^*|_Z \to S^*$ which gives a lift to the projectiviation). We need to compute the cohomology class of this $2$-cycle $\wt{Z}$ in $H^\bullet(X_2)$. One problem is that the cokernel of the map
\[ \d{\sigma}|_{V_1|_Z} : V_1|_Z \to \struct{X_1}(m)|_Z \ot \F^{-1}|_Z \]
may have torsion along a $1$-cycle $G_1 \subset Z$. If the foliation is generic, however, the cokerne of $\d{\sigma}|_{V_1|_Z}$ will have no torsion in codimension $1$ and $\d{\sigma}$ then induces a section of
\[ \struct{X_2}(1) \ot \pi_{2,1}^* \struct{X_1}(m) \ot \pi^*_{2,0} \F^{-1} \]
whose zero locus in $\wt{Z}$. As $Z \sim m u_1 - c_1(\F)$ the cohomology class of $\wt{Z}$ is given by,
\[ [\wt{Z}] = (m u_1 - c_1(\F)) \cdot (u_2 + m u_1 - \F) = m^2 u_1^2 + m u_1 \cdot u_2 - 2 m u_1 \cdot c_1(\F) - u_2 \cdot c_1(\F) + c_1(\F)^2 \]
A Chern class computation yields
\[ (2 u_1 + u_2)^2 \cdot \wt{Z} = m^2 (4 c_1^2 - 3 c_2) + m (5 c_1^2 - 3 c_2) + (8 m + 4) c_1 \cdot c_1(\F) + 3 c_1(\F)^2 \]
If the $1$-cycle $G_1$ is nonzero, the numerical formula becomes,
\[ [\wt{Z}] = (m u_1 - \F) \cdot (u_2 + m u_1 - c_1(\F)) - \pi_{2,1}^* [G_1] \]
The generical formula for $(2 u_1 + u_2)^2 \cdot [\wt{Z}]$ is therefore,
\[ (2 u_1 + u_2)^2 \cdot [\wt{Z}] = m^2 (4 c_1^2 - 3 c_2) + (5 c_1^2 - 3 c_2) + (8m + 4) c_1 \cdot c_1(\F) + 3 c_1(\F)^2 - (3 u_1 - c_1) \cdot [G_1] \]
Using the obvious exact sequences $H^2(\wt{Z}, \struct{X_2}(2m, m)|_{\wt{Z}})$ is a quotient of
\[ H^2( \pi_{2,1}^{1}(Z), \struct{X}(2m,m) |_{\pi_{2,1}^{-1}(Z)}) \]
which is controlled by $H^2(X_2, \struct{X_2}(2m,m))$ and $H^3(X_2, \struct{X_2}(2m,m) \ot \struct{}(-Z))$ a direct image argument shows that the latter groups are controlled by groups of the form $H^2(X, E_{2,3m} \Omega \ot L)$ for suitable $L$. s in the proof of Theorem 3.4 one can check that the latter $H^2$ groups vanish. The positivity of $(2 u_1 + u_2)^2 \cdot [\wt{Z}]$ therefore implies that $\struct{X_2}(2,1)|_{\wt{Z}}$ is big and therefore all parabolic leave of the multifoliation $\F$ are algebraically degenerate because the parabolic leaves of $\F$ must lie in the base locus of all invariant differentials and this cuts out a $1$-dimensional space on $\wt{Z}$ by the bigness and the fact that $\wt{Z}$ is $2$-dimensional.



\section{Bigness of $K_\F$}

What is the condition on $\omega$ for it to define a foliation with $K_\F$ big?

Consider, $Y \to V(\omega) \subset \P(\Omega_X)$ the desingularization. Denote the maps $p : Y \to \P(\Omega_X)$ and $f : Y \to X$. Then we have,
\begin{center}
\begin{tikzcd}
f^* \Omega_X \arrow[d] \arrow[r] & \Omega_Y \arrow[d]
\\
L \arrow[r] & \pushout Q
\end{tikzcd}
\end{center}
Let $L = p^* \struct{X}(1)$.
Since $f^* \Omega_X \to L$ is surjective, it is immediate that $Q = \coker{(\ker{(f^* \Omega_X \to L)} \to \Omega_Y)}$.
Then we can take the torsion-free quotient of $Q$ which is rank $1$ and hence is $\I_Z \cdot \M$ for some $Z \subset Y$ closed subscheme of codimenson $2$. 

\begin{lemma}
Let $X$ be a smooth variety and $\F$ a torsion-free coherent sheaf of rank $1$. Then there is a unique decomposition $\F \cong \I_Z \cdot \L$ where $\L$ is a line bundle and $\I_Z$ is the ideal sheaf of a codimension $2$ subscheme.
\end{lemma}

\begin{proof}
Since $\F$ is torsion-free the map $\F \to \F^{\vee \vee}$ is injective. Furthermore, $\F^{\vee \vee}$ is a line bundle since it is reflexive of rank $1$ on a smooth variety. Let $\L = \F^{\vee \vee}$ and consider $\F \ot \L^{-1} \embed \struct{X}$. It suffices to show that this map is an isomorphism in codimension $1$. Indeed, if $\dim{\stalk{X}{x}} \le 1$ then $\stalk{X}{x}$ is a DVR and $\F_x$ is torsion-free and hence free therefore $\F$ is a vector bundle over an open $U$ of complementary codimension at least $2$ and thus $\F \to \F^{\vee \vee}$ is an isomorphism over $U$. 
\end{proof}


Now the claim is, if $L$ is big then $\M$ is big. Indeed, there is a map $L \to \M$ and hence $H^0(Y, L^{\ot m}) \embed H^0(Y, \M^{\ot m})$, 


WEIRD from McQuillan's paper: if $\Omega_X$ is big then it seems that the foliation is always big. WHAT ABOUT THE BIDISK FOLIATIONS?

PROBLEM: no reason for $L$ to be BIG, IS EXACTLY,
\[ 0 \to \Sym{k-m}{\Omega_X} \xrightarrow{\omega} \Sym{k}{\Omega_X} \to f_* L^{\ot m} \to 0 \]
for $m > k$ and taking cohomology we have to control a connecting map. Therefore, its only clearly big if some asymtotic cohomology statement holds which is possibly stronger than the asymtotic cohomology statement we need for the jet obstruction. 
\bigskip\\
In fact, if $L^{\ot m}$ has a nonconstant section at all then the base locus does not dominate $X$ so we immediately win by the resultant method. Thus we expect no sections. 


It looks like if $\F$ is $p$-closed then we ask are there finitely many or infinitely many $C$ such that $K_{\F} \cdot C < 0$. In the former case we win by McQuillan's theorem. In the latter case, we loose by Miyaoka's theorem since then there are infinitely many rational curves of bounded degree by Bend-and-Break hence uniruled. 

\section{Foliation Basics}

\begin{prop}
Let $f : X \to Y$ be a morphism of $S$-schemes and a morphism $\F \to \T_{Y/S}$ a subsheaf. Then there exists, a morphism $f^+ \F \to \T_{X/S}$ and the linear Lie bracket maps are compatible,
\begin{center}
\begin{tikzcd}
\wedge^2 f^+ \F \arrow[d] \arrow[r, "\ell"] & \T_{X/S} / f^+ \F \arrow[d]
\\
f^* \wedge^2 \F \arrow[r, "f^* \ell"] & f^* \T_{Y/S} / \F  
\end{tikzcd}
\end{center}
and if $S$ has characteristc $p$ then the $p$-curvature maps are also compatible,
\begin{center}
\begin{tikzcd}
\Frob^*_X f^+ \F \arrow[d] \arrow[r, "\psi_p"] & \T_{X/S} / f^+ \F \arrow[d]
\\
f^* \Frob^*_X\F \arrow[r, "f^* \psi_p"] & f^* \T_{X/S} / \F 
\end{tikzcd}
\end{center}
\end{prop}

\begin{proof}
We define $f^+ \F$ via the diagram,
\begin{center}
\begin{tikzcd}
f^+ \F \arrow[d] \pullback \arrow[r] & f^* \F \arrow[d]
\\
\T_{X/S} \arrow[r] & f^* \T_{Y/S}
\end{tikzcd}
\end{center}
Now the commutativity is obvious from the fact that $\T_{X/S} \to \pi^* \T_{Y/S}$ is compatible with bracket and $p$-power operations.
\end{proof}

Therefore, as long as $X$ does not map into the singular locus of the foliation $\F$, then we can define a foliation as the saturated image of $f^+ \F$ on $Y$ and then it will also be involutive and $p$-closed if $\F$ is. {\color{red} PROVE THIS!!}


\begin{cor}
If $f : C \to X$ is an entire curve invariant under a foliation $\F$ then 
\[ \im{f} \subset \Sing{\F} \cup (V^1(\psi_p) \cap V^1(\ell)) \]
where $V^1$ is the locus where a map of vector bundles is not full rank. For example, if $\F$ is a rank $1$ foliation then,
\[ \im{f} \subset \Sing{\F} \cup V(\psi_p) \]
If $X$ is a surface and $\F$ is not $p$-closed then $\Sing{\F}$ is a collection of points so any invariant curve is contained in $\Delta_p := V(\psi_p)$.
\end{cor}

\begin{theorem}[Miyaoka]
Let $L$ be a nef $\RR$-divisor on $X$. Let $f : C \to X$ be a nonconstant morphism from a smooth projective curve $C$ such that $X$ is a smooth along $f(C)$. Let $\F \subset \T_X$ be a ($p$-closed) $1$-foliation, smooth along $f(C)$. Assume that,
\[ c_1(\F) \cdot C > \frac{K_X \cdot C}{p-1} \]
then for every $x \in f(C)$ there is a rational curve $B_x \subset X$ passing through $x$ such that,
\[ L \cdot B_x \le 2n \frac{p L \cdot C}{(p-1) c_1(\F) \cdot C - K_X \cdot C} \]
\end{theorem}

\begin{proof}
Let $\rho : X \to Y = X / \F$ be the quotient. Then,
\[ K_X = (p-1) c_1(\F) + \rho^* K_Y \] 
and therefore,
\[ K_Y \cdot C = K_X \cdot C - (p-1) c_1(\F) \cdot C < 0 \]
Since the foliation is smooth along $f(C)$, the variety $Y$ is also smoth along the image of $C$. Therefore, by Bend-and-Break [Kollar rational curves, Theorem 5.8] we know that there exists such a family of rational curves $B_x'$ throught each point of the image of $C$ such  that for any nef $\RR$-divisor $M$ of $Y$ we have
\[ M \cdot B_x' \le 2n \frac{M \cdot C}{-K_Y \cdot C} \]
Let $L'$ be the pullback of $L^{(1)}$ via the canonical map $Y \to X^{(1)}$ then $\rho^* L' = (F_X)^* L = p L$. Since $\rho$ is purely inseparable of degree $p^{\rank{\F}}$ the reduced pre-image $B_x$ of $B_x'$ is also a rational curve. Note that,
\[ p (L \cdot B_x) = \rho^* L' \cdot B_x \le p L' \cdot B_x' \]
If the induced morphism $B_x \to B_x'$ is inseparable then we have equality. Otherwise, $\rho^* L' \cdot B_x = L' \cdot B_x'$ and the inequality is obvious. Therefore,
\[ L \cdot B_x \le L' \cdot B_x' \le 2n \frac{L' \cdot C}{-K_Y \cdot C} \le 2n \frac{p L \cdot C}{(p-1) c_1(\F) - K_X \cdot C} \]
proving the claim.
\end{proof}

\begin{rmk}
$c_1(\F) = - K_{\F}$ and therefore the theorem applies to curves $C$ with $K_{\F} \cdot C < 0$.
\end{rmk}


\section{Surfaces over $\ZZ$}

\begin{example}
Consider the deformation $X \to \A^1$ of a smooth quartic surface in $\P^3$ to the singular Kummer quadric. Then $\pi_1(X_0^{\text{sm}}) \neq 1$ even though $\pi_1(X_1 \sm S) = 1$ for any finite set of points $S$. This shows we cannot use deformation theory and Grothendieck's existence theorem to conlude that the fundamental group is zero. However, the resolution of singularities of $X_0$ is indeed simply-connected. This shows we need to carefully analyze the singularities of the special fiber in order to conclude.
\end{example}

\begin{prop}
Let $X / K$ be a smooth proper suface over a number field. Let $\p \subset K$ be a prime such that $N(\p) = q = p^n$ and $K / \Q$ is unramfied over $p$. Let $\ell \neq p$ be another prime. Suppose,
\begin{enumerate}
\item $\pi_1^{\et}(X_{\ol{K}}) = 0$
\item $\Frob_{\q}^n \acts H^2_{\et}(X_{\ol{K}}, \Q_{\ell})$ by $q^n \cdot \id$ for some $n > 0$ 
\item there is a flat proper model $\X \to \Spec{\stalk{K}{\p}}$ of $X$ such that $\X_\p$ is integral with isolated (GOOD DO THIS!!!) singularities.
\end{enumerate}
Then let $Y$ be a smooth proper surface over $\FF_q$ obtained as a resolution of singularities of $\X_{\p}$. Then,
\begin{enumerate}
\item $\pi_1^{\et}(Y_{\ol{\FF}_q}) = 0$
\item $Y$ is supersingular meaning $\Frob_{\q} \acts H^2(Y_{\ol{\FF}_q}, \Q_\ell)$ by $q^n \cdot \id$ for some $n > 0$. 
\end{enumerate}
\end{prop}

\begin{proof}
DO THIS PROOF!!!
\end{proof}







\section{Surfaces to Try}

\subsection{Horikawa surfaces}

Properties:
\begin{enumerate}
\item always $\pi_1 = 0$ (REF?)
\item usually general type 
\item can be orbifold hyperbolic
\end{enumerate}

Cons:

\begin{enumerate}
\item supersingularity seems hard
\item often are actually unirational
\end{enumerate}

\begin{enumerate}
\item 
\end{enumerate}

References:

\begin{enumerate}
\item \chref{https://arxiv.org/pdf/1201.5822.pdf}{How is the orbifold jets relevant here?}
\item \chref{https://arxiv.org/pdf/0805.3986.pdf}{there are lots of unirational horikawa surfaces}
\item 
\end{enumerate}

\subsection{Hilbert Modular Surfaces}


\subsection{Quotient of Products of Curves}

\subsection{Line Arrangement Varities}

\subsection{Explicit Complete Intersections}


\subsection{REU Examples}


\begin{theorem}
Let $p, q, w$ be primes such that $p, q, w \equiv 1 \mod{s}$ for some $s$ and let $X$ be the variety defined by,
\[ x_0^p + x_1^{ps} + x_2^{q} + x_3^{qs} = 0 \]
over $\FF_{w}$. If $w$ is a primitive root modulo $p$ and $q$ then $X$ is supersingular.  
\end{theorem}

\begin{example}
This works for,
\begin{enumerate}
\item $p = 7$
\item $q = 13$
\item $s = 3$
\item $w = 19$
\end{enumerate}
\end{example}

This example is supersingular for infinitely many explicit primes. Also it should be simply connected {\color{red} WHY?}
\bigskip\\
What the singularities? Note this is clearly quasi-smooth. Therefore, the singularities are only at the singularities of $P = \P(qs, ps, q, p)$. This is well-formed. 

\section{Foliations Methodology}

\subsection{What people say about McQuillan's paper}

\subsubsection{AN EXPLICIT BOUND FOR THE LOG-CANONICAL DEGREE OF CURVES ON
OPEN SURFACES}

Bogomolov proved the well known result according to which irreducible curves of fixed
geometric genus on X form a bounded family. Since Bogomolov’s argument depended on the
analysis of curves contained in a certain closed set (see [Des79] for an exposition), his remark-
able result was not effective. Indeed, Bogomolov was able to prove that curves in this closed set
form a bounded family by considerations involving algebraic foliations but without providing an
explicit bound on their degree. Because of this, in a deformation of the surface X , the number of
either rational or elliptic curves might in principle tend to infinity. This situation can be ruled
out providing an upper bound on the canonical degree of irreducible curves on X by a function
of the invariants of X and the geometric genus of the curve. The existence of such a function
and its form was then conjectured in various places and in slightly different contexts, see for
instance [Tia96, §9], with the function depending only on K2
X , c2(X ) and the geometric genus of
the curve. The conjecture was proved with some restrictive hypothesis on the singularities of the
curve involved by Langer in [Lan03] and finally in its full generality by Miyaoka in [Miy08]. It
is interesting to note that part of Miyaoka’s result can be recovered by methods closer in spirit to
the original argument of Bogomolov, see McQuillan [McQ17, Corollary 1.3], though one is able to
prove the existence of the afore mentioned function no explicit form can be established.


\subsubsection{Effective algebraic integration in bounded genus, Most interesting part of McQuillan's classification of foliations}

\newcommand{\HH}{\mathbb{H}}

\begin{theorem}
Let $\F$ be a relatively minimal foliation on a smooth projective surface $X$. If the numerical dimension of $\F$ does not coincide with the Kodaira dimension of $\F$ then,
\begin{enumerate}
\item $\nu(\F) = 1$
\item $\kappa(\F) = -\infty$
\item $X$ is the minimal desingularization of the Baily-Borel compactification of an irreducible quotient of $\HH \times \HH$ 
\item $\F$ is induced by one of the two natural fibrations on $\HH \times \HH$
\end{enumerate}
\end{theorem}

\subsubsection{Definitions}

(WHAT IS ALMOST ETALE??)

When the foliation is p-closed it follows from [2, Th´eor`eme 1] that there are infinitely
many algebraic solutions. On the other hand, if the foliation F is not p-closed then there
is a divisor F , the p-divisor, which is defined as the degeneracy locus of the p-curvature
morphism F . An interesting property of the p-divisor is that every irreducible algebraic
solution of the foliations is contained in the support of the p-divisor.


QUESTION: DO THE LEAVES OF SUCH A FOLIATION ALWAYS HAVE TO BE RATIONALLY CONNECTED? LOOK AT THE PAPER PROVING THIS USING MORI THEORY?

\section{Place to Collect Foliation Results}

\subsubsection{Paper of Langer}

Proves Popa-Schnell for surfaces in all characteristics

\chref{https://www.mimuw.edu.pl/~alan/papers/foliations.pdf}{Theorem }

Look at Theorem 5.1

Also gives a nice foliation producing rational curves result: Theorem 2.1

\subsection{Foliaitons To Read}

\begin{enumerate}
\item Myoka
\item Bogomolov-McQuillen
\end{enumerate}

\section{Product-Quotient Surfaces}

Let $G \acts C_1$ and $G \acts C_2$ be two faithful actions on smooth projective curves. Then we consider $X = (C_1 \times C_2) / G$ with the diagonal action and $\pi : S \to X$ the minimal desingualarizaton. Here we write down some results.

\begin{prop}
If $C$ is any curve and $H$ is a finite group acting faithfully on $C$ and $p \in C$ then the fixed point $H_p$ is cyclic. 
\end{prop}

\begin{proof}
 Hershel M. Farkas , Irwin Kra III.7.7 It follows from the next result.
\end{proof}

\begin{prop}
Let $h_1, \dots, h_n$ be holomorphic functions in the neighborhood of $0 \in \CC$ with $h_j(0) = 0$ and suppose these form a group $H$ under composition. Then $H$ is a rotation group, meaning there is an open disc around $0$ on which all $h_i$ are defined and a biolomorphism to $\Delta$ such that the $h_j$ are identidied with rotations. 
\end{prop}

\begin{cor}
The singularities of $X$ are cyclic quotient singulairites. If $\ol{(x,y)} \in X$ then analytically locally this point is $\CC^2 / C_n$ where $n = \# G_{(x,y)}$ defined by $\xi \cdot (z_1, z_2) = (\xi z_1, \xi^a z_2)$ for some $a$ coprime to $n$ and $\xi$ a primitive $n^{\text{th}}$-root of unity. This is a \textit{singularity of type} $\frac{1}{n}(1,a)$.
\end{cor}

\begin{rmk}
Note that $(n,a) = 1$ because otherwise some element of $G$ would act trivially on the second factor in an analytic neighborhood but by rigidity this means it acts trivially on one curve and hence the action is not faithful. 
\end{rmk}

\subsection{Singularity Types}

Note that $\frac{1}{n}(1,a) = \frac{1}{n}(1,a')$ where $a, a'$ are inverses in $(\Z / n \Z)^\times$. The exceptional fiber of the minimal resolution of a cyclic quotient singularity of type $\frac{1}{n}(1,a)$ are well-known and correspond to Hizerbruch-Jung strings:
\[ L = \sum_{i = 0}^\ell Z_i \]
a connected union of smmoth rational curves $Z_1, \dots, Z_\ell$ with self-intersection numbers at most $-2$ and ordered linearly so that $Z_i \cdot Z_{i+1} = 1$ and all other intersections are zero. Then the exceptional divisor $E$ on $S$ is the disjoint union of these HJ strings. 
\bigskip\\
The self-intersection numbers $Z_i^2 = -b_i$ are determined by the formula,
\[ \frac{n}{a} = b_i - \frac{1}{b_2 - \frac{1}{\cdots}} \]
We denote this fraction by the notation $[b_1, \dots, b_{\ell}]$ so we write,
\[ \frac{n}{a} = [b_1, \dots, b_\ell] \]
Moreover,
\[ \frac{n}{a} = [b_1, \dots, b_\ell] \iff \frac{n}{a'} = [b_\ell, \dots, b_1] \]
Cyclic quotient singularities of type $\frac{1}{n}(1, n-1)$ are particular cases of \textit{rational double points} or $A_n$ singularity: all the curves $Z_i$ have self-intersection $-2$. Singularities of type $\frac{1}{2}(1,1)$ are called \textit{ordinary double points} or $A_1$ singularities. However, type $\frac{1}{n}(1,a)$ for $a \neq n-1$ do not need to be canonical singularities.  


\begin{prop}[Serrano, Prop. 2.2]
$q(S) : = h^1(S, \struct{S}) = g(C_1/G) + g(C_2/G)$.
\end{prop}

Therefore, for our purposes we will always take $C_1 / G \cong C_2 / G \cong \P^1$. 
\bigskip\\
Furthermore, let $p_g = h^0(S, \omega_S) = h^2(S, \struct{S})$. Thus if $q(S) = 0$ we see that,
\[ \chi(\struct{S}) = 1 + p_g \]
and hence by Noether's formula,
\[ c_1^2 + c_2 = 12 (1 + p_g) \]

\begin{theorem}[Serrano, Theorem 4.1]
Let $S$ be as above and $\sigma_1 : S \to C_1 / G$ and $\sigma_2 : S \to C_2 / G$ the associated fibrations. Let $\{ n_i N_i \}_{i \in I}$ and $\{ m_j M_j \}_{j \in J}$ denote the components of all singular fibers of $\sigma_1$ and $\sigma_2$ respectively with the multiplicites attached. Finally, let $\{ Z_t \}_{t \in T}$ be the set of curves contracted to points by $\sigma_1 \times \sigma_2$ (i.e. the exceptional locus of $\pi : S \to X$) then,
\[ K_S = \sigma_1^* (K_{C_1/G}) + \sigma_2^*(K_{C_2 / G}) + \sum_{i \in I} (n_i - 1)N_i + \sum_{j \in J} (m_i - 1) M_i + \sum_{t \in T} Z_t \]
The fibrations induce foliations $\F_1, \F_2$ on $S$ such that Serrano's formula can be written as,
\[ K_S = \cN_{\F_1}^\vee \ot \cN_{\F_2}^\vee \ot \struct{S}(E) \]
\end{theorem}


\begin{center}
\begin{tikzcd}
& S  \arrow[d, "\pi"] \arrow[ldd, "\sigma_1"'] \arrow[rdd, "\sigma_2"]
\\
& (C_1 \times C_2) / G \arrow[ld, "p_1"] \arrow[rd, "p_2"'] 
\\
C_1 / G & & C_2 / G
\end{tikzcd}
\end{center}
We know that $X$ has only cyclic quotient singularities. 

\subsection{Surfaces with $p_g = 0$ and $c_1^2 = c_2$}

The conditions imply that $c_1^2 = c_2 = 6$ and $p_g = q = 0$. These have been classified and all have exactly two $A_1$ singularities.
\bigskip\\
Around one of the two singular points, $X$ is analytically isomorphic to the quotient $\CC^2 / C_2$ with action $(z_1, z_2) \mapsto (-z_1, -z_2)$. This is isomorphic to an affine subvariety of $\CC^3$ with coordinates,
\[ u = z_1^2 \quad v = z_1 z_2 \quad w = z_2^2 \]
defined by the equation $uw  =  v^2$. Moreover, if $\mu_1, \mu_2$ are the local coordinates on $S$, the resolution morphism is locally given by,
\[ \pi(\mu_1, \mu_2) = (\mu_1, \mu_1 \mu_2, \mu_1 \mu_2^2) \]
given by blowing up $(u,v,w)$. Therefore, we have the following relation between the local coordinates,
\[ z_1 = \mu_1^{1/2} \quad z_2 = \mu_1^{1/2} \mu_2 \]
The exceptional fiber is a single $(-2)$-curve. Using the local coordinates $\mu_1, \mu_2$ on $S$ we see that it is given by the set of point $\{ \mu_1 = 0 \}$. 

\subsection{Bigness of the Cotangent Bundle}

\renewcommand{\sm}{\mathrm{sm}}

Let $\Lambda$ be the set of points of $C_1 \times C_2$ with nontrivial stabilizer. We consider a procedure to produce sections of $S^{2m} \Omega_X$ from sections of $K_S^{\ot m}$. 
\bigskip\\
Let $\omega \in H^0(S, K_S^{\ot m})$. The pushforward $\pi_*$ gives a section of $K_X^{\ot m}$ away from the singularities which can be pulled back to a section of $(K_{C_1 \times C_2}^{\ot m})^G$ defined outside $\Lambda$. Since $\Lambda$ is a set of points it has codimension $\ge 2$ so this extends to a section over $C_1 \times C_2$. Moreover, we can identify sections of $(K_{C_1 \times C_2})^G$ with $(S^{2m} \Omega_{C_1 \times C_2})^G$ which descend to a section of $S^{2m} \Omega_X$ over the smooth locus which pulls back to a section $\Gamma(\omega)$ of $S^{2m} \Omega_S$ defined outside of $E$. 
\begin{center}
\begin{tikzcd}
H^0(S, K_S^{\ot m}) \arrow[ddd, "\Gamma"] \arrow[rddd, "\Theta"] \arrow[r, "\varphi_*"] & H^0(X^{\sm}, K_{X}^{\ot m}) \arrow[r, "p^*"] & H^0(C_1 \times C_2 \sm \Lambda, K_{C_1 \times C_2}^{\ot m})^G \arrow[d]
\\
& & H^0(C_1 \times C_2, K_{C_1 \times C_2}^{\ot m})^G \arrow[d]
\\
& & H^0(C_1 \times C_2, \Omega_{C_1}^{\ot m} \ot \Omega_{C_2}^{\ot m})^G \arrow[d]
\\
H^0(S \sm E, S^{2m} \Omega_S) & H^0(X^{\sm}, S^{2m} \Omega_X) \arrow[l, "\varphi^*"] & H^0(C_1 \times C_2, S^{2m} \Omega_{C_1 \times C_2})^G \arrow[l, "p_*"] 
\end{tikzcd}
\end{center} 

\begin{prop}
If $\omega \in H^0(S, \struct{S}(m(K_S - E))$ then $\Gamma(\omega)$ naturally extends to a global section $\Gamma(\omega) \in H^0(S, S^{2m} \Omega_S)$.
\end{prop}

\begin{proof}
We get $\Theta(\omega) \in H^0(X^{\sm}, S^{2m} \Omega_X)$ which, by definition, can be written locally on a punctured disk around an $A_1$ singularity,
\[ a(z_1, z_2) \d{z_1}^m \d{z_2}^m \]
Using the change of coordinates $z_1 = \mu_1^{1/2}$ and $z_2 = \mu_1^{1/2} \mu_2$ given by $\pi$ we get the pullback by $\pi^*$ of $\Theta(\omega)$ which is $\Gamma(\omega)$ can be written locally as,
\[ \sum_{j = 0}^m {m \choose j} \frac{\mu_2^{m-j} (a \circ \pi)(\mu_1, \mu_2)}{2^{2m - j} \mu_1^{m-j}} \d{\mu_1}^{2m-j} \d{\mu_2}^j \]
but $a \circ \pi$ canishes along $E$ at multiplicity at least $m$ so this is well-defined on $S$. 
\end{proof}

\begin{rmk}
Notice that all of these forms are of type $\d{z_1}^m \d{z_2}^m$ and therefore the resultant of any two is zero. Therefore, we need to use methods beyond the resultant. 
\end{rmk}


\begin{prop}
The line bundle $\struct{S}(K_S - E)$ is big.
\end{prop}

\begin{proof}
Since $S$ is a minimal surface of general type, the canonical divisor $K_S$ is nef. Therefore, by the asymtotic Riemann-Roch theorem,
\[ h^0(S, K_S^{\ot m}) = \tfrac{1}{2} m^2 c_1(S)^2 + O(m) \]
but $c_1(S)^2 = 6$ so we have,
\[ h^0(S, K_S^{\ot m}) = 3 m^2 + O(m) \]
Thus, there exists a positive real number $M$ such that,
\[ 3 m^2 - M m \le h^0(S, K_S^{\ot m}) \]
for $m$ large enough.
\bigskip\\
On the other hand, let $\omega$ be a section of $K_S^{\ot m}$. The corresponding section on $C_1 \times C_2$ can be written locally, around a fixed point, as
\[ a(z_1, z_2) (\d{z_1} \wedge \d{z_2})^m \]
where $a$ is a holomorphic function,
\[ a(z_1, z_2) = \sum_{i,j} a_{ij} z_1^i z_2^j \]
Using change of coordinates $z_1 = \mu_1^{1/2}$ and $z_2 = \mu^{1/2} \mu_2$ given by $\pi$ at an $A_1$ singularity. Then $\omega$ vanishes along $E$, at least with multiplicity $m$ if $a_{i,j} = 0$ for all $i,j$ such that $i + j < 2m$. This gives $1 + 2 + \cdots + 2m$ sufficent conditions. However, the section is invariant by the $G$-action, then $a_{i,j} = 0$ for all $i + j$ odd since any function defined on the quotient must be invariant under the stabilizer $C_2$ acting via $(z_1, z_2) \mapsto (-z_1, -z_2)$. Therefore, 
\begin{align*}
h^0(S, \struct{S}(m(K_S - E))) & \ge h^0(S, K_S^{\ot m}) - \# \{ A_1 \text{ singularities} \} \frac{1 + 2 + \cdots + 2 m}{2} 
\\ 
& \ge (3 m ^2 - Mm) - (2m^2 + m) = m^2 - (M+1)m 
\end{align*}
\end{proof}

It looks like the innequality we need is $c_1^2 > 2 a_1$ where $a_1$ is the number of $a_1$ singularities and we only have $a_1$ singularities. {\color{red} COMPARE TO ROUSEAUX}


\section{Explicit Formulas for Bigness}

\subsection{Orbifold}

Let $\X \to (X, \Delta)$ be a 2-dimensional orbifold for which $\Delta = 0$ and the singularities are ADE. Let $Y \to X$ be the minimal desingularization.

\begin{prop}
The Chern numbers are $c_1^2(\X) = c_1^2(X) = c_1^2(Y)$ and,
\[ c_2(\X) = c_2(Y) - \sum (n+1) (a_n + d_n + e_n) + \sum \left( \frac{a_n}{n+1} + \frac{d_n}{4(n-2)} + \frac{e_6}{24} + \frac{e_7}{48} + \frac{e_8}{180} \right) \]
Where $a_n$ is the number of $A_n$ singularities etc.
\end{prop}

\begin{rmk}
Note these numbers are the orders of the group which acts to form the standard group quotient representation of these singularities.
\end{rmk}

\begin{theorem}[\chref{https://arxiv.org/pdf/1303.3377.pdf}{RR13}]
Suppose that $s_2(Y) + s_2(\X) > 0$ then $\Omega_Y$ is big.
\end{theorem}

\begin{cor}
If there are only $a_n$ singularities this number is,
\[ 2 c_1^2(Y) - 2 c_2(Y) + \sum \frac{n^2 + 2n}{n + 1} a_n \]
\end{cor}

\subsection{Local Euler Characteristic}

Again let $\pi : Y \to X$ be the resolution of a surface. Then we have,
\[ \chi(Y, S^m \Omega^1_Y) = \frac{1}{12} \left( 2 (K^2 - \chi) m^3 - 6 \chi m^2 - (K^2 + 3 \chi) m + K^2 + \chi \right) \]
where $K^2 = c_1(Y)$ and $\chi = c_2(Y)$. 

\begin{theorem}[\chref{https://arxiv.org/pdf/1912.08908.pdf}{Bruin1}]
Let $X$ be an irreducible complex projective surface whose singular locus $S$ is a finite set of isolated du Val singularities. Let $\tau : Y \to X$ be a minimal resolution. Then for $m \ge 3$ we have,
\[ h^0(Y, S^m \Omega_Y^1) \ge \chi(Y, S^m \Omega^1_Y) + \sum_{s \in S} \chi^1(s, S^m \Omega^1_Y) \]
and for all $m \ge 1$
\[ h^0(Y, S^m \Omega^1_Y) \ge h^0(X, \hat{S}^m \Omega^1_X) - \sum_{s \in S} \chi^0(s, S^m \Omega_Y^1) \]
\end{theorem}

\begin{theorem}
If $s \in S$ is an $A_1$ singularity then $\chi^0 \sim \frac{11}{108} m^3$ and $\chi^1 \sim \frac{4}{27} m^3$. See the paper for exact formulas. 
\end{theorem}


\subsection{Complete Intersections of Quadrics}

Let $X \subset \P^n$ be the complete intersection of $n - 2$ with $\ell$ isolated $A_1$ singularities. Let $Y \to X$ be the minimal resolution. In this case we have,
\[ K^2 = c_1(Y)^2 = (n - 5)^2 2^{n-2} \quad \chi = c_2(Y) = (n^2 - 7n + 16)n^{n-3} \]
which is general type for $n \ge 6$.  Thene we have,
\[ \chi(Y, S^m \Omega_Y^1) = \tfrac{1}{3} 2^{n-5} (2 (n^2 - 13 n + 34) m^3 - 6(n^2 - 7n + 16) m^2 - (5 n^2 - 41 n + 98) m + 3 n^2 - 27 + 66) \]


\section{Surfaces with Many Nodes}

\begin{enumerate}
\item Barth's sextic is NOT known to have big $\Omega$ I think NOT know to be quasi-hyperbolic {\color{red} I THINK BOGOMOLOV'S PAPER HAS THE MISTAKE THAT MAKES PEOPLE THINK IT HAS BIG OMEGA}

\item Barth decic and Sarti's surface both are known to have big $\Omega$.
\end{enumerate}


{\color{red} magic squares surface looks to be not supersingular in most characteristics via quick computer search}



\subsection{Surface of degre 5 with four $A_9$ singularities}


\section{Questions}

Does McQuillan's Theorem actually imply that if $\Omega$ is big then it cannot be uniruled in any characteristic of good reduction? In particular in finitely many characteristics? 


\section{Ideas}

Maybe the following works: for $H^1(\X, S^n \Omega_{\X})$ I can get the usual Euler characteristic estimate for the stack but also use the formulas for $H^0(X, \hat{S}^n \Omega_X)$ which should be the same as the differentials on the stack (right?) and therefore the difference gives me a bound on the $H^1$ term. 

\end{document}