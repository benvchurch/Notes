\documentclass[12]{article}

\usepackage{amsfonts}
\usepackage{amsmath}

\begin{document}
\section{The Tautological Bundle}

\newcommand{\C}{\mathbb{C}}
\renewcommand{\P}{\mathbb{P}}
\newcommand{\embed}{\hookrightarrow}
\newcommand{\Span}[1]{\mathrm{Span}\left( #1 \right)}

Consider the fibre bundle, $\pi : S^{2 n + 1} \to \P^n_{\C}$ given by consider ing $S^{2n +1} \subset \C^{n+1}$ and restricting the projection $\C^{n+1} \to \P^n_\C$. Then $\pi$ is a principal $S^1$-bundle. Consider the tautological representation $\rho : U(1) \to \mathrm{GL}_1(\C)$ which is the inclusion $U(1) \embed \C^\times$, which gives an associated line bundle $S^{2 n + 1} \times_\rho \C$. We call this the tautological bundle since its fibre above a point is the line in $\C^{n+1}$ which that point on $\P^n_\C$ corresponds to.
\bigskip\\
To see this explicitly, consider the following bundle,
\[ T = \{ (L, v) \mid L \in \P^n_\C \textrm{ and } v \in L \subset \C^{n+1} \} \subset \P^n_\C \times \C^{n+1} \]
with the projection $\pi : T \to \P^n_\C$ via $(L, v) \mapsto L$. I claim that this bundle is isomorphic to the tautological bundle constructed above. 
\bigskip\\
Consider the map $f : S^{2n + 1} \times_\rho \C \to T$ via $f : [x, \lambda] \mapsto ( \mathrm{Span}(x), \lambda x)$. This is clearly a bundle map since $\pi([x, \lambda]) = \pi(x) = \Span(x) = \pi(\Span{x}, \lambda x)$. Furthermore it is well-defined because $f([x, \mu \lambda]) = (\Span{x}, \mu \lambda x) = (\Span{\mu x}, \lambda \mu x) = f([\mu x, \lambda])$. We need to check that this map is injective and surjective. First, if $f([x, \lambda]) = f([y, \mu])$ then $\Span{x} = \Span{y}$ so $y = \gamma x$ for $\gamma \in \C^\times$ and $\lambda x = \mu y$ so $\lambda = \mu \gamma$ (since these vectors are nonzero) and thus,
\[ [x, \lambda] = [x, \gamma \mu] = [\gamma x, \mu] = [y, \mu] \]
For surjectivity note that given $(L, v)$ with $v \in L$ then $L = \Span{x}$ for $x \in S^{2n + 1}$ and $v = \lambda x$ with $\lambda \in \C$ since $L$ is a line. Thus $f([x, \lambda]) = (L, v)$. 
\bigskip\\
The tautological bundle has no nonzero (holomorphic) global sections.   However, there are $n+1$ independent glboal sections of its dual. To see this consder the global $\mathrm{Hom}(T, \mathcal{O}_\P)$. There exist $n+1$ idependent functions defined by the $n+1$ projections $p_k : \C^{n+1} \to \C$ via the construction, 
\[ T \embed \mathcal{O}^{n+1}_\P = \P^n_\C \times \C^{n+1} \xrightarrow{p_k} \P^n_\C \times \C = \mathcal{O}_\P \]
These sections are refered to as $X_k$ ,the $k^{\mathrm{th}}$ coordinate function on $\P^n_\C$.
\bigskip\\
Producing the coordinate functions $X_k$ as sections of the dual $X^\vee$ identifies the tautological bundle $T$ with the algebraic twist $\mathcal{O}_\P(-1)$ and thus its dual is the Serre twisting sheaf $T^\vee = \mathcal{O}_\P(1)$. 

\section{MATH 275A 2021 Lecture 2}

\newcommand{\ket}[1]{\left| #1 \right>}
\newcommand{\bra}[1]{\left< #1 \right|}

Using the Stern-Gerlach boxes we define spin operators $\hat{S}_i$ on our Hilbert space $H = \C^2$. These have eigenstate $\ker{\pm}$ along each axis. Furthermore, we have a Hamiltonian $\hat{H}$. For a constant magnetic field, up to a constant,
\[ \hat{H} = \hat{S} \cdot \vec{B} \]
For $B$ along the $z$-direction,
\[ \hat{H} = \hat{S}_z B \]
Then the evolution follows the Schrodinger equation,
\[ i \partial_t \ket{\psi} = \hat{H} \ket{\psi} \]
For any observable (i.e. operator $\hat{A}$) we can define the expected value,
\[ \left< \hat{A} \right>_\psi = \bra{\psi} \hat{A} \ket{\psi} \]
Then,
\[ i \partial_t \left< \hat{A} \right>_\psi= \left< [\hat{A}, \hat{H}] \right>_{\psi} \]
Now for example, we choose $\ket{\psi(0)} = \ket{+_x}$. Then we expand,
\[ \ket{\psi(0)} = \frac{1}{\sqrt{2}} \left( \ket{+_z} + \ket{-_z} \right) \]
Then applying the evolution operator,
\[ \ket{\psi(t)} = e^{-i H t} \frac{1}{\sqrt{2}} \left( \ket{+_z} + \ket{-_z} \right) = \frac{1}{\sqrt{2}} \left( e^{-i \frac{B}{2}t} \ket{+_z} + e^{i \frac{B}{2}t} \ket{-_z} \right) \]
Now we consider,
\[ i \partial_t \left< \hat{S}_x \right> = \bra{\psi} \hat{S}_x \ket{\psi} = \left< [\hat{S}_x, \hat{H}] \right> = B \left< [\hat{S}_x, \hat{S}_z] \right> = - i B \hat{S}_y \]
and therefore,
\[ \partial_t \left< \hat{S}_x \right> = - B \left< \hat{S}_y \right> \]
Likewise,
\[ \partial_t \left< \hat{S}_y \right> = B \left< \hat{S}_x \right> \]
This coupled system has solution,
\[ \left< \hat{S}_x \right> = \cos{(B t)} \quad \text{and} \quad \left< \hat{S}_y \right> = \sin{(B t)} \]

\subsubsection{Operators}

\newcommand{\R}{\mathbb{R}}

Infinite dimensional space $H = L^2(\R) = \{ f : \R \to \C \mid \int |f|^2 < \infty \}$. We take observables to be ``self-adjoint'' operators on $H = L^2(\R)$. For example, $\hat{x} = x \cdot$ and $\hat{p} = - \partial_x$. However, the eigenfunctions of these operators are not $L^2$ they are tempered distributions. We say,
\[ \left< \frac{1}{\sqrt{2\pi}} e^{i p x} \middle| \frac{1}{\sqrt{2\pi}} e^{i q x} \right> = \delta(p - q) \]

\subsubsection{Uncertainty Principle}
Define,
\[ \Delta \hat{A} = \hat{A} = - \left< \hat{x} \right> I \]
and likewise for $B$ two self-adjoint operators $A, B$. Then,
\[ \left< (\Delta \hat{x})^2 \right>_\psi \left< (\Delta \hat{p})^2 \right>_\psi \ge \frac{1}{4} \left| \bra{\psi} [\hat{A}, \hat{B}] \ket{\psi} \right|^2 \]
\bigskip\\
For example,
\[ [\hat{x}, \hat{p}] = \hat{x} \hat{p} - \hat{p} \hat{x} = i I \]
because,
\[ (\hat{x} \hat{p} - \hat{p} \hat{x}) \psi = x (-i \partial_x \psi) + i \partial_x (x \psi) = - i \partial_x \psi + i \psi + x \partial_x \psi = i \psi \]
Therefore,
\[ \sigma_x^2 \sigma_p^2 \ge \frac{1}{4} \]

\subsubsection{Angular Momentum}

Classical angular momentum $\vec{L} = \vec{x} \times \vec{p}$. We upgrade these to quantum self-adjoint operators. Thus we get, for example,
\[ \hat{L}_z = - i (x \partial_y - y \partial_x) \]
Then $L^2 = L_x^2 + L_y^2 + L_z^2$.  

\end{document}
