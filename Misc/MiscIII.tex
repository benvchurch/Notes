\documentclass[12]{article}

\usepackage{amsfonts}
\usepackage{amsmath}

\begin{document}
\section{The Tautological Bundle}

\newcommand{\C}{\mathbb{C}}
\renewcommand{\P}{\mathbb{P}}
\newcommand{\embed}{\hookrightarrow}
\newcommand{\Span}[1]{\mathrm{Span}\left( #1 \right)}

Consider the fibre bundle, $\pi : S^{2 n + 1} \to \P^n_{\C}$ given by consider ing $S^{2n +1} \subset \C^{n+1}$ and restricting the projection $\C^{n+1} \to \P^n_\C$. Then $\pi$ is a principal $S^1$-bundle. Consider the tautological representation $\rho : U(1) \to \mathrm{GL}_1(\C)$ which is the inclusion $U(1) \embed \C^\times$, which gives an associated line bundle $S^{2 n + 1} \times_\rho \C$. We call this the tautological bundle since its fibre above a point is the line in $\C^{n+1}$ which that point on $\P^n_\C$ corresponds to.
\bigskip\\
To see this explicitly, consider the following bundle,
\[ T = \{ (L, v) \mid L \in \P^n_\C \textrm{ and } v \in L \subset \C^{n+1} \} \subset \P^n_\C \times \C^{n+1} \]
with the projection $\pi : T \to \P^n_\C$ via $(L, v) \mapsto L$. I claim that this bundle is isomorphic to the tautological bundle constructed above. 
\bigskip\\
Consider the map $f : S^{2n + 1} \times_\rho \C \to T$ via $f : [x, \lambda] \mapsto ( \mathrm{Span}(x), \lambda x)$. This is clearly a bundle map since $\pi([x, \lambda]) = \pi(x) = \Span(x) = \pi(\Span{x}, \lambda x)$. Furthermore it is well-defined because $f([x, \mu \lambda]) = (\Span{x}, \mu \lambda x) = (\Span{\mu x}, \lambda \mu x) = f([\mu x, \lambda])$. We need to check that this map is injective and surjective. First, if $f([x, \lambda]) = f([y, \mu])$ then $\Span{x} = \Span{y}$ so $y = \gamma x$ for $\gamma \in \C^\times$ and $\lambda x = \mu y$ so $\lambda = \mu \gamma$ (since these vectors are nonzero) and thus,
\[ [x, \lambda] = [x, \gamma \mu] = [\gamma x, \mu] = [y, \mu] \]
For surjectivity note that given $(L, v)$ with $v \in L$ then $L = \Span{x}$ for $x \in S^{2n + 1}$ and $v = \lambda x$ with $\lambda \in \C$ since $L$ is a line. Thus $f([x, \lambda]) = (L, v)$. 
\bigskip\\
The tautological bundle has no nonzero (holomorphic) global sections.   However, there are $n+1$ independent glboal sections of its dual. To see this consder the global $\mathrm{Hom}(T, \mathcal{O}_\P)$. There exist $n+1$ idependent functions defined by the $n+1$ projections $p_k : \C^{n+1} \to \C$ via the construction, 
\[ T \embed \mathcal{O}^{n+1}_\P = \P^n_\C \times \C^{n+1} \xrightarrow{p_k} \P^n_\C \times \C = \mathcal{O}_\P \]
These sections are refered to as $X_k$ ,the $k^{\mathrm{th}}$ coordinate function on $\P^n_\C$.
\bigskip\\
Producing the coordinate functions $X_k$ as sections of the dual $X^\vee$ identifies the tautological bundle $T$ with the algebraic twist $\mathcal{O}_\P(-1)$ and thus its dual is the Serre twisting sheaf $T^\vee = \mathcal{O}_\P(1)$. 

\end{document}
