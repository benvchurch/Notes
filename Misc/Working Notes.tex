\documentclass[12pt]{article}
\usepackage{hyperref}
\hypersetup{
    colorlinks=true,
    linkcolor=blue,
    filecolor=magenta,      
    urlcolor=blue,
}

\usepackage{import}
\import{"../Algebraic Geometry/"}{AlgGeoCommands}

\newcommand{\Loc}[1]{\mathfrak{Loc}\left( #1 \right)}
\newcommand{\AbGrp}{\mathbf{AbGrp}}
\usepackage{bbm}


\begin{document}

\section{TODO!!}

\begin{enumerate}
\item Finish symplectic geometry course
\begin{enumerate}
\item figure out if symplectic toric is the same as projective toric variety (projectivity needed to come from a polytope and also to be Kahler)
\item review coisotropic reduced and write some notes
\item hyperkahler reduction examples 
\item are there examples of noncompact hyperkahlers?
\item work out the kinks in notes on hamiltonian actions
\end{enumerate}
\item review killing homotopy groups columbia lectures and write some notes
\item figure out those damn jet bundles and connections on principal bundles
\begin{enumerate}
\item RMK: $\pi^* E$ is NOT trivial for a vector bundle let alone a fiber bundle. it does get equiped with a canonical section but for a vector bundle this is just the trivial section, only for a principal bundle does giving a section trivialize it.
\item role of atiyah sequence vs jet bundle sequence
\item 
\end{enumerate}
\item spectral sequences for tor and ext in derived category (FIND MY NOTES ON THIS!)
\begin{enumerate}
\item application to universal coefficient theorem
\item Kunneth spectral sequence
\item Kunneth formula for smash product?
\item why are derived functors triangulated
\item derived functors in terms of Kan extensions (NOTES)
\end{enumerate}
\item write notes on universal morphisms
\item $G$-action of $X/Y$ induces map Descent data $X/Y$ to $G$-equivariant sheaves
\begin{enumerate}
\item isomorphism when $X/Y$ is a $G$-cover i.e. $X \to Y$ is a $G$-torsor
\item write down explicit $G$-equivariant structure on $\Omega_X$
\item Galois descent derive explicit form
\end{enumerate}
\item Weil restriction
\begin{enumerate}
\item write down trivialization after going back up
\item Galois descent in explicit form
\end{enumerate}
\item notes on Galois actions on schemes
\item notes on Frobenii
\item notes on universal constructions in math with examples

\item fix notes on Tor in category of sheaves and Tor symmetry (do I need symmetry of flat objects a priori?).

\item Finish stable homotopy theory course.

\item Finish vector bundles and connections notes (in AG folder)
\begin{enumerate}
\item Kahler iff $\nabla I = 0$ where $\nabla$ is the Levi-Civita connection
\item Ricci tensor and the trace bullshit 
\item Riemann-Hilbert and existence of flat frames for integrable connections
\end{enumerate}
\end{enumerate}

\section{What I Want to Think About}

\begin{enumerate}
\item Flat cohomology equal etale cohomology for smooth (affine groups) apply this to that counting rational points things
\item work out the details for the group fixing $\CC$ inside endomorphism group. What does an integrable structure of this kind look like, how close to a complex manifold can we get? In dimension two this should be exactly a conformal (not necessarily orientable) structure. 
\item FINISH CONFORMAL NOTES!
\item Hilbert Class Field of curves (ASK BRIAN FOR REFERENCE)
\item Read about Fredholm index and Riemann-Roch
\item Cohmology and inclusion-exclusion: cohomology for vectorspaces?
\end{enumerate}

\section{Some Questions I Have}

\begin{enumerate}
\item Reduction of structure group for a scheme.
\begin{enumerate}
\item what about the algebraic group $\mathrm{SL}^{\pm} = \det^{-1}(\mu)$ what does reduction of structure group give. For a manifold this is supposed to be a pseudo-volume form but obviously that's not right.
\item what about $\Res{\CC}{\RR}{\Gm} \embed \mathrm{GL}_2$ from the action $\Gm \acts \A^1_{\CC}$ restricted giving an action $\Res{\CC}{\RR} \Gm \acts \A^2_{\RR}$. I feel like this should give an almost complex structure. What properties does it have? What about for other fields?
\item What is an almost complex structure on a scheme look like?
\end{enumerate}
\item Is my calculation of an ``almost almost complex structure'' as reduction of structure group to $\left< \sigma \right> \ltimes \mathrm{GL}(n, \CC) \subset \mathrm{GL}(2n, \RR)$. For the case $n = 1$ this should be the conformal group justfying that I think this should correspond to the non-oriented case of a complex manifold since Riemann surfaces are exactly oriented conformal manifolds.
\end{enumerate}


\section{TEST}



\section{Introduction}

\newcommand{\cO}{\mathcal{O}}
\newcommand{\cN}{\mathcal{N}}

Let $X$ be a minimal variety of dimension $n$ over $\C$. Throughout this note, we will assume that there are $k$ linearly independent nowhere-vanishing holomorphic 1-forms on $X$. In other words, there exists a short exact sequence
\[ 0 \rightarrow \cO^{\oplus k} \rightarrow \Omega_{X}^{1} \rightarrow \F^{\vee} \rightarrow 0 \]
of vector bundles.

\begin{lemma}
The dual short exact sequence
\[ 0 \rightarrow \F \rightarrow T_{X} \rightarrow \cO_{X}^{\oplus k} \rightarrow 0 \]
gives a foliation.
\end{lemma}

\textbf{Q1.} When is $\F$ algebraic?

This is very much related to a paper of Campana and Peternell, where they ask this question in the context

\subsection{Foliations}

Let $X$ be an algebraic variety over $\CC$ or a complex manifold. We will usually assume $X$ projective and smooth so there will be no distinction between analytic and algebraic sheaves. On $X$ we define $\T_X = \Omega_X^\vee$ which is a reflexive sheaf but not necessarily a vector bundle if $X$ is singular. 

\begin{defn}
A \textit{(singular) foliation} on $X$ is a subsheaf $\F \subset T_X$ such that,
\begin{enumerate}
\item $[\F, \F] \subset \F$ 

\item $\F$ is \textit{saturated} meaning $\T_X / \F$ is torsion-free.
\end{enumerate}
We write $\cN_{\F} = \T_X / \F$ is the \textit{normal bundle} of the foliation and $r_\F = \rank{\F}$ is the \textit{rank} of the foliation. The \textit{canonical bundle} of $\F$ is $\omega_{\F} = \det{\F}^\vee$ and any Weil divisor $K_{\F}$ such that $\struct{X}(K_\F) = \omega_{\F}$ is called the \textit{canonical class}. 
\end{defn}

\begin{prop}
If $\F$ is a foliation then $\F$ is reflexive.
\end{prop}

\begin{proof}
This follows from $\T_X / \F$ being torsion-free by \chref{https://stacks.math.columbia.edu/tag/0EBG}{Tag 0EBG}.
\end{proof}

\begin{prop}
The map $\ell_\F : \F \ot \F \to \T_X / \F$ is $\struct{X}$-linear and,
\[ [\F, \F] \subset \F \iff \ell_\F = 0 \]
\end{prop}


\begin{proof}
Indeed, we just need to compute,
\[ [f X, Y] = f [X, Y] + Y(f) X \]
so if $X, Y$ are sections of $\F$ then $Y(f) X$ is zero in $\T_X / \F$ proving the claim.
\end{proof}

\renewcommand{\Sing}{\mathrm{Sing}}

We now define a bunch of notions of singular sets of a foliation.

\begin{defn}
The \textit{singular set} of a coherent subsheaf $\F \subset T_X$ is,
\[ \Sing(\F) = \{ x \in X \mid \F_x \to (\T_X)_x \text{ is not a free direct factor} \} \]
is union of the points where the rank of $\F$ jumps up and where the rank of $\F \to T_X$ jumps down. Explicitly, 
\[ S_1(\F) = \{ x \in X \mid \F_x \text{ is not free} \} \]
and likewise,
\[ S_2(\F) = \{ x \in X \mid \F(x) \to \T_X(x) \text{ is not injective} \} \]
\end{defn}

\begin{prop}

\end{prop}

\begin{defn}
A foliation $\F \subset \T_X$ is \textit{regular} if $\Sing{(\F)} = \empty$ or, equivalently, if $\F \embed \T_X$ is a sub-bundle (in the sense of locally being a free direct factor).
\end{defn}

\begin{rmk}
If $\T_X$ is a vector bundle then the following are equivalent,
\[ \F \text{ is regular} \iff \F \subset \T_X \text{ is a sub-bundle} \iff \cN_\F \text{ is a vector bundle} \iff \Sing{\F} = \empty \]
\end{rmk}

\begin{defn}
We say that $\F$ is $1$-\textit{Gorenstein} if $\omega_{\F}$ is a line bundle. 
\end{defn}


\begin{rmk}
Note that $\F$ is automatically reflexive so $\det{\F}$ is reflexive so $\det{\F}$ is a line bundle iff $\omega_\F$ is a line bundle. 
\end{rmk}

\begin{rmk}
If $X$ is regular then any $\F$ is automatically $1$-Gorenstein because rank $1$ reflexive sheaves on regular schemes are line bundles (this should follow from every Weil divisor being Cartier) it is in Hartshorne's Stable Reflexive Sheaves somewhere. 
\end{rmk}


\subsection{The Analytic Theory}

Recall the following theorem from real analysis,

\begin{theorem}[Frobenius]
Let $M$ be a smooth manifold and $\F$ a regular foliation. Then there is a collection $\{ L_\alpha \}_\alpha$ of connected injectively immersed smooth manifolds $L_\alpha \to M$ (not closed) such that,
\begin{enumerate}
\item $M$ is a disjoint union of the $L_\alpha$
\item for each $p \in M$ there is a chart $U_p \subset M$ such that $U_p \cap L_\alpha$ is a countable union of slices (there are coodinates $(x^1, \dots, x^n)$ such that the components are $x^{r+1} = c_1, \dots, x^n = c_{n-r}$)
\item for each $p \in L_\alpha$ we have $T_p L_\alpha = \F_p$ inside $T_M$.
\end{enumerate}
If $X$ is a complex manifold and $\F \subset T_X$ is a complex regular foliation then we can assume that the leaves $L_\alpha$ are immersed complex submanifolds. 
\end{theorem}

\begin{cor}[Lemma 2.6, OFF]
Let $(X, \F)$ be a $1$-Gorenstein foliated variety. Suppose that $\F$ is Pfaff-regular and locally free at a point $x \in X$ then there exists an analytic open $U$ of $x$ a complex analytic space $W$, and a smooth morphism $U \to W$ of relative dimension $r_{\F}$ such that $\F|_U = \T_{U/W}$. 
\end{cor}

\begin{defn}
Let $X$ be an algebraic variety. 
We say an immersed manifold $\iota : L \subset X$ is \textit{algebraic} if $\iota(L) \cap Z^{\text{sm}} \subset Z$ is (analytically) open where $Z$ is the Zariski closure of $\iota(L)$.
\end{defn}

\begin{rmk}
I choose this slightly strange condition to capture the following phenomenon. Let $\A^1 \to \A^2$ be $t \mapsto (t^2 - 1, t(t^2 - 1))$ whose image is the nodal curve $X$. This is an immersed complex manifold. However, the image of the open set $\overline{B_{1/2}(1)}^C$ is not open in $X$. We need to remove the node to get an open set.
\end{rmk}

\begin{rmk}
I think the definition of algebraic leaves in Campana 2021 (that the Zariski and topological closures coincide) is wrong. For example, it predicts that the dense irrational slope foliation on an abelian surface is algebraic since each leaf is topologically and hence Zariski dense. 
\end{rmk}

\begin{lemma}
Let $\iota : L \to X$ be an immersed submanifold such that $Z = \overline{\iota(L)}^{\text{Zar}}$ and $L$ have the same dimension. Then $L$ is algebraic.
\end{lemma}

\begin{proof}
Consider the map $\iota : \iota^{-1}(Z^{\text{sm}}) \to Z^{\text{sm}}$ is a local diffeomorphism of smooth manifolds since it is an immersion of manifolds of the same dimension and hence is open. 
\end{proof}

\begin{rmk}
Algebraicity of course implies that $L$ is a complex (immersed) submanifold.
\end{rmk}

\begin{lemma}
Let $\iota : L \to X$ be an immersed submanifold with $L$ connected. Then if $Z = \overline{\iota(L)}^{\text{Zar}}$ has the same dimension as $L$ then $Z$ is irreducible. 
\end{lemma}

\begin{proof}
We know $\iota : L \to Z$ is analysically open away from the singularities. However, $Z^{\text{sing}} \subset Z$ has codimension at least $2$ and thus $\iota^{-1}(Z^{\text{sing}})$ also has codimension at least $2$ so $L \sm \iota^{-1}(Z^{\text{sing}})$ is connected. Thus $\iota(L \sm \iota^{-1}(Z^\text{sing})) \subset Z \sm Z^{\text{sing}}$ so it must lie in some irreducible component (the irreducible components have become disconnected by removing the singularities). Since $L \sm \iota^{-1}(Z^{\text{sing}})$ is dense in $L$ then $\iota : L \to Z$ is contained in some irreducible component.  
\end{proof}

\begin{rmk}
This is false if we don't assume that $\dim{Z} = \dim{L}$. For example, there are embedded curves $\RR \to \A^3_{\CC}$ whose closure is the union of two planes. Ineed, consider a curve which wanders in the $xy$-plane before following the $x$-axis then smoothly transitions to wandering in the $xz$-plane. 
\end{rmk}


\begin{defn}
A foliation $\F$ on $X$ is \textit{algebraic} if the leaf of the regular foliation $\F|_{U_{\F}}$ on $U_{\F} = X \sm \Sing(\F)$ through a general point is algebraic.
\end{defn}

\begin{prop}
If $X$ is a smooth variety and $\F$ regular algebraic foliation on $X$ then every algebraic leaf is an \textit{embedded} submanifold which is the analytification of a smooth algebraic subvariety.
\end{prop}

\begin{proof}
It suffices to show that each leaf $L$ is Zariski closed. Let $Z$ be the Zariski closure of $L$. Choose $p \in X$ and an open $U$ such that $U \cap L$ is a union of slices. Since $Z$ is closed we may shrink $U$ so that $U \cap Z$ is connected. Since $Z$ is irreducible, $Z^{\text{sm}}$ is a connected embedded submanifold dense in $Z$. Then $L \cap Z^{\text{sm}} \cap U \subset Z^{\text{sm}} \cap U$ is open and its closure in $U$ is a union of slices but since $Z \cap U$ is connected of dimension equal to the dimension of the slices it cannot contain more than one. Hence $L \cap U$ is a single slice and is closed. Thus taking closures $L \cap U = Z \cap U$ so $L = Z$ and hence $L$ is smooth.
\end{proof}

\begin{rmk}
Without the algebraicity assumption, the leaves of $\F$ do not even need to be closed. For example, the irrational slope foliation on an abelian variety. 
\end{rmk}

\section{The Complex Geometric Picture}

\begin{prop}
There is a covariant equivalence of categories,
\[ \{ \Z\text{-Hodge Structures of type } (1,0) \oplus (0,1) \} \iff \{ \text{complex Tori} \} \]
which specializes to,
\[ \{ \text{polarized } \Z\text{-Hodge Structures of type } (1,0) \oplus (0,1) \} \iff \{ \text{abelian varities} \} \] 
\end{prop}

\begin{rmk}
If the polarization is required to be principal then the corresponding abelian varitety is princiaplly polarized. 
\end{rmk}

\begin{cor}
Nonconstant morphisms $f : X \to A$ to a simple abelian variety (considered up to translation and isogeny) correspond to irreducble sub-$\Q$-Hodge structures of $H^1(X, \Z)$.
\end{cor}

\newcommand{\Alb}{\mathrm{Alb}}

\begin{proof}
Morphisms $f : X \to A$ sending a fixed base point $x_0 \in X$ to $0 \in A$ are equivalent to homomorphisms $\mathrm{Alb}_X \to A$ which correspond to maps of Hodge structures $H^1(A, \Z) \to H^1(\Alb_X, \Z) = H^1(X, \Z)$. Considered in the isogeny category, these correspond to maps $H^1(A, \Q) \to H^1(X, \Q)$ and since $A$ is simple $H^1(A, \Q)$ is irreducible so the map is either zero or injective.
\end{proof}

The smoothness of the morphism $f : X \to A$ is not in question. Smoothness is equvalent to $f^* \Omega_A \to \Omega_X$ being a subbundle (of the correct dimension) i.e. if $\omega_1, \dots, \omega_g \in H^0(A, \Omega_A)$ is a basis of holomorphic $1$-forms then $f^* \omega_1, \dots, f^* \omega_g \in H^0(X, \Omega_X)$ should form a partial frame (meaning they are everywhere independent). 


\subsection{Maps to Circles}

\begin{prop}
Let $M$ be a compact smooth manifold and $\omega \in M$ a closed nonvalishing $1$-form. For any $\epsilon > 0$ there exists a submersive map $f_\epsilon : M \to S^1$ and an integer $n_\epsilon$ such that,
\[ || \omega - n_\epsilon^{-1} f_\epsilon^* \d{t} || < \epsilon \]
in the $L^\infty$ norm.
\end{prop}

\begin{proof}
Because $S^1$ is a $K(\Z, 1)$ there are continuous maps $f_i : M \to S^1$ such that $f_i^* [S_1]$ form a basis of $H^1(X, \Z)$. By [Prop. 17.8 of Bott, Tu, Differential Forms in Algebraic Topology] up to homotopy, we may choose the $f_i$ smooth. By naturality of the de Rham comparision theorem, 
\[ \eta_i = f^* \d{t} \] 
form a basis of $H^1_{\dR}(X)$. Thus we can write,
\[ \omega = \sum_{i} \alpha_i \eta_i + \d{g} \]
The idea is to rationally approximate the numbers $\alpha_i \in \RR$. Indeed, we can choose rational numbers $\frac{a_i}{n_\epsilon} \in \QQ$ such that if let let,
\[ \tilde{\omega} = \sum_i \frac{a_i}{n_\epsilon} \eta_i  + \d{g} \]
then we get,
\[ || \omega - \tilde{\omega} || = \left| \left| \sum_i \left( \alpha_i - \frac{a_i}{n_\epsilon}  \right) \eta_i \right| \right| < \epsilon \]
this requires choosing the rational apprxomation on the order of $\frac{\epsilon}{\mathrm{Vol}(M)}$. The let,
\[ f_\epsilon = \left( \prod_i f_i^{a_i} \right) \cdot (\Pi \circ g)^{n_\epsilon} \]
where $\Pi : \RR \to S^1$ is the universal cover. Therefore, 
\[ f_\epsilon^* \d{t} = \sum_i a_i f_i^* \d{t} + n_\epsilon \d{g} = \sum_i a_i \eta_i + n_\epsilon \d{g} = n_\epsilon \tilde{\omega} \]
proving the required inequality. Finally, for sufficiently small $\epsilon$, since $\omega$ is nonvanishing we see that $\tilde{\omega}$ is also nonvanishing so $f_\epsilon$ is smooth. 
\end{proof}

\begin{rmk}
Let's see what happens when we try to do this for a holomorphic $1$-form $\omega \in H^0(X, \Omega_X)$. Write $\omega = \omega_1 + i \omega_2$ into its real an imaginary parts. Note that because $\omega$ is holomorphic $\d{\omega} = 0$ (this requires $X$ compact Kahler). Therefore, we can rationally approximate, $\tilde{\omega}_1$ and $\tilde{\omega}_2$ to get a submersive (because for small enough $\epsilon$ we can ensure that $\tilde{\omega}_1$ and $\tilde{\omega}_2$ are everywhere independent) map $f : X \to S^2$ with $f^* \d{z} = n \tilde{\omega}$ with $\tilde{\omega} = \tilde{\omega}_1 + i \tilde{\omega}_2$ but there is no reason that $\tilde{\omega}_2$ should be holomorphic. Suppose we could approximate $\omega$ by a rational form which is holomorphic. This is exactly a $\Q$-Hodge submodule of $H^1(X, \Q)$ of rank $1$. Therefore, we are in the buisness of showing that $H^1(X, \Q)$ is a reducible Hodge structure.  
\end{rmk}


\section{Morphisms}

Note: $\Omega_X$ on a normal variety is often not torsion-free.

\newcommand{\sat}{\mathrm{sat}}

\begin{prop}
Let $f : X \to Y$ be a dominant morphism of normal varities Then the subsheaf $\T_f = \ker{(T_X \to f* T_Y)}$ is a (possibly singular) foliation.
\end{prop}

(SHIT DOES THIS EVEN MAKE SENSE IF PULLBACK IS NOT INJECTIVE)

\begin{proof}
Consider the sequence,
\begin{center}
\begin{tikzcd}
f^* \Omega_Y \arrow[r] & \Omega_X \arrow[r] & \Omega_f \arrow[r] & 0
\end{tikzcd}
\end{center}
then the left-exactness (as a functor on the opposite category) of $\Hom{\struct{X}}{-}{\struct{X}}$ gives a sequence,
\begin{center}
\begin{tikzcd}
0 \arrow[r] & \T_f \arrow[r] & \T_X \arrow[r] & f^* \T_Y
\end{tikzcd}
\end{center}
where we define $\T_f = \Omega_f^\vee$. Note that these sheaves are reflexive\footnote{See \chref{https://stacks.math.columbia.edu/tag/0AY4}{Tag 0AY4}}) and that $\T_X / \T_f \embed f^* \T_Y$ which is reflexive and hence torsion-free. Therefore, it suffices to show that $\T_f$ is closed under Lie bracket. For two local sections $\xi, \eta$ of $\T_f$ then,
\[ [\xi, \eta] \circ \d = (\xi \circ \d)(\eta \circ \d) - (\eta \circ \d)(\xi \circ \d) \]
as derivations. This acts trivially on $f^{-1} \struct{Y}$ since $\xi \circ \d$ and $\eta \circ \d$ both act trivially by definition. Hence $[\xi, \eta] \circ \d$ factors through $\Omega_f$ menaing $[\xi, \eta]$ arises from a section of $\T_f$. 
\end{proof}

\subsection{Pfaff Fields and Leaves}


\begin{defn}
Let $X$ be a variety. A \textit{A Pfaff field of rank} $r$ is a nonzero map $\eta : \Omega^r_X \to \L$ where $\L \in \Pic{X}$. The singular locus $S$ of $\eta$ is the vanishing locus of $\eta$ which is defined by the ideal $\I_S = \im{(\Omega_X^r \ot \L^\vee \to \struct{X})}$.  
\end{defn}

\begin{defn}
A closed subscheme $Z \subset X$ is \textit{invariant} under a Pfaff field $\eta$ is,
\begin{enumerate}
\item no irredcible component of $Y$ is contained in the singular locus of $\eta$

\item the restriction $\eta|_Y : \Omega_X^r|_Y \to \L|_Y$ factors as,
\begin{center}
\begin{tikzcd}
\Omega_X^r|_Y \arrow[r, "\eta|_Y"] \arrow[d] & \L|_Y 
\\
\Omega^r_Y \arrow[ru]
\end{tikzcd}
\end{center}
\end{enumerate}
\end{defn}

\begin{defn}
If $\F$ is a 1-Gorenstein foliation of rank $r$ then there is an associated Pfaff field,
\[ \eta_\F : \Omega^r_X \to \wedge \T_X^\vee \to \wedge^r \F^\vee = \det{\F}^\vee = \omega_\F \]
\end{defn}

\begin{defn}
We say that a Pfaff field $\eta$ is \textit{regular} if $\Sing(\eta) = \empty$ and a $1$-Gorenstein foliation is \textit{Pfaff-regular} if $\Sing(\eta_\F) = \empty$.
\end{defn}

\subsection{Relations between notions of singularities}

\begin{lemma}
Let $X$ be a smooth variety, $\F$ a foliation of rank $r$. Then,
\[ S(\eta_{\F}) \subset S_1(\F) \cup S_2(\F) = \Sing(\F) \]
and,
\[ \Sing(\eta_\F) \sm S_1(\F) = S_2(\F) \]
\end{lemma}

\begin{proof}
If $\F_x \to (\T_X)_x$ is a direct factor then $\F(x) \to \T_X(x)$ is clearly injective (the exact sequence is split) so $S_1(\F) \cup S_2(\F) \subset \Sing(\F)$. Moreover, since $X$ is smooth $(\T_X)_x$ is free so if $\F_x$ is also free then I claim if $\F(x) \to \T_X(x)$ is injective then it is a direct factor. Consider $\T_X / \F$. By upper-semi-continuity, the locus where this has rank $n - r$ is open. By the injectivity, $x$ in this locus and hence $\T_X / \F$ is constant rank on a neighborhood of $x$. Since $X$ is reducedm $\T_X / \F$ is locally free at $x$ hence $(\T_X / \F)_x$ is free so the sequence splits proving the claim. 
\bigskip\\
Now we need to consider singularities of the Pfaff field. Let $x \in S_1(\F)^c$ then choose an open $U$ on which $\F$ and hence $\omega_\F$ is locally free. Therefore,
\[ x \in \Sing(\eta_\F) \iff x \in x \in S_2(\F) \]
since the map $\F(x) \to \T_X(x)$ is injective iff the map $\Omega_X^r(X) \to \det{\F^\vee(x)}$ is nonzero. 
\end{proof}

\subsection{Pullbacks of Foliations (WIP)}

\begin{defn}
Let $U \subset X$ be open. We say $U$ is \textit{big} if $\codim{U^C}{X} \ge 2$.
\end{defn}

\begin{defn}
Let $f : X \to Y$ be a dominant separable map of varieties and $\F \subset \T_Y$ a foliation. Then we can form a foliation $f^{\#} \F$ by taking the pullback,
\begin{center}
\begin{tikzcd}
\cP \arrow[r] \arrow[d] \pullback & f^* \F \arrow[d] 
\\
\T_X \arrow[r] & f^* \T_Y
\end{tikzcd}
\end{center}
and then taking the saturation of the image, $f^{\#} \F = (\im{(\cP \to \T_X)})^\sat$.
\end{defn}

\begin{prop}
If $\F \subset \T_X$ is closed under lie bracket then $\F^\sat$ is a foliaiton.
\end{prop}

\begin{proof}
It suffices to check that $\F^\sat$ is also closed under the Lie bracket. Ineed, 
\[ \F^\sat \otimes \F^\sat \to \T_X / \F^\sat = (\T_X / \F)_{\text{tors-free}} \]
but there is a dense open $U$ on which $\F|_U = \F^\sat|_U$ and on $U$ this map is zero so because the target is torsion-free the map must be zero (since the image of the map is torsion). 
\end{proof}

WHAT IS RELATIONSHIP BETWEEN PULLBACK FOLIATION AND PFAFF FIELD

\subsection{Existence of a Morphism}

\begin{lemma} \label{leaf-pfaff}
Let $X$ be a smooth variety and $\F$ a rank $r$ foliation. Let $Z \subset X$ be an irreducible subvariety of dimension $r_\F$ such that $Z \not\subset \Sing(\F)$. Then,
\[ Z \sm \Sing(\F) \text{ is a leaf of } \F|_{X \sm \Sing(\F)} \iff Z \text{ is invariant under } \eta_{\F} \]
\end{lemma}

\begin{proof}
Let $x \in Y \sm \Sing(\F)$ be a smooth point of $Y$ and $v_1, \dots, v_r$ a local frame of $\F$. Then the map,
\[ \eta|_U : \Omega_X^r|_U \to \omega_\F|_U \]
is given by,
\[ \alpha \mapsto \alpha(v_1, \dots, v_r) \omega \]
where $\omega \in H^0(U, \omega_\F)$ is a generator such that $\omega(v_1, \dots, v_r) = 1$. Since $\omega_\F$ is torsion-free, to show that $Y$ is invariant under $\eta_{\F}$ it suffices to show that it kills the kernel of $\Omega^r_U|_Y \to \Omega^r_Y$ at the dense set of smooth points of $Y$. Therefore, $Y$ is invariant under $\eta$ if and only if any $f \in H^0(U, \struct{U})$ vanishing on $Y \cap U$ and any local $(r-1)$-form $\beta$ has $(\d{f} \wedge \beta)(v_1, \dots, v_r) = 0$ (this generates the kernel of $\Omega_X^r|_Y \to \Omega^r_Y$) which happens exactly if $\d{f}(v_i) = 0$ for each $i$ meaning $Y$ is tangent to $\F$ at $x$. Since the smooth points of $Y$ are dense and $Y$ is irreducible this means that $Y$ is the closure of a leaf of $\F|_{X \sm \Sing(\F)}$ iff it is invariant under $\eta$. 
\end{proof}

\newcommand{\Chow}{\mathrm{Chow}}

Now consider the following situation. Let $X$ be a normal projective vareity and $\F$ an algebraically integrable foliation on $X$. Then let $W \subset \Chow(X)$ be the closure of the set of $\CC$-points corresponding to the closures of the algebraic leaves of $\F$. Then let $U \subset X \times W$ be the universal cycle giving a diagram,
\begin{center}
\begin{tikzcd}
U \arrow[r, "e"] \arrow[d, "\pi"] & X 
\\
W
\end{tikzcd}
\end{center}
Notice that $e$ is dominant since by assumption a general point of $X$ is contained in an algebraic leaf of $\F$. 

\begin{rmk}
Why in OFF do they choose the component $W$ such that the universal cycle dominates $X$ isn't this immediate as I said above?
\end{rmk}

\begin{rmk}
Notice that Pfaff fields and foliations don't usually pullback well (the intution is that sections of $\Omega$ pullback nicely not dual sections). However, in this case there is the following trick using the product structure.
\end{rmk}

\begin{defn}
Given a $1$-Gorenstein foliation, $\F \subset \T_X$ and Pfaff field $\eta_\F : \Omega_X^r \to \omega_\F$ define,
\[ \eta_{W \times X} : \Omega^r_{W \times X} = \wedge^r (\pi_1^* \Omega_W \oplus \pi_2^* \Omega_X) \to \wedge^r (\pi_2^* \Omega_X) = \pi_2^* \Omega^r_X \xrightarrow{\pi_2^* \eta_\F} \pi_2^* \omega_{\F} \]
and likewise we can define a foliation,
\[ \F_{W \times X} = \pi_2^* \F \subset \pi_2^* \T_X \subset \T_{W \times X} \]
Indeed, $\eta_{W \times X} = \eta_{\F_{W \times X}}$ since, 
\[ \Omega_{W \times X}^r \to \det{\F_{W \times X}^\vee} \]
factors through $\Omega_{W \times X}^r \to \pi_2^* \Omega_X^r$. 
\end{defn}

\begin{lemma}
The Pfaff field $\eta_{W \times X}|_U : \Omega_{W \times X}^r|_U \to e^* \omega_{\F}$ factors through $\Omega^r_{W \times X}|_U \onto \Omega^r_{U/W}$ giving a Pfaff field $\eta_U : \Omega^r_{U/W} \to e^* \omega_{\F}$. Similarly, the map $\Omega_{W \times X}|_U \to \pi_2^* \F^\vee$ (note I was careful here about notation since it need not be the case that $\pi_2^* \F^\vee = (\pi_2 \F)^\vee$) factors as,
\[ \Omega_{W \times X}|_U \onto \Omega_{U/W} \to (e^* \F^\vee)_{\text{tors-free}} \]
\end{lemma}

\begin{proof}
Let $K = \ker{(\Omega^r_{W \times X}|_U \to \Omega^r_{U/W})}$ then we know that $K \to \Omega^r_{W \times X}|_U \to e^* \omega_{\F}$ vanishes on the dense set of fibers which are leaves (with the reduced structure) by Lemma \href{leaf_pfaff}. Since $e^* \omega_{\F}$ is torsion-free this map is zero.
\bigskip\\
Similarly, let $K = \ker{(\Omega_{W \times X}|_U \to \Omega_{U/W})}$ and consider the map $K \to \Omega_{W \times X}|_U \to e^* \F^\vee$. As before, this map vanishes on the set of fibers which are leaves (with the reduced structure). Since $e$ is not flat, it is posible for $e^* \F^\vee$ to not be torsion-free. Passing to the torsion-free quotient then the map must be zero since it is zero on a dense set. 
\end{proof}


\begin{lemma}
Let $f : X \to Y$ be a dominant map of integral characteristic zero schemes. Consider the reflexive hull $\Omega_{X/Y} \to \Omega_{X/Y}^{\vee\vee}$. Then for any fiber, the map, $\Omega_{X_y}|_{(X_y)_{\red}} \to \Omega_{(X_y)_{\red}}$ factors as,
\begin{center}
\begin{tikzcd}
\Omega_{X_y}|_{(X_y)_{\red}} \arrow[r, two heads] \arrow[rd, two heads] & \Omega_{(X_y)_{\red}} \arrow[d, two heads]
\\
& (\Omega_{X/Y})^{\vee \vee}|_{(X_y)_\red}
\end{tikzcd}
\end{center}
\end{lemma}

\begin{proof}
Let $\cN_y$ be the conormal bundle of $(X_y)_\red \embed X_y$ i.e. the sheaf of nilpotents. It suffices to show that $\cN_y \to (\Omega_{X/Y})_{\text{tors-free}} |_{X_y}$ is zero. This is an affine local question so reduce to an injective map of domains $\varphi : R \to A$ and localizing we may assume that $R$ is local. Let $I = \m_R A$ and $J = \sqrt{I}$. We need to show that $J/J^2 \to \Omega_{A/R}^{\vee \vee} \ot (R / J)$ is zero. This is equivalent to showing: for all $f \in J$ and $\ell \in \Hom{A}{\Omega_{A/R}}{A}$ that $\ell(f) \in J$. Now $f^n \in I$ then I claim that for all $0 \le k \le n$,
\[ f^{n-k} \ell(\d{f})^{2k} \in I \]
The case $k = 0$ is obvious. For induction, notice that $\d{I} \subset I \Omega_{A/R}$ since $\d{\m_R} = 0$ and thus,
\[ \d{(f^{n-k} \ell(\d{f})^{2k})} = (n-k) f^{n-k-1} \d{f} \ell(\d{f})^{2k} + (2k) f^{n-k} \ell(\d{f})^{2k-1} \d{\ell(\d{f})} \in I \Omega_{A/R} \]
then applying $\ell$ we see,
\[ (n-k) f^{n-k-1} \ell(\d{f})^{2k+1} + (2k) f^{n-k} \ell(\d{f})^{2k-1} \ell(\d{\ell(\d{f})}) \in I \]
multiplying by $\ell(\d{f})$ preserves $I$ so we get,
\[ (n-k) f^{n-k-1} \ell(\d{f})^{2(k+1)} + (2k) f^{n-k} \ell(\d{f})^{2k} \ell(\d{\ell(\d{f})}) \in I \]
but by the induction hypothesis the second term lies in $I$ and hence,
\[ (n-k) f^{n-k-1} \ell(\d{f})^{2(k+1)} \in I \]
so if $n > k$ the claim is proved by induction (using characteristic zero).
\end{proof}

\begin{lemma}
Each fiber of $\pi : U \to W$ is connected and pure dimension $r_\F$. Fibers with no irreducible component contained in $S_1(\F)$ are irreducible and $e(\pi^{-1}(w))_{\red}$ is the closure of a leaf\footnote{Recall that $e$ is a closed immersion when restricted to any fiber since $U \subset W \times X$ is a closed immersion}.
\end{lemma}

(I THINK THERES A CONFLATION OF $(e^* \F^\vee)$ and $(e^* \F)^\vee$ FIGURE THIS OUT)

\begin{proof}
First, note that every cycle in a connected component of $\Chow(X)$ whose general fiber is geometrically connected and equidimensional is connected and equidimensional. To see this, note that a flat limit of a proper geometrically connected equidimensional scheme is (geometrically) connected and equidimensional so this follows from [Kollar, RCAV, Cor 3.16]. Therefore, it suffices to show that each irreducible component of $\pi^{-1}(w)$ is a leaf. Indeed, leafs are disjoint so this implies that $e(\pi^{-1}(w))$ with the reduced structure is a leaf.
\bigskip\\
We apply Lemma \ref{leaf-pfaff} using that $\eta_U : \Omega_{U/W}^r \to e^* \omega_{\F}$ factors the map $e^* \Omega_X^r \to e^* \omega_{\F}$. Restricting to an irreducible component $Z \subset F = \pi^{-1}(w)$ and recalling that $e : F \to X$ is a closed immersion we know that $\Omega_{X}^r|_Z \to \omega_\F |_Z$ factors through $\Omega^r_{F}|_Z \to \omega_\F|_Z$. If $F$ is reduced at the generic point of $Z$ then $\Omega^r_{F}|_Z \to \Omega^r_Z$ has torsion kernel so the Pfaff field factors through $\Omega^r_Z \to \omega_\F|_Z$ proving that $Z$ is a leaf. Thus we need to show that $F$ is reduced at the generic point of each component not contained in $S_1(\F)$. Let $Z$ be such a component and $\xi$ its generic point. Since $\F$ is a vector bundle generically on $Z$ we know that $e^* \F$ is reflexive on a neighborhood of $\xi$ (CHECK THIS). Therefore, $\Omega_{U/W} \to e^* \F^\vee$ factors through $\Omega_{U/W}^{\vee \vee}$. Therefore, by the lemma, restricting to $Z$ we see that,
\begin{center}
\begin{tikzcd}
\Omega_{U_w}|_{Z} \arrow[r, two heads] \arrow[rd, two heads] & \Omega_{Z} \arrow[d, two heads] \arrow[r, dashed] & \F^\vee|_Z
\\
& (\Omega_{U/W})^{\vee \vee}|_{Z} \arrow[ru]
\end{tikzcd}
\end{center}
proving that $Z$ is a leaf.
\end{proof}

\begin{cor}
If $X$ is smooth and $\F$ is a regular foliation then every fiber of $\pi : U \to W$ is a leaf with the reduced structure. 
\end{cor}

\begin{theorem}
Let $\F \subset T_X$ be an algebraic (regular) foliation of rank $r$. Then there exists a morphism $f : X \to W$ of relative dimension $r$ such that $\F \subset \T_f$ and this is an isomorphism away from codimension $2$ (IS THIS LAST PART TRUE, NEED REFLEXIVE MAYBE).
\end{theorem}

\begin{proof}
Let 
\end{proof}

\begin{example}
The conclusion that $\F = \T_f$ \textit{only on a dense open} cannot be strengthended. Indeed, 
\end{example}


\begin{prop}
Suppose that a foliation $\F \subset T_X$ is algebric. Then,
\begin{enumerate}
\item there is a Zariski open $U \subset X$ and a morphism $f : U \to S$ such that $T_f = \F|_U$.
\item if $\F$ is a vector bundle then $f$ is smooth
\item if $\F$ is furthermore a sub-bundle then $S$ is  smooth. (WAIT BUT IF $X$ IS SMOOTH AND $f$ IS SMOOTH IT IMPLIES THAT $S$ IS SMOOTH). 
\end{enumerate}
\end{prop}



\subsection{The Case $\kappa = -\infty$}

Suppose we assume that $\mu_{\alpha, \text{min}}(\F^\vee) > 0$ (in the notation of Campana) then the Foliation is algebraic.







\subsection{An Idea}

Consider the canonical fibration $f : X \to S$. If $X$ has a nonvanishing $1$-form $\omega$ then Popa-Schnell shows that the map $X \to \Alb_X$ cannot fully contract the fiber $F$ of $X \to S$. If we can show that $\dim{\Alb_F} < \dim{\Alb_X}$ then this means that $\Alb_X$ is reducible. 

\section{Reflexivity}

\begin{prop}
Let $\F$ be a coherent sheaf on an integral scheme $X$. Then $\F^*$ is reflexive. 
\end{prop}

\begin{proof}
Let $\psi : \id \to (-)^{* *}$ be the double dual natural transformation.
There are maps,
\[ \F^* \xrightarrow{\psi_{\F^*}} \F^{***} \xrightarrow{\psi_{\F}^*} \F^* \]
which acts as follows, for $\varphi \in \F^*$ and $\ell \in \F^{**}$,
\[ \varphi \mapsto (\ell \mapsto \ell(\varphi)) \mapsto [(\ell \mapsto \ell(\varphi)) \circ \psi_{\F}]   \]
the result of which is the function,
\[ x \mapsto^{\psi_{\F}} (\varphi' \mapsto \varphi'(x)) \mapsto \varphi(x) \]
which is just $\varphi$ so indeed this is the identity. Thus it suffices to show that the second map is injective. However, $\F^{***}$ is torsion-free and hence it suffices to check this at the generic point where it becomes finite-dimensional linear algebra. 
\end{proof}

Consider a separable map $f : X \to C$ where $X$ is a smooth proper (integral) surface over $k$ and $C$ a smooth proper (integral) curve over $k$. Then there is a sequence,
\begin{center}
\begin{tikzcd}
0 \arrow[r] & f^* \Omega_C \arrow[r] & \Omega_X \arrow[r] & \Omega_{X/C} \arrow[r] & 0
\end{tikzcd}
\end{center} 
which is injective on the left by generic smoothness and the fact that $\Omega_C$ and $\Omega_X$ are vector bundles. Then we get an exact sequence,
\begin{center}
\begin{tikzcd}
0 \arrow[r] & \F \arrow[r] & \T_X \arrow[r] & f^* \T_C \arrow[r] & \shExt{1}{\struct{X}}{\Omega_{X/C}}{\struct{X}} \arrow[r] & 0
\end{tikzcd}
\end{center}
again using that $\Omega_X$ is a vector bundle where we set $\F = \Omega_{X/C}^\vee$. Since $\F$ is reflexive and $X$ is a regular surface we see that $\F$ is a vector bundle. By generic smoothness, $\F$ has rank $1$. I claim that $\F$ is closed under Lie bracket. 
\bigskip\\
Indeed, this a local question so we reduce to $R  \to A \to B$ are ring maps then $\Hom{B}{\Omega_{B/A}}{B} \to \Hom{B}{\Omega_{A/R}}{B}$ is closed under Lie bracket. Indeed, for $X, Y \in \Hom{B}{\Omega_{B/A}}{B}$ we need to show that $(X \circ \d) (Y \circ \d) - (Y \circ \d) (X \circ \d)$ is a $A$-derivation not just an $R$-derivation. This is basically obvious because $X \circ \d$ and $Y \circ \d$ kill $A$. 
\bigskip\\
Now we get an exact sequence,
\begin{center}
\begin{tikzcd}
0 \arrow[r] & \T_X / \F \arrow[r] & f^* \T_C \arrow[r] & \shExt{1}{\struct{X}}{\Omega_{X/C}}{\struct{X}} \arrow[r] & 0
\end{tikzcd}
\end{center}
Since $f^* \T_C$ is torsion-free we see that so is $\T_X / \F$ so $\F$ is automatically saturated. 

\end{document}