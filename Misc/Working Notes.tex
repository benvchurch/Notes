\documentclass[12pt]{article}
\usepackage{hyperref}
\hypersetup{
    colorlinks=true,
    linkcolor=blue,
    filecolor=magenta,      
    urlcolor=blue,
}

\usepackage{import}
\import{"../Algebraic Geometry/"}{AlgGeoCommands}

\newcommand{\Loc}[1]{\mathfrak{Loc}\left( #1 \right)}
\newcommand{\AbGrp}{\mathbf{AbGrp}}
\usepackage{bbm}


\begin{document}

\section{TODO!!}

\begin{enumerate}
\item Finish symplectic geometry course
\begin{enumerate}
\item figure out if symplectic toric is the same as projective toric variety (projectivity needed to come from a polytope and also to be Kahler)
\item review coisotropic reduced and write some notes
\item hyperkahler reduction examples 
\item are there examples of noncompact hyperkahlers?
\item work out the kinks in notes on hamiltonian actions
\end{enumerate}
\item review killing homotopy groups columbia lectures and write some notes
\item figure out those damn jet bundles and connections on principal bundles
\begin{enumerate}
\item RMK: $\pi^* E$ is NOT trivial for a vector bundle let alone a fiber bundle. it does get equiped with a canonical section but for a vector bundle this is just the trivial section, only for a principal bundle does giving a section trivialize it.
\item role of atiyah sequence vs jet bundle sequence
\item 
\end{enumerate}
\item spectral sequences for tor and ext in derived category (FIND MY NOTES ON THIS!)
\begin{enumerate}
\item application to universal coefficient theorem
\item Kunneth spectral sequence
\item Kunneth formula for smash product?
\item why are derived functors triangulated
\item derived functors in terms of Kan extensions (NOTES)
\end{enumerate}
\item write notes on universal morphisms
\item $G$-action of $X/Y$ induces map Descent data $X/Y$ to $G$-equivariant sheaves
\begin{enumerate}
\item isomorphism when $X/Y$ is a $G$-cover i.e. $X \to Y$ is a $G$-torsor
\item write down explicit $G$-equivariant structure on $\Omega_X$
\item Galois descent derive explicit form
\end{enumerate}
\item Weil restriction
\begin{enumerate}
\item write down trivialization after going back up
\item Galois descent in explicit form
\end{enumerate}
\item notes on Galois actions on schemes
\item notes on Frobenii
\item notes on universal constructions in math with examples

\item fix notes on Tor in category of sheaves and Tor symmetry (do I need symmetry of flat objects a priori?).

\item Finish stable homotopy theory course.

\item Finish vector bundles and connections notes (in AG folder)
\begin{enumerate}
\item Kahler iff $\nabla I = 0$ where $\nabla$ is the Levi-Civita connection
\item Ricci tensor and the trace bullshit 
\item Riemann-Hilbert and existence of flat frames for integrable connections
\end{enumerate}
\end{enumerate}

\section{What I Want to Think About}

\begin{enumerate}
\item Flat cohomology equal etale cohomology for smooth (affine groups) apply this to that counting rational points things
\item work out the details for the group fixing $\CC$ inside endomorphism group. What does an integrable structure of this kind look like, how close to a complex manifold can we get? In dimension two this should be exactly a conformal (not necessarily orientable) structure. 
\item FINISH CONFORMAL NOTES!
\item Hilbert Class Field of curves (ASK BRIAN FOR REFERENCE)
\item Read about Fredholm index and Riemann-Roch
\item Cohmology and inclusion-exclusion: cohomology for vectorspaces?
\end{enumerate}

\section{Some Questions I Have}

\begin{enumerate}
\item Reduction of structure group for a scheme.
\begin{enumerate}
\item what about the algebraic group $\mathrm{SL}^{\pm} = \det^{-1}(\mu)$ what does reduction of structure group give. For a manifold this is supposed to be a pseudo-volume form but obviously that's not right.
\item what about $\Res{\CC}{\RR}{\Gm} \embed \mathrm{GL}_2$ from the action $\Gm \acts \A^1_{\CC}$ restricted giving an action $\Res{\CC}{\RR} \Gm \acts \A^2_{\RR}$. I feel like this should give an almost complex structure. What properties does it have? What about for other fields?
\item What is an almost complex structure on a scheme look like?
\end{enumerate}
\item Is my calculation of an ``almost almost complex structure'' as reduction of structure group to $\left< \sigma \right> \ltimes \mathrm{GL}(n, \CC) \subset \mathrm{GL}(2n, \RR)$. For the case $n = 1$ this should be the conformal group justfying that I think this should correspond to the non-oriented case of a complex manifold since Riemann surfaces are exactly oriented conformal manifolds.
\end{enumerate}


\section{Project}

\section{Foliations}

\subsection{Riemannian Foliations}

\begin{rmk}
Given a transverse metric $g_\F$ to a foliation $\F \subset T M$ we can always choose a metric $g$ on $M$ such that $g$ is bundle-like. Indeed, just choose a splitting $T M \cong \F \oplus Q$ and choose any metric on $\F$ then take the sum. 
\end{rmk}

\subsection{Holonomy}

\subsection{Linearization}

HOW IS THIS RELATED TO THE NORMAL BUNDLE?

\end{document}