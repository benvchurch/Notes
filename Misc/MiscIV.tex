\documentclass[12pt]{article}
\usepackage{hyperref}
\hypersetup{
    colorlinks=true,
    linkcolor=blue,
    filecolor=magenta,      
    urlcolor=blue,
}

\usepackage{import}
\import{"../Algebraic Geometry/"}{AlgGeoCommands}

\newcommand{\Loc}[1]{\mathfrak{Loc}\left( #1 \right)}
\newcommand{\AbGrp}{\mathbf{AbGrp}}

\begin{document}

\section{Lang-Nishimura}

\begin{theorem}[Lang-Nishimura]
Let $f : X \rat Y$  be a rational map of $k$-varieties with $Y$ proper. If $X$ has a smooth $k$-point then $Y$ has a point.
\end{theorem}

\begin{proof}
First we prove the case that $X$ is a curve. Shrink to the smooth locus $U \subset X$ which intersects some generic point since $X$ has a smooth point $x \in X$ and $U$ is open. Hence we get a rational map $U \rat Y$ which extends to $U \to Y$ since $U$ is a regular curve and $Y$ is proper. 
\bigskip\\
Now we reduce to the curve case. We may shrink $X$ such that it affine and integral with $x \in X(k)$ a smooth $k$-point. The goal is to show that there exists a (nonproper) curve $C \to X$ mapping to $X$ whose image intersects the locus of definition of $f : X \rat Y$ and contains a lift $x' \in C(k)$ as a smooth $k$-point of $C$. There is an \etale neighborhood $U \to X$ of $x$ with a lift $x' \in U(k)$ with an \etale map $U \to \A^n_k$. Let $V \subset X$ be the domain of $f$ then pushing and pulling gives a dense open of $\A^n_k$. Therefore, choose a line $L \subset \A^n_k$ through the origin intersecting this locus. Then the preimage $L' \subset U$ is a smooth curve passing through $x'$ and hence $L' \to X$ satisfies the hypotheses. 
\end{proof}

\begin{example}
The condition that $x \in X(k)$ is a \textit{smooth point} is necessary. For example, consider,
\[ X = \Proj{\RR[X,Y,Z]/(X^2 + Y^2)} \]
and let $Y = \P^1_{\CC}$ be its normalization and consider the inverse of the normalization $X \rat Y$. Now $X$ contains a nonsmooth $\RR$-point $[0:0:1] \in X(\RR)$ but $Y$ does not have an $\RR$-point. 
\end{example}

\newcommand{\bb}{\mathbb}

\section{$\bb{E}_8$ lattice}

Let $X = \Bl_{P_1, \dots, P_9}(\P^2)$ be the blowup at $9$ points sufficiently general so there is a unique cubic $C$ through these points and it is smooth. Then there there is a genus $1$ curve $\wt{C} \subset X$ which is the strict transform of the unique conic through the points $P_1, \dots, P_9$. Let $E_1, \dots, E_9$ be the exceptional divisors. Then,
\[ \wt{C} = 3 H - (E_1 + \cdots + E_9) \]
so indeed we see that $\wt{C}^2 = 0$. Now the claim is that the lattice,
\[ \Pic{X} = \NS{X} \]
contains the $\bb{E}_8$ lattice as a subquotient. Indeed,
\[ \left< \wt{C} \right>^\perp / \left< \wt{C} \right> \cong \bb{E}_8 \]

\section{Root Stacks}

\section{Weierstrass Points}


\section{MAPP}

\begin{remark}
Note that the isomorphism $X \iso (B' \times Z)/G$ is \textit{not} compatible with any map to $A$. Indeed, there may not even be a map to $A$ since $B'/G = B$ may only be isogenous to an abelian subvariety. Even if $G$ is trivial, the isomorphism may not be compatible with $f$ and the projection. For example, consider $X = E \times C$ where $E$ is an elliptic curve and $C$ is a genus $2$ curve with Jacobian $E \times E'$. Mapping to the Albanese $E \times E \times E'$, our construction gives the identity $\id : E \times C$. However, the map to the Al banese does not factor through the first projection $\pr_1 : X \to E$. 
\end{remark}

\end{document}