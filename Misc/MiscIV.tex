\documentclass[12pt]{article}
\usepackage{hyperref}
\hypersetup{
    colorlinks=true,
    linkcolor=blue,
    filecolor=magenta,      
    urlcolor=blue,
}

\usepackage{import}
\import{"../Algebraic Geometry/"}{AlgGeoCommands}

\newcommand{\Loc}[1]{\mathfrak{Loc}\left( #1 \right)}
\newcommand{\AbGrp}{\mathbf{AbGrp}}

\newcommand{\pr}{\mathrm{pr}}


\newcommand{\cM}{\mathcal{M}}

\usepackage{slashed}
\usepackage{mathtools}
\DeclarePairedDelimiter\bra{\langle}{\rvert}
\DeclarePairedDelimiter\ket{\lvert}{\rangle}
\DeclarePairedDelimiterX\braket[2]{\langle}{\rangle}{#1\,\delimsize\vert\,\mathopen{}#2}

\usepackage{amssymb}% http://ctan.org/pkg/amssymb
\usepackage{pifont}% http://ctan.org/pkg/pifont
\newcommand{\cmark}{\ding{51}}%
\newcommand{\xmark}{\ding{55}}%
\newcommand{\pvec}[1]{\vec{#1}\mkern2mu\vphantom{#1}}

\begin{document}

\section{Lang-Nishimura}

\begin{theorem}[Lang-Nishimura]
Let $f : X \rat Y$  be a rational map of $k$-varieties with $Y$ proper. If $X$ has a smooth $k$-point then $Y$ has a point.
\end{theorem}

\begin{proof}
First we prove the case that $X$ is a curve. Shrink to the smooth locus $U \subset X$ which intersects some generic point since $X$ has a smooth point $x \in X$ and $U$ is open. Hence we get a rational map $U \rat Y$ which extends to $U \to Y$ since $U$ is a regular curve and $Y$ is proper. 
\bigskip\\
Now we reduce to the curve case. We may shrink $X$ such that it affine and integral with $x \in X(k)$ a smooth $k$-point. The goal is to show that there exists a (nonproper) curve $C \to X$ mapping to $X$ whose image intersects the locus of definition of $f : X \rat Y$ and contains a lift $x' \in C(k)$ as a smooth $k$-point of $C$. There is an \etale neighborhood $U \to X$ of $x$ with a lift $x' \in U(k)$ with an \etale map $U \to \A^n_k$. Let $V \subset X$ be the domain of $f$ then pushing and pulling gives a dense open of $\A^n_k$. Therefore, choose a line $L \subset \A^n_k$ through the origin intersecting this locus. Then the preimage $L' \subset U$ is a smooth curve passing through $x'$ and hence $L' \to X$ satisfies the hypotheses. 
\end{proof}

\begin{example}
The condition that $x \in X(k)$ is a \textit{smooth point} is necessary. For example, consider,
\[ X = \Proj{\RR[X,Y,Z]/(X^2 + Y^2)} \]
and let $Y = \P^1_{\CC}$ be its normalization and consider the inverse of the normalization $X \rat Y$. Now $X$ contains a nonsmooth $\RR$-point $[0:0:1] \in X(\RR)$ but $Y$ does not have an $\RR$-point. 
\end{example}

\newcommand{\bb}{\mathbb}

\section{$\bb{E}_8$ lattice}

Let $X = \Bl_{P_1, \dots, P_9}(\P^2)$ be the blowup at $9$ points sufficiently general so there is a unique cubic $C$ through these points and it is smooth. Then there there is a genus $1$ curve $\wt{C} \subset X$ which is the strict transform of the unique conic through the points $P_1, \dots, P_9$. Let $E_1, \dots, E_9$ be the exceptional divisors. Then,
\[ \wt{C} = 3 H - (E_1 + \cdots + E_9) \]
so indeed we see that $\wt{C}^2 = 0$. Now the claim is that the lattice,
\[ \Pic{X} = \NS{X} \]
contains the $\bb{E}_8$ lattice as a subquotient. Indeed,
\[ \left< \wt{C} \right>^\perp / \left< \wt{C} \right> \cong \bb{E}_8 \]

\section{Root Stacks}

\section{Weierstrass Points}



\section{MAPP}

\begin{remark}
Note that the isomorphism $X \iso (B' \times Z)/G$ is \textit{not} compatible with any map to $A$. Indeed, there may not even be a map to $A$ since $B'/G = B$ may only be isogenous to an abelian subvariety. Even if $G$ is trivial, the isomorphism may not be compatible with $f$ and the projection. For example, consider $X = E \times C$ where $E$ is an elliptic curve and $C$ is a genus $2$ curve with Jacobian $E \times E'$. Mapping to the Albanese $E \times E \times E'$, our construction gives the identity $\id : E \times C$. However, the map to the Al banese does not factor through the first projection $\pr_1 : X \to E$. 
\end{remark}


\section{Etale fundamental groups are NOT profinitely complete}

I allways thought that \etale fundamental groups are profinitely complete i.e. equal to their own profinite completion. This is false in general. They are always profinite but this is weaker in general. It is true that a profinite group is the limit over its finite \textit{continuous} quotients or equivalently,
\[ G = \ilim G/H \]
as $H$ runs over the finite index \textit{open} (actually every open subgroup in a compact group is automatically finite index) normal subgroups. However, this does not necessarily include every finite index subgroup.

\begin{rmk}
However, if a \textit{topological} group $G$ is profinite then $G \to \wh{G}^{\text{top}}$ is an isomorphism by definition where,
\[ \wh{G}^{\text{top}} = \ilim_{\substack{H \triangleleft G \\ H \text{ open}}} G/H \]
\end{rmk}

\begin{example}
$\pi_1^{\et}(\Spec{\Q}) = \Gal{\overline{\Q}/\Q}$ is \textit{not} profinitely complete. Indeed, see Chapter 7 of Milne's Class Field Theory.
\end{example}

\begin{rmk}
See these answers:
\begin{enumerate}
\item \chref{https://math.stackexchange.com/questions/3734250/are-normal-subgroups-of-finite-index-in-an-absolute-galois-groups-open}{Silverman's incorrect definition}

\item \chref{https://mathoverflow.net/questions/82177/a-profinite-group-which-is-not-its-own-profinite-completion/}{examples of noncomplete profinite groups}
\end{enumerate}
\end{rmk}

\begin{prop}
However, if $X$ is a scheme of finite type over $\CC$ then,
\[ \pi_1^{\et}(X) = \wh{\pi_1(X(\CC))} \]
is the profinite completion of a finitely presented group and hence is profinitely complete.
\end{prop}

\begin{proof}
Indeed, by Riemann-Existence,
\[ \pi_1^{\et}(X)\text{-FinSets} \cong \FEt_X \cong \{ \text{finite covering spaces of } X \} \cong \pi_1(X(\CC))\text{-FinSets} \]
where $\pi_1^{\et}(X)\text{-FinSets}$ means \textit{continuous} finite $\pi_1^{\et}(X)$-sets. This identifies $\pi_1^{\et}(X)$ as a topological group with $\wh{\pi_1(X(\CC))}$
\end{proof}

\begin{lemma}
If $G$ is a finitely presented group then $\wh{G} \to \wh{\wh{G}}$ is an isomorphism.
\end{lemma}

\begin{proof}
\chref{https://math.stackexchange.com/questions/4534747/do-groups-with-finitely-many-finite-index-subgroups-of-each-index-have-strongly}{See here}.
\end{proof}

\begin{rmk}
Note that the above theorem is nontrivial. In fact, it is false without the finite presentation assumption. See \chref{https://arxiv.org/pdf/0801.2955.pdf}{here}.
\end{rmk}

\section{Grothendieck Abelian Categories}

\begin{defn}
Let $\cA$ be an abelian category. We say that satisfies,
\begin{enumerate}
\item[(AB3)] $\cA$ has all direct sums
\item[(AB4)] $\cA$ is AB4 and taking direct sums is exact
\item[(AB5)] $\cA$ is AB3 and taking filtered colimits is exact
\item[(AB6)] $\cA$ is AB3 and given a family of filtered categories $\{ I_j \}_{j \in J}$ and maps $D_j : I_j \to \cA$ we have,
\[ \prod_{j \in J} \colim_{I_j} D_j = \colim_{(i_j) \in \prod\limits_{j \in J} I_j} \left( \prod_{j \in J} D_j(i_j) \right) \]
\end{enumerate}
We say that $\cA$ \textit{has a generator} if there is an object $M \in \cA$ such that $\Hom{\cA}{M}{-}$ is faithful. We say that $\cA$ is a \textit{Grothendieck category} if $\cA$ is AB5 and has a generator.
\end{defn}

\begin{lemma}
We have the following implications:
\[ \text{AB6} \implies \text{AB5} \implies \text{AB4} \implies \text{AB3} \]
\end{lemma}

\begin{lemma}
For any unital ring $R$, the category $\mathrm{Mod}_R$ satisfies AB6 and AB4$^*$ but not AB5$^*$.
\end{lemma}

\begin{example}
$\Ab$ thus satisfies AB6 and AB4$^*$ but not AB5$^*$. Hence $\Ab^\op$ which is isomorphic to the category of compact Hausdorff topological groups by Pontriagin duality saitsifes AB6$^*$ and AB4 but not AB5. 
\end{example}

\begin{lemma}
The only abelian category satisfying AB5 and AB5$^*$ is the zero category. 
\end{lemma}

\begin{lemma}
An AB3 abelian category $\cA$ has a generator $M$ if and only if for every $A \in \cA$ there is an epimorphism,
\[ \bigoplus_I M \onto A \]
\end{lemma}

\begin{proof}
Suppose that,
\[ \bigoplus_I M \onto A \]
By definition, if $f,g : A \to B$ are two maps such that the induced maps,
\[ M \to A \to B \]
are pairwise equal then $f = g$. Therefore,
\[ \Hom{\cA}{A}{B} \xrightarrow{\Hom{\cA}{M}{-}} \Hom{}{\Hom{\cA}{M}{A}}{\Hom{\cA}{M}{B}} \]
is injective since it is injective after evaluation at the inclusions $\{ M \to A \}_I$.
\bigskip\\
Conversely, suppose that $\cA$ has a generator. For each $A \in \cA$ let $I = \Hom{\cA}{M}{A}$ which is a set and there is a canonical map,
\[ c : \bigoplus_I M \to A \]
via evaluation. We need to show this is an epimorphism. Indeed, if $f, g : A \to B$ are two maps such that $f \circ c = g \circ c$ this means that $f_* = g_*$ and since $\Hom{\cA}{M}{-}$ is faithful we see that $f = g$ so we conclude that $c$ is an epimorphism.  
\end{proof}

\begin{theorem}[1.10.1 in T\^{o}hoku] 
Let $\cA$ be a Grothendieck abelian category then $\cA$ has enough injectives.
\end{theorem}

\newcommand{\PSh}{\mathrm{PSh}}

(IS THIS CORRECT??)

\begin{prop}
Let $\C$ be a category and $\cA$ satisfies any of,
\begin{enumerate}
\item AB3
\item AB4
\item AB5
\item AB6
\item AB3$^*$
\item AB4$^*$
\item AB5$^*$
\item AB6$^*$
\item $\cA$ has a generator
\item $\cA$ is a Grothendieck abelian category 
\end{enumerate} 
then the same is true of $\PSh(\C, \cA) = \mathrm{Fun}(\C^\op, \cA)$.
\end{prop}

DO THIS!!

\begin{theorem}
Let $\C$ be a site and $\cA$ satisfies any of,
\begin{enumerate}
\item AB3
\item AB4
\item AB5
\item AB6
\item AB3$^*$
\item $\cA$ has a generator
\item $\cA$ is a Grothendieck abelian category
\end{enumerate}
then the same is true of $\Sh(\C, \cA)$.
\end{theorem}

\begin{rmk}
Note tha even for $\cA = \Ab$ the sheaf category $\Sh(\C, \cA)$ need not be AB4$^*$ because infinite products are only left exact and do not, in general, preserve epimorphisms. For example, \chref{https://math.stackexchange.com/questions/674482/example-of-epimorphisms-such-that-the-product-is-not-an-epimorphism-in-the-categ}{see here}.
\end{rmk}

DO THIS!!

\begin{theorem}
Let $\cA$ is a Grothendieck abelian category and $\C$ is a site then the inclusion,
\[ \Sh(\C, \cA) \embed \PSh(\C, \cA) \]
has a left adjoint called ``sheafificaion''. 
\end{theorem}

DO THIS PROOF

\begin{cor}
If $\cA$ is a Grothendieck abelian category and $\C$ is a site then $\Sh(\C, \cA)$ has enough injectives. 
\end{cor}

\begin{theorem}[\chref{https://www.sciencedirect.com/science/article/pii/S000187080400194X}{Gabber\footnote{Which as usual he let other people publish.}}]
Let $X$ be a scheme. Then $\QCoh{X}$ is a Grothendieck abelian category and hence has enough injectives. Furthermore, $\QCoh{X} \embed \Mod{\struct{X}}$ has a right adjoint and hence is also AB3$^*$. 
\end{theorem}

IS IT TRUE THAT ALL GROTHENDIECK ABELIAN CATEGORIES HAVE ALL PRODUCTS?? WHY DOES QCoh HAVE PRODUCTS?? JUST BECAUSE OF THE COHERATOR?

\begin{rmk}
Note that products in $\QCoh{X}$ do not agree with products in $\Mod{\struct{X}}$ in general. GIVE EXAMPLE They are also not exact \chref{https://arxiv.org/pdf/1810.08752.pdf}{see here}
\end{rmk}

\begin{enumerate}
\item \chref{https://amathew.wordpress.com/2011/07/30/quasi-coherent-sheaves-presentable-categories-and-a-result-of-gabber}{CMB}
\item \chref{https://mathoverflow.net/questions/27168/injective-objects}{Leo's answer}.

\item \chref{https://mathoverflow.net/questions/40587/quasi-coherent-envelope-of-a-module}{quasi-coherent module}

\item \chref{https://www.math.mcgill.ca/barr/papers/gk.pdf}{Tohoku}.
\end{enumerate}

\section{Griffiths Conjecture}

\begin{conj}
Let $X$ be a smooth projective complex variety. If $E$ is an ample vector bundle on $X$ then it admits a hermitian metric with positive bisectional curvature.
\end{conj}

\begin{rmk}
In the case that $\rank{E} = 1$ this is exactly the Kodiara embedding theorem. 
\end{rmk}

\begin{rmk}
This conjecture is almost false as follows: 
\begin{enumerate}
\item \chref{https://link.springer.com/article/10.1007/BF01077641}{this} paper proves that if $X$ admits a \textit{Kahler} metic with negative bisectional curvature then $\pi_1(X)$ is infinite
\item \chref{}{Brotbek and Darondeau} proved that a generic complete intersection of large enough codimension and degree in $\P^N$ has ample cotangent bundle
\item by Lefschetz hyperplane theorem the above examples have $\pi_1 = 0$. 
\end{enumerate}
The reason this does not give a counterexample to Griffiths' conjecture is exactly the stipulation that the metric on $X$ is Kahler not just some arbitrary hermitian metric on $T X$. 
\end{rmk}

Some other references:

\begin{enumerate}
\item \chref{https://mathoverflow.net/questions/43010/griffiths-positive-metric}{MO Griffiths positivity}
\item \chref{https://arxiv.org/pdf/1710.10034.pdf}{Approach to the conjecture}

\item \chref{https://mathoverflow.net/questions/106743/relationship-between-sectional-curvature-bisectional-curvature-and-conjugate-po}{MO reference on holomorphic (bi)sectional curvature}. 
\end{enumerate}

\section{Infinite Products are not quasi-coherent}

Usually the sheaf,
\[ \F = \prod_{i \in \N} \struct{X} \]
is not quasi-coherent. This may be surprising since the inclusion of presheaves into sheaves admiting a left-adjoint shows that,
\[ (\lim_i \F_i)(U) = \lim_i \F_i(U) \]
and therefore,
\[ \F(U) = \prod_{i \in \N} \struct{X}(U) \]
However this is just because localization does not commute with products. Indeed, if $\F$ were quasi-coherent, over an affine $\Spec{A}$ we must have,
\[ \F = \wt{A^{\times \N}} \]
but this does not hold as an equality of sheaves because localization and infinte products do not commute. Indeed, consider $A = k[x]$ and localization at the element $f = x$. Then there is a natural map,
\[ (A^{\times n})_f \to (A_f)^{\times n} \]
but the element $(1, x^{-1}, x^{-2}, \cdots)$ is not in the image since elements on the right must have only bounded below powers of $x$ since they can be writen as $f^{-n} s$ for $s \in A^{\times n}$. 

\section{Positronium Lifetimes}

\newcommand{\CM}{\mathrm{CM}}


Let an electron with (four) momentum $p_1$ and positron with momentum $p_2$ annihilate to two photons (or vector bosons) with momenta $k_1, k_2$. The leading-order Feynman diagrams give,
\begin{align*}
i \cM = (-i e)^2 & \epsilon(k_1)^*_{\mu} \epsilon(k_2)^*_{\nu} \, \, \bar{v}^{s_2}(p_2) \left[ \gamma^{\nu} \frac{i(\slashed{q}_1 + m)}{q_1^2 - m^2 + i \epsilon} \gamma^\mu + \gamma^{\mu} \frac{i(\slashed{q}_2 + m)}{q_2^2 - m^2 + i \epsilon} \gamma^{\nu} \right] u^{s_1}(p_1)
\end{align*}
where $q_1 = p_1 - k_1$ and $q_2 = p_1 - k_2$ corresponding to the $t$-channel and $u$-channel respectively. First we work out some formulas. Expanding the momenta to first-order,
\[ \bar{v}(p_2) \slashed{a} u(p_1) = \sqrt{E_1 E_2} \, \xi'^{\dagger} \left[ - a^0 \left( \frac{\vec{p}_1}{E_1} + \frac{\vec{p}_2}{E_2} \right) \cdot \vec{\sigma} + 2 \vec{a} \cdot \vec{\sigma} \right] \xi  \]
Similarly,
\[ \bar{v}(p_2) \slashed{a} \gamma^5 u(p_1) = \sqrt{E_1 E_2} \, \xi'^{\dagger} \left[ - 2 a^0 + \vec{a} \cdot \left( \frac{\vec{p}_1}{E_1} + \frac{\vec{p}_2}{E_2} \right) - i \vec{\sigma} \cdot \left[ \left( \frac{\vec{p}_1}{E_1} - \frac{\vec{p}_2}{E_2} \right) \times \vec{a} \right] \right] \xi \]
And finally,
\begin{align*}
\bar{v}(p_2) \slashed{a} \slashed{b} u(p_1) = \sqrt{E_1 E_2} \, \xi'^{\dagger} & \left[ a^\mu b_\mu \left( \frac{\vec{p}_1}{E_1} - \frac{\vec{p}_2}{E_2} \right) \cdot \vec{\sigma} - 2 a^0 (\vec{b} \cdot \vec{\sigma}) + 2 b^0 (\vec{a} \cdot \vec{\sigma}) - i (\vec{a} \times \vec{b}) \cdot \left( \frac{\vec{p}_1}{E_1} - \frac{\vec{p}_2}{E_2} \right) \right.
\\
+ & \left. (\vec{a} \times \vec{b}) \times \left( \frac{p_1}{E_1} + \frac{p_2}{E_2} \right) \cdot \vec{\sigma} \right] \xi 
\end{align*}
We use the identity,
\[ \gamma^\mu \gamma^\nu \gamma^\rho = g^{\mu \nu} \gamma^\rho + g^{\nu \rho} \gamma^\mu - g^{\mu \rho} \gamma^\nu - i \varepsilon^{\alpha \mu \nu \rho} \gamma_\alpha  \gamma^5 \]
Work in the CM frame where,
\begin{align*}
p_1 &= (\tfrac{1}{2} E_{\CM}, \vec{p})
\\
p_2 & = (\tfrac{1}{2} E_{\CM}, -\vec{p})
\\
k_1 & = (\tfrac{1}{2} E_{\CM}, \vec{k})
\\
k_2 & = (\tfrac{1}{2} E_{\CM}, -\vec{k})
\end{align*}
Therefore,
\[ q_1^2 = (p_1 - k_1)^2 = m^2 + m_B^2 - 2 p_1 \cdot k_1 = m^2 + m_B^2 - \tfrac{1}{2} E_{\CM}^2 + 2 \vec{p} \cdot \vec{k} \]
and likewise,
\[ q_2^2 = (p_1 - k_2)^2 = m^2 + m_B^2 - 2 p_1 \cdot k_2 = m^2 + m_B^2 - \tfrac{1}{2} E_{\CM}^2 - 2 \vec{p} \cdot \vec{k} \]
Also in the CM frame,
\[ \bar{v}(p_2) \slashed{a} u(p_1) = E_{\CM} \, \xi'^{\dagger} \left[ \vec{a} \cdot \vec{\sigma} \right] \xi  \]
Similarly,
\[ \bar{v}(p_2) \slashed{a} \gamma^5 u(p_1) = \xi'^{\dagger} \left[ - E_{\CM} a^0 - 2i \vec{\sigma} \cdot \left[ \vec{p} \times \vec{a} \right] \right] \xi \]
And finally,
\begin{align*}
\bar{v}(p_2) \slashed{a} \slashed{b} u(p_1) = \xi'^{\dagger} & \left[ 2 a^\mu b_\mu  (\vec{p} \cdot \vec{\sigma}) - E_{\CM} a^0 (\vec{b} \cdot \vec{\sigma}) + E_{\CM} b^0 (\vec{a} \cdot \vec{\sigma}) - 2i (\vec{a} \times \vec{b}) \cdot \vec{p} \right] \xi 
\end{align*}
We need to simplify,
\begin{align*}
\cM = e^2 & \epsilon(k_1)^*_{\mu} \epsilon(k_2)^*_{\nu} \, \, \bar{v}^{s_2}(p_2) \left[ \frac{\gamma^\nu \slashed{q}_1 \gamma^\mu + m \gamma^\nu \gamma^\mu}{\tfrac{1}{2} E_{\CM}^2 - m_B^2 - 2 \vec{p} \cdot \vec{k}} + \frac{\gamma^\mu \slashed{q}_2 \gamma^\nu + m \gamma^\mu \gamma^\nu}{\tfrac{1}{2} E_{\CM}^2 - m_B^2 + 2 \vec{p} \cdot \vec{k}} \right] u^{s_1}(p_1)
\end{align*}
To do this, we define two quantities,
\begin{align*}
A = & \epsilon(k_1)^*_{\mu} \epsilon(k_2)^*_{\nu} \, \, \bar{v}^{s_2}(p_2) [ \gamma^\nu \slashed{q}_1 \gamma^\mu + m \gamma^\nu \gamma^\mu] u^{s_1}(p_1)
\\
B = & \epsilon(k_1)^*_{\mu} \epsilon(k_2)^*_{\nu} \, \, \bar{v}^{s_2}(p_2) [ \gamma^\mu \slashed{q}_2 \gamma^\nu + m \gamma^\mu \gamma^\nu] u^{s_1}(p_1)
\end{align*}
such that,
\[ \cM = \frac{e^2}{2m^2 - m_B^2} \left[ A \cdot \left( \frac{2 m^2 - m_B^2}{\tfrac{1}{2} E_{\CM}^2 - m_B^2 - 2 \vec{p} \cdot \vec{k}} \right) + B \cdot \left( \frac{2 m^2 - m_B^2}{\tfrac{1}{2} E_{\CM}^2 - m_B^2 + 2 \vec{p} \cdot \vec{k}} \right) \right]  \]
To first-order in $\vec{p}$ this is,
\[ \cM = \frac{e^2}{2m^2 - m_B^2} \left[ (A + B) + \left[ \frac{2 \vec{p} \cdot \vec{k}}{2 m^2 - m_B^2} \right] \cdot (A - B) \right] \]
Now we expand $A$ and $B$ to first-order in $\vec{p}$. Then,
\begin{align*}
A &= E_{\CM} \xi'^{\dagger} \left[ (\epsilon_1^* \cdot q_1) (\vec{\epsilon^*_2} \cdot \vec{\sigma}) + (\epsilon_2^* \cdot q_1) (\vec{\epsilon^*_1} \cdot \vec{\sigma}) - (\epsilon_1^* \cdot \epsilon_2^*) (\vec{q}_1 \cdot \vec{\sigma}) + i \varepsilon^{\alpha \nu \rho \mu} (\epsilon_1^*)_\nu (q_1)_\rho (\epsilon_2^*)_\mu \left< \gamma_\alpha \right>_5 \right.
\\
& \left. + 2m (\epsilon_1^* \cdot \epsilon_2^*) \frac{\vec{p} \cdot \vec{\sigma}}{E_{\CM}} - m \epsilon^{*0}_2 (\vec{\epsilon^*_1} \cdot \vec{\sigma}) + m \epsilon^{*0}_1 (\vec{\epsilon^*_2} \cdot \vec{\sigma}) + 2i m (\vec{\epsilon^*_1} \times \vec{\epsilon^*_2}) \cdot \frac{\vec{p}}{E_{\CM}}  \right] \xi
\end{align*}
where we define $\left< \gamma_\alpha \right>_5$ as the matrix $M$ in $\bar{v}^{s_2}(p_2) \gamma_\alpha \gamma^5 u^{s_1}(p_1) = E_{\CM} \xi'^{\dagger} M \xi$.
We need to be careful expanding the $\varepsilon$ term. There are four terms depending on where the $0$ index appears. These are (including a minus sign from index lowering),
\[ - i \left< \gamma_0 \right>_5 (\vec{\epsilon_1^*} \times \vec{q}_1) \cdot (\vec{\epsilon^*_2}) + i \epsilon_1^{*0} (\left< \vec{\gamma} \right>_5 \times \vec{q}_1) \cdot \vec{\epsilon_2^*} - i q_1^0 (\left< \vec{\gamma} \right>_5 \times \vec{\epsilon_1^*}) \cdot \vec{\epsilon_2^*} + i \epsilon^{*0}_2 (\left< \vec{\gamma} \right>_5 \times \vec{\epsilon_1^*}) \cdot \vec{q}_1 \]
But $q_1^0 = 0$ and we can compute the $\left< \vec{\gamma} \right>_5$ terms by rearranging them into the form $\left< \vec{a} \cdot \vec{\gamma} \right>_5$ so we can use the above identities since,
\[ \left< \vec{a} \cdot \vec{\gamma} \right>_5 = \left< \slashed{a} \right>_5 = \frac{\vec{p} \times \vec{a}}{E_{\CM}} \cdot (2i \vec{\sigma})  \]
where $a = (0, -\vec{a})$. Therefore,
\[ i \varepsilon^{\alpha \nu \rho \mu} (\epsilon_1^*)_\nu (q_1)_\rho (\epsilon_2^*)_\mu \left< \gamma_\alpha \right>_5 = i (\vec{\epsilon^*_1} \times \vec{q}_1) \cdot \vec{\epsilon^*_2} - 2 \epsilon_1^{*0} \frac{\vec{p} \times (\vec{q}_1  \times \vec{\epsilon_2^*})}{E_{\CM}} \cdot \vec{\sigma} - 2 \epsilon_2^{*0} \frac{\vec{p} \times (\vec{\epsilon_1^*} \times \vec{q}_1)}{E_{\CM}} \cdot \vec{\sigma} \]
And putting everything together (and using that $q_1^0 = 0$) we get,
\begin{align*}
A &= -E_{\CM} \xi'^{\dagger} \left[ (\vec{\epsilon_1^*} \cdot \vec{q}_1) (\vec{\epsilon^*_2} \cdot \vec{\sigma}) + (\vec{\epsilon_2^*} \cdot \vec{q}_1) (\vec{\epsilon^*_1} \cdot \vec{\sigma}) + (\epsilon_1^* \cdot \epsilon_2^*) (\vec{q}_1 \cdot \vec{\sigma}) \right.
\\
& - i (\vec{\epsilon^*_1} \times \vec{q}_1) \cdot \vec{\epsilon^*_2} + 2 \epsilon_1^{*0} \frac{\vec{p} \times (\vec{q}_1  \times \vec{\epsilon_2^*})}{E_{\CM}} \cdot \vec{\sigma} + 2 \epsilon_2^{*0} \frac{\vec{p} \times (\vec{\epsilon_1^*} \times \vec{q}_1)}{E_{\CM}} \cdot \vec{\sigma}
\\
& \left. - m \left( 2(\epsilon_1^* \cdot \epsilon_2^*) \frac{\vec{p} \cdot \vec{\sigma}}{E_{\CM}} - \epsilon^{*0}_2 (\vec{\epsilon^*_1} \cdot \vec{\sigma}) + \epsilon^{*0}_1 (\vec{\epsilon^*_2} \cdot \vec{\sigma}) + 2i (\vec{\epsilon^*_1} \times \vec{\epsilon^*_2}) \cdot \frac{\vec{p}}{E_{\CM}} \right) \right] \xi
\end{align*}
And $B$ is identical except for swapping $\epsilon_1$ and $\epsilon_2$ and swapping $q_1$ for $q_2$. Now write $\hat{A}$ and $\hat{B}$ for the unitless quantities inside the spinor inner product menaing that,
\[ A = - 2 m E_{\CM} \, \, \xi'^{\dagger} \hat{A} \xi \]
and likewise for $B$. Therefore, since to first-order in $\vec{p}$ we have $E_{\CM} = 2 m$ we have,
\[ \cM = - \left( \frac{2 e^2}{1 - \frac{m_B^2}{2m^2}} \right) \xi'^{\dagger} \left[ \hat{A} + \hat{B} + \left[ \frac{2 \vec{p} \cdot \vec{k}}{2 m^2 - m_B^2} \right] \cdot (\hat{A} - \hat{B}) \right] \xi \]
Now we consider, using that $\vec{q}_1 + \vec{q}_2 = 2 \vec{p}$ and $\vec{q}_1 - \vec{q}_2 = - 2 \vec{k}$ the quantity
\begin{align*}
\hat{A} + \hat{B} &= (\vec{\epsilon_1^*} \cdot \tfrac{\vec{p}}{m}) (\vec{\epsilon^*_2} \cdot \vec{\sigma}) + (\vec{\epsilon_2^*} \cdot \tfrac{\vec{p}}{m}) (\vec{\epsilon^*_1} \cdot \vec{\sigma}) + (\epsilon_1^* \cdot \epsilon_2^*) (\tfrac{\vec{p}}{m} \cdot \vec{\sigma})
\\
& + i (\vec{\epsilon^*_1} \times \tfrac{\vec{k}}{m}) \cdot \vec{\epsilon^*_2} - \epsilon_1^{*0} \frac{\vec{p} \times (\vec{k}  \times \vec{\epsilon_2^*})}{m^2} \cdot \vec{\sigma} +  \epsilon_2^{*0} \frac{\vec{p} \times (\vec{k} \times \vec{\epsilon_1^*})}{m^2} \cdot \vec{\sigma}
\\
& - (\epsilon_1^* \cdot \epsilon_2^*) (\tfrac{\vec{p}}{m} \cdot \vec{\sigma})
\\
& = (\vec{\epsilon_1^*} \cdot \tfrac{\vec{p}}{m}) (\vec{\epsilon^*_2} \cdot \vec{\sigma}) + (\vec{\epsilon_2^*} \cdot \tfrac{\vec{p}}{m}) (\vec{\epsilon^*_1} \cdot \vec{\sigma}) - i (\vec{\epsilon^*_1} \times \vec{\epsilon^*_2}) \cdot \tfrac{\vec{k}}{m} - \epsilon_1^{*0} \frac{\vec{p} \times (\vec{k}  \times \vec{\epsilon_2^*})}{m^2} \cdot \vec{\sigma} +  \epsilon_2^{*0} \frac{\vec{p} \times (\vec{k} \times \vec{\epsilon_1^*})}{m^2} \cdot \vec{\sigma}
\end{align*}
and likewise,
\begin{align*}
\hat{A} - \hat{B} &= - (\vec{\epsilon_1^*} \cdot \tfrac{\vec{k}}{m}) (\vec{\epsilon^*_2} \cdot \vec{\sigma}) - (\vec{\epsilon_2^*} \cdot \tfrac{\vec{k}}{m}) (\vec{\epsilon^*_1} \cdot \vec{\sigma}) - (\epsilon_1^* \cdot \epsilon_2^*) (\tfrac{\vec{k}}{m} \cdot \vec{\sigma})
\\
& - i (\vec{\epsilon^*_1} \times \tfrac{\vec{p}}{m}) \cdot \vec{\epsilon^*_2}
\\
& + \epsilon_2^{*0} (\vec{\epsilon_1^*} \cdot \vec{\sigma}) - \epsilon_1^{*0} (\vec{\epsilon_2^*} \cdot \vec{\sigma}) - i (\vec{\epsilon_1^*} \times \vec{\epsilon_2^*}) \cdot \tfrac{\vec{p}}{m}
\\
& = - (\vec{\epsilon_1^*} \cdot \tfrac{\vec{k}}{m}) (\vec{\epsilon^*_2} \cdot \vec{\sigma}) - (\vec{\epsilon_2^*} \cdot \tfrac{\vec{k}}{m}) (\vec{\epsilon^*_1} \cdot \vec{\sigma}) - (\epsilon_1^* \cdot \epsilon_2^*) (\tfrac{\vec{k}}{m} \cdot \vec{\sigma}) + \epsilon_2^{*0} (\vec{\epsilon_1^*} \cdot \vec{\sigma}) - \epsilon_1^{*0} (\vec{\epsilon_2^*} \cdot \vec{\sigma})
\end{align*}
Finally, dropping terms to higher order in $\vec{p}$ we get,
\begin{align*}
\cM & = - \left( \frac{2 e^2}{1 - \frac{m_B^2}{2 m^2}} \right) \xi'^{\dagger} \bigg[ (\vec{\epsilon_1^*} \cdot \tfrac{\vec{p}}{m}) (\vec{\epsilon^*_2} \cdot \vec{\sigma}) + (\vec{\epsilon_2^*} \cdot \tfrac{\vec{p}}{m}) (\vec{\epsilon^*_1} \cdot \vec{\sigma}) - i (\vec{\epsilon^*_1} \times \vec{\epsilon^*_2}) \cdot \tfrac{\vec{k}}{m} 
\\
& \quad \quad \quad - \epsilon_1^{*0} \frac{\vec{p} \times (\vec{k}  \times \vec{\epsilon_2^*})}{m^2} \cdot \vec{\sigma} +  \epsilon_2^{*0} \frac{\vec{p} \times (\vec{k} \times \vec{\epsilon_1^*})}{m^2} \cdot \vec{\sigma}
\\
& \quad \quad - \left[ \frac{2 \vec{p} \cdot \vec{k}}{2 m^2 - m_B^2}  \right]  \left( (\vec{\epsilon_1^*} \cdot \tfrac{\vec{k}}{m}) (\vec{\epsilon^*_2} \cdot \vec{\sigma}) + (\vec{\epsilon_2^*} \cdot \tfrac{\vec{k}}{m}) (\vec{\epsilon^*_1} \cdot \vec{\sigma}) + (\epsilon_1^* \cdot \epsilon_2^*) (\tfrac{\vec{k}}{m} \cdot \vec{\sigma}) - \epsilon_2^{*0} (\vec{\epsilon_1^*} \cdot \vec{\sigma}) + \epsilon_1^{*0} (\vec{\epsilon_2^*} \cdot \vec{\sigma}) \right) \bigg] \xi
\end{align*}
For photon polarizations this simplifies greatly since $\vec{\epsilon} \perp \vec{k}$ and $\epsilon^0 = 0$. Therefore, the photon amplitude is,
\begin{align*}
\cM & = -2 e^2 \xi'^{\dagger} \bigg[ (\vec{\epsilon_1^*} \cdot \tfrac{\vec{p}}{m}) (\vec{\epsilon^*_2} \cdot \vec{\sigma}) + (\vec{\epsilon_2^*} \cdot \tfrac{\vec{p}}{m}) (\vec{\epsilon^*_1} \cdot \vec{\sigma}) + (\vec{\epsilon_1^*} \cdot \vec{\epsilon_2^*})(\tfrac{\vec{p}}{m} \cdot \hat{k})(\hat{k} \cdot \vec{\sigma}) - i (\vec{\epsilon^*_1} \times \vec{\epsilon^*_2}) \cdot \hat{k} \bigg] \xi
\end{align*}
using that $\frac{\vec{k}}{m} = \hat{k}$ to first-order in $\vec{p}$ since the photons carry away all the energy and hence $|\vec{k}| = m$. All but the last term are suppressed by a factor of $\vec{p}/m$ which for positronium will be proportional to $\alpha$.

\subsection{Selection rules for 2-photon annihilation}

We are free to orient our spinor basis along any direction  (recall that $\xi'$ is the flipped spinor of the physical positron). We choose to orient along the $z$-direction which is chosen as the direction along which $\vec{k}$ points. Then,
\[ \cM_{\pm \pm} = - 2 e^2 \xi'^{\dagger} (\pm 1 + \tfrac{\vec{p}}{m} \cdot \vec{\sigma}) \xi \]
\[ \cM_{\pm \mp} = - 2 e^2 \xi'^{\dagger} (\tfrac{p_x \mp i p_y}{m}) (\sigma_x \mp i \sigma_y) \xi \]
Spin-orbit coupling means that the positronium will naturally be split into eigenstates of total angular momentum. For $n = 1$ there is only an $\ell = 0$ wavefunction so there are only two states, para (singlet $s = 0$) and ortho (triplet $s = 1$) given by,
\[ \ket{{1}^1 S_0} = \frac{1}{\sqrt{2}} \left( \ket{\uparrow \downarrow} - \ket{\downarrow \uparrow} \right) \otimes \ket{\psi_1} \]
and
\begin{align*}
\ket{{1}^3 S_1, m = 1} &= \ket{\uparrow \uparrow} \otimes \ket{\psi_1}
\\
\ket{{1}^3 S_1, m = 0} &= \frac{1}{\sqrt{2}} \left( \ket{\uparrow \downarrow} + \ket{ \downarrow \uparrow} \right) \otimes \ket{\psi_1}
\\
\ket{{1}^3 S_1, m = -1} &= \ket{\downarrow \downarrow} \otimes \ket{\psi_1}
\end{align*}
where $\ket{\psi_0}$ is the $1S$ state wavefunction. For $n = 2$ we have $\ell = 0,1$ and hence there are more states. We have the excited $2S$ (meaning $\ell = 0$) versions of para ($s = 0$) and ortho ($s = 1$) positronium which are identical but with $\ket{\psi_0}$ replaced by $\ket{\psi_{1,0}}$. More interesting are the following,
the $s = 0$ states,
\begin{align*}
\ket{{2}^1 P_1, m = 1} &= \frac{1}{\sqrt{2}} \left( \ket{\uparrow \downarrow} - \ket{\downarrow \uparrow} \right) \otimes \ket{\psi_{1,1,1}}
\\
\ket{{2}^1 P_1, m = 0} &= \frac{1}{\sqrt{2}} \left( \ket{\uparrow \downarrow} - \ket{\downarrow \uparrow} \right) \otimes \ket{\psi_{1,1,0}}
\\
\ket{{2}^1 P_1, m = 0} &= \frac{1}{\sqrt{2}} \left( \ket{\uparrow \downarrow} - \ket{\downarrow \uparrow} \right) \otimes \ket{\psi_{1,1,0}}
\end{align*}
the $s = 1$ state with $j = 0$,
\[ \ket{2^{3} P_0} = \frac{1}{\sqrt{3}} \left( \ket{\uparrow \uparrow} \ot \ket{\psi_{1,1,-1}} + \ket{\downarrow \downarrow} \ot \ket{\psi_{1,1,1}}  - \frac{1}{\sqrt{2}} (\ket{\uparrow \downarrow} + \ket{\downarrow \uparrow}) \ot \ket{\psi_{1,1,0}} \right) \]
the $s = 1$ states with $j = 1$,
\begin{align*}
\ket{2^{3} P_1, m = 1} &= \frac{1}{\sqrt{2}} \left( \ket{\uparrow \uparrow} \ot \ket{\psi_{1,1,0}} - \frac{1}{\sqrt{2}} (\ket{\uparrow \downarrow} + \ket{\downarrow \uparrow}) \ot \ket{\psi_{1,1,1}} \right) 
\\
\ket{2^{3} P_1, m = 0} &= \frac{1}{\sqrt{2}} \left( \ket{\uparrow \uparrow} \ot \ket{\psi_{1,1,-1}} - \ket{\downarrow \downarrow} \ot \ket{\psi_{1,1,1}} \right) 
\\
\ket{2^{3} P_1, m = -1} &= \frac{1}{\sqrt{2}} \left( \ket{\downarrow \downarrow} \ot \ket{\psi_{1,1,0}} - \frac{1}{\sqrt{2}} (\ket{\uparrow \downarrow} + \ket{\downarrow \uparrow}) \ot \ket{\psi_{1,1,-1}} \right) 
\end{align*}
and finally the $s = 1$ sates with $j = 2$,
\begin{align*}
\ket{2^3 P_2, m = 2} & = \ket{\uparrow \uparrow} \ot \ket{\psi_{1,1,1}}
\\
\ket{2^3 P_2, m = 1} & = \frac{1}{\sqrt{2}} \left( \ket{\uparrow \uparrow} \ot \ket{\psi_{1,1,0}} + \frac{1}{\sqrt{2}} (\ket{\uparrow \downarrow} + \ket{\downarrow \uparrow}) \ot \ket{\psi_{1,1,1}} \right)
\\
\ket{2^3 P_2, m = 0} & = \frac{1}{\sqrt{6}} \left( \ket{\uparrow \uparrow} \ot \ket{\psi_{1,1,-1}} + \frac{\sqrt{2}}{\sqrt{2}} (\ket{\uparrow \downarrow} + \ket{\downarrow \uparrow}) \ot \ket{\psi_{1,1,0}} + \ket{\downarrow \downarrow} \ot \ket{\psi_{1,1,1}} \right)
\\
\ket{2^3 P_2, m = -1} & = \frac{1}{\sqrt{2}} \left( \ket{\downarrow \downarrow} \ot \ket{\psi_{1,1,0}} + \frac{1}{\sqrt{2}} (\ket{\uparrow \downarrow} + \ket{\downarrow \uparrow}) \ot \ket{\psi_{1,1,-1}} \right)
\\
\ket{2^3 P_2, m = -2} & = \ket{\downarrow \downarrow} \ot \ket{\psi_{1,1,-1}}
\end{align*}
The spin-flip rule assigns, for the positron,
\[ \ket{\uparrow} \mapsto \xi' = \begin{pmatrix}
0 
\\
1
\end{pmatrix}
\quad \quad
\ket{\downarrow} \mapsto \xi' = \begin{pmatrix}
-1
\\
0
\end{pmatrix} \]
The amplitudes can be rewritten in the form of the matrix shown times $\xi \xi'^{\dagger}$. Then using the spin flip we easily see that $s = 0$ state corresponds to,
\[ \xi \xi'^\dagger = \frac{1}{\sqrt{2}} \id \]
and the $s = 1$ state with spin along $\hat{n}$ corresponds to,
\[ \xi \xi'^\dagger = \frac{1}{\sqrt{2}} \hat{n} \cdot \vec{\sigma} \]
(this is consistent with Peskin as can be seen by daggering the above expression which exchanges $\xi$ and $\xi'$ and replaces $\hat{n}$ by $\hat{n}^*$)
where $\hat{n}$ points along the direction of $m = 0$ meaning spin $+ \hat{z}$ corresponds to $\hat{n} = \hat{x} + i \hat{y}$. Therefore, taking the trace of the inner matrix,
\[ \cM(s = 0) = 2i  \sqrt{2} e^2 (\vec{\epsilon^*_1} \times \vec{\epsilon^*_2}) \cdot \hat{k} \]
and therefore,
\begin{align*}
\cM_{\pm \pm}(s = 0) &= \mp 2 \sqrt{2} e^2 
\\
\cM_{\pm \mp}(s = 0) &= 0 
\end{align*}
so any $s = 0$ state decays into an odd $S$-wave EPR $j = 0$ entangled state. Since the wavefunction of a $P$ orbital vanishes at the origin, this protects against decay of the $j = 1$ and $s = 0$ states into two photons (angular momentum of two photons cannot be $j = 1$). This is also due to $C$ conservation since the $2^{1} P_1$ state is odd under $C$ but any two photon state is even under $C$. Therefore, we expect this decay to be forbidden into any $C$-odd vector particles with a $C$-invariant interaction term. 
Likewise,
\begin{align*}
\cM(s = 1) &= -\sqrt{2} e^2 \, \, \Tr{\bigg[ (\vec{\epsilon_1^*} \cdot \tfrac{\vec{p}}{m}) (\vec{\epsilon^*_2} \cdot \vec{\sigma}) + (\vec{\epsilon_2^*} \cdot \tfrac{\vec{p}}{m}) (\vec{\epsilon^*_1} \cdot \vec{\sigma}) + (\vec{\epsilon_1^*} \cdot \vec{\epsilon_2^*})(\tfrac{\vec{p}}{m} \cdot \hat{k})(\hat{k} \cdot \vec{\sigma}) - i (\vec{\epsilon^*_1} \times \vec{\epsilon^*_2}) \cdot \hat{k} \bigg] (\hat{n} \cdot \sigma)}
\\
& = - 2 \sqrt{2} e^2 \bigg[ (\vec{\epsilon_1^*} \cdot \tfrac{\vec{p}}{m}) (\vec{\epsilon^*_2} \cdot \hat{n}) + (\vec{\epsilon_2^*} \cdot \tfrac{\vec{p}}{m}) (\vec{\epsilon^*_1} \cdot \hat{n}) + (\vec{\epsilon_1^*} \cdot \vec{\epsilon_2^*})(\tfrac{\vec{p}}{m} \cdot \hat{k})(\hat{k} \cdot \hat{n}) \bigg]
\end{align*}
and therefore, using the completness relation
\begin{align*}
\cM_{\pm \pm}(s = 1) &= - 2 \sqrt{2} e^2 (\tfrac{\vec{p}}{m} \cdot \hat{n})
\\
\cM_{\pm \mp}(s = 1) &= -2 \sqrt{2} e^2 (\tfrac{p_x}{m} \mp i \tfrac{p_y}{m})(n_x \mp i n_y)
\end{align*}
Hence for the three spin orientations we get,
\begin{align*}
\cM_{\pm \pm}(s = 1, m = 1) &= - 2 e^2 (\tfrac{p_x}{m} + i \tfrac{p_y}{m})
\\
\cM_{\pm \pm}(s = 1, m = 0) &= - 2 \sqrt{2} e^2 (\tfrac{p_z}{m})
\\
\cM_{\pm \pm}(s = 1, m = -1) &= - 2  e^2 (\tfrac{p_x}{m} - i \tfrac{p_y}{m})
\\
\cM_{\pm \mp}(s = 1, m = \pm 1) &= -4  e^2 (\tfrac{p_x}{m} \mp i \tfrac{p_y}{m})
\\
\cM_{\pm \mp}(s = 1, m = 0) &= 0
\\
\cM_{\pm \mp}(s = 1, m = \mp 1) &= 0
\end{align*}
This alows the $s = 1$ and $j = 0$ state to decay into an even $S$-wave EPR $j = 0$ state. The first two terms do not couple since they require both $\vec{S}$ and $\vec{L}$ in the same orientation. However, the $\M_{\pm \pm}$ does couple to the last term and is even under exchange of RHC and LHC photons. This is necessary to preserve parity since any ${}^3 P$ state has even parity. Note that $S$-wave $\ket{++}$ and $\ket{--}$ are both $j = 0$ states of the photon field since they are identical particles so are even under exchange facilitated by a $\pi$-rotation. Thus either linear combination is a valid $j = 0$ state with $s = 0$ and $\ell = 0$. Therefore we see that it is not $P$ but $C$ that really forbids various positronium decays. 
\bigskip\\
Finally, we analyize the $2^{3} P_1$ and $2^{3} P_2$ states. The easiest is $2^{3} P_2$ for which $m = \pm 2$ clearly couple to $\M_{\pm \mp}$. The $m = \pm 1$ states do not couple. Although it at first appears that the $m = 0$ state couples, it does not. Indeed, the amplitude is,
\[ - 2 \sqrt{2} e^2 [-1 + (\sqrt{2})^2 - 1] = 0 \]
where the minus signs arise from the Condon–Shortley phase convention which is chosen such that the raising and lowering operators act on spherical harmonics in the way expected for the derivation of the Clebsch-Gordan coefficients in use. Therefore we get $D$-wave (using the rotation matrices for $j = 2$ we get $\cos{2 \theta}$ angular dependence on amplitudes) $s = 2$ photon emission with $j = 2$. 
\bigskip\\
For $2^{3} P_1$ we see that none of the states couple since $\ket{\uparrow \uparrow}$ only couples to two photons when paired with $\ket{\psi_{1,1,1}}$ and likewise only the $m = 0$ states and $m = -1$ states couple to eachother. This decay is not forbidden by $C$ it is forbidden by angular momentum selection since a two photon state cannot have odd spin along an axis and thus cannot have $j = 1$ since $L_z$ is zero along the direction of motion and hence $J_z$ has an even eigenvalue. (BETTER EXPLINATION)

\subsection{Positronium Decay Rate}

We build a Positronium states. For $\ell = 0$ we consider,
\[ \ket{\text{Ps}(\vec{k}) \, {}^{1} S_0} = \sqrt{2 E_{\vec{k}}} \int \frac{\dn{3}{p}}{(2 \pi)^3} \frac{1}{\sqrt{4 E_{\vec{p} + \frac{\vec{k}}{2}} E_{-\vec{p} + \frac{\vec{k}}{2}}}} \tilde{\psi_0}(\vec{p}) \frac{\sqrt{4 E_{\vec{p} + \frac{\vec{k}}{2}} E_{-\vec{p} + \frac{\vec{k}}{2}}}}{\sqrt{2}} \sum_s a_{\vec{p} + \frac{\vec{k}}{2}}^{s \dagger} b_{-\vec{p} + \frac{\vec{k}}{2}}^{s \dagger} \ket{\Omega}   \]
we need to show that this is properly normalized. Indeed,
\begin{align*}
\braket{\text{Ps}(\vec{k}') \, {}^{1} S_0}{\text{Ps}(\vec{k}) \, {}^{1} S_0 } & = 2 E_{\vec{k}} \int \frac{\dn{3}{\pvec{p}'}}{(2 \pi)^3} \frac{\dn{3}{\vec{p}}}{(2 \pi)^3} \tilde{\psi_0}^*(\pvec{p}') \tilde{\psi_0}(\vec{p}) \left[ \frac{1}{2} \sum_{s's} \bra{\Omega}  b_{-\pvec{p}' + \frac{\vec{k}'}{2}}^{s'} a_{\pvec{p}' + \frac{\vec{k}'}{2}}^{s'} a_{\vec{p} + \frac{\vec{k}}{2}}^{s \dagger} b_{-\vec{p} + \frac{\vec{k}}{2}}^{s \dagger} \ket{\Omega} \right]
\\
& = 2 E_{\vec{k}} \int \frac{\dn{3}{\pvec{p}'}}{(2 \pi)^3} \frac{\dn{3}{\vec{p}}}{(2 \pi)^3} \tilde{\psi_0}^*(\pvec{p}') \tilde{\psi_0}(\vec{p})
\\
& \quad \cdot \left[ \frac{1}{2} \sum_{s's} (2 \pi)^3 \delta^{(3)}(\pvec{p}' + \tfrac{\vec{k}'}{2} - \vec{p} - \tfrac{\vec{k}}{2}) \delta_{ss'} \delta_{ss'} (2 \pi^3) \delta^{(3)}(-\vec{p} + \tfrac{\vec{k}}{2} + \pvec{p}' - \tfrac{\vec{k}'}{2}) \right]
\end{align*}
Call the arguments of the $\delta$-functions $A,B$. Then $\tfrac{1}{2}(A + B) = (\pvec{p}' - \vec{p})$ and $A - B = \vec{k}' - \vec{k}$ and the matrix,
\[ \begin{pmatrix}
\tfrac{1}{2} & \tfrac{1}{2}
\\
1 & -1
\end{pmatrix} \]
has determinant $-1$ and therefore we can perform this change of variables on the $\delta$-functions to get,
\begin{align*}
\braket{\text{Ps}(\vec{k}') \, {}^{1} S_0}{\text{Ps}(\vec{k}) \, {}^{1} S_0 } 
& = 2 E_{\vec{k}} \int \frac{\dn{3}{\pvec{p}'}}{(2 \pi)^3} \frac{\dn{3}{\vec{p}}}{(2 \pi)^3} \tilde{\psi_0}^*(\pvec{p}') \tilde{\psi_0}(\vec{p}) \left[ (2 \pi)^6 \delta^{(3)}(\pvec{p}' - \vec{p})  \delta^{(3)}(\vec{k}'-\vec{k}) \right]
\\
& = 2 E_{\vec{k}} (2 \pi)^3 \delta^{(3)}(\vec{k}' - \vec{k}) \int \frac{\dn{3}{\vec{p}}}{(2 \pi)^3} \tilde{\psi}_0^*(\vec{p}) \tilde{\psi}(\vec{p}) = 2 E_{\vec{k}} (2 \pi)^3 \delta^{(3)}(\vec{k}' - \vec{k})
\end{align*}
which is the desired relativistic normalization. Now we consider the $P$-states. Let $\psi_i$ be a set of $P$-wave wavefunctions of the form $\psi_i = x^i f(|x|)$ normalized such that,
\[ \int \dn{3}{x} \psi_i^*(x) \psi_j(x) = \delta_{ij} \]
Then consider the states,
\[ \ket{\text{Ps}(\vec{k}) \, P_{\Sigma}} = \sqrt{2 E_{\vec{k}}} \int \frac{\dn{3}{p}}{(2 \pi)^3} \sum_i \frac{1}{\sqrt{4 E_{p+\frac{\vec{k}}{2}} E_{-p+\frac{\vec{k}}{2}}}} \tilde{\psi_i}(p) \sqrt{4 E_{p+\frac{\vec{k}}{2}} E_{-p+\frac{\vec{k}}{2}}} \sum_{s's} a_{p + \frac{\vec{k}}{2}}^{s' \dagger} \Sigma^i_{s's} b_{-p + \frac{\vec{k}}{2}}^{s \dagger} \ket{\Omega}   \]
where $\Sigma_i$ are a set of 2 x 2 matrices such that $\sum_i \tr{\Sigma^{i \dagger} \Sigma^i} = 1$. We need to check the normalization of these states,
\begin{align*}
\braket{\text{Ps}(\vec{k}') \, P_{\Sigma}}{\text{Ps}(\vec{k}) \, P_{\Sigma}} & = 2 E_{\vec{k}} \int \frac{\dn{3}{p'}}{(2 \pi)^3} \frac{\dn{3}{p}}{(2 \pi)^3} \sum_{ij} \tilde{\psi}_i^*(p') \tilde{\psi}_j(p) \sum_{t' t s' s} \Sigma^{j *}_{t' t} \Sigma^i_{s' s} \bra{\Omega} b_{-p' + \frac{\vec{k}'}{2}}^t  a^{t'}_{p' + \frac{\vec{k}'}{2}} a^{s' \dagger}_{p + \frac{\vec{k}}{2}}  b^{s \dagger}_{-p + \frac{\vec{k}}{2}} \ket{\Omega} 
\\
& = 2 E_{\vec{k}} \int \frac{\dn{3}{p'}}{(2 \pi)^3} \frac{\dn{3}{p}}{(2 \pi)^3} \sum_{ij} \tilde{\psi}_i^*(p') \tilde{\psi}_j(p) \left[ \sum_{t' t s' s} (\Sigma^{j \dagger})_{t t'} \Sigma^i_{s' s} \delta_{t s} \delta_{t' s'} \right] (2 \pi)^6 \delta^{(3)}(p'-p) \delta^{(3)}(\vec{k}'-\vec{k})
\\
& = 2 E_{\vec{k}} (2 \pi)^3 \delta(\vec{k}'-\vec{k}) \int \frac{\dn{3}{p}}{(2 \pi)^3} \sum_{ij} \left[ \tilde{\psi}_i^*(p') \tilde{\psi}_j(p) \tr{\Sigma^{j \dagger} \Sigma^i} \right]
\\
& = 2 E_{\vec{k}} (2 \pi)^3 \delta(\vec{k}'-\vec{k}) \sum_{ij} \delta_{ij} \tr{\Sigma^{j \dagger} \Sigma^i} = 2 E_{\vec{k}} (2 \pi)^3 \delta(\vec{k}'-\vec{k}) 
\end{align*}
using the normalization condition on $\Sigma$. Hence we get the correct relativistic normalization. Then we can consider the decay rate. For a positronium state of the form,
\[ \ket{\text{Ps}} = \sqrt{2 M} \int \frac{\dn{3}{p}}{(2 \pi)^3} \sum_{iss'} C_{i s s'} \tilde{\psi}_i(p) a^{s \dagger}_{p} b^{s' \dagger}_{-p} \ket{\Omega}  \]
then we get an amplitude for two photon decay, recalling the relativisitc normalization convention for the definition of $\M$,
\[ \M(\text{Ps} \to 2 \gamma)  = \sqrt{2 M} \int  \frac{\dn{3}{p}}{(2 \pi)^3} \sum_{iss'} C_{i s s'} \tilde{\psi}_i(p) \frac{1}{2 m} \M(e^- (\vec{p},s) + e^{+}(-\vec{p},s') \to 2 \gamma) \]
Now we let,
\[ \M^{ss'}_{\alpha \beta}(\vec{p}, \vec{k}) := \M(e^-_{s} (\vec{p}) + e^{+}_{s'}(-\vec{p}) \to \gamma_{\alpha}(\vec{k}) + \gamma_{\beta}(-\vec{k})) \]
where $\alpha, \beta = + \text{ or } -$ label the photon polarizations and $a,b = \uparrow \text{ or } \downarrow$ are spinor indices for the polarizations of the electron and the positron (recall we use Peskin's terrible convention that the spinor indices for antiparticles are flipped with respect to the physical spin).
We will need to expand this in $\vec{p}$. 
Then the decay rate for a fixed polarization is given by,
\begin{align*}
\Gamma &= \frac{1}{2} \int \frac{1}{2M} \frac{\dn{3}{k'}}{(2 \pi)^3} \frac{\dn{3}{k}}{(2 \pi)^3} \frac{1}{4 |k|^2} |\M(\text{Ps} \to 2 \gamma)|^2 (2 \pi)^4 \delta^{(4)}(k + k' - p_{\text{Ps}}) 
\\
& = \frac{2 \pi}{16 M} \int \frac{\dn{3}{k}}{(2 \pi)^3} \frac{1}{|k|^2} \delta(2 |k| - M) |\M(\text{Ps} \to 2 \gamma)|^2
\\
& = \frac{2 \pi}{16 M} \int \frac{\dn{3}{k}}{(2 \pi)^3} \frac{1}{|k|^2} \delta(2 |k| - M) |\M(\text{Ps} \to 2 \gamma)|^2
\\
& = \frac{1}{16 M} \frac{1}{(2 \pi)^2}  \frac{1}{2} \int |\M(\text{Ps} \to 2 \gamma)|^2 \d{\Omega}
\\
& = \frac{\pi}{(4 \pi)^3} \frac{1}{2 M} \int |\M(\text{Ps} \to 2 \gamma)|^2 \d{\Omega}
\end{align*}
the first $\frac{1}{2}$ from the fact that the final state photons are identical particles and another $\frac{1}{2}$ comes from the $\delta$-function. Therefore,
\[ \Gamma = \frac{1}{(4 \pi)^3} \cdot \frac{\pi}{4 m^2} \int \sum_{\alpha \beta} \left| \int \frac{\dn{3}{\vec{p}}}{(2 \pi)^3} \sum_{iss'} C_{i s s'} \tilde{\psi}_i(\vec{p}) \M^{ss'}_{\alpha \beta}(\vec{p}, \vec{k}) \right|^2 \d{\Omega} \]
Now we first compute the $S$-wave decays. Consider the $\ell = 0$ wavefunctions,
\[ \psi_n(x) = \sqrt{ \left( \frac{2}{n a_0} \right)^3 \frac{(n-1)!}{2n n!}} e^{-\rho} L^1_{n-1}(\rho) \cdot \frac{1}{\sqrt{4 \pi}} \]
where $\rho = \frac{2 r}{n a_0}$ and $a_0 = (\mu \alpha^2)^{-1}$ where $\mu$ is the reduced mass and $L^1_{n-1}$ is the generalized Laguerre polynomial which are normalized so that $L^1_{n-1}(0) = 1$. Therefore,
\[ \psi_n(0) = \sqrt{ \left( \frac{2}{n a_0} \right)^3 \frac{(n-1)!}{2n \, n!}} \cdot \frac{1}{\sqrt{4 \pi}} \]
Since these have zero gradient at the origin, to first-order in $\vec{p}$ we find,
\[ \Gamma_{{n}^1 S_0} = \frac{1}{(4 \pi)^3} \cdot \frac{\pi}{4 m^2} \int 2 |\psi_n(0)|^2 [8 e^4] \d{\Omega} \]
the first two arises from the two final allowed polarizations and the $8e^4$ is the square of,
\[ \M_{\pm \pm}(s = 0) = \mp 2 \sqrt{2} e^2 \]
and using that the other polarization states have zero amplitude by conservation of angular momentum. Alternatively, we can use the ${}^1 S_0$ state we constructed earlier which has $C_{0ss'} = \frac{1}{\sqrt{2}} \delta_{ss'}$ and that $\M^{ss'}_{\pm \pm} = \mp 2 e^2 \delta_{s s'}$ so therefore the internal sum gives the same result: $\mp 2 \sqrt{2} e^2 \psi_n(0)$.  Therefore, we get,
\[ \Gamma_{{}^1 S_0} = \frac{1}{(4 \pi)^2} \cdot \frac{\pi}{4 m^2} \cdot \left( \frac{m \alpha}{n} \right)^3 \frac{1}{2n^2} \cdot \frac{1}{4 \pi} \cdot (16 e^4) = \frac{1}{2 n^5} m \alpha^5 \]
In the case, $n = 1$ we get,
\[ \Gamma_{1^1 S_0} = \tfrac{1}{2} m \alpha^5 \]
and for $n = 2$ we get,
\[ \Gamma_{2^1 S_0} = \tfrac{1}{64} m \alpha^5 \]
Likewise the ${}^3 S_1$ state decay to two photons is forbidden by C invariance. 
\bigskip\\
Now we compute the decay of the $P$-states. From our expression for the Positronium state, we get a decay rate, 
\[ \Gamma = \frac{1}{(4 \pi)^3} \cdot \frac{\pi}{4 m^2} \int \sum_{\alpha \beta} \left| \int \frac{\dn{3}{\vec{p}}}{(2 \pi)^3} \sum_i \tilde{\psi}_i(\vec{p}) \tr{(\Sigma^\top \M_{\alpha \beta}(\vec{p}, \vec{k}))} \right|^2 \d{\Omega} \]
Since $\psi_i(0) = 0$ the zeroth-order term of $\M(\vec{p})$ integrates to zero. Therefore, we should write,
\[ \M^{ss'}(\vec{p},\vec{k}) = \M^{ss'}(0, \vec{k}) + \vec{F}^{ss'}(\vec{k}) \cdot \vec{p} + O(p^2) \]
Then to first-order,
\[ \Gamma = \frac{1}{(4 \pi)^3} \cdot \frac{\pi}{4 m^2} \int \sum_{\alpha \beta} \left| \int \frac{\dn{3}{\vec{p}}}{(2 \pi)^3} \sum_i \tilde{\psi}_i(\vec{p}) \, [\vec{p} \cdot \tr{(\Sigma^{i \top} \vec{F}_{\alpha \beta}(\vec{k}))}] \right|^2 \d{\Omega} \]
Furthermore,
\[ \int  \frac{\dn{3}{\vec{p}}}{(2 \pi)^3} \tilde{\psi}_i(\vec{p}) \, \vec{p} = - i \nabla \psi_i(x) \bigg|_{x = 0} = - i \vec{e_i} f(0) \]
because the other term $x^i \partial_j f(|x|)$ is zero at $\vec{x} = 0$. Therefore, 
\[ \Gamma = \frac{1}{(4 \pi)^3} \cdot \frac{\pi}{4 m^2} |f(0)|^2 \int \sum_{\alpha \beta} \left| \tr{(\vec{\Sigma}^{\top} \cdot \vec{F}_{\alpha \beta}(\vec{k}))} \right|^2 \d{\Omega} \]
Now we set,
\[ \Sigma^i = 
\begin{cases}
\frac{1}{\sqrt{6}} \sigma^i & j = 0
\\
\frac{1}{2} \epsilon^{ijk} n^j \sigma^k & j = 1
\\
\tfrac{1}{\sqrt{2}} h^{ij} \sigma^j & j = 2
\end{cases} \]
where $n$ is a unit vector and $h^{ij}$ is a symmetric traceless tensor such that $\sum_{ij} h^{ij} (h^{ij})^* = 1$. Note! Peskin has a mistake here, in order for the normalization to work correctly we need $\frac{1}{\sqrt{2}}$ not $\frac{1}{\sqrt{3}}$ in the $j = 2$ case. 
To compute $\tr{(A^\top \M)}$ write,
\[ \M^{s s'} = \xi^{s' \dagger} M \xi^s \] 
and therefore,
\[ \tr{(A^\top \M)} = \sum_{s s'} A_{ss'} \M^{s s'} = \sum_{s s'} A_{ss'} \xi^{s' \dagger} M \xi^s = \tr{\left( M \sum_{s s'} A_{ss'} \xi^s \xi^{s' \dagger} \right)} = \tr{(M A^\top)} \] 
We computed,
\[ \tr{(M^i_{\pm \pm} \sigma^j)} = -\frac{4 e^2}{m} \delta_{ij} \quad \quad \tr{(M^i_{\pm \mp} \sigma^j)} = -\frac{4 e^2}{m} (\delta_{i1} \mp i \delta_{i2}) (\delta_{j 1} \mp i \delta_{j 2}) \]
Therefore, for the $j = 0$ angular momentum state,
\[ \tr{(\vec{\Sigma}^{\top} \cdot \vec{F}_{\pm \pm})} = - \frac{2 \sqrt{6} e^2}{m}  \quad \quad \tr{(\vec{\Sigma}^{\top} \cdot \vec{F}_{\pm \mp})} = -\frac{4 e^2}{m} \, \tfrac{1}{\sqrt{6}} (1 - 1) = 0 \]
Therefore,
\[ \Gamma_{2^{3} P_0} = \frac{1}{(4 \pi)^2} \cdot \frac{\pi}{4m^2} |f(0)|^2 \cdot 2 \left( \frac{24 e^4}{m^2} \right) \]
and 
\[ f(r) = \frac{1}{4 \sqrt{2 \pi} a_0^{3/2}} \frac{1}{a_0} e^{-r / 2 a_0} \] 
where $a_0 = (\mu \alpha)^{-1}$ so plugging in gives,
\[ \Gamma_{2^{3} P_0} = \frac{1}{(4 \pi)^2} \cdot \frac{\pi}{4m^2}  \frac{(m/2)^5 \alpha^5}{32 \pi}  \cdot 2 \left( \frac{24 e^4}{m^2} \right) = \tfrac{3}{256} m \alpha^7 \]
For the $j = 1$ angular momentum state,
\[ \tr{(\vec{\Sigma}^{\top} \cdot \vec{F}_{\pm \pm})} = 0  \quad \quad \tr{(\vec{\Sigma}^{\top} \cdot \vec{F}_{\pm \mp})} = - \frac{4 e^2}{2m} (\mp i \epsilon^{132} n^3 \mp i \epsilon^{231} n^3) = 0 \]
and we recover the fact that the $j = 1$ does not decay into two photons.
For the $j = 2$ angular momentum state,
\[ \tr{(\vec{\Sigma}^{\top} \cdot \vec{F}_{\pm \pm})} = - \frac{4 e^2}{\sqrt{2} m} h^{ij} \delta_{ij} = 0  \quad \quad \tr{(\vec{\Sigma}^{\top} \cdot \vec{F}_{\pm \mp})} = - \frac{4 e^2}{\sqrt{2} m} (h^{11} - h^{22} \mp i h^{12} \mp i h^{21}) \]
We need to average over the possible polarization tensors $h$. However, the Peskin solutions have an error in that the standard basis of symmetric traceless tensors do not give an \textit{orthogonal} basis of spin 2 states. Indeed, consider the matrices,
\[ h_1 = \frac{1}{\sqrt{2}}
\begin{pmatrix}
1 & 0 & 0 
\\
0 & -1 & 0
\\
0 & 0  & 0
\end{pmatrix}
\quad \quad
h_2 = \frac{1}{\sqrt{2}}
\begin{pmatrix}
0 & 0 & 0
\\
0 & 1 & 0
\\
0 & 0 & -1 
\end{pmatrix} \]
these correspond to states,
\[ \ket{\psi_1} = \frac{1}{\sqrt{2}} (\ket{\hat{x}_0} \ot \ket{\hat{x}_0} - \ket{\hat{y}_0} \ot \ket{\hat{y}_0}) \quad \quad \ket{\psi_2} = \frac{1}{\sqrt{2}} (\ket{\hat{y}_0} \ot \ket{\hat{y}_0} - \ket{\hat{z}_0} \ot \ket{\hat{z}_0}) \]
but the $m = 0$ states along perpendicular axes are orthogonal. Therefore $\braket{\psi_1}{\psi_2} = \frac{1}{2}$. Instead we need to choose a basis of symmetric traceless matrices which is orthogonal for the physical states. A good choice are the states of definite $J_z$. These correspond to,
\begin{align*}
h_{+2} & = 
\frac{1}{2} \begin{pmatrix}
1 & i & 0
\\
i & -1 & 0
\\
0 & 0 & 0
\end{pmatrix}
\\
h_{+1} & = 
\frac{1}{2} \begin{pmatrix}
0 & 0 & 1
\\
0 & 0 & i
\\
1 & i & 0
\end{pmatrix}
\\
h_{0} & = 
\frac{1}{\sqrt{6}} \begin{pmatrix}
1 & 0 & 0
\\
0 & 1 & 0
\\
0 & 0 & -2
\end{pmatrix}
\\
h_{-1} & = 
\frac{1}{2} \begin{pmatrix}
0 & 0 & 1
\\
0 & 0 & -i
\\
1 & -i & 0
\end{pmatrix}
\\
h_{+2} & = 
\frac{1}{2} \begin{pmatrix}
1 & -i & 0
\\
-i & -1 & 0
\\
0 & 0 & 0
\end{pmatrix}
\end{align*}
In the calculation of $\M$ we put $\vec{k}$ along $\hat{z}$ so to perform the integral over $\vec{k}$ we instead need to integrate over the orientations of the spin $2$ particle. If we choose coordinates with $\theta = 0$ corresponding definite spin along $z$ then the amplitude squared is constant in the azimuthal angle $\phi$. Then,
\begin{align*}
\deriv{\Gamma_{h}^{\pm}}{\Omega} &= \frac{1}{(4 \pi)^3} \cdot \frac{\pi}{4m^2}  \frac{(m/2)^5 \alpha^5}{32 \pi}  \cdot \left( \frac{2 \sqrt{2} e^2}{m} \right)^2  |h^{11}(\theta) - h^{22}(\theta) \mp 2 i h^{12}(\theta)|^2
\\
& = \frac{m \alpha^7}{512} \cdot \frac{1}{4 \pi} |h^{11}(\theta) - h^{22}(\theta) \mp 2 i h^{12}(\theta)|^2 
\\
& = \frac{m \alpha^7}{512} \cdot \frac{1}{4 \pi} f_h(\theta)
\end{align*}
Applying the rotation matrix and then computing the amplitude and summing over the two nonzero photon polarization states,
\begin{align*}
f_{+2}(\theta) &= \tfrac{1}{16} (35 + 28 \cos{2 \theta} + \cos{4 \theta})
\\
f_{+1}(\theta) &= \tfrac{1}{4} (5 - 4 \cos{2 \theta} - \cos{4 \theta})
\\
f_0(\theta) &= \tfrac{1}{8} (9 - 12 \cos{2 \theta} + 3 \cos{4 \theta})
\\
f_{-1}(\theta) &= \tfrac{1}{4} (5 - 4 \cos{2 \theta} - \cos{4 \theta})
\\
f_{-2}(\theta) &= \tfrac{1}{16} (35 + 28 \cos{2 \theta} + \cos{4 \theta})
\end{align*}
Then notice that,
\[ f_{+2}(\theta) + f_{+1}(\theta) + f_0(\theta) + f_{-1}(\theta) + f_{-2}(\theta) = 8 \]
so an unpolarized collection of emitts photons uniformly as they must. Therefore, averaging over polarizations and integrating we get,
\[ \Gamma_{2^3 P_2} = \frac{m \alpha^7}{512} \cdot \frac{8}{5} = \tfrac{1}{320} m \alpha^7  \]
Morover, integrating each $f_i(\theta)$ we get the same value: $\frac{32 \pi}{5}$ and hence,
\[ \Gamma_h = \tfrac{1}{320} m \alpha^7 \]
and therefore each state in the $j = 2$ multiplet has the same decay probability. We can see why this is true by angular momentum conservation. Since $\vec{J}$ commutes with the Hamiltonian and hence the rasing and lower operators $J_{\pm}$ commute with the Hamiltonian. These annihilate the electromagnetic vacuum and hence only act on the positronium state. Therefore the different $m$-states of the $J$ multiplets of positronium must have the same overall decay rate and must decay to photon states which are connected by raising and lowering operators of total angular momentum.



\subsection{Decay to Massive $B$}

For a massive vector particle there are three possible physical polarizations. The $\epsilon^{\pm}$ are identical but the third polarization in the rest frame $\vec{\epsilon} = (0,0,1)$ is transformed via Lorentz bost into,
\[ \epsilon^\mu = (\beta \gamma, 0, 0, \gamma) \]
For the particle with momentum $\vec{k}$ and the opposite sign on $\beta$ for the other particle. From the traces, we again see that,
\[ \M_{\pm \pm}(s = 0) = \mp \left( \frac{2 \sqrt{2} e^2}{1 - \frac{m_B^2}{2 m^2}} \right) \left( \frac{k}{m} \right)
\]
and all other polarizations have zero amplitude. Similarly, for $s = 1$ the transverse polarizations give a similar result,
\begin{align*}
\cM_{\pm \pm}(s = 1) &= - \left( \frac{2 \sqrt{2} e^2}{1 - \frac{m_B^2}{2 m^2}} \right) \left( \tfrac{\vec{p}}{m} \cdot \hat{n} - \left( \tfrac{m_B^2}{2 m^2 - m_B^2}\right) (\tfrac{\vec{p}}{m} \cdot \hat{k}) (\hat{k} \cdot \hat{n}) \right)
\\
\cM_{\pm \mp}(s = 1) &= - \left( \frac{2 \sqrt{2} e^2}{1 - \frac{m_B^2}{2 m^2}} \right) (\tfrac{p_x}{m} \mp i \tfrac{p_y}{m})(n_x \mp i n_y)
\end{align*}
Now we need to compute the amplitudes with at least one transverse polarization. 
\begin{align*}
\M_{00}(s = 1) & = - \frac{2 \sqrt{2} e^2}{\left( 1 - \frac{m_B^2}{2m^2} \right)^2} \left( \frac{m_B}{m} \right)^2 \left( \frac{\vec{p} \cdot \hat{n}}{m} \right)
\\
\M_{\pm 0}(s = 1) & = - \frac{4 e^2 m_B}{2 m^2 - m_B^2} \left[ n_z (p_x \mp i p_y) + \frac{m_B^2}{2 m^2 - m_B^2} (n_x \mp i n_y) p_z  \right]
\\
\M_{0\pm}(s = 1) & = - \frac{4 e^2 m_B}{2 m^2 - m_B^2} \left[ n_z (p_x \pm i p_y) + \frac{m_B^2}{2 m^2 - m_B^2} (n_x \pm i n_y) p_z  \right]
\end{align*}
This gives $S$-wave emission for states with $m_\ell = 0$ and $s = 1$.
Notice that these go to zero as $m_B \to 0$ as they must since this is an unphysical polarization of the photon. 
\bigskip\\
Let's see if any of these amplitudes break any forbidden decays into two photons. The ${}^3 S_1$ states remain forbidden since the new polarizations only couple to nonzero momenta. Indeed C conservation shows this decay remains forbidden as does $2^1 P_1$. Indeed, the new polarizations couple to $\vec{\sigma}$ and hence give zero on $s = 0$. Thus we need only consider $2^3 P_1$. The new terms of interest are in $\M_{\pm 0}(s = 1)$ and $\M_{0 \pm}(s = 1)$ which couple $m = 0$ to $m = \pm 1$ states. This indeed allows for decay of $2^3 P_1$. This shows that the massless photon imposes an additional restriction on the angular momentum selection rules. 

\section{Parity (and C) Violating Decay}

Consider a new interaction term,
\[ \mathcal{H}_{\text{int}} = \bar{\psi} (g_s + i g_p \gamma^5) \slashed{B} \psi \]
Therefore, the new vertex contribution in the Feynman rules is,
\[ - i \Gamma^\mu = - i (g_s + i g_p \gamma^5) \gamma^\mu  \]
This term is P and C violating unless $g_p = 0$ and $B^\mu$ is a vector or $g_s = 0$ and $B^\mu$ is a pseudo-vector. 

\subsection{$e^- + e^+ \to B$}

The amplitude for this process is,
\[ \cM = - \epsilon_\mu^* \bar{v}^{s_2}(p_2) \Gamma^\mu u^{s_1}(p_1) \]
In the nonrelativistic limit in the CM frame we get,
\[ \cM = - E_{\CM} \xi'^{\dagger} \left[ g_s ( \vec{\epsilon^*} \cdot \vec{\sigma}) + i g_p \epsilon^{*0} - g_p \vec{\sigma} \cdot (\tfrac{\vec{p}}{m} \times \vec{\epsilon^*})  \right] \xi \]
For $s = 0$ we get,
\[ \cM(s = 0) = - E_{\CM} \sqrt{2} i g_p \epsilon^{*0} \]
but in the CM frame $\epsilon^{*0} = 0$ since there is no timelike polarization. Thus $\cM(s = 0) = 0$ as expected since it must decay to a spin $1$ particle. The orbital angular momentum cannot produce a $B$ through either interaction, interesting. For $g_s$ this is explained by P-invariance since ${}^1 P_1$ is even under P but for $g_p = 0$ we get P conservation if $B$ is odd so $g_s$ cannot couple to ${}^1 P_1$. For $g_p$ this is explained by C-invariance since ${}^1 P_1$ is odd under P but for $g_s = 0$ we get C conservation if $B$ is even so the term $g_p$ cannot couple to ${}^1 P_1$.
\bigskip\\
We would expect if $\M = \epsilon_\mu \M^\mu$ then $\M^0 = 0$ by the Ward identity since we are in the rest frame of the produced $B$. However, this does not happen since the Ward identity is violated by this interaction. Does this create a problem? For the $s = 1$ states we get, using the trace tricks,
\[ \cM(s = 1) = - E_{\CM} \sqrt{2} \left[ g_s (\vec{\epsilon^*} \cdot \hat{n}) - g_p (\tfrac{\vec{p}}{m} \times \vec{\epsilon^*}) \cdot \hat{n} \right] \]
If $g_s$ is nonzero then any $s = 1$ state nonvanishing at the origin can decay to form a $B$ polarized along $\hat{n}$. If $g_p = 0$ then $\mathcal{H}_{\text{int}}$ is P-invariant with $B_\mu$ odd under parity. Since the ${}^3 S_1$ states are P odd the $g_s$ coupling is alowed. Likewise $P$ states are P even explaining why $g_s$ does not couple to $P$ states. Furthermore, the ${}^3 S_1$ are odd under P but if $g_s = 0$ then the coupling is $P$ invariant with $B$ even hence the $g_p$ can only couple to states even under P.  
\bigskip\\
If $g_p$ is nonzero there is a more complicated coupling. This coupling vanishes on $S$ states since it is proportional to $\vec{p}$. Furthermore it vanishes on the $2^3 P_2$ and $2^3 P_0$ states because it does not couple states with parallel $\vec{p}$ and $\hat{n}$. This must be true since a single $B$ has $j = 1$ in its rest frame. However, by inspection, it does couple to $2^3 P_1$ with $j = 1$ as is allowed by angular momentum conservation. This cannot be explained by P or C since all $2^{3} P_j$ are even under P and C only by $j = 1$ selection. 



\begin{table}
\begin{center}
\begin{tabular}{||c | c c c c c c||} 
 \hline 
 & ${}^1 S_0$ & ${}^3 S_1$ & ${}^1 P_1$ & ${}^3 P_0$ & ${}^3 P_1$ & ${}^3 P_2$ \\
 \hline\hline
P & \checkmark & \checkmark & \xmark & \xmark & \xmark & \xmark
\\
C & \xmark & \checkmark & \checkmark & \xmark & \xmark & \xmark 
\\
$j$ & \xmark & \checkmark & \checkmark & \xmark & \checkmark & \xmark
\\
\hline
\end{tabular}\caption{$e^- + e^+ \to B$ vector ($g_s$) decays allowed by P, C, and $J$ conservation.}.
\end{center}
\end{table}

\begin{table}
\begin{center}
\begin{tabular}{||c | c c c c c c||} 
 \hline 
 & ${}^1 S_0$ & ${}^3 S_1$ & ${}^1 P_1$ & ${}^3 P_0$ & ${}^3 P_1$ & ${}^3 P_2$ \\
 \hline\hline
P & \xmark & \xmark & \checkmark & \checkmark & \checkmark & \checkmark
\\
C & \checkmark & \xmark & \xmark & \checkmark & \checkmark & \checkmark 
\\
$j$ & \xmark & \checkmark & \checkmark & \xmark & \checkmark & \xmark
\\
\hline
\end{tabular}\caption{$e^- + e^+ \to B$ pseduo-vector ($g_p$) decays allowed by P, C, and $J$ conservation.}.
\end{center}
\end{table}

The tables show that only one state is allowed to decay for each of $g_s$ and $g_p$ and these both occur at leading order.

\subsection{$e^- + e^+ \to 2 B$}

The leading-order Feynman diagrams give,
\begin{align*}
i \cM = (-i)^2 & \epsilon(k_1)^*_{\mu} \epsilon(k_2)^*_{\nu} \, \, \bar{v}^{s_2}(p_2) \left[ \Gamma^{\nu} \frac{i(\slashed{q}_1 + m)}{q_1^2 - m^2 + i \epsilon} \Gamma^\mu + \Gamma^{\mu} \frac{i(\slashed{q}_2 + m)}{q_2^2 - m^2 + i \epsilon} \Gamma^{\nu} \right] u^{s_1}(p_1)
\end{align*}
With two particles in the final state, the configuration may contribute to overall parity so we cannot simply use P to rule out decays. Indeed, we saw that states both odd -- ${}^1 S_0$ -- and even -- ${}^3 P_0$ under P decay to two photons. However, C allows us to forbid decays in the case that one coupling constant is zero so the Hamiltonian is C invariant. In either case $B^\mu$ is a vector or pseudovector i.e. has definite C so the $2 B$ state has C eigenvalue $+1$. Hence the same decays are C forbidden in the vector and pseudovector cases. Since we showed that the states which are not C protected already decay to a massive (only ${}^3 P_1$ is protected in the massless case) vector we will not get anything new in the pseudovector case. Therefore we look for C violating decays. The candidates are ${}^3 S_1$ and ${}^1 P_1$. Note that both have $j = 1$ so we need to work in the massive case to have a chance of seeing such decays.
\bigskip\\
For ${}^3 S_1$ we need to consider only the nonrelativistic limit to zeroth-order in momenta. Consider,
\begin{align*}
\bar{v}^{s_2}(p_2) & \Gamma^\nu (\gamma^\alpha + m) \Gamma^\mu u^{s_1}(p_1)
\\
& = m 
\overline{\begin{pmatrix}
\xi'
\\
-\xi'
\end{pmatrix}}
\begin{pmatrix}
0 & (g_s - i g_p) \sigma^\nu
\\
(g_s + i g_p) \bar{\sigma}^\nu & 0
\end{pmatrix}
\begin{pmatrix}
m & \sigma^\alpha
\\
\bar{\sigma}^\alpha & m
\end{pmatrix}
\begin{pmatrix}
0 & (g_s - i g_p) \sigma^\mu
\\
(g_s + i g_p) \bar{\sigma}^\mu & 0
\end{pmatrix}
\begin{pmatrix}
\xi
\\
\xi
\end{pmatrix}
\\
& = m 
\overline{\begin{pmatrix}
\xi'
\\
-\xi'
\end{pmatrix}}
\begin{pmatrix}
0 & (g_s - i g_p) \sigma^\nu
\\
(g_s + i g_p) \bar{\sigma}^\nu & 0
\end{pmatrix}
\begin{pmatrix}
\sigma^\alpha (g_s + i g_p) \bar{\sigma}^\mu & m (g_s - i g_p) \sigma^\mu 
\\
m (g_s + i g_p) \bar{\sigma}^\mu & \bar{\sigma}^\alpha (g_s - i g_p) \sigma^\mu 
\end{pmatrix}
\begin{pmatrix}
\xi
\\
\xi
\end{pmatrix}
\\
& = m 
\overline{\begin{pmatrix}
\xi'
\\
-\xi'
\end{pmatrix}}
\begin{pmatrix}
0 & (g_s - i g_p) \sigma^\nu
\\
(g_s + i g_p) \bar{\sigma}^\nu & 0
\end{pmatrix}
\begin{pmatrix}
\sigma^\alpha (g_s + i g_p) \bar{\sigma}^\mu & m (g_s - i g_p) \sigma^\mu 
\\
m (g_s + i g_p) \bar{\sigma}^\mu & \bar{\sigma}^\alpha (g_s - i g_p) \sigma^\mu 
\end{pmatrix}
\begin{pmatrix}
\xi
\\
\xi
\end{pmatrix}
\\
& = m 
\begin{pmatrix}
\xi'
\\
-\xi'
\end{pmatrix}^{\dagger}
\begin{pmatrix}
0 & I 
\\
I & 0
\end{pmatrix}
\begin{pmatrix}
(g_s - i g_p) \sigma^\nu m (g_s + i g_p) \bar{\sigma}^\mu & (g_s - i g_p) \sigma^\nu \bar{\sigma}^\alpha (g_s - i g_p) \sigma^\mu 
\\
(g_s + i g_p) \bar{\sigma}^\nu \sigma^\alpha (g_s + i g_p) \bar{\sigma}^\mu & (g_s + i g_p) \bar{\sigma}^\nu m (g_s - i g_p) \sigma^\mu
\end{pmatrix}
\begin{pmatrix}
\xi
\\
\xi
\end{pmatrix}
\\
& = m 
\begin{pmatrix}
\xi'
\\
-\xi'
\end{pmatrix}^{\dagger}
\begin{pmatrix}
(g_s + i g_p) \bar{\sigma}^\nu \sigma^\alpha (g_s + i g_p) \bar{\sigma}^\mu & (g_s + i g_p) \bar{\sigma}^\nu m (g_s - i g_p) \sigma^\mu
\\
(g_s - i g_p) \sigma^\nu m (g_s + i g_p) \bar{\sigma}^\mu & (g_s - i g_p) \sigma^\nu \bar{\sigma}^\alpha (g_s - i g_p) \sigma^\mu 
\end{pmatrix}
\begin{pmatrix}
\xi
\\
\xi
\end{pmatrix}
\\
& = m \xi'^{\dagger} \big[ (g_s + i g_p) \bar{\sigma}^\nu \sigma^\alpha (g_s + i g_p) \bar{\sigma}^\mu + (g_s + i g_p) \bar{\sigma}^\nu m (g_s - i g_p) \sigma^\mu 
\\
& \quad \quad \quad - (g_s - i g_p) \sigma^\nu m (g_s + i g_p) \bar{\sigma}^\mu - (g_s - i g_p) \sigma^\nu \bar{\sigma}^\alpha (g_s - i g_p) \sigma^\mu  \big] \xi
\\
& = m \xi'^{\dagger} \big[ (g_s^2 - g_p^2) [\bar{\sigma}^\nu \sigma^\alpha \bar{\sigma}^\mu - \sigma^\nu \bar{\sigma}^\alpha \sigma^\mu] + 2i g_s g_p [\bar{\sigma}^\nu \sigma^\alpha \bar{\sigma}^\mu + \sigma^\nu \bar{\sigma}^\alpha \sigma^\mu] + (g_s^2 + g_p^2) m [\bar{\sigma}^\nu \sigma^\mu - \sigma^\nu \bar{\sigma}^\mu] \big] \xi 
\end{align*}
To zeroth order in momenta $q_1 = (0, -\vec{k})$ and $q_2 = (0, \vec{k})$ so $q^1 = q^2 = -k^2 = -(m^2 - m_B^2)$. Since $\alpha$ is spatial the second term, the P violating term, is only nonzero if exactly one of $\mu$ or $\nu$ is spatial. Therefore, we expect P violating decays into one transverse polarization and one longitudinal polarization $B$. Therefore, 
\begin{align*}
\M & = \left( \frac{2m}{2 m^2 - m_B^2} \right) \xi'^{\dagger} \big[ (g_s^2 - g_p^2) [(\vec{\epsilon_2^*} \cdot \vec{\sigma}) (\vec{k} \cdot \vec{\sigma}) (\vec{\epsilon_1^*} \cdot \vec{\sigma}) - (\vec{\epsilon_1^*} \cdot \vec{\sigma}) (\vec{k} \cdot \vec{\sigma}) (\vec{\epsilon_2^*} \cdot \vec{\sigma})]
\\
& \quad \quad \quad \quad \quad + 2i g_s g_p [\epsilon_2^{*0} (\vec{k} \cdot \vec{\sigma})(\vec{\epsilon_1^*} \cdot \vec{\sigma}) + (\vec{\epsilon_2^*} \cdot \vec{\sigma}) (\vec{k} \cdot \vec{\sigma})  \epsilon_1^{*0} - \epsilon_1^{*0} (\vec{k} \cdot \vec{\sigma})(\vec{\epsilon_2^*} \cdot \vec{\sigma}) - (\vec{\epsilon_1^*} \cdot \vec{\sigma}) (\vec{k} \cdot \vec{\sigma})  \epsilon_2^{*0}]  \big] \xi
\end{align*}
Now,
\[ (\vec{a} \cdot \vec{\sigma})(\vec{b} \cdot \vec{\sigma}) (\vec{c} \cdot \vec{\sigma}) = (\vec{a} \cdot \vec{\sigma})(\vec{b} \cdot \vec{c} + i (\vec{b} \times \vec{c})\cdot \vec{\sigma}) = i \vec{a} \cdot (\vec{b} \times \vec{c}) + (\vec{b} \cdot \vec{c}) (\vec{a} \cdot \vec{\sigma}) - (\vec{a} \times (\vec{b} \times \vec{c})) \cdot \vec{\sigma} \]
Antisymmetrizing over $\vec{a}$ and $\vec{c}$ gives,
\begin{align*}
(\vec{a} \cdot \vec{\sigma})(\vec{b} \cdot \vec{\sigma}) (\vec{c} \cdot \vec{\sigma}) - (\vec{c} \cdot \vec{\sigma})(\vec{b} \cdot \vec{\sigma}) (\vec{a} \cdot \vec{\sigma}) & = 2i \vec{a} \cdot (\vec{b} \times \vec{c}) + [ (\vec{b} \cdot \vec{c}) (\vec{a} \cdot \vec{\sigma}) - (\vec{b} \cdot \vec{a}) (\vec{c} \cdot \vec{\sigma})] + (\vec{b} \times (\vec{c} \times \vec{a})) \cdot \vec{\sigma}
\\
& = 2i \vec{a} \cdot (\vec{b} \times \vec{c})
\end{align*}
Therefore, 
\begin{align*}
\M & = \left( \frac{2i}{1 - \frac{m_B^2}{2m^2}} \right) \xi'^{\dagger} \big[ (g_s^2 - g_p^2) [\tfrac{\vec{k}}{m} \cdot ( \vec{\epsilon_1^*} \times \vec{\epsilon_2^*})] + 2 i g_s g_p [ \tfrac{\vec{k}}{m} \times ( \epsilon_2^{*0}  \vec{\epsilon_1^*} - \epsilon_1^{*0} \vec{\epsilon_2^*})] \cdot \vec{\sigma} ] \big] \xi
\end{align*}
We get a reduction of the main term in $\M(s = 0)$ but only transverse polarizations can be emitted from ${}^1 S_0$ states still (WHY IS THERE SOME CONSERVATION?) Our trace tricks give,
\begin{align*}
\M(s=1) & = - \left( \frac{4 \sqrt{2} g_s g_p}{1 - \frac{m_B^2}{2m^2}} \right)  [ \tfrac{\vec{k}}{m} \times ( \epsilon_2^{*0}  \vec{\epsilon_1^*} - \epsilon_1^{*0} \vec{\epsilon_2^*})] \cdot \hat{n}
\end{align*}
\begin{align*}
\cM_{\pm \pm}(s = 1) & = 0
\\
\cM_{\pm \mp}(s = 1) & = 0
\\
\cM_{0 0}(s = 1) & = 0
\\
\cM_{\pm 0}(s = 1) & = \pm 8 i g_s g_p \left( \frac{m}{m_B} \right) \left( \frac{m^2 - m_B^2}{2m^2 - m_B^2} \right) (n_x \mp i n_y) 
\\
\cM_{0 \pm}(s = 1) & = \mp 8 i g_s g_p \left( \frac{m}{m_B} \right) \left( \frac{m^2 - m_B^2}{2m^2 - m_B^2} \right) (n_x \pm i n_y) 
\end{align*}
These alow decay of ${}^3 S_1$ into $P$-wave (since the total probability for $B + B$ production varies as $\cos^2{\theta}$ away from the $s = 1$ spin axis) $j = 1$ with $s = 1$ state of two $B$ particles.
\bigskip\\
Finally, we need to consider the decay of ${}^1 P_1$. To do this we need to expand $\M(s = 0)$ to second-order in $\vec{p}$. We get,
\begin{align*}
\cM_{\pm \pm}(s = 1) & = \pm \frac{2 \sqrt{2} (g_p^2 - g_s^2) \sqrt{1 - \frac{m_B^2}{m^2}}}{1 - \frac{m_B^2}{2m^2}} 
\\
\cM_{\pm \mp}(s = 1) & = 0
\\
\cM_{0 0}(s = 1) & = 0
\\
\cM_{\pm 0}(s = 1) & = + 8 i g_s g_p \left( \frac{m}{m_B} \right) \left( \frac{\sqrt{m^2 - m_B^2}}{2m^2 - m_B^2} \right) (p_x \mp i p_y) 
\\
\cM_{0 \pm}(s = 1) & = - 8 i g_s g_p \left( \frac{m}{m_B} \right) \left( \frac{\sqrt{m^2 - m_B^2}}{2m^2 - m_B^2} \right) (p_x \pm i p_y) 
\end{align*}
The transverse polarizations only couple to zero momentum. However, we see that the orbital angular momentum can now couple though the $\M_{\pm 0}$ and $\M_{0 \pm}$ amplitudes to create spin $1$ particles. This alows for the decay of ${}^1 P_1$ into $P$-wave $j = 1$ with $s = 1$ state of two $B$-particles.
\bigskip\\
Questions: notice that these amplitudes diverge as $m_B \to 0$. Does this show that the parity violating coupling is somehow inconsistent for massless particles. It is indeed not gauge invariant so the Ward identity is violated so we may not expect the longitudal polarization to cancel in the limit. Is this a problem? Is there any reason that these extra coupling for $n$ and for $p$ look very similar up to signs and a factor of $k$?


\section{The Ward Identity}

Peskin shows that the Ward identity implies $Z_1 = Z_2$ where $Z_1^{-1}$ is the vertex factor of QED and $Z_2$ is the electron field strength renormalization. This means that the electric charge $e$ in terms of the bare charge $e_0$ is renormalized as follows,
\[ e = Z_1^{-1} Z_2 \sqrt{Z_3} e_0 = \sqrt{Z_3} e_0 \]
See Peskin 10.37. Why is this cancellation necessary. I think it is to preserve gauge invariance of the renormalized Lagrangian. The bare Lagrangian,
\[ \L = -\tfrac{1}{4} F_{0 \mu \nu} F_0^{\mu \nu} + \bar{\psi}_0(i \slashed{\partial} - e_0 \slashed{A}_{0} - m_0) \psi_0 \]
Now we renomalize, first doing field strength renormalization via rescaling,
\[ \psi_0 = Z_2^{1/2} \psi \quad \quad A_0^\mu = Z_3^{1/2} A_\mu \]
and rearranging into counterterms to get,
\[ \L = - \tfrac{1}{4} F_{\mu \nu} F^{\mu \nu} + \bar{\psi}( i \slashed{\partial} - e \slashed{A})\psi  - \tfrac{1}{4} \delta_3 F_{\mu \nu} F^{\mu \nu} + \bar{\psi}( i \delta_2 \slashed{\partial} - e \delta_1 \slashed{A} - \delta_m)\psi
\]
where,
\[ \delta_3 = Z_3 - 1 \quad \delta_2 = Z_2 - 1 \quad \delta_1 = Z_1 - 1 = (e_0/e) Z_2 Z_3^{1/2} - 1 \quad \delta_m = Z_2 m_0 - m \]
Now the gauge tansformation takes,
\[ \psi \mapsto e^{i \alpha} \psi \quad A^\mu \mapsto A^\mu - e_0^{-1} Z_3^{-1/2} \partial^\mu \alpha \]
where the \textit{bare} charge appears as well as the field strength renormalization since this is the gauge transformation of $A_0^{\mu}$ multiplied by $Z_{3}^{-1/2}$. But notice that the Ward identity forces $e = e_0 Z_3^{1/2}$ and therefore,
\[ \psi \mapsto e^{i \alpha} \psi \quad A^\mu \mapsto A^\mu - e^{-1} \partial^\mu \alpha \]
meaning that the renormalized fields satisfy gauge invariance! Notice, there is no reason (or need) for the coincidence $Z_1 = Z_2$ in a non-gauge theory as we can simply renormalize away a shift in the coupling constant using the vertex counterterm. In QED it is actually photon field strength renormalization that renormalizes $e$ and accounts for its running so only the photon self-energy is needed in the Callan–Symanzik equation to compute the $\beta$ function. A consequence of this is that the coupling constants of $A^\mu$ to each charged species are renormalized exactly the same way (since the entire renormalization is via photon-self energy which does not depend on the particular vertex defining the coupling to the charged species) meaning there is a universal electric interaction strength for all species. Thus in the renormalized theory $A^\mu$ couples to the conserved Noether charge current as it must since it satisifes a full gauge symmetry even after renormalization. 
\bigskip\\
I think Peskin's treatment is a bit backwards. First he computes the bare vertex factor $\Gamma^\mu$ and sees that it gives a form factor $F_1$ which is divergent at $q^2 = 0$ in violation of the principal that $F_1(0) = 1$ since this corresponds to $e$ being the physical electric charge. Then he uses the LSZ reduction formula to see that we didn't include the effects of electron field-strength renormalization $Z_2$ whcih cancels the divergence and exactly gives $F_1(0) = 1$ via the Ward identity forced equality $Z_1 = Z_2$. However, this is still wrong! Peskin did not include the photon field-strength renormalization in this analysis. We should really have $Z_3^{1/2} Z_2 \Gamma^\mu$ as our vertex factor and this $Z_3^{1/2}$ accomodates for the fact that we're still using the bare $e_0$ and shifts it to $e$ the physical charge. The difference between $e_0$ and $e$ is pushed under the rug in chapter 7.

\section{Coulomb Scatting in QFT (Peskin 4.4 and 5.1)}

Consider the interaction term in the Hamiltonian,
\[ H_I = \int \dn{3}{x} e \bar{\psi} \gamma^\mu A_\mu \psi \]
where $A_\mu$ is a source field (i.e. not quantized). Then we compute the $S$-matrix elements using the convention,
\[ S = I + i T \]
and to leading-order,
\[ \bra{p' s'} i T \ket{p s} = \bra{p' s'} - i \int \d{t} H_I \ket{ps} \]
However,
\[ \psi(x) \ket{ps} = u^s(p) e^{-i p x} \]
and therefore,
\[  \bra{p' s'} i T \ket{p s} = - i e \bar{u}^{s'}(p') \gamma^\mu u(p) \int e^{-i(p - p') x} A_\mu \dn{4}{x} = - i e \bar{u}^{s'}(p') \gamma^\mu u(p) \wt{A}_\mu(q) \]
where $q = p - p'$. 

\subsection{Time-independent potentials}

Suppose that $A_\mu$ is a time-independent potential the particle is scattering off. Then from the Hamiltonian or Lagrangian formalism we expect that the energy of the particle is conserved. Indeed, the time component of the Fourier transform gives a $\delta$-function since $A_\mu$ is constant in $t$. Therefore we define the scattering amplitude via the formula,
\[ \bra{p' s'} i T \ket{p s} = (2 \pi) \delta(E_f - E_i) i \M \]
Therefore, we get,
\[ \M = - e \bar{u}^{s'}(p') \gamma^\mu u(p) \int e^{-i(p - p') x} A_\mu \dn{4}{x} = - i e \bar{u}^{s'}(p') \gamma^\mu u(p) \wt{A}_\mu(\vec{q}) \]
where now $\wt{A}_\mu(\vec{q})$ is the spatial Fourier transform evaluated at the 3-vector $\vec{q} = \vec{p} - \pvec{p}'$. There is an overall $\delta$-function enforcing energy conservation because of time-indpendence but not momentum conservation because the potential can absorb arbitrary momentum. 

\subsection{Building Wavepackets}

The incident wave packet $\ket{\psi}$ is built as follows,
\[ \ket{\psi}_{\text{in}} = \int \frac{\dn{3}{k}}{(2\pi)^3} \frac{1}{\sqrt{2 E_k}} \tilde{\psi}(\vec{k}) e^{i \vec{b} \cdot \vec{k}} \ket{\vec{k}}_{\text{in}} \]
with impact parameter $\vec{b}$. Then we consider asymtotic final states of pure momentum,
\[ {}_{\text{out}} \bra{\phi} = \int \frac{\dn{3}{p}}{(2 \pi)^3} \frac{1}{\sqrt{2 E_p}} \tilde{\phi}(\vec{p}) \, {}_{\text{out}} \bra{\vec{p}} \]
Then the probability of scattering into a sector of momentum space is,
\begin{align*}
P(\vec{b}) &= \frac{\dn{3}{p}}{(2 \pi)^3} \frac{1}{2 E_p} \left| {}_{\text{out}} \braket{\vec{p}}{\psi}_{\text{in}} \right|^2 
\\
& = \frac{\dn{3}{p}}{(2 \pi)^3} \frac{1}{2 E_p} \int \frac{\dn{3}{k'}}{(2\pi)^3}  \frac{\dn{3}{k}}{(2\pi)^3} \frac{1}{\sqrt{4 E_k E_{k'}}}   \tilde{\psi}^*(\vec{k}')  \tilde{\psi}(\vec{k}) e^{i \vec{b} \cdot (\vec{k} - \vec{k}')}
({}_{\text{out}} \braket{\vec{p}}{\vec{k}}_{\text{in}}) ({}_{\text{out}} \braket{\vec{p}}{\vec{k}'}_{\text{in}})^*
\end{align*}
Now we define,
\[ \d{\sigma} = \int \dn{2}{b} P(\vec{b}) \]
And hence, the integration over $\vec{b}$ gives a factor of $(2 \pi)^2 \delta^{(2)}(k^\perp - k'^\perp)$. Furthermore, we showed that
\[ ({}_{\text{out}} \braket{\vec{p}}{\vec{k}}_{\text{in}}) = i \M(\vec{k} \to \vec{p}) (2 \pi) \delta(E_k - E_p) \]
and likewise,
\[ ({}_{\text{out}} \braket{\vec{p}}{\vec{k}'}_{\text{in}})^* = -i \M^*(\vec{k}' \to \vec{p}) (2 \pi) \delta^{(4)}(E_{k'} - E_p) \]
We use this second delta function and the delta function arising from integration over $\vec{b}$. Inspect,
\[ \int \d{k'_z} \delta(E_{k'} - E_p) = \int \d{k'_z} \, \delta \left( \sqrt{(k'^\perp)^2 + k'^2_z + m^2} - E_p \right) = \left| \frac{k'_z}{E_{k'}} \right|^{-1} = \frac{1}{v'} \]
Therefore,
\begin{align*}
\d{\sigma} & = \frac{\dn{3}{p}}{(2 \pi)^3} \frac{1}{(2 E_p)^2} \int \frac{\dn{3}{k}}{(2\pi)^3} \frac{1}{v'}  \tilde{\psi}^*(\vec{k}')  \tilde{\psi}(\vec{k}) \M(\vec{k} \to \vec{p}) \M^*(\vec{k}' \to \vec{p}) (2 \pi) \delta(E_k - E_p) 
\end{align*}
where we fix $k^\perp = k'^\perp$ and $E_{k'} = E_p = E$ hence since $E_{k} = E_p$ we have $k_z' = \pm k_z$. If the wavefunctions are well-localized in momentum space we can ignore the $k'_z = - k_z$ solution to the $\delta$-functions and take $\vec{k} = \vec{k}'$. Therefore, if the wavefunction is well-peaked we can more smooth functions through the integral over $\vec{k}$ to get,
\begin{align*}
\d{\sigma} & = \frac{\dn{3}{p}}{(2 \pi)^3}  \frac{1}{(2 E)^2 v} (2 \pi) \delta(E_k - E_p)  | \M(\vec{k} \to \vec{p})|^2  \int \frac{\dn{3}{k}}{(2\pi)^3} \tilde{\psi}^*(\vec{k})  \tilde{\psi}(\vec{k}) 
\\
& = \frac{\dn{3}{p}}{(2 \pi)^3} \frac{1}{(2 E)^2} \cdot \frac{1}{v} \cdot  (2 \pi) \delta(E_k - E_p) \, | \M(\vec{k} \to \vec{p})|^2
\end{align*}
Now we integrate over $\vec{p}$ to get the full cross section for scattering with this $S$-matrix element,
\begin{align*}
\sigma & = \int \frac{\dn{3}{p}}{(2 \pi)^3}  \frac{1}{(2 E)^2 v} (2 \pi) \delta(E_k - E_p)  | \M(\vec{k} \to \vec{p})|^2
\\
& = \frac{1}{4 \pi^2 (2 E)^2 v} \int \d{|p|} \d{\Omega} |\vec{p}|^2  \delta(E_k - E_p) | \M(\vec{k} \to \vec{p})|^2 
\\
& = \frac{1}{16 \pi^2 E^2 v} \cdot \frac{E}{|k|} |k|^2 \int |\M(\vec{k} \to \vec{p})|^2  \d{\Omega}
\\
& = \frac{1}{(4 \pi)^2} \int |\M(\vec{k} \to \vec{p})|^2  \d{\Omega}
\end{align*}
We have constraints $|p| = |k|$ and $E_p = E_k$ but do not constrain the direction of $\vec{p}$.

\subsection{Coulomb Potential}

Consider,
\[ A^\mu(x) = \left( \frac{Z e}{4 \pi r}, 0 \right) \]
Then we compute the Fourier transform,
\begin{align*}
\wt{A}_0(\vec{q}) &= \int e^{-i q x} \frac{Z e}{4 \pi r} \dn{3}{r}
\\
& = \frac{Z e}{4 \pi} \int e^{-i r |q| \cos{\theta}} 2 \pi r \, \d{r} \, \d{\cos{\theta}}
\\
& = \frac{Z e}{2} \int \frac{1}{-i r |q|} \left[ e^{-i r |q|} - e^{i r |q|} \right] r  \, \d{r} 
\\
& = \frac{Z e}{|q|} \int_0^{\infty} \sin{(r |q|)} \d{r} = \frac{Z e}{|q|^2} 
\end{align*}
where we must compute these integrals in the sense of distributions. 

\subsection{The scattering ampltitude}



Now,
\[ |\M|^2 = \left( \frac{Z e}{|q|^2} \right)^2 e^2 | u^{s'}(p') \gamma^0 u(p) |^2 \]
We will compute the cross section in terms of the scattering angle $\theta$ between $p$ and $p'$. Since $E_f = E_i$ we see that $|p| = |p'|$ and hence,
\[ |q| = 2 |p| \sin{\tfrac{\theta}{2}} \]
First we compute the non-relativistic limit in which,
\[ u^{s'}(p') \gamma^0 u(p) \approx 2 m \xi'^\dagger \xi \]
therefore the spins are unaffected in the scattering Approximating $|p'| = |p| \approx m v$ we get,
\[ |\M|^2 = \frac{Z^2 e^4}{(2 m)^4 v^4 \sin^4{\frac{\theta}{2}}} (2 m)^2 \]
\[ \deriv{\sigma}{\Omega} = \frac{1}{(4 \pi)^2} |\M|^2 = \frac{\alpha^2 Z^2}{4 m^2 v^4 \sin^4(\frac{\theta}{2})} \]
Putting in dimensionful constants we get,
\[ \deriv{\sigma}{\Omega} = \frac{\alpha^2 Z^2 \hbar^2 c^2}{4 (mc^2)^2 (v/c)^4 \sin^4{\frac{\theta}{2}}} \]
Now we compute the fully relativistic case. We need to compute the spin average,
\begin{align*}
\tfrac{1}{2} \sum_{\text{spins}} | \bar{u}^{s'}(p') \gamma^0 u^s(p)|^2 & = \tfrac{1}{2} \tr{(\gamma^0 (\slashed{p}' + m) \gamma^0 (\slashed{p} + m))}
\\
& = \tfrac{1}{2} \tr{(\gamma^0 \slashed{p}' \gamma^0 \slashed{p} + \gamma^0 m \gamma^0 \slashed{p} + \gamma^0 \slashed{p}'\gamma^0 m + m^2)}
\end{align*}
The middle terms have an odd number of $\gamma$ matrices and thus have zero trace. Furthermore,
\[ \gamma^0 \slashed{p}' \gamma^0 \slashed{p} = - \gamma^0 \gamma^0 \slashed{p}' \slashed{p} + \gamma^0  \{ \slashed{p}', \gamma^0 \} \slashed{p} = - \slashed{p}' \slashed{p} + 2 \gamma^0 p'^0 \slashed{p} \]
Therefore, we get, using that $\tr{(\gamma^\mu \gamma^\nu)} = 4 g^{\mu \nu}$
\begin{align*}
\tfrac{1}{2} \sum_{\text{spins}} | \bar{u}^{s'}(p') \gamma^0 u^s(p)|^2 & = \tfrac{1}{2} \left[ - \tr{(\slashed{p}' \slashed{p})} + 2 p'^0 \tr{(\gamma^0 \slashed{p})} + 4m^2 \right]
\\
& = - 2 p' \cdot p + 4 p'^0 p^0 + 2 m^2
\\
& = 2 E^2 + 2 \vec{p} \cdot \pvec{p}' + 2 m^2 
\\
& = 2 m^2(1 + \gamma^2 + \beta^2 \gamma^2 \cos{\theta})
\\
& = 2 m^2 \gamma^2 ([1 - \beta^2] + 1 + \beta^2 \cos{\theta}) 
\\
& = 2 m^2 \gamma^2 (2 - \beta^2 (1 - \cos{\theta}))
\\
& = 4 E^2 (1 - \beta^2 \sin^2{\tfrac{\theta}{2}})
\end{align*}
Then we have,
\[ |\M|^2 = \frac{Z^2 e^4}{(2 |p| \sin{\tfrac{\theta}{2}})^4} 4 E^2 (1 - \beta^2 \sin^2{\tfrac{\theta}{2}}) = \frac{Z^2 e^4}{4 |p|^2 \beta^2 \sin^4{\frac{\theta}{2}}} (1 - \beta^2 \sin^2{\tfrac{\theta}{2}}) \]
Therefore,
\[ \deriv{\sigma}{\Omega} = \frac{|\M|^2}{(4 \pi)^2} = \frac{Z^2 \alpha^2}{4 |p|^2 \beta^2 \sin^4{\frac{\theta}{2}}} (1 - \beta^2 \sin^2{\tfrac{\theta}{2}}) \]
Putting in the dimensionful parameters we get,
\[ \deriv{\sigma}{\Omega} = \frac{\alpha^2 Z^2 \hbar^2}{4 |p|^2 \beta^2 \sin^4{\frac{\theta}{2}}} (1 - \beta^2 \sin^2{\tfrac{\theta}{2}}) = \frac{\alpha^2 Z^2 \hbar^2}{4 m^2 c^2 \gamma^2 \beta^4 \sin^4{\frac{\theta}{2}}} (1 - \beta^2 \sin^2{\tfrac{\theta}{2}}) \] 

\subsection{Helicity Structure of the Scattering Cross Section}

Notice that this formula for $e^{-}$-scattering off a hard Coulomb target is well-defined in the limit $m \to 0$ for fixed (relativistic) momentum (unlike the non-relativistic case). For $m \to 0$ and hence $\beta \to 1$ we get,
\[ \deriv{\sigma}{\Omega} = \frac{\alpha^2 Z^2 \hbar^2 \cos^2{\frac{\theta}{2}}}{4 |p|^2 \sin^4{\frac{\theta}{2}}} \]
We want to explain this additional structure in terms of relativity and the spin/helicity structure. We need to compute,
\[ u^{s'}(p') \gamma^0 u^s(p) \]
in the limit $m \to 0$. We can choose states of definite helicity as our basis. We choose $p = (E, 0, 0, E)$ then $\xi^{\uparrow} = \xi^{+}$ satisfies $(\vec{p} \cdot \vec{\sigma}) \xi^{+} = E \xi^{+}$ and likewise $\xi^{-} = \xi^{\downarrow}$ is the negative helicity eigenstate $(\vec{p} \cdot \vec{\sigma}) \xi^{-} = - E \xi^{-}$. These give definite Helicity spinors,
\[ u^{+}(p) = \sqrt{2 E}
\begin{pmatrix}
0
\\
0
\\
1
\\
0
\end{pmatrix}
\quad \quad
u^{-}(p) = \sqrt{2 E}
\begin{pmatrix}
0
\\
1
\\
0
\\
0
\end{pmatrix}  \]
Now we need to rotate these spinors to get the definite helicity states for $p' = (E, E \cos{\theta}, 0, E \cos{\theta})$. Rotation around the $y$-axis is generated by,
\[ e^{-i \Sigma_2 \theta} =
\begin{pmatrix}
\cos{\tfrac{\theta}{2}} & - \sin{\tfrac{\theta}{2}} & 0 & 0
\\
\sin{\tfrac{\theta}{2}} & \cos{\tfrac{\theta}{2}} & 0 & 0
\\
0 & 0 & \cos{\tfrac{\theta}{2}} & - \sin{\tfrac{\theta}{2}}
\\
0 & 0 & \sin{\tfrac{\theta}{2}} & \cos{\tfrac{\theta}{2}}
\end{pmatrix} \] 
and therefore,
\[ u^{+}(p') = \sqrt{2 E}
\begin{pmatrix}
0
\\
0
\\
\cos{\tfrac{\theta}{2}}
\\
\sin{\tfrac{\theta}{2}}
\end{pmatrix}
\quad \quad
u^{-}(p') = \sqrt{2 E}
\begin{pmatrix}
- \sin{\tfrac{\theta}{2}}
\\
\cos{\tfrac{\theta}{2}}
\\
0
\\
0
\end{pmatrix}  \]
Therefore, 
\begin{align*}
\bar{u}^{+}(p') \gamma^0 u^{+}(p) & = 2 E \cos{\tfrac{\theta}{2}} 
\\
\bar{u}^{+}(p') \gamma^0 u^{-}(p) & = 0
\\
\bar{u}^{-}(p') \gamma^0 u^{+}(p) & = 0
\\
\bar{u}^{-}(p') \gamma^0 u^{-}(p) & = 2 E \cos{\tfrac{\theta}{2}} 
\end{align*}
Therefore, the sum over final polarization and average over initial spins gives,
\[ 2 E \cos{\tfrac{\theta}{2}} \]
which is indeed the square-root of the extra factor that appears in the scattering cross section compared to the non-relativistic cross section for Rutherford scattering. Furthermore, notice that the process is Helicity conserving and therefore not angular momentum conserving (for $\theta \neq 0$) similar to how it does not conserve momentum. The extra factor of $\cos{\tfrac{\theta}{2}}$ can be interpreted as the amplitude to connect positive (resp. negative) helicity states after rotation.

\section{QFT notes}

\subsection{QED conserves chirality}

Note that a vector interaction with a Dirac fermion like QED preserved chirality. Indeed, the interaction term,
\[ e \bar{\psi} \slashed{A} \psi = e (\chi_L^{\dagger} A_\mu \bar{\sigma}^\mu \chi_L + \chi_R^{\dagger} A_\mu \sigma^\mu \chi_R) \]
which couples only fermions of the same chirality together. Hence for massless fermions interacting with QED all processes preserve helicity. We saw this for example with Compton scattering where the helicity is conserved in the limit $m \to 0$ and with relativistic Coulomb scattering where again helicity is conserved in the limit $m \to 0$.

\subsection{Weinberg-Witten Theorem}

\section{Homological. numerical, and algebraic equivalence}

\begin{defn}

\end{defn}

\begin{prop}
Let $X$ be a variety then $\fPic^0_X$ parametrizes algebraically trivial line bundles and hence Cartier divisors. If $X$ is locally factorial this coincides with algebraically trivial Weil divisors.
\end{prop}

\begin{prop}
Let $X$ be a proper normal variety. Then for Weil divisors the following are equivalent:
\begin{enumerate}
\item algebraic and $\Z$-homological equivalence
\item numerical and $\Q$-homological equivalence
\end{enumerate}
\end{prop}

SHOW THAT THESE RESULTS FAIL FOR CODIM $>1$ CLASSES. EXAMPLE OF LAZARSFELD?

\section{Fibral Conditions}

``Fibral'' criteria have two meanings:
\begin{enumerate}
\item if $f : X \to Y$ is a map of good enough $S$-schemes then $f$ has property $\cP$ iff all $f_s$ has property $P$ for all $s \in P$
\item if $f : X \to Y$ is good enough then $f$ has property $\cP$ iff all $f_y : X_y \to \Spec{\kappa(y)}$ have property $\cP$.
\end{enumerate}
We will consider both types of fibral criteria in this order. We start with type (a) critera for flatness, smoothness, and being an isomorphism.

\subsection{Conrad Math 248B Homework 8 Problem 2}

Let $S$ be a scheme and $f : X \to Y$ a map between flat and locally finitely  presented $S$-schemes.

\subsection{(i)}

For $s \in S$, prove that if $f_s : X_s \to Y_s$ is flat at $x \in X_s$ then $f$ is flat at $x$.
\bigskip\\
This is local on the source and target so we reduce to the case that $X,Y,S$ are affine. By finite presentation we may further assume that $S$ is finite type over $\ZZ$ by spreading out. Recall the following theorem,

\begin{theorem}[Mat CRT, 22.5]
Let $A \to B$ be a local map of local rings and $u : M \to N$ a morphism of finite $B$-modules. If $N$ is flat over $A$ then the following are equivalent,
\begin{enumerate}
\item $u$ is injective and $N/u(M)$ is flat over $A$
\item $\tilde{u} : M \ot_A \kappa_A \to N \ot_A \kappa_A$ is injective.
\end{enumerate}
\end{theorem}

Note that $f$ is locally finitely presented (since the diagonal of a lfp morphism is lfp and so are the composition of two and the base change). Therefore, shrinking further we can write $X$ in affine space over $Y$ to get $X \embed \A^n_Y$. Localize so that $R$ is a local ring. Indeed let $S = \Spec{R}$ and $Y = \Spec{A}$ and $X = \Spec{B/J}$ where $B = A[x_1, \dots, x_n]$ and $J = (f_1, \dots, f_r)$ with $x \in X$ a maximal ideal $\m \subset B$ containing $J$ and $f(\m) = \p$. Now we apply the theorem from Matsumura to the localization $A_{\p} \to B_{\m}$ and the map of modules $\tilde{u} : J_\m \to B_\m$. Then $\tilde{u} = (u \ot_R (R / \m_R)) \ot_{A/m_R A} (A / \m_A)$ but $A \to B$ becomes flat after applying $- \ot_R (R / \m_R)$. Note that,
\[ 0 \to J_\m \to B_\m \to B_\m/J_\m \to 0 \]
stays exact after applying $- \ot_R (R / \m_R)$ since $B_\m/J_\m$ is $R$-flat by assumption. Also $(B_\m / J_\m) \ot_R (R / \m_R)$ is $(A/m_R A)$-flat by assumption hence,
\[ 0 \to J_\m \ot_{R} (R / \m_R) \to B_\m \ot_{R} (R / \m_R) \to (B_\m / J_\m) \ot_B (R / \m_R) \to 0 \]
remains flat after applying $- \ot_{A / \m_R A} (A / \m_A)$ and therefore $\tilde{u}$ is injective. Therefore by the theorem $(B / J)_\m$ is $A_\p$-flat proving the claim.
\bigskip\\
Now we prove:

\begin{prop}
Let $f : X \to Y$ be a morphism of lfp flat $S$-schemes. If $f_{\bar{s}} : X_{\bar{s}} \to Y_{\bar{s}}$ is flat for each geometric fiber over $S$ then $f$ is flat.
\end{prop}

\begin{proof}
By the above it suffices to show that $f_s$ is flat for each $s \in S$. This follows from the flatness of $f_{\bar{s}}$ by faithfully flat descent. Indeed, let $A \to B$ be a map of $k$-algebras such that $A_{\bar{k}} \to B_{\bar{k}}$ is flat. Then since $B \to B_{\bar{k}}$ is faithfully flat and $A \to A_{\bar{k}} \to B_{\bar{k}}$ hence $A \to B \to B_{\bar{k}}$ is flat we see that $A \to B$ is flat. 
\end{proof}

\subsection{(ii)}

Prove that if $f_s$ is smooth (resp. \etale) for all $s \in S$ then $f$ is smooth (resp. \etale). Likewise for $f_{\bar{s}}$ replacing $f_s$.
\bigskip\\
By implication of properties, $f$ is lfp.
By part (i) we see that $f$ is flat. Since $f_s$ is smooth we see that the geometric fibers $X_{\bar{y}}$ of $f$ are regular (since these are also the geometric fibers of $f_s$ or $f_{\bar{s}}$) and hence $f$ is smooth.
\bigskip\\
Alternatively we can use $\Omega_{X/Y}$. It suffices to show that $\Omega_{X/Y}$ is locally free of the correct rank. By checking over $f_s$ we see that $\Omega_{X/Y}$ has the correct rank. Hence if $X$ is reduced we would win immediately since a constant rank coherent sheaf is a vector bundle. As above we reduce to $A \to B$ a map of flat $R$-algebras with $B = A[x_1, \dots, x_n]/J$ then consider the sequence,
\[ J/J^2 \to B^n \to \Omega_{B/A} \to 0 \]
we need to show that $J/J^2 \to B^n$ is locally a split injection. Since this is true after applying $- \ot_R \kappa_R$ and hence after applying $- \ot_B \kappa_B = (- \ot_R \kappa_R) \ot_{B/\m_R B} \kappa_B$ it suffices to prove the following lemma.

\begin{lemma}
Let $(A, \m, \kappa)$ be a local ring. Let $\varphi : M \to N$ be a map from a finitely presented $A$-module $M$ to a finite projective $A$-module $N$. Then the following are equivalent,
\begin{enumerate}
\item $\varphi$ is a split injection
\item $\varphi \ot_A \kappa$ is an injection.
\end{enumerate}
\end{lemma}

\begin{proof}
(a) clearly implies (b). Assume (b). Since injections over $\kappa$ are split, we can choose a section $N \ot_A \kappa \to M \ot_A \kappa$ and consider,
\begin{center}
\begin{tikzcd}
& & M \arrow[d, two heads]
\\
N \arrow[rru, dashed] \arrow[r] & N \ot_A \kappa \arrow[r] & M \ot_A \kappa
\end{tikzcd}
\end{center}
the lift exists since $N$ is projective. Hence we get a map $\psi : N \to M$ such that $\psi \circ \varphi : M \to M$ is an endomorphism which equals the identity over $\kappa$. Hence $\varphi \circ \varphi$ is an isomorphism by Lemma~\ref{lemma:endo_local_ring} so $\varphi$ is a split injection. Indeed we just need to modify the map $\psi : M \to N$ to $\psi' = (\psi \circ \varphi)^{-1} \circ \psi$ and then clealy $\psi'$ is a section since $\psi' \circ \varphi = (\psi \circ \varphi)^{-1} \circ (\psi \circ \varphi) = \id$.
\end{proof}

\subsection{(iii)}

Prove that if $f$ is finite type and $f_s$ is an isomorphism for all $s \in S$ then $f$ is quasi-finite flat with fibral-degree $1$.
\bigskip\\
Isomorphisms are smooth and hence by the previous part we conclude that $f$ is smooth. Furthermore, it is finite-type and its fibers are the fibers of some $f_s$ hence are a single point with degree $1$ so we conclude that $f$ is quasi-finite flat with constant fibral-degree $1$.

\subsection{(iv)}

Prove the following lemma of Deligne and Rapoport.

\begin{prop}
Let $f : X \to Y$ be a quasi-finite\footnote{By definition this means $f$ is finite type. Indeed it means finite type and finite fiber degree see \chref{https://stacks.math.columbia.edu/tag/02NH}{Tag 02NH} } separated map of noetherian schemes that is flat with constant fibral degree. Then $f$ is finite.
\end{prop}

\begin{proof}
Since $f$ is quasi-finite, to prove that $f$ is finite it suffices to show it is proper. Therefore we must simply verify the valuative criterion of properness. Hence reduce to the case that $Y = \Spec{R}$ is a dvr (we can further assume that $X$ has a $K = \Frac{R}$ point and find a section but it suffices to just show that $X \to \Spec{R}$ is proper since then this property holds). Since $f : X \to Y$ is quasi-finite separated it factors by ZMT as an open immersion $X \embed \ol{X}$ and a finite map $\ol{X} \to \Spec{R}$. We need to show that $f$ has a section. Consider the scheme-theoretic closure $Z \subset \ol{X}$ of the generic fiber. Because $X \to \Spec{R}$ is flat $X \subset Z$ and $Z$ is $R$-flat and finite since it is a closed subscheme of a finite $R$-scheme. Hence the fibral degree of $Z$ is constant. Furthermore, $X \subset \ol{X}$ is open so $X \subset Z$ is open but the fibral degree of $X$ is constant by assumption so $X = Z$ 
\end{proof}

As a consequence, if $f$ in (iii) is also separated then $f$ is finite of degree $1$ and hence an isomorphism.

\begin{rmk}
Separatedness is necessary. For example, let $X$ be $\A^1$ with two origins and consider $X \sqcup \Gm \to \A^1$ which is flat (since locally it the inclusion of an open) with constant fibral degree $2$ but not finite.
\end{rmk}

\subsection{Some Lemma}

\begin{lemma}
Let $f : X \to Y$ be a morphism of schemes and $\F$ a coherent $\struct{X}$-module flat over $Y$. Suppose that $\F|_{X_y}$ is locally free of rank $r$ at $x \in X_y$ then $\F$ is locally free of rank $r$ at $x \in X$.
\end{lemma}

\begin{proof}
We reduce to a statement on local rings. Let $A \to B$ be a local homomorphism and $M$ a finitely presented $B$-module flat over $A$ such that $M / \m_A M$ is free of rank $r$ then $M$ is free of rank $r$. Lifting a basis gives a map $\varphi : B^r \to M$ such that $\varphi \ot_A \kappa_A$. Consider,
\[ 0 \to \ker{\varphi} \to B^r \to M \to \coker{\varphi} \to 0 \]
then since $M$ is a finite $B$-module we see that $\coker{\varphi}$ is finite. Since $(\coker{\varphi}) \ot_A \kappa_A = 0$ then $(\coker{\varphi}) \ot_B \kappa_B = 0$ and hence $\coker{\varphi} = 0$ by Nakayama. Therefore $\varphi$ is surjective. Since $M$ is finitely presented $\ker{\varphi}$ is $B$-finite and $M$ is $A$-flat so $(\ker{\varphi}) \ot_A \kappa_A = 0$ so $(\ker{\varphi}) \ot_B \kappa_B = 0$ and hence $\ker{\varphi} = 0$ by Nakayama so $\varphi$ is an isomorphism.
\end{proof}

\begin{rmk}
Of course flatness is necessary e.g. consider $k[x] \to k[x]$ and $M = k[x]/(x)$.
\end{rmk}

\begin{rmk}
Consider $k[x,y]/(x^2, xy) \to k[x,y]/(x^2, xy) \to k[y]$. This is an example where $\F$ is a vector bundle when restricted to the fiber over any irreducible subscheme on the base but not a vector bundle.
\end{rmk}


\begin{lemma} \label{lemma:endo_local_ring}
Let $(R, \m, \kappa)$ be a local ring. Suppose that $M$ is a finite $R$-module with an endomorphism $\phi : M \to M$ such that $\phi \otimes \id : M \otimes_R \kappa \to M \otimes_R \kappa$ is an isomorphism then $\phi$ is an isomorphism. 
\end{lemma}

\begin{proof}
Consider the exact sequence,
\begin{center}
\begin{tikzcd}
M \arrow[r, "\phi"] & M \arrow[r] & \coker{\phi} \arrow[r] & 0
\end{tikzcd}
\end{center}
and apply the right-exact functor $(-) \otimes_R \kappa$ to get,
\begin{center}
\begin{tikzcd}
M \otimes_R \kappa \arrow[r, "\phi \otimes \id"] & M \otimes_R \kappa \arrow[r] & (\coker{\phi}) \otimes_R \kappa \arrow[r] & 0
\end{tikzcd}
\end{center}
But $\phi \otimes \id$ is an isomorphism and the sequence is exact so $(\coker{\phi}) \otimes_R \kappa = 0$ and thus $\coker{\phi} = 0$ by Nak. Therefore $\varphi$ is an isomorphism by a general result on endomorphisms of finite modules.
\end{proof}

\subsection{Type (a) fibral finiteness and isomorphsism}

\begin{prop}[EGA III, tome 1, Proposition 4.6.7]
Let $S$ be a locally noetherian scheme. Let $f : X \to Y$ be a morphism of proper $S$-schemes. Let $s \in S$ and consider $f_s : X_s \to Y_s$. 
\begin{enumerate}
\item if $f_s$ is a finite morphism (resp. a closed immersion) then there exists an open neighborhood $U \subset S$ of $s \in S$ such that $f|_U : X_U \to Y_U$ is finite (resp. a closed immersion).
\item If $X \to S$ is flat then if $f_s$ is an isomorphism then there exists an open neighborhood $U \subset X$ of $s \in S$ such that $f|_U : X_U \to Y_U$ is an isomorphism.
\end{enumerate}
\end{prop}

\begin{cor}
Let $S$ be a locally noetherian scheme. Let $f : X \to Y$ be a morphism of proper $S$-schemes. Let $s \in S$ and consider $f_s : X_s \to Y_s$. 
\begin{enumerate}
\item if $f_s$ is finite (resp. a closed immersion) for each $s \in S$ then $f$ is finite (resp. a closed immersion).
\end{enumerate}
\end{cor}



\subsection{Type (b) Fibral Properness and Isomorphism}

\begin{prop}[EGA IV.15.7.10]
If $f : X \to Y$ is universally submersive (e.g. flat), finite type, separated and has proper and geometrically connected fibers then $f$ is proper.
\end{prop}

\begin{rmk}
Note! Geometrically connected implies nonempty! This is very important or else Grothendieck would be claiming that open immersions are proper!
\end{rmk}

\begin{rmk}
Universally submersive is necessary e.g. consider $\Gm \sqcup * \to \A^1$.
\end{rmk}

\begin{rmk}
For a local version consider \chref{https://mathoverflow.net/questions/371196/flat-with-geometrically-connected-and-proper-fibers-is-proper}{this} question.
\end{rmk}

\begin{cor}
If $f : X \to Y$ is universally submersive (e.g. flat), finite type, separated and has fibral-degree $1$. Then $f$ is an isomorphism.
\end{cor}

\begin{proof}
Indeed since $f$ has geometrically connected fibers we see that $f$ is proper but it is quasi-finite and hence finite. Therefore $f$ is an isomorphism sinice it is a finite map of degree $1$. 
\end{proof}

\begin{rmk}
This also follows from the lemma of Deligne-Rapoport.
\end{rmk}

\section{Flatness of Hilbert and Picard Schemes}

\begin{prop}
Let $f : X \to Y$ be a lfp morphism of schemes such that,
\begin{enumerate}
\item the geometric fibers $X_{\bar{y}}$ are regular
\item the fiber dimension $\dim{X_y}$ is constant
\item the geometric fibers $X_{\bar{y}}$ are irreducible
\item $f$ is proper
\item $Y$ is reduced
\end{enumerate}
then $f$ is smooth.
\end{prop}

\begin{proof}
This is local on the source and target so spreading out we reduce to the case that $X$ and $Y$ are affine and finite type over $\ZZ$. Then by the valuative criterion of flatness we reduce to the case that $Y = \Spec{R}$ is a dvr. Let $Z \subset X$ be the scheme-theoretic closure of the generic fiber. Since the generic fiber is irreducible $Z$ is irreducible. Furthermore, the fiber dimension of $Z$ is can only jump up but the fiber dimension of $X$ is constant hence the special fibers of $X$ and $Z$ have the same dimension and $X_s$ is irreducible so $X = Z$ as closed subsets and hence $X$ is irreducible. Then we use the following lemma.
\end{proof}

\begin{lemma}
Let $f : X \to Y$ be a lpf morphism of schemes such that,
\begin{enumerate}
\item $Y$ is regular (hence locally Noetherian)
\item the fibers $X_y$ are regular and equidimensional of constant dimension
\item $X$ is equidimensional
\end{enumerate}
then $X$ is regular and $f$ is flat.
\end{lemma}

\begin{proof}
For $x \in X$ let $y = f(x)$. Consider the map $\stalk{Y}{y} \to \stalk{X}{x}$
Since $Y$ is regular, $\m_y$ is generated by $\dim{\stalk{Y}{y}}$ elements therefore,
\[ \dim_{\kappa(x)} \m_x / (\m_x^2 + \m_y) \ge \dim_{\kappa(x)} \m_x / \m_x^2 - \dim_{\kappa(x)} (\m_y / \m_y^2) \ot_{\kappa(y)} \kappa(x) \ge \dim{\stalk{X}{x}} - \dim{\stalk{Y}{y}} \]
But $X_y$ is regular so,
\[ \dim{\stalk{X_y}{x}} = \dim_{\kappa(x)} \m_x / (\m_x^2 + \m_y) \]
so we need to show that,
\[ \dim{\stalk{X_y}{x}} = \dim{\stalk{X}{x}} - \dim{\stalk{Y}{y}} \]
then this will imply that the inequalities are equalities hence that $\stalk{X}{x}$ is a regular local ring. Also it will show flatness by the miracle flatness theorem applied to $\stalk{Y}{y} \to \stalk{X}{x}$. Thus we apply the following.
\end{proof}

\begin{lemma}
Let $f : X \to Y$ be a locally finite type morphism of locally noetherian schemes. Suppose that,
\begin{enumerate}
\item $X$ is equidimensional
\item the fibers are equidimensional of constant dimension
\item $Y$ is universally catenary
\end{enumerate}
then for each $x \in X$ set $y = f(x)$ the equality,
\[ \dim{\stalk{X_y}{x}} = \dim{\stalk{X}{x}} - \dim{\stalk{Y}{y}} \]
holds. 
\end{lemma}

\begin{proof}
Pulling back along $\Spec{\stalk{Y}{y}} \to Y$ and shrinking to an affine open we reduce to a finitely presented morphism $\Spec{B} \to \Spec{A}$ where $A$ is a regular local ring and hence universally catenary. For each irreducble (reduced) component $\Spec{B_i} \subset \Spec{B}$ we have by assumption $\dim{B_i} = \dim{B}$. Since $X \to Y$ is dominant (either $X$ is empty and the result is trivial or $f$ is dominant since every fiber has constant hence nonnegative dimension) the map $A \to B_i$ is injective. Therefore, the dimension formula holds:
\[ \dim{(B_i)_\m} - \dim{A} = \trdeg{K(A)}{K(B_i)} - \trdeg{\kappa_A}{\kappa(\m)} \]
but by constancy and equidimensionality of the fibers we have,
\[ \dim{(B \ot_A \kappa_A)} = \dim{(B_i \ot_A \kappa_A)} = \trdeg{K(A)}{K(B_i)} \]
since the fiber of each $B_i$ is an union of irreducible components of the fiber each of which has the same dimension by assumption. Likewise we apply the dimension formula to the fiber $\Spec{B \ot_A \kappa_A} \to \Spec{\kappa_A}$ to get that for each irreducible component $\Spec{C_j}$ we have,
\[ \dim{(C_j)_\m} = \trdeg{\kappa_A}{K(C_j)} - \trdeg{\kappa_A}{\kappa(\m)} \]
and since $B \ot_A \kappa_A$ is an equidimensional finite-type $\kappa_A$-scheme,
\[ \dim{(B \ot_A \kappa_A)} = \dim{C_j} = \trdeg{\kappa_A}{K(C_j)} \] 
Thus finally,
\[ \dim{(B_i)_\m} - \dim{A} = \dim{(C_i)_\m} \]
However, $B_i$ are the irreducible components of $B$ so by definition $\dim{B_\m} = \max_i \dim{(B_i)_\m}$ and likewise $\dim{(B \ot_A \kappa_A)_\m} = \max_i \dim{(C_i)_\m}$ so we conclude that,
\[ \dim{(B \ot_A \kappa_A)} = \dim{B} - \dim{A} \]
which is what we needed to show.
\end{proof}

\begin{prop}
Let $\X \to S$ be a smooth projective family of surfaces meaning the geometric fibers are smooth varities of dimension $2$. Then the relative Hilbert scheme of points $\Hilb_{\X / S}^n$ is smooth over $S$.
\end{prop}

\begin{proof}
Since $\X \to S$ is projective the Hilbert scheme exists and is projective. Furthermore, since $\X_{\bar{s}}$ is a smooth surface, $(\Hilb_{\X/S})_{\bar{s}} = \Hilb_{\X_{\bar{s}}}$ is smooth and irreducible of dimension $2n$. Therefore, we conclude by the previous results that $\Hilb_{\X/S} \to S$ is smooth.
\end{proof}

\newcommand{\cN}{\mathcal{N}}

\begin{example}
Let $E$ be an elliptic curve and consider the nontrivial extension,
\begin{center}
\begin{tikzcd}
0 \arrow[r] & \struct{E} \arrow[r] & \E \arrow[r] & \struct{E} \arrow[r] & 0
\end{tikzcd}
\end{center}
and let $X = \P_E(\E)$. Choose the ample class $\L = \pi^*( \struct{E}([0]) ) \ot \struct{X}(1)$ and hilbert polynomial $p(t) = t + 1$. Then I claim that,
\[ \Hilb_{X}^{\L, p} \cong \Spec{k[\epsilon]} \]
The exact sequence gives a section $\sigma : E \to X$. Then $T_{[\sigma]} \Hilb_X = H^0(E, \cN_{E|X})$. We'll next compute that $\cN_{E|X} = \struct{E}$ and hence $T_{[\sigma]} \Hilb_X = k$ so we get a $1$-dimensional tangent space. Thus it suffices to show that $\Hilb_X(k) = \{ \sigma \}$. Consider, DO THIS!!
\end{example}

\begin{example}
Consider $X = \P_Y(\E)$ for some vector bundle and consider a section $\sigma : Y \to X$ given by,
\begin{center}
\begin{tikzcd}
0 \arrow[r] & \E_0 \arrow[r] & \E \arrow[r] & \L \arrow[r] & 0
\end{tikzcd}
\end{center}
Then we need to compute,
\begin{center}
\begin{tikzcd}
0 \arrow[r] & \C_{\sigma} \arrow[r] & \sigma^* \Omega_X \arrow[r] & \Omega_Y \arrow[r] & 0
\end{tikzcd}
\end{center}
and there is an Euler sequence,
\begin{center}
\begin{tikzcd}
0 \arrow[r] & \Omega_{X/Y}(1) \arrow[r] & \pi^* \E \arrow[r] & \struct{X}(1) \arrow[r] & 0
\end{tikzcd}
\end{center}
where the second map is the universal quotient. Therefore,
\begin{center}
\begin{tikzcd}
0 \arrow[r] & (\sigma^* \Omega_{X/Y}) \ot \L \arrow[r] & \E \arrow[r] & \L \arrow[r] & 0
\end{tikzcd}
\end{center}
thus we see that $\sigma^* \Omega_{X/Y} = \E_0 \ot \L^{-1}$. Furthermore, consider the cotangent sequence,
\begin{center}
\begin{tikzcd}
0 \arrow[r] & \pi^* \Omega_Y \arrow[r] & \Omega_X \arrow[r] & \Omega_{X/Y} \arrow[r] & 0
\end{tikzcd}
\end{center}
when we apply $\sigma^*$ there is a section $\d{\sigma} : \sigma^* \Omega_X \to \Omega_Y$ and hence,
\[ \C_{\sigma} = \ker{\d{\sigma}} = \sigma^* \Omega_{X/Y} = \E_0 \ot \L^{-1} \]
Therefore, 
\[ T_{[\sigma]} \Hilb_X = H^0(\cN_{\sigma}) = \Hom{}{\E_0}{\L} = T_{[\sigma]} \Hom{\pi}{Y}{X} \]
The last equality comes from considering surjections $\E[\epsilon] \onto \L'$ over $Y \times \Spec{k[\epsilon]}$ up to isomorphism or equivalently flat subbundles $\E_0' \embed \E[\epsilon]$ fits into a sequence,
\[ \Hom{}{\E_0}{\E} \to \{ \E_0' \embed \E[\epsilon] \} \to \ker{(\mathrm{Def}(\E_0) \to \mathrm{Ob}(\E_0 \to \E))} \]
which is,
\[ \Hom{}{\E_0}{\E} \to \{ \E_0' \embed \E[\epsilon] \} = \ker{(\Ext{1}{\struct{X}}{\E_0}{\E_0} \to \Ext{1}{\struct{X}}{\E_0}{\E})} \]
this is the section of the long exact sequence giving,
\[ \{ \E_0' \embed \E[\epsilon] \} = \Hom{\struct{X}}{\E_0}{\L} \]
\end{example}

\begin{example}
Let $E$ be an elliptic curve and consider the bundle $\E$ on $Y = E \times \A^1$ given by the extension,
\begin{center}
\begin{tikzcd}
0 \arrow[r] & \struct{Y} \arrow[r] & \E \arrow[r] & \struct{Y} \arrow[r] & 0
\end{tikzcd}
\end{center}
corresponding to the extension class,
\[ \xi = t \in \Ext{1}{\struct{Y}}{\struct{Y}}{\struct{Y}} = k[t] \]
Now set $X = \P_Y(\E)$ and consider the family of smooth surfaces $X \to \P^1$ and the ample line bundle $\L = \pi^* (\struct{Y}(\sigma_0)) \ot \struct{X}(1)$ where $\sigma_0$ is the zero section of $Y \to \A^1$. For $t \neq 0$ we have seen that \[ \Hilb^{\L, p}_{X_t} = \Spec{k[\epsilon]} \]
However, $X_0 = E \times \P^1$ with ample $\L_0 = \struct{E}([0]) \boxtimes \struct{\P^1}(1)$ and $\Hilb^{\L_0, p}_{X_0}$ contains a $\P^1$ parameterizing the closed subschemes $E \times \{ s \} \subset E \times \P^1$. Therefore $\Hilb^{\L, p}_{X/\P^1}$ is not flat over $\P^1$.
\end{example}


\section{Notes}

$H^1(\struct{X})$ constant in the smooth case by Hodge theory. 
\bigskip\\
Can $H^1(\struct{X})$ jump in the normal case?
\bigskip\\
Looks like no, using vanishing cycle theory. 
\bigskip\\
Questions:

\begin{enumerate}
\item Hilbert schemes of points of smooth families are flat
\item Find example of flat family where Hilbert schemes of points are not flat
\item Example where Hilbert scheme with fixed polynomial is not flat.
\end{enumerate}

\section{Neutrino Cosmic Background}

\subsection{Relativistic Gas}

Consider a relativistic gas of bosons or fermions with $s$ species (e.g. for electrons/positrons we have $s = 4$ (two polarizations times two particle types) for neutrios we have $s = 12$ (two polarizations times six particle types)  when the temperature is large compared to the heaviest neutrino mass which is basically always true). We compute,
\[ Z = \left( \prod_{ \{ n_{\vec{k}} \}_{\vec{k}}} \prod_{\vec{k}} e^{- \beta n_{\vec{k}} E_{\vec{k}}} \right)^s = \left( \prod_{\vec{k}} \prod_{n_{\vec{k}}} e^{-\beta n_{\vec{k}} E_{\vec{k}}} \right)^s = \left[ \prod_{\vec{k}} \left( \frac{1}{1 \mp e^{-\beta E_{\vec{k}}}} \right)^{\pm 1} \right]^s \]
where the energy associated to a certain mode with wavenumber $\vec{k}$ is,
\[ E_{\vec{k}}^2 = \hbar^2 |k|^2 c^2 + m^2 c^4 \]
For a gas in a box of large volume with respect to the thermal wavelength we compute,
\[ -\log{Z} = \pm s \int_0^{\infty} \d{k} \left( \frac{V}{\pi^3} \right) \left( \frac{4 \pi k^2}{8} \right) \log{(1 \mp e^{-\beta E_{\vec{k}}})} = \pm \frac{s V}{2 \pi^2} (\beta \hbar c)^{-3} \int_0^{\infty} x^2 \log{\left(1 \mp e^{-\sqrt{x^2 + (\beta m c^2)^2}}\right)} \d{x} \] 
Then we derive the total energy from the formula,
\[ U = - \pderiv{\log{Z}}{\beta} \] 
to get,
\begin{align*}
U &= \pm \frac{s V}{2 \pi^2 \hbar^3 c^3} \left[ - 3 \beta^{-4} \int_0^{\infty} x^2 \log{\left(1 \mp e^{-\sqrt{x^2 + (\beta m c^2)^2}}\right)} \d{x}  \pm \beta^{-3} \int_0^{\infty} \frac{x^2 (x^2 + (\beta m c^2)^2)^{-\frac{1}{2}} \beta (m c^2)^2}{e^{\sqrt{x^2 + (\beta m c^2)^2}} \mp 1} \d{x} \right] 
\\
& = \pm \frac{s V}{2 \pi^2 \beta^4 \hbar^3 c^3} \left[ - \int_0^{\infty} (3 x^2) \log{\left(1 \mp e^{-\sqrt{x^2 + (\beta m c^2)^2}}\right)} \d{x}  \pm \int_0^{\infty} \frac{x^2 (x^2 + (\beta m c^2)^2)^{-\frac{1}{2}} (\beta m c^2)^2}{e^{\sqrt{x^2 + (\beta m c^2)^2}} \mp 1} \d{x} \right] 
\end{align*}
Therefore, integrating the first by parts gives,
\begin{align*}
U & = \frac{s V}{2 \pi^2 \beta^4 \hbar^3 c^3} \left[ \int_0^{\infty} \frac{x^4 (x^2 + (\beta m c^2)^2)^{-\frac{1}{2}}}{e^{\sqrt{x^2 + (\beta m c^2)^2}} \mp 1} \d{x} + \int_0^{\infty} \frac{x^2 (x^2 + (\beta m c^2)^2)^{-\frac{1}{2}} (\beta m c^2)^2}{e^{\sqrt{x^2 + (\beta m c^2)^2}} \mp 1} \d{x} \right] 
\\
& = \frac{s V}{2 \pi^2 \beta^4 \hbar^3 c^3} \int_0^{\infty} \frac{x^2 \sqrt{x^2 + (\beta m c^2)^2}}{e^{\sqrt{x^2 + (\beta m c^2)^2}} \mp 1} \d{x}
\end{align*}
which is much more easily derived from the expression for $-\log{Z}$ before the introduction of the dimensionless integration parameter $x$. Regardless, introduce functions,
\[ u_{\pm}(\alpha) := \int_0^\infty \frac{ x^2 \sqrt{x^2 + \alpha^2}}{e^{\sqrt{x^2 + \alpha^2}} \mp 1} \d{x} \]
and
\[ f_{\pm}(\alpha) := \pm \int_0^{\infty} x^2 \log{\left( 1 \mp e^{-\sqrt{x^2 + \alpha^2}} \right)} \d{x} \]
then we see,
\[ U = \frac{s V (k_B T)^4}{2 \pi^2 \hbar^3 c^3} u_{\pm}(\alpha) \]
and,
\[ F = \frac{s V (k_B T)^4}{2 \pi^2 \hbar^3 c^3} f_{\pm}(\alpha) \]
where,
\[ \alpha = \frac{m c^2}{k_B T} \]
is the ratio of the Compton temperature to $T$.
In particular,
\[ \frac{F}{U} = \frac{f_{\pm}(\alpha)}{u_{\pm}(\alpha)} \]
We compute the high and low temperature limits. In the high temperature limit $\beta \to 0$ or equivalently $m \to 0$ we get,
\begin{align*}
u_{+,0}(\alpha) & = u_{+}(0) = \frac{\pi^4}{15}
\\
u_{-,0}(\alpha) & = u_{-}(0) = \frac{7 \pi^4}{120}
\\
f_{+,0}(\alpha) & = f_{+}(0) = -\frac{\pi^4}{45}
\\
f_{-,0}(\alpha) & = f_{-}(0) = -\frac{7 \pi^4}{360}
\end{align*}
Therefore, for an ultra-relativistic gas we find that the ratio of free energy to total energy is a constant independent of the nature of the gas,
\[ \frac{F}{U} = - \frac{1}{3} \]
This gives the familiar result, that the pressue
\[ P = - \pderiv{F}{V} \bigg|_T = - \frac{F}{V} = \frac{1}{3} \frac{U}{V} \]
of a relativistic gas is $\frac{1}{3}$ the energy density. Now in the low-temperature limit, which corresponds to large $\alpha$, we can approximate,
\begin{align*}
u_{\pm}(\alpha) & \approx \int_0^{\infty} \frac{x^2 \sqrt{x^2 + \alpha^2}}{e^{\sqrt{x^2 + \alpha^2}}} \d{x} = \int_{\alpha}^{\infty} \frac{x u^2 \d{u}}{e^u} = e^{-\alpha} \int_0^{\infty} \frac{\sqrt{(y + \alpha)^2 - \alpha^2} \, (y + \alpha)^2 \d{y}}{e^y}
\\
& \approx e^{-\alpha} \int_0^{\infty} \frac{ \frac{1}{\sqrt{2}} [ y^2 + 2 \sqrt{\alpha y} \, ] (y + \alpha)^2 \d{y}}{e^y} \approx \sqrt{\tfrac{\pi}{2}} \, \alpha^{\frac{5}{2}} e^{-\alpha}
\end{align*}
Therefore as $T \to 0$ for $m > 0$ we have $\alpha \to \infty$ hence the energy content of this gas component freezes out. Furthermore,
\begin{align*}
f_{\pm}(\alpha) & \approx \pm \int_0^{\infty} x^2 \left( \mp e^{-\sqrt{x^2 + \alpha^2}} \right) \d{x} = - \int_{\alpha}^{\infty} x u \d{u} e^{-u} = e^{-\alpha} \int_0^{\infty} \sqrt{(y + \alpha)^2 - \alpha^2} (y + \alpha) e^{-y} \d{y} 
\\
& \approx e^{-\alpha} \int_0^{\infty} \tfrac{1}{\sqrt{2}} [y + 2 \sqrt{y \alpha}] (y + \alpha) e^{-y} \d{y} \approx \sqrt{\tfrac{\pi}{2}} \alpha^{\frac{3}{2}} e^{-\alpha} 
\end{align*}
Therefore, 
\[ \frac{F}{U} \approx \alpha^{-1} = \frac{k_B T}{m c^2} \]

Now the entropy is,
\[ S = \frac{U - F}{T} = k_B \frac{s V (k_B T)^3}{2 \pi^2 \hbar^3 c^3} [u_{\pm}(\alpha) - f_{\pm}(\alpha)]  \]

\subsection{Neutrino Decoupling}

Consider a gas of photons, electrons, and positrons. Then we get,
\[ S_{\text{EM}} = k_B \frac{V (k_B T)^3}{\pi^2 \hbar^3 c^3} \left[ \frac{\pi^4}{15} + \frac{\pi^4}{45} + 2 u_{-}(\alpha_e) - 2 f_{-}(\alpha_e) \right] \]
Likewise, the entropy of a neutrino gas (assuming $\alpha_\nu \gg 1$ meaning $k_B T \gg m_{\nu} c^2$) is that of a fermionic gas with $s = 12$ so,
\[ S_{\nu} = k_B \frac{12 \pi^2 V (k_B T_\nu)^3}{2 \hbar^3 c^3} \cdot \frac{7}{120} \cdot \frac{4}{3} = k_B \frac{7 \pi^2 V (k_B T_\nu)^3}{15 \hbar^3 c^3}  \]
Therefore,
\[ \frac{S_{\text{EM}}}{S_{\nu}} = \frac{30}{7 \pi^4} \left( \frac{T}{T_\nu} \right)^3 \left[ \frac{2 \pi^4}{15} + u_{-}(\alpha_e) - f_{-}(\alpha_e) \right] \]
At electroweak unification temperature the neutrinos are thermalized with the other particles. However, as the universe cools sufficiently for the weak sector to decouple from the electromagnetic sector the entropy of each sector are independently conserved since the expansion of the universe is adiabatic. Therefore the ratio is a constant. However, at high temperature we have,
\[ \frac{S_{\text{EM}}}{S_{\nu}} = \frac{30}{7 \pi^4}\left[ \frac{2 \pi^4}{45} + \frac{7 \pi^4}{120} + \frac{7 \pi^4}{360} \right] = \frac{11}{21} \]
Therefore, at very low temperature the electrons and positrons freeze out so we get,
\[ \frac{11}{21} = \frac{4}{21} \left( \frac{T}{T_\nu} \right)^3 \]
and therefore,
\[ T_\nu = \left( \frac{4}{11} \right)^{\frac{1}{3}} T \]
For the CMB temperature of $T = 2.73 K$ we get $T_\nu = 1.95 K$. One interpretation of this lower temperature is that the neutrinos decouple before the electron and positrons freeze out of the soup but when these freeze out they dump their energy into the photon gas giving it extra heating. 

\section{MIT OCW 8.06 Darwin Term}

\subsection{HM 2.4}

Recall the Feynman-Hellman lemma that if $H_\lambda$ is a continuous familiy of Hamiltonians with a continuous family of eigenstates $\ket{\psi_\lambda}$ which energy $E_\lambda$ then,
\[ \deriv{}{\lambda} E_\lambda = \bra{\psi_\lambda} \deriv{H_\lambda}{\lambda} \ket{\psi_\lambda} \]
We apply this to the Hydrogen effective Hamiltonian,
\[ H = - \frac{\hbar^2}{2m} \frac{\d^2}{\d{r}^2} + \frac{\hbar^2}{2m} \frac{\ell(\ell+1)}{r^2} - \frac{e^2}{r} \]
The hydrogen atom energies are,
\[ E_n = - \frac{e^2}{2a_0} \frac{1}{n^2} \quad \quad a_0 = \frac{\hbar^2}{m e^2} \]
In solving the radial equation one sets $n = N + \ell + 1$ where $N$ is the degree of the radial polynomial. 

\begin{enumerate}
\item Let $\lambda = e^2$ be the parameter then we get $\deriv{H_\lambda}{\lambda} = - \frac{1}{r}$ and therefore,
\[ \left< \frac{1}{\lambda} \right> = - \deriv{E_\lambda}{\lambda} = \frac{m e^2}{\hbar^2 n^2} = \frac{1}{a_0 n^2} \]

\item For the parameter $\lambda = \ell$ where in the radial equation we can consider $\ell$ as a continuous parameter we get,
\[ \deriv{H_\lambda}{\lambda} = \frac{\hbar^2}{2m} \frac{2 \ell + 1}{r^2} \]
Therefore,
\[ \left< \frac{1}{r^2} \right> = \frac{2m}{\hbar^2 (2 \ell + 1)} \deriv{E_\lambda}{\lambda} \]
where we fix $N$ because it corresponds to the number of nodes in the radial equation and hence is an adiabatic invariant and we vart $n$ according to $n = N + \ell + 1$. Therefore we get,
\[ \deriv{E_\lambda}{\lambda} = \frac{m e^4}{ \hbar^2} \frac{1}{n^3} \]
and hence,
\[ \left< \frac{1}{r^2} \right> = \frac{2m}{\hbar^2 (2 \ell + 1)} \frac{m e^4}{ \hbar^2} \frac{1}{n^3} = \left( \frac{m e^2}{\hbar^2} \right)^2 \frac{2}{2 \ell + 1} \frac{1}{n^3} = \frac{1}{a_0^2} \frac{2}{2 \ell + 1} \frac{1}{n^3} \]
\end{enumerate}

\subsection{HW 3.2}

Consider the radial equation,
\[ u'' = \frac{2m}{\hbar^2} \left[ V_{\text{eff}}(r) - E \right] u \]
multiply this by $u'$ and integrate to get,
\[ \int_0^{\infty} u'' u' \d{r} = \frac{2m}{\hbar^2} \int_0^{\infty} \left[ V_{\text{eff}}(r) - E \right] u u' \d{r} \]
Now $u'' u'$ is the derivative of $\tfrac{1}{2} (u')^2$ and $u u'$ is the derivative of $\tfrac{1}{2} u^2$ so integrating the RHS by parts we get,
\[ \tfrac{1}{2} (u')^2 \big|_{0}^{\infty} = \frac{2 m}{\hbar^2} \left[ \left[ V_{\text{eff}}(r) - E \right] \, \tfrac{1}{2} u^2 \big|^{\infty}_0 - \int_0^{\infty} V_{\text{eff}}' \, \tfrac{1}{2} u^2 \, \d{r} \right] \]
However, the boundary conditions for bound states in the radial equation are $u(0) = u(\infty) = 0$ and likewise since $u = r \psi$ we get $u'(0) = \psi(0)$ and $u'(\infty) = 0$ and therefore,
\[ \psi(0)^2 = \frac{2m}{\hbar^2} \int_0^{\infty} V_{\text{eff}}' \, u^2 \d{r} = \frac{2 m}{\hbar^2} \int_0^{\infty} \deriv{V_{\text{eff}}}{r} \psi(r)^2 \, r^2 \d{r} = \frac{2m}{4 \pi \hbar^2} \int \deriv{V_{\text{eff}}}{r} \psi(r)^2 \,  \dn{3}{r} \]
Now if we consider a state with $\ell = 0$ we get $V_{\text{eff}} = V$ and $\psi(\vec{r}) = \psi(r)$ and we can choose $\psi$ real so we conclude,
\[ |\psi(0)|^2 = \frac{2 m}{4 \pi \hbar^2} \left<  \deriv{V_{\text{eff}}}{r} \right> \]

\subsection{The Darwin Term}

Therefore using the previous two exercises,
\[ |\psi_{n,0}(0)|^2 = \frac{2 m e^2}{4 \pi \hbar^2} \left< \frac{1}{r^2} \right> = \frac{1}{2 \pi} \cdot \frac{1}{a_0^3} \cdot \frac{2}{n^3} = \frac{1}{\pi a_0^3 n^3} \]
Therefore,  for the Darwin term,
\[ \delta H_{\text{Darwin}} = \frac{\hbar^2}{8 m^2 c^2} \nabla^2 V = \frac{\pi \hbar^2 e^2}{2 m^2 c^2} \delta^{(3)}(\vec{r}) \]
we only get an energy shift for $\ell = 0$ which in first-order perturbation theory is,
\[ \Delta E_{n,0} = \left< \delta H_{\text{Darwin}} \right> = \nabla^2 V = \frac{\pi \hbar^2 e^2}{2 m^2 c^2} | \psi_{n,0}(0) |^2 = \frac{\hbar^2 e^2}{m^2 c^2 a_0^3} \cdot \frac{1}{n^3} = \alpha^4 mc^2 \cdot \frac{1}{2 n^3} \]


\section{Semisimple Algebras}

\subsection{Jacobson Radical}

Here a ring is always unital and associative but not necessarily commutative.

\begin{defn}
The \textit{Jacobson radical} $J(R)$ is defined as,
\[ J_{l}(R) = \bigcap_{\substack{M \text{ simple} \\ \text{left } R\text{-mod}}} \Ann{R}{M} \]
\[ J_{r}(R) = \bigcap_{\substack{M \text{ simple} \\ \text{right } R\text{-mod}}} \Ann{R}{M} \]
\end{defn}

Note that these are two-sided ideals because $\Ann{R}{M}$ is always a two-sided ideal. Indeed if $M$ is a left $R$-module then if $x M = 0$ then clearly $rxM = 0$ but al $x r M = x (r M) = x M = 0$. Similarly for $M$ a right $R$-module.

We use the terminology ``the'' Jacobson radical because of the following theorem we will now prove.

\begin{theorem}
For any unital ring $R$,
\[ J_l(R) = J_r(R) \]
as two-sided ideals.
\end{theorem}

\begin{rmk}
This is false for nonunital rings. GIVE EXAMPLE
\end{rmk}

\begin{rmk}
Due to the symmetry one might ask if we can define the Jacobson radical more symmetrically as the intersection of the annihlators of simple bimodules. (CAN YOU DO THIS??)
\end{rmk}

We now will prove this result as follows.

\begin{prop}
The following hold:
\[ J_l(R) = \bigcap_{\substack{\m \text{ maximal} \\ \text{left ideal}}} \m = \{ x \in R \mid \forall r \in R : 1 + r x \text{ is left invertible} \} \]
and likewise
\[ J_r(R) = \bigcap_{\substack{\m \text{ maximal} \\ \text{right ideal}}} \m = \{ x \in R \mid \forall r \in R : 1 + xr \text{ is right invertible} \} \]
\end{prop}

\begin{proof}
The argument for the two statements is identical so we will only do it for the left case. Since if $\m$ is a maximal left ideal then $R/\m$ is a simple left $R$-module it is clear that $J_l(R) \subset \m$ since if $x (R / \m) = 0$ then $x = x \cdot 1 \in \m$. 
\bigskip\\
Conversely suppose that $x \in \m$ for each maximal left ideal $\m$. Let $M$ be a simple left $R$-module and $m \in M$ nonzero. Then $R m \subset M$ is a nonzero submodule so $R m = M$ and hence ${}_R R \to M$ sending $r \mapsto r m$ is surjective and hence $M \cong R / \Ann{R}{m}$ and $\Ann{R}{m}$ is a maximal left\footnote{Unlike $\Ann{R}{M}$ the ideal $\Ann{R}{m}$ is \textit{not} two-sided. The point is that for any $r \in R$ we have $r M = M$ so if $x M = 0$ then $xr M = x(rM) = x M = 0$ but the same does not work for $x \in \Ann{R}{m}$ since $r m$ might not be annihilated by $x$. This shows that even if $m$ generates $M$ we may have $\Ann{R}{m} \supsetneq \Ann{R}{M}$ in the noncommutative case because we cannot take $x m = 0$ and use it to conclude that $xrm = 0$. Indeed, if $M = R/\m$ then $\Ann{R}{\bar{1}} = \m$ but $\Ann{R}{M} = \{ x \in R \mid xR \subset \m \}$ the largest two-sided ideal contained in $\m$, which is, in general, smaller than $\m$ since $\m$ is only a left ideal.} ideal (maximal otherwise there would be proper submodules of $R / \Ann{R}{m}$). Now it is clear that,
\[ \Ann{R}{M} = \bigcap_{m \in M} \Ann{R}{m} \]
which is an intersection of maximal left ideals and hence $x \in \Ann{R}{M}$ so $x \in J_l(R)$.
\bigskip\\
Now we show the second equality. Suppose that $x \in \m$ for each maximal left ideal. Then if $1 + rx$ does not have a left inverse then $R (1 + rx)$ is a proper left ideal and hence contained in a maximal left ideal $\m$ but then $x \in \m$ so $r x \in \m$ so $1 \in \m$ giving a contradiction. Conversely, if $(1 + rx)$ does have a left inverse for all $r \in R$ suppose that $x \notin \m$ then there exists $r$ such that $1 + rx \in \m$ so $1 \in R (1 + rx) \subset \m$ giving a contradiction. To see the existence of such $r$ consider the left ideal $\m + R x$ which is strictly larget than $\m$ since $x \notin \m$. Hence by maximality $\m + R x = R$ so it contains $1$ and hence there exists $r \in R$ so that $1 + r x \in \m$.
\end{proof}

\begin{lemma}
The following conditions on $x \in R$ are equivalent,
\begin{enumerate}
\item $\forall r \in R : 1 + rx$ has a left inverse
\item $\forall r \in R : 1 + rx$ has a two-sided inverse
\end{enumerate}
and similarly,
\begin{enumerate}
\item $\forall s \in R : 1 + xs$ has a two-sided inverse
\item $\forall s \in R : 1 + xs$ has a right inverse
\end{enumerate}
\end{lemma}

\begin{proof}
We just need to show (a) $\implies$ (b) since other implications are similar or trivial. Suppose (a) then there exists $s$ such that $s(1 + rx) = 1$ and hence,
\[ s = 1 - srx \]
so applying (a) with $r$ replaced by $-sr$ we see that $s$ has a left inverse $s'$ which means that $s$ is invertible (since it also has a right inverse $1 + rx$) and hence $1 + rx$ is also invertible.
\end{proof}

\begin{theorem}
For any unital ring $R$,
\[ J_l(R) = \{ x \in R \mid 1 + R x R \subset R^\times \} = J_r(R) \]
\end{theorem}

\begin{proof}
Indeed, we showed that if $x \in J_l(R)$ then $\forall r \in R : 1 + rx$ has a two-sided inverse by the above lemma. But $J_l(R)$ is a two-sided ideal so this means that $x s \in J_l(R)$ hence $\forall r,s \in R : 1 + rxs$ is a unit. The reverse inclusion is clear. Likewise, if $x \in J_r(R)$ then $\forall s \in R : 1 + x s$ has a two-sided inverse but $J_r(R)$ is a two-sided ideal so this means that $r x \i nJ_r(R)$ hence $\forall r, s \in R : 1 + rxs$ is a unit. Thus we conclude.
\end{proof}

From now on we write $J(R) = J_l(R) = J_r(R)$.

\subsection{Nilpotence}

\begin{prop}
Let $I$ be a left (resp. right) ideal consisting of nilpotent elements then $I \subset J(R)$.
\end{prop}

\begin{proof}
Since $I$ consists of nilpotent elements for each $x \in I$ we have $rx$ is nilpotent for each $r$. Hence $1 + rx$ is a unit so $x \in J(R)$.
\end{proof}

\begin{defn}
We say that a module $M$ is
\begin{enumerate}
\item \textit{Noetherian} if every ascending chain of submodules stabilizes
\item \textit{Artinian} if every descending chain of submodules stabilizes
\end{enumerate}
We say that $R$ is
\begin{enumerate}
\item \textit{left (resp. right) Noetherian} if ${}_R R$ (resp. $R_{R}$) is Noetherian as a left (resp. right) $R$-module
\item \textit{left (resp. right) Artinian} if ${}_R R$ (resp. $R_{R}$) is Artinian as a left (resp. right) $R$-module
\end{enumerate} $R$ is left (resp. right)
\end{defn}

\begin{rmk}
Note that there exist left artinian rings that are not right artinian (see Lam, a First Course in Noncommutative Rings, p.22). 
\end{rmk}

\begin{prop}
Let $R$ be left (resp. Artinian) noetherian. Then $J(R)$ is nilpotent.
\end{prop}

\begin{proof}
Let $J = J(R)$.
Consider the descending chain of left ideals,
\[ J \supset J^2 \supset J^2 \supset \cdots \]
This must stabilize so we have $J^m = J^{n}$ for $m \ge n$ and some fixed $n$. Let $m = 2n$ and $I = J^n$ so $I^2 = I$. If $I \neq 0$ then there exists a left ideal $K$ such that $I K \neq 0$ (e.g. $K = R$) since $R$ is left Artinian there is a minimal such $K$ by Zorns' lemma. If $y \in K$ then $Iy \subset $ FINISH
\end{proof}

\begin{prop}
Let $R$ be left (resp. right) Artinian then $R$ is left (resp. right) Noetherian.
\end{prop}

\begin{proof}
DO THIS PROOF!!
\end{proof}

\subsection{Semisimple Rings}

(SHOW AUTOMATIC FINITNESS CONDITIONS NOETHERIAN AUTOMATICALLY!!)

(GIVE EXAMPLE WHY INFINITE PRODUCT OF FIELDS NOT SEMISIMPLE)

\begin{defn}
A module $M$ is \textit{semisimple} if one of the following equivalent properties holds,
\begin{enumerate}
\item $M$ is a direct sum of simple modules
\item $M$ is the sum of its irreducible submodules
\item every submodule of $M$ is a direct summand
\end{enumerate}
\end{defn}

\begin{defn}
A ring $R$ is left (resp. right) semisimple if ${}_R R$ (resp. $R_{R}$) is a semisimple left (resp. right) $R$-module.
\end{defn}

\begin{prop}
If $R$ is left (resp. right) semisimple if and only if the category of left (resp. right) $R$-modules is semisimple in the sense that all exact sequences split.
\end{prop}

\begin{proof}
See [Rotman, An Introduction to Homological Algebra, Prop. 4.5] for details. If $R$ is left semisimple then ${}_R R$ hence hence all free left $R$-modules are semisimple. However, every left $R$-module is a quotient of a free module and hence semisimple because submodules of semisimple module are direct summands so quotients are also direct summands. Then every module is both injective and projective since all injections and surjections split since the modules are all semisimple.
\end{proof}

\begin{lemma}
Let $M$ be an $R$-module such that the intersection of all maximal submodules is zero. If $M$ is Artinian then $M$ is semisimple.
\end{lemma}

\begin{proof}
Since $M$ is Artinian the poset of finite intersections of maximal submodules satisfies Zorn's lemma and hence has a minimal element which must equal the intersection of al maximal submodules which is zero. Hence there is a collection $\{ N_i \}$ of maximal submodules such that $\bigcap_i N_i = (0)$. Therefore, the map,
\[ M \to \bigoplus_{i} M/N_i \]
is injective but $M/N_i$ is simple and hence $M$ is a submodule of a semisimple module and hence is semisimple.
\end{proof}

\begin{prop}
The following are equivalent,
\begin{enumerate}
\item $R$ is left semisimple
\item $R$ is right semisimple
\item $J(R) = 0$ and $R$ is left Artinian 
\item $J(R) = 0$ and $R$ is right Artinian.
\end{enumerate}
\end{prop}

\begin{proof}
Recall that $J(R)$ is the intersection of all maximal left ideals and also all maximal right ideals. Hence if $J(R) = 0$ then both ${}_R R$ and $R_{R}$ satisfy that the intersection of their maximal submodules is $(0)$.
If $J(R) = 0$ and $R$ is left (resp. right) Artinian then by the above lemma ${}_R R$ (resp $R_R$) is semisimple so we conclude that $R$ is left (resp. right) semisimple. Hence it suffices to show that if $R$ is left (resp. right) semisimple then $J(R) = (0)$ and $R$ is both left and right Artinian.
\bigskip\\
If $R$ is left semisimple then ${}_R R$ is a direct sum of simple modules. However, ${}_R R$ is trivially finitely generated but an infinite direct product cannot be finitely generated so it is a finite direct sum of simple modules. Hence $R$ is left Artinian.
\end{proof}

(HOW DO I SHOW IT IS RIGHT ARTINIAN!???)

\begin{defn}
A ring $R$ is \textit{simple} if it has no nontrivial two-sided ideals.
\end{defn}

\begin{rmk}
Note that $R$ being simple is much weaker than ${}_R R$ being a simple left $R$-module. For example, $R = M_n(k)$ is simple but is certainally not simple as a left module since it has nontrivial left ideals (e.g. matrices with some columns zero). Indeed, we have the following result.
\end{rmk}

\begin{prop}
The following are equivalent,
\begin{enumerate}
\item $R$ is a division ring
\item ${}_R R$ is a simple left $R$-module
\item $R_{R}$ is a simple right $R$-module
\end{enumerate}
\end{prop}

\begin{proof}
Indeed, these are equivalent to $R$ having no nontrivial left (resp. right) ideals. Thus if $x \in R$ is nonzero then $Rx = R$ so there is $yx = 1$ so every nonzero element has a left inverse. However, $y \in R$ then also has a left inverse hence every nonzero element is invertible. The same argument holds if $R_{R}$ is a simple right $R$-module.
\end{proof}


\begin{prop}
Let $R$ be semisimple. Then $R$ is left and right Artinian. 
\end{prop}


\subsection{Artin-Wedderburn Theorem}

\begin{lemma}[Shur]
Let $M_1, M_2$ be simple $R$-modules. Then any nonzero endomorphism $\varphi : M_1 \to M_2$ is an isomorphism. Hence $\End[R]{M}$ is a division ring.
\end{lemma}

\begin{proof}
Indeed $\ker{\varphi} \subset M_1$ and $\im{\varphi} \subset M_2$ are submodules. Since $M_1$ and $M_2$ are simple these are either $(0)$ or $M_1$ respectively $M_2$. If $\ker{\varphi} = M_1$ then $\varphi = 0$. If $\im{\varphi} = (0)$ then $\varphi = 0$ hence $\varphi$ is a bijection and hence invertible.
\end{proof}


\begin{theorem}[Artin-Wedderburn]
Let $R$ be a semisimple ring. Then, $R$ is isomorphic to a finite direct product of matrix rings:
\[ R \cong M_{n_1}(D_1) \times \cdots \times M_{n_r}(D_r) \]
where $D_i$ are division rings. Moreover the division rings $D_i$ and the integers $r, n_1, \dots, n_r$ are a complete set of invariants of $R$.
\end{theorem}

\begin{proof}
Since $R$ is semisimple, $R$ is semisimple as a right $R$-module. Therefore we can decompose,
\[ R = \bigoplus_{i = 1}^r I_i^{\oplus n_i} \]
where $I_i$ are the minimal nonzero right ideals (i.e. the simple submodules of $R_R$). This sum is finite since $R$ is right Artinian. Since $I_i$ is simple we see that $D_i = \End[R]{I_i}$ is a division ring by Shur's lemma. Therefore,
\[ R \cong \End[R]{R_R} \cong \End[R]{I_1^{\oplus n_i}} \times \cdots \times \End[R]{I_r^{\oplus n_r}} \]
since the $I_i$ are distinct simple modules there are no nonzero maps between them by Shur's lemma. Therefore,
\[ R \cong M_{n_1}(D_1) \times \cdots \times M_{n_r}(D_r) \]
\end{proof}

\section{TEST}

$\mathop{\mathscr{Pic}}$

\section{Affine Torsors}

\newcommand{\Zar}{\mathrm{Zar}}

$\Ga^n$-torsors in the Zariski, \etale, fppf topologies all coincide. This is because they are classified by $H^1(X, \Ga^n)$ but $\Ga^n = \struct{X}^{\oplus n}$ as a sheaf of rings on these sites which is coherent. Therefore, 
\[ H^1_{\Zar}(X, \Ga^n) = H^1_{\et}(X, \Ga^n) = H^1_{\fppf}(X, \Ga^n) \]
Now consider affine bundles, meaning the transition maps are allowed to be any affine linear transformation of $\A^n$ not just translations. Let $E_n$ be the affine algebraic group of affine linear transformations meaning there is an exact sequence,
\begin{center}
\begin{tikzcd}
0 \arrow[r] & \Ga^n \arrow[r] & E_n \arrow[r] & \GL_n \arrow[r] & 0
\end{tikzcd}
\end{center}
Then affine bundles are classified by $H^1(X, E_n)$. For any topology coarser than fppf (should work for any topology such that descent is effective for covers) we get comparison maps,
\begin{center}
\begin{tikzcd}[column sep = tiny]
0 \arrow[r] & H^0(X, \Ga^n) \arrow[d, equals] \arrow[r] & H^0(X, E_n) \arrow[d] \arrow[r] & H^0(X, \GL_n)  \arrow[d, equals]  \arrow[r] & H^1(X, \Ga^n)  \arrow[d, equals] \arrow[r] & H^1(X, E_n) \arrow[d] \arrow[r] & H^1(X, \GL_n) \arrow[d, equals] \arrow[r] & H^2(X, \Ga^n) \arrow[d, equals] 
\\
0 \arrow[r] & H^0_{\tau}(X, \Ga^n) \arrow[r] & H^0_{\tau}(X, E_n) \arrow[r] & H^0_{\tau}(X, \GL_n) \arrow[r] & H^1_{\tau}(X, \Ga^n) \arrow[r] & H^1_{\tau}(X, E_n) \arrow[r] & H^1_{\tau}(X, \GL_n) \arrow[r] & H^2_{\tau}(X, \Ga^n)
\end{tikzcd}
\end{center}
where the maps $H^i(X, \GL_n) \to H^i_{\tau}(X, \GL_n)$ are isomorphism for $i = 0,1$ by Hilbert 90 and the sheaf property. The maps $H^i(X, \Ga^n) \to H^i_{\tau}(X, \Ga^n)$ are isomorphisms for all $i$ since the sheaf is quasi-coherent. Hence, the maps $H^i(X, E_n) \to H^i_{\tau}(X, E_n)$ are also isomorphisms by a diagram chase. Hence \etale or fppf $\A^n$-bundles are Zariski-locally trivial.

\section{DGLA}


\begin{defn}
A \textit{Differential graded Lie algebra} (DGLA) is a $\Z$-graded vector space $L = \bigoplus_{i \in \Z} L^i$ with a bilinear map $[\bullet, \bullet] : L \times L \to L$ and a linear map $\d : L \to \L$ satisfying the conditions on homogeneous elements,
\begin{enumerate}
\item $[L^i, L^j] \subset L^{i+j}$ and $[a,b] + (-1)^{|a|\cdot |b|}[b,a] = 0$
\item $[a,[b,c]] = [[a,b], c] + (-1)^{|a|\cdot|b|} [b, [a,c]]$
\item $\d(L^i) \subset L^{i+1}$ and $\d^2 = 0$ and $\d{[a,b]} = [\d a, b] + (-1)^p [a, \d{b}]$ so $\d$ is derivation with respect to the bracket.
\end{enumerate}
\end{defn}

\begin{rmk}
Note that $L^0$ and $L^{\text{even}}$ are Lie algebras.
\end{rmk}

\begin{defn}
A linear map $f : L \to L$ is a \textit{derivation of degree} $n$ if $f(L^i) \subset L^{i+n}$ such that,
\[ f([a,b]) = [f(a), b] + (-1)^{n |a|} [a, f(b)] \]
\end{defn}

\newcommand{\ad}{\mathrm{ad}}

\begin{rmk}
Note that if $a \in L^i$ then $\ad_a : L \to L$ given by $[a, -]$ is a derivation of degree $i$ and $\d$ is a derivation of degree $1$.
\end{rmk}

\begin{defn}
Denote,
\[ Z^i(L) = \ker{(\d : L^i \to L^{i+1})} \]
and
\[ B^i(L) = \im{(\d : L^{i-1} \to L^i)} \]
and
\[ H^i(L) = Z^i(L)/B^i(L) \]
The Mauer-Cartan equation of a DGLA $L$ is,
\[ \d{a} + \tfrac{1}{2} [a,a] = 0 \]
\end{defn}

\begin{defn}
A morphism $f : L_1 \to L_2$ of DGLAs is a graded linear map commuting with the bracket and differential. We say that $f$ is a \textit{quasi-isomorphism} if $f_* : H^i(L_1) \to H^i(L_2)$ are isomorphisms.
\end{defn}

\begin{defn}
A DGLA $L$ is \textit{Formal} if it is quasi-isomorphic to $H^\bullet(L)$.
\end{defn}

\begin{prop}
Let $D : L \to L$ be a derivation, then $\ker{D}$ is a graded Lie subalgebra.
\end{prop}

\begin{proof}
We need to show that if $a, b \in \ker{D}$ then, $[a,b] \in \ker{D}$ since as a subalgebra it automatically satifies the other properties. Indeed, 
\[ D([a,b]) = [Da, b] + (-1)^{|a|} [a, Db] = 0 \]
so $[a,b] \in \ker{D}$.
\end{proof}

\begin{prop}
Let $L$ be a DGLA and $a \in L^i$ then,
\begin{enumerate}
\item if $i$ is even $[a,a] = 0$
\item if $i$ is odd $[a, [a,b]] = \tfrac{1}{2} [[a,a],b]$ for $b \in L$ and $[[a,a],a] = 0$.
\end{enumerate}
\end{prop}
\begin{proof}
Indeed,
\[ [a,a] + (-1)^{i^2} [a,a] = 0 \]
but $i$ is even so we see that $[a,a] = 0$. Now we use the Jacobi identity,
\[ [a,[a,b]] = [[a,a], b] + (-1)^{i^2} [a,[a,b]] \]
but $i$ is odd so we see that,
\[ [a,[a,b]] = \tfrac{1}{2} [[a,a],b] \]
Furthermore, for $a = b$ then,
\[ [a,[a,a]] + (-1)^{i^2} [[a,a],a] = 0 \]
hence these are minus eachother so from the above we conclude,
\[ [[a,a],a] = 0 \]
\end{proof}

\begin{example}
Let $M$ be a complex manifold and $E$ a holomorphic vector bundle on $M$. Let $\cA^{p,q}(E)$ be the sheaf of $C^{\infty}$-differential forms valued in $E$ of type $(p,q)$. Then consider,
\[ L = \bigoplus_q \Gamma(M, \cA^{0,q}(\End{E})) [-q] \]
with differential $\bar{\partial}_E$ and the natural bracket defined by,
\[ [\varphi \ot \omega, \varphi' \ot \omega'] = [\varphi, \varphi'] \ot (\omega \wedge \omega') \]
This is a DGLA. Indeed, it is clear that $[\bullet, \bullet]$ is graded skewsymmetric since there is an extra sign in $[\varphi, \varphi']$ additionally to $\wedge$ which is graded commutative. Then the Jacobi identity follows from the Jacobi identity for $\End{E}$ and associativity of $\wedge$. The derivation property of $\d$ follows from the fact that,
\[ \d (\omega \wedge \omega') = \d{\omega} \wedge \omega' + \omega (-1)^{q} \wedge \d{\omega'} \]
\end{example}


\section{Deformation Complexes}

\newcommand{\LL}{\mathbb{L}}
\newcommand{\ob}{\mathrm{ob}}

\begin{defn}
Let $\X$ be a DM stack. Let $L^\bullet \in D(\struct{\X_{\et}})$. We say that, 
\begin{enumerate}
\item $L^\bullet$ is \textit{admissible} if,
\begin{enumerate}
\item $h^i(L^\bullet) = 0$ for all $i > 0$
\item $h^i(L^\bullet)$ is coherent for $i = 0,-1$.
\end{enumerate}
\item $L^\bullet$ is \textit{perfect} (with amplitude contained in $[a,b]$) if locally it quasi-isomorphic to a complex,
\[ 0 \to \E^a \to \cdots \to \E^b \to 0 \]
where each $\E^i$ is a vector bundle living in degree $i$. 
\end{enumerate}
\end{defn}

\begin{defn}
Let $E^\bullet \in D(\struct{\X_\et})$ be admissible. Then a homomorphism $\psi : E^\bullet \to \LL_{\X}$ is an \textit{obstruction theory} if,
\begin{enumerate}
\item $h^0(\phi)$ is an isomorphism
\item $h^{-1}(\phi)$ is surjective.
\end{enumerate}
\end{defn}

\begin{rmk}
Given a lifting problem,
\begin{center}
\begin{tikzcd}
T \arrow[r, "g"] \arrow[d, hook] & \X
\\
\ol{T} 
\end{tikzcd}
\end{center}
with ideal sheaf $\J$, we use the map $\phi$ to pullback the obstruction class $\omega(g) \in \Ext{1}{}{g^* \LL_{\X}}{\J}$ to an obstruction,
\[ \ob_{E}(g) := \phi^* \omega(g) \in \Ext{1}{}{g^* E^\bullet}{\J} \]
\end{rmk}


\begin{defn}
An obstruction theory $E^\bullet \to \LL_{\X}$ is \textit{perfect} if $E^\bullet$ is of perfect amplitude contained in $[-1,0]$.
\end{defn}

\begin{theorem}
The following are equivalence,
\begin{enumerate}
\item $\phi : E^\bullet \to \LL_{\X}$ is an obstruction problem

\item for any lifting problem $(T, \ol{T}, g)$ the obstruction $\ob_{E^\bullet}(g) \in \Ext{1}{}{g^* E^\bullet}{\J}$ vanishes if and only if there is a solution to the lifting problem and if $\ob_{E^\bullet}(g) = 0$ then the extensions form a torsor under $\Ext{0}{}{g^* E^\bullet}{\J} = \Hom{}{g^* h^0(E^\bullet)}{\J}$.
\end{enumerate}
\end{theorem}

\subsection{Examples}

\newcommand{\R}{\mathbf{R}}

\begin{example}
If $\X$ is smooth then $\LL_{\X}$ is a perfect deformation theory and there are no obstructions to lifting. 
\end{example}

\begin{example}
Let $C, X \to S$ be proper $S$-schemes. Consider the Hom scheme $\uHom{S}{C}{X}$ constructed from the Hilbert scheme. Recall that when $X$ is smooth, there is a classical tangent-obstruction theory,
\[ T^i = \Ext{i}{\struct{C}}{f^* \Omega_{X/S}}{\struct{C}} \]
to deforming a map $f : C \to X$ over $S$.
\bigskip\\
Let $H = \uHom{S}{C}{X}$ be the Hom scheme and consider the diagram,
\begin{center}
\begin{tikzcd}
H \times C \arrow[d, "\pi_1"] \arrow[r, "\ev"] & X
\\
C
\end{tikzcd}
\end{center}
Therefore we get maps,
\[ \ev^* \LL_{X} \to \LL_{H \times C} = \pi_1^* \LL_{H} \oplus \pi_2^* \LL_{C} \to \pi^*_1 \LL_{H} \]
using the functoriality of the cotangent complex. We want to push this forward to $H$ but the adjunction goes the wrong way. To fix this, we use Grothendieck duality. Assume that $C$ admits a dualizing complex $\omega_{C/S}^\bullet$. Then applying $- \ot^{\LL} \omega_{C/S}^\bullet$ we get,
\[ (\ev^* \LL_X) \ot^{\LL} \omega_{C/S}^\bullet \to (\pi_1^* \LL_{H}) \ot^{\LL}_{C/S} \iso \pi_1^! \LL_{H} \]
Now we can use the correct adjunction to get,
\[ \R \pi_* ([\ev^* \LL_{X}] \ot^{\LL} \omega_{C/S}^\bullet) \to \LL_{H} \]
Therefore, 
\[ E^\bullet = \R \pi_* ([\ev^* \LL_{X}] \ot^{\LL} \omega_{C/S}^\bullet) \]
is our candidate for an obstruction theory.
\end{example}

\begin{rmk}
This makes sense because we should be computing $\Ext{1}{\struct{C}}{f^* \Omega_X}{\struct{C}} = H^1(C, f^* \T_X)$ on the fibers in the smooth case where as $\R \pi_* \LL_{X}$ would be more like computing $H^1(f^* \Omega_X)$ on the fibers. In either case we will take $\RHom{}{-}{\struct{C}}$ on the base but this does not dualize $f^* \Omega_X$ to $f^* \T_X$ \textit{inside} the $\R \pi_*$.
\end{rmk}

\begin{theorem}
Let $S = \Spec{k}$. If $C$ is gorenstein then $E^\bullet \to \LL_{H}$ is an obstruction theory. If $C$ is a curve and $X$ is smooth the obstruction theory is perfect.
\end{theorem}

\begin{rmk}
IS GORENSTEIN REALLY NEEDED??
\end{rmk}

\begin{proof}
DO THIS!!!
\end{proof}

\begin{lemma}[Existence of the Mumford Complex]
Let $\pi : X \to S$ be a projective map 
\end{lemma}

\subsection{Moduli Stack of PRojective Varieties}

Let $\M$ and $\X$ be DM stacks and $p : \M \to \X$ be a flat relatively Gorenstein projective morphism (has constant relative dimension and that the relative dualizing complex $\omega_{\M/\X}^{\bullet}$ is a line bundle $\omega_{\M/\X}$. 
\bigskip\\
If $G^\bullet \in D^+(\struct{\X})$ when $p^! G^\bullet = p^* G^\bullet \ot^{\LL} \omega_{\M/\X}$ and thus for any complex $F^\bullet \in D^{-}(\struct{\M})$ there are natural isomorphisms,
\[ \Ext{k}{\struct{\M}}{F^\bullet}{p^* G^\bullet} \to \Ext{k}{\struct{\M}}{F^\bullet \ot^{\LL}}{p^! G^\bullet} \to \Ext{k}{\struct{\X}}{ \R p_* (F^\bullet \ot^{\LL} \omega_{\M/\X})}{G^\bullet} \]
The connecting map on cotangent complexes,
\[ \LL_{\M/\X} \to p^* \LL_{X}[1] \]
which we call the Kodaira-Spencer map since it reproduces it in the classical setting then induces a map,
\[ E^\bullet = \R p_* (\LL_{\M/\X} \ot^{\LL} \omega_{\M/\X}^\bullet)[-1] \to \LL_{\X}^\bullet \]

\begin{proof}
In the above situation. If $p : \M \to \X$ is \textit{universal} then $E^\bullet \to \LL^\bullet_X$ is an obstruction theory for $X$.
\end{proof}

\begin{proof}
DO THIS!!
\end{proof}

\begin{cor}
If $p$ is smooth of relative dimension $\le 2$ then $E^\bullet$ is a perfect obstruction theory. 
\end{cor}
\subsection{Cones}

\subsection{Construction of Virtual Fundamental Classes}


\begin{prop}
If $E^\bullet$ is a perfect obstruction theory with $h^0(E^\bullet)$ locally free and $h^1(E^\bullet) = 0$ then $X$ is smooth, the virtual dimension of $[X, E^\bullet]$ is $\dim{X}$ and $[X, E^\bullet] = [X]$.
\end{prop}

\begin{prop}
Let $X$ be smooth and $E^\bullet$ a perfect obstruction theory for $X$. If $h^0(E^\bullet)$ is locally free then the virtual fundamental class is,
\[ [X, E^\bullet] = c_r(h^1(E^{\bullet \vee})) \cdot [X] \]
where $r = \rank{h^1(E^{\bullet \vee})}$.
\end{prop}

\begin{proof}
If $F^\bullet \to E^\bullet$ is a global resolution of $E^\bullet$, then $C(F^\bullet) = \im{(F_0 \to F_1)}$.
\end{proof}

\section{pseudo-Torsors}

\begin{defn}
Let $G \to X$ be an $X$-group scheme. Then a \textit{pseudo $G$-torsor} $T \to X$ is an $X$-scheme together with an $X$-action $\rho : G \times_X T \to T$ such that the natural map of $T$-schemes,
\[ G \times_X T \xrightarrow{(\rho, \pi_2)} T \times_X T \]
is an isomorphism. A morphism $f : T \to T'$ of pseudo-torsors is a $G$-equivariant morphism of $X$-schemes.
\end{defn}

\begin{rmk}
If $T \to X$ is a pseudo $G$-torsor then for any $X' \to X$ we have $T \times_X X' \to X'$ is a $G \times_X X'$-pseudo-torsor.
\bigskip\\
Indeed, the map,
\[ G_{X'} \times_{X'} T_{X'} \to T_{X'} \times_{X'} T_{X'} \]
is just the base change of $G \times_X T \to T \times_X T$ along $X' \to X$ over the unique projection to $X$. 
\end{rmk}

\begin{rmk}
This is a map of $T$-schemes for the second projection structure. Note that it is $G$-equivariant for the actions $g \cdot (g', t) = (g g', t)$ and $g \cdot (t, t') = (g \cdot t, t')$ which is an action over $T$. Note that,
\[ T \times_X G \xrightarrow{(\pi_1, \rho)} T \times_X T \]
is likewise an isomorphism of $T$-schemes using the first projection structure and is equivariant for the actions $g \cdot (t, g') = (t, g g')$ and $g \cdot (t, t') = (t, g \cdot t')$.
\bigskip\\
We can also consider,
\[ G \times_X T \xrightarrow{(\rho, \pi_2)} T \times_X T \]
as an isomorphism of $X$-schemes. Now over $X$ these both have $G \times_X G$-actions (the second $G$ makes $G \times_X T \to T$ equivariant but it is not an action over $T$ since $G \times_X T \to T$ is a map of $T$-schemes only if $G$ acts trivially). However, it is not an isomorphism of $G \times_X G$-pseudo-torsors since it does not intertwine both actions. The action on the section factor of the RHS corresponds to the anti-diagonal action $g \cdot (g', t) = (g' g^{-1}, g \cdot t)$ on the LHS not the action on the second factor. Indeed, as $G \times_X G$-torsors over $X$, in general, $G \times_X T$ and $T \times_X T$ are not isomorphic. For exmaple, if $T$ is the frame bundle of a vector bundle $E$ such an isomorphism would imply that $E \oplus E \cong \struct{X}^{\oplus \rank{E}} \oplus E$.
\bigskip\\
What is confusing is that $T \times_X T$ is isomorphic to $G \times_X T$ for either $G$-action \textit{individually} but not simultaneously (IS THIS FALSE!!!!). The other isomorphism is given by,
\[ G \times_X T \xrightarrow{(\rho, \pi_2)} T \times_X T \]
then $g \cdot (g', t) = (g', g \cdot t)$ is compatible with $g \cdot (t, t') = (t, g \cdot t')$
\end{rmk}

\begin{prop}
If a pseudo-torsor $\pi : T \to X$ has a section if and only if it is a trivial torsor i.e. there is a $G$-equivariant $X$-isomorphism,
\[ G \iso T \]
\end{prop}

\begin{proof}
Given an isomorphism $G \iso T$ we immediately transfer the zero section. Thus if $\sigma : X \to T$ is a section then consider the map $\varphi : G \to T$,
\[ G \xrightarrow{\id \times \sigma} G \times_X T \xrightarrow{\rho} T \]
which is clearly $G$-equivariant. We claim this is an isomorphism. Indeed, consider the diagram,
\begin{center}
\begin{tikzcd}
G \pullback \arrow[d, "\id \times \sigma"] \arrow[r, "\varphi"] & T \arrow[d, "\id \times \sigma"]
\\
G \times_X T \arrow[r] & T \times_X T
\end{tikzcd}
\end{center}
which is Cartesian because if $(f, g) : S \to G \times_X T$ and $h : S \to T$ agree as maps $G \to T \times_X T$ then $g = \sigma \circ s$ where $s : S \to X$ is the structure map and $h = \rho \circ (f \times g) = (f \times \sigma \circ s)$ hence this data is the same as $f : S \to G$ and we compute $g = \sigma \circ s = \sigma \circ \pi \circ f$ and $h = \varphi \circ f$. Hence by base change $\varphi$ is an isomorphism.   
\end{proof}

\begin{prop}
A scheme $T \to X$ equipped with a $G$-action over $X$ is a pseudo $G$-torsor if and only if the sheaf $h^T$ is a pseudo $h^G$-torsor in the sense that for any test scheme $S \to X$ then $T(S)$ is either empty or a $G(S)$-torsor.
\end{prop}

\begin{proof}
First suppose that $T$ is a pseudo $G$-torsor and that $T(S)$ is nonempty. Choose a map $S \to T$ then pulling back to $S$ the $G_S$-torsor $T_S \to S$ has a section and hence is isomorphic to $G_S \to S$. Therefore, $T(S) = T_S(S) = G_S(S) = G(S)$.
\bigskip\\
Conversely, suppose that $h^T$ is a pseudo $h^G$-torsor. The action $h^G \acts h^T$ induces $G \acts T$ over $X$. To show that $G \times_X T \to T \times_X T$ is an isomorphism it suffices to show that the associated map of sheaves is an isomrphism. However, working in the category of $X$-schemes, the map of sets,
\[ G(S) \times T(S) \to T(S) \times T(S) \]
is a bijection if and only if $T(S)$ is a $G(S)$-torsor or is empty. Indeed, if $T(S)$ is nonempty then considering the fiber over $T \times \{ t_0 \}$ this says that $G(S) \to T(S)$ via $g \mapsto g \cdot t_0$ is bijective.
\end{proof}

\begin{example}
It is clear from the definitions that $T = \empty$ is a pseduo $G$-torsor. For any pseudo $G$-torsor $T \to U$ and $U \embed X$ an open embedding we also see that $T \to X$ is a pseduo $G$-torsor. Clearly we should require some sort of surjectivity or we're not get the right notion 
\end{example}

However, the worst example is probably the following:

\begin{example}
Let $G = *$ and $X = \Spec{R}$ with $(R, \m, \kappa)$ a dvr with $K = \Spec{R}$. Then $\Spec{K} \sqcup \Spec{\kappa} \to \Spec{R}$ is a $G$-torsor. Indeed for $G = \{ * \}$ we are just considering maps $T \to X$ such that $\Delta_{T/X} : T \to T \times_X T$ is an isomorphism i.e. monomorphisms of schemes. Given such an example we can base change it by any group to get a nonflat but surjective $G$-torsor. Now monomorphisms of affine schemes are the same as monomorphisms in the category of affine schemes (since maps into an affine scheme are controlled at the level of global sections) hence correspond to epimorphisms of rings. The map $R \to K \times \kappa$ is an epimorphism of rings because if $\varphi, \psi : K \times \kappa \to A$ agree on $R$ then we just need to show that $\varphi, \psi$ agree on $\{ 0 \} \times \kappa$ and $K \times \{ 0 \}$. But these maps are obtained via applying $- \ot_R K$ or $- \ot_R \kappa$ as maps of $R$-modules since the images of these submodules are automatically $K$ or $\kappa$-modules since $\varphi, \psi$ are ring maps. Either functor kills one factor so the map from $R$ beceomes an isomorphism and hence the two morphisms agree. 
\bigskip\\
Another such example of a surjective nonflat monomorphism is $\Spec{k} \to \Spec{k[\epsilon]}$.
\end{example}


\begin{defn}
A $G$-pseduo-torsor is \textit{split} or \textit{trivial} if it has a section (meaning it is isomorphic to the trivial torsor). A pseudo-torsor is a $G$-\textit{torsor} if it is fppf-locally split. This is equivalent to saying that for some fppf cover $U \to X$ we have $T(U) \neq \empty$.
\end{defn}



\begin{prop}
Let $\pi : T \to X$ be a $G$-torsor. If $\cP$ is an fppf-local property and $G \to X$ has $\cP$ then $T \to X$ has $\cP$.
\end{prop}

\begin{proof}
Obvious since after an fppf cover $U \to X$ we get that $T_U \to U$ is isomorphic to $G_U \to U$.
\end{proof}

\begin{cor}
A pseudo $G$-torsor $\pi : T \to X$ is a $G$-tosor if and only if $\pi : T \to X$ is fppf hence $T \to X$ is itself an fppf cover that splits $T$ (torsors kill themselves).
\end{cor}

\begin{example}
Let $G \acts X$ over a base scheme $S$. Then $\rho : G \times_S X \to X$ is a $G$-torsor over $X$ if $G$ is a fppf $S$-group. Furthermore, if $G \to S$ has property $\cP$ (where $\cP$ is true of isomorphisms and is preserved under composition and base change) then $\rho : G \times_S X \to X$ has $\cP$.
\bigskip\\
This is because the following diagram commutes,
\begin{center}
\begin{tikzcd}
G \times_S X \arrow[rr, "(\pi_1 \, \rho)"] \arrow[rd, "\rho"'] & & G \times_S X \arrow[ld, "\pi_2"]
\\
& X 
\end{tikzcd}
\end{center}
and $(\pi_1, \rho) : G \times_S X \to G \times_S X$ is an isomorphism because it has inverse $(\pi_1, \rho \circ (\iota \times \id))$. Therefore, if $\pi$ has $\cP$ so does $\rho$.
\end{example}

We saw that maps of pseudo torsors need not be isomorphisms (e.g. pullbacks of open or closed embeddings on the base). However maps of torsors are always isomorphisms.

\begin{prop}
Let $f : T \to T'$ be a map of $G$-torsors. Then $f$ is an isomorphism.
\end{prop}

\begin{proof}
It suffices to check after an fppf cover so we may assume that $T$ is split. Then the map $f$ gives a section of $T'$ so $T'$ is also split. Hence there is a commutative diagram,
\begin{center}
\begin{tikzcd}
G \arrow[d] \arrow[r, dashed] & G \arrow[d]
\\
T \arrow[r, "f"] & T'
\end{tikzcd}
\end{center}
since the downward maps are $G$-equivariant isomorphisms we get a $G$-equivariant map $G \to G$. However, these are always of the form $m \circ (\id \times \sigma)$ for a section $\sigma : X \to G$ (indeed consider the image of $e : X \to G$) and hence are isomorphisms. Thus $f$ is an isomorphism.
\end{proof}


Now we ask the following question: suppose $\pi : T \to X$ has a $G$-action over $X$ such that for each $x \in X$ then $G_x \acts T_x$ is isomorphic to $G_x$ with its left translation action. Then is $\pi : T \to X$ a $G$-torsor? 

\begin{example}
The answer is, in general, no. We saw an example,
\[ G_K \sqcup G_\kappa \to \Spec{R} \]
for $R$ a dvr with fraction field $K$ and residue field $\kappa$ which satisfies the hypothesis but is not flat. Another such example is $G_k \to \Spec{k[\epsilon]}$.
\end{example}

Therefore, we should restrict to $T$ and $X$ varieties. However, if we don't controll the singularities, this can still go wrong.

\begin{example}
Let $X = \Spec{k[x,y]/(y^2 - x^3)}$ and $T = \A^1 \sm \{ - 1 \}$ with the map $x \mapsto t^2$ and $y \mapsto t^3$. This map,
\[ k[x,y]/(y^2 - x^3) \to k[t, (t+1)^{-1}] \]
of rings is an epimorphism and a monomorphism. Indeed, $(t^2 - 1)/(t + 1) = t - 1$ and hence $t$ is in the image. Furthermore, $k[t] \to k[t, (t+1)^{-1}]$ is an epimorphism since if you know where $(t+1)$ goes then you specify where $(t+1)^{-1}$ goes. Therefore, this gives an example of constant fiber degree $1$ but not flat. 
\bigskip\\
Base change along any $G$ gives an example of a pseudo $G$-torsor 
\end{example}

However, failure of flatness is the only issue as the following result shows.

\begin{prop}
If $X' \to X$ is a finite type separated map of noetherian schemes with constant fiber degree $1$ then,
\begin{center}
\begin{tikzcd}
\text{proper} \arrow[r, "\text{quasi-finite}"] & \text{finite} \arrow[l] \arrow[d]
\\
\text{flat} \arrow[u, "(a)"] \arrow[ru, "(b)"] & \text{isomorphism} \arrow[l]
\end{tikzcd}
\end{center}
where the implication (a) is from the following result and (b) is from the lemma of Deligne-Rappoport (see Brian exercise done in previous section). Hence any of these properties implies it is an isomorphism.
\end{prop}

\begin{prop}[EGA IV.15.7.10]
If $f : X \to Y$ is universally submersive (e.g. flat), finite type, separated and has proper and geometrically connected fibers then $f$ is proper.
\end{prop}


Therefore, miracle flatness works in the regular case. 

\begin{prop}
Let $G$ be a smooth $k$-group and $T \to X$ a morphism of $k$-varities with $X$ regular and a $G$-action on $T$ relative to $X$. If each fiber $T_x \cong G_{\kappa(x)}$ with its left translation action then $T \to X$ is a $G$-torsor. 
\end{prop}

\begin{proof}
First we claim that $T$ is regular. Since $X$ is regular the fibers over closed points are cut out by $\dim{X}$ elements and since $G_{\kappa(x)}$ is regular the fibers are regular. Therefore, $T$ is regular since the dimension of the fibers are $\dim{T} - \dim{X}$ (DO BETTER LOOK AT WHAT I WROTE EARLIER ABOUT MAPS WITH REGULAR FIBERS).
\bigskip\\
First, notice that $T \to X$ is flat by miracle flatness (FINISH PROOF, HOW TO GET PSEUDO-TORSOR CONDITION).
\end{proof}


IS NORMALITY ENOUGH IN THE $G$ IS sFINITE CASE?? NON-NORMAL EXAMPLE FOR $G$ NOT FINITE

\section{Applications of Zariski Connectedness}

\begin{prop}
Let $f : X \to S$ me a lfp morphism of schemes. Consider the function,
\[ n(s) := \# \pi_0(X_{\bar{s}}) \]
sending $s$ to the number of geometric components of the fiber. If $f$ is flat and proper then $n : S \to \Z$ is lower semi-continuous. If morover, the fibers of $f$ are geometrically reduced then $n$ is locally constant.
\end{prop}

\begin{rmk}
Without the reducedness assumption, $n$ can certainally jump down e.g. consider a ramified covering of curves. Morover, all of the assumptions are necessary:
\begin{enumerate}
\item flatness: consider $\A^1 \sqcup \{ * \} \to \A^1$ mapping $* \to 0$
\item properness: consider $X = \Proj{k[t][X,Y,Z]/(X^2  - Y^2 + t Z^2)} \to \Spec{k[t]}$ which is the degeneration of a conic to two lines. Then consider $X \sm \{ (0, [0,0,0]) \}$ where we remove the point connecting the two conics hence the number of connected components jumps up
\end{enumerate}
\end{rmk}

\begin{proof}
Consider the Stein factorization $f : X \to S' \to S$ then the connected components of $X_{\bar{s}}$ are in bijection with the connected components of $S'_{\bar{s}}$ since $X \to S'$ has geometrically connected components. Hence if $S' \to S$ were flat this would be easy. However, this is not always the case\footnote{Suppose the schemes are noetherian, then $S' \to S$ finite and is moreover flat if and only if $f_* \struct{X}$ is a vector bundle (we need $S' \to S$ to be lfp for this equivalence) but there are examples of flat proper maps of noetherian schemes $f : X \to S$ such that $f_* \struct{X}$ is not a vector bundle. Whenever there is a failure of cohomological flatness this can occur for a thickening of the fiber via the theorem of formal functions e.g. \chref{https://mathoverflow.net/questions/65267/global-sections-of-flat-scheme-also-flat}{here}. DOD BETTER  }. Instead we use etale localization. Suppose there are $n$ geometric points in the fiber over $s$. Then by \chref{https://stacks.math.columbia.edu/tag/0BSR}{Tag 0BSR} there exists an \etale neighbrohood $U \to S$ of $s$ such that $S'_U = V_1 \sqcup V_2 \sqcup \cdots V_m$ with $s_i \in V_i$ and $\kappa(s_i) / \kappa(s)$ purely inseparable. Then $X_{V_i} \to X$ are \etale and $X \to S$ is flat and lfp so $X_{V_i} \to S$ has open image. Since there are finitely many, the intersection of their images is an open neighborhood of $s$ on which there are at least $m$ geometric points in each fiber.
\bigskip\\
Now suppose that $f : X \to S$ has geometrically reduced fibers. Then $n(s) = \dim_{\kappa(s)} H^0(X_s, \struct{X_s})$ because $H^0(X_s, \struct{X_s})$ is a finite product of local Artinain $\kappa(s)$-algebras corresponding to the connected components. By flat base change, we compute $H^0(X_{\bar{s}}, \struct{X_{\bar{s}}})$ as the base change and this is reduced hence a product of $\kappa(\bar{s})$ since this is an algebraically closed fields and the number of products is eactly $n(s)$ and equal to the dimension of $H^0(X_s, \struct{X_s})$. However, the function $s \mapsto H^0(X_{s}, \struct{X_s})$ is upper semicontinuous hence we conclude.
\end{proof}

\begin{rmk}
It is interesting to me that ``proper pushforward of coherent is coherent'' is only considered in the locally noetherian case in EGA see [EGA III.1, Theorem 3.2.1]. Can you do anything by noetherian approximation? Maybe not if you can't make the base change from the noetherian setting flat in order to use flat base change. Hmm?
\end{rmk}

\newcommand{\Y}{\mathscr{Y}}

\begin{theorem}
Let $f : \X \to \Y$ me a lfp morphism of DM-stacks. Consider the function,
\[ n(s) := \# \pi_0(\X_{\bar{s}}) \]
sending $s$ to the number of geometric components of the fiber. If $f$ is flat and proper then $n : S \to \Z$ is lower semi-continuous. If morover, the fibers of $f$ are geometrically reduced then $n$ is locally constant.
\end{theorem}

\begin{proof}
The proof is identical up to some reductions. Taking an \etale cover, we can assume $\Y = S$ is a scheme. Just as for schemes, there is a Stein factorization,
\[ f : \X \to \rSpec{S}{f_* \struct{\X}} = S' \to S \]
and $S'$ is affine over $S$ is hence a scheme. Therefore, we conclude by applying \etale localiation to $S' \to S$ exactly as above. Then we get opens $\X_{V_i}$ and the maps $\X_{V_i} \to S$ are flat and lpf hence open so taking the intersection of these images gives an open where the function $n$ is at least $n(s)$. To conlude we need to prove upper semicontinuity for stacks. 
\end{proof}

\begin{prop}
Let $\X$ be a normal noetherian connected DM-stack. Then $\X$ is irreducible.
\end{prop}

\begin{proof}
Let $U \to \X$ be an \etale neighborhood then $U$ is normal. Shrinking so that $U$ is connected (noetherianity implies that the connected components are clopen) we see that $U$ is irreducible. Hence its image is open and irreducible. Therefore, $\X$ is locally irreducible and connected and hence irreducible. Indeed, for any irreducible component $T \subset |\X|$ then either $T \cap U_i$ is empty or $T \cap U_i$ is an irreducible component because if $T \cap U_i \subset Z \subset U_i$ is a larger irreducible then $\overline{Z}$ is irreducible and contains $T$ since $T \subset \ol{Z} \cup U_i^C$ are closed but $T \not\subset U_i^C$ because $T \cap U_i \neq \empty$ hence $T \subset \ol{Z}$ and $Z$ is irreducible so $T = \ol{Z}$ by maximality and hence $T \cap U_i = \ol{Z} \cap U_i = Z$ so we conclude that $T \cap U_i$ is an irreducible component of $U_i$. But $U_i$ is irreducible so $U_i \supset T$ and hence $T = |\X|$ so we conclude.  
\end{proof}

\begin{theorem}
The moduli space of curves $\M_{g,n}$ over any field is geometrically irreducible.
\end{theorem}

\begin{proof}
For $n' \ge n$ we have $\M_{g,n'} \to \M_{g,n}$ is surjective so it suffices to prove the claim for $n' \gg 0$ so we may assume that $\M_{g,n}$ is DM.
It suffices to prove this over prime subfields. Then consider $\M_{g,n} \to \Spec{\Z}$ and the DM-compactification of stable curves $\ol{\M}_{g,n} \to \Spec{\Z}$ which is smooth and proper map of noetherian DM-stacks. Hence, we can apply Zariski connectedness to conclude that the number of connected components is constant and by smoothness (hence geometric normality) if the fiber is geometrically connected then it is geometrically irreducible. Therefore we reduce to $\Q$. Then we base change to $\CC$ so we win by classical methods. Then $\M_{g,n} \embed \ol{\M}_{g,n}$ is open so we win.
\end{proof}
\end{document}
