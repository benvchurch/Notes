\documentclass[12pt]{article}
\usepackage{hyperref}
\hypersetup{
    colorlinks=true,
    linkcolor=blue,
    filecolor=magenta,      
    urlcolor=blue,
}

\usepackage{import}
\import{"../Algebraic Geometry/"}{AlgGeoCommands}

\newcommand{\Loc}[1]{\mathfrak{Loc}\left( #1 \right)}
\newcommand{\AbGrp}{\mathbf{AbGrp}}

\begin{document}

\section{Picard Scheme}

\begin{theorem}
Let $X$ be a proper $k$-scheme. Then $\fPic_{X/k}$ is represented by a lft $k$-scheme. 
\end{theorem}

\begin{rmk}
However, this does not hold for proper flat families in general even for curves. 
\end{rmk}

From here on let $f : X \to S$ be a flat, locally finitely presented, proper morphism where $S = \Spec{R}$ is a DVR. 

\begin{theorem}[8.3.2]
$\fPic_{X/S}$ is represented by an algebraic space if and only if $f$ is cohomologically flat in degree $0$.
\end{theorem}

\begin{rmk}
In fact, the above holds when $S$ is any reduced scheme.
\end{rmk}

This is a problem since we want to study non-cohomologically flat situations. We fix this in the next section.

\subsection{Rigidified Picard Scheme}

\begin{prop}[8.1.6]
$f$ admits a rigidifying subscheme meaning a closed subscheme $Y \subset X$ which is flat, locally finitely presented, proper and such that for any $T \to S$ the map,
\[ \Gamma(X_T, \struct{X_T}^\times) \to \Gamma(Y_T, \struct{Y_T}^\times) \]
is injective.  
\end{prop}


\begin{defn}
Let $Y \embed X$ be a rigidifying subscheme. Then we define the rigidified Picard  functor,
\[ \fPic_{X/S|Y} : (T \to S) \mapsto \{ (\L, \varphi) \mid \L \in \Pic{X_T} \text{ and } \varphi : \L|_Y \iso \struct{Y} \} / \cong \]
The condition of being a rigidifying subscheme shows exactly that there are no nontrivial automorphism of $(\L, \varphi)$. 
\end{defn}

FGA shows that the functor,
\[ (T \to S) \mapsto (f_T)_* \struct{X_T} \]
is representable by a linear scheme $V_X$ over $X$. This is a vector bundle over $X$ iff $f$ is cohomologically flat in degree $0$. Furthermore, the subsheaf of units,
\[ (T \to S) \mapsto (f_T)_* \struct{X_T}^\times \]
is represented by an open subscheme,
\[ V_X^\times \subset V_X \]
Now $V_X$ is a ring scheme and $V_X^\times$ is a group scheme. 

\begin{prop}
Let $Y \embed X$ be a rigidifier. 
There is an exact sequence of fppf sheaves of abelian groups,
\begin{center}
\begin{tikzcd}
0 \arrow[r] & V_X^\times \arrow[r] & V_Y^\times \arrow[r] & \fPic_{X/S} \arrow[r] & \fPic_{X/S|Y} \arrow[r] & 0
\end{tikzcd}
\end{center}
where the last map forgets the rigidification. It is surjective in the fppf topology because by definition any class in $\fPic_{X/S}$ is fppf locally represented by a line bundle.
\end{prop}

\begin{theorem}[8.3.3]
Let $Y \embed X$ be a rigidifier. Then $\fPic_{X/S|Y}$ is representable by an algebraic space over $S$ which admits a universal rigidified line bundle.
\end{theorem}

\begin{prop}
Let $s \in S$ be a point such that $H^2(X_s, \struct{X_s}) = 0$. Then there is an open neighborhood $s \in U \subset S$ such that, both $\fPic_{X/S|Y}|_U$ and $\fPic_{X/S}|_U$ are formally smooth over $U$.
\end{prop}

\subsection{Relative Curves}

Now suppose that $f$ has relative dimension $1$ and has geometrically connected fibers.


\section{Overview of the Proof}

\newcommand{\gon}{\mathrm{gon}}

\begin{defn}
Let $C/k$ be an integral curve over a field $k$. Then the \textit{gonality} of $C$ is the smallest degree of a finite map $C \to \P^1$ \textit{over} $k$. The \textit{geometric gonality} of $C$ is the maximum of the gonality over $\bar{k}$ of the irreducible components of $C_{\bar{k}}$. 
\end{defn}

\begin{lemma}
Let $f : X \to B$ be a proper morphism of relative dimension $1$ between normal varities.
\end{lemma}

\begin{lemma}
Let $f : X \to B$ a proper morphism of relative dimension $1$ of varities over a perfect field $k$ whose generic fiber is a smooth connected curve. Let $n = \dim{X}$. Suppose there is a line bundle $\L \embed \Omega_X^{n-1}$ whose sections separate $d$ general points on $X$. Then the general fiber of $f$ has gonality $> d$.
\end{lemma}

\begin{proof}
We can shrink $B$ such that the base and the map are smooth. Choose a general fiber $C \embed X$ which is a smooth irreducible curve. Therefore, there is an exact sequence,
\[ 0 \to \C_{C|X} \to \Omega_{X}|_C \to \Omega_C \to 0 \]
of vector bundles. Since $\Omega_C$ is a line bundle there is an exact sequence,
\[ 0 \to \C_{C|X}^{n-1} \to \Omega^{n-1}_X|_C \to (\wedge^{n-2} \C_{C|X}) \ot \Omega_C \to 0 \]
However, since $C$ is a fiber of $f$ we have $\C_{C|X} = \struct{X}^{n-1}$. Therefore, we get $n-1$ projection maps,
\[ \L \to \Omega_X^{n-1}|_C \to \Omega_C \]
which are all zero exactly if $\L \embed \Omega_X^{n-1}$ factors through $\C_{C|X}^{n-1}$ but these forms are constant along fibers so sections of $\L$ would not be able to separate any points on $C$. Therefore, one of the projections $\L \to \Omega_C$ is a nonzero map of line bundles hence injective meaning that,
\[ H^0(C, \L) \to H^0(C, \Omega_C) \]
must be injective. Since we chose $C$ generically $H^0(C, \L)$ and hence $H^0(C, \Omega_C)$ can separate $d$ general points on $C$. Therefore $\gon(C) > d$. 
\end{proof}

\newpage 

\section{Meaning of Supersingular  on even cohomology}

What does it mean to have a $\Frob$ eigenvalue $\alpha = \zeta q^{i/2}$. This happens exactly when $\Frob^n$ has an eigenvalue $(q^n)^{i/2}$. In other words an eigenvector with eigenvalue $\alpha = \zeta q^{i/2}$ is the same as a vector fixed under $\Frob^n / (q^n)^{i/2}$ for some $n$. By the Tate conjecture, these are classes should be algebraic cycles defined over $\FF_{q^n}$. Therefore, for $i$ even, the supersingular eigenspaces are exactly the set of ``potentially algebraic cycles'' meaning the cycles that are represented by cycle classes of varities defined over possibly lager fields. 

\section{Characters}

\newcommand{\lcm}{\mathrm{lcm}}

We are considering the projective variety $X$ defined by the polynomial,
\[ f = a_0 x_0^{n_0} + \cdots + a_r x_r^{n_r} \]
Let $m = \lcm(n_0, \dots, n_r)$ and denote by $\mu_n$ the group of $n^{\text{th}}$-roots of unity in $\FF_q$. Then there is an action of the group,
\[ \mu_{n_0} \times \cdots \times \mu_{n_r} \]
on $X$. However, the map,
\[ \mu_{n_0} \times \cdots \times \mu_{n_r} \to \Aut{X} \]
is not injective since $X$ is defined as the quotient under the action,
\[ \lambda \cdot (x_0, \dots, x_n) = (\lambda^{\frac{m}{n_0}} x_0, \dots, \lambda^{\frac{m}{n_r}} x_r) \] 
therefore the kernel of the map 
\[ \mu_{n_0} \times \cdots \times \mu_{n_r} \to \Aut{X} \]
is exactly the image of
\[ \mu_m \to \mu_{n_0} \times \cdots \times \mu_{n_r} \]
under the map 
\[ \lambda \mapsto (\lambda^{\frac{m}{n_0}}, \dots, \lambda^{\frac{m}{n_r}}) \]
Therefore, we get a map,
\[ G = (\mu_{n_0} \times \cdots \times \mu_{n_r})/\mu_m \to \Aut{X} \]
Since $G \acts X$ by functoriality it also acts on the middle cohomology,
\[ G \acts H^{r-1}_{\et}(X, \Q_\ell) \]
Then $G$ is abelian so its irreducible representations are all one-dimensional characters. Therefore, we get a decomposition into spaces on which $G$ acts through a given character,
\[ H^{r-1}_{\et}(X, \Q_\ell) = \bigoplus_{\chi \in \wh{G}} H^{r-1}_{\et}(X, \Q_\ell)(\chi) \]
Weil proved that in our case for each character $\chi$ we have,
\[ \dim H^{r-1}_{\et}(X, \Q_\ell)(\chi) \le 1 \]
Furthermore, since $G$ acts by automorphisms and the action of $\Frob$ is natural meaning that the action of $\Frob$ and $G$ commute. Therefore, $\Frob$ preserves the irreducible decomposition of $G$. Since each factor is $1$-dimensional, 
\[ \Frob \acts H^{r-1}_{\et}(X, \Q_\ell) \]
is just multiplication by a corresponding $\Frob$ eigenvalue $\alpha_{\chi}$. Furthermore, since if $[Z]$ is the class of a subvariety then $g \cdot [Z] = [g \cdot Z]$ so the action of $G$ preserves the algebraic cycles. Therefore,
\[ H_{\text{alg}}^{2i}(X, \Q_\ell) \subset H^{2i}_{\et}(X, \Q_\ell) \]
is a $G$-subrepresentation. Therefore, since each character space is $1$-dimensional then the space of algebraic cycles is a sum of a subset of the characters. These are exactly the ``algebraic characters''. By the Tate conjecture, they are also the ``supersingular characters'' i.e. those characters such that $\alpha_{\chi} = \zeta q^{i/2}$. 
\bigskip\\
Now we need to make the connection to the set $A_{\ul{n}, q^f}$. To do this, we fix compatible isomorphisms $\mu_n \cong \mu_n(\CC)$ for each $n$ dividing $q^f - 1$ (recall that $f = \ord_m(q)$. This just amounts to a choice of generator $g \in \FF_{q^f}^\times$ which we identify with $\zeta_{q^f-1} = e^{\frac{2 \pi i}{q^f - 1}}$. Now for each $i$ and a character,
\[ \chi : \mu_{n_i} \to \mu_{n_i}(\CC) \]
consider the map,
\[ \FF_{q^f}^\times \to \mu_{n_i} \xrightarrow{\chi} \mu_{n_i}(\CC) \]
where the map is,
\[ x \mapsto x^{\frac{q^f - 1}{n_i}} \mapsto \chi(x^{\frac{q^f - 1}{n_i}}) \]
This gives a map,
\[ \wh{G} \to \Hom{}{(\FF_{q^f}^\times)^{r+1}}{\CC^\times} \]
The compatible isomorphism then enters when we identify,
\[ \wh{G} = \{ (a_0, \dots, a_r) \mid a_i \in (\Z / n_i \Z) \text{ and } (m/n_0) a_0 + \cdots + (m/n_r) a_r \equiv 0 \mod m \} \]
By definition, the character corresponding to $a = 1$ is given by taking the generator of $\mu_{n_i}$ which is $g^{\frac{q^f - 1}{n_i}}$ and sending to the generator of $\mu_{n_i}(\CC)$ which is $\zeta_{n_i}$ hence the corresponding character of $\FF_{q^f}^\times$ is defined on the generator by
\[ g \mapsto \zeta_{n_i} \]
which corresponds to $\alpha_i = \frac{1}{n_i}$ as we defined previously. Therefore, this identification of $\wh{G}$ shows that its image in $\Hom{}{(\FF_{q^f}^\times)^{r+1}}{\CC^\times}$ is almost the set $A_{\ul{n},q^f}$. Notice we have ``explained'' where the sum condition comes from but not the conditional $0 < \alpha_i < 1$ i.e. corresponding to a condition that all $a_i \neq 0$. To do this let,
\[ \wh{G}^{\text{prim}} \subset \wh{G} \]
be the subset where $\chi$ is nontrivial when restricted to each $\mu_{n_i} \to G$. The the image of,
\[ \wh{G}^{\text{prim}} \to \Hom{}{(\FF_{p^f}^\times)^{r+1}}{\CC^\times} \]
is exactly the set $A_{\ul{n}, q^f}$. The reason geometrically for considering only primitive characters is, it turns out, 
\[ \dim H^{r-1}_{\et}(X, \Q_\ell)(\chi) = 1 \]
exactly for the primitive characters and is zero otherwise. 

\section{Potential Examples}

General references,
\begin{enumerate}
\item \chref{https://arxiv.org/pdf/1310.3402.pdf}{Beauville}

\item \chref{https://arxiv.org/pdf/1009.0388.pdf}{cuboids}

\item \chref{https://magma.maths.usyd.edu.au/magma/citations/pdf/topic/AlgSrf.pdf}{review of surfaces}
\end{enumerate}

\subsection{Desingularizing Branched covers}

\subsection{Hilbert Modular Surfaces}

Let $X$ be a desingularization of the Baliey-Borel compactification.

\begin{enumerate}
\item $\pi_1(X) = 0$ \chref{https://core.ac.uk/download/pdf/82371496.pdf}{Minimal Models}
\end{enumerate}

\subsection{Explicit Complete intersection}

We need a complete intersection $X = X_{d_1, d_2} \subset \P^4$ of degrees $d_1, d_2 \ge 9$ which is CM. This seems hard to compute.

\begin{enumerate}
\item $\pi_1(X) = 0$ by Lefschetz
\item $\Omega_X$ is big by results of Debarre I think or explicit calculation
\item $H^2$ CM is unknown
\end{enumerate}

Some references,
\begin{enumerate}
\item \chref{https://arxiv.org/pdf/1406.7848.pdf}{Brotbek explicit}

\item \chref{https://arxiv.org/abs/math/0306066}{Debarre}

\item \chref{https://www.jstor.org/stable/pdf/1971522.pdf}{weak torelli}

\item \chref{https://theses.hal.science/tel-00677065/document}{debarre thesis}
\end{enumerate}

\subsection{Special Fermat Complete intersections}

Consider a complete intersection of Fermat 3-folds in $\P^4$. This is dual to a complete intersection Fermat curve in $\P^4$ but of rather high genus.

\begin{enumerate}
\item $\pi_1(X) = 0$ by Lefschetz
\item $\Omega_X$ is big by results of Debarre I think or explicit calculation
\item $H^2$ CM is unknown
\end{enumerate}

References,
\begin{enumerate}
\item \chref{https://www.jstage.jst.go.jp/article/math1924/14/2/14_2_309/_pdf}{Terasoma}

\item \chref{https://projecteuclid.org/journals/tohoku-mathematical-journal/volume-44/issue-3/Unirationality-of-certain-complete-intersections-in-positive-characteristics/10.2748/tmj/1178227304.full}{some are unirational} (Does this work for only degree $q+1$ or for any degree such that $p^v \equiv -1 \mod d$).
\end{enumerate}

\subsection{Torodorov Surfaces}

\begin{enumerate}
\item \chref{http://www.numdam.org/item/10.24033/asens.1375.pdf}{TODOROV}

\item \chref{https://arxiv.org/pdf/1310.3402.pdf}{Beauville}

\item \chref{https://eudml.org/doc/142799}{counterexamples to Global Torelli}

\item \chref{http://www.numdam.org/item/CM_1982__45_3_293_0.pdf}{usui}
\end{enumerate}

\subsection{Moduli of Vector bundles}

Are these simply connected?

\begin{enumerate}
\item \chref{https://math.univ-cotedazur.fr/~beauvill/pubs/2Q.pdf}{beauville}

\item \chref{https://www.researchgate.net/publication/243105001_On_Moduli_Spaces_of_Parabolic_Vector_Bundles_of_Rank_2_over_CP1}{parabolic rank $2$ on $\P^1$}
\end{enumerate}

\section{Jets Referencs}

\begin{enumerate}
\item Two Applications of Algebraic Geometry to Entire Holomorphic
Mappings
MARK GREEN AND PHILLIP GRIFFITHS

\item \chref{http://eroussea.perso.math.cnrs.fr/ggbsd.pdf}{exceptional set}

\item \chref{https://www-fourier.ujf-grenoble.fr/~demailly/manuscripts/hyperbolic.pdf}{demailly}

\item \chref{https://arxiv.org/pdf/math/0001093.pdf}{log jet bundles}

\item \chref{https://arxiv.org/pdf/1801.04765.pdf}{green-griffiths-lang}

\item \chref{https://arxiv.org/pdf/1612.07847.pdf}{semple jets}
\end{enumerate}

\section{Relative BB}

References:

\begin{enumerate}
\item \chref{https://arxiv.org/pdf/2004.08261.pdf}{Campana}

\item \chref{https://arxiv.org/abs/2210.08767}{Higgs}

\item \chref{https://arxiv.org/pdf/2112.12448.pdf}{algebraic foliations}

\item \chref{https://arxiv.org/pdf/2202.01295.pdf}{schnell-singular metrics}

\item \chref{https://arxiv.org/pdf/2004.08261.pdf}{Campana - without tears}

\item \chref{https://link.springer.com/article/10.1007/s00222-018-00853-2}{Horing and Peternell - Algebraic integrability} 
\end{enumerate}


\section{Shioda}

We develop a novel obstruction to unirationality in positive characteristic and apply it to produce a counterexample to the Shioda conjecture. The construction of this obstruction is based on jet-bundle techniques.

\subsection{Introduction}

We say that a variety over $\FF_q$ is \textit{supersingular} if the Newton polygon of $\Frob_q \acts H^i(X, \Q_\ell)$ has a single slope for each $i$. 

(CHECK EQUIVALENCE)

\newcommand{\tdF}{\mathrm{tdF}}

If $X$ is smooth and projective, then supersingularity is equivalent to the eigenvalues of $\Frob_q \acts H^i(X, \Q_\ell)$ are all of the form $\zeta q^{i/2}$ where $\zeta$ is a root of unity. Sawin and [OTHER AUTHOR] (CITE) introduced a cohomological birational invariant $H^i_{\tdF}$ based on divisibility of Frobenius eigenvalues in the ring of integers. If $X$ is a smooth projective simply-connected surface then $X$ is supersingular if and only if $H^2_{\tdF}(X) = 0$. 

In 1973 Shioda made the following conjecture (CITE)

\begin{conj}[Shioda]
Let $X$ be a smooth projective surface over $\overline{\FF}_p$ with $\pi_1^{\et}(X) = 1$. Then $X$ is supersingular if and only if $X$ is unirational.
\end{conj} 

He remarked that ``if false, this conjecture would be very difficult to verfiy'' (FIND QUOTE). His reasoning was presumably based on the dearth of known obstructions to unirationality since the setup of his conjecture ensures that the only as-yet known obstructions, $\pi_1^{\et}$ and the Galois representation of $H^2(X, \Q_\ell)$ do not suffice. To provide a counterexample therefore, we must develop a novel obstruction.

\subsection{Jet Bundles}

Jet bundles capture higher-order differential information. There are unfortunately two dual notions which both go under the title ``jets''. Our jet bundles are jets of maps $X \to \A^1$ -- known in EGA as ``bundles of principal parts'' because they can be interpreted as symbols for differential operators  -- while the ``jets'' more commonly studied in the complex geometry literature are jets of maps $\A^1 \to X$. Here we take the former notion. 

\begin{defn}
Let $X /S$ be a separated $S$-scheme. We denote by $\Delta^n_X \embed X \times_S X$ the $n$-th order thickening of the diagonal. If $\I_\Delta \subset \struct{X \times_S X}$ is the ideal corresponding to the closed embedding $\Delta_{X/S} : X \embed X \times_S X$ then $\Delta^n$ is defined by $\I^{n+1}$. The closed subscheme $\Delta^n_X \subset X \times_S X$ is then equipped with projection maps $\pi_i : \Delta^n_X \to X$ for $i = 1,2$.
\end{defn}

\begin{defn}
Let $X / S$ be a smooth separated scheme. Let $\E$ be a vector bundle on $X$. Then the $n$-th \textit{jet bundle} of $\E$ is defined as the $\struct{X}$-module,
\[ J^n(\E) := \pi_{1*} \pi_2^* \E = (\struct{X \times_S X} / \I_\Delta^{n+1}) \ot_{\struct{X}} \E \]
using the projections $\pi_i : \Delta^n_X \to X$. We write $J^n(X) = J^n(\struct{X})$.
\end{defn}

\begin{prop}
There are exact sequences,
\begin{center}
\begin{tikzcd}
0 \arrow[r] & \nSym{n}{\Omega_X} \ot \E \arrow[r] & J^n(\E) \arrow[r] & J^{n-1}(\E) \arrow[r] & 0
\end{tikzcd}
\end{center}
\end{prop}

\begin{proof}
DO IT.
\end{proof}

\begin{prop}
Let $f : X \to Y$ be morphism of smooth varities and $\E$ a vector bundle on $Y$. Then there is a pullback map,
\[ f^* J^n(\E) \to J^n(f^* \E) \]
compatible with the pullback on $\Omega_X$ and the projection maps.
\end{prop}

\begin{proof}
This follows immediately from the map $\Delta_X^n \to \Delta_Y^n$ which induces the natural pullback $f^* \nSym{n}{\Omega_Y} \to \nSym{n}{\Omega_X}$ DO THIS!!
\end{proof}

Finally, we show that sections of jet bundles are birational invariants in the category of smooth projective varities as are symmetric differentials.

\begin{prop}
Let $f : X \rat Y$ be a rational map of smooth projective varieties. Then there are functorial pullback maps on global sections,
\[ H^0(Y, J^m(Y)) \to H^0(X, J^m(X)) \]
comptatible with the pullback on global symmetric differentials and the projection maps.
\end{prop}

\begin{proof}
Let $U \subset X$ be the domain of $f$. Since $X$ is smooth and $Y$ is proper then $\codim{U^C,X} \ge 2$. Since $J^m(X)$ is a vector bundle we then get by Harthog,
\[ H^0(Y, J^m(Y)) \to H^0(U, J^m(U)) = H^0(X, J^m(X)) \]
This is also how the pullback on $\nSym{n}{\Omega_Y}$ is defined and is compatible with projections because the pullback along $U \to Y$ is. 
\end{proof}


\subsection{The Obstruction}

The obstruction is based on a higher-order generalization of the observation that a unirational variety in characteristic zero cannot carry any nonzero symmetric differentials.

\begin{thm}
Suppose that $\omega \in H^0(X, \nSym{r}{\Omega_X})$ is a symmetric differential form that admits a lift to some section,
\[ \wt{\omega} \in H^0(X, J^{p^h r}(X)) \]
meaning $\wt{\omega} \mapsto \omega$ under the projection $J^{p^h r}(X) \to J^r(X)$ and the inclusion $\nSym{r}{\Omega_X} \embed J^r(X)$. Then there does not exist a dominant purely-inseparable map $\P^n \rat X$ of height $h$.
\end{thm} 


\section{Every Abelian variety over a finite field has "CM"}

This is because $\End{A}$ is always bigger than $\Z$ since there is relative $\Frob_q : A \to A$ if $A$ is defined over $\FF_q$. However, this does not work if $A$ is defined over an infinite field $k$. This is because we have a relative Frobenius $\Frob_q : A \to A^{(q)}$ but these are not isomorphic $k$-schemes unless $A$ is defined over $\FF_q$.

\section{Cartier Operator}

Setting: let $X \to S$ be a morphism of schemes of characteristic $p$. Then there is absolute Frobenius $F_S : S \to S$ which is identity on the underlying spaces and $a \mapsto a^p$ as a map of sheaves $\struct{S} \to \struct{S}$. Consider the diagram,
\begin{center}
\begin{tikzcd}
X \arrow[rd] \arrow[r, dashed] \arrow[rr, bend left, "F_X"] & X' \arrow[d] \pullback \arrow[r] & X \arrow[d]
\\
& S \arrow[r, "F_S"] & S
\end{tikzcd}
\end{center}
then the dashed map is called the relative Frobenius $\Frob_{X/S} : X \to X'$. Then $\Frob_{X/S}$ is finite if $X \to S$ is lft and is finite locally free if $X \to S$ is smooth. 

\begin{example}
Let $X = \A^n$ and $S = \Spec{k}$ then $\Frob_{X/S} : x_i \mapsto x_i^p$ is a $k$-linear map while $F_X : f \mapsto f^p$ is not $k$-linear since $a f \mapsto a^p f^p$. 
\end{example}

\begin{theorem}[Cartier]
There exists a unique morphism of graded $\struct{X'}$-algebras.
\[ C^{-1} : \bigoplus_{i \ge 0} \Omega_{X'/S}^i \to \bigoplus_{i \ge 0} \cH^i(F_* \Omega^\bullet_{X/S}) \]
such that,
\begin{enumerate}
\item $C^{-1}(1) = 1$
\item $C^{-1}(\alpha \wedge \beta) = C^{-1}(\alpha) \wedge C^{-1}(\beta)$
\item $C^{-1}(1 \ot \d{f}) = [f^{p-1} \d{f}]$ where $1 \ot \d{f}$ denotes the pullback of $\d{f} \in \Gamma(X', \Omega_{X'/S}^1)$ along $ U : X' \to X$
\end{enumerate}
Furthermore, if $\pi : X \to S$ is smooth then $C^{-1}$ is an isomorphism. 
\end{theorem}

\begin{proof}
It suffices to construct the degree $1$ part of $C^{-1}$ via the universal property of exterior algebras. To get,
\[ C^{-1} : \Omega^1_{X'/S} \to \cH^1(F_* \Omega^\bullet_{X/S}) \]
Suffices to construct,
\[ \Omega_{X/S}^1 \to u_* \Omega_{X'/S}^1 \to \cH^1(F_* \Omega_X^\bullet) \]
which is the same data as a derivation,
\[ \delta : \struct{X} \to \cH^1(F_* \Omega_{X/S}^\bullet) \]
meaning,
\[ \delta(fg) = f^p \delta(g) + g^p \delta(f) \]
Try $\delta(f) = [f^{p-1} \d{f}]$. It satisfies Leibnitz law and additivity since,
\[ \delta(f+g) - \delta(f) - \delta(g) = \left[ \d \left( \frac{(f+g)^p - f^p - g^p}{p} \right) \right] = 0 \]
This takes care of existence and also uniqueness. Then for smoothness we reduce to $\A^n_S$ and thus to $\A^1_{k}$. Then we set $X = \Spec{k[x]}$ and $X' = \Spec{k[y]}$. Now $F_* \Omega^1_{X/S}$ is the complex,
\[ 0 \to k[x^p] \left< 1, x, \dots, x^{p-1} \right> \to k[x^p] \left< \d{x}, x \d{x}, \dots, x^{p-1} \d{x} \right> \to 0 \]
Therefore it has,
\[ H^0 = \ker{\d} = k[x^p] \cdot 1 \quad \text{and} \quad H^1 = \coker{\d} = k[x^p] x^{p-1} \d{x} \]
Therefore, we see explicitly that $C^{-1}$ is an isomorphism. Then we use Kunneth to recover the statement for $\A^n_S$ and then for $\pi : X \to S$ smooth it is \etale locally the projection from affine space so we win.
\end{proof}

\section{Curve Degenerations}

Use Theorem 9.3.7 to show if special fiber is geometrically reduced then we win.

\section{Kawamata Result}

\begin{defn}
Let $f : V \to W$ be a fiber space (surjective with connected geometric generic fiber) between nonsingular projective varities over $\CC$. This is the same as saying $f_* \struct{V} = \struct{W}$. Then $\Var(f)$ is the minimal transcendence degree over $\CC$ of a subfield $L \subset \ol{\CC(W)}$ such that there is a variety $F$ defined over $L$ so that $F \times_{\Spec{L}} \Spec{\ol{\CC(W)}}$ is birationally equivalent to $V_{\bar{\eta}}$.
\end{defn}

\begin{defn}
Let $f : X \to Y$ be a map between algebraic varities whose fibers are reduced and irreducible. Let $p_i : Y \times Y \to Y$ be the coordinate projections. Define $Br E(f, X, Y)$ to be the set of points $z \in Y \times Y$ such that $z \times_{p_1} X$ and $z \times_{p_2} X$ (the fibers over the two points in the pair $z$) are birational over $k(z)$.
\end{defn}

\begin{prop}
If $f : X \to Y$ is projective then $Br E(f, X, Y)$ is a pro-algebraic equivalence relation. If $f$ is smooth and none of the fibers are birationally, then $Br E(f, X, Y)$ is closed. 
\end{prop}

\begin{theorem}
Let $E \subset Y \times Y$ be a closed pro-algebraic equivalence relation. Then there exists an open set $Y_0 \subset Y$ and a surjective map $g : Y_0 \to Z$ with connected fibers such that for $E_0 = E \cap (Y_0 \times Y_0)$ the following hold,
\begin{enumerate}
\item any equivalence class of $E_0$ is a union of fibers of $g$
\item there are countably many proper closed subvarities $Z_i \subset Z$
such that if $u \in Y_0$ and $g(u) \notin \bigcup Z_i$ then the equivalence class of $E_0$ that contains $u$ is a countable union of fibers of $g$.
\end{enumerate}
Furthermore, $g$ is unique viewed as a rational map of $Y$. 
\end{theorem}

\begin{prop}
Let $f : X \to Y$ be a smooth projective map and assume that none of the fibers are ruled. Let $E = Br E(f, X,Y)$ and let $g$ and $Z$ be as in above theorem. Then,
\begin{enumerate}
\item $\ol{k(Z)}$ or more precisely $g^* \ol{k(Z)} \subset \ol{k(Y)}$ is called the minimal closed field of definition of $X/Y$ (or of its generic fiber).
\item $\Var(f) = \dim{Z}$. 
\end{enumerate}
\end{prop}

\begin{prop}
Let $X$ be a normal projective varitey having only canonical singularities, and let $\alpha_X : X \to A = \Alb_X$ be the Albanese morphism. Assume that $m K_X \sim 0$ for some positive integer $m$. Then $\alpha_X$ is an \etale fiber bundle, i.e., there is an \etale covering $\pi : B \to A$ such that,
\[ X \times_A B \cong F \times B \]
for some projective variety $F$.
\end{prop}

\begin{proof}
By [KAWAMATA] Theorem 1, $\alpha_X$ is an algebraic fiber space. By Theorem 1.1, $\alpha_X$ has zero variation. Let $\mu : Y \to X$ be an equivariant (WHAT DOES THIS MEAN) resolution, as before. Then by (HOW DO THESE LEMMAS APPLY) there is a nonempty open $U \subset A$ and an \etale covering $U' \to U$ such that,
\[ X \times_U U' \cong F \times U' \quad \text{and} \quad Y \times_U U' \cong F' \times U' \]
for some projective varities $F$ and $F'$. This should just be from the fact that variation is zero so over some locus the fibers are isomorphic. 
\bigskip\\
Then by Viehweg, Lemma 9.5-(ii), there is an \etale covering $\pi : B \to A$ such that $Y \times_A B$ is birationally equivalent to $F' \times B$ over $B$. We note that we used there the following two facts
\end{proof}

\begin{lemma}[Viehweg Lemma 9.5]
Let $V$ be a nonsingular projective variety with $\kappa(V) = 0$ and $A$ an abelian variety and $f : V \to A$ a fibre spae. Assume one of the following:
\begin{enumerate}
\item $\Var{f} = 0$ and the non-singular fibres of $f$ have a minimal model
\item There exists an open subvariety $U$ of $A$ and an \etale covering $U'$ of $U$, such that $U' \times_A V \cong U' \times F$ for some variety $F$.
\end{enumerate}
Then there exists a variety $V'$ birational to $V$ such that the induced map $f' : V' \to A$ is an \etale fibre bundle with fiber $F$.
\end{lemma}

\begin{proof}
First show that (i) $\implies$ (ii).
\bigskip\\
Let's assume (ii). Let $W'$ be a nonsingular compactification of $U'$. In the case
\end{proof}

\section{Clemens Proof}

\newcommand{\cN}{\mathcal{N}}

\begin{defn}
An immersed curve $f : C \to X$ is a map from a smooth curve which is everywhere maximal rank menaing the map is unramified. Associated to such a mapping is the conormal bundle $\C_f := \ker{(f^* \Omega_X \to \Omega_C)}$ and the normal bundle $\cN_f = f^* \T_X / \T_C$ which is also a bundle by the assumptions.
\end{defn}

\begin{theorem}
Let $X$ be a generic hypersurface of degree $m$ in $\P^n$> Then $X$ sodes not admit an irreducible family of immersed curves of genus $g$ and degree $d$ which cover a variety of codimension $< D$ where,
\[ D = \frac{2 - 2 g}{d} + m - (n+1) \]
\end{theorem}

\begin{cor}
Therefore, if $(n-1) - D \le 0$ i.e.
\[ m - 2 n \ge \frac{2g - 2}{d} \]
then there do not exist any curves of genus $g$ and degree $d$ on $X$.
\end{cor}

\subsection{Semipositivity of the Normal Bundle}

\newcommand{\Gr}{\mathbf{Gr}}

\begin{defn}
Let $C$ be a complete nonsingular curve and $\E$ a vector bundle. We say that $\E$ is \textit{semipositive} if every quotient bundle of $\E$ has non-negative degree.
\end{defn}

\begin{lemma}
Suppose that $\E$ is a flat family of vector bundles on $C$ (meaning a vector bundle on $C \times T$ flat over $T$). Let $x, y \in T$ with $x \spto y$ then
\[ \E_y \text{ is semipositive} \implies \E_x \text{ is semipositive} \]
\end{lemma}

\begin{proof}
Suppose not, then there exists $\E_x \onto \G$ of rank $s$ and $\deg{\G} < 0$. Chosoe a dvr $\Spec{R} \to T$ hitting $x \spto y$ so we reduce to $C_R$. The quotient defines $C_K \to \Gr_{C_R}(\E, s)$ which we want to extend to $C_R$. Indeed, this extends over codimension $1$ so it intersects $C_\kappa$ and then this extends to a morphism $C_\kappa \to \Gr_{C_R}(\E, s)$ hence we get a quotient $\E_y \onto \G'$ and $\deg{\G'} \le \deg{\G} < 0$ giving a contradiction. Indeed, note that $\deg{\G}$ is the degree of the map $C_K \to \Gr_{C_K}(\E, s)$ wrt the universal determinant bundle so we apply the following lemma. In order to make the universal determinant ample, we twist by a constant ample on $C$ which just shifts all degrees to be positive.
\end{proof}

\begin{lemma}
Let $C$ be a smooth curve and $R$ a dvr. Let $X \to \Spec{R}$ be proper flat with an ample line bundle $\L$ on $X$. Suppose that $\varphi_K : C_K \to X_K$ is a map over $R$ then this extends to a map $\varphi_U  : U \to X$ on some open $U \subset C_R$ of codimension at least $2$ hence we get a map $\varphi_\kappa : C_\kappa \to X_\kappa$ called the specialization. Then $\deg{\varphi_K^* \L_K} \ge \deg{\varphi_\kappa^* \L_\kappa}$.
\end{lemma}

\begin{proof}
Since $\deg{\L^{\ot n}} = n \deg{\L}$ we can replace $\L$ with a power such that it is very ample and hence we can replace $X$ by $\P^n_R$ and set $\L = \struct{\P^n}(1)$. (HOW TO FINISH)
\end{proof}

\begin{lemma}
If the global sections of $\E$ span the fiber at some point $p \in C$ then $\E$ is semi-positive.
\end{lemma}

\begin{proof}
Indeed, if $\E \onto \G$ then $\wedge^{\rank{\G}} \E \to \det{\G}$ is surjective so since there is a nonzero section of $\wedge^{\rank{\G}}{\E}$ there is also a nonzero section of $\det{\G}$ hence $\deg{\G} \ge 0$.
\end{proof}

\begin{lemma}
Consider an exact sequence,
\begin{center}
\begin{tikzcd}
0 \arrow[r] & \E_1 \arrow[r] & \E_2 \arrow[r] & \E_3 \arrow[r] & 0
\end{tikzcd}
\end{center}
if $\E_1$ and $\E_3$ are semi-positive then $\E_2$ is semi-positive.
\end{lemma}

\begin{proof}
Suppose not and take $T = \ker{(\E_2 \to \G)}$ which falsifies the semi-positivity. Let $S$ be the minimal subbundle of $\E_2$ containing $T$ and $\E_1$. Consider the map,
\[ \eta : T \oplus \E_1 \to S \]
Then there exists a subbundle $K$ of $T \oplus \E_1$ such that, for almost all $p \in C$, the mapping $\eta$ gives an injection,
\[ ((T \oplus \E_1)/K)_p \to S_p \]
Since $K$ is a subbundle of $\E_1$ we have $\deg{K} \le \deg{\E_1}$ and hence, 
\[ \deg{((T \oplus \E_1)/K)} \ge \deg{T} \]
Therefore $\deg{S} \ge \deg{T}$ (WHY????). Thus $\deg{\E_2/S} < 0$ because $\deg{\E_2} - \deg{T} = \deg{\E_2/T} < 0$ so $\deg{\E_2 / S} = \deg{\E_2} - \deg{S} < 0$. However, $S$ contains $\E_1$ so $\E_2 / S$ is a quotient of $\E_3$ contradicting the semipositivity of $\E_3$. 
\end{proof}


Let $X$ be a smooth hypersurface of degree $m$ in $\P^n$ and let $f : C \to X$ be an immersion of degree $d$. Let $W$ be a generically chosen hypersurface of degree $m$ in $\P^{n+m}$ such that $\P^n \cap W = X$. 

\begin{lemma}
The normal bundle $N_{f}$ to the mapping,
\[ f : C \to X \subset W \]
is semi-positive.
\end{lemma}

\begin{proof}
Since we assume throughout that $m \ge 2$, we can specialize $W$ to a hypersurface $W'$ of degree $m$ in $\P^{n+m}$ which contains $\P^n$ and is non-singular at point of $f(C)$. By the specialization lemma, it suffices to prove the assertion of the lemma for $f : C \to W$ where $W$ is generic such that it contains the $\P^n$. There is a sequence of normal bundles,
\[ 0 \to N_{f, \P^n} \to N_{f, W} \to f^* N_{\P^n, W} \to 0 \]
and the fact that $N_{f, \P^n}$ is semi-positive since it is globally generated (since $\T_{\P^n}$ is globally generated) thus we just need to find some $W$ such that $f^* N_{\P^n, W}$ is semi-positive. Consider the sequence,
\[ 0 \to f^* N_{\P^n, W} \to f^* N_{\P^n, \P^{n+m}} \to f^* M_{W, \P^{n+m}} \to 0 \]
If we can find some special $W$ for which,
\[ f^* N_{\P^n, W} \cong \struct{C}^{\oplus (m-1)} \]
then the proof will be complete since any quotient of this is, by definition, globally generated hence will be semi-positive.
\bigskip\\
We do this by direct computation. (DO THIS!!)
\end{proof}

\subsection{Proof of the Main Theorem}

Let $X$ be a generic hypersurface of degree $m$ in $\P^n$ and we suppose that there is an irreducible algebraic family

\section{Support of a Complex}

\begin{defn}
Let $\F$ be an abelian sheaf on a space $X$. Then we define the support as
\[ \Supp{X}{\F} = \{ x \in X \mid \F_x \neq 0 \} \]
\end{defn}

\begin{prop}
Let $f : X \to Y$ be a continuous map and $\F$ an abelian sheaf on $Y$. Then
\[ \Supp{X}{f^{-1} \F} = f^{-1}(\Supp{Y}{\F}) \]
\end{prop}

\begin{proof}
Because $(f^{-1} \F)_x = \F_{f(x)}$ this is immediate. However, the similar-looking proposition for modules over a sheaf of rings will be more difficult because of the tensor product.
\end{proof}

\begin{prop}
Let $X$ be a scheme and $\F$ a finite-type quasi-coherent $\struct{X}$-module then
\[ \Supp{X}{\F} = V(\shAnn{\struct{X}}{\F}) \]
and hence $\Supp{\struct{X}}{\F}$ is closed with the natural structure of a closed subscheme.
\end{prop}

\begin{proof}
This is local on $X$ so we may assume that $X = \Spec{A}$ and $\F = \wt{M}$ with $M$ finite. Then it suffices to show that
\[ \Supp{A}{M} = V(\Ann{A}{M}) \]
Indeed, suppose that $\p \not\supset \Ann{A}{M}$ then there is some $s \in \Ann{A}{M} \setminus \p$ and hence $s M = 0$ but $s \in (A \setminus \p)$ so $M_\p = (A \setminus \p)^{-1} M = 0$. Conversely, if $M_\p = 0$ then for any generating set $m_1, \dots, m_r \in M$ there exist $s_1, \dots, s_r \in A \setminus \p$ such that $s_i m_i = 0$ in $M$. Hence $s = s_1 \cdots s_r \in \Ann{A}{M}$ but $s_i \notin \p$ so $s \notin \p$ since $\p$ is prime proving the claim.
\end{proof}

\begin{example}
Finiteness is essential otherwise the support may not be cloed. For example,
\[ M = \bigoplus_{p \neq 2} \Z / p \Z \]
satisfies $\Supp{\Z}{M} = \Spec{\Z} \setminus \{ 2 \}$.
\end{example}

\begin{prop}
Let $A \to B$ be a local ring homomorphism and $M$ a finite $A$-module. Then $M = 0$ if and only if $M \ot_A B = 0$.
\end{prop}

\begin{proof}
Indeed, 
\[ M \ot_A \kappa_A \ot_{\kappa_A} \kappa_B = M \ot_A \kappa_B = M \ot_A B \ot_B \kappa_B = 0 \]
but $\kappa_A \embed \kappa_B$ is faithfully flat hence $M \ot_A \kappa_A = 0$ so by Nakayama $M = 0$ since $M$ is finite.
\end{proof}

\begin{example}
$\Z_{(p)} \to \Z / p$ is a local ring homomorphism but $\Q \ot_{\Z_{(p)}} \Z / p = 0$ so the finitness of $M$ is essential.
\end{example}

\begin{prop}
Let $f : X \to Y$ be a map of locally ringed spaces and $\F$ is a finitely-generated $\struct{Y}$-module then $\Supp{X}{f^* \F} = f^{-1}(\Supp{Y}{\F})$.
\end{prop}

\begin{proof}
We need to show that $(f^* \F)_x = 0$ if and only if $\F_{f(x)} = 0$ but $(f^* \F)_x = \F_x \ot_{\stalk{Y}{f(x)}} \stalk{X}{x}$ and $\stalk{Y}{f(x)} \to \stalk{X}{x}$ is a local ring homomorphism and $\F_x$ is a finite $\stalk{X}{x}$-module so we apply the previous lemma.
\end{proof}

\begin{defn}
Let $X$ be a topological space and $C^\bullet$ a complex of abelian sheaves on $X$ then we define,
\[ \Supp{X}{C^\bullet} = \bigcup_n \Supp{X}{\cH^n(C^\bullet)} \]
\end{defn}

\begin{rmk}
Let $X$ be a scheme and $C^\bullet \in D^\flat(\Coh{X})$ then $\Supp{X}{C^\bullet}$ is well-defined and closed since $C^\bullet$ is represented by a bounded (so the union is finite) complex of coherent sheaves (so the homology sheaves are coherent hence the supports are closed).
\end{rmk}

\begin{prop}
Let $X$ be a ringed space and $C^\bullet \in D(\Mod{\struct{X}})$. Then,
\[ \Supp{X}{C^\bullet} = \{ x \in X \mid C^\bullet|_x \in D(\Ab) \text{ is nonzero } \} \]
\end{prop}

\begin{rmk}
If $C^\bullet \in D^\flat(\Coh{X})$ then we get a stalk restriction $C^\bullet|_x \in D^\flat(\stalk{X}{x})$ and this is the complex we need to check is nonzero in the derived category.
\end{rmk}

\begin{proof}
Recall that $C^\bullet$ is zero in the derived category iff it is exact. Indeed, consider the map $C^\bullet \to 0$ which is a quasi-isomorphism if and only if $C^\bullet$ is exact. Since talking stalks is exact, we see that $\cH^n(C^\bullet)_x = H^n(C^\bullet|_x)$ and therefore $C^\bullet|_x$ is exact if and only if $x \notin \Supp{X}{\cH^i(C^\bullet)}$ for all $n$.
\end{proof}

\begin{rmk}
It is not true that $\Supp{}{f^* C^\bullet} \supset f^{-1}(\Supp{}{C^\bullet})$ or $\Supp{}{f^* C^\bullet} \subset f^{-1}(\Supp{}{C^\bullet})$ in general even for $C^\bullet \in D^\flat(\Coh{Y})$. For example, on $\A^1$ the complex,
\[ [0 \to \struct{} \xrightarrow{x} \struct{} \to \struct{0} \to 0] \]
is exact so it has trivial support but restriction to $\{ 0 \}$ gives the complex,
\[ [0 \to \struct{0} \xrightarrow{0} \struct{0} \to \struct{0} \to 0] \]
which is not exact so it has support $\{ 0 \}$ giving a counterexample to the second. For the first, consider the complex,
\[ [ 0 \to \struct{} \to \struct{0} \to 0] \]
which has kernel $x \struct{}$ which has support $\A^1$ but its restriction to $0$ is,
\[ [0 \to \struct{0} \to \struct{0} \to 0] \]
is trivial in the derived category so it has empty support. We need to consider $L f^*$ to make such formulas work but we will be able to salvage this statement for complexes of locally free modules.
\end{rmk}

\newcommand{\LL}{\mathbb{L}}

\begin{lemma}
Let $f : X \to Y$ be a morphism of schemes and $C^\bullet \in D^\flat(\Coh{Y})$,
\[ \Supp{X}{\LL f^* C^\bullet} = f^{-1}(\Supp{Y}{\C^\bullet}) \]
\end{lemma}

\begin{proof}
Indeed, $(\LL f^* C^\bullet)|_{x} = C^\bullet|_{f(x)} \ot^{\LL}_{\stalk{Y}{f(x)}} \stalk{X}{x}$ so if $C^\bullet|_{f(x)} = 0$ then $(\LL f^* C^\bullet)|_x = 0$. Conversely, we first replace $C^\bullet$ by a finite flat resolution and apply the following lemma.
\end{proof}

\begin{rmk}
First, note the following easier fact. Let $C^\bullet$ be an exact bounded above complex of flat $A$-modules. Then $C^\bullet \ot_A M$ is also exact for any $A$-module $M$. This is implicit in the definition of $\LL f^* C^\bullet$ since if $C^\bullet = 0$ in the derived category then we must have $\LL f^* C^\bullet = 0$. Indeed, we show that all kernels and images are flat by induction on sequences of the form,
\[ 0 \to K^n \to C^n \to C^{n+1} \to 0 \]
to get that $K^n$ is flat and then,
\[ 0 \to K^i \to C^i \to K^{i+1} \to 0 \]
implies that $K^i$ is flat assuming $K^{i+1}$ is flat. 
\end{rmk}

\begin{lemma}
Let $A \to B$ be a local ring homomorphism and $C^\bullet$ be a bounded above complex of finitely presented flat\footnote{This is the same as finite projective or finite locally free.} $A$-modules. Then $C^\bullet$ is exact if and only if $C^\bullet \ot_A B$ is exact. 
\end{lemma}

\begin{proof}
Since $C^\bullet \ot_A B$ is an exact complex of flat $B$-modules by the fact we just discussed $C^\bullet \ot_A B \ot_B (B / \m_B)$ is also exact. But $A/\m_A \to B/\m_B$ is faithfully flat so we see that $C^\bullet \ot_A (A /\m_A)$ is also exact since it becomes exact after applying $- \ot_{A/\m_A} B / \m_B$. Therefore, we apply the following lemma.
\end{proof}

\begin{lemma}
Let $A$ be a local ring and $C^\bullet$ be a bounded above complex of finitely presented flat $A$-modules. Then $C^\bullet$ is exact if and only if $C^\bullet \ot_A (A/\m_A)$ is exact. 
\end{lemma}

\begin{proof}
Let $\kappa = A / \m_A$ and assume that $C^\bullet \ot_A \kappa$ is exact.
Consider the upper bound of the complex,
\[ C^n \xrightarrow{\varphi^n} C^{n+1} \to \coker{\varphi^n} \to 0 \]
and $\varphi \ot \id_\kappa$ is surjective so because $\coker{\varphi^n}$ is finite by Nakayama $\coker{\varphi^n} = 0$. Thus $\ker{\varphi^n} = 0$ is flat so its formation commutes with base change. Furthermore, since $C^n$ and $C^{n-1}$ are projective $\ker{\varphi^n}$ is finite projective and hence finitely presented. Next consider,
\[ \cdots \to C^{n-1} \xrightarrow{\varphi^{n-1}} \ker{\varphi^n} \to 0 \]
The last map is surjective after applying $- \ot_A \kappa$ since the formation of $\ker{\varphi^n}$ commutes with base change and $C^\bullet \ot_A \kappa$ is exact. Since $\ker{\varphi^n}$ is finitely presented then by Nakayama we conclude that $\im{\varphi^{n-1}} = \ker{\varphi^n}$ and are finite flat and thus $\ker{\varphi^{n-1}}$ is projective hence finitely presented flat. Thus we continue by induction to show that $C^\bullet$ is exact.
\end{proof}
\end{document}