\documentclass[12pt]{article}
\usepackage{hyperref}
\hypersetup{
    colorlinks=true,
    linkcolor=blue,
    filecolor=magenta,      
    urlcolor=blue,
}

\usepackage{import}
\import{"../Algebraic Geometry/"}{AlgGeoCommands}

\newcommand{\Loc}[1]{\mathfrak{Loc}\left( #1 \right)}
\newcommand{\AbGrp}{\mathbf{AbGrp}}

\renewcommand{\tr}{\operatorname{tr}}

\newcommand{\LL}{\mathbb{L}}
\newcommand{\ob}{\mathrm{ob}}
\newcommand{\cM}{\mathcal{M}}
\newcommand{\cT}{\mathcal{T}}
\newcommand{\vir}{\mathrm{vir}}

\begin{document}

\section{Counting Elliptic Curves on 3-folds}

\subsection{Hartshorne-Serre Correspondence}

Let $k$ be an algebraically closed field.

\begin{rmk}
If $R \onto A$ is a quotient of rings with $R$ regular and $A$ local complete intersection then $\ker{(R \to A)}$ is generated by a regular sequence (Tag \chref{https://stacks.math.columbia.edu/tag/00S8}{00S8}. Therefore, for a subscheme $Y \embed X$ of a regular scheme the following are equivalent,
\begin{enumerate}
\item $Y$ is a local complete intersection scheme
\item $Y \embed X$ is a regular embedding
\end{enumerate} 
Hence we will say in this situation that $Y \embed X$ is a ``local complete intersection subscheme''.
\end{rmk}

\begin{theorem}[\chref{https://arxiv.org/pdf/math/0610015.pdf}{Arrondo, Thm 1}]
Let $X$ be a smooth algebraic variety and $Y \subset X$ a local complete intersection subscheme of codimension $2$ in $X$. Let $N_{Y|X}$ be the normal bundle of $Y$ in $X$ and let $L \in \Pic{X}$ be a line bundle such that,
\begin{enumerate}
\item $H^2(X, L^\vee) = 0$
\item $\bigwedge^2 N \ot L|_Y^{\vee}$ is generated by $r-1$ global sections $s_1, \dots, s_{r-1}$.
\end{enumerate}
Then there exists a rank $r$ vector bundle $E$ on $X$ such that,
\begin{enumerate}
\item $\bigwedge^r E \cong L$
\item $E$ has $r-1$ global sections $\alpha_1, \dots, \alpha_{r-1}$ whose dependency locus in $Y$ and such that,
\[ s_1 \alpha_1 |_Y + \cdots + s_{r-1} \alpha_{r-1} |_Y = 0 \]
\end{enumerate}
Moreover, if $H^1(X, L^\vee) = 0$ then there is a unique such $E$ up to isomorphism in the sense that for any two tuples $(E, \alpha_1, \dots, \alpha_{r-1})$ and $(E', \alpha_1', \dots, \alpha_{r-1}')$ satisfying the above conditions, there exists a (not necessarily unique) isomorphism,
\[ \psi : E \iso E' \]
such that $\alpha'_i = \psi \circ \alpha_i$ for each $0 \le i \le r-1$.
\end{theorem}


\begin{defn}
A subvariety $Y \embed X$ is \textit{subcanonical} if,
\begin{enumerate}
\item $Y \embed X$ is a local complete intersection subscheme
\item $\det{N_{Y|X}}$ is in the image of $\Pic{X} \to \Pic{Y}$.
\end{enumerate} 
\end{defn}

\begin{rmk}
We call the above sitution ``subcanonical'' because if $Y$ has a canonical sheaf $\omega_Y$ then $\omega_Y$ extends to a line bundle on $X$. 
\end{rmk}

\begin{cor}
Let $X$ be a smooth algebraic variety and $Y \subset X$ a local complete intersection subscheme of codimension $2$ in $X$. Let $N_{Y|X}$ be the normal bundle of $Y$ in $X$ and let $L \in \Pic{X}$ be a line bundle such that,
\begin{enumerate}
\item $H^2(X, L^\vee) = 0$
\item $L|_Y \cong \bigwedge^2 N_{Y|X}$
\end{enumerate}
Then there exists a rank $2$ vector bundle $E$ on $X$ such that,
\begin{enumerate}
\item $\bigwedge^2 E \cong L$
\item $E$ has a global section $\alpha$ such that $Y = V(\alpha)$.
\end{enumerate}
Moreover, if $H^1(X, L^\vee) = 0$ then there is a unique such $E$ up to isomorphism giving a correspondence,
\[ \{ C \embed X \mid \text{embedded smooth genus } 1 \} \iso \{ (E, \alpha) \mid \det{E} = \omega_X \text{ and } V(\alpha) \text{ a smooth curve } \} / \cong \]
\end{cor}

\subsection{The Case of Elliptic Curves}

Suppose that $C \embed X$ is an embedded smooth genus $1$ curve in a regular $3$-fold $X$. Then $\omega_C \cong \struct{C}$ and hence $\det{N_{C|X}} = \det{\T_X|_Y} = \omega_X^\vee|_Y$. Hence set $L = \omega_X^\vee$ so if $H^2(X, \omega_X) = H^1(X, \struct{X})^\vee = 0$ then we can apply the Hartshorne-Serre correspondence to conclude:

\begin{theorem}
For each embedded genus $1$ curve $C \embed X$ there is a rank $2$ vector bundle $E$ with a section $\alpha \in H^0(X, E)$ such that,
\begin{enumerate}
\item $\det{E} \cong \omega_X$
\item $V(\alpha) = C$
\end{enumerate}
furthermore if $H^1(X, \omega_X) = H^2(X, \struct{X})^\vee = 0$ then $(E, \alpha)$ is determined up to isomorphism by $C$.
\end{theorem}

By applying the Kozul resolution to the map $E^\vee \to \struct{X}$ we get the following resolution,
\[ 0 \to \omega_X^\vee \to E^\vee \to \struct{X} \to \struct{C} \to 0 \]
Furthermore, because $\rank{E} = 2$ and $\det{E} = \omega_X$ we get a canonical (one we fix the isomorphism $\det{E} \iso \omega_X$) isomorphism,
\[ E^\vee \iso E \ot \omega_X^\vee \]

\begin{rmk}
In general, if $Y \embed X$ is a scheme of codepth $r$ (meaning,
\[ \depth{\stalk{X}{y}}{\stalk{Y}{y}} = \depth{\stalk{X}{y}}{\stalk{X}{y}} - r = \dim{\stalk{X}{y}} - r \]
for each $y \in Y$ viewing $\stalk{Y}{y}$ as a $\stalk{X}{y}$-module) and finite projective dimension then by the Auslander–Buchsbaum formula we see that each $\stalk{Y}{y}$ has projective dimension $r$. Hence if $X$ has the resolution property then $\struct{Y}$ globally has projective dimension $r$ on $X$ meaning exists a finite locally free resolution,
\[ 0 \to \E_{r} \to \cdots \to \E_{1} \to \struct{X} \to \struct{Y} \to 0 \]
For example, this holds whenever $Y \embed X$ is codimension $r$ regular embedding in a regular scheme $X$.
Therefore, the fact that the ideal sheaf of an embedded genus $1$ curve in a smooth 3-fold has a length two resolution is not at all surprising. The surprising fact is that we can take $\E_1$ to have rank $2$ and $\E_2$ to have rank $1$.
\end{rmk}

\subsection{Duality}

Now the section $\alpha : \struct{X} \to E$ is self-dual in the following sense. The map $\alpha^\vee : E^\vee \to \struct{X}$ generates $\I_C$ and hence, from the Kozul resolution we get $q : \omega_X^\vee \to E^\vee$ is $\ker{\alpha^\vee}$. However, I claim that, canonically $q \ot \omega_X \cong \alpha$. To see this, fix isomorphisms,
\begin{enumerate}
\item $\psi : \bigwedge^2 E \iso \omega_X$
\item $\wt{\psi} : E \ot \omega_X^\vee \iso E^\vee$ via $s \ot \varphi \mapsto \varphi(\psi(s \wedge -))$
\item $q : \omega_X^\vee \to E^\vee$ via $\psi^\vee : \omega_X \iso \bigwedge^2 E^\vee$ composed with $\delta : \eta_1 \wedge \eta_2 \mapsto \alpha^\vee(\eta_1) \eta_2 - \alpha^\vee(\eta_2) \eta_1$
\end{enumerate}
then the claim is that the following diagram commutes,
\begin{center}
\begin{tikzcd}
\omega_X^\vee \arrow[d, equals] \arrow[r, "q"] & E^\vee \arrow[from=d, "\wt{\psi}"']
\\
\omega_X^\vee \arrow[r, "\alpha \ot \omega_X^\vee"] & E \ot \omega_X^\vee
\end{tikzcd}
\end{center}
Indeed, for a local section $\varphi$ of $\omega_X^\vee$ the top map is,
\[ q(\varphi) = \delta(\psi^\vee(\varphi)) \]
We need to show this equals $\wt{\psi}(\alpha \ot \varphi)$ which is the map $s \mapsto \varphi(\psi(\alpha \wedge s))$. Indeed, $\psi^\vee(\varphi)$ is the map $s_1 \wedge s_2 \mapsto \varphi(\psi(s_1 \wedge s_2))$. Now, $\bigwedge^2 E^\vee$ is one-dimensional so let $\eta_1 \wedge \eta_2$ be a local generator then $\varphi(\psi(- \wedge -)) = f \eta_1 \wedge \eta_2$ so $\delta(\varphi^\vee(\varphi)) = f [\alpha^\vee(\eta_1) \eta_2 - \alpha^\vee(\eta_2) \eta_1]$ is the map $s \mapsto f \eta_1(\alpha) \eta_2(s) - f \eta_2(\alpha) \eta_2(s) = f \cdot (\eta_1 \wedge \eta_2)(\alpha \wedge s) = \varphi(\psi(\alpha \wedge s))$ so we conclude.  
\bigskip\\
We write this duality via the equation,
\[ \ker{\alpha^\vee} = \alpha \ot \omega_X^\vee \]
where by kernel we mean the inclusion $\ker{\alpha^\vee} \to E^\vee$.

\subsection{Tangent-Obstruction Theories a la Behrend and Fantechi}

\subsubsection{virtual fundamental classes arise from the normal cone}

Fulton defines intersection classes via a clever reduction to the case of intersection with the zero section of a vector bundle $\pi : E \to X$ which is easily defined via the inverse of the isomorphism $\pi^* : A_k(X) \to A_{k+r}(E)$. To do this, suppose that $\iota : X \embed Y$ is a regular embedding of codimension $d$ and $V \embed Y$ is any subscheme. We want to define an intersection product $X \cdot V \in A_*(X \cap V)$ or equivalently the refined Gysin map $\iota^!_V : A_k(V) \to A_{k-d}(X \cap V)$. To do this, we deform $X \embed Y$ to the embedding of the zero section in the normal bundle $X \embed N_{X|Y}$ using the deformation to the normal cone $M^\circ_{X|Y}$. Inside this, we deform $W = X \cap V \embed V$ via $M^\circ_{W|V} \embed M^\circ_{X|Y}$ whose special fiber is the normal cone $C_{W|V} \embed N_{X|Y}|_V \embed N_{X|Y}$ with $W$ embedded via the zero section. We want intersection products to satisfy the following invariance property,

\begin{center}
Let $\mathcal{Y} \to \P^1$ be a flat family and $\mathcal{V} \subset \mathcal{Y}$ a subscheme flat over $\P^1$. If $X \times \P^1 \to \mathcal{Y}$ is a family of regular embeddings, then the classes $X \cdot_{\mathcal{Y}_t} \mathcal{V}_t \in A_*(X)$ are constant. 
\end{center}

Therefore, applying this to the deformation to the normal cone $\mathcal{Y} = M_X^\circ(Y)$ and $\mathcal{V} = M^\circ_{X \cap V}(V)$ we must define $X \cdot V = s^*[C_{W|V}]$ where $s : W \to N_{X|Y}|_W$ is the zero section of the normal bundle.
\bigskip\\
The ability to intersect regular embedding gives an interesting new structure on the zero locus $Z = Z(s)$ of a section $s : X \to E$ of a vector bundle $\pi : E \to X$ of rank $r$ on $X$. Indeed, scheme-theoretically $Z = V_0 \cap V_s$ where $V_s \subset E$ is the image of the section $s$. Hence $\pi : V_s \to X$ is an isomorphism and $V_s \embed X$ is a regular embedding for any section. If the ``coordinates'' of $s$ induce independent conditions, then $\dim{Z} = n - r$ and it is regularly embedded. However, as $s$ varies, $Z$ may change dimension and behave poorly. Regardless, $V_0 \cdot V_s = s_0^* [C_{Z|V_s}] \in A_{n-r}(Z)$ (since $V_0 \embed E$ is already the zero locus of its normal bundle) gives a class of the expected dimension $n-r$ which in good situations is the fundamental class of the scheme-theoretic intersection $Z = V_0 \cap V_s$. We call this the ``virtual fundamental class'' and $n - r$ the ``virtual dimension''. Now $[V_s] = [V_0]$ since the subschemes $V_s$ and $V_0$ are deformation equivalent over $\A^1$ (and hence rationally-equivalent) so the virtual fundamental class $[Z]^{\vir}$ satisfies,
\[ (\iota_{Z \embed X})_* [Z]^{\vir} = (\iota_{Z \embed X})_* (\iota_{V_0 \embed E})^!_{E} [V_s]  = \iota_{V_0 \embed E}^*[V_s]  = \iota_{V_0 \embed E}^*[V_0] = c_r(E) \frown [X] \]
This means $[Z]^{\vir}$ refines the Euler class of $E$ whereas the fundamental class $[Z]$ will not satisfy this relation if $Z$ has the wrong dimension.
\par
For stacks carrying an obstruction theory (e.g. moduli stacks) we will define a virtual fundamental class locally modeled on the above construction. This can be done analytically locally using Kuranishi theory which identifies a formal neighborhood of the moduli space with the kernel of the holomorphic obstruction map,
\[ U \to T^2 \]
for an open $U \subset T^1$ where $T^1, T^2$ are the tangent and obstruction spaces. Behrend and Fntechi instead define an intrinsic normal cone via local embeddings into smooth schemes. Then a perfect obstruction theory defines an embedding of this intrinsic normal cone into an obstruction bundle and we may define the virtual fundamental class as the intersection of the normal cone with the zero section as before. Here we explain this construction.
\par
Let $\X$ be a DM-stack and $v : \X_{\fppf} \to \X_{\et}$ be the morphism of topoi from the big fppf site to the small \etale site. The canonical map $v^{-1} \struct{\X_{\et}} \to \struct{\X_\et}$ induces a morphism of ringed topoi,
\[ v : (\X_{\fppf}, \struct{\X_{\fppf}}) \to (\X_{\et}, \struct{\X_{\et}}) \]

(WHY DO WE NEED TO PASS TO FPPF TO DEFINE NORMAL BUNDLE)

\begin{defn}
We say an object $L^\bullet \in D(\struct{\X_{\et}})$ satisfies condition $\star$ if,
\begin{enumerate}
\item $h^i(L^\bullet) = 0$ for all $i > 0$
\item $h^i(L^\bullet)$ is coherent for $i = 0,-1$.
\end{enumerate}
\end{defn}

\begin{prop}
There exists a cotangent complex $\LL_{\X}^\bullet \in D(\struct{\X_{\et}})$ and it satisfies condition $\star$.
\end{prop}

\begin{defn}
Let $E^\bullet \in D(\struct{\X_{\et}})$ satisfy condition $\star$. Then a homomorphism $\phi : E^\bullet \to \LL_X^\bullet$ in $D(\struct{\X_{\et}})$ is called an \textit{obstruction theory} for $X$ if
\begin{enumerate}
\item $h^0(\phi)$ is an isomorphism
\item $h^{-1}(\phi)$ is surjective.
\end{enumerate}
\end{defn}

\begin{rmk}
These two conditions are equivalent to asking that the induced morphism of cones $\phi^\vee : \mathfrak{N}_X \to \mathfrak{C}_{E^\bullet}$ is a closed immersion. Thus if $\mathfrak{C}_X \subset \mathfrak{N}_X$ is the intrinsic normal cone then $\phi^\vee(\mathfrak{C}_X)$ is a closed subcone of $\mathfrak{C}_{E^\bullet}$ called the \textit{obstruction cone}. If $\mathfrak{C}_{E^\bullet}$ is smooth over $X$ then it is a vector bundle stack and hence we can intersect the obstruction cone with the zero section using the gysin map to obtain a virtual fundamental class. This smoothness occurs when the obstruction complex is \textit{perfect}. 
\end{rmk}

\begin{defn}
We call an obstruction theory $E^\bullet \to \LL^\bullet_{\X}$ \textit{perfect}, if $E^\bullet$ is of perfect amplitude contained in $[-1,0]$.
\end{defn}

\newcommand{\fC}{\mathfrak{C}}

Let $E^\bullet$ be a perfect obstruction theory for $\X$ and let $\fC_X \embed h^1/h^0(E^\vee)$ be the intrinsic normal cone. We call $\rank{E^\bullet}$ the \textit{virtual dimension} of $\X$ with respect to the obstruction theory $E^\bullet$. Recall that $\rank{E^\bullet} = \rank{E^0} - \rank{E^{-1}}$ if locally $E^\bullet$ is written as a complex of vector bundles $[E^{-1} \to E^0]$. This is a well-defined locally constant function on $\X$. We shall assume that the virtual dimension of $\X$ with respect to $E^\bullet$ is a constant $n$.
\par 
To construct the \textit{virtual fundamental class} $[\X, E^\bullet] \in A_n(\X)$ of $\X$ with respect to the obstruction theory $E^\bullet$, we would like to simply intersect the intrinsic normal cone $\mathfrak{C}_X$ with the zero section of $h^1/h^0(E^\vee)$ which is smooth of relative dimension $-n$ over $X$ and hence this intersection has dimension $n$. Unfortuntely, this construction would require Chow groups for Artin stacks, which we do not have at our disposal. This is why we need the assumption that $E^\bullet$ has a global resolution.

\begin{defn}
Let $F^\bullet = [F^{-1} \to F^0]$ be a homomorphism of vector bundles on $\X$ considered as a complex of $\struct{\X}$-modules concentrated in degrees $-1$ and $0$. An isomorphism $F^\bullet \to E^\bullet$ in $D(\struct{\X_{\et}})$ is called a \textit{global resolution} of $E^\bullet$.
\end{defn}

Let $F^\bullet$ be a global resolution of $E^\bullet$. Then,
\[ h^1/h^0(E^\vee) = [(F^{-1})^\vee / (F^0)^\vee] \]
so that $F_1 = (F^{-1})^\vee$ is a global presentation of $h^1/h^0(E^\vee)$. Let $C(F^\bullet)$ be the fiber product,
\begin{center}
\begin{tikzcd}
C(F^\bullet) \pullback \arrow[r] \arrow[d] & F_1 \arrow[d]
\\
\fC_X \arrow[r] & h^1/h^0(E^\vee) 
\end{tikzcd}
\end{center}
Then $C(F^\bullet)$ is a closed subcone of the vector bundle $F_1$. We define the \textit{virtual fundamental class} $[X, E^\bullet]$ to be the intersection of $C(F^\bullet)$ with the zero section of $F_1$. Note that $C(F^\bullet) \to \fC_X$ is smooth of relative dimension $\rank{F_0}$, so that $C(F^\bullet)$ has pure dimension $\rank{F_0}$ nd $[X, E^\bullet]$ has degree,
\[ \rank{F_0} - \rank{F_1} = \rank{E^\bullet} = n \]


\subsubsection{Obstruction Theories}

If $T \to \ol{T}$ is a square zero extension of $k$-schemes with ideal $J$ and $g : T \to \X$ is a morphism. Consider the morphisms,
\[ g^* \LL_{\X}^\bullet \to \LL_{T}^\bullet \to \LL^\bullet_{T/\ol{T}} \]
but $\tau_{\ge -1} \LL^\bullet_{T/\ol{T}} = J[1]$ and therefore this morphism defines an obstruction,
\[ \omega(g) \in \Ext{1}{}{g^* \LL_{\X}^\bullet}{J} \]
to the existence of an extension $\bar{g} : \ol{T} \to \X$. Then the extensions form a torsor under $\Ext{0}{}{g^* \LL_{\X}^\bullet}{J} = \Hom{}{g^* \Omega_X}{J}$. Then the set of extensions $\bar{g} : \ol{T} \to X$ of $g$ forms a sheaf $\ul{\mathrm{Ext}}(g, \ul{T})$ on $T_{\et}$ and there is a canonical isomorphism,
\[ \ul{\mathrm{Ext}}(g, \ul{T}) \iso \Hom{\struct{T}}{\ob(g)}{0(g)} \]
where $\ob(g) : C(J) \to g^* \mathfrak{N}_X$ is the induced morphism of cones and $0(g)$ is the zero section and we consider the sheaf of $2$-isomorphisms of cone stacks (meaning they are $\A^1$-equivariant 2-morphisms). 
\par
Let $E^\bullet \in D(\struct{\X_{\et}})$ satisfy $(\star)$ and let $\phi : E^\bullet \to \LL_{\X}^\bullet$ be a homomorphism. Let $\fC = h^1/h^0((E^\bullet_{\fppf})^\vee)$ and $\phi^\vee : \mathfrak{N}_{\X} \to \fC$ the induced morphism of cone stacks. We denote by $\phi^* \omega(g)$ the image of $\omega(g) \in \Ext{1}{}{g^* \LL_{\X}^\bullet}{J}$ in $\Ext{1}{}{g^* E^\bullet}{J}$ and by $\phi^\vee(\ob(g))$ the composition,
\[ C(J) \xrightarrow{\ob(g)} g^* \mathfrak{N}_X \xrightarrow{g^* \phi^\vee} g^* \fC \]
of morphisms of cone stacks over $T$.

\begin{theorem}
The following are equivalent
\begin{enumerate}
\item $\phi : E^\bullet \to \LL_{\X}^\bullet$
\item $\phi^\vee : \mathfrak{N}_X \to \fC$ is a closed immersion of cone stacks over $X$.
\item For any $(T, \ol{T}, g)$ as above, the obstruction $\phi^* \omega(g) \in \Ext{1}{}{g^* E^\bullet}{J}$ vanishes if and only if there exists an extension $\bar{g}$ of $g$ to $\ol{T}$ and if $\phi^* \omega(g) = 0$ then the extensions form a torsor under $\Ext{0}{}{g^* E^\bullet}{J} = \Hom{}{g^* h^0(E^\bullet)}{J}$
\item For any $(T, \ol{T}, g)$ as above, the sheaf of extensions $\ul{\mathrm{Ext}}(g, \ol{T})$ is isomorphic to the sheaf $\Hom{}{\phi^\vee(\ob(g))}{0}$ of $\A^1$-equivariant isomorphisms from $\phi^\vee(\ob(g)) : C(J) \to g^* \fC$ to the zero section $0 : C(J) \to g^* \fC$.
\end{enumerate}
\end{theorem}

\subsection{The Tangent-Obstruction Complex for DT-Theory}

Donaldson-Thomas theory operates based on the following observations.

\begin{theorem}[\chref{https://arxiv.org/pdf/2104.12736.pdf}{Lieblich Olsson}]
Let $E$ be a perfect complex on a scheme $X$. If $K$ is a bounded complex then there is a trace map,
\[ \tr : \Ext{i}{X}{E}{E \ot^{\LL} K} \to H^i(X, K) \]
and for the obstructions to deformations we have,
\[ \tr \ob(E) = \ob(\det{E}) \in H^2(X, K) \]
and the tangent spaces satisfy: if $(E', \sigma)$ is a deformation of $E$ over $X \embed X'$ with kernel $K$ and $\alpha \in \Ext{1}{X}{E}{E \ot^{\LL} K}$ is a class, then
\[ \det{(\alpha * (E', \sigma))} = \tr{\alpha} * (\det{E'}, \det{\sigma}) \]
Therefore, the traceless subspaces $\Ext{1}{X}{E}{E \ot^{\LL} K}_0, \Ext{2}{X}{E}{E \ot^{\LL} K}_0$ form a tangent-obstruction theory for perfect complexes with fixed determinant. 
\end{theorem}

Donaldson-Thomas theory studies the moduli space of ideal sheaves with fixed chern classes. However, the standard tangent-obstruction theory on the Hilbert scheme is not ammenable to defining a perfect obstruction theory on $\Hilb$ in order to define a vitrual fundamental class via the intrinsic normal cone. Instead, we restrict to the moduli space of Gieseker stable sheaves which does have a perfect obstruction theory on a smooth threefold. 

\begin{defn}
Let $X$ be an integral projective $k$-scheme with ample line bundle $\struct{X}(1)$. A coherent sheaf $\F$ is \textit{Gieseker stable} if it is torsion-free and for any subsheaf $0 \subsetneq \G \subsetneq \F$ we have, 
\[ \frac{\chi(\G(n))}{\rank{\G}} < \frac{\chi(\F(n))}{\rank{\F}} \]
for $n \gg 0$.
\end{defn}

\begin{lemma}
Gieseker stability is an open condition in flat (locally\footnote{We want a flat proper family $f : X \to S$ with an $f$-ample line bundle $\L$ on $X$.}) projective families.
\end{lemma}

\begin{proof}
Let $\F$ be a coherent sheaf on $X$ flat over $S$ which is torsion-free on fibers. The unstable locus on $S$ is the image of $\Quot_{\F/X/S}^{\L, \Phi} \to S$ ranging over hilbert polynomials $\Phi$ such that $\Phi(\F) - \Phi$ is the hilbert polynomial of a destabilizing subsheaf on some fiber. Therefore, by properness of the Quot scheme, it suffices to show that this set of $\Phi$ is finite.  

(WHY IS IT FINITE)
\end{proof}


\begin{prop}
Let $X$ be an integral locally factorial projective $k$-scheme and $\I$ a rank $1$ Gieseker stable sheaf with $\det{\I} \cong \struct{X}$. Then $\I$ is an ideal sheaf in a canonical (fixing the isomorphism $\det{\I} \cong \struct{X}$) way. Moreover, every nonzero ideal sheaf is Gieseker stable of rank $1$.
\end{prop}

\begin{proof}
Since $\I$ is Gieseker stable it is torsion-free and hence $\I \to \I^{\vee \vee}$ is injective but $\I^{\vee \vee}$ is a line bundle since it is reflexive of rank $1$ and $X$ is locally factorial. Since $\det{\I} = \det{\I^{\vee \vee}} = \struct{X}$ we see that $\I \embed \struct{X}$ in a canonical way. Furthermore, if $\I \subset \struct{X}$ is an ideal sheaf it either is nilpotent (hence zero) or nonzero at the generic point and hence rank $1$. Furthermore, for any $0 \subsetneq \G \subsetneq \I$ we see that $\rank{\G} = 1$ so supoose that $\chi(\G(n)) \ge \chi(\I(n))$ for $n \gg 0$. Thus $\chi((\I / \G)(n)) \le 0$ for $n \gg 0$. However, by Serre vanishing, if $\chi(\F(n)) \le 0$ for $n \gg 0$ then $\F$ is represented by the zero module so $\F = 0$ hence $\I$ is stable.
\end{proof}


Now we define a tangent-obstruction theory for sheaves. Fix a smooth quasi-projective scheme $X$ and Chern classes $c_i \in H^{2i}(X)$ and consider the moduli functor $\cM$ that assigns to any scheme $S$ the groupoid of Gieseker stable sheaves on $X \times S$ flat over $S$ with chern classes $c_i$. Consider the standard data:

\begin{enumerate}
\item $S$ an affine scheme
\item $\E_0$ a sheaf on $X \times S$ Gieseker stable on each fiber of $p : X \times S \to S$ and flat over $S$
\item Chern classes $c_i(\E_0) \in H^{2i}(X)$ and rank $r(\E) \in H^0(X)$ and a line bundle $L$ on $X$ with $c_1(L) = c_1$
\item the corresponding classifying morphism $f : S \to \cM$
\item an $\struct{S}$-module $\I$.
\end{enumerate}

\begin{theorem}
Let $X$ be a smooth, polarized, complex projective variety and fix Chern classes $c_i \in H^{2i}(X)$ and a line bundle $L \in \Pic{X}$ with $c_1(L) = c_1$. Let $\cM = \cM(X, c_i)$ denote the moduli space of Gieseker stable sheaves and $\cM_L$ the sublocus of those with determinant $L$. If the numbers,
\[ \dim{\Ext{i}{X}{\E}{\E}} \quad i \ge 3 \]
are constant over $\E \in \cM$, then the tangent-obstruction complex given by,
\[ \cT^i_{\E_0} = \shExt{i}{p}{\E_0}{\E_0 \ot p^* \I} \]
is perfect.
\end{theorem}


\section{Duality Calculation}

\newcommand{\bfL}{\mathbf{L}}

Let $X$ be a 3-fold and $\struct{C}$ the ideal sheaf of an embedded curve $\iota : C \embed X$. Then we compute,
\[ \RHom{X}{\iota_* \struct{C}}{\struct{X}} = \iota_* \RHom{C}{\struct{C}}{\R \iota^! \struct{X}} = \iota_* \R \iota^! \struct{X} \]
where $\R \iota^!$ is the cohomology with supports along $C$. 

\begin{prop}
Let $f : X \to Y$ be a proper morphism of smooth proper varities over $k$ then for any perfect object $\F$,
 \[ \R f^! \F = (\bfL f^* \F \ot^{\bfL} \omega_X \ot^{\bfL} \bfL f^* \omega_Y^{-1}) [\dim{X} - \dim{Y}] \]
\end{prop}

\begin{proof}
By Serre duality, there is a unique line bundle $\omega_X$ such that,
\[ \Hom{X}{\G}{\omega_X} = H^n(X, \G)^\vee \]
Since everything is CM we see that,
\[ \RHom{X}{\G}{\omega_X} = \R \Gamma(\G)^\vee [-\dim{X}] \]
This is just Grothendieck duality for $p : X \to *$ since,
\[ \RHom{X}{\G}{\omega_X[\dim{X}]} = \RHom{X}{\G}{\R p^! k} = \RHom{X}{\R \Gamma(\G)}{k}  = \R \Gamma(\G)^{\vee} \]
Since $\F$ and $\omega_X$ and $\omega_Y$ are all perfect we can move them around as follows,
\begin{align*}
\RHom{X}{\G}{(\R f^! \F) \ot^{\bfL} f^* \omega_Y} &= \RHom{X}{\G \ot^{\bfL} f^* \omega_Y^{-1}}{\R f^! \F} = \RHom{Y}{\R f_* \G \ot^{\bfL} \omega_Y^{-1}}{\F} 
\\
& = \RHom{Y}{\R f_* \G \ot^{\bfL} \F^\vee}{\omega_Y} = \R\Gamma(Y, \R f_* \G \ot^{\bfL} \F^\vee)^\vee [-\dim{Y}] 
\end{align*}
But notice also by the projection formula and Serre duality,
\[ \RHom{X}{\G \ot^{\bfL} \bfL f^* \F^\vee}{\omega_X} = \R \Gamma(X, \G \ot^{\bfL} \bfL f^* \F^\vee)^\vee [-\dim{X}] = \R\Gamma(Y, \R f_* \ot^{\bfL} \F^\vee)^\vee [-\dim{X}]  \]
Therefore, by the uniqueness in Serre duality we see that,
\[ ((\R f^! \F) \ot^{\bfL} f^* \omega_Y)[\dim{Y}] = (\bfL f^* \F \ot \omega_X)[\dim{X}] \]
proving the claim.
\end{proof}
Therefore, we see that,
\[ \R \iota^! \struct{X} = \omega_C \ot (\iota^* \omega_X^{-1}) [-2] \]
Hence if $\omega_C \ot (\iota^* \omega_X^{-1}) \cong \struct{C}$ then we see that
\[ \RHom{X}{\iota_* \struct{C}[-1]}{\struct{X}} = \iota_* \R \iota^! \struct{X}[1] = \iota_* \struct{C}[-1] \]
meaning that $\iota_* \struct{C}[-1]$ is self-dual as prediced by the isomorphism,
\[ \iota_* \struct{C}[-1] \cong [0 \to \struct{X} \to \E \to \struct{X} \to 0] \]
to a self-dual complex (with $\E$ in degree $0$ and cohomology -- the cokernel of the map $\E \to \struct{X}$ whose image generates $\I_C$ being $\struct{C}$ -- in degree $1$). 


\end{document}