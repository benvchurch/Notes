\documentclass[12pt]{article}
\usepackage{hyperref}
\hypersetup{
    colorlinks=true,
    linkcolor=blue,
    filecolor=magenta,      
    urlcolor=blue,
}

\usepackage{import}
\import{"../Algebraic Geometry/"}{AlgGeoCommands}

\newcommand{\Loc}[1]{\mathfrak{Loc}\left( #1 \right)}
\newcommand{\AbGrp}{\mathbf{AbGrp}}

\renewcommand{\tr}{\operatorname{tr}}

\newcommand{\LL}{\mathbb{L}}
\newcommand{\ob}{\mathrm{ob}}
\newcommand{\cM}{\mathcal{M}}
\newcommand{\cT}{\mathcal{T}}
\newcommand{\vir}{\mathrm{vir}}
\newcommand{\cO}{\mathcal{O}}
\newcommand{\ad}{\mathrm{ad}}

\newcommand{\Y}{\mathscr{Y}}

\DeclareMathOperator{\coeff}{\mathrm{coeff}}
\DeclareMathOperator{\cent}{\mathrm{center}}


\begin{document}

\section{Facts About the Picard Scheme}

\tableofcontents

\newcommand{\Set}{\mathbf{Set}}

\begin{defn}
Let $f : X \to S$ be a morphism of schemes. Consider the sheaf
\[ \fPic_{X/S} : (\Sch_S)_{\fppf} \to \Set \]
given by the sheafification in the big $\fppf$ site of the presheaf $T \mapsto \Pic{X_T} / \Pic{T}$. We say that $\fPic_{X/S}$ exists if this sheaf is representable by an algebraic space. 
\end{defn}

\begin{theorem}[Artin]
Let $f : X \to S$ be proper, flat and finitely presented. Then if the formation of $f_* \struct{X}$ commutes with all base change then $\fPic_{X/S}$ is represented by an algebraic space over $S$.
\end{theorem}

\subsection{Some Remarks About Sites}

\renewcommand{\Sh}{\mathrm{Sh}}
\newcommand{\PSh}{\mathrm{PSh}}

Functors between sites are just functors of the underlying category.

\begin{defn}
Given a functor $u : \C \to \D$ we can form the following operations on presheaves
\[ u^p : \PSh(\D) \to \PSh(\C) \quad u^p : \F \mapsto \F \circ u \]
\[ u_p : \PSh(\C) \to \PSh(\D) \quad u_p : \F \mapsto (V \mapsto \dlim_{V \to u(U)} \F(U)) \]
these are the familiar pushforward and pullback maps if $u = f^{-1}$ for a morphism of topological spaces. Furthermore there is another pushforward
\[ {}_p u : \PSh(\C) \to \PSh(\D) \]
\[ {}_p u : \F \mapsto (V \mapsto \ilim_{u(U) \to V} \F(U)) \]
If $\C$ and $\D$ are sites then composing these functors with sheafification produces functors $u^s, u_s, {}_s u$ on the category of sheaves.
\end{defn}

\begin{lemma}[\chref{https://stacks.math.columbia.edu/tag/00VE}{Tag 00VE}]
The functors $(u_p, u^p)$ and $(u^p, {}_p u)$ are adjoint pairs.
\end{lemma}

In good situations $f = (u_s, u^s)$ will define a morphism of topoi $\Sh(\D) \to \Sh(\C)$. The canonical example of such behavior is when $u = f^{-1}$ for $f : X \to Y$ a continuous map in which case $f : \Sh(X) \to \Sh(Y)$ is the usual morphism of topoi. Indeed, this holds in the following situation. 

\begin{lemma}
If $u : \C \to \D$ is a continuous functor of sites then $(u_s, u^s)$ is an adjoint pair and $u_s = u_p$.
\end{lemma}

\begin{proof}
It is clear that $u^p \F$ is a sheaf since $(u^p \F)(U) = \F(u(U))$ and $u$ preserves coverings so $u^p \F$ satisfies the sheaf condition. Furthermore,
\[ \Hom{\Sh(\D)}{u_s \F}{\G} = \Hom{\PSh(\D)}{u_p \F}{\G} = \Hom{\PSh(\C)}{\F}{u^p \G} = \Hom{\Sh(\C)}{\F}{u^s \G} \]
The first step follows by the universal property of sheafification, the second by adjointness of $(u_p, u^p)$ and the third by $u^p = u^s$. 
\end{proof}

\begin{defn}
A \textit{morphism} $f : \D \to \C$ of sites is a continuous functor $f^{-1} : \C \to \D$ (meaning it preserves fiber products by covers and takes covers to covers) such that $(f^{-1})_s$ is exact. Therefore, it defines a morphism of topoi $((f^{-1})_s, (f^{-1})^s) : \Sh(\D) \to \Sh(\C)$. 
\end{defn}

In dual good situations $(u^s, {}_s u)$ will define a morphism of topoi $\Sh(\C) \to \Sh(\D)$ (note the different directions of these maps). The canonical of such behavior is the postcomposition map $(\Sch_{S'})_{\fppf} \to (\Sch_{S})_{\fppf}$ along a map $S' \to S$. Indeed, this holds in the following situation.

\begin{defn}
Let $u : \C \to \D$ be a functor between sites. The functor is called \textit{cocontinuous} if for each $U \in \C$ and covering $\{ V_j \to u(U) \}_{j \in J}$ in $\D$ there exists a covering $\{U_i \to U\}_{i \in I}$ such that the family of maps $\{ u(U_i) \to u(U) \}_{i \in I}$ refines the covering $\{ V_j \to u(U) \}_{j \in J}$.
\end{defn}

\begin{rmk}
Note that in general $\{ u(U_i) \to u(U) \}_{i \in I}$ is \textit{not} in general a covering in $\D$. 
\end{rmk}

\begin{lemma}[\chref{https://stacks.math.columbia.edu/tag/00XO}{Tag 00XO}]
Let $u : \C \to \D$ be a cocontinuous functor between sites. Then it defines a morphism $g = (u^s, {}_s u) : \Sh(\C) \to \Sh(\D)$ of topoi where
\[ u^s \F := (\F \circ u)^{\#} \]
and 
\[ {}_s u \F := (U \mapsto \lim_{u(U) \to V} \F(U))^{\#} \]
meaning they form an adjoint pair and $u^s$ is exact. 
\end{lemma}

\begin{lemma}[\chref{https://stacks.math.columbia.edu/tag/00XX}{Tag 00XX}] \label{lemma:cocontinuous_adjoint_pair}
Let $\C$ and $\D$ be sites. Let $u : \C \to \D$ and $v : \D \to \C$ be functors. Assume that $u$ is cocontinuous and that $(u, v)$ is an adjoint pair. Let $g = (u^s, {}_s u) : \Sh(\C) \to \Sh(\D)$ be the morphism of topoi in the previous lemma. Then $g_* \F = v^p \F$ as presheaves so $g_* = v^s = v^p$ and also $g^{-1} = v_s$. Hence the map of topoi $(v_s, v^s)$ associated to $v$ is equal to $g$. Moveover, if $v$ is continuous then $v$ defines a morphism of sites $f : \C \to \D$ whose associated morphism of topoi is equal to $g$.
\end{lemma}

\begin{proof}
We have $u^p h^V = h^{v(V)}$ by adjointness since,
\[ (u^p h^V)(U) = h^V(u(U)) = \Hom{\D}{u(U)}{V} = \Hom{\C}{U}{v(V)} = h^{v(V)}(U) \]
Therefore $g^{-1} (h^V)^{\#} = (u^p (h^V)^{\#})^{\#} = (u^p h^V)^{\#} = (h^{v(V)})^{\#}$. Hence for any sheaf $\F$ on $\C$ we have
\[ (g_* \F)(V) = \Hom{\Sh(\D)}{(h^V)^{\#}}{g_* \F} = \Hom{\Sh(\C)}{g^{-1} (h^V)^{\#}}{\F} = \Hom{\Sh(\C)}{(h^V)^{\#}}{\F} = \F(v(V)) \]
To prove that $g^{-1} := u^s$ equals $v_s$ we use that $(v_s, v^s)$ and $(u^s, {}_s u)$ are adjoint pairs and $v^s = {}_s u$ so we conclude by uniqueness of adjoints.
\end{proof}

\begin{lemma}
Let $g : S' \to S$ be a map of objects of a site $\C$ which has fiber products against $g$. Consider the functors 
\[ g^{-1} : \C_S \to \C_{S'} \quad \quad g^{\to} : \C_{S'} \to \C_{S} \]
given by
\[ g^{-1} : X/S \mapsto X \times_S S' / S' \quad \quad g^{\to} : X / S' \mapsto X / S \]
Then $(g^{\to}, g^{-1})$ is an adjoint pair with $g^{\to}$ is continuous and cocontinuous and $g^{-1}$ continuous. They induce the same morphism of topoi $f : \Sh(\C_{S'}) \to \Sh(\C_{S})$ meaning
\[ f^{-1} = (g^{-1})_s = (g^{\to})^s = (g^{\to})^p \]
and 
\[ f_* = (g^{-1})^p = {}_s (g^{\to}) \]
\end{lemma}

\begin{proof}
Indeed,
\[ \Hom{S}{g^{\to} X}{Y} = \Hom{S'}{X}{S' \times_S Y} = \Hom{S'}{X}{g^{-1} Y} \]
Note that $g^{\to}$ is continuous, cocontinuous, and commutes with fiber products and equalizers. By continuity $(g^{\to})^s = (g^{\to})^p$ (which is also obvious). Now we conclude using the previous lemma.
\end{proof}

\begin{cor}
Let $g : S' \to S$ be a morphism of schemes and consider the corresponding maps between sites
\[ g^{\to} : \Sch_{S'} \to \Sch_S \quad \quad g^{-1} : \Sch_{S} \to \Sch_{S'} \]
Then the corresponding morphism of topoi for the $\tau$-topology
\[ f : \Sh((\Sch_{S'})_{\tau}) \to \Sh((\Sch_{S})_{\tau}) \]
is computed as $(f^{-1} \F)(U/S') = \F(U/S)$ and $(f_* \F)(V/S) = \F(V_{S'}/S')$.
\end{cor}

\begin{rmk}
We call $f^{-1} \F$ the ``base change of $\F$ along $g$''. Indeed, if $\F = h^X$ is representable by an $S$-scheme then $f^{-1} \F = h^{X_{S'}}$ indeed we proved this in Lemma~\ref{lemma:cocontinuous_adjoint_pair} applied to the adjoint pair $(g^{\to}, g^{-1})$. It is just adjointness, recall
\[ (f^{-1} h^X)(U/S') = h^X(U/S) = \Hom{S}{U}{X} = \Hom{S'}{U}{X_{S'}} = h^{X_{S'}}(U) \]
Therefore, we will sometimes write $\F \times_S S' := f^{-1} \F$.
\end{rmk}

\begin{lemma}[\chref{https://stacks.math.columbia.edu/tag/00WY}{Tag 00WY}]
Let $\F : \C^\op \to \Set$ be a presheaf. Let $\F^{\#} \in \Sh(\C)$ be its sheafification. If $u : \C \to \D$ is a continuous functor of sites then  $u_s \F^{\#}$ is the sheafification of $u_p \F$.
\end{lemma}

\begin{proof}
Indeed, for any sheaf $\G$ on $\D$
\begin{align*}
\Hom{\Sh(\D)}{u_s \F^{\#}}{\G} &= \Hom{\Sh(\C)}{\F^{\#}}{u^s \G} = \Hom{\PSh(\C)}{\F}{u^s \G} = \Hom{\PSh(\C)}{\F}{u^p \G}
\\
& = \Hom{\PSh(\C)}{u_p \F}{\G} = \Hom{\Sh(\C)}{(u_p \F)^{\#}}{\G} 
\end{align*}
using that $u^s = u^p$ because $u$ is continuous. Thus we conclude by the Yoneda lemma.
\end{proof}

\begin{lemma}
In the situation of Lemma~\ref{lemma:cocontinuous_adjoint_pair} with $v$ continuous, let $\F : \D^\op \to \Set$ be a presheaf. Then $u^s (\F^{\#}) = (u^p \F)^{\#}$. 
\end{lemma}

\begin{proof}
Indeed, for any sheaf $\G$ on $\C$
\begin{align*}
\Hom{\Sh(\C)}{u^s \F^{\#}}{\G} &= \Hom{\Sh(\D)}{\F^{\#}}{{}_s u \, \G} = \Hom{\PSh(\D)}{\F}{{}_s u \, \G} = \Hom{\PSh(\C)}{\F}{v^p \, \G}
\\
& = \Hom{\PSh(\C)}{v_p \F}{\G} = \Hom{\Sh(\C)}{u^p \F}{\G} = \Hom{\Sh(\C)}{(u^p \F)^{\#}}{\G} 
\end{align*}
\end{proof}

\begin{cor}
If $g : S' \to S$ is a morphism in a site $\C$ and $f : \Sh(\C_{S'}) \to \Sh(\C_{S})$ is the corresponding morphism of topoi. If $\F : \C_{S}^\op \to \Set$ is a presheaf then $f^{-1} (\F^{\#}) = (\F \circ g^{\to})^{\#}$. 
\end{cor}

\begin{cor}
The Picard functor $\fPic_{X/S}$ commutes with base change, meaning for any $g : S' \to S$ we have $\fPic_{X/S} \times_S S' = \fPic_{X_{S'}/S'}$.
\end{cor}

\begin{proof}
Indeed, this is exactly the previous lemma applied to the Picard sheaf which is the sheafification of the relative Picard functor $T/S \mapsto \Pic{X_T}/\Pic{T}$. Finally, this composed with $g^{\to}$ produces,
\[ T/S' \mapsto T/S \mapsto \Pic{X_T} / \Pic{T} = \Pic{(X_{S'})_T} / \Pic{T} \]
which is the relative Picard functor for $X_{S'}/S'$ applied to $T/S'$.
\end{proof}


\subsection{Exceptional Pushforward}

Here let $g : S' \to S$ be a morphism of schemes and 
\[ f : \Sh((\Sch_{S'})_{\tau}) \to \Sh((\Sch_{S})_{\tau}) \]
the corresponding morphism of topoi for the $\tau$-topology. 
\bigskip\\
Note that $f_* \F = (g^{-1})^s \F$ is Weil-restriction because exactly its value on $T/S$ is $\F(T \times_S S')$.
\bigskip\\
Now, there is another operation we can perform on schemes which we upgrade to sheaves. Let $g : S' \to S$ be a morphism and $f : X \to S'$ be an $S'$-scheme. Then we can view $X/S'$ as a $S$-scheme which is called $g^{\to} X$. This gives,
\[ \Hom{S}{T}{g^{\to} X} = \{ (a,b) \mid a : T \to S' \text{ and } b \in \Hom{S'}{T}{X} \} \]
where $b$ is a $S'$-morphism through the structure map $a$. The claim is that there is an exceptional left adjoint $f_!$ such that $f_! h^{X/S'} = h^{X/S}$ and thus $f_!$ extends this operation to all sheaves.

\begin{lemma}
Let $u : \C \to \D$ be cocontinuous functor and $g = (u^s, {}_s u) : \Sh(\C) \to \Sh(\D)$ the associated morphism of topoi. If $u$ is also continuous then
\begin{enumerate}
\item $g^{-1} = u^p$ (no sheafification required)
\item $g^{-1}$ has a left adjoint $g_! = u_s$
\item $g^{-1}$ commutes with arbitrary limits and colimits
\item $g_! (h^{X})^{\#} = (h^{u(X)})^{\#}$.
\end{enumerate}
\end{lemma}

\begin{proof}
Indeed, since $u$ is continuous $u^p$ sends sheaves to sheaves and $g^{-1} = u^s = (u^p)^{\#}$ so we see that $g^{-1} = u^p$. Therefore, 
\begin{align*}
\Hom{\Sh(\C)}{\F}{g^{-1} \G} = \Hom{\PSh(\C)}{\F}{u^p \G} = \Hom{\PSh(\D)}{u_p \F}{\G} = \Hom{\Sh(\D)}{g_! \F}{\G} 
\end{align*}
and (c) follows from (b) and that $g^{-1}$ is left adjoint to $g_*$. Finally,
\[ \Hom{\Sh(\D)}{g_! (h^X)^{\#}}{\G} = \Hom{\Sh(\C)}{h^X}{g^{-1} \G} = (g^{-1} \G)(X) = \G(u(X)) = \Hom{\Sh(\C)}{(h^X)^{\#}}{\G} \]
using (a) to get $g^{-1} \G = G \circ u$.
\end{proof}

\begin{rmk}
Although $(u_p, u^p)$ are always adjoint, $(u_s, u^s)$ are not necessarily adjoint if $u$ is not continuous because there is a sheafification in $u^s$ which does not have good mapping-in behavior.
\end{rmk}


\begin{cor}
Let $g : S' \to S$ be a morphism of schemes and $f : \Sh((\Sch_{S'})_{\tau}) \to \Sh((\Sch_{S})_{\tau})$
the corresponding morphism of $\tau$-topoi. Then $f_! h^{X/S'} = h^{X/S}$ for any $S'$-scheme $X$.
\end{cor}

\begin{proof}
This follows immediately from (d) above applied to $u = g^{\to}$. We can also just check,
for any sheaf $\F$ on $\C_{S}$ where $\C = (\Sch)_{\tau}$ use adjunction,
\begin{align*}
\Hom{\Sh(\C_{S})}{f_! h^{X/S'}}{\F} & = \Hom{\Sh(\C_{S'})}{h^{X/S'}}{f^{-1} \F} = (f^{-1} \F)(X/S')
\\
& = \F(g^{\to}(X/S')) = \F(X/S) = \Hom{\Sh(\C_{S})}{h^{X/S}}{\F}
\end{align*}
so we conclude by the Yoneda lemma.
\end{proof}

\section{Remarks about Moduli Theory}

It is usually advisable from the perspective of moduli theory to work in ``big'' sites not ``small'' sites. The reason is if you restrict your objects to only those with certain nice properties over the base $S$ then these covers may not be able to see any representing object that has worse properties. For example, if we take the representable sheaf given over $\A^1$-schemes by the object $* \to \A^1$ then in the small \etale, fppf, etc site this is the trivial sheaf because no nonempty finitely presented flat scheme $T \to \A^1$ can factor through a point since its image must be open. Therefore, this functor is even representable in the fppf site, by the empty scheme, but this is not the correct object. 

\section{Abelian Schemes}

In this section $k$ is an arbitrary field.

\begin{lemma}
Let $f : A \to B$ be a map of abelian varities and $\phi^\vee : B^\vee \to A^\vee$ be the dual. Consider the maps
\[ A \times_k A^\vee \leftarrow A \times_k B^\vee \to B \times_k B^\vee \]
then there is a canonical isomorphism of rigidified sheaves on $A \times_k B^\vee$
\[ (\id_A \times \phi^\vee)^* \cP_A \cong (\phi \times \id_{B^\vee})^* \cP_B \]
\end{lemma}

\begin{proof}
Indeed, one way of defining $\phi^\vee$ is as the classifying map for the bundle $(\phi \times \id_{B^\vee})^* \cP_B$ on $A \times_k B^\vee$ which is rigidifed along $e_A \times B^\vee$. To see why this agrees with the map defined by functoriality, we need to show that for any $t : T \to B^\vee$ that $\phi^\vee \circ t$ represents $\phi^*_T \L_t$ on $A_T$. Indeed, this is by definition. To say that $\phi^\vee \circ t$ represents $\phi^* \L_t$ means exactly
\[ (\id_A \times \phi^\vee \circ t)^* \cP_A = (\id_A \times t)^* (\id_A \times \phi^\vee)^* \cP_A = (\id_A \times t)^* (\phi \times \id_{B^\vee})^* \cP_B = (\phi \times \id_T)^* (\id_B \times t)^* \cP_B \]
\end{proof}

\begin{defn}
Let $(X, x_0)$ be a pointed $k$-scheme meaning $x_0 \in X(k)$. Recall that the \textit{albanese} $\Alb_X$ of $X$ is the initial object in the category of pointed maps $(X, x_0) \to (A, e)$ where $A$ is an abelian variety.
\end{defn}

\begin{theorem}
Let $X$ be proper, geometrically integral, and geometrically normal. Then the abanese exists and $\Alb_X = (\fPic_{X/k}^\circ)_{\red}^\vee$.
\end{theorem}

\begin{proof}
Recall that $\fPic_{X/k}^\circ$ is a proper group scheme [Kleinman, Theorem 5.4]. Thus $(\fPic^\circ_{X/k})_{\red}$ is an abelian variety {\color{red} does this require $k$ perfect}. By interpreting $\fPic_{X/k}$ as the moduli space of line bundles rigidified along $x \in X$ there is a universal Poincare bundle $\cP$ on $\fPic_{X/k} \times_k X$. We restrict to $(\fPic_{X/k}^\circ)_{\red} \times_k X$ which defines a morphism $X \to \Alb_X$ sending $x \mapsto e$ where $\Alb_X := (\fPic_{X/k}^\circ)_{\red}^\vee$. We need to check that $\Alb_X$ satisfies the universal property. Indeed, let $f : (X, x_0) \to (A, e)$ be a pointed morphism. Then we get a morphism $f^\vee : A^\vee \to \fPic_{X/k}^\circ$ and hence $A^\vee \to (\fPic_{X/k}^\circ)_{\red}$. We need to show that the dual morphism $\Alb_X \to A$ recovers $(X, x_0) \to (A, e)$ as the composition $X \to \Alb_X \to A$. By definition, $a : X \to \Alb_X$ satisfies $(a \times \id)^* \cP_{\Alb_X} = \cP_X$. It suffices to show that the following diagram commutes,
\begin{center}
\begin{tikzcd}
(X \times_k A^\vee, \L) \arrow[d, "f \times \id"] \arrow[r, equals] & (X \times A^\vee, \L) \arrow[r, "\id \times f^\vee"] \arrow[d, "a \times \id"] & (X \times (\fPic_{X}^\circ)_{\red}, \cP_X) \arrow[d, "a \times \id"]
\\
(A \times_k A^\vee, \cP_A) \arrow[from=r, "\tilde{f} \times \id"] & (\Alb_X \times A^\vee, \M) \arrow[r, "\id \times f^\vee"'] & (\Alb_X \times (\fPic_X^\circ)_{\red}, \cP_{\Alb_X})
\end{tikzcd}
\end{center}  
where $\L := (\id \times f^\vee)^* \cP_X$ and $\M := (\id \times f^\vee)^* \cP_{\Alb_X}$. Commutativity of the right hand square is obvious. Note that $(\tilde{f} \times \id)^* \cP_A = \M$ by the previous lemma. Therefore, by commutativity of the right square
\[ \L = (\id \times f^\vee)^* \cP_X = (a \times \id)^* (\id \times f^\vee)^* \cP_{\Alb} = (a \times \id)^* \M = (\tilde{f} \times \id)^* \cP_A \]
Hence we see that
\[ (a \times \id)^* (\tilde{f} \times \id)^* \cP_A = (f \times \id)^* \cP_A \]
so by definition of $A^\vee$ this means the classifing maps of these bundles $f : X \to A$ and $\tilde{f} \circ a : X \to A$ must be equal.
\bigskip\\
This argument is \chref{http://tinyurl.com/4333a3ac}{stolen from here}.
\end{proof}

\subsection{The Relative Case}

\begin{rmk}
Note that there may not exist a map $X \to \Alb_{X/S}$ of $S$-schemes unless $X \to S$ has a section. Indeed, we need a section in order to define the Poincare bundle which is usually used to define the morphism. The Albanese is best presented as an initial object in the category of \textit{pointed} maps to abelian schemes.
\end{rmk}

\begin{lemma}
Let $f : X \to Y$ be a morphism of pointed\footnote{meaning $X$ and $Y$ are equipped with sections $\sigma_X$ and $\sigma_Y$ over $S$ such that $f \circ \sigma_X = \sigma_Y$} $S$-schemes such that $X^\vee = \fPic_{X/S}^\circ$ and $Y^\vee = \fPic_{Y/S}^\circ$ exist as algebraic spaces over $S$. Let $\phi^\vee : Y^\vee \to X^\vee$ be the dual. Consider the maps
\[ X \times_S X^\vee \leftarrow X \times_S Y^\vee \to Y \times_k Y^\vee \]
then there is a canonical isomorphism of rigidified sheaves on $X \times_S Y^\vee$
\[ (\id_X \times f^\vee)^* \cP_X \cong (f \times \id_{Y^\vee})^* \cP_Y \]
\end{lemma}

\begin{proof}
Again, immediate from the definition of $f^\vee$ as the classifying map for the bundle $(f \times \id_{Y^\vee}) \cP_Y$ which is basically the usual definition. 
\end{proof}

\begin{prop}
Let $\pi : X \to S$ be a morphism with a section $\sigma : S \to X$ such that both $\fPic_{X/S}^\circ$ and $\Alb_{X/S} := \fPic_{\fPic_{X/S}^\circ}^\circ$ exist\footnote{Here we refer to the Picard scheme rigidified along the section $\sigma$ so that it admits a universal bundle. I expect this will imply that $f$ and $\fPic_{X/S}$ are both proper but the proof is basically formal given the universal properties.} (as algebraic spaces). Then for any abelian scheme $\cA \to S$ and a pointed $S$-morphism $f : (X, \sigma) \to (\cA, e)$ there is a unique factorization 
\[ X \xrightarrow{a} \Alb_{X/S} \xrightarrow{\tilde{f}} \cA \]
\end{prop}

\begin{proof}
The proof is basically idential. The Poincare bundle $\cP_X$ on $X \times_S \fPic^\circ_{X/S}$ defines a morphism $a : X \to \Alb_{X/S}$ such that $(a \times \id)^* \cP_{\Alb_{X/S}} = \cP_X$. From here, the proof is identical. We define the morphism $f^\vee : \cA^\vee \to \fPic_{X/S}^\circ$ and then $\tilde{f}$ is its dual. We need to show that the dual morphism $\Alb_{X/S} \to \cA$ recovers $(X, \sigma) \to (\cA, e)$ as the composition $X \to \Alb_{X/S} \to \cA$. It suffices to show that the following diagram commutes,
\begin{center}
\begin{tikzcd}
(X \times_S \cA^\vee, \L) \arrow[d, "f \times \id"] \arrow[r, equals] & (X \times_S \cA^\vee, \L) \arrow[r, "\id \times f^\vee"] \arrow[d, "a \times \id"] & (X \times_S \fPic_{X}^\circ, \cP_X) \arrow[d, "a \times \id"]
\\
(\cA \times_S \cA^\vee, \cP_A) \arrow[from=r, "\tilde{f} \times \id"] & (\Alb_X \times_S \cA^\vee, \M) \arrow[r, "\id \times f^\vee"'] & (\Alb_X \times_S \fPic_X^\circ, \cP_{\Alb_X})
\end{tikzcd}
\end{center}  
where $\L := (\id \times f^\vee)^* \cP_X$ and $\M := (\id \times f^\vee)^* \cP_{\Alb_X}$. Commutativity of the right hand square is obvious. Note that $(\tilde{f} \times \id)^* \cP_A = \M$ by the previous lemma. Therefore, by commutativity of the right square
\[ \L = (\id \times f^\vee)^* \cP_X = (a \times \id)^* (\id \times f^\vee)^* \cP_{\Alb} = (a \times \id)^* \M = (\tilde{f} \times \id)^* \cP_A \]
Hence we see that
\[ (a \times \id)^* (\tilde{f} \times \id)^* \cP_A = (f \times \id)^* \cP_A \]
so by definition of $\cA^\vee$ this means the classifing maps of these bundles $f : X \to \cA$ and $\tilde{f} \circ a : X \to \cA$ must be equal. 
\end{proof}

\begin{cor}
Let $X$ be a proper, geometrically integral, and geometrically normal $k$-variety and $A$ an abelian variety. Choose $x_0 \in X(k)$. Then $\Hom{k}{(X, x_0)}{(A, e)} = \Hom{k\text{-grp}}{\Alb_X}{A}$ and $\Hom{k}{X}{A} = \Hom{k}{\Alb_X}{A}$. If $k$ has characteristic zero, the latter is isomorphic to a disjoint union of copies of $A$. 
\end{cor}

\begin{proof}
For any $T \to \Spec{k}$, apply the previous theorem to the pointed map $X_T \to A_T$ to produce the bijection at the level of $T$-points. In this generality, we know the Hom scheme is representable by an algebraic space. For the unpointed case, given any $f : X_T \to A_T$ we can translate by $f(x_0)$ to get a pointed map and hence get a factorization $X_T \to (\Alb_{X})_T \to A_T$ where the second map is translation by $f(x_0)$ composed with a homomorphism. This proves the second statement. In characteristic zero, $\Hom{k\text{-grp}}{\Alb_X}{A}$ is \etale over $k$ by considering lattices. Moreover, $A \times_k \Hom{k\text{-grp}}{\Alb_X}{A} \cong \Hom{k}{\Alb_X}{A}$ via the map $(a, f) \mapsto f + a$ because any map of abelian varities preserving the identity is a homomorphism.   
\end{proof}


\newcommand{\fHom}{\mathrm{Hom}}

\begin{lemma}
Let $G$ be a sheaf of groups on a site $\C$ and $T$ a sheaf with a $G$-action. Suppose that for any $U \in \C$ there is a covering $\{ U_i \to U \}$ such that $T|_{U_i}$ is a $G|_{U_i}$-torsor.
\end{lemma}

\begin{proof}
It suffices to show that if $T(U)$ is nonempty then $G(U) \acts T(U)$ simply transitively. Let $x,y \in T(U)$ for each $U_i$ there is a unique $g_i \in T(U_i)$ such that $g_i \cdot x|_{U_i} = y_{U_i}$. By the uniqueness applied to $U_{ij} = U_i \times_U U_j$ we see that 
\end{proof}

\begin{rmk}
Note that freeness cannot be checked locally. Indeed, 
\end{rmk}

\begin{prop}
Let $\pi : X \to S$ be a flat proper morphism (now without a section) such that both $\fPic_{X/S}^\circ$ and $\Alb_{X/S} := \fPic_{\fPic_{X/S}^\circ}^\circ$ exist (as algebraic spaces) and are abelian schemes. Consider, the sheaf,
\[ \F = \fHom_S^1(X, \Alb_{X/S}) \] 
which is given by 
\[ T \mapsto \{f \in \Hom{T}{X_T}{(\Alb_{X/S})_T} \mid f^\vee : (\Alb_{X/S})_T^\vee \to \fPic_{X_T/T}^\circ \text{ is the canonical isomorphism} \} \]
This is an open subscheme of $\Hom{S}{X}{\Alb_{X/S}}$ if this is representable. Moreover, it is a torsor for $\Alb_{X/S}$ in the fppf-topology.
\end{prop}

\begin{proof}
The openess follows from the openessness of the locus where $f^\vee : (\Alb_{X/S})_T^\vee \to \fPic^\circ_{X_T/T}$ is an isomorphism over $T$ because this is a map of smooth proper $T$-schemes. The action $\Alb_{X/S} \acts \F$ gives $\F$ the structure of an $\Alb_{X/S}$-torsor because fppf-locally $\pi :X \to S$ acquires a section and thus by the previous results $\Alb_{X/S}$ show that $\F$ is nonemtpy and the action is simply transitive by the corresponding fact for abelian varities {\color{red} DO THIS}
\end{proof}

Since $\Alb_{X/S}$ is an abelian scheme, any torsor over it is smooth and hence trivial in the \etale topology.

\section{Examples}

Consider $C$ a hyperelliptic curve, $E$ an elliptic curve $G = \Z/2$ and let $X = (C \times E)/G$ where the action is via $(x,y) \mapsto (\iota(x), y + p)$ where $\iota : C \to C$ is the hyperelliptic involution and $p \in E[2]$. Then consider $\pi : X \to \P^1$ whose fibers are $E$ over $t \notin B$ with $B \subset \P^1$ the branch locus of $C \to \P^1$. 
\bigskip\\
Note that $f : X \to \P^1$ is flat, proper, and finitely presented. Moreover, by Raynaud's theorem (or direct calculation that can be done with Claim 1) we have $f_* \struct{X} = \struct{\P^1}$ universally. Therefore, $\fPic_{X/\P^1}$ exists as algebraic spaces. Moreover, since $H^2(X_t, \struct{X_t}) = 0$ it is a smooth algebraic space over $S$. The by [EGA IV$_3$, 15.6.5] $\fPic_{X/\P^1}^\circ$ is represented by an open subspace. We will see that they are far from separated.

\subsubsection{Claim 1}

\newcommand{\br}[1]{\llbracket #1 \rrbracket}

The fibers over $t \in B$ are a nonreduced structure on $\wt{E} := E / \left< p \right>$ defined by the split extension of $\struct{\wt{E}}$ by $\L$ corresponding to the unique nontrivial $2$-torsion point of $\wt{E}$ in the image of $E[2]$.
\bigskip\\
Analytically locally near the branch point, the morphism has the form $(E \times \Spec{k\br{s}})/G \to \Spec{k\br{t}}$ where $t \mapsto s^2$. The structure sheaf of the quotient is given by the $G$-invariants in
\[ \struct{E} \oplus s \struct{E} \]
where $\sigma \in G$ acts by $s \mapsto -s$ and on $\struct{E}$ by translation by $p$. Therefore, the $s$-component must correspond to a nontrivial $G$-equivariant line bundle on $E$ which is $2$-torsion and whose underlying line bundle is $\struct{E}$ which desends to the line bundle $\L$ extending $\struct{\wt{E}}$. However, 
\[ \ker{(\fPic(\wt{E}) \iso \fPic^G(E) \to \fPic(E))} = \ker{(\wt{E} \to E)} \]  
and the unique nontrivial point of this kernel is the dual $2$-torsion point corresponding to $\L$ as claimed.

\subsubsection{Claim 2}

The fibers of $\fPic_{X/\P^1}^\circ$ are
\[ \fPic_{X/\P^1}^\circ \cong
\begin{cases}
E & t \notin B
\\
\wt{E} & t \in B
\end{cases} \]
Indeed, $X_t \cong E$ for $t \notin B$ and otherwise we need to compute $\fPic^\circ_{X_t}$ where $X_t$ is the nonreduced structure on $\wt{E}$. However, we have the sequence
\begin{center}
\begin{tikzcd}
0 \arrow[r] & \L \arrow[r] & \struct{X_t}^\times \arrow[r] & \struct{\wt{E}}^\times \arrow[r] & 0
\end{tikzcd}
\end{center}
but $H^1(\L) = 0$ so $\fPic_{X_t}^\circ = \wt{E}$ for $t \in B$.

\subsubsection{Claim 3}

$\fPic^\circ_{X/\P^1}$ is not separated. Indeed, consder the divisor $D \subset X$ given by the reduction of some double fiber. Then $\struct{X}(D)$ is trivial away from this fiber but $\struct{X}(D)|_D = \L^\vee$ because this is the normal bundle to $D = (X_t)_{\red} \subset X_t$. This is nontrivial in the special fiber and hence this shows that the Picard scheme is not separated. 

\subsubsection{Claim 4}

There is an \etale equivalence relation pushout diagram,
\begin{center}
\begin{tikzcd}
\wt{E} \times (\P^1 \sm B) \arrow[r] \arrow[d] & \wt{E} \times \P^1 \arrow[d]
\\
\wt{E} \times \P^1 \arrow[r] & \fPic^\circ_{X/\P^1}
\end{tikzcd}
\end{center}
where the downward map is $(x,y) \mapsto (x + \hat{p}, y)$ where $\hat{p} \in \wt{E}[2]$ is the dual $2$-torsion point to $p$. Moreover, we can conisder the map $X \to \wt{E}$ whcih produces a dual map
\[ \wt{E} \times \P^1 \to \fPic^\circ_{X/\P^1} \]
on fibers this is the dual map $\wt{E} \to E$ for $t \notin B$ and an isomorphism $\wt{E} \to \wt{E}$ for $t \in B$. This also shows nonseparatedness because of the following result.

\begin{prop}
Let $f : X \to Y$ be a morphism of smooth proper $S$-schemes whose fibers are connected. Then the locus over $S$ where $f$ is finite is open and $s \mapsto \deg{f_s}$ is locally constant.
\end{prop}

\begin{proof}
Openness of the finite locus is given by [EGA III$_1$, Prop. 4.6.7(i)] flat (by fibral flatness). Therefore, $f_* \struct{X}$ is a vector bundle whose formation is compatible with base change and whose rank at a point computes the degree over that point which is therefore constant. 
\end{proof}

The claim is that this fails if we only assume the schemes are universally closed and not separated. 

\begin{example}
Let $X$ be the affine line two origins and $X \to \A^1$ the projection. As a map of $\A^1$-schemes the fiber degree jumps from $1$ to $2$. Likewise the map $\A^1 \sqcup \A^1 \to X$ has fiber degree jumping from $2$ to $1$. Indeed, in these examples $f_* \struct{X}$ does not correctly compute the fiber degree at nonseparated points. 
\end{example}

\section{Torsors For Abelian Schemes}

\begin{theorem}[Raynaud's Thesis, XIII 2.8.]
Let $X$ be a torsor over an abelian scheme $\cA \to S$. If $S$ is a regular and noetherian then there exists a finite \etale cover $S' \to S$ such that $X_{S'}$ is trivial.
\end{theorem}



\section{Proof of The Main Theorem}

In this section we work over $\CC$.

\begin{prop} \label{prop:inf_auts}
Let $X$ be a projective klt variety with $K_X \sim_{\Q} 0$ and $H^1(X, \struct{X}) = 0$. Then $H^1(X, \T_X) = 0$ meaning that $\mathrm{Aut}_X$ is an \etale $k$-group.
\end{prop}

{\color{red} Maybe we only know how to prove this if $K_X = 0$.}

\begin{defn}
Consider the following situation $(*)$
\begin{center}
$f : X \to S$ is a flat projective morphism whose fibers are normal varities such that $R^1 f_* \struct{X}$ is a vector bundle whose formation commutes with all base changes.
\end{center}
\end{defn}

\begin{lemma}
In situation $(*)$ then $\fPic_{X/S}^\circ \to S$ is representable by an abelian scheme. In this case we define $\Alb_{X/S} := (\fPic_{X/S}^{\circ})^\vee$.
\end{lemma}

\begin{proof}
{\color{red} ANDRES POINTED OUT A GOOD PROOF}
\end{proof}


\begin{theorem}[Ambro]
Let $(X, B)$ be a projective log variety with klt singularities such that $K + B \equiv_{\text{num}} 0$. Then,
\begin{enumerate}
\item there exists a positive integer $b$ such that $b(K_X + B) \sim 0$ 
\item The Albanese map $X \to \Alb_X$ is surjective with connected fibers. Furthermore, there exists an \etale covering $A' \to \Alb_X$, a projective log variety $(F, B_F)$ and an isomorphism
\begin{center}
\begin{tikzcd}
(X, B) \times_{\Alb_X} A' \arrow[rd] \arrow[rr, "\sim"] & & (F, B_F) \times A' \arrow[ld]
\\
& A'
\end{tikzcd}
\end{center}
\end{enumerate}
\end{theorem}


\newcommand{\Isom}{\mathrm{Isom}}

\begin{theorem}
Let $f : X \to S$ be a flat projective morphism of varities over $\CC$ such that
\begin{enumerate}
\item each fiber $X_s$ is normal, klt, $\Q$-factorial and has $K_{X_s} \equiv_{\text{num}} 0$ 
\item $S$ is regular
\end{enumerate}
then there exists a finite \etale cover $S' \to S$ and a finite \etale cover $X' \to X_{S'}$ such that as $S'$-schemes there is an isomorphism $X' \cong F \times_{S'} \cA$ where $\cA \to S'$ is an abelian scheme and $F \to S'$ is a flat projective morphism satisfying the same properties in the assumptions of the theorem and additionally for any finite \etale cover $Y \to F_s$ of a fiber we have $H^1(Y, \struct{Y}) = 0$.
\end{theorem}

\begin{proof}
Note that because we assume that $X_s$ is geometrically integral and since klt implies Du Bois {\color{red} cite} by [CITE DUBOIS CONSTANCY] we see that $X \to S$ satisfies $(*)$.

\subsubsection{Step 1}
We want to produce a map $X \to \Alb_{X/S}$. We can only do this if we can find a section of
\[ \fHom_S^1(X, \Alb_{X/S}) \]
However, this is a torsor over an abelian scheme $\Alb_{X/S}$ and $S$ is regular so by Raynaud, this torsor is killed by a finite \etale cover $S_1 \to S$. Replacing $X$ by $X_{S_1}$ we may assume that there is an morphism $X \to \Alb_{X/S}$ whose fibers $X_s \to \Alb_{X_s}$ are choices of albanese maps (note that we are \textit{not} claiming the existence of a section of $X \to S$, the choices of points $x_s \in X_s$ such that $X_s \to \Alb_{X_s}$ sends $x_s \mapsto e$ may not be made in a compatible way, indeed $X \to S$ may not have a section \textit{finite}-etale locally on $S$ even in the case it is smooth\footnote{For example, consider $\P(\Omega_{\P^2}) \to \P^2$ which has no sections and $\P^2$ is simply connected. Of course its relative albanese is trivial and hence it admits an Albanese morphism}.
\bigskip\\
Set $\cA := \Alb_{X/S}$ and consider the diagram,
\begin{center}
\begin{tikzcd}
F \arrow[r] \arrow[rd] & X \arrow[d] \arrow[r, "a"] & \cA \arrow[ld]
\\
& S
\end{tikzcd}
\end{center}
where $F = X \times_{\cA} e_{\cA}$ is the fiber over the zero section. We claim that $F \to S$ is flat. Indeed, by fibral flatness $a$ is flat since its fibers $X_s \to \Alb_{X_s}$ are actually isotrivial, hence flat, by Ambro's theorem. Therefore, the base change $F \to S$ of $a : X \to \cA$ is also flat.

\subsubsection{Step 2}

Ambro's result applied to each fiber proves that $F$ has normal, klt, $\Q$-factorial fibers and satisfies $K_{F_s} \sim_{\Q} 0$ and hence $F \to S$ also satisfies $(*)$.
Choose a point $s_0 \in S$. By Ambro's theorem there is a finite \etale cover $A' \to \Alb_{X_{s_0}}$ splitting $a_{s_0}$. 

\begin{lemma}
Let $\varphi : A \to B$ be an isogeny of abelian varities. Then there is a futher isogeny $B \to A$ such that the composition $B \to A \to B$ is $[n]$ for some\footnote{We can only take $n = \deg{(A \to B)}$ if $A$ and $B$ are principally polarized} integer $n$.
\end{lemma}

\begin{proof}
Consider $\varphi^{\vee} : B^\vee \to A^{\vee}$ and choose a polarization $A^{\vee} \iso A$ then the composition gives $B^\vee \to B$ and composing with a polarization gives $B \to B^\vee \to B$. Therefore, we reduce to the statment for a map $\varphi : B \to B$. Since $\ker{\varphi^{\vee}}$ is a torsion finite abelian group scheme there is $n$ such that $n \cdot \ker{\varphi^{\vee}}$ hence $\ker{\varphi^{\vee}} \subset \ker{[n]}$ so there is a factorization
\begin{center}
\begin{tikzcd}
B^\vee \arrow[r, "\times n"] \arrow[d,"\varphi^\vee"'] & B^\vee
\\
B^\vee \arrow[ru, dashed, "\psi"]
\end{tikzcd}
\end{center}
hence $[n] = \psi \circ \varphi^\vee$ and therefore dualizing $[n] = [n]^\vee = \varphi \circ \psi^\vee$ proves the claim.
\end{proof}

Using this lemma, we can take the covering in Ambro's theorem to be multiplication by $n$. Hence consider the diagram
\begin{center}
\begin{tikzcd}
F \arrow[d, equals] \arrow[r] & X' \pullback \arrow[d, "\et"] \arrow[r] & \cA \arrow[d, "\times n"]
\\
F \arrow[r] & X \arrow[r, "a"] & \cA
\end{tikzcd}
\end{center}
such that on the restriction to the fiber over $X_{s_0}$ there exists an isomorphism $X'_{s_0} \cong \cA_{s_0} \times F_{s_0}$ preserving the diagram. Likewise this proves that $F$ has normal, klt, $\Q$-factorial fibers and satisfies $K_{F_{s_0}} \sim_{\Q} 0$. It is not necessarily\footnote{For example, consider a surface of the form $X = (E \times E')/(\Z / 2 \Z)$ where the group acts via $(x,y) \mapsto (-x, y + p)$ for $p \in E'[2]$. This is a free action so $X$ is smooth and its fibers over $S = \P^1 \sm B$ are $E'$. Then $X \to \Alb_{X_S/S} = E' / (\Z / 2 \Z)$ has fibers $E$ which do not have vanishing $h^{0,1}$.} true that $H^1(F_s, \struct{F_s}) = 0$. However, we can now iterate the steps so far to the family $F \to S$ to get $F' \to S$ and if $H^1(F_s, \struct{F_s}) \neq 0$ then $\dim{F'} < \dim{F}$ so this process must terminate after finitely many steps. Hence we can assume that $H^1(F_s, \struct{F_s}) = 0$ for all $s \in S$ because of its constancy over $S$.

\subsubsection{Step 3}

Consider functor sending $T \to S$ to the set of isomorphisms $\varphi : X_T \to (F \times_S \cA)_T$ making the diagram
\begin{center}
\begin{tikzcd}
F_T \arrow[r, equals] \arrow[d] & F_T \arrow[d, "\id \times e"]
\\
X_T \arrow[r, "\varphi"] \arrow[d] & (F \times_S \cA)_T \arrow[d]
\\
\cA \arrow[r, equals] & \cA
\end{tikzcd}
\end{center}
commute.
These are ``upper-triangular'' isomorphisms. We denote this by
\[ \Isom^{F,\cA}_S(X, F \times_S \cA) \embed \Isom_S(X, F \times_S \cA) \]
Notice that the above is a closed embedding because these $S$-schemes are separated and hence the equalizer of two morphisms is a closed subscheme. 

\subsubsection{Some Deformation Theory}


\begin{lemma} \label{def_theory}
Let $A$ be an Artin local ring with residue field $\kappa$. Let $X_A$ be a smooth $A$-scheme and $C_A, B_A$ be any $A$-schemes and the data,
\begin{enumerate}
\item $A$-morphisms $g_A : B_A \to C_A$ and $f_A : C_A \to X_A$ with $g_A$ a closed embedding

\item a small extension of Artin local rings,
\[ 0 \to I \to A' \to A \to 0 \]

\item deformations $B_{A'}$ of $B_A$ and $C_{A'}$ of $C_A$ and $X_{A'}$ of $X_A$ over $A'$

\item deformations $g_{A'} : B_{A'} \to C_{A'}$ of $g_A$ and $h_{A'}$ of $h_{A} = f_A \circ g_A$
\end{enumerate}
then, denoting the data over $\kappa$ with a $0$, there exists a class,
\[ \ob(f_A) \in \Ext{1}{C_0}{f_0^* \Omega_{X_0}}{\I_{B_0}} \ot_k I \]
obstructing the existence of a map $f_{A'} : C_{A'} \to X_{A'}$ such that $f_{A'} \circ g_{A'} = h_{A'}$ where,
\[ \I_{B_0} = \ker{(\struct{C_0} \to g_{0*} \struct{B_0})} \]
\end{lemma}

\begin{proof}
Using the flatness properties, we need to consider the following lifting problem,
\begin{center}
\begin{tikzcd}
& & g_* \struct{B_{A'}} \arrow[from=dd, bend left = 70, near start, "h_{A'}^{\#}"] \arrow[r] & g_* \struct{B_A}
\\
0 \arrow[r] & I \ot \struct{C_0} \arrow[r, crossing over] & \struct{C_{A'}} \arrow[u] \arrow[r] & \struct{C_A} \arrow[r] \arrow[u] & 0
\\
& & f^{-1}_0 \struct{X_{A'}} \arrow[u, dashed]  \arrow[r] & f^{-1}_0 \struct{X_A} \arrow[u, "f_A^{\#}"] 
\end{tikzcd}
\end{center}
such that the diagram commutes. The set of such liftings locally on $C_0$ is a torsor over,
\[ \Der[A']{f_0^{-1} \struct{X_{A'}}}{\I_{B_0} \ot_\kappa I} = \Der[k]{f_0^{-1} \struct{X_0}}{\I_{B_0}} \ot_\kappa I = \shHom{\struct{C_0}}{f_0^* \Omega_{X_0}}{\I_{B_0}} \ot_\kappa I \]
Therefore, it suffices to show that lifts exist locally. The local picture,
\begin{center}
\begin{tikzcd}
0 \arrow[r] & I \ot B_0 \arrow[r] & B' \arrow[from=dd, bend left = 70] \arrow[r] & B \arrow[r] & 0
\\
0 \arrow[r] & I \ot S_0 \arrow[u] \arrow[r, crossing over] & S' \arrow[u] \arrow[r] & S \arrow[r] \arrow[u] & 0
\\
& I \ot J_{B_0} \arrow[u] & R' \arrow[u, dashed, "\ell'"] \arrow[r] & R \arrow[u, "\ell"] 
\\
& 0 \arrow[u]
\end{tikzcd}
\end{center}
where $R'$ is smooth over $A'$ so there exists a lift $\ell' : R \to S'$ over $A'$ since $S' \to S$ is a square-zero extension. We need to modify the lift to make it commute with $R' \to B'$. The difference is an element $\delta \in \Hom{R'}{\Omega_{R'/A'}}{B_0 \ot_\kappa I}$  
so we consider the exact sequence,
\begin{center}
\begin{tikzcd}
\Hom{R'}{\Omega_{R'}}{S_0 \ot_\kappa I} \arrow[r] & \Hom{R'}{\Omega_{R'}}{B_0 \ot_\kappa I} \arrow[r] & \Ext{1}{R'}{\Omega_{R'/A'}}{J_{B_0} \ot_\kappa I}
\end{tikzcd}
\end{center}
which arises from tensoring the exact sequence (using that $g_A$ is a closed embedding for surjectivity)
\begin{center}
\begin{tikzcd}
0 \arrow[r] & J_{B_0} \arrow[r] & S_0 \arrow[r] & B_0 \arrow[r] & 0
\end{tikzcd}
\end{center}
by $- \ot_\kappa I$ which remains exact since $\kappa$ is a field and applying $\Hom{R'}{\Omega_{R'/A'}}{-}$. Now $\Omega_{R'/A'}$ is a projective $R'$-module so the Ext is zero and hence we can lift $\delta$ to a derivation $\tilde{\delta} : R' \to S_0 \ot_\kappa I$ then $\ell' - \tilde{\delta}$ gives the required lift. 
\end{proof}

\begin{lemma}
Given morphisms $Z \xrightarrow{\iota} X,Y$ and $X,Y \xrightarrow{\pi} W$ of flat proper $S$-schemes consider the algebraic space
\[ \Isom_S^{Z,W}(X, Y) \to S \]
representing the functor sending $T$ to the isomorphisms $\varphi : T_X \to Y_T$ such that the diagram
\begin{center}
\begin{tikzcd}
Z_T \arrow[r, equals] \arrow[d, "\iota_X"] & Z_T \arrow[d, "\iota_Y"]
\\
X_T \arrow[r, "\varphi"] \arrow[d, "\pi_X"] & Y_T \arrow[d, "\pi_Y"]
\\
W_T \arrow[r, equals] & W_T
\end{tikzcd}
\end{center}
commutes. Suppose that
\[ \Hom{\struct{X_s}}{\Omega_{X_s/W_s}}{\J} = 0 \]
where $\J$ is the ideal sheaf of $Z_s \embed X_s$
then the structure map $\Isom_S^{Z,W}(X, Y) \to S$ is unramified in a neighborhood of $s \in S$. Moreover, there are iterated obstructions
\[ \ob_1(\varphi) \in H^0(\shExt{1}{\struct{X_s}}{\Omega_{X_s/W_s}}{\J}) \]
and 
\[ \ob_2(\varphi) \in H^1(\shHom{\struct{X_s}}{\Omega_{X_s/W_s}}{\J}) \]
whose vanishing ensures that the structure map is also \etale in a neighborhood of $s \in S$.
\end{lemma}

\begin{proof}
We check the valuative criterion for unramifiedness. Suppose we are given the following data
\begin{enumerate}
\item a square-zero extension of Artin rings $0 \to I \to A' \to A \to 0$
\item a map $\Spec{A} \to S$
\item an isomorphism $\varphi : X_A \iso Y_A$ making the requisite diagrams commute
\end{enumerate}
we need to show there is at most one extension to an isomorphism $\varphi' : X_{A'} \iso Y_{A'}$ making the requisite diagrams commute. Let $s \in S$ be the image of $\Spec{A} \to S$ and $\kappa$ the residue field. We use $0$ to denote the base change to $\kappa$. The set of $\varphi'$ correspond to dashed arrows in the diagram
\begin{center}
\begin{tikzcd}
0 \arrow[r] & I \ot \iota_* \struct{Z_0} \arrow[r] & \iota_{*} \struct{Z_{A'}} \arrow[from=dd, bend left = 70, near end, "\iota_Y^{\#}"] \arrow[r] & \iota_* \struct{Z_A} \arrow[r] & 0
\\
0 \arrow[r] & I \ot \struct{X_0} \arrow[u] \arrow[r, crossing over]  & \struct{X_{A'}} \arrow[from=dd, bend left = 70, near start, "\pi_X^{\#}", crossing over] \arrow[u] \arrow[r] & \struct{X_A} \arrow[r] \arrow[u] & 0
\\
& & \varphi^{-1}_0 \struct{Y_{A'}} \arrow[u, dashed]  \arrow[r] & \varphi^{-1}_0 \struct{Y_A} \arrow[u, "\varphi^{\#}"] 
\\
& & \pi^{-1}_Y \struct{W_{A'}} \arrow[u]  \arrow[r] & \pi_Y^{-1} \struct{W_A} \arrow[u, "\varphi^{\#}"] 
\end{tikzcd}
\end{center}
The difference $\delta = \varphi'_1 - \varphi'_2$ is a derivation $\varphi_0^{-1} \struct{Y_{A'}} \to I \ot \struct{X_0}$ landing inside $I \ot \J_{Z_0}$ which is zero on $\pi^{-1}_Y \struct{W_{A'}}$. Therefore, this is a torsor over
\[ \Der[\pi_Y^{-1} \struct{W_{A'}}]{\varphi_0^{-1} \struct{Y_{A'}}}{I \ot \J_{Z_0}} = \Der[\pi_U^{-1} \struct{W_0}]{\struct{Y_0}}{\I_{Z_0}} \ot I = \Hom{\struct{X_0}}{\varphi_0^{*} \Omega_{Y_0/W_0}}{\J_0} \ot I \]
Because $\varphi_0$ is an isomorphism, the assumption applies to show there is at most one extension. Since the space of lifts is a pseudo-torsor over
\[ \shHom{\struct{X_s}}{\Omega_{X_s/W_s}}{\J} \]
it suffices to show there exists such a class $\ob_1(\varphi)$ whose vanishing is equivalent to the existence of $\varphi'$ locally on $X_A$. {\color{red} TODO}
\end{proof}

Hence applying this result to our situation, we need to check that $\Hom{\struct{F_s \times \cA_s}}{\Omega_{F_s}}{\struct{F_s \times \cA_s}} =  0$. By the Kunneth formula, $H^1(F_s, \T_{F_s}) = 0$ which is the content of Proposition~\ref{prop:inf_auts}.

\subsubsection{(TODO) Step 5 Deformation Theory of Products}

We need to prove the following claim:

\newcommand{\Def}{\mathrm{Def}}

\begin{prop}
Let $X,Y$ be proper geometrically connected and reduced $k$-schemes with
\begin{enumerate}
\item $H^1(Y, \struct{Y}) = 0$
\item $H^0(Y, \T_Y) = 0$
\end{enumerate}
Then the canonical map of deformation functors
\[ \Def_X \times \Def_Y \to \Def_{X \times Y} \]
is an isomorphism.
\end{prop}

{\color{red} TODO, also is it automatic that the splitting of the deformation must be compatible with the maps $F \to X \to \cA$ }

\begin{cor}
The algebraic space $\Isom_S^{F,\cA}(X, F \times_S \cA) \to S$
is \etale over $S$.
\end{cor}

\subsubsection{Step 6}

By the previous result, the image of the Isom scheme is open, hence we have split $X$ not only at $s_0$ but in an open neighborhood also. Therefore, to each point $s \in S$ we can associated a pair $(n_s, U_s)$ of an integer $n_s > 0$ and an open $U_s$ containing $s$ such that after base change along $[n_s] : \cA \to \cA$ there is a splitting over $U_s$. Since $S$ is quasi-compact, there is a finite set $\{ s_1, \dots, s_r \}$ such that $U_1, \dots, U_r$ cover $S$. Take $n = n_1 \cdots n_r$ then base changing over 

\begin{prop}
The fibers of $\Isom_S^{F,\cA}(X, F \times_S \cA) \to S$ are either empty or are exactly one reduced point. 
\end{prop}

\begin{proof}
Since it is \etale, it suffices to show that there is a unqiue $k$-point in each fiber. This is equivalent to showing that there is a unique automorphism $\varphi : F_s \times \cA_s \to F_s \times \cA_s$ compatible with the projection to $\cA_s$ and the inclusion of $F_s$ at the zero section. The former condition means that $\pi_2 \circ \varphi : F_s \times \cA_s \to \cA_s$ is $\pi_2$ hence it suffices to prove that $\pi_1 \circ \varphi = \pi_1$. However, $\pi_1 \circ \varphi : F_s \times \cA_s \to F_s$ is the same as a morphism $\cA_s \to \Hom{}{F_s}{F_s}$ and the condition that $\varphi \circ (\id \times e) = \id \times e$ implies that $e \mapsto \id$. Since $\mathrm{Aut}_{F_s}$ is \etale and $\cA_s$ is connected we conclude that the classifying map is constant and therefore $\pi_1 \circ \varphi = \pi_1$. 
\end{proof}

Therefore $\Isom_S^{F,\cA}(X, F \times_S \cA) \to S$ is radicial and unramified hence a monomorphism. Furthermore a flat monomorphism locally of finite presentation is an open immersion. However, after base change along $[n] : \cA \to \cA$ we know that $\Isom_S^{F,\cA}(X, F \times_S \cA) \to S$ is surjective and therefore is an isomorphism. Hence $X \cong F \times_S \cA$ compatibly with the extension structure.

\end{proof}

\section{Goals}

\subsection{klt Log Pairs}

Would want a statement of the form: if $(X, D) \to S$ is a family of klt Calabi-Yau pairs, meaning $(X_s, D_s)$ is klt and $K_{X_s} + D_s \sim_{\Q} 0$ (or numerically equivalent to zero?) then there are finite \etale covers $S' \to S$ and $X' \to X_{S'}$ such that $X' \to S'$ is a product of $\cA \to S'$ an abelian scheme and $(F, D_F)$ which satisies all the same good properties as $(X, D)$.
\bigskip\\
It sounds like we should be able to apply this somehow to moduli problems of CY pairs. I don't know.

To make the argument work we better check the various steps
\begin{enumerate}
\item Ambro's result is already in this form
\item vital part is the vanishing of infinitesimal deformations, I think this still works (see below)
\item how do we construct $D_F$ is it just $D_F := F \cap D$ we should look at Ambro's proof to see if this works on fibers
\item the isom scheme arguments will need minor modifications to take into account the $D$.
\end{enumerate}

\begin{lemma}
Let $(X, D)$ be a log-CY klt pair with $H^1(X, \struct{X}) = 0$ and $D$ a $\Z$-divisor (I'm not sure what the correct thing is if $K_X$ is only a $\Q$-divisor). Then $H^0(X, \T_X(-D)) = 0$.
\end{lemma}

\newcommand{\reg}{\mathrm{reg}}

\begin{proof}
Indeed, the canonical pairing $\Omega_{X^{\reg}} \ot \Omega_{X^{\reg}}^{n-1} \to \omega_{X^{\reg}}$ over the regular locus induces an isomorphism of reflexive sheaves $\T_X(-D) \iso \Omega_X^{[n-1]}(-K_X - D) = \Omega_X^{[n-1]}$. Consider a log resolution $\pi : \wt{X} \to X$. By [Greb Kebekus Diff Forms LC Spaces, Theorem 1.5] there is an extension $H^0(X, \Omega_X^{[n-1]}) = H^0(\wt{X}, \Omega_{\wt{X}}^{[n-1]})$. By Hodge theory, the dimension of this space is the same as the dimension of $H^1(\wt{X}, \struct{\wt{X}}) = H^1(X, \struct{X})$ {\color{red} THIS IS WRONG GET $H^1(\wt{X}, \omega_{\wt{X}}) = H^1(X, \struct{X})$} using that $X$ has rational singularities {\color{red} this is still true for klt pairs right? YES} so we just need the assumption $H^1(X, \struct{X}) = 0$.
\end{proof}

\subsection{Applications to the Iitaka fibration}

\section{Deformation Theory}

\begin{lemma}[BHPS, Lemma 3.4]
Let $k$ be a field and $X,Y$ proper DM stacks over $k$. Assume that either
\begin{enumerate}
\item $\Ext{0}{X}{\LL_X}{\struct{X}} = 0$ i.e. $X$ has no infinitesimal automorphisms or
\item $H^1(Y, \struct{Y}) = 0$ 
\end{enumerate}
and $H^0(X, \struct{X}) = k$. Then the natural map
\[ \Ext{i}{X}{\LL_X}{\struct{X}} \to \Ext{i}{X \times Y}{p_1^* \LL_X}{\struct{X \times Y}} \]
induces by pulling back along the projection $p_1 : X \times Y \to X$ is bijective for $i = 0,1$ and injective for $i = 2$.
\end{lemma}

\newcommand{\cQ}{\mathcal{Q}}

\begin{proof}
By the projection formula and adjointness
\[ \Ext{i}{X \times Y}{p_1^* \LL_X}{\struct{X \times Y}} = \Ext{i}{X}{\LL_X}{\RR p_{1*} \struct{X \times Y}} = \Ext{i}{X}{\LL_X}{\RR \Gamma(Y, \struct{Y}) \ot \struct{X}} \]
Consider the exact triangle
\[ \struct{X} \xrightarrow{u} \RR \Gamma(Y, \struct{Y}) \ot \struct{X} \to \cQ \to + 1 \]
where $\cQ$ is defined a the cone of $u$. Applying $\Hom{X}{\LL_X}{-}$ gives
\[ \Ext{i}{X}{\LL_X}{\cQ[-1]} \to \Ext{i}{X}{\LL_X}{\struct{X}} \to \Ext{i}{X \times Y}{p_1^* \LL_X}{\struct{X \times Y}} \to \Ext{i}{X}{\LL_X}{\cQ} \]
Thus, it suffies to check that $\Ext{i}{X}{\LL_X}{\cQ} = 0$ for $i \le 1$. Because $H^0(X, \struct{X}) = k$ the map $\H^0(u)$ is an isomorphism so $\cQ \in D^{\ge 1}(X)$. Therefore, since $\LL_X$ is connective we immediatley see that $\Ext{0}{X}{\LL_X}{\cQ} = 0$. To check that $\Ext{1}{X}{\LL_X}{\cQ} = 0$ note that the exact trangle
\[ \tau_{\ge 2} \cQ[-1] \to \H^1(\cQ)[-1] \to \cQ \]
shows that $\Ext{1}{X}{\LL_X}{\cQ} = \Ext{0}{X}{\LL_X}{\H^1(\cQ)}$. By construction $\H^1(\cQ) = H^1(Y, \struct{Y}) \ot \struct{X}$ is a free $\struct{X}$-module. Therefore, its vanishing follows by either the assumption $H^1(Y, \struct{Y}) = 0$ or by the assumption that $\Ext{0}{X}{\LL_X}{\struct{X}} = 0$.  
\end{proof}

\begin{cor} \label{cor:ext_product_basechange_asymmetric}
Therefore if $Y$ satisfies the \textit{both} the following properties
\begin{enumerate}
\item $\Ext{0}{Y}{\LL_Y}{\struct{Y}} = 0$ i.e. $Y$ has no infinitesimal automorphisms \textit{and}
\item $H^1(Y, \struct{Y}) = 0$
\end{enumerate}
and both $X,Y$ have $H^0(\struct{}) = k$ then the previous lemma applies to $X,Y$ in either order. 
\end{cor}

\begin{proof}
For the projections $p_1 : X \times Y \to Y$ and $p_2 : X \times Y \to Y$ we get the statement for $p_1$ by using the fact that $H^1(Y, \struct{Y}) = 0$. Swapping the roles of $X$ and $Y$ we get the statement for $p_2$ because $Y$ has not infinitesimal automorphisms. 
\end{proof}


\newcommand{\Def}{\mathrm{Def}}
\newcommand{\Art}{\mathrm{Art}}

\begin{prop}
Let $X,Y$ be proper DM-stacks over $k$ with $H^0(X, \struct{X}) = H^0(Y, \struct{Y}) = k$ and suppose
\begin{enumerate}
\item $H^1(Y, \struct{Y}) = 0$
\item $\Ext{0}{Y}{\LL_Y}{\struct{Y}} = 0$ i.e. $Y$ has no infinitesimal automorphisms.
\end{enumerate}
Then the canonical map of deformation spaces
\[ \Def_X \times \Def_Y \to \Def_{X \times Y} \]
is an isomorphism.
\end{prop}

\begin{proof}
We will show that
\[ \mathrm{prod}_{X,Y}(A) : \Def_X(A) \times \Def_Y(A) \to \Def_{X \times Y}(A) \]
is an equivalence of groupoids for $A \in \Art_k$ by working inductively on $\dim_k(A)$. {\color{red} maybe we can use the extension structure to rigidify and therefore get no infinitesimal automorphisms}

For $\dim_k(A) = 1$ we have $A = k$ and there is nothing to prove. By induction, we may assume the claim for all $A \in \Art_k$ with $\dim_k(A) \le n$ for some fixed interger $n$. Given $\wt{A} \in \Art_k$ with $\dim_k(\wt{A}) = n + 1$ choose a surjection $\wt{A} \onto A$ with kernel $k$ as an $A$-module giving a digram
\begin{center}
\begin{tikzcd}
\Def_X(\wt{A}) \times \Def_Y(\wt{A}) \arrow[r, "\mathrm{prod}_{X,Y}"] \arrow[d] & \Def_{X \times Y}(\wt{A}) \arrow[d]
\\
\Def_X(A) \times \Def_Y(A) \arrow[r, "\mathrm{prod}_{X,Y}"] & \Def_{X \times Y}(A)
\end{tikzcd}
\end{center}
with the bottom map an equivalence by assumption. We will show that the top map is also an equivalence. There is nothing to show if the bottom row is empty, we may fix a base point of the bottom row, meaning a flat deformation $f : \X \to \Spec{A}$ and $g : \Y \to \Spec{A}$ of $X, Y$ to $\Spec{A}$. Let $\pi_{f,g} : \X \times_A \Y \to \Spec{A}$ denote the fiber product and $p, q$ the projection maps.
\bigskip\\
We first show that all fibers of $\mathrm{prod}_{X,Y}(\wt{A})$ are nonempty {\color{red} I actually think this argument only shows if there is a point on the right then there exists a point on the left not that \textit{every} fiber is nonempty} meaning if $\pi_{f,g}$ admits a deformation to $\wt{A}$ then so do each $f, g$. Let $D_A : \LL_A \to k[1]$ be the derivation classifying the surjection $\wt{A} \onto A$. Associated to this derivation, we have obstruction classes
\[ \ob(f) := \ob(f, f^* D_A) : \LL_{\X/A}[-1] \to \struct{X}[1] \quad \quad \ob(g) := \ob(g, g^* D_A) : \LL_{\Y/A}[-1] \to \struct{Y}[1] \]
on $\X$ and $\Y$ and the obstruction class
\[ \ob(\pi_{f,g}, \pi_{f,g}^* D_A) : \LL_{\X \times_A \Y/A} [-1] \to \struct{\X \times_A \Y}[1] \]
on $\X \times \Y$ which satisfy the following compatiblity diagram
\begin{center}
\begin{tikzcd}[column sep = large]
p^* \LL_{\X/A}[-1] \arrow[d] \arrow[r, "p^* \ob(f)"] & \struct{\X \times_A \Y}[1] \arrow[d, equals]
\\
\L_{\X \times_A \Y/A}[-1] \arrow[r, "\ob(\pi_{f,g})"] & \struct{\X \times_A \Y}[1] \arrow[d, equals]
\\
q^* \LL_{\Y/A}[-1] \arrow[u] \arrow[r, "q^* \ob(g)"] & \struct{\X \times_A \Y}[1]
\end{tikzcd}
\end{center}
The assumption that $\pi_{f,g}$ admits a deformation across $A' \onto A$ ensures that the midle horizontal arrow in the above diagram is $0$. It follows by commutativity that the same is true of the other horizontal arrows, i,e, that $p^* \ob(f) = q^* \ob(g) = 0$. To show that $\ob(f) = 0$ it suffices to show that the pullback
\[ \pi_0(\Hom{\X}{\LL_{\X/A}[-1]}{k \ot_A \struct{\X}[1]} \to \pi_0(\Hom{\X \times_A \Y}{p_1^* \LL_{\X/A}[-1]}{p_1^* (k \ot_A \struct{\X})[1]} \]
is injective, and similarly for $Y$. Simplifying, this amounts to showing that the pullback
\[ \Ext{2}{\X}{\LL_{\X/A}}{\struct{\X}} \to \Ext{2}{\X \times_A \Y}{p_1^* \LL_{\X/A}}{\struct{\X \times \Y}} \]
is injective and similarly for $Y$. By base change and adjunction, it is enough to check that the pullback
\[ \Ext{2}{X}{\LL_X}{\struct{X}} \to \Ext{2}{X \times Y}{p_1^* \LL_X}{\struct{X \times Y}} \]
is injective and similarly for $Y$. Both follow from Corollary~\ref{cor:ext_product_basechange_asymmetric}.
\bigskip\\
Now we show that $\mathrm{prod}_{X,Y}(\wt{A})$ is an equivalence on fibers over a fixed deformation to $A$. This amounts to showing that for any deformation of $\X \times_A \Y$ over $\wt{A}$ it is uniquely isomorphic to a defomration of $\X$ times a deformtion of $\Y$. These are torsors over the requisite $\Ext{1}{X}{\LL_{X}}{\struct{X}}$ etc by base change {\color{red} elaborate} and the space of automorphisms is $\Ext{0}{X}{\LL_{X}}{\struct{X}}$ etc.
\bigskip\\
Note that $\LL_{X \times Y} = p_1^* \LL_X \oplus p_2^* \LL_Y$ and the natural map
\[ \Ext{i}{X}{\LL_X}{\struct{X}} \oplus \Ext{i}{Y}{\LL_Y}{\struct{Y}} \to \Ext{i}{X \times Y}{\LL_{X \times Y}}{\struct{X \times Y}} = \Ext{i}{X \times Y}{p_1^* \LL_{X \times Y}}{\struct{X \times Y}} \oplus \Ext{i}{X \times Y}{\LL_{X \times Y}}{\struct{X \times Y}} \]
is simply the sum of the maps described in the preceeding lemmas. Therefore, this map is an isomorphism for $i = 0, 1$ and injective for $i = 2$. 
\bigskip\\
Any map of torsors over isomorphic groups is an isomorphism so we conclude.
\end{proof}

\section{Deformation Theory a la Abramovich-Hasset}

LSBA = Kollar-Shepherd-Barron-Alexeev

\begin{defn}
A proper geometrically connected $k$-variety is \textit{stable} if $X$ has semi-log canonical singularities and $K_X$ is a $\Q$-Cartier ample divisor.
\end{defn}

\begin{defn}
Let $S$ be a $k$-scheme, a \textit{stable family} $f : X \to S$ is a proper flat morphism whose fibers are stable varieties nd such that $\omega_{X/S}^{[m]}$ is flat over $S$ and commutes with base change for all $m \in \ZZ$.
\end{defn}

\begin{defn}
Let $h(m)$ be an integer-valued function. The moduli functor $\ol{\M}_h$ of stable varieties with Hilbert function $h$ is the functor taking $S$ to the groupoid of stable families $f : X \to S$ whose fibers have Hilbert function $h$ with respect to $\omega_{X/S}$. 
\end{defn}

\begin{defn}
Given a stable variety $X$ over $k$ let $\M(X)$ denote the connected component of $\ol{\M}_{h(X)}$ containing $X$. Say that $X$ and $Y$ are \textit{deformation equivalent} if $\M(X) = \M(Y)$.
\end{defn}

\begin{lemma}[BHPS]
Let $X$ be a stable variety over a field $k$ of characteristic $0$. Then $X$ has no infinitesimal automorphisms. 
\end{lemma}

\subsection{Koll\'{a}r families}

\begin{defn}
A \textit{Koll\'{a}r family} of $\Q$ line bundles $(f : X \to B, F)$ is a pair where
\begin{enumerate}
\item $f : X \to B$ is a flat family of equidimensional connected reduced $S_2$ schemes
\item $F$ is a coherent sheaf on $X$ such that
\begin{enumerate}
\item for each fiber $X_b$ the restriction $F|_{X_b}$ is reflexive of rank $1$
\item for every $n$, the formation of $F^{[n]}$ commutes with arbitrary base change
\item for each geometric point $\bar{s} \in B$ there is an integer $N_{\bar{s}} \neq 0$ such that $F^{[N_{\bar{s}}]}|_{X_s}$ is invertible.
\end{enumerate} 
\end{enumerate}
A \textit{morphism} $(\phi, \alpha) : (X \to B, F) \to (X' \to B', F')$ of Koll\'{a}r families is a cartesian diagram
\begin{center}
\begin{tikzcd}
X \arrow[r, "\phi"] \pullback \arrow[d] & X' \arrow[d]
\\
B \arrow[r] & B'
\end{tikzcd}
\end{center}
along with an isomorphism $\alpha : F \to \phi^* F'$. Therefore, this defines the category of \textit{Koll\'{a}r families of $\Q$-line bundle} as a fibered category over $\Sch_k$. There is an important open subcategory $\K^{\L}$ of \textit{Koll\'{a}r families of polarizing $\Q$-line bundles} where $X \to B$ is proper and $F^{[N_s]}|_{X_s}$ is ample. 
\end{defn}

\begin{defn}
Let $(f : X \to B, F)$ be a Koll\'{a}r family of $\Q$-line bundles.
\begin{enumerate}
\item the $\Gm$\textit{-space} of $F$ is the $X$-scheme
\[ \cP_F = \rSpec{X}{\bigoplus_{j \in \ZZ} F^{[j]}} \]
Note that $\cP_F \to B$ is flat by [AH, Prop 5.1.4]
\item the \textit{associated stack} is
\[ \X_F = [\cP_F / \Gm] \]
Which carries the \textit{natural line bundle} $\L = \L_F$ associated to the principal $\Gm$-bundle
\[ \cP_F \to \X_F \]
\end{enumerate}
\end{defn}

\begin{prop}[AH, Prop. 5.3.2]
In the above situation
\begin{enumerate}
\item the family $\X \to B$ is a family of cyclotomic orbispaces uniformaized by $\L$ whose fibers are $S_2$
\item The morphism $\pi : \X \to X$ makes $X$ into the coarse moduli space of $\X$. This is an isomorphism on the locus where $F$ is invertible which is a big open.
\item For any integer $a$, we have $\pi_* (\L^a) = F^{[a]}$ (in particular it is reflexive)
\item The construction is functorial, given a morphism $(\phi, \alpha) : (X \to B, F) \to (X' \to B', F')$ we have canonicall $\cP_F \cong \phi^* \P_{F'}$ and $\X_F \cong \phi^* \X_{F'}$ via $\alpha$.
\end{enumerate}
\end{prop}

\begin{theorem}[AH, Theorem 5.3.6]
The category of Koll\'{a}r families of $\Q$-line bundles is equivalent to the category of uniformized twisted varities via the base change preserving functors
\[ (X \to B, F) \mapsto (\X_F \to B, \L_F) \]
with $\X_F = [\cP_F / \Gm]$ and its unverse
\[ (\X \to B, \L) \mapsto (X \to B, \pi_* \L) \]
where $X$ is the coarse moduli space of $\X$.
\end{theorem}




\subsection{$\Q$-Gorenstein Deformations}

\begin{defn}
let $X$ be a normal projective surface with quotient singularities, a $\Q$-Gorenstein smoothing is a $1$-parameter flat family of projective surfaces $\psi : \X \to \Delta$ such that
\begin{enumerate}
\item the generic fiber $X_t$ is a smooth projective surface
\item the central fiber $X_0$ is isomorphic to $X$
\item the canonical divisor $K_{\X / \Delta}$ is $\Q$-Cartier.
\end{enumerate}
\end{defn}

\begin{rmk}
The third condition is equivalent to $\X$ being $\Q$-Gorenstein because then there is an integer multiple $m K_{\X}$ which is Cartier and 
\[ m K_{\X / \Delta} = m K_{\X} - m \phi^* K_{\Delta} \]
is also Cartier. 
\end{rmk}


\subsection{References}

\begin{enumerate}
\item \chref{https://arxiv.org/pdf/1206.0438.pdf}[BHPS]
\item \chref{https://www.math.brown.edu/bhassett/papers/projective/slcmodEMS2.pdf}{[AH]}
\item \chref{https://arxiv.org/pdf/2402.15117.pdf}{[AAB]}
\item \chref{https://arxiv.org/pdf/2307.06522.pdf}{[ABBDILW]}
\end{enumerate}

\newpage

\section{Talk}

\subsection{Motivation: unirationality}

Unirationaltiy is the weakest notion of a polynomial system being ``solvable'' by polynomials. 

\begin{defn}
Let $X$ be a variety over $k$. We say that $X$ is \textit{unirational} (over $k$) if there is a dominant rational map $\P^n \rat X$.
\end{defn}

This says exactly that (most) points of $X$ can be parametrized by some rational functions. 

\subsubsection{Characteristic Zero}

Proving that a variety is not unirational is often quite sublte. For example, even in characteristic zero there is no known example of a rationally connected variety where we can prove it is not unirational.
\par 
However, if the variety admits any sort of forms we can use these as an obstruction.

\begin{prop}
Let $k$ be of characteristic zero and $X$ a proper $k$-variety. Then if $\omega \in H^0(X, \Omega_X^{\ot m})$ is a nonzero form for $m > 0$ then $X$ is not unirational.
\end{prop}

\begin{proof}
Indeed, if $f : \P^n \rat X$ is a unirational parametrization with $n = \dim{X}$ then $f$ is generically \etale. Furthermore, since $X$ is proper $f$ is defined in codimension $1$ so $f^* \omega$ extends to a section of $H^0(\P^n, \Omega_{\P^n}^{\ot m})$ giving a contradiction. 
\end{proof}

This argument fails completely in positive characteristic because of the existence of inseparable maps for which $f^* \omega = 0$.

\begin{example}
Consider a surface $X / \FF_p$ defined as a smooth compactification of an equation of the form,
\[ z^p = f(x,y) \]
for some $f \in \FF_p[x,y]$. Then if we choose $f$ to have large degree $X$ may be of general type. However, notice that if we adjoin $p^{\text{th}}$-roots $s,t$ of $x$ and $y$ then the equation becomes
\[ z^p = f(s^p, t^p) = f(s,t)^p \]
therefore $k(X) \subset k(s,t)$ given by $z = f(s,t)$ and $x = s^p$ and $y = t^p$. This shows that $X$ is always uniratonal. 
\end{example}

This demonstrates that unirationality in positive characteristic is a phenmenon that persists even for quite ``complex'' varieties from the perspective of the usual classification of surfaces. 

\subsection{Fundamental Groups}

Another property enjoyed by unirational varities in characteristic is that their fundamental groups vanish.

\begin{prop}
Let $X / \CC$ be a smooth proper complex variety. If $X$ is unirational then $\pi_1(X) = 0$.
\end{prop} 

\begin{proof}
Let $f : \P^n \sm Z \to X$ be a unirational parametrization and we can choose $\codim{Z} \ge 2$. Thus $\pi_1(\P^n \sm Z) = 1$ so the map lifts to the universal cover $\wt{X} \to X$ as a holomorphic map. However $f$ is generically a finite cover so this forces $\wt{X} \to X$ to be a finite cover also. Hence $\wt{X}$ is also a smooth proper variety. Now consider the diagram,
\begin{center}
\begin{tikzcd}
Y \arrow[r] \arrow[d] \pullback & \wt{X} \arrow[d]
\\
\P^n \sm Z \arrow[r] & X 
\end{tikzcd}
\end{center}
since $\pi_1(\P^n \sm Z) = 1$ we see that $Y$ is a finite union of copies of $\P^n \sm Z$ and hence $\wt{X}$ is also unirational. Therefore we showed that $h^0(\wt{X}, \Omega_{\wt{X}}^p) = 0$ so by hodge symmetry $h^p(\wt{X}, \struct{\wt{X}}) = 0$ so $\chi(\wt{X}) = 1$. However, it is well-known that if $\wt{X} \to X$ has degree $n$ then $\chi(\struct{\wt{X}}) = n \chi(\struct{X})$ hence $n = 1$. 
\end{proof}

Some version of this caries over to positive characteristic as well. 

\begin{theorem}[Serre]
Let $k$ be of characteristic $p$ and $X$ is a smooth proper $k$-variety. If $X$ is unirational then $\pi_1(X)$ is finite of order coprime to $p$. 
\end{theorem}

We will see in the next section an example where $\pi_1(X)$ is nonzero in positive characteristic.


\subsection{Shioda's Example}

Shioda used some very clever tricks and computations to compute the following.

\begin{theorem}[Shioda]
Let $F_n$ be the Fermat surface over $\FF_p$ ($p > 2$) given by the equation
\[ \{ X^n + Y^n + Z^n + W^n = 0 \} \subset \P^3 \]
Then $F_n$ is unirational if and only if $p^\nu \equiv -1 \mod n$ for some $\nu$.
\end{theorem}

\begin{example}
Let $G = \Z / 5 \Z$ act on $F_5$ by $\lambda \cdot [X : Y : Z : W] = [X : \lambda Y : \lambda^2 Z : \lambda^3 W]$. This action is free and hence $X = F_5 / G$ is a smooth projective surface with $\pi_1(X) = \Z / 5 \Z$ called the Godeaux surface. Shioda's result shows that this is supersingular over infinitely many primes. However, we saw this cannot happen over $\CC$, indeed $X$ has general type. 
\end{example}

How does Shioda prove this theorem: for the ``if'' part he uses an extremely clever change of variables to just write down the unirational parametrization. To show the ``only if'' he needs an obstruction, this comes from \etale cohomology. 

\begin{prop}
Let $X$ be a smooth projective unirational surface over $\FF_q$. Then $\Frob_q \acts H^2_{\et}(X, \QQ_\ell)$ has eigenvalues of the form $\zeta \cdot q$ for $\zeta$ a root of unity. We call this property \textit{supersingularity}. 
\end{prop}

\begin{proof}
Pass to an extension $\FF_{q^n}$ such that the unirational parametrization and all points of indeterminacy are defined over $\FF_{q^n}$. Resolving the indeterminacy locus, we get $\Bl_S \P^2 \to X$ where $S \subset \P^2$ is a finite set of points. Therefore $H^2_{\et}(X, \QQ_\ell) \embed H^2_{\et}(\Bl_S \P^2, \Q_\ell) = \Q_\ell(-1)^{1 + \# S}$. Thus $\Frob_{q^n} = \Frob_{q}^n$ acts via $q^n$.  
\end{proof}


\begin{lemma}
Let $f : X \to Y$ be surjection of smooth proper varieties. Then $f^* : H^i_{\et}(Y, \Q_\ell) \embed H^i_{\et}(X, \Q_\ell)$ is injective.
\end{lemma}

\begin{proof}
Poincare duality defines a pushforward map $f_* = D^{-1} \circ (f_*)^{\vee} \circ D$ where $D(\alpha) = \tr{(\alpha \smile -)}$. If $f$ is generically finite, it suffices to show that $f_* f^* = \deg{f}$ since this is invertible on $\Q_\ell$. Indeed, 
\[ D(f_* f^* \alpha) = \tr{(f^* \alpha \smile f^* - )} = \tr{f^* (\alpha \smile -)} = (\deg{f}) D(\alpha) \]
because $f^* = \deg{f}$ on top cohomology. If $f$ is not generically finite then we may take a generic multisection $Z \subset X \to Y$ and then $(f \circ \iota)^*$ is injective so $f^*$ is injective.
\end{proof}

Shioda then computes that $F_n$ is supersingular exactly when $p^\nu \equiv - 1 \mod p$ for some $\nu$ and this covers all the cases his construction works to show $F_n$ is moreover unirational. This, and a handful of other examples, leads him to make the following conjecture.

\begin{conj}[Shioda]
Let $X$ be a smooth projective surface over $\FF_q$. Suppose that
\begin{enumerate}
\item $\pi_1(X) = 1$
\item $X$ is supersingular
\end{enumerate}
then $X$ is unirational.
\end{conj}

\begin{rmk}
One might ask why doesn't Shioda just require that $\pi_1(X)$ is finite as suggested by Serre's theorem and the Godeaux surface example: this conjecture would be false because it turns out the Godeaux surface is supersingular in every characteristic (of good reduction) but it unirational only in the characteristics that $F_5$ is unirational in.
\end{rmk}

The goal of this talk is to describe new obstructions to unirationality that could be used to test Shioda's conjecture.

\subsection{Jet Obstructions}


Here we describe one approach to obstructing unirationality. The idea is very simple: recall that in characteristic zero we may use forms to obstruct unirationality. Hover, the issue is that degree $p$ purely inseparable maps can kill first-order derivatives. The fundamental observation is that they cannot kill $p^{\text{th}}$-order derivatives. Therefore, if we can find higher-order forms on our varities then these will pullback nontrivially even along certain inseparable maps to produce obstructions. This idea can be formalized using various notions of jet bundles.

\begin{defn}
Let $X \to S$ be a smooth scheme. Then $J^n_{X/S} := \pi_{1*} \pi_2^* \struct{X}$ where the projections are along the $n^{\text{th}}$-order thickening of the diagonal $\Delta^n_{X/S} := V(\I_{\Delta}^{n+1}) \subset X \times_S X$. This satisfies the following universal property,
\[ \shHom{\struct{X}}{J^n_{X/S}}{\E} = \shDiff{\le n}{X}{\struct{X}}{\E} \]
\end{defn}

Now I define a notion of bigness for lifts of jets.

\newcommand{\jetamp}{\mathrm{jetamp}}

\begin{defn}
Let $\wt{J}^n_X := \ker{(J^n_X \to \struct{X})}$ using the natural projection $J^n_X \to J^0_X = \struct{X}$.
\end{defn}

\begin{defn}
Let $X$ be a smooth proper variety. Define the \textit{jet-amplitude} of $X$ to be
\[ \jetamp(X) := \sup \left \{ \frac{n}{r} : \exists \, \omega \in H^0(X, \wt{J}^{n}_X) \text{ such that } \omega|_{\wt{J}^{r}_X} \neq 0 \right\} \]
\end{defn}

\begin{theorem}[C]
Let $X$ be a smooth proper variety over a perfect field $k$. Suppose that $\log_p \jetamp(X) > \ell$ then there does not exist a unirational parametrization $\P^n \rat X$ of inseparability degree $\ell$.
\end{theorem}

\begin{proof}
Since $\jetamp$ only increases anlong separable maps we may assume that $X$ has a purely inseparable unirational parametrization $f : \P^n \rat X$ of degree $k$. Now suppose that $\jetamp \ge p^{\ell} := q$ then there are integers $r,s \ge 0$ such that $\omega \in H^0(X, \wt{J}_X^{s})$ such that $\omega|_{\wt{J}^{r}} \neq 0$ and $s \ge q r$. By decreasing $r$ we may assume that $\omega|_{\wt{J}^{r-1}} = 0$ so $\omega|_{\wt{J}^{r}} \in H^0(X, \nSym{r}{\Omega_X})$ then decreasing $s$ we may assume $s = q r$. There is a factorization of rational maps
\begin{center}
\begin{tikzcd}
X^{(1/q)} \arrow[r, dashed, "f'"] \arrow[rd, "F"'] & \P^n \arrow[d, dashed, "f"] 
\\
& X
\end{tikzcd}
\end{center}
where $F : X^{(1/q)} \to X$ is the $\ell^{\text{th}}$-relative Frobenius. 
Using extension in codimension $\ge 2$, there are well-defined pullback maps
\begin{center}
\begin{tikzcd}
H^0(X, \wt{J}^{s}_X) \arrow[r] \arrow[d, "f^*"] & H^0(X, \wt{J}^{r}_X) \arrow[d, "f^*"] 
\\
H^0(\P^n, \wt{J}^{s}_X) \arrow[d, "f'^*"] \arrow[r] & H^0(\P^n, \wt{J}^{r}) \arrow[d, "f'^*"] 
\\
H^0(X^{(1/q)}, \wt{J}^{s}_{X^{(1/q)}}) \arrow[r] & H^0(X^{(1/q)}, \wt{J}^{r}_{X^{(1/q)}}) 
\end{tikzcd}
\end{center} 
where the composition along column is $F^*$. Therefore, it suffices to show that $F^* \omega \neq 0$. To do this, we may work \etale-locally. Indeed, if $c : U \to X$ is an \etale chart then it suffices to show that $F_U^* \omega|_U \neq 0$ since the diagram
\begin{center}
\begin{tikzcd}
U^{(1/q)} \arrow[r, "F_U"] \arrow[d] & U \arrow[d]
\\
X^{(1/q)} \arrow[r, "F"] & X 
\end{tikzcd}
\end{center}
commutes. Since $X$ is smooth, there are \etale charts $U \to X$ with $g : U \to \A^n_k$ \etale. The diagram, 
\begin{center}
\begin{tikzcd}
\wt{J}^s_{\A^n} \arrow[d, "F^*"] \arrow[r] & \wt{J}^r_{\A^n} \arrow[d, "F^*"] \arrow[ld, "t", dashed]
\\
\wt{J}^s_{\A^n} \arrow[r] & \wt{J}^r_{\A^n}
\end{tikzcd}
\end{center}
has a commuting lift $t$. Indeed, $J^r_{\A^n}$ is represented by the algebra,
\[ k[x_1, \dots, x_n][\d{x_1}, \dots, \d{x_n}] / (\d{x_1}, \dots, \d{x_n})^{r+1} \] and $F^* \d{x_i} = \d{x_i^{q}} = (\d{x_i})^{q}$. Therefore, the kernel of $\wt{J}^s_{\A^n} \to \wt{J}^r_{\A^n}$ are polynomials $h(\d{x_1}, \dots, \d{x_n})$ with coefficients in $k[x_1, \dots, x_n]$ of minimal degree $\ge r + 1$ and hence $F^* h = h(\d{x_1^{q}}, \dots, \d{x_n^{q}}) = 0$ in $J^s_{\A^n}$ since it has minimal degree $\ge q (r + 1) \ge s + 1$. Furthermore, consider a symmetric form $\eta \in H^0(\A^n, \nSym{r}{\Omega})$ write
\[ \eta = \sum_{I} f_I(x_1, \dots, x_n) \d{x_{i_1}} \cdots \d{x_{i_r}} \]
then we see
\[ t(\eta) = \sum_I f_I(x_1^q, \dots, x_n^q) \d{x_{i_1}^q} \cdots \d{x_{i_r}^q} \in H^0(\A^n, \nSym{s}{\Omega}) \]
but for different $I \subset \{1, \dots, n\}$ of size $r$, these are distinct basis elements of $H^0(\A^n, \nSym{s}{\Omega})$. Therefore, $t(\eta) = 0$ if and only if all $f_I(x_1^q, \dots, x_n^q) = 0$ if and only if all $f_I = 0$. Hence $t$ is injective. Since $g$ is \etale, we get isomorphisms $g^* \wt{J}_{\A^n}^r \iso \wt{J}^r_{U}$ for all $r$ so the diagram pulls back to
\begin{center}
\begin{tikzcd}
\wt{J}^s_{U} \arrow[r] \arrow[d, "F^*_U"] & \wt{J}^r_{U} \arrow[d, "F^*_U"] \arrow[dl, "t_U"]
\\
\wt{J}^s_{U} \arrow[r] & \wt{J}^r_{U}
\end{tikzcd}
\end{center}
such that the restriction $t_U : \nSym{r}{\Omega_U} \to \nSym{s}{\Omega_U}$ is injective. Therefore $F^*_U \omega|_U = t(\omega|_{\wt{J}^r_U}) \neq 0$ because we assumed that $\omega|_{\wt{J}^r} \in H^0(X, \nSym{r}{\Omega_X})$ and is nonzero.
\end{proof}

Unfortunately it is quite difficult to compute $\jetamp(X)$. For ``generic'' complete intersections of large degree, it is known that $\jetamp(X) = \infty$. However, for any particular example it is very hard to get ones hands on this thing. There is a trivial bound as follows.


\begin{prop}
Let $X$ be a smooth projective variety. Then,
\[ \jetamp(X) \ge \left(\frac{\ul{\hat{h}}^0(\Omega_X)}{\hat{h}^1(\Omega_X)}\right)^{\frac{1}{2 \dim{X}}} \]
where $\hat{h}^i$ these are so-called \textit{asymtotic cohomology}
\[ \hat{h}^i(X, \E) := \limsup_{m \to \infty} \frac{h^0(X, \nSym{m}{\E})}{m^{\dim{X} + \rank{\E} - 1}} \]
and likewise $\ul{\hat{h}}^i$ are the \textit{lower asymtotic cohomology}
\[ \ul{\hat{h}}^i(X, \E) := \liminf_{m \to \infty} \frac{h^0(X, \nSym{m}{\E})}{m^{\dim{X} + \rank{\E} - 1}} \]
\end{prop}


\begin{proof}
This arises from the exact sequence
\begin{center}
\begin{tikzcd}
0 \arrow[r] & \nSym{n}{\Omega_X} \arrow[r] & J^n_X \arrow[r] & J^{n-1}_X \arrow[r] & 0 
\end{tikzcd}
\end{center}
Therefore, 
\[ \dim \im{(H^0(X, J^{rs}_X) \to H^0(X, J^r_X))} \ge h^0(X, J^r_X) - \sum_{k = r+1}^{rs} \left[h^1(X, \nSym{k}{\Omega_X}) \right] \]
but also
\[ h^0(X, J^r_X) \ge \sum_{i = 1}^r \left[ h^0(X, \nSym{i}{\Omega_X}) - h^1(X, \nSym{i}{\Omega_X}) \right] \] 
therefore
\[ \dim \im{(H^0(X, J^{rs}_X) \to H^0(X, J^r_X))} \ge \sum_{i = 1}^r h^0(X, \nSym{i}{\Omega_X}) - \sum_{i = 1}^{rs} h^1(X, \nSym{i}{\Omega_X}) \]
Applying the lemma, for any $\epsilon > 0$ for $r \gg_{\epsilon} 0$ we have
\[ \sum_{i = 1}^r h^0(X, \nSym{i}{\Omega_X}) - \sum_{i = 1}^{rs} h^1(X, \nSym{i}{\Omega_X}) \ge \frac{1}{2 \dim{X}} \left[ (\ul{\hat{h}}^0 - \epsilon) r^{2 \dim{X}} - (\hat{h}^1 + \epsilon) (rs)^{2 \dim{X}} \right] \]
Thus,
\[ s^{2 \dim{X}} \le \frac{\ul{\hat{h}}^0 - \epsilon}{\hat{h}^1 + \epsilon} \implies \jetamp(X) \ge s  \]
\end{proof}

\begin{lemma}
Let $f : \N \to \RR^+$ be an integer function. Suppose that
\[ \liminf_{n \to \infty} \frac{f(n)}{n^k} = a \quad \text{ and } \quad \limsup_{n \to \infty} \frac{f(n)}{n^k} = b \]
then\footnote{Notice that the outer inequalities are not usually equalities. For example let 
\[ f(n) = \begin{cases} 0 & n \text{ even} \\ 1 & n \text{ odd} \end{cases} \]
then the hypotheses are satisfies with $k = 0$ and $a = 0$ and $b = 1$ but
\[ \sum_{i = 1}^n f(i) = \ceil{\frac{n}{2}} \] 
and therefore the limit
\[ \lim_{n \to \infty} \frac{1}{n} \sum_{i = 1}^n f(i) = \frac{1}{2} \]
exists so the inequalities in the conclusion of the lemma are strict. }
\[ \frac{b}{k+1} \ge \limsup_{n \to \infty} \frac{1}{n^{k+1}} \sum_{i = 1}^n f(i) \ge \liminf_{n \to \infty} \frac{1}{n^{k+1}} \sum_{i = 1}^n f(i) \ge \frac{a}{k+1}  \]
\end{lemma}

\begin{proof}
The assumptions give for all $\epsilon > 0$ there is $n_\epsilon$ such that $n \ge n_\epsilon$ implies 
\[ (b + \epsilon) n^k \ge f(n) \ge (a - \epsilon) n^k \]
summing over this we get,
\[ (b + \epsilon) \frac{1}{n^{k+1}} \sum_{i = 1}^n i^k \ge \frac{1}{n^{k+1}} \sum_{i = n_\epsilon}^n f(i) + \frac{1}{n^{k+1}} \sum_{i = 1}^{n_\epsilon} i^k \ge (a - \epsilon) \frac{1}{n^{k+1}} \sum_{i = 1}^n i^k \]
Therefore,
\[ \frac{b + \epsilon}{k + 1} (1 + O(n^{-1})) \ge \frac{1}{n^{k+1}} \sum_{i = 1}^n f(i) + \frac{1}{n^{k+1}} \sum_{i = 1}^{n_\epsilon} (i^k - f(i)) \ge \frac{a - \epsilon}{k+1} (1 + O(n^{-1}))  \]
Taking the limit $n \to \infty$ and then $\epsilon \to 0$ gives the bounds
\[ \frac{b}{k+1} \ge \limsup_{n \to \infty} \frac{1}{n^{k+1}} \sum_{i = 1}^n f(i) \ge \liminf_{n \to \infty} \frac{1}{n^{k+1}} \sum_{i = 1}^n f(i) \ge \frac{a}{k+1}  \]
Furthermore, for any $\epsilon > 0$ and $n_0$ there is $n \ge n_0$ such that $(a + \epsilon) n^k \ge f(n)$ 
\end{proof}

\begin{example}
Suppose $h^0(X, \nSym{2}{\Omega_X}) \ge h^1(X, \nSym{3}{\Omega_X}) + h^1(X, \nSym{4}{\Omega_X})$ then $X$ is not a Zariski space in characteristic $2$.
\end{example}

\subsection{Hyperbolicity}

\begin{defn}
A complex variety $X$ is \textit{hyperbolic} if ever entre map $f : \CC \to X$ is constant.
\end{defn}

In particular, this says that $X$ does not contain any rational or elliptic curves. It is expected that general type varities are \textit{close} to being hyperbolic in the following precise sense.

\begin{conj}[Green-Griffiths-Lang]
Let $X$ be a smooth projective complex surface of general type. Then there exists a proper Zariski closed subset $Z \subsetneq X$ such that any entire curve $f : \CC \to X$ has image inside $Z$.
\end{conj}

The remarkable feature of this conjecture is that we expect every entire curve to be \textit{algebraically degenerate} meaning it satisfies a polynomial relation (i.e. it lives in $Z$). Moreover, we expect that these relations can be chosen independently of the map! 
\bigskip\\
The strategy developed by Green, Griffiths, Demailly and many others towards this conjecture is to produce algebraic differential equations that entire curves must satisfy. We can think of a first-order differential equation as a closed subspace of the tangent bundle of $X$. If we could produce enough differential equations so that the intersection of these loci in $T X$ does not dominate $X$ then we win. The problem is that many general type surfaces have no first-order algebraic differential equations or, as an algebraic geometer would say, $H^0(X, \nSym{n}{\Omega_X}) = 0$ for all $n > 0$. Instead, we consider higher-order differential equations via Demailly-Semple jet bundles.

\subsection{Semple Jets}

\begin{defn}
A \textit{directed variety} $(X, \E)$ is a pair of a variety $X$ with a subbundle $\E \subset \T_X$. A morphism of directed varities $f : (X, \E) \to (Y, \E')$ is a morphism $f : X \to Y$ such that under $f_* \T_X \to \T_Y$ we have $f_* \E \to \E'$.
\end{defn}

\begin{rmk}
Demailly's philosophy is that it is usefull to study this ``relative notion'' even for the absolute case $\E = \T_X$ since it has better functoriality properties.
\end{rmk}

\begin{rmk}
Here our convention is that $\P(\E) := \rProj{X}{\Sym{}{\E^\vee}}$ so that $\cO(-1)$ is the universal subbundle. Hence $\cO(1)$ on $\P(\T_X)$ is what I usually call $\cO(1)$ on $\P(\Omega_X)$.
\end{rmk}

\begin{defn}
To a directed pair $(X, \E)$ we introduce the \textit{projectivization} to produce a new pair $\P(X, \E) := (\wt{X}, \wt{\E})$ where $\wt{X} := \P(\E)$ and $\wt{\E}$ is defined via the diagram,
\begin{center}
\begin{tikzcd}[row sep = small, column sep = large]
0 \arrow[r] & \T_{\wt{X}/X} \arrow[dd] \arrow[r] & \wt{\E} \arrow[dd] \pullback \arrow[r] & \cO(-1) \arrow[d, hook] \arrow[r] & 0
\\
& & & \pi^* \E \arrow[d, hook]
\\
0 \arrow[r] & \T_{\wt{X}/X} \arrow[r] & \T_{\wt{X}} \arrow[r] & \pi^* \T_{X} \arrow[r] & 0
\end{tikzcd}
\end{center}
Then we have,
\[ \dim{\wt{X}} = \dim{X} + \rank{\E} - 1 \quad \quad \rank{\wt{\E}} = \rank{\E} \] 
\end{defn}

\begin{rmk}
Note that the Euler exact sequence takes the form,
\begin{center}
\begin{tikzcd}
0 \arrow[r] & \cO \arrow[r] & \pi^* \E \ot \cO(1) \arrow[r] & \T_{\wt{X}/X} \arrow[r] & 0
\end{tikzcd}
\end{center}
\end{rmk}

\begin{prop}
Given a morphism of directed varities $f : (X, \E) \to (Y, \F)$ we get a rational map $\wt{f} : (\wt{X}, \wt{\E}) \rat (\wt{Y}, \wt{\F})$ such that the diagram,
\begin{center}
\begin{tikzcd}
(\wt{X}, \wt{\E}) \arrow[d,"\pi"] \arrow[r, "\wt{f}", dashed] & (\wt{Y}, \wt{\F}) \arrow[d, "\pi"]
\\
(X, \E) \arrow[r, "f"] & (Y, \F)
\end{tikzcd}
\end{center} 
commutes in the category of directed manifolds (with rational maps). Moreover, if $f$ is ``immersive along $\E$'', meaning $f_{\#} : \E \to f^* \F$ is injective, then $\wt{f}$ is a morphism.
\end{prop}

\begin{defn}
Let $(X, V)$ be a directed manifold. The \textit{projectivized Semple $k$-jet bundle} $P_k V = X_k$ is defined iteratively via,
\[ (X_0, V_0) := (X, V) \quad \quad (X_{k+1}, V_{k+1}) := (\wt{X_k}, \wt{V_k}) \]
and we have,
\[ \dim{P_k V} = \dim{X} + k (\rank{V} - 1) \quad \quad \rank{V_k} = \rank{V} \]
\end{defn}


The semple tower is defined so that the following holds. Suppose that $f : C \to X$ is an immersed curve such that $\d{f} : \T_C \to f^* \T_X$ factors through $f^* \E \subset f^* \T_X$. Since $\d{f}$ is a subbundle this gives a subbundle $\T_X \embed \pi^* \E$ and hence a lift $f' : C \to \wt{X}$ such $\d{f} : \T_C \to f^* \E \to f^* \T_X$ is $f'^* [\struct{\wt{X}}(-1) \to \pi^* \E \to \pi^* \T_X]$. Therefore, consider $\d{f'} : \T_C \to f'^* \T_{\wt{X}}$. Since this map lifts $\d{f}$ we see that $\d{f'} : \T_X \to f'^* \wt{\E}$.

Hence, if we start with an immersed curve $f : C \to X$ then there are lifts $f_k : C \to P_k$ for all $k$. The fundamental property is:

\begin{prop}
Let $f : \P^1 \to X$ be a rational curve. Then the lift $f_k : \P^1 \to P_k X$ lies in the base locus of $H^0(P_k X, \struct{P_k X}(m))$ for all $m > 0$.
\end{prop}

\begin{proof}
Analogous to the statement that for any symmetric form $\omega \in H^0(X, \nSym{n}{\Omega_X})$ we must have $f^* \omega = 0$ because there are no global pluri-forms on $\P^1$.
\end{proof}

The following result gives some hope of proving the GGL conjecture:

\begin{theorem}[Green-Griffiths, Demailly]
Let $(X, \E)$ be a smooth projective directed variety that lifts to characteristic zero. Suppose that $\det{\E^\vee}$ is big. Then
\[ H^0(P_k(X, \E), \struct{}(m)) \]
has (many) nonzero sections for $m \gg k \gg 1$.
\end{theorem}

\subsection{Induced Foliations}

Suppose that $\omega \in H^0(P_k X, \struct{P_k X}(m))$ is a Demailly-Semple jet. Then the vanishing locus $V(\omega) \subset P_k X$ is pure codimension $1$. For each irreducible component $Z$, consider the following subbundle of the tangent bundle,
\[ \F := \ker{(\T_Z \oplus \struct{P_k X}(-1)|_Z \to \pi^* \T_{P_{k-1}X} |_Z)} \]
Since $X$ is a surface, $\F$ is generically rank $1$ and therefore automatically closed under Lie bracket i.e. it defines a foliation. {\color{red} CHECK THIS} Since every curve $f : C \to X$ lifts to a curve $f_k : C \to P_k X$ parallel to the directed structure (meaning its differential $\d{f_k} : \T_C \to \T_{P_k X}$ factors through $\E_k$)

\begin{rmk}
The curvature map $[-,-] : \wedge^2 \wt{\E} \to \cQ$ where $\cQ$ is the universal quotient bundle must be zero when pulled back to any directed subspace. We call a map $f : (X', \E') \to (X, \E)$ \textit{holonomic} if $\wedge^2 \E' \to f^* \wedge^2 \E \to f^* \cQ$ is zero. {\color{red} IS THIS THE RIGHT DEF}
\end{rmk}


\subsection{$p$-Curvature}

\begin{defn}
Let $k$ be a field of characteristic $p$ and $\partial : A \to A$ a derivation. Then there is a well-defined derivation $\partial^p$ given by 
\[ x \mapsto \underbrace{\partial \cdots \partial}_{p \text{ times}} x \]
This defines a nonlinear map $\T_X \to \T_X$. Given an involutive subsheaf $\F$ the $p$-\textit{curvature} map 
\[ \psi_p : \Frob^* \F \to \T_X / \F \]
is a \textit{linear} $\struct{X}$-module map measuring the failure of $\F$ to be $p$-closed.
\end{defn}

\begin{proof}
To show $\psi_p(\partial_1 + \partial_2)$ we use that
\[ (\partial_1 + \partial_2)^p = \partial_1^p + \partial_2^p + \sum_{i = 1}^{p-1} s_i(\partial_1, \partial_2) \]
but the $s_i$ are Lie polynomials so $s_i(\partial_1, \partial_2) \in \F$ because $\F$ is clsoed under Lie bracket.
We need to show that $\psi_p(f \partial) = f^p \psi_p(\partial)$. Indeed, 
\[ (f \partial)^p = f^p \partial^p + f \ad_{\partial}^{p-1}(f^{p-1}) \partial \]
and because $\ad_{\partial}$ is just the action of $\partial$ on a function so the second term lies in $\F$.
\end{proof}

\newcommand{\Lie}{\mathrm{Lie}}
\newcommand{\g}{\mathfrak{g}}

\begin{lemma}
Let $R$ be a (possibily noncommutative) associative ring of characteristic $p$. Then,
\begin{enumerate}
\item there are universal Lie polynomials $s_i(x,y) \in R \left< x, y \right>$ such that
\[ (a + b)^p = a^p + b^p + \sum_{i = 1}^{p-1} s_i(a,b) \]
where $s_i$ is defined by the relation,
\[ (\ad_{a t + b})^{p-1}(a) = \sum_{i = 1}^{p-1} i s_i(a,b) t^{i-1} \]
\item for all $g, \theta \in R$ such that $\{ \ad_{\theta}^n(g^m) \}_{n,m \ge 0}$ commute with each other,
\[ (g \theta)^p = g^p \theta^p + g \cdot \ad_{\theta}^{p-1}(g^{p-1}) \theta \]
\end{enumerate}
\end{lemma}

\begin{rmk}
$s_i(x,y) \in \Lie(x,y)_{p-1}$ for the $(p-1)^{\text{th}}$-stage of the lower central series $\g_{i+1} = [\g_i, \g]$.
\end{rmk}


\begin{example}
For example, if $p = 2$ then $(a + b)^2 = a^2 + ab + ba + b^2$ so $s_1(a,b) = ab + ba = [a,b]$ because the characteristic is 2.
\end{example}


\begin{lemma}
Let $A,B$ be (possibily noncommutative) associative rings of characteristic $p$ and $M$ is an $(A,B)$-bimodule. Suppose that for some $a \in A$ and $m \in M$,
\[ a \cdot m = \sum_i f_i \cdot m \cdot b_i \]
such that the $f_i \in A$ commute with each other and $b_i \in B$ then
\[ a^p \cdot m - \sum_i f_i^p \cdot m \cdot b_i^p \in \left< r \cdot m \cdot s \mid r \in \ZZ[f_1, \dots, f_r, \Lie(a, f_1, \dots, f_r)_{p-1}] \text{ and } s \in \Lie(b_1, \dots, b_r)_{p-1} \right>  \] 
where the $1$ indicates the first stage of the lower central series.
\end{lemma}


\begin{proof}
$M$ is an $A \ot B^\op$-module and $(a \ot 1 - \sum_i f_i \ot b_i)$ acts as zero. Furthermore,
\[ (a \ot 1 - \sum_i f_i \ot b_i)^p = a^p \ot 1 + \left(- \sum_i f_i \ot b_i \right)^p + \sum_{j = 1}^{p-1} s_j(a \ot 1, -\sum_i f_i \ot b_i) \]
so the last term is in the span of $\Lie(a, f_1, \dots, f_r)_{p-1} \ot b_i$. By Jacobson again,
\[  \left(- \sum_i f_i \ot b_i \right)^p = - \sum_i f_i^p \ot b_i^p + g \]
where $g$ is in the span of $f_i \ot \Lie(b_1, \dots, b_r)_{p-1}$. Therefore,
\[ a^p \cdot m - \sum_i f_i^p \cdot m \cdot b_i^p = -\sum_{j = 1}^{p-1} s_j(a \ot 1, \sum_i f_i \ot b_i) \cdot m  -g \cdot m \]
\end{proof}

\begin{prop}
Suppose that $f : C \to X$ is a map from a curve to a foliated variety $(X, \F)$ meaning that $\d{f} : \T_C \to f^* \F \to f^* \T_X$. Then $f^* \psi_p = 0$. Hence if $X$ is a surface and $\F$ is a foliation by curves then $\im{f} \subset \Delta_p := V(\psi_p)$.
\end{prop}

\begin{proof}
Indeed, this follows from the following lemma.
\end{proof}

\begin{prop}
Let $f : (X, \F_X) \to (Y, \F_Y)$ be a map of foliated varities. Then the diagram
\begin{center}
\begin{tikzcd}
\Frob^* \F_X \arrow[r, "\psi_p"] \arrow[d] & \T_X / \F_X \arrow[d]
\\
f^* \Frob^* \F_Y \arrow[r, "f^* \psi_p"] & f^* (\T_Y / \F_Y) 
\end{tikzcd}
\end{center}
of $\struct{X}$-linear maps commutes.
\end{prop}

\begin{proof}
This is a local statement so we reduce to a ring map $\phi : A \to B$ with a diagram,
\begin{center}
\begin{tikzcd}
\Der[R]{B}{B} \arrow[r, "\partial \mapsto \partial \circ \phi"] & \Der[R]{A}{B} 
\\
M \arrow[u] \arrow[r, "\kappa"] & B \ot_A N \arrow[u] 
\end{tikzcd}
\end{center}
Suppose that $\partial \in M$ we need to show that $\psi_p(\partial) \circ \phi - \psi_p(\kappa(\partial)) \in B \ot_A N$. Indeed, let
\[ \kappa(\partial) = \sum_i b_i \ot \partial_i \]
for $\partial_i \in N$. Then 
\[ \psi_p(\Frob^* \kappa(\partial)) = \psi_p \left(\sum_i b_i^p \ot \partial_i \right) = \sum_i b_i^p \ot \partial_i^p \]
We apply the lemma to $B \ot_A \D_A$ as a $(\D_B, \D_A)$-bimodule and $\kappa$ shows that $\partial \cdot 1 = \sum_i b_i \cdot 1 \cdot \partial_i$. Therefore,
\[ \kappa(\psi_p(\partial)) - \psi_p(\Frob^* \kappa(\partial)) \]
is in the submodule spanned by elements of the form $r \ot s$ for $r$ generated by iterated application of $\partial$ to the $b_i$ and products of $b_i$ and $s$ is generated by commutators of $\partial_i$ which lie in $N$ so we see this is zero in $B \ot_A (\Der[R]{A}{A} / N)$.
\end{proof}


\begin{theorem}
Let $X$ be a smooth projective surface over $k$ a field of characteristic $p$. Suppose that $\omega \in H^0(X, \nSym{n}{\Omega_X})$ is a nonzero symmetric form and $\F_{\omega}$ is the foliation induced on $V(\omega)$. If $\F_{\omega}$ is \textit{not} $p$-closed (on each irreducible component) then $X$ has finitely many rational curves.
\end{theorem}

\begin{proof}
Indeed, $V(\omega) \subset \P(\Omega_X)$ is pure codimension $1$. The induced foliation $\F_\omega$ is assumed to be not $p$-closed on each irreducible component $Z \subset V(\omega)$ therefore $\psi_p : \Frob^* \F \to \T_Z/\F$ is a nonzero map of generically rank $1$-torsion-free sheaves. Hence its zero locus is not dense in $Z$ and hence is dimension $1$. Therefore, the union of these images in $X$ is a finite union of curves. However, we have seen that any rational curve on $X$ must lie inside this locus.
\end{proof}


\begin{example}
Consider two curves $C_1, C_2$ defined as $A_6$-covers of $\P^1$ branched over three points with monodromies
\begin{enumerate}
\item $(1 6)(3 4) \quad (25436) \quad (16452)$
\item $(123)(456) \quad (125) \quad (1465)(23)$
\end{enumerate}
Let $X$ be the minimal resolution of $(C_1 \times C_2)/A_6$. This is a minimal surface with $c_1(X)^2 = c_2(X) = 6$ and $\pi_1(X) = A_4 \times \Z / 5 \Z$ which is finite. Furthermore, $h^{2,0} = 0$ so the cohomology of $X_{\CC}$ is generated by algebraic cycles by the Lefschetz $(1,1)$ theorem. Therefore, $X$ is supersingular when reduced mod any prime of good reduction. From the main two invariants we cannot see that $X$ is not unirational except at $p = 2,3,5$. However, one can show that $\Omega_X$ is big. Choosing a symmetric form $\omega$ we get a foliation $\F_\omega$ and whenever $\F_{\omega}$ is not $p$-closed $X$ is not unirational. The Grothendieck $p$-curvature conjecture predicts this happens infinitely often. One can put this on a computer and start generating a list of primes where $X$ is not unirational.  
\end{example}


\section{AWS Questions}

\subsection{For Moonen}

\begin{enumerate}
\item why do we care about integral Fourier transforms on Chow? To produce these we are taking the chern character of something that more naturally lives in K-theory. Why not just work in K-theory where we have a god-given integral Fourier transform. 
\end{enumerate}

\subsection{For Pries}

\begin{enumerate}
\item Tell here about this family of curves arising from intersections of Fermats, can we apply these results?
\end{enumerate}

\section{}

$k[u,v] \subset k[x,y]$ given by $u \mapsto x + y$ and $v \mapsto xy$ get 
\[ \d{u} \mapsto \d{x} + \d{y} \quad \d{v} \mapsto x \d{y} + y \d{x} \]
can only have forms that look like 
\[ f(x) \d{x} + f(y) \d{y} \]
This is generated by
\[ x^n \d{x} + y^n \d{y} = (n+1)^{-1} \d{(x^{n+1} + y^{n+1})} \]
which is a pullback of a form downstairs. For $n=1$ get
\[ x \d{x} + y \d{y} = 1/2 \d{(u^2 - 2 v)}= u \d{u} - \d{v} \]
Therefore we get $\d{v}$ as long as we get $n = 0,1$. Need a nonvanishing form and a form with a simple pole at each point.

\section{Stein Factorization and Flatness}

\begin{example}
Consider a flat proper curve $f : X \to \Spec{R}$ which is not cohomologically flat. These exist even for $X$ regular by Raynaud's examples. Then for some $R_n := R / \pi^n$ the pullback $X_n \to \Spec{R_n}$ does not have flat global sections. Indeed, by flatness, $\pi^n$ is a non-zerodivisor on $X$ so there is an exact sequence
\begin{center}
\begin{tikzcd}
0 \arrow[r] & \struct{X} \arrow[r, "\pi^n"] & \struct{X} \arrow[r] & \struct{X_n} \arrow[r] & 0
\end{tikzcd}
\end{center} 
therefore we get an exact sequence
\begin{center}
\begin{tikzcd}
0 \arrow[r] & H^0(X, \struct{X}) / \pi^n \arrow[r] & H^0(X_n, \struct{X_n}) \arrow[r] & H^1(X, \struct{X})[\pi^n] \arrow[r] & 0  
\end{tikzcd}
\end{center}
Since $H^0(X, \struct{X}) = R$ we see that $H^0(X_n, \struct{X_n})$ fits into the sequence
\begin{center}
\begin{tikzcd}
0 \arrow[r] & R / \pi^n \arrow[r] & H^0(X_n, \struct{X_n}) \arrow[r] & H^1(X, \struct{X}) [\pi^n] \arrow[r] & 0 
\end{tikzcd}
\end{center}
By cohomology and base change $H^1(X, \struct{X})$ has $\pi^n$-torsion since otherwise it would be flat and this would imply cohomological flatness. By finiteness, the last term must stabilize for $n \gg 0$ and hence we see that $H^0(X_n, \struct{X_n})$ cannot be flat at some stage.
\end{example}

\begin{prop}[Conrad]
If $f : X \to S$ is smooth and proper then its Stein factorization is also smooth and proper. Hence the above behavior cannot occur. 
\end{prop}

\begin{prop}
If $f : X \to S$ is a smooth proper map of varities then $R^i f_* \struct{X}$ are vector bundles that commute with finite base change.
\end{prop}

\begin{proof}
Indeed, the fibers are Du Bois. Is there an easier proof of this fact?
\end{proof}

\begin{prop}[\chref{https://stacks.math.columbia.edu/tag/0E1E}{Tag 0E1E}]
Let $f : X \to S$ be a flat proper finitely presented morphism with reduced fibers. Then the number of geomtric components in the fibers is locally constant.
\end{prop}

Does this imply that $f_* \struct{X}$ is a vector bundle whose formation commutes with base change?

\section{Cohomology and Base Change}

\begin{prop}[\chref{https://stacks.math.columbia.edu/tag/0A1K}{Tag 0A1K}]
Let $S$ be a scheme. Let $f : X \to Y$ be a qcqs morphism of algebraic spaces over $S$. Let $E \in D_{\mathrm{QCoh}}(\struct{X})$. Let $\G^\bullet$ be a bounded above complex of quasi-coherent $\struct{X}$-modules flat over $Y$. Then formation of 
\[ R f_* (E \ot_{\struct{X}}^{\LL} \G^\bullet) \]
commutes with all base change.
\end{prop}

\begin{example}
Some flatness hypothesis is necessary. Otherwise let $X$ be the blowup of $\P^2$ at a point $x$ and consider the base change diagram
\begin{center}
\begin{tikzcd}
E \arrow[d, "f'"] \arrow[r, "g'"] & X \arrow[d, "f"]
\\
x \arrow[r, "g"] & \P^2
\end{tikzcd}
\end{center}
Let $E = \struct{X}$ and $\G^\bullet = \struct{X}(E)$. Then $E \ot^{\LL} (-) = \id$ so we ignore it. Since $\G^\bullet$ is a complex of locally free coherent sheaves we have $\LL (g')^* \G^\bullet = (g')^* \struct{X}(E) = \struct{E}(-1)$ with respect to any isomorphism $E \cong \P^1$. Therefore $\RR f'_* \LL (g')^* \G^\bullet = 0$ since $\struct{\P^1}(-1)$ has no cohomology in any degree. However, $\RR f_* \struct{X}(E) = \struct{\P^2}$ and therefore $\LL g^* \RR f_* \G^\bullet \neq 0$.
\end{example}

\begin{prop}[\chref{https://stacks.math.columbia.edu/tag/0A1P}{Tag 0A1P}]
Let $S$ be a scheme. Let $f : X \to Y$ be a morphism of finite presentation between algebraic spaces over $S$. Let $E \in D(\struct{X})$ be a perfect object. Let $\G^\bullet$ be a bounded complex of finitely resented $\struct{X}$-modules flat over $Y$ with support proper over $Y$. Then,
\[ K = R f_* (E \ot_{\struct{X}}^{\LL} \ot \G^\bullet) \]
is a perfect object of $D(\struct{Y})$ whose formation commutes with arbitrary base change. 
\end{prop}

\begin{proof}
The commutativity with base changes follows from the previous lemma. Therefore, we just need to show that $K$ is perfect. This is local on $Y$ so we may assume that $Y$ is affine. If $Y$ is noetherian the result follows from the construction of the Mumford complex (a detailed proof is at \chref{https://stacks.math.columbia.edu/tag/08IS}{Tag 08IS}). 
\bigskip\\
To reduce to the noetherian case we need to use noetherian approximation and spreading out. The important thing is that the previous proposition gives that the cohomology computed in the noetherian setting commutes with base change to the non-noetherian setting so we win because $\LL f^*$ preserves perfect complexes. 
\end{proof}

\begin{cor}
Let $f : X \to Y$ be a flat proper morphism of algebraic space over $S$. If $E$ is perfect then $\RR f_* E$ is perfect. Hence $R^i f_* E$ are cohomologies of a perfect complex.
\end{cor}

\begin{rmk}
I don't know how to show that pushforward of a coherent sheaf has any nicer finiteness conditions than these even for something like $\P^n_A \to \Spec{A}$ where $A$ is a non-noetherian ring exactly because cohomology and base change don't commute at each level. However, we can conclude that if the sheaf can be resolved by a perfect complex then it total cohomology is perfect. 
\end{rmk}

\begin{rmk}
A perfect complex $E$ over a non-noetherian ring $A$ may have non-finite cohomology. Indeed, the top cohomology is finite but the kernel of a map $A^n \to A^m$ may not be finite. 
\end{rmk}

\section{Quotient Singularities}

\begin{theorem}[\chref{https://arxiv.org/pdf/1908.01416.pdf}{RPST}]
Let $(R. \m, \kappa)$ be an excellent strictly Henselian local ring of mixed characteristic\footnote{I think this includes pure characteristic zero}. Suppose that $R$ is a Gorenstein rational singularity of dimension $2$. Then $R$ is a rational double point and there exists a finite cover $Y \to X = \Spec{R}$ with $Y$ regular such that $\struct{X} \to \pi_* \struct{Y}$ splits as a map of $\struct{X}$-modules. 
\end{theorem}

\begin{rmk}
In pure characteristic zero, this is a classic \chref{https://projecteuclid.org/journals/duke-mathematical-journal/volume-34/issue-2/Local-classification-of-quotients-of-complex-manifolds-by-discontinuous-groups/10.1215/S0012-7094-67-03441-2.short}{theorem of Prill} who proved that $R$ is the quotient of a finite subgroup of $\SL_2$. 
\end{rmk}


\section{Smoothness is fpqc local}

\begin{lemma}
Let $f : A \to B$ be a ring map and $M$ an finitely presented $A$-module. Then $\Hom{A}{M}{N} \ot_A B = \Hom{B}{M \ot_A B}{N \ot_A B}$.
\end{lemma}

\begin{proof}
Let $A^n \to A^m \to M \to 0$ be a presentation. Then consider the diagram,
\begin{center}
\begin{tikzcd}
0 \arrow[r] & \Hom{A}{M}{N} \ot_A B \arrow[d] \arrow[r] & \Hom{A}{A^m}{N} \ot_A B \arrow[d] \arrow[r] & \Hom{A}{A^n}{N} \arrow[d]
\\
0 \arrow[r] & \Hom{B}{M \ot_A B}{N \ot_A B} \arrow[r] & \Hom{B}{B^m}{N \ot_A B} \arrow[r] & \Hom{B}{B^n}{N \ot_A B}
\end{tikzcd}
\end{center}
The second two downward maps are isomorphisms and hence so is the first.
\end{proof}

\begin{lemma}
Let $\phi : A \to B$ a flat local ring map. Let $M$ be a finitely presented $A$-module such that $M \ot_A B$ is free. Then $M$ is free.
\end{lemma}

\begin{proof}
Indeed, it suffices to show that $M$ is projective. To show that $\Hom{A}{M}{-}$ is exact it suffices to show that $\Hom{A}{M}{-} \ot_A B$ is exact because $\phi$ is automatically faithfully flat since it is local. Because $M$ is finitely presented $\Hom{A}{M}{-} \ot_A B = \Hom{B}{M \ot_A B}{(-) \ot_A B}$ which is exact since $\phi$ is flat and $M \ot_A B$ is projective.
\end{proof}

\begin{lemma}
Let $\phi : A \to B$ be a flat local ring map of Noetherian rings. Suppose that $B$ is regular then $A$ is regular.
\end{lemma}

\begin{proof}
We will show that $A$ has finite global dimension. Let $M$ be a finite $A$-module and $P^\bullet \to M$ a resolution by finite free $A$-modules. We want to show that $K = \ker{(P^{d-1} \to P^{d-2})}$ is free where $d = \dim{B}$. Because $\phi$ is flat, $-\ot_A B$ produces a finite free resolution $P^\bullet \ot_A B \to M \ot_A B$ with $K \ot_A B = \ker{(P^{d-1} \ot_A B \to P^{d-2} \ot_A B)}$. Since $B$ is regular, by Schanuel's lemma, $K \ot_A B$ is free and hence $K$ is free because $K$ is finitely presented using that $A$ is noetherian.
\end{proof}

\begin{prop}
Let $f : X \to S$ be a morphism of finite presentation. Let $\{ X_i \to X \}_{i \in I}$ be an fpqc cover of $X$ such that each $X_i \to S$ is smooth. Then $f : X \to S$ is smooth.
\end{prop}

\begin{proof}
We will show that $f$ is flat with geometrically regular fibers. For any $x \in X$ we can choose affine opens $f : U \to V$ restricting $f : X \to S$ such that $x \in U$. By the definition of fpqc covering, there is some $U' \subset X_i$ for some $i$ an affine open so that $U' \to U$ is flat and $x \in U$ is in the image. Let $x' \in U'$ be a preimage of $x$ and let $s \in V$ be the image in $V$. Then there are ring maps $\stalk{V}{y} \to \stalk{U}{x} \to \stalk{U'}{x'}$ whose composition is flat and the second is a flat local map hence faithfully flat. Thus $\stalk{V}{y} \to \stalk{U}{x}$ is flat. Now we base change along $\Spec{\bar{\kappa(s)}} \to V$. The fiber $U'_{\bar{s}}$ is regular (and noetherian) because it is smooth by assumption and $U_{\bar{s}}$ is noetherian because it is finite type over $\bar{s}$. The map $U_{\bar{s}}' \to U_{\bar{s}}$ is flat and hits every point of $U_{\bar{s}}$ over $x \in U$ so we reduce to the lemmas to prove that each local ring is regular.
\end{proof}

\section{Measures of Irrationality}

\begin{example}
If $C$ is a trigonal curve (meaning it has gonality $3$) then $C \embed \P^{g-1}$ enbedded via $|K_C|$. Then $C$ has a $1$-dimensional family of tri-secant lines. 
\end{example}

\newcommand{\gon}{\mathrm{gon}}

\begin{theorem}[Noether]
If $C_d \subset \P^2$ is smooth then $\gon(C) = d-1$.
\end{theorem}

\begin{lemma}
Suppose $K_C$ separates $r$ points meaning
\[ H^0(C, K_C) \onto H^0(Z, K_C) \]
is surjective
for $Z \subset C$ any collection of $r$ distinct points. Then $\gon(C) \ge r  + 1$.
\end{lemma}

\begin{proof}
Suppose there exists a map $f : C \to \P^1$ of degree $r$. Then there is some line bundle $\L$ of degree $r$ and $h^0(C, \L) = 2$ (otherwise I could lower the degree). Let $Z$ be a genercal fiber of $C \to \P^1$ then there is a sequence
\begin{center}
\begin{tikzcd}
H^0(K_C) \arrow[r] & H^0(K_C|_Z) \arrow[r] & H^1(K_C \ot \I_Z) \arrow[r] & H^1(K_C) \arrow[r] & 0
\end{tikzcd}
\end{center}
but $H^1(K_C)$ is nonzero so we see that $H^1(K_C \ot \I_Z) = H^0(A)^\vee$ has dimension at least $2$ and therefore the first map is not surjective.
\end{proof}

\begin{proof}[Proof of Noether's Theorem]
Let $C \subset \P^2$ be a smooth curve of degree $d$. Then $K_C = \struct{C}(d-3)$ separates at least $d - 2$ points so by the lemma $\gon(C) \ge d - 1$. 
\end{proof}

\subsection{Higher Dimensions}

\newcommand{\irr}{\mathrm{irr}}
\newcommand{\covgon}{\mathrm{covgon}}

\begin{defn}
Let $X$ be a smooth projective variety of dimension $n$. The \textit{degree of irrationality} $\irr(X)$ is the minimal degree of a dominant rational map $f : X \rat \P^n$. 
\end{defn}

\begin{rmk}
We consider all rational maps so that $\irr(X)$ is clearly irrational. Furthermore $\irr(X) = 1 \iff X$ is rational. 
\end{rmk}

This was defined by Heinzer-Moh where they considered the set
\[ \{ e > 0 \mid \exists X \rat \P^n \text{ degree } e \}  \subset \N \]
If $X$ is a cure, this is a semigroup under $+$ given by tensor product on the defining line bundles. 
\bigskip\\
Let $X_d \subset \P^{n+1}$ hypersurface.

\begin{theorem}[BCdP, BdPELU]
If $n \ge 2$ and $d \ge 2 n + 2$ and $X$ is very general then $\irr(X_d) = d - 1$ and any map of degree $d-1$ is birationally equivalent to the projection map. 
\end{theorem}

\subsection{Covering Gonality}

\begin{defn}
The \textit{covering gonality} of $X$ is the smallest $\delta > 0$ such that there is a covering family of curves on $X$ so that the general member has gonality $\delta$. 
\end{defn}

\begin{theorem}[LP,BdPELU]
Let $n \ge 2$ and $d \ge n + 2$ and $X$ is a hypersurface with canonical singularities then $\covgon(X) \ge d - n$. 
\end{theorem}

\begin{example}[Lopez-Pirola]
For $n = 2$ we have $X \subset \P^3$ a hypersurface of degree $d$. Consider the covering family given by the intersection of $X$ with a tangent plane. Then $\gon(T_p X \cap X) =  d - 2$ because generically the are nodal and we can project from the node. 
\end{example}

\subsection{Gonality of Curves}

Suppose we define $\gon(C) = \gon(\wt{C})$ for any integral curve $C$. Then given a flat family $\C \to T$ of integral curves then
\[ \gon(C_t) \ge \gon(C_0) \]
for $t$ a general point. Moreover, this function is constructible and hence is lower semi-continuous. 

\begin{rmk}
Ciliberto-de Poi-Flamini-Supino compute covgon for $X_d \subset \P^{n+1}$ very general. 
\end{rmk}

\begin{rmk}
G. Smith: covgon in characteristic $p$.
\end{rmk}

\subsection{Products of Curves}

What do the measures of irrationality of $C \times D$ look like?

\begin{lemma}[Heinzer-Moh]
Let $X$ be smooth projective and $\phi : X \rat C$ be dominant map. Then $\irr(X) \ge \gon(C)$.
\end{lemma}

\begin{proof}
Suppose $\irr(X) = d$ then we need to show $\gon(C) \le d$. Consider 
\begin{center}
\begin{tikzcd}
X \arrow[d, dashed] 
\\
\P^n \arrow[r, dashed] & \nSym{d}{X} \arrow[r, "\nSym{d}{\phi}"] & \nSym{d}{C} \arrow[d]
\\
& & \fPic^d_C
\end{tikzcd}
\end{center}
Since $\p^n \to \fPic^d_C$ must be constant and therefore the image of $\P^n$ which is nonconstant lies in a fiber of $\nSym{d}{C} \to \fPic^d_C$ which is exactly the linear series $|L_d|$ of some line bundle of degree $d$. Since this map is nonconstant, $|L_d|$ has dimension at least $1$ so we win.
\end{proof}

\begin{rmk}
This is false if we replace $C$ by a higher-dimensional variety. Examples of Yoshihara. 
\end{rmk}

Therefore we get,
\[ \gon(C) \cdot \gon(D) \ge \irr(C \times D) \ge \max \{ \gon(C), \gon(D) \} \]

\begin{conj}[C-Martin]
If $C,D$ are very general then $\irr(C \times D) = \gon(C) \cdot \gon(D)$.
\end{conj}

\begin{theorem}[C-Martin]
If $C, D$ are smooth hyperelliptic and genus $\ge 2$ then $\irr(C \times D) = 4$. 
\end{theorem}

\begin{theorem}
Let $C, D$ be curves such that $g(C) \ge gon(C)^2$ and $g(D) \ge \gon(D)^2$ then $\irr(C \times D) = \gon(C) \cdot \gon(D)$.
\end{theorem}

\subsection{Open problems}

\begin{enumerate}
\item general products of curves
\item elliptic curves is $\irr(E \times E') = 3$ or $4$
\item If $X \to T$ is a smooth family, is $\irr$ lower semicontinuous? 
\end{enumerate}

\begin{example}[Yoshihara]
$C \embed E \times E'$ with $C$ a smooth curve of genus $3$ this implies $\irr(E \times E') = 3$.
\end{example}

\subsection{Abelian Surfaces}

$\cA_{(1,d)}$ moduli of $(1,d)$-polarized abelian surfaces, these are $3$-dimensional moduli spaces parametrizing $(A, L_d)$ where $L_d^2 = 2d$ and $h^0(L_d) = d$

\begin{theorem}[C-Stapleton]
$\irr(A_d) \le 4$
\end{theorem}

The general $A$ has $\rho(A) = 1$. For $d \neq 1,3$ then $\irr(A_d) = 4$ (Martin) 


\section{Stacks and Du Bois Complex}

\newcommand{\R}{\mathbf{R}}

Let $\X$ be a separated DM-stack and $\pi : \X \to X$ its coarse space. Then we need to claim that $\R \pi_* \ul{\Omega}_{\X} = \ul{\Omega}_{X}$. Because we know $\ul{\Omega}_{\X}$ makes sense if it is the $h$-topology thing. Then 


\begin{defn}
A morphism $p : \wt{X} \to X$ in $\Sch$ is a \textit{topological epimorphism} if $p$ is surjective and the Zariski topology of $X$ is the quotient topology of the Zariski topology of $\wt{X}$. It is a \textit{universal topological epimorphism} if this remains true after any base change.
\end{defn}

\begin{defn}
The $h$-topology on $\Sch$ is the Grothendieck topology whose covers $\{ U_i \to X \}$ are such that
\[ \bigsqcup U_i \to X \]
is a universal topological epimorphism.
\end{defn}

\begin{example}
The following are $h$-covers
\begin{enumerate}
\item flat covers
\item proper surjective morphisms
\item quotients by the operation of a finite group
\end{enumerate}
\end{example}

\begin{prop}
Let $(X', Z)$ be an abstract blow-up. This means there is a cartesian square
\begin{center}
\begin{tikzcd}
E \pullback \arrow[d] \arrow[r] & X' \arrow[d, "f"]
\\
Z \arrow[r] & X
\end{tikzcd}
\end{center}
where $f$ is proper, $Z \subset X$ is closed and $f$ induces an isomorphism $X' \sm E \iso X \sm Z$. Let $\F$ be an $h$-sheaf. Then the square
\begin{center}
\begin{tikzcd}
\F(X') \arrow[r] \arrow[d] & \F(E)
\\
\F(X) \arrow[u] \arrow[r] & \F(Z) \arrow[u]
\end{tikzcd}
\end{center}
is Cartesian.
\end{prop}

\newcommand{\Zar}{\mathrm{Zar}}

\begin{defn}
Let $\Omega^p \in \PSh(\Sch)$ be the presheaf
\[ \Omega^p : X \mapsto \Gamma(X, \Omega^p_X) \]
Let $\Omega^p_h$ be its sheafification in the $h$-topology.
\end{defn}

Let $k$ be an algebraically closed field of characteristic zero.

\begin{prop}[AJ Thm. 3.6]
Let $X$ be a smooth variety over $k$. Then $\Omega_h^p(X) = \Gamma(X, \Omega^p_X)$. 
\end{prop}

\begin{defn}
Let $\rho : \Sch_h \to \Sch_{\Zar}$ be the canonical continuous morphism of sites. For $X \in \Sch$ let $\rho_X : \Sch_h \to X_{\Zar}$ be the inclusion of $X$ with the Zariski topology. Since $\rho^*$ is $h$-sheafification which is exact, this defines a morphism of topoi. 
\end{defn}

\begin{prop}[Prop 6.19]
Let $\pi : \wt{X} \to X$ be a resolution of a variety of dimension $d$. Then 
\[ \R (\rho_X)_* \Omega^{d}_h = \R \pi_* \Omega_{\wt{X}}^d \]
\end{prop}

\subsection{Hypercovers}

We need special hypercoverings that satisfy cohomological descent. 

\begin{defn}
We say that 
\end{defn}

\begin{defn}[Delgine]
We say that a simplicial map $X_\bullet \to Y_\bullet$ of topological spaces is a \textit{k-truncated hypercovering for the universal cohomological descent topology} if the maps 
\[ \varphi_n : \cosk{X_\bullet} \to \cosk \sk_n X_\bullet \]
induces
\[ (\varphi_n)_{n+1} : X_{n+1} \to (\cosk \sk_n X_\bullet)_{n+1} \]
is a universal cohomological descent map. 
\end{defn}

\begin{defn}
An $h$-hypercover  



\end{defn}

\begin{prop}
If $X_\bullet \to Y_\bullet$ is an $h$-hypercover then for all $h$-sheaves, the map of cosimplicial sets
\[ H_h^i(Y_\bullet, \F) \to H^i_h(X_\bullet, \F) \]
is an isomorphism.
\end{prop}

\subsection{The Du Bois Complex}

Let $\d : \Omega^p_h \to \Omega^{p+1}_h$ be the $h$-sheafification of the exterior differential on $p$-forms. We call $\Omega_h^\bullet$ the \textit{the algeraic de Rham complex in the h-topology}. We denote by $H_h^i(X, \Omega^\bullet_h)$ its hypercohomology in the $h$-topology. The stupid filtration
\[ F^p \Omega^\bullet_h := [ \cdots \to 0 \to \Omega^p_h \to \Omega_h^{p+1} \to \cdots] \subset \Omega^\bullet_h \]
is called the \textit{Hodge filatration}.

\begin{prop}
Let $X \in \Sch$. Then $H^i_{\dR}(X) = H_h^i(X, \Omega_h^\bullet)$.
\end{prop}

\begin{proof}
Let $X_\bullet \to X$ be a proper hypercover with all $X_n$ smooth. By the hypercohomology spectral sequence and Prop. 6.10
\[ H^i_h(X, \Omega_h^\star) = H^i_h(X_\bullet, \Omega^\bullet_h) = H^i_{\Zar}(X_\bullet, \Omega^\star_h) \]
\end{proof}

\begin{cor}
Let $k = \CC$. The embedding induces a natural isomorphism
\[ H^i_h(X, \Omega_h^i) \cong H^i_{\text{sing}}(X_{\CC}^\an, \CC) \]
\end{cor}

\begin{theorem}[Theorem 7.7]
Let $X \in \Sch$ be proper. Then the Hodge-to-de Rham spectral sequence
\[ E_1^{p,q} = H^q_h(X, \Omega_h^p) \implies H^{p+q}_h(X, \Omega^\bullet_h) \]
degenerates at $E_1$. The Hodge filtration on the complex induces the Hodge filtration on $H_{\dR}^i(X)$.
\end{theorem}

\begin{defn}
Let $X$ be a variety, $\pi : X_\bullet \to X$ a proper hypercover with all $X_n$ smooth (but not necessarily connected) then
\[ \ul{\Omega}_X^\bullet := \R \pi_* \Omega^\bullet_{X_\bullet} \]
is the Du Bois complex of $X$ filtered by
\[ F^p \ul{\Omega}_X^{\bullet} = \R \pi_* F^p \Omega^\bullet_{X^\bullet} \]
\end{defn}

\begin{theorem}
Consider $\Omega^\bullet_h$ with the Hodge filtration as an object of $D^+ F(\Sh(\Sch_h))$ where $\Sh(\Sch_h)$ is the category of sheaves of abelian groups on $\Sch_h$. Then
\[ \R (\rho_X)_* \Omega^\bullet_X \iso \ul{\Omega}_X^\bullet \]
\end{theorem}

\begin{proof}
Choose a proper hypercover $\pi : X_\bullet \to X$ with $X_n$ smooth. We need to show that
\[ \R (\rho_X)_* \Omega_h^\bullet \cong \R \pi_* \Omega^\bullet_X \]
Using the Hodge filtration on both sides  we reduce to Corollary 6.16.
\end{proof}

\subsection{h-differentials on stacks}


Let $\X$ be an algebraic stack and $\X_{\fppf}$ the big fppf site.

Some questions: 

\begin{enumerate}
\item How do we deal with the 2-categorical site for sheaves on stacks
\item Does alpers notes say that lisse-etale is funtorial (get map of topoi?)
\item How does $\Sch_h \to X_{\Zar}$ become a continuous map of sites since it does not preserve fiber products. Can I do the same with stacks?
\item is exceptional pushforward or pullback the one that commutes with colimits in lemma 1.2.1?
\item how is functoriality for a map of stacks $f : \X \to \Y$ actually defined since the pullback of a scheme \etale over $\Y$ is not a scheme anymore? Maybe only the map of topoi exists? 
\end{enumerate}

Answers

\begin{enumerate}
\item you dont, $\X_{\fppf}$ is an ordinary $1$-site just give $\X$ as a category the topology given by fppf covers on the underlying maps of schemes.

\item I think 

\item you dont need absolute (fiber) products, just fiber products along maps $U_i \to U$ part of a covering family. These do pullback correctly to fiber products since we remember the base of the fiber product.  

\item it is $g^{-1}$ in question, but commutativity with \textit{limits} is the operational part

\item you don't use the pullback map you use the inclusion $\X_{\fppf} \to \Y_{\fppf}$ which is both cocontinuous (since covers are the same) and continuous since it sends covers to covers and preserves fiber products by covers. For a scheme this is just the postcomposition map $g^{\to}$ that we saw gives the same map of topoi as fiber product in the other direction.
\end{enumerate}

First step, check that under $\Sch_h \to X_{\et}$ we get the right object. Well $\Sch_h \to X_{\et} \to X_{\Zar}$ does give the correct object. We need to show that if $\tau_X : \Sch_h \to X_{\et}$ is the canonical map then $\tau_X : \Sch_h \to \Sch_{\et} \to X_{\et}$ gives a map of topoi and I want to show that $\R (\tau_X)_* \Omega_h^\bullet = (\ul{\Omega}_X^\bullet)^{\et}$ the sheafification of the complex of Zariski sheaves.

If $g : U \to X$ is an \etale map then I want to show that $g^* \ul{\Omega}_X^\bullet = (\R (\tau_X)_* \Omega_h^\bullet)|_{U_{\Zar}}$ but $\Sch_h \to \Sch_\et \to U_{\et} \to U_{\Zar}$ is $\Sch_h \to \Sch_{\Zar} \to U_{\Zar}$ and thereore 
\[ (\R (\tau_X)_* \Omega_h^\bullet)|_{U_{\Zar}} = \R (\rho_U)_* \Omega_h^\bullet = \ul{\Omega}_U^\bullet \]
which is isomorphic to what we want because the Du Bois complex is compatible with \etale covers {\color{red} WHY?}

\begin{prop}
Let $\X$ be a smooth separated DM-stack over $k$. Then $\R (\tau_{\X})_* \Omega^\bullet_h = \Omega^\bullet_{\X}$.
\end{prop}

\begin{proof}
Indeed, by the previous calculation $\R (\tau_{\X})_* \Omega_h^\bullet |_{U_{\Zar}} = \Omega_{U}^\bullet$ for any $U \to \X$ \etale. {\color{red} CHECK}
\end{proof}

\begin{cor}
Let $\X$ be a smooth separated DM-stack and $\pi : \X \to X$ its coarse space then $\ul{\Omega}_X^k = \Omega_X^{[k]}$.
\end{cor}

\begin{proof}
Indeed $\ul{\Omega}_X^\bullet = \R (\rho_X)_* \Omega^\bullet_h = \R \pi_* \R (\rho_{\X})_* \Omega^\bullet_h = \R \pi_* \Omega^\bullet_{\X}$. Moreover, the filtrations are preserved so $\ul{\Omega}_X^k = \R \pi_* \Omega^k_{\X}$ {\color{red} CHECK THIS BIGTIME} Since $\pi$ is a GMS we see that $\ul{\Omega}_X^k$ is supported in degree zero and it is reflexive by a calculation {\color{red} DO THIS!!}
\end{proof}

We want to show some cohomology vanishing of $\R (\tau_\X)_* \Omega_h^\bullet$ for $\X$ a smooth algebraic stack. In particular, we want to show that $\R (\tau_\X)_* \Omega_h^k$ is supported in one degree. To do this, we restrict to smooth covers $U \to \X$ and we compute but there it is just the Du Bois complex of $U$ so we win because $U$ is smooth over $k$. Now maybe reflexivity of the pushforward is hard. By the same logic, we should be able to show that it is a vector bundle on $\X$ in some sense? Maybe along a GMS the pushforward of a vector bundle is reflexive? 

\subsection{Checks}

Consider the maps of categories $\X_{\fppf} \to \Sch_{\fppf} \to \Sch_h$ which are just the projection and then the identiy on underlying categories. These are continuous cocontinuous maps. 

The first map probably does not define a morphism of topoi but it does give an exact pullback $g_!$ on the category of abelian sheaves by \chref{https://stacks.math.columbia.edu/tag/00XT}{00XT}

The second map defines an opposite morphism of topoi via \chref{https://stacks.math.columbia.edu/tag/00X6}{Tag 00X6}.

\begin{rmk}
One place we need exactness of $f^*$ is for $f_*$ to preserve injectives. Otherwise, we may not have a Grothendieck spectral sequence. 
\end{rmk}


\begin{prop}
Consider functors between sites $\C \xrightarrow{u} \C' \xrightarrow{v} \C''$. Suppose these are continuous. Then $v^s = u_s \circ (v \circ u)^s$.
\end{prop}

\begin{proof}

\end{proof}

\section{Remarks on the Du Bois Complex of Stacks}

\newcommand{\Sm}{\mathbf{Sm}}

Following [HJ] fix a field $k$ of characteristic $0$. By \textit{scheme} and \textit{stack} we mean a separated scheme/algebraic stack of finite type over $k$. A \textit{variety} is a reduced (separated finite type over $k$) scheme. Let $\Sch$ and $\Var$ and $\Sm$ be the categories of schemes, varities, and smooth varities respectively. 
\bigskip\\
A continuous map/morphism of sites $f : \C \to \D$ is a functor $f^{-1} : \D \to \C$ in the opposite direction that preserves covering families and base change along morphisms in a covering family. We call such a functor $f^{-1} : \D \to \C$ continuous. 
\bigskip\\
See [HJ, section 1] for more conventions I will try to follow and the stacks project sections on sites and sheaves for the definitions of continuous, cocontinuous functors and their associated functoriality for sheaves.

\subsection{The Du Bois Complex for a Scheme}

Let $\d : \Omega^p_h \to \Omega^{p+1}_h$ be the $h$-sheafification of the exterior differential on $p$-forms. We call $\Omega_h^\bullet$ the \textit{the algeraic de Rham complex in the h-topology}. We denote by $H_h^i(X, \Omega^\bullet_h)$ its hypercohomology in the $h$-topology. The stupid filtration
\[ F^p \Omega^\bullet_h := [ \cdots \to 0 \to \Omega^p_h \to \Omega_h^{p+1} \to \cdots] \subset \Omega^\bullet_h \]
is called the \textit{Hodge filatration}.



\begin{defn}
Let $X$ be a variety, $\pi : X_\bullet \to X$ a proper hypercover\footnote{We always mean a hypercover in the sense of Hodge III which are required to satisfy cohomological descent.} with all $X_n$ smooth (but not necessarily connected) then
\[ \ul{\Omega}_X^\bullet := \R \pi_* \Omega^\bullet_{X_\bullet} \]
is the Du Bois complex of $X$ filtered by
\[ F^p \ul{\Omega}_X^{\bullet} = \R \pi_* F^p \Omega^\bullet_{X^\bullet} \]
\end{defn}

\begin{defn}
Let $\rho : \Sch_h \to \Sch_{\Zar}$ be the canonical continuous morphism of sites. For $X \in \Sch$ let $\rho_X : \Sch_h \to X_{\Zar}$ be the inclusion of $X$ with the Zariski topology. Since $\rho^*$ is $h$-sheafification which is exact, this defines a morphism of topoi. 
\end{defn}

\begin{theorem}[HJ Theorem 7.12]
Consider $\Omega^\bullet_h$ with the Hodge filtration as an object of $D^+ F(\Sh(\Sch_h))$ where $\Sh(\Sch_h)$ is the category of sheaves of abelian groups on $\Sch_h$. Then
\[ \R (\rho_X)_* \Omega^\bullet_X \iso \ul{\Omega}_X^\bullet \]
\end{theorem}

\begin{proof}
Choose a proper hypercover $\pi : X_\bullet \to X$ with $X_n$ smooth. We need to show that
\[ \R (\rho_X)_* \Omega_h^\bullet \cong \R \pi_* \Omega^\bullet_X \]
Using the Hodge filtration on both sides the problem reduces to the following.
\end{proof}

\begin{lemma}[HJ Lemma 6.16]
Let $X$ be a scheme and $\pi : X_\bullet \to X$ a proper hypercover with $X_n$ smooth for all $n$. Then
\[ \R (\rho_X)_* \Omega^p_h = \R \pi_* \Omega^p_{X_\bullet} \]
moreover:
\begin{enumerate}
\item the complex is concentrated in degrees at most $\dim{X}$
\item it vanishes for $p > \dim{X}$
\item all cohomology sheaves are coherent.
\end{enumerate}
\end{lemma}

As Ravi suggested, this makes Du Bois' theorem immediate. 

\begin{cor}
Let $X$ be a variety. Then $\ul{\Omega}_{X}^\bullet \in D^+ F(\Sh(X_{\Zar})$ is independent of the choice of hyperresolution $X_\bullet \to X$.
\end{cor}

\subsection{Functoriality}

\renewcommand{\big}{\mathrm{big}}
\renewcommand{\small}{\mathrm{small}}

Notice that functoriality of $\rho_X$ is a little weird. For a map $f : X \to Y$ of schemes, the obvious diagram
\begin{center}
\begin{tikzcd}
\Sch_h \arrow[r, "\rho_X"] \arrow[rd, "\rho_Y"'] & X_{\Zar} \arrow[d, "f"]
\\
& Y_{\Zar}
\end{tikzcd}
\end{center}
does NOT commute because $\rho$ is given by inclusion by $f$ is given by preimage. However, the underlying functors for the big sites do commute
\begin{center}
\begin{tikzcd}
\Sch_h \arrow[from=r, "u_X"'] \arrow[from=rd, "u_Y"] & (\Sch_X)_{\Zar} \arrow[d, "u_f"]
\\
& (\Sch_Y)_{\Zar} 
\end{tikzcd}
\end{center}
with $u_f$ the cocontinuous continuous map 
\[ (U \to X) \mapsto (U \to Y) \]
whose associated morphism of topoi $(u^s, {}_s u)$ is $f_{\big} : \Sh((\Sch_X)_{\Zar}) \to \Sh((\Sch_X)_{\Zar})$ the same map associated to the continuous map $v_f : (\Sch_Y)_{\Zar} \to (\Sch_X)_{\Zar}$ given by pullback. Furthermore $f_{\big !} = u_s$ is left-adjoint to $f_{\big}^{-1}$. Therefore, as a morphism of topoi,
\[ \rho_X^\big = ((u_X)_s, (u_X)^s) = ( (u_Y)_s \circ (u_f)_s, (u_f)^s \circ (u_Y)^s) \]
Therefore, we conclude that
\[ (\rho_X^\big)^{-1} = (\rho_Y^\big)^{-1} \circ f_{\big !} \quad \quad \rho_{X*}^\big = f_{\big}^{-1} \circ \rho_{Y *}^\big \]
Therefore we have functoriality in the sense of
\begin{center}
\begin{tikzcd}
\Sh(\Sch_h) \arrow[r, "\rho_X^{\big}"] \arrow[rd, "\rho_Y^{\big}"'] & \Sh((\Sch_X)_{\Zar}) \arrow[from=d, "(f_! \dashv f^{-1})"']
\\
& \Sh((\Sch_Y)_{\Zar}) 
\end{tikzcd}
\end{center}
Note that $f_!$ not in general exact. Moreover, the Zariski version of \chref{https://stacks.math.columbia.edu/tag/021G}{Tag 021G} gives maps of topoi
\[ \Sh(X_{\Zar}) \xrightarrow{\iota_X} \Sh((\Sch_X)_{\Zar}) \xrightarrow{\pi_X} \Sh(X_{\Zar})\] 
composiing to the identiy with $\pi_X$ arising from the obvious morphism $\pi_X : (\Sch_X)_{\Zar} \to X_{\Zar}$ of sites. The morphism of sites commutes,
\begin{center}
\begin{tikzcd}
\Sh(\Sch_X)_{\Zar}) \arrow[d, "f_{\big}"] \arrow[r, "\pi_X"] & \Sh(X_{\Zar}) \arrow[d, "f_{\small}"] 
\\
\Sh((\Sch_Y)_{\Zar}) \arrow[r, "\pi_Y"] & \Sh(Y_{\Zar})
\end{tikzcd}
\end{center}
there are natural maps
\[ f_{\small}^{-1} \circ \pi_{Y*} \to \pi_{X*} \circ f_{\big}^{-1} \]
arising by adjunction from
\[ \pi_{Y_*} \to \pi_{Y*} \circ f_{\big *} \circ f_{\big}^{-1} = f_{\small *} \circ \pi_{X*} \circ f_{\big}^{-1} \] 
This gives a map
\[ \rho_{X*} = \pi_{X*} \circ \rho_{X*}^\big = \pi_{X*} \circ f_{\big}^{-1} \circ \rho_{Y *}^{\big} \leftarrow f_{\small}^{-1} \circ \pi_{Y *} \circ \rho_{Y*}^\big = f_{\small}^{-1} \circ \rho_{Y*}  \]
If we take derived functors this gives the map
\[ f^{-1}_{\small} \ul{\Omega}_Y^\bullet = f^{-1}_{\small} \R \rho_{X*} \Omega_h^\bullet \to \R \rho_{Y*} \Omega_h^\bullet = \ul{\Omega}_X^\bullet \]
expressing functoriality of the Du Bois complex. Here we use that $f^{-1}_{\small}$ is exact. 

\subsection{The Du Bois Complex for Stacks}

\newcommand{\Et}{\mathrm{\'{E}t}}

Our proposed definition is to imitate the definition using the $h$-topology but replacing the target Zariski topology by the \etale topology. Let $\X_{\et}$ denote the small \etale site of a DM-stack or scheme and $\X_{\Et}$ denote the big \etale site of an arbitrary stack or scheme.

\begin{defn}
Let $\X$ be a DM-stack. Then let $\rho_{\X}^{\et} : \Sch_h \to \X_{\et}$ be the continuous map of sites whose underlying functor is $u^{\et} : (U \to \X) \mapsto U$.
\end{defn} 

\begin{defn}
Let $\X$ be any stack. Then let $\rho_{\X}^{\Et} : \Sch_h \to \X_{\Et}$ be the continuous map of sites whose underlying functor is $u^{\Et} : (U \to \X) \mapsto U$.
\end{defn} 

\begin{lemma}
The induced maps $\rho_{\X}^{\et} : \Sh(\Sch_h) \to \Sh(\X_{\et})$ and $\rho_{\X}^{\Et} : \Sh(\Sch_h) \to \Sh(\X_{\Et})$ are morphisms of topoi. 
\end{lemma}

\begin{proof}
It suffices to show $(\rho_{\X}^{\et})^{-1}$ and $(\rho_{\X}^{\Et})^{-1}$ are exact. This follows from \chref{https://stacks.math.columbia.edu/tag/00XS}{Tag 00XS} since the functor $(U \to \X) \mapsto U$ is continuous, cocontinuous, and preserves fiber products and equalizers.
\end{proof}

\subsubsection{Functoriality for DM-Stacks}

As before, there is a comparison diagram between the big and small \etale topoi. 

\begin{center}
\begin{tikzcd}
\Sh(\X_{\Et}) \arrow[d, "f_{\big}"] \arrow[r, "\pi_{\X}"] & \Sh(\X_{\et}) \arrow[d, "f_{\small}"] 
\\
\Sh(\Y_{\Zar}) \arrow[r, "\pi_{\Y}"] & \Sh(\Y_{\et})
\end{tikzcd}
\end{center}
and $\rho_{\X}^{\et} = \pi_{\X} \circ \rho_{\X}^{\Et}$. Furthermore, the following relations hold
\[ (\rho_{\X}^{\Et})^{-1} = (\rho_{\Y}^\Et)^{-1} \circ f_{\big !} \quad \quad \rho_{\X*}^\Et = f_{\big}^{-1} \circ \rho_{\Y *}^\et \]
As before, the natural maps
\[ f_{\small}^{-1} \circ \pi_{\Y*} \to \pi_{\X*} \circ f_{\big}^{-1} \]
give
\[ \rho_{\X*}^{\et} = \pi_{\X*} \circ \rho_{\X*}^\Et = \pi_{\X*} \circ f_{\big}^{-1} \circ \rho_{Y *}^{\Et} \leftarrow f_{\small}^{-1} \circ \pi_{\Y *} \circ \rho_{Y*}^\Et = f_{\small}^{-1} \circ \rho_{Y*}^{\et}  \]
If we take derived functors this gives the map
\[ f^{-1}_{\small} \ul{\Omega}_{\Y}^\bullet = f^{-1}_{\small} \R \rho_{\X*}^{\et} \Omega_h^\bullet \to \R \rho_{\Y*}^{\et} \Omega_h^\bullet = \ul{\Omega}_{\X}^\bullet \]
expressing functoriality of the Du Bois complex. Here we use that $f^{-1}_{\small}$ is exact. 

\begin{lemma}
If $f : \X \to \Y$ is \etale then $f^{-1}_{\small} \ul{\Omega}^\bullet_{\Y} \to \ul{\Omega}_{\X}^\bullet$ is an isomorphism of filtered complexes. 
\end{lemma}

\begin{proof}
The morphism $f_{\small} : \Sh(\X_{\et}) \to \Sh(\Y_{\et})$ is induced by the cocontinuous, continuous functor $\X_{\et} \to \Y_{\et}$ given by $(U \to \X) \mapsto (U \to \Y)$ since $f$ is \etale. Since everything commutes we see that $\rho_{X*}^{\et} = f^{-1}_{\small} \circ \rho_{Y*}^{\et}$ tracing through an identical argument to the big case. Taking derived functors we conclude.
\end{proof}

\begin{rmk}
This may look somewhat weird with $f^{-1}$ instead of $f^*$ if is one is thinking about bundles. Remember $\ul{\Omega}_{\X}$ is defined on the entire small \etale site and therefore $f^{-1}_{\small}$ is more like a restriction to its values on $\Y$. The same phenomenon happens with $\Omega_{\X}^\bullet$ which is the complex of sheaves
\[ (U \to \X) \mapsto \Gamma(U, \Omega_U^\bullet) \]
\end{rmk}

\subsubsection{Du Bois Complex for DM-Stacks}

\begin{defn}
The \textit{Du Bois complex} of a DM-stack $\X$ is $\ul{\Omega}^\bullet_{\X} := \R (\rho_{\X}^{\et})_* \Omega^\bullet_h \in D^+ F(\Sh(\X_{\et}))$ and its associated graded parts are $\ul{\Omega}_{\X}^k = \R (\rho_{\X}^{\et}) \Omega^k_h$.
\end{defn}

We should check this gives ``the same'' thing as the Du Bois complex when $\X$ is a scheme.

\begin{prop}
Let $X$ be a scheme and let $\nu : \Sh(X_{\et}) \to \Sh(X_{\Zar})$ be the natural map of topoi. Then there are natural isomorphisms
\[ \R \nu_* (\ul{\Omega}_{X}^\bullet)_{\et} = \R \nu_* \R (\rho_X^{\et})_* \Omega^\bullet_h = \R (\rho_{X *}) \Omega^\bullet_h = \ul{\Omega}_X^\bullet \]
and furthermore for all $k \ge 0$,
\[ \nu^* \R (\rho_X)_* \Omega^k_h = \R (\rho_{X}^\et)_* \Omega^k_h \]
Therefore, if $X$ is a variety, the three possible definitions of the Du Bois complex of $X$ agree. 
\end{prop}

\begin{proof}
Notice that the composition
\[ \Sch_h \xrightarrow{\rho_X^{\et}} X_{\et} \xrightarrow{\nu} X_{\Zar} \]
of continuous morphisms is $\rho_X$. Therefore, the first statement hold immediately from the compositionality of derived functors. Adjunction of the first statement gives a map,
\[ \alpha : \nu^{-1} \R (\rho_X)_* \Omega^\bullet_h \to \R (\rho_{X}^\et)_* \Omega^\bullet_h   \]
of complexes of abelian sheaves. Taking filtered parts both sides become complexes of $\struct{X}$-modules so we get a map
\[ \alpha^k : \nu^* \R (\rho_X)_* \Omega^k_h \to \R (\rho_{X}^\et)_* \Omega^k_h \]
To show this is an isomorphism, we check on the cohomology sheaves,
\[ \alpha^{k,q} : \nu^* \R^q (\rho_X)_* \Omega^k_h \to \R^q (\rho_{X}^\et)_* \Omega^k_h \]
using that $\nu^*$ is exact. To prove these are isomorphisms, it suffices to show that for any \etale map $f : U \to X$ the restricton of each to $U_{\Zar}$ (i.e. pullback along $U_{\et} \to X_{\et}$ compsed with pushforward along $U_{\et} \to U_{\Zar}$) is an isomorphism. We do this by induction on $q$. We want to show that all maps in the diagram,
\begin{center}
\begin{tikzcd}
(\nu^* \R^q (\rho_X)_* \Omega^k_h)|_{U_{\Zar}} \arrow[d, equals] \arrow[r, "\alpha^{k,q}"] & (\R^q (\rho_{X}^\et)_* \Omega^k_h)|_{U_{\Zar}} \arrow[d]
\\
f^* \H^q(\ul{\Omega}_X^k) \arrow[r] & \H^q(\ul{\Omega}_U^k)
\end{tikzcd}
\end{center}
are isomorphisms where the second downward map is the natural edge map arising from the spectral sequence,
\[ E_2^{a,b} = \R^a \nu_{U*} f_{\small}^{-1} \R^b (\rho_{X}^\et)_* \Omega^k_h \implies \R^{a+b} (\rho_U)_* \Omega^k_h = \H^{a+b}(\ul{\Omega}_U^k) \]
This map is automatically isomorphism for $q = 0$ and the bottom map is always an isomorphism because the Du Bois complex is compatible with \etale localization (meaning $f^* \ul{\Omega}_X^k \iso \ul{\Omega}_U^k$ is a qis). Therefore, we win for $q = 0$. If $\alpha^{k,q'}$ is an isomorphism for $q' < q$ then the spectral sequence is converged at $E_2$ for $a + b \le q$ with only one nonzero column: $a = 0$. This is because $\nu_{U*}$ is exact on quasi-coherent sheaves by \chref{https://stacks.math.columbia.edu/tag/03P2}{Tag 03P2} and affine vanishing. Thus the edge map is an isomorphism so we win by induction.
\end{proof}

\begin{prop}
Let $\X$ be a smooth DM-stack. Then the natural map
\[ \Omega_{\X}^\bullet \to \ul{\Omega}_{\X}^\bullet \]
is an isomorphism. 
\end{prop}

\begin{proof}
It suffices to show that each $\Omega_{\X}^k \to \ul{\Omega}_{\X}^k$ is an isomorphism. This can be checked \etale locally. Let $f : U \to \X$ be an \etale cover by a scheme. Then we need to show that the pullback $f^{-1}_{\small}$, we which have shown above is the natural map
\[ \nu_U^* \Omega_U^k \to \nu^*_U \ul{\Omega}_U^{k} \]
which is an isomorphism since $U$ is smooth and we know for a smooth variety that $\Omega_U^k \iso \ul{\Omega}_U^k$ (these are Zariski sheaves) and so we conclude by applying $\nu^*_U$. 
\end{proof}

Now we recover another result of Du Bois.

\begin{cor}
Let $X$ be a variety with finite quotient singularities. Then $\ul{\Omega}_X^k = \Omega_X^{[k]}$.
\end{cor}

\begin{proof}
By a result of Vistoli, there is a smooth DM-stack $\X$ such that $\pi : \X \to X$ is the coarse space. Since $\pi$ is a good moduli space the projection formula gives,
\[ \ul{\Omega}_{X}^k = \R \pi_* \pi^* \ul{\Omega}_X^k \]
\end{proof}

\subsubsection{The Du Bois Complex of any stack}

\newcommand{\lisset}{\ell-\text{\'{e}t}}

We make the same definitions but replace the \etale site with the lisse-\etale site for a general stack.

\begin{rmk}
We can also work with $\R (\rho_{\X}^{\Et})_* \Omega_{h}^\bullet$ but it is annoying to work with quasi-coherent sheaves defined on the big sites because the embedding
\[ \mathrm{QCoh}(\X) \embed \Mod{\struct{\X_{\Et}}} \]
is NOT usually left-exact. Indeed, consider $\Spec{\Z}$ and the map $2 : \struct{} \to \struct{}$ which is injective as a map of quasi-coherent sheaves but not injective as a map of sheaves on the big \etale site because its value on $\Spec{\FF_2}$ -- which is an object of the big site -- is not injective. I stole this example from \chref{https://stacks.math.columbia.edu/tag/06VE}{here}.
\end{rmk}

\begin{defn}
Let $\X$ be an stack. Then let $\rho_{\X}^{\lisset} : \Sch_h \to \X_{\lisset}$ be the obvious morphism of sites $(U \to \X) \mapsto U$.
\end{defn} 

\begin{lemma}
The induced map $\rho_{\X}^{\lisset} : \Sh(\Sch_h) \to \Sh(\X_{\lisset})$ is a morphism of topoi. 
\end{lemma}

\begin{proof}
It suffices to show $(\rho_{\X}^{\lisset})^{-1}$ is exact. This follows from \chref{https://stacks.math.columbia.edu/tag/00XS}{Tag 00XS} since the functor 
\[ (U \to \X) \mapsto U \]
is continuous, cocontinuous, and preserves fiber products and equalizers.
\end{proof}

\begin{defn}
The \textit{Du Bois complex} of a stack $\X$ is $\ul{\Omega}^\bullet_{\X} := \R (\rho_{\X}^{\lisset})_* \Omega^\bullet_h \in D^+ F(\Sh(\X_{\lisset}))$ and its associated graded parts are $\ul{\Omega}_{\X}^k = \R (\rho_{\X}^{\lisset}) \Omega^k_h$. 
\end{defn}

\begin{lemma}
Let $\X$ be a stack. Then $\ul{\Omega}_{\X}^k$ and $\R (\rho_{\X}^{\Et}) \Omega^k_h$ are complexes of locally quasi-coherent $\struct{\X}$-modules.  
\end{lemma}

\begin{proof}
For any morphism $f : U \to \X$ this is just the claim that $(f_{\big}^{-1} \R (\rho_{\X}^{\Et}) \Omega^k_h) |_{U_{\et}} = (\ul{\Omega}^k_{U})^{\et}$ is a complex of quasi-coherent sheaves. 
\end{proof}




\section{Nowhere vanishing for singular varities}

\begin{defn}
Let $\omega$ be a reflexive differential form on a variety $X$ which extends to a global form $\wt{\omega}$ on a resolution $\wt{X} \to X$. We say that $\omega$ is \textit{nowhere vanishing} if $\omega$ does not vanish on $\wt{X} \sm E$ nor is identically zero along any connected component of $E$. 
\end{defn}

Question? Does PS14 still hold?


\section{Questions March 28}

\begin{enumerate}
\item How is Ambro's result consistent with nontrivial projective bundles on abelian varities? Answer: Ambro only says its an \etale-local fiber bundle NOT a finite \etale-local fiber bundle.

\item if $C$ is a genus $3$ hyperelliptic curve. Is it true that symmetric product is general type? Is it minimal? How does this work? Yes to all. 

\item What is the correct version of the question $(X, D)$ have no automorphisms? Which result was Dori talking about that fixes the automorphisms? Dori proves that its a torus for boundary polarized CY. I think the correct question: the linear part of the connected component of Aut of $(X, D)$ is a torus. 
\end{enumerate}

\section{Inversion of Adjunction}


\subsection{Pairs}

\begin{defn}
A \textit{pair} $(X, \Delta)$ over a base scheme $B$ is a $B$-scheme $X$ and a $\Q$-divisor $\Delta$ on $X$ satisfying
\begin{enumerate}
\item $B$ is regular, excellent, and pure dimensional
\item $X$ is reduced, pure dimensional, $S_2$, excellent scheme that has a canonical sheaf $\omega_{X/B}$
\item the canonical sheaf $\omega_{X/B}$ is locally free outside a codimension $2$ subset (automatic if $X$ is normal)
\item $\Delta = \sum a_i D_i$ is a $\Q$-linear combination of distinct prime divisors non contained in $\Sing(X)$. We allow the $a_i$ to be arbitrary rational numbers. 
\end{enumerate}
\end{defn}

It is tricky to give the most general conditions in which a canonical sheaf exists. However, we have one in the following situation:

\begin{prop}
Suppose there is an open $j : X^0 \embed X$ an a locally closed embedding $\iota : X^0 \embed \P^N_B$ such that
\begin{enumerate}
\item $Z : X \sm X^0$ has codimension $\ge 2$ in $X$
\item $\iota(X^0)$ is lci in $\P^N_B$
\end{enumerate}
let $I$ denote the ideal sheaf of $\ol{\iota(X^0)}$ then $I / I^2$ is locally free on $\iota(X^0)$ so we set
\[ \omega_{X^0/B} = \iota^* (\omega_{\P^N_B/B} \ot \det(I/I^2)^{-1}) \]
then the canonical sheaf of $X$ over $B$ is
\[ \omega_{X/B} := j_* \omega_{X^0 / B} \]
\end{prop}

\renewcommand{\Diff}{\mathrm{Diff}}

\subsection{Birational Maps}

\begin{defn}
A rational map of reduced schemes $f : X \rat Y$ is \textit{birational} if there are dense open subschemes $U_X \subset X$ and $U_Y \subset Y$ such that $f|_{U_X} : U_X \to U_Y$ is an isomorphism. Among all pairs $(U_X, U_Y)$ there is a maximal one $(U_X^m, U_Y^m)$. The complement $\Exc(f) := X \sm U^m_X$ is called the \textit{exceptional set} or \textit{locus} of $f$. 
\end{defn}

\begin{rmk}
Note that if $X$ is pure dimensional and $f : X \rat Y$ is birational then $Y$ is pure dimensional.
\end{rmk}

\begin{defn}
Let $f : X \rat Y$ be birational and $Z \subset X$ be a closed subscheme such that $Z \cap U^m_X$ is dense in $Z$. Then the closure of $f(Z \cap U_X^m) \subset Y$ is called the \textit{birational transform} of $Z$ denoted as $f_* Z$.  
\end{defn}

\begin{rmk}
CAUTION: if $f$ is not a morphism then $f_*$ need not preserve linear or algebraic equivalence. Furthermore, if $D := Z$ is a divisor then $\struct{Y}(f_* D)$ and $f_* \struct{X}(D)$ agree on $U_Y$ but not elsewhere. 
\end{rmk}

\begin{lemma}
If $f : Y \to X$ is a birational morphism then $f^{-1} : X \rat Y$ has maximal opens $(U_X^m, U_Y^m)$ satisfying $(X \sm U_X^m) \sm \Sing(X)$ has codimension $2$ in $X$ and $U_Y^m = \Exc(f)$.
\end{lemma}

\begin{proof}
Indeed, it is clear that $(U_X^m, U_Y^m)$ is the flip of the maximal open for $f$ so the second statement is immediate. We need to show that every prime divisor on $X \sm \Sing(X)$ meets $U_X^m$. Since we removed the singular locus we can assume that $X$ is smooth and irreducible. By Zariski's main lemma, either $f$ is an isomorphism over the generic point of $D$ or the fiber is positive dimensional meaning, by dimension reasons that there is a component of $Y$ mapping into $D$ which violates the definition of a birational map. 
\end{proof}

\begin{rmk}
Hence if $D \subset X$ is a divisor with no component inside $\Sing(X)$ and $Y \to X$ is a birational morphism then $f^{-1}_* D$ is well-defined. 
\end{rmk}

\begin{defn}
Let $X$ be a reduced scheme. A \textit{divisor over} $X$ is a pair $(f : Y \to X, E)$ of a birational morphism $f : Y \to X$ and a divisor $E \subset Y$. The \textit{center} of $E$, denoted $\cent_X E$ is the closure of $f(E) \subset X$.   
\end{defn}

\section{Discrepancies and Singularities}

\begin{defn}
Let $(X, \Delta)$ be a pair. Assume that $m (K_X + \Delta)$ is Cartier for some $m > 0$. This means $m \Delta$ is integral and $\omega_X^{[m]}(m \Delta)$ is locally free. Suppose $f : Y \to X$ is a (not necessarily proper) birational morphism from a reduced scheme $Y$. Let $E \subset Y$ denote the exceptional locus of $f$ and $E_i \subset E$ the irreducible exceptional divisors. We assume $Y$ is regular at the generic point of each $E_i$ (automatic if $Y$ is normal). Let 
\[ f_*^{-1} \Delta := \sum a_i f_*^{-1} D_i \quad \text{ where } \quad \Delta = \sum a_i D_i \]
denote the birational transform. The natural isomorphism
\[ t_{Y \sm E} : \omega_Y^{[m]}(m f_*^{-1} \Delta)|_{Y \sm E} \iso f^* \left( \omega_X^{[m]}(m \Delta) \right) |_{Y \sm E} \]
implies that there are rational numbers $a(E_i, X, \Delta)$ such that $m a(E_i, X, \Delta)$ are integers and $t_{Y \sm E}$ extends to an isomorphism
\[ t_Y : \omega_Y^{[m]}(m f_*^{-1} \Delta) \iso f^* \left( \omega_X^{[m]}(m \Delta) \right) \left( \sum_i m \cdot a(E_i, X, \Delta) E_i \right) \]
For any divisor $D \subset Y$ that is not exceptional, we define 
\[ a(D, X, \Delta) := - \coeff_{f_* D}(\Delta) \]
These numbers are called \textit{discrepacies} of $D$ with respect to $(X, \Delta)$. The above fundamental relation can be writen as
\[ K_Y + f_*^{-1} \Delta \sim_{\Q} f^* (K_X + \Delta) + \sum_i a(E_i, X, \Delta) E_i \]
Furthermore, we define the \textit{log discrepancy} $b(E_i, X, \Delta)$ by the relation
\[ K_Y + f_*^{-1} \Delta + E \sim_{\Q} f^* (K_X + \Delta
) + \sum_i b(E_i, X, \Delta) \]
where $\Delta_Y = f_*^{-1} \Delta + E$ and hence $b(E_i, X, \Delta) = a(E_i, X, \Delta) + 1$.
\end{defn}

\begin{lemma}
Suppose that $f : Y \to X$ and $f' : Y' \to X$ are birational morphisms and $E \subset Y$ and $E' \subset Y'$ are prime divisors such that the rational map $f^{-1} \circ f' : Y' \rat Y$ satisfies $(f^{-1} \circ f')_* E' = E$ (meaning we assume that the rational map is defined at the generic point of $E'$ and it maps this point to $E$). Then $a(E, X, \Delta) = a(E', X, \Delta)$ and $\cent_X E = \cent_{X'} E'$.
\end{lemma}

\begin{rmk}
Because of this lemma, we say heuristically that ``the discrepancy of a divisor over $X$ does not depend on the birational model''. 
\end{rmk}
\begin{rmk}
The following relations are satisfied
\begin{align*}
K_Y + f_*^{-1} \Delta & \sim_{\Q} f^* (K_X + \Delta) + \sum_{E_i \subset E} a(E_i, X, \Delta) E_i
\\
K_Y  & \sim_{\Q} f^* (K_X + \Delta) + \sum_{D \subset Y} a(D, X, \Delta) D
\end{align*}
the first sum over the exceptional divisors and the second over all divisors. Furthermore, we set $\Delta_Y$ to be any divisor such that $f_* (K_Y + \Delta_Y) = K_X + \Delta$ then {\color{red} IS THIS RIGht}
\[ K_Y + \Delta_Y \sim_{\Q} f^* (K_X + \Delta) \]
\end{rmk}

\begin{rmk}
Notice above that $a(D, Y, \Delta_Y) = a(D, X, \Delta)$ for any divisor $D \subset Y$. We say that a birational morphism $(Y, \Delta_Y) \to (X, \Delta_X)$ is \textit{crepant} if it satisfies this property. 
\end{rmk}


\begin{defn}
A proper birational morphism of pairs $f : (Y, \Delta_Y) \to (X, \Delta_X)$ is \textit{crepant} if 
\[ K_Y + \Delta_Y \sim_{\Q} f^* (K_X + \Delta_X) \]
In particular,
\[ a(F, Y, \Delta_Y) = a(F, X, \Delta_X) \]
for any divisor $F$ over $Y$. 
\end{defn}


\begin{defn}
Let $(X, \Delta)$ be a pair where $X$ is normal of dimension $\ge 2$ and $\Delta = \sum a_i D_i$ is a subboundary (i.e. a formal sum of distinct prime divisors and $a_i$ are rational numbers $\le 1$). Assume $m(K_X + \Delta)$ is Cartier for some $m > 0$. We say that $(X, \Delta)$ is,
\begin{enumerate}
\item \textit{terminal} if $a(E, X, \Delta) > 0$ for all exceptional $E$ over $X$
\item \textit{canonical} if $a(E, X, \Delta) \ge 0$ for all exceptional $E$ over $X$
\item \textit{klt} if $a(D, X, \Delta) > -1$ for all divisors $D$ over $X$
\item \textit{plt} if $a(E, X, \Delta) > -1$ for all exceptional $E$ over $X$
\item \textit{dlt} if $a(D, X, \Delta) > -1$ for any $D$ with $\cent_X D \subset \text{non-snc}(X, \Delta)$
\item \textit{lc} if $a(D, X, \Delta) \ge -1$ for all divisors $D$ over $X$
\end{enumerate}
\end{defn}

\begin{rmk}
Notice that $(X, \Delta)$ klt implies that the coefficients of $\Delta$ satisfy $a_i < 1$. Indeed, $a(D_i, X, \Delta) = - \coeff_{D_i} \Delta = - a_i$ and we require $a(D_i, X, \Delta) > - 1$. 
\end{rmk}

\begin{rmk}
Each singularity type implies the next downwards except canonical only implies klt if every coefficient of $\Delta$ satisfies $a_i < 1$. 
\end{rmk}

\begin{lemma}[Kollar, Singularities of MMP, Corollary 2.13]
Let $X$ be a normal scheme and $\Delta = \sum d_j D_j$ a $\Q$-divisor such that $K_X + \Delta$ is $\Q$-Cartier. Let $f : Y \to X$ be a log resolution. Write,
\[ K_Y \sim_{\Q} f^* (K_X + \Delta) + \sum a_i E_i \]
Then
\begin{enumerate}
\item $(X, \Delta)$ is log canonical if and only if every $a_i \ge -1$
\item $(X, \Delta)$ is klt if and only if every $a_i > -1$
\end{enumerate}
notice that in this convention the birational transforms are amount the $E_i$ with coefficient $a_i = -d_j$.
\end{lemma}

\subsection{Adjunction}

Let $(X, \Delta)$ be a pair. 

\begin{theorem}[Kawmata]
Let $(X, S + B)$ be a pair such that $S$ is a reduced divisor which has no common component with the support of $B$. Let $S^\nu$ denote the normalization of $S$, and let $B^\nu$ denote the different of $B$ on $S^\nu$. Then $(X, S + B)$ is log canonical near $S$ if and only if $(S^\nu, B^\nu)$ is log canonical.
\end{theorem}

\begin{defn}
A \textit{pair} $(X, \Delta)$ where $X$ is a normal projective variety, and $\Delta$ is an effective $\Q$-divisor such that $K_X + \Delta$ is $\Q$-Cartier. 
\end{defn}

Now assume that $(X, \Delta)$ is a pair where $\Delta = S + B$ with $S$ normal prime divisor. Let $\pi : Y \to X$ a \textit{log resolution} of $(X, \Delta)$ meaning a birational morphism with $Y$ smooth and $\Supp{}{\pi_*^{-1} \Delta} + \Exc(\pi)$ is a snc divisor. Denote by $\wt{S}$ the strict transform of $S$. Then we can write,
\[ K_Y + \Delta_Y = \pi^* ( K_X + \Delta) \]
and then define
\[ \Diff(B) := \pi_{\wt{S}, *} ((\Delta_Y - \wt{S})|_{\wt{S}}) \]
where $\pi_{\wt{S}} : \wt{S} \to S$ is the restrictyion of $\pi$. Then we obtain a new pair $(S, \Diff(B))$ that does not depend on the choice of $Y$.




MAIN QUESTION IF $(X, \Delta)$ IS A PAIR WHAT DOES AUT MEAN AND IS IT CONTAINED IN THE FOLLOWING GROUP

Let $(\hat{X}, \hat{\Delta}) \to (X, \Delta)$ be an snc resolution then consider $\Aut{\hat{X}, \ceil{\hat{\Delta}}}$. 

\section{Questions}

What about the example quadric surface to union of two planes. 

$V(xy - t zw)$ then have $[t s_0 : \alpha s_1 : s_0  : \alpha s_1 ]$ and $[s_0 : t \alpha s_1 : s_0 : s_1 \alpha]$ lines.






\end{document}