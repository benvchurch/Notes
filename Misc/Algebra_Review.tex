\documentclass[12pt]{article}
\usepackage{import}
\import{"../Algebraic Geometry/"}{AlgGeoCommands}

\newcommand{\Loc}[1]{\mathfrak{Loc}\left( #1 \right)}
\newcommand{\AbGrp}{\mathbf{AbGrp}}

\renewcommand{\K}{\mathbb{K}}

\newcommand{\inner}[2]{\left< #1, #2 \right>}

\newcommand{\B}{\mathcal{B}}
\newcommand{\R}{\mathbb{R}}

\newcommand\eqae{\mathrel{\stackrel{\makebox[0pt]{\mbox{\normalfont\tiny a.e.}}}{=}}}
\renewcommand{\F}{\mathcal{F}}
\renewcommand{\K}{\mathcal{K}}

\begin{document}

\tableofcontents

\section{Nakayama's Lemma}

\begin{prop}
Let $R$ be a (possibly noncommutative) ring and $M$ a finitely generated left $R$-module and $I \subset R$ a left-ideal. Then if $I \cdot M = M$ then there exists some $r \in R$ such that $1 - r \in R$ and $r M = 0$.
\end{prop}

\begin{proof}

\end{proof}

\section{Galois Theory}

\begin{prop}
Let $E$ be the splitting field of a $f \in K[x]$. Then,
\[ | \Aut{E/K} | \le [E : K ] \]
with equality if and only if $f$ is separable.
\end{prop}

\begin{proof}
Dummit and Foote p.561.
\end{proof}

\begin{lemma}[Independence of Characters]
Let $\sigma_1, \dots, \sigma_n : G \to E^\times$ be distinct linear characters. Then in $E[G]$ the elements $\sigma_1, \dots, \sigma_n$ are independent.
\end{lemma}

\begin{proof}
We proceed by induction on $n$. For the case $n = 1$ this is obvious because a character $G \to E^\times$ is nonzero as a map $G \to E$. 
\bigskip\\
Now suppose that,
\[ a_1 \sigma_1 + \cdots + a_n \sigma_n = 0 \]
Now, this must hold for both $x \in G$ and $g x \in G$ so,
\[ a_1 \sigma_1(x) + \cdots + a_n \sigma_n(x) = 0 \]
and likewise,
\[ a_1 \sigma_1(gx) + \cdots + a_n \sigma_n(gx) = 0 \]
but $\sigma_i(gx) = \sigma_i(g) \sigma_i(x)$. Multiplying the first equation by $\sigma_n(g)$ and subtracting we find,
\[ a_1 [\sigma_n(g) - \sigma_1(g)] \sigma_n(x) + \cdots + a_{n-1} [\sigma_n(g) - \sigma_{n-1}(g)] \sigma_n(x) = 0 \]
Therefore by the independence of $\sigma_1, \dots, \sigma_{n-1}$ by assumption, we see that,
\[ a_1 [\sigma_n(g) - \sigma_1(g)] = 0 \]
Therefore either $a_1 = 0$ or $\sigma_1 = \sigma_n$ for all $g$. Since we assumed the characters are distinct this shows that $a_1 = 0$ reducing to the $n-1$ case so $a_i = 0$ for all $i$ by the induction hypothesis. Thus $\sigma_1, \dots, \sigma_n$ are independent.
\end{proof}

\begin{cor}
Distinct field embeddings $\sigma_1, \dots, \sigma_n : K \embed L$ are independent.
\end{cor}

\begin{proof}
Indeed, these are independent as characters $K^\times \to L^\times$ inside the $L$-vectorspace of maps $K^\times \to L$. Therefore, they must be independent as maps $K \to L$.
\end{proof}

\begin{cor}
Let $x_1, \dots, x_n \in E$ be a basis for $E / K$ and $n = [E : K]$. Let $G \subset \Aut{E/K}$ then the vectors $v_\sigma \in E^n$ defined by $(v_\sigma)_i = \sigma(x_i)$ are independent over $E$.
\end{cor}

\begin{proof}
Suppose that,
\[ \sum_{\sigma \in G} \alpha_\sigma v_\sigma = 0 \]
for $\alpha_\sigma \in E$. Then for each $i = 1,\dots,n$ we have,
\[ \sum_{\sigma \in G} \alpha_\sigma \sigma(x_i) = \sum_{\sigma \in G} \alpha_\sigma (v_\sigma)_i = 0 \]
Furthermore, we can write any $x \in E$ as,
\[ x = \beta_1 x_1 + \cdots + \beta_n x_n \]
for $\beta_i \in K$. Since $\sigma$ is a $K$-algebra map, multiplying the $i^{\text{th}}$ equation by $\beta_i$ and adding them gives,
\[ \sum_{i = 1}^n \beta_i \sum_{\sigma \in G} \alpha_\sigma \sigma(x_i) = \sum_{\sigma \in G} \alpha_\sigma \sum_{i = 1}^n \beta_i \sigma(x_i) = \sum_{\sigma \in G} \alpha_\sigma \sigma(\beta_1 x_1 + \cdots + \beta_n x_n) = \sum_{\sigma \in G} \alpha_\sigma \sigma(x) \]
and thus,
\[ \sum_{\sigma \in G} \alpha_\sigma \sigma(x) = 0 \]
Since $x \in E$ is arbitrary, we see that,
\[ \sum_{\sigma \in G} \alpha_\sigma \sigma = 0 \]
showing that $\alpha_\sigma = 0$ for all $\sigma \in G$ by the independence of the characters thus proving that the $v_\sigma \in E^n$ are independent. 
\end{proof}

\begin{cor}
If $G \subset \Aut{E/K}$ then $|G| \le [E : K]$.
\end{cor}

\begin{prop}
Let $E/K$ be a field extension and $G \subset \Aut{E/K}$. Then,
\[ |G| = [E : K] \iff K = E^G \]
\end{prop}

\begin{proof}
Suppose that $|G| = [E : K]$. Take $F = E^G$ giving a tower $K \subset F \subset E$. We know that $[E : K] = [E : F][F : K] = |G|$. However, $G \subset \Aut{E/F}$ because each automorphism fixes $F$ by definition. Thus $|G| \le [E : F]$ meaning that,
\[ |G| \le [E : F] \le [E : K] = |G| \]
proving that $[E : F] = [E : K]$ so $F = K$.
\bigskip\\
Now suppose that $K = E^G$. See Dummit and Foote p.571.
\end{proof}

\begin{rmk}
The proof shows that in general,
\[ [E : K] = |G| \cdot [E^G : K] \]
\end{rmk}

\begin{defn}
We say that $E / K$ is \textit{Galois} if $K = E^{\Aut{E/K}}$ and write $\Gal{E/K} := \Aut{E/K}$.
\end{defn}

\begin{cor}
We see that $E / K$ is Galois if and only if $|\Aut{E/K}| = [E : K]$.
\end{cor}

\subsection{The Galois Correspondence}

\begin{prop}
Let $E/K$ be a finite extension and $G \subset \Aut{E/K}$. Let $F = E^G$ then $E/F$ is Galois and $G = \Aut{E/F}$.
\end{prop}

\begin{proof}
By definition, $G \subset \Aut{E/F}$. Since $F = E^G$ we have $|G| = [E : F]$ and therefore,
\[ |G| \le |\Aut{E/F}| \le [E : F] = |G| \]
proving that $|G| = |\Aut{E/F}| = [E : F]$ and thus $G = \Aut{E/F}$ and that $E/F$ is Galois (note we actually automatically get that $E/F$ is Galois because $F = E^G = E^{\Aut{E/F}}$ using that $G = \Aut{E/F}$).
\end{proof}

\begin{prop}[Galois Connection]
Let $E/K$ be a finite extension and $G = \Aut{E/K}$.
\[ \{ \text{subgroups } H \subset G \} \substack{ \xrightarrow{\quad H \mapsto E^H \quad}  \\ \xleftarrow[F \mapsto \Aut{E/F}]{} } \{ \text{intermediate extensions } K \subset F \subset E \} \]
satisfy the following properties,
\begin{enumerate}
\item $H \mapsto E^H \mapsto \Aut{E/E^H} = H$ meaning that 
\end{enumerate}
\end{prop}

\section{Groups of Lie Type}

\section{Galois Groups of Cubics}

\section{Products of Ideals}

\begin{lemma}
Let $I, J \subset R$ be ideals. Then,
\[ V(IJ) = V(I \cap J) = V(I) \cup V(J) \]
\end{lemma}

\begin{proof}
If $I \subset \p$ then $\p \supset I \cap J \subset IJ$ so it is clear that,
\[ V(I) \cup V(J) \subset V(I \cap J) \subset V(IJ) \]
Thus suppose that $\p \supset IJ$ but $\p \notin V(I) \cup V(J)$. Then there is $x \in I$ and $y \in J$ such that $x, y \notin \p$ so that $\p \not\supset I$ and $\p \not \supset J$. Then $x y \in IJ \subset \p$ so $x y \in \p$ contradicting the primality of $\p$ and proving the claim.
\end{proof}

\begin{prop}
Let $R$ be a comutative ring and $I, J \subset R$ are ideals.
If any of the following are true,
\begin{enumerate}
\item $I + J = R$
\item $\nilrad{R / IJ} = (0)$
\end{enumerate}
then $I \cap J = IJ$.
\end{prop}

\begin{proof}
If $I + J = R$ then for any $r \in I \cap J$ consider $1 = x + y$ with $x \in I$ and $y \in J$ and $r = r x + ry \in IJ$ so $I \cap J \subset IJ \subset I \cap J$ proving equality. 
\bigskip\\
Now suppose that $\nilrad{R / IJ} = (0)$. Consider the ideal $(I \cap J)/IJ \subset R / IJ$. I claim that it is contained in the nilradical. Indeed, for any prime $\p$ of $R / IJ$, that is a prime of $R$ above $IJ$ because $V(IJ) = V(I \cap J)$ and thus $(I \cap J)/IJ \subset \nilrad{R / IJ}$ so $I \cap J = IJ$.
\end{proof}

\section{Induced Representations}

\subsection{Restriction}

\begin{rmk}
There is a functor $\Rep_R : \mathbf{Grp}^\op \to \mathbf{Cat}$ sending $G \mapsto \mathrm{Rep}_R(G)$ taking $\phi : G \to H$ to the functor $\Res{}{\phi}{-} : \Rep_R(H) \to \Rep_R(G)$ via $\rho_W \mapsto \rho_W \circ \phi$ and $(T : W \to W') \mapsto (T : W \to W')$ which still commutes with $\rho_W \circ \phi$ by definition.
\bigskip\\
This restriction functor is just restriction of modules from the ring map $R[G] \to R[H]$.
\bigskip\\
Therefore we get a map $\Aut{G}^\op \to \Aut{\Rep_R(G)}$ and thus a natural right action (which we turn into a left action via $\Aut{G} \to \Aut{G}^\op$ sending $g \mapsto g^{-1}$) on $G$-representations. 
\end{rmk}

\begin{prop}
If $\phi : G \to H$ is surjective then $\Rep_R(H) \to \Rep_R(G)$ preserves irreducibles.
\end{prop}

\begin{proof}
If $W$ is an irreducible $H$-rep then if $V \subset \Res{}{\phi}{W}$ is a $G$-invariant subspace then $\rho_W(\phi(g)) \cdot V = V$ and thus $\rho_W(h) \cdot V = V$ so $V$ is $H$-invariant because $\phi$ is surjective.
\end{proof}

\subsubsection{The Case of a Normal Subgroup}

\begin{rmk}
For the special case of a normal subgroup $H \subset G$ we denote the conjugation action $c : G \to \Aut{H}$ and then applying the above construction we find the following.
\end{rmk}

\begin{defn}
Let $H \subset G$ be a normal subgroup and $W$ an $H$-representation. Then for $g \in G / H$ we define $g * W$ to be the $H$-representation given by $\rho_W \circ c_g^{-1}$ 
\end{defn}

\begin{rmk}
Notice that if $g' = g h$ then $\rho_W \circ c_{g'}^{-1} = \rho_W \circ c_h^{-1} \circ c_g^{-1}$ but $\rho_W \circ c_h^{-1} \cong \rho_W$ so we get $g * W \cong g' * W$ as required. This is a manifestation of the fact that $\Rep_R : \mathbf{Grp}^\op \to \mathbf{Cat}$ is really a $2$-functor sending the natural transformation (isomorphism) $\eta : \phi \to \phi'$ (which just says that $\phi' = c_h \circ \phi$ for some $h = \eta_* \in H$) to the natural isomorphism $\Res{}{\eta}{V} : \Res{}{\phi}{V} \to \Res{}{\phi'}{V}$ given by $v \mapsto h \cdot v$ because then,
\[ h \cdot (g \cdot_{\phi} v) = h \cdot (\phi(g) \cdot v) = (h \phi(g) h^{-1}) \cdot (h \cdot v) = g \cdot_{\phi'} (h \cdot v) \]
\end{rmk}

\begin{prop}
If $H \subset G$ is normal and $V$ is a $G$-representation then $g * \Res{G}{H}{V} \cong \Res{G}{H}{V}$.
\end{prop}

\begin{proof}
Consider the map $\eta : V \to V$ by sending $\eta: v \mapsto g \cdot v$. I claim this is an isomorphism $\eta : g * \Res{G}{H}{V} \to \Res{G}{H}{V}$. Indeed it is clearly bijective and linear. Now,
\[ (g * \rho)(h) \cdot v = g^{-1} h g \cdot v \mapsto g \cdot (g^{-1} h g) \cdot v = hg \cdot v = h \cdot (g \cdot v) = \rho(h) \cdot v \]
so $\eta \circ (g * \rho)(h) = \rho(h) \circ \eta$. 
\end{proof}

\begin{prop}
Let $H \subset G$ be normal and $V$ a $G$-representation. Then $G / H$ acts on the $H$-subrepresentations $W \subset \Res{G}{H}{V}$ via $W \mapsto g \cdot W$ where $g \cdot W \cong g * W$ as $H$-representations.
\end{prop}

\begin{proof}
We need to show that $g \cdot W$ is a well-defined subrepresentation. First, for $v \in W$,
\[ h \cdot (g \cdot v) = hg \cdot v = g(g^{-1} h g) \cdot v = g \cdot ((g^{-1} h g) \cdot v) \]
proving that $g \cdot W$ is indeed $H$-invariant since $g^{-1} h g \in H$ so $g^{-1} h g \cdot v \in W$ and also that $g * W \cong g \cdot W$ via $v \mapsto g \cdot v$ by the same argument above. Furthermore, if $g' = gh$ then $g' \cdot W = g \cdot (h \cdot W) = g \cdot W$ because $W$ is $H$-invariant. 
\end{proof}

\begin{rmk}
It is clear that the $G$-invariant subspaces of $V$ are exactly the fixed points under the $G/H$-action.
\end{rmk}

\subsection{Induction and Coinduction}

\begin{prop}
Let $H \subset G$ then $R[G]$ is a free $R[H]$-module.
\end{prop}

\begin{proof}
Consider, 
\[ R[G] \cong \bigoplus_{g \in H  G} g R[H] \]
as \textit{right} $R[H]$-modules (we can make them left modules by $R[H]^\op \cong R[H]$) via sending $g \cdot h \mapsto gh$. This is clearly surjective because $gh$ covers each coset. Furthermore, this is injective because if,
\[ \sum_{g \in G/H} g \left( \sum_{h \in H} \alpha_{g,h} h \right) = \sum_{g \in G / H} \sum_{h \in H} \alpha_{g,h} gh = 0 \]
but there is an bijection $G / H \times H \to G$ via $(g,h) \mapsto gh$ then $\alpha_{g,h} = 0$. Finally, this map is $R[H]$-linear because $g \cdot h h' \mapsto gh h' = (gh) \cdot h'$.
\end{proof}

\begin{prop}
If $H \subset G$ is normal then for any $H$-representation $W$,
\[ \Res{G}{H}{\Ind{G}{H}{W}} \cong \bigoplus_{g \in G / H} g * W \]
\end{prop}

\begin{prop}
If $H \subset G$ is normal then for any $G$-representation $V$,
\[ \Ind{G}{H}{\Res{G}{H}{V}} \cong R[G/H] \otimes_R V \]
as $R[G]$-modules.
\end{prop}

\begin{proof}
Consider the map, $\Ind{G}{H}{\Res{G}{H}{V}} \cong R[G] \otimes_{R[H]} V \to R[G/H] \otimes_R V$ defined by,
\[ g \otimes v \mapsto [g] \otimes g \cdot v \]
This is well-defined because,
\[ gh \otimes v \mapsto [gh] \otimes gh \cdot v \quad \text{ and } \quad g \otimes (h \cdot v) \mapsto [g] \otimes g h \cdot v = [gh] \otimes gh \cdot v \]
This is clearly surjective and both sides are free $R$-modules of equal rank so it is an isomorphism.
\end{proof}

(DEFINITION OF INDUCTION AND COINDUCTION)
(WHEN ARE THEY EQUAL)
(EXPLICIT DESCRIPTIONS)
(CHARACTER FORMULAE)
(FORMULA FOR IND(RES))
(NON-NORMAL CASE?)

\section{Noetherian Normalization}

\begin{theorem}
Let $A$ be a finitely generated $K$-algebra domain. Then there are algebraically independent $x_1, \dots, x_d \in A$ where $d = \dim{A}$ such that,
\[ K[x_1, \dots, x_d] \subset A \]
is a finite extension of domains.
\end{theorem}

\begin{proof}
We proceed by induction on the number of generators of $A$ as a $K$-algebra. If $n = 0$ then $A = K$ and we are done. Now we apply an induction hypothesis and assume that $A$ is generated by $n$ elements $y_1, \dots, y_n$ over $K$. If these are algebraically independent then we are done. Otherwise there is some relation $f \in K[x_1, \dots, x_n]$ such that,
\[ f(y_1, \dots, y_n) = 0 \]
in $A$. Let $z_i = y_i - y_n^{r^i}$ for $i < n$. Then obviously,
\[ f(z_1 + y_n^r, \dots, z_{n-1} + y_n^{r^{n-1}}, y_n) = 0 \]
The monomials in this expansion are of the form,
\[ \alpha \left( \prod_{i = 1}^{n-1} (z_i + y_n^{r^i})^{a_i} \right) y_n^{a_n} = \alpha y^{a_n + a_1 r + \cdots a_{n-1} r^{n-1}}_n + \cdots \]
However the exponent of $y_n$ encodes a unique base $r$ number if we choose $r$ larger than every exponent in $r$. Therefore, there is only one term of $f$ that can contribute to this largest $y_n$ exponent term (each monomial has a different $y_n$ exponent). Dividing by $\alpha$ we get a monic polynomial $f' \in K[z_1, \dots, z_{n-1}][x]$ such that $f'(y_n) = 0$ and thus $y_n$ is integral over $K[z_1, \dots, z_{n-1}]$. By using the induction hypothesis, there exist algebraically independent $x_1, \dots, x_d \in K[z_1, \dots, z_{n-1}]$ (the dimensions are the same because the extension is integral) such that,
\[ K[x_1, \dots, x_d] \subset K[z_1, \dots, z_{n-1}] \subset A \]
is a sequence of integral extensions proving the claim for $A$ and thus for all $A$ by induction on the number of generators. 
\end{proof}

\section{Cohen's Theorem}

\begin{lemma}
Let $A \subset B$ be an integral extension of domains. Then $A$ is a field iff $B$ is a field.
\end{lemma}

\begin{proof}
If nonzero $b \in B$ is integral over $a$ then $b^{-1} \in B$ from the polynomial since its trailing term is invertible. Thus $A$ a field implies $B$ a field. If $B$ is a field then since $a^{-1}$ is integral over $A$ we see that $a^{-1} \in A$ from the polynomial so $A$ is a field.
\end{proof}

\begin{lemma}
Let $f : A \to B$ be an integral map of rings and $\p \subset B$ a prime. Then $f^{-1}(\p)$ is maximal if and only if $\p$ is maximal.
\end{lemma}

\begin{proof}
Indeed, consider $A / f^{-1}(\p) \subset A / \p$ which is an integral extension of domains. Thus $\p$ is maximal iff $A / \p$ is a field iff $A / f^{-1}(\p)$ is a field iff $f^{-1}(\p)$ is maximal.
\end{proof}

\begin{prop}[Lying Over]
Let $A \subset B$ be an integral extension of rings. Then the continuous map $f^* : \Spec{B} \to \Spec{A}$ is surjective.
\end{prop}

\begin{proof}
Let $\p \subset A$ be a prime. Consider, $S = A \setminus \p$ then there is a diagram,
\begin{center}
\begin{tikzcd}
A \arrow[d] \arrow[r, hook] & B \arrow[d]
\\
A_\p \arrow[r, hook] & S^{-1} B
\end{tikzcd}
\end{center}
and the bottom extension is integral. Choose a maximal ideal $\m \subset S^{-1} B$ which is nonzero because $A_\p$ is contained inside it. Then $\m$ pulls back to a maximal ideal in $A_\p$ which must be $\p A_\p$ since $A_\p$ is local and thus under $A \to A_\p \to S^{-1} B$ we see that $\m \mapsto \p$. By commutativity the pullback of $\m$ in $B$ maps to $\p$.
\end{proof}

\begin{cor}[Going Up]
If $f : A \to B$ is an integral map of rings then $f$ satisfies going up and $f^*(V(I)) = V(f^{-1}(I))$.
\end{cor}

\begin{proof}
Let $I \subset B$ be an ideal. Consider $\p \supset f^{-1}(I)$ and the map $A / f^{-1}(I) \embed B / \p$ which is an integral extension of domains. Thus $\Spec{B/I} \to \Spec{A / f^{-1}(I)}$ is surjective. If $\q \in V(I)$ then $f^{-1}(\q) \supset f^{-1}(I)$ so $f^*(V(I)) \subset V(f^{-1}(I))$ and the surjectivity proves that $f^*(V(I)) = V(f^{-1}(I))$. In particular, if $I = \q$ is prime then we recover going up. Namely if $\p = f^{-1}(\q)$ and $\p' \supset \p$ then there exists $\q' \supset \q$ such that $\q' \mapsto \p$. 
\end{proof}

\begin{rmk}
Therefore the image is closed because if $Z \subset \Spec{B}$ is closed then $Z = V(I) = \Spec{B/I}$ and $\Spec{B/I} \to \Spec{A}$ factors as $\Spec{B/I} \to \Spec{A/f^{-1}(I)} \to \Spec{A}$ and $f^*(V(I)) = V(f^{-1}(I))$ meaning $\Spec{B/I} \to \Spec{A / f^{-1}(I)}$ is surjective so the image is closed.
\end{rmk}

\begin{prop}[Incompatibility]
If $A \to B$ is an integral map and $\p \subset \p'$ are primes of $B$ above $\q \subset A$ then $\p = \p'$.
\end{prop}

\begin{prop}[Going Down] 
Since $A / \q \embed B / \p$ is an integral extension of domains then $(A/\q)_\q \embed (B / \p)_\q$ is an integral extension of domains with $(A / \q)_\q$ a field so $(B / \p)_\q$ is a field. Therefore $\p' = \p$ since there is a unique prime prime ideal. 
\end{prop}

\end{document}