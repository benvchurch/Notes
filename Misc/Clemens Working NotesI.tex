\documentclass[12pt]{article}
\usepackage{hyperref}
\hypersetup{
    colorlinks=true,
    linkcolor=blue,
    filecolor=magenta,      
    urlcolor=blue,
}

\usepackage{import}
\import{"../Algebraic Geometry/"}{AlgGeoCommands}

\newcommand{\Loc}[1]{\mathfrak{Loc}\left( #1 \right)}
\newcommand{\AbGrp}{\mathbf{AbGrp}}

\renewcommand{\tr}{\operatorname{tr}}

\newcommand{\LL}{\mathbb{L}}
\newcommand{\ob}{\mathrm{ob}}
\newcommand{\cM}{\mathcal{M}}
\newcommand{\cT}{\mathcal{T}}
\newcommand{\vir}{\mathrm{vir}}
\newcommand{\cO}{\mathcal{O}}
\newcommand{\ad}{\mathrm{ad}}

\newcommand{\Y}{\mathscr{Y}}


\DeclareMathOperator{\covdeg}{\text{cov.deg}}
\DeclareMathOperator{\cd}{\text{cd}}

\theoremstyle{plain}
\newtheorem{Lthm}{Theorem}
\renewcommand*{\theLthm}{\Alph{Lthm}}
\newtheorem{Lcor}[Lthm]{Corollary}
\newtheorem{Lprop}[Lthm]{Proposition}
%\newtheorem*{Claim*}{Claim}

\DeclareMathOperator{\irr}{irr}
\DeclareMathOperator{\BAV}{(BAV)}
\DeclareMathOperator{\gon}{gon}
\DeclareMathOperator{\cov}{cov.gon}
\DeclareMathOperator{\amp}{Amp}
\DeclareMathOperator{\nef}{Nef}
\DeclareMathOperator{\eff}{Eff}
\DeclareMathOperator{\mon}{Mon}
\DeclareMathOperator{\val}{Val}
\DeclareMathOperator{\ind}{ind}
\DeclareMathOperator{\seg}{seg}
\DeclareMathOperator{\pic}{Pic}
\DeclareMathOperator{\aut}{Aut}
\DeclareMathOperator{\dps}{dP}
\DeclareMathOperator{\ns}{NS}
\DeclareMathOperator{\NKLT}{NKLT}
\DeclareMathOperator{\divib}{div}

\newcommand{\mb}[1]{\mathbb{#1}}
\DeclareMathOperator{\cg}{cov.gon}
\DeclareMathOperator{\covgon}{cov.gon}
\DeclareMathOperator{\mindeg}{min.deg}

\begin{document}

\section{Introduction}

The result is interesting for two reasons: first because we can do it for all rational an elliptic surfaces rather than those having $-K_X$ nef and seconly because $X$ is covered by unirational surfaces. This is another indication of the strong distinction between rational and unirational surfaces in positive characteristic. 

\section{Clemens Conjecture}

\begin{lemma}
Let $f : S \to X$ be a morphism from a smooth birationally ruled surface to a smooth 3-fold. Suppose $\varphi : \L \embed \wedge^2 \Omega_X$ is a line bundle embedded in $\wedge^2 \Omega_X$ and $\L$ has a nonzero section $s$. Let $\ol{S} = \im{f}$ then one of the following must hold:
\begin{enumerate}
\item $\ol{S} \subset V(s)$
\item $f^*(\L \ot \struct{X}(\ol{S}))$ intersects non-positively with the general fiber of $S \to C$
\item $\ol{S} \subset V(\varphi)$
\end{enumerate}
\end{lemma}

\begin{proof}
Suppose (a) does not hold.
Because $H^0(S, \omega_S) = 0$ since $S$ is ruled and $f^* \L$ has a nonzero section because we are not in case (a), the composition is zero
\[ f^* \L \to f^* \wedge^2 \Omega_X \to \omega_S \]
since $\omega_S$ has no sections and $f^* \L$ is big. 
\bigskip\\
Now consider the sequence
\[ 0 \to \C \to f^* \Omega_X \to \Omega_S \]
Let $\ol{S}$ be the image of $S$. Then we have a sequence,
\[ 0 \to \C \to \Omega_X |_{\ol{S}} \to \Omega_{\ol{S}} \to 0 \]
and the sequence is left exact because $\ol{S}$ is a prime divisor and hence is Cartier and so $\C$ is a line bundle. 
Consider the exact sequence
\[ 0 \to f^* \C \to f^* \Omega_X \onto \F \subset \Omega_S \]
where $\Omega_S / \F$ has support over the exceptional locus of $S \to \ol{S}$. Then I claim there is a sequence
\[ 0 \to \F \ot \C \to \wedge^2 f^* \Omega_X \to \omega_S \]
Indeed, consider the map $f^* \Omega_X \ot \C \to \wedge^2 f^* \Omega_X$. I claim this surjects onto the kernel. Indeed, if $\alpha \wedge \beta \mapsto 0$ then $\alpha - \lambda \beta$ is in the kernel. Therefore, $\alpha \wedge \beta = (\alpha - \lambda \beta) \wedge \beta$ thus is in the image of the claimed map. Moreover, since $\C \ot \C$ maps to zero we get a map $\F \ot \C \to \wedge^2 f^* \Omega_X$. This is injective because $\C$ is a line bundle and $\F$ is torsion-free and rank $1$ so we can check injectivity at the generic point.
\par 
Therefore, since $f^* \L \to \wedge^2 f^* \Omega_X \to \omega_S$ is zero we get that the map factors through $f^* \L \to \F \ot \C$. Hence, if the map $f^* \L \to \wedge^2 f^* \Omega_X$ is nonzero then we get an embedding
\[ f^* \L \embed \Omega_S \ot f^* \C \]
We need that $f^*(\L \ot \C^\vee)$ is big since $\Omega_S$ cannot contain a big line bundle. Indeed, there is map $S \to C$ whose general fiber is $\P^1$. Then we know $\Omega_S|_F \cong \struct{\P^1}(-2) \oplus \struct{\P^1}$ but a big line bundle must restrict positively to the generic fiber. 
\end{proof}

{\color{red} If $X$ is singular this might be an issue unless the singularities are not so bad that forms do not extend to the resolution}

Note if $S \to X$ hits a singular point of $X$ that needs to be resolved then the modification to the normal bundle is only over exceptional loci of $S$ I think and therefore do not interact with the general fiber of $S$ maybe?? Unless the map contracts something to the singularity which seems very possible. 

\section{Chang and Ran}

let $X \subset \P^4$ be a general quintic hypersurface. Let it be a general hyperplane section of $Y \subset \P^5$ another fixed quintic. Let $S \to X$ be a smooth surface of negative kodiara dimension mapping birationally onto its image in $X$. There are two cases:
\begin{enumerate}
\item either $S$ fills $Y$ as we move $H$
\item $S$ extends to a divisor of $Y$ such that $S$ is a section.
\end{enumerate}

I THINK they show (b) does not occur and when $-K_S$ is nef (a) does not occur either.

\subsection{(a)}

Consider the sequences,
\[ 0 \to T_S \to f^* T_{X} \to N_f \to 0 \]
and 
\[ 0 \to N_f \to N_{\tilde{f}} \to L \to 0 \]
where $\tilde{f}$ is the composite 
\[ S \xrightarrow{f} X \embed Y \]
and $L = f^* \struct{}(1)$. 
\bigskip\\
Note that the second sequence splits in any neighbrohood of a fiber of $f$. Let $\tau = (N_f)_{\tors}$ which is supported purely in codimension $1$ (because $T_S$ has corank $1$ in $f^* T_X$). Since $S$ fills $Y$ we see that $N_{\tilde{f}}$ is generated generically by global sections. Thus 
\[ c_1(N_{\tilde{f}} / \tau) = c_1(N_{\tilde{f}}) - c_1(\tau) \]
is nef {\color{red} WHY? maybe I don't know what generically globally generated means in this context?}

\subsection{(b)}



\section{Wang 2000}

Let $X$ be a non-singular complete intersection of type $(m_1, \dots, m_k)$ in a Grassmanian $G(r, n+1)$ such that $\dim{X} \ge 3$ and $m = m_1 + \cdots + m_k \ge n + 1$, and supposet $\ol{D} \subset X$ is an irreducible and reduced divisor. Let $f : D \to \ol{D} \subset X$ be a desingularization, $\ell$ denote the dimension of $D$ and $L = f^* \struct{G}(1)$. Obviously, $L$ is big and nef. Let $K_D$ be the canonical bundle of $D$. Let $S$ and $Q$ be the universal subbundle and universal quotient bundle on $G$. 

\begin{prop}
$X$ does not cotain any reduced irreducible divisor which admits a designularization having
\[ H^0(K_D \ot f^* Q^\vee) = 0 \quad \text{ and } \quad H^1(K_D - L^{\ot m_i}) = 0 \]
for any all $i = 1, \dots, k$.
\end{prop}

\subsection{Reflexive Sheaves}

Let $\F^{\vee \vee}$ be the double dual of $\F$. A coherent sheaf $\F$ is reflexive if the natural map $\F \to \F^{\vee \vee}$ is an isomorphism. Define the singularity set of $\F$ to be the locus where $\F$ is not free over the local ring.
\par
It is well-known that the sigularity set of a torsion-free sheaf on $D$ is in codimension $\ge 2$. Moreover, the singularity set of a reflexitve sheaf on $D$ is in codimension $\ge 3$. It is also well-known that, in general, any reflexive rank $1$ sheaf on an integral locally factoral scheme is a line bundle.

\subsection{The Proof}

Assume such $\ol{D}$ exists. Consider the sequence
\[ 0 \to Q^\vee \to \struct{G}^{n+1} \to S^\vee \to 0 \]
Pull this back and tensor with $f^* Q$ to get
\[ 0 \to f^* Q \ot f^* Q^\vee \to (f^* Q)^{n+1} \to f^* T_G \to 0 \]
The top cohomology
\[ h^\ell(f^* Q) = h^0(K_D \ot f^* Q^\vee) = 0 \]
vanishes by assumption and hence $H^\ell(f^* T_G) = 0$. Now we pull back the normal bundle sequence of $X$
\[ 0 \to f^* T_X \to f^* T_G \to \bigoplus L^{\ot m_i} \to 0 \]
Note that we need the smoothness of $X$ to get the above sequence. Then we have,
\[ h^{\ell-1}(L^{\ot m_i}) = h^1(K_D - L^{\ot m_i}) = 0 \]
also by assumption and hence using this and the above calculation
\[ H^\ell(f^* T_X) = 0 \]
Next, consider the defining sequence of the normal sheaf
\[ 0 \to T_D \to f^* T_X \to N_f \to 0 \]
with the above three sequences we obtain
\[ H^\ell(N_f) = 0 \]
and
\[ c_1(N_f) = K_D + (n + 1 - m) L \]
where
\[ m  = m_1 + \cdots + m_k \]
Let $N_f^{\vee \vee}$ be the double dual of $N_f$ which is a line bundle. The image of $N_f \to N_f^{\vee \vee}$ is torsion-free. The singularity set of the image is in codimension $\ge 2$ so there is an exact sequence
\[ 0 \to \tau \to N_f \to N_f^{\vee \vee} \to \phi \to 0 \]
with $\dim{\Supp{}{\phi}} \le 0$. Devide these into sequences
\[ 0 \to \tau \to N_f \to \psi \to 0 \]
and
\[ 0 \to \psi \to N_f^{\vee \vee} \to \phi \to 0 \]
Then $H^\ell(N_f) = 0$ implies that likewise
\[ H^\ell(N_f^{\vee \vee}) = 0 \]
because $H^\ell(\phi) = 0$ by dimension reasons. On the other hand, we have 
\[ c_1(N_f^{\vee \vee}) = K_D + (n + 1 - m)L - c_1(\tau) \]
Note that $c_1(\tau)$ is always effective. Therefore,
\[ h^\ell(N_f^{\vee \vee}) = h^0(K_D - N_f^{\vee \vee}) = h^0((m - n - 1)L + c_1(\tau)) > 0 \]
which is a contradiction. 

\subsection{Main Theorem}

For $r = 1$ we identify $G(1, n+1) = \P^n$. 

\begin{prop}
A nonsingular complete intersection $X$ of type $(m_1, \dots, m_k)$ in $\P^n$ for $n \ge 4$ such that
\[ m = m_1 + \cdots + m_k \ge n + 1 \]
does not contain a reduced irreducible divisor which admits a desingularization having $H^0(K_D - L) = 0$ and $H^1(K_D - m_i L) = 0$ for all $i = 1,\dots, k$. 
\end{prop}

We get thiis immediately if we identify $\P^n$ with $G(n, n+1)$. 

\begin{theorem}
A non-singular complete intersection $X$ of type $(m_1, \dots, m_k)$ in $\P^n$ such that $\dim{X} \ge 3$ and $m = m_1 + \cdots + m_k \ge n + 1$ does not contain a reduced irreducible divisor which admits a desingularization having nef anticanonical bundle.
\end{theorem}

\begin{proof}
If $-K_D$ is nef, $-K_D + L$ and $-K_D + m_i L$ are nef and big. Therefore by Kawamata-Viehweg vanishing we obtain
\[ H^0(K_D - L) = 0 \quad H^1(K_D - m_i L) = 0 \]
for all $i$. Note that $\dim{D} = \dim{X} - 1 \ge 2$ so we may apply the vanishing results. 
\end{proof}

\section{Wang's Thesis Filling Result}

\section{Mori's Construction}

Let $S$ be a scheme, $t \in \struct{S}$. Let $f, g \in \struct{S}[x_0, \dots, x_n]$ be homogeneous polynomials of degrees $cd$ and $d$ respectively such that $g^c - f$ is not identically zero in $\kappa(s)$ for any $s \in S$. The scheme
\[ Z = V(y^c - f, ty - g) \subset \P_S(x_0, \dots, x_n, y) = \P_S(1, \dots, 1, d) \]
defines a family of weighted complete intersections over $S$. If $s \in S$ and $t(s) \neq 0$ then the fiber $Z_s$ is isomophic to the hypersurface
\[ V(g(s)^c - t(s)^c f(s)) \subset \P_{\kappa(s)}(x_0, \dots, x_n) \]
If $t(s) = 0$ then the fiber $Z_s$ is isomorphic to a $\mu_c$-cover of the hypersurface $V(g(s))$ branched over $V(f(s))$. 

\subsection{$\mu_c$-cyclic covers}

For polynomials $f,g \in k[x_0, \dots, x_n]$ as before we consider
\[ Z = V(y^c - f, g) \subset \P_k(x_0, \dots, x_n, y) = \P_k(1, \dots, 1, d) \]
which is a $\mu_c$-cover of the hypersurface $V(g)$ branched along $V(f)$. 
\bigskip\\
The map $Z \to V(f) \to \P^n$ is given by the projection $\P(1,\dots,1,d) \to \P^n$ away from $[0 : \cdots : 0 : 1]$ and therefore $\struct{Z}(1)$ is unambiguous and is an ample line bundle (it is a line bundle since it is pulled back from $\struct{\P^n}(1)$ and ample since it is pulled back from a closed embedding into $\P(1,\dots,1, d)$). 
\bigskip\\
We are going to be interested in the case $g = x_0$ and $f$ has degree $c$. In this way we can just drop $g$ and reduce the number of variables and work in an unweighted projective space since $\deg{y} = d = 1$. 
\bigskip\\
Note that we can always increase the number of variables $k[x_0, \dots, x_n, x_{n+1}]$ and extend to $\tilde{f}, \tilde{g} \in k[x_0, \dots, x_{n}, x_{n+1}]$ that restrict to $f,g$ when we set $x_{n+1} = 0$. Hence we get extensions of $Z$ in a larger weighted projective space. 

\subsection{Non-filling case}

Let $f$ be a generic polynomial of degree $c$ in $k[x_0, \dots, x_n]$ and suppose that $D \subset Z$ is a uniruled divisor. Then if we extend $Z^m$ for $m \ge n-1$ to a generic higher degree cyclic cover then there is not for all $m$ an extension of a divisor $D^m \subset Z^m$ such that $D \subset Z$ is a hyperplane slice. 

\begin{proof}
We need $D^m$ is uniruled. We can get this in two ways:
\begin{enumerate}
\item Mori-Miyaoka (works in all characteristics)
\item don't prove it in general just take the image of the Hilbert scheme component for slices of $Z^m$ in $Z^m$ this is either a divisor or everything, if its everything then we get sections of $N_{\tilde{f}}$ and everything.
\end{enumerate}
Let $h : \P^1 \to D$ be a rational curve through a general point of $D$ and consider
\[ h_m : \iota_m \circ h \quad \text{ with } \iota : D \embed D^m \]
Consider the sequence
\[ 0 \to N_{h_{n+1}} \to N_{h_m} \to h^* \struct{}(1)^{\oplus (m-n)} \to 0 \]
where the last term is the normal bundle of $D^{n+1} \embed D^{m}$ since $D^{n+1}$ is a linear slice. 
\bigskip\\
Now $N_{h_{n+1}}$ is semipositive {\color{red} HERE WE NEED TO USE THE ARGUMENT OF CLEMENS MAYBE $D^{n+1}$ IS GENERIC} (every quotient has nonnegative degree) and hence nonspecial. Ths means a general deformation $\hat{h_m}$ of $h_m$ for $m \gg 0$ will be linearly normal (meaning that $H^0(\P^N, \struct{}(1)) \to H^0(\P^1, \hat{h}_m^* \struct{}(1))$ is surjective) and hence projectively normal because $\hat{h}_m^* \struct{}(1)$ is $0$-regular with respect to itself. This is just because for degree less than $N$ a generic rational curve in $\P^N$ is projectively normal (and degenerate) {\color{red} BUT IT IS NOT GENERIC SINCE IT IS IN $D^m$ WHAT IS UP?}


Let $L = g^*_{m} \struct{}(1)$ which is independent of $m$.

The previous sequence implies that
\[ c_1(N_{h_{n+1}} \ot L^\vee) = c_1(N_{h_{m}} \ot L^\vee) \]
is also independent of $m$. Likewise,
\[ c_1(\hat{h}_m^* N_{g_m}(-1)) = c_1(h_m^* N_{g_m}(-1)) = c_1(h_{n+1}^* N_{g_{n+1}}(-1)) \]
where $g_m : D^m \embed X^m$ is the inclusion,
becuase $h_{m}^* N_{g_m} = h_{m+1}^* N_{g_{m+1}}$ and then
\[ c_1(h_{n+1}^* N_{g_{n+1}}(-1)) = \deg{h^*_{n+1} K_{D^{n+1}}) \le -2 \]
{\color{red} WHERE DOES THE $-2$ COME FROM}
I think the point is that because $D^{n+1}$ is uniruled there should be a nonzerp map
\[ h^*_{n+1} K_{D^{n+1}} \to K_{\P^1} \]
{\color{red} NEED SEPARABLY UNIRULED FOR THIS, IS IT TRUE?}
because 
\[ 0 \to T_{D^{n+1}} \to T_{X^{n+1}} |_{D^{n+1}} \to N_{g_{n+1}} \to 0 \]
and $\det{T_{X^{n+1}}} = \struct{}(1)$ so $\det{N_{g_{n+1}}(-1)} = K_{D^{n+1}}$.
\bigskip\\
Now consider the sequence
\[ 0 \to N_{\hat{h}_m} \to N_{g_m \circ \hat{h}_m} \to \hat{h}_m^* N_{g_r} \to 0 \]
From above
\[ H^1(\hat{h}^*_m N_{g_m}(-1)) \neq 0 \]
and therefore
\[ H^1(N_{g_m \circ h_m}(-1)) \neq 0 \]
this gives a contradiction from the following lemma.

\begin{lemma}
Let $r : \P^1 \to X^n \subset \P^{n+1}$ be a projectively normal rational curve on a smooth hypersuface. Then there exists an extension $X^m \supset X^n$ in $\P^{m+1}$ such that the map $r_m : \P^1 \to X^m$ has $H^1(N_{r_m}(-1)) = 0$. 
\end{lemma}

\begin{proof}
Consider a potential extension $X^m \supset X^n$ defined by $y^c - F$ and $j : X^m \to \P^{m+1}$ the inclusion. Then we have an exact sequence
\[ 0 \to N_{r_m}(-1) \to N_{j \circ r_m}(-1) \to r_m^* \struct{}(c-1) \to 0 \]
and the natural map
\[ \delta : H^0(\T_{\P^{m+1}}(-1)) \to H^0(N_{j \circ r_m}(-1)) \to H^0(r^*_m \struct{}(c-1)) \]
is given by
\[ \frac{\partial}{\partial x_i} \mapsto -\frac{\partial F}{\partial x_i} \quad \frac{\partial}{\partial y} \mapsto 0 \]
Since $N_{j \circ r_m}(-1)$ is semipositive (it is a quotient of $\T_{\P^{m+1}}(-1)$) it has vanishng $H^1$ and hence $H^1(N_{r_m}(-1)) = 0$ if and only if $\delta$ is surjective. {\color{red} SHIT MEGA SHIT}. Note that $y^c = F$ on $X^m$ but lower powers of$y$ are independent so this map is not surjective, it does not hit anything with a $y$ in it. 
\end{proof}

{\color{red} CAN WE DO IT WITH GENERIC CYCLIC COVERS RATHER THAN GENERIC HYPERSURFACES?}


\end{proof}



\end{document}