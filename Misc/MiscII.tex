\documentclass[12pt]{article}
\usepackage{import}
\import{"../Algebraic Geometry/"}{AlgGeoCommands}

\newcommand{\Loc}[1]{\mathfrak{Loc}\left( #1 \right)}
\newcommand{\AbGrp}{\mathbf{AbGrp}}

\begin{document}

\section{Local Cohomology}

\begin{definition}
A local group or local system of groups $\L$ is a locally-constant sheaf of abelian groups. We write $\Loc{X}$ for the category of local systems on $X$.  
\end{definition}

\begin{theorem}
Let $X$ be a locally-path-connected (AND) topological space. Then there is a equivalence of categories between the category of local groups on $X$ and the category of actions of the fundamental groupoid $\Pi(X)$ on abelian groups. 
\end{theorem}

\begin{proof}
There is a functor $\Loc{X} \to \AbGrp^{\Pi(X)}$ sending a local system to its monodromy action. For any path $\gamma : I \to X$ and a point $\gamma(t)$ there is a open connected neighborhood $\gamma(t) \in U_t$ small enough such that $\L |_U{_t} \cong \underline{G}|_{U_t}$ for some abelian group $G$. Then $\gamma^{-1}(U_t)$ cover $I$ which is compact so we may choose finitely many $U_{i}$ which cover the path and we may assume that $U_i \cap U_{i + 1} \neq \empty$. Then since both are connected and $\L$ is constant on each we get isomorphisms,
\begin{center}
\begin{tikzcd}
\L(U_i) \arrow[rd, "\sim"] & & \L(U_{i+1}) \arrow[ld, "\sim"']
\\
& \L(W)
\end{tikzcd}
\end{center}
where $W$ is a connected component of $U_i \cap U_{i + 1}$. Thus $\L(U_i) \xrightarrow{\sim} \L(U_{i+1})$. Inductivly, this gives $\L(U_0) \xrightarrow{\sim} \L(U_n)$ which, since it is well-defined after shrinking the neighborhoods admits restricting to stalks, gives the monodromy map $[\gamma] : \L_{\gamma(0)} \to \L_{\gamma(1)}$. Clearly this construction respects composition. Furthermore, we can do the exact same construction for maps $I^2 \to X$ showing that the identifications everywhere commute under homotopy. Explicitly, let $h : I^2 \to X$ be a path homotopy between $\gamma_1 : I \to X$ and $\gamma_2 : I \to X$ then for each $t$ let $h(t, -) : I \to X$ be the path homotoping the point $\gamma_1(t)$ to $\gamma_2(t)$. Then $[h(t_2, -)] \circ [\gamma_1(t_1 \mapsto t_2)] = [\gamma_2(t_1 \mapsto t_2)] \circ [h(t_1, -)]$ as maps $\L_{\gamma_1(t_1)} \to \L_{\gamma_2(t_2)}$. Since at the endpoints $h(0, - ) = h(1, -)$ is the constant path then we see that $[\gamma_1] = [\gamma_2]$. Therefore, monodromy defined a functor $M_\L : \Pi(X) \to \AbGrp$. 
\bigskip\\
Now I claim this association $\L \mapsto M_\L$ is functorial. Given a morphism $\eta : \L \to \L'$ of local groups we get get maps $\eta_x : \L_x \to \L_x'$ which commute with restriction and thus with the monodromy construction i.e. a natural transformation between functors $M_\L$ and $M_{\L'}$.
\bigskip\\
Now we need to show that $\L \mapsto M_\L$ is fully faithful. 
\bigskip\\
Finally, $M : \Loc{X} \to \AbGrp^{\Pi(X)}$ is essentially surjective. (PROVE THIS)
\end{proof}

\begin{remark}
When $X$ is connected, then groupoid $\Pi(X)$-representations are simply group representations of $\pi_1(X,x_0)$.
\end{remark}

\begin{definition}
Let $X$ be a locally-path-connected. For each $n > 1$ (for $n = 1$ the representation is simply the inner automorphism representation of a groupoid) there is a groupoid representation $\pi_n(X) : \Pi(X) \to \AbGrp$ which generalizes the action at each point $\pi_1(X, x_0) \acts \pi_n(X, x_0)$. By the above theorem, this corresponds to a local group $\underline{\pi_n(X)}$. 
\end{definition}

\section{Maps of a Proper Curve are Finite}

\begin{theorem}
Let $C$ be a proper curve over $k$ and $X$ is separated of finite type over $k$. Then any nonconstant map $f : C \to X$ over $k$ is finite.
\end{theorem}

\begin{proof}
Since $C \to \Spec{k}$ is proper and $X \to \Spec{k}$ is separated then by Tag 01W6 the map $f : C \to X$ is proper. The fibres of closed points $x \in X$ are proper closed subschemes $C_x \embed C$ (since if $C_x = C$ then $f : C \to X$ would be the constant map at $x \in X$) and thus finite since proper closed subsets of a curve are finite. Now I claim that if the fibres $f^{-1}(x)$ are finite at closed points $x \in X$ then all fibres are finite. Assuming this, $f : C \to X$ is proper with finite fibres and thus is finite by Tag 02OG.
\bigskip\\
To show the claim consider,
\[ E = \{ x \in X \mid \dim{C_x} = 0 \} \]   
Since $C$ is Noetherian, $\dim{C_x} = 0$ iff $C_x$ is finite (suffices to check for affine schemes since quasi-comact and dimension zero Noetherian rings are exactly Artinian rings which have finite spectrum). Then $E$ is locally constructible by Tag 05F9 and contains all the closed points of $X$. Since $X$ is finite type over $k$ then $X$ is Jacobinson which implies that $E$ is dense in every closed set. Then for any point $\xi \in X$ then $Z = \overline{\{ \xi \}}$ is closed and irreducible with generic point $\xi$ and thus $E \cap Z$ is dense in $Z$. Then by Tag 005K we have $\xi \in E$ so $E = X$ proving that all fibres are finite.
\end{proof}

\begin{remark}
The only facts about $C$ that I used were that $C \to \Spec{k}$ is proper and that $C$ is irreducible of dimension one. The second two properties are needed for the following to hold.
\end{remark}

\begin{lemma}
If $X$ is an irreducible Noetherian scheme of dimension one then every nontrivial closed subset of $X$ is finite.
\end{lemma}

\begin{proof}
Since $X$ is quasi-compact it suffices to show this property for affine schemes $X = \Spec{A}$ with $\dim{A} = 1$ and prime nilradical. Any nontrivial closed subset is of the form $V(I)$ for some proper radical ideal $I \subset X$ with $I \supsetneq \nilrad{A}$. Then $\height{I} = 1$ since any prime above $I$ must properly contain $\nilrad{A}$ and thus have height at least one but $\dim{A} = 1$. Then,
\[ \height{I} + \dim{A / I} \le \dim{A} \]
so $\dim{A / I} = 0$. Since $A$ is Noetherian so is $A / I$ but $\dim{A / I} = 0$ and thus $A / I$ is Artianian. Therefore $\Spec{A / I}$ is finite proving the proposition. 
\end{proof}

\begin{remark}
Since $C \to \Spec{k}$ is proper it is finite type over $k$ and thus $C$ is Noetherian.
\end{remark}

\section{Action on Fibres of Fibration}

\begin{theorem}
Let $F \embed E \xrightarrow{\sim} B$ be a fibration. Then there is a groupoid action $\Pi(B)$ on the space of fibres and in particular $\pi_1(B, x_0) \to \Aut{F}$. 
\end{theorem}

\begin{proof}
Consider a path $\gamma : I \to B$ from $x_1$ to $x_2$ and then the diagram,
\begin{center}
\begin{tikzcd}[row sep = large, column sep = large]
F_{x_1} \arrow[r, hook] \arrow[d, hook] & E \arrow[d, "p"]
\\
F_{x_1} \times I \arrow[ru, dashed, "\tilde{\gamma}"] \arrow[r, "\gamma"] & B
\end{tikzcd}
\end{center}
By homotopy lifting we get a map $\tilde{\gamma} : F_{x_1} \times I \to E$ lifitng $\gamma : F_{x_1} \times I \to B$. Then $p \circ \tilde{\gamma} = \gamma$ so $\tilde{\gamma}(-, 1) \subset F_{x_2}$ since $p \circ \tilde{\gamma}(-,1) = \gamma(1) = x_2$. Therefore we get a map $[\gamma] : F_{x_1} \to F_{x_2}$ via $[\gamma](x) = \tilde{\gamma}(x, 1)$. 
\bigskip\\
I claim that two lifts of homotopic paths are homotopic. Given two paths $\gamma_1, \gamma_2 : I \to B$ and a path homotopy $h : I^2 \to B$ and two lifts $\tilde{\gamma_1}, \tilde{\gamma_2} : F_{x_1} \times I \to E$ we want a map $F_{x_1} \times I^2 \to E$ above $h : \times I^2 \to B$. This map is defined on $F_{x_1} \times (I \times \{ 0, 1 \} \cup \{ 0 \} \times I)$ via $\tilde{\gamma}_1$ on $F_{x_1} \times I \times \{0\}$ and $\tilde{\gamma}_2$ on $F_{x_1} \times I \times \{0\}$ any by inclusion of the fibre $F_{x_1}$ on $F_{x_1} \times \{ 0 \} \times I$ (constant on $I$) since $h_ {\{0\} \times I}$ is constant since it is a path homotopy. Then by homotopy lifitng, we get $\tilde{h} : F_{x_1} \times I \times I \to E$ such that $p \circ \tilde{h} = h$ and thus $\tilde{h}(-, 1, -) : F_{x_1} \times I \to F_{x_2}$ gives a homotopy from $[\gamma_1] : F_{x_1} \to F_{x_2}$ to $[\gamma_2] : F_{x_1} \to F_{x_2}$. 
\bigskip\\
Therefore, we have a representation of $\Pi(B)$ on $\mathbf{hTop}$ sending $x \mapsto F_x$ and $\gamma \mapsto [\gamma]$. 
\end{proof}


\section{Serre - Vanishing}

\begin{rmk}
First we prove the result for the case $\P^n_R$. 
\end{rmk}

\begin{theorem}
Let $\P^n = \P^n_R$. For any coherent $\struct{\P^n}$-module $\F$ there is some $r > 0$ such that,
\[ H^i(\P^n_R, \F \otimes_{\struct{\P^n}} \struct{\P^n}(s)) = 0 \]
for all $i > 0$ and $s \ge r$.
\end{theorem}

\begin{proof}
Since this holds for $i > n$ we may apply reverse induction on $i$. Assume the theorem holds for $i + 1$ and let $\F$ be some coherent sheaf. Since $\struct{\P^n}(1)$ is ample, for some $\ell > 0$ the sheaf $\F \otimes_{\struct{\P^n}} \struct{\P^n}(\ell)$ is generated by global sections,
\[ \bigoplus_{j = 1}^N \struct{\P^n} \onto \F \otimes_{\struct{\P^n}} \struct{\P^n}(\ell) \]
and thus tensoring by $\struct{\P^n}(-\ell)$ we get a surjection,
\[ \bigoplus_{j = 1}^N \struct{\P^n}(-\ell) \onto \F \]
which we may extend to an exact sequence,
\begin{center}
\begin{tikzcd}
0 \arrow[r] & \G \arrow[r] &  \bigoplus\limits_{j = 1}^N \struct{\P^n}(-\ell)  \arrow[r] & \F \arrow[r] & 0
\end{tikzcd}
\end{center}
Since $\struct{\P^n}(d)$ is locally free it is flat (exactness can be checked on stalks) so we get a short exact sequence,
\begin{center}
\begin{tikzcd}
0 \arrow[r] & \G(d) \arrow[r] &  \bigoplus\limits_{j = 1}^N \struct{\P^n}(d - \ell)  \arrow[r] & \F(d) \arrow[r] & 0
\end{tikzcd}
\end{center}
Applying the LES of homology we get,
\begin{center}
\begin{tikzcd}
\bigoplus\limits_{j = 1}^N H^i(\P^n_R, \struct{\P^n}(d - \ell)) \arrow[r] & H^i(\P^n_R, \F(d)) \arrow[r] & H^{i+1}(\P^n_R, \G(d))
\end{tikzcd}
\end{center}
By the induction hypothesis, for all sufficently large $d \ge r_\G$ the cohomology $H^{i+1}(\P^n_R, \G(d)) = 0$ vanishes and furthermore by explicit calcuation, $H^i(\P^n_R, \struct{\P^n}(d - \ell)) = 0$ for $i > 0$ and $d \ge \ell$ so take $r_\F = \max\{ \ell, r_\G \}$ and then for $d \ge r_\F$ we find,
\[ H^i(\P^n_R, \F \otimes_{\struct{\P^n}} \struct{\P^n}(d)) = 0 \]
proving the result by induction. 
\end{proof}

\begin{theorem}
Let $R$ be a noetherian ring and $X \to \Spec{R}$ proper. Furthermore, let $\L$ be an ample line bundle on $X$. Then for any coherent $\struct{X}$-module $\F$ there is some $r > 0$ such that,
\[ H^i(X, \F \otimes_{\struct{X}} \L^{\otimes s}) = 0 \]
for all $i > 0$ and $s \ge r$.
\end{theorem}

\begin{proof}
Since $X \to \Spec{R}$ is finite type and $X$ has an ample line bundle $\L$ then $X$ must be quasi-projective over $R$ for some immersion $\iota : X \to \P^N_R$ where $\L^{\otimes d} = \iota^* \struct{\P^N}(1)$. Since $X \to \Spec{R}$ is proper and $\P^N_R \to \Spec{R}$ is separated then $\iota : X \to \P^N_R$ is automatically proper hence a closed immersion so $X$ is projective. 
\bigskip\\
Being a closed immersion $\iota : X \to \P^N_R$ is affine so we may compute (the Leray spectral sequence degenerates),
\[ H^i(X, \G) = H^i(\P^N_R, \iota_* \G) \]
for any quasi-coherent sheaf on $X$. Therefore, considering the coherent sheaf $\G = \F \otimes_{\otimes{\struct{X}}} \L^{\otimes s}$ it suffices to compute,
\[ H^i(\P^N_R, \iota_* ( \F \otimes_{\struct{X}} \L^{\otimes s} ) ) \]
We will apply the projection formula noting that writing $s = nd + r$ gives,
\[ \L^{\otimes s} = (\L^{\otimes d})^{\otimes n} \otimes_{\struct{X}} \L^{\otimes r} = (\iota^* \struct{\P^N}(1))^{\otimes n} \otimes_{\struct{X}} \L^{\otimes r} = \iota^* \struct{\P^n}(n) \otimes_{\struct{X}} \L^{r} \]
Therefore, let $\E = \struct{\P^n}(n)$ in the projection formula to find that,
\[ \iota_* ( \F \otimes_{\struct{X}} \L^{s} ) = \iota_* (\F \otimes_{\struct{X}} \L^{r} \otimes_{\struct{X}} \iota^* \struct{\P^N}(n)) = \iota_* (\F \otimes_{\struct{X}} \L^{r}) \otimes_{\struct{\P^n}} \struct{\P^N}(n) \]
Since $\iota_* (\F \otimes_{\struct{X}} \L^{r})$ is coherent the previous proposition alows us to choose $n$ large enough (taking the maximum of the $n$ large enough to kill the cohomology of each of $r = 0, 1, \dots, d-1$) so that,
\[ H^i(\P^N_R, \iota_* (\F \otimes_{\struct{X}} \L^{r}) \otimes_{\struct{\P^n}} \struct{\P^N}(n))  = 0 \]
for any $r = 0, 1, \dots, d-1$ and $n \gg 0$.
Therefore, for all sufficiently large $s$ we have,
\[ H^i(X, \F \otimes_{\struct{X}} \L^{\otimes s}) = H^i(\P^N_R, \iota^* (\F \otimes_{\struct{X}} \L^{\otimes s})) = H^i(\P^N_R, \iota_* (\F \otimes_{\struct{X}} \L^{r}) \otimes_{\struct{\P^n}} \struct{\P^N}(n)) = 0 \]
\end{proof}

\begin{theorem}[projection formula]
Let $f : X \to Y$ be a morphism of ringed spaces $\F$ a $\struct{X}$-module and $\E$ a finite locally free $\struct{Y}$-module. Then,
\[ R^q f_* (\F \otimes_{\struct{X}} f^* \E) = R^q f_* \F \otimes_{\struct{Y}} \E \]
\end{theorem}

\begin{theorem}
Let $X$ be projective, then the functors $\Ext{i}{\struct{X}}{-}{\G} : \Coh{\struct{X}} \to \Mod{\Gamma(X, \struct{X})}$ for a fixed quasi-coherent $\struct{X}$-module $\G$ are universal contravariant $\delta$-functors. 
\end{theorem}

\begin{proof}
It suffices to show that $\Ext{i}{}{-}{\G}$ are coeffaceable for all $i > 0$. Since $X$ is projective there is an ample line bundle $\L$ on $X$ and for the coherent $\struct{X}$-module $\G$ there is some $r > 0$ such that,
\[ H^i(X, \G \otimes_{\struct{X}} \L^{\otimes s}) = 0 \]
for any $s \ge r$ and $i > 0$. Then since $\L$ is ample, for any coherent $\struct{X}$-module $\F$ for some $n_0$ such that for $n \ge n_0$ the coherent sheaf $\F \otimes_{\struct{X}} \L^{\otimes n}$ is generated by global sections. Choosing $n \ge \max \{ n_0, r \}$ we get a surjection,
\[ \bigoplus_{j = 1}^N \struct{X} \onto \F \otimes_{\struct{X}} \L^{\otimes n} \]
However, since $\L$ is a line bundle we may tensor by $\L^{\otimes -n} = (\L^{\otimes n}])^\vee  $ to get a surjection,
\[ \H = \bigoplus_{j = 1}^N \L^{\otimes -n} \onto \F \]
Furthermore, since $\L$ is locally free of rank one,
\[ \Ext{i}{\struct{X}}{\H}{\G} = \bigoplus_{j = 1}^N \Ext{i}{\struct{X}}{(\L^{\otimes n})^\vee}{\G} = \bigoplus_{j = 1}^N \Ext{i}{\struct{X}}{\struct{X}}{\L^{\otimes n} \otimes_{\struct{X}} \G} = \bigoplus_{j = 1}^N H^i(X,  \G \otimes_{\struct{X}} \L^{\otimes n}) = 0 \]
for $i > 0$ by Serre vanishing showing that $\Ext{i}{\struct{X}}{-}{\G}$ is coeffaceable for all $i > 0$.  
\end{proof}

\section{Computing Ext and Tor in the Second Argument}

\subsection{Ext}

\begin{definition}
Let $\C$ be an abelian category (possibly enriched over another category $\D$). Then if $\C$ has enough injectives, $\Ext{i}{\C}{A}{-} : \C \to \D$ are the right-derived functors of $\Hom{\C}{A}{-} : \C \to \D$. 
\end{definition}

\begin{lemma}
$\Ext{i}{\C}{-}{M} : \C \to \D$ is a contravariant functor. 
\end{lemma}

\begin{proof}
Given an injective resolution $M \to \I^\bullet$ and a map $A \to B$ we get a morphism of complexes $\Hom{\C}{B}{\I^\bullet} \to \Hom{\C}{A}{\I^\bullet}$ and thus a morphims of cohomology,
\[ \Ext{i}{\C}{B}{M} \to \Ext{i}{\C}{A}{M} \]
which clearly respects composition.
\end{proof}

\begin{lemma}
If $P$ is projective then $\Ext{i}{\C}{P}{-} = 0$ for $i > 0$.
\end{lemma}

\begin{proof}
This follow immediatly from the defining property that $\Hom{\C}{P}{-}$ is exact.
\end{proof}

\begin{prop}
Given a short exact sequence,
\begin{center}
\begin{tikzcd}
0 \arrow[r] & A \arrow[r] & B \arrow[r] & C \arrow[r] & 0
\end{tikzcd}
\end{center}
in $\C$ and some $M \in \C$ then there is a long exact sequence, 
\begin{center}
\begin{tikzcd}
0 \arrow[r] & \Hom{\C}{C}{M} \arrow[draw=none]{d}[name=Z, shape=coordinate]{}  \arrow[r] & \Hom{\C}{B}{M} \arrow[r] & \Hom{\C}{A}{M} 
\arrow[dll,
rounded corners, crossing over,
to path={ -- ([xshift=2ex]\tikztostart.east)
|- (Z) [near end]\tikztonodes
-| ([xshift=-2ex]\tikztotarget.west)
-- (\tikztotarget)}]
\\
& \Ext{1}{\C}{C}{M} \arrow[draw=none]{d}[name=ZZ, shape=coordinate]{}  \arrow[r] & \Ext{1}{\C}{B}{M} \arrow[r] & \Ext{1}{\C}{A}{M} \arrow[dll,
rounded corners, crossing over,
to path={ -- ([xshift=2ex]\tikztostart.east)
|- (ZZ) [near end]\tikztonodes
-| ([xshift=-2ex]\tikztotarget.west)
-- (\tikztotarget)}]
\\
& \Ext{2}{\C}{C}{M} \arrow[r] & \Ext{2}{\C}{B}{M} \arrow[r] & \Ext{2}{\C}{A}{M} \arrow[r] & \cdots
\end{tikzcd}
\end{center}
\end{prop}

\begin{proof}
Take an injective resolution $M \to \I^\bullet$. Then since each $\I^n$ is injective the functor $\Hom{\C}{-}{\I^n}$ is exact so we get an exact sequence of complexes,
\begin{center}
\begin{tikzcd}
0 \arrow[r] & \Hom{\C}{C}{\I^\bullet} \arrow[r] & \Hom{\C}{B}{\I^\bullet} \arrow[r] & \Hom{\C}{A}{\I^\bullet} \arrow[r] & 0
\end{tikzcd}
\end{center}
Taking the cohomology sequence of this short exact sequence of complexes gives the desired long exact sequence. 
\end{proof}

\begin{lemma}
If $P_\bullet \to A$ is a projective resolution then $\Ext{i}{\C}{A}{-} = H^i(\Hom{\C}{P_\bullet}{-})$. 
\end{lemma}

\begin{proof}
We may use the acyclicity lemma which may be proven by the above exact sequence for $\Hom{\C}{-}{M}$ noting that $\Ext{i}{\C}{P_n}{M} = 0$. However, a more elegant argument goes as follows. Since $P_\bullet$ is a complex of projectives the functor $\Hom{\C}{P_n}{-}$ is exact so for any exact sequence,
\begin{center}
\begin{tikzcd}
0 \arrow[r] & M \arrow[r] & N \arrow[r] & K \arrow[r] & 0
\end{tikzcd}
\end{center}
we get an exact sequence of complexes,
\begin{center}
\begin{tikzcd}
0 \arrow[r] & \Hom{\C}{P_\bullet}{M} \arrow[r] & \Hom{\C}{P_\bullet}{N} \arrow[r] & \Hom{\C}{P_\bullet}{K} \arrow[r] & 0
\end{tikzcd}
\end{center}
which gives a long exact sequence in the cohomology functors $H^i(\Hom{\C}{P_\bullet}{-})$ which shows that $H^i(\Hom{\C}{P_\bullet}{-})$ form a $\delta$-functor. Furthermore, since $\C$ has enough injectives, for any $M \in \C$ we can embed $M \embed I$ into an injective $I$ and $H^i(\Hom{\C}{P_\bullet}{I}) = 0$ since $\Hom{\C}{-}{I}$ is exact. Therefore,  $H^i(\Hom{\C}{P_\bullet}{-})$ is an effaceable $\delta$-functor and thus universal by Grothendieck. Furthermore, since $\Hom{\C}{-}{M}$ is left-exact,
\begin{align*}
H^0(\Hom{\C}{P_\bullet}{-}) & = \ker{(\Hom{\C}{P^0}{-} \to \Hom{\C}{P^1}{-})} = \Hom{\C}{\coker{(P^1 \to P^0)}}{-} 
\\
& = \Hom{\C}{A}{-} 
\end{align*}
However, $\Ext{i}{\C}{A}{-}$ are the derived functors of $\Hom{\C}{A}{-}$ so they too form a universal $\delta$-functor over $\Hom{\C}{A}{-}$. Thus, since universal $\delta$-functors with naturally isomorphic first terms are unique,
\[ \Ext{i}{\C}{A}{-} = H^i(\Hom{\C}{P_\bullet}{-}) \]
\end{proof}

\begin{rmk}
The above formalism applies exactly to any bifunctor $F : \C^\op \times \C \to \D$ such that for any $A \in \C$ there are enough $F(A, -)$-acyclics $I$ for which $F(-,I)$ is exact and replacing `injective' with this class of acyclics and `projective' by any class of onjects $P$ such that $F(P, -)$ is exact. Furthermore we assume $\C$ is abelian with enough injectives, $\D$ is additive, and $F : \C^\op \times \C \to \D$ is additive. 
\bigskip\\
For example, in the category of $\struct{X}$-modules on a scheme, the bifunctor,
\[ \shHom{\struct{X}}{-}{-} : \shMod{\struct{X}}^\op \times \shMod{\struct{X}} \to \shMod{\struct{X}} \] satisfies the following properties. First, for injective sheaves $\I$ we have $\shHom{\struct{X}}{-}{\I}$ is exact (and there are enough injectives which are obviously acyclic for $\shHom{\struct{X}}{\F}{-}$. Second, if $\E$ is a locally-free sheaf then,
\[ \shHom{\struct{X}}{\E}{-} = \E^\vee \otimes_{\struct{X}} (-) \]
and $\E^\vee$ is locally free and thus flat so $\shHom{\struct{X}}{\E}{-}$ is exact. Therefore, we see that $\shExt{i}{\struct{X}}{-}{\G}$ is a contravariant $\delta$-functor, vanishing on locally free sheaves, which may be computed via cohomology of locally-free complexes. Furthermore, whenever $\shMod{\struct{X}}$ has enough locally frees (for example whenever $X$ has an ample line bundle) then $\shExt{i}{\struct{X}}{-}{\G}$ forms a universal contravariant $\delta$-functor.
\end{rmk}

\subsection{Tor}

\begin{defn}
When $\C$ has a right-exact cmonoid structure $- \otimes_\C -$ and $\C$ has enough projectives then define $\Tor{\C}{i}{A}{-} : \C \to \C$ as the left-derived functors of $A \otimes_\C - : \C \to \C$. 
\end{defn}

\begin{rmk}
Here it will be necessary to assume that $\C$ has enough flat objects ($- \otimes_\C F$ is exact) which happens say when projectives are flat. 
\end{rmk}

\begin{lemma}
$\Tor{\C}{i}{-}{M}$ is a covariant functor.
\end{lemma}

\begin{proof}
Given a map $A \to B$ and a projective resolution $P_\bullet \to M$ we get a morphism of complexes, $A \otimes_\C P_\bullet \to B \otimes_\C P_\bullet$ and thus a morphism of homology,
\[ \Tor{\C}{i}{A}{M} \to \Tor{\C}{i}{B}{M} \]
\end{proof}

\begin{defn}
We say an object $P \in \C$ is \textit{flat} if $P \otimes_\C - $ is an exact functor.
\end{defn}

\begin{lemma}
The following are equivalent,
\begin{enumerate}
\item $P$ is flat
\item $\Tor{\C}{i}{P}{-} = 0$ for all $i > 0$
\item $\Tor{\C}{1}{P}{-} = 0$.
\end{enumerate}
\end{lemma}

\begin{proof}
Clearly $(a) \implies (b) \implies (c)$. Now, if $\Tor{\C}{1}{P}{-} = 0$ then for any exact sequence,
\begin{center}
\begin{tikzcd}
0 \arrow[r] & A \arrow[r] & B \arrow[r] & C \arrow[r] & 0
\end{tikzcd}
\end{center}
we get an exact sequence,
\begin{center}
\begin{tikzcd}
\Tor{\C}{1}{P}{C} \arrow[r] & P \otimes_\C A \arrow[r] & P \otimes_\C B \arrow[r] & C \otimes_\C P \arrow[r] & 0
\end{tikzcd}
\end{center}
so if $\Tor{\C}{1}{P}{-} = 0$ then $P \otimes_\C -$ is exact i.e. $P$ is flat.
\end{proof}


\begin{prop}
Given a short exact sequence,
\begin{center}
\begin{tikzcd}
0 \arrow[r] & A \arrow[r] & B \arrow[r] & C \arrow[r] & 0
\end{tikzcd}
\end{center}
in $\C$ and some $M \in \C$ then there is a long exact sequence, 
\begin{center}
\begin{tikzcd}
\cdots \arrow[r] & \Tor{\C}{2}{A}{M} \arrow[draw=none]{d}[name=Z, shape=coordinate]{}  \arrow[r] & \Tor{\C}{2}{B}{M} \arrow[r] & \Tor{\C}{2}{C}{M} 
\arrow[dll,
rounded corners, crossing over,
to path={ -- ([xshift=2ex]\tikztostart.east)
|- (Z) [near end]\tikztonodes
-| ([xshift=-2ex]\tikztotarget.west)
-- (\tikztotarget)}]
\\
& \Tor{\C}{1}{A}{M} \arrow[draw=none]{d}[name=ZZ, shape=coordinate]{}  \arrow[r] & \Tor{\C}{1}{B}{M} \arrow[r] & \Tor{\C}{1}{C}{M} \arrow[dll,
rounded corners, crossing over,
to path={ -- ([xshift=2ex]\tikztostart.east)
|- (ZZ) [near end]\tikztonodes
-| ([xshift=-2ex]\tikztotarget.west)
-- (\tikztotarget)}]
\\
 & A \otimes_\C M \arrow[r] & B \otimes_\C M \arrow[r] & C \otimes_\C M \arrow[r] & 0
\end{tikzcd}
\end{center}
\end{prop}

\begin{proof}
Take a flat resolution $F_\bullet \to M$. Then since each $F^n$ is flat the functor $F^n \otimes_\C -$ is exact so we get an exact sequence of complexes,
\begin{center}
\begin{tikzcd}
0 \arrow[r] & A \otimes_\C F_\bullet \arrow[r] & B \otimes_\C F_\bullet \arrow[r] & C \otimes_\C F_\bullet \arrow[r] & 0
\end{tikzcd}
\end{center}
Taking the homology sequence of this short exact sequence of complexes gives the desired long exact sequence since by the acylicity lemma we may commute $\Tor{\C}{i}{A}{M}$ via a flat resolution of $M$.  
\end{proof}

\begin{lemma}
If $F_\bullet \to A$ is a free resolution then $\Tor{\C}{i}{A}{-} = H_i(F_\bullet \otimes_\C -)$.
\end{lemma}

\begin{proof}
We may use the acyclicity lemma which may be proven by the above exact sequence for $\Tor{\C}{i}{-}{M}$ showing that $\Tor{\C}{i}{-}{M}$ forms a $\delta$-functor and noting that $\Tor{\C}{i}{F_n}{M} = 0$. However, a more elegant argument goes as follows. Since $F_\bullet$ is a complex of frees the functor $F_n \otimes -$ is exact so for any exact sequence,
\begin{center}
\begin{tikzcd}
0 \arrow[r] & M \arrow[r] & N \arrow[r] & K \arrow[r] & 0
\end{tikzcd}
\end{center}
we get an exact sequence of complexes,
\begin{center}
\begin{tikzcd}
0 \arrow[r] & F_\bullet \otimes M \arrow[r] & F_\bullet \otimes N \arrow[r] & F_\bullet \otimes K \arrow[r] & 0
\end{tikzcd}
\end{center}
which gives a long exact sequence in the homology functors $H_i(F_\bullet \otimes -)$ which shows that $H_i(F_\bullet \otimes -)$ form a (homological) $\delta$-functor. Furthermore, since $\C$ has enough frees, for any $M \in \C$ we have a surjection $F \onto M$ for some free $F$ and $H_i(F_\bullet \otimes_\C F) = 0$ since $- \otimes \F$ is exact (both rows and columns stay exact, it is the exactness of the columns here ensured by freeness of $F$ which is needed for the vanishing). Therefore,  $H_i(F_\bullet \otimes -)$ is a coeffaceable $\delta$-functor and thus universal by Grothendieck. Furthermore, since $- \otimes_\C M$ is right-exact,
\begin{align*}
H_0(F_\bullet \otimes_\C -) & = \coker{([F_1 \otimes_\C -] \to [F_0 \otimes_\C -])} = \coker{(F_1 \to F_0)} \otimes_\C (-) = A \otimes_\C (-)
\end{align*}
However, $\Tor{\C}{i}{A}{-}$ are the derived functors of $A \otimes_\C -$ so they too form a universal $\delta$-functor over $A \otimes_\C -$. Thus, since universal $\delta$-functors with naturally isomorphic first terms are unique,
\[ \Tor{\C}{i}{A}{-} = H_i(F_\bullet \otimes_\C -) \]
\end{proof}

\begin{prop}
Tor is symmetric: there is a natural isomorphism $\Tor{\C}{i}{A}{B} = \Tor{\C}{i}{B}{A}$.
\end{prop}

\begin{proof}
Choose a flat resolution $F_\bullet \to A$. By the above lemma $\Tor{\C}{i}{A}{B} = H_i(F_\bullet \otimes_\C B)$. However, by the symmetry of $- \otimes_\C -$ we have, $H_i(F_\bullet \otimes_\C B) = H_i(B \otimes_\C F_\bullet)$. Furthermore, because $\Tor{\C}{i}{B}{-}$ is the left-derived functor of $B \otimes_\C -$ we may compute it via acyclics (since it is a $\delta$-functor) so $\Tor{\C}{i}{B}{A} = H_i(B \otimes_\C F_\bullet)$ and thus,
\[ \Tor{\C}{i}{A}{B} = H_i(F_\bullet \otimes_\C B) = H_i(B \otimes_\C F_\bullet) = \Tor{\C}{i}{B}{A} \]
\end{proof}

\begin{remark}
These arguments apply to the satellites of any symmetric bifunctor $F : \C \times \C \to \D$ between abelian categories such that $\C$ has enough objects $A$ for which $F(A, -)$ is exact, in particular, if $F(P, -)$ is exact for projectives (as is the tensor product). 
\end{remark}

\newcommand{\Tot}{\mathrm{Tot}}

\begin{remark}
Symmetry follows directly from the following spectral sequence argument. Let $F^A_\bullet \to A$ and $F^B_\bullet \to B$ be free resolutions. Then consider the double complex $C_{p,q} = F^A_p \otimes_\C F^B_q$. There are two spectral sequences which compute the homology of the total complex $\Tot(C_{\bullet. \bullet})$. These two spectral sequences agree on their zeroth page, ${}^A E^0_{p,q} = {}^B E^0_{p,q} = F^A_p \otimes_\C F^B_q$. Now, the first pages are,
\begin{align*}
{}^A E^1_{p,q} & = H_{p}(C_{\bullet, q}) = H_{p}(F^A_\bullet \otimes_\C F^B_q) = A \otimes_\C F_p^B \quad \text{ in p degree zero}
\\
{}^B E^1_{p,q} & = H_{q}(C_{p, \bullet}) = H_{q}(F^A_p \otimes_\C F^B_\bullet) = F^A_p \otimes_\C B \quad \text{ in q degree zero}
\end{align*}
where we have used the fact that $- \otimes_\C F^B_q$ and $F^A_p \otimes_\C - $ are exact (since the resolutions are free) and thus commute with taking homology. Then the second pages are,
\begin{align*}
{}^A E^2_{p,q} & = H_{q}({}^A E^1_{p, \bullet}) = H_{q}(A \otimes_\C F_\bullet^B) = L^q(A \otimes_\C -)(B) \quad \text{ in p degree zero}
\\
{}^B E^2_{p,q} & = H_{p}({}^B E^1_{\bullet, q}) = H_{p}(F^A_p \otimes_\C B) = L^p (- \otimes_\C B)(A) \quad \text{ in q degree zero}
\end{align*}
Since the second pages are supported in a single row or collumn both spectral sequences are converged. Therefore, we find,
\[ H_n(\Tot(C_{\bullet, \bullet})) = {}^A E^2_{0,n} = {}^B E^2_{n, 0} = L^n(A \otimes_\C -)(B) = L^n(- \otimes_\C B)(A) \] 
Therefore, for a bifunctor we may derive in either component to get the same satellite functors. Furthermore, when $- \otimes_\C -$ is symmetric then, 
\begin{align*}
L^n(A \otimes_\C -)(B) & = L^n(- \otimes_\C A)(B) = L^n(B \otimes_\C -)(A) 
\\
L^n(- \otimes_\C B)(A) & = L^n(B \otimes_\C -)(A) = L^n(- \otimes_\C A)(B) 
\end{align*}
so the derived functors are symmetric.
\end{remark}

\subsection{Acyclicity}

\begin{lemma}
Let $F$ be a $\delta$-functor. Suppose there is an exact sequence,
\begin{center}
\begin{tikzcd}
0 \arrow[r] & A \arrow[r] & I^0 \arrow[r] & \cdots \arrow[r] & I^n \arrow[r] & K \arrow[r] & 0
\end{tikzcd}
\end{center}
where $I^i$ are $F$-acyclic. Then for $i > 0$,
\[ F^{n + 1 + i}(A) = F^i(A) \]
and $F^{n+1}(A) = \ker{(F^0(I^{n}) \to F^0(K))}$. 
\end{lemma}

\begin{proof}
We prove this by induction on $n$. For $n = 0$, we are given a short exact sequence,
\begin{center}
\begin{tikzcd}
0 \arrow[r] & A \arrow[r] & I^0 \arrow[r] & K \arrow[r] & 0
\end{tikzcd}
\end{center}
Taking the long exact sequence,
\begin{center}
\begin{tikzcd}
0 \arrow[r] & F^0(A) \arrow[r] & F^0(I^0) \arrow[r] & F^0(K) \arrow[r] & F^1(A) \arrow[r] & F^1(I^0)
\end{tikzcd}
\end{center}
and
\begin{center}
\begin{tikzcd}
F^i(I^0) \arrow[r] & F^i(K) \arrow[r] & F^{i+1}(A) \arrow[r] & F^{i+1}(I^0)
\end{tikzcd}
\end{center}
However, $I^0$ is $F$-acyclic so $F^i(I^0) = 0$ for $i > 0$ and thus $F^{i + 1}(A) = F^i(K)$ for $i > 0$. Furthermore, for the second sequence $F^1(A) = \ker{(F^0(I^0) \to F^0(K))}$. 
\bigskip\\
Now we assume the result holds for $n-1$. We split the exact sequence into,
\begin{center}
\begin{tikzcd}
0 \arrow[r] & A \arrow[r] & I^0 \arrow[r] & \cdots \arrow[r] & I^n \arrow[r] & \tilde{K} \arrow[r] & 0
\end{tikzcd}
\end{center}
and 
\begin{center}
\begin{tikzcd}
0 \arrow[r] & \tilde{K} \arrow[r] & I^n \arrow[r] & K \arrow[r] & 0
\end{tikzcd}
\end{center}
Applying the induction hypothesis we see that,
$F^{n + i}(A) = F^{i}(\tilde{K})$
for $i > 0$. In particular, we will use, $F^{n+1}(A) = F^{1}(\tilde{K})$. Now, by the LES of the second exact sequence we find,
\begin{center}
\begin{tikzcd}
0 \arrow[r] & F^0(\tilde{K}) \arrow[r] & F^0(I^n) \arrow[r] & F^0(K) \arrow[r] & F^1(\tilde{K}) \arrow[r] & F^1(I^n)
\end{tikzcd}
\end{center}
and
\begin{center}
\begin{tikzcd}
F^i(I^n) \arrow[r] & F^i(K) \arrow[r] & F^{i+1}(\tilde{K}) \arrow[r] & F^{i+1}(I^n)
\end{tikzcd}
\end{center}
However, $I^n$ is $F$-acyclic so for $i > 0$ we get,
\[ F^i(K) = F^{i+1}(\tilde{K}) \quad \text{ and } \quad F^1(\tilde{K}) = \coker{(F^0(I^n) \to F^0(K))} \] 
Therefore, we have $F^{n + i + 1}(A) = F^{i+1}(\tilde{K}) = F^i(K)$ for $i > 0$. Furthermore, \[ F^{n+1}(A) = F^1(\tilde{K}) = \coker{(F^0(I^n) \to F^0(K))} \]
proving the lemma.
\end{proof}

\begin{theorem}[acyclicity]
If $F$ is a $\delta$-functor and $A \to I^\bullet$ a resolution of $F$-acyclic objects,
\[ F^n(A) = H^n(F^0(I^\bullet)) \]
\end{theorem}

\begin{proof}
We may truncate the resolution by adding a cokernel $K$ to give an exact sequence,
\begin{center}
\begin{tikzcd}
0 \arrow[r] & A \arrow[r] & I^0 \arrow[r] & \cdots \arrow[r] & I^n \arrow[r] & K \arrow[r] & 0
\end{tikzcd}
\end{center}
By the previous lemma, we can compute,
\[ F^{n+1}(A) = \coker{(F^0(I^n) \to F^0(K))} \]
However, by exactness, $K = \coker{(I^{n-1} \to I^n)} = \ker{(I^{n+1} \to I^{n+2})}$. Furthermore, $F^0$ is left-exact so $F^0(K) = \ker{(F(I^{n+1}) \to F(I^{n+2}))}$. Therefore, for $n \ge 0$ we find,
\[ F^{n+1}(A) = \coker{(F^0(I^n) \to F^0(K))} = \coker{(F^0(I^n) \to \ker{(F(I^{n+1}) \to F(I^{n+2}))})} = H^{n+1}(F^0(I^\bullet)) \]
Furthermore, $F^0$ is left-exact so,
\[ F^0(A) = F^0(\ker{(I^0 \to I^1)}) = \ker{(F^0(I^0) \to F^0(I^1))} = H^0(F^0(I^\bullet)) \]
\end{proof}

\subsection{Tor for Sheaves}

\begin{remark}
Often the categories $\shMod{\struct{X}}$, $\QCoh{\struct{X}}$, and $\Coh{\struct{X}}$ do not have enough projectives. Therefore, we cannot define Tor for sheaves as a left-derived functor we need an alternative definition.  
\end{remark}

\begin{defn}
Let $X$ be a scheme such that $\Coh{\struct{X}}$ has enough locally-frees (e.g. $X$ has an ample line bundle). Given a coherent sheaf $\F$ and a resolution $\E_\bullet \to \F$ by locally free coherent sheaves, we define,
\[ \shTor{\struct{X}}{i}{\F}{-} = H_i( \E_\bullet \otimes_{\struct{X}} - ) \]
\end{defn}

\begin{prop}
$\shTor{\struct{X}}{i}{\F}{-}$ is a universal homological $\delta$-functor. 
\end{prop}

\begin{proof}
First, given an exact sequence of coherent sheaves,
\begin{center}
\begin{tikzcd}
0 \arrow[r] & \G_1 \arrow[r] & \G_2 \arrow[r] & \G_3 \arrow[r] & 0
\end{tikzcd}
\end{center}
we get an exact sequence of complexes,
\begin{center}
\begin{tikzcd}
0 \arrow[r] & \E_\bullet \otimes_{\struct{X}} \G_1 \arrow[r] & \E_\bullet \otimes_{\struct{X}} \G_2 \arrow[r] & \E_\bullet \otimes_{\struct{X}} \G_3 \arrow[r] & 0
\end{tikzcd}
\end{center}
since $\E^n$ is locally-free and thus flat. Taking homology gives a long exact sequence of $\Tor{\struct{X}}{i}{\F}{-}$ sheaves making it a homological $\delta$-functor.
It suffices to show that $\shTor{\struct{X}}{i}{\F}{-}$ is coeffaceable. Since there are enough locally-free sheaves for any coherent $\G$ we can find a locally-free and a surjection $\E' \onto \G$. Then, since $- \otimes_{\struct{X}} \E'$ is exact then,
\[ \shTor{\struct{X}}{i}{\F}{\E} = H_i(\E_\bullet \otimes_{\struct{X}} \E) = 0 \]
where $\E_\bullet \to \F$ is a locally-free resolution. Therefore, $\shTor{\struct{X}}{i}{\F}{-}$ is coeffaceable.
\end{proof}

\begin{rmk}
Since $\shTor{\struct{X}}{i}{\F}{-}$ is universal it will agree with any other reasonable definition (any definition which is a universal $\delta$-functor) because there is a unique universal $\delta$-functor over,
\[ \shTor{\struct{X}}{0}{\F}{-} = H_0(\E_\bullet \otimes_{\struct{X}} -) = \coker{(\E^1 \to \E^0)} \otimes_{\struct{X}} - = \F \otimes_{\struct{X}} - \]
where the second equality follows from right-exactness of $- \otimes_{\struct{X}} \G$. 
\end{rmk}

\begin{rmk}
Since $- \otimes_{\struct{X}} - : \shMod{\struct{X}} \times \shMod{\struct{X}} \to \shMod{\struct{X}}$ is a symmetric bifunctor with enough locally-frees which are flat. Then since $\shTor{\struct{X}}{i}{\F}{-}$ are the left-satellite functors of $\F \otimes_{\struct{X}} -$ we can apply the acyclicity lemma to show that we map compute sheaf Tor from a locally free resolution $\E_\bullet \onto \G$,
\[ \shTor{\struct{X}}{i}{\F}{\G} = H_i(\F \otimes_{\struct{X}} \E_\bullet) \]
which shows the symmetry of $\shTor{\struct{X}}{i}{-}{-}$. 
\end{rmk}

\section{Depth of Field}

First we calculate the size of the circle of confusion. Let the lense have aperature $D$ and focal length $f$. The image distance is given by,
\[ \frac{1}{i} + \frac{1}{o} = \frac{1}{f} \]
then,
\[ i = \frac{f o}{o - f} \]
Therefore, we can compute the change in image distance as $o$ changes,
\[ \deriv{i}{o} = \frac{f}{o - f} - \frac{f o}{(o - f)^2} = - \frac{f^2}{(o - f)^2} \]
For a depth of $\Delta o$ we have a spread of image depths,
\[ \Delta i  \approx \frac{f^2 \Delta o}{(o - f)^2} \]
Then the width of the circle of confusion is given by,
\[ \frac{C}{D} = \frac{\Delta i}{f + \Delta i} \approx \frac{\Delta i}{f} \]
Therefore,
\[ C = \frac{f D}{(o - f)^2} \Delta o \]
For a fixed allowable circle of confusion $C_{\text{max}}$ for the desired resolution, we find the depth of field,
\[ \text{DOF} = 2 \frac{C}{D} \cdot \frac{(o - f)^2}{f} = \frac{2 (o - f)^2 N C}{f^2} \]
where $N = f / D$ is the focal ratio. 

\subsection{Hyperfocal Distance}

At some focal distance $H$, all objects beyond $H$ are in focus. This occurs when,
\[ \frac{i - f}{f} = \frac{C}{D} \]
and
\[ i = \frac{f H}{H - f} \]
Then,
\[ \frac{H}{H - f} - 1 = \frac{f}{H - f} = \frac{C}{D} \]
Therefore,
\[ H = \frac{f (D + C)}{C} = \frac{f^2}{CN} + f \]
\bigskip\\
Alternativly,
if we focus at infinity and ask beyond which everything is in focus then,
\[ \frac{i - f}{i} = \frac{C}{D} \]
and

\[ i = \frac{f H}{H - f} \]
Then,
\[ 1 - \frac{H - f}{H} = \frac{f}{H} = \frac{C}{D} \]
Therefore,
\[ H = \frac{f D}{C} = \frac{f^2}{NC} \]

\section{Morphisms from Proper to Affine Schemes}

Let $X \to \Spec{R}$ be proper and $\Spec{A} \to \Spec{R}$ be affine. Now,
\[ \Hom{R}{X}{\Spec{A}} = \Hom{R}{A}{\Gamma(X, \struct{X})} \]
The map $X \to \Spec{A}$ is given as follows, consider $\varphi_x : A \to \Gamma(X, \struct{X}) \to \stalk{X}{x}$ then $x \mapsto \varphi_x^{-1}(\m_x)$. Therefore, all maps $X \to \Spec{A}$ are constant if $x \mapsto \res_x^{-1}(\m_x)$ is a fixed ideal independent of $x$.  

\section{Irreducible Polynomials over $\Z$}

\newcommand{\Fin}{\mathbb{F}}

Consider the map $\Spec{\Z[X]} \to \Spec{\Z}$. The fibres are, over the generic point $(0)$, we have $\Spec{\Q[X]} \to \Spec{\Q}$ which corresponds to ideals of the form $(f(X))$ for $f$ an irreducible polynomial $f \in \Q[X]$. The fibres over $(p)$ are $\Spec{\Fin_p[X]} \to \Spec{\Fin_p}$ whose primes are of the form $(f(X))$ for $f$ an irreducible polynomial $f \in \Fin_p[X]$. Therefore we get an explicit description of $\Spec{\Fin[X]}$, we have the primes, $(f(X))$ for irreducible $f \in \Q[X]$ (for which we may clear denominators to get $f \in \Z[X]$) and $(p, f(X))$ for irreduclbe $f \in \Fin_p[X]$ (choosing some representative in $\Z[X]$) and finally of course $(0)$ and $(p)$ are prime (corresponding to the generic points of the fibres). 
\bigskip\\
Suppose $f \in \Z[X]$ were irreducible then any prime (strictly) above $(f)$ must be of the form $(p, f(X))$ otherwise $f$ would be a nontrivial product. Then we have $\dim{\Z[X]/(f)} = 1$ furthermore, (COMPLETE THIS ARGUMENT ... )

\section{Normalization}

\begin{example}
Consider $X = \Spec{A}$ with $A = k[x,y]/(y^2 - x^2(x + 1)$. Then consider,
\[ A \to k[t] \quad \quad x \mapsto t^2 - 1 \quad y \mapsto t(t^2 - 1) \]
Then $y^2 = t^2 (t^2 - 1)$ and $x^2 (x - 1) = t^2 (t^2 - 1)^2$ so this map is well-defined. This gives a dominant map,
\[ \A^1_k \to \Spec{A} \]
Furthermore, I claim that,
\[ \Frac{A} \embed k(t) \]
is an isomorphism, clearly $\Frac{A} \to k(t)$ is injective. The inverse map is $t \mapsto y/x$ then $y/x \mapsto t \mapsto y/x$ and $t \mapsto y/x \mapsto t$. Furthermore, $x \mapsto (t^2 - 1) \mapsto (y^2/x^2 - 1) = x$ and $y \mapsto t(t^2 -1) \mapsto y/x(y^2/x^2 - 1) = y$. Thus the map $\A^1 \to \Spec{A}$ is a dominant birational morphism. Furthermore,
\[ \overline{A} = k[y/x]  = k[t] \subset \Frac{A} \]
because $t = y/x$ satisfies the monic $t^2 - x - 1$ so $\A^1 \to \Spec{A}$ is the normalization.
\end{example}

\begin{example}
Consider the cusp $X = \Spec{A}$ with $A = k[x,y] = (y^2 - x^3)$. Then consider,
\[ A \to k[t] \quad \quad x \mapsto t^2 \quad y \mapsto t^3 \]
Then $y^2 \mapsto t^6$ and $x^2 \mapsto t^6$ so this is well-defined. This gives a dominant map,
\[ \A^1_k \to \Spec{A} \]
Furthermore, I claim that,
\[ \Frac{A} \embed k(t) \]
is an isomorphism. Send $t \mapsto y/x$ then $t \mapsto y/x \mapsto t$ and $y/x \mapsto t \mapsto y/x$. Then $y \mapsto t^3 \mapsto y^3 / x^3 = y$ and $x \mapsto t^2 \mapsto y^2 / x^2 = x$. Therefore, $\A^1_k \to \Spec{A}$ is a dominant birational morphism. Furthermore,
\[ \overline{A} = k[y/x] = k[t] \subset k(t) = \Frac{A} \]
because $t = y/x$ satisfies the monic $t^2 - x$. 
\end{example}

\begin{example}
Consider the tachnode $X = \Spec{A}$ with $A = k[x,y]/(x^2 - y^4)$. Then consider,
\[ A \to k[t, s]/(s^2 - 1) \quad \quad x \mapsto t \quad y \mapsto t^2 s \]
Then $x^4 \mapsto t^4$ and $y^2 \mapsto t^4$ so this is well-defined. this gives a dominant map,
\[ \Spec{k[t,s]/(s^2 - 1)} = \A^1_k \coprod \A^1_k \to \Spec{A} \]
On the irreduclble componenets $\p_{+} = (y - x^2)$ and $\p_{-} = (y + x^2)$ of $\Spec{A}$ we have,
\[ \stalk{X}{\p_{+}} = \Frac{k[x,y]/(y - x^2)} \quad \quad \stalk{X}{\p_{-}} = \Frac{k[x,y]/(y + x^2)} \]
and thus the map $\Spec{k[t, s]/(s^2 - 1)} \to \Spec{A}$ gives an isomorphism on each component and $\spec{k[t, x]/(s^2 - 1)}$ is normal. 
\end{example}

\section{A Very Werid Scheme}

For finite products we have,
\[ \Spec{A \times B} = \Spec{A} \coprod \Spec{B} \]
where we take the coproduct in the category of schemes. In particular, the primes of $A \times B$ are simply $\p_1 \times B$ or $A \times \p_2$ for primes $\p_1 \subset A$ and $\p_2 \subset B$. However, for infinite product this fails. Consider,
\[ X = \Spec{\prod\limits_{i = 0}^\infty k} \quad \quad R = \prod\limits_{i = 0}^\infty k \]
where $k$ field. The prime ideals of this ring are not just the kernels of the projections $R \to k$ which are maximal ideals. To see this, consider the ideal $I$ of functions $\N \to k$ which have finite support. Clearly $I \to R \to k$ is surjective for each projection so $I$ is not contained in any of the described primes. It turns out that prime ideals of $R$ correspond to ultrafilters $\F$ of $\N$ where $\p(\F)$ for some ultrafilter is the following,
\[ \p(\F) = \{ (a_i) \mid \{ i \mid a_i = 0 \} \in \F \} \]
Therefore, the principal ultrafilter $\F_i$ above $\{ i \}$ gives exactly $\p(\F_i) = \ker{\pi_i}$ but there are many more nonprincipal ultrafilters. 

\section{Coproducts in the Category of Schemes}

\begin{prop}
The forgetful functor $F : \Sch \to \Top$ preserves colimits. 
\end{prop}

\begin{rmk}
Let $\Hom{\Top}{F(X)}{S} = \Hom{\Sch}{X}{T(S)}$ 
\end{rmk}

\section{NOTE LOOK UP THE PROOF FOR PROJ -> LOCALLY FREE}

\section{Ravi Excersies}

\begin{rmk}
Maps $\Spec{k} \to \P^n_k$ are equivalent to giving a line bundle $\L$ on $\Spec{k}$ i.e. a one-dimensional $k$-vectorspace $V \cong k$ and $n+1$ sections $s_i \in V$ not all zero. We call this point $[s_0, \dots, s_n] \in \P^n_k$ up to isomorphism $\varphi : V \cong V'$ and $\varphi(s_i) = s_i'$ This is simply global scalling by $k^\times$. Furthermore, for any extension $K / k$ we can describe $\P^n_k(K)$ similarly but with $s_i \in K$. 
\end{rmk}

\begin{definition}
Projection from a rational point $\P^n_k \rat \P^{n-1}_k$ given a projection point $p \in \P^n_k$. We define this as follows: by an automorphism of $\P^n_k$ let $p = [1 : 0 : \cdots : 0]$. Take the dense open $U = D(X_0) \setminus \{ 0 \} = \Spec{x_1, \dots, x_n} \setminus \{ (0) \}$. Then consider the map $U \to \P^{n-1}_k$ via $\L = \struct{U}$ and $s_i = x_i$. These global sections generate because we have removed the point at which they all vanish. This rational map $\P^n_k \rat \P^{n-1}_k$ has domain $\Dom{f} = \P^n_k \setminus \{ p \}$.   
\end{definition}


\subsection{6.5 F}

Consider the conic $C = V(X^2 + Y^2 = Z^2) \subset \P^2_k$. Consider the map $\P^1_k \to \P^2_k$ defined by the line bundle $\L = \struct{\P^1}(2)$ and the sections $X_0^2 - X_1^2, 2 X_0 X_1, X_0^2 + X_1^2$. The image is exactly $C = V(X^2 + Y^2 = Z^2)$ and thus $C \cong \P^1_k$. However, if characteristic of $k = 2$ then these sections are $X_0^2 + X_1^2, 0, X_0^2 + X_1^2$ which does not define a map since these may all vanish simultaneously. In fact, $V(X^2 + Y^2 = Z^2)$ is not smooth in characteristic two since $X^2 + Y^2 = (X + Y)^2$ so we get $X + Y = \pm Z$ the union of two lines in $\P_k^2$. 
\bigskip\\ 
We can also describe an isomorphism as follows. First, lets do a change of coordinates $X \mapsto \tfrac{1}{2}(X + Z)$ and $Z \mapsto \tfrac{1}{2}(X - Z)$ then $C = V(XZ + Y^2)$. 
Take the point $p = [1 : 0 : 0]$ use the projection $\P^2_k \rat \P^1_k$ away from $p$. On the affine $D(X)$ this is the map $U = \Spec{k[y, z]/(z + y^2)} \setminus \{ 0 \} \to \P^1_k$ via $(y, z) \mapsto [y : z]$. Now $U = \Spec{k[y, y^{-1}]} = \Gm^k$ and the map is $\Gm^k \to \P^1_k$ via $t \mapsto [t, t^2]$. This is a rational map $C \rat \P^1_k$ of smooth projective curves so it extends to $C \to \P^1_k$ which is inverse to the previous map. 

\subsection{6.5 G}

Consider $C = \Spec{k[x, y]/(y^2 - x^3 - x^2)}$. Then we construct a rational map $C \rat \A^1_k$ via projecting from $p = (0,0)$. Explicitly, consider $U = D(x)$ and consider, $f : U \to \A^1_k$ via $t \mapsto y/x$. Inversely we define $g : \A^1_k \to C$ generated by the ring map $x \mapsto t^2 - 1$ and $y \mapsto  t(t^2 - 1)$. Note that we have seen this is the normalization $\A^1_k \to C$ of $C$. Then $g \circ f : U \to C$ is $x \mapsto y^2/x^2 - 1 = x$ and $y \mapsto y/x(y^2/x^2 - 1) = y$. Furthermore, $f \circ g : \Gm^k \to \A^1_k$ is $t \mapsto y/x \mapsto t$. Therefore, these are inverse rational maps showing that $C \birat \A^1_k$ is birational. However we cannot extend this rational map to $p$ since $\stalk{C}{p} = \Spec{(k[x, y]/(y^2 - x^2))_{(x, y)}}$ is not a domain and thus not regular.
\bigskip\\
This gives a formula for the rational points of $C$ by $\A^1_L \rat C_L$. Via $t \mapsto (t^2 - 1, t(t^2 - 1))$ which hit every $L$-rational point on $C$. Thus,
\[ C(L) = \{ (t^2 - 1, t(t^2 - 1)) \mid t \in L \} \]
We see that $C$ is a rational curve i.e. $C \birat \P^1_k$. 

\subsection{6.5 H}

Consider the quadric surface,
\[ Q = V(X^2 + Y^2 - Z^2 - W^2) \subset \P^3_k \]
First, we do a change of variables,
\[ X \mapsto \tfrac{1}{2}(X + Z) \quad Z \mapsto \tfrac{1}{2}(X - Z) \quad Y \mapsto \tfrac{1}{2}(Y + W) \quad W \mapsto \tfrac{1}{2}(Y - W) \]
which gives,
\[ Q = V(XZ + YW) \subset \P^3_k \]
Now we project from the point $p = [1 : 0 : 0 : 0]$ on $U = D(X) \setminus \{ p \}$ this gives the map, 
\[ f : \Spec{k[y, z, w]/(z + yw)} \setminus \{ 0 \} \to \P^2_k \]
via sections $y, z, w$. We describe an inverse $\P^2_k \rat Q$ as follows, consider $\P^2_k = \Proj{k[T_0, T_1, T_2]}$ then on $D(T_0 T_2)$ take $\Spec{k[t_0, t_1]} \to \Spec{k[y,z,w]/(z + yw)}$ via $y \mapsto -t_1$ and $z \mapsto -t_1^2/t_0$ and $w \mapsto -t_1 / t_0$ which is the map $(t_0, t_1) \mapsto (-t_1, -t_1^2/t_0, -t_1/t_0)$. This gives,
\[ g : D(T_0 T_2) \to D(XW) \]
and thus $\P^2_k \rat Q$. Furthermore, $g \circ f : D(XW) \to U$ is,
\[ (y, z, w) \mapsto [y : z : w] = [y/w : z/w : 1] \mapsto (-z/w, -z^2/wy, -w/y) = (y, z, w) \] restriction of the identity since $z + wy = 0$. Furhtermore, $f \circ g : D(T_0 T_1 T_2) \to D(T_0 T_1 T_2)$ is,
\[ (t_0, t_1) \mapsto (-t_1, t_1^2/t_0, -t_1/t_0) \mapsto [-t_1 : -t_1^2/t_0 : -t_1/t_0] = [-t_0 t_1 : -t_1^2 : -t_1] = [ t_0 : t_1 : 1] = (t_0, t_1) \]
Thus we have $\P^2_k \birat Q$ via $(t_0, t_1) \mapsto (-t_1, -t_1^2/t_0, -t_1/t_0)$ on $D(T_0 T_1 T_2) \cong D(XZW)$ and thus, clearing denominators and sending $t_1 \mapsto - t_1$, we get,
\[ Q(L) = \{ [t_0 : t_1 t_0 : - t_1^2 : t_1] \mid t_0, t_1 \in L^\times \} \cup \{ [0 : t_0 : t_1 : 0] \mid t_1, t_2 \in L^\times \} \cup \{ [0 : t_0 : 0 : t_1] \mid t_1, t_2 \in L^\times \} \]

\subsection{6.5 I}

Consider the rational map $c : \P^2_k \rat \P^2_k$ given by $[x : y : z] \mapsto [1/x : 1/y : 1/z]$ on $D(xyz)$. Since $\P^2_k$ is smooth, we can extend over smooth codimension one irreducibles i.e. $V(x)$ and $V(y)$ and $V(z)$ such that $c$ is defined on a dense open of each. In particular, on $D(yz)$ we have $[x : y : z] \mapsto [1 : x/y : x/z]$ is equivalent to $c$ restriced to $D(xyz)$ and likewise on $D(xy)$ and $D(xz)$. Thus,
\[ \Dom{f} \supset D(xy) \cup D(yz) \cup D(zx) = \P^2_k \setminus \{ [1 : 0 : 0], [0 : 1 : 0], [0 : 0 : 1] \} \]
The remaining closed set is codimension two so we generically will not be able to extend over it. Indeed, if we try $[x : y : z] \mapsto  [y : x : xy / z]$ on $D(z)$ then at $[0 : 0 : 1]$ this is not defined so it does not work.


\subsection{6.5 J}

Show that there are no dominant rational maps $\P^1_k \to F^n_k$ where $F^n_k = \Proj{k[X, Y, Z]/(X^n + Y^n - Z^n)}$ is the Fermat curve for $n > 2$.


\section{Which Hypersurfaces are Isomorphic to Projective Space?}

First, what is a hypersurface. 

\begin{defn}
A hypersurface $H \subset \P^n_k$ is a codimension one integral closed subscheme i.e. a prime divisor on $\P^n_k$. 
\end{defn}

\begin{theorem}
Every hypersurface $H \subset \P^n_k$ is of the form $V(F)$ for some $F \in \Gamma(\P^n_k, \struct{\P^n_k}(d))$. 
\end{theorem}

\begin{proof}
Since $H$ is a prime divisor and $\P^n_k$ is locally factorial (in particular regular) then $H$ is Cartier so its associated sheaf of ideals $\I \cong \struct{\P^n}(-d)$ is invertible. Then the inclusion map $\struct{\P^n_k}(-d) \embed \struct{\P^n_k}$ is given by some regular section $F \in \Gamma(\P^n_k, \struct{\P^n_k}(d))$ and thus $H = V(F)$. 
\end{proof}

\begin{rmk}
In the case $n = 1$ hypersurfaces are exactly points and since $\P^0_L = \Spec{L}$ then for any finite extension $L / k$ we can easily find $\Spec{L} \to \P^1_k$ so hypersurfaces of $\P^1_k$ are exactly of the form $\P^0_L$. We wonder how this generalizes to $n > 1$. Furthermore, note that we will use the fact that $H$ is effective Cartier and argue, from the exact sequence,
\begin{center}
\begin{tikzcd}
0 \arrow[r] & \struct{\P^n_k}(-d) \arrow[r] & \struct{\P^n_k} \arrow[r] & \iota_* \struct{H} \arrow[r] & 0
\end{tikzcd}
\end{center}
and the associated LES,
\begin{center}
\begin{tikzcd}
H^0(\P^n_k, \struct{\P^n_k}) \arrow[d, equals] \arrow[r] & H^0(H, \struct{H}) \arrow[r] & H^1(\P^n_k, \struct{\P^n_k}(-d)) \arrow[d]
\\
k & & 0
\end{tikzcd}
\end{center}
to argue that for $n > 1$ we get a surjection $k \onto H^0(H, \struct{H})$ showing that we cannot have extensions of $k$. Note that this argument does not hold for $n = 1$ since $H^1(\P^1_k, \struct{\P^1_k}(-d)) \neq 0$ and we can, in fact, have extensions of the base field.
\end{rmk}

\begin{theorem}
Let $H \subset \P^n_k$ be a degree $d$ hypersurface i.e. $H = V(F)$ for $F \in \Gamma(\P^n_k, \struct{\P^n_k}(d))$ and $n > 1$. Then $H \cong \P^{n-1}_L$ for some $L / k$ exactly when $L = k$ and either $d = 1$ or $n = 2$ and $d = 2$. 
\end{theorem}

\begin{proof}
Suppose that $H \cong \P^{n-1}_L$ and consider the inclusion $\iota : H \embed \P^n_k$ and let $X = \P^n_k$. Then for the ample sheaf $\L = \iota^* \struct{X}(1)$ we have $\L \in \Pic{X} \cong \Pic{\P^{n-1}_L}$ so $\L$ correspond to $\struct{\P^{n-1}_k}(k)$ for some $k \in \Z$. Therefore, we must have,
\[ H^p(H, \L^{\otimes \ell}) = H^p(\P^{n-1}_k, \struct{\P^{n-1}_k}(k\ell)) \]
In particular, 
\begin{align*}
\dim_k H^p(H, \L^{\otimes \ell}) & = (\dim_k L) \cdot
\begin{cases}
{ k \ell + n-1 \choose n-1 } & p = 0
\\
0 & p \neq 0, n-1 
\\
{ - k \ell - 1 \choose n - 1} & p = n-1
\end{cases} 
\end{align*}
Furthermore, since $\iota$ is a closed immersion (and thus affine) we have,
\[ H^p(H, \L^{\otimes \ell}) = H^p(X, \iota_* \L^{\otimes \ell}) = H^p(X, \iota_* \struct{H} \otimes_{\struct{X}} \struct{X}(\ell)) \]
using the projection formula. Then, there is an exact sequence of sheaves,
\begin{center}
\begin{tikzcd}
0 \arrow[r] & \I \arrow[d, equals] \arrow[r] & \struct{X} \arrow[r] & \iota_* \struct{H} \arrow[r] & 0
\\
& \struct{X}(-d)
\end{tikzcd}
\end{center}
Twisting by $\struct{X}(\ell)$ gives,
\begin{center}
\begin{tikzcd}
0 \arrow[r] & \struct{X}(\ell - d) \arrow[r] & \struct{X}(\ell) \arrow[r] & \iota_* \struct{H} \otimes_{\struct{X}} \struct{X}(\ell) \arrow[r] & 0
\end{tikzcd}
\end{center}
Now denote $\F = \iota_* \struct{H}$ and $\F(\ell) = \F \otimes_{\struct{X}} \struct{X}(\ell)$ which is the sheaf whose cohomology we wish to compute. Taking the LES of cohomology we get,
\begin{center}
\begin{tikzcd}
0 \arrow[r] & H^0(X, \struct{X}(\ell - d)) \arrow[r] & H^0(X, \struct{X}(\ell)) \arrow[r] & H^0(H, \L^{\otimes \ell}) \arrow[r] & H^1(X, \struct{X}(\ell-d)) \arrow[r, equals] & 0
\end{tikzcd}
\end{center}
since $n > 1$. First, for $\ell = 0$ the first sequence gives $H^0(X, \struct{X}) \onto H^0(H, \struct{H})$ and thus $k \onto L$ which is a $k$-morphism so $L = k$ since it is an extension. Furthermore, from the above short exact sequence, we see that,
\[ h^0(H, \L^{\otimes \ell}) = h^0(X, \struct{X}(\ell)) - h^0(X, \struct{X}(\ell - d)) = { \ell + n \choose n} - {\ell - d + n \choose n} \]
In particular, for $d > 1$ and $\ell = 1$ we have,
\[ h^0(H, \L) = h^0(X, \struct{X}(1)) = n + 1 \]
This must equal (since $L = k$),
\[ h^0(H, \L) = { k + n - 1 \choose n - 1} = { k + n - 1 \choose k} = r(k) \]
which is is zero for $k < 0$ and monotonically increasing for $k > 0$. Note that $r(0) = 1$ and $r(1) = n$ and $r(2) = \tfrac{1}{2}(n+1)n$. Since $r(1) < r(2) < r(3)$ and $r(1) = n$ then either $r(2) = n+1$ or $r(k) \neq n + 1$ for all $k$. However, $\tfrac{1}{2} n(n+1) = n + 1$ exacly when $n = 2$ for $n > 0$ forcing the case $n = 2$ when $d > 1$. In particular for the case $n = 2$ and $d = 2$ we get a plane conic which we know is isomorphic to $\P^1_k$. Also, we need to consider the case $d = 1$ in which $H$ is a hyperplane and it is easy to see that $H \cong \P^{n-1}_k$ via the map $\P^{n-1}_k \embed \P^n_k$ defined by $\struct{\P^{n-1}_k}(1)$ and the $n$ sections perpendicular to $F \in \Gamma(\P^n_k, \struct{\P^n_k}(1))$ which has image $H$ proving the claim.
\bigskip\\
Note further that we get,
\begin{center}
\begin{tikzcd}
H^{n-1}(X, \struct{X}(\ell)) \arrow[r] & H^{n-1}(H, \L^{\otimes \ell}) \arrow[r] & H^n(X, \struct{X}(\ell-d)) \arrow[r] & H^n(X, \struct{X}(\ell)) \arrow[r] & H^n(H, \struct{H})
\end{tikzcd}
\end{center}
and otherwise $H^p(X, \struct{X}(\ell)) = H^{p+1}(X, \struct{X}(\ell - d))$ so $H^p(H, \struct{H}) = 0$ for $p \neq 0, n-1$. Since $\dim{H} = n-1$ we have $H^n(H, \struct{H}) = 0$ and also $H^{n-1}(X, \struct{X}(\ell)) = 0$ so we find,
\begin{center}
\begin{tikzcd}
0 \arrow[r] & H^{n-1}(H, \L^{\otimes \ell}) \arrow[r] & H^n(X, \struct{X}(\ell-d)) \arrow[r] & H^n(X, \struct{X}(\ell)) \arrow[r] & 0
\end{tikzcd}
\end{center}
so we have,
\[ h^{n-1}(H, \L^{\otimes \ell}) = h^{n-1}(X, \struct{X}(\ell - d)) - h^{n-1}(X, \struct{X}(\ell)) = { d - \ell - 1 \choose n } - { - \ell - 1 \choose n} \]
which does have the correct degree in $(-\ell)$ i.e. $n-1$ to be $h^{n-1}(\P^{n-1}_k, \struct{\P^{n-1}_k}(k\ell))$.
\end{proof}




\section{Random Comalg Facts}

\begin{lemma}
Let $(p_1)$ and $(p_2)$ be incommensurable prime ideals. Then $(p_1) \cap (p_2) = (p_1 p_2)$.
\end{lemma}

\begin{proof}
Clearly $(p_1 p_2) \subset (p_1) \cap (p_2)$ so it suffices to show that if $a = p_1 x = p_2 y$ then $a \in (p_1 p_2)$. Since $a \in (p_1)$ and $p_2 \notin (p_1)$ we get $y \in (p_1)$ and likewise $x \in (p_2)$ showing that $a \in (p_1 p_2)$. 
\end{proof}

\section{Open Questions}

\begin{enumerate}
\item Coproducts in the Category of Schemes vs Affine Schemes why are they different but agree with LRS coproducts in the first case which agree with Top coproducts since the Forget : LRS $\to$ Top has a right-adjoint (Raymond chat).
\item Which Hypersurfaces are Rational?  GOOD QUESTION. I think all quadric hypersurfaces are rational even though only the conic $X^2 + Y^2 - Z^2$ is on the nose isomorphic to $\P^1_k$. Can we prove this? Projection from a point?
\item Example of an affine curve which does not embed in $\A^2_k$
\item Does unirational imply finite domination by rational variety in general?
\end{enumerate}

\section{To Do on Thesis}

\begin{enumerate}
\item Example of non-arithmetic curve with no $\Delta_{\nu}$-regular equation, try the think with weakly $\Delta$-nondegenerate by never $\Delta$-nondegenerate.
\item Is the elliptic curve example I gave toric?
\item find example which is toric: use the 
\item Explicit example of curve not on toric surface?
\item Explicit example of curve not on a Hirzburch surface?
\item Example of curve which is toric but never weakly $\Delta$-nondegenerate?
\item 
\end{enumerate}


\section{When is a Sheaf a Pushforward}

THE FOLLOWING IS NOT QUITE CORRECT BUT APPROXIMATELY

\begin{lemma}
Let $\iota : f : Z \embed X$ be a closed embedding and $U = X \setminus Z$. Then if $\F$ is a sheaf of $\struct{X}$-modules then $\F = \iota_* \iota^{-1} \F$ if and only if $\F |_U = 0$. Furthermore, $\F = \iota_* \iota^* \F$ if and only if $\I \cdot \F = 0$ where $\I$ is the ideal sheaf of $Z \embed X$. Furthermore, if $Z$ is reduced then these notions agree. 
\end{lemma}

\begin{proof}

\end{proof}

\begin{rmk}
Given simply topological maps, a sheaf $\F$ is a pushforward of some sheaf on a closed subset exactly when it is zero on the complement. However, if we ask for this sheaf to be the pushforward of a sheaf of $\struct{Z}$-modules then we need the stronger $\I \cdot \F = 0$. 
\end{rmk}

\section{Cayley-Hamilton}

\begin{theorem}
Let $A \in \mathrm{M}_{n}(R)$ be a square matrix over a ring $R$ and $p_A(\lambda) = \det{(\lambda I - A)}$ be its characteristic polynomial. Then $p_A(A) = 0$.
\end{theorem}

\begin{proof}
First, I argue in the case that $R = k$ is a field.
Matrices $A \in \mathrm{M}_n(k)$ correspond to closed points of $X = \A^{n^2}_k = \Spec{k[a_{ij}]}$. Now the fundamental observation is that $p_A(A)$ is a matrix of polynomials in $a_{ij}$ and thus gives a morphism $p : X \to X$ via the ring map $k[a_{ij}] \to k[a_{ij}]$ sending $a_{ij}$ to the $i,j$ entry of the matrix $p_A(A)$ with $A = (a_{ij})$. 
\bigskip\\
Now, if $p_A$ is seperable (i.e. has distinct roots over $\bar{k}$) then $A$ is diagonalizable over $\bar{k}$ (eigenvectors with distinct eignevalues are independent). Then $A = B D B^{-1}$ with $D$ diagonal (these matrices defined over $\bar{k}$) and it is clear that $p_A(B D B^{-1}) = B p_A(D) B^{-1} = 0$ since $p_A(\lambda) = 0$ for each eigenvalue. Furthermore, this case occurs exactly when the discriminant $\Delta(p_A) \neq 0$ which is a polynomial in $a_{ij}$ so $\Delta : X \to \A^1_k$ gives a global function. We have shown that for any closed point $A \in D(\Delta)$, i.e. some matrix over $\bar{k}$ with $\Delta(p_A) = 0$, that  $p_A(A) = 0$ so the map $p : X \to X$ vanishes on the closed points of $D(\Delta)$ which is dense since it is open and nontrivial (any diagonal matrix over $\bar{k}$ with nonrepeated entries satisfies this, I guess I used $\bar{k}$ is infinite here) in an irreducible variety $X$. Thus $p : X \to X$ is the zero map since it vanishes on a dense set (using that $X$ is a variety). In particular $p$ is the zero polynomial in $a_{ij}$.
\bigskip\\
Now, for an arbitrary ring $R$ take a matrix $A \in \mathrm{M}_n(R)$ then $p(a_{ij}) = p_A(A)$ is an integer coefficient polynomial in $a_{ij}$ (meaning the coefficients are in the image $\Z \to R$). However, for each prime $\p \in \Spec{R}$, the above argument shows that $\overline{p_A(A)} \in \kappa(\p)$ is zero since it is the characteristic polynomial applied to the matrix $\overline{A} \in \mathrm{M}_n(\kappa(\p))$ over the field $\kappa(\p)$. Thus $p_A(A) \in \p$ for each $\p \in \Spec{R}$ so $p_A(A) \in \nilrad{R}$ for any $A$ thus the coefficents are in $\nilrad{R}$ (we can see this because reducing $p$ in $\kappa(\p)$ gives the zero polynomial). However, the coefficients are in the image of $\Z \to R$ then $\nilrad{R} \cap \Im{\Z} = \nilrad{\Z / (n)}$ where $n = \ker{(\Z \to R)}$ 
(DAMN DOESNT WORK)
\end{proof}


\section{Quasi-Compactness and Noetherian Spaces}

\begin{defn}
A topological space $X$ is Noetherian if every descending chain of closed sets stabilizes.
\end{defn}

\begin{lemma}
Subspaces of Noetherian subspaces are Noetherian.
\end{lemma}

\begin{proof}
Let $S \subset X$ with $X$ noetherian. Then the closed sets of $S$ are exactly $S \cap Z$ for $Z \subset X$ closed. Thus descending chains of closed sets in $S$ stabilize.
\end{proof}

\begin{defn}
A space is quasi-compact if every open cover has a finite subcover.
\end{defn}

\begin{lemma}
Noetherian spaces are quasi-compact. 
\end{lemma}

\begin{proof}
Let $U_{\alpha}$ be an open cover of $X$ which is Noetherian. Then consider the poset $A$ under inclusion of finite unions of the $U_\alpha$ all of which are open sets of $X$. Since $X$ is Noetherian any ascending chain of opens must stabilize so any chain in $A$ has a maximum. Then by Zorn's lemma $A$ has a maximal element which must be $X$ since the $U_\alpha$ form a cover. Therefore there exists a finite subcover.
\end{proof}

\begin{cor}
Every subset of a noetherian topological space is quasi-compact.
\end{cor}

\begin{defn}
A continuous map $f : X \to Y$ is quasi-compact if for each quasi-compact open $U \subset Y$ then $f^{-1}(U)$ is quasi-compact open.
\end{defn}

\subsection{The Case for Schemes}


\begin{lemma}
Affine schemes are quasi-compact.
\end{lemma}

\begin{proof}
Let $U_i$ be an open cover of $\Spec{A_i}$. Since $D(f)$ for $f \in A$ forms a basis of the topology on $\Spec{A_i}$ we can shrink to the case $U_i = D(f_i)$. Then.
\[ X = \bigcup_{i  \in I} D(f_i) = \bigcup_{i \in I} D(( \{ f_i \mid i \in I \} )) \]
And thus the ideal $I = ( \{ f_i \mid i \in I \} )$ is not contained in any maximal ideal so $I = (1)$. Therefore, there are $f_1, \dots, f_n$ such that $a_1 f_1 + \cdots a_n f_n = 1$ and thus $(f_1, \dots, f_n) = (1)$ which implies that,
\[ X = D((f_1, \dots, f_n)) = \bigcup_{i = 1}^n D(f_i) \]
so $X$ is quasi-compact.
\end{proof}

\begin{defn}
A scheme $X$ is \textit{locally Noetherian} if for every affine open $U$ the ring $\struct{X}(U)$ is Noetherian. $X$ is \textit{Noetherian} if it is quasi-compact and locally-Noetherian. 
\end{defn}

\begin{lemma}
If $(f_1, \dots, f_n) = A$ and $A_{f_i}$ is Noetherian then $A$ is Noetherian.
\end{lemma}

\begin{proof}
For any ideal $I \subset A$ we know $I_{f_i} \subset A_{f_i}$ is finitely generated. Clearing denominators and collecting the finite union of these finite generators gives a map $A^N \to I$ which is surjective when localized $A^N_{f_i} \onto I_{f_i}$. Consider the $A$-module $K = \coker{(A^N \to I)}$ then for any $x \in K$ we have $f_i^{n_i} \cdot x = 0$ for each $i$ but $f_i^{n_i}$ generate the unit ideal (since $D(f_i^{n_i}) = D(f_i)$ which cover $\Spec{A}$) so $x = 0$ to $A^N \onto I$ so $I$ is finitely generated showing that $A$ is Noetherian.
\end{proof}

\begin{lemma}
If $X$ has an open affine cover $U_i = \Spec{A_i}$ with $A_i$ noetherian then $X$ is locally noetherian. Moreover, if the cover can be made finite then $X$ is noetherian. 
\end{lemma}

\begin{proof}
Let $V = \Spec{B} \subset X$ be an affine open, Then $V \cap U_i \subset V$ is open so it may be covered by principal opens $D(f_{ij}) \subset V \cap U_i$ for $f_{ij} \in B$. Since $V$ is quasi-compact we may find a finite subcover. We need to show that $B_{f_{ij}}$ is Noetherian then since $D(f_{ij})$ cover $V$ we use the lemma to conclude that $B$ is Noetherian. However, $D(f_{ij}) \subset V \cap U_i$ can be covered by principal opens (of $U_i = \Spec{A_i}$) $W_{ijk} \subset D(f_{ij}) \subset U_i = \Spec{A_i}$ and each $(A_i)_{f_{ijk}}$ is Noetherian since $A_i$ is, so using the same lemma we find that $B_{f_{ij}}$ is Noetherian. 
\bigskip\\
Now suppose the cover is finite and let $V_j$ be any open cover of $X$. We need to show $X$ is quasi-compact so we must show that $V_i$ has a finite subcover. Consider $U_i \cap V_j$  which is open in the affine $U_i = \Spec{A_i}$ so it may be covered by principal opens $D(f_{ijk}) \subset U_i \cap V_j$. Now,
\[ U_i = \bigcup_{j,k} D(f_{ijk}) \]
but $U_i$ is affine and thus quasi-compact so we may find an finite subcover which only uses finitely many $V_i$ but the cover $U_i$ of $X$ is also finite so only finitely many $V_i$ are needed to cover $X$.  
\end{proof}

\begin{cor}
$X = \Spec{A}$ is Noetherian iff $A$ is a Noetherian ring.
\end{cor}

\begin{proof}
If $X$ is Noetherian then $\struct{X}(X) = A$ is a Noetherian ring ($X$ is affine and thus quasi-compact). Conversely $\Spec{A}$ is a finite Noetherian affine cover so $X$ is Noetherian.
\end{proof}

\begin{rmk}
It is not the case that for a Noetherian scheme we must have $\struct{X}(X)$ a noetherian ring even for varieties. See http://sma.epfl.ch/~ojangure/nichtnoethersch.pdf. 
\end{rmk}

\begin{lemma}
If $A$ is Noetherian then $\Spec{A}$ is a Noetherian topological space.
\end{lemma}

\begin{proof}
Every descending chain of subsets is of the form $V(I_1) \supsetneq V(I_2) \supsetneq V(I_3) \supsetneq \cdots$ but the ideals,
\[ \sqrt{I_1} \subsetneq \sqrt{I_2} \subsetneq \sqrt{I_3} \subsetneq \cdots \]
satbilize since $A$ is Noetherian and thus so does the chain of closed subsets.
\end{proof}

\begin{lemma}
If $X$ is a Noetherian scheme then its underlying topological space is Noetherian.
\end{lemma}

\begin{proof}
Choose a finite covering $U_i = \Spec{A_i}$ by Noetherian rings. Then for any descending chain of closed subsets $Z_1 \supsetneq Z_2 \supsetneq Z_3 \supsetneq \cdots$ we know $Z \cap U_i$ stabilizes at $n_i$ since $\Spec{A_i}$ is a Noetherian space. Thus, $Z$ satibilizes at $\max{n_i}$ which exists since the cover is finite. 
\end{proof}

\begin{rmk}
The converses of the above are false and so is $X$ Noetherian. Let $R$ be a non-Noetherian valuation ring. Then $\Spec{R}$ has two points and thus is Noetherian as a topological space but not as a scheme since $R$ is not a Noetherian ring. 
\end{rmk}

\begin{lemma}
If $X$ is locally Noetherian then any immersion $\iota : Z \embed X$ is quasi-compact.
\end{lemma}

\begin{proof}
Closed immersions are affine and thus quasi-compact so it suffices okay to show that open immersions are quasi-compact. Let $j : U \to X$ be an open immersion. It suffices to check that $j^{-1}(U_i)$ is quasi-compact on an affine open cover $U_i = \Spec{A_i}$ with $A_i$ Noetherian. But $j : j^{-1}(U_i) \to U_i \cap U$ is a homeomorphism and $\Spec{A_i}$ is a Noetherian topological space so every subset is quasi-compact and, in particular, $U_i \cap U$ is quasi-compact so $j^{-1}(U_i)$ is also.
\end{proof}

\begin{rmk}
When $X$ is Noetherian then it is a Noetherian space so any inclusion map $\iota : Z \embed X$ for \textit{any} subset $Z \subset X$ is quasi-compact since every subset is quasi-compact. In particular, every subset of $X$ is retrocompact. 
\end{rmk}

\subsection{Quasi-Compact Morphisms}

\begin{lemma}
A morphism $f : X \to Y$ is quasi-compact iff $Y$ has a cover by affine opens $V_i$ such that $f^{-1}(V_i)$ is quasi-compact.
\end{lemma}

\begin{proof}
Clearly if $f$ is quasi-compact then any affine open cover $V_i$ of $Y$ satisfies $f^{-1}(V_i)$ is quasi-compact since $V_i$ is a quasi-compact open by virtue of being affine open.
\bigskip\\
Now assume that such a cover exists. Let $U \subset Y$ be a quasi-compact open. Then $U$ is covered by finitely may $V_1, \dots, V_n$. Then $U \cap V_i$ is open in $V_i$ which is affine so it is covered by standard opens $W_{ij}$. Since $U$ is quasi-compact then we can choose finitely many $W_{ij}$. Now $f^{-1}(V_i)$ is quasi-compact by assumption so it has a finite cover by affine opens,
\[ f^{-1}(V_i) = \bigcup_{j = 1}^N \tilde{V}_{ij} \]
Then $f : \tilde{V}_{ik} \to V_i$ is a morphism of affine schemes so $f^{-1}(W_{ij}) \cap \tilde{V}_{ik}$ is a principal affine. Therefore,
\[ f^{-1}(U) = \bigcup_{i = 1}^n f^{-1}(V_i \cap U) = \bigcup_{i,j} f^{-1}(W_{ij}) = \bigcup_{i,j,k} f^{-1}(W_{ij}) \cap \tilde{V}_{ik} \]
is a finite union of principal affines each of which is quasi-compact so $f^{-1}(U)$ is quasi-compact. 
\end{proof}

\begin{proposition}
$X$ is quasi-compact iff any morphism $X \to T$ for some affine scheme $T$ is quasi-compact.
\end{proposition}

\begin{proof}
If $X$ is quasi-compact then $f : X \to T$ is quasi-compact since $T$ is an affine open cover of itself and $f^{-1}(T)$ is quasi-compact. Conversely, if $f : X \to T$ is quasi-compact with $T$ affine then $T$ is quasi-compact open in $T$ so $X = f^{-1}(T)$ is quasi-compact.
\end{proof}

\begin{lemma}
The base change of a quasi-compact morphism is quasi-compact.
\end{lemma}

\begin{proof}
(DO THIS)
\end{proof}

\section{Affine Morphisms}

\begin{defn}
A morphism $f : X \to Y$ is \textit{affine} if the preimage of every affine open is affine.
\end{defn}

\begin{lemma}
Every morphism of affine schemes is affine and thus quasi-compact.
\end{lemma}

\begin{proof}
Let $X = \Spec{A}$ and $Y = \Spec{B}$ and $f : X \to Y$ be a morphism of affine schemes given by a ring map $\varphi : B \to A$. Then, any affine open $\Spec{C} = V \subset Y$ can be covered by principal opens $D(f_i)$ for $f_i \in B$. Note that under $\psi : B \to C$ we see that $D(f_i) = D(\psi(f_i))$ since $D(f_i) \subset \Spec{C}$. Since $D(\psi(f_i))$ cover $\Spec{C}$ then $\psi(f_i) \in C$ generate the unit ideal. Then we have $f^{-1}(D(f_i)) = D(\varphi(f_i))$ which is affine and $\varphi(f_i)$ generate the unit ideal of $\Gamma(f^{-1}(V), \struct{X})$ so $f^{-1}$ is affine.
\end{proof}

\begin{rmk}
An alternative proof goes as follows. Consider the pullback diagram,
\begin{center}
\begin{tikzcd}
f^{-1}(U) \arrow[d, hook] \arrow[r] & U \arrow[d, hook]
\\
X \arrow[r] & Y
\end{tikzcd}
\end{center}
then open immersions are stable under base change so $f^{-1}(U) = U \times_Y X = \Spec{C \otimes_B A}$ if affine.
\end{rmk}

\begin{rmk}
In fact, by Tag 01S8, a morphism $f : X \to S$ is affine iff $X$ is relatively affine over $S$ meaning $X = \rSpec{S}{\mathcal{A}}$ for some quasi-coherent $\struct{S}$-algebra $\mathcal{A}$. 
\end{rmk}

\begin{lemma}
Let $f : X \to Y$ be a morphism and $W_i$ an affine open cover of $Y$ such that $f^{-1}(W_i)$ is affine. Then $f$ is affine.
\end{lemma}

\begin{proof}
Let $\Spec{A} = V \subset Y$ be affine open. Then $V_i = V \cap W_i$ is open in the affine open $V = \Spec{A}$ so it can be covered by principal opens $D(f_{ij}) \subset V \cap W_i$ for $f_{ij} \in A$. Since $f : f^{-1}(W_i) \to W_i$ is a morphism of affine schemes, the preimage of the affine open $D(f_{ij}) \subset V \cap W_i$ is affine $f^{-1}(D(f_{ij}))$ (note that $D(f_{ij}) \subset V \cap W_i$ is not necessarily a prinicpal affine open of $W_i$). But since $D(f_{ij})$ cover $\Spec{A}$ the $f_{ij} \in A$ generate the unit ideal and thus $f^\#(f_{ij}) \in \Gamma(f^{-1}(V), \struct{X})$ generate the unit ideal and $(f^{-1}(V))_{f_{ij}} = f^{-1}(D(f_{ij}))$ is affine so $f^{-1}(V)$ is affine.
\end{proof}

\begin{lemma}
The base change of an affine morphism is affine. 
\end{lemma}

\begin{proof}
(DO THIS)
\end{proof}


\begin{lemma}
Affine morphisms are quasi-compact.
\end{lemma}

\begin{proof}
If $f : X \to Y$ is affine then any affine open cover $V_i$ of $Y$ gives $f^{-1}(V_i)$ is affine and thus quasi-compact so $f$ is quasi-compact. 
\end{proof}



\section{Separatedness}

\begin{defn}
A morphism $f : X \to Y$ with diagonal $\Delta_{X/Y} : X \to X \times_Y X$ is,
\begin{enumerate}
\item \textit{separated} if the diagonal $\Delta_{X/Y}$ is a closed immersion
\item \textit{affine-separated} if the diagonal $\Delta_{X/Y}$ is affine
\item \textit{quasi-separated} if the diagonal $\Delta_{X/Y}$ is quasi-compact
\end{enumerate}
\end{defn}

\begin{lemma}
Any morphism of affine schemes is separated. Furthermore, affine morphisms are separated.
\end{lemma}

\begin{proof}
For a map $\Spec{A} \to \Spec{B}$ the diagonal is $\Spec{A} \to \Spec{A \otimes_B A}$ given by $A \otimes_B A \to A$ via $a_1 \otimes a_2 \mapsto a_1 a_2$ which is surjective so the diagonal is a closed immersion. The second fact is Tag 01S7.
\end{proof}

\begin{lemma}
The composition of (quasi/affine)-separated morphisms are (quasi/affine)-separated. 
\end{lemma}

\begin{proof}
(DO THIS)
\end{proof}

\begin{lemma}
For any morphism $f : X \to Y$ the diagonal $\Delta_{X / Y} : X \to X \times_Y X$ is an immersion.
\end{lemma}

\begin{proof}
Let $V_i$ be an affine cover of $Y$ then choose an affine open cover $U_{ij}$ of $X$ with $f(U_{ij}) \subset V_i$. Then the diagonal of the affine map $U_{ij} \to V_j$ is $U_{ij} \to U_{ij} \times_{V_i} U_{ij}$ which is a closed immersion since it corresponds to $A_{ij} \otimes_{B_i} A_{ij} \to A_{ij}$ via $a_1 \otimes a_2 \mapsto a_1 a_2$ is surjective. Therefore $f : X \to Y$ is locally on $X$ a closed immersion and thus an immersion. 
\end{proof}

\begin{rmk}
Therefore, to show that $f : X \to Y$ is separated, it suffices to show that the diagonal is closed (here equivalently meaning that the map or its image is closed). 
\end{rmk}

\begin{lemma}
If $X$ is Noetherian then every morphism $f : X \to S$ is quasi-compact and quasi-seperated. 
\end{lemma}

\begin{proof}
Every subset of $X$ is quasi-compact since $X$ is (topologically) Noetherian. Then apply the first part to the diagonal $\Delta_{X/S} : X \to X \times_S X$ which is then quasi-compact and thus $f : X \to S$ is quasi-separated.
\end{proof}

\begin{lemma}
Let $f : X \to S$ be affine-separated/quasi-separated with $S = \Spec{A}$ affine. Then for any two affine opens $U, V \subset X$ the intersection $U \cap V$ is affine/quasi-compact. 
\end{lemma}

\begin{proof}
Consider the pullback diagram,
\begin{center}
\begin{tikzcd}
U \cap V \arrow[d, hook] \arrow[r] & U \times_S V \arrow[d, hook]
\\
X \arrow[r, "\Delta_{X/S}"] & X \times_S X 
\end{tikzcd}
\end{center}
where $U \cap V = \Delta_{X/S}(U \times_S V)$ using the basechange of an open immersion is an open immersion. Then since $S$ is affine, $U \times_S V$ is affine and thus quasi-compact open of $X \times_S X$. Then if $f$ is affine-separated then $\Delta_{X/S}$ is affine so $U \cap V = \Delta_{X/S}(U \times_S V)$ is affine. If $f$ is quasi-separated then $\Delta_{X/S}$ is quasi-compact so $U \cap V = \Delta_{X/S}(U \times_S V)$ is quasi-compact.
\end{proof}

\begin{rmk}
In the separated case, we see that $U \cap V$ is affine and $\struct{X}(U) \otimes_{\struct{S}(S)} \struct{X}(V) \to \struct{X}(U \cap V)$ is surjective.
\end{rmk}

\begin{rmk}
Tag 01KO gives a generalization of this lemma. For the separated case see Tag 01KP.
\end{rmk}

\begin{lemma}
Let $f : X \to Y$ be quasi-compact and quasi-separated and $\F$ be a quasi-coherent $\struct{X}$-module then $f_* \F$ is a quasi-coherent $\struct{Y}$-module.
\end{lemma}

\begin{proof}
Sinsce this is local on $Y$ we can restrict to the case that $Y$ is affine. Then   $X = f^{-1}(Y)$ is quasi-compact (when $Y$ is not affine $f^{-1}(V)$ will be quasi-compact) so take a finite affine open cover $U_i$ and since $f : X \to Y$ is quasi-seperated over an affine then by the above lemma $U_i \cap U_j$ is quasi-compact so it has a finite affine open cover $U_{ijk}$. Then, by the sheaf property, there is an exact sequence of sheaves on $Y$
\begin{center}
\begin{tikzcd}
0 \arrow[r] & f_* \F \arrow[r] & \bigoplus\limits_{i} f_* (\F|_{U_i}) \arrow[r] & \bigoplus\limits_{ijk} f_* (\F|_{U_{ijk}}) 
\end{tikzcd}
\end{center}
which works because these are finite sums. However, $f : U_{ijk} \to Y$ is a morphism of affine schemes and since $\F$ is quasi-coherent we have $\F |_{U_{ijk}} = \wt{M_{ijk}}$ so $f_* (\F|_{U_{ijk}}) = \wt{M_{ijk}}$ as an $\struct{Y}(Y)$-module. Thus, $f_* \F$ is a kernel of quasi-coherent $\struct{Y}$-modules and thus is quasi-coherent. 
\end{proof}

\begin{rmk}
If $X$ is Noetherian then $f : X \to Y$ is automatically quasi-compact and quasi-separated so there is no issue in the above lemma.
\end{rmk}


\section{Sets Cut Out By Some Function}

\newcommand{\R}{\mathbb{R}}

\begin{theorem}
Every closed subset $E \subset \R^n$ is the vanishing of some smooth function.
\end{theorem}

\begin{proof}
Since $\R^n$ is a metric space, it is hereditarily paracompact so the complement $E^C \subset \R^n$ is paracomapct. Since $\R^n$ is seperable, $E^C$ is covered by countably many balls $B_{r_i}(a_i)$ for $a_i \in E^C$ since it is open so, by paracompactness, we may shrink the radii such that this cover is locally finite. Choose a smooth bump function, 
\[ g : [0, \infty) \to [0, \infty) \]
such that $g([0, 1)) > 0$ and $g([1, \infty)) = 0$ e.g. 
\[ g(x) = 
\begin{cases}
\exp{\left( - \frac{1}{1 - x} \right)} & x < 1
\\
0 & x \ge 1
\end{cases} \]
Then consider,
\[ f(x) = \sum_{x \in X} g(|x - a_i|/r_i) \]
Since $g(|x - a_i|/r_i) = 0$ for $x \notin B_{r_i}(a_i)$ and the cover is locally finite, this is a finite sum so $f$ is well-defined and smooth. Furthermore, 
\[ f(x) = 0 \iff x \notin \forall i \in I : x \notin B_{r_i}(a_i) \iff x \notin E^C \iff x \in E \]
\end{proof}

\begin{rmk}
This esaily generalizes to show that any closed subset $Z \subset X$ of a smooth manifold is cut out by closed sets.
\end{rmk}
\noindent
Our next question is what does the vanishing of analytic or holomorphic functions look like. We have one result.

\begin{prop}
A nontrivial vanishing set of analytic functions in $\R^n$ (or holomophic functions in $\mathbb{C}^n$) has positive codimension. Explicitly, it does not contain any nonempty open. 
\end{prop}

\begin{proof}
This is clear because analytic and holomorphic functions which vanish on a nonempty open vanish everywhere. 
\end{proof}

\section{Cousins Problems}

Here we let $X$ be a complex manifold and $\struct{X}$ be its sheaf of holomorphic functions and $\K_X$ be its sheaf of meromorphic functions. The Cousins problems are the following questions given a cover $U_i$ and a meromorphic function $f_i \in \Gamma(U_i, \K_X)$ on $U_i$. 

\begin{defn}
The Cousins problems ask the following. 
\begin{enumerate}
\item (First or additive Cousin Problem) if $(f_i - f_j)|_{U_i \cap U_j}$ is holomorphic for each pair $i,j$ then does there exist a global meromorphic function $f \in \Gamma(X, \K_X)$ such that $f|_{U_i} - f_i$ is holomorphic?
\item (Second or multiplicative Cousin Problem) if $(f_i / f_j)|_{U_i \cap U_j}$ is non-vanishing holomoprhic for each pair $i,j$ then does there exist a global meromorphic function $f \in \Gamma(X, \K_X)$ such that $f|_{U_i} / f_i$ is holomorphic and non-vanishing?
\end{enumerate}
\end{defn}
\noindent\\
Notice that set of pairs $\{ (U_i, f_i) \}$ in the first Cousin problem defines a global section of the sheaf $\K_X / \struct{X}$ exactly because $(f_i - f_j)|_{U_i \cap U_j} \in \struct{X}(U_i \cap U_j)$ is holomorphic. Likewsie, the set of pairs $\{ (U_i, f_i) \}$ in the second Cousin problem defined a global section of the sheaf $\K_X^\times / \struct{X}^\times$ exactly because $(f_i / f_j) |_{U_i \cap U_j} \in \struct{X}^\times(U_i \cap U_j)$ is holomorphic and nonvanishing. Therefore, we can restate the Cousins problems as follows.

\begin{defn}
The Cousins problems ask the following. 
\begin{enumerate}
\item (First Cousin Problem) is the map $H^0(X, \K_X) \to H^0(X, \K_X / \struct{X})$ surjective?
\item (Second Cousin Problem) is the map $H^0(X, \K_X^\times) \to H^0(X, \K_X^\times / \struct{X}^\times)$ surjective?
\end{enumerate}
\end{defn}
\noindent\\
Now we can solve these problems using the following two exact sequences of sheaves,
\begin{center}
\begin{tikzcd}
0 \arrow[r] & \struct{X} \arrow[r] & \K_X \arrow[r] & \K_X / \struct{X} \arrow[r] & 0
\\
0 \arrow[r] & \struct{X}^\times \arrow[r] & \K_X^\times \arrow[r] & \K_X^\times / \struct{X}^\times \arrow[r] & 0
\end{tikzcd}
\end{center}
and we can relate the sheaf cohomology needed in the two problems via the exponential exact sequence,
\begin{center}
\begin{tikzcd}
0 \arrow[r] & \underline{\Z} \arrow[r, "2 \pi i"] & \struct{X} \arrow[r, "\exp"] & \struct{X}^\times \arrow[r] & 0
\end{tikzcd}
\end{center}

\begin{theorem}
The first cousin problem is solvable when $H^1(X, \struct{X}) = 0$. 
\end{theorem}

\begin{proof}
The first exact sequence gives a cohomology exact sequence,
\begin{center}
\begin{tikzcd}
0 \arrow[r] & H^0(X, \struct{X}) \arrow[r] & H^0(X, \K_X) \arrow[r] & H^0(X, \K_X/\struct{X}) \arrow[r] & H^1(X, \struct{X}) \arrow[r] & H^1(X, \K_X)
\end{tikzcd}
\end{center}
Clearly, if $H^1(X, \struct{X}) = 0$ then, by exactness, $H^0(X, \K_X) \to H^0(X, \K_X / \struct{X})$ is surjective. 
\end{proof}

\begin{rmk}
By Cartan's theorem B, we know $H^1(X, \struct{X}) = 0$ for any Stein manifold. So the first Cousin problem is always solvable for Stein manifolds.
\end{rmk}

\begin{theorem}
The second cousin problem is solvable when $H^1(X, \struct{X}^\times) = 0$ or when $H^1(X, \struct{X}) = H^2(X, \struct{X}) = 0$ and $H^2(X; \Z) = 0$.
\end{theorem}

\begin{proof}
The second exact sequence gives a cohomology exact sequence,
\begin{center}
\begin{tikzcd}
0 \arrow[r] & H^0(X, \struct{X}^\times) \arrow[r] & H^0(X, \K_X^\times) \arrow[r] & H^0(X, \K_X^\times/\struct{X}^\times) \arrow[r] & H^1(X, \struct{X}^\times) \arrow[r] & H^1(X, \K_X^\times)
\end{tikzcd}
\end{center}
Clearly, if $H^1(X, \struct{X}^\times) = 0$ then, by exactness, $H^0(X, \K_X) \to H^0(X, \K_X / \struct{X})$ is surjective. Now consider the cohomology of the exponential sequence,
\begin{center}
\begin{tikzcd}
H^1(X; \Z) \arrow[r] & H^1(X, \struct{X}) \arrow[r] & H^1(X, \struct{X}^\times) \arrow[r] & H^2(X; \Z) \arrow[r] & H^2(X, \struct{X}) 
\end{tikzcd}
\end{center}
Then if $H^1(X, \struct{X}) = 0$ and $H^2(X, \struct{X}) = 0$ we get an isomorphism (the first Chern class) $H^1(X, \struct{X}^\times) = H^2(X; \Z)$ so if $H^2(X; \Z) = 0$ then $H^1(X, \struct{X}^\times) = 0$ giving the surjection. 
\end{proof}

\begin{rmk}
For Stein manifolds we always have $H^p(X, \struct{X}) = 0$ for $p > 0$ by Cartan's theorem B. Therefore, the second cousin problem is solvable for Stein manifolds when $H^2(X; \Z) = 0$. 
\end{rmk}

\section{The Topology of Schemes}

Here I want to ask what the topology of schemes ``looks like'' from the perspective of algebraic topology. The importance of the analytification functor $X \mapsto X^\an$ is that it alows us to compute the ``correct'' topological invariants to complex varieties. However, what happens if we try to compute algebraic topology on the Zariski topology?

\begin{lemma}
Suppose $X$ is a topological space with a dense point $\xi \in X$. Then $X$ is contractible.
\end{lemma}

\begin{proof}
Consider the homotopy $h : X \times I \to X$ defined by,
\[ h(x, t) = 
\begin{cases}
x & t = 0
\\
\eta & t > 0
\end{cases} \]
This is continuous because no nontrivial closed set $Z \subset X$ contains $\xi$ so $h^{-1}(Z) = Z \times \{ 0 \}$ which is closed. Furthermore $h^{-1}(X) = X \times I$ so $h$ is continuous. 
\end{proof}

\begin{rmk}
In particular, we see that every irreducible scheme is contractible. 
\end{rmk}
\noindent\\
However, there are example of varieties which have nontrivial homotopy type. 

\begin{example}
https://math.stackexchange.com/questions/2701914/connected-non-contractible-schemes
\end{example}

\section{Ample Invertible Sheaves}

DO THIS!!!!!

\subsection{ of Ample Divisor is Affine}

\begin{rmk}
Recall that $X_s = \{ x \in X \mid s_x \notin \m_x \L_x \}$ is open since under a local trivialization this is $\tilde{s}_x \notin \m_x$ and this happens exactly when $s$ is locally invertible an open condition.
\end{rmk}

\begin{rmk}
The following is Grothendieck's definition of Ampleness.
\end{rmk}

\begin{defn}
Let $X$ be quasi-compact. Then an invertible $\struct{X}$-module $\L$ is ample if for each $x \in X$ there exists $n \ge 1$ and $s \in \Gamma(X, \L^{\otimes n})$ such that $x \in X_s$ and $X_s$ is affine.
\end{defn}

\begin{theorem}
Let $\L$ be ample on quasi-compact $X$ and $s \in \Gamma(X, \L)$ then $X_s$ is affine.
\end{theorem}

\begin{proof}
We know that $s : \struct{X_s} \to \L|_{X_s}$ is an isomorphism. For each $x_i \in X_s$ we can choose $n_i \ge 1$ and $s_i \in \Gamma(X, \L^{\otimes n_i})$ such that $X_{s_i}$ is affine and $x_i \in X_{s_i}$.  
\end{proof}

\begin{rmk}
Since $\L$ is smple iff $\L^{\otimes n}$ is ample for any $n \ge 1$ we see that $X_s$ is affine for any $s \in \Gamma(X, \L^{\otimes n})$. 
\end{rmk}

\section{test}

\begin{align*}
\struct{X}
\\
A_{_{f_1}} = A_{f_1}
\\
\Hom{\struct{X}}{\F}{\G}
\\
\shHom{\struct{X}}{\F}{\G}
\\
\shExt{i}{\struct{X}}{\F}{\G}
\\
\shTor{\struct{X}}{i}{\F}{\G}
\\
\shEnd{\struct{X}}{\F}
\\
\shDer{\F}{\G}
\end{align*}

\section{Backup}

\section{Rational Maps}

\begin{definition}
Let $X, Y$ be schemes over $S$. Consider the set of pairs of opens and $S$-morphisms,
\[ \{ (U, f_U) \mid U \subset X \text{ dense open } f_U : U \to Y \} \]
And an equivalence relation $(U, f_U) \sim (V, f_V)$ if for some dense (in $X$) open $W \subset U \cap V$ we have $(f_U)_{W} = (f_V)_{W}$. A rational $S$-morphism $f : X \rat Y$ is an equivalence class of pairs $(U, f_U)$. 
\bigskip\\
The domain of the rational function $f : X \rat Y$ is,
\[ \Dom{f} = \bigcup \{ U \mid (U, f_U) \in f \} \]
The set of rational maps $X \rat Y$ is exactly,
\[ \mathrm{Rat}(X, Y) = \varinjlim_{U \in \mathcal{D}(X)} \Hom{\Sch}{U}{Y} \]
where $\mathcal{D}(X)$ is the set of dense open subset $U \subset X$.
\end{definition}

\begin{remark}
This is an equivalence relation since if $(U_1, f_1) \sim (U_2, f_2) \sim (U_3, f_3)$ then there exist dense opens $V \subset U_1 \cap U_2$ and $W \subset U_2 \cap U_3$. Then $V \cap W \subset U_1 \cap U_2 \cap U_3 \subset U_1 \cap U_3$ and $V \cap W$ is a dense open. Futhermore,
\[ (f_1|_{V})_{V \cap W} = (f_2|_{V})_{V \cap W} = (f_2|_{W})_{V \cap W} = (f_3|_{W})_{V \cap W} \]
\end{remark}

\begin{lemma}
If $U, V \subset X$ are dense opens then $U \cap V$ is a dense open.
\end{lemma}

\begin{proof}
For any nonempty open $W \subset X$ we know $W \cap U$ is non empty open since $U$ is dense and thus $W \cap U \cap V$ is nonempty since $V$ is dense. Thus $U \cap V$ is dense. 
\end{proof}

\subsection{Glueing Rational Maps}

\subsection{The Locus on Which Morphisms Agree}

\begin{lemma}
Let $(R, \m, \kappa)$ be a local ring. Then for schemes $X$ there is a natural bijection,
\[ \Hom{\Sch}{\Spec{R}}{X} \cong \{ x \in X \text{ and local map } \stalk{X}{x} \to R \} \]
\end{lemma}

\begin{proof}
Given $\Spec{R} \to X$ we automatically get $\m \mapsto x$ and $\stalk{X}{x} \to R_\m = R$. 
Now, note that taking any affine open neighbrohood $x \in \Spec{A} \subset X$ and then $A \to A_\p = \stalk{X}{x}$ to give $\Spec{\stalk{X}{x}} \to \Spec{A} \to X$. Clearly, this map sends $\m_x \mapsto x$ and at $\m_x$ has stalk map $\id : \stalk{X}{x} \to \stalk{X}{x}$ since it is the localization at $\p$ of $A \to A_\p$. 
\bigskip\\
Thus we get an inverse as follows. Given a point $x \in X$ and a local map $\phi : \stalk{X}{x} \to R$ then take,
\[ \Spec{R} \to \Spec{\stalk{X}{x}} \to X \]
This is inverse since $\m \mapsto \m_x$ (because $\stalk{X}{x} \to \m_x$ is local) and $\m_x \mapsto x$ and the stalk at $\m$ gives $\stalk{X}{x} \xrightarrow{\id} \stalk{X}{x} \xrightarrow{\phi} R$. 
\bigskip\\
Finally, I claim that any $f : \Spec{R} \to X$ factors through $\Spec{R} \to \Spec{\stalk{X}{x}} \to X$ and thus is reconstructed from $x \in X$ and $\stalk{X}{x} \to R$. Choose an affine open neighbrohood $x \in \Spec{A} \subset X$ then consider $f^{-1}(\Spec{A})$ which is open in $\Spec{R}$ and contains the unique closed point $\m \in \Spec{R}$ so there is some $f \in R$ s.t. $\m \in D(f) \subset f^{-1}(\Spec{A})$ so $f \notin \m$ so $f \in R^\times$ and thus $D(f) = \Spec{R}$. Therefore, we get a map $\Spec{R} \to \Spec{A}$ and thus $\phi : A \to R$ where $\phi^{-1}(\m) = \p = x$ so $A \setminus \p$ is mapped inside $R^\times$ so this map factors through $A \to A_\p \to R$ giving the desired factorization $\Spec{R} \to \Spec{\stalk{X}{x}} \to \Spec{A} \to X$.  
\end{proof}

\begin{definition}
The locus $Z$ on which two maps $f, g : X \to Y$ over $S$ agree is given as the pullback,
\begin{center}
\begin{tikzcd}[row sep = large, column sep = large]
Z \arrow[r] \arrow[dr, phantom, "\usebox\pullback" , very near start, color=black] \arrow[d] & Y \arrow[d, "\Delta_Y"]
\\
X \arrow[r, "F"] & Y \times_S Y
\end{tikzcd}
\end{center}
with $F = (f, g)$. Furthermore $Z \to X$ is an immersion. 
\end{definition}

\begin{lemma}
Topologically, the locus on which $S$-morphisms $f, g : X \to Y$ agree is,
\[ Z = \{ x \in X \mid f(x) = g(x) \text{ and } f_x = g_x : \kappa(f(x)) \to \kappa(x) \} \]
\end{lemma}

\begin{proof}
On some $S$-subscheme $G \subset X$, the maps $f|_G = g|_G$ agree iff there exists $G \to Y$ such that,
\begin{center}
\begin{tikzcd}
G \arrow[d, hook] \arrow[r, dashed] & Y \arrow[d, "\Delta"]
\\
X \arrow[r, "F"] & Y \times_S Y
\end{tikzcd}
\end{center}
commutes. In particular, for any point $x \in X$ consider $\iota : \Spec{\kappa(x)} \to X$ then $f \circ \iota = g \circ \iota$ iff $f(x) = g(x)$ and $f_x = g_x : \kappa(f(x)) \to \kappa(x)$. Consider a point $z \in Z$ and $\Spec{\kappa(z)} \to Z$, such a point is equivalent to giving a diagram,
\begin{center}
\begin{tikzcd}[row sep = large, column sep = large]
\Spec{\kappa(z)} \arrow[rd, dashed] \arrow[rrd, bend left] \arrow[rdd, bend right]
\\
& Z \arrow[r] \arrow[dr, phantom, "\usebox\pullback" , very near start, color=black] \arrow[d] & Y \arrow[d, "\Delta_Y"]
\\
& X \arrow[r, "F"] & Y \times_S Y
\end{tikzcd}
\end{center}
However, $\iota : Z \to X$ is an immersion so $f_x : \kappa(f(x)) \xrightarrow{\sim} \kappa(x)$ is an isomorphism. Therefore, points $\Spec{\kappa(z)} \to Z$ of $z$, are exactly points of $X$ for which a lift $\Spec{\kappa(x)} \to Y$ exists i.e. points such that $f$ and $g$ agree in the required way.
\end{proof}

\begin{lemma}
If $f : X \to Y$ is an immersion then $f_x : \stalk{Y}{f(x)} \onto \stalk{X}{x}$ is injective for each $x \in X$ and $f_x : \kappa(f(x)) \xrightarrow{\sim} \kappa(x)$ is an isomorphism.
\end{lemma}

\begin{proof}
First note that $f^{\#} : \struct{Y} \to f_* \struct{X}$ is surjective by definition (surjective for the closed immersion factor and isomorphism for the open immersion factor). Thus we get an injection $f_x : \stalk{Y}{y} \to (f_* \struct{X})_{f(x)}$. Furthermore, topologically, $f : X \to Y$ is a homeomorphism onto its image so for any open $U \subset X$ there exists an open $V \subset Y$ s.t. $U = f^{-1}(V)$ showing that,
\[ (f_* \struct{X})_{f(x)} = \varinjlim_{f(x) \in V} \struct{X}(f^{-1}(V)) = \varinjlim_{x \in U} \struct{X}(U) = \stalk{X}{x} \]
Finally, since $f_x : \stalk{Y}{f(x)} \onto \stalk{X}{x}$ is local we get $f_x : \kappa(f(x)) \onto \kappa(x)$ which is a surjection of fields and thus an isomorphism. 
\end{proof}

\begin{lemma}
If $Y \to S$ is separated then the locus on which $f,g : X \to Y$ over $S$ agree is closed.
\end{lemma}

\begin{proof}
Since $X \to S$ is separated, $\Delta_{Y/S} : Y \to Y \times_S Y$ is a closed immersion. So $Z \to X$ is the base change of a closed immersion and thus a closed immersion. 
\end{proof}

\begin{lemma}
Let $X$ be a reduced and $Y$ be a separated scheme over $S$ and $f ,g : X \to Y$ be morphims over $S$. If $f \circ j = g \circ j$ agree on a dense subscheme $j : G \embed X$ then $f = g$.
\end{lemma}

\begin{proof}
Consider $F = (f, g) : X \to Y \times_S Y$. Since $\Delta : Y \to Y \times_S Y$ is a closed immersion (by separatedness). Then $F^{-1}(\Delta)$ is the locus on which $f = g$ which is closed because $\Delta : Y \to Y \times_S Y$ is a closed immersion. Since $f|_G = g|_G$ we get a diagram,
\begin{center}
\begin{tikzcd}[row sep = large, column sep = large]
G \arrow[rd, dashed] \arrow[rrd, bend left] \arrow[rdd, bend right, hook]
\\
& Z \arrow[r, "\tilde{F}"] \arrow[dr, phantom, "\usebox\pullback" , very near start, color=black] \arrow[d, hook, "\iota"'] & Y \arrow[d, "\Delta_Y"]
\\
& X \arrow[r, "F"] & Y \times_S Y
\end{tikzcd}
\end{center}
Since $\iota : Z \embed X$ is a closed immersion with dense image, $Z \embed X$ is surjective. By the following, $\iota : Z \to X$ is an isomorphism. Thus, $F = F \circ \iota \circ \iota^{-1} = \Delta_Y \circ \tilde{F} \circ \iota^{-1}$. By the univeral property of maps $X \to Y \times_S Y$ this implies that $f = g = \tilde{F} \circ \iota^{-1}$.
\end{proof}

\newcommand{\Nil}{\mathcal{N}}

\begin{lemma}
Let $X$ be a scheme and consider an exact sequence of quasi-coherent $\struct{X}$-modules,
\begin{center}
\begin{tikzcd}
0 \arrow[r] & \I \arrow[r] & \struct{X} \arrow[r] & \mathcal{A} \arrow[r] & 0
\end{tikzcd}
\end{center}
and $\A$ is a sheaf of $\struct{X}$-algebra. 
Suppose that $\F_x \neq 0$ for each $x \in X$. Then $\I \embed \Nil$ where $\Nil$ is the sheaf of nilpotents.
\end{lemma}

\begin{proof}
Take an affine open $U = \Spec{R} \subset X$ such that $\mathcal{A} |_{U} = \wt{A}$. Then we have an surjection of rings $R \onto A$ giving $R/I = A$ for $I = \ker{(R \to A)}$. Now, for each $\p \in \Spec{R}$ we know $R_\p = \stalk{X}{\p} \neq 0$. However, if $\p \not\supset I$ then $(R/I)_\p = A_\p = 0$ so we must have $\p \supset I$ for all $\p \in \Spec{R}$ i.e. $I \subset \nilrad{R}$. Therefore, $\I |_U \embed \Nil|_U$ for any affine open $U \subset X$ showing that $\I$ is comprised of nilpotents. 
\end{proof}

\begin{corollary}
If $X$ is reduced and $\iota : Z \embed X$ is a surjective closed immersion then $\iota : Z \xrightarrow{\sim} X$ is an isomorphism. 
\end{corollary}

\begin{proof}
Since $\iota: Z \embed X$ is a homeomorphism onto its image $X$ it suffices to show that the map of sheaves $\iota^\# : \struct{X} \to \iota_* \struct{Z}$ is an isomorphism. Since $\iota : Z \to X$ is a closed immersion $\iota^\# : \struct{X} \onto \iota_* \struct{Z}$ is a surjection and $\struct{Z}$ is a quasi-coherent sheaf of $\struct{X}$-algebras giving an exact sequence,
\begin{center}
\begin{tikzcd}
0 \arrow[r] & \I \arrow[r] & \struct{X} \arrow[r] & \iota_* \struct{Z} \arrow[r] & 0
\end{tikzcd}
\end{center} Furthermore, 
\[ \Supp{\struct{X}}{\iota_* \struct{Z}} = \Im{\iota} = X \]
since $(\iota_* \struct{Z})_x = \stalk{Z}{x}$ when $x \in \Im{\iota}$ (and zero elsewhere). by the above, $\I \embed \Nil = 0$ since $X$ is reduced to $\iota^\# : \struct{X} \to \iota_* \struct{Z}$ is an isomorphism.  
\end{proof}

\begin{lemma}
A rational $S$-map $f : X \rat Y$ with $X$ reduced and $Y \to S$ separated is equivalent to a morphism $f : \Dom{f} \to Y$. 
\end{lemma}

\begin{proof}
For any $(U, f_U)$ and $(V, f_V)$ representing $f$ there must be a dense (in $X$) open $W \subset U \cap V$ on which $f_U|_W = f_V|_W$ and thus $f_U |_{U \cap V} = f_V |_{U \cap V}$ since $f_U, f_V : U \cap V \to Y$ are morphisms from reduced to irreducible schemes. Now $\Dom{f}$ has an open cover $(U_i, f_i)$ for which $f_i |_{U_i \cap U_j} = f_j |_{U_i \cap U_j}$ so these morphisms glue to give $f : \Dom{f} \to Y$ ($\Hom{S}{-}{Y}$ is a sheaf on the Zariski site).  
\end{proof}

\subsection{Dominant Morphisms}

\begin{definition}
A morphism $f : X \to Y$ is dominant if its image (topologically) is dense.
\end{definition}


\begin{lemma}
If $X$ and $Y$ are irreducible with generic points $\xi \in X$ and $\eta \in Y$ then $f : X \to Y$ is dominant iff $f(\xi) = \eta$. 
\end{lemma}

\begin{proof}
Clearly, if $f(\xi) = \eta$ then,
\[ \overline{f(X)} \supset \overline{f(\xi)} = X \]
so $f$ is dominant. Conversely, suppose that $f : X \to Y$ is dominant. Then,
\[ f(X) = f(\overline{\{ \xi \}}) \subset \overline{f(\xi)} \]
but $f(X)$ is dense so $\overline{f(\xi)} = Y$ but $Y$ has a unique generic point so $f(\xi) = \eta$.
\end{proof}

\begin{definition}
Let $X, Y$ be irreducible. A rational map $f : X \rat Y$ is \textit{dominant} if any representative $f : U \to Y$ is dominant.
\end{definition}

\begin{rmk}
Since, on an irreducble scheme $X$ every nonempty open $W \subset X$ contains the generic point $\xi \in W \subset X$. Therefore, if $f_U : U \to Y$ and $f_V : V \to Y$ agree on some dense open $W \subset U \cap V$ then $f_U(\xi) = \eta \iff f_V(\xi) = \eta$ so some representative is dominant iff every representative is dominant. 
\end{rmk}

\begin{prop}
Irreducible schemes with dominat rational maps form a category. 
\end{prop}

\begin{proof}
It suffices to show how dominant rational maps may be composed. Given $f : X \rat Y$ and $g : Y \rat Z$ and representatives $f_U : U \to Y$ and $g_V : V \to Y$. Then, consider $g \circ f : f^{-1}(V) \to Z$. Since $f$ is dominant $\xi_X \in f^{-1}(\xi_Y) \subset f^{-1}(V)$ so $f^{-1}(V)$ is nonempty (since $\Im{f} \cap V$ is nonempty because $\Im{f}$ is dense). Furthermore, $f(\xi_X) = \xi_Y$ and $g(\xi_Y) = \xi_Z$ so $g \circ f(\xi_X) = \xi_Z$ and thus $g \circ f$ is dominant so it defines a dominant rational map $g \circ f : X \rat Z$. Furthermore, if $(U_1, f_1) \sim (U_2, f_2)$ and $(V_1, g_1) \sim (V_2, g_2)$ then $f_1 |_W = f_2 |_W$ and $g_1 |_{W'} = g_2 |_{W'}$ for dense opens $W \subset U_1 \cap U_2$ and $W' \subset V_1 \cap V_2$. Then, $(g_1 \circ f_1) |_{f^{-1}(W') \cap W} = (g_2 \circ f_2) |_{f^{-1}(W') \cap W}$ so composition is well-defined. 
\end{proof}

\begin{remark}
We really need irreducibility to compose rational maps. Consider,
\[ \Spec{k[x,y]/(xy)} \xrightarrow{f} \Spec{k[x]} \rat \Gm{k} \]
where $\Spec{k[x]} \to \Gm{k}$ is defined on the dense open $D(x)$. However, $f^{-1}(D(x)) \subset \Spec{k[x,y]/(xy)}$ is $\Spec{k[x, x^{-1}]} \embed \Spec{k[x,y]/(xy)}$ contained in the $x$-axis and thus is not dense.
\end{remark}

\subsection{Rational Functions}

\begin{definition}
A \textit{rational function} on a scheme $X$ is a rational map $f : X \rat \A^1_\Z$ or for $X \to S$ equivalently a rational $S$-map $f : X \rat \A^1_S$. Since $\A^1$ is a ring object in the category of schemes and thus gives a ring structure on $\Hom{\Sch}{U}{\A^1}$. This puts a ring structure on the set of rational functions forming the ring of rational functions,
\[ R(X) = \varinjlim_{U \in \mathcal{D}(X)} \Hom{\Sch}{U}{\A^1} \]
where $\mathcal{D}(X)$ is the set of dense open subsets $U \subset X$.
\end{definition}

\begin{prop}
Suppose that $X$ has finitely many irreducible compoents with generic point $\xi_i$. Then,
\[ R(X) = \stalk{X}{\xi_1} \times \cdots \times \stalk{X}{\xi_n} \]
\end{prop}

\begin{proof}
For any dense open and there are finitely many irreducible components $Z_i$ then $Z_i \cap U \neq \empty$ so $\xi_i \in U$ for each $i$ since otherwise,
\[ U \subset \bigcup_{i \neq j} Z_i \]
which is closed (since the union is finite) contradicting denseness of $U$. Now,
\[ U_i = (Z_i \cap U) \setminus \bigcup_{j \neq i} Z_i \]
is open and $\xi \in U_i \subset Z_i$ and,
\[ \bigcup_{i = 1}^n U_i \subset U \subset X \]
is dense since it contains all $\xi_i$. However, $U_i \cap U_j = \varnothing$ and thus,
\begin{align*}
R(X) & = \varinjlim_{U \in \mathcal{D}(X)} \Hom{\Sch}{U}{\A^1}
\\
& = \varinjlim_{U \in \mathcal{D}(X)} \struct{X}(U)
\\
& = \varinjlim_{U \in \mathcal{D}(X)} \prod_{i = 1}^n \struct{X}(U_i)
\\
& = \prod_{i = 1}^n \varinjlim_{\xi_i \in U_i} \struct{X}(U_i)
\\
& = \prod_{i = 1}^n \stalk{X}{\xi_i} 
\end{align*}
since all opens containing each generic point are dense.
\end{proof}

\begin{cor}
If $X$ is reduced then,
\[ R(X) = \kappa(\xi_1) \times \cdots \times \kappa(\xi_n) \] 
If $X$ is irreducible then,
\[ R(X) = \stalk{X}{\xi} \]
If $X$ is integral then,
\[ R(X) = \kappa(\xi) = K(X) \]
so the ring of rational functions is exactly the function field on an integral scheme.
\end{cor}

\begin{lemma}
A dominant rational map $X \rat Y$ (over $S$) between irreducible schemes induces a $\stalk{S}{s}$-algebra map $\stalk{Y}{\xi_Y} \to \stalk{X}{\xi_X}$. 
\end{lemma}

\begin{proof}
A morphism $X \rat Y$ in the category of domiant rational $S$-maps gives by composition $R(Y) = \Hom{\mathbf{Rat}}{Y}{\A^1_S} \to \Hom{\mathbf{Rat}}{X}{\A^1_S} = R(X)$. Alternatively, since $X \rat Y$ is defined on some nonempty open (dense is automatic for irreducible schemes) $U \to Y$ and $\xi_X \in U$. Since $X \rat Y$ is dominant $\xi_X \mapsto \xi_Y$ and thus we get $\stalk{X}{\xi_Y} \to \stalk{X}{\xi_X}$ over $\stalk{S}{s}$ for $\xi_X, \xi_Y \mapsto s$. 
\end{proof}

\begin{cor}
A dominant rational $k$-map $X \rat Y$ of integral schemes over $\Spec{k}$ induces an extension of function fields $K(Y) \embed K(X)$ over $k$.
\end{cor}

\begin{rmk}
This is an extension of fields because a ring map $K(Y) \to K(X)$ is automatically injective on fields. 
\end{rmk}

\begin{cor}
There is a functor $\mathbf{Rat}_A^\op \to \Ring_A$ from the category of irreducble schemes over $\Spec{A}$ and dominant rational maps to the category of $A$-algebras sending $X \rat Y$ to $R(Y) \to R(X)$. 
\bigskip\\
Likewise, there is a functor $\mathbf{Rat}_{\text{int}, A}^\op \to \mathbf{Field}_A$ from the category of integral schemes over $\Spec{A}$ and dominant rational maps to the category of fields over $A$ sending $X \rat Y$ to $K(Y) \embed K(X)$ over $A$. 
\end{cor}

\subsection{Birational Maps}

\begin{definition}
Irreducible $S$-schemes are $S$-\textit{birational} if they are isomorphic in the category of irreducible $S$-schemes with dominant rational $S$-maps. We say that a rational $S$-map $f : X \rat Y$ is a birational morphism if it is dominant and there is a domiant rational $S$-morphism $g : Y \rat Y$ such that $g \circ f = \id_X$ and $f \circ g = \id_Y$ as rational maps. 
\end{definition}

\begin{prop}
If irreducible schemes $X$ and $Y$ are birational then $R(X) = R(Y)$. 
\end{prop}

\begin{prop}
In particular, if integral $k$-schemes $X$ and $Y$ are $k$-birational then $K(X) = K(Y)$ via a $k$-isomorphism. 
\end{prop}

\begin{prop}
Let $X$ and $Y$ be irreducible $S$-schemes. Then $X$ and $Y$ are $S$-birational iff there are dense opens $U \subset X$ and $V \subset Y$ which are isomorphic $U \cong V$ over $S$.
\end{prop}

\begin{proof}
If $f : U \to V$ and $g : V \to U$ are inverse $S$-isomorphisms then they represent inverse dominat (since they are surjective onto $U,V$ which are dense) rational $S$-maps $f : X \rat Y$ and $f : Y \rat X$ so $X$ and $Y$ are birational.
\bigskip\\
The reverse direction is Tag 0BAA.
\end{proof}

\begin{theorem}
There is an equivalence of categories between the following,
\begin{enumerate}
\item the category of integral schemes locally of finite type over $k$ with dominant rational maps
\item the category of affine integral schemes of finite type over $k$ with dominant rational maps
\item the opposite category of finitely-generated $k$-algebra domains with dominant rational maps
\item the opposite category of finitely-generated fields over $k$ with inclusions over $k$
\end{enumerate}
\end{theorem}

\begin{proof}
We need to show that an embedding $K(Y) \embed K(X)$ over $k$ for integral scheme locally of finite type over $k$ induces a rational map and for any finitely generated field $K$ over $k$ there is (DO THIS PROOF).
\end{proof}

\begin{rmk}
The restiction on the category of schemes is necessary. $\Spec{k(x)}$ is not finte type over $k$ and there is no rational map $\Spec{k[x]} \rat \Spec{k(x)}$ induced by $k(x) \embed k(x)$ since it would simply be a morphism and would be given by a ring map $k(x) \embed k[x]$ inducing $\id : k(x) \to k(x)$ which is impossible since it needs to send $x \mapsto x$ which is a unit in the source but not the target.
\end{rmk}

\begin{rmk}
See Tag 0BAD for a generalization.
\end{rmk}

\subsection{Rational Varieties}

\subsection{Extending Rational Maps}

\begin{lemma}
Regular local rings of dimension $1$ exactly correspond to DVRs.
\end{lemma}

\begin{proof}
Any DVR $R$ has a uniformizer $\varpi \in R$ then $\dim{R} = 1$ and $\m / \m^2 = (\varpi)/(\varpi^2) = \varpi \kappa$ which also has $\dim_{\kappa}(\m / \m^2) = 1$ so $R$ is regular.
\bigskip\\
Conversely, if $R$ is a regular local ring of dimension $\dim{R} = 1$ then, by regularity, $R$ is a normal noetherian domain so by $\dim{R} = 1$ then $R$ is Dedekind but also local and thus is a DVR. 
\end{proof}

\begin{proposition}
Let $X$ be a Noetherian $S$-scheme and $Z \subset X$ a closed irreducible codimension $1$ generically nonsingular subset (with generic point $\eta \in Z$ such that $\stalk{X}{\eta}$ is regular). Let $f : X \rat Y$ be a rational map with $Y$ proper over $S$. Then $Z \cap \Dom{f}$ is a dense open of $Z$.
\end{proposition}


\begin{proof}
Choose some representative $(U, f_U)$ for $f : X \rat Y$. Note that $\stalk{X}{\eta}$ is a regular dimension one (see Lemma \ref{codimension_loc_rings}) ring and thus a DVR. Consider the generic point $\xi \in X$ of $X$ then, by localizing, we get an inclusion of the generic point $\Spec{\stalk{X}{\xi}} \to \Spec{\stalk{X}{\eta}} \to X$ and $\stalk{X}{\xi} = K(X) = \Frac{\stalk{X}{\eta}}$. Furthermore, the inclusion of the generic point gives $\Spec{K(X)} \to U \xrightarrow{f_U} Y$ and thus we get a diagram,
\begin{center}
\begin{tikzcd}
\Spec{K(X)} \arrow[d, hook] \arrow[r] & Y \arrow[d]
\\
\Spec{\stalk{X}{\eta}} \arrow[ru, dashed, "\ell"] \arrow[r] & \Spec{k} 
\end{tikzcd}
\end{center}
and a lift $\Spec{\stalk{X}{\eta}} \to Y$ by the valuative criterion for properness applied to $Y \to \Spec{k}$ since $\stalk{X}{\eta}$ is a DVR. Choose an affine open $\Spec{R} \subset Y$ containing the image of $\Spec{\stalk{X}{\eta}} \to Y$ (i.e. choose a neighborhood of the image of $\eta$ which automatically contains $f(\xi)$ since the map factors $\Spec{\stalk{X}{\eta}} \to \Spec{\stalk{Y}{f(\eta)}} \to \Spec{R} \to Y$) and let $\eta \in V = \Spec{A} \subset X$ be an affine open neighbrohood of $\xi$ mapping onto $\Spec{R}$. By Lemma \ref{open_domain}, since $\stalk{X}{\eta}$ is a domain, we may shrink $V$ so that $A$ is a domain. Since $X$ is irreducible $U \cap V$ is a dense open. Note that if $\eta \in U$ then $\eta \in \Dom{f}$ and thus $Z \cap \Dom{f}$ is a nonempty open of the irreducible space $Z$ and therefore a dense open so we are done. Otherwise, let $\p \in \Spec{A}$ correspond to $\eta \in Z$ then $A_\p = \stalk{X}{\eta}$ is a  DVR. Take some principal affine open $D(f) \subset U \cap V$ for $f \in A$ so $f \in \p$ since $\p \notin D(f) \subset U \cap V$. Since $A_\p$ is a DVR we may choose a uniformizer $\varpi \in \p$ so the map $A \to \p$ via $1 \mapsto \varpi$ is as isomorphism when localized at $\p$. Since $A$ is Noetherian both are f.g. $A$-modules so there must be some $s \in A \setminus \p$ such that $A_s \to \p_s$ is an isomorphism. Replacing $A$ by $A_s$ we may assume $\p = (\varpi) \subset A$ is principal. Since $f \in \p$ we can write $f = t \varpi^k$ for some $a \in A \setminus \p$ (see Lemma \ref{principal_ideal_powers}). Then consider $\tilde{V} = \Spec{A_t}$. Since $t \notin \p$ then $\eta \in \tilde{V}$ and since $f = t \varpi^k$ we have $D(f) \subset D(t) = \tilde{V}$.
Now we get the following diagram, 
\begin{center}
\begin{tikzcd}[row sep = large]
& & \Spec{R}
\\
\Spec{A_\p} \arrow[rru, bend left, "\ell"] \arrow[r] &  \Spec{A_t} \arrow[ru, dashed, "f_V"]
\\
\Spec{\Frac{A}} \arrow[r] \arrow[u] & \Spec{A_f} \arrow[u] \arrow[ruu, bend right, "f_U"'] 
\end{tikzcd}
\end{center}
I claim the square is a pushout in the category of affine schemes because maps $R \to A_\p$ and $R \to A_f$ which agree under the inclusion to $\Frac{A}$ gives a map $R \to A_\p \cap A_f \subset \Frac{A}$. However, consider,
\[ x \in A_\p \cap A_t \implies x = \frac{u \varpi^r}{s} = \frac{a}{f^n} \]
for $u, s, t \in A \setminus \p$ and $a \in A$. Thus we get,
\[ u t^n \varpi^{r + nk} = s a \]
so $a \in \p^{r + nk} \setminus \p^{r + nk + 1}$ ($s \notin \p$ which is prime) and thus $a = u' \varpi^{r + nk}$ for $u' \in A \setminus \p$. Therefore,
\[ x = \frac{u' \varpi^{r + nk}}{t^n \varpi^{nk}} = \frac{u' \varpi^{r}}{t^n} \in A_t \]
Thus, $A_\p \cap A_f \subset A_f$ so we get a map $R \to A_t$. Therefore we get a map $f_{\tilde{V}} : \tilde{V} \to Y$ such that $(f|_{\tilde{V}})|_{D(f)} = (f_U)|_{D(f)}$ which implies that $\eta \in \tilde{V} \subset \Dom{f}$ so $Z \cap \Dom{f}$ is a dense open of $Z$. 
\end{proof}

\begin{prop}
Let $C \to S$ be a proper regular noetherian scheme with $\dim{C} = 1$ and $f : C \rat Y$ a rational $S$-map with $Y \to S$ proper. Then $f$ extens unquely to a morphism $f : C \to Y$. 
\end{prop}

\begin{proof}
For any point $x \notin \Dom{f}$ let $Z = \overline{\{ x \}} \subset D$ for $D = C \setminus \Dom{f}$. Since $\Dom{f}$ is a dense open, by lemma \ref{codimension_opens}, we have $\codim{Z, C} \ge \codim{D, C} \ge 1$ but $\dim{C} = 1$ so $\codim{Z, C} = 1$. Furthermore, since $C$ is regular $\stalk{C}{x}$ is regular and thus, by the previous proposition, $Z \cap \Dom{f}$ is a dense open and in particular $x \in \Dom{f}$ meaning that $\Dom{f} = C$ so we get a morphism $C \to Y$. This is unique because $C$ is reduced (it is regular) and $Y$ is separated (it is proper over $S$) so morphisms $C \to Y$ are uniquely determined on a dense open which any representative for $f : C \rat Y$ is defined on.   
\end{proof}

\begin{defn}
A \textit{curve} over $k$ is an integral separated dimension one scheme finite type over $\Spec{k}$.
\end{defn}

\begin{cor}
Rational maps between normal proper curves are morphisms.
\end{cor}

\begin{cor}
Birational maps between normal proper curves are isomorphisms.
\end{cor}

\begin{proof}
Let $f : C_1 \rat C_2$ and $g : C_2 \rat C_1$ be birational inverses of smooth proper curves. Then we know that these extend to morphisms $f : C_1 \to C_2$ and $g : C_2 \to C_1$. Furthermore, the maps $g \circ f : C_1 \to C_1$ must extend the identity on some dense open. However, since curves are separated and reduced there is a unique extension of this map so $g \circ f = \id_{C_1}$ and likewise $f \circ g = \id_{C_2}$. 
\end{proof}

\begin{thm}
If $k$ is perfect then there exists a unique normal curve in each birational equivalence class of curves.
\end{thm}

\begin{proof}
It suffices to show existence. Given a curve $X$, we consider the projective closure $X \to \overline{X}$ (WHY THIS EXISTS) which is birational and $\overline{X} \to \Spec{k}$ is proper. Then take the normalization $\overline{X}^\nu \to \overline{X}$ which remains proper over $\Spec{k}$ (CHECK THIS) and is birational. Then $\overline{X}^\nu$ is regular and thus smooth over $k$ since $k$ is perfect and $\overline{X}^\nu \to X$ is birational.
\end{proof}

\subsection{Lemmas}

\begin{lemma} \label{principal_ideal_powers}
Let $A$ be a Noetherian domain and $\p = (\varpi)$ a principal prime. Then any $f \in \p$ can be written as $f = t \varpi^k$ for $f \in A \setminus \p$. 
\end{lemma}

\begin{proof}
From Krull intersection,
\[ \bigcap_{n \ge 0}^\infty \p^n = (0) \]
so there is some $n$ such that $f \in \p^n \setminus \p^{n+1}$. Thus $f = t \varpi^n$ for some $f \in A$ but if $t \in \p$ then $f \in \p^{n+1}$ so the result follows.
\end{proof}

\begin{lemma} \label{codimension_opens}
Consider a closed subset $Y \subset X$ and an open $U \subset X$ with $U \cap Z \neq \empty$. Then $\codim{Y, X} = \codim{Y \cap U, U}$. 
\end{lemma}

\begin{proof}
Consider a chain of irreducibles $Z_i \supsetneq Z_{i+1}$ with $Z_0 \subset Y$. I claim that $Z_i \mapsto Z_i \cap U$ and $Z_i \mapsto \overline{Z_i}$ are inverse functions giving a bijection between closed irreducible chains in $X$ with final terms containined in $Y$ and closed irreducible chains in $U$ with final term contained in $Y \cap  U$. Note, if $Z_i \subset Y \cap U$ then $\overline{Z_i} \subset Y$ since $Y$ is closed in $X$.
\bigskip\\
First, $\overline{Z_i \cap U} \subset Z_i$ and is closed in $X$. Then $\overline{Z_i \cap U} \cup U^C \supset Z_i$ so because $Z_i$ is irreducible $\overline{Z_i \cap U} = Z_i$ since by assumption $Z_i \not\subset U^C$. Conversely, if $Z_i \subset U$ is a closed irreducible subset then $\overline{Z_i}$ is closed and irreducible in $X$ and $Z_i \subset \overline{Z_i} \cap U$ but $Z_i = C \cap U$ for closed $C \subset X$ so $Z_i \subset C$ and thus $\overline{Z_i} \subset C$ so $\overline{Z_i} \cap U \subset C \cap U = Z_i$ meaning $Z_i = \overline{Z_i} \cap U$. Thus we have shown these operations are inverse to eachother.
\bigskip\\
Finally, if $Z_i \cap U - Z_{i+1} \cap U$ then $\overline{Z_i \cap U} = \overline{Z_i \cap U}$ so $Z_i = Z_{i+1}$ so the chain does not degenerate. Likewise, if $\overline{Z_i} = \overline{Z_{i+1}}$ then $\overline{Z_i} \cap U = \overline{Z_{i+1}} \cap U$ so $Z_i = Z_{i+1}$. Therefore, we get a length-preserving bijection between the chains defining $\codim{Y,X}$ and $\codim{Y \cap U, U}$. 
\end{proof}

\begin{lemma}
If $Z \subset X$ is irreducible and $U$ is open and $U \cap Z \neq \empty$ then $Z \cap U$ is irreducible. Furthermore, if $Z \subset X$ is irreducible then $\overline{Z}$ is irreducible.
\end{lemma}

\begin{proof}
If we have closed $Z_1, Z_2 \subset X$ with $Z_1 \cup Z_2 \supset Z \cap U$ then $Z_1 \cup Z_2 \cup U^C \supset Z$ so one must cover $Z$ since it is irreducible but $Z \not\subset U^C$ so either $Z_1 \supset Z \cap U$ or $Z_2 \supset Z \cap U$.
\bigskip\\
Likewise, for closed $Z_1, Z_2 \subset X$ with $Z_1 \cup Z_2 \supset \overline{Z} \supset Z$ then by irreducibility $Z_1 \supset Z$ or $Z_1 \supset Z$ but these are closed so $Z_1 \supset \overline{Z}$ or $Z_2 \supset \overline{Z}$. 
\end{proof}

\begin{lemma} \label{codimension_loc_rings}
Let $Z \subset X$ be a closed irreducible subset with generic point $\eta \in Z$. Then $\codim{Z,X} = \dim{\stalk{X}{\eta}}$. 
\end{lemma}


\begin{proof}
Take affine open neighborhood $\eta \in U = \Spec{A} \subset X$. Then for $\p \in \Spec{A}$ corresponding to $\eta$ we get $A_\p = \stalk{X}{\eta}$. However, $\codim{Z, X} = \codim{Z \cap U, U}$ and $Z \cap U = \overline{\{ \p \}} = V(\p)$. Therefore,
\[ \codim{Z, X} = \codim{Z \cap U, U} = \height{\p} = \dim{A_\p} = \dim{\stalk{X}{\eta}} \]
\end{proof}

\begin{lemma}
Let $X$ be a Noetherian scheme then the nonreduced locus,
\[ Z = \{ x \in X \mid \nilrad{\stalk{X}{x}} \neq 0 \} \]
is closed.
\end{lemma} 

\begin{proof}
The subsheaf $\Nil \subset \struct{X}$ is coherent since $X$ is Noetherian. Thus $Z = \Supp{\struct{X}}{\Nil}$ is closed and $\Nil_x = \nilrad{\struct{X}{x}}$. Locally, on $U = \Spec{A}$ we have $\Nil |_U  = \wt{\nilrad{A}}$ and $\nilrad{A}$ is a f.g. $A$-module since $A$ is Noetherian so,
\[ \Supp{\struct{X}}{\Nil} \cap U = \Supp{A}{\nilrad{A}} = V(\Ann{A}{\nilrad{A}} \]
is closed in $\Spec{A}$. 
\end{proof}

\begin{lemma}
Let $X$ be a Noetherian scheme then $X$ has finitly many irreducible components.
\end{lemma}

\begin{proof}
First let $X = \Spec{A}$ for a Noetherian ring $A$. Then the irreducible components of $A$ correspond to minimal primes $\p \in \Spec{A}$. Then $\dim{A_\p} = 0$ and $A_\p$ is Noetherian so $A_\p$ is artinian. $A_\p$ must have some associated prime so $\Ass{A_\p}{A_\p} = \{ \p A_\p \}$.  By Tag 05BZ, then $\Ass{A}{A} \cap \Spec{\A_\p} = \Ass{\A_\p}{\A_\p} = \{ \p \}$ so every minimal prime is an associated prime. However, for $A$ noetherian then $A$ admits a finite composition series so there are finitely many associated primes.
\bigskip\\
Now let $X$ be a Noetherian scheme. For any affine open $U \subset X$ we have shown that $U$ has finitely many irreducible components. However, since $X$ is quasi-compact there is a finite cover of affine opens and thus $X$ must have finitely many irreducible components. 
\end{proof}

\begin{lemma}
Let $X$ be a Noetherian scheme and $Y$ is the complement of some dense open $U$. Then $\codim{Y, X} \ge 1$.
\end{lemma}

\begin{proof}
It suffices to show that $Y$ does not contain any irreducible component since theny any irreducible contained in $Y$ cannot be maximal. Since $X$ is Noetherian, it has finitely many irreducible components $Z_i$. Then if $Z_j \subset Y$ for some $i$ we would have $Z_i \cap U = \varnothing$ but then,
\[ U = \bigcup_{i \neq j} Z_i \]
which is closed so $\overline{U} \subsetneq X$ contradicitng our assumption that $U$ is dense.
\end{proof}

\begin{example}
This may not hold when $X$ is not Noetherian. For example, (FIND EXAMPLE)
\[ X = \bigcup_{i = 1}^\infty V(x_i) \subset k[x_1, x_2, \cdots] \] 
\end{example}

\begin{lemma} \label{open_domain}
Let $X$ be a Noetherian scheme and $x \in X$ such that $\stalk{X}{x}$ is a domain. Then there is an affine open neighborhood $x \in U \subset X$ with $U = \Spec{A}$ and $A$ is a domian.
\end{lemma}

\begin{proof}
Take any affine open neighbrohood $x \in U \subset X$ with $U = \Spec{A}$ and $\p \in \Spec{A}$ corresponding to $x$. Then $A_\p = \stalk{X}{x}$ is a domain. Since $X$ is Noetherian then $A$ is Noetherian so it has finitely many minimal primes $\p_i$ (corresponding to the generic points of irreducible components of $U$) with $\p_0 \subset \p$. Since $A_\p$ is a domain, it has a unique minimal prime and thus $\p_0$ is the only minimal prime contained in $\p$ (geometrically $A_\p$ being a domain corresponds to the fact that $\p$ is the generic point of a generically reduced irreducible subset which lies in only one irreducible component)
\bigskip\\
Now for any $i \neq 0$ take $f_i \in \p \setminus \p_0$. This is always possible else $\p \subset \p_0$ contradicting the minimality of $\p_0$. If $f \notin \q$ then $\q \not\supset \p_i$ for any $i \neq 0$ so $\q \supset \p_0$ since it must lie above some minimal prime. Thus $\nilrad{A_f} = \p_0 A_f$ is prime and $f \notin \p$ since else $\p \supset \p_1 \cap \cdots \cap \p_n$ which is impossible since $\p \not\supset \p_i$ for any $i$. Now we know that $\nilrad{A_\p} = 0$ and $A_f$ is Noetherian so $\nilrad{A_\p}$ is finitely generated. Thus, there is some $g \notin \p$ such that $\nilrad{A_{fg}} = (\nilrad{A_f})_g = 0$. Thus $A_{fg}$ is a domain since $\nilrad{A_{fg}} = (0)$ and is prime and $\p \in A_{fg}$ because $fg \notin \p$. Therefore, $x \in \Spec{A_{fg}} \subset U$ is an affine open satisfying the requirements. 
\end{proof}

\begin{rmk}
This does not imply that $X$ is integral if $\stalk{X}{x}$ is a domain for each $x \in X$ (which is false, consider $\Spec{k \times k}$) because it only shows there is an integral cover of $X$ not that $\struct{X}(U)$ is a domain for each $U$. 
\end{rmk}

\begin{example}
Let $X = \Spec{k[x,y]/(xy, y^2)}$. Then for the bad point $\p = (x, y)$ we have $\nilrad{\stalk{X}{\p}} = (y)$. Away from the bad point, say $\p = (x - 1, y)$ we have, $\stalk{X}{\p} = \Spec{k[x]_{(x-1)}}$ so $\nilrad{\stalk{X}{\p}} = (0)$. Furthermore, at the generic point $\p = (y)$, we have, $\stalk{X}{\p} = \Spec{k(x)}$ so $\nilrad{\stalk{X}{\p}} = (0)$. 
\end{example}

\begin{example}
Consider $X = \Spec{k[x,y,z]/(yz)}$ which is the union of the $x$-$y$ and $x$-$z$ planes. Consider the generic point of the $z$-axis $\p = (x, y)$ then $\stalk{X}{\p} = \Spec{k[x, z]_{(x)}}$ is a domain since the $z$-axis only lies in one irreducible component. However, at the generic point of the $x$-axis, $\p = (y, z)$ we get $\stalk{X}{\p} = \Spec{(k[x, y, z]/(yz))_{(y, z)}}$ has zero divisors $yz = 0$ so is not a domain since the $x$-axis lives in two irreducible components.
\end{example}

\section{Reflexive Sheaves}

\newcommand{\RPic}[1]{\mathrm{RPic}\left( #1 \right)}
\renewcommand{\R}{\mathcal{R}}

\begin{defn}
Recall the dual of a $\struct{X}$ module $\F$ is the sheaf $\F^\vee = \shHom{\struct{X}}{\F}{\struct{X}}$. We say that a coherent $\struct{X}$-module $\F$ is \textit{reflexive} if the natural map $\F \to \F^{\vee \vee}$ is an isomorphism. 
\end{defn}

\begin{lemma}
Let $X$ be an integral locally Noetherian scheme and $\F, \G$ be coherent $\struct{X}$-modules. If $\G$ is reflexive then $\shHom{\struct{X}}{\F}{\G}$ is reflexive.
\end{lemma}

\begin{proof}
[Tag. 0AY4]
\end{proof}
\noindent\\
In particular, since $\struct{X}$ is clearly reflexive, this lemma shows that for any coherent $\struct{X}$-module then $\F^\vee$ is a reflexive coherent sheaf. We say the map $\F \to \F^{\vee \vee}$ gives the reflexive hull $\F^{\vee \vee}$ of $\F$.

\begin{defn}
Let $\R$ be the full subcategory $\Coh{\struct{X}}$ of coherent reflexive $\struct{X}$-modules. $\R$ is an additive category   and in fact has all kernels and cokernels defined by taking reflexive hulls of the sheaf kernel and cokernel. Furthermore, $\R$ inherits a monoidal structure from the tensor product defined using the reflexive hull as follows,
\[ \F \otimes_\R \G = (\F \otimes_{\struct{X}} \G)^{\vee \vee} \]
Finally, we define $\RPic{X}$ to be group of constant rank one reflexives induced by the monoidal structure on $\R$. Explicitly, $\RPic{X}$ is the group of isomorphism clases of constant rank one reflexive coherent $\struct{X}$-modules with multiplication $(\F, \G) \mapsto (\F \otimes_{\struct{X}} \G)^{\vee \vee}$ and inverse $\F \mapsto \F^\vee$. 
\end{defn}
\noindent\\
The importance of reflexive sheaves derives from their correspondence to Weil divisors. Here we let $X$ be a normal integral seperated Noetherian scheme. 

\begin{prop}
If $D$ is a Weil divisor then $\struct{X}(D)$ is reflexive of constant rank one. 
\end{prop}

\begin{proof}
(CITE OR DO).
\end{proof}

\begin{theorem}
Let $X$ be a normal integral seperated Noetherian scheme. There is an isomorphism of groups $\Cl{X} \xrightarrow{\sim} \RPic{X}$ defined by $D \mapsto \struct{X}(D)$.
\end{theorem}

\begin{proof}
(DO OR CITE)
\end{proof}
\noindent\\
We summarize the important results as follows.
\begin{theorem}
Let $X$ be a Noetherian normal integral scheme. Then for any Weil divisors $D, E$,
\begin{enumerate}
\item $\struct{X}(D + E) = (\struct{X}(D) \otimes_{\struct{X}} \struct{X}(E))^{\vee \vee}$
\item $\struct{X}(-D) = \struct{X}(D)^\vee$
\item $\shHom{\struct{X}}{\struct{X}(D)}{\struct{X}(E)} = \struct{X}(E - D)$
\item if $E$ is Cartier then $\struct{X}(D + E) = \struct{X}(D) \otimes_{\struct{X}} \struct{X}(E)$
\end{enumerate}
\begin{center}

\begin{proof}
(DO OR CITE)
\end{proof}

\end{center}
\end{theorem}
\noindent\\
Finally, we have a result which controls when the dualizing sheaf can be expressed in terms of a divisor.
\begin{prop}
Let $X$ be a projective variety over $k$. Then,
\begin{enumerate}
\item if $X$ is normal then its dualizing sheaf $\omega_X$ is reflexive of rank $1$ and thus $X$ admits a canonical divisor $K_X$ s.t. $\omega_X = \struct{X}(K_X)$
\item if $X$ is Gorenstein then $\omega_X$ is an invertible module so $K_X$ is Cartier.
\end{enumerate}
\end{prop}

\begin{proof}
(FIND CITATION OR DO).
\end{proof}



\end{document}
