\documentclass[12pt]{article}
\usepackage{import}
\import{"../Algebraic Geometry/"}{AlgGeoCommands}

\newcommand{\Loc}[1]{\mathfrak{Loc}\left( #1 \right)}
\newcommand{\AbGrp}{\mathbf{AbGrp}}

\begin{document}

\section{Affine Curves with No Immersions}

\begin{exercise}
Find a smooth affine curve $C$ with no immersion $C \embed \P^2_k$. Unlike in plane case, $K_X \neq 0$ is not an obstruction ($g = 3$ example). 
\end{exercise}

\begin{exercise}
An easier version, find a curve with no immersion $C \embed \A^2_k$. My guess is that here $K_X \neq 0$ is an obstruction but I can't prove it.
\end{exercise}

\section{Curves with Intersection at One Point in Ambiant Space}

\begin{exercise}
First question: given a curve $C$ and a closed point $P \in C$ when is there a function $f \in \Gamma(C, \struct{C})$ such that $V(f) = \{ P \}$ (topologically it may be nonreduced that is fine). 
\bigskip\\
For affine curves: the previous question is asking: given a one-dimensional Noetherian domain $A$ and a maximal ideal $\m_0 \in \Spec{A}$ when is there $f \in A$ such that $\sqrt{(f)} = \m_0$. 
\end{exercise}

Suppose that $A$ is a Dedekind domain. Then by unique factorization, $\sqrt{(f)} = \m_0$ iff $(f) = \m_0^n$ meaning that this is possible iff $\m_0$ is torsion in the ideal class group. Chooing any nontorision point of an elliptic curve is then a counter example.
\bigskip\\
Is the class group not being torsion the only obstruction?

\begin{exercise}
Given a closed immersion $\iota : C \embed S$ with $S$ a surface, given a point $P \in C$ when does there exist a closed curve $Z \subset S$ such that $Z \cap C = \{ P \}$?
\end{exercise}
\noindent
First, an example. Consider the surface $S = C \times_k \P^1_k$ with $C \embed S$ via the closed point $0 \in \P^1_k$. Then for any closed point $P \in C$ we can consider the curve $\{ P \} \times_k \P^1_k \subset C \times_k \P^1_k$ and $\{ P \} \times_k \P^1_k \cap C = \{ P \}$. Therefore, this is certainally possible for an arbitrary curve. Perhaps we need to fix 
\bigskip\\
The affine curve $Z \subset X$ is defined by some sheaf of ideals $\I \subset \struct{S}$. Then $\iota^{-1} \I \cdot \struct{C}$ gives the sheaf of ideals for $\iota^{-1}(Z)$. 
\bigskip\\
First, consider $S = \A^2_k$ then we have $C = V(f)$ for some $f \in k[x,y]$ and $Z = V(g)$ for some $g \in k[x,y]$ we can assume these are irreducible so $C$ and $Z$ are integral curves. Then, under the map $k[x,y] \to k[x,y]/(f)$ we get $\iota^{-1} \I \cdot \struct{C} = g \cdot k[x,y]/(f) = (g)$ and thus we reduce exactly to the previous problem for the affine curve $C = \Spec{k[x,y]/(f)}$.
\bigskip\\
Likewise, consider $S = \P^2_k$ then we have $Z = V(I)$ for a height one homogeneous prime ideal $I$ (CHECK THIS). I am guessing that we can take $I = (f)$ since $k[X_0, X_1, X_2]$ is a UFD so height one primes are principal (MAKE SURE $f$ is HOMOGENEOUS). Then $Z = \Proj{k[X_0, X_1, X_2]/(g)}$ and I think $\iota^{-1}(Z) = \Proj{k[X_0, X_1, X_2]/(f, g)}$ or equivalently $V(g)$ in $C = \Proj{k[X_0, X_1, X_2]/(f)}$. So now reduce to when $V(g)$ on a projective curve have support at only one point. 


\section{Weakly but not Strongly Toric Curve}

Find a curve which is toric (i.e. weakly $\Delta$-nondegenerate for some $\Delta$) but is never nondegnerate i.e. (never $\Delta$-nondegenerate for any $\Delta$). 

\section{Affines in the Plane}

\begin{exercise}
Given an example of an immersed curve $C \embed \A^2_k$ such that $C$ is not (closed) embedded in any affine open of $\A^2_k$. Likewise, give an example of an immersed curve $C \embed \P^2_k$ such that $C$ is not (closed) embedded in any affine open of $\P^2_k$.
\end{exercise}

\section{Why is this sor hard}


\begin{prop}
There exists a smooth affine curve $C$ over $k$ with no immersion $C \embed \A^2_k$ and, in particular, no immersion $C \embed \Gm{k}^2$. Thus, there are smooth affine curves which are not affine plane curves. 
\end{prop}

\begin{proof}
First, we show that if $j : C \embed \A^2_k$ is an immersion then $\Omega_{C/k} \cong \struct{C}$ is trivial. We can factor $j$ as $C \embed \overline{C} \embed \A^2_k$ into an open immersion followed by a closed immersion [Stacks, Tag 03DQ]. Then $\overline{C}$ 
\end{proof}

Take an algebraically closed field $k$. Show the following,
\begin{enumerate}
\item if $C \embed \P^2_K$ is an immersion then there is a plane curve $\bar{C} \subset \A^2_k$ (closed immersion) an an open immersion $C \embed \bar{C}$.
\item for any closed curve $\bar{C} \subset \A^2_k$ we have $\Omega_{\bar{C}/k} = \struct{C}$
\item thus, since $C \embed \bar{C}$ is \etale we have $\Omega_{C/k} = \struct{C}$ so it suffices to construct a smooth affine curve with nontrivial canonical bundle $\Omega_{C/k}$.
\item Choose a curve $C$ with genus $g(C) \ge 2$ then $\deg{\Omega_{C/k}} \ge 2$ and choose a point $P \in C$ such that $K_X \not\sim (2g - 2) [P]$ for any $k \in \Z$.
\item Show that $U = C \setminus \{ P \}$ is affine,
\item Then $U \embed C$ is \etale so $\Omega_{U / k} = f^* \Omega_{C/k}$ so $K_C \sim [P]$. 
\item Show that this is nontrivial using the exact sequence,
\begin{center}
\begin{tikzcd}
\Z \arrow[r] & \Cl{C} \arrow[r] & \Cl{U} \arrow[r] & 0
\end{tikzcd}
\end{center}
the first map sending $1 \mapsto [P]$ so we need to show that $K_X \not\sim (2g - 2) [P]$ for any $k$. 
\end{enumerate}

\section{Supersingular Stuff Here}

\section{TODO}

Milne's Notes on Etale Cohomology
\\
Write notes on GAGA and analytification of a Scheme
\\
Finish K-book
\\
Finish Milne's Notes on Shimura Varieties

\section{Questions For Johan}

\begin{exercise}
Which hypersurfaces $X \subset \P^n_k$ are rational. I managed to prove that only hyperplanes and conics $X \subset \P^2_k$ are isomorphic to $\P^{n-1}_k$ but which are actually rational?
\end{exercise}

\begin{exercise}
Which hypersurfaces $X \subset \P^n_k$ are of the form $\P^{a_1} \times_k \cdots \times_k \P^{a_r}$?
\end{exercise}

This might be trivial since in projective space all lines intersect but this is not true of these schemes.

\begin{exercise}
Is there a splitting principal in algebraic geometry? 
\\
How difficult is it to compute the Aytiah class for a vector bundle? What tools are there to do this computation?
\end{exercise}

\begin{exercise}
What is the ``correct'' definition of the Chern classes in algebraic geometry?
\end{exercise}

\begin{exercise}
Ravi says:
Having made a spirited case for smoothness, we
should be clear that regularity is still very useful. For example, it is the only concept which makes sense in mixed characteristic. In particular, Z is regular at its
points (it is a regular ring), and more generally, discrete valuation rings are incredibly useful examples of regular rings.
\bigskip\\
Why does smoothness not make sense in mixed characteristic?
\end{exercise}

\begin{exercise}
what is the difference between the neron model and the minimal model?
\end{exercise}

\begin{exercise}
When is a smooth hypersurface $X \subset \P^n_k$ of degree $d$ uniruled. My guess is exactly when $d < n + 1$ at least for $k = \mathbb{C}$. I think I can prove this for small $n$. Is it true in general? What about other fields? 
\end{exercise}

\begin{exercise}
Why do people talk so much about supersingular K3 surfaces in particular? What is it about K3 surfaces which make the notion of supersingularity so interesting. 
\end{exercise}

\begin{exercise}
In the section on Etale cohomology, you make a distiction between the big and small pushforward of a sheaf on a site. Why? These seem to me to basically be the same functor. I understand that their adjoint depends on the site, is this why we distinguish them? On the same note, why do you only consider small sites for the Zariski and Etale topologies and not for fppf and other topologies?
\end{exercise}

\begin{exercise}
Why is unramified defined as locally finite type as opposed to locally of finite presentation the way smooth and \etale are?
\end{exercise}

\section{Questions For Stack Exchange}

\begin{exercise}
The thing about a function which only cuts out one point.
\end{exercise}

\begin{exercise}
What does supersingularity actually mean.
\end{exercise}

\begin{exercise}
Soft question. Non mathematical questions with mathematical answers. 
\end{exercise}

\end{document}