\documentclass[12pt]{extarticle}
\usepackage[utf8]{inputenc}
\usepackage[english]{babel}
\usepackage[a4paper, total={6in, 9in}]{geometry}
 
\usepackage{amsthm, amssymb, amsmath, centernot}

\newcommand{\notimplies}{%
  \mathrel{{\ooalign{\hidewidth$\not\phantom{=}$\hidewidth\cr$\implies$}}}}
 
\renewcommand\qedsymbol{$\square$}
\newcommand{\cont}{$\boxtimes$}
\newcommand{\divides}{\mid}
\newcommand{\ndivides}{\centernot \mid}
\newcommand{\Z}{\mathbb{Z}}
\newcommand{\N}{\mathbb{N}}
\newcommand{\Zplus}{\mathbb{Z}^{+}}
\newcommand{\Primes}{\mathbb{P}}

 
\newtheorem{theorem}{Theorem}[section]
\newtheorem{lemma}[theorem]{Lemma}
\newtheorem{proposition}[theorem]{Proposition}
\newtheorem{corollary}[theorem]{Corollary}


\newenvironment{definition}[1][Definition:]{\begin{trivlist}
\item[\hskip \labelsep {\bfseries #1}]}{\end{trivlist}}


\newenvironment{lproof}{\begin{proof} \renewcommand{\qedsymbol}{}}{\end{proof}}

\renewcommand{\mod}[3]{\: #1 \equiv #2 \: \mathrm{mod} \: #3 \:}

\newcommand{\nmod}[3]{\: #1 \centernot \equiv #2 \: mod \: #3 \:}

\newcommand{\ndiv}{\hspace{-4pt}\not \divides \hspace{2pt}}

\begin{document}

\author{Ben Church}
\title{\Huge Elements of Number Theory}
\date{}

\maketitle
\tableofcontents
\newpage

\section{The Integers and Divisibility}
Number Theory is primarily the study of the integers $\Z$. In this course we will study specific subsets and supersets of the integers.

\subsection{Basic Definitions}

\textbf{Definitions of Common Sets:}
\begin{itemize}
\item $\> \Zplus \ = \{n \in \Z  \mid n > 0\}$
\item $\> \Primes \ = \{p \in \Zplus \mid p$ is prime$\}$
\item $\> \N \ = \{n \in \Z \mid n \ge 0\}$
\item $\> \mathbb{Q} \ = \{\frac{m}{n} \mid m,n \in \Z \enspace \wedge \enspace n \neq 0\}$
\end{itemize}

\begin{definition}[Fact:]$\indent \Primes \subset \Zplus \subset \N \subset \Z \subset \mathbb{Q} \subset \mathbb{R}$
\end{definition}

\begin{definition} Let $n,a \in \Z$ then $ n \divides a \iff \exists x \in \Z: nx = a$
\end{definition}

\begin{definition} $p \in \Zplus$ and $p > 1$ is prime iff $\forall a \in \Zplus (a \divides p \implies a = 1 \vee a = p)\\ $ \indent equivalently: if $a,b \in \Zplus$ s.t. $p = ab$ then $a = 1$ or $b = 1$
\end{definition}

\begin{definition} Let $n,a,b \in \Z$ then $\mod{a}{b}{n} \iff n \divides a-b$ 
\end{definition}
\subsection{Well-Ordering and Induction}
The natural numbers have two particularly important properties known as the well-ordering principle and the principle of mathematical induction. These two properties are logically equivalent (see Theorem~\ref{oderingisinduction}) but neither can be proved from other elementary properties. One of these two statements is conventionally included as an axiom of the natural numbers. 
\\\\
\textbf{The Well-Ordering Principle:}\\
\indent If $A \subseteq \N \wedge A \neq \{\}$ then $\exists x \in A : \forall y \in A : x \le y$.\\
\indent I.e. every non-empty subset of $\N$ has a least element. 
\\\\
\textbf{The Principle of Mathematical Induction:}\\
\indent If $\phi$ is a logical predicate which satisfies: 
\begin{enumerate}
\item $\exists g \in \N : \phi (g)$
\item $\forall x \in \N : \phi (x) \implies \phi (x+1)$
\end{enumerate}

\quad then $\forall x \in \N : x \ge g \implies \phi (x)$
\newpage

\begin{theorem}
\label{oderingisinduction}
well-ordering is equivalent to induction
\end{theorem}
\begin{proof} Suppose well-ordering. Let $\phi$ satisfy the induction criteria.\\
Define: \[S = \{n \in \N \mid \neg \phi (n) \: \wedge \: n \ge g\}\] Assume $S \subseteq \N$ is not empty. By well-ordering,
$S$ has a least element $l$. \\ $l-1 < l$ so $l-1 \notin S$. Also, $l > g$ because $\phi (g)$ and therefore, $l-1 \ge g$. \\
But $l-1 \notin S$ thus, $\phi (l-1)$ so by criterion 2, $\phi (l)$ thus $l \notin S$. \cont \: (a contradiction!) \\ Therefore, $S = \{\}$ so $\forall x \in \N : \neg (\neg \phi (n) \: \wedge \: n \ge g)$ i.e. $n \ge g \implies \phi (x)$
\\\\
Suppose induction. Let $A \subseteq \N$ have no least element.\\
Define: \[\phi (n) \iff \forall x \le n : x \notin A\] 
Since $0$ is the least element of all the natural numbers, $0 \notin A$ thus $\phi (0)$\\
Suppose $\phi (n)$, if $n+1 \in A$ then because $\forall x < n+1 : x \notin A$\\
$n+1$ would be the least element of A but $A$ does not have a least element. \cont  \\
Thus, $n+1 \notin A$ so $\forall x \le n+1 : x \notin A$ which implies $\phi (n+1)$\\
By induction, $\forall x \in \N : \phi (n)$ so $\forall n \in \N : n \notin A$ equivalently $A \cap \N = \{\}$\\
note that $A \subseteq \N$. Thus: $A = \{\}$\\\\
Induction proves that if $A$ has no least element then $A$ is empty.\\
Equivalently, if $A$ is not empty then $A$ has a least element.\\
\end{proof}

\subsection{Properties of Modular Arithmetic}

\begin{lemma} If $n \divides a$ and $n \divides b$ then $\forall x,y \in \Z: n \divides ax+by$ 
\end{lemma}

\begin{lproof}
\label{pmlemma}
Let $a = ns$ and $b = nr$ then $ax + by = n(sx + ry)$. \\
Because $sx + ry \in \Z$ we conclude that $n \divides ax+by$ 
\end{lproof}

\begin{lemma} If $an \divides bn$ then $a \divides b$
\end{lemma}

\begin{lproof}
\label{dividecancelation}
Let $ank = bn$ therefore, $ak = b$ so $a \divides b$
\end{lproof}

\begin{lemma} 
modular congruence is what is known as an equivalence relation
\begin{enumerate}
\item (Reflexivity) For every $a \in \Z$, $\mod{a}{a}{n}$
\item (Symmetry) If $\mod{a}{b}{n}$ then $\mod{b}{a}{n}$
\item (Transitivity) If $\mod{a}{b}{n}$ and $ \mod{b}{c}{n}$ then $\mod{a}{c}{n}$
\end{enumerate}
\end{lemma}

\begin{lproof}
$a-a = 0$ but $n   0 = 0$ so $\exists x$ (namely 0) $: nx = a-a \:$ therefore, $\: n \divides a-a$. \\
If $n \divides a-b$ then $nk = a-b$ so $n(-k) = b-a$ thus $n \divides b-a$. \\
If $n \divides a-b$ and $n \divides b-c$ then $n \divides (a-b) + (b-c)$ so $n \divides a-c$.
\end{lproof}

\newpage


\begin{lemma}
\label{modadd}
If $\mod{a}{b}{n} \:$ and $\: \mod{c}{d}{n} \:$ then $\: \mod{a+c}{b+d}{n}$
\end{lemma}

\begin{lproof}
$n \divides a-b$ and $n \divides c-d \:$ thus, $\: n \divides (a-b) + (c-d)$ so $n \divides (a+c)-(b+d)$ 
\end{lproof}

\begin{lemma}
\label{modmult}
If $\: \mod{a}{b}{n} \:$ and $\: \mod{c}{d}{n} \:$ then $\: \mod{ac}{bd}{n}$
\end{lemma}

\begin{lproof}
Let $n \divides a-b$ and $n \divides c-d$ then $n \divides (a-b)c + (c-d)b$ thus $n \divides ac -bd$
\end{lproof}

\begin{lemma}
\label{modpower}
If $\mod{a}{b}{n}$ then $\forall k \geq 0: \mod{a^k}{b^k}{n}$
\end{lemma}

\begin{lproof}
Base Case: by hypothesis $ a^1 \equiv b^1 \: mod \: n$\\
Assume that $\mod{a^k}{b^k}{n}$ then by Lemma~\ref{modmult}, $\mod{a^k \cdot a}{b^k   \cdot b}{n}$\\ thus $\mod{a^{k+1}} {b^{k+1}}{n}$ and by induction, the result holds for all $k$.\\
\textit{note:} $\mod{a^k}{b^k}{n} \notimplies \mod{a}{b}{n}$
\end{lproof}



\subsection{Primes and Factorization}

\begin{theorem}
\label{primeorprod}
every $1 < x \in \N$ is either prime or a product of primes
\end{theorem}

\begin{proof}
Let $S$ be the set of natural numbers that are neither prime nor a product of primes. Suppose $S$ is non-empty. By well-ordering, $S$ has a least element: $g$.\\
$g$ is not prime so $g = ab$ where $1 < a,b < g$ and since $g$ is minimal, $a,b \notin S$.\\
Therefore, $a$ and $b$ are either prime or products of primes. \\
Because products of products of primes are products of primes, \\
$ab = g$ is a product of primes so $g \notin S$ \cont \\\\
Thus, $S$ is empty.\\
\end{proof}

\begin{theorem}
There are infinitely many primes.
\end{theorem}

\begin{proof}
Suppose that $p_1, p_2, \dots , p_k$ are all the primes.\\
Then we can define: $\Pi = p_1   p_2   \dots   p_k + 1$. \\
By Theorem~\ref{primeorprod}, $\exists \tilde{p} \in \Primes$ s.t. $\tilde{p}  \divides  \Pi$ \\
Because $\tilde{p}$ is prime, it must be one of the $p_i$ so $\tilde{p}  \divides  p_1   p_2   \dots   p_k$. \\
Thus, $\tilde{p}  \divides  \Pi - p_1   p_2   \dots   p_k$ therefore $\tilde{p}  \divides  1$ \cont
\end{proof}


\begin{theorem}[The Fundamental Theorem of Arithmetic]
\label{fundarith}
Every natural number has a unique prime factorization.
\end{theorem}

\begin{proof}
Let $s$ be the least natural number with non-unique prime factorization so that $s = p_1   p_2   \dots   p_n$ and that $s = q_1   q_2   \dots   q_m$ where $p_1$ is not any $q_i$\\
Consider $t = (q_1-p_1)   q_2   \dots   q_m = q_1   q_2   \dots   q_m - p_1   q_2   \dots   q_m$ \\
Therefore, $t = s - p_1 (q_2   \dots   q_m)$ but $p_1  \divides  s$ so $p_1 \divides t$. \\
$t < s$ so $t = (q_1-p_1) (  q_2   \dots   q_m)$ has unique factorization.\\
Because $p_1  \divides  t$ and $p_1$ is not any $q_i$ then $p_1  \divides  q_1-p_1$. \\ Since $p_1  \divides  p_1$ by Lemma~\ref{pmlemma}, $p_1 \divides q_1$ but $p_1 \neq q_1$ so $q_1$ is not prime. \cont \\ By well-ordering, every natural number must have a unique prime factorization.
\end{proof}



\subsection{The Euclidean Property and Modular Residues}

\begin{theorem}[The Euclidean Property]
\label{euclidprop}
Let $a,b \in \N$ then there exist unique $q,r \in \N$ s.t. $a = qb + r$ and $0 \le r < b$
\end{theorem}

\begin{proof}
Define: \[S = \{x \in \N \mid \exists q \in \Z : x = a - bq\}\] \\
Because $a \in S$, $S$ is non-empty so by well-ordering, $S$ has a least element $r$. \\ Since $r \in S$ we know that $r \ge 0$ and for some $q$, $a = bq + r$.\\
Suppose that $r > b$ then, $a- bq > a - bq - b = a-b(q+1) > 0$ \\
so $r - b \in S$ but is less than $r$, the least element. \cont     
\end{proof}

\begin{lemma}
Let $a = nq_a + r_a$ and $b = nq_b+r_n$ s.t. $0\le r_a,r_b < n$\\ then $\mod{a}{b}{n} \iff r_a = r_b$
\end{lemma}

\begin{lproof}
Let $n \divides a-b$ then $n \divides r_a - r_b + n(q_a-q_b)$ so $n \divides r_a-r_b$. \\
However, $r_a - r_b < n$ so $r_a - r_b = 0$. Thus $r_a = r_b$.
\\\\
If $r_a = r_b$ then $a-b = n(q_a-q_b)$ so $n \divides a-b$ i.e. $\mod{a}{b}{n}$. 
\end{lproof}

\begin{definition}
Let $a = qn + r$ s.t. $0 \le r < n$, then $r$ is the residue of $a$ modulo $n$ i.e. $a \: mod \: n = r$. By the above, $a \: mod \: n = b \: mod \: n \iff \mod{a}{b}{n}$.
\end{definition}


\begin{lemma}
The facts of modular arithmetic can be applied directly to the modular residues which are the most reduced form of that number:
\begin{itemize}
\item $((a \: mod \: n) + (b \: mod \: n)) \: mod \: n = (a + b) \: mod \: n$
\item $((a \: mod \: n)   (b \: mod \: n)) \: mod \: n = (a   b) \: mod \: n$
\item $(a \: mod \: n)^k \: mod \: n = a^k \: mod \: n$.
\end{itemize}
\end{lemma}

\begin{lproof}
The statements rely on: $\mod{(a \: mod \: n)}{a}{n}$ and $\mod{(b \: mod \: n)}{b}{n}$ by Lemma~\ref{modadd} we have that $\mod{(a \: mod \: n) + (a \: mod \: n)}{a+b}{n}$ and therefore, $((a \: mod \: n) + (b \: mod \: n)) \: mod \: n = (a + b) \: mod \: n$. The other results can be proved in an identical manner.
\end{lproof}



\newpage

\subsection{Examples of Congruence Calculations}
\textbf{Show:} \quad $13 \divides 3^{45}-1$. \quad $3^{45} = 27^{15}$ and $\mod{27}{1}{13}$ so $ \mod{27^{15}}{1^{15}}{13}$ \\ \indent thus, $\mod{3^{45}}{1}{13}$ i.e $\: 13 \divides 3^{45}-1$
\\\\
\textbf{Show:} \quad $31 \divides 2^{45}-1$. \quad $2^{45} = 32^{9}$ and $\mod{32}{1}{31}$ so $\mod{2^{45}}{1^{9}}{31}$ 
\\\\
\textbf{Show:} \quad $13 \divides 2^{24}-1$. \quad $2^{24} = 16^{6}$ and $\mod{16}{3}{13}$ so $ \mod{2^{24}}{3^{6}}{n}$\\
\indent however, $3^{6} = 27^{2}$ and $\mod{27}{1}{13}$ so $\mod{2^{24}}{1}{13}$
\\\\
\textbf{Show:} \quad $13 \divides 7^{24}-1$. \quad $7^{24} = 49^{12}$ and $\mod{49}{-3}{13}$\\
\indent so $\mod{49^{12}}{(-3)^{12}}{13}$ however, $(-3)^{12} = 81^{4}$ and $\mod{81}{3}{13}$.\\
\indent Thus, $\: 81^{3} \equiv 3^3 \equiv 27 \equiv 1 \: mod \: 13 \:$ so $\mod{7^{24}}{1}{13}$.
\\\\
\textbf{Show:} \quad $28 \divides 3^{92}-9$. \quad $92 = 3 \cdot 30 + 2$ and $\mod{3^3 = 27}{1}{28}$ so \\ \indent $\mod{3^{90}}{1^{30}}{13}$ thus, $\mod{3^90 \cdot 3^2}{9}{28}$ therefore $\mod{3^{92}}{9}{28}$ 
\\\\

\noindent If we write out an integer in base-10 notation, we can use properties of modular arithmetic to test for divisibility or congruence on the digits.\\\\
 Let $n = 10^{k}a_{k} + 10^{k-1}a_{k-1} + \dots + 10^1a_1 + a_0$ \\\\
\textbf{To Test Congruence and Divisibility:}
\begin{enumerate}
\item[By 2:] $\mod{10}{0}{2}$, only the first digit is relevant, $\mod{n}{a_0}{2}$
\item[By 3:] $\mod{10}{1}{3}$ so $\mod{10^k}{1}{3}$ thus $\mod{n}{a_{k} + a_{k-1} + \dots + a_{1}+ a_0}{3}$
\item[By 4:] $\mod{10}{2}{4}$ and $\forall k > 1 : \mod{10^k}{0}{4}$ so $\mod{n}{2a_1 + a_0}{4}$.
\item[By 5:] $\mod{10}{0}{5}$, only the first digit is relevant, $\mod{n}{a_0}{5}$
\item[By 6:] $\forall k \ge 1:\mod{10^k}{4}{6}$, by substitution: $\mod{n}{4(a_{k} + \dots + a_1) + a_0}{6}$.
\item[By 7:] Suppose that $n = 10a + b$ then because $\mod{20}{-1}{7}$, \\$\mod{10a + b}{10a-20b}{7}$ so if $7 \divides a-2b$ then $7 \divides n$ 
\item[By 8:] $\mod{10}{2}{8}$, $\mod{10^2}{4}{8}$, and $\forall k > 2 : \mod{10^k}{0}{8}$.
\\ by substitution: $\mod{n}{4a_2 + 2a_1 + a_0}{8}$.
\item[By 9:] $\mod{10}{1}{9}$ so $\mod{10^k}{1}{9}$ thus $\mod{n}{a_{k} + a_{k-1} + \dots + a_{1}+ a_0}{9}$
\item[By 10:] $\mod{10}{0}{10}$ so $\mod{n}{a_0}{10}$
\item[By 11:] $\mod{10}{-1}{11}$, powers of $10$ are alternately $1$ and $-1$ modulo $10$  \\Thus, $\mod{n}{a_0 - a_1 + a_2 - a_3 + \dots + (-1)^ka_k}{11}$\\
\end{enumerate}

\section{The Greatest Common Divisor}
\subsection{Definition and Properties}
\begin{definition}
Let $a,b \in \Z$ then the greatest common divisor of $a$ and $b$ is a positive integer: $d \in \Zplus$ s.t. $d \divides a$, $d \divides b$, and $c \divides a \wedge c \divides b \implies c \divides d$
\end{definition}

\begin{lemma}
\label{gcduniqe}
The greatest common divisor is unique.
\end{lemma}

\begin{lproof}
Suppose that $d$ and $d^*$ are greatest common divisors of $a$ and $b$. \\
Since $d \divides a$ and $d \divides b$ the definition of $d^*$ implies that $d \divides d^*$. \\ 
Likewise, $d^* \divides a$ and $d^* \divides b$ so $d^* \divides d$ thus $d = d^*$.\\\\
note: because of uniqueness, we can define the function: $gcd(a,b)$ 
\end{lproof}

\begin{definition}
If $gcd(a,b) = 1$ then $a$ and $b$ are coprime, relatively prime, or $a \perp b$
\end{definition}

\begin{lemma}
$\forall a \in \N : a \perp a+1$
\end{lemma}

\begin{lproof}
Let $d = gcd(a,a+1)$ then $d \divides a$ and $d \divides a+1$. Then $d \divides 1$ so $d = 1$.
\end{lproof}

\begin{lemma}
$gcd(a,b) = gcd(a,a \pm b)$
\end{lemma}

\begin{lproof}
Let $d = gcd(a,b)$ then $d \divides a$ and $d \divides b$ so $d \divides a \pm b$. \\
If $c \divides a$ and $c \divides a \pm b$ then $c \divides b$ so $c \divides d$. Thus, $d = gcd(a,a \pm b)$.
\end{lproof}


\begin{lemma}
\label{congruentgcdequal}
if $\mod{a}{b}{n}$ then $gcd(a,n) = gcd(b, n)$
\end{lemma}

\begin{lproof}
Let $d = gcd(a,n)$. Since $n \divides a-b$ then $d \divides a$ and $d \divides n \implies d \divides b$. \\ Also if $c \divides n$ and $c \divides b$ then $c \divides a$ therefore $c \divides d$ so $d = gcd(b,n)$.
\end{lproof}

\begin{corollary}
Since $\mod{(a \: mod \:b)}{a}{b}$ then $gcd(a,b) = gcd(a \: mod \: b, b)$.
\end{corollary}

\noindent \textbf{The Euclidean Algorithm:}
\\
We can use the fact that $gcd(a,b) = gcd(b, a \: mod \: b)$ to find $gcd(a,b)$ recursively.\\ Let $b \le a$ then $a \: mod \: b < b$ so the arguments are reduced in the recursion.\\
The base case is: $gcd(a,0) = a$.\\\\
{\tt gcd(a,b):}
\begin{enumerate}
\item {\tt if $a < b$: return gcd(b, a)}
\item {\tt if $b = 0$: return a}
\item {\tt else: return gcd(b, a mod b)}
\end{enumerate}
This algorithm requires strictly fewer than $2\log_2(a)$ steps even in the worst case. 
\newpage

\begin{theorem}
$gcd(a,b)$ is the least element of \[T_{a,b} = \{n \in \Zplus  \mid  \exists x,y \in \Z : ax+by = n \}\]
\end{theorem}

\begin{proof}
$T_{a,b}$ is non-empty because $a,b \in T_{a,b}$  by well-ordering, $T_{a,b}$ has a least element: $d = ax_0 + by_0$. Use the Euclidean property to write for any $x$ and $y$: $ax+by = (ax_0 + by_0)q + r$ s.t. $0 \le r < d$.  Thus, $r = a(x-x_0) + b(x-y_0)$.\\
If $r > 0$ then $r \in T_{a,b}$ but $r < d$, the least element. \cont \: Thus $r = 0$ and $d \divides ax+by$. 
\\\\
Therefore, $\forall n \in T_{a,b} : d \divides n$. $a,b \in T_{a,b}$ so $d \divides a$ and $d \divides b$. \\ Suppose $c \divides a$ and $c \divides b$ but because $d = ax_0 + by_0$ this implies that $c \divides d$. \\
Thus $d$ satisfies the definition of the greatest common divisor.
\end{proof}

\begin{corollary}
\label{gcdintsol}
If $n = ax + by$ then $gcd(a,b) \divides n$ \\
Furthermore, if $gcd(a,b) \divides n$ then $n = k(ax_0+by_0) = a(kx_0) + b(ky_0)$. \\
Thus $ax+by = c$ has integer solutions iff $gcd(a,b) \divides  c$
\end{corollary}

\begin{corollary}
\label{gcdcoprime}
$a \perp b$ i.e. $gcd(a,b) = 1 \iff \exists x,y \in \Z : ax+by = 1$
\end{corollary}

\subsection{Properties of the Greatest Common Divisor}

\begin{lemma}
\label{coprimeprodsarecoprime}
$a \perp mn$ iff $a \perp m$ and $a \perp n$
\end{lemma}

\begin{lproof}
Let $a \perp m$ and $a \perp n$ then by Corollary~\ref{gcdcoprime}, \\$ax+my = 1$ and $ax'+ny' = 1$. Multiply: $axax'+axny'+myax'+myny' = 1$ thus $a(axx' + xny'+myx') + mnyy' = 1$.
Therefore, integer solutions exist for $ax'' + mny'' = 1$ so by Corollary~\ref{gcdcoprime}, $gcd(a,mn) = 1$
 \\\\ Let $a \perp mn$ i.e. $d \divides a,mn \implies d=1$ then $d \divides a,m \implies d \divides mn \implies d = 1$ i.e. $a \perp m$ similarly, $d \divides a,n \implies d \divides mn \implies d = 1$ i.e. $a \perp n$.
\end{lproof}


\begin{lemma}
\label{coprimeprod}
If $a \perp b$, $a \divides c$, and $b \divides c$ then $ab \divides c$
\end{lemma}

\begin{lproof}
By Corollary~\ref{gcdcoprime}, $ax+by = 1$. Multiply by $c$, $axc + byc = c$. \\
Since $b \divides c$ and $a \divides c$ we know that $ab \divides axc + byc$ so $ab \divides c$
\end{lproof}

\begin{corollary}
\label{coprimeprodmods}
If $m \perp n$ and $\mod{a}{b}{m}$ and $\mod{a}{b}{n}$\\ then $m \divides a-b$ and $n \divides a-b$ so $mn \divides a-b$ i.e. $\mod{a}{b}{mn}$
\end{corollary}


\begin{lemma}[Euclid's Lemma]
\label{euclidlem}
If $n \perp a$ and $n \divides ab$ then $n \divides b$
\end{lemma}

\begin{lproof}
Let $ab = nk$.
By Corollary~\ref{gcdcoprime}, $ax+ny = 1$. Multiply by $b$, $abx + nby = b$. \\
Thus by substitution, $nkx + nby = b$ so $n(kx+by) = b$ and so $n \divides b$
\end{lproof}

\begin{corollary}
Let $p$ be prime then for any $a$, either $p \divides a$ or $p \perp a$.\\ Thus if $p \divides ab$ then either $p \divides a$ or $p \divides b$ (or both)
\end{corollary}

\begin{corollary}
\label{congruencecancelation}
Let $a \perp n$. Suppose $\mod{ab}{ac}{n}$ \\ then $n \divides a(b-c)$ so $a \divides b-c$  i.e. $\mod{b}{c}{n}$
\end{corollary}

\begin{lemma}
\label{coprimeresidues}
Let $d = gcd(a,b)$ then $\tilde{a} \perp \tilde{b}$ where $a = d\tilde{a}$ and $b = d\tilde{b}$
\end{lemma}

\begin{lproof}
By Corollary~\ref{gcdintsol}, $ax + by = d$. Because $d = gcd(a,b)$ we can write $\tilde{a}dx + \tilde{b}dy = d$ therefore $\tilde{a}x + \tilde{b}y = 1$ and so by Lemma~\ref{gcdcoprime}, $\tilde{a} \perp \tilde{b}$
\end{lproof}


\section{Fermat's Little Theorem}
\subsection{Statement of the Theorem}

\begin{theorem}[Fermat's Little Theorem]
If $p$ is prime and $p \ndivides a$ then $p \divides a^{p-1} - 1$
\end{theorem}

\begin{proof}
Consider: $ak$ for $1 \le k < p$: \\ $p \ndivides a$ and $k < p$ i.e $p \ndivides k$ so by Lemma~\ref{euclidlem}, $p \ndivides ak$. Suppose $\mod{ak_1}{ak_2}{p}$ for $1 \le k_1,k_2 < p$ then $p \divides a(k_1-k_2)$ so by Lemma~\ref{euclidlem}, $p \divides k_1-k_2$ However, $k_1 - k_2 < p$ so $k_1 = k_2$. By Theorem~\ref{euclidprop}, $ak = pq + r$ for $0 \le r < p$ and since $p \ndivides ak$ we know that $r \neq 0$ and $\mod{ak}{r}{p}$. Each $ak$ matches to a unique $r$. Thus $\mod{1a \cdot  2a \cdot \ldots  \cdot (p-1)a}{1  \cdot 2 \cdot  \ldots  \cdot (p-1)}{p}$. Rewrite this as: \\ $\mod{(p-1)!   a^{p-1}}{(p-1)!}{p}$ i.e. $p \divides (p-1)!   (a^{p-1}-1)$. 
By Lemma~\ref{euclidlem}, \\ $p \divides (p-1)!$ or $p \divides a^{p-1} - 1$.  If $p \divides (p-1)!$ since $p \ndivides p-1$ then $p \divides (p-2)!$ ect \cont \\ Thus, $p \divides a^{p-1} - 1$   
\end{proof}

\begin{corollary}
\label{fermatlittlealternate}
If $p$ is a prime then $p \divides a^p - a$ for every $a \in \Z$
\end{corollary}

\begin{lproof}
Suppose that $p \ndivides a$ then by Fermat, $p \divides a^{p-1} - 1$ so $p \divides a^p - a$.\\
Otherwise, $p \divides a$ so $p \divides a(a^{p-1} - 1)$ i.e. $p \divides a^p - a$
\end{lproof}

\subsection{Examples and Applications}

\begin{lemma}
Let $p$ be prime and $p \ndivides a$. If $p - 1 \divides n$ then $p \divides a^n - 1$
\end{lemma}

\begin{lproof}
Let $p - 1 \ndivides n$ then $k(p-1) = n$. Also, $p \ndivides a$ so by Fermat's Little Theorem, $\mod{a^{p-1}}{1}{p}$. Thus, $\mod{a^{k(p-1)}}{1}{p}$ so $p \divides a^n - 1$
\end{lproof}

\begin{definition}[Example:]
Find prime divisors of $2^{64} - 1$ \\ Factor $64: 1,2,4,8,16,32,64 \xrightarrow{+1} 2, 3, 5, 9, 17, 33, 65$. Select those factors that are one less than a prime. By the previous lemma, if a factor of $64$ is one less than a prime then that prime divides $2^{64} - 1$ Thus $3,5,17$ are prime factors of $2^{64} - 1$. \\ (Not $2$ because it divides the base)
\end{definition}

\begin{definition}[Example:]
Find prime divisors of $10^{234} - 1$ \\ Factor $234: 1,2,3,6,9,13,18,26,39,78,117,234 \xrightarrow{+1} 2,3,4,7,10,14,19,27,40,79,118,235$. Select those factors that are one less than a prime and do not divide $10$. Thus, $3,7,19,79$ are prime factors of $10^{234} - 1$ 
\end{definition}

\begin{definition}[Example:]
Evaluate $17^{361} \: mod \: 7$ \\
Write: $361 = 60 \cdot 6 + 1$ thus, $17^{361} = 17 \cdot (17^6)^{60}$. By Fermat's Little Theorem, $17^6 \: mod \: 7 = 1$. Therefore, $17^{361} \: mod \: 7 = 17 \cdot 1 = 3$
\end{definition}

\begin{definition}[Composite Test:]
either proves that a number is composite or is inconclusive. \\
Suppose that $n$ is prime. Choose an $a$ that not divisible by $n$ then by Fermat's Little Theorem, $n \divides a^{n-1} - 1$. Thus if $n \ndivides a^{n-1} - 1$ then $n$ is composite. If $n \divides a^{n-1} - 1$ often $n$ would have to be prime. However, as we will soon see this is not always the case. 
\end{definition}


\newpage
\subsection{Fermat Pseudoprimes}

\begin{definition}
$n$ is a Fermat pseudoprime for base $a$ if $n \divides a^{n-1} - 1$ and $n$ is composite. 
\end{definition}

\noindent\textbf{Examples:}
\begin{itemize}
\item $91 = 7  \cdot 13$ and $7,13 \divides 3^{90} -1$. However, $7 \perp 13$ so $7 \cdot 13 \divides 3^{90} - 1$

\item $341 = 11  \cdot 31$ and $11,31 \divides 2^{340} - 1$. However, $11 \perp 31$ so $11 \cdot  31 \divides 2^{340} - 1$

\item Let $\mod{a}{1}{n}$ or if $n$ is odd $\mod{a}{-1}{n}$ then $\mod{a^{n-1}}{1}{n}$

\end{itemize}

\begin{theorem}
There exist infinitely many Fermat pesudoprimes for every base. 
\end{theorem}

\begin{proof}
For any base $a$ we will construct a number $m$ such that $m \divides a^{m-1} - 1$
Choose an odd prime $p$ such that $p \ndivides a$ and $p \ndivides a^2 - 1$.\\ Define: \[m = \frac{a^{2p} - 1}{a^2 - 1} \]\\
Claims:
\begin{enumerate}
\item $m \in \Zplus$. 
\begin{lproof}
$\mod{a^2}{1}{(a^2-1)}$ so $\mod{a^{2p}}{1}{(a^2-1)}$
\end{lproof}

\item $2p \divides m-1$. 
\begin{lproof}
$m(a^2 - 1) = a^{2p} - 1$ so $(m-1)(a^2-1) = a^{2p} - a^2$ \\
factor this as $a^2(a^{p-1}-1)(a^{p-1}+1)$. $p \ndivides a$ so by Fermat, $p \divides a^{p-1} - 1$. \\
$p$ is odd so $p-1 = 2k$ thus $ \mod{a^{2k}}{1}{(a^2-1)}$ so $a^2-1 \divides a^{p-1}-1$. \\
Because $p \perp a^2-1$ by Lemma~\ref{coprimeprod}, $p(a^2-1) \divides a^{p-1} - 1$. \\
Either $a^2$ or $a^{p-1} + 1$ is even so $2 \divides a^2(a^{p-1} + 1)$.\\ Thus $2p(a^2-1) \divides a^2(a^{p-1}-1)(a^{p-1}+1) = (m-1)(a^2-1)$ so $2p \divides m-1$
\end{lproof}


\item $m \divides a^{m-1} - 1$.

\begin{lproof}
$m(a^2-1) = a^{2p} - 1$ so $\mod{a^{2p}}{1}{m}$. \\ Because $2p \divides m-1$ by Lemma~\ref{modpower}, $\mod{a^{m-1}}{1}{m}$
\end{lproof}

\item $m$ is composite.

\begin{lproof}
\[m = \frac{a^{2p} - 1}{a^2 - 1} = \frac{a^{2} - 1}{a - 1}  \cdot \frac{a^{p} + 1}{a^2 + 1}\]
Since $\mod{a}{\mp 1}{a \pm 1}$ and $p$ is odd, $\mod{a^p}{\mp 1}{a \pm 1}$ thus $a \pm 1 \divides a^p \pm 1$ so both factors are in $\Zplus$.
\end{lproof}
\end{enumerate}
Since any prime greater than $a^2 - 1$ with satisfy the conditions and produce a unique pseudoprime $m$, there are infinitely many pseudoprimes for each base $a$.
\end{proof}


\section{Divisor Sums and Perfect Numbers}

\begin{lemma}
\label{primepowdivision}
Let $p$ be a prime and let $p \divides a^n$ then $p \divides a$.
\end{lemma}

\begin{lproof}
Base Case: if $p \divides n^1$ then $p \divides n$. \\
Assume: $p \divides a^n \implies p \divides n$. \\ 
If $p \divides a^{n+1}$ then by Lemma~\ref{euclidlem}, either $p \divides n$ or $p \divides a^n$.\\ 
By assumption, $p \divides a^n \implies p \divides a$. Thus either way, $p \divides a$. \\
Therefore, $p \divides a^{n+1} \implies p \divides a$ and the result holds by induction.
\end{lproof}

\begin{theorem}
Let $n = p_1^{a_1}   p_2^{a_2}   \dots   p_k^{a_k}$ with distinct $p_i \in \Primes$ then:
\begin{enumerate}
\item Let $q \in \Primes$ and $q \divides n$ then $q$ is some $p_i$ i.e. prime factorizations are unique.
\item the number of divisors of $n$ is $\prod\limits_{i=1}^{k} (a_i+1)$
\end{enumerate}
\end{theorem}

\begin{proof}
Suppose that $q \divides p_1^{a_1}   p_2^{a_2}   \dots   p_k^{a_k}$ then by repeated application of Lemma~\ref{euclidlem}, $q \divides p_i^{a_i}$ for some $i$. By Lemma~\ref{primepowdivision}, $q \divides p_i$ but $q, p_i \in \Primes$ so $q = p_i$. 
\\\\
If $d \divides n$ and $q \divides d$ then $q$ is some $p_i$. Therefore, the prime factorization of $d$ must have the same distinct primes as $n$ but may have different powers. \\ Thus, $d = p_1^{b_1}   p_2^{b_2}   \dots   p_k^{b_k}$ where $\forall i$ : $ 0 \le b_i \le a_i$. There are $a_i + 1$ possibilities for the power $b_i$. Therefore, the number of distinct divisors is: $\prod\limits_{i=1}^{k} (a_i+1)$.  
\end{proof}

\begin{definition}
$\sigma (n)$ is the sum of the positive divisors of n. i.e. $\sigma (n) = \sum\limits_{d \divides n} d$.
\end{definition}

\begin{lemma}
If $a \perp b$ then $\sigma (a \cdot b) = \sigma (a) \cdot \sigma (b)$
\end{lemma}

\begin{lproof}
write the divisors of $a$ as $\{d_1, d_2, \dots , d_k\}$ and those of $b$ as $\{f_1, f_2, \dots ,f_r\}$. \\
Let $s \divides ab$ then by Lemma~\ref{coprimeresidues} we write $g = gcd(s, a)$ and $\tilde{s} \perp \tilde{a}$ and $\tilde{s}g \divides \tilde{a}gb$. \\ Thus, $\tilde{s} \divides \tilde{a}b$ and 
by Lemma~\ref{euclidlem}, $\tilde{s} \divides b$. Since $g \divides a$ and $\tilde{s} \divides b$ and $s = \tilde{s}g$ then $s = d_i   f_j$. Thus every factor of $ab$ is a factor of $a$ multiplied by a factor of $b$.  
\\\\
Suppose that $d_i   f_j = d_r   f_s$. Let $k \divides d_i$ and $k \divides f_s$ then $k \divides a$ and $k \divides b$ but because $a \perp b$ then $k = 1$. Therefore, $d_i \perp f_s$. Similarly, $d_r \perp f_j$. Since $d_i \divides d_r   f_s$ and $d_r \divides d_i   f_j$. By Lemma~\ref{euclidlem}, $d_i \divides d_r$ and $d_r \divides d_i$ so $d_i = d_r$. Similarly, $f_j = f_s$. \\ Therefore, no two products of divisors are equal. \\ So each product of divisors of $a$ and $b$ is a unique divisor of $ab$.
\\\\
We can write: $\sigma (ab) = d_1   f_1 + d_2   f_1 + \dots + d_k   f_1 + d_1   f_2 + \dots + d_k   f_r$. \\
Which factors as: $\sigma (ab) = (d_1 + d_2 + \dots + d_k)   (f_1 + f_2 + \dots + f_r) = \sigma (a)   \sigma (b)$
\end{lproof}

\begin{corollary}
Let $n = p_1^{a_1}  p_2^{a_2}  \dots   p_k^{a_k}$ with distinct $p_i \in \Primes$ then \\ $\sigma (n) = \sigma (p_1^{a_1})   \sigma (p_2^{a_2})   \dots   \sigma (p_k^{a_k})$. Also, $\sigma (p_i^{a_i}) = 1 + p_i + p_i^2 + \dots + p_i^{a_i} = \frac{p^{a_i + 1} - 1}{p_i - 1}$. \\
Therefore, \[\sigma (n) = \prod\limits_{i = 1}^{k} \frac{p_i^{a_i + 1} - 1}{p_i - 1} \]
\end{corollary}

\begin{definition}
The divisor sum function partitions $\Zplus$ into three sets:
\begin{enumerate}
\item $n$ is \textit{abundant} if $\sigma (n) > 2n$
\item $n$ is \textit{perfect} if $\sigma (n) = 2n$
\item $n$ is \textit{deficient} if $\sigma (n) < 2n$
\end{enumerate}
\end{definition}

\begin{proposition}
Let $p, q \in \Primes$ then $p^k$ and (if $p,q > 2$) then $pq$ are deficient.  
\end{proposition}

\begin{lproof}
$2(p-1)p^k - p^{k+1} + 1 = p^{k+1} - 2p^k + 1 > 0$ so $2(p-1)p^k > p^{k+1} - 1$.\\ Therefore, $2p^k > (p^{k+1} - 1)/(p-1)$ i.e. $2p^k > \sigma (p^k)$. If $p,q > 2$ then \\ $pq - p - q - 1 > 0$ so $2pq > pq + p + q + 1 = (p+1) (q+1)$ i.e. $2pq > \sigma (pq)$.\\
Thus there exist infinitely many both even and odd deficient numbers.
\end{lproof}

\begin{lemma}
\label{abundantmultiples}
proper multiples of abundant and perfect numbers are abundant. \end{lemma}

\begin{lproof}
Suppose $\sigma (n) > 2n$. For every $d \divides n$, and for any multiple by $a$, $ad \divides an$.\\ Thus $\sigma (an) \ge a \sigma (n) > 2an$ so $an$ is abundant. \\\\
Suppose that $\sigma (n) = 2n$. For every $d \divides n$, and for any proper multiple by $a$, $ad \divides an$. Furthermore, $a > 1$ so no $ad = 1$. Thus $\sigma (an) \ge a \sigma (n) + 1 > 2an$.
\end{lproof}

\begin{corollary}
Because $945$ is an odd abundant number, any multiple of $945$ by an odd number is abundant and still odd. Therefore there are infinitely many odd abundant numbers. Furthermore, any multiple of $945$ by an even number is even and abundant. Therefore there are infinitely many even abundant numbers.
\end{corollary}

\begin{corollary}
proper divisors of deficient and perfect numbers are deficient.  
\end{corollary}

\begin{lproof}
Let $a$ be a proper divisor of $n$. Suppose that $a$ is not deficient. \\ Therefore it is perfect or abundant. But $n$ is a proper multiple of $a$ so by Proposition~\ref{abundantmultiples}, $n$ is abundant. The contrapositive of this statement is: \\
If $n$ is not abundant then its proper divisors are deficient. 
\end{lproof}

\begin{theorem}[Euclid-Euler Theorem]
$n$ is an even perfect number iff \\ $n = 2^{k}(2^{k+1}-1)$ where $2^{k+1}-1$ is prime.
\end{theorem}

\begin{proof}
$2^k \perp 2^{k+1}-1$ so
$\sigma (2^{k}(2^{k+1}-1)) = \sigma (2^k)   \sigma (2^{k+1}-1)$ but since $2^{k+1} - 1$ is prime, $\sigma (2^k)   \sigma (2^{k+1}-1)  = (2^{k+1} - 1)   (2^{k+1} - 1 + 1) = 2^{k+1} (2^{k+1} - 1) = 2n$.\\

\noindent Let $n = 2^km$ be perfect and $2^k \perp m$. then $\sigma (2^km) = (2^{k+1} - 1)   \sigma (m)$ \\
$n$ is perfect so $\sigma(2^km) = 2^{k+1}m$ and thus $(2^{k+1} - 1)   \sigma (m) = 2^{k+1}m$. \\
$2^{k+1} - 1 \divides 2^{k+1}m$ but $2^{k+1} - 1 \perp 2^{k+1}$ so $2^{k+1} - 1 \divides m$, write: $m = r(2^{k+1} - 1)$ \\
Thus, $(2^{k+1} - 1)   \sigma (m) = 2^{k+1}r(2^{k+1} - 1)$ so $\sigma (r(2^{k+1} - 1)) = 2^{k+1}r$. \\
Rewrite this as $\sigma (r   (2^{k+1} - 1)) = r + r   (2^{k+1}-1)$. Both terms divide $r   (2^{k+1} - 1)$ and if $n$ is even i.e. $k > 0$ the two divisors are distinct. Therefore, the only two divisors of $r   (2^{k+1} - 1)$ are $r$ and $2^{k+1} - 1$. Since $1$ is a divisor of every number, $r = 1$ and thus $2^{k+1} - 1$ has only two factor so $2^{k+1} - 1$ is prime. 
Recombining, $n = 2^k   (2^{k+1} - 1)$.
\end{proof}

\noindent Primes of the form $2^{r} -1$ are known as Mersenne primes. There is a one-to-one correspondence between perfect numbers and Mersenne primes.

\section{Linear Congruences}

Consider congruences of the form: $\mod{ax}{b}{n}$. 

\begin{lemma}
\label{congsolutions}
$\exists x \in \Z$ s.t. $\mod{ax}{b}{n}$ iff $gcd(a,n) \divides b$
\end{lemma}

\begin{lproof}
Suppose $\mod{ax}{b}{n}$ then $ax - b = nk$ for some $k \in \Z$. \\
Therefore, $ax - nk = b$ so $b \in T_{a,b}$ and by Corollary~\ref{gcdintsol}, $gcd(a,n) \divides b$ \\\\
Let $gcd(a,n) \divides b$ then by Corollary~\ref{gcdintsol}, $\exists x,y \in \Z$ s.t. $ax + ny = b$. \\
Therefore, $ax - b = -ny$ so $\mod{ax}{b}{n}$. 
\end{lproof}

\begin{lemma}
\label{numsoltocongruence}
Let $g = gcd(a,n)$ then if $g \divides b$ there exist exactly $g$ solutions up to congruence modulo $n$ to the equation $\mod{ax}{b}{n}$.
\end{lemma}

\begin{lproof}
By Lemma~\ref{coprimeresidues}, $\tilde{a} \perp \tilde{n}$. Also by Lemma~\ref{congsolutions} there exists a solution $x_0$.  \\
Suppose that $\mod{ax}{b}{n}$ has two solutions $x_1,x_2$ then $\mod{ax_1}{ax_2}{n}$. \\ Therefore, $n \divides a(x_1-x_2)$ so $g\tilde{n} \divides g\tilde{a}(x_1-x_2)$ and thus $\tilde{n} \divides \tilde{a}(x_1-x_2).$ \\ By Lemma~\ref{euclidlem}, $n \divides x_1 - x_2$ and thus $\mod{x_1}{x_2}{\tilde{n}}$ and so there is exactly one solution modulo $\tilde{n}$, $x_0$ which is reduced modulo $\tilde{n}$. \\
Every solution must be of the form: $x = x_0 + \tilde{n}k$. This is a solution for any k. $a(x_0 + \tilde{n}k) = \mod{ax_0 + \tilde{a}nk}{ax_0}{n}$. Since $x_0 < \tilde{n}$, so $x < n = \tilde{n}g$ iff $k < g$. \\
Therefore, if $0 \le k < g$ then $x_0 + \tilde{n}k$ is a distinct solution modulo $n$. \\ If $\mod{x_0 + \tilde{n}k_1}{x_0 + \tilde{n}k_2}{n}$ then $\tilde{n}g \divides \tilde{n}(k_1-k_2)$ so $g \divides k_1 - k_2$. \\ Therefore, there are exactly $g$ solutions.
\end{lproof}

\subsection*{Systems of Linear Congruences}

\begin{theorem}[Chinese Remainder Theorem]
If $n_1, n_2, \dots , n_k$ are pairwise coprime then the system: $\mod{x}{a_1}{n_1}$, $\mod{x}{a_2}{n_2}$, \dots ,$\mod{x}{a_k}{n_k}$ has a unique solution modulo $n_1   n_2   \dots   n_k$
\end{theorem}

\begin{proof}
Let $N_i = n_1   n_2   \dots   n_{i-1}   n_{i+1}   \dots   n_k$. Suppose $\forall i,j : n_i \perp n_j$. \\ Therefore, $N_i \perp n_i$. Since $1 \divides a_i$ by Lemma~\ref{congsolutions}, $\exists x_i \in \Z$ s.t. $\mod{N_ix_i}{a_i}{n_i}$ \\
However, $\mod{N_i}{0}{n_j}$ for all $i \neq j$. Let $x = N_1x_1 + N_2x_2 + \dots + N_kx_k$. \\
Therefore, for each $i$, $\mod{x}{a_i}{n_i}$ and thus a solution exists. \\
Suppose that $x_1$ and $x_2$ solve the system. Then for each $i$, $\mod{x_1}{x_2}{n_i}$. \\
Because $n_i \perp n_j$ by Lemma~\ref{coprimeprodmods}, $\mod{x_1}{x_2}{n_1   n_2   \dots   n_k}$. 
\end{proof}

\begin{corollary}
Suppose that $\forall i$ : $a_i \perp n_i$ and $\forall i \neq j$ : $n_i \perp n_j$ then the system: $\mod{a_1x}{b_1}{n_1}$, $\mod{a_2x}{b_2}{n_2}$, \dots ,$\mod{a_kx}{b_k}{n_k}$ has a unique solution modulo
$n_1   n_2   \dots   n_k$
\end{corollary}

\begin{lproof}
$a_i \perp n_i$ so by Lemma~\ref{congsolutions} there exists $z_i$ s.t. $\mod{a_iz_i}{1}{n_i}$. Thus, \\ $\mod{xa_iz_i}{x}{n_i}$. Plugging into the system, $\mod{x}{b_iz_i}{n_i}$. Since the moduli are pairwise coprime, by the Chinese Remainder Theorem, there is a unique solution modulo $n_1   n_2   \dots   n_k$ to the new system. If $x$ solves the new system: $\mod{x}{b_iz_i}{n_i}$ then multiplying by $a_i$ gives $\mod{a_ix}{b_i}{n_i}$.
\end{lproof}

\begin{corollary}
There are arbitrarily large gaps between primes. 
\end{corollary}

\begin{lproof}
Let $p_i$ be the $i^{th}$ prime. By the Chinese Remainder Theorem: \\
$\mod{x}{-1}{p_1}$, $\mod{x}{-2}{p_2}$, \dots ,$\mod{x}{-k}{p_k}$ has a solution modulo $p_1 p_2 \dots p_k$. Then for each $i$, $p_i \divides x+i$. However, no $p_i = x+i$ else $p_{i+1} \divides p_i + 1$. A contradiction for $k > 2$. Each $x+i$ is composite so each solution begins a gap of atleast size $k$ but $k$ can be arbitrarily large.
Alternatively, since $2,3,4, \dots ,n \divides n!$ then $2 \divides n!+2$, $3 \divides n!+3$, \dots ,$n \divides n!+n$ and thus $n! + 2$, $n!+3$, \dots , $n!+n$ are composite  comprising a prime gap of at least size $n-1$ for arbitrarily large $n$.
\end{lproof}

\begin{lemma}
\label{splitcongintoprimes}
Let $n = p_1^{a_1} p_2^{a_2} \dots   p_k^{a_k}$ \\ then $\mod{x}{y}{n}$ iff $\mod{x}{y}{p_1^{a_1}} \wedge \mod{x}{y}{p_2^{a_2}} \wedge \dots \wedge \mod{x}{y}{p_k^{a_k}}$.
\end{lemma}

\begin{lproof}
Suppose $\mod{x}{y}{n}$ then $n \divides x-y$ and therefore, $p_1^{a_1}   p_2^{a_2}   \dots   p_k^{a_k} \divides x-y$. \\
Thus, for each $i$, $p_i^{a_i} \divides x-y$ so $\mod{x}{y}{p_i^{a_i}}$.\\\\
Suppose that $\mod{x}{y}{p_1^{a_1}} \wedge \mod{x}{y}{p_2^{a_2}} \wedge \dots \wedge \mod{x}{y}{p_k^{a_k}}$. Since, $p_i \perp p_j$ for $i \neq j$, we have by Corollary~\ref{coprimeprodmods} that $\mod{x}{y}{p_1^{a_1}   p_2^{a_2}   \dots   p_k^{a_k}}$ and therefore, $\mod{x}{y}{n}$.

\end{lproof}

\begin{theorem}
The system: $\mod{x}{a_1}{n_1}$, $\mod{x}{a_2}{n_2}$, \dots ,$\mod{x}{a_k}{n_k}$ has a solution iff $gcd(n_i,n_j) \divides a_i - a_j$ for each $i$ and $j$.
\end{theorem}

\begin{proof}
Suppose the system has a solution $x$. Then for any two congruences we can write: $x = a_i + n_i k_j$ and $x = a_j + n_i k_j$. Therefore, $a_i - a_j = n_jk_j - n_ik_i$. By Corollary~\ref{gcdintsol}, $gcd(n_i,n_j) \divides a_i - a_j$.\\

\noindent Let $\forall i,j$: $gcd(n_i,n_j) \divides a_i - a_j$ and $\{p \in \Primes \mid \exists i$ s.t. $p \divides n_i\} = \{p_1, p_2, \dots, p_r\}$. \\ Let $n_i = p_1^{b_{i,1}}   p_2^{b_{i,2}}   \dots   p_r^{b_{i,r}}$. Where some $b_{i,j}$ may be zero if $p_j \ndivides n_i$. Consider: \\\\
$\begin{matrix}
\mod{x}{a_1}{p_1^{b_{1,1}}} & \mod{x}{a_1}{p_2^{b_{1,2}}} & \dots & \mod{x}{a_1}{p_r^{b_{1,r}}} \\

\mod{x}{a_2}{p_1^{b_{2,1}}} & \mod{x}{a_2}{p_2^{b_{2,2}}} & \dots & \mod{x}{a_2}{p_r^{b_{2,r}}} \\

\vdots & \vdots & \ddots & \vdots \\

\mod{x}{a_k}{p_1^{b_{k,1}}} & \mod{x}{a_k}{p_2^{b_{k,2}}} & \dots & \mod{x}{a_k}{p_r^{b_{k,r}}} 
\end{matrix}$ \\

\noindent Assuming that $p_j^{b_{m,j}}$ is the largest power of $p_j$ in row $j$ for any $p_j^{b^{i,j}}$, we have that $b^{i,j} < b^{m,j}$ and thus $\mod{x}{a_m}{p_j^{b_{i,j}}}$. Since $\mod{a_m}{a_i}{gcd(n_m,n_i)}$ and $p_j^{b^{i,j}} \divides n_i$ and since $b^{i,j} < b^{m,j}$ also $p_j^{b^{i,j}} \divides n_m$. Therefore, $p_j^{b^{i,j}} \divides gcd(n_i,n_m)$ so $\mod{a_m}{a_i}{p_j^{b^{i,j}}}$. By transitivity, $\mod{x}{a_i}{p_j^{b_{i,j}}}$ so if $x$ solves $\mod{x}{a_m}{p_j^{b_{m,j}}}$ then $x$ solves any $\mod{x}{a_i}{p_j^{b_{i,j}}}$ we can replace the column with the congruence with the largest power. Since in the reduced system each prime appears in only one equation (the one derived from that prime's column) the moduli are  parwise coprime. By the Chinese Remainder Theorem, there is a solution to the reduced system. Therefore there is a solution to the full system. Since by Lemma~\ref{splitcongintoprimes} row $i$ of the full solution is solved iff $\mod{x}{a_i}{n_i}$ and the full system has a solution so the original system must also be solved by $x$.
\end{proof}

\section{Euler's $\mathbf{\Phi}$ Function and Primitive Roots}

\subsection{Euler's Function and Theorem}

\begin{definition}
$\Phi (n) = \{x \in \N \mid x < n \wedge x \perp n \}$ and $\phi (n) = | \Phi (n) |$
\end{definition}

\begin{lemma}
\label{coprimephimuliplicative}
if $a \perp b$ then $\phi (a \cdot b) = \phi (a) \cdot \phi (b)$
\end{lemma}

\begin{lproof}
Consider the system: $\mod{x}{k}{a}$ and $\mod{x}{g}{y}$. \\ Since $a \perp b$, by the Chinese Remainder Theorem there exists a solution less than $ab$. \\ Suppose the solution $x$ solves a different system $\mod{x}{k'}{a}$ and $\mod{x}{g'}{y}$.\\Then by transitivity, $\mod{k}{k'}{a}$ and $\mod{g}{g'}{b}$ so no two system have the same solution so each reduced pair $(g,k)$ corresponds to exactly one $x < ab$. \\ By Lemma~\ref{congruentgcdequal} $k \perp a$ and $g \perp b$ is equivalent to $x \perp a$ and $x \perp b$. \\ Also, by Lemma~\ref{coprimeprodsarecoprime}, $x \perp a$ and $x \perp b$ is equivalent to $x \perp ab$. \\ Therefore, $k \perp a$ and $g \perp b\iff x \perp ab$. There are $\phi (ab)$ such $x$ and $\phi (a) \phi(b)$ such pairs $(k,g)$ so $\phi (a \cdot b) = \phi (a) \cdot \phi (b)$. 
\end{lproof}

\begin{theorem}
\label{phisumoffactors}
$\sum \limits_{d \divides n} \phi (d) = n$
\end{theorem}

\begin{proof}
Define: \[A_g = \{a \in \Zplus \mid a \le n \wedge gcd(a,n) = g \} \] By Lemma~\ref{coprimeresidues}, $a \in A_g$ iff $\tilde{a} \perp \tilde{n}$ and $\tilde{a} \le \tilde{n}$ where $\tilde{a}g = a$, $\tilde{n}g = n$.\\ Thus, $|A_g| = \phi (\tilde{n})$. Alternatively, for $d \divides n$ we have that $\phi (d) = |A_{\frac{n}{d}}|$. \\ Each number has exactly one $gcd$ so the sets $A_g$ for each $g \divides n$ partition $\{1,2, \cdots , n\}$. \\ Therefore, $\sum \limits_{g \divides n} |A_g| = n$ and so $\sum \limits_{d \divides n} \phi (d) = n$
\end{proof}

\begin{corollary}
\label{phiofprime}
If $p \in \Primes$ then for $k \in \Zplus$, $\phi (p^k) = p^k - p^{k-1}$
\end{corollary}

\begin{lproof}
$p^k$ has factors $1,p,p^2, \cdots ,p^k$ and so by Theorem~\ref{phisumoffactors}, \\$p^k = \phi (p^k) + \phi (p^{k-1}) + \cdots + \phi (1)$ and $p^{k-1} = \phi (p^{k-1}) + \phi (p^{k-2}) + \cdots + \phi (1)$. \\ Therefore, $p^k - p^{k-1} = \phi (p^k)$.
\end{lproof}

\begin{corollary}
\label{primeiffphi}
$p$ is prime iff $\phi(p) = p - 1$
\end{corollary}

\begin{lproof}
Suppose that $\phi (n) = n - 1$ and $n = ab$ where $a < n$ then because every $0 < k < n$ must be coprime to $n$ because $\phi(n) = n - 1$, then $a \perp n$. However, $a \divides a$ and $a \divides n$ thus $a = 1$ so $n$ is prime. By Corollary~\ref{phiofprime}, if $p$ is prime then $\phi(p) = p - 1$
\end{lproof}

\begin{proposition}
\label{phiformula}
$\phi (n) = n \displaystyle\prod \limits_{p \divides n} \bigg(1-\cfrac{1}{p}\bigg)$ for $p \in \Primes$
\end{proposition}

\begin{lproof}
Let $n = p_1^{a_1}   p_2^{a_2}   \dots   p_k^{a_k}$ for distinct primes. Then by Lemma~\ref{coprimephimuliplicative}, \\ $\phi (n) = \phi (p_1^{a_1}) \phi (p_2^{a_2}) \dots \phi (p_k^{a_k})$. Also, by Corollary~\ref{phiofprime}, $\phi (p_i^{a_i}) = p_i^{a_i} - p_i^{a_i-1} =$ \\ $ p_i^{a_i}\bigg(1 - \cfrac{1}{p_i}\bigg)$. Thus, $\phi (n) =  p_1^{a_1}   p_2^{a_2}   \dots   p_k^{a_k} \cdot \bigg(1 - \cfrac{1}{p_1}\bigg)\bigg(1 - \cfrac{1}{p_2}\bigg) \dots \bigg(1 - \cfrac{1}{p_k}\bigg)$. \\
Because $n = p_1^{a_1}   p_2^{a_2}   \dots p_k^{a_k}$ then $\phi (n) = n \bigg(1 - \cfrac{1}{p_1}\bigg)\bigg(1 - \cfrac{1}{p_2}\bigg) \dots \bigg(1 - \cfrac{1}{p_k}\bigg)$
\end{lproof}


\newpage

\begin{corollary}
\label{phidivison}
$\phi (ab) = \phi (a) \phi (b) \cfrac{g}{\phi(g)}$ where $g = gcd(a,b)$
\end{corollary}

\begin{lproof}
If $p \divides ab$ then by Lemma~\ref{euclidlem}, $p \divides a$ or $p \divides b$ or both. Thus to remove double counting, $\phi (ab) = ab \displaystyle\prod \limits_{p \divides ab} \bigg(1-\cfrac{1}{p}\bigg) = a \prod \limits_{p \divides b} \bigg(1-\frac{1}{p}\bigg) \cdot b \prod \limits_{p \divides b} \bigg(1-\frac{1}{p}\bigg) \cdot \prod \limits_{p \divides a,b} \bigg(1-\frac{1}{p}\bigg)^{-1}$. \\
However, $\displaystyle\cfrac{\phi (g)}{g} =  \displaystyle\prod \limits_{p \divides a,b} \bigg(1-\cfrac{1}{p}\bigg)$ so $\phi (ab) = \phi (a) \phi (b) \cfrac{g}{\phi(g)}$.
\end{lproof}

\begin{lemma}
If $a \divides b$ then $\phi (a) \divides \phi (b)$
\end{lemma}

\begin{lproof}
Let $b = ak$ then by Lemma~\ref{phidivison}, $\phi (b) = \phi (a) \phi (k) \cfrac{g}{\phi(g)}$ \\ Thus, $\phi (b) = \phi (a) \: k \displaystyle\prod \limits_{p \divides k \wedge p \ndivides a} \bigg(1-\cfrac{1}{p}\bigg)$ and therefore, $\phi (a) \divides \phi (b)$
\end{lproof}

\begin{corollary}
\label{phieven}
If $n > 2$ then $\phi (n)$ is even.
\end{corollary}

\begin{lproof}
Suppose $p \divides n$ where $p$ is an odd prime. \\ Then $\phi (p) \divides \phi (n)$.  However, $\phi (p) = p-1$ is even so $\phi (n)$ is even. \\ Else, $n = 2^k$ so $\phi (n) = 2^k - 2^{k-1}$ which is even if $k > 1$ i.e. $n > 2$.
\end{lproof}

\begin{theorem}[Euler's Theorem]
if $a \perp n$ then $\mod{a^{\phi (n)}}{1}{n}$
\end{theorem}

\begin{proof}
Enumerate the elements of $\Phi (n)$ as $b_1, b_2, \cdots , b_{\phi (n)}$. Let $a \perp n$. Suppose that $\mod{ab_i}{ab_j}{n}$ then by Corollary~\ref{congruencecancelation}, $\mod{b_i}{b_j}{n}$ so the elements $ab_i$ are distinct. $a \perp n$ and $b_i \perp$ so by Lemma~\ref{coprimeprodsarecoprime}, $ab_i \perp n$. Thus $(ab_i \: mod \: n) \perp n$ so $ab_i \: mod \: n \in \Phi (n)$. Therefore, $\exists b_j \in \Phi (n)$ s.t. $\mod{ab_i}{b_j}{n}$. Since the $ab_i$ are distinct modulo $n$, $ab_1, ab_2, \cdots, ab_{\phi (n)}$ is a permutation of $b_1, b_2, \cdots , b_{\phi (n)}$ modulo $n$. \\
In particular, $\mod{ab_1 ab_2 \cdots ab_{\phi (n)}}{b_1 b_2 \cdots b_{\phi (n)}}{n}$. And thus, \\ $\mod{a^{\phi (n)}b_1 b_2 \cdots b_{\phi (n)}}{b_1 b_2 \cdots b_{\phi (n)}}{n}$.  Since every $b_i \perp n$, \\ by repeated application of Corollary~\ref{congruencecancelation}, $\mod{a^{\phi (n)}}{1}{n}$.
\end{proof}

\noindent Fermat's Little Theorem is the special case of Euler's Theorem \\ where if $p$ is prime then $\phi (p) = p-1$ and $p \perp a$ whenever $p \ndivides a$.

\begin{definition}[RSA Encryption]
Select large distinct primes $p$ and $q$ and let the modulus $n = pq$. Generate an encryption key $e$ coprime with $\phi (n)$. Next, Lemma~\ref{gcdintsol} implies that a decryprion key $d$ exists such that $ed = \phi (n)r + 1$ for $d,r > 0$. \\ Publish the encryption key and modulus but keep the decryption key secret. \\
To send a message, use a standard scheme to code that message as an integer $m \in \Phi (n)$. Then use the public encryption key and modulus to send out cypher-text  $c = m^e \: mod \: n$. Exponentiate the cypher-text using the private description key, $c^d = m^{ed} = m \cdot m^{\phi (n) r}$. Since $m \perp n$ by Euler's Theorem, $\mod{m^{\phi (n)}}{1}{n}$ so $\mod{m \cdot m^{\phi (n) r}}{m}{n}$ and thus, $\mod{c^d}{m}{n}$. The message has been decrypted. \\
RSA is secure because to calculate $d$ requires knowledge of $\phi (n)$. In general this is an extremely computationally expensive problem. However, beginning with the prime factorization $n = pq$, the computation: $\phi (n) = (p-1)(q-1)$ becomes trivial.

\end{definition}


\newpage
\subsection{Order and Primitive Roots}

\begin{definition}
Let $g \perp n$ then $ord_n(g)$ is the least positive $r$ s.t. $\mod{g^r}{1}{n}$.
\end{definition}

\begin{lemma}
\label{orderdivides}
If $\mod{g^s}{1}{n}$ then $ord_n(g) \divides s$.
\end{lemma}

\begin{lproof}
By the Euclidean Property, $s = ord_n(g)q + r$ for $0 \le r < ord_n(g)$. \\ Since $\mod{g^{ord_n(g)}}{1}{n}$ therefore $\mod{g^s = (g^{ord_n(g)})^q \cdot g^r}{g^r}{n}$.\\ However, $\mod{g^s}{1}{n}$ so $\mod{g^r}{1}{n}$ but $r < ord_n(g)$ the least positive exponent. \\ Therefore, $r = 0$ and thus, $ord_n(g) \divides s$.
\end{lproof}

\begin{corollary}
\label{orderdividesphi}
If $g \perp n$ then $ord_n(g) \divides \phi (n)$
\end{corollary}

\begin{lproof}
By Euler's Theorem, $g \perp n$ implies that $\mod{g^{\phi (n)}}{1}{n}$. \\ Therefore, by Lemma~\ref{orderdivides}, $ord_n(g) \divides \phi (n)$
\end{lproof}

\begin{lemma}
\label{congruentorders}
If $\mod{x}{y}{n}$ then $ord_n(x) = ord_n(y)$.
\end{lemma}

\begin{lproof}
Let $\mod{x}{y}{n}$ then by Lemma~\ref{modpower}, $\mod{x^s}{y^s}{n}$. \\ Therefore, $\mod{x^s}{1}{n} \iff\mod{y^s}{1}{n}$ so the least exponents are equal.
\end{lproof}

\begin{proposition}
$n \divides \phi (a^n - 1)$
\end{proposition}

\begin{lproof}
$ord_{a^n-1}(a) = n$ because if $r < n$ then $a^r - 1 < a^n - 1$. \\ Thus by Lemma~\ref{orderdividesphi}, $n \divides \phi (a^n - 1)$
\end{lproof}

\begin{lemma}
\label{orderofpowers}
$ord_n(g^k) = ord_n(g)/gcd(k, ord_n(g))$
\end{lemma}

\begin{lproof}
Let $s = ord_n(g^k)$ and $ord_n(g) = r$. Since $\mod{(g^k)^r = g^{kr}}{1}{n}$ then $s \divides r$. Let $r = ds$. $\mod{(g^k)^s}{1}{n}$ then $r \divides ks$ thus, $ds \divides ks$ so $d \divides k$. Suppose that $c \divides k$ and $c \divides r$ then $\mod{(g^k)^{r/c} = (g^r)^{k/c}}{1}{n}$ therefore $sc \divides r$ i.e. $cs \divides ds$ and thus $c \divides d$.  Collectively, $d \divides r$ and $d \divides k$ and $c \divides k \wedge c \divides r \implies c \divides d$  therefore $d = gcd(k,r)$ \\
However, $r = ds$ so therefore, $ord_n(g) = gcd(g,ord_n(g)) \cdot ord_n(g^k)$
\end{lproof}

\begin{definition}
$g$ is a primitive root modulo $n$ if $g \perp n$ and  $ord_n(g) = \phi (n)$.
\end{definition}


\begin{proposition}
\label{prootsgenerate}
If $g$ is a primitive root modulo $n$ then $\{g^k \: mod \: n \mid k \in \Zplus \} = \Phi (n)$
\end{proposition}


\begin{lproof}
Let since $g \perp n$ by Lemma~\ref{congruentgcdequal}, $g^k \: mod \: n \perp n$ so for every $k$, $g^k \: mod \: n \in \Phi (n)$. Suppose that $\mod{g^i}{g^j}{n}$ for $i < j < \phi (n)$ then $\mod{g^i \cdot g^{\phi (n) - j}}{g^j \cdot g^{\phi (n) - j}}{n}$ but by Euler's Theorem $\mod{g^j \cdot g^{\phi (n) - j} = g^{\phi (n)}}{1}{n}$ and thus $\mod{g^{\phi (n) + i - j}}{1}{n}$ but $\phi (n) + i - j < \phi (n)$ contradicting $ord_n (g) = \phi (n)$. Therefore there are atleast $\phi (n)$ elements of $\{g^k \: mod \: n \mid k \in \Zplus \} \subset \Phi (n)$ and therefore, $\{g^k \: mod \: n \mid k \in \Zplus \} = \Phi (n)$
\end{lproof}

\begin{lemma}
\label{numberofprimroots}
If $d \divides \phi (n)$ then either $|\{x \in \Phi (n) \mid ord_n(x) = d\} | \ge \phi (d)$ or is zero. In particular, if $\Phi (n)$ contains a primitive root then the are exactly $\phi (\phi (n))$ in $\Phi (n)$.
\end{lemma}

\begin{lproof}
Suppose that $\{x \in \Phi (n) \mid ord_n(x) = d\}$ is not empty. Then $\exists g \in \Phi (n)$ s.t. $ord_n(g) = d$. Consider $ord_n(g^k)$ by Lemma~\ref{orderofpowers}, $ord_n(g^k) = d$ iff $gcd(k, d) = 1$. Furthermore, for $s,t < d$ if $\mod{g^s}{g^t}{n}$ then $s = t$ so exponents less that $d$ give distinct elements. There are $\phi (d)$ exponents $k$ s.t. $k \perp d$ and $k < d$ this gives a lower bound on the number of elements with order $d$. \\ Let $d = \phi (n)$ and let a primitive root $g$ exist. By Proposition~\ref{prootsgenerate}, there are no elements other than $g^k$ in $\Phi (n)$ so there are exactly $\phi (\phi (n))$ elements of order $\phi (n)$.
\end{lproof}

\subsection{The Primitive Root Theorem}


\begin{theorem} [Lagrange's Polynomial Theorem]
\label{lagrangepoly}
If $p$ is a prime and $f(x)$ is a polynomial of order $n$ with a leading coefficient not divisible by $p$ then $\mod{f(x)}{0}{p}$ has at most $n$ distinct solutions modulo $p$. 
\end{theorem}

\begin{proof}
Base Case: for $n = 0$, $\mod{a_0}{0}{p}$ has zero solutions if $p \ndivides a_0$.\\
Assume the theorem holds for degree $n$ polynomials. Let $f$ have degree $n+1$ and have a non-zero leading coefficient modulo p. Suppose $p \divides f(r)$ then $f(x) = (x-r)g(x) + pk$ where $g$ has degree $n$. $g$ cannot have a leading coeficient divisible by $p$ because then $f(x) = (x-r)g(x) + kp$ would as well. If $p \divides f(x)$ then $p \divides (x-r)g(x)$ so by Lemma~\ref{euclidlem}, $p \divides x-r$ or  $p \divides g(x)$. Suppose that $p \divides x_1 - r$ and $p \divides x_2 - r$ then $p \divides x_1 - x_2$ so the solutions are congruent. Since $g$ has degree $n$, by hypothesis there are at most $n$ solutions to $p \divides g(x)$. Therefore, including $r$, there are at most $n+1$ solutions to $p \divides f(x)$. The result holds for all degrees by induction.
\end{proof}



\begin{theorem}
\label{numelementsordprimes}
If $p$ is prime then if $d \divides p-1$ there are exactly $\phi (d)$ elements of $\Phi (p)$ with order $d$. In particular, there are $\phi (p-1)$ primitive roots.
\end{theorem}

\begin{proof}
Suppose that $\exists g \in \Phi (p)$ s.t. $ord_p (g) = d$. Then $\mod{x^d - 1}{0}{p}$ has a solution $g$. Furthermore, $g^i$ for $0 \le i < d$ are also solutions and are distinct because $ord_p(g) = d$ and each exponent is less than $d$ (see Proposition~\ref{prootsgenerate}). These are $d$ distinct solutions to a degree $d$ polynomial so by Theorem~\ref{lagrangepoly}, there are no others. If $ord_p (x) = d$ then $\mod{x^d - 1}{0}{p}$ so there is some $k$ s.t. $x = g^k$. By Lemma~\ref{orderofpowers}, $ord_p(g^k) =d $ iff $gcd(k,d) = 1$. Therefore, there are exactly $\phi (d)$ such $k$ and by extension exactly $\phi (d)$ such $x$.
\\\\
However, this proof presupposes that at least one element of order $d$ exists. \\
Let $S_d = \{x \in \Phi (p) \mid ord_n(x) = d\}$ if $S_d$ is non-empty then $|S_d| = \phi (d)$. \\ Therefore, either $|S_d| = 0$ or $|S_d| = \phi (d)$. Every element of $\Phi (p)$ has exactly one order and therefore the sets $S_d$ partition $\Phi (p)$. In particular if any $|S_d| = 0$ then \[\sum \limits_{d \divides p-1} |S_d| = \phi (p) = p -1 < \sum \limits_{d \divides p-1} \phi (d)\] However, by Theorem~\ref{phisumoffactors}, $\sum \limits_{d \divides p-1} \phi (d) = p-1$ so for every $d$, $|S_d| = \phi (d)$
\end{proof}

\begin{lemma}
\label{primrootsofdouble}
Let $n$ be odd. For each primitive root modulo $n$ there is a primitive root modulo $2n$. Also, $\Phi(n)$ and $\Phi(2n)$ contain the same number of primitive roots.\\
In particular, if primitive roots exist modulo $n$ then primitive roots exist modulo $2n$. 
\end{lemma}

\begin{lproof}
Because $2 \perp n$, $\phi(2n) = \phi (2) \cdot \phi (n) = \phi (n)$. Chose $g$ to be an odd primitive root modulo $n$. If $g$ is even then choose $g + n$ which is odd but has equal order. $g \perp 2$ and $g \perp n$ so by Lemma~\ref{coprimeprodsarecoprime}, $g \perp 2n$. By Euler's Theorem, $\mod{g^{\phi (2n)}}{1}{2n}$. However, if $k < \phi(n) = \phi (2n)$ then $n \ndivides g^k - 1$ because $ord_n(g) = \phi (n)$ and thus $2n \ndivides g^k - 1$ so if $k < \phi(2n)$ then $\nmod{g^k}{1}{2n}$. Thus, $ord_{2n}(g) = \phi (2n)$ i.e. $g$ is a primitive root modulo $2n$. Since $\phi(2n) = \phi (n)$, by Lemma~\ref{numberofprimroots}, both $\Phi(2n)$ and $\Phi(n)$ contain $\phi (\phi(n))$ primitive roots.   
\end{lproof}

\begin{theorem}[The Primitive Root Theorem]
There exist primitive roots modulo: $2,4, p^k$, and  $2p^k$ where $p$ is an odd prime and not for any other moduli.
\end{theorem}

\begin{proof}
Let $p$ be an odd prime. Claims:

\begin{enumerate}
\item $\exists g$ s.t. $ord_p(g) = \phi (p)$ and $\nmod{g^{\phi (p)}}{1}{p^2}$

\begin{lproof}
By Theorem~\ref{numelementsordprimes}, there exists a primitive root, $x$ modulo $p$. Suppose that $\mod{x^{\phi (p)}}{1}{p^2}$ Then, $(x + p)^{\phi (p)} = x^{\phi(p)} + \phi(p) x^{\phi(p) - 1} p + \cdots + p^{\phi(p)}$ \\ However, $p\ndivides \phi(p)$ and $p \ndivides x^{\phi(p) - 1}$ thus $p \ndivides \phi(p) x^{\phi(p) - 1}$ therefore, $p^2 \ndivides \phi(p) x^{\phi(p) - 1}p$ thus $\nmod{(x+p)^{\phi(p)}}{1}{p^2}$ Furthermore, $\mod{x+p}{x}{p}$ so $x+p$ is also a primitive root modulo $p$. Choose $g = x+p$ otherwise choose $g = x$.
\end{lproof}

\item $\forall k \ge 1$ : $\nmod{g^{\phi(p^k)}}{1}{p^{k+1}}$ 

\begin{lproof}
 By Euler's Theorem, $g^{\phi(p^k)} = 1 + mp^k$. Let $\nmod{g^{\phi(p^k)}}{1}{p^{k+1}}$ i.e. $p \ndivides m$. $\phi(p^{k+1}) = p^k(p-1) = p\phi(p^k)$. Thus, $g^{\phi(p^{k+1})} = g^{p\phi(p^k)} = (1+mp^k)^p$. Expand, $(1+mp^k)^p = 1 + mp^{k+1} + \dots + (mp^k)^p$ but $p \ndivides m$ so $\nmod{g^{\phi(p^{k+1})}}{1}{p^{k+2}}$. Since, $\nmod{g^{\phi (p)}}{1}{p^2}$ by Induction the result holds for all $k \ge 1$. \\ The induction only holds if $kp > k+1$. However, for $p = 2$ and $k = 1$ this is false: $g^{\phi(2^2)} = (1+2m)^2 = 1+2^2m + 2^2m^2$ but $2 \divides m + m^2$ so $2^3 \divides (1+2m)^2-1$. Therefore, $\mod{g^{\phi(2^2)}}{1}{2^3}$ although $\nmod{g^{\phi(2^1)}}{1}{2^2}$
\end{lproof}

\item $\forall k \ge 1$ : $ord_{p^k}(g) = \phi(p^k)$ i.e. $g$ is a primitive root for $p^k$

\begin{lproof}
Assume: $ord_{p^k}(g) = \phi(p^k)$. Let $m = ord_{p^{k+1}}(g)$. Then $p^k \divides p^{k+1} \divides g^m - 1$ Thus, $ord_{p^k}(g) = \phi(p^k) \divides m$. Also, $m \divides \phi(p^{k+1})$ so $\phi(p^k) \divides m \divides \phi(p^k) p$ thus, \\
$m = \phi(p^k)$ or $\phi(p^{k+1})$. However, $\nmod{g^{\phi(p^k)}}{1}{p^{k+1}}$ so $ord_{p^{k+1}}(g) = \phi(p^{k+1})$ \\
By induction, the result holds for all $k$.
\end{lproof}

\item If $a,b > 2$ and $a \perp b$ then no primitive roots exist modulo $n = ab$

\begin{lproof}
If $g \perp ab$ then $g \perp a$ and $g \perp b$ so by Euler's Theorem, $\mod{g^{\phi(a)}}{1}{a}$ and $\mod{g^{\phi(b)}}{1}{b}$. Since $a,b > 2$ by Corollary~\ref{phieven}, $2 \divides \phi(a)$ and $2 \divides \phi(b)$. Therefore, $\mod{g^{\phi(a) \cdot \phi(b)/2}}{1}{a}$ and $\mod{g^{\phi(b) \cdot \phi(a) /2}}{1}{b}$. Since $a \perp b$ by Corollary~\ref{coprimeprodmods}, $\mod{g^{\phi(a) \cdot \phi(b) /2}}{1}{ab}$ and $\phi(ab) = \phi(a) \cdot \phi(b)$. Thus, $\phi(a) \cdot \phi(b) /2 < \phi(n)$ so $\forall g \in \Phi(n)$ : $ord_n(g) < \phi(n)$ i.e. no primitive roots modulo $n$ exist.
\end{lproof}

\item There exist primitive roots modulo $2^k$ only for $2$ and $4$

\begin{lproof}
$1$ and $3$ are primitive roots for $2$ and $4$ respectively. However,  \\ $\forall g$ : $\mod{g^{\phi(2^2)}}{1}{2^3}$. Let $\forall g$ : $\mod{g^{\phi(2^k)}}{1}{2^{k+1}}$ i.e. $g^{\phi(2^k)} = 1 + m2^{k+1}$ $g^{\phi(2^{k+1})} = (1+m2^{k+1})^2 = \mod{1 + m2^{k+2} + m^22^{2k+2}}{1}{2^{k+2}}$ thus by induction: $\forall g$ : $\mod{g^{\phi(2^k)}}{1}{2^{k+1}}$ in particular, $\forall g$ : $ord_{2^{k+1}}(g) \le \phi(2^k) < \phi(2^{k+1})$
\end{lproof}
\end{enumerate}
Primitive roots exist modulo $2$, $4$, and, $p^k$. By Lemma~\ref{primrootsofdouble}, there also exist primitive roots modulo $2p^k$ because $p$ is odd. If $n \neq 2,4,p^k,2p^k$ then $n = 2^k > 4$ or $n$ has coprime factors besides $2$  and thus there do not exist primitive roots modulo $n$. 
\end{proof}

\section{Quadratic Residues}

\begin{definition}
$a$ is a quadratic residue modulo $p$ if $p \ndivides a$ and $\exists x$ s.t. $\mod{x^2}{a}{p}$
\end{definition}

\begin{lemma}[Euler's Criterion]
Let $p$ be an odd prime and $p \ndivides a$ then $a$ is a quadratic residue modulo $p$ iff $\mod{a^{(p-1)/2}}{1}{p}$.
\end{lemma}

\begin{lproof}
Let $g$ be a primitive root modulo $p$. Then $\mod{a}{g^m}{p}$. \\ Thus, $\mod{a^{(p-1)/2} = g^{m(p-1)/2}}{1}{p}$.  $ord_p(g) = p-1$ so $p-1 \divides m(p-1)/2$ thus, $2(p-1) \divides m(p-1)$ so $2 \divides m$. Thus, $\mod{a}{(g^{m/2})^2}{p}$ so $a$ is a quadratic residue. \\
Suppose $a$ is a quadratic residue then $\mod{a}{x^2}{p}$. Write $\mod{x}{g^n}{p}$ then $\mod{a}{g^{2n}}{p}$ therefore, $\mod{a^{(p-1)/2} = g^{n(p-1)}}{1}{p}$.
\end{lproof}

\begin{definition}
The Legendre Symbol: 
\begin{itemize}
\item $(a|p) = 1$ if $a$ is a quadratic residue modulo $p$
\item $(a|p) = -1$ if $a$ is a quadratic non-residue modulo $p$ 
\item $(a|p) = 0$ if $p \divides a$
\end{itemize}
\end{definition}

\begin{lemma}
If $p$ is an odd prime then $\mod{(a|p)}{a^{(p-1)/2}}{p}$ 
\end{lemma}

\begin{lproof}
If $p \divides a$ then $p \divides a^{(p-1)/2}$ so $\mod{a^{(p-1)/2}}{0}{p}$ and $(a|p) = 0$.
Else, $p \ndivides a$ so by Fermat's Little Theorem, $p \divides a^{p-1}-1$ since $p$ is odd, $p \divides (a^{(p-1)/2}-1)(a^{(p-1)/2}+1)$. Thus, $p \divides a^{(p-1)/2}-1$ or $p \divides a^{(p-1)/2}+1$. If $a$ is a quadratic residue modulo $p$ then both $(a|p) = 1$ and $\mod{a^{(p-1)/2}}{1}{p}$. If $a$ is a quadratic non-residue, $p \ndivides a^{(p-1)/2}-1$ so $p \divides a^{(p-1)/2}+1$ i.e. $\mod{a^{(p-1)/2}}{-1}{p}$ but also $(a|p) = -1$.
\end{lproof}

\begin{lemma}
$(ab|p) = (a|p) \cdot (b|p)$
\end{lemma}

\begin{lproof}
$\mod{(a|p)}{a^{(p-1)/2}}{p}$ and $\mod{(b|p)}{b^{(p-1)/2}}{p}$ thus \\ $\mod{(a|p)(b|p)}{a^{(p-1)/2} \cdot a^{(p-1)/2}}{p}$ but $\mod{(ab|p)}{(ab)^{(p-1)/2}}{p}$ so $\mod{(ab|p)}{(a|p)(b|p)}{p}$ but the values are $\{-1,0,1\}$ and $p > 2$ so $(ab|p) = (a|p)(b|p)$
\end{lproof}

\begin{lemma}
\label{congruentlegs}
If $\mod{a}{b}{p}$ then $(a|p) = (b|p)$
\end{lemma}

\begin{lproof}
$\mod{a}{b}{p}$ so $\mod{a^{(p-1)/2}}{b^{(p-2)/2}}{p}$ thus $\mod{(a|p)}{(b|p)}{p}$ but the values are $\{-1,0,1\}$ and $p > 2$ so $(a|p) = (b|p)$ 
\end{lproof}

\begin{proposition}
\label{quadresneg1}
$-1$ is a quadratic residue modulo $p$ iff $\mod{p}{1}{4}$
\end{proposition}

\begin{lproof}
$\mod{(-1|p)}{(-1)^{(p-1)/2}}{p}$ thus $(-1|p) = 1$ iff $\frac{p-1}{2} = 2k$ i.e. $p = 4k + 1$
\end{lproof}

\begin{proposition}
There are infinitely many primes of the form $4k + 1$
\end{proposition}

\begin{lproof}
Let $p \divides (n!)^2 + 1$. Because $\mod{(n!)^2}{-1}{p}$ then $(-1|p)$ so $p = 4k + 1$. However, if $p \le n$ then $p \divides n!$ implying that $p \divides 1$ \cont\ Thus, $p > n$. Therefore there are arbitrarily large primes of the form $4k + 1$ and thus no upper bound. 
\end{lproof}

\begin{proposition}
There are infinitely many primes of the form $4k - 1$
\end{proposition}

\begin{lproof}
Suppose that $p_1, p_2, \dots , p_r$ were all the primes of the form $4k - 1$. Consider, \\ $N = 4 (p_1 \cdot p_2 \cdot \ldots \cdot p_r) - 1$. No $p_i \divides N$ else $p_i \divides -1$. Thus, every prime divisor of $N$ has the form $4k + 1$. If $\mod{p \equiv q}{1}{4}$ then $\mod{pq}{1}{4}$ however, $\mod{N}{-1}{4}$ \cont 
\end{lproof}

\newpage

\begin{lemma}[Gauss's Lemma]
Let $p$ be an odd prime s.t. $p \ndivides a$ and $n$ be the number of residues modulo $p$ of $1 \cdot a, 2 \cdot a, \dots , \frac{p-1}{2} \cdot a$ that are greater than $\frac{p-1}{2}$ then $(a|p) = (-1)^n$
\end{lemma}

\begin{lproof}
Suppose that $\mod{xa}{ya}{p}$ because $p \ndivides a$ by Corollary~\ref{congruencecancelation}, $\mod{x}{y}{p}$. If $ax \: mod \: p = r > \frac{p-1}{2}$ then $p - r < \frac{p-1}{2}$ and $\mod{-(p - r)}{ax}{p}$. Suppose that $\mod{-ax}{ay}{p}$ then $p \divides x+y$, impossible because $x,y < \frac{p-1}{2}$. Each $ax$ reduces upto sign to a unique $r < \frac{p-1}{2}$ Thus, $\mod{1a \cdot 2a \cdot \ldots \cdot \frac{p-1}{2}a}{(-1)^n \cdot 1 \cdot 2 \cdot \ldots \cdot \frac{p-1}{2}}{p}$ However, $1a \cdot 2a \cdot \ldots \cdot \frac{p-1}{2}a = a^{(p-1)/2} \cdot (\frac{p-1}{2})!$ therefore, $\mod{a^{(p-1)/2} \cdot (\frac{p-1}{2})!}{(-1)^n \cdot (\frac{p-1}{2})!}{p}$ so $\mod{a^{(p-1)/2}}{(-1)^n}{p}$. Furthermore, $\mod{a^{(p-1)/2}}{(a|p)}{p}$ thus $(a|p) = (-1)^n$
\end{lproof}

\begin{theorem}[Quadratic Reciprocity]
If $p$ and $q$ are distinct odd primes then \[(p|q)(q|p) = (-1)^{\frac{p-1}{2} \cdot \frac{q-1}{2}}\]
\end{theorem}

\begin{proof}
Let $a = \frac{p-1}{2}$ and $b = \frac{q-1}{2}$. For each $r < pq$ there is a unique $(x,y)$ s.t. $\mod{r}{x}{p}$ and $\mod{r}{y}{q}$ with $x < p$ and $y < q$. Let $f(r) = (x,y)$ where $f$ is a bijection by the Chinese Remainder Theorem. Let $P = f([1,\frac{pq-1}{2}])$ and let $u$ be the number of $r \in P$ s.t. $f(r) = (x,0)$ for $x > a$ and $v$, the number of $r$ s.t. $f(r) = (0,y)$ for $y > b$. 
Consider: \\
\[N_1 = |\{(x,y) \in P \mid 0 < x  \le a \wedge 0 < y \le b\}|\] 
\[N_2 = |\{(x,y) \in P \mid a < x < p \wedge 0 \le y \le b\}|\]  
\[N_3 = |\{(x,y) \in P \mid 0 \le x  \le a \wedge b < y < q\}|\] 
$\frac{pq-1}{2} = \frac{(p-1)q}{2} + \frac{q-1}{2} = aq + b$. Thus $r = lq + k < aq + b$ for $0 \le l \le a$ and $1 \le k \le b$. Furthermore, $\mod{r}{k}{q}$ so there are $(a+1)b$ elements of $P$ s.t. $0 < y \le b$ also there are $b - v$ elements of $P$ s.t. $x = 0$ and $0 \le y \le b$ also there are $u$ elements of $P$ s.t. $a < x < p$ and $y = 0$ thus $N_1 + N_2$ = $(ab + b) - (b - v) + u = ab + u + v$ \\ Swapping $q$ for $p$ the same argument shows that $N_1 + N_3 = ab + u + v$. \\ Let $f(r) = (x,y)$ with $1 \le x  \le a$ and $1 \le y \le b$. $f(pq-r) = (p-x,q-y)$ then exactly one of $(x, q-y)$ and $(p-x, y)$ is in $P$. There are $ab$ such $x$ and $y$. When $x = 0$ there are $v$ such $y > b (N_3)$ and when $y = 0$ there are $u$ such $x > a (N_2)$. Therefore, $N_2 + N_3 = ab + u + v$. Summing: $2(N_1 + N_2 + N_3) = 3(ab + u + v)$ thus $2 \divides ab + u + v$ so $(-1)^{ab + u + v} = 1$    multiplying by $(-1)^{u + v}$ gives $(-1)^{ab} = (-1)^u \cdot (-1)^v$. Furthermore, if $f(r) = (x, 0)$ then $r = kq$ where $k \le a$ because  $((a+1)q > aq + b)$ so $u$ is the number of $1 \cdot q, 2 \cdot q, \dots , a \cdot q$ which are greater than $a$ modulo $p$ thus by Gauss's Lemma, $(-1)^u = (q|p)$ likewise $(-1)^v = (p|q)$. Thus, $(p|q)(q|p) = (-1)^{ab} = (-1)^{\frac{p-1}{2} \cdot \frac{q-1}{2}}$ 
\end{proof}

\begin{proposition}
\label{quadres2}
$2$ is a quadratic residue modulo $p$ iff $\mod{p}{\pm 1}{8}$
\end{proposition}

\begin{lproof}
By Gauss's Lemma, $(2|p) = (-1)^n$ where $n$ is the number of elements, \\ $2, 4, \dots, p-1$ that are greater than $\frac{p-1}{2}$. If $\frac{p-1}{2}$ is even then $n = \frac{p-1}{4}$ thus $(2|p) = 1$ iff $2 \divides n$ i.e. $\mod{p}{1}{8}$. If $\frac{p-1}{2}$ is odd then $n$ counts the middle element i.e. $n = \frac{1}{2}(\frac{p-1}{2} + 1) = \frac{p+1}{4}$ thus $(2|p) = 1$ iff $2 \divides n$ i.e. $\mod{p}{-1}{8}$.
\end{lproof}

\newpage
\section{Special Families of numbers}
This section is devoted to a survey of a selection of the most important families of numbers that have been areas of theoretical and applied research in number theory.
\subsection{Carmichael Numbers}

\begin{lemma}
\label{coprimeifdividespowerminusone}
If $n \divides a^k \pm 1$ then $a \perp n$
\end{lemma}

\begin{lproof}
Let $a^k \pm 1 = ng$ or $a (\mp  a^{k-1}) + n   (\pm g) = 1$ so there exist integer solutions to the equation: $ax + ny = 1$ and thus by Corollary~\ref{gcdintsol} $a \perp n$.
\end{lproof}

\begin{definition} a Carmichael number is a composite number $n$ such that \\ $\forall a \in \Zplus : a \perp n \implies n \divides a^{n-1} - 1$.\\
Equivalently, $n$ is a pseduoprime for every base $a$ for which $a \perp n$.
\end{definition}

\begin{theorem}[Korselt's Criterion]
$n$ is a Carmichael number iff $n$ is composite and $\forall p \in \Primes$ : $p \divides n \implies p-1 \divides n-1$ and n is square-free i.e. $\centernot \exists p \in \Primes$ s.t. $p^2 \divides n$
\end{theorem}

\begin{proof}
Let $n$ be square-free i.e the product of distinct primes: $n = p_1   p_2   \dots   p_k$ \\ Let $a \perp n$ therefore for each prime divisor $p$ of $n$, $p \ndivides a$. Therefore by \\ Fermat's Little Theorem, $p \divides a^{p-1} - 1$. Since $p - 1 \divides n-1$ this implies that $p \divides a^{n-1} - 1$ for each $p \divides n$.
Because distinct prime are coprime, by Lemma~\ref{coprimeprod}: $p_1   p_2   \dots   p_k \divides a^{n-1} - 1$ so $n \divides a^{n-1} - 1$ for each $a \perp n$ and by hypothesis $n$ is composite.
\\\\
Suppose that $n$ is a Carmichael number. \\
Let $n = p_1^{a_1}   p_2^{a_2}   \dots   p_k^{a_k}$ with distinct $p_i \in \Primes$. For any particular $p_i$, \\
by The Primitive Root Theorem, there exists a number, $g$, s.t. $ord_{p_i}(g) = p_i-1$. \\ 
Consider the system: 
\\ \indent $\mod{x}{1}{p_1}$ , $\mod{x}{1}{p_2}$ , \dots ,  $\mod{x}{g}{p_i}$ , \dots , $\mod{x}{1}{p_k}$\\ he primes are distinct and thus are coprime so there exists a solution by \\ the Chinese Remainder Theorem. $x \perp n$ because for every $p \divides n$ we have that $p \ndivides x$. \\
Thus because $n$ is Carmichael, $n \divides x^{n-1} - 1$ therefore, $p_i \divides x^{n-1} - 1$. Since $\mod{g}{x}{p_i}$ by Lemma~\ref{congruentorders}, $ord_{p_i}(x) = ord_{p_i}(g) = p_i-1$ and so Lemma~\ref{orderdivides} implies that $p_i - 1 \divides n - 1$.
Thus if $p \divides n$ then $p-1 \divides n-1$. \\\\

\noindent Suppose that some $p_i^2 \divides n$. The Primitive Root Theorem implies that there exists a number, $g$, s.t. $ord_{p_i^2}(g) = p_i(p_i-1)$. Consider the system: \\ \indent $\mod{x}{1}{p_1}$ , $\mod{x}{1}{p_2}$ , \dots ,  $\mod{x}{g}{p_i^2}$ , \dots , $\mod{x}{1}{p_k}$\\ The primes are distinct and thus are coprime so there exists a solution by the Chinese Remainder Theorem. $x \perp n$ because for every $p \divides n$ we have that $p \ndivides x$. \\
Thus because $n$ is Carmichael, $n \divides x^{n-1} - 1$ therefore, $p_i^2 \divides x^{n-1} - 1$. \\ Since $\mod{g}{x}{p_i}$ by Lemma~\ref{congruentorders}, $ord_{p_i^2}(x) = ord_{p_i^2}(g) = p_i(p_i-1)$ and so by Lemma~\ref{orderdivides}, $p_i(p_i - 1) \divides n - 1$ so $p_i \divides n-1$ but $p_i \divides n$ and thus $p_i \divides 1$ \cont \\ Therefore, $\centernot \exists p$ s.t $p^2 \divides n$ i.e. $n$ is square-free. 
\end{proof}

\begin{proposition} No even Carmichael numbers exist.
\end{proposition}
\begin{lproof}
Carmichael numbers are square-free and composite so they have at least one odd prime factor. Let $p$ be an odd prime and $p \divides n$ then by Korselt's Criterion, \\ $p-1 \divides n-1$. However, $p-1$ is even so $n-1$ is even and thus $n$ is odd. \\ Alternatively, if $n$ is even then $n-1$ is odd so $\mod{(n-1)^{n-1}}{-1}{n}$ \\ However, $n-1 \perp n$ so  there exists an $a$ coprime with $n$ s.t. $n \ndivides a^{n-1}-1$
\end{lproof}

\begin{lemma}
Let $n = ab$ then $a - 1 \divides n - 1 \iff a - 1 \divides b - 1$
\end{lemma}

\begin{lproof}
\label{divoneless}
Let $a-1 \divides ab - 1$, since $a-1 \divides (a-1)b$, we have that $a-1 \divides ab - 1 - (ab - b)$. \\
Regrouping, $a-1 \divides b-1$.
\\

\noindent Let $a - 1 \divides b - 1$ and therefore $a - 1 \divides a(b - 1)$. Since $a-1 \divides a-1$, \\ $a-1 \divides ab - a + (a-1)$. Regrouping, $a-1 \divides ab - 1$.
\end{lproof}

\begin{proposition} No Carmichael numbers with two prime factors exist.
\end{proposition}
\begin{lproof}
Let $n = pq$ where $p-1 \divides n - 1$ and $q-1 \divides n - 1$. \\ By Lemma~\ref{divoneless}, $p-1 \divides q-1$ and $q-1 \divides p-1$. Therefore, $p-1 = q-1 \implies p = q$ which contradicts the fact that Carmichael numbers are square-free.
\end{lproof}

\begin{proposition} $\forall a \in \Z$ : $n \divides a^n - a$ and $n$ is composite iff $n$ is Carmichael.
\end{proposition}

\begin{lproof}
Let $n = p_1   p_2   \dots   p_k$ be a Carmichael number and therefore it is composite. Suppose $a \perp n$ then $n \divides a^{n-1} - 1$ and therefore, $n \divides a^{n} - a$.\\
Otherwise, $\exists p_i \divides n$ s.t. $p_i \divides a$. For each prime factor $p$ of $n$ s.t. $p \ndivides a$,\\ by Fermat, $p \divides a^{p-1} - 1$ and by Korselt, $p \divides a^{n-1} - 1$. Thus, $p \divides a^{n} - a$. \\
For each prime factor $p'$ of $n$ s.t. $p' \divides a$ we also have that $p' \divides a^n - a$. \\
Thus for every $p \divides n$, we have, $p \divides a^n - a$. By Lemma~\ref{coprimeprod}, $p_1   p_2   \dots   p_k = n \divides a^n - a$.\\

\noindent If $\forall a \in \Z$ : $n \divides a^n - a$ then if $a \perp n$ by Lemma~\ref{euclidlem}, $n \divides a^{n-1} - 1$.\\ Since n is composite, n is Carmichael.    
\end{lproof}

\begin{proposition}
$(6k+1)(12k+1)(18k+1)$ is Carmichael if each factor is prime.
\end{proposition}

\begin{lproof}
Let $6k+1$, $12k + 1$, $18k + 1$ each be prime and let $n = (6k+1)(12k+1)(18k+1)$. \\
We check that each factor minus one divides the product of the other two minus one:
\begin{itemize}

\item $(12k+1)(18k+1) - 1 = 12 \cdot 18  \cdot k^2 + (12 + 18) k = 6k(36k + 5)$ thus, \\ $(6k+1) - 1 \divides (12k+1)(18k+1) - 1$

\item $(6k+1)(18k+1) - 1 = 6 \cdot 18  \cdot k^2 + (6 + 18) k = 12k(9k + 2)$ thus, \\ $(12k+1) - 1 \divides (6k+1)(18k+1) - 1$

\item $(6k+1)(12k+1) - 1 = 6 \cdot 12  \cdot k^2 + (6 + 12) k = 18k(4k + 1)$ thus, \\ $(6k+1) - 1 \divides (12k+1)(18k+1) - 1$
\end{itemize}

\noindent Thus for each prime $p$ that divides $n$, by Lemma~\ref{divoneless}, $p - 1 \divides n - 1$. If each factor is prime then they are distinct so by Korselt's Criterion, $n$ is Carmichael. \\ For example, $k = 1 \implies 7 \cdot 13 \cdot 19 = 1729$ is a Carmichael number. However, this formula does not produce all Carmichael numbers in particular it does not generate the first two Carmichael numbers: $3 \cdot 11 \cdot 17 = 561$ and $3 \cdot 13 \cdot 17 = 1105$
\end{lproof}

\newpage

\subsection{Mersenne Numbers}

\begin{proposition}
If $a^k - 1$ is prime then either $k = 1$ or $a = 2$ 
\end{proposition}

\begin{proof}
$\mod{a}{1}{a-1}$ thus $\mod{a^k}{1}{a-1}$ so $a-1 \divides a^k-1$ if $a^k-1$ is prime then either $a-1 = a^k-1$ i.e. $k = 1$ or $a - 1 = 1$ i.e. $a = 2$.
\end{proof}

\begin{proposition}
If $2^k - 1$ is prime then $k$ is prime
\end{proposition}

\begin{lproof}
Let $k = rt$. $\mod{2^r}{1}{2^r-1}$ and thus $\mod{2^{rt}}{1}{2^r-1}$ i.e. $2^r - 1 \divides 2^k - 1$. Because $2^k - 1$ is prime, either $2^r - 1 = 2^k - 1$ or $2^r - 1 = 1$ therefore, \\ either $r= k$ or $r = 1$ thus $r \divides k \implies r = 1 \vee r = k$ i.e. $k$ is prime.
\end{lproof}

\begin{definition}
$M_p = 2^p - 1$ is a Mersenne number if $p$ is prime. 
\end{definition}

\begin{proposition}
\label{mersennedivisorlemma}
All prime divisors of $M_p$ are of the form $q = 2pk + 1$
\end{proposition}

\begin{lproof}
Let $q \divides 2^p - 1$. $ord_q(2) \divides p$ so $ord_q(2) = 1$ or $ord_q(2) = p$. However, if $ord_q(2) = 1$ then $q \divides 2^1-1$ \cont\ Thus $ord_q(2) = p$. By Corollary~\ref{orderdividesphi}, $ord_q(2) \divides q-1$ thus $p \divides q-1$ so $q = pr + 1$. Furthermore, since $M_p$ is odd then $q$ must also be odd. Thus, $q-1$ is even so $2 \divides r$ and thus, $q = 2pk + 1$
\end{lproof}

\begin{proposition}
If $p$ is a prime of the form $4k - 1$ and $2p + 1$ is prime then $2p + 1 \divides M_p$ and therefore $M_p$ is composite.  
\end{proposition}

\begin{lproof}
Let $p = 4k - 1$ then $2p + 1 = 8k - 1$. Thus, by Propostion~\ref{quadres2}, $(2|p) = 1$. By Euler's Criterion, $\mod{2^{(2p+1-1)/2} = 2^p}{1}{2p + 1}$ thus $2p + 1 \divides 2^p - 1$.
\end{lproof}

\begin{proposition}
If $q \divides M_p$ where $q$ is an odd prime then $\mod{q}{\pm 1}{8}$
\end{proposition}

\begin{lproof}
$\mod{2^p}{1}{q}$ thus $\mod{2^{p+1}}{2}{q}$ and because $q$ is odd, $\mod{(2^{(p+1)/2})^2}{2}{q}$. Thus $(2|q) = 1$ so by Proposition~\ref{quadres2}, $\mod{q}{\pm 1}{8}$.\\\\
Alternatively, by Proposition~\ref{mersennedivisorlemma}, $q = 2pk + 1$ thus $\mod{2^{(q-1)/2} = 2^{pk}}{1}{q}$ thus by Euler's Criterion, $(2|q)$ so by Proposition~\ref{quadres2}, $\mod{q}{\pm 1}{8}$.
\end{lproof}

\begin{proposition}
If $M_p$ is composite then it is a Fermat pseudoprime for base $2$    
\end{proposition}

\begin{lproof}
$\mod{2^p}{1}{M_p}$ By Corollary~\ref{fermatlittlealternate}, $p \divides 2^p - 2$ because $p$ is prime, thus,  \\ $\mod{2^{2^p- 2}}{1}{M_p}$ therefore, $\mod{2^{M_p - 1}}{1}{M_p}$ and by hypothesis $M_p$ is composite. \\
\end{lproof}

\noindent The first composite Mersenne number is $M_{11} = 23 \cdot 89$ and after that many Mersenne numbers are composite. In all, only 48 Mersenne primes are known as of 2015. However, Mersenne numbers are an efficient form to store large prime numbers and are especially useful for cryptography. Therefore, there has been significant research and computing power into finding Mersenne primes. In 2013, a distributed computing project known as The Great Internet Mersenne Prime Search proved that $M_{57,885,161}$ is prime making it the largest known prime number. It is an open question whether or not infinitely many Mersenne primes exist.


\newpage

\subsection{Fermat Numbers}

\begin{proposition}
If $a^n + 1$ is prime then $n = 2^k$ for some $k$.
\end{proposition}

\begin{lproof}
Suppose that $n = sr$ where $s$ is odd. Then $\mod{a^r}{-1}{a^s+1}$ so \\ $\mod{a^sr}{(-1)^s}{a^r+1}$. Since $a$ is odd, $(-1)^s = -1$ so $a^r + 1 \divides a^n + 1$. $a^n + 1$ is prime so $a^r + 1 = a^n + 1$ thus $s = 1$. The only odd factor of $n$ is $1$ therefore, $n = 2^k$
\end{lproof}

\begin{definition}
$F_k = 2^{2^k} + 1$ is the $k^{th}$ Fermat Number.
\end{definition}

\begin{lemma}
$F_{n+1} = F_0 \cdot F_1 \cdot \ldots \cdot F_n + 2$
\end{lemma}

\begin{lproof}
For $n = 0$, $F_1 = 2^2 = 2^1 + 2$ Assume true for $n$. \\ Consider, $F_0 \cdot F_1 \cdot \ldots \cdot F_n \cdot F_{n+1} + 2 = (F_{n+1} - 2) \cdot F_{n+1} + 2 = (2^{2^{n+1}} - 1)(2^{2^{n+1}} + 1) + 2 = 2^{2 \cdot 2^{n+1}} + 1 = 2^{2^{n+2}} + 1$ by induction the result holds for all $n$.  
\end{lproof}

\begin{proposition}
For all $i$ and $j$ s.t. $i \neq k$, $F_i \perp F_j$ 
\end{proposition}

\begin{lproof}
Let $i > j$ then $F_i - F_0 \cdot F_1 \cdot \ldots \cdot F_{i-1} = 2$. Thus id $d \divides F_i$ and $d \divides F_j$ then $d \divides 2$. However, Fermat numbers are odd so $d = 1$. 
\end{lproof}

\begin{proposition}
If $F_n$ is composite then it is a Fermat pseudoprime for base $2$  
\end{proposition}

\begin{lproof}
$\mod{2^{2^n}}{-1}{F_n}$ thus $\mod{(2^{2^n})^{2^{(2^n-n)}}}{(-1)^{2^{(2^n-n)}}}{F_n}$ because $2^{(2^n-n)}$ is even, $\mod{2^{2^n \cdot 2^{(2^n-n)}}}{1}{F_n}$ thus, $\mod{2^{2^{2^n}}}{1}{F_n}$ Furthermore, $F_n = 2^{2^{2^n}}$ so $\mod{2^{F_n - 1}}{1}{F_n}$ and by hypothesis $F_n$ is composite.  
\end{lproof}

\begin{theorem}[Pepin's Test]
$F_k$ is prime iff $\mod{3^{(F_k-1)/2}}{-1}{F_k}$ for $F_k > 3$.
\end{theorem}

\begin{proof}
If $\mod{3^{(F_k-1)/2}}{-1}{F_k}$ then $\mod{3^{F_k-1}}{1}{F_k}$ so $ord_{F_k}(3) \divides F_k - 1$  \\ however, $ord_{F_k}(3) \ndivides (F_k - 1)/2$ because $F_k - 1 = 2^{2^n}$ then $2^{2^n-1} \ndivides ord_{F_k}(3) \divides 2^{2^n}$, \\ thus $ord_{F_k}(3) = F_k - 1 \le \phi (F_k)$ thus by Corollary~\ref{primeiffphi}, $F_k$ is prime. \\\\ If $F_k \neq 3$ is prime then by Quadratic Reciprocity, $(F_k|3)(3|F_k) = (-1)^{2^{2^{k-1}}} = 1$. Furthermore, $\mod{2}{-1}{3}$ so $\mod{F_k = 2^{2^k} + 1}{2 \equiv - 1}{3}$. However, $\nmod{3}{1}{4}$ so by Proposition~\ref{quadresneg1}, $(-1|3) = -1$ and thus $(F_k|3) = -1$ by Lemma~\ref{congruentlegs}. Because $(F_k|3)(3|F_k) = 1$ then $(3|F_k) = -1$, by Euler's Criterion, $\mod{3^{(F_k-1)/2}}{-1}{F_k}$\\
\end{proof}

\noindent After computing that $F_0 = 3, F_1 = 5, F_2 = 17, F_3 = 257, F_4 = 65537$ are all primes, Fermat conjectured that all Fermat numbers are prime. However, Euler proved that $641 \divides F_5$  and thus $F_5$ is composite. \\

\noindent Note: $641 = 5^4 + 2^4 = 5 \cdot 2^7 + 1$. \\ Thus, $\mod{5^4 + 2^4}{0}{641}$ so $\mod{2^{28} \cdot (5^4 + 2^4)}{0}{641}$ \\
Write: $2^{28} \cdot (5^4 + 2^4) = 5^4 \cdot (2^7)^4 + 2^{32} = (5 \cdot 2^7)^4 + 2^{2^5}$ thus $\mod{(5 \cdot 2^7)^4 + 2^{2^5}}{0}{641}$. \\ However, $\mod{5 \cdot 2^7}{-1}{641}$ thus $\mod{(5 \cdot 2^7)^4}{1}{641}$ \\ therefore, $\mod{2^{2^5} + 1}{0}{641}$ \\

\noindent Using Pepin's and other primality tests, all Fermat numbers $F_5$ through $F_{32}$ have been proven to be composite. It is unknown whether any other Fermat primes exist.


\newpage

\section{Primality Tests}

\begin{theorem}[Wilson's Theorem]
$p \divides (p - 1)! + 1$ iff $p$ is prime.
\end{theorem}

\begin{proof}
By Lemma~\ref{numsoltocongruence}, for each $a \in \Phi(p)$, there exists a unique $x \in \Phi(p)$ s.t. \\ $\mod{ax}{1}{p}$ because $a \perp p$. Now consider $x$, because $\mod{xa}{1}{p}$ then $a$ is the unique solution for $x$ Suppose that $\mod{a^2}{1}{p}$ then $p \divides a^2-1$ so by Euclid's Lemma, $p \divides a-1$ or $p \divides p+1$ i.e. $\mod{a}{\pm 1}{p}$. Thus every element of $Phi(p)$ has a unique inverse distinct from itself except for $1$ and $p-1$. Therefore, $\mod{(p-1)!}{1 \cdot (p-1)}{p}$ so $\mod{(p-1)!}{-1}{p}$ i.e. $p \divides (p-1)! + 1$. \\\\
Conversely, let $n = ab$ where $a < n$. Then $a \divides (n-1)!$ but if $n \divides (n-1)! + 1$ then $a \divides (n-1)! + 1$ so $a \divides (n-1)!  + 1 - (n-1)!$ so $a \divides 1$. Thus, $n$ is prime. 
\end{proof}

\begin{theorem}[Lucas's Test]
If there exists an $a$ s.t. $\mod{a^{n-1}}{1}{n}$ and for every proper divisor $m$ of $n-1$, $\nmod{a^m}{1}{n}$ then $n$ is prime.
\end{theorem}

\begin{proof}
Let $\mod{a^{n-1}}{1}{n}$ thus, $ord_n(a) \divides n - 1$. Assume that $ord_n(a) < n - 1$. However, since $ord_n(a)$ is a divisor of $m$ it is a proper divisor and therefore, \\ $\nmod{a^{ord_n(a)}}{1}{n}$ but by definition the order satisfies this relation \cont\ The assumption is false, $ord_n(a) = n - 1 $ and thus, $n - 1 \divides \phi (n)$ so by Corollary~\ref{primeiffphi}, $n$ is prime. 
\end{proof}

\begin{theorem}[Proth's Theorem]
Let $n = h \cdot 2^m + 1$ where $h < 2^m$ and $2 \ndivides h$. \\ If $p$ is an odd prime and $\mod{p^{(n-1)/2}}{-1}{n}$ then $n$ is prime.  
\end{theorem}

\begin{proof}
Let $\mod{p^{(n-1)/2}}{-1}{n}$ and let $q$ be a prime divisor of $n$ i.e. $n = qc$. Because $q \divides n$ then $\mod{p^{(n-1)/2}}{-1}{q}$ so by Lemma~\ref{coprimeifdividespowerminusone}, $p \perp q$. Because $\mod{p^{(n-1)/2}}{-1}{q}$ then $\mod{p^{n-1}}{1}{q}$ therefore, $ord_q(p) \ndivides (n-1)/2$ and $ord_q(p) \divides n-1$ thus \\ $h \cdot 2^{m-1} \ndivides ord_q(p) \divides h \cdot 2^m$ therefore, $2^m \divides ord_q(p)$. However, $ord_q(p) \divides q-1$ so $2^m \divides q-1$ Write $q - 1 = r \cdot 2^m$ for $ r \ge 1$. Thus, $n - q$ = $(h-r) \cdot 2^m$ so $2^m \divides n - q$ However, $n = qc$ so $n - q = q(c-1)$ so $ 2^m \divides q(c-1)$. $q$ is odd so $q \perp 2^m$ so by Euclid's Lemma, $2^m \divides c-1$. Write $c - 1 = s \cdot 2^m$ thus $n = qc$ so $h \cdot 2^m + 1 = (r \cdot 2^m + 1)(s \cdot 2^m + 1)$. Simplifying, $h = rs \cdot 2^m + r + s$ but by hypothesis, $h < 2^m$ so $s = 0$ thus $c = 1$ therefore $n = q$ implying that $n$ is prime.
\end{proof}

\begin{lemma}
\label{primesarestrongcriteria}
Let $p = d \cdot 2^s + 1$ where $2 \ndivides d$ be prime. Then for any $a$ s.t. $p \ndivides a$, \\ either $\mod{a^d}{1}{p}$ or for some $0 \le r \le s-1$ it holds that $\mod{a^{d \cdot 2^r}}{-1}{p}$
\end{lemma}

\begin{proof}
If $a \perp p$ then by Fermat's Little Theorem, $p \divides a^{p-1} - 1$. However, \\ $a^{p-1} - 1 = (a^d)^{2^s} - 1$ factoring, $a^{p-1} - 1 = ((a^d)^{2^{s-1}} - 1)((a^d)^{2^{s-1}} + 1)$ factoring each difference of squares, $a^{p-1} - 1 = (a^d - 1)(a^d + 1)(a^{d \cdot 2} + 1)(a^{d \cdot 2^2} + 1) \cdots (a^{d \cdot 2^{s-1}} + 1)$ but $p \divides a^{p-1} - 1$ so  by Euclid's Lemma $p$ divides at least one factor. \\ Thus, $p \divides a^d - 1$ or for some $0 \le r \le s-1$, $p \divides a^{d \cdot 2^r} + 1$. 
\end{proof}

\begin{theorem}[Miller-Rabin Test]
Let $n = d \cdot 2^s + 1$ where $2 \ndivides d$ and choose $a \perp n$.\\ If $\nmod{a^d}{1}{n}$ and for every $0 \le r \le s-1$, $\nmod{a^{d \cdot 2^r}}{-1}{n}$ then $n$ is composite.
\end{theorem}

\begin{proof}
The theorem follows from the contrapositive of Lemma~\ref{primesarestrongcriteria}. Numbers that fail the Miller-Rabin Test for $a$ are known as strong probable primes for base $a$. 
\end{proof}

\end{document}