\documentclass[12pt]{extarticle}
\usepackage[utf8]{inputenc}
\usepackage[english]{babel}
\usepackage[a4paper, total={6in, 9in}]{geometry}
\usepackage{tikz-cd}
 
\usepackage{amsthm, amssymb, amsmath, centernot}

\newcommand{\notimplies}{%
  \mathrel{{\ooalign{\hidewidth$\not\phantom{=}$\hidewidth\cr$\implies$}}}}

\renewcommand\qedsymbol{$\square$}
\newcommand{\cont}{$\boxtimes$}
\newcommand{\divides}{\mid}
\newcommand{\ndivides}{\centernot \mid}
\newcommand{\Z}{\mathbb{Z}}
\newcommand{\N}{\mathbb{N}}
\newcommand{\C}{\mathbb{C}}
\newcommand{\Zplus}{\mathbb{Z}^{+}}
\newcommand{\Primes}{\mathbb{P}}
\newcommand{\ball}[2]{B_{#1} \! \left(#2 \right)}
\newcommand{\Q}{\mathbb{Q}}
\newcommand{\R}{\mathbb{R}}
\newcommand{\Rplus}{\mathbb{R}^+}
\newcommand{\invI}[2]{#1^{-1} \left( #2 \right)}
\newcommand{\End}[1]{\text{End}\left( A \right)}
\newcommand{\legsym}[2]{\left(\frac{#1}{#2} \right)}
\renewcommand{\mod}[3]{\: #1 \equiv #2 \: (\mathrm{mod} \: #3) \:}
\newcommand{\nmod}[3]{\: #1 \centernot \equiv #2 \: (\mathrm{mod} \: #3) \:}
\newcommand{\ndiv}{\hspace{-4pt}\not \divides \hspace{2pt}}
\newcommand{\finfield}[1]{\mathbb{F}_{#1}}
\newcommand{\finunits}[1]{\mathbb{F}_{#1}^{\times}}
\newcommand{\ord}[1]{\mathrm{ord}\! \left(#1 \right)}
\newcommand{\quadfield}[1]{\Q \small(\sqrt{#1} \small)}
\newcommand{\vspan}[1]{\mathrm{span}\! \left\{#1 \right\}}
\newcommand{\galgroup}[1]{Gal \small(#1 \small)}
\newcommand{\ints}[1]{\mathcal{O}_{#1}}
\newcommand{\sm}{\! \setminus \!}
\newcommand{\norm}[3]{\mathrm{N}^{#1}_{#2}\left(#3\right)}
\newcommand{\qnorm}[2]{\mathrm{N}^{#1}_{\Q}\left(#2\right)}
\newcommand{\quadint}[3]{#1 + #2 \sqrt{#3}}
\newcommand{\pideal}{\mathfrak{p}}
\newcommand{\inorm}[1]{\mathrm{N}(#1)}
\newcommand{\tr}[1]{\mathrm{Tr} \! \left(#1\right)}
\newcommand{\delt}{\frac{1 + \sqrt{d}}{2}}
\renewcommand{\Im}[1]{\mathrm{Im}(#1)}
\newcommand{\modring}[1]{\Z / #1 \Z}
\newcommand{\modunits}[1]{(\modring{#1})^\times}
\renewcommand{\empty}{\varnothing}
\renewcommand{\d}[1]{\mathrm{d}#1}
\newcommand{\deriv}[2]{\frac{\d{#1}}{\d{#2}}}
\newcommand{\pderiv}[2]{\frac{\partial{#1}}{\partial{#2}}}
\newcommand{\parsq}[2]{\frac{\partial^2{#1}}{\partial{#2}^2}}
\newcommand{\id}{\mathrm{id}}
\newcommand{\adele}[1]{\mathbb{A}_{#1}}
\newcommand{\idele}[1]{\mathbb{I}_{#1}}
\newcommand{\udele}[1]{\mathbb{I}^1_{#1}}
\newcommand{\sadele}[2]{\mathbb{A}_{#1, #2}}
\newcommand{\sidele}[2]{\mathbb{I}_{#1, #2}}
\newcommand{\sudele}[2]{\mathbb{I}^1_{#1, #2}}
\newcommand{\Hom}[2]{\mathrm{Hom}\left( #1, #2 \right)}


\theoremstyle{definition}
\newtheorem{theorem}{Theorem}[section]
\newtheorem{lemma}[theorem]{Lemma}
\newtheorem{proposition}[theorem]{Proposition}
\newtheorem{corollary}[theorem]{Corollary}
\newtheorem{remark}[theorem]{Remark}
\newtheorem{example}[theorem]{Example}


\newenvironment{definition}[1][Definition:]{\begin{trivlist}
\item[\hskip \labelsep {\bfseries #1}]}{\end{trivlist}}


\newenvironment{lproof}{\begin{proof} \renewcommand{\qedsymbol}{}}{\end{proof}}


\begin{document}
\section{Innequalities of Class Field Theory }
\begin{theorem}[First Innequality]
Let $L / K$ be a finite extension of number fields and $C_K = \idele{K} / K^\times$ and $C_L = \idele{K} / K^\times$. There is a norm map $N_{L/K} : C_L \to C_K$ then,
\begin{itemize}
\item the group $CK/N_{L/K}$ is finite.
\item Let $h = |CK/N_{L/K}|$, we have $h \le [L : S]$ 
\item If $L/K$ is abelian then $h \ge [L K]$
\end{itemize}
\end{theorem}

\begin{theorem}[Second Innequality]
If $L/K$ is abelian, then $h \le [L : K]$
\end{theorem}

\section{Artin Reciprocity Existence April (16)}

\begin{theorem}
The theorem has two parts,
\begin{enumerate}
\item[(a)]
Let $r : \mathbf{A}^\times \to \galgroup{L/K}$ be denoted by $r((a_v)) = \prod_v r_v(a_v)$ then $r(a) = 1$ for all $a = (a) \in K^\times \subset \mathbf{A}^\times$.

\item[(b)]
Let $\alpha \in Br(L/K) = H^2(\galgroup{L/K}, L^\times)$ then $\sum_{v} inv_v(\alpha) = 0$.
\end{enumerate}
\end{theorem}

\begin{lemma}
Let $(a_v) \in \mathbf{A}^\times_K$ and $G = Gal(L/K)$ for $L/K$ abelian, let $\chi \in \hat{G} = H_m(G, \Q / \Z) = \hat{H}^1(G, \Z /\Z) \xrightarrow{\delta} H^2(G, \Z)$. Then,
\[ \sum_{v} inv_v(\bar{a}_v, \delta(x)) = \chi(r_{L/K}((a_v)))\]
Where $(\bar{a}_v) \in \hat{H}^0(G, \mathbf{A}_K^\times)$ and
$\delta(x) \in \hat{H}^2 (G, \Z)$ is mapped to,
\[\bar{a}_v \cup \delta(x) \in \hat{H}^2(G, \mathbf{A}_K^\times) \hookrightarrow \bigoplus_v H^2(G^\nu, (L^\nu)^\times)\]
\end{lemma}

\begin{corollary}
Let $L/K$ be a cyclic cyclotomic extension. Suppose that $(a)$ of reciprocity theorem holds for $K$ then part $(b)$ holds fror any $\alpha \in Br(L/K)$. In particular, $(a) \implies (b)$. 
\end{corollary}

\begin{proof}
Since $L/K$ is cyclic, we may take $\chi \in \hat{G}$ injective then, 
\[\cup \delta \chi : \hat{H}^0(G, A) \to \hat{H}^2(G, A) = \hat{H}^2(G, A \otimes \Z)\]
is an isomorphism. Apply this to $K^\times$ so we have,
\begin{center}
\begin{tikzcd}[column sep = huge]
 K^\times / N_{L/K} L^\times = \hat{H}^0(G, L^\times) \arrow[r, "\cup \delta \chi"] & Br(L/K) = H^2(G, L^\times)
\end{tikzcd}
\end{center}
which is an isomorphism. Take $\alpha = \bar{a} \cup \delta \chi$ for some $a \in K^\times$ and $\bar{a}$ is the image of $a$ in $\hat{H}^0$. The lemma above then yields,
\[ \sum_v inv_v(\alpha) = \sum_v inv_v(\bar{a} \cup \delta \chi) = \chi(r_{L/K}(a))\]
Assuming $(a)$ we then have $r_{L/K}(a) = 1$ so $\chi(r_{L/K}(a)) = 1$ so $\sum_{v} inv_v(\alpha_v) = 0$. Thus, $(a) \implies (b)$. 

\end{proof}

Summary of Proof.

\begin{proof}
We have shown that,
\begin{enumerate}
\item $(a) \implies (b)$ for $\alpha$ split by a cyclic extension for which $(a)$ holds. 

\item Every $\alpha$ is split by some cyclic extension for which $(a)$ holds.

\item We know $a)$ if $L/K$ is cyclotomic.

\item Therefore, we have proven $(b)$ in generality. 

\item Finally, $(b) \implies (a)$.
\end{enumerate}
\end{proof}

\begin{proof}[$(b) \implies (a)$]
It suffices to show that $\forall \chi \in \hat{G}$ that $\chi(r_{L/K}(a)) = 1$. Let $a \in K^\times$ for $a^\ast$ the image of $a$ in $K^\times / N_{L/k} L^\times = \hat{H}^2(G, L^\times)$. Then, using the fact that the cup product is natural,
\begin{center}
\begin{tikzcd}
\hat{H}^0(G, L^\times) \arrow[r, "\cup \delta \chi"] \arrow[d] & \hat{H}^2(G, L^\times) \arrow[d] 
\\
\hat{H}^0(G, \mathbb{A}^\times_L) \arrow[r, "\cup \delta \chi"] & \hat{H}^2(G, \mathbb{A}^t\times_L)
\end{tikzcd}
\end{center}
we see that,
\begin{center}
\begin{tikzcd}[column sep = tiny, row sep = huge]
a^\ast \cup \delta \chi \in \arrow[d] &  \hat{H}^2(G, L^\times) \subset Br(K) \arrow[d]
\\
\bar{a} \cup \delta \chi \in & \hat{H}^2(G, \mathbb{A}^\times_L) 
\end{tikzcd}
\end{center}
By part $(b)$, we know that,
\[ \sum_{v} inv_v(a^\ast \cup \delta \chi) = \sum_{v} inv_v( \bar{a} \cup \delta \chi) = \chi(r_{L/K}(a)) \]
Therefore, $\chi(r_{L/K}(a)) = 1$ for each $\chi \in \hat{G}$. 
\end{proof}

\begin{corollary}
Suppose $F \subset F' \subset E$ is a tower of abelian extensions of $p$-adic fields. Let $G = \galgroup{E/K} \supset H = \galgroup{E/F}$. Suppose that, 
\[\chi' \in \Hom{G/H}{\Q/\Z} = \widehat{G/H} = H^1(G/H, \Q/ \Z) \xrightarrow{\delta} H^2(G/H, \Z) \]
is mapped to,
\[ \chi \in \Hom{G}{\Q / \Z} \xrightarrow{\sim} H^2(G, \Z) \] 
with $inf_{G/H}^G(\delta(\chi')) = \delta(inf_{G/H}^G(\chi') = \delta(\chi)$. Suppose $a \in F^\times$ then $\chi(r_{E/F}(a)) = \chi^1(r_{F'/F}(a))$.
\end{corollary}

\begin{proof}
$\chi(r_{E/F}(a)) = inv_F(\bar{a} \cup \delta (\chi)) = inv_F(\bar{a} \cup \delta (\chi')) = \chi'(r_{F'/F}(a))$ then $\bar{a} \in \hat{H}^0(G, E^\times)$ and $a \in \hat{H}^0(G/H, F'^\times)$. 

\begin{center}
\begin{tikzcd}
F^\times \arrow[r, "r_{F^{ab}/F}"] & W_F \arrow[d] \arrow[r] & \galgroup{F^{ab}/F} \arrow[d] 
\\
& Frob^\Z \arrow[r, hook] & \galgroup{F^{un}/ F} = \lim_{\leftarrow} \Z / n \Z 
\end{tikzcd}
\end{center}
\end{proof}

\begin{theorem}[Artin Existence Theorem]
Let $K$ be a number field let $C_K = \mathbb{A}^\times_K / K^\times$ let $U \subset C_K$ be an open subgroup of finite index. Then, there is an ablian extension $L/K$ such that $U = N_{L/K} C_L$. We call $U$ a norm subgroup. 
\end{theorem}

\begin{lemma}
Suppose $V \subset U$ and $V$ is a norm subgroup then $U$ is a norm sugbgroup.
\end{lemma}

\begin{proof}
Let $V = N_{E/K} C_E$  thus $r : C_K / N_{E/K} C_E  \xrightarrow{\sim} \galgroup{E/K}$ maps $U/V \xrightarrow{r} H = r(V)$.  Let $L = E^H$. Thus, $C_K / U \xrightarrow{\sim} (C_K/V)/(U/V) = \galgroup{E/K} / \galgroup{E/L} = \galgroup{L/K}$. By the corollary, $\forall a \in C_k$ then $r_{E/K}(a) |_L = r_{L/K}(a)$. Therefore, $\ker{r_{L/K}} = U$ but by the reciprocity law, $\ker{r_{L/K}} = N_{L/K} C_L$ so $U$ is a norm subgroup. 
\end{proof}

\begin{proposition}
Suppose that $\zeta_n \in K$ let $S \supset S_{\infty}$ be a sufficiently large set of primes containing all primes dividing $n$ and generators of $Cl(K)$. Suppose $a \in K^\times$ satisfies, 
\[ [a] \forall v \in S, a \in (K_n^\times)^n \quad [b] \forall v \notin S, a \in \mathcal{O}_v^\times \]
then $a \in (K^\times)^n$. 
\end{proposition}

\begin{proof}
Let $L = K(\sqrt[n]{a})$. We know that $H^1$ is abelian over $K$ by  hypothesis, and we show $L = K$. y $[a]$, every $v \in S$ splits completely in $L$. On the other hand, every $v \notin S$ is unramified in $L$ because $a \in \mathcal{O}_v^\times$ and $K_v(\sqrt[n]{a})$. If $a \in 1 + m_v$ then it is already an $n^{\text{th}}$ root. $\bar{a} \in ( \mathcal{O}_v / m_v )^\times$ which implies $K_v(\sqrt[n]{a}) = K_v(\sqrt[n]{\omega(\bar{a})})$ is unramified. However, $K_v^\times = N_{Lw/K_v} L_w^\times$ for $v \in S$ which implies that $\mathcal{O}_v^\times = N_{L_w/K_v} \mathcal{O}_w^\times \subset N_{L_w/K} L_w^\times$ for all $n \notin S$. Thus, $N_{L/K} \mathbb{A}_L^\times \supset \mathbb{A}^\times_{K,S}$ However, $S$ has been chosen to contain a set of generators of $CL(K)$ so we see that $\mathrm{A}^\times_{K,S} \cdot K^\times = \mathbb{A}_K^\times$ so $N_{L/K} C_L = C_K$. So $[L : K] = C_K / N_{L/K} C_L = 1$ so $\sqrt[n]{a} \in K$.  
\end{proof}

\begin{lemma}
LEt $p$ be a prime, $\zeta_p \in K$. Let $\bar{V} \subset C_K$ be an open subgroup (the image of $V \subset \mathbb{A}_K^\times$) such that $C_K / V$ is anihilated by $p$. Then $\bar{V}$ is a norm subgroup. 
\end{lemma}

\section{April 18}

\begin{lemma}
Let $p$ be a prime, $\zeta_p \in K$, and $V \subset \mathbb{A}_K^\times$ open with $\delta = V \cdot K^\times / K^\times$ such that $(C_K)^p \subset \bar{V}$ i.e. $p \cdot D_K / \bar{V} = 0$ then $\bar{V}$ is a norm subgroup. 
\end{lemma}

\begin{proof}
Let $S$ be a set of places as in the proposition $\mathbb{A}_{K,S}^\times \cdot K^\times / K^\times = C_K$ such that $S \supset S_{\infty}$. Let $\mathcal{U} = U_{K,S}$  be the groups of $S$-units of $K$ then $\mathcal{U}/T(\mathcal{U}) = \Z^{|S|-1}$ by the unit theorem. Let $L = K(\sqrt[p]{u}) = K(u^{1/p}, u \in \mathcal{U})$ This a finite extension. Let,
\[W = W_S = \prod_{v \in S} (K^\times_v)^p \times \prod_{v \notin S} \ints{v}^\times \subset \mathbb{A}_{K,S}^\times\]
We prove that $\bar{W} = W \cdot K^\times / K^\times \subset C_K = N_{L/K}(C_L)$. In particular we need to prove two facts,
\begin{enumerate}
\item $W \subset N_{L/K} \mathbb{A}_{L}^\times$

\item $[C_K : \bar{W} ] = [ C_K : N_{L/K} C_L ] = p^|S|$
\end{enumerate}
The proof of the first fact is purely local. For any $v$, we have $N_{L/K} L_v^\times \supset (K_v^\times)^p$ and $L_v = K_v(\sqrt[p]{u})$. By the local Artin map, $K^\times_v / N_{L_w^\times / K_v^\times} L_w^\times \cong \galgroup{L_w^\times / K_n^\times} \cong (\Z / p \Z)^r$ so $K^\times_v / N_{L_w^\times / K_v^\times} L_w^\times$ has exponent $p$ and thus $(K_v^\times)^p \subset N_{L/K} L_w^\times$. For $n \notin S$, $L$ is unramified at $v$ so the local units are contained in the image of the local norm.
\bigskip\\
To prove $2$, we know that,
\[ [C_K : N_{L/K} C_L] = |\galgroup{L/K} | = [ \mathcal{U} \cdot (K^\times)^p : (K^\times)^p ] \]
by local reciprocity and kummer theory. To see this, consider the short exact sequence,
\begin{center}
\begin{tikzcd}
1 \arrow[r] & \mu_p \arrow[r] & \bar{K}^\times \arrow[r, "u \mapsto u^p"] & \bar{K}^\times \arrow[r] & 1
\end{tikzcd}
\end{center}
Appllying the functor $H^1(\galgroup{L/K}, -)$ by Kummer theory,
\[ K^\times / (K^\times)^p \cong \Hom{\galgroup{\bar{K}/K}}{\mu_p}\] since $\mu_p \subset K$. It suffices to show that, 
\[[C_K : \bar{W} ] = [ \mathcal{U} \cdot (K^\times)^p : (K^\times)^p ] = |\mathcal{U}/\mathcal{U}^p | = p^{|S| - 1} \cdot p \]
from the torsion-free and torsion groups respectively.
However, $\mathcal{U} \cap (K^\times)^p = \mathcal{U}^p$. Now, 
\[ [C_K : \bar{W}] = [ \mathbb{A}_{K,S}^\times \cdot K^\times : W \cdot K^\times ] = \frac{ [ \mathbb{A}_{K, S}^\times : W ] }{ [ \mathbb{A}_{K, S}^\times \cap K^\times : W \cap K^\times ]} = \frac{ [ \mathbb{A}_{K,S}^\times : W ] }{ [ \mathcal{U} : \mathcal{U}^p ] } \]
because $K^\times \cap W \subset (K^\times)^p$ by rhe proposition.  However, 
\[ \mathbb{A}_{K, S}^\times / W = \prod_{v \in S} K_v^\times / (K_v^\times)^p \]
But if $F$ is a local field containing $\zeta_p$ then $[ F^\times : (F^\times)^p ] = \frac{ p^2 }{ || p ||_F}$ thus,
\[ | \mathbb{A}_{K,S}^\times / W | = \prod_{v \in S} \frac{p^2}{|| p ||_v} - \prod_{v \in S} p^2 = p^{2 |S|} \]
since $\prod_{v \in S} || p ||_v = \prod_v ||p ||_v = 1$ since $||p||_v = 1$ for $v \notin S$. Similarly,
\[  [ \mathcal{U} : \mathcal{U}^p ] = p^{|S|} \]
which implies that,
\[ [C_K : \bar{W} ] = \frac{ p^{2 |S|} }{ p^{|S|} } = p^{|S|} \]
as required. We have shown that $\bar{\mathcal{U}}$ is an open subgroup. Furthermore, $\bar{V} \supset C_K^p$ and $V$ also contains $\prod_{v \notin S} \ints{v}^\times$ for some $S$. Therefore, $\bar{V} \supset \bar{W}$ for some sufficiently large $S$. Thus $\bar{V}$ is a norm subgroup.  
\end{proof}

\begin{lemma}
Let $U \subset C-k$ be an open subgroup of finite index. Suppose there exists a finite cyclic extension $K' / K $ such that $U' = N_{K'/K}^{-1}(U)$ is a norm subgroup of $C_{K'}$. Then $U$ is a norm subgroup.
\end{lemma}

\begin{corollary}
The lemma holds for $K$ such that $\zeta_p \notin K$. 
\end{corollary}

\begin{proof}
Let $K' = K(\zeta_p)$ this is cyclic and $\bar{V}' = N_{K'/K}^{-1}(\bar{V})$ satisfies the hypothesis of the above lemma. Thus,
\[ N_{K'/K} : C_{K'} / \bar{V}' \hookrightarrow C_K / \bar{V} \]
Therefore $C_{K'} / \bar{V}'$ is $p$-torsio. Thus, we reduce to the case of $K'$.
\end{proof}

\begin{theorem}[Global Existence]
Let $U \subset C_K$ be an open subgroup of finite index then $U$ is a norm subgroup.
\end{theorem}

\begin{proof}
Let $D = [C_K : V]$ and let $p$ be a prime dividing $D$. Take $K' = K(\zeta_p)$. It suffies, by the above lemma, to prove that $U' = N_{K'/K}^{-1}(U) \subset C_{K'}$ is a norm subgroup. Let $D' = [C_{K'} : U' ] $ we have $D' \divides D$. By inductin on $D$, we may assume $D' = D$. Let $C_{K'} \supset V \supset V'$ with $[C_{K'} : V] = p$. By the lemma, $V$ is a norm subgroup corresponding to the cyclic extension $L/K'$. Let $U'' = N_{L/K'}^{-1}(U')$ if can show $U'' \subset C_L$ is a norm subgroup then we are done by the lemma. It suffices, by induction, to show that $[C_L : U''] < [C_{K'} : U' ] = D$. However, we have the injection,
\[ N_{L/K} : C_L / U'' \hookrightarrow C_{K'} / U' \]
With $N_{L/K}(C_L) = V$ and the image is $V / U'$ of order $D/p < D$.
\end{proof}

\begin{theorem}[Kronecker-Weber]
Any abelian extension of $\Q$ is contained in a cyclotomic field.
\end{theorem}

\begin{proof}
Consider $\mathbb{A}_{\Q}^\times / \Q^\times = \left( \mathbb{A}_{\Q, \infty}^\times \cdot \Q^\times \right) / \Q^\times$ where $\mathbb{A}_{\Q, \infty}^\times = \R_{+}^\times \times \prod_{p} \Z_p^\times$. Let $L/\Q$ be an abelian extension and $\bar{U} \subset C_\Q$ the corresponding norm subgroup $U \subset \mathbb{A}^\times_{\Q}$ so the map $\prod_p \Z_p^\times \to \galgroup{L / \Q}$ is surjective. The kernel contains $U_m$ for some $m$ where \[U_m = \{ (x_n) \in \prod_{p} \Z_p^\times \mid \mod{(x_v)}{1}{m} \}\]
so it suffices to show that $N_{ \Q(\zeta_n)/\Q } C_{\Q(\zeta_m)} \supset (\R_+^\times \times U_m) \cdot \Q^\times / \Q^\times$ which implies that $L \subset \Q(\zeta_m)$. We reduce to the case $m = q^r$ for a prime $q$ then the norm at $p$ is surjective on units $p \neq q$. At $q$,
\[ N_{\Q_q(\zeta_{q^r}) / \Q_q} \Q_q (\zeta_{q^r})^\times \]   
\end{proof}

\section{April 23: Proofs of Local Class Field Theory}

\begin{lemma}
Any subgroup $U \subset K^\times$ containing $N(L^\times)$ for some $L$ is a norm subgroup. 
\end{lemma}

\begin{remark}
For $\pi \in K^\times$ a uniformizer. Let $K_{\pi, n}^{Art.} / K$ be the finite abelian extension such that $N((K_{\pi, n}^{Art.})^\times) = \pi^\Z \times (1 + \pi^n \ints{K})$ if $n \ge 1$. This exists by local existence. 
\end{remark}


\section{Hilbert Theorem 90}

\newcommand{\Gal}[1]{\mathrm{Gal} (#1)}
\newcommand{\Nm}{\mathrm{Nm}}
\newcommand{\Tr}{\mathrm{Tr}}
\newcommand{\Frob}{\mathrm{Frob}}
\newcommand{\m}{\mathfrak{m}}
\newcommand{\inv}{\mathrm{inv}}

\begin{theorem}
Let $L / K$ be a finite Galois extension then $H^1(\Gal{L/K}, L^\times) = 0$.
\end{theorem}

\begin{proof}
Let $\varphi \in Z^1(G, L^\times)$. We need to show that $\varphi \in B^1(G, L^\times)$ i.e. there exists some $m \in L^\times$ such that $\varphi(g) = (g \cdot m)m^{-1}$. Pick $a \in L^\times$. Construct,
\[ m = \sum_{g \in G} \varphi(g) \cdot g(a) \]
We may ensure that $m \neq 0$ because the characters $g : L^\times \to L^\times$ are linearly independent the linear combination,
\[ \sum_{g \in G} \varphi(g) g \]
is a nonzero character and thus does not vanish at some $a \in L^\times$. Now take,
\[ g \cdot m = \sum_{h \in G}  g \left( \varphi(h) \cdot h(a) \right) = \sum_{h \in G} g(\varphi(h)) \cdot gh(a) \]
However, since $\varphi \in Z^1(G, L^\times)$ then $g \varphi(h) = \varphi(gh) \varphi(g)^{-1}$ which implies that,
\[ g \cdot m = \varphi(g)^{-1} \sum_{h \in G} \varphi(gh) \cdot gh(a) = \varphi(g)^{-1} m \]
Therefore,
\[ \varphi(g) = \frac{m}{g \cdot m} = \frac{g(m^{-1})}{m^{-1}} \in B^1(G, L^\times) \]  
\end{proof}

\begin{example}
Assume that $L / K$ is cyclic then,
\[ H^1(G, L^\times) = \hat{H}^{-1}(G, L^\times) = \ker{\Nm_G} / \Im{(\sigma - 1)}  = 0 \]
Therefore, for any $a \in L^\times$ such that $\Nm_{L / K}(a) = 1$ then $\exists b \in L^\times$ such that 
\[ a = (\sigma b) b^{-1} \]
\end{example}

\begin{example}
For $L / K = \Q(i) / \Q$ let $a = x + i y$ and $b = m + i n$ for $x, y \in \Q$. Then $\Nm_{L/K}(a) = x^2 + y^2 = 1$ implies that,
\[ a = \bar{b} b^{-1} = \frac{m - i n}{m + i n} = \frac{m^2 - n^2}{m^2 + n^2} + \frac{2 mn}{m^2 + n^2} i \]
i.e.
\[ x = \frac{m^2 - n^2}{m^2 + n^2} \quad \quad y = \frac{2 mn}{m^2 +  n^2} \quad \text{for} \quad m,n \in \Q \]
\end{example}

\begin{remark}
We may also consider $L / K$ infinite Galois. Then, over the finite subextensions $K \subset L' \subset L$,
\[ \Gal{L/K} = \varprojlim_{L' \subset L} \Gal{L'/K} \]
is a profinite group. Then we define the continuous group cohomology,
\[ H^r_{\text{cts}}(\Gal{L/K}, L^\times) = \varinjlim_{L' \subset L} H^r(\Gal{L'/K}, (L')^\times) \] 
and more generally, if $G$ is a profinite group,
\[ G = \varprojlim_{H} G / H \]
then,
\[ H^r_{\text{cts}}(G, M) = \varinjlim_{H} H^r(G / H, M^H) \]
This can be computed using continuous cochains under the profinite topology. 
\end{remark}

\section{$H^2$ of Unramifield Extensions}

Let $L / K$ be a finite unramified extension of local fields. Recall that the Galois groups is computed on the residue fields,
$\Gal{L / K} = \Gal{\ell / k} = \left< \Frob_{L/K} \right>$ which is cyclic. We need to prove that,
\[ H^2(\Gal{L/K}, L^\times) \cong \Z / n \Z \]

\begin{definition}
Let $U_K = \ints{K}^\times$ be the units and,
\[ U_K^{(i)} = 1 + \m_K^i \]
be the $i$-units. Therefore, we have a filtration,
\[ U_k \supset U_K^{(1)} \supset U_K^{(2)} \supset U^{(3)}_K \supset \cdots \]
\end{definition}

\begin{proposition}
There are exact sequences.
\begin{center}
\begin{tikzcd}
1 \arrow[r] & U_K^{(1)} \arrow[r] & U_K \arrow[r] & k^\times \arrow[r] & 1
\end{tikzcd}
\end{center}
and similarly
\begin{center}
\begin{tikzcd}
1 \arrow[r] & U_K^{(i+1)} \arrow[r] & U_K^{(i)} \arrow[r] & k \arrow[r] & 0
\end{tikzcd}
\end{center}
via $1 + a \pi_K^n \mapsto a \in k$. 
\end{proposition}

\begin{lemma}
Let $G = \Gal{L/K} = \Gal{\ell / k}$. Then,
\[ \hat{H}^r(G, \ell^\times) = \hat{H}^r(G, \ell) = 0 \]
\end{lemma}

\begin{proof}
By Hilbert's theorem 90, $H^1(G, \ell^\times) = 0$. Since $G$ is cylic and $\ell^\times$ is finite, we know its Herbrand quotient $h(\ell^\times) = 1$ and thus $H^2(G, \ell^\times) = 0$. Since $G$ is cylic the entire cohomology is determined by these two terms. 
\end{proof}

\begin{corollary}
The maps $\Nm : \ell^\times \to k^\times$ and $\Tr : \ell \to k$ are surjective. 
\end{corollary}

\begin{proof}
This follows from the vanishing of Tate cohomology via,
\begin{align*}
\hat{H}^0(G, \ell^\times) & = \frac{k^\times}{\Nm{\ell^\times}} = 0 
\\
\hat{H}^0(G, \ell) & = \frac{k}{\Tr{\ell}} = 0 
\end{align*}
\end{proof}

\begin{lemma}
The norm map $\Nm : U_L \to U_K$ is surjective. 
\end{lemma}

\begin{proof}
Consider the diagrams,
\begin{center}
\begin{tikzcd}
U_L \arrow[d] \arrow[r, "\Nm"] & U_K \arrow[d] 
\\
\ell^\times \arrow[r, "\Nm"] \arrow[r] & k^\times
\end{tikzcd}
\end{center}

\begin{center}
\begin{tikzcd}
U_L^{(i)} \arrow[d] \arrow[r, "\Nm"] & U_K^{(i)} \arrow[d] 
\\
\ell \arrow[r, "\Tr"] \arrow[r] & k
\end{tikzcd}
\end{center}
Given $a \in U_K$. We want $b \in U_L$ such that $a \in \Nm(b)$. Since $\Nm : \ell^\times \to k^\times$ is surjective we way find $b_0 \in U_L$ such that $\Nm(b_0) \equiv a$ modulo $\m_k$. Lt $a_1 = a (\Nm(b_0))^{-1} \in U_K^{(1)}$. Since $\Tr : \ell \to k$ is surjective we may find $b_1 \in U_L$ such that $a_2 = a_1 (\Nm(b_1)) \in U_K^{(2)}$. Continue this way and let,
\[ b = \prod_{ i = 0}^\infty b_i \]
Then,
\[ \frac{a}{\Nm(b)} \in \bigcap_{i = 1}^\infty U_K^{(i)} = 1 \]
\end{proof}

\begin{corollary}
$\hat{H}^r(G, U_L) = 0$ for each $r \in \Z$. 
\end{corollary}

\begin{proof}
The lemma implies that $\hat{H}^0(G, U_L) = 0$. By Hilbert 90, $H^1(G, L^\times) = 0$. Furthermore, $L^\times = U_L \oplus \Z$ and thus,
\[ H^1(G, L^\times) = H^1(G, U_L) \oplus H^1(G, \Z) \]
which implies that $H^1(G, U_L) = 0$. Therefore, $\hat{H}^r(G, U_L) = 0$ for all $r \ge 0$ by the periodicity. 
\end{proof}

\begin{lemma}
We may identify cohomology of the trivial module $\Z$ with homs to the trivial module $\Q / \Z$,
\[ H^2(G, \Z) \cong \Hom{G}{\Q / \Z} \]
\end{lemma}

\begin{proof}
Consider the exact sequence of trivial $G$-modules,
\begin{center}
\begin{tikzcd}
0 \arrow[r] & \Z \arrow[r] & \Q \arrow[r] & \Q / \Z \arrow[r] & 0
\end{tikzcd}
\end{center}
which gives rise to a long exact sequence of cohomology. 
\end{proof}

\begin{theorem}
\[ H^2(G, L^\times) \cong \Hom{G}{\Q / \Z} \]
\end{theorem}

\begin{proof}
\[ H^2(G, L^\times) = H^2(G, U_L) \oplus H^2(G, \Z) \]
Furthermore, $H^2(G, U_L) = 0$ and $H^2(G, \Z) = \Hom{H}{\Q / \Z}$. 
\end{proof}

\begin{definition}
Let $L / K$ be a finite unramified extension. The \textit{invariant map} is the above isomorphism,
\[ \inv_{L/K} : H^2(G, L^2) \xrightarrow{\sim} \Hom{G}{\Q / \Z}  \]
Furthermore, $G$ is cyclic of degree $n$ so,
\[ \Hom{G}{\Q / \Z} \subset \tfrac{1}{n} \Z / \Z \subset \Q / \Z \]
Taking the direct limit we obtain,
\[ \inv_K : H^2(\Gal{K^{\text{un}} / K}, (K^\text{un})^\times) \xrightarrow{\sim} \Q / \Z \]
\end{definition}

\section{March 11}

Today's Goal:
\[ \inv_K : H^2(\Gal{K^{\text{sep}} / K}, (K^{\text{sep}})^\times) \cong \Q / \Z \]

\begin{remark}
For any Galois extension of fields $L / K$, we use the shorthand notation,
\[ H^2(L/K) = H^2(\Gal{L/K}, L^\times) \] 
\end{remark}

\begin{lemma}
Let $L / K$ be a finite extension of local fields of degree $n = [L : K]$. Then the following diagram commutes,
\begin{center}
\begin{tikzcd}
H^2(K^{\text{ur}} / K) \arrow[r, "\mathrm{Res}"] \arrow[d, "\inv_K"] & H^2(L^{\text{ur}} / L) \arrow[d, "\inv_L"]
\\
\Q / \Z \arrow[r, "\times n"] & \Q / \Z
\end{tikzcd}
\end{center}
\end{lemma}

\begin{remark}
Note that $L^{\text{un}} = L \cdot K^{\text{ur}}$ by adjoing all coprime to $p$ power roots of unity. So we may view,
\[ \Gal{L^{\text{un}} / L} \subset \Gal{K^{\text{ur}} / K} \]
which is compatible with $(K^\text{un})^\times \subset (L^\text{un})^\times$ which give a map,
\[ \mathrm{Res} : H^2(K^{\text{ur}} / K) \to H^2(L^{\text{ur}} / L) \]
\end{remark}


\newcommand{\sep}{\mathrm{sep}}
\newcommand{\ur}{\mathrm{ur}}
\newcommand{\Inf}{\mathrm{Inf}}
\newcommand{\Res}{\mathrm{Res}}


\begin{proof}
Consider the valuation map $\mathrm{val}_K : (K^\text{ur})^\times \to \Z$ via $\varpi_K^n \mapsto n$ which gives a diagram with isomorphisms in the columns,
\begin{center}
\begin{tikzcd}
H^2(K^\text{ur} / K) \arrow[d, "\mathrm{val}_K"] \arrow[r, "\mathrm{Res}"] & H^2(L^\text{ur} / L)  \arrow[d, "\mathrm{val}_L"] 
\\
H^2(K^\text{ur} / K, \Z) \arrow[d] \arrow[r, dashed] & H^2(L^\text{ur} / L, \Z)  \arrow[d] 
\\
H^1(K^\text{ur} / K, \Q / \Z) \arrow[d] \arrow[r, dashed] & H^1(L^\text{ur} / L, \Q / \Z) \arrow[d]
\\
\Q / \Z  \arrow[r, dashed] & \Q / \Z
\end{tikzcd}
\end{center}
The valuation maps give isomorphisms because the units have trivial cohomology. Let $e = e(L / K)$ be the ramification index, $f = f(L/K)$ the residue degree and chose $\varpi_K = \varpi_L^e$. We have diagrams,
\begin{center}
\begin{tikzcd}
(K^\text{ur} )^\times \arrow[d] \arrow[r, "\mathrm{val}_K"] & \Z \arrow[d, "\times e"]
\\
(L^\text{ur} )^\times \arrow[r, "\mathrm{val}_L"] & \Z
\end{tikzcd}
\end{center}
Since $\Frob_L = \Frob_K^f$ then $\Res(\varphi)(\Frob_L) = f \varphi(\Frob_K)$. 
Therefore, the final map fives multiplication by $ef = n$. 
\end{proof}

\begin{corollary}
Let $L / K$ be a finite Galois extension of degree $n = [L : K]$. Then, $H^2(L/K)$ contains a cyclic subgroup of order $n$.
\end{corollary}

\begin{proof}
By Hilbert 90, we can apply the inflation-restriction sequence to get,
\begin{center}
\begin{tikzcd}
0 \arrow[r] & H^2(L/K) \arrow[r,"\inf"] & H^2(K^{\text{sep}} / K) \arrow[r, "\Res"] & H^2(L^\sep / L) 
\\
0 \arrow[r] & \frac{1}{n} \Z / \Z \arrow[u, dashed, hook] \arrow[r] & H^2(K^\ur / K) \arrow[u, "\Inf", hook] \arrow[r, "\Res"] & H^2(L^\ur / L) \arrow[u, "\Inf", hook]
\\
& &  0 \arrow[u] & 0 \arrow[u]
\end{tikzcd}
\end{center}
Because the inflation map is injective then the induced map $\tfrac{1}{n} \Z / \Z \to H^2(L/K)$ is also injetive. 
\end{proof}

\begin{proposition}
$|H^2(L/K)| \le [L : K]$. 
\end{proposition}

\begin{proof}
If $L / K$ is cyclic, then the Herbrand quotient,
\[ h(L^\times) = [L : K] \]
However, by Hilbert 90, $H^1(L/K) = 0$ and thus $|H^2(L / K)| = [L : K]$. Otherwise, if $L / K$ is not cyclic, we use, the fact that $\Gal{L/K}$ for local fields is always solvable. Therefore, we may induct on the degree of the extension. Choose a nontrivial tower, $K \subset K' \subset L$ with $K' / K$ cyclic. Then by inflation-restriction,
\begin{center}
\begin{tikzcd}
0 \arrow[r] & H^2(K' / K) \arrow[r, "\Inf"] & H^2(L/ K) \arrow[r, "\Res"] & H^2(L / K')
\end{tikzcd}
\end{center} 
However, $K' / K$ is cyclic and thus $|H^2(K' / K)| = [K' : K]$. Furthermore, by the induction hypothesis, $|H^2(L / K')| \le [L ; K']$. Thus,
\[ |H^2(L / K) | \le [L : K'] [K' : K]  = [L  : K] \]  
\end{proof}

\begin{theorem}
Let $L / K$ be an extension of local fields. Then $H^2(L/K) = \tfrac{1}{n} \Z / \Z$. Furthermore, the following diagram commutes,
\begin{center}
\begin{tikzcd}
0 \arrow[r] & H^2(L/K) \arrow[r,"\inf"] & H^2(K^{\text{sep}} / K) \arrow[r, "\Res"] & H^2(L^\sep / L) 
\\
0 \arrow[r] & \frac{1}{n} \Z / \Z \arrow[u, "\sim"] \arrow[r] & H^2(K^\ur / K) \arrow[u, "\Inf", hook] \arrow[r, "\Res"] & H^2(L^\ur / L) \arrow[u, "\Inf", hook]
\end{tikzcd}
\end{center}
\end{theorem}

\begin{proof}
This follows immediately from the previous two propositions. 
\end{proof}

\begin{remark}
In particular, we may view $H^2(L/ K) \subset H^2(K^\ur / K)$ since is is isomorphic to a subgroup of $H^2(K^\ur / K)$ which inflates to $H^2(L / K)$.
\end{remark}

\begin{theorem}
There exists an isomorphism,
\[ H^2(K^\sep / K) \xrightarrow{\sim} H^2(K^\ur / K) \]
\end{theorem}

\begin{proof}
Notice $H^2(L / K) \subset H^2(K^\ur / K) \subset H^2(K^\sep / K)$. Taking the direct limit over finite $L / K$ we find that,
\begin{center}
\begin{tikzcd}
H^2(K^\sep / K) \arrow[r, hook] & H^2(K^\ur / K) \arrow[r, hook] & H^2(K^\sep / K) 
\end{tikzcd}
\end{center} 
since the composition of these surjections is the identity, each must be surjective and thus an isomorphism. 
\end{proof}

\begin{definition}
Composing with $\inv_K : H^2(K^\ur / K) \xrightarrow{\sim} \Q / \Z$ we obtain an \textit{invariant map}:
\[ \inv_K : H^2(K^\sep / K) \xrightarrow{\sim} \Q / \Z \]
defined on all seperable extensions. 
\end{definition}

\begin{definition}
We call the element $u_{L/K} \in H^2(L / K)$ with 
\[ \inv_K(u_{L/K}) = \frac{1}{[L : K]} \in \frac{1}{[L : K]} \Z / \Z \subset \Q / \Z \]
the \textit{fundamental class}.
\end{definition}

\newcommand{\Br}{\mathrm{Br}}

\begin{remark}
For any field $K$, the group
\[ \Br(K) = H^2(K^\sep / K, (K^\sep)^\times)  \]
is known as the \textit{Brouer Group} of $K$. It is the group of central simple algebras under Brouer equivalence. We have shown,
\[ \Br(K) \cong \Q / \Z \]
for any local field. For example, on the quaternion algebra, this map acts as, $\mathbb{H}_K \mapsto \tfrac{1}{2}$.
\end{remark}

\begin{remark}
The notion of a Brouer group generalizes to any scheme $X$:
\[ \Br(X) : = H^2_{\text{\'{e}t}}(X, \mathbb{G}_m) \]
\end{remark}

\begin{remark}
The local Artin reciprocity map, is defined to be the inverse of,
\begin{center}
\begin{tikzcd}
\hat{H}^{-2}(G, \Z) \arrow[d, "\sim"] \arrow[r, "\sim"] & \hat{H}^0(G, L^\times) \arrow[d, "\sim"]
\\
G^\mathrm{ab} \arrow[r] \arrow[r, "\sim"] & K^\times / \mathrm{Nm}(L^\times) 
\end{tikzcd}
\end{center}
This can be described explicitly in terms of $u_{L/K}$. Thre is a cup product pairing:
\[ \hat{H}^{-2}(G, \Z) \times H^2(G, L) \xrightarrow{\smile} \hat{H}^0(G, L^\times) \]
However, $H^2(G, L^\times) = H^2(L / K) = \left< u_{L/K} \right>$ so the fundamental class induces the map: $\hat{H}^{-2}(G, \Z) \to \hat{H}^0(G, L^\times)$
\end{remark}

\section{March 27}

\begin{remark}
Our goal is to construct the global Artin map,
\[ \phi_K : C_K \to \Gal{K^{\text{ab}} / K} \]
\end{remark}

\begin{definition}
Let $L / K$ be a finite Galois extension and $v$ be a place of $K$ and $w \divides v$ a place of $L$. The decomposition group,
\[ D(w) = \{ \sigma \in \Gal{L/K} \mid \sigma(w) = w \} \cong \Gal{L_w / K_v} \]
\end{definition}

\begin{remark}
For a different $w' \divides v$ with $w' = \tau(w)$ we have $D(w') = \tau D(w) \tau^{-1}$. In particular, if $L / K$ is abelian then $D(w) = D(w')$ so the decomposition group of a place $v$ is well-defined. 
\end{remark}

\begin{remark}
Therefore, for abelian $L / K$ at each place $v$ of $K$ the local Artin map gives a canonical map,
\[ \phi_v : K_v^\times \to \Gal{L_w/K_v} = D(v) \subset \Gal{L/K} \]
which is independent of the choice of $w \divides v$. 
\end{remark}

\newcommand{\ab}{\mathrm{ab}}

\begin{proposition}
There exists a continuous homomorphism,
\[ \phi_K : \idele{K} \to \Gal{K^{\ab} / K} \]
such that for any finite abelian extension $L / K$ and any place $v$ of $K$ the following diagram commutes,
\begin{center}
\begin{tikzcd}
K_v^\times \arrow[r, "\phi_v"] \arrow[d, hook] & \Gal{L_w / K_v} \arrow[d, hook] 
\\
\idele{K} \arrow[r, "\phi_{L/K}"] & \Gal{L/K}
\end{tikzcd}
\end{center}
\end{proposition}

\begin{proof}
Let $a = (a_v)_v \in \idele{K}$ if $a_v \in \ints{v}^\times$ and $L_w / K_v$ is unramified then $\phi_v(a_v) = 1$. Therfore, $\phi_{L/K}(a)$ is uniquely determined locally by,
\[ \phi_{L/K}(a) = \prod_v \phi_v(a_v) \]
since the product is finite and thus well-defined since all but finitely many places are unramifield. Furthermore, varying $L$ we recover the map $\phi_K : \idele{K} \to \Gal{K^\ab / K}$. It remains to show that $\phi_K$ is continuous i.e. that $\ker{\phi_{L/K}}$ is an open subgroup of $\idele{K}$. Take $S$ to be the ramified places of $L/K$. By the compatibility of local Artin maps,
\begin{center}
\begin{tikzcd}
\sidele{L}{S} \arrow[d, "\Nm_{L/K}"] \arrow[r, "\phi_{L/L}"] & \Gal{L/L} \arrow[d, hook]
\\
\sidele{K}{S} \arrow[r, "\phi_{L/K}"] & \Gal{L/K}
\end{tikzcd}
\end{center}
Therfore, $\ker{\phi_{L/K}} \supset \Nm_{L/K}{\sidele{L}{S}}$ which contains an open subgroup of $\sidele{K}{S}$ and thus $\ker{\phi_{L/K}}$ is an open subgroup. 
\end{proof}

\begin{theorem}[Global CFT]
Let $K$ be a number field. Then,
\begin{enumerate}
\item The homomorphism $\phi_K : \idele{K} \to \Gal{K^\ab / K}$ satisfies,
\begin{enumerate}
\item $\phi_K(K^\times) = 1$ thus it descends to $\phi_K : C_K \to \Gal{K^\ab / K}$

\item for any finite abelian $L / K$ the Artin map induces an isomorphism,
\[ \phi_{L/K} : C_K / \Nm(C_L) \xrightarrow{\sim} \Gal{L/K} \]
\end{enumerate}
\item For any finite index open subgroup $N \subset C_K$ there exists a finite abelian extension $L / K$ such that $\Nm(C_L) = N$. 
\end{enumerate}
\end{theorem}

\begin{corollary}
The map $L \mapsto \Nm_{L/K}(C_L)$ gives a bijection between finite abelian extensions of $K$ and finite index open subgroups of $C_K$. Furthermore,
\begin{enumerate}
\item $L_1 \subset L_2 \iff \Nm(C_{L_1}) \supset \Nm(C_{L_2})$
\item $\Nm(C_{L_1 \cdot L_2}) = \Nm{C_{L_1)}} \cap \Nm{C_{L_2}}$
\item $\Nm(C_{L_1 \cap L_2}) = \Nm(C_{L_1}) \cdot \Nm(C_{L_2})$
\end{enumerate} 
\end{corollary}

\subsection{Ray Class Fields}

\newcommand{\p}{\mathrm{p}}
\newcommand{\Cl}{C\ell}

\begin{definition}
A \textit{modulus} of $K$ is a function $m : \{ \text{places of } K \} \to \Z_{\ge 0}$ such that,
\begin{enumerate}
\item $m(v) = 0$ for all but finitely many $v$
\item $m(v) \in \{0, 1\}$ if $v$ is a real place
\item $m(v) = 0$ if $v$ is a complex place
\end{enumerate}
\end{definition}

\begin{definition}
Associated to a modulus $m$ we have the \textit{principal congruence subgroup},
\[ \idele{K}^m = \prod_{v \ndivides \infty} U_{v,m(v)} \times \prod_{v \divides \infty} K_{v, m(v)}^\times \]
where, $U_{v,0} = \ints{v}^\times$ and $U_{v,i} = 1 + \p_v^i$ and $K_{v,0}^\times = K_v^\times$ and $K_{v,1}^\times = \R_{\ge 0}$. 
\end{definition}

\begin{definition}
Define $C_K^m = (\idele{K}^m \cdot K^\times) / K^\times \subset C_K$ which is an open subgroup of $C_K$ of finite index. And define the \textit{ray class group},
\[ \Cl_m = C_K / C_K^m \]
which is a finite abelian group. 
\end{definition}

\begin{remark}
Write formally,
\[ \m = \prod_{v} \p_v^{m(v)} = \m_0 \cdot \m_{\infty} \]
where $m_0 \subset \ints{K}$ can be viewed as an ideal. Then,
\[ \Cl_m \cong \frac{\{ \text{fractional ideals coprime to } m_0 \}}{\{ x \in K^\times \mid \forall v \divides \m_0 : x \in U_{v,m(v)} \text{ and } \forall v \divides \m_{\infty} : x \in \R_{v, > 0} \}} \]
\end{remark}

\begin{example}
Let $K = \Q$ and $\m = (m)$ for $m \in \Z$. Then,
\[ \Cl_{\m} = \frac{\{ \left( \frac{r}{s} \right) \mid (r,m) = (s,m) = 1 \}}{\{ \frac{r}{s} \in \Q^\times \mid r \equiv_m s \}} \cong (\Z / m \Z)^\times / \{ \pm  1 \} \]
Furthermore, for $\m = (m) \cdot \infty$ we find,
\[ \Cl_\m  \cong (\Z / m \Z)^\times \]
\end{example}

\begin{definition}
The abelian extension $L_m / K$ corresponding to $\Cl_m$ via global class field theory with $\Nm(L_\m) = C^m_K$ and thus $\Gal{L_m / K} \cong \Cl_m$ is called the \textit{ray class field} for $m$. When $m = 1$ is trivial then $\Cl_m = \Cl_K$ and the corresponding ray class field is called the Hilbert class field $H$.
\end{definition}

\begin{remark}
By local CFT, the ramified places of the ray class field are contained in $m$. Therefore, $H$ is the maximal unramified abelian extension. An infinite place $v \divides \infty$ is unramified if $L_w / K_v = \R / \R$ or $\C / \C$.   
\end{remark}

\begin{definition}
When the modulus,
\[ \m = \prod_{v \divides \infty} \p_v \]
over only real places the corresponding ray class field is the \textit{narrow} Hilbert class field $H^+$ which is the maximal abelian extension of $K$ unramified at finite places. 
\end{definition}

\begin{example}
For $K = \Q$ we have $H = H^+ = \Q$. However, for $K = \Q(\sqrt{3})$ then $H = K$ since $\Cl_K = 1$ but $H^+ = K(i)$ since $(\Cl_K^+ = Z / 2 \Z)$ as $2 + \sqrt{3} > 0$ and $2 - \sqrt{3} > 0$. 
\end{example}

\begin{example}
For $K = \Q$ and $\m = (m) \cdot \infty$ then $\Cl_\m = (\Z / m \Z)^\times$ and thus $L_\m = \Q(\zeta_m)$. For $\m = (m)$ then $\Cl_m = (\Z / m \Z)^{\times} / \{ \pm 1 \}$ and thus $L_\m = \Q(\zeta_m)^+ = \Q(\zeta_m + \overline{\zeta_m})$. 
\end{example}

\begin{corollary}[Kronecker-Weber]
\[ \Q^\ab = \bigcup_{m \ge 1} \Q(\zeta_m) \]
\end{corollary}

\begin{proof}
For $K = \Q$ every modulus takes the form,
\[ \m = \prod_{p \in \Q} (p)^{m(p)} \cdot \infty \text{ or } \m = \prod_{p \in \Q} (p)^{m(p)} \]
However this is simply $\m = (m) \cdot \infty$ or $\m = (m)$ where, $m \in \Z$ is,
\[ m = \prod_{p \in \Q} p^{m(p)} \]
We have seen that both possible moduli give abelian extensions contained in some cyclotomic field. 

\end{proof}

\section{April 3}

\subsection{Cohomology of Units}

Let $L/K$ be a Galois extension of number fields and $S \supset S_\infty$ a finite set containing all infinite places of $K$ and $T = \{  w : w \divides v : v \in S \}$ a finite set of places of $L$.

\begin{definition}
The group of $T$-units of $L$ is defined to be,
\[ U(T) = \{ \alpha \in L^\times \mid \forall \omega \notin T : \alpha \in U_\omega  \} = L^\times \cap \sidele{L}{T} \]
\end{definition}

\begin{proposition}
Assume $L / K$ is cyclic then the Herbrand quotient is given by,
\[ h(U(T)) = \frac{\prod\limits_{v \in S} [L_v : K_v]}{[L : K]} \] 
\end{proposition}

\newcommand{\Ind}{\mathrm{Ind}}

\begin{proof}
Consider $V = \R^T = \Hom{T}{\R}$ and $N = \Hom{T}{\Z} \subset V$ a $G$-stable lattice. Then,
\[ N \cong \bigoplus_{v \in S} \Hom{G/D(v)}{\Z} \cong \bigoplus_{v \in S} \Ind_{D(v)}^G \Z \]
Therefore, by Shapiro,
\[ h(N) = \prod_{v \in S} h(\Ind_{D(v)}^G \Z) = \prod_{v \in S} h_{D(v)}(\Z) = \prod_{v \in S} = \prod_{v \in S} |D(v)| = \prod_{v \in S} [L_v : K_v] \]
Consider $\lambda : U(T) \to V$ given by, $\alpha \mapsto (\log{|\alpha|_w} )_{\omega \in T}$. Let $M^0 = \Im{\lambda} \subset V$. Dirichlet's unit theorem for $T$-units gives $M^0$ a lattice in the hyperplane,
\[ \{ \sum_{\omega \in T} X_\omega = 0 \} \subset V \]
Consider $M = M^0 \oplus \Z (1,\cdots, 1) \subset V$ a $G$-stable lattice in $V$. Therefore, because $\ker{\lambda}$ is finite its Herbrand quotient vanishes and thus $h(U(T)) = h(M^0)$ because there is an exact sequence,
\begin{center}
\begin{tikzcd}
0 \arrow[r] & \ker{\lambda} \arrow[r] & U(T) \arrow[r] & M^0 \arrow[r] & 0
\end{tikzcd}
\end{center}
Furthermore,
\[ h(M) = h(\Z) h(M^0) = |G| h(U(T) \]
Therefore, the following lemma gives the desired conclusion,
\[ h(U(T)) = \frac{h(M^0)}{[L : K]} = \frac{h(N)}{[L : K]} = \frac{\prod\limits_{v \in S} [L_v : K_v]}{[L : K]} \]
\end{proof}

\begin{lemma}
Let $G$ be a finite cyclic group and $V$ a $\R[G]$-module i.e. a $G$-module and $\R$-vectorspace. Let $M$ and $N$ be two $G$-stable lattices in $V$ then $h(M) = h(N)$. 
\end{lemma}

\subsection{Cohomology of Idele Class Group}

\begin{lemma}
$H^0(G, C_L) = C_K$
\end{lemma}

\begin{proof}
Consider the short exact sequence,
\begin{center}
\begin{tikzcd}
1 \arrow[r] & L^\times \arrow[r] & \idele{L} \arrow[r] & C_L \arrow[r] & 1 
\end{tikzcd}
\end{center}
Then the long exact sequence of cohomology gives,
\begin{center}
\begin{tikzcd}
1 \arrow[r] & H^0(G, L^\times) \arrow[r] & H^0(G, \idele{L}) \arrow[r] & H^0(G, C_L) \arrow[r] & H^1(G, L^\times) 
\end{tikzcd}
\end{center}
However $H^1(G, L^\times) = 0$ and $H^0(G, L^\times) = 0$ and $H^0(G, \idele{L}) = \idele{K}$. Therefore we have a short exact sequence,
\begin{center}
\begin{tikzcd}
1 \arrow[r] & K^\times \arrow[r] & \idele{K} \arrow[r] & H^0(G, C_L) \arrow[r] & 0
\end{tikzcd}
\end{center}
showing that,
\[ H^0(G, C_L) = \idele{K}/K^\times = C_K \]
\end{proof}

\begin{lemma}
Let $S \supset S_\infty$ contain a generating set of primes for $\Cl_K$ then,
\[ \idele{K} = K^\times \cdot \sidele{K}{S} \]
\end{lemma}

\begin{proof}
There is a surjection $\idele{K} \to I_K$ given by sending,
\[ (a_v)_v \mapsto \prod_v \p_v^{v_v(a_v)} \]
which has kernel $\sidele{K}{S_\infty}$. 
Therefore,
\[ \Cl_K = \frac{I_K}{K^\times} = \frac{\idele{K}}{K^\times \cdot \sidele{K}{S_\infty}} \]
Since $\Cl_K$ is finite, by a choice of $S$ I can set,
\[ \frac{\idele{K}}{K^\times \cdot \sidele{K}{S}} = 0 \]
by quotienting $\{ \p_v \}_{v \in S}$. 
\end{proof}

\begin{theorem}
Let $L / K$ be a finite cyclic extension then $h(C_L) = [L : K]$. 
\end{theorem}

\begin{proof}
Let $S \supset S_\infty$ be a finite set of places of $K$ such that $T$ contains a generating set of primes of $\Cl_L$ and all ramified places. Then,
\[ C_L = \frac{\idele{L}}{L^\times} = \frac{L^\times \sidele{L}{T}}{L^\times} = \frac{\sidele{L}{T}}{L^\times \cap \idele{L}{T}} = \frac{\sidele{L}{T}}{U(T)} \]
Therefore,
\[ h(C_L) = \frac{h(\sidele{L}{T})}{h(U(T))} = \frac{\prod\limits_{v \in S} n_v}{\left( \frac{\prod\limits_{v \in S} [L_v : K_v]}{[L : K]} \right)} = [L : K] \]
\end{proof}

\begin{corollary}
$[ C_K : \Nm(C_L) ] \ge [L : K]$. 
\end{corollary}

\begin{lemma}
Assume $L / K$ is solvable. If $\exists D \supset \idele{K}$ subgroup such that $D \subset \Nm(I_L)$ and $K^\times D$ is dense in $\idele{K}$ then $L = K$.
\end{lemma}

\begin{proof}
If not then since $L / K$ is solvable there exists a nontrivial cyclic subextension $K' / K$. Then $D \subset \Nm(\idele{L}) \subset \Nm(\idele{K'})$. By local class field theory, $\Nm(\idele{K'})$ is an open subgroup of $\idele{K}$ which implies that $K^\times \cdot \Nm(\idele{K'})$ is an open subgroup and hence closed. However, $K^\times D$ is dense if $\idele{K}$ and thus $K^\times \Nm(\idele{K'}) = \idele{K}$. Therefore, $[C_K : \Nm(C_{K'})] = 1$ and thus by the first innequality $[K' : K] = 1$ contradicting the fact that $K' / K$ is nontrivial. 
\end{proof}

\begin{proposition}
Let $L / K$ be a nontrivial solvable extension. Then there exist infinitely many places $v$ of $K$ such that $v$ does not split completely in $L$. 
\end{proposition}

\begin{proof}
If not, let $S \supset S_\infty$ contain all such $v$ which would then be finite. Consider the subgroup $D = \{ (a_v) \in \idele{K} \mid \forall v \in S : a_v = 1 \}$. Then $D \subset \Nm(\idele{L})$ since if $v \notin S$ then $L_v = K_v$ because $v$ splits completely. However, any $(a_v)_{v \in S}$con be approximated by a global element $a \in K^\times$ (as $S$ is finite). Therefore $K^\times \cdot D$ is dense in $\idele{K}$. By the previous lemma then extension $L / K$ must be trivial.   
\end{proof}

\begin{proposition}
Assume $L / K$ is solvable then $\{ \Frob_{w/v} \mid w/v \text{ is unramified} \}$ generate $\Gal{L/K}$.  
\end{proposition}

\begin{proof}
Let $H$ be the group generated by the Frobenius elements and $E = L^H$. Then if $v$ is unramified so $v$ splits completely in $E / K$ because the Frobenius acts trivially. Thus $E = K$ by the lemma so $H = G$. 
\end{proof}

\begin{corollary}
Let $L / K$ be an abelian extension. Then the Artin map,
\[ \phi_K : \idele{K} \to \Gal{L/K} \]
is surjective.
\end{corollary}

\begin{proof}
Since $\phi_K(\varpi_v) = \Frob_{w/v}$ the image of $\phi_K$ contains all Frobenius elements and thus generates $\Gal{L/K}$. 
\end{proof}

\section{Second Innequality}

\begin{theorem}
Let $L / K$ be a finite abelian extension of number fields then,
\[ [C_K : \Nm(C_L)] \le [ L : K ] \]
\end{theorem}

\begin{proof}
Equivalently, we need to show that,
\[ \frac{1}{[L : K]} \le \frac{1}{[C_K : \Nm(C_L)]} \]
which we may interprete as relating the density of a set of primes of $K$. since $1/[L : K]$ is the density of completely split primes of $L/K$. Let $\m$ be a modulus of $K$ and let $I_K^\m$ be the group of fractional ideals of $K$ coprime to $\m$. Let 
\[ K^\m = \{ a \in K^\times \mid \forall v \divides m_0 a \in U_{v, m(v)} \text{ and } \forall v \divides \m_{\infty} : a \in K_{v, m(v)}^\times = \R_{> 0} \} \]
Then,
\[ \Cl_\m \cong \frac{I_K^\m}{K^\m} \]
\end{proof}

\begin{theorem}
Let $K^\m \subset H \subset I_K^\m$ let $A \in I_K^\m / H$ which is a quotient of $\Cl_\m$. Then,
\[ \delta(\p \in A) = \frac{1}{[I_K^\m : K]} \]
\begin{proof}
Let $\chi$ be any character of $I^\m / H$ then construct the Weber L-function $L(s, \chi)$. Consider,
\[ \log{L(s, \chi)} \sim \sum_{\p \ndivides \m} \frac{\chi(\p)}{N(\p^s)} = \sum_{B \in I^\m/H} \chi(B) \sum_{\p \in B} \frac{1}{N \p^s} \]
Recall that if $G$ is a finite abelian group $g \in G$ then,
\[ \sum_{\chi \in \Hom{G}{\C^\times}} \chi(g) = \begin{cases}
|G| & g = 1 
\\
0 & g \neq 1
\end{cases} \]
Therefore, consider,
\begin{align*}
\sum_{\chi} \chi(A^{-1}) \log{L(s, \chi)} \sim \sum_{\chi} \sum_{B \in I^\m_K/H} \chi(A^{-1} B) \sum_{\p \in B} \frac{1}{N \p^s} = [I_K^\m : H] \sum_{\p \in A} \frac{1}{N \p^s} 
\end{align*}
\end{proof}
Since the sum over all characters picks out the element $A^{-1} B = 1$. Furthermore, for $\chi \neq \chi_0$ we know that $L(s, \chi)$ is finite for $s \to 1$ but,
\[ \log{L(s, \chi_0)} \sim \log{\zeta_K(s)} \sim \log{\frac{1}{s-1}} \]
which implies that,
\[ \log{\frac{1}{s-1}} \sim [I_K^\m : H] \sum_{\p \in A} \frac{1}{N \p^s}  \]
Therefore,
\begin{align*}
\delta(\p \in A) = \lim_{s \to 1^+} \frac{\sum\limits_{\p \in A} \frac{1}{N \p^s}}{\log{\frac{1}{s-1}}} = \frac{1}{[I_K^\m : H]} 
\end{align*}
\end{theorem}


\begin{theorem}
Let $L / K$ be a finite abelian extension of number fields then,
\[ [C_K : \Nm(C_L)] \le [ L : K ] \]
\end{theorem}

\begin{proof}
Apply the previous theorem to $H = K^\m \cdot \Nm(I^\m_L)$ and $A = 0 \in I_K^\m  / K)$. Then,
\[ \delta(\p \in A) = \frac{1}{[I_K^\m : H]} \]
However, if $\p$ is a prime of $K$ that splits completely in $L$ then $\p \in \Nm(I^\m_L)$ since the Galois group permutes the factors of $\p$ exactly once. This holds if $\p$ is coprime to $\m$ which only removes a finite number of primes from the set of all completely split primes. Therefore, the density of completely split primes is less than the density of primes $\p \in \Nm(I^\m_L)$ i.e. $\p \in A$. Therfore,
\[ \frac{1}{[L : K]} \le \frac{1}{[I^\m_K : H]} \] 
\end{proof}

\begin{corollary}
If $L / K$ is finite Galois then,
\[ H^1(\Gal{L/K}, C_L) = 0 \]
\end{corollary}

\begin{proof}
When $G = \Gal{L/K}$ is cyclic then the first innequality gives,
\[ h(C_L) = \frac{[C_K : \Nm(C_L)]}{|H^1(G, C_L)} = [L : K] \]
However,
\[ [C_K : \Nm(C_L)] \le [L : K] \] 
and thus $[C_K : \Nm(C_L)] = [L : K]$ and therefore $H^1(G, C_L) = 0$. 
If $G$ is not cyclic, then consider $G$ solvable. Take normal $H \triangleleft G$ such that $G / H$ is cyclic. By inflation-restriction, there is an exact sequence,
\begin{center}
\begin{tikzcd}
0 \arrow[r] & H^1(G/H, C_L^H) \arrow[r] & H^1(G, C_L) \arrow[r] & H^1(H, C_L)  
\end{tikzcd}
\end{center} 
By induction we have $H^1(H, C_L) = 0$ and $H^1(G/H, C_L^H) = 0$ since $G/H$ is cyclic. Therefore, $H^1(G, C_L) = 0$. Finally, for a generaly group $G$, use the embedding,
\begin{center}
\begin{tikzcd}
H^1(G, C_L) \arrow[r, hook] & \prod_{p \divides |G|} H^1(G_p, C_L) 
\end{tikzcd}
\end{center}
where $G_p$ is a $p$-Sylow subgroup of $G$ which is solvable so $H^1(G_p, C_L) = 0$. Therefore, $H^1(G, C_L) = 0$.
\end{proof}

\begin{theorem}[Chebotarev Density]
Let $L / K$ be a finite Galois extension of number fields. Let $\sigma \in G = \Gal{L/K}$ and $C_\sigma \subset G$ its conjugacy class. Then,
\[ \delta(\{ \p \subset \ints{K} \mid \Frob_{\p} \in C_\sigma \}) = \frac{|C_\sigma|}{|G|} \]
\end{theorem}

\begin{example}
For $\sigma = 1$ then $C_\sigma = \{ 1 \}$. Furthermore, $\Frob_\p = 1$ exactly when $\p$ splits completely. Then the Chebotarev Density theorem gives,
\[ \delta(\{ \p \subset \ints{K} \mid \Frob_\p = 1 \}) = \frac{1}{|G|} \]
which implies that,
\[ \delta(\{ \p \text{ splits completely} \}) = \frac{1}{[L : K]} \]
\end{example}

\begin{example}
Take $K = \Q$ and $L / K$ abelian. In particular, take $L = \Q(\zeta_N)$ then $\Gal{L/K} = (\Z / N \Z)^\times$ and consider $\sigma = a \in (\Z / N \Z)^\times$. Then the Chebotarev Density theorem states that,
\[ \delta(\{ p  \mid \Frob_p = a \}) = \frac{1}{|G|} \]
However, if $\Frob_p = a$ then the actions on $\zeta_N$ are equal meaning that $\zeta_N^p = \zeta_N^a$ and thus $p \equiv_N a$. Therefore,
\[ \delta(\{ p \equiv_N a \}) = \frac{1}{\phi(N)} \]
\end{example}

\newcommand{\F}{\mathbb{F}}

\begin{example}
Take $K = \Q$ and $L = \Q(\zeta_3, \sqrt[3]{2})$ a nonabelian Galois number field which is the splitting field of $x^3 - 2$. Then $G = S_3$ . There are three conjugacy classes, $C_1 = \{1\}$, $C_2 = \{ (1 \: 2), (2 \: 3), (3 \: 4) \}$, and $C_3 = \{ (1 \: 2 \: 3), (1 \: 3 \: 2) \}$. $\Frob_p \in C_1$ iff $p$ splits completely iff $x^3 - 2$ splits completely in $\F_p$. $\Frob_p \in C_2$ iff $x^3 - 2$ has one linear factor in $\F_p$. Finally, $\Frob_p \in C_3$ iff $x^3 - 2$ is irreducible in $\F_p$. Then the Chebotarev Density theorem tells us that these three conditions occur with frequency $\frac{1}{6}, \frac{1}{2}, \frac{1}{3}$ respectivly. 
\end{example}

\begin{remark}
There is \underline{no} simple congruence condition on $p$ to determine what conjugacy class $p$ lies in.
\end{remark}

\end{document}

