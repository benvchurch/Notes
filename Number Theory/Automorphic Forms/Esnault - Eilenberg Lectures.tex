\documentclass[12pt]{article}
\usepackage{import}
\import{../}{NumberTheoryCommands}

\begin{document}

\section{Lecture 1 (Sept. 20)}

\newcommand{\Msq}{M^{\square}}
\newcommand{\LL}{\mathbb{L}}
\newcommand{\alg}{\mathrm{alg}}
\newcommand{\X}{\mathfrak{X}}
\newcommand{\tame}{\mathrm{tame}}
\newcommand{\irred}{\mathrm{irred}}
\newcommand{\cH}{\mathscr{H}}
\renewcommand{\gr}{\mathbf{gr}}
\newcommand{\Fil}{\mathrm{Fil}}

In these lectures we will strudy the arithmetic properties of complex local systems (not $\ell$-adic local systems that arise in arithmetic geometry).
\bigskip\\
We begin with Poincare. Consider the topological $\pi_1(X, x)$ where $x$ is a base point of a topological space $X$. For fintie CW complex, $\pi_1(X, x)$ is finitely presented (finitely many generators and relations). Furthermore, every finitely presented group is $\pi_1(X, x)$ for some finte CW complex which is constructed via generators and relations (in fact, this can be down with a compact 3-manifold). 
\bigskip\\
Among these spaces we can consider $X$ a smooth quasi-projective variety over $\CC$ then $X(\CC)$ is a compact topological manifold and hence a finite CW complex. Thus $\pi_1(X(\CC), x)$ is a finitely-presented group. However, not every finitely-presented group is a fundamental group of some smooth quasi-projective variety. 
\bigskip\\
To study $\pi_1(X(\CC), x)$ we study its complex linear representations,
\[ \rho : \pi_1(X(\CC), x) \to \GL_n(\CC) \]
There is a space,
\[ \Msq(X, r)(\CC) = \{ M_1, \dots, M_g \in \GL_r(\CC) \mid \text{relations} \} \]
which is the complex points of a modui space parametrizing (framed) local systems. To remove the dependence on the basepoint, we write $M(X, r)$ for the $\GL_r$-quotient of $\Msq(X, r)$. This is called the character variety of $X$. 
\bigskip\\
Affine algebraic variety defined over $\ZZ$. This parametrizes semi-simple $\rho$.

\begin{theorem}[Toledo]
There exist projective varieties $X$ such that $\pi_1(X(\CC), x)$ is \textit{not} residually finite (meaning it does not inject into its profinite completion). 
\end{theorem}

\begin{rmk}
Therefore, we cannot just study the finite quotients we actually need to study the complex representations. 
\end{rmk}

\begin{defn}
We say that a local system $\LL$ is \textit{geometric} if there is a Zariski open $U \subset X$ and $g : Y \to U$ smooth and projective such that $\LL|_U$ is a subquotient of $R^i g_* \underline{\C}$.
\end{defn}

\begin{rmk}
Because $\pi_1(U, x) \to \pi_1(X, x)$ is surjective we don't lose any information in the restriction $\LL|_U$ but we cannot expect the geometric families to always extend. 
\end{rmk}

\begin{rmk}
If $\phi  : Y \to X$ is finite and unramified then the subquotients of $g_* \underline{\C}$ are called the finite local systems (they are trivialized by $g$). 
\end{rmk}

These are all the examples that anyone knows how to write down. The geometric local systems have monodromy defined over $\Z$ and thus define maps $\rho : \pi_1(X(\CC), x) \to \GL_r(\overline{\Z})$ because when we take a subquotient we have to extend to algebraic integers. However, since $M(X, r)$ is positive dimensional it has lots of transcendental points so where do these arise? Does every local system arising integrally come from geometry? 

\subsection{Riemann-Hilbert Correspondence}

There is a correspondence between local systems $\LL$ and differential systems $(E, \nabla)$ where $E$ is an algebraic vector bundle and $\nabla$ is an integrable connection (with, in the noncompact case, regular singularities at the boundary). This correspondence is given by,
\[ L \mapsto (L \ot_{\CC} \struct{X}, \id \ot \d{}) \]
which in the proper case is automatically algebraic by GAGA and in the nonproper case we need a result to Deligne to see that it is algebraizable. The other direction is given by taking,
\[ (E, \nabla) \mapsto E^{\nabla} \]

\begin{prop}
There is a moduli space of integrable connections such that $M_{\dR}(X, r)(\CC) = \{ (E, \nabla) \} $. Then by the Riemann-Hilbert correspondence, \[ M_B \cong M_{\dR} \]
\end{prop}

\begin{theorem}[Simpson, '90s] Higgs bundles $(V, \theta)$ where $\theta : V \to \Omega^1 \ot V$ is linear and $\theta \wedge \theta = 0$. There is an isomorphism between the $M_{\dR}$ and the moduli space of semi-stable Higgs bundles.
\end{theorem}

\begin{example}
For $r = 1$ we have $M_B(\CC) = H^1(X(\CC), \CC^\times) = \Hom{}{\pi_1(X)}{\CC^\times}$. Also
\[ M_{\dR}(\CC) = H^1(X_{\CC}, \struct{X}^\times \to \Omega^1 \to \Omega^2) \] 
\end{example}

\subsection{Grothendieck $p$-curvature Conjecture}

For motivation, let $X = \P^1 \sm \{ 0,1, \infty \}$ (perhaps the most important variety) and $(E, \nabla)$ be a vector bundle with flat connection. Then $(X, E, \nabla)$ is defined over $S = \Spec{R}$ where $R$ has finite type over $\Z$ (by spreading out if you like). Then we can write the following differential equation,
\[ \frac{\d{f}}{f} = b \frac{\d{t}}{t} + c \frac{\d{t}}{t - 1} \]
We are looking for conditions that ensure the existence of a complete set of solutions algebraic over $\C[t, t^{-1}, (t-1)^{-1}]$. In fact, we can take $R = \struct{K}[1/S]$ the ring of integers of some number field with some finite set of primes inverted. The conjecture is that this is equivalent requiring the the differential equaion modulo $\p$,
\[ \frac{\d{f}}{f} = \bar{b} \frac{\d{t}}{t} + \bar{c} \frac{\d{t}}{t-1} \]
has a full set of solutions over $(\struct{K} / \p)[t, t^{-1}, (t-1)^{-1}]$ for almost all primes $\p$.
\bigskip\\
A simpler analog of Grothendieck's conjecture goes back to Kronecker's criterion for checking that an algebraic number is rational (or an algebraic integer is a root of unity). Here is the easy version:
\begin{center}
Let $b \in \overline{\Q}$ then $b \in \Q$ iff $\bar{b} \in \FF_p \subset \overline{\ZZ} / (p)$ for all primes not appearing the denominator of $b$.
\end{center}

\subsection{Further Questions}

Under what conditions are there no deformations of a local system? In particular when is $M_B = *$? 
\bigskip\\
By the Riemman existence theorem, 
\[ \pi_1(X(\CC), x)^{\wedge} = \pi_1^{\et}(X_{\CC}, x) \]
Furthermore, we claim if the \etale fundamental group is zero (meaning $X(\CC)$ has no nontrivial finite covers i.e. $\pi(X(\CC), x)$ has no finite quotients) then actually there are no nontrivial local systems or equivalenty the algebraic completion of $\pi_1(X(\CC), x)$ also vanishes. Indeed, if $\rho : \pi_1 \to \GL_r(\CC)$ Then there is some finite type $\Z$-algebra $A \subset \CC$ thusch that $\rho : \pi_1 \to \GL_r(A) \subset \GL_r(\CC)$ since $\pi_1$ is finitely generated. Then for each maximal ideal of $A$ we get a finite quotient of $\pi_1$ which is trivial and hence $\rho$ is trivial. 
\bigskip\\
Similary, if $X$ is smooth projective over $k = \bar{k}$ in characteristic $p > 0$ then $\pi_1(X, x) = 1$ implies no nontrivial crystals. Johan conjectured: there are no nontrivial isocrystals. 

\begin{rmk}
If we assume $X / \FF_q$ and the isocrystal is over $\FF_q$ then ????
\end{rmk}  

Simpson: $M_B(X, r)$ is an affine variety so it has some dimension but it may have zero dimensional components. These correspond to rigid local systems. Conjecture: the isolated points of $M_B(X, r)$ are geometric and in particular integral. 


In the case the isolated point it reduced, then we can prove the conjecture that rigid implies integral. However, it is not always reduced. The proof usses the Langlands program via theory ``companions'' of Deligne (Weil II). 


\section{Lecture 2 (Sept. 27) The $p$-curvature conjecture}

\subsection{Kronecker}

Last time we mentioned a classical result due to Kronecker which is an analog of the Grothendieck $p$-curvature conjecture. The question is as follows: for $a \in \C$ when is $a \in \mu_{\infty}$ meaning $a = \exp(2 \pi i b)$ for some $b \in \Q$?

\begin{prop}[Kronecker]
If $a \in \overline{\Z}$ and for all embeddings $\iota(a) \in \CC$ we have $|\iota(a)| = 1$ then $a \in \mu_{\infty}$.
\end{prop}

\begin{proof}
Consider the minimal polynomial,
\[ f(X) = X^d + c_1 X^{d-1} + \cdots + c_d \]
Then,
\[ f = \prod (x - a_i) \in \Z[x] \]
and therefore using the theory of symmetric functions,
\[ f_n = \prod (x - a_i^n) \in \Z[x] \]
but the coefficients have bounded norm because all the absolute values of the $a_i$ are $1$. Therefore there are finitely many such $f_n$ and hence there are repetitions in the $a_i^n$ so $a_i^n = a_j$ for some $i$ and $j$ and thus $a_i$ is a root of unity. 
\end{proof}

\begin{rmk}
This is an analytic version of Kronecker's problem. Here is an algebraic version for the question $b \in \Q$. 
\end{rmk}

\begin{theorem}
Assume $b \in \overline{\Q}$ so $\Q \subset \Q(b) \subset \overline{\Q}$. Then $b \in \struct{\Q(b)}[S^{-1}]$ so for any prime of this ring consider $\bar{b} \in (\struct{\Q(b)} / \p)$ which is a finite extension of $\FF_p$. Assume that for all but finitely many $\p$ that $\bar{b} \in \FF_p$ in the prime subfield. Then $b \in \Q$.
\end{theorem}

\begin{proof}
Note that $(\struct{\Q(b)} / \p) = \FF_p[b]$ and hence if $b \in \FF_p$ then $p$ is totally split. However, this can only happen for almost all $p$ if $b \in \Q$.
\end{proof}

\subsection{Grothendieck}

Let $X = \A^1 \sm \{ 0 \}$ with parameter $t$. Consider the differential equation,
\[ \frac{\d{f}}{f} = b \frac{\d{t}}{t} \]
which has solutions $f = a t^b$. We want conditions for those solutions to be algebraic over $\struct{X}(X) = \CC[t, t^{-1}]$. In this case, this is equivalent to $b \in \Q$. This is also equivalent to the monodromy group being finite. 
\bigskip\\
However, $b \in \Q$ is equivalent to saying that $\bar{b} \in \FF_p$ for all but finitely many $\p$. Therefore, we should consider the differential equation mod $\p$. Therefore, the differential equation has a full set of algebraic solutions if and only if the mod $\p$ equations has a full set of solutions. 

\begin{conj}[Grothendieck]
Let $X$ be quasi-projective variety over $\CC$, and $(\E, \nabla)$ is a vector bundle with an integrable algebraic connection. Spread out over some finite type affine $\Z$-scheme $S$ to get $(X_S, (\E, \nabla)_S)$. Then there is an open dense $S^0 \subset S$ such that for all closed points $s \in |S^0|$ then $(\E, \nabla)_{s}$ has locally a full set of solutions if and only if $(\E, \nabla)$ has a full set of algebraic solutions locally on $X$. 
\end{conj}

\begin{rmk}
The conclusion of the conjecture is equivalent to having finite monodromy. 
\end{rmk}

\begin{rmk}
It suffices to prove the conjecture for $X$ is a proper cuve because,
\begin{enumerate}
\item by Lefschetz, there is $C \to X$ with $\pi_1(C) \onto \pi_1(X)$ so we can check the conclusion after restricting $(\E, \nabla)$ to $C$.
\item By restricting to disks around the punctures we reduce to the Kronecker example. Therefore, given the assumption, there exists a finite \etale cover $Y \to X$ which trivializes the differential equation at the punctures so that we can extend the problem to $\overline{Y}$.
\end{enumerate}
\end{rmk}

\begin{prop}
Let $X$ be a smooth projective curve over $\CC$. We can assume that $X$ is defined over a number field. 
\end{prop}

\begin{prop}[Jordan]
There is a number $N(r)$ such that for any finite $\Gamma \subset \GL_r(\CC)$ there is an abelian normal subgroup $A \subset \Gamma$ such that $\Gamma / A$ has order at most $N(r)$.
\end{prop}

\begin{theorem}[Belyi]
Any smooth projective curve $X$ over a number field $F$, there is a map $f : X \to \P^1$ branched over $\{ 0, 1, \infty \}$. 
\end{theorem}

\begin{cor}
Therefore, we reduce the $p$-curvature conjecture to the case of $\P^1 \sm \{ 0, 1, \infty \}$ meaning the ring $\CC[t, t^{-1}, (t-1)^{-1}]$. We have the universal form,
\[ \frac{\d{f}}{f} = b \frac{\d{t}}{t} + c \frac{\d{t}}{t} \]
for $b, c \in M_r(\overline{\Q})$ . Ww want $f$ a matrix of algebraic functions over the ring. 
\end{cor}

\begin{theorem}[Katz]
Let $g : Y \to X$ be smooth projective (here we require the family is defined over all of $X$ not just a dense open) and $\L = R^i g_* \underline{\CC}$ gives the Gauss-Manin local system corresponding to $(\E, \nabla)$ the Gauss-Manin connection on the relative de Rham cohomology. Then the $p$-conjecture is true for $(\E, \L)$. 
\end{theorem}

\begin{proof}
\begin{enumerate}
\item Kronecker analytic criterion in a more general version. Then the $(\E, \nabla)_s$ condition implies 

\item 
\end{enumerate}
\end{proof}

\section{Lecture 3 (Oct. 4)}


Let $\Gamma$ be a finitely generated group. We consider the following completions,
\begin{center}
\begin{tikzcd}
\Gamma \arrow[r] \arrow[rd] & \widehat{\Gamma} \arrow[r, equals] & \varprojlim\limits_{\substack{\Gamma \onto Q \\ \text{Q is finite}}} Q
\\
& \Gamma^{\alg} \arrow[u, dashed, two heads] \arrow[r, equals] & \varprojlim\limits_{\varphi : \Gamma \to \GL_r(\CC)} \overline{\im{\varphi}} \arrow[u, dashed]
\end{tikzcd}
\end{center}

\begin{theorem}[Mal\v{c}ev, 1940]
If $\widehat{\Gamma} = \{ 1 \}$ then $\Gamma^{\alg} = \{ 1 \}$.
\end{theorem}

\begin{proof}
Let $\widehat{\Gamma} = \{ 1 \}$. For any map $\rho : \Gamma \to \GL_r(\CC)$ because $\Gamma$ is finitely generated, there is a finite type $\Z$-algebra $A \subset \CC$ such that $\rho$ factors through $\GL_r(A) \subset \GL_r(\CC)$. Let $\m \subset A$ be some maximal ideal and $\kappa = A / \m$ which is finite since $A$ is finite type over $\Z$. Then,
\[ A \embed \widehat{A} = \varprojlim_n (A / \m^n) \]
Then we get a map,
\[ \hat{\rho} : \Gamma \to \GL_r(A) \to \GL_r(\hat{A}) \]
Then since,
\[ \GL_r(\hat{A}) = \widehat{\GL_r(A)} = \varprojlim_{n} \GL_r(A/\m^n) \]
is profinite $\hat{\rho}$ factors through $\widehat{\Gamma} = \{ 1 \}$ and hence is trivial. Therefore, $\rho = \id$ so $\Gamma^{\alg} = \{ 1 \}$.  
\end{proof}

\begin{theorem}[Grothendieck]
Let $X$ be a smooth quasi-projective $\CC$-variety. If the \etale fundamental group,
\[ \pi_1^{\et}(X) = \widehat{\pi_1(X(\CC))} = \{ 1 \} \]
is trivial then there are no non-trivial $\struct{X}$-coherent regular singular $D$-modules.
\end{theorem}

\begin{proof}
This is a consequence of Mal\v{c}ev since $\pi_1(X(\CC))$ is finitely generated. 
\end{proof}

\begin{rmk}
Then $X(\CC)$ is a complex manifold and the Riemann-Hilbert map,
\[ \{ \CC\text{-local systems} \} \xrightarrow{RH} \{ (E, \nabla) \text{ analytic vector bundles with flat analytic connection} \} \]
given by sending $\L \mapsto (\L \ot \struct{X^{\an}}, \id \ot \d)$ is an equivalence of Tannakian categories. The inverse map is given by $(\E, \nabla) \mapsto \E^{\nabla}$ which using Kovalevskaya is a local system of the proper rank.
\end{rmk}

\begin{theorem}[Deligne]
There is an enhancement of Riemann-Hilbert to,
\[ \{ \CC\text{-local systems} \} \xrightarrow{RH} \{ (E, \nabla) \text{ algebraic vector bundles with flat regular singular connection} \} \]
\end{theorem}

\begin{rmk}
We can think of $(E, \nabla)$ having regular singularities as meaning that there is a good compactification $\overline{X}$ of $X$ with an extension of $(E, \nabla)$ to $(\overline{E}, \overline{\nabla})$ where the connection maps to logarithmic forms on the boundary. This can then be extended to the theory of $D$-modules. 
\end{rmk}

The finial formulation descends to,
$X_{\overline{F}}$ where $F \subset \CC$ is a finitely genrated field over $\Q$. Then since,
\[ \pi_1^{\et}(X_{\overline{F}}) = \pi_1^{\et}(X_{\CC}) \]
we can conclude that if $\pi_1^{\et}(X_{\overline{F}}) = \{ 1 \}$ then $X$ has no nontrivial regular singular $\struct{X}$-coherent $D$-modules.

\begin{rmk}
There is no known purely algebraic proof of this statement. We would also like to find an analg for $F$ a field of characteristic $p > 0$. 
\end{rmk}

\begin{conj}[Gieseker, 1975]
Let $X$ be smooth projective over $k$ with $k$ an algebraically closed field of charactersitic $p > 0$. Suppose that $\pi_1(X) = \{ 1 \}$ then there are no nontrivial $\struct{X}$-coherent $D$-modules. 
\end{conj}

\begin{theorem}[E-Mehta, 2010]
Gieseker's conjecture is true. 
\end{theorem}

\begin{rmk}
The work of Katz: with $X$ quasi-projective smooth over $k = \bar{k}$ there is a Riemann-Hilbert correspondence,
\[ \mathrm{Loc}_{\FF_p} \iso \{ (E, \varphi) \} \]
where $E$ is a vector bundle and $\varphi : E \iso E$ is a Frob-semilinear isomorphism of abelian sheaves (not $\struct{X}$-linear). 
\end{rmk}

\begin{rmk}
The dream is that for a $\struct{X}$-coherent $D$-module then construct a $\varphi$-module which thus is trivial since it corresponds to a local system. 
\end{rmk}

\begin{prop}
$\struct{X}$-coherent $D$-modules correspond to $\Frob$-divided sheaves.
\end{prop}

\begin{rmk}
A $\Frob$-divided sheaf is a sequence of sheaves $E_i$ equipped with isomorphisms $E_i \iso \Frob^*_X E_{i+1}$. This implies that $E_0$ is localyl free. 
\end{rmk}

\begin{rmk}
We want to show that if $\pi_1(X_{\overline{k}}) = 0$ then there are no nontrivial $\Frob$-divided sheaves. Suppose that $E_\bullet$ is nontrivial. Then find a new locally free sheaf $M$ of the same rank sch that,
\[ \Frob_X^* M \iso M \]
We claim this is enough. In the case $M$ has rank $1$ this says $M^{\ot p} \iso M$ and therefore $M^{\ot (p^m - 1)} \iso \struct{X}$ and thus we get a Kummer $\mu_{p^{n}-1}$ cover which contradicts $\pi_1(X_{\overline{k}}) = \{ 1 \}$ undless $M \cong \struct{X}$. More generally $\Frob_X^* M \iso M$ produces a $\GL_r(\FF_{p^n})$-cover $\pi : Y \to X$ and hence we can conclude that $M$ is trivial.
\end{rmk}

\subsection{Grothendieck Specialization}

If we spread out $X$ to $\X \to \Spec{S}$ where $S$ is a finite type $\FF_p$-algebra. Then for any closed point $s$ the map,
\[ \pi_1(X_{\overline{k}}) \onto \pi_1(\X_{\bar{s}}) \]
is surjective. Therefore, we can work over $\overline{\FF}_p$.

\subsection{Model Theory}

Let $k = \overline{\FF}_p$. Let $X$ be defined over $\FF_q$. Haushouski considered correspondences $Y \subset X_{k} \ot_k X_k$ of dimension $\dim{Y} = \dim{X} = d$ which is defined over some $\FF_q$. There exists a dense set of clsoed points of $X_{k}$ such that $(x, (\Frob_X^{\m_x}) x) \in Y(k)$. 
\bigskip\\
In particular, for any map $\psi : X \to X$ there is a dense set of fixed points of powers of $\psi$. 

\section{Lecture 4 (Oct. 11)}

\begin{defn}
Let $\Gamma$ be an abstract group. Then $\Gamma$ is \textit{finitely generated} if there is a surjection $F_n \onto \Gamma$. Morover, it is \textit{finitely presented} if the kernel $K = \ker{(F_n \onto \Gamma)}$ is finitely generated as a normal subgroup of $F_n$ meaning there are finitely many elements $r_i \in K$ such that $K$ is the smallest normal subgroup containing $r_i$.  
\end{defn}

\begin{defn}
Let $\Gamma$ be a profinite group then we say that $\Gamma$ is \textit{finitely generated} if there is a surjection $\widehat{F}_n \onto G$ where $\widehat{F}_n$ is the profinite completino of the free group. Then $\Gamma$ is \textit{finitely presented} if the kernel $K = \ker{(\widehat{F}_n \to \Gamma)}$ is finitely generated as a closed normal subgroup menaing there are finitely many elements $r_i \in K$ such that $K$ is the smallest closed normal subgroup containig $r_i$. 
\end{defn}

\begin{example}
For $\Gamma = \pi_1(X(\CC), x)$ for $X$ quasi-projective smooth over $\CC$ then $\Gamma$ is finitely presented and $\widehat{\Gamma}$ is also finitely presented. But there exist finitely presented profinite groups with no discrete structure.
\end{example}

\begin{rmk}
Question: if $X$ is quasi-projective smooht over $k = \bar{k}$ and $\ch{k} = p > 0$ then is $\pi_1(X)$,
\begin{enumerate}
\item finitely generated - Yes if $X$ is projective by Lefschetz 
\item finitely presented 
\end{enumerate}
\end{rmk}

\begin{example}
Let $X = \A^1$ then,
\[ \pi_1(\A^1) \onto \bigoplus_{i = 1}^S (\Z / p) \]
with $S$ as big as we want from Artin-Schrier theory so $\pi_1(\A^1)$ is not f.g.
\end{example}

\begin{defn}
Let $\overline{X} = \Spec{R}$ where $R$ is a DVR and $X$ is the punctured DVR. Then,
\[ \pi_1(X) \onto \pi_1^{\tame}(X) \]
is defined by considering the tame covers (a cover of $X$ is tame if the normalization $R' / R$ is a tame extension meaning the ramification index is coprime to $p$ and the residue field extension is separable). 
\end{defn}

\begin{defn}
We say that a finite \etale cover $f : Y \to X$ is \textit{tame} if for every map from a smooth curve $C \to X$ the map $Y_C \to C$ is a tame \etale cover. This allows us to define the tame fundamental group $\pi_1^{\tame}(X)$. 
\end{defn}

\begin{rmk}
There are universal covers $\wt{X} \to X^{\tame} \to X$ with Galois groups $K$ and $\pi_1^\tame(X)$ respectively and total Galois group $\pi_1(X)$ fitting into the exact sequence,
\begin{center}
\begin{tikzcd}
1 \arrow[r] & K \arrow[r] & \pi_1(X) \arrow[r] & \pi_1^{\tame}(X) \arrow[r] & 1
\end{tikzcd}
\end{center}
\end{rmk}

\begin{lemma}
If $X \embed \overline{X}$ with $\overline{X}$ smooth projective and $\overline{X} \sm X$ is normal crossings divisor (good compactification) then $\pi_1^{\tame}(X)$ is finitely generated.
\end{lemma}

\begin{rmk}
We want to know if $\pi_1^{\tame}(X)$ is finitely presented? 
\end{rmk}

\begin{rmk}
There is no cohmological criterion for an \textit{abstract} group to be finitely presented. However, there is the following theorem.
\end{rmk}

\begin{theorem}[Zebotzky]
Let $\Gamma$ be a finitely generated profinite group. Then $\Gamma$ is finitely presented if and only if for all primes $\ell \in \Z$ and continuous representation $\rho : \Gamma \to \GL_r(\FF_\ell)$ then $\dim_{\FF_\ell} H^2(\Gamma, \rho) \le C \rank{\rho}$ for some uniform constant $C$ (independent of $\ell$ and $\rho$). 
\end{theorem}

\begin{theorem}[E-Shusterman-Srinivas]
If $X$ is quasi-projective over $\FF_q$ and admits a good compactification then $\pi_1^{\tame}(X)$ is finitely presented. 
\end{theorem}

\begin{rmk}
For $\ell \neq p$ the proof goes by considering,
\[ H^2(\pi_1^{\tame}(X), \rho) \embed H^2(X, \L) \]
where $\L$ is the associated local system to $\rho : G \to \GL_r(\FF_{\ell})$. 
\bigskip\\
For $\ell = p$ we prove that,
\[ H^2(\pi_1^{\tame}(X), \rho) \embed H^2(\overline{X}, j_* \L) \]
\end{rmk}

\begin{rmk}
Is the assumption necessary? 
\end{rmk}

\begin{lemma}[Delgine]
If $\L$ is tame then,
\[ \chi(X, \L) = \sum (-1)^i \dim H^i(X, \L) = (\rank{\L}) \cdot \chi(X, \FF_\ell) \]
\end{lemma}

\begin{rmk}
For the $\dim{X} = 1$ case this follows from the Grothendieck-Ogg-Shafarevich formula.
\end{rmk}

There is an exact sequence,
\begin{center}
\begin{tikzcd}
H^0(C, \L) \arrow[r] & H^2(X, \L) \arrow[r] & H^2(X \sm C, \L) 
\end{tikzcd}
\end{center}
So we replace $H^2$ by $H^1(X, \L) = H^1(\pi_1, \L) = H^1(\pi_1^\tame, \L)$. This is given by a cocycle $\pi_1^\tame \to \FF_\ell^{\op r}$ and there are a bounded number of such cocycles because $\pi_1^\tame$ is finitely generated and $\FF_{\ell}^{\op r}$ is finite. 


\begin{rmk}
We don't actually need the good compactification assumption for the $\ell \neq p$ cases using alterations.
\end{rmk}

Now for the case $\ell = p$ we consider the compactification $X \embed \overline{X}$. 

\section{Lecture 5 (Oct. 18)}

\begin{rmk}
Recall, if we take an abstract group $G$ which is finitely generated (resp. finitely presented) then $\widehat{G}$ is topologicaly finitely generated (resp. topologically finitely presented). 
\end{rmk}

\begin{prop}
Over $\CC$ we know that $\pi_1^\et(X)$ is the profinite completion of a finitely generated group nametly $\pi_1(X(\CC))$. However, today we will see that this is fale in positive characteristic. 
\end{prop}

\begin{defn}
Let $\pi$ be a profinite group. We say that $\pi$ is \textit{quasi-$p'$-finitely-generated} if there is a finitely generated abstract group $\Gamma$ with a map $\Gamma \to \pi$ such that,
\begin{enumerate}
\item the completion $\widehat{\Gamma} \onto \pi$ is surjective
\item the diagram,
\begin{center}
\begin{tikzcd}
\widehat{\Gamma} \arrow[r, two heads] \arrow[d, two heads] & \pi \arrow[d, two heads]
\\
(\widehat{\Gamma})^{p'} \arrow[r, "\sim"] & \pi^{p'}
\end{tikzcd}
\end{center}
gives an isomrophism on the pro-$p'$-completion.
\end{enumerate}
We say that $\pi$ is $p'$-finitely presented if we can take $\Gamma$ to be fintiely presented as an abstract group.
\end{defn}

\begin{example}
If $X$ is smooth quasi-projective over $\CC$ then $\Gamma = \pi_1(X(\CC))$ and $\pi_1 = \pi_(X_{\CC})$ then is $p'$-finitely presented for all $p$.
\end{example}

\begin{example}
Let $X \to S$ smooth projective with $S \to \Spec{\Z}$ affine and finite type. For any closed point $s \in |S|$ then the specialization map $\sp : \pi_1(X_{\CC}) \onto \pi_1(X_{\bar{s}})$ is surjective and an isomorphism on the $p'$-completion. The same is true for $\pi_1^{\tame}$ for $X$ quasi-projective with a good relative compactification. Then Grothendieck existence also shows that any finite \etale cover $Y \to X_{\bar{s}}$ lifts to characteristic zero so we also get a finite generation on the right sort. 
\end{example}

\begin{defn}
We say that $\pi$ is $p'$-finitely generated (resp. presented) if all open subgroups $U \subset \pi$ are quasi-$p'$-finitely generate (resp. presented).
\end{defn}

\begin{theorem}
$p$'-finite generation is an obstruction for the existence of a lift to characteristic zero. Explicitly, there is $X$ smooth projective over $\overline{\FF}_p$ with $\pi_1^{\et}(X)$ not $p'$-finite-generated and thus cannot lift to characteristic zero.
\end{theorem}

\subsubsection{Serre}

Let $k = \bar{k}$. There is a $G$-cover $Y \to X$ with $Y \subset \P^n$ a smooth complete intersection of $\dim{Y} \ge 3$ and thus $\pi_1(Y) = 1$. Then $\pi_1(X) = G \embed \PGL_n(k)$. If $X$ lifts then so does $Y$ and this shows that the representation $G \to \PGL_n(k)$ lifts but Serre shows that this is impossible. 

\subsubsection{Deligne-Illusie}

Let $X$ be smooth projective. If $X$ lifts over $W_2(k)$ then we have HtdR degeneration,
\[ b_n = \sum_{p + q = n} h^{p,q} \]
However, this is not an obstruction to lift to characteristic zero! Enriques surfaces over lift to characteristic zero but these do not satisfy HrdR degeneration. These Enriques surfaces do lift to characteristic zero but over a ring ramified over $W_2(k)$.

\subsubsection{Achinger-Zdanowing}

Explicit example (Tate 

van Dobben de Bruyn. If $X \subset C$ with $C$ supersingular hyperelliptic $g \ge 2$. Then any alteration of $X$ does not lift to characteristic zero. 

This theorem is in contrast to a theorem by Achinger: 

\begin{thm}[Artin]
if $X$ is smooth over $\CC$ there is basis of the Zariski topology which are $K(\pi_1)$. In particular, for any localy system $\L$ with finite monodromy (irrelevant over characteristic zero) then $H^1(\pi_1(U), \L_{\bar{s}}) \iso H^1(U, \L)$ is an isomorphism. 
\end{thm}

\begin{theorem}[Achinger]
Let $X$ be over $k = \overline{k}$ then every affine $U \subset X$ is a $K(\pi_1)$ meaning every local system with finite monodromy the map $H^1(\pi_1(U), \L_{\bar{s}}) \iso H^1(U, \L)$. 
\end{theorem}

\subsection{Main Tool From the Defn}

Consider a finite $G$ which fits into a quotient diagram,
\begin{center}
\begin{tikzcd}
1 \arrow[r] & U_{G} \arrow[r] & \pi \arrow[r] & G \arrow[r] & 1
\end{tikzcd}
\end{center}
This gives an outer action $G \to \mathrm{Out}(U_G)$ which lifts to an action when $\pi \to G$ has a section. Therefore, we get an action $G \to \Aut{U_G^\ab}$. If $\pi = \pi_1(X)$ then $\pi_1(X) \to G$ defines a connected finite \etale cover $Y \to X$ then $\pi_1(Y) = U_G$. If we were over $\CC$ then $U_G^\ab = H_1(Y, \ZZ)$ so $G$ would act on homology. We can actually show that we get an action,
\[ G \to \GL(H^1(Y_\et, \Q_\ell)) \]
In fact, there is an underling $\Q$-vectorspace such that $V \ot_{\Q} \Q_{\ell} = H^1(Y_{\et}, \Q_\ell)$ whcih is a representation $\rho : G \to \Aut{V}$ compatibly with $\rho = \rho_\ell$. 

\begin{rmk}
Let $\pi$ be $p'$-finitely generated then $\rho_{\varphi, \ell}$ is Schol rational. We will exhibit an example for which $\rho_{\varphi, \ell}$ is not Schol rational.  
\end{rmk}

Let $C$ be a curve of genus $g \ge 2$ over $k = \bar{k}$ characteristic zero. Then,
\[ |\Aut{C}| \le 84(g - 1) \]
However, if $C$ is over $k = \bar{k}$ of characteristic $p > 0$ then $\Aut{C}$ is still finite but may be larger than the Hurwitz bound.

\begin{example}
Hoquette curve: $C \to \P^1$ hyperelliptic curve over $\FF_p$ branched exactly over the $p+1$ rational points of $\P^1$. Thus we can write it as,
\[ y^2 = x^p - x \]
Thus has genus,
\[ g = \frac{p-1}{2} \]
and we assume $p \ge 5$ such that $g \ge 2$. Then,
\[ |\Aut{C}| = 2p(p^2 - 1) > 84 (g - 1) = 42 (p - 3) \]
Then the map,
\[ \Aut{C} \to \GL(H^1(C, \Q_\ell)) \]
is injective using the trace formula. Indeed, if $g \in G$ then the graph $\Gamma_g \subset C \times C$ then the fixed points $\Gamma_g \cdot \Delta$ is up to a constant the trace acting on cohomology. So if $g$ acts trivially on $H^1$ then $\Gamma_g \cdot \Delta = 2 - 2 g < 0$ which is impossible unless $\Delta_g = \Delta$. 
\bigskip\\
Then $\rho_{\ell}$ is not Schur rational. Consider,
\[ \Q_{\ell}[G] \to \End{H^1(C, \Q_\ell)}^{\Frob} \]
Then you learn that this image is $\mathrm{End}^0(C) \ot \Q_{\ell}$. There is some ramification that obstructs rationality. 
\end{example}

Now we return to Serre. Set $P = Y \subset \P^n$ then $G \acts P$ fixed point free. Then $Z  = C \times P$ equipped with the diagonal $G$-action is also a fixed-point free action. Then consider $X = Z / G \to P/G$ is an isotrivial fiber bundle with fiber $C$. Then either using the homotopy fiber sequence for the fibration or the quotient of $Z$ sequence. Then there is an exact sequence,
\begin{center}
\begin{tikzcd}
1 \arrow[r] &  \pi_1(C) \arrow[r] & \pi_1(X) \arrow[r] & G \arrow[r] & 1
\end{tikzcd}
\end{center}

\section{Lecture 6 (Oct. 25)}

\begin{defn}
A matrix $M$ is \textit{quasi-unipotent} if one of the following equivalent conditions holds,
\begin{enumerate}
\item $M^n$ is unipotent for some $n \ge 0$
\item $M$ has eigenvalues which are roots of unity.
\end{enumerate}
\end{defn}

Let $X$ be a normal variety over $\CC$. Let $X \embed \overline{X}$ be a normal compact completion. Let $D_i$ be the irreducible components of $\overline{X} \sm X$. Let $T_i$ be a small loopp contained at $D_i$. Such that $\im{T_i}$ are defined in $\pi_1(X(\CC))$ up to conjugacy. Let $\rho : \pi_1(X(\CC)) \to \GL_r(\CC)$ be a local system $L_{\CC}$ which we require to have quasi-unipotent monodromy at infity meaning $T_i$ acts quasi-unipotently. Kashiwana showed that this does not depend on the compactification.

\begin{defn}
Let $x \in X$ and a map $\rho : \pi_1(X(\CC), x) \to \GL_r(\CC)$ for $r \in \NN_{\ge 1}$. Then,
\[ \Hom{}{\pi_1(X(\CC), x)}{\GL_r} \]
This gives a funcor on rings over $\ZZ$,
\[ R \mapsto \{ \rho_{R} : \pi \to \GL_r(R) \} \]
This is representable by an affine scheme over $\ZZ$ called,
\[ \Ch(\pi, r)^{\square} \quad \text{ or } \quad M_B(X, r)^{\square} \]
\end{defn}

\begin{defn}
Consider the quasi-unipotent locus $QU \subset M_B(X, r)(\CC)$ of $L_{\CC}$ which are quasi-unipotent monodromy at $\infty$.
\end{defn}

\begin{theorem}[E-M. Kerz, 20] 
$QU \subset M_B(X, r)^{\square}$ is Zariski dense. 
\end{theorem}

\begin{theorem}[Grothendieck]
If $L_{\CC}$ is geometric\footnote{Recall this means there is a dense open $U \subset X$ and a smooth projective morphism $f : Y \to U$ such that $L_{\CC}|_U$ is a subquotient of the Gauss-Mannin system $\bigoplus_{i} R^i g_* \underline{\CC}$. By Deligne semi-simplicitly we can assume it is actually a direct summand.} then $L_{\CC} \in QU$
\end{theorem}

\begin{theorem}[Briskorn]
Suppose that $Y \to U \embed X$ is defined over $F$ where $F$ is a finitely generated field over $\Q$. Let $F = \Frac{\struct{S}}$ with $S$ affine finite type over $\Q$. Then by base change we may assume that $F$ is a number field. 
\end{theorem}

\begin{proof}
Riemann-Hilbert correspondence:
\[ \bigoplus_{i} R^i g_* \underline{\CC}  = \left( \bigoplus_{i} R^i g_* \Omega^\bullet_{Y/U} \right)^{\nabla} \]
and since this local system is defined over $\ZZ$ it has eigenvalues of $\rho(T_j)$ algebraic integers $\mu_i \in \overline{\ZZ} \subset \overline{\QQ}$. This has residues $\lambda_i \in \overline{\QQ}$. Then,
\[ \exp(2 \pi i \lambda_i) = \mu_i \] 
Gelfand's theorem implies that $\lambda \in \QQ$. 
\end{proof}

Also $X_F \to X_S$ affine scheem finite type over $\Z$. Let $S^\circ \subset S$ nonempty open. Consider the diagram,
\begin{center}
\begin{tikzcd}
\pi_1(X_{\CC}, x) \arrow[r] & \pi_1^{\tame}(X_s, x_S)
\\
\pi_1(X(\CC), x) \arrow[u] \arrow[ru]
\end{tikzcd}
\end{center}
Then the images of $T_i$ are also quasi-unipotent. Then for $L_{\CC}$ we can consider this for $L_{\ell}$ an $\ell$-adic local system. For $\ch{s} \gg 0$ something ((!(!!!))


\begin{prop}[Deligne]
Let $T_i^{\et} \in \pi_1(X_{\CC}, x)$ be the image of the $T_i$. Consider the homotopy exact sequence,
\begin{center}
\begin{tikzcd}
1 \arrow[r] & \pi_1(X_{\CC}, x) \arrow[r] & \pi_1(X_{F}, x) \arrow[r] & G_F \arrow[r] & 1
\end{tikzcd}
\end{center}
Given a rational point $\sigma \in X_F(F)$ we get a section $\sigma : G_F \to \pi_1(X_F, x)$. Depending on the choice of $\sigma$ we get an action,
\[ \delta_{\sigma} : G_F \to \Aut{\pi_1(X_{\C}, x)} \]
Then for $\gamma \in G_F$ then $\gamma(T_i^{\et}) = (T_i^{\et})^{\chi(\gamma)}$ where $\chi : G_F \to \widehat{\ZZ}^\times$ is the cyclotomic character\footnote{$\chi = \prod_{\ell} \chi_{\ell}$ where $\chi_{\ell} : G_F \to \ZZ_\ell^\times$ is the standard cyclotomic character.}. 
\end{prop}

\begin{proof}
Consider the moduli space $M^\square$ of framed representations. Consider the conjugacy classes of $g_i \subset \pi_1(X(\CC), x)$ of the $T_i$. The characteristic polynomials,
\[ \ch{\rho(g)} = \det{T - \rho(g_i)} = T^r - S_n(\rho(g_i) T^{r-1} + \cdots + s_r(\rho(g_i))  \]
Then consider the diagram,
\begin{center}
\begin{tikzcd}
M^\square \arrow[rd, "\psi"] & S = \prod_{1}^s (\Gm^r) \arrow[d, "\varphi"] 
\\
& T \subset \prod^s_1 \left( \A^{r-1} \times \Gm \right) 
\end{tikzcd}
\end{center}
where $R$ is the closure of $\psi(M^{\square})$. The goal is to show that the image under $\varphi$ of the quasi-unipotent part is dense in $T$. We can spread out this construction to $B$ for some affine scheem finite type over $\ZZ$ with $\struct{B} \subset \CC$. 
\bigskip\\
Assume for contradiction that $\overline{QU} \subsetneq M^{\square}(\CC)$. When we consider $M^{\square}_B \to B$ the map $M_B^{\square} \sm \overline{Q}_B \to B$ is dominant (WHY)
\end{proof}

\section{Lecture 7 (Nov. 1) Companions}

Let $X$ be a smooth projective variety over $\CC$ and $\LL_{\CC}$ a complex local system. Let $\tau : \CC \iso \CC$. This corresponds to a representation,
\[ \rho : \pi_1(X(\CC)) \to \GL_r(\CC) \]
The considering $\rho^{\tau} = \tau \circ \rho$ we get a new local system $\LL^\tau$. 

This gives a diagram,
\begin{center}
\begin{tikzcd}
M_B(X, r)_{\CC} \arrow[d, "(-)^{\tau}"] \arrow[r, "\psi"] & \prod_{1} (\A^{r-1} \times \Gm)_{\CC} = N_{\CC} \arrow[d]
\\
M_B(X, r) \arrow[r] & N_{\CC}
\end{tikzcd}
\end{center}

\subsection{Over Finite Fields}

Let $X$ be a quasi-projective variety over $\FF_q$ and $M^{\irred}_\ell(X,r)$ be the set of simple arithmetic $\LL_{\ell}$ on $X_{\overline{\FF}_p}$ with $\ell \neq p$. For a field isomorphism $\sigma : \Qbar_{\ell} \to \Qbar_{\ell'}$ we get a diagram,
\begin{center}
\begin{tikzcd}
\rho : \pi_1(X_{\overline{\FF}_p}) \arrow[r, "\rho"] \arrow[rd, "\rho^{\sigma}"] & \GL_r(\Qbar_\ell) \arrow[d, "\sigma"]
\\
& \GL_r(\Qbar_{\ell'}) 
\end{tikzcd}
\end{center}
however now $\rho^{\sigma}$ is \textit{not} continuous because there is no continuous isomorphism $\Qbar_{\ell} \iso \Qbar_{\ell'}$ if $\ell \neq \ell'$. Thus $\rho^{\sigma}$ does not define a $\ell'$-local system. However, what we need is a diagram,
\begin{center}
\begin{tikzcd}
M^{\irred}_{\ell}(X, r) \arrow[r, "\psi^{\infty}"] & N^{\infty}(\Qbar) \arrow[d, "\sigma"]
\\
& N^{\infty}(\Qbar)
\end{tikzcd}
\end{center}
so that the downward map preserves being a local system. 
Let,
\[ N^\infty = \prod_{|X|} (\A^{r-1} \times \Gm) \]
where by Chebotarev we know that the representation is characterized by where the Frobenii land. We know,
\[ \LL_{\ell} \mapsto \mathrm{char}(\LL_\ell) = \det{(T - \rho(g))} \]
for $x \in X(\FF_{q'})$ and $g = \Frob_x$ gives characteristic polynomial with algebraic coefficients.

\begin{conj}[Deligne, Companions]
There is a diagram,
\begin{center}
\begin{tikzcd}
M_\ell^{\irred}(X, r) \arrow[r, hook, "\psi^\infty"] & N^\infty(\Qbar) \arrow[d, "\sigma"]
\\
M^{\irred}_{\ell'} \arrow[r] & N^{\infty}(\Qbar)
\end{tikzcd}
\end{center}
\end{conj}

\begin{theorem}[Lafforge]
Holds for $X$ a smooth curve over $\FF_q$ and irred arithmetic $\ell$-adic local systems are geometric.
\end{theorem}

\begin{theorem}[Drinfeld]
Let $X$ be smooth $\dim{X} > 1$
\end{theorem}

\begin{proof}
Step zero (Deligne): if $\LL_{\ell}$ is a $\ell$-adic local system then,
\[ \psi^{\infty}(\LL_{\ell}) \in N^{\infty}(E) \]
for a fixed number field so there is a lattice $E^r$ such that $\rho_{\LL_\ell} : \pi_1(X_{\overline{\FF}_q}, x) \to \GL_r(\struct{E})$. Since $\sigma(\struct{E}) = \struct{E}$ the companion, if it exists, are defined over $A$ for $A \embed \Qbar_{\ell'}$ is the completion at a place over $\ell'$. 
First show upper bound to a cover of $X$ domination all galois coveres which come from: for $\varphi : C \to X$ the companion,
\[ (\varphi^* \LL_{\ell})^{\sigma} \iff \pi_1(C, x) \to \GL_r(\Q) \]


Second step: there exists a curve such that $\pi_1(C) \to \pi_1(X) / H \to \text{finite}$

\end{proof}


\begin{rmk}
Warning: there is \textit{no} Lefschetz theory over $\FF_q$. Indeed, there is no curve $C \embed \A^2$ such that $\pi_1(C) \to \pi_1(\A^2)$ is surjective.
\end{rmk}

\section{Lecture 8 (Nov. 15)}

\begin{rmk}
Simpson's conjecture says that rigid integrable connections arise from geometry. This implies the following conjecture.
\end{rmk}

\begin{theorem}[E. Grocdenz, 2018]
Let $X$ be smooth projective over $\CC$ and $(E, \Delta)$ a rigid (isolated in the moduli space $M_{\dR}(X, r, \L)$ where $\L$ is the determinant character). Then we can spread out to $\X \to \Spec{S}$ where $S$ is a finite type smooth $\Z$-algebra and spread out to $(E, \nabla)_S$. For any $s \in |S|$ closed (really need $\kappa(s)$ perfect not just $\FF_q$) then along $\Spec{W(\kappa(s))} \to S$ we restrict to the formal scheme $\widehat{X_W}$ completed along $X_{\kappa(s)}$ then $(E, \nabla)|_{\wh{X_W}}$ has the structure of an $F$-isocrystal.
\end{theorem}

\subsection{$F$-isocrystals}

Where does the $F$-isocrystal structure come from if the connection is geometric? Let's take the special case that $X = U$ and we have an actual cover $g : Y \to X$. Then we spread out to $g : Y_S \to X_S$ and restrict to characteristic $p$ to get,
\begin{center}
\begin{tikzcd}
Y_{\FF_q} \arrow[d] & Y^{(p)}_{\FF_q} \arrow[r] \arrow[d, "g'"] & Y_{\FF_q} \arrow[d]
\\
X_{\FF_q} \arrow[r, "\Frob_q"] & X^{(p)}_{\FF_q} \arrow[d] \arrow[r] & X_{\FF_q} \arrow[d]
\\
& \Spec{\FF_q} \arrow[r, "F"] & \Spec{\FF_q}
\end{tikzcd}
\end{center}
Then $GM(g') = \Frob_{\text{arith}}^* GM(g)$. Consider the good filtration (the one that passes to the derived category) with $\cH^i(F^\bullet) = \Frob_{\text{geom}}^* (\Omega^\bullet_{Y^{(p)}/X^{(p)}})$ by Cartier isomorphism. Then $GM(g)$ is filtered by subconnections such that the graded parts are spanned by flat sections.

\begin{defn}
Let $\wh{X_W}$ be a smooth formal scheme. Let $(\wh{E}, \wh{\nabla})_W = \varprojlim_n (E_n, \nabla_n)$ where each is a connection on $\wh{X}_W / p^n = X_{W_n}$. There is a filtration $F^\bullet \subset (E_1, \nabla_1)$ such that $\nabla_1(F^i) \subset \Omega_X^1 \ot F^i$ and $\gr^F(E_1)$ is spanned by flat sections and by Carter isomorphism,
\[ F^* \]
Then $(\wh{E}, \wh{\nabla})$ is a crystal.
\end{defn}

\begin{defn}
The category of crystals is $W$-linear. The category of isocrystals is the isogeny category of the category of crystals. This is the same as just remembering the generic fiber $(\wh{E}, \wh{\nabla})_K$ but knowing that it comes from an object with a $W$-lattice. 
\end{defn}

\begin{rmk}
Let $F : X_{\FF_q} \to X_{\FF_q}$ be the frobenius. If $\hat{F} : \wh{X} \to \wh{X}$ is a lift to the formal model. Then $\hat{F}^* (\wh{E}, \wh{\nabla})$ is defined.
\end{rmk}

\begin{theorem}
If $(\wh{E}, \wh{\nabla})$ is a crystal. By smoothness, locally on $X_1$ there is a lift $\hat{F}$ so we have locally defined $\hat{F}^* (\wh{E}, \wh{\nabla})$ and these glue.
\end{theorem}

\begin{defn}
An $F$-isocrystal is an isocrystal equipped with an isomorphism (as isocrystals) $\hat{F}^* (\wh{E}, \wh{\nabla})_K \iso (\wh{E}, \wh{\nabla})_K$.
\end{defn}


\section{Lecture 9 (Nov. 22)}



\section{Lecture 10 (Nov. 29)}

Let $X$ be a smooth quasi-projective variety over $\CC$ and $(E, \nabla)$ a flat connection. We spread out to a model $X_S$ over some finite type $\Z$-scheme $S$. Consider,
\begin{center}
\begin{tikzcd}
X \arrow[d] \arrow[r] & X_S \arrow[d]
\\
\Spec{\CC} \arrow[r] & S
\end{tikzcd}
\end{center}

If for every $s \in |S^\circ| \subset S$ with $S^\circ$ is a dense open we have $(E, \nabla)_s$ on $X_s$ is spanned by flat sections then there is $h : Y \to X$ finite \etale such that $h^*(E, \nabla)$ is trivial (i.e. is spanned by flat sections). 
\bigskip\\
We replace the local condition with a weaker one: that there exists a filtration $\Fil_s \subset (E, \nabla)_s$ such that the graded parts are spanned by flat sections. This implies the weaker concludsion that $(E, \nabla)$ is \textit{geometric} meaning there is $g : Y \to U \subset X$ open dense a smooth projective morphism $g$ such that $(E, \nabla)$ is a subquotient of $R^i g_* (\struct{}, \d)$. 
\bigskip\\
Valuge programs:

what if you replace $X$ with the formal completion of $X$ along a subscheme? Why would you do this? Over $\FF_q$, we do have $\ell$-adic ``companions'' in the sense of Delgine if $X$ is smooth. But Deligne (Weil II) conjectured this for $X$ normal. This is a serious generalization though it seems like a minor modification. Let $X$ be normal, then we know resolution of singularities in characteristic $p$ giving $h : \wt{X} \to X$. Then there is a smooth open $U \embed \wt{X}$ mapping isomorphically into $X$. Let $C \subset \wt{X}$ be a contracted curve. For an $\ell$-adic local system $\LL_{\ell}$ and an isomorphism $\sigma : \Qbar_{\ell} \iso \Qbar_{\ell'}$ we ask if there is a companion: $\LL_{\ell'}$ which is required to satisfy,
\[ \forall x \in |X| : \sigma(\det{(T - F_x | \LL_{\ell, \bar{x}})}) = \det{(T - F_x | \LL_{\ell', \bar{x}})} \]
If we consider $h^* \LL_{\ell}$ then since $\wt{X}$ is smooth, there is a companion $\LL_{\ell'/\wt{X}}$. For $x \in |C|$ we have,
\[ \det{(T - F_x | \LL_{\ell, \bar{x}})} = \text{trivial} \]
meaning up to a field extension it is a fixed polynomial. Therefore, likewise for $\LL_{\ell', \wt{X}}$. However, there can be interesting topology to $C$, combinatorial loops coming from multiple components.

\subsection{Alex Fubobsky}

\begin{theorem}
Let $\pi$ be a profinite group, topologically finitely generated. Then $\pi$ is finitely presented iff there exists $C \ge 0$ such that for all primes $\ell$ and integer $r$ for any,
\[ \rho : \pi \to \GL_r(\FF_\ell) \]
then,
\[ \dim_{\FF_\ell} H^2(\pi, \rho) \le C \cdot r \]
\end{theorem}

\begin{cor}
If $X$ is smooth and $X \embed \overline{X}$ is a good compactification then $\pi^{\tame}(X)$ is finitely presented.
\end{cor}

\begin{enumerate}
\item Let $\pi$ be profintie and $\rho : \pi \to \GL_r(\FF_\ell)$ then,
\[ \dim_{\FF_\ell} H^0(\pi, \rho) \le r \]
because it is the invariants of the action. 

\item If $\pi$ is finitely generated with $d$ generators then,
\[ \dim_{\FF_\ell} H^i(\pi, \rho) \le d \cdot r \]
since a cocycle of $\pi$ is determined by its values on the generators. A cocycle is a continuous map $\chi : \pi \to M$ with $\chi(ab) = a \chi(b) + \chi(a)$ which is determined by the generators.

\item Let $\pi$ be finitely presented, assume there is some bound on $\dim_{\FF_\ell} H^i(\pi, \rho)$ is there a nice associated property? 
\end{enumerate}

Obstruction on $\pi$ to be $\pi^\tame(X)$ for $X$ smooth projective over $k = \bar{k}$ in char $p > 0$ which lifts to characteristic zero. Yes: $p'$-finitely presented. 

\begin{conj}
Replace $X$ by a rigid analytic variety, this is still an obstruction.
\end{conj}

Let $X$ be normal over $\CC$ quasi-projective. Then consider $M_B^{\text{irred}}(X, r)(\CC)$ contains a Zariski dense subset of local systems with quasi-unipotent monodromy at $\infty$. 
\bigskip\\
We we make the same definition with $\FF_\ell$ local systesms we can ask the same question: are the quasi-unipotent at $\infty$ dense? 



\end{document}

