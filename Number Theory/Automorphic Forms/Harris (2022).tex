\documentclass[12pt]{article}
\usepackage{import}
\import{../}{NumberTheoryCommands}


\newcommand{\X}{\mathfrak{X}}
\newcommand{\bs}{\backslash}

\begin{document}

\section{Introduction}

\subsection{Lue Pan}

\begin{enumerate}
\item Shimura varieties and Hodge theory (Complex geometry, Lie groups, automorphic forms)
\item $p$-adic geometry
\begin{enumerate}
\item starting with Tate's $p$-divisible groups paper
\end{enumerate}
\item $p$-adic functional analysis
\item representation theory of eveloping algebras.
\end{enumerate}

\subsection{Schedule}

\begin{enumerate}
\item introduction of methods of differential geometry in the study of perfectoid modular curves
\end{enumerate}

\begin{theorem}[Pan, 2022]
Let $E$ be a finite extension of $\Q_p$ and $\rho : G_{\Q} \to \GL_2(E)$ a continuous representation (using the $p$-adic topology). Suppose,
\begin{enumerate}
\item $\Hom{E[G_{\Q}]}{\rho}{\hat{H}^1(K^p, E)} \neq 0$ where $K = K^p K_p \subset \GL_2(\A_f)$ with $K_p \subset \GL_2(\Q_p)$ is compact open 
\item and $G_{\Q_p} \subset G_{\Q}$ is a decomposition group and $\rho|_{G_{\Q_p}}$ is de Rham of Hodge-Tate weights $0,k$ for $k \in \Z^+$.  
\end{enumerate}
Then $\rho$ arises from a (classical) cusp $(k+1)$-form and an eigenvector for the Hecke operators.
\end{theorem}

\begin{rmk}
Arises from a classical cusp form means arises from the following theorem.
\end{rmk}

\begin{thm}[Eichler-Shimura, Deligne]
Let $f$ a cusp form of weight $k+1$ and level $N$, eigenvector for the Hecke operators. Write,
\[ f(z) = \sum_{n \ge 1} a_n e^{2 \pi i n z} \]
with $a_1 = 1$. Then there is a $\rho_f : G_{\Q} \to \GL_2(E)$ with for $q \ndivides N p$ we have $\Tr{\rho_f(\Frob_q)} = a_q$, and $\rho_f |_{G_{\Q_p}}$ is de Rham of Hodge-Tate weights $0,k$. 
\end{thm}

\subsection{Alterior Motives}

\subsubsection{Simpson Correspondence}

The Simpson correspondence. For a complex variety with a local system $\alpha : \pi_1(Z) \to \GL_n(\C)$ then Simpson produced a Higgs field (nonabelian Hodge theory for $n > 1$). There is a $p$-adic simpson correspondence (Faltings, Abber-Gros-Traji) which studies,
\[ \alpha : \pi_1(Z) \to \GL_n(\bar{\Q_p}) \]
and produces $p$-adic Higgs field. However, in the Simpson correspondence we need by holomorphic and antiholomorphic differentiation. Lue Pan develops a $p$-adic version of this. 

\subsubsection{Perfectoid Geometry}

In a perfectoid ring, the map $(-)^p : R \to R$ is surjective (by construction) and therefore, $\d{f} = \d{g^{p^n}} = p^n g^{p^n-1} \d{g} = 0$ so we need to clarify what it means to do differential geometry in a perfectoid setting.
\bigskip\\
Also want to clarify relation between complex and $p$-adic differential operators for $p$-adic $L$-functions.

\subsubsection{$p$-adic representation theory of $p$-adic groups}

complex nonabelian Hodge theory is related to the theory of complex Lie groups. Similarly, we might hope to make progress on $p$-adic groups beyond $\GL_2(\Q_p)$ using nonabelian Hodge theory. 

\subsection{Fontain-Mazur Conjecture}

Let $\rho : G_{\Q} \to \GL_2(E)$ be a continuous representation. Suppose,
\begin{enumerate}
\item $\rho$ is unramified at all but finitely many primes (this is implied by (1) in Pan 2022. 
\item $\rho|_{G_{\Q_p}}$ is de Rham of Hodge-Tate weights $0,k$.
\end{enumerate}
Then $\rho = \rho_f$ for some cuspidal Hecke eigenform of weight $k+1$. 

\begin{rmk}
Emerton proved this special case of Fontain-Mazur (only thinking about dimension $2$ here) with some condition on the irreducibility of the residual representation. 
\end{rmk}

Suppose that $\rho$ is the representation on the Tate module of an elliptic curve $A$,
\[ \rho : G_{\Q} \to \mathrm{Aut} \left( \varprojlim_n A[p^n] \right) \]
Then for $k = 1$ this satisfies the conditions of the theorem and thus we get that $\rho$ is modular which implies Tanayama-Shimura and hence FLT. 

\section{Review of Modular Forms}

\newcommand{\cS}{\mathcal{S}}

Going back 200 years. We consider,
\[ \begin{pmatrix}
a & b
\\
c & d
\end{pmatrix}
\in \SL(2, \R) \]
and $\h = \{ z = x + i y \mid y > 0 \} \subset \CC$ is the upper half plane. Then we have an action,
\[ \begin{pmatrix}
a & b
\\
c & d
\end{pmatrix} \cdot z = \frac{a z + b}{c z + d} \]
Then for $N \ge 3$ the open modular curve $Y(N) = \h / \Gamma(N)$ and $X(N) = Y(N) \cup \{ \text{cusps / points at infinity} \}$ which has a unique algebraic structure. Then $X(N)$ is defined over $\Q(\zeta_N)$ is a moduli space for elliptic curves with torsion structure. Consider $\Omega = \Omega^1_{X(N)}$ is the cotangent bundle. Consider $\Omega(S)$ are the differential $1$-forms with simple poles at $S$ (the set of cusps). There is a natural line bundle $\omega$ on $X(N)$ such that $\omega^{\ot 2} \cong \Omega(S)$.
\bigskip\\
For $k \in \Z$ the space of modular forms $\M_k(\Gamma(N)) = \Gamma(X(N), \omega^{\ot k}) = H^0(X(N), \omega^{\ot k})$. For $k > 0$ there are lots of modular forms. For $k < 0$ there are none. For $k = 0$ there are only constants. Then the cusp forms $\cS_k(\Gamma(N)) \embed \M_k(\Gamma(N))$ are the modular forms vanishing at the cusps. There is a Hecke algebra acting on the cusp forms. An eigenvector is called an eigenform. For $k \ge 2$, Deligne produces from an eigenform a ``modular'' representation $\rho_f$. and for $k = 1$ this is via Deligne-Serre.
\bigskip\\
Consider $\h \subset \CC \embed \P^1(\CC)$ by adding $\infty$ via $z \mapsto [z : 1]$. From the action of $G = \PGL_2(\CC)$ we get,
\[ \gamma \cdot [z : 1] = [az + b : c z + d] = \left[ \frac{az + b}{cz + d} : 1 \right] \]
Notice that the inclusion is invariant under $\SL_2(\RR) \subset \PGL_2(\CC)$. 
Then $\P^1$ has a collection of $G$-equivariant line bundles called $\struct{}(n) = \struct{}(1)^{\ot n}$ for $n \in \ZZ$.    
\[ \iota^* (\struct{}(n)) / \Gamma(N) = \omega^{-n} |_{Y(N)} \]
for $n = \deg{\struct{}(n)}$. 
\bigskip\\
Let $\tau : G \to \GL(V)$ be any algebraic representation. Let $G$ act on $V \times \P^1$ via $g \cdot (v, x) = (\tau(g) v, g \cdot x)$ then the projectionn to $\P^1$ is equivariant. Therefore we can pull back and descent to $Y(N)$ to get $\wt{V}$. Then,
\[ \nabla : V \ot \struct{\P^1} \to V \ot \Omega^1_{\P^1} \]
given by $v \ot \varphi \mapsto v \ot \d{\varphi}$. This is somehow equivariant so we get an integrable connection,
\[ \wt{\nabla} : \wt{V} \to \wt{V} \ot \Omega^1_{Y(N)} \]
By Deligne, we can extend this to $X(N)$ so that $\nabla$ acquires logarithmic poles on the cusps. 
\bigskip\\
Suppose that $\tau$ is integral, meaning $V_{\Z} \subset V$ is a lattice stabilized by the action of $\SL(2, \Z) \subset \GL(2, \Z)$. Then,
\[ \wt{V}_{\Z} = (V_\Z \times \h) / \Gamma(N) \]
Then $\wt{V}_n = \wt{V}_{\Z} / p^n$ for $n \ge 1$. Define,
\[ H^1_{!}(Y(N), \wt{V}_n) = \im{(H^1_c(Y(N), \wt{V}_n) \to H^1(Y(N), \wt{V}_n))} \]
Then, Eichler-Shimura relations say,
\[ H^1_{!}(Y(N), \wt{V}) \ot \C \iso \cS_{k+2}(\Gamma(N)) \oplus \overline{\cS_{k+2}(\Gamma(N))} \]
this is a sort of Hodge decomposition which is equivariant for the Hecke operators. Let,
\[ H^1_{!}(Y(N), \wt{V} \ot \Q_p) = \varprojlim_n H^1_{!}(Y(N), \wt{V}_n) \ot_{\Z_p} \Q_p \]
This is \etale cohomology and therefore is equipped with an action of $G_{\Q}$. Consider, $Y(N p^m)$ for $m \to \infty$. If $N \divides N'$ then get a Galois cover $Y(N') \to Y(N)$. Now we consider the inverse system of $Y(N p^m)$ whose limit has covering group $\SL(2, \Z_p)$. Then the completed cohomology is,
\[ \hat{H}^1_{!}(\Gamma(N), E) = \varinjlim_n \varprojlim_m H^1_{!}(Y(N p^m), \wt{V}_N) \ot_{\Z_p} E \]
First taking the limit in the level and then taking the inverse limit. The order of limits is important. This is $p$-adically complete (a $p$-adic Banach space over $E$) but the other order of limits does not give a $p$-adically complete space. This is equipped with a continuous $\Gal(\Qbar / \Qbar(\zeta_{N p^\infty}) \times \SL(2, \Z_p)$ action. This is $H^1$ of a perfectoid space $\X_N$ and there is a Hodge-Tate map,
\[ \pi_{HT} : \X_N \to \P^1_{\hat{\overline{\Q_p}}} \]
of adic spaces which is ``anti-holomorphic'' relative to $\iota : \h \embed \P^1_{\CC}$. The story is, differential operators on $Y(N p^m)$ pull back to operators on the completed cohomology and same with along $\iota : \h \embed \P^1_{\CC}$. However, these operators only act on the locally analytic elements. 

\section{Sep. 8}

\subsection{Benefits of registering for this course}

\begin{enumerate}
\item NO DRAWBACKS (homework, exams, etc)

\item Sends a message to the administration

\item You get messages from me (mixed perhaps): will be away Sep. 29th
\end{enumerate}

\subsection{Shimura Varieties as Moduli Spaces for Hodge Structures (Complex Theory)}

Modular curve - parameter space for elliptic curves,
\[ E(\CC) = \CC / \Lambda \quad \Lambda \subset \CC \text{ is a lattice} \]
Any smooth projective curve of genus $1$ over $\CC$ is isomorphic to $\CC / \Lambda$ for some $\Lambda$. Furthermore, if $\alpha \in \CC^\times$ then $\alpha : \CC \to \CC$ takes $\Lambda \mapsto \alpha \Lambda$ so $\CC / \Lambda \cong \CC / \alpha \Lambda$ but these are the only relations. 
\bigskip\\
Orient $\CC$ such that $\{ 1, i \}$ is a positive $\RR$-basis. Let,
\[ \Omega = \{ (\omega, \omega') \in \CC^2 \mid \{ \omega, \omega' \} \text{ is an oriented } \RR\text{-basis} \} \]
Then $\SL_2(\Z) \acts \Omega$ via,
\[ \begin{pmatrix}
a & b
\\
c & d
\end{pmatrix}
\cdot (\omega, \omega') = (a \omega + b \omega', c \omega + d \omega') \]
fixing the lattice $\Z \omega \oplus \Z \omega'$. Thus $\Omega / \SL_2(\Z)$ is the set of lattices in $\CC$. Therefore, the set of complex elliptic curves is,
\[ \SL_2(\Z) \backslash \Omega / \CC^\times \]
We are assuming $\tau = \frac{\omega'}{\omega} \in \h$ and therefore, we can write $(\omega, \omega') \sim (1, \tau)$. Furthermore, we can write $\omega = a i + b$ and $\omega' = c i + d$ and,
\[ \begin{pmatrix}
a & b 
\\
c & d
\end{pmatrix}
\in \GL_2(\RR)_+  \]
so we see that $\Omega \cong \GL_2(\RR)_+$. Then $\GL_2(\RR) \acts \h$ and the stabilizer of $\tau = i$ is $\CC^\times$. Therefore, we get,
\[ \SL_2(\Z) \backslash \Omega / \CC^\times \cong \SL_2(\Z) \backslash \h \]
An alternative way to think about this is,
\[ \Z^2 \embed \Q^2 \embed \RR^2 \]
with a varying complex structure on $\RR^2$.

\begin{defn}
A complex structure on $\RR^2$ is a homomorphism,
\[ h : \CC^\times \to \GL_2(\RR) \]
such that the eigenvalues of $h(z)$ are $z$ and $\bar{z}$. Such an $h$ extends to a homomorphism of $\RR$-algebras,
\[ \CC \to \End{\RR^2} \]
Then $h$ defnes an isomorphism $\iota_h : \RR^2 \to \C$. 
\end{defn}

\begin{rmk}
Then $\CC / \iota_h(\Z^2)$ is an elliptic curve and this gives everything from varying the complex structure rather than the lattice. 
\end{rmk}

For all $z \in \CC \sm \RR$ the map $h(z)$ has two distinct eigenvalues $z, \bar{z}$. Let $V^{-1,0}_h$ and $V^{0,-1}_h$ in $V \ot_{\RR} \CC$ be the eigenspaces. Then,
\[ V \ot_{\Q} \CC \cong V^{-1,0} \oplus V^{0, -1} \]
and $\overline{V^{-1,0}} = V^{0,-1}$.

\begin{rmk}
The standard complex structure is given by,
\[ h_0 : \CC^\times \to \GL_2(\RR) \quad h_0(x+iy) = 
\begin{pmatrix}
x & y
\\
-y & x
\end{pmatrix} \]
In this case, $V^{-1,0} = \C \cdot (1,i)$ and $V^{0,-1} = \C (1, -i)$.
\end{rmk}

It is better to consider $\h^{\pm} = \h \cup \bar{\h} = \CC \sm \RR$ and $\GL_2(\RR) \acts \h^{\pm}$ by,
\[ \begin{pmatrix}
a & b
\\
c & d
\end{pmatrix}
\cdot z = \frac{az + b}{cz + d} \]
Choose $\gamma$ such that, $\gamma^{-1} h(i) \gamma = h_0(i)$ ger a map,
\[ \pi : \{ \text{complex structures} \} \to \h^{\pm} \quad \pi(h) = \tau_h = \gamma(i) \]
Check that if $\gamma'$ is another choice that $k = (\gamma')^{-1} \gamma$ centralizes $h_0$ and thus belongs to $h_0(\CC)^\times$. Then $\tau_h$ is independent of the choice of $\gamma$ such that $\gamma^{-1} h(i) \gamma = h_0(i)$.
\bigskip\\
The upshot,
\[ \{ \text{complex structures} \} \cong \GL_2(\RR) / K_\infty \cong \h^\pm \]
where
\[ K_\infty = k_0(\CC^\times) \subset \GL_2(\RR) \]
Therefore, these sets acquire the structure of a complex manifold. Also $V_h^{0,-1} \subset \CC^2$ is a varible line defining a point $p_h \in \P^1(\CC) = \P(V_\CC)$. If $[\alpha : \beta]$ is a homogeneous coordinate on $\P^1$ and $t = \alpha / \beta$ the inhomogeneous coordinate,
\[ \GL_2(\RR) / K_\infty \embed \P(V_{\CC}) \]
Furthermore, $\iota_h : \RR^2 \to \CC$ extends by linearity to $V_{\CC} = \RR^2 \ot_{\RR} \CC \to \CC$. Then we get an isomorphism $V_{\CC} / V_h^{0,-1} \cong \CC$. Thus the elliptic curve parametrized by $h$ becomes,
\[ E_h(\CC) = \iota_h(\Z^2) \backslash \CC \cong \Z^2 \backslash \Z^2 \ot_{\Z} \CC / V_h^{0, -1} \]
However, 
\[ (\Z^2 \times \SL_2(\Z)) \backslash \h^{\pm} \times V_{\CC} / V^{0,-1} \]
is not a family of elliptic curves because of the fixed points. The fibers are $E_h / \Stab{h}$. Therefore, we replace $\SL_2(\Z)$ by $\Gamma(N)$ for $N \ge 3$ to get trivial stabilizers. 
\bigskip\\
Then $Y_N = \Gamma(N) \bs \h^+$ and $G = \GL(2)$ let $\wt{V}_\Z = \Gamma(N) \sm \h^+ \times \Z^2$ where $\Gamma(N) \subset \SL_2(\Z) \acts \Z^2$ by the standard action. Then $\Gamma(N)$ acts trivially on $\Z^2 / N \Z^2$ and let $\wt{V}[N] = \wt{V}_{\Z} / N \wt{V}_{\Z}$ is a trivial $\Z / N \Z$-local system of rank $2$. We see that,
\[ \wt{V}_{\Z} |_{E_h} \cong H_1(E_h, \Z) \]
then we see, 
\[ \wt{V}[N]|_{E_h} \cong E_h[N] \]
is trivialized over $Y_N$ so $Y_N(\CC) = \{ (E, \alpha) \mid \alpha : (\Z / N \Z)^2 \iso E[N] \text{ plus condition of the Weil pairing} \}$ level $N$ structure. For $E / \CC$ and elliptic cuve. There exists $Y^*_N$ defined over $\Q$ represending the Moduli problem,
\[ Y^*_N(F) = \{ (E, \alpha) \mid \alpha : (\Z / N \Z)^2 \iso E[N] \} \]
for $F \supset \Q$. Furthermore, $Y_N^*(\CC)$ is $(\Z / N \Z)^\times$ copies of $Y_N(\CC)$ permuted by the Galois action since the Weil-paring requires the choice of a root of unity $\zeta_N$. Therefore, $Y_N$ is algebraic and defined over $\Q(\zeta_N)$ but not over $\Q$. 

\subsection{Some Algebraic Geometry}

Let $f : Z \to X$ be a proper smooth morphis of $\Q$-varieties of relative dimension $d$. Then $\wt{H}^d = R^d f_* \Z$ is a local system with $(R^d f_* \Z)_* \cong H^d(Z_x, \Z)$. In a neighborhood of $x$, this is trivial. If we suppose this is free of rank $m$ then $\wt{H}^d / N \wt{H}^d$ is a free $\Z / N \Z$-modules of rank $N$. This fives a representation of $\pi_1(X(\CC), x_0)$ on $(\Z / N \Z)^m$. If $K \subset \pi_1(X, x_0)$ is the kernel, then there is a covering space $X_N / X(\CC)$ with group $\pi_1(X, x_0) / K$. Then by Riemann existence $X_N$ is a variety. 
\bigskip\\
Let $N_0 \ge 3$. Then for any $N$ we have, $Y_{N_0 N} \to Y_{N_0}$ trivializes $\wt{V}[N N_0]$. 
\bigskip\\
Define,
\[ \hat{H}^i(N_0, \Z_p) = \varprojlim_m \varinjlim_n H^i(X_{N_0 p^n}, \Z / p^m \Z) \]
completed cohomology. 

\subsection{Hodge Structres}

\begin{defn}
Let $V / \Q$ be a finite dimensional vector space. Let $V_{\CC} = V \ot_{\Q} \CC$. A \textit{Hodge structure} $V$ pure of weight $w$ is a decomposition,
\[ V_{\CC} = \bigoplus_{p + q = w} V^{p,q} \]
such that $\overline{V^{p,q}} = V^{q,p}$. 
\end{defn}

\begin{rmk}
We have seen that elliptic curves over $\CC$ are in correspondence with weight $-1$-Hodge structures. 
\end{rmk}

\begin{defn}
A morphism of Hodge structures $g : V \to V'$ is a $\Q$-linear map such that $g_{\CC}(V^{p,q}) \subset V'^{p,q}$. Then we get a category of Hodge structures. If we allow for mixed weights we also get direct sums. 
\end{defn}

\begin{prop}
There is an equivalence of categories,
\[ \{ (V, h : \CC^\times \to \Aut{V_{\RR}}) \cong \{ \text{Hodge structures} \} \]
where $h$ is a homomorphism of $\RR$-linear algebraic groups (in particular all the eigenvalues of $h(z)$ are of the form $z^p \bar{z}^q$). 
\end{prop}

\section{Hodge Stuctures}

If $G$ is an algebraic group then $\Rep_F(G)$ of finite dimensional $G$-representations forms a catgory where $F$ is a field. In fact, this is a symmetric monoidal category:

\begin{enumerate}
\item additive: direct sums exist
\item $F$-linear it is enriched over $F$-vectorspaces
\item there is a neutral object $\rho : G \to \Aut{V}$ trivial
\item there is a monoidal functor $\otimes$ such that the neutral element is an identity
\item symmetric $V \otimes W \cong W \otimes V$. 
\end{enumerate}

Hodge structures correspond to $\Rep_{\RR}(\S)$ along with $\Q$-structure.
\bigskip\\
We saw that $\h^{\pm}$ is a parameter space for Hodge structure with $V_{\CC} \cong V^{-1,0} \oplus V^{0,1}$ or equivalently for Hodge structures on $\GL_2$-representations. 

\begin{theorem}
Let $\rho_0 : G \to \GL(V)$ is a faithful representation of a reductive group $G$ (over characteristic zero) then any representation $\rho : G \to \GL(W)$ is a direct summand of $V^{\ot a} \ot (V^*)^{\ot b}$ for some $a$ and $b$. 
\end{theorem}

\begin{rmk}
Applying this to $V = \Q^2$ with natural $\GL_2$-structure. Then $W \subset V^{\ot a} \ot (V^*)^{\ot b}$. Then a Hodge structure on $V$ determines on $V^*$ and thus on $V^{\ot a} \ot (V^*)^{\ot b}$. Then clearly, for any $h : \S \to \GL_2$ the direct summand as $\GL_2$-representations is also stable under the $\S$-action and hence defines a Hodge structure (really we also need to worry about the $\Q$-structure).  
\end{rmk}

\begin{defn}
A Hodge structure on a $G$-representation $\rho : G \to \GL(V)$ is a map $h : \S \to G$ which then endows $V$ with a Hodge structure. 
\end{defn}

\subsection{Real Algebraic Groups}

\begin{defn}
Let $\g$ be a Lie algebra. Then for $\ad : \g \to \End{\g}$ we have,
\[ \ad([X,Y]) = \ad(X) \cdot \ad(Y) - \ad(Y) \circ \ad(X) \]
Then the Killing form is,
\[ B(X,Y) = \Tr{(\ad(X) \cdot \ad(Y))} \]
which is a symmetric bilinear form invariant under $\Aut{\g}$.
\end{defn}

\begin{defn}
A complex Lie algebra $\g$ is \textit{semisimple} if the Killing form
is nondegenerate. 
\end{defn}

\begin{defn}
A \textit{real form} of a complex Lie algebra $\g$ is a Lie algebra $\g_0$ over $\RR$ such that $\g \cong \g_0 \ot_{\RR} \CC$.
\end{defn}

\begin{rmk}
A real form is equivalent to a $\CC$-antilinear involution $\sigma : \g \to \g$ in which case,
\[ \g_0 = \g^{\sigma} = \{ X \in \g \mid \sigma(X) = X \} \]
\end{rmk}

\newcommand{\Lie}{\mathrm{Lie}}
\newcommand{\inner}[2]{\left< #1 , #2 \right>}

\begin{theorem}
Lie subgroups of $G$ correspond to Lie subalgebras of $\g = \Lie(G)$.
\end{theorem}

\begin{defn}
Say the real form $\g_0$ is \textit{compact} if the assocated subgroup of $G$ is compact.
\end{defn}

\begin{rmk}
We can recover an algebraic group $G$ with $\Lie(G) = \g$ as the subgroup $G = \Aut{\g}^\circ / \text{scalars} \subset \GL(\g)$ then $\Lie(G) = \g$. (WHY?)
\end{rmk}

\begin{lemma}
$\g_0$ is compact iff the Killing form $B_\g$ is negative-definite on $\g$. 
\end{lemma}

\begin{proof}
If $B_\g$ is negative-definite, then $G_0$ the adjoint group of $\g_0$ maps faithfully to $\Aut(\g_0)$ and preserves $B_\g$ hence is compact (since its image lies inside $O(\g, B_\q)$ which is compact. Conversely, if $G_0$ is compact then $\g_0$ admits a $G_0$-invariant (positive-definite) inner produce $\inner{-}{-}$ via integrating over an arbitrary non-degenerate inner product on $\g$. Now $\inner{-}{-}$ is $G_0$-invariant which implies that for infinitessimal transformations,
\[ \forall X \in \g_0 : \inner{\ad(X) Y}{Z} = - \inner{Y}{\ad(X) Z} \]
To see this, consider,
\[ \deriv{}{t} \inner{ \exp(\ad(X)(Y))}{\exp(\ad(X)(Z))} = 0 \]
The nonzero eigenvalues of a real skew-symmetric matrix are pure imaginary. Then $\ad(X)^2$ is smmetric with eigenvalues $\le 0$ so $B_{\q}(X,X) = \Tr(\ad(X)^2) \le 0$ and is zero iff all eigenvalues are zero and thus $\ad(X) = 0$ so it is negative-definite. 
\end{proof}

\begin{theorem}
Let $\g$ be a complex semisimple $\h \subset \g$ a Cartan subalgebra. Then there is a compact real form $\g_u$ such that $\h$ is stable under the involution. Then there is a real vector space $\h_{\RR}$ which is a real form of $\h$ (just a real basis since $\h$ is abelian) and $i \h_{\RR}$ is a maximal abelian subalgebra of $\g_u$. In fact, $\h_{\RR}$ is spanned by the coroot vectors $H_\alpha$ for $\alpha \in \Delta(\g, \h)$ with $\alpha(H) = B_\g(H_\alpha, H)$. 
\end{theorem}

\begin{theorem}
$\g, \g_u$ as above and $\tau$ is the conjugation associated to $(\g, \g_u)$. Let $\sigma$ be a second conjugation of $\g$. There is a $1$-parameter group $t \mapsto A_t \subset \Aut{\g}$ such that $A(0) = \id$ and $A(1) \tau A(1)^{-1}$ commutes with $\sigma$. 
\end{theorem}

\begin{cor}
Any two compact real forms of $\g$ are conjugate by an element of $\Aut(\g)^\circ$. 
\end{cor}

\begin{rmk}
Let $\h = \Lie(H)$ then,
\[ \Delta \subset X^*(H) = \Hom{}{H}{\Gm} \to \Hom{}{\h}{\CC} \cong X^*(H) \ot_{\ZZ} \CC \]
\end{rmk}

\begin{defn}
Let $\g$ be a real semisimple Lie algebra. A \textit{Cartan involution} is an involution $\sigma : \g \to \g$ such that $\mathfrak{k} = \g^{\sigma = 1}$ and $\mathfrak{\p} = \g^{\sigma = -1}$ has $B_{\g}$ negative-definite and positive-definite respectively and gives a Cartan decomposition,
\[ \g \cong \mathfrak{k} \oplus \mathfrak{p} \]
\end{defn}

\begin{theorem}
Let $\g$ be real semisimple. Then $\g$ has a Cartan involution and any two are conjugate by $\Aut{\g}^\circ$. 
\end{theorem}

\begin{proof}
Let $\g_u$ be a compact real form of $\g_{\CC}$ and $\g = \g_{\CC}^\sigma$ and $\g_u = \g_{\CC}^{\tau}$ where $\sigma$ and $\tau$ are anti-linear involutions. By the theorem, we may assume that $\sigma$ and $\tau$ commute. Then $\tau$ defines an involution of $\g$ since it commutes with $\sigma$. Then the fixed points are $\mathfrak{k} = \g \cap \g_u$ which is maximal compact and $\tau$ is a Cartan involution on $\g$. 
\end{proof}

\subsection{Deligne's Axioms for Shimura Data}

\renewcommand{\k}{\mathfrak{k}}
\renewcommand{\p}{\mathfrak{p}}

\begin{defn}
A Shimura datum is a pair $(G, X)$ where $G$ is a reductive group over $\Q$ and $X$ is a $G_{\RR}$-conjugacy class of homomorphism $h : \S \to G_{\RR}$ satisfying,
\begin{enumerate}
\item $\forall h \in X$ we have $\im{h|_{\mathbb{G}_{m, \RR}}} \subset Z(G)$
\item $\forall h \in X : \Ad(h(i))$ is a Cartan involuton of $\g^{\ad} = \Lie(G^\ad(\RR))$ where $G^\ad = G / Z(G)$. 
\item For any $h \in X : \Ad \circ h : \S \to \GL(\g_{\RR})$ is a Hodge structure such that $\g^{p,q} = 0$ unless $(p, q) \in \{ (0,0), (-1,1), (1, -1) \}$. 
\end{enumerate}
\end{defn}

\begin{rmk}
For $h \in X$ let $K_h = Z_{G_{\RR}}(h(\S))$ then $\k_h = \Lie(K_h) = \g^{0,0}$ is is the part where $\S$ acts trivially through $h$. Furthermore, $\ad(h(i))$ actis as $-1$ pn $\g^{-1,1}$ or $\g^{1,-1}$. 

\begin{enumerate}
\item $\g_{\RR} = \k_h \oplus \p_h$ is a Cartan decomposition
\item $K_h / Z_G$ is maximal compact in $G^\ad(\RR)$
\item $\p_h \cong \p_h^{+} \oplus \p_h^{-}$ with $\p_h^{-} = \g^{1,-1}$ and $\p_h^{+} = \g^{-1,1}$.
\end{enumerate}
Then $X$ has a $G(\RR)$-invariant complex structure. And $G / K$ is a symmetric space with an action of $G$. This has a complex structure invariant under the $G$-action iff $X$ is constructed from a Shimura datum. 
\end{rmk}

\section{September 15}

\newcommand{\GmX}[1]{\mathbb{G}_{m, #1}}

To produce $\CC$-local systems on a connected $X$ the best way is to take a smooth proper morphism $p : Y \to X$ then $\F = R^i p_* \underline{\CC}$ then $\F_x = \cong H^i(Y_x, \CC)$.
\bigskip\\
Let $X = \wt{X} / \Gamma$ where $\wt{X}$ is simply connected. Then $\Gamma \cong \pi_1(X, x)$ for any $x \in X$. If $\rho : \Gamma \to \GL(W)$ for $W$ finite dimensional over $\CC$. Then we can take $\wt{W} = (\wt{X} \times W) / \Gamma$ which is a local system on $X$ whose associated representation is $\rho$. 
\bigskip\\
For $\wt{X}$ a hermitian symmetric space for $G(\RR)^\circ$. Then for $\Gamma \subset G(\Q)$ discrete such that,
\begin{enumerate}
\item $\Gamma \acts \wt{X}$ has no fixed points
\item $\wt{X} / \Gamma$ is compact
\item $\wt{X} / \Gamma$ has finite invariant volume. 
\end{enumerate}
Then we get not only a complex manifold but actually a variety. 
\bigskip\\
Let $G$ be a reductive algebraic group over $\Q$. Let $X$ be a $G(\RR)$-conjugacy class of maps $h : \S \to G_{\RR}$ satisfying,
\begin{enumerate}
\item $\ad(h(i))$ is a Cartan involution of $\g$
\item $h(\GmX{\RR}) \subset Z_G$
\item $\forall z \in \S(\RR)$ then $h(z)$ has eigenvalues $z / \bar{z}, 1, \bar{z}/z$.
\end{enumerate}

\begin{rmk}
Recall that $\S = \Res{\CC}{\RR}{\mathbb{G}_{m,\CC}}$ and thus contains $\mathbb{G}_{m,\CC} \subset \S$. This corresponds to the lattice $\Z^2$ with complex conjugation $\sigma(a,b) = (b,a)$ with the invariant map $\Z^2 \to \Z$ sending $(a,b) \mapsto a + b$.
\end{rmk}

\begin{example}
Let $G = \GmX{\RR}$ and $h(z) = z \bar{z} = N(z)$. 
\end{example}

\newcommand{\GU}{\mathrm{GU}}
\newcommand{\fin}{\mathrm{fin}}
\newcommand{\st}{\mathrm{st}}

\begin{defn}
$\GU(p,q) = \{ g \in \GL_{p+q}(\CC) \mid g^\dagger I_{p,q} g = \nu(g) I_{p,q} \}$ where $I_{p,q}$ is the matrix for the standard quadratic form with signature $(p,q)$. Then there is an exact sequence,
\begin{center}
\begin{tikzcd}
1 \arrow[r] & \SO(p,q) \arrow[r] & \GU(p,q) \arrow[r] & \Gm \arrow[r] & 1
\end{tikzcd}
\end{center} 
Then we let,
\[ h_0(z) = 
\begin{pmatrix}
z I_p & 0
\\
0 & \bar{z} I_q
\end{pmatrix} \]
such that $h_0(i) = i I_{p,q}$. Then $U(p,q) = \ker{\nu}$. 
\end{defn}

\section{Sept 20}


Let $G$ be a group over $\struct{}$ and $X$ a $G(\RR)$-conjugacy class of $h : \S \to \G_{\RR}$ with the required properties. Consider $K \subset G(\A_{\fin})$ then,
\[ \Sh_K(G,X) = G(\Q) \bs (X \times \G(\A_{\fin}) / K \]
For $K' \subset K$ get $\Sh_{K'}(G, X) \to \Sh_K(G, X)$. If $K'$ is normal in $K$ then,
\[ \Sh_{K'}(G, X) \to \Sh_{K}(G, X) \]
is a $/K'$-covering. Write $K = K^[ \cdot K_p$ with,
\[ K^p \subset \prod_{q \neq p}' G(\struct{q}) \]
Fix $K_p$ let $K_{p,n} \to \{ 1 \}$ then consider,
\[ S(K^p) = \varprojlim_n \Sh_{K^{p} K_{p,n}}(G, X) \]
Note that for affine schemes,
\[ \varprojlim_N \Spec{R_n} = \Spec{\varinjlim_n R_n} \]
Then $S(K^p) \to \Sh_{K^p K_{p,n}}(G, X)$ is a pro-\etale cover with group $K_{p, n}$. We cam also from the adic space $\hat{S}(K^p)$.

\begin{defn}
A map of Shimura data $(H, Y) \to (G, X)$ is a homomorphism $\phi : H \to G$ and a map of complex analytic spaces $Y \to X$ which is equivariant for the $H_{\RR}$-action.
\end{defn}

\begin{rmk}
This gives rise to a map of Shimura varities,
\[ \Sh_{K \cap H(\A_{\fin})}(H, Y) \to \Sh_K(G, X) \]
of quasi-projective algebraic varities over $E(H, Y) \supset E(G, X)$.
\end{rmk}

The complex structure is given by $X \embed \hat{X} = G(\CC) / P(\CC)$ the Borel embedding where for $h \in X$ we have $P = P_h$ is the subgroup with $\Lie(P_h) = \k_{h, \CC} \oplus \p_h^{-}$ with $\p_h^{-} = \g^{1,-1}$. 

\begin{rmk}
Recall any algebraic map $h : \S \to \GL(W)_{\RR}$ with $W$ defined over $\Q$ gives a Hodge structure on $W$. 
\end{rmk}

Let $\hat{X}$ be a $G$-homogeneous space (over a number field $E$). Consider $G$-equivariant vector bundles on $\hat{G}$ meaning $V \to \hat{X}$ which is locally isomorphic to $U \times \Ga^n$ for an open $U \subset \hat{X}$ and the transition maps are algebraic. If the transition functions lie in $\GL(n, E)$ then it is a local system. Furthermore, we want it to be equivariant meaning it is equipped with a fiberwise linear action $G \acts V$ such that $\pi : V \to \hat{X}$ is equivariant. 

\begin{center}
\begin{tikzcd}
G \times V \arrow[r] \arrow[d] & V \arrow[d]
\\
G \times \hat{X} \arrow[r] & \hat{X}
\end{tikzcd}
\end{center}

Suppose that $\hat{X} = G / P$ (in full generality $\hat{X}$ might not have $E$-points and hence the parabolic might lie over some extension).
Let $x$ be a fixed point of $P \acts G$ meaning $x \in \hat{X}$. Then we get $V \mapsto V_x$ with $P \acts V_x$. This defines an equivalence of $\ot$-categories,
\[ \{ \text{equivariant vector bundles on } \hat{X} \} \iso \Rep_{E(x)}(P) \]
To go the other way, given a representation $\rho : P \to \GL(W)$ then we let,
\[ V = (G \times V) / P \to G / P = \hat{X} \quad \text{ with } (g, v) \cdot p = (gp, \rho(p)^{-1} v) \]
Consider the restriction map, $\Rep(G) \to \Rep(P)$ then given $\rho : G \to \GL(W)$ we get,
\[ (G \times W) / P \cong (G/P) \times W \]
If we have $\rho  : K = P / R_n(P) \to \GL(W)$ then $\rho$ is completely reducible and thus,
\[ V(\rho) = (G \times W)/P \]
is completely reducible. 
\bigskip\\
Let $V$ be a homogeneous vector bundle on $\hat{X}$. For $\beta : X \embed \hat{X}$ consider,
\begin{center}
\begin{tikzcd}
[V] \arrow[r] \arrow[d] & G(\Q) \bs \beta^*(V) \times G(\A_{\fin}) / K \arrow[d]
\\
\Sh_{K}(G,X) \arrow[r, equals] & G(\Q) \bs (X \times G(\A_{\fin}) / K 
\end{tikzcd}
\end{center}

Therefore we get a functor $V \mapsto [V]_K$ from homogeneous vector bundles on $\hat{X}$ to ``automorphic'' vector bundle son $\Sh_K(G, X)$. 

\begin{rmk}
There is often an extra axiom: $\rank_{\RR} Z_G = \rank_{\QQ} Z_G$ to exclude the following example: $G = \Res{F}{\Q}{\Gm}$ for $F$ totally real. Since $G$ is commutative then $X = \{ h \}$ there is trivial conjugation. Then,
\[ F^\times \bs (h \times V \times \A_{\fin, F}^\times / K \]
will have some fixed points. Equivalently we can consider vectorbundles over $\hat{X}$ with trivial $Z_G^0 = \ker{\Nm}$ action which ensures that the above quotient is actually a vector bundle.
\end{rmk}

\begin{defn}
$\Sh(G, X) = \varprojlim_{K \subset G(\A_{\fin})} \Sh_K(G, X)$ on which $G(\A_{\fin})$ acts. Then we get $\left< [V]_K \right> = \varprojlim_{K} [V}_K$ is equivariant under $G(\A_{\fin})$. 
\end{defn}

For any $h$, then $r \circ h : \S \to \GL(W)$ defines a Hodge structure on $W$ which we write,
\[ W_{\CC} = \bigoplus_{p,q} W_h^{p,q} \]
If $g \in G(\RR)$ then $r \circ g(h)$ gives the Hodge structure,
\[ W_{\GG} = \bigoplus_{p,q} W_{g(h)}^{p,q} \quad \text{with} \quad W_{g(h)}^{p,q} = r(g) \cdot W_h^{p,q} \]

Recall,
\[ W^{p,q} = F^p W \cap \overline{F^q W} \]
But $W$ also has a filtration by $P$-invariant subspaces. Writing,
\[ \g_{\CC} = \g^{-1,1} \oplus \g^{0,0} \oplus \g^{1,-1} \]
Then,
\[ \g^{a,b} \ot W^{p,q} \subset W^{p+a,q+b} \]
WHAT!!





Let $r : G \to \GL(W)$ and,
\[ \wt{W}_K = G(\Q) \bs (X \times W_F \times G(\A_{\fin})) / K \]
is a local system in $F$-vectorspaces where as $[W]_$ is an algebraic vector bundle attached to $\hat{X} \times W$ as homogeneous vector bundles. Then $F^p [W] = [F^p W]$ the Hodge filtration on $[W]$ over the Shimura variety.

\begin{example}
Let $(G, X) = (\GL(2), \h^{\pm})$ and $\hat{X} = \P^1$ and $W = \Sym^k{V}$ where $V = \Q^2$ with standard representation $\rho_{\st}$. Then,
\[ \Sh_{K(N)}(\GL(2), \h^{\pm}) = Y(N) \]
For $N \ge 3$ there is a universal curve $p : \E \to Y(N)$ then $[V] = (R^1 p_* \underline{\Q})^\vee$ or really the flat sections of the relative de Rham vector bundle under the Gauss-Manin connection so $[V]_x = H^1_{\dR}(\E_x)^\vee$.
\bigskip\\
For $V / \P^1 = \GL_2 / B$ is isomorphic to a sum of $\struct{}(k)$ for some $k$ and $\Omega = \struct{}(-2)$ so it pulls back to $\Omega$ on the modular curve which is $\omega^{\ot 2}$. Therefore, $[\struct{}(k)] = \omega^{\ot (-k)}$. 
\bigskip\\
For $r : G \to \GL(W)$ and $\hat{X} \times W = W \ot \struct{\hat{X}}$ as algebraic vector bundles. There is a trivial connection on here. Therefore, we get $\nabla : [W] \to [W] \ot \Omega^1_{\Sh(G, X)}$ which is an integrable connection on the Shimura variety and,
\[ \nabla (F^p [W]) \subset F^{p-1} [W] \ot \Omega^1_{\Sh(G, X)} \]
which we check using Lie algebras, another manifestation of Griffiths transversality. Using the de Rham complex associated to $[W]$ we get an equivalence in the derived category,
\[ \wt{W}(\C) \sim [0 \to [W] \to [W] \ot \Omega^1 \to \cdots \to [W] \ot \Omega^d \to 0] \]
for holomorphic vectorspaces. Therefore,
\[ H^i(\Sh, \wt{W}) = \H^i_{\dR}(\Sh, [W]) \]
\end{example} 

\section{Sept 27}

\subsection{Perfectoid Spaces}

Let $C_p = \widehat{\Qbar}_p$ for any characteristic zero field complete with repsecte to the $p$-adic topology and algebraically closed. Let $[K : \Q_p] < \infty$ and $G = \Gal{\Qbar_p / K}$. Let $\rho : G \to \GL{V}$ and $\rho_{C_p} : G \to \GL{V \ot_{\Q_p} C_p}$ the action is diagonal. 
\bigskip\\
Let $K_{00} / K$ be with Galois group $\Z_p$, totally ramified (e.g. $K = \Q_p(\zeta_{p^n})^{\FF_p^\times}$. 

\begin{defn}
A \textit{pseudo-uniformizer} $\varpi$ is an extension of $\Q_p$ is a topologically nilpotent unit meaning,
\[ \lim_{n \to \infty} \varpi^n = 0 \]
\end{defn}

\begin{rmk}
This is called topologically nilpotent because it says that for any open neighborhood $U$ of $0$ there is $\varpi^n \in U$ for sufficiently large $n$.
\end{rmk}

\begin{prop}
For any $L / K_{\infty}$ with integer ring $\struct{L}$ we have,
\[ \Tr_{L/K_{\infty}} \struct{L} \supset \m_{\infty} \]
which is called being almost unramified or almost etale. 
\end{prop}

\begin{proof}
Let $L = L_0 K_{\infty}$ with $L_0 / K$ and $L_h = L_0 K_h$ gives $\Nm (\cD_{L_n/K_N}) \to 9$ as $n \to \infty$. Therefore, it 
\end{proof}

\section{Oct. 4 Sen's Theory}

Recall the situation: $\Q_p \subset K \subset K_{\infty} \subset C$ with $\Gamma = \Gal{K_{\infty} / K}$. 

\begin{cor}
The inflation map,
\[ H^1_{\cont}(\Gamma, \GL(n, \hat{K}_\infty) \to H^1_{\cont}(G, \GL(n, C)) \]
is an isomorphism.
\end{cor}

\begin{rmk}
We are using the fact that $\hat{K}_{\infty} = C^H$ where $H = \Gal(C / K_{\infty})$. 
\end{rmk}

\begin{rmk}
$H^1_{\cont}(H, \GL(n, C)) = 1$ 
\end{rmk}

Now we take $W$ a finite dimensional $C$-vectorspace with $n = \dim_C{W}$. We equip it with the structure of a semi-linear $G$-representaion $g \cdot (\lambda w) = g(\lambda) g(w)$ for $w \in W$ and $g \in G$ and $\lambda \in C$. 

\begin{prop}
Let $\hat{W}_{\infty} = W^H$ is an $n$-dimensional $\hat{K}_{\infty}$ subspace. Then the inclusion $\hat{W}_{\infty \ot_{\hat{K}_{\infty}} \to W$ is an isomorphism. 
\end{prop}

\begin{proof}
This is Galois descent
\end{proof}

\begin{prop}[Decompletion]
The inclusion $H^1(\Ga, \GL(n, K_{\infty})) \to H^1(\Gamma, \GL(n, \hat{K}_{\infty}))$.
\end{prop}

\begin{proof}
The map $g \mapsto U_g$ descends to a cocycle $g \in \Gamma$ maps to $U_g \in \GL(n, \hat{K}_{\infty})$. But it $\gamma$ is a topological generator then,
\[ U_\gamma \in \GL(n, K_{\infty}) = \bigcup_r \GL(n, K_r) \]
Therefore, because this is a topological generator we can choose a uniform $r$ such that,
\[ \forall g \in \Gamma : U_g \in \GL(n, K_r) \]
\end{proof}

\begin{prop}
There is a $K_r$-representation $W_r$ of $\Gamma$ of dimension $n$ such that $W_r \ot_{K_r} \hat{K}_{\infty} \cong \hat{W}_{\infty}$.
\end{prop}

\begin{cor}
Let $W_\infty \subset \hat{W}_{\infty}$ be the sset of all vectors whose $\Gamma$-orbit is contained in a $K$-vector space of finite dimension. Then $W_r \ot_{K_r} K_\infty = W_\infty$ and hence $W_{\infty} \ot_{K_\infty} \hat{K}_\infty = \hat{W}_\infty$. 
\end{cor}

\begin{proof}
Clearly $\dim_{K_{\infty}} W_r \ot_{K_r} K_{\infty} = \dim_{K_r} W_r = n$. Also, $W_{\infty} \supset W_r \ot_{K_r} K_{\infty}$ so $\dim{W_\infty} \ge n$. On the other hand, Sen proves that any element of $\hat{K}_{\infty}$ whose $\Gamma$-orbit is finite is contained in $K_{\infty}$. One writes $D_{\mathrm{Sen}}(W) = W_\infty$ is an $n$-dim v.s. over $K_{\infty}$ attached to the $G$-action on $W$. ...
\end{proof}

\begin{rmk}
Write $\log{\chi}$ for the map $G \to \Gamma_r \iso \Z_p$ for $\Gamma_r = \Gal{K_{\infty} / K_r}$. Then $\gamma_r$ acts linearly on $W_r$. 
\end{rmk}

\begin{defn}
The Sen operator $\phi = \phi_W$ is the $K_r$-linear endomorphism fo $W_r$ whose matrix in the basis $\{ e_1, \dots, e_r \}$ is given by,
\[ \Phi = \frac{\log{U_{\gamma_r}}}{\log{\chi(\gamma_r)}} \]
is independent of the choice of $\gamma_r$. 
\end{defn}

\begin{theorem}
Sen's operator is the unique $K_{\infty}$-linear endomorphism fo $W_{\infyy} = D_{\text{Sen}}(W)$ wuch that for all $w \in W_{\infty}$ there exists $\Gamma_W \subset \Gamma$ open such that $\forall \sigma \in \Gamma_W$,
\[ \sigma(e) = [ \exp(\phi(\log{\chi(r)})) ](w) \]
\end{theorem}

\begin{theorem}[Sen]
Let $V$ be a $\Q_p$-vectorspace and $\rho : \Gal(\Qbar_p / K) \to \GL(V)$ a linear represetnation. Let $W = V \ot C$ and $\phi = \phi_W$ its SEn operator. Suppose that residue field of $K$ is algebraically closed. Then $\Lie(\rho(\Gal(\Qbar_p / K)))$ is the smallest $\Q_p$-rational subspace $S \subset \End[\Q_p]{V}$ such that $\phi \in S \ot C$.
\end{theorem}

\begin{rmk}
We want a geometric version where the residue field may not be perfect.
\end{rmk}


\section{Oct 6}

\begin{defn}
A \textit{strict $p$-ring} is a $p$-adically complete ring $S$, flat over $\Z_p$ such that $S / p S$ is perfect. 
\end{defn}

\begin{prop}
Suppose that $S$ is a strict $p$-ring and $A = S / pS$. Then there is a multiplicative Teichmuller lift $A \to S^\times \{ 0 
}$ given by,
\[ a \mapsto [a] = \varprojlim_n \tilde{a}^{p^n} \]
\end{prop}

\begin{theorem}[Witt]
The functor $S \mapsto S / p S$ from,
\[ \{ \text{strict p-rings} \} \to \{ \text{perfect } \FF_p\text{-algebas} \} \]
is an equivalence of catefories. The inverse map $A \mapsto W(A)$ is called the ring of Witt vectors. Any $w \in W(A)$ has a unique expansion,
\[ w = \sum_{n \ge 0} [a_n(w)] p^n \]
with $a_n(w) \in A$. 
\end{theorem}

\section{Oct. 11}

\begin{rmk}
No class next tuesday.
\end{rmk}

Consider the perfectoid modular curve $X_{\infty}(N)$ over $X(N)$ (with $p \ndivides N$) then we get a vector bundle $\wt{M} = (X_\infty(N) \times M) / G$ where $G = \GL(2, \Z_p)$ is the Galois group of the perfectoid cover. To $\wt{M}$ there is an associated Sen operator. 
\bigskip\\
Working over $C$ a complete algebraically closed field of characteristic zero and $p$-adic normalized such that $| \bullet |_C$ restricts to the standard norm on $\Q_p$. Consider $\struct{C} \subset C$ then $\Spa(C, \struct{C})$ and $X = \Spa(A, A^+)$ is a $1$-dimensional smooth affinoid adic space over $\Spa(C, \struct{C})$. 

\begin{defn}
Profinite inverse limit: let $\mathbb{X}$ be a perfectoid space and $\mathbb{X}_i$ for $i \in I$ a filtered system of noetherian adic spaces over $C$ with maps,
\begin{center}
\begin{tikzcd}
\mathbb{X} \arrow[rd, "\varphi_j"] \arrow[r, "\varphi_i"] & \mathbb{X}_i \arrow[d]
\\
& \mathbb{X}_j
\end{tikzcd}
\end{center}
We say that $\mathbb{X} \sim \varprojlim \mathbb{X}_i$ if,
\begin{enumerate}
\item $|\mathbb{X}| \to \varprojlim | \mathbb{X}_i |$ is a homeomorphism
\item for any $x \in \mathbb{X}$ and $\varphi_i(x) = x_i$ the map on residue fields,
\[ \varinjlim_i k(x_i) \to k(x) \]
has dense image.
\end{enumerate}
\end{defn}

\begin{rmk}
If the limit exists as a perfectoid space then it is unique. However, it does not always exist as a perfectoid space. 
\end{rmk}

\begin{example}
$\A^1(C) \supset D_\alpha$ closed disk of radius $\alpha$. Then consider $A = C \left< X \right>$ is the ring of functions on $D_1$. Furthermore, $\sup_{D_\alpha} f$ is a norm which corresponds to a point. If we write,
\[ f = \sum a_i x^i \]
then $\sup |a_i|$ is another norm. 
\end{example}

\section{Mysterious Functor}

We want a functor relating,
\[ H^n_{\et}(X_{\bar{k}}, \Q_p) \ot K \]
and the deRham cohomology,
\[ H^n_{\dR}(X/K) \]

\begin{theorem}[Fantaine-Messing, and others]
There is a canonical isomorphism,
\[ H^n_{\et}(X_{\bar{k}}, \Q_p) \ot_{\Q_p} B_{\dR} \cong H^n_{\dR}(X/k) \or_k B_{\dR} \]
of Galois modules.
\end{theorem}


\section{Oct 27}

\begin{defn}
A \textit{strict $p$-ring} is a $p$-adically complete ring $S$ flat over $\Z_p$ such that $S / p S$ is perfect. 
\end{defn}

\begin{rmk}
Can replace $\Z_{p}$ by $\Z_{(p)}$. 
\end{rmk}

If $S$ is a strict $p$-ring and $A = S / p S$ then,
\[ A \to S^\times \cup \{ 0 \} \quad a \mapsto [a] = \lim_n \tilde{a}^{p^n} \]
is well-defined and multiplicative where $\tilde{a}$ is an arbitrarily chosen lift. 

\begin{theorem}[Witt]
The map $S \mapsto S / p S$ is an equivalence of categories,
\[ \{ \text{strict} p\text{-rings} \} \to \{ \text{perfect} \FF_p\text{-algebras} \} \]
The inverse map $A \mapsto W(A)$ is the ring of Witt vectors. Any $w \in W(A)$ can be written uniquely as,
\[ w = \sum_{n \ge 0} [a_n] p^n \]
for $a_n \in A$.
\end{theorem}

\begin{example}
For $A = \FF_{p^r}$ with $r \ge 1$ thne $W(A)$ is the ring of integers of $\Q(\zeta_{p^r - 1})$ in that case $[a_n] \in \mu_{p^r-1} \cup \{ 0 \}$ for all $a_n \in \FF_{p^r}$. 
\end{example}

\begin{example}
If $A = \FF_q[x^{\pm 1}_1, \dots, x_n^{\pm 1}]$ then $\wt{A} = \FF_q[x_i^{\pm \frac{1}{p^r}}]_{i,r}$ is a perfect ring.
\end{example}

\begin{rmk}
Can prove Witt's theorem by showing that deformations of a perfect $\FF_p$-algebra to characteristic zero are unique. Then the unique deformation will be $W(A)$. 
\end{rmk}

\subsection{Addition and Multiplication}

Consider an infintie sequence of variables $\alpha_i, \beta_j$ corresponding to $\alpha_i^p$ and $\alpha_j^p$. Let 

\subsection{Universal Property}

Let $S$ be $p$-adically complete, $p$-torsion-free. Consider a map of multiplicative monoids $\varphi : R \to S$ such that the composition $R \to S \to S / p$ is a ring homomorphism. Then $\varphi : R \to S$ factors uniquely through $[] : R \to W(R)$.




\subsection{Perfectoid}

Let $C = \CC_p$ be the complete normed field and $\struct{C}$ the elements of norm $\le 1$. Define $\struct{C}^\flat$ the \textit{tilt} of $\struct{C}$ via,
\[ \struct{C}^\flat = \varprojlim_{x \mapsto x^p} \struct{C} \]
These maps are surjective. In fact, the reduction map,
\[  \struct{C}^\flat = \varprojlim_{x \mapsto x^p} \struct{C} \iso \varprojlim_{x \mapsto x^p} \strut{C}/p \]
where the inverse map takes,
\[ (a_n \mod p) \mapsto b_n = \lim_m a_{n+m}^{p^m} \]

\begin{lemma}
Let $K$ be a perfectoid field, $\struct{K}$ is surjective, $\ker{\struct{K}} = (\zeta)$ is principal generated by an element of the form $(\tilde{\omega}) + p \alpha$ where $\bar{\omega}$ is a pserudo-uniformizer and $\alpha \in W(K^\flat)^\times$. 
\end{lemma}

\begin{example}
$C^\flat \iso \overline{\FF}_p((t^{\frac{1}{p^\infty}}))$. Moreover,
\[ \left( \overline{\FF}_p((t^{\frac{1}{p^\infty}}) \right)^\flat = \overline{\FF}_p((t^{\frac{1}{p^\infty}})) \]
\end{example}


\begin{defn}
Fontaine's ring $B_{\dR}$. Consider $R = C$ then $C^{\flat \circ} = \struct{C}^\flat$  then,
\[ B^+_{\dR} = \widehat{W(\struct{C}^\flat[\tfrac{1}{p}]} \]
completed with respect to $\ker{\struct{C}}$. Then,
\[ B_{\dR}(C) = B_{\dR}^_(C)[\tfrac{1}{\xi}] = \Frac{B^+_{\dR}(C)) \]
\end{defn}

\begin{prop}
\begin{enumerate}
\item The action of $\Gal{\Qbar_p / \Q_p}$ on $C$ extends functorially to $B_{\dR}$.
\item the powers of $\xi$ define an exhaustive filtration
\item As $\Gal{\Qbar_p / \Q_p}$-module we have $(\xi^i)/(\xi^{i+1}) \cong C(i)$ where $C(i)$ is the $i$-th power of the cyclotomic character.
\end{enumerate}
\end{prop}

\begin{rmk}
We can take $\xi = \log{[\epsilon]}$ where,
\[ \epsilon = (1, \zeta_p, \zeta_{p^2}, \dots, \zeta_{p^n}, \dots) \in \struct{C}^\flat \]
and $[\epsilon]$ is the Teichmuller lift and $\log$ is given by the standard power series. 
\end{rmk}

\begin{theorem}[$p$-adic Hodge Theory]
Let $X / K$ be smooth and proper with $K / \Q_p$ finite. Then,
\begin{enumerate}
\item there is a canonical isomorphism $H_{\et}^\bullet(X, \Z_p) \ot_{\Z_p} B_{\dR} \cong H^\bullet_{\dR}(X/K) \ot_K B_{\dR} \]
\item the diagonal Galois-action on the left corresponds to the action on $B_{\dR}$ on the right
\item the filtration on $B_{\dR}$ on the left corresponds to the tensor-product filtration on the right,
\[ \Fil^n(W \ot V) = \sum_{i+j = n} \Fil^i(W) \ot \Fil^j(V) \]
In particular,
\[ H^\bullet_{\et}(X, \Z_p) \ot C \iso \Fil^0(H_{\dR}^\bullet(X, K) \ot_K B_{\dR}) \]
and moreover,
\[ (H^\bullet_{\et}(X, \Z_p) \ot_{\Z} B_{\dR})^{\Gal} \iso H_{\dR}^\bullet(X/K) \]
Furthermore, 
\[ \bigoplus_{i + j = 0} \gr^i H^\bullet_{\dR}(X/K) \ot \gr^j B_{\dR} = \bigoplus_j H^\bullet_{\dR}(X/K) \ot C(-j) \]
Recall that,
\[ \gr^j H^n_{\dR}(X/K) = H^{n-j}(X, \Omega^j_{X/K}) \]
and therefore,
\[ H^n_{\et}(X, \Z_p) \ot_{\Z_p} C \iso \bigoplus_{j} H^{n-j}(X, \Omega^j_{X/K}) \ot C(-j) \]
which is called the Hodge-Tate decomposition. 
\end{enumerate}
\end{theorem}

\newcommand{\LL}{\mathbb{L}}

\begin{theorem}[Scholze, Faltings: primitive comparison theorem]
Let $\LL$ be an $\FF_p$-local system on $X_{\et}$. Then there is an almost isomorphism,
\[ \forall i \ge 0 : (R^i f_{\et x} \LL) \ot \struct{Y}^+ / p = R^i f_{\et x} (\LL \ot \struct{X}^+ / p) \]
In particular, if $Y$ is a a point, this gives,
\[ Hi(X_{\et}, \LL) \ot (\struct{C} / p) = H^i(X_{\et}, \LL \ot \struct{X}^+ / p) \]
\end{theorem}

\begin{rmk}
We can replace $\FF_p$-local systems by $\ZZ / p^n \ZZ$-local systems, by $\ZZ_p$-local systems then invert $p$.
\end{rmk}

\section{Nov 3}

Recall the spectral sequence,
\[ E^{p,q}_2 = H^p(X, \Omega^q_{X/C})(-j) \implies H^{i+j}(X, C) := H^{p+q}(X_{\et}, \Z_p) \ot C \]

Scholze's primitive comparison theorem:
\[ H^\bullet(X_{\et}, \Q_p) \ot_{\Q_p} C = H^{\bullet(X_{\proet}, \widehat{\struct{X}}) \]
The Leray spectral sequence for $\nu : X_{\proet} \to X_{\et}$,
\[ E^{p,q}_2 = H^p(X_{\et}, R^q \nu_* \widehat{\struct{X}}) \implies H^{p+q}(X_{\proet}, \widehat{\struct{X}}) \]
It suffices to prove,
\begin{prop}
For $X / C$ a smooth curve. Then,
\[ \struct{X_{\et}} \cong \nu_* \widehat{\struct{X}} \]
and there is a natural isomorphism,
\[ \Omega_{X_{\et}}^1(-1) \cong R^1 \nu_* \widehat{\struct{X}} \]
which by cup products induces isomorphisms in all degrees,
\[ \Omega_{X_{\et}}^i(-j) \cong R^i \nu_* \widehat{\struct{X}} \]
\end{prop}

\end{document}
