\documentclass[12pt]{article}
\usepackage{import}
\import{./}{NumberTheoryCommands}


\begin{document}

\section{CM Types}

\section{Algebraic de Rham Cohomology}

\newcommand{\HH}{\mathbb{H}}
\newcommand{\cH}{\mathcal{cH}}

Let $X$ be a smooth quasi-projective variety over $K$ a characteristic zero field. Let $\dim{X} = n$. The algebraic de Rham complex is,
\[ \Omega^\bullet_X := [0 \to \struct{X} \to \Omega^1_X \to \Omega^2_X \to \cdots] \]
where $\Omega^p_X := \wedge^p \Omega_{X/K}$ is the sheaf of Kahler differentials and differential is defined as,
\[ \d{(f_0 \, \d{f_1} \wedge \cdots \wedge \d{f_r})} = \d{f_0} \wedge \d{f_1} \wedge \cdots \wedge \d{f_r} \]
Then we define the algebraic de Rham cohomology,
\[ H^i_{\dR}(X/K) = \HH^i(X, \Omega^\bullet_X) \]
There is a filtration on $\Omega_X^{\bullet}$ denoted,
\[ \Omega_X^{\bullet \ge p} := [ 0 \to \cdots \to 0 \to \Omega^p_{X} \to \Omega^{p+1}_X \to \cdots ] \]
The graded parts are,
\[ \Omega_X^{\bullet \ge p} / \Omega_X^{\bullet \ge p + 1} = \Omega_X^p[-p] \]
This gives a filtration on $H_{\dR}^i(X/K)$ by,
\[ F^p H_{\dR}^i(X/K) := \im{(\HH(X, \Omega_X^{\bullet \ge p}) \to \HH(X, \Omega_X^\bullet))} \]
Then we get the Hodge-de Rham spectral sequence,
\[ E^{p,q}_1 = H^q(X, \Omega^p_X) \implies \HH^{p+q}(X, \Omega_X^\bullet) \]
If $X / \CC$ has structure of a complex manifold then we can consider the analogous construction in the analytic topology:
\[ \Omega_{X^\an}^\bullet := [ 0 \to \struct{X^\an} \to \Omega^1_{X^\an} \to \Omega^2_{X^\an}] \]
Then GAGA implies (in the proper case),
\[ H^i_{\dR}(X^\an/\CC) := \HH^i(X, \Omega^{\an}_X) = H^i_{\dR}(X/\CC) \]
Note that we need to use the compatibility of the algebraic and analtyic Hodge-de Rham spectral sequences since the de Rham complex is not a complex of $\struct{X}$-modules (differentials are nonlinear) but we can compare the Hodge cohomology directly from GAGA).
This is true in more generality (FIND CITATION)

\begin{prop}[Grothendieck]
Let $X$ be smooth over $\CC$. Then,
\[ H_{\dR}^i(X/\CC) \cong H^i_{\text{sing}}(X^\an, \CC) \]
\end{prop}

\begin{proof}
In the analytic topology there is a quasi-isomorphism from the Poincare lemma,
\[ \CC[0] \to \Omega_{X^\an}^\bullet \]
And therefore,
\[ H^i(X^\an, \CC) \cong H_{\dR}^i(X^\an/\CC) \]
since this is an isomorphism on hypercohomology.
\end{proof}

Moreover, if $X$ is a compact Kahler manifold then degeneration of Hodge-de Rham spectral sequence comes down to the Hodge decomposition in terms of harmonic forms. Therefore,
\[ H^n(X, \CC) = H_{\dR}^n(X/\CC) \cong \bigoplus_{p+q=n} H^q(X, \Omega^p_{X^\an}) \]
Then $H_{\dR}^n$ is an example of a $\Q$-Hodge structure of pure weight $n$.

\begin{defn}
Let $Q \subset \RR$ be a subring. A $\Q$-Hodge structure of pure weight $k$ is a projective $Q$-module $V$ with bigrading of $V_{\CC} = V \ot_{Q} {\CC}$ meaning,
\[ V_{\CC} = \bigoplus_{p+q = k} H^{p,q} \]
such that,
\[ \ol{H}^{p,q} = H^{q,p} \]
\end{defn}

Let $X / \CC$ an consider $f : X \to Y$ a family of varities over $\CC$ with $f$ smooth proper. Then there are cohomology sheaves,
\[ \cH^q_{\dR}(X/Y) := R^q f_* \Omega^\bullet_{X/Y} \]
finite locally free over $Y$ which is filtered by submodules,
\[ H^p \cH^q = R^q f_* \Omega_{X/Y}^{\bullet \ge p} \]
These are called Hodge bundles and define a variation of Hodge structures. Spectral sequences applied to $\Omega^\bullet_{X/\CC}$ has,
\[ E^{p,q}_1 = \Omega_{Y}^p \ot_{\struct{Y}} \cH_{\dR}^{q}(X/Y) \implies H^{p+q}_{\dR}(X/\CC) \]

\subsection{Gauss-Manin Connection}

$\Omega_{Y/\CC}^p \ot_{\struct{Y}} \cH_{\dR}^q \implies R^{p+q} f_* \Omega_{X}^\bullet$ 
differentials on $E_1$-page give,
\[ \d_1^{p,q} : \Omega^p_{Y} \ot_{\struct{Y}} \cH^q_{\dR} \to \Omega^{p+1}_Y \ot_{\struct{Y}} \cH^q_{\dR} \]
For $p = 0$ this is the Gauss-Manin connection $\nabla$. This is an algebraic definition so if $f : X \to Y$ is efined over $K \subset \CC$ then $\nabla$ is defined over $K$ and $\cH^q_{\dR}$ is defined over $K$.
\bigskip\\
Cycle class map on de Rham or Betti cohomology,
\[ H^i(X^\an, \Q) \ot_\Q \CC \iso H^i_{\dR}(X/K) \ot_K \CC \]
rational structures on both sides not the same (even if $K = \Q$) there are at least factors of $2 \pi i$. Then we define cycle class maps, for a line bundle $\L$ we get,
\[ c_1(\L) \in H^2(X, \Z(1)) \]
Indeed from the exponential sequence,
\begin{center}
\begin{tikzcd}
0 \arrow[r] & \Z(1) \arrow[r] & \struct{X} \arrow[r, "\exp"] & \struct{X}^\times \arrow[r] & 0
\end{tikzcd}
\end{center}
then we get,
\begin{center}
\begin{tikzcd}
H^1(X, \struct{X}) \arrow[r] & \Pic{X} \arrow[r] & H^2(X, \Z(1)) \arrow[r] & H^2(X, \struct{X}) 
\end{tikzcd}
\end{center}
from the long exact sequence. Alternatively, we get a map,
\[ \d{\log} : \struct{X}^\times \to \Omega^1_{X/K} \]
and this gives,
\[ H^1(X, \struct{X}^\times) \to H^1(X, \Omega^1_{X/K}) \]
However we want a class in de Rham cohomology. To do this, notice that $\d{log}$ lands inside closed forms so we get a map of complexes,
\[ \d{\log} : \struct{X}^\times[-1] \to \Omega^\bullet_X \]
and taking $\HH^2(X, -)$ gives,
\[ c_1 : \Pic{X} = H^1(X, \struct{X}^\times) \to H^2_{\dR}(X/K) \]
In general for a vector bundle $\E$ we want to define,
\[ c_p(\E) \in F^p H^{2p}_{\dR}(X/K) \]
and compatibly get a class,
\[ c_p(\E) \in H^{2p}(X^{\an}, \Z(p)) \]
We define these inductively from $c_1$ using,
\[ H^{2p}(\P(\E), \Z(p)) = \bigoplus_{i=0}^{r-1} \xi^i \cdot \pi^* H^{2p - 2i}(X^\an, \Z(p-i)) \]
where $\xi := c_1(\struct{}(-1))$ and then,
\[ \xi^r + \pi^*(c_1) \xi^{r-1} + \cdots + \pi^*(c_p) = 0 \]
where we define $c_r(\E) := c_r$. This definition extends from vector bundle to Grothendieck group of coherent sheaves,
\[ K_0(X) = G_0(X) \]
since $X$ is smooth. Then for some subvariety $Z \subset X$ we get $[\struct{Z}] \in G_0(X) = K_0(X)$. Then we define,
\[ [Z] = \frac{(-1)^{p-1}}{(p-1)!} c_p([\struct{Z}]) \in H^{2p}(X^\an, \Z(p)) \]
where $p$ is the codimension of $Z$.
\end{document}