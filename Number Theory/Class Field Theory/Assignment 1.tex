\documentclass[12pt]{extarticle}
\usepackage[utf8]{inputenc}
\usepackage[english]{babel}
\usepackage[a4paper, total={7in, 9.5in}]{geometry}
 
\usepackage{amsthm, amssymb, amsmath, centernot}
\usepackage{mathtools}
\DeclarePairedDelimiter{\floor}{\lfloor}{\rfloor}

\newcommand{\notimplies}{%
  \mathrel{{\ooalign{\hidewidth$\not\phantom{=}$\hidewidth\cr$\implies$}}}}
 
\renewcommand\qedsymbol{$\square$}
\newcommand{\cont}{$\boxtimes$}
\newcommand{\divides}{\mid}
\newcommand{\ndivides}{\centernot \mid}
\newcommand{\Z}{\mathbb{Z}}
\newcommand{\N}{\mathbb{N}}
\newcommand{\C}{\mathbb{C}}
\newcommand{\Zplus}{\mathbb{Z}^{+}}
\newcommand{\Primes}{\mathbb{P}}
\newcommand{\ball}[2]{B_{#1} \! \left(#2 \right)}
\newcommand{\Q}{\mathbb{Q}}
\newcommand{\R}{\mathbb{R}}
\newcommand{\Rplus}{\mathbb{R}^+}
\newcommand{\invI}[2]{#1^{-1} \left( #2 \right)}
\newcommand{\End}[1]{\text{End}\left( A \right)}
\newcommand{\legsym}[2]{\left(\frac{#1}{#2} \right)}
\renewcommand{\mod}[3]{\: #1 \equiv #2 \: (\mathrm{mod} \: #3) \:}
\newcommand{\nmod}[3]{\: #1 \centernot \equiv #2 \: (\mathrm{mod} \: #3) \:}
\newcommand{\ndiv}{\hspace{-4pt}\not \divides \hspace{2pt}}
\newcommand{\finfield}[1]{\mathbb{F}_{#1}}
\newcommand{\finunits}[1]{\mathbb{F}_{#1}^{\times}}
\newcommand{\ord}[1]{\mathrm{ord}\! \left(#1 \right)}
\newcommand{\quadfield}[1]{\Q \small(\sqrt{#1} \small)}
\newcommand{\vspan}[1]{\mathrm{span}\! \left\{#1 \right\}}
\newcommand{\galgroup}[1]{Gal \small(#1 \small)}
\newcommand{\ints}[1]{\mathcal{O}_{#1}}
\newcommand{\sm}{\! \setminus \!}
\newcommand{\norm}[3]{\mathrm{N}^{#1}_{#2}\left(#3\right)}
\newcommand{\qnorm}[2]{\mathrm{N}^{#1}_{\Q}\left(#2\right)}
\newcommand{\quadint}[3]{#1 + #2 \sqrt{#3}}
\newcommand{\pideal}{\mathfrak{p}}
\newcommand{\inorm}[1]{\mathrm{N}(#1)}
\newcommand{\tr}[1]{\mathrm{Tr} \! \left(#1\right)}
\newcommand{\delt}{\frac{1 + \sqrt{d}}{2}}
\renewcommand{\Im}[1]{\mathrm{Im}(#1)}
\newcommand{\modring}[1]{\Z / #1 \Z}
\newcommand{\modunits}[1]{(\modring{#1})^\times}
\renewcommand{\empty}{\varnothing}
\renewcommand{\d}[1]{\mathrm{d}#1}
\newcommand{\deriv}[2]{\frac{\d{#1}}{\d{#2}}}
\newcommand{\pderiv}[2]{\frac{\partial{#1}}{\partial{#2}}}
\newcommand{\parsq}[2]{\frac{\partial^2{#1}}{\partial{#2}^2}}

\newcommand{\atitle}[1]{\title{% 
	\large \textbf{Mathematics W4043 Algebraic Number Theory
	\\ Assignment \# #1} \vspace{-2ex}}
\author{Benjamin Church \\ \textit{Worked With Matthew Lerner-Brecher} }
\maketitle}

 
\newtheorem{theorem}{Theorem}[section]
\newtheorem{lemma}[theorem]{Lemma}
\newtheorem{proposition}[theorem]{Proposition}
\newtheorem{corollary}[theorem]{Corollary}


\begin{document}
\atitle{1}
 
\begin{enumerate}
\item 
\begin{enumerate}
\item Let $V$ be a finite-dimensional $\Q_p$-vectorspace with a norm $| \bullet |_V$ such that for $v \in V$ and $a \in \Q_p$ we have $|a v|_V = |a|_p |v|_V$. Then, let $v_1, \cdots, v_r$ be a basis of $V$ and define the supremum norm,
\[ \left|\left| \sum_{i = 1}^r c_i v_i \right|\right| = \sup\limits_{1 \le i \le r} |c_i|_p \]
This norm agrees with $|\bullet|_V$ (up to the scalar $|v_i|_V$) on the subspace spanned by $v_i$ which is a copy of $\Q_p$ inside $V$. Therefore, the two norms are equivalent on $V$ i.e. there exist nonzero constants $C$ and $C'$ such that,
\[ C||v|| \le |v|_V \le C' ||v|| \]
We must show that $V$ is complete with respect to $|\bullet|_V$. Given a Cauchy sequence $\{u^{(i)} \}$, we write it in components with respect to the basis $\{v_i\}$. That is, from $\{u^{(i)}\}$ we get $n$ sequences of the form $\{u^{(i)}_j\}$ with $u^{(i)}_j \in \Q_p$ such that,
\[ \sum_{j = 1}^r u^{(i)}_j v_j = u^{(i)}\]
Because the sequence is Cauchy, for any $\varepsilon > 0$ there exists some $N$ such that for $n, m > N$ we have,
\[ C |u^{(n)}_j - u^{(m)}_j|_p \le C ||u^{(n)} - u^{(m)}|| \le |u^{(n)} - u^{(m)} |_V < \varepsilon \]
for any $1 \le j \le r$. Therefore, each sequence $\{u^{(i)}_j\}$ is also Cauchy because,
\[ |u^{(n)}_j - u^{(m)}_j|_p < \frac{\varepsilon}{C}\]
By the completeness of $\Q_p$, each of these sequences has a limit, $u_j = \lim\limits_{n \to \infty} u^{(n)}_j$. Define $u = \sum_{i = 1}^r u_i v_i$. Then, for any $\varepsilon > 0$ there exists $N_j$ such that $n > N_j \implies |u^{(n)}_j - u_j|_p < \epsilon$. Therefore, for $n > \sup\limits_{1 \le i \le r} N_i$, we have,  
\[ |u^{(i)} - u|_V \le C' ||u^{(i)} - u|| \le \sup\limits_{1 \le j \le r} |u^{(i)}_j - u_j|_p < \varepsilon\]
so $\lim\limits_{n \to \infty} u^{(n)} = u$. Therefore, $V$ is complete with respect to the norm $|\bullet|_V$. 
\item Let $K$ be a field of characteristic zero and therefore there is an embedding $\iota : \Q \to K$. Futhermore, suppose that $K$ is complete with respect to $|| \bullet||$ such that $||\iota(a)|| = |a|_p$ for any $a \in \Q$. The embedding map extends to $\iota : \Q_p \to K$ by,
\[\iota( \lim \limits_{n \to \infty} a_n) = \lim\limits_{n \to \infty} \iota(a_n) \]
This is defined on all of $\Q_p$ because any element of $\Q_p$ can be written as the limit of a cauchy sequence whose terms are in $\Q$. Also, if 
\[ \lim\limits_{n \to \infty} a_n = \lim\limits_{n \to \infty b_n} \]
for two sequences $a_n, b_n \in \Q$ then,
\[ ||\iota(a_n) - \iota(b_n)|| = || \iota(a_n - b_n) || = |a_n - b_n|_p \to 0\]
so $\iota(\lim\limits_{n \to \infty} a_n) = \iota(\lim\limits_{n \to \infty} b_n)$ which means that $\iota$ is well-defined. 
Furthermore, because the norm on $a_n$ is preserved under $\iota$ which is a homomorphism so $\iota(a_n)$ is also Cauchy ($|a_n - a_m|_p = ||\iota(a_n - a_m)|| = ||\iota(a_n) - \iota(a_m)||$) and $K$ is complete so the limit exists. It suffices to show that $\iota$ is a continuous homomorphism,
\[\iota( \lim \limits_{n \to \infty} a_n + b_n) = \lim\limits_{n \to \infty} \iota(a_n + b_n) = \lim\limits_{n \to \infty} \left[ \iota(a_n) + \iota(b_n) \right] = \lim\limits_{n \to \infty} \iota(a_n) + \lim\limits_{n \to \infty}  \iota(b_n) \]
and 
\[\iota( \lim \limits_{n \to \infty} a_n b_n) = \lim\limits_{n \to \infty} \iota(a_n + b_n) = \lim\limits_{n \to \infty} \left[ \iota(a_n) \iota(b_n) \right] = \left( \lim\limits_{n \to \infty} \iota(a_n) \right) \left( \lim\limits_{n \to \infty}  \iota(b_n) \right) \]
Furthermore, $||\iota(a)|| = |a|_p$ for any $a \in \Q_p$ so if $\lim\limits_{n \to \infty} a_n = a$ for any sequence $a_n \in \Q_p$, then for any $\varepsilon$, there exists $N$ such that for $n > N$,
\[ ||\iota(a_n) - \iota(a)|| = |a_n - a|_p  < \varepsilon\]
so $\lim\limits_{n \to \infty} \iota(a_n) = \iota(a)$ and thus $\iota$ is continuous.  
\end{enumerate}

\item We consider quadratic extensions of $\Q_p$. As proven in class, every finite extension of $\Q_p$ is generated by an element of $\Q$. Thus, we can restrict our study to extensions of the form $\Q_p(\sqrt{d})$ where $d \in \Q$. 
\begin{enumerate}
\item Take $p \neq 2$. Any quadratic extension of $\Q_p$ is of the form, $\Q_p(\sqrt{d})$. First, suppose $\legsym{d}{p} = 1$ then, $d$ is a square modulo $p$ so $x^2 - d$ has a root in $\finfield{p} \cong \Q_p/\Z_p$ and $\mod{(x^2 - d)' = 2d}{0}{p}$ because $p \neq 2$ so by Hensel's lemma $x^2 - d$ has a root in $\Q_p$. Thus, $\Q_p(\sqrt{d}) = \Q_p$. Else, suppose that $\legsym{d}{p} = -1$ then $d$ cannot have a root in $\Q_p$ because otherwise its image in $\Q_p/\Z_p$ would be a root of $x^2 - d$ in $\finfield{p}$ which does not exist. Thus, $\Q_p(\sqrt{d}) \neq \Q_p$. However, the product of two nonresidues is a residue so if $\legsym{d}{p} = \legsym{d'}{p}$ then $\legsym{dd'}{p} = 1$ and thus $dd'$ is a square in $\Q_p$ so $dd' = a^2$ and thus $d' = \frac{a^2}{d}$ so $\Q_p(\sqrt{d}) = \Q_p(\sqrt{d'})$. Therefore, there is exactly one quadratic extension of $\Q_p$ generated by a nonresidue. \bigskip \\
Next, take $d = p$. We know that $p$ is not a square in $\Q_p$ because $p$ is a uniformizer. Thus, $\Q_p(\sqrt{p}) \neq \Q_p$. Furthermore, $p$ is ramified in $\Q_p(\sqrt{p})$ but unramified in $\Q_p(\sqrt{d})$ since $ef = 2$ but $f > 1$ because $k(\Q_p(\sqrt{d}) \supset \finfield{p}[\sqrt{d}] \supsetneq \finfield{p}$ because $d$ is a nonresidue. Therefore, $\Q_p(\sqrt{p}) \neq \Q_p(\sqrt{d}) \neq \Q_p$. \bigskip \\
Finally, consider $p \divides d$ but $p \neq d$. Take $d = p k$. Since $d$ can be choosen to be square-free, we require $p \ndivides k$. Thus, if $k$ is a quadratic residue modulo $p$ then $k$ is a square in $\Q_p$ so $d$ is not squre-free. Therefore, take $k$ to be a nonresidue. In this case, all such $\Q_p(\sqrt{p k})$ are equivalent because any two nonresidues differ by a square in $\Q_p$. Futhermore, suppose that $\Q_p(\sqrt{d}) = \Q_p(\sqrt{p k})$ then $p k d$ is a square but $k$ and $d$ are nonresidues so $k d$ is a square in $\Q_p$. Thus, $p$ must be a square which contradicts the fact that $\Q_p(\sqrt{d}) \neq \Q_p(\sqrt{p})$. Finally, if $\Q_p(\sqrt{p}) = \Q_p(\sqrt{p k})$ then both $p$ and $pk$ are square which implies that $k$ is a square, again contradicting the fact that $\Q_p(\sqrt{d}) \neq \Q_p(\sqrt{p})$. Thus, $\Q_p(\sqrt{pk})$ is distinct from any of the quadratic fields so far constructed. Also, $p \ndivides k$ so $k \in \Q_p^\times$ so $(\sqrt{pk})^2 = (p)$ and thus $p$ is ramified in $\Q_p(\sqrt{pk})$. In particular, $\Q_p(\sqrt{pk}) \neq \Q_p$. Thus, there are exactly $3$ quadratic extensions of $\Q_p$. These are,
\[ \Q_p(\sqrt{d}) \quad \Q_p(\sqrt{p}) \quad \Q_p(\sqrt{pd}) \]
where $d$ is any quadratic nonresidue modulo $p$. 
 
\item The classification of quadratic extensions of $\Q_2$ follows from an important lemma from elementary number theory,
\begin{lemma}
Let $d$ be odd, $d$ is a quadratic residue modulo $2^r$ for every $r \in \Zplus$ if and only if $\mod{d}{1}{8}$. 
\end{lemma}
By the inverse limit construction of $\Z_2$, a number $d \in \Q$ is a square in $\Z_p$ if and only if its reduction to $\Z/2^r\Z$ is a square for every $r$. Therefore, for $2 \ndivides d$, the field $\Q_2(\sqrt{d})$ is a quadratic extension of $\Q_p$ if and only if $\nmod{d}{1}{8}$ such that $x^2 - d$ has no roots (in particular) modulo $8$ and thus no roots in $\Q_2$. Furthermore, for two odd $d,d'$ i.e. $d, d' \in (\Z/8\Z)^\times$, the following are equivalent,
\[\mod{d}{d'}{8} \iff \mod{d^{-1}d'}{1}{8} \iff d^{-1} d' \in (\Q_2)^2 \iff \Q_2(\sqrt{d}) = \Q_2(\sqrt{d'})
\] 
Therefore, the quadratic extensions $\Q_2(\sqrt{d})$ with odd $d$ are in one-to-one correspondence with the nontrivial elements of $(\Z/8\Z)^\times$. Thus, we can choose representatives,
\[ \Q_2(\sqrt{3}) \quad \Q_2(\sqrt{5}) \quad \Q_2(\sqrt{7})\]
for the set of all quadratic extensions of $\Q_2$ by odd $d$. \bigskip\\
Now we consider $\Q_p(\sqrt{d})$ for even $d$. Write $d = 2 k$ with $2 \ndivides k$ because we can take $d$ to be square-free since $\Q_2(\sqrt{a^2 k}) = \Q_2(\sqrt{k})$. Using the same argument as above,  
\[\mod{k}{k'}{8} \iff \mod{k^{-1}k'}{1}{8} \iff k^{-1} k' \in (\Q_2)^2 \iff \Q_2(\sqrt{2k}) = \Q_2(\sqrt{2k'})
\] 
However, $\Q_2(\sqrt{2k}) \neq \Q_p$ because $2 \ndivides k$ so $k$ is a unit in $\Q_2$. Thus, $(\sqrt{2k})^2 = (2k) = (2)$ so $2k$ cannot be a square in $\Q_2$ because $2$ is a uniformizer for the prime ideal $(2)$ in $\Z_2$. 
Therefore, the quadratic extensions $\Q_2(\sqrt{d})$ with even $d$ are in one-to-one correspondence with \textit{all} elements of $(\Z/8\Z)^\times$ with representatives,
\[ \Q_2(\sqrt{2}) \quad \Q_2(\sqrt{6}) \quad \Q_2(\sqrt{10}) \quad \Q_2(\sqrt{14}) \] 
However, if $\sqrt{d} = a + b \sqrt{2k}$ then $d = a^2 + 2k b^2 + 2 a b \sqrt{2k}$ so $a = 0$ or $b = 0$ which either way implies that $d$ is a square or $d$ is even. Thus, the quadratic extensions derived from the two different case must be distinct. Thus, in total, there are $7$ distinct quadratic extensions of $\Q_2$. 
\end{enumerate}

\item Let $L / K$ be an extension of p-adic fields and let $\Pi$ be a uniformizer of $L$ with $\mathfrak{m}_L = (\Pi)$ and $\pi$ a uniformizer of $K$ with $\mathfrak{m}_K = (\pi)$ such that $\mathfrak{m}_K \ints{L} = (\Pi)^e $.
\begin{enumerate}
\item 
Define the groups,
\[ I_i = \{ g \in \galgroup{L/K} \mid \mod{g(a)}{a}{\mathfrak{m}_L^{i + 1}} \} \]
take $g \in I_0 = I$ then we know that $g$ permutes prime ideals. However, $\ints{L}$ is a DVR so it has a unique prime ideal. Thus, $g(\mathfrak{m}_L) = \mathfrak{m}_L$ so $(g(\Pi)) = (\Pi)$ which implies that $g(\Pi) = u \Pi$ where $u$ is a unit. Let $\alpha(g)$ be the image of $u$ in $k(L)^\times = \left(\ints{L}/\mathfrak{m}_L\right)^\times$ which is nonzero because $u$ is invertable in $k(L)$. Thus, $\mod{u}{\alpha(g)}{\mathfrak{m}_L}$ so,
\[\mod{g(\Pi)}{\alpha(g) \Pi} {\uniform{L}^2} \]
Then, Furthermore, let $g, h \in I$ then consider,
\[ \mod{g \circ h(\Pi)}{\alpha(g \circ h) \Pi}{\uniform{L}^2} \]
However,
\[ \mod{g(\Pi)}{\alpha(g) \Pi}{\uniform{L}^2} \quad \quad \mod{h(\Pi)}{\alpha(h) \Pi}{\uniform{L}^2}\]
Therefore,
\[ (g \circ h)(\Pi) \equiv g( \alpha(h) \Pi) = g(\alpha(h)) g(\Pi) \equiv \alpha(g) g(\alpha(h)) \Pi \equiv \alpha(g) \alpha(h) \Pi \modulo{\uniform{L}^2} \]
where I have used the fact that $g \in I$ so $\mod{g(\alpha(h))}{\alpha(h)}{\uniform{L}}$. Therefore, the map, $\alpha : I \to k(L)^\times$ is a homomorphism. \bigskip \\
Now, we will show that $\ker{\alpha} = I_1$. First, if $g \in I_1$ then by definition, $\mod{g(\Pi)}{\Pi}{\uniform{L}^2}$ so $g \in \ker{\alpha}$. Conversely, suppose that $g \in \ker{\alpha}$ then, $g \in I$ and $\mod{g(\Pi)}{\Pi}{\uniform{L}^2}$. Any $a \in \ints{K}$ can be written as $a = \sum\limits_{i \ge 0} a_i \Pi^i$ with $a_i \in k(L)$. Thus,
\[ g(a) = \sum\limits_{i \ge 0} g(a_i) g(\Pi)^i \equiv \sum\limits_{i \ge 0} a_i \Pi^i = a \modulo{\uniform{L}^2} \]
thus, $g \in I_1$ so $\ker{\alpha} = I_1$. Since the map $\alpha : I \to k(L)^\times$ is a homomorphism, the map decends to an injective homomorphism, $\alpha : I/I_1 \to k(L)^\times$. In particular, $I/I_1$ is isomorphic to a subgroup of $k(L)^\times$ which has order $q - 1 = p^f - 1$. Therefore, by Lagrange's theorem, $I/I_1$ has order dividing $p^f - 1$ which must thus be coprime to $p$.
   
\item For $ i > 0$ define the map $\alpha : I_i \to k(L)$ by,
\[ \mod{g(\Pi)}{\Pi + \alpha(g) \Pi^{i + 1}}{\uniform{L}^{i + 2}} \] 
This map is well defined because $g \in I_i$ so $\mod{g(\Pi)}{\Pi}{\uniform{L}^{i + 1}}$ so $g(\Pi) = \Pi + a \Pi^{i + 1}$. For $g, h \in I_i$ consider,
\[(g \circ h)(\Pi) \equiv g(\Pi) + g(\alpha(h)) g(\Pi)^{i + 1} \equiv \Pi + \alpha(g) \Pi^{i + 1} + \alpha(h) \Pi^{i + 1} \modulo{\uniform{L}^{i + 2}} \]
because $g$ fixes $k(L)$ and preserves $\Pi^{i + 1}$ modulo $\Pi^{i + 2}$ since \[g(\Pi)^{i + 1} = (\Pi + \alpha(g) \Pi^{i + 1} + r \Pi^{i + 2})^{i + 1} = \Pi^{i + 1} + (i + 1)\alpha(g) \Pi^{2 i + 2} + \cdots \]
Thus, $\alpha(g \circ h) = \alpha(g) + \alpha(h)$ so $\alpha$ is a homomorphism. Furthermore, if $g \in I_{i + 1}$ then, 
\[ \mod{g(\Pi)}{\Pi}{\uniform{L}^{i+2}} \]
so $g \in \ker{\alpha}$. Conversely, if $g \in \ker{\alpha}$ then $g$ preserves $\Pi$ modulo $\uniform{L}^{i + 2}$ i.e.
\[ \mod{g(\Pi)}{\Pi}{\uniform{L}^{i + 2}} \]
But any element $a \in \ints{L}$ can be written as $\sum\limits_{i \ge 0} a_i \Pi^i$ with $a_i \in k(L)$. But $g$ fixes $k(L)$ so,
\begin{align*}
g(a) = \sum\limits_{i \ge 0} g(a_i) g(\Pi)^i \equiv \sum\limits_{i \ge 0} a_i \Pi^i = a \modulo{\uniform{L}^{i + 2}} 
\end{align*}
Thus, for any $a \in \ints{L}$ we have, $\mod{g(a)}{a}{\uniform{L}^{i + 2}}$ so $g \in I_{i + 1}$. Thus, $\ker{\alpha} = I_{i + 1}$. Because $\alpha : I_i \to k(L)$ is a homomorphism with kernel $I_{i + 1}$, $\alpha$ decends to an injective homomorphism $\alpha : I_i/I_{i + 1} \to k(L)$. Thus, $I_i/I_{i + 1}$ is embedded in the additive group $k(L)$ which has order $p^f$. By Lagrange's theorem, $I_i/I_{i + 1}$ has order $p^k$ for some $k \le f$. \bigskip\\

For any $g \in \galgroup{L/K}$, consider $a = g(\Pi) - \Pi$. If $a$ were a unit then since $(g(\Pi)) = (\Pi)$ we would have $a \in \uniform{L}$ so $\uniform{L} = \ints{L}$ which is a contradiction. Thus, if $a \neq 0$ i.e. $g \neq \id$ then by Dedekind prime factorization, $(a) = \uniform{L}^k$ for some $k$. Thus, $g \in I_{i}$ for $i < k$ but by the uniqueness of prime ideal factorization, $g \notin I_k$. Therefore, because $\galgroup{L/K}$ is finite, the chain of subgroups,
\[ I_0 \supset I_1 \supset I_2 \supset \cdots \]
must terminate at some $I_N = \{ \id_L \}$ since every element has a maximum index at which it appears in the sequence. Furthermore,
\[ |I_1| = \left(\prod_{i = 1}^{N-1} |I_i|/|I_{i + 1}| \right) \cdot |I_N| = \left(\prod_{i = 1}^{N-1} |I_i/I_{i + 1}| \right) \cdot |I_N| = \prod_{i = 1}^{N-1} |I_i/I_{i + 1}|\]
but each factor group is a $p$-group. Thus, $I_1$ is a $p$-group. 

\item Let $K$ be a p-adic field and consider the extension $K(\sqrt[d]{\pi})$ where $q = |k(K)|$ and $d \divides q - 1$. First, I will argue that $x^d - \pi$ splits completely in $K$. The polynomial $x^d - 1$ has a root in $k(K) \cong \finfield{q}$ because $\finfield{q}^\times$ is cyclic so let $g \in k(K)$ have order $q - 1$. Then, since $d \divides q - 1$, take $x = g^{\frac{q - 1}{d}}$ such that $x^d - 1 = 0$ and $x$ is a primitive $d^{\mathrm{th}}$-root of unity.  Because $(x^d - 1)' = d x^{d - 1} \neq 0$ in $k(K)$ then by Hensel's lemma, there exists a root $\omega$ of $x^d - 1$ in $K$ such that $\mod{\omega}{x}{\uniform{L}}$. Thus, $K$ contains $d$, $d^{\mathrm{th}}$-root of unity. Thus, the we can factor the polynomial,
\[ x^d - \pi = (x - \sqrt[d]{\pi})(x - \omega \sqrt[d]{\pi}) \cdots (x - \omega^{d - 1} \sqrt[d]{\pi}) \]
Thus, $K(\sqrt[d]{\pi})$ is the splitting field of $x^d - \pi$ over $K$ and therefore Galois because $K$ has characteristic zero. By the Cyclic Extension theorem, since $K$ contains all $d^{\mathrm{th}}$-roots of unity and $(\sqrt[d]{\pi})^d = \pi \in K$, we have that $K(\sqrt[d]{\pi})/K$ is a cyclic extension and thus abelian. Futhermore, with $L = K(\sqrt[d]{\pi})$, the ideal $\mathfrak{m}_K \ints{L} = \pi \ints{L} = (\sqrt[d]{\pi})^d$. Therefore, the ramification index must be at least $d$. However, $n = [L : K] = d$ because $x^d - \pi$ is irreducible in $K$ else $\pi$ could not be a uniformizer. Thus, $e = d = n$ so $L/K$ is totally ramified. Furthermore, $e = d$ and $d \divides q - 1$ with $q = p^f$ by assumption. Thus, $p \ndivides d$ since $p \ndivides q - 1$. Therefore, $p \ndivides e$ since $e = d$. Therefore, $L/K$ is tamely and totally ramified.  

\end{enumerate}
\end{enumerate}

\end{document}