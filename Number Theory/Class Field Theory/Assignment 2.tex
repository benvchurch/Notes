\documentclass[12pt]{extarticle}
\usepackage[utf8]{inputenc}
\usepackage[english]{babel}
\usepackage[a4paper, total={7in, 9.5in}]{geometry}
 
\usepackage{amsthm, amssymb, amsmath, centernot}
\usepackage{mathtools}
\DeclarePairedDelimiter{\floor}{\lfloor}{\rfloor}

\newcommand{\notimplies}{%
  \mathrel{{\ooalign{\hidewidth$\not\phantom{=}$\hidewidth\cr$\implies$}}}}
 
\renewcommand\qedsymbol{$\square$}
\newcommand{\cont}{$\boxtimes$}
\newcommand{\divides}{\mid}
\newcommand{\ndivides}{\centernot \mid}
\newcommand{\Z}{\mathbb{Z}}
\newcommand{\N}{\mathbb{N}}
\newcommand{\C}{\mathbb{C}}
\newcommand{\Zplus}{\mathbb{Z}^{+}}
\newcommand{\Primes}{\mathbb{P}}
\newcommand{\ball}[2]{B_{#1} \! \left(#2 \right)}
\newcommand{\Q}{\mathbb{Q}}
\newcommand{\R}{\mathbb{R}}
\newcommand{\Rplus}{\mathbb{R}^+}
\newcommand{\invI}[2]{#1^{-1} \left( #2 \right)}
\newcommand{\End}[1]{\text{End}\left( A \right)}
\newcommand{\legsym}[2]{\left(\frac{#1}{#2} \right)}
\renewcommand{\mod}[3]{\: #1 \equiv #2 \: (\mathrm{mod} \: #3) \:}
\newcommand{\nmod}[3]{\: #1 \centernot \equiv #2 \: (\mathrm{mod} \: #3) \:}
\newcommand{\ndiv}{\hspace{-4pt}\not \divides \hspace{2pt}}
\newcommand{\finfield}[1]{\mathbb{F}_{#1}}
\newcommand{\finunits}[1]{\mathbb{F}_{#1}^{\times}}
\newcommand{\ord}[1]{\mathrm{ord}\! \left(#1 \right)}
\newcommand{\quadfield}[1]{\Q \small(\sqrt{#1} \small)}
\newcommand{\vspan}[1]{\mathrm{span}\! \left\{#1 \right\}}
\newcommand{\galgroup}[1]{Gal \small(#1 \small)}
\newcommand{\ints}[1]{\mathcal{O}_{#1}}
\newcommand{\sm}{\! \setminus \!}
\newcommand{\norm}[3]{\mathrm{N}^{#1}_{#2}\left(#3\right)}
\newcommand{\qnorm}[2]{\mathrm{N}^{#1}_{\Q}\left(#2\right)}
\newcommand{\quadint}[3]{#1 + #2 \sqrt{#3}}
\newcommand{\pideal}{\mathfrak{p}}
\newcommand{\inorm}[1]{\mathrm{N}(#1)}
\newcommand{\tr}[1]{\mathrm{Tr} \! \left(#1\right)}
\newcommand{\delt}{\frac{1 + \sqrt{d}}{2}}
\renewcommand{\Im}[1]{\mathrm{Im}(#1)}
\newcommand{\modring}[1]{\Z / #1 \Z}
\newcommand{\modunits}[1]{(\modring{#1})^\times}
\renewcommand{\empty}{\varnothing}
\renewcommand{\d}[1]{\mathrm{d}#1}
\newcommand{\deriv}[2]{\frac{\d{#1}}{\d{#2}}}
\newcommand{\pderiv}[2]{\frac{\partial{#1}}{\partial{#2}}}
\newcommand{\parsq}[2]{\frac{\partial^2{#1}}{\partial{#2}^2}}

\newcommand{\atitle}[1]{\title{% 
	\large \textbf{Mathematics W4043 Algebraic Number Theory
	\\ Assignment \# #1} \vspace{-2ex}}
\author{Benjamin Church \\ \textit{Worked With Matthew Lerner-Brecher} }
\maketitle}

 
\newtheorem{theorem}{Theorem}[section]
\newtheorem{lemma}[theorem]{Lemma}
\newtheorem{proposition}[theorem]{Proposition}
\newtheorem{corollary}[theorem]{Corollary}


\begin{document}
\atitle{2}
 
\begin{enumerate}
\item Let $G \subset H$ be a pair of locally compact (Hausdorff) abelian groups with $H/G$ finite cyclic of order $n$. Consider the restriction map $r : \hat{H} \to \hat{G}$ defined by $r(\chi)(g) = \chi(g)$ for $g \in G$. Given any contiuous character $\chi \in \hat{G}$ we can lift $\chi$ to one of $n$ continuous characters in $\hat{G}$. Since $H/G$ is cyclic, $H/G = \left< a H \right>$ so any element of $H$ can be written uniquely as $a^k g$ for some $g \in G$. Write $\chi(a) = e^{i \theta}$ for $\theta \in [0, 2\pi)$. Therefore, define the lift,
\[\tilde{\chi}_r (a^k g) = \chi(g) e^{ i (2 \pi r + \theta) k /n } \]
This character is well-defined because $a^n \in G$ but,
\begin{align*}
\tilde{\chi}_r (a^{k + zn} g) & = \chi(g) e^{ i (2 \pi r + \theta) (k + zn) / n } = \chi(g) e^{i (2 \pi r + \theta) z} e^{ i (2 \pi r + \theta) k /  n } = \chi(g) e^{i \theta z} e^{ i (2 \pi r + \theta) k /n }
\\
& = \chi(g) \chi(a^n)^z e^{ i (2 \pi r + \theta) k /n } = \chi(a^{nz} g) e^{ i (2 \pi r + \theta) k /n } = \tilde{\chi}(a^k a^{zn} g)
\end{align*}
We must show that this character is a continuous homomorphism. Take $h_1 = a^{k_1} g_1$ and $h_2 = a^{k_2} g_2$ then $h_1 h_2 = a^{k_1 + k_2} g_1 g_2$ because $H$ is abelian. Then,
\[\tilde{\chi}_r(h_1 h_2) = \chi(g_1 g_2) e^{i (2 \pi r + \theta) (k_1 + k_2) /n} = \chi(g_1) e^{i (2 \pi r + \theta) k_1 /n} \chi(g_2) e^{i (2 \pi r + \theta) k_2 /n} = \tilde{\chi}_r(h_1) \tilde{\chi}_r(h_2)  \]
Furthermore, let $U \subset S^1$ be open. I claim that,
\[ \tilde{\chi}_r^{-1}(U) = \bigcup_{k = 0}^{n - 1} a^k \cdot \chi^{-1}(e^{- i (2 \pi r + \theta) k /n} \cdot U )  \] 
To prove this claim, suppose that $h \in \tilde{\chi}_r^{-1}(U)$ with $h = a^k g$ then,
\[\tilde{\chi}_n(a^k g) = \chi(g) e^{i (2 \pi r + \theta) k /n} \in U \implies \chi(g) \in e^{- i (2 \pi r + \theta) k /n} \cdot U  \implies g \in \chi^{-1}(e^{- i (2 \pi r + \theta) k /n} \cdot U) \]
and thus, $h = a^k \chi(g) \in a^k \cdot \chi^{-1}(e^{- i (2 \pi r + \theta) k /n} \cdot U )$ so $h$ is an element of the RHS. Conversely, suppose,
\[ h \in \bigcup_{k = 0}^{n - 1} a^k \cdot \chi^{-1}(e^{- i (2 \pi r + \theta) k /n} \cdot U )  \]  
Then for some $k$, we know that,
\[h \in a^k \cdot \chi^{-1}(e^{- i (2 \pi r + \theta) k /n} \cdot U ) \implies a^{-k} h \in \chi^{-1}(e^{- i (2 \pi r + \theta) k /n} \cdot U ) \implies  e^{i (2 \pi r + \theta) k /n} \chi(a^{-k} h) \in U\]
but $\chi^{-1}(e^{- i (2 \pi r + \theta) k /n} \cdot U ) \subset G$ so $a^{-k} h = g$ and therefore, $h = a^k g$ which implies that $\tilde{\chi}_r(h) =  e^{i (2 \pi r + \theta) k /n} \chi(a^{-k} h) \in U$ so $h \in \tilde{\chi}_r^{-1}(U)$. Thus, we have proven the claim. However,
\[ \tilde{\chi}_r^{-1}(U) = \bigcup_{k = 0}^{n - 1} a^k \cdot \chi^{-1}(e^{- i (2 \pi r + \theta) k /n} \cdot U )  \] 
is a union of open sets because $\chi^{-1}(U)$ is open by continuity and multiplication by a group element is a homeomorphism. Thus, $\tilde{\chi}_r$ is continuous. In particular, the reduction map $r$ is surjective because for any $\chi \in \hat{G}$ we have, $r(\tilde{\chi}_r)(g) = \tilde{\chi}_r(g) = \chi(g)$. Furthermore, we have constructed $n$ distinct lifts of $\chi$ because $\tilde{ \chi}_r(ag) = \chi(g) e^{2 \pi i k / n}$ which are distinct elements of $S^1$ for $0 \le k \le n - 1$. Also, if any $\tilde{\chi}$ is a lift of $\chi$ then $\tilde{\chi}(a^k g) = \tilde{\chi}(a^k) \chi(g) = \chi(g) \tilde{\chi}(a)^k$. Thus, $\tilde{\chi}$ is determined by the image of $a$. However, $a^n \in G$ so $\tilde{\chi}(a^n) = \tilde{\chi}(a)^n = \chi(a^n)$ so $\tilde{\chi}(a)$ is an $n^{\mathrm{th}}$-root of $\chi(a^n)$ which immediately leads to the formula,
\[\tilde{\chi}_r (a^k g) = \chi(g) e^{ i (2 \pi r + \theta) k /n } \]
Thus, there are exactly $n$ lifts of $\chi$.  \bigskip \\
Consider the map $\iota : \widehat{H/G} \to \hat{H}$ given by $\iota(\chi) = \tilde{\chi}$ where $\tilde{\chi}(h) = \chi(hG)$. If $\tilde{\chi} = 1$ then take any coset $hG$ in $H$ which gives $\chi(hG) = \tilde{\chi}(h) = 1$ so $\chi$ is trivial. Thus, $\iota$ is an injection. Furthermore, suppose that $\tilde{\chi} 
\in \Im{\iota}$ then $\tilde{\chi}$ is constant on $G$-cosets and, in particular, $\tilde{\chi}(g) = \chi(gG) = 1$ so $r(\tilde{\chi})(g) = \tilde{\chi}(g) = 1$. Thus, $\Im{\iota} \subset \ker{r}$. Likewise, if $\tilde{\chi} \in \ker{r}$ then $\tilde{\chi}(g) = 1$ so if $h' \in h G$ then $\tilde{\chi}(h') = \tilde{\chi}(hg) = \tilde{\chi}(h) \tilde{\chi}(g) = \tilde{\chi}(h)$ so $\tilde{\chi}$ is constant on $G$-cosets. Thus, $\tilde{\chi}$ descends to a map on the quotien space $\chi : H/G \to S^1$ with the property that $\chi(hG) = \tilde{\chi}(h)$ which means that $\tilde{\chi} = \iota(\chi)$. Therefore, $\ker{r} \subset \Im{\iota}$ and thus $\ker{r} = \Im{\iota}$. In summary, the following sequence is exact,
\begin{center}
\begin{tikzcd}
1 \arrow[r] & \widehat{H/G} \arrow[r, "\iota", hook] & \hat{H} \arrow[r, "r", two heads] & \hat{G} \arrow[r] & 1 
\end{tikzcd}
\end{center}

\item I will follow the discussion in Ramakrishnan's book: Fourier Analysis on Number Fields (\S 7.5). First, we will find the volume of $\udele{K}/K^\times$ in terms of $\sudele{K}{S}$ for any finite set $S$ of places. 

\begin{lemma}
The following sequence is exact,
\begin{center}
\begin{tikzcd}
1 \arrow[r] & \sudele{K}{S} / (K^\times \cap \sudele{K}{S}) \arrow[r, hook] & \udele{K}/K^\times \arrow[r, two heads] & \udele{K} / (K^\times \cdot \sudele{K}{S}) \arrow[r] & 1
\end{tikzcd}
\end{center}
\end{lemma}

\begin{proof}
The natural map $\sudele{K}{S} / (K^\times \cap \sudele{K}{S}) \to \udele{K} / K^\times$ is a well defined injection because it is an inclusion of the restricted quotient \footnote{If $A, H \subset G$ then $A/(H \cap A) \hookrightarrow G/H$ by $a + (H \cap A) \mapsto a + H$. This is a well-defined injection because $a \in A$ so $a \in (H \cap A) \iff a \in H$ }. Likewise, the map $\udele{K} / K^\times \to \udele{K} / (K^\times \cdot \sudele{K}{S})$ is surjective because $K^\times \subset K^\times \cdot \sudele{K}{S}$. It suffices to show that the image of,
\[f : \sudele{K}{S} / (K^\times \cap \sudele{K}{S}) \to \udele{K}/K^\times \quad \text{equals the kernel of} \quad g : \udele{K}/K^\times \to \udele{K} / (K^\times \cdot \sudele{K}{S})\]
The kernel of $g$ consists of $[K^\times \cdot \sudele{K}{S}]$ where the brackes denotes cosets with respect to $K^\times$. However, multiplication by an element of $K^\times$ does not affect the coset. Therefore, 
\[ \ker{g} = [K^\times \cdot \sudele{K}{S}] = [\sudele{K}{S}] = \Im{f}\]
\end{proof}

From the above lemma, 
\begin{equation}
\vol{\udele{K}/K^\times} = \vol{\udele{K} / (K^\times \cdot \sudele{K}{S})} \cdot \vol{\sudele{K}{S} / (K^\times \cap \sudele{K}{S})}
\end{equation}
However, we proved in class that $\vol{\udele{K} / (K^\times \cdot \sudele{K}{S})} = h_S$ is finite. In fact, when $S = S_{\infty}$ then $h_S = h_K$ the class number. In particular, we will want to consider the case where $S = S_\infty$. In that case, $K^\times \cap \sudele{K}{S_\infty} = \ints{K}^\times$. This equality holds since every norm of a unit is $1$ so $\ints{K}^\times \subset K^\times \cap \sudele{K}{S_\infty}$. Furthermore, if there were some element $x \in K^\times \cap \sudele{K}{S_\infty}$ with $x \notin \ints{K}^\times$ then $(x) \neq \ints{K}$ so in particular it must be a power of prime (possibly fractional) ideals and thus could not have norm $1$ with respect to the valuation of any prime in its factorization. Thus, $x \in \ints{K}^\times$. Rewriting the volume formula, 
\begin{equation} \label{master}
\vol{\udele{K}/K^\times} = h_K \cdot \vol{\sudele{K}{S} / \ints{K}^\times} 
\end{equation}
Now, we need to define the logarithmic and regulator maps, $\lambda : \sudele{K}{S_\infty} \to \R^{r_1 + r_2}$ given by,
\[ \lambda(x) = (\log{|x_\nu|_\nu})_\nu\]
and its restriction to $K^\times \cap \sudele{K}{S_\infty}$ called the regulator map $\reg : K^\times \cap \sudele{K}{S_\infty} \to \R^{r_1 + r_2}$.

\begin{lemma}
Properties of the logarithmic map,
\begin{itemize}
\item $\Im{\lambda} = H = \{(t_\nu) \mid \sum\limits_{\text{real }\nu} t_\nu + 2 \sum\limits_{\text{complex }\nu} t_\nu = 0\}$ 
\item $\ker{\lambda} = \sudele{K}{\varnothing}$
\end{itemize}
\end{lemma}

\begin{proof}
The first statement is clear from the fact that the domain of $\lambda$ is $\sudele{K}{S_\infty}$. From the definition of the unit ideles,
\[\prod_{\nu \in S_\infty} |x_\nu|_\nu = 1 \quad \text{and thus} \quad \sum_{\nu \in S_{\infty}} \log{|x_\nu|_\nu}  = 0\]
which sums twice over each pair of complex embeddings. Furthermore, $\lambda(x) = 0$ if and only if $|x_\nu|_\nu = 1$ for each $\nu \in S_\infty$. This is exactly the definition of $x \in \sudele{K}{\varnothing}$. 
\end{proof}

In particular, $\ker{\reg} = K^\times \cap \sudele{K}{\varnothing} = \mu_K$ because if every absolute value of $x$ is $1$ then the field norm must be $1$ which implies that $x$ is a unit on the unit circle and thus a root of unity. Also, define $w_K = |\mu_K|$ and $L = \reg(\ints{K}^\times)$. Now, consider the commutative diagram,
\begin{center}
\begin{tikzcd}[row sep = large]
& 1 \arrow[d] & 1 \arrow[d] & 1 \arrow[d] & 
\\
1 \arrow[r] & \mu_K \arrow[r] \arrow[d] & \ints{K}^\times \arrow[r, "\reg"] \arrow[d] & L \arrow[r] \arrow[d] & 1
\\
1 \arrow[r] & \sudele{K}{\varnothing} \arrow[r] \arrow[d] & \sudele{K}{S_{\infty}} \arrow[r, "\lambda"] \arrow[d] & H \arrow[r] \arrow[d] & 1
\\
1 \arrow[r] & \sudele{K}{\varnothing} / \mu_K \arrow[r] \arrow[d] & \sudele{K}{S_{\infty}} / \ints{K}^\times \arrow[r, "\lambda_*"] \arrow[d] & H/L \arrow[r] \arrow[d] & 1
\\
& 1 & 1 & 1 &
\end{tikzcd}
\end{center}
which I claim is exact in each row and column. The columns are simply inclusions and projections onto the corresponding quotiens which are trivially exact. The first row is exact because $\ker{\reg} = \mu_K$ under the inclusion into $\ints{K}^\times$. The second row is exact again because $\ker{\lambda} = \sudele{K}{\varnothing}$ included into $\sudele{K}{S_\infty}$. The third row is exact by the snake lemma where the bottom row forms the cokernels of the downward maps and the top row of trivial groups forms the kernels of the downwards maps by the exactness of the columns. That is,
\begin{center}
\begin{tikzcd}[row sep = large]
& 1 \arrow[d] \arrow[r] & 1 \arrow[draw=none]{ddd}[name=Z, shape=coordinate]{} \arrow[d] \arrow[r]  & 1 \arrow[d]  & 
\\
1 \arrow[r] & \mu_K \arrow[r] \arrow[d] & \ints{K}^\times \arrow[r, "\reg"] \arrow[d] & L \arrow[r] \arrow[d] & 1 
\\
1 \arrow[r] & \sudele{K}{\varnothing} \arrow[r] \arrow[d] & \sudele{K}{S_{\infty}} \arrow[r, "\lambda"] \arrow[d] & H \arrow[r] \arrow[d] & 1
\\
& \sudele{K}{\varnothing} / \mu_K \arrow[r] \arrow[d]
\arrow[from=uuurr,
rounded corners, crossing over,
to path={ -- ([xshift=2ex]\tikztostart.east)
|- (Z) [near end]\tikztonodes
-| ([xshift=-2ex]\tikztotarget.west)
-- (\tikztotarget)}]
& \sudele{K}{S_{\infty}} / \ints{K}^\times \arrow[r, "\lambda_*"] \arrow[d] & H/L \arrow[r] \arrow[d] & 1
\\
& 1 & 1 & 1 &
\end{tikzcd}
\end{center} 
Using the properties of quotient measures, we arive at formulas for various groups in this diagram by noting that the center group is isomprphic to the quotient of the last over the first. In particular,
\begin{equation} \label{modOKtimes}
\vol{\sudele{K}{S_\infty} / \ints{K}^\times} = \vol{\sudele{K}{\varnothing} / \mu_K} \cdot \vol{H/L} 
\end{equation}
where $R_K = \vol{H/L}$ is the regulator of $K$ i.e. the volume of the fundamental domain $H/L$ with respect to the quotien measure induced by $\lambda_*$. Furthermore,
\begin{equation} \label{muK}
\vol{\sudele{K}{\varnothing} / \mu_K} = \vol{\sudele{K}{\varnothing}} / \vol{\mu_K} 
\end{equation}
where the measure on $\mu_K$ is the discrete measure so $\vol{\mu_K} = |\mu_K| = w_K$ by definition. Putting equations \ref{master}, \ref{modOKtimes}, and \ref{muK} together we get,
\begin{equation} \label{together}
\vol{\udele{K} / K^\times} = \frac{h_K}{w_K} R_K \cdot \vol{\sudele{K}{\varnothing}}  
\end{equation}
Therefore, we have reduced the task of computing $\vol{\udele{K} / K^\times}$ to that of computing $\vol{\idelempty}$ which I claim is easier. 

\begin{lemma}
\[\vol{\idelempty} = 2^{r_1} (2 \pi)^{r_2} |d_K|^{-1/2}\]
\end{lemma}

\begin{proof}
$\idelempty$ admits the product decomposition,
\[ \idelempty = \prod_{\text{real } \nu} U_\nu \times \prod_{\text{complex } \nu} U_\nu \times \prod_{\text{non arch. } \nu} U_\nu \]
where $U_\nu$ denotes the subset of $K_\nu$ with absolute value $1$. Ramakrishnan classifies how the Haar measure on $\idelempty$ restricts to these sets:
\begin{itemize}
\item For real $\nu$, the measure on $\{\pm 1\}$ is the counting measure.
\item For complex $\nu$, the measure on $S^1 \subset \C^\times$ is the standard Lebesgue measure.
\item For non archimedean $\nu$, the measure on $\ints{\nu}^\times$ is the standard self-dual Haar measure. 
\end{itemize}
Therefore,
\begin{equation}
\vol{U_\nu} = \begin{cases}
2 & \nu \text{ is real} \\
2 \pi & \nu \text{ is complex} \\
N(\mathscr{D}_\nu)^{-1/2} & \nu \text{ is non archimedean}
\end{cases}
\end{equation}
where $\mathscr{D}_\nu$ is the different for the place $\nu$.
Then, using the fact that,
\begin{equation}
|d_K| = \prod_{\text{non arch. } \nu} N(\mathscr{D}_\nu)
\end{equation}
we find that the total measure becomes,
\begin{equation}
\vol{\idelempty} = 2^{r_1} (2 \pi)^{r_2} |d_K|^{-1/2}
\end{equation}
\end{proof}

Combinging the above lemma and equation \ref{together} we arrive at the final result,
\begin{equation}
\vol{\udele{K} / K^\times} = \frac{2^{r_1} (2 \pi)^{r_2} h_K R_K}{w_K \sqrt{|d_K|}} 
\end{equation}

\item Let $K/\Q$ be a finite Galois extension with a complex embedding $\tau : K \to \C$ such that $\bar{\tau} \neq \tau$. Because $\tau$ is an embedding, $K \cong \tau(K)$. Therefore, we can consider the field $\Q \subset \tau(K) \subset \C$. I claim that $\tau(K)/ \Q$ is normal. For any $\alpha \in \tau(K)$ there exists a $\beta \in K$ such that $\tau(\beta) = \alpha$. Now consider the minimal polynomial $q \in \Q[X]$ of $\alpha$. Because $\tau(1) = 1$ we know that $\tau$ preserves $\Q$. Therefore, $q(\alpha) = q(\tau(\beta)) = \tau(q(\beta)) = 0$ so because $\tau$ is injective we know that $\beta$ is a root of $q$. Since $q$ is irreducible over $\Q$ we know that $q$ is the minimal polynomial of $\beta$. Because $K/\Q$ is Galois, $q$ splits in $K$. Therefore, the image of $q$ must also split in $\tau(K)$. Thus, $\tau(K) / \Q$ is normal. \bigskip \\
Now, for $\alpha \in \tau(K)$ take the minimal polynomial of $q$ over $\Q$,
\[ q(X) = a_n X^n + \cdots + a_0 \]
Because the coeficients are real, $\overline{q(\alpha)} = q(\bar{\alpha}) = 0$. Therefore, $\alpha$ is a root of $q$. Hovever, $\tau(K)$ is normal over $\Q$ so $q$ splits completely in $\tau(K)$ so $\bar{\alpha} \in \tau(K)$. Therefore, complex conjugation, $\sigma : \tau(K) \to \tau(K)$ is an automorphism of $\tau(K)$. However, $\tau \neq \bar{\tau}$ so $\exists \beta \in K$ such that $\tau(\beta) \neq \overline{\tau(\beta)}$. Therefore, $\sigma \neq \id$ but $\sigma^2 = \id$ so $\sigma$ has order $2$ in the group $\galgroup{\tau{K}/\Q}$. Thus, $\galgroup{K/\Q} \cong \galgroup{\tau(K)/ \Q}$ must have even order and thus $[K : \Q]$ is even.  

\item Let $K / \Q$ be real with $r_1 = d$ and $r_2 = 0$ and let $L / K $ have degree two with $L$ having complex embeddings, $r_1 = 0$ and $r_2 = d$. Take $c \in \galgroup{L/K}$ with $c \neq \id$. By Dirichlet's unit theorem, we know that $\ints{K}^\times \cong \Z^{d - 1} \times \mu(\ints{K})$ and likewise, $\ints{L}^\times \cong \Z^{d - 1} \times \mu(\ints{L})$ because $r_1 + r_2 - 1$ is equal in both cases. Let $u_1, \cdots, n_{d - 1}$ be a basis for the free abelian part of $\ints{K}^\times$. However, $\ints{L}^\times$ has free abelian part of equal rank so $u_1, \cdots, n_{k}$ with $k = d - 1$, must also be a basis for the units of infinite order in $\ints{L}$. Thus, any unit $u \in \ints{L}^\times$ can be written as,
\[ u = u_1^{n_1} \cdots u_{k}^{n_{k}} \omega \]
where $\omega \in \mu(\ints{L})$ is a root of unity in $\ints{L}$. However, $u_i \in \ints{K} \subset K$ so $c \in \galgroup{L/K}$ fixes $u_i$ for each $i$. Therefore, suppose that $c(u) = u^{-1}$, then,
\[c(u) = c(u_1^{n_1} \cdots u_{k}^{n_{k}}) c(\omega) = u_1^{n_1} \cdots u_{k}^{n_{k}} c(\omega) = u^{-1} = u_1^{-n_1} \cdots u_{k}^{-n_{k}} \omega^{-1} \]
In this case,
\[u_1^{2 n_1} \cdots u_{k}^{2n_{k}} c(\omega) \omega = 1 \]
but since $c(\omega) \omega$ has finite order and $u_1 \cdots u_k$ is a basis of a free abelian group, this implies that $n_1 = \cdots = n_k = 0$ and thus $c(\omega) \omega = 1$. Therefore, $u = \omega \in \mu(\ints{L})$. However, $L / \Q$ is a finite extension so it has a finite number of roots of unity and thus, $\mu(\ints{L})$ is finite. However, the set of units such that $c(u) = u^{-1}$ is a subset of $\mu(\ints{L})$ and is therefore finite.  
\end{enumerate}

\end{document}