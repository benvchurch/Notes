\documentclass[12pt]{extarticle}
\usepackage[utf8]{inputenc}
\usepackage[english]{babel}
\usepackage[a4paper, total={7in, 9.5in}]{geometry}
 
\usepackage{amsthm, amssymb, amsmath, centernot}
\usepackage{mathtools}
\DeclarePairedDelimiter{\floor}{\lfloor}{\rfloor}

\newcommand{\notimplies}{%
  \mathrel{{\ooalign{\hidewidth$\not\phantom{=}$\hidewidth\cr$\implies$}}}}
 
\renewcommand\qedsymbol{$\square$}
\newcommand{\cont}{$\boxtimes$}
\newcommand{\divides}{\mid}
\newcommand{\ndivides}{\centernot \mid}
\newcommand{\Z}{\mathbb{Z}}
\newcommand{\N}{\mathbb{N}}
\newcommand{\C}{\mathbb{C}}
\newcommand{\Zplus}{\mathbb{Z}^{+}}
\newcommand{\Primes}{\mathbb{P}}
\newcommand{\ball}[2]{B_{#1} \! \left(#2 \right)}
\newcommand{\Q}{\mathbb{Q}}
\newcommand{\R}{\mathbb{R}}
\newcommand{\Rplus}{\mathbb{R}^+}
\newcommand{\invI}[2]{#1^{-1} \left( #2 \right)}
\newcommand{\End}[1]{\text{End}\left( A \right)}
\newcommand{\legsym}[2]{\left(\frac{#1}{#2} \right)}
\renewcommand{\mod}[3]{\: #1 \equiv #2 \: (\mathrm{mod} \: #3) \:}
\newcommand{\nmod}[3]{\: #1 \centernot \equiv #2 \: (\mathrm{mod} \: #3) \:}
\newcommand{\ndiv}{\hspace{-4pt}\not \divides \hspace{2pt}}
\newcommand{\finfield}[1]{\mathbb{F}_{#1}}
\newcommand{\finunits}[1]{\mathbb{F}_{#1}^{\times}}
\newcommand{\ord}[1]{\mathrm{ord}\! \left(#1 \right)}
\newcommand{\quadfield}[1]{\Q \small(\sqrt{#1} \small)}
\newcommand{\vspan}[1]{\mathrm{span}\! \left\{#1 \right\}}
\newcommand{\galgroup}[1]{Gal \small(#1 \small)}
\newcommand{\ints}[1]{\mathcal{O}_{#1}}
\newcommand{\sm}{\! \setminus \!}
\newcommand{\norm}[3]{\mathrm{N}^{#1}_{#2}\left(#3\right)}
\newcommand{\qnorm}[2]{\mathrm{N}^{#1}_{\Q}\left(#2\right)}
\newcommand{\quadint}[3]{#1 + #2 \sqrt{#3}}
\newcommand{\pideal}{\mathfrak{p}}
\newcommand{\inorm}[1]{\mathrm{N}(#1)}
\newcommand{\tr}[1]{\mathrm{Tr} \! \left(#1\right)}
\newcommand{\delt}{\frac{1 + \sqrt{d}}{2}}
\renewcommand{\Im}[1]{\mathrm{Im}(#1)}
\newcommand{\modring}[1]{\Z / #1 \Z}
\newcommand{\modunits}[1]{(\modring{#1})^\times}
\renewcommand{\empty}{\varnothing}
\renewcommand{\d}[1]{\mathrm{d}#1}
\newcommand{\deriv}[2]{\frac{\d{#1}}{\d{#2}}}
\newcommand{\pderiv}[2]{\frac{\partial{#1}}{\partial{#2}}}
\newcommand{\parsq}[2]{\frac{\partial^2{#1}}{\partial{#2}^2}}

\newcommand{\atitle}[1]{\title{% 
	\large \textbf{Mathematics W4043 Algebraic Number Theory
	\\ Assignment \# #1} \vspace{-2ex}}
\author{Benjamin Church \\ \textit{Worked With Matthew Lerner-Brecher} }
\maketitle}

 
\newtheorem{theorem}{Theorem}[section]
\newtheorem{lemma}[theorem]{Lemma}
\newtheorem{proposition}[theorem]{Proposition}
\newtheorem{corollary}[theorem]{Corollary}

\newcommand{\uniform}[1]{\mathfrak{m}_{#1}}
\newcommand{\modulo}[1]{\: \: (\mathrm{mod} \: #1)}
\newcommand{\vol}[1]{\mathrm{vol}\left( #1 \right)}
\newcommand{\reg}{\mathrm{reg}}
\newcommand{\idelempty}{\sudele{K}{\varnothing}}

\begin{document}
\atitle{5}
 
\begin{enumerate}
\item Let $G$ be a group. Consider the map $\Psi : \Z[G] \to \Z$ given by,
\[ \Psi : \sum_{g \in G} n_g \, g \mapsto \sum_{g \in G} n_g \]
Take $I_G = \ker{\Psi}$. Consider the map,
\[ \Phi : G \to I_G / I_G^2 \quad \text{given by} \quad g \mapsto (g - 1) \: (\mathrm{mod} \: I_G^2) \]

First, we need to show that $\Phi$ is a homomorphism,
\[ g_1 g_2 \mapsto \mod{(g_1 g_2 - 1) = (g_1 - 1)(g_2 - 1) + (g_1 - 1) + (g_2 - 1)}{(g_1 - 1) + (g_2 - 1)}{I_G^2} \] 
since $(g_1 - 1) (g_2 - 1) \in I_G^2$. Given an element,
\[ \sum_{i = 1}^r n_i \, g_i  \quad \text{such that} \quad \sum_{i = 1}^r n_i = 0 \]
we can write,
\[ \sum_{i = 1}^r n_i \, g_i  = \sum_{i = 1}^r (n_i \, g_i - 1) + \sum_{i = 1}^r n_i = \sum_{i = 1}^r (n_i \, g_i - 1) \]
and we know that,
\[ \Phi : \prod_{i = 1}^r g_i^{n_i} \mapsto \sum_{i = 1}^r n_i \, (g_i - 1)  \] 
so $\Phi$ is surjective. Furthermore, $I_G/I_G^2$ is an abelian group so the map $\Phi : G \to I_G/I_G^2$ factors through $G^{\mathrm{ab}} = G/[G, G]$. Take the map $\Phi^{\mathrm{ab}} : G^{\mathrm{ab}} \to I_G/I_G^2$. We construct an inverse map by $\Xi : I_G \to G^{\mathrm{ab}}$ given by $\Xi : (g - 1) \mapsto g$ which is well defined because both groups are abelian so the map is invariant under reordering. Consider any product, 
\[(g_1 - 1)(g_2 - 1) = (g_1 g_2 - 1) - (g_1 - 1) - (g_2 - 1) \mapsto g_1 g_2 g_1^{-1} g_2^{-1} \in [G, G] \] 
so $\Xi$ is trivial on $I_G^2$. Equivalently, we see that the kernel of $\Phi$ sending elements into $I_G^2$ is generated by commutators so $\Phi^{\mathrm{ab}}$ is injective. Continuing, we find that $\Xi$ factors through the quotient as a map,
\[ \Xi : I_G / I_G^2 \mapsto G^{\mathrm{ab}} \quad \text{acting as} \quad \Xi : (g - 1) \mapsto g\] 
which is clearly and inverse of $\Phi^{\mathrm{ab}}$. Therefore, $\Phi^{\mathrm{ab}}$ is an isomorphism. 

\item Let $G$ be a finite group. Set $\Lambda = \Z[G]$ and consider the map,
\[ \Psi : \Homover{\Z}{\Lambda}{B} \to \Lambda \otimes_{\Z} B \quad \text{given by} \quad \Psi : \varphi \mapsto \sum_{g \in G} g \otimes \varphi(g)\]
First, suppose that,
\[ \Psi(\varphi) = \sum_{g \in G} g \otimes \varphi(g) = 0 \]
then each term is zero and thus $\varphi(g) = 0$ which means that $\varphi = 0$. Therefore, $\Phi$ is injective. Furthermore, for $g \in G$ and $h \in B$, consider the map,
\[ \delta^h_g(x) = \begin{cases}
h &  x = g \\ 
0 & x \neq g
\end{cases} \]
This is a morphism of $\Z$-modules $\Z[G] \to B$. Furthermore,
\[ \Psi : \delta^h_g \mapsto \sum_{x \in G} g \otimes \delta^h_g(x) = g \otimes h \]
which implies that,
\[ \Psi : \sum_{g, h} n_{g,h} \delta^h_g \mapsto \sum_{g, h} n_{g, h} g \otimes h \]
and thus $\Psi$ is surjective. It remains to check that $\Phi$ is a morphism of $\Lambda$-modules. 
\[ \Psi(\varphi_1 + \varphi_2) = \sum_{g \in G} g \otimes (\varphi_1(g) + \varphi_2(g)) = \sum_{g \in G} g \otimes \varphi_1(g) + \sum_{g \in G} g \otimes \varphi_2(g) = \Psi(\varphi_1) + \Psi(\varphi_2) \]
Furthermore,
\[ \Phi( h \cdot \varphi) = \sum_{g \in G} g \otimes \varphi(h^{-1} g) = \sum_{g' \in G} hg' \otimes \varphi(g') = h \cdot \sum_{g' \in G} g'
\otimes \varphi(g') = h \cdot \Phi(\varphi) \]
so $\Phi$ is an isomorphism of $\Lambda$ modules.
\item
\begin{theorem}[Tate]
Let $G$ be a finite group and let $C$ be a $G$-module. Suppose that all (not necessarily proper) subgroups $H$ of $G$ satisfy $G^1(H, C) = 0$ and $H^2(H, C)$ is cyclic of order $|H|$. Then, there is an isomorphism,
\[ \hat{H}^r(G, \Z) \to \hat{H}^{r+2}(G, C) \]
depending only on the choice of a generator for $H^2(G, C)$.
\end{theorem}

\begin{proof}
I will follow Milne's notes on Class Field Theory (p. 81 - 82). First we choose a generator $\gamma$ of the cyclic group $H^2(G, C)$ which when restricted to any subgroup $H$ must also generate $H^2(H, C)$. Given a cocycle $\varphi \in C_2(G, C) = \Homover{G}{\Z[G \times G]}{C}$ which goes to the generator of $H^2(G, C)$ when homology is taken we can define a $G$-module,
\[ C(\varphi) = C \oplus \bigoplus_{\sigma \in G} [x_{\sigma}] \Z \]
Where $G$ acts on the free group by its action on the basis symbols,
\[ \sigma \cdot x_{\tau} = x_{\sigma \circ \tau} - x_{\sigma} + \varphi(\sigma, \tau) \]
An easy computation shows that this defines an action of $G$ on $C(\varphi)$. We now need to show that,
\[ H^1(H, C(\varphi)) = H^2(H, C(\varphi)) = 0 \]
with the action defined above. We have a short exact sequence,
\begin{center}
\begin{tikzcd} 
0 \arrow[r] & I_G \arrow[r] & \Z[G] \arrow[r] & \Z \arrow[r] & 0 
\end{tikzcd}
\end{center}

since $I_G$ is the kernel of the map $\Z[G] \to \Z$ and is thus generated by elements of the form $\sigma - 1$ for $\sigma \in G \sm \{1\}$. We know that $\Z[G]$ is an induced module and thus $H^r(G, \Z[G])$ for all $r$\footnote{$\Z[G] \cong \mathrm{Ind}^G_1(\Z)$ so by Shapiro's Lemma, $H^r(G, \mathrm{Ind}^G_1(\Z) \cong H^r(1, \Z) = 0$}. We can define a map $\alpha : C(\varphi) \to \Z[G]$ such that, $\alpha(c) = 0$ for $c \in C$ and $\alpha(x_\sigma) = \sigma - 1$. Which gives rise to a short exact sequence of $G$-modules,
\begin{center}
\begin{tikzcd}
0 \arrow[r] & C \arrow[r] & C(\varphi) \arrow[r, "\alpha"] & I_G \arrow[r] & 0 
\end{tikzcd}
\end{center}
This short exact sequence of $G$-modules gives rise to a long exact sequence of homology,
\begin{center}
\begin{tikzcd}[column sep = small]
H^0(H, C) \arrow[r] & H^0(H, C(\varphi)) \arrow[r]  \arrow[draw=none]{d}[name=Z, shape=coordinate]{} & H^0(H, I_G) \arrow[r] &  H^1(H, C) \arrow[r] & H^1(H, C(\varphi)) \arrow[r] & H^1(H, I_G) \arrow[dllll,
rounded corners, crossing over,
to path={ -- ([xshift=2ex]\tikztostart.east)
|- (Z) [near end]\tikztonodes
-| ([xshift=-2ex]\tikztotarget.west)
-- (\tikztotarget)}]
& &
\\ 
& H^2(H, C) \arrow[r] & H^2(H, C(\varphi)) \arrow[r] & H^2(H, I_G) \arrow[r] & \cdots
\end{tikzcd}
\end{center}

However, we know that $H^1(H, C) = 0$ and $H^2(H, I_G) \cong H^1(H, \Z) = 0$. Therefore, we have the exact sequence,

\begin{center}
\begin{tikzcd}
0 \arrow[r] & H^1(H, C(\varphi)) \arrow[r] & H^1(H, I_G) \arrow[r] & H^2(H, C) \arrow[r] & H^2(H, C(\varphi)) \arrow[r] & 0 
\end{tikzcd}
\end{center} 

We use this sequence to argue that $H^1(G, C(\varphi)) = H^2(G, C(\varphi)) = 0$ and therefore that all the cohomology groups vanish because $G$ is finite. 
\bigskip\\
As we have already seen, there are exact sequences,
\begin{center}
\begin{tikzcd}
0 \arrow[r] & C \arrow[r] & C(\varphi) \arrow[r, "\alpha"] & I_G \arrow[r] & 0 
\end{tikzcd}
\end{center}
and, 
\begin{center}
\begin{tikzcd}
0 \arrow[r] & I_G \arrow[r] & \Z[G] \arrow[r] & \Z \arrow[r] & 0
\end{tikzcd}
\end{center}
which together give a sequence,
\begin{center}
\begin{tikzcd}
0 \arrow[r] & C \arrow[r] & C(\varphi) \arrow[r, "\alpha"] & \Z[G] \arrow[r] & \Z \arrow[r] & 0
\end{tikzcd}
\end{center}
which remains exact by a homology computation. However, we know that $H^r(G, C(\varphi)) = 0$ and $H^r(G, \Z[G]) = 0$ so the map $H^r(G, \Z) \to H^{r+2}(G, C)$ is an isomorphism by proposition 1.13 in Milne. 

\end{proof}
\end{enumerate}

\end{document}