\documentclass[12pt]{extarticle}
\usepackage[utf8]{inputenc}
\usepackage[english]{babel}
\usepackage[a4paper, total={7in, 9.5in}]{geometry}
 
\usepackage{amsthm, amssymb, amsmath, centernot}
\usepackage{mathtools}
\DeclarePairedDelimiter{\floor}{\lfloor}{\rfloor}

\newcommand{\notimplies}{%
  \mathrel{{\ooalign{\hidewidth$\not\phantom{=}$\hidewidth\cr$\implies$}}}}
 
\renewcommand\qedsymbol{$\square$}
\newcommand{\cont}{$\boxtimes$}
\newcommand{\divides}{\mid}
\newcommand{\ndivides}{\centernot \mid}
\newcommand{\Z}{\mathbb{Z}}
\newcommand{\N}{\mathbb{N}}
\newcommand{\C}{\mathbb{C}}
\newcommand{\Zplus}{\mathbb{Z}^{+}}
\newcommand{\Primes}{\mathbb{P}}
\newcommand{\ball}[2]{B_{#1} \! \left(#2 \right)}
\newcommand{\Q}{\mathbb{Q}}
\newcommand{\R}{\mathbb{R}}
\newcommand{\Rplus}{\mathbb{R}^+}
\newcommand{\invI}[2]{#1^{-1} \left( #2 \right)}
\newcommand{\End}[1]{\text{End}\left( A \right)}
\newcommand{\legsym}[2]{\left(\frac{#1}{#2} \right)}
\renewcommand{\mod}[3]{\: #1 \equiv #2 \: (\mathrm{mod} \: #3) \:}
\newcommand{\nmod}[3]{\: #1 \centernot \equiv #2 \: (\mathrm{mod} \: #3) \:}
\newcommand{\ndiv}{\hspace{-4pt}\not \divides \hspace{2pt}}
\newcommand{\finfield}[1]{\mathbb{F}_{#1}}
\newcommand{\finunits}[1]{\mathbb{F}_{#1}^{\times}}
\newcommand{\ord}[1]{\mathrm{ord}\! \left(#1 \right)}
\newcommand{\quadfield}[1]{\Q \small(\sqrt{#1} \small)}
\newcommand{\vspan}[1]{\mathrm{span}\! \left\{#1 \right\}}
\newcommand{\galgroup}[1]{Gal \small(#1 \small)}
\newcommand{\ints}[1]{\mathcal{O}_{#1}}
\newcommand{\sm}{\! \setminus \!}
\newcommand{\norm}[3]{\mathrm{N}^{#1}_{#2}\left(#3\right)}
\newcommand{\qnorm}[2]{\mathrm{N}^{#1}_{\Q}\left(#2\right)}
\newcommand{\quadint}[3]{#1 + #2 \sqrt{#3}}
\newcommand{\pideal}{\mathfrak{p}}
\newcommand{\inorm}[1]{\mathrm{N}(#1)}
\newcommand{\tr}[1]{\mathrm{Tr} \! \left(#1\right)}
\newcommand{\delt}{\frac{1 + \sqrt{d}}{2}}
\renewcommand{\Im}[1]{\mathrm{Im}(#1)}
\newcommand{\modring}[1]{\Z / #1 \Z}
\newcommand{\modunits}[1]{(\modring{#1})^\times}
\renewcommand{\empty}{\varnothing}
\renewcommand{\d}[1]{\mathrm{d}#1}
\newcommand{\deriv}[2]{\frac{\d{#1}}{\d{#2}}}
\newcommand{\pderiv}[2]{\frac{\partial{#1}}{\partial{#2}}}
\newcommand{\parsq}[2]{\frac{\partial^2{#1}}{\partial{#2}^2}}

\newcommand{\atitle}[1]{\title{% 
	\large \textbf{Mathematics W4043 Algebraic Number Theory
	\\ Assignment \# #1} \vspace{-2ex}}
\author{Benjamin Church \\ \textit{Worked With Matthew Lerner-Brecher} }
\maketitle}

 
\newtheorem{theorem}{Theorem}[section]
\newtheorem{lemma}[theorem]{Lemma}
\newtheorem{proposition}[theorem]{Proposition}
\newtheorem{corollary}[theorem]{Corollary}

\newcommand{\uniform}[1]{\mathfrak{m}_{#1}}
\newcommand{\modulo}[1]{\: \: (\mathrm{mod} \: #1)}
\newcommand{\vol}[1]{\mathrm{vol}\left( #1 \right)}
\newcommand{\reg}{\mathrm{reg}}
\newcommand{\idelempty}{\sudele{K}{\varnothing}}

\begin{document}
\atitle{4}
 
\begin{enumerate}
\item Consider the standard free resolution of $\Z$ given by,
\begin{center}
\begin{tikzcd}
\cdots \arrow[r] & \Lambda_i \arrow[r, "\delta_i"] & \Lambda_{i-1} \arrow[r, "\delta_{i-1}"] & \cdots \arrow[r, "\delta_2"] & \Lambda_1 \arrow[r, "\delta_1"] & \Z \arrow[r] & 0
\end{tikzcd}
\end{center}
where $\Lambda_i = \Z[G^{i+1}]$ and $\delta_i : \Lambda_i \to \Lambda_{i-1}$ is given by,
\[ \delta_i(g_0, g_1, \dots, g_i) = \sum_{j = 0}^i (-1)^j (g_0, \dots, \hat{g}_j, \dots, g_i) \]
First we need to show that this sequence is a complex. Take any $(g_0, \cdots, g_i) \in \Lambda_i$ then consider the composition $\delta_{i-1} \circ \delta_{i}(g_0, \dots, g_i) = 0$,
because the composition of the two boundary maps will remove each pair of positions but in two different ways. Either, $\delta_i$ removes the first of the pair or $\delta_{i-1}$ removes the first. These two options form the same term but with opposite sign because if $\delta_i$ removes the first of the pair then the index of the second is shifted by one so there is a relative minus sign between the two terms. Therefore, the sum is zero. Explicitly,
\begin{align*}
\delta_{i-1} \circ \delta_{i}(g_0, \dots, g_i) & = \sum_{k < j} (-1)^{j + k} (g_0, \dots, \hat{g}_k, \dots, \hat{g}_j, \dots, g_i) + \sum_{k \ge j} (-1)^{j + k} (g_0, \dots, \hat{g}_j, \dots, \hat{g}_{k+1}, \dots, g_i)
\\
& = \sum_{k < j} (-1)^{j + k} (g_0, \dots, \hat{g}_k, \dots, \hat{g}_j, \dots, g_i) - \sum_{k' \ge j} (-1)^{j + k'} (g_0, \dots, \hat{g}_j, \dots, \hat{g}_{k'}, \dots, g_i)
\\
& = 0
\end{align*}
Therefore $\Im{\delta_{i}} \subset \ker{\delta_{i-1}}$ and the resolution forms a complex. Furhermore, define the chain homotopy, $h_i : \Lambda_{i-1} \to \Lambda_i$ by the map,
\[ h_i(g_1, \dots, g_i) = (1, g_1, \dots, g_i) \]
Now, consider the composition,
\begin{align*}
\delta_{i} \circ h_i(g_1, \dots, g_i) & = (g_1, \dots, g_i) + \sum_{j = 1}^i  (-1)^j (1, g_1, \dots, \hat{g}_j, \dots, g_i) 
\\
& = (g_1, \dots, g_i) + \sum_{j = 1}^i  (-1)^j h_{i-1}(g_1, \dots, \hat{g}_j, \dots, g_i) 
\\
& = (g_1, \dots, g_i) - h_{i-1} \left( \sum_{j' = 0}^i  (-1)^{j'} (g_1, \dots, \hat{g}_{j'+1}, \dots, g_i) \right)
\\
& = (\id - h_{i-1} \circ \delta_{i-1}) (g_1, \dots, g_i)
\end{align*}
Therefore, take any cycle $X \in \ker{\delta_{i-1}}$ then $\delta_{i} \circ h_i(X) = X - h_{i-1} \circ \delta_{i-1}(X) = X$ so $X \in \Im{\delta_{i}}$ therefore $X$ is a boundary. Thus, $\ker{\delta_{i}} \subset \Im{\delta_i}$ so in total $\ker{\delta_i} = \Im{\delta_i}$. Therefore, the free resolution is exact. 
 
\item Let $G$ be a group and $A$ a $G$-module. Define the group of homogeneous $i$-cochains,
\newcommand{\homcochain}[1]{C_{\mathrm{hom}}^{#1}(G,A)}
\[\homcochain{i} = \{ f : G^{i+1} \to A \mid f(gX) = g f(X) \forall X \in G^{i+1} \} \cong \Homover{G}{\Lambda_i}{A} \]
and the group of inhomogeneous $i$-cochains,
\newcommand{\inhomcochain}[1]{C_{\mathrm{in}}^{#1}(G,A)}
\[\inhomcochain{i} = \{ f : G^{i} \to A  \} \]
There is a bijection $F_i : \homcochain{i} \to \inhomcochain{i}$ given by, $F_i : f \mapsto \phi$ such that
\[ \phi(g_1, \cdots, g_i) = f(1, g_1, g_1 g_2, \dots, g_1 \cdots g_i) \]
Now, consider the homogeneous differential,
\[d^i_{\mathrm{hom}} : \homcochain{i} \to \homcochain{i+1}\]
given by sending $d^i_{\mathrm{hom}} : f \mapsto f \circ \delta_{i+1}$. Consider, $F_{i+1} \circ d^i_{\mathrm{hom}}(f) = F_{i+1}(f \circ \delta_{i+1})$. This map is a inhomogeneous $i+1$-cochain acting as,
\begin{align*}
F_{i+1}(f \circ \delta_{i+1})(g_1, \dots, g_{i+1}) & = (f \circ \delta_{i+1})(1, g_1, g_1 g_2, \dots, g_1 \cdots g_{i+1}) 
\\
& = \sum_{j = 0}^{i+1} (-1)^j f(1, g_1, g_1 g_2, \dots, \widehat{g_1 g_2 \dots g_j}, \dots, g_1 \cdots g_{i+1}) 
\\
& = f(g_1, g_1 g_2, \dots, g_1 \cdots g_{i+1}) 
\\
& \quad \quad + \sum_{j = 1}^{i+1} (-1)^j f(1, g_1, g_1 g_2, \dots, \widehat{g_1 g_2 \dots g_j}, \dots, g_1 \cdots g_{i+1}) 
\end{align*}
However, for $j < i+1$,
\begin{align*} 
f(1, g_1, g_1 g_2, \dots, & \widehat{g_1 g_2 \dots g_j}, \dots, g_1 \cdots g_{i+1})
\\
& = f(1, g_1, g_1 g_2, \dots, g_1 \dots g_{j-1}, g_1 \dots g_{j-1} (g_k g_{j+1}), \dots, g_1 \cdots g_{i+1}) 
\\
& = \phi(g_1, g_2, \dots, g_j g_{j+1}, \dots g_{i+1})
\end{align*}
Thus,
\begin{align*}
F_{i+1}(f \circ \delta_{i+1})(g_1, \dots, g_{i+1}) 
& = g_1 \cdot f(1, g_2, \dots, g_2 \cdots g_{i+1}) + \sum_{j = 1}^{i} (-1)^j \phi(g_1, g_2, \dots, g_j g_{j+1}, \dots g_{i+1})
\\
& + (-1)^{i+1} f(1, g_1, g_1 g_2, \dots, g_1 \dots g_{i})
\\
& = g_1 \cdot \phi(g_2, \dots, g_{i+1}) 
\\ & \quad \quad + \sum_{j = 1}^{i} (-1)^j \phi(g_1, g_2, \dots, g_j g_{j+1}, \dots g_{i+1}) + (-1)^{i+1} \phi(g_1, \dots, g_i)
\end{align*}
which is the formula for the inhomogeneous differential,
\[d^i : \inhomcochain{i} \to \inhomcochain{i+1}\] Therefore, $d^{i}_{\mathrm{in}} = F_{i+1} \circ d^i_{\mathrm{hom}}$
\item Let $G$ be the trivial group. We know that $(-)^G$ is an equivalent functor to $\Homover{G}{\Z}{-}$ which is left exact. Furthermore, for any $G$-module $A$, the cohomology $H^i(G, -)$ is isomorphic to the derived functor of $\Homover{G}{\Z}{-}$. However, since $G$ is trivial, $M^G \cong M$ and therefore $\Homover{G}{\Z}{-}$ is the identity functor. Therefore, choosing any injective resolution of $A$, 
\begin{center}
\begin{tikzcd}
0 \arrow[r] & A \arrow[r] & I^0 \arrow[r] & I^1 \arrow[r] & I^2 \arrow[r] & \cdots 
\end{tikzcd}
\end{center} 
the complex obtained by applying the functor $\Homover{G}{\Z}{-}$,
\begin{center}
\begin{tikzcd}
0 \arrow[r] & \Homover{G}{\Z}{A} \arrow[r] & \Homover{G}{\Z}{I^0} \arrow[r] & \Homover{G}{\Z}{I^1} \arrow[r] & \Homover{G}{\Z}{I^2} \arrow[r] & \cdots 
\end{tikzcd}
\end{center}
remains exact because we have done nothing by applying the identity functor. Therefore, the derived functor of $\Homover{G}{\Z}{-}$ is trivial because it is the cohomology of an exact sequence.  
\end{enumerate}

\end{document}