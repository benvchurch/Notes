\documentclass[12pt]{extarticle}
\usepackage[utf8]{inputenc}
\usepackage[english]{babel}
\usepackage[a4paper, total={7in, 9.5in}]{geometry}
 
\usepackage{amsthm, amssymb, amsmath, centernot}
\usepackage{mathtools}
\DeclarePairedDelimiter{\floor}{\lfloor}{\rfloor}

\newcommand{\notimplies}{%
  \mathrel{{\ooalign{\hidewidth$\not\phantom{=}$\hidewidth\cr$\implies$}}}}
 
\renewcommand\qedsymbol{$\square$}
\newcommand{\cont}{$\boxtimes$}
\newcommand{\divides}{\mid}
\newcommand{\ndivides}{\centernot \mid}
\newcommand{\Z}{\mathbb{Z}}
\newcommand{\N}{\mathbb{N}}
\newcommand{\C}{\mathbb{C}}
\newcommand{\Zplus}{\mathbb{Z}^{+}}
\newcommand{\Primes}{\mathbb{P}}
\newcommand{\ball}[2]{B_{#1} \! \left(#2 \right)}
\newcommand{\Q}{\mathbb{Q}}
\newcommand{\R}{\mathbb{R}}
\newcommand{\Rplus}{\mathbb{R}^+}
\newcommand{\invI}[2]{#1^{-1} \left( #2 \right)}
\newcommand{\End}[1]{\text{End}\left( A \right)}
\newcommand{\legsym}[2]{\left(\frac{#1}{#2} \right)}
\renewcommand{\mod}[3]{\: #1 \equiv #2 \: (\mathrm{mod} \: #3) \:}
\newcommand{\nmod}[3]{\: #1 \centernot \equiv #2 \: (\mathrm{mod} \: #3) \:}
\newcommand{\ndiv}{\hspace{-4pt}\not \divides \hspace{2pt}}
\newcommand{\finfield}[1]{\mathbb{F}_{#1}}
\newcommand{\finunits}[1]{\mathbb{F}_{#1}^{\times}}
\newcommand{\ord}[1]{\mathrm{ord}\! \left(#1 \right)}
\newcommand{\quadfield}[1]{\Q \small(\sqrt{#1} \small)}
\newcommand{\vspan}[1]{\mathrm{span}\! \left\{#1 \right\}}
\newcommand{\galgroup}[1]{Gal \small(#1 \small)}
\newcommand{\ints}[1]{\mathcal{O}_{#1}}
\newcommand{\sm}{\! \setminus \!}
\newcommand{\norm}[3]{\mathrm{N}^{#1}_{#2}\left(#3\right)}
\newcommand{\qnorm}[2]{\mathrm{N}^{#1}_{\Q}\left(#2\right)}
\newcommand{\quadint}[3]{#1 + #2 \sqrt{#3}}
\newcommand{\pideal}{\mathfrak{p}}
\newcommand{\inorm}[1]{\mathrm{N}(#1)}
\newcommand{\tr}[1]{\mathrm{Tr} \! \left(#1\right)}
\newcommand{\delt}{\frac{1 + \sqrt{d}}{2}}
\renewcommand{\Im}[1]{\mathrm{Im}(#1)}
\newcommand{\modring}[1]{\Z / #1 \Z}
\newcommand{\modunits}[1]{(\modring{#1})^\times}
\renewcommand{\empty}{\varnothing}
\renewcommand{\d}[1]{\mathrm{d}#1}
\newcommand{\deriv}[2]{\frac{\d{#1}}{\d{#2}}}
\newcommand{\pderiv}[2]{\frac{\partial{#1}}{\partial{#2}}}
\newcommand{\parsq}[2]{\frac{\partial^2{#1}}{\partial{#2}^2}}

\newcommand{\atitle}[1]{\title{% 
	\large \textbf{Mathematics W4043 Algebraic Number Theory
	\\ Assignment \# #1} \vspace{-2ex}}
\author{Benjamin Church \\ \textit{Worked With Matthew Lerner-Brecher} }
\maketitle}

 
\newtheorem{theorem}{Theorem}[section]
\newtheorem{lemma}[theorem]{Lemma}
\newtheorem{proposition}[theorem]{Proposition}
\newtheorem{corollary}[theorem]{Corollary}


\begin{document}
\atitle{8}
 
\begin{enumerate}
\item 
\begin{enumerate}
\item Because $\modunits{p}$ is cyclic, take a generator $g \in \modunits{p}$ then for any $a \in \modunits{p}$ we have that $a = g^n$ for some $n$ so $\chi(a) = \chi(g)^n$. Thus for any Dirichlet character, $\chi$ is determined by $\chi(g)$ because $\chi(0) = 0$. However, $\chi(g)$ is a $(p-1)$-st root of unity in $\C$ so there are at most $p-1$ possible values of $\chi(g)$ and thus at most $p-1$ characters.  

\item Take $\chi_1, \chi_2 \in X(p)$ and define $\chi_1 \cdot \chi_2$ to be the Dirichlet character $\chi_1 \cdot \chi_2 : a \mapsto \chi_1(a) \chi_2(b)$. This is a character because, 

\[\chi_1 \cdot \chi_2 : ab \mapsto \chi_1(ab) \chi_2(ab) = \chi_1(a) \chi_2(a) \chi_1(b) \chi_2(b) = (\chi_1 \cdot \chi_2)(a) (\chi_1 \cdot \chi_2)(b)\]
and since $(\chi_1 \cdot \chi_2)(a) \mapsto 0$ if and only if $\chi_1(a) = 0$ or $\chi_2(a) = 0$ if and only if $(a, p) \neq 1$. Furthermore, this operation is assoicative and commutative by properties of complex multiplication. For any $\chi \in X(p)$, the character $\chi \cdot \chi_0 = \chi$ because if $(a, p) = 1$ then $(\chi \cdot \chi_0)(a) = \chi(a) \chi_0(a) = \chi(a)$ and if $(a, p) \neq 1$ then $\chi(a) = 0$ and so $(\chi \cdot \chi_0)(a) = 0$. Also, consider the character $\bar{\chi} : a \mapsto \overline{\chi(a)}$ which is a character because $z \mapsto \bar{z}$ is an automorphism of $\C$. Futhermore, if $(a, p) = 1$ then $(\chi \cdot \bar{\chi})(a) = \chi(a) \overline{\chi(a)} = 1$ because $\chi(a)$ is a root of unity in $\C$ and therefore lies on the unit circle. If $(a, p) \neq 1$ then $(\chi \cdot \bar{\chi})(a) = \chi(a) \overline{\chi(a)} = 0$ so $\chi \cdot \bar{\chi} = \chi_0$. Thus, $X(p)$ contains an identity and inverses. 

\item Let $g \in \modunits{p}$ be a generator and define $\lambda : \modring{p} \to \C$ by,
\[\lambda(g^k) = e^{\frac{2 \pi i k}{p-1}} \quad \quad \lambda(0) = 0\]
Suppose that $a,b \in \modunits{p}$ then $ab, \modunits{p}$ so we can write $a = g^n$ and $b = g^m$ so $ab = g{n+m}$ and thus, 
\[\lambda(ab) = e^{\frac{2 \pi i (n+m)}{p-1}} =  e^{\frac{2 \pi i n}{p-1}} e^{\frac{2 \pi i m}{p-1}} = \lambda(a) \lambda(b)\]
Furthermore, if $a = 0$ or $b = 0$ then $ab = 0$ so $\lambda(ab) = 0 = \lambda(a) \lambda(b)$. Thus for any $a,b \in \modring : \lambda(ab) = \lambda(a) \lambda(b)$. Furthermore, if $\mod{a}{b}{p}$ then if $p \divides a$ then $p \divides b$ so $\lambda(a) = 0 \iff \lambda(b) = 0$. If the residue class is nonzero, then $\mod{a}{g^n}{p}$ and $\mod{b}{g^m}{p}$ so $p \divides g^{n} - g^m = g^{n} ( g^{n-m} - 1)$ so $\mod{g^{n-m}}{1}$ and therefore, because $g$ is a generator, $p - 1 \divides n - m$ and thus, 
\[\lambda(a) = e^{\frac{2 \pi i n}{p - 1}} = e^{\frac{2 \pi i (m + (p - 1)k)}{p - 1}} = e^{\frac{2 \pi i m}{p - 1}} e^{2 \pi i k} = e^{\frac{2 \pi i m}{p - 1}} = \lambda(b)\] 
By definition, $\lambda(a) = 0$ if and only if $a \notin \modunits{p}$ if and only if $(a, p) \neq 1$. Thus, $\lambda \in X(p)$. Suppose that $\lambda^n = \chi_0$ then in particular, $g \in \modunits{p}$ so $\lambda^n(g) = 1$ and thus, $(e^{\frac{2 \pi i}{p-1}})^n = 1$ which holds when $n = p - 1$ but if $n < p - 1$ then $\lambda^n(g) = e^{2 \pi i x}$ for $0 < x < 1$ which cannot equal $1$. Thus, $\ord{\lambda} = p - 1$ and there are exactly $p - 1$ elements of $X(p)$ so $\lambda$ generates the group. 

\item Write $\lambda(g^k) = \zeta_{p-1}^k$ and $\lambda(0) = 0$. Let $a \in \modunits{p}$ and $a \neq 1$ then $a = g^k$ for $k < p - 1$ and thus, $\lambda(g) = \zeta_{p-1}^k \neq 1$ because $\zeta_{p-1}$ is primitive. 

\end{enumerate}

\item Let $a \in \modunits{p}$ and $a \neq 1$. Because $X(p)$ is generated by $\lambda$, 
\[ \sum\limits_{\chi \in X(p)} \chi(a) = \sum\limits_{n = 0}^{p - 2} \lambda^n(a)\]
Write $a = g^k$ then plugging in for the action of $\lambda$, 
\[\sum\limits_{\chi \in X(p)} \chi(a) = \sum\limits_{n = 0}^{p - 2} (\zeta_{p-1}^{k})^n = \frac{(\zeta_{p-1}^{k})^{p-1} - 1}{\zeta_{p-1}^{k} - 1}\]
However, $\zeta_{p-1}^{k}$ is a $(p-1)$-st root of unity and therefore a root of the polynomial $X^{p-1} - 1$. Furthermore, $a \neq 0$ so $\lambda(a) = \zeta_{p-1}^{k} \neq 1$ Thus, $\zeta_{p-1}^{k}$ is a root of  $X^{p-1} - 1$ but not of $X - 1$ and therefore, $\zeta_{p-1}^{k}$ is a root of the polynomial,
\[\frac{X^{p-1} - 1}{X - 1}\] 
so,
\[\sum\limits_{\chi \in X(p)} \chi(a) = \frac{(\zeta_{p-1}^{k})^{p-1} - 1}{\zeta_{p-1}^{k} - 1} = 0\] 

\item Let $d \divides p - 1$. Because $X(p)$ is a cylic it is abelian so by Lemma \ref{abeliansubgroup}, $X(p)$ contains a subgroup $H$ of all $\chi \in X(p)$ such that $\chi^d = \chi_0$. Furthermore, since $X(p)$ is cyclic, $H$ is also cyclic. Also, $\kappa = \lambda^{\frac{p-1}{d}}$ has order $d$ so $\kappa \in H$ and it has the maximum order because every elemet of $H$ satisifies $\chi^d = \chi_0$ so $\kappa$ generates $H$ which thus must have order $d$. 

\item

\begin{enumerate}
\item For $n > 0$, take $Q_n(X_1, \cdots, X_n) = X_1^2 + \cdots + X_n^2$ then if $Q_n(a_1, \cdots, a_n) = 0$ in $\Z$ we must have each $a_1 = 0$ because every term is positive. 

\item Let $n \ge 3$ and $p$ be prime. Let $Q$ be a quadratic form in $n$ variables with coeficients in $\Z$. Because $Q$ is quadratic, $\deg{Q} = 2 < n$ so we may apply Chevalley-Warning to conclude that the number of solutions to $Q(x_1, \cdots, x_n) = 0$ in $\finfield{p}$ or equivalently, to $\mod{Q(x_1, \cdots, x_n)}{0}{p}$ with solutions equal modulo $p$, is divisible by $p$. However, $Q$ is homogeneous order $2$ so $Q(0, \cdots, 0) = 0$ and thus, the number of solutions is non-zero and thus must be at least $p$. Therefore, there is a solution distinct modulo $p$ from $(0, \cdot, 0)$ which must have the form $(a_1, \cdots, a_n)$ with not every $\mod{a_i}{0}{p}$ i.e. not every $a_i \in \Z$ divisible by $p$.     

\item We want to prove that for any quadratic form $Q(x,y) = a^2 + bxy + cy^2$ the congruence $\mod{Q(x,y)}{m}{p}$ has a solution for any integer $m \in \Z$ such that $p \ndivides a$. \\ 

I claim this proposition only holds under the assumption that $\nmod{\Delta = b^2 - 4ac}{0}{p}$. For example, $\mod{x^2 + 2xy + y^2}{2}{3}$ has no solutions becaue $x^2 + 2xy + y^2 = (x + y)^2$ is a square but $2$ is not. This is because $b^2 - 4ac = 0$ which is divisible by $p$. 

Under this assumption, the proof goes as follows. Consider the quadratic form in three variables, $\tilde{Q}(x,y,z) = a x^2 + bxy + cy^2 - mz^2$. Consider, $\mod{\tilde{Q}(x,y,z)}{0}{p}$. Now, we want to show that this congruence has a soution with nonzero $z$ in $\finfield{p}$. Suppose that $(x,y,0)$ is a solution, then, $\mod{\Q(x,y)}{0}{p}$. \\

First, consider the case that $p \divides a$. Then, $\mod{bxy + c y^2}{0}{p}$. The solutions are $(0, 0, 0)$ and $(-b^{-1}cy, y, 0)$ for any $y \in \finfield{p}$ because $\nmod{b^2 - 4ac}{0}{p}$ and $p \divides a$ implies that $p \ndivides b$ and thus $b^{-1}$ exists modulo $p$. Therefore, there are $p + 1$ solutions. \\

In the case that $p \ndivides a$, if $y = 0$ then $\mod{ax^2}{0}{p}$ and $p \ndivides a$ so $x = 0$. This is one solution, $(0, 0, 0)$. If $y \neq 0$ then let $\mod{z}{xy^{-1}}{p}$ then $\mod{az^2 + bz + c}{0}{p}$ implies that $z = (2a)^{-1} \left[-b \pm \sqrt{b^2 - 4 ac} \right]$. This has two solutions when $b^2 - 4ac$ is a square modulo $p$ and no solutions otherwise. Now, $(zy, y, 0)$ is a solution. Therefore, the number of solutions is either $1$ if $b^2 - 4ac$ is not a square (only the trivial solution) or $1 + 2 (p - 1) = 2p - 1$ (two for each nonzero $y$) when $b^2 - 4ac$ is a square. \\

In every case, the number of solutions with $z = 0$ is not divisible by $p$. However, because $\deg{\tilde{Q}} = 2 < 3$, by Chevalley-Warning, the total number of solutions is divisible by $p$. Thus, there exist solutions with $z \neq 0$ to $\mod{\tilde{Q}(x,y,z)}{0}{p}$. Take such a solution $(x, y, z)$. Then, $\mod{ax^2 + bxy + cy^2 - mz^2}{0}{p}$ so let $x' = z^{-1}x$ and $y' = z^{-1}y$ where the inverses exist because $\nmod{z}{0}{p}$. Thus, $\mod{(ax'^2 + bx'y' + cy'^2 - m)z^2}{0}{p}$ but $\nmod{z}{0}{p}$ so $\mod{ax'^2 + bx'y' + cy'^2}{m}{p}$. Therefore, there exists a solution to $\mod{Q(x,y)}{m}{p}$. 
\item Let $F(X,Y,Z) = X^3 + Y^3 + Z^3 + XY^2 + YZ^2 + ZX^2 + XYZ$ which is homogeneous of order $3$. I claim that the only solution in $\finfield{2}$ to $F(a,b,c) = 0$ is $(0,0,0)$. Eqivalently, that if
\[\mod{F(a,b,c)}{0}{2}\]
for $a,b,c \in \Z$ then $2 \divides a, b, c$. We can check this property by considering the $8$ possibilities for the residues of $a,b,c$ modulo $2$.

\begin{align*}
(a, b, c) \equiv_2 (0, 0, 0) & \quad \quad F(a,b,c) \equiv_2 0 + 0 + 0 + 0 + 0 + 0 + 0 \equiv_2 0 \\
(a, b, c) \equiv_2 (1, 0, 0) & \quad \quad F(a,b,c) \equiv_2 1 + 0 + 0 + 0 + 0 + 0 + 0 \equiv_2 1 \\
(a, b, c) \equiv_2 (0, 1, 0) & \quad \quad F(a,b,c) \equiv_2 0 + 1 + 0 + 0 + 0 + 0 + 0 \equiv_2 1 \\
(a, b, c) \equiv_2 (0, 0, 1) & \quad \quad F(a,b,c) \equiv_2 0 + 0 + 1 + 0 + 0 + 0 + 0 \equiv_2 1 \\
(a, b, c) \equiv_2 (1, 1, 0) & \quad \quad F(a,b,c) \equiv_2 1 + 1 + 0 + 1 + 0 + 0 + 0 \equiv_2 1 \\
(a, b, c) \equiv_2 (0, 1, 1) & \quad \quad F(a,b,c) \equiv_2 0 + 1 + 1 + 0 + 1 + 0 + 0 \equiv_2 1 \\
(a, b, c) \equiv_2 (1, 0, 1) & \quad \quad F(a,b,c) \equiv_2 1 + 0 + 1 + 0 + 0 + 1 + 0 \equiv_2 1 \\
(a, b, c) \equiv_2 (1, 1, 1) & \quad \quad F(a,b,c) \equiv_2 1 + 1 + 1 + 1 + 1 + 1 + 1 \equiv_2 1 \\
\end{align*}
Therefore, the only solution modulo $2$ is $(0,0,0)$. 
\end{enumerate}
\end{enumerate}

\section*{Lemmas}

\begin{lemma} \label{abeliansubgroup}
Let $A$ be an abelian group. For $n \in \N$, $A_n = \{a \in A \mid a^n = e\}$ is a subgroup of $A$.
\end{lemma}
\begin{proof}
For any $n \in \N$, we have $e^n = e$ so $e \in A_n$. Also, if $a, b \in A$ then $(ab)^n = a^n b^n = e$ so $ab \in A_n$. Also, $(a^{-1})^n = (a^n)^{-1} = e^{-1} = e$ so $a^{-1} \in A_n$. Thus $A_n$ is a subgroup of $A$. 
\end{proof}

\end{document}