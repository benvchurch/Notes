\documentclass[12pt]{extarticle}
\usepackage[utf8]{inputenc}
\usepackage[english]{babel}
\usepackage[a4paper, total={7in, 9.5in}]{geometry}

\usepackage{amsthm, amssymb, amsmath, centernot}

\newcommand{\notimplies}{%
  \mathrel{{\ooalign{\hidewidth$\not\phantom{=}$\hidewidth\cr$\implies$}}}}
 
\renewcommand\qedsymbol{$\square$}
\newcommand{\cont}{$\boxtimes$}
\newcommand{\divides}{\mid}
\newcommand{\ndivides}{\centernot \mid}
\newcommand{\Z}{\mathbb{Z}}
\newcommand{\N}{\mathbb{N}}
\newcommand{\Zplus}{\mathbb{Z}^{+}}
\newcommand{\Primes}{\mathbb{P}}
\newcommand{\ball}[2]{B_{#1} \! \left(#2 \right)}
\newcommand{\Q}{\mathbb{Q}}
\newcommand{\R}{\mathbb{R}}
\newcommand{\Rplus}{\mathbb{R}^+}
\newcommand{\invI}[2]{#1^{-1} \left( #2 \right)}
\newcommand{\legsym}[2]{\left(\frac{#1}{#2} \right)}
\renewcommand{\mod}[3]{\: #1 \equiv #2 \: \mathrm{mod} \: #3 \:}
\newcommand{\nmod}[3]{\: #1 \centernot \equiv #2 \: mod \: #3 \:}
\newcommand{\ndiv}{\hspace{-4pt}\not \divides \hspace{2pt}}
\newcommand{\finfield}[1]{\mathbb{F}_{#1}}
\newcommand{\finunits}[1]{\mathbb{F}_{#1}^{\times}}
\newcommand{\ord}[1]{\mathrm{ord}\! \left(#1 \right)}
\newcommand{\quadfield}[1]{\Q \small(\sqrt{#1} \small)}
\newcommand{\vspan}[1]{\mathrm{span}\! \left\{#1 \right\}}
\newcommand{\galgroup}[1]{Gal \small(#1 \small)}


\newcommand{\atitle}[1]{\title{% 
	\large \textbf{Mathematics W4043 Algebraic Number Theory
	\\ Assignment \# #1} \vspace{-2ex}}
\author{Benjamin Church }
\maketitle}

 
\newtheorem{theorem}{Theorem}[section]
\newtheorem{lemma}[theorem]{Lemma}
\newtheorem{proposition}[theorem]{Proposition}
\newtheorem{corollary}[theorem]{Corollary}
\begin{document}
\atitle{1}
 
\begin{enumerate}
\item
\begin{enumerate}
\item By Euler's Criterion: $\mod{\legsym{7}{13}}{7^{\frac{13-1}{2}}}{13}$ but $7^{\frac{13-1}{2}} = 7^6 = \mod{49^3}{(-3)^3}{13}$ and $(-3)^3 = \mod{-27}{-1}{13}$ so $\legsym{7}{13} = -1$

\item By Euler's Criterion: $\mod{\legsym{13}{7}}{13^{\frac{7-1}{2}}}{7}$ but $13^{\frac{7-1}{2}} = \mod{13^3}{-1}{7}$ so $\legsym{13}{7} = -1$. \\ 

Thus, $\legsym{7}{13} \legsym{13}{7} = 1 = (-1)^{\frac{13-1}{2} \frac{7-1}{2}}$ since $6 \cdot 3$ is even. 

\item By Euler's Criterion: $\mod{\legsym{23}{19}}{23^{\frac{19-1}{2}}}{19}$ but $23^{\frac{19-1}{2}} = \mod{23^9}{4^9}{19}$ and $4^9 = \mod{64^3}{7^3}{19}$ and $\mod{7^3 = 49 \cdot 7}{11 \cdot 7}{19}$ and $11 \cdot 7 = \mod{77}{1}{19}$ so $\legsym{23}{19} = 1$. 

\item By Euler's Criterion: $\mod{\legsym{19}{23}}{19^{\frac{23-1}{2}}}{23}$ but $19^{\frac{23-1}{2}} = \mod{19^11}{19 \cdot 16^5}{23}$ and $19 \cdot 16^5 = \mod{19 \cdot 16 \cdot 16^4}{19 \cdot 16 \cdot 3^2}{23}$ and $19 \cdot 16 \cdot 9 = \mod{2736}{-1}{23}$. Therefore, $\legsym{23}{19} = -1$. \\ 

Thus, $\legsym{23}{19} \legsym{19}{23} = -1 = (-1)^{\frac{23-1}{2} \frac{19-1}{2}}$ since $11 \cdot 9$ is odd. 

\end{enumerate}

\item 
\begin{enumerate}
\item For $g \in \finunits{17}$, $\ord{g} \mid 17-1 = 16$ so $\ord{g} = 2^k$ for some $k \in \{0, 1, 2, 3, 4\}$. If $g$ is not a primitive root i.e. $\ord{g} \neq 2^4$ then $\ord{g} = 2^k$ for $k \leq 3$ so $g^{\frac{17-1}{2}} = g^{\ord{g} 2^{3-k}} = (g^{\ord{g}})^{2^{3-k}} = 1$. Therefore, by Euler's Criterion, if $\legsym{g}{17} = -1$ then $g$ is a primitive root. \\ \\
Now guess $g = 3$. $\legsym{3}{17} \legsym{17}{3} = (-1)^{\frac{17-1}{2} \frac{3-1}{2}} = (-1)^8 = 1$ but $\legsym{17}{3} = \legsym{2}{3} = -1$ so $\legsym{3}{17} = -1$ so $3$ is a primitive root of $\finunits{17}$. \\ \\

\item For $g \in \finunits{23}$, $\ord{g} \mid 23-1 = 22$ so $\ord{g} \in \{1, 2, 11, 22\}$. The only element of order $1$ is $1$ and by uniqueness of subgroups of cyclic groups, $\left<-1\right>$ is the unique subgroup of order $2$ and therefore, the only element of order $2$ is $-1$. Any other non-primitive root must have order $11$. Thus, if $g \in \finunits{23} \setminus \{1, -1\}$ is not a primitive root then $g^{11} = 1$ but $11 = \frac{23-1}{2}$ so by Euler's Criterion $\legsym{g}{23} = 1$. Thus, if $\legsym{g}{13} = -1$ and $g \neq \pm 1$ then $g$ is a primitive root of $\finunits{23}$. \\ \\
Guess $g = 3$. $\legsym{3}{23} \legsym{23}{3} = (-1)^{\frac{23-1}{2} \frac{3-1}{2}} = (-1)^{11} = -1$ but $\legsym{23}{3} = \mod{\legsym{2}{3}}{2^1}{3}$. Thus, $\legsym{2}{3} = -1$ so $\legsym{3}{23} = 1$ try again! \\ \\
Guess $g = 5$. $\legsym{5}{23} \legsym{23}{5} = (-1)^{\frac{23-1}{2} \frac{5-1}{2}} = (-1)^{22} = 1$ but $\legsym{23}{5} = \mod{\legsym{3}{5}}{2^2}{5}$. Thus, $\legsym{23}{5} = -1$ so $\legsym{5}{23} = -1$ thus $5$ is a primitive root in $\finunits{23}$.
\end{enumerate}

\item 

\begin{enumerate}
\item $\quadfield{d} = \{a + b \sqrt{d} \mid a,b \in \Q \}$. By construction, $\{1, \sqrt{d}\}$ spans $\quadfield{d}$ and if $a + b\sqrt{d} = 0$ then $\sqrt{d} = -\frac{a}{b}$ which contradicts $d$ being a non-square in $\Q$. Thus, $[\quadfield{d} : \Q] = 2$ because $\{1, \sqrt{d}\}$ is a basis for $\quadfield{d}$ over $\Q$. \\ \\
Now, let $K$ be any quadratic extension of $\Q$ then take $x \in K \setminus \vspan{1}$. Then $|\{1, x\}| = 2$ and is independent so it is a basis. Thus, $\exists a,b \in \Q : x^2 = a + bx$ so $x = \frac{1}{2} (b \pm \sqrt{b^2 + 4a})$. Since $a, b \in \Q$, $b^2 + 4 a = \frac{p}{q} \in \Q$. Since $x = \frac{b}{2} \cdot 1 + (\pm \frac{1}{2q}) \sqrt{pq}$, both  $x$ and $1$ can be reexpressed in the new basis: $\{1, \sqrt{b^2-4ac} \}$ therefore, $K = \quadfield{pq}$ with $pq \in \Z$.  \\ \\
Let $d \in \Z$ be a non-square so $x^2 - d$ is irreducible over $\Q$. Since $\pm \sqrt{d} \in \quadfield{d}$, $x^2 - d$ splits over $\quadfield{d}$. But $\quadfield{d}$ is an extension of $\Q$ of order $2$ so it is the minimal field over which $x^2 - d$ splits. Thus $\quadfield{d}$ is the splitting field of $x^2 - d$ and therefore $\quadfield{d}/ \Q$ is a Galois extension.  \\ \\
Since $|\galgroup{\quadfield{d}/\Q}| = [\quadfield{d} : \Q] = 2$ then $\galgroup{\quadfield{d}/\Q} \cong \Z / 2\Z$. Besides the identity automorphism, there is $\sigma \in \galgroup{\quadfield{d}/\Q}$ given by $\sigma : a + b\sqrt{d} \mapsto a - b\sqrt{d}$ and $\sigma^2 = \mathrm{id}$.   

\item Let $d/d' = q^2 \in \Q$ be a square. $\quadfield{d} = \{a + b\sqrt{d} \mid a,b \in \Q \} = \\ \{ a + b \sqrt{q^2 d'} \mid a, b \in \Q\} = \{ a + bq \sqrt{d'} \mid a, b \in \Q\} = \quadfield{d'}$ \\ \\
Let $\quadfield{d} = \quadfield{d'}$ then for any $a,b \in \Q$, $a + b\sqrt{d} \in \quadfield{d} = \quadfield{d'}$. $\sqrt{d} \in \quadfield{d} = \quadfield{d'}$ so $\exists a,b \in \Q : \sqrt{d} = a+b\sqrt{d'}$. Then, $d = a^2 + 2ab\sqrt{d'} + b^2 d'$ so if $a \neq 0$ then $\sqrt{d'} = \frac{d - a^2 - b^2 d'}{2ab}$ which is impossible since $\sqrt{d'} \notin \Q$. Thus $a = 0$ and therefore, $\sqrt{d} = b \sqrt{d'}$ so $d/d' = b^2$ with $b \in \Q$. \\ \\
If $K$ is a quadratic extension of $\Q$ then $\exists q \in \Z$ s.t. $K = \quadfield{q}$ then write the prime factorization, $q = p_1^{a_1} \dots p_k^{a_k}$. Let $q' = p_1^{\tilde{a}_i} \dots p_k^{\tilde{a}_i}$ where $\tilde{a}_i = a_i \: \textrm{mod} \: 2$ so $q/q' = p_1^{a_i - \tilde{a}_i} \dots p_k^{a_i - \tilde{a}_i}$ which is a square since \\ $\mod{a_i}{\tilde{a}_i}{2}$ therefore, $\quadfield{q} = \quadfield{q'}$ so every quadratic extension is $\quadfield{p_1 \dots p_k}$ with distinct primes $p_i$ since every $\tilde{a}_i = 0,1$.   
\\
\item Let $P(x) = ax^2 + bx + c \in Z[x]$ be irreducible over $\Q$. Then $r_{\pm} = \frac{1}{2}(-b \pm \sqrt{b^2 - 4ac})$ are the roots of $P$. Let $\Delta = b^2 - 4ac$ then $r_{\pm} \in \quadfield{\Delta}$ so $P$ splits over $\quadfield{\Delta}$. Since $\quadfield{\Delta}$ is a quadratic extension over $\Q$, there are no proper subfields besides $\Q$ thus $\quadfield{\Delta}$ is the splitting field of $P$. Also,
$\mod{\Delta}{b^2}{4}$ so $\Delta$ is a quadratic residue modulo $4$. Thus, $\mod{\Delta}{0, 1}{4}$. 

\item Let $d \in \Z$ be a square-free integer. If $\mod{d}{1}{4}$ then for any $b$ s.t. $2 \ndivides b$ we have that $\mod{b^2}{1}{4}$ so $\mod{d - b^2}{0}{4}$ then $d = b^2 - 4c$ for $c \in \Z$. Take $Q(x) = x^2 + bx + c$ then $\Delta = b^2 - 4c = d$ so $\quadfield{d}$ is the splitting field of $Q$. In particular, let $b = 1$ then $c = (1-d)/4$ so $Q(x) = x^2 + x + \frac{1-d}{4}$ \\ \\
If $\nmod{d}{1}{4}$ then take $Q(x) = x^2 - d$ so $\Delta = 4d$ and by above $\quadfield{d}$ is the splitting field of $Q$. 

\end{enumerate}

\end{enumerate}

\end{document}