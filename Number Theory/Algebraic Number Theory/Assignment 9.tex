\documentclass[12pt]{extarticle}
\usepackage[utf8]{inputenc}
\usepackage[english]{babel}
\usepackage[a4paper, total={7in, 9.5in}]{geometry}
 
\usepackage{amsthm, amssymb, amsmath, centernot}
\usepackage{mathtools}
\DeclarePairedDelimiter{\floor}{\lfloor}{\rfloor}

\newcommand{\notimplies}{%
  \mathrel{{\ooalign{\hidewidth$\not\phantom{=}$\hidewidth\cr$\implies$}}}}
 
\renewcommand\qedsymbol{$\square$}
\newcommand{\cont}{$\boxtimes$}
\newcommand{\divides}{\mid}
\newcommand{\ndivides}{\centernot \mid}
\newcommand{\Z}{\mathbb{Z}}
\newcommand{\N}{\mathbb{N}}
\newcommand{\C}{\mathbb{C}}
\newcommand{\Zplus}{\mathbb{Z}^{+}}
\newcommand{\Primes}{\mathbb{P}}
\newcommand{\ball}[2]{B_{#1} \! \left(#2 \right)}
\newcommand{\Q}{\mathbb{Q}}
\newcommand{\R}{\mathbb{R}}
\newcommand{\Rplus}{\mathbb{R}^+}
\newcommand{\invI}[2]{#1^{-1} \left( #2 \right)}
\newcommand{\End}[1]{\text{End}\left( A \right)}
\newcommand{\legsym}[2]{\left(\frac{#1}{#2} \right)}
\renewcommand{\mod}[3]{\: #1 \equiv #2 \: (\mathrm{mod} \: #3) \:}
\newcommand{\nmod}[3]{\: #1 \centernot \equiv #2 \: (\mathrm{mod} \: #3) \:}
\newcommand{\ndiv}{\hspace{-4pt}\not \divides \hspace{2pt}}
\newcommand{\finfield}[1]{\mathbb{F}_{#1}}
\newcommand{\finunits}[1]{\mathbb{F}_{#1}^{\times}}
\newcommand{\ord}[1]{\mathrm{ord}\! \left(#1 \right)}
\newcommand{\quadfield}[1]{\Q \small(\sqrt{#1} \small)}
\newcommand{\vspan}[1]{\mathrm{span}\! \left\{#1 \right\}}
\newcommand{\galgroup}[1]{Gal \small(#1 \small)}
\newcommand{\ints}[1]{\mathcal{O}_{#1}}
\newcommand{\sm}{\! \setminus \!}
\newcommand{\norm}[3]{\mathrm{N}^{#1}_{#2}\left(#3\right)}
\newcommand{\qnorm}[2]{\mathrm{N}^{#1}_{\Q}\left(#2\right)}
\newcommand{\quadint}[3]{#1 + #2 \sqrt{#3}}
\newcommand{\pideal}{\mathfrak{p}}
\newcommand{\inorm}[1]{\mathrm{N}(#1)}
\newcommand{\tr}[1]{\mathrm{Tr} \! \left(#1\right)}
\newcommand{\delt}{\frac{1 + \sqrt{d}}{2}}
\renewcommand{\Im}[1]{\mathrm{Im}(#1)}
\newcommand{\modring}[1]{\Z / #1 \Z}
\newcommand{\modunits}[1]{(\modring{#1})^\times}
\renewcommand{\empty}{\varnothing}
\renewcommand{\d}[1]{\mathrm{d}#1}
\newcommand{\deriv}[2]{\frac{\d{#1}}{\d{#2}}}
\newcommand{\pderiv}[2]{\frac{\partial{#1}}{\partial{#2}}}
\newcommand{\parsq}[2]{\frac{\partial^2{#1}}{\partial{#2}^2}}

\newcommand{\atitle}[1]{\title{% 
	\large \textbf{Mathematics W4043 Algebraic Number Theory
	\\ Assignment \# #1} \vspace{-2ex}}
\author{Benjamin Church \\ \textit{Worked With Matthew Lerner-Brecher} }
\maketitle}

 
\newtheorem{theorem}{Theorem}[section]
\newtheorem{lemma}[theorem]{Lemma}
\newtheorem{proposition}[theorem]{Proposition}
\newtheorem{corollary}[theorem]{Corollary}


\begin{document}
\atitle{1}
 
\begin{enumerate}
\item Take $x, y \in \Q$ then write $x = p^{v_p(x)} \cdot \frac{a_1}{b_1}$ and $y = p^{v_p(x)} \cdot \frac{a_2}{b_2}$ where $a_1, b_1, a_2, b_2$ are all relativly prime to $p$ and $v_p : \Q \to \N$ is the $p$-adic valuation. First, if $x = 0$ then by definition $|x|_p = 0$. Otherwise, $a_1$ and $b_1$ are well defined and thus, $|x|_p = p^{-v_p(x)} \neq 0$ because $p \neq 0$. Now, $|xy|_p = |p^{v_p(x) + v_p(y)} \cdot \frac{a_1 a_2}{b_1 b_2}|_p$. However, because $p$ does not divide $a_1$ or $a_2$ we have $p \ndivides a_1 a_2$ ad similarly $p \ndivides b_1 b_2$. Thus, $v_p(xy) = v_p(x) + v_p(y)$ so, \[|xy|_p = p^{-v_p(xy)} = p^{-v_p(x)} \cdot p^{-v_p(y)} = |x|_p |y|_p\]
Finally, because $v_p(x), v_p(y) \ge \min\{v_p(x), v_p(y)\}$ we can write, \[x + y = p^{\min\{v_p(x), v_p(y)\}} \left(p^x \cdot \frac{a_1}{b_1} + p^y \cdot \frac{a_2}{b_2} \right)\] where $x$ and $y$ are nonnegative. Thus, $v_p(x + y) \ge \min\{v_p(x), v_p(y)\}$ because $\left(p^x \cdot \frac{a_1}{b_1} + p^y \cdot \frac{a_2}{b_2} \right)$ can only contain positive powers of $p$. Therefore, 
\[|x + y|_p \le p^{-\min\{v_p(x), v_p(y)\}} = \max\{p^{-v_p(x)}, p^{-v_p(y)}\} = \max\{|x|_p, |y|_p\}\]
where I have used the fact that $-\min\{a,b\} = \max\{-a, -b\}$. Thus, $| \bullet |_p$ is a norm. 

\item 

\begin{enumerate}
\item Given a sequence satisfying $a_i \in \Q$ and $\lim\limits_{i \to \infty} |a_i|_p = 0$ then the norm of the series is \[ \left| \sum\limits_{i = 0}^\infty a_i \right|_p = \lim\limits_{n \to \infty} \left| \sum\limits_{i = 0}^n a_i \right|_p \]
Consider the mapping $f : \Q \to \Q$ given by $f(x) = |x|_p$. The norm, is the limit of a sequence of elements pf the image of $f$. Therefore, by Lemma \ref{seqs}, the limit must be contained in the closure of the image. Since $\Im{f} = \{p^r \mid r \in \Z\}$ we have that the norm of any series is an element of $\overline{\Im{f}} = \{p^r \mid r \in \Z\} \cup \{0 \}$ because there is finite seperation between powers of $p$ but there are also arbitrarily small powers of $p$. It remains to show that this limit exists. Because $\lim\limits_{i \to \infty} |a_i|_p = 0$ we have for any $\epsilon > 0$ there exists $k$ such that $n > k \implies |a_i|_p < \epsilon$. By the ultrametric property, fot $n, m > k$,
\[ \left| \sum\limits_{i = n}^m a_i \right|_p \le \max\{|a_n|, |a_{n+1}|, \cdots, |a_m| \} < \epsilon \]
Furthermore,
\[ \left| \left| \sum\limits_{i = 0}^m a_i \right|_p - \left| \sum\limits_{i = 0}^n a_i \right|_p \right| = \left| \left| \sum\limits_{i = 0}^{n-1} a_i + \sum\limits_{i = n}^n a_i \right|_p - \left| \sum\limits_{i = 0}^m a_i \right|_p \right|  \le \left| \sum\limits_{i = n}^m a_i \right|_p  < \epsilon \]
thus, the sequence of norms is Cauchy so the limit in $\R$ exists. 

\item Suppose that the series $\sum\limits_{i = 0}^n a_i$ and $\sum\limits_{i = 0}^n b_i$ are equivalent. Then,
\[ \lim\limits_{n \to \infty} \left| \sum\limits_{i = 0}^n a_i - \sum\limits_{i = 0}^n b_i \right|_p = 0 \]
Now suppose that $\left| \sum\limits_{i = 0}^\infty a_i \right|_p = L$. Therefore, for any $\epsilon > 0$ there exist $k_1, k_2 \in \N$ such that, 
\[n > k_1 \implies \left| \sum\limits_{i = 0}^n a_i - \sum\limits_{i = 0}^n b_i \right|_p  < \frac{\epsilon}{2}\]
and likewise, 
\[n > k_2 \implies \left| \left| \sum\limits_{i = 0}^n a_i \right|_p - L \right| < \frac{\epsilon}{2}\]  
Now, we can write, when $n > k_1$, 
\[   - \frac{\epsilon}{2} < \left| \sum\limits_{i = 0}^n b_i \right|_p - \left| \sum\limits_{i = 0}^n a_i \right|_p <  \frac{\epsilon}{2} \]
Therefore, for $n > \max\{k_1, k_2\}$,
\[   - \epsilon < \left| \sum\limits_{i = 0}^n b_i \right|_p - \left| \sum\limits_{i = 0}^n a_i \right|_p + \left| \sum\limits_{i = 0}^n a_i \right|_p - L <  \epsilon \]
or rearranging,
\[ \left| \left| \sum\limits_{i = 0}^n b_i \right|_p - L \right| < \epsilon \]  
Thus,
\[ \left| \sum\limits_{i = 0}^\infty b_i \right|_p = \lim\limits_{n \to \infty} \left| \sum\limits_{i = 0}^n b_i \right|_p = L =  \left| \sum\limits_{i = 0}^\infty a_i \right|_p\]
\end{enumerate}

\item

\begin{enumerate}
\item Take $x = \frac{a}{b} \in \Q$. By the fundamental theorem of arithmetic, $a$ and $b$ factor into products of primes. We can write $x = \frac{a}{b} = \pm p_1^{r_1} \cdots p_k^{r_k}$ with possibly negative powers where $p_i$ runs through the primes in both $a$ and $b$. Therefore, $|x|_{p_i} = p_i^{-r_i}$ and for any prime $p$ not in the factorization, $p \ndivides x$ so $|x|_p = p^0 = 1$. Because the factoization is finite, for all but a finite number of primes, $|x|_p = 1$. 

\item Take $a \in \Q$ with $a \neq 0$. Write the prime factoization, $a = \pm p_1^{r_1} \cdots p_k^{r_k}$ remembering that the powers may be negative. Consider the product, which by part (a) is finite,
\[ |a| \prod_{p} |a|_p = |a| \prod_{i = 1}^k |a|_{p_i} = |a| \prod_{i = 1}^k p_i^{-r_i} = \frac{|a|}{p_1^{r_1} \cdots p_r^{r_k}} = \frac{|a|}{|a|} = 1 \] 
\end{enumerate}

\item We can define the homomorphism $i : \Q \to \bf{A} = \R \otimes \prod_{p} \Q_p$ by $i : x \to (x, (x))$ where I have set $a_\R = a \in \Q \subset \R$ and for every $p$, set  $a_p = a \in \Q \subset \Q_p$. This function is well-defined because $x \in \Q$ so by the previous problem, $|x|_p = 1$ for all but a finite number of $p$ so the sequence is in the adele group. Therefore, $i(x) \in \bf{A}$. This map is injective because if $i(x) = i(y)$ then $(x, (x)) = (y, (y))$ so $x = y$. Finally, this map is a homomorphism because $i(x + y) = (x + y, (x + y)) = (x, (x)) + (y, (y)) = i(x) + i(y)$ since the sum is also a rational number and $p$-adic addition restricts to rational addition on $\Q$. 
\\\\
Next, take $x \in \Z$ then $x = \pm p_1^{r_1} \cdots p_k^{r_k}$ with positive $r_i$ so $|x|_{p_i} = p_i^{-r_i}$ and for any $p$ not in the factorization $|x|_p = 1$. Thus, $|x|_p \le 1$ for every $p$. Suppose that  $x \in \Q$ satisfies $|x|_p \le 1$ for every $p$. We can write $x = \frac{a}{b}$ and $a$ and $b$ factor into products of primes so $x = \frac{a}{b} = \pm p_1^{r_1} \cdots p_k^{r_k}$ with possibly negative powers where $p_i$ runs through the primes in both $a$ and $b$. Therefore, $|x|_{p_i} = p_i^{-r_i} \le 1$ so by hypothesis $r_i \ge 0$. Using the factorization, $x = \pm p_1^{r_1} \cdots p_k^{r_k} \in \Z$ because no power can be negative. 

\end{enumerate}

\section*{Lemmas}

\begin{lemma} \label{seqs}
Let $X$ be a metric space and $a_n$ a seqeunce in $S \subset X$. If it exists, $\lim\limits_{n \to \infty} a_n \in \overline{S}$. 
\end{lemma}
\begin{proof}
Suppose that for every $\epsilon > 0$ there exists an $k \in \N$ such that $n > k \implies d(a_n, L)  < \epsilon$. Then, $a_n \in S$ so $\ball{\epsilon}{L} \cap S \neq \empty$ for every $\epsilon > 0$ which implies that $L \in \overline{S}$. 
\end{proof}

\end{document}