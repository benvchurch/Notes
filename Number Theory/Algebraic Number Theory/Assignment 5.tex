\documentclass[12pt]{extarticle}
\usepackage[utf8]{inputenc}
\usepackage[english]{babel}
\usepackage[a4paper, total={7in, 9.5in}]{geometry}
 
\usepackage{amsthm, amssymb, amsmath, centernot}
\usepackage{mathtools}
\DeclarePairedDelimiter{\floor}{\lfloor}{\rfloor}

\newcommand{\notimplies}{%
  \mathrel{{\ooalign{\hidewidth$\not\phantom{=}$\hidewidth\cr$\implies$}}}}
 
\renewcommand\qedsymbol{$\square$}
\newcommand{\cont}{$\boxtimes$}
\newcommand{\divides}{\mid}
\newcommand{\ndivides}{\centernot \mid}
\newcommand{\Z}{\mathbb{Z}}
\newcommand{\N}{\mathbb{N}}
\newcommand{\C}{\mathbb{C}}
\newcommand{\Zplus}{\mathbb{Z}^{+}}
\newcommand{\Primes}{\mathbb{P}}
\newcommand{\ball}[2]{B_{#1} \! \left(#2 \right)}
\newcommand{\Q}{\mathbb{Q}}
\newcommand{\R}{\mathbb{R}}
\newcommand{\Rplus}{\mathbb{R}^+}
\newcommand{\invI}[2]{#1^{-1} \left( #2 \right)}
\newcommand{\End}[1]{\text{End}\left( A \right)}
\newcommand{\legsym}[2]{\left(\frac{#1}{#2} \right)}
\renewcommand{\mod}[3]{\: #1 \equiv #2 \: (\mathrm{mod} \: #3) \:}
\newcommand{\nmod}[3]{\: #1 \centernot \equiv #2 \: (\mathrm{mod} \: #3) \:}
\newcommand{\ndiv}{\hspace{-4pt}\not \divides \hspace{2pt}}
\newcommand{\finfield}[1]{\mathbb{F}_{#1}}
\newcommand{\finunits}[1]{\mathbb{F}_{#1}^{\times}}
\newcommand{\ord}[1]{\mathrm{ord}\! \left(#1 \right)}
\newcommand{\quadfield}[1]{\Q \small(\sqrt{#1} \small)}
\newcommand{\vspan}[1]{\mathrm{span}\! \left\{#1 \right\}}
\newcommand{\galgroup}[1]{Gal \small(#1 \small)}
\newcommand{\ints}[1]{\mathcal{O}_{#1}}
\newcommand{\sm}{\! \setminus \!}
\newcommand{\norm}[3]{\mathrm{N}^{#1}_{#2}\left(#3\right)}
\newcommand{\qnorm}[2]{\mathrm{N}^{#1}_{\Q}\left(#2\right)}
\newcommand{\quadint}[3]{#1 + #2 \sqrt{#3}}
\newcommand{\pideal}{\mathfrak{p}}
\newcommand{\inorm}[1]{\mathrm{N}(#1)}
\newcommand{\tr}[1]{\mathrm{Tr} \! \left(#1\right)}
\newcommand{\delt}{\frac{1 + \sqrt{d}}{2}}
\renewcommand{\Im}[1]{\mathrm{Im}(#1)}
\newcommand{\modring}[1]{\Z / #1 \Z}
\newcommand{\modunits}[1]{(\modring{#1})^\times}
\renewcommand{\empty}{\varnothing}
\renewcommand{\d}[1]{\mathrm{d}#1}
\newcommand{\deriv}[2]{\frac{\d{#1}}{\d{#2}}}
\newcommand{\pderiv}[2]{\frac{\partial{#1}}{\partial{#2}}}
\newcommand{\parsq}[2]{\frac{\partial^2{#1}}{\partial{#2}^2}}

\newcommand{\atitle}[1]{\title{% 
	\large \textbf{Mathematics W4043 Algebraic Number Theory
	\\ Assignment \# #1} \vspace{-2ex}}
\author{Benjamin Church \\ \textit{Worked With Matthew Lerner-Brecher} }
\maketitle}

 
\newtheorem{theorem}{Theorem}[section]
\newtheorem{lemma}[theorem]{Lemma}
\newtheorem{proposition}[theorem]{Proposition}
\newtheorem{corollary}[theorem]{Corollary}


\begin{document}
\atitle{5}
 
\begin{enumerate}
\item We first define two quantities, 
\[r_k(m) = \big| \{(x_1, \dots, x_n) \in \Z^k \mid x_1^2 + \cdots + x_n^2 = m \} \big| \]
and
\[N_k(m) = \big| \{(x_1, \dots, x_n) \in \N^k \mid x_1^2 + \cdots + x_n^2 = m \text{ and } x_i \text{ is odd} \} \big| \]
\begin{enumerate}
\item I can't figure this out. Please have mercy upon my soul.     

\item Any solution to $x_1^2 + x_2^2 + x_3^2 + x_4^2 = m$ in $N_4(m)$ i.e. every $x_i$ is odd and nonnegative. Thus, $\mod{x_i}{1}{4}$ so there is a bijection between the solutions in $N_4(m)$ and the pairs of solutions $(x_1, x_2)$ and $(x_3, x_)$ to $x_1^2 + x_2^2 = m_1$ and $x_3^2 + x_4^2 = m_2$ with $\mod{m_1 \equiv m_2}{2}{4}$ and $m_1 + m_2 = m$. Thus, each solution in $N_4(m)$ corresponds to a unique pair of solutions in $N_2(m_1)$ and $N_2(m_2)$ with $(m_1, m_2) \in R$. Therefore,
\[N_4(m) = \sum_{R} N_2(m_1) N_2(m_2)\]
Using the result of $(a)$, 
\[N_4(m) = \sum_{R} \sum_{a \divides m_1} \chi(a) \sum_{c \divides m_2} \chi(c) \] 
Whenever $a$ is even, the character $\chi(a) = 0$ so we can ignore all even divisors. The same holds for $c$. Since $m_1$ and $m_2$ are even we can write them as $2ab = m_1$ and $2dc = m_2$ which is possible because the divisors $a$ and $c$ are not even. Also, $\mod{m_1 \equiv m_2}{2}{4}$ so they are not divisible by $4$ and thus $a,b,c,d$ are all odd. The set of odd divisors of $m_1$ and $m_2$ for all $m_1$ and $m_2$ satisfying the required properties is therefore in bijection with the set of odd positive integers $(a,b,c,d)$ with $2ab + 2cd = m$. We call this set $S$. Thus,

\[N_4(m) = \sum_{S} \chi(a) \chi(c) \] 
For odd $x$, the character is $\chi(x) = (-1)^{\frac{x-1}{2}}$ so
\[N_4(m) = \sum_{S} (-1)^{\frac{a-1}{2}} (-1)^{\frac{c-1}{2}} \]
but $(-1)^x = (-1)^{-x}$ so
\[N_4(m) = \sum_{S} (-1)^{\frac{a-1}{2}} (-1)^{\frac{1-c}{2}} =  \sum_{S} (-1)^{\frac{a-c}{2}} \]

\item Let $a = x + y$, $b = z - t$, $c = x - y$, and $d = z + t$. We solve to get $x = \frac{1}{2}(a + c)$, $y = \frac{1}{2} (a - c)$, $z = \frac{1}{2} (b + d)$, $t = \frac{1}{2} (d - b)$. We need to show that this mapping is a bijection between $S$ and $S'$. Since we can easily invert the mapping, it must be a bijection if both directions are well defined. Take $a,b,c,d > 0$ then $|y| < x$ and $|t| < z$. We also know that $a,b,c,d$ are odd and that $2ab + 2cd = m$. Therefore,
\begin{align*}
(xz - yt) &= 4 \frac{1}{4} \left((a + c) \cdot (b + d) - (a - c) \cdot (d - b) \right) \\
&= (ab + cb + ad + cd - ad - cb + cd + ab \\ 
& = 2ab + 2cd = m
\end{align*}
But, $|y| < x$ because $a, c > 0$ and $|t| < z$ because $b,d > 0$. Also, $a,b,c,d$ are odd so $x = y + c$ implies that $x$ and $y$ have different parity and, similarly, $z = t + b$ so $z$ and $t$ have different parity. Thus, $S to S'$ is well defined. Each of these demonstrations can be applied in the opposite direction to show that $S' \to S$ is a bijection. From above, $(xz - yt) = 2ab + 2cd$ so $(xy - yt) = m \iff 2ab + 2cd = m$. Also if $|y| < x$ then $-a - c < a - c < a + c$ so $a > 0$ and $c > 0$ similarly, $|t| < z \implies b, d > 0$. Finally, if $x$ and $y$ have different parity then $a = x + y$ and $c = x - y$ are odd. Likewise for $b$ and $d$. Thus these sets are in bijection so using part (b),
\[N_4(m) = \sum_{S'} (-1)^y\]
\item Restricting the sum to only elements of $S$ in which $y = 0$, we have
\[\mathcal{N}_0 = \sum_{S', y = 0} (-1)^y =  \sum_{S', y = 0} 1 = \Big| \{(x,y,z,t) \in S \mid y = 0\} \Big| \] 
we know that $|t| < z$ and $m = 4xz$ with $x$ odd and $t$ and $z$ having different parity. There are $z$ possible values for $t$ so 
\[ \sum_{S', y = 0} 1 = \sum_{m = 4xz} z = \sum_{z \divides m/4} z \]
We can restrict $z$ to odd divisors because $\mod{m}{4}{8}$. And therefore,
\[\mathcal{N}_0 = \sum_{d \divides m} d\]
Next, $\mathcal{N}_1 = \mathcal{N}_2$ because there is a bijective correspondence between positive solutions and negative solutions given by negating every comonent. Also, $(-1)^y = (-1)^-y$ so the sums that define $\mathcal{N}_1$ and $\mathcal{N}_2$ are equal. Next, we must check that the map from $S'$ to itself is a bijection. In fact, it is its own inverse. Apply the transformation twice, $z'' = y' = z$ and $y'' = z' = y$ we must also check the $x$ and $t$ variables. For this, we need to know that the integers $u' = u$. This is true because 
\[ 2 u' - 1 < \frac{x'}{y'} < 2 u' + 1\]
but \[\frac{x'}{y'} = \frac{2uz - t}{z} = 2u - \frac{t}{z}\] 
and we know that $|t| < z$ so $|\frac{t}{z}| < 1$.
Therefore, \[2u - 1 < \frac{x'}{y'} < 2u + 1\]
so $u' = u$ thus $x'' = 2 u z' - t' = 2uy - (2uy - x) = x$ and $t'' = 2uy' - x' = 2uz - (2 u z - t) = t$. Therefore the map is a bijection because it is invertable. Because $x$ and $y$ have opposite parity and $z$ and $t$ have opposite parity we must have $4(xy - yt) = \mod{m}{4}{8}$ so $xy - yt$ is odd then $y$ and $z$ have opposite parity. Then, we can reparametrize the sum because the map is a bijection,
\[\mathcal{N}_1 = \sum_{S', y > 0} (-1)^y = \sum_{S', y' > 0} (-1)^{y'} = \sum_{S', z > 0} (-1)^{z} = - \mathcal{N}_0 \]
because $y$ and $z$ have opposite parity. Therefore, $\mathcal{N}_0 = 0$. However, it is clear from the definitions of $\mathcal{N}_0, \mathcal{N}_1, \mathcal{N}_2$ that the three possibilities ($y = 0, y > 0, y < 0$) cover all possible cases. Thus, \[N_4(m) = \mathcal{N}_0 + \mathcal{N}_1 + \mathcal{N}_2 = \mathcal{N}_0 = \sum_{d \divides m} d\]      
 
\item First, if we have a solution to $x_1^2 + x_2^2 + x_3^2 + x_4^2 = 2m$ then, using the given identity,
\[(x_1 + x_2)^2 + (x_1 - x_2)^2 + (x_3 + x_4)^2 + (x_3 - x_4)^2 = 4m\]
therefore, there is a bijective correspondence between solutions for $2m$ and for $4m$ since this mapping can be inverted. The reverse direction takes a solution for $4m$, say,
\[{x_1'}^2 + {x_2'}^2 + {x_3'}^2 + {x_4'}^2 = 4m \] then we can take 
\[(x_1' + x_2')^2 + (x_1' - x_2')^2 + (x_3' + x_4')^2 + (x_3 - x_4)^2 = 8m \] 
and thus,
\[\left(\frac{x_1' + x_2'}{2}\right)^2 + \left(\frac{x_1' - x_2'}{2} \right)^2 + \left(\frac{x_3' + x_4'}{2}\right)^2 + \left(\frac{x_3 - x_4}{2}\right)^2 = 2m \]
which are integer solutions because the terms have the same parity. Thus the map is a bijection so the number of solutions in each case is identical, $r_4(4m) = r_4(2m)$. \\ 

Let $m$ be odd. Then $\mod{4m}{4}{8}$. We prodceed by breaking up the solutions in $r_4(m)$ into two cases. Either, all $x_i$ are odd in which case, up to sign the solution is one of $N_4(4m)$ or they are all even because their sum is 4 modulo 8. In the first case we have 4 sign choices and thus a total of $16 N_4(4m)$ solutions. In the second case, because all $x_i$ are even we divide $x_1^2 + x_2^2 + x_3^2 + x_4^2 = 4m$ by $4$ to get a solution for $m$. Therefore, every solution is either one of $N_4(4m)$ or $r_4(m)$. Thus,
\[r_4(4m) = N_4(4m) + r_4(m)\] \\\\
Since $m$ is odd we have $\mod{m}{2,6,4}{8}$ and, any solution to \[x_1^2 + x_2^2 + x_3^2 + x_4^2 = 2m\] must have two even parity and two odd parity squares. There are $6$ ways to arrange these solutions. Also given any solution to \[x_1^2 + x_2^2 + x_3^2 + x_4^2 = 2m\] we can form \[(x_1 + x_2)^2 + (x_1 - x_2)^2 + (x_3 + x_4)^2 + (x_3 - x_4)^2 = 2m\] but these solutions are fixed to have the first two and last two squares having equal parity. Thus, we can only form $2$ out of the $6$ possible solutions for $2m$ given a solution to $m$. As before, this can be inverted as long as adjacent terms have the same parity. Therefore there is a bijection between the $2$ out of $6$ cases of solution for $2m$ in which the first two and last two have equal parity and all solutions for $m$. Therefore, $r_4(2m) = 3 r_4(m)$. \\\\
At last, we prove the Jacobi Four Square fomula. Take odd $m$,
\[3 r_4(m) = r_4(2m) = r_4(4m) = 16 N_4(4m) + r_4(m)\]
but we know that $N_4(m) = \sum\limits_{d \divides \frac{m}{4}} d$ so,
\[2 r_4(m) = 16 \sum\limits_{d \divides \frac{m}{4}} d\] and thus 
\[r_4(m) = 8 \sum\limits_{d \divides \frac{m}{4}} d\]  
Now we consider the case that $m$ is even. Write $m = 2m'$; if $m'$ is odd then by above, $_4(m') = 8 \sum\limits_{d \divides \frac{m'}{4}} d$ and $r_4(2m') = 3 r_4(m')$ so 
\[r_4(m) = 24 \sum\limits_{d \divides \frac{m'}{4}} d = 24 \sum\limits_{d \divides \frac{m}{2}} d\]
On the other hand, if $m'$ is even then using $r_4(4m) = r_4(2m)$ we reduce until $m'$ is odd or $2k$ with odd $k$. Then the above cases can be applied.       

\end{enumerate}
\item

\begin{enumerate}

\item As proven in class, every solution to the equation $x^2 - d y^2 = 1$ is in the form $(\pm a_n, \pm b_n)$ where $a_n + b_n \sqrt{d} = (a_1 + b_1 \sqrt{d})^n$ with $a_1 + b_1 \sqrt{d}$ being the positive solution ($a_1, b_1 > 0$) with the smallest $b_1$. Now consider the set $\Sigma \subset \Gamma$ of $u_i = a_i - b_i \sqrt{d}$ which solve $a_i^2 - d b_i^2 = \pm 1$ with $b_1 \le b_2 \le b_3 \dots$. If the negative sign Pell's equaton has no solutions, then we are done because every $x \in \Gamma$ is of the form $\pm a_n \pm b_n \sqrt{d}$ but $(a_1 + b_1 \sqrt{d})^{-n} = a_n - b_n \sqrt{d}$ since $(a_n - b_n \sqrt{d})(a_n + b_n \sqrt{d}) = a_n^2 - d b_n^2 = 1$. Also, $a_1 - b_1 \sqrt{d} = (a_1 + b_1 \sqrt{d})^{-1}$ because $(a_1 - b_1 \sqrt{d})(a_1 + b_1 \sqrt{d}) = a_1^2 - d b_1^2 = 1$. Therefore, $-1$ and $a_1 - b_1 \sqrt{d}$ generate every sign combination and thus the entire group of units. Now we suppose that the negative equation has solutions. There must be a $u_k$ with minimal $b_k$ which has norm $-1$ and thus solves the negative equation. If $u_1 \neq u_k$ then $u_1$ solves the positive equation by minimality. But $\qnorm{K}{u_k^2} = \qnorm{K}{u_k}^2 = (-1)^2 = 1$ so by the above classification, $u_k^2 = (u_1)^n$ and thus either $n$ is even or $u_1$ is a square. If $u_1 = w^2$ then $\qnorm{K}{w^2} = \qnorm{K}{w}^2 = 1$ so $\qnorm{K}{w} = \pm 1$ therefore $w$ would be a smaller solution than $u_1$ which contradicts minimality. Thus, $n$ is even so $u_k = \pm (u_1)^{n/2}$ and thus, \[\qnorm{K}{u_k} = \qnorm{K}{\pm (u_1)^{n/2}} = \qnorm{K}{u_1}^{n/2} = 1\] But by assumption, $u_k$ has norm $-1$. Thus, $u_1$ must have a negative norm and be minimal. Similarly, in this case, $a_1 - b_1 \sqrt{d} = u_1^2$ else since $u_1^2$ has norm $+1$ we would have $u_1^2 = (a_1 - b_1 \sqrt{d})^n$ so either $n$ is even or $u_1$ is a square. But $u_1 = (a_1 - b_1 \sqrt{d})^{n/2}$ is a contradiction because $u_1$ has norm $-1$ and $a_1 - b_1 \sqrt{d}$ has norm $+1$ and $u_1$ being a square would contradict its minimality. Take $r \in \Gamma$, then $r^2$ has norm $+1$ and thus $r^2 = (u_1^2)^n$ because $u_1^2$ is the minimal solution to the positive equation. Thus, $r = \pm (u_1)^n$ therefore, $\Gamma = \left< u_1, -1 \right>$.       

\item For $b \in \Zplus$ take $q^{\pm}(b) = db^2 \pm 1$ and let $b_1$ be the smallest $b$ such that either $q^+(b_1)$ or $q^-(b_1)$ is a square. Let $u_1 = a - b \sqrt{d}$ then $a^2 - d b^2 = \pm 1$ and hence $a^2 = db^2 \pm 1 = q^{\pm}(b)$ thus $b_1 \le b$ because $q^{\pm}(b)$ is a square and $b_1$ is minimal. However, if $b_1 < b$ then $q^{\pm}(b_1) = a^2$ so $a^2 - d b_1^2 = \pm 1$ so $u_1$ is not minimal thus, $b_1 = b$. Thus, $u_1 = a - b_1 \sqrt{d}$ but $a_1 = \sqrt{q^{\pm}(b_1)} = \sqrt{d b_1^2 \pm 1} = \sqrt{a^2}$ so $a_1 = a$ and therefore, $u_1 = a_1 - b_1 \sqrt{d}$. 

\item  Case $d = 6$: \\
$q^+(1) = 6 + 1 = 7$, $q^-(1) = 6 - 1 = 5$, $q^-(2) = 24 - 1 = 23$, $q^+(2) = 24 + 1 = 25$ thus the smallest square is for $b_1 = 2$ and $a_1 = \sqrt{25} = 5$ so $u_1 = 5 - 2 \sqrt{6}$ then $\qnorm{K}{u_1} = +1$. \\\\
Case $d = 10$: \\
$q^+(1) = 10 + 1 = 11$, $q^-(1) = 10 - 1 = 9$ thus the smallest square is for $b_1 = 1$ and $a_1 = \sqrt{9} = 3$ so $u_1 = 3 - \sqrt{10}$ then $\qnorm{K}{u_1} = -1$. \\\\
Case $d = 14$: \\
$q^+(1) = 14 + 1 = 15$, $q^-(1) = 14 - 1 = 13$, $q^-(2) = 56 - 1 = 55$, $q^+(2) = 56 + 1 = 57$, $q^+(3) = 126 + 1 = 127$, $q^-(3) = 126 - 1 = 125$, $q^+(4) = 224 + 1 = 225$, $q^-(3) = 224 - 1 = 223$ thus the smallest square is for $b_1 = 4$ and $a_1 = \sqrt{225} = 15$ so $u_1 = 15 - 4 \sqrt{14}$ then $\qnorm{K}{u_1} = +1$. \\\\

\end{enumerate}

\item

\begin{enumerate}
\item $9 = 3 \cdot 3 = -(1 - \sqrt{10})(1 + \sqrt{10})$. We want to show that these factorizations are not related by units. In particular, that $3$ and $1 + \sqrt{10}$ are not equivalnt. For the case $d = 10$ the group of units of $\Z[\sqrt{10}]$ is generated by $-1$ and $u_1 = 3 - \sqrt{10}$. We must show that \[1 + \sqrt{10} \neq (-1)^k \cdot 3 \cdot u_1^n = (-1)^k \cdot 3 \cdot (3 - \sqrt{10})^n = \pm 3 \cdot (a_n + b_n \sqrt{10})\]
so but the coefficient of $\sqrt{10}$ is $1$ and thus cannot be divisible by $3$.  

\item By a similar method to problem 2d on Assignment \# 3, we claim that \[(3) = (3, 1 + \sqrt{10}) (3, 1 - \sqrt{10}) = AB\] by the previous problem, neither $A$ nor $B$ can be equivalent to $(3)$ and thus $AB$ is the prime factorization of $(3)$. This must hold because if $A$ or $B$ were not prime they could be decomposed as a product of primes but in a quadratic ring of integers, any $p \ints{K}$ factors as at most two primes. Thus, $A$ and $B$ must be primes themselves.  

\item First, we compute the Minkowski bound. Since $d = \mod{10}{2}{4}$ we have $\Delta_K = 4 d = 40$ and $\Q(\sqrt{10})$ is a real quadratic extension so $r_1 = 2$ and $r_2 = 0$. Thus,
\[ c_1 = \left(\frac{4}{\pi}\right)^{r_2} \frac{2!}{2^2} \sqrt{\Delta_K} = \sqrt{10} \approx 3.16\]
Therefore, every ideal class contains an ideal with norm less than or equal to $3$. We look at the ideals which factor $(2)$ and $(3)$ because these must include an element in each ideal class except for the unit norm ideal which coresponds to the class of principal ideals. These ideals are $A$, $B$, and $C$ where $C^2 = (2)$ (which we know is ramified because $\mod{10}{2}{4}$). We have shown that $\Z[\sqrt{10}]$ is not a UFD and therefore certainally not a PID so the class number is at least two. There are only four possible minimal ideals and also four possible groups with orders $2,3,4$ which are, $\Z/2\Z$, $\Z/3\Z$, $\Z/4\Z$, and $\Z/2\Z \times \Z/2\Z$. However, the products of generators of $A$ are \[3 \cdot (1 + \sqrt{10}), 3 \cdot 3 = -(1 - \sqrt{10})(1 + \sqrt{10}), (1 + \sqrt{10})^2\]
each of which is divisible by $1 + \sqrt{10}$ so $A^2 \subset (1 + \sqrt{10})$ but 
\[ -(1 + \sqrt{10})^2 + 3 (1 + \sqrt{10}) + 3 \cdot 3 = -1 - 2\sqrt{10} - 10 + 3 + 3 \sqrt{10} + 9 = 1 + \sqrt{10}\]
so $(1 + \sqrt{10}) \subset A^2$ and thus $A^2 = (1 + \sqrt{10})$. However, an indentical argument shows that $B^2 = (1 - \sqrt{10})$. Therefore, since all principal ideals are in the identity ideal class, the orders of $A$, $B$, and $C$ must be $2$. Thus, the class group cannot be $\Z/3\Z$ (else two elements would have order 3) or $\Z/4\Z$ (else only one element would have order 2 and if these elements are not distinct we don't have enough elements anyway). However, $A^2$ and $AB$ are both principal so $A$ and $B$ must be equivalent because inverses are unique. Therefore, we have at most $3$ ideal classes which makes $\Z/2\Z \times \Z/2\Z$ impossible. Therefore, the ideal class group is $\Z/2\Z$ so the class number is $2$.        

 
\end{enumerate}

\item Let $\{\pideal_1, \dots, \pideal_r+1 \}$ be all the prime ideals of a Dedekind domain $R$. Now consider the ideal \[ I = \pideal_i^2  + \prod\limits_{j \neq i}^n \pideal_j\]
then both $\pideal_i^2 \subset I$ and $\prod\limits_{j \neq i}^n \pideal_j \subset I$ so $\pideal_i^2 = IJ$ and $\prod\limits_{j \neq i}^n \pideal_j = IJ'$ which implies that these ideals share prime factors. This is a contradiction unless $I = R$. Thus, by the Chinese remainder theorem, the projection \[\pi : R \to R/\pideal_i^2 \times \prod_{j \neq i}^n R/\pideal_j\]
is a surjection. Therefore, $\exists x \in R$ s.t. $\pi(x) = ([z], [1], [1], \cdots, [1])$
where I have choosen $z \in \pideal_i \sm \pideal_i^2$. This is always possible because $\pideal_i = \pideal_i^2$ would contradict the uniqueness of prime factorization. Now, $x \in \pideal_i$ because $x - z \in \pideal_i^2 \subset \pideal_i$ and $z \in \pideal_i^2$. Furthermore, $x \notin \pideal_i^2$ because $x - z \in \pideal_i^2$ but $z \notin \pideal_i^2$. Also, if $x \in \pideal_j$ then because $x - 1 \in \pideal_j$ we have that $1 \in \pideal_j$ which contradicts its primality. Thus, $x \in \pideal_i$ but $x \notin \pideal_i^2$ and $x \notin \pideal_j$ for $i \neq j$. Thus, $(x) \subset \pideal_i$ so $(x) = \pideal_i J$ but then $x \in J$ so $J$ cannot have any factors of $\pideal_j$ for $i \neq j$ else $I \subset \pideal_j$ so then $x \in \pideal_j$. Thus, $(x) = \pideal_i \pideal_i^k$ but $x \notin \pideal_i^2$ so $\pideal_i \pideal_i^k \supsetneq \pideal_i^2$ so $k = 0$. Thus, $(x) = \pideal_i$ so every prime ideal is principal. Now any ideal $I$ can be written as,
\[I = \prod_{i = 1}^{n} \pideal_i^{\mathrm{ord}_{\pideal_i}(I)} = \prod_{i = 1}^{n} (x_i)^{\mathrm{ord}_{\pideal_i}(I)} = \left(\prod_{i = 1}^{n} x_i^{\mathrm{ord}_{\pideal_i}(I)} \right) \] 
so every ideal is principal. 
\end{enumerate}

\end{document}