\documentclass[12pt]{extarticle}
\usepackage[utf8]{inputenc}
\usepackage[english]{babel}
\usepackage[a4paper, total={7in, 9.5in}]{geometry}
 
\usepackage{amsthm, amssymb, amsmath, centernot}
\usepackage{mathtools}
\DeclarePairedDelimiter{\floor}{\lfloor}{\rfloor}

\newcommand{\notimplies}{%
  \mathrel{{\ooalign{\hidewidth$\not\phantom{=}$\hidewidth\cr$\implies$}}}}
 
\renewcommand\qedsymbol{$\square$}
\newcommand{\cont}{$\boxtimes$}
\newcommand{\divides}{\mid}
\newcommand{\ndivides}{\centernot \mid}
\newcommand{\Z}{\mathbb{Z}}
\newcommand{\N}{\mathbb{N}}
\newcommand{\C}{\mathbb{C}}
\newcommand{\Zplus}{\mathbb{Z}^{+}}
\newcommand{\Primes}{\mathbb{P}}
\newcommand{\ball}[2]{B_{#1} \! \left(#2 \right)}
\newcommand{\Q}{\mathbb{Q}}
\newcommand{\R}{\mathbb{R}}
\newcommand{\Rplus}{\mathbb{R}^+}
\newcommand{\invI}[2]{#1^{-1} \left( #2 \right)}
\newcommand{\End}[1]{\text{End}\left( A \right)}
\newcommand{\legsym}[2]{\left(\frac{#1}{#2} \right)}
\renewcommand{\mod}[3]{\: #1 \equiv #2 \: (\mathrm{mod} \: #3) \:}
\newcommand{\nmod}[3]{\: #1 \centernot \equiv #2 \: (\mathrm{mod} \: #3) \:}
\newcommand{\ndiv}{\hspace{-4pt}\not \divides \hspace{2pt}}
\newcommand{\finfield}[1]{\mathbb{F}_{#1}}
\newcommand{\finunits}[1]{\mathbb{F}_{#1}^{\times}}
\newcommand{\ord}[1]{\mathrm{ord}\! \left(#1 \right)}
\newcommand{\quadfield}[1]{\Q \small(\sqrt{#1} \small)}
\newcommand{\vspan}[1]{\mathrm{span}\! \left\{#1 \right\}}
\newcommand{\galgroup}[1]{Gal \small(#1 \small)}
\newcommand{\ints}[1]{\mathcal{O}_{#1}}
\newcommand{\sm}{\! \setminus \!}
\newcommand{\norm}[3]{\mathrm{N}^{#1}_{#2}\left(#3\right)}
\newcommand{\qnorm}[2]{\mathrm{N}^{#1}_{\Q}\left(#2\right)}
\newcommand{\quadint}[3]{#1 + #2 \sqrt{#3}}
\newcommand{\pideal}{\mathfrak{p}}
\newcommand{\inorm}[1]{\mathrm{N}(#1)}
\newcommand{\tr}[1]{\mathrm{Tr} \! \left(#1\right)}
\newcommand{\delt}{\frac{1 + \sqrt{d}}{2}}
\renewcommand{\Im}[1]{\mathrm{Im}(#1)}
\newcommand{\modring}[1]{\Z / #1 \Z}
\newcommand{\modunits}[1]{(\modring{#1})^\times}
\renewcommand{\empty}{\varnothing}
\renewcommand{\d}[1]{\mathrm{d}#1}
\newcommand{\deriv}[2]{\frac{\d{#1}}{\d{#2}}}
\newcommand{\pderiv}[2]{\frac{\partial{#1}}{\partial{#2}}}
\newcommand{\parsq}[2]{\frac{\partial^2{#1}}{\partial{#2}^2}}

\newcommand{\atitle}[1]{\title{% 
	\large \textbf{Mathematics W4043 Algebraic Number Theory
	\\ Assignment \# #1} \vspace{-2ex}}
\author{Benjamin Church \\ \textit{Worked With Matthew Lerner-Brecher} }
\maketitle}

 
\newtheorem{theorem}{Theorem}[section]
\newtheorem{lemma}[theorem]{Lemma}
\newtheorem{proposition}[theorem]{Proposition}
\newtheorem{corollary}[theorem]{Corollary}


\begin{document}
\atitle{7}
 
\begin{enumerate}
\item
\begin{enumerate}
\item Take the quadratic form $a X^2 + b XY + c Y^2$ with discriminant $\Delta = b^2 - 4ac = -7$ and $a \le \sqrt{|\Delta| / 3} \approx 1.53$. Because $\mod{-7}{1}{4}$ we know that $b$ is odd. For $b = 1$, we have $1 - 4 ac = -7$ so $4 ac = 8$ and therefore, $ac = 2$. Thus, $a = 1$ and $c = 2$ under the requirement that $|b| \le a \le c$. We have the reduced solution $(1, 1, 2)$. No other values are possible because $|b| \le a \le \sqrt{|\Delta|/3}$ implies that $b = \pm 1$ or $b = 0$. However, $b$ must be odd and in both cases when $|b| = 1$ we have $a = |b| = 1$ so $b \le 0$ by the definition of a reduced form.  

\item Let $K = \quadfield{-7}$ and $I$ be an ideal of $\ints{K}$ with a $\Z$ basis $\{\alpha_1, \alpha_2\}$. On assignment \# 4, we proved that \[q_I(a,b) = \frac{\qnorm{K}{a \alpha_1 + b \alpha_2}}{\inorm{I}}\] is a quadratic form with discriminant $\Delta_I = \Delta_N$ where $\Delta_N$ is the discriminant of the quadratic form given by the norm over the standard basis $\{1, \delta\}$ where $\delta = \frac{1+\sqrt{-7}}{2}$ because $\mod{-7}{1}{4}$. However, for $\mod{d}{1}{4}$ the field $K = \quadfield{d}$ has discriminant $d$ (which equals the discriminant of the form $\qnorm{K}{x}$). Thus, $\Delta_I = -7$. However, by part (a), there is a single equivalence class of ideals with discriminant $\Delta = -7$. In particular, $q_I$ must be equivalent to the reduced form $q(X, Y) = X^2 + XY + 2 Y^2$. Therefore, there exist integers  $r,s,t,u \in \Z$ such that $q(a,b) = q_I(ar + bs, at + bu)$. Let $a = 1$ and $b = 0$ then $q(a, b) = a^2 + ab + 2b^2 = 1$ and thus,
\[q_I(r,t) = \frac{\qnorm{K}{r \alpha_1 + t \alpha_2}}{\inorm{I}} = 1\]
Thus, $\qnorm{K}{r \alpha_1 + t \alpha_2} = \inorm{I}$. Let $\beta = r \alpha_1 + t \alpha_2$. Because $\alpha_1, \alpha_2 \in I$ we have that $\beta \in I$ so $(\beta) \subset I$ and therefore there exists and ideal $J$ such that $(\beta) = IJ$. Thus, $\inorm{\beta \ints{K}} = \inorm{I} \inorm{J}$ but $\inorm{\beta \ints{K}} = \qnorm{K}{\beta} = \inorm{I}$ so $\inorm{J} = 1$. Therefore, $J = \ints{K}$ so $(\beta) = I \ints{K} = I$ so $I$ is a pricipal ideal. Since every ideal of $\ints{K}$ is therefore principal, the class number of $K = \quadfield{-7}$ is $1$.      

\item The prime $p$ can be represented by a quadratic form if and only if $\Delta$, the discriminant, is a square modulo $4p$. If $\Delta$ is a square modulo $4p$, then $\Delta$ is a square modulo $p$ and $4$. Conversely, suppose that $\Delta$ is a square modulo $4$ and modulo $p$ (for odd $p$) then there are numbers $a,b$ s.t. $\mod{\Delta}{a^2}{4}$ and $\mod{\Delta}{b^2}{p}$ so by CRT there is a solution modulo $4p$ to $\mod{x}{a}{4}$ and $\mod{x}{b}{p}$. Thus, $\mod{x^2}{\Delta}{4}$ and $\mod{x^2}{\Delta}{p}$ so $\mod{x^2}{\Delta}{4p}$. For the case that $\Delta = -7$, because $\mod{-7}{1}{4}$ is a square, $p$ is represented iff $\legsym{\Delta}{p} = 1$ or $\legsym{\Delta}{p} = 0$. By quadratic reciprociy,
\[\legsym{\Delta}{p} = \legsym{-7}{p} = (-1)^{\frac{p-1}{2}} \legsym{7}{p} =  (-1)^{\frac{-8}{2}} \legsym{p}{7} = \legsym{p}{7}\]
Thus, $p$ is represented iff $\mod{p}{0,1,2,4}{7}$. We have excluded $p = 2$ from the previous discussion and will now consider whether $2$ is represented. Since $\mod{-7}{1}{8}$ we have that $-7$ is a square modulo $4p$ for $p = 2$ so $2$ is represented. Since an odd prime $p$ is split in $\quadfield{d}$ iff $\legsym{d}{p} = 1$, we have that every split prime is represented but $p = 7$ is ramified rather than split although $7$ is also represented. Because $\mod{-7}{1}{8}$, the prime $p = 2$ is split so we have that every prime excluding $7$ is split if and only if it is represented.  

\end{enumerate}

\item Let $\chi : \modring{p} \to \C$ be a Dirichlet character modulo $p$. Then for $1 \in \Z /p \Z$ we have $1 \cdot 1 = 1$ so $\chi(1) \chi(1) = \chi(1)$. Since $\C$ is a field, either $\chi(1) = 0$ or $\chi(1) = 1$. However, $(1, p) = 1$ so $\chi(1) \neq 0$ therefore $\chi(1) = 1$. By Lagrange's theorem, $\forall a \in (\modring{p})^\times : a^{p-1} = 1$ therefore $\chi(a)^{p-1} = \chi(a^{p-1}) = \chi(1) = 1$ so $\chi(a)$ is a $(p-1)$-st root of unity. 

\item Take $a \in (\modring{p})^\times$ then $\exists a^{-1} \in (\Z/ p \Z)^\times$ so $\chi(a) \chi(a^{-1}) = \chi(a a^{-1}) = 1$. Because $\C$ is a field, $\chi(a^{-1}) = \chi(a)^{-1}$. However, for $z \in \C$, we have $z^{-1} = \frac{1}{|z|^2} \cdot \bar{z}$. However, $\chi(a)$ is a root of unity so, because the magnitude is multiplicative, $\chi(a)$ has magnitude $1$. Thus, $\chi(a^{-1}) = \chi(a)^{-1} = \bar{\chi}(a)$.  

\item Suppose that $\chi \neq \chi_0$. Now, $[a] \notin \modunits{p} \iff (a, p) \neq 1 \iff \chi(a) = 0$ so,
\[ \sum\limits_{a \in \modring{p}} \chi(a) = \sum\limits_{a \in \modunits{p}} \chi(a)\]
However, $(\modring{p})^\times$ is a finite multiplicative subgroup of a field and therefore is cyclic. Take a generator $g \in \modunits{p}$. Now, $\modunits{p} = \{ g^k \mid 0 \le k \le p - 2\}$ so the sum is,
\[ \sum\limits_{a \in \modunits{p}} \chi(a) = \sum\limits_{k = 0}^{p-2} \chi(g^k) = \sum\limits_{k = 0}^{p-2} \chi(g)^k = \frac{\chi(g)^{p-1} - 1}{\chi(g) - 1}\]
However, $g \in \modunits{p}$ so $\chi(g)$ is a $(p-1)$-st root of unity and therefore a root of the polynomial $X^{p-1} - 1$. Furthermore, if $\chi(g) = 1$, then $\chi(a) = \chi(g^k) = \chi(g)^k = 1$ so $\chi = \chi_0$ which we assumed was false. Thus, $\chi(g)$ is a root of  $X^{p-1} - 1$ but not of $X - 1$ and therefore, $\chi(g)$ is a root of the polynomial,
\[\frac{X^{p-1} - 1}{X - 1}\] 
In full,
\[ \sum\limits_{a \in \modring{p}} \chi(a) = \sum\limits_{a \in \modunits{p}} \chi(a) = \sum\limits_{k = 0}^{p-2} \chi(g^k) = \sum\limits_{k = 0}^{p-2} \chi(g)^k = \frac{\chi(g)^{p-1} - 1}{\chi(g) - 1} = 0\]

\item Let $\chi : \modring{p} \to \C$ be given by $\chi : a \mapsto \legsym{a}{p}$ if $(p, a) = 1$ and $a \mapsto 0$ otherwise. The definining properties of a Dirichlet character follow from basic properties of the Legendre Symbol. First, 
\[\mod{\legsym{ab}{p}}{(ab)^{\frac{p-1}{2}} = a^{\frac{p-1}{2}} b^{\frac{p-1}{2}}}{p} \implies  \mod{\legsym{ab}{p}}{\legsym{a}{p} \legsym{b}{p}}{p}\]
Also, $p \divides ab \iff p \divides a \text{ or } p \divides b$ therefore $(ab, p) = 1 \iff (a, p) = 1 \text{ and } (b, p) = 1$. Thus, $\chi(ab) = \chi(a) \chi(b)$ since if $(a, p) \neq 1$ then $(ab, p) \neq 1$ so $\chi(ab) = 0 = \chi(a) \chi(b)$. Furthermore, let $\mod{a}{b}{p}$ then, if $p \divides a$ then $p \divides b$ and
\[\legsym{a}{p} \equiv a^{\frac{p-1}{2}} \equiv \mod{b^{\frac{p-1}{2}}} {\legsym{b}{p}}{p}\]
Therefore, $\chi(a) = \chi(b)$. Finally, $a$ and $p$ have a common factor if and only if $p \divides a$ if and only if $\chi(a) = 0$. Thus, $\chi$ is a Dirichet character. Also, the kernel of the function $s : a \mapsto a^2$ has order $2$ so the image of the function cannot be the entire ring $\Z / p \Z$. Therefore, there exists at least one quadratic non-residue so $\Im{\chi} = \{0, \pm 1\}$ so this function must be distinct from $\chi_0$ which has image $\{0, 1\}$. 

\end{enumerate}


\end{document}