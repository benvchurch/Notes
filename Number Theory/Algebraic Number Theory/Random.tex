
Case p = 7: First, we consider the factorization of $(7)$ in $E = \Q(\zeta_3) = \Q(\sqrt{-3})$. Since $\legsym{-3}{7} = 1$ we know that $7$ is split in $E$, that is,
\[7 \ints{E} = \mathfrak{p}_1 \mathfrak{p}_2\] 
Now, consider the extension, $[K : E] = 3$, which is Galois because $[K : \Q]$ is Galois. Therefore, in the extension $K/E$ we can factor,
\[ \mathfrak{p}_i \ints{K} = \prod_{j = 1}^g \mathfrak{P}_j^{e}\]
where the indeces are the same for each prime factor because the extension is Galois. Also, $[K : E] = efg$ where $[K : E] = 3$ and $f = [\ints{K}/\mathfrak{P}_j : \ints{E}/\mathfrak{p}_i]$. However, $\ints{E} = \Z[\zeta_3]$ which is (proven on assigment 3) a PID. Therefore, we can write $\mathfrak{p}_i = (\alpha_i)$. Because $\sqrt[3]{15} \in K$ is a root of $X^3 - 15$ and thus an algebraic integer and $\ints{E} \subset \ints{K}$ we know that, \[\ints{E}[\sqrt[3]{15}] \subset \ints{K}\] Because $X^3 - 15$ has no roots in $E$ we know that $\ints{E}[\sqrt[3]{15}] \cong \ints{E}[X]/(X^3 - 15)$. Now, consider the quotient,
\[ \ints{K}/ \alpha_i \ints{K} \supset \ints{E}[\sqrt[3]{15}]/(\alpha_i) \cong \ints{E}[X]/(\alpha_i, X^3 - 15) \cong \finfield{7}[X]/(X^3 - 15) \]
where I have used the fact that $\ints{E}/(\alpha_i) = \ints{E}/\mathfrak{p}_i \cong \finfield{7}$ because $7$ is split in $E$. Furthermore, $X^3 - 15$ has roots, $1, 2, 4$ in $\finfield{7}$ so in the field $\finfield{7}[X]$ the ideal factors as, $(X^3 - 15) = (X - 1)(X - 2)(X - 4)$. Because these roots are all distinct, the factors are pairwise relatively prime. Thus,
\[\frac{\finfield{7}[X]}{(X^3 - 15)} = \frac{\finfield{7}[X]}{(X - 1)(X - 2)(X - 4)} \cong \frac{\finfield{7}[X]}{(X - 1)} \oplus \frac{\finfield{7}[X]}{(X - 2)}  \oplus \frac{\finfield{7}[X]}{(X - 4)}  \cong \finfield{7} \oplus \finfield{7} \oplus \finfield{7}\] 
Therefore, the ideal $\alpha_i \ints{K}$ cannot be prime because its quotient ring contains zero divisors. Thus, $g > 1$ but because $efg = 3$ then we know that $e = f = 1$ and $g = 3$. Therefore, three prime ideals of $\ints{K}$ lie above each $\mathfrak{p}_i$. Thus, in total, six prime ideals of $\ints{K}$ lie above $7$. Futhermore, we have proven that $7$ is unramified in $K$. This holds, because $e_{K/\Q}(\mathfrak{P}) = e_{K/E}(\mathfrak{P}) \: \: e_{E/\Q}(\mathfrak{p}) = 1$ since $7$ is unramified in $E$ and each of its prime divisors in $E$ are likewise unramified in $K$. 