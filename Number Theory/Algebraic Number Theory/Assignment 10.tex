\documentclass[12pt]{extarticle}
\usepackage[utf8]{inputenc}
\usepackage[english]{babel}
\usepackage[a4paper, total={7in, 9.5in}]{geometry}
 
\usepackage{amsthm, amssymb, amsmath, centernot}
\usepackage{mathtools}
\DeclarePairedDelimiter{\floor}{\lfloor}{\rfloor}

\newcommand{\notimplies}{%
  \mathrel{{\ooalign{\hidewidth$\not\phantom{=}$\hidewidth\cr$\implies$}}}}
 
\renewcommand\qedsymbol{$\square$}
\newcommand{\cont}{$\boxtimes$}
\newcommand{\divides}{\mid}
\newcommand{\ndivides}{\centernot \mid}
\newcommand{\Z}{\mathbb{Z}}
\newcommand{\N}{\mathbb{N}}
\newcommand{\C}{\mathbb{C}}
\newcommand{\Zplus}{\mathbb{Z}^{+}}
\newcommand{\Primes}{\mathbb{P}}
\newcommand{\ball}[2]{B_{#1} \! \left(#2 \right)}
\newcommand{\Q}{\mathbb{Q}}
\newcommand{\R}{\mathbb{R}}
\newcommand{\Rplus}{\mathbb{R}^+}
\newcommand{\invI}[2]{#1^{-1} \left( #2 \right)}
\newcommand{\End}[1]{\text{End}\left( A \right)}
\newcommand{\legsym}[2]{\left(\frac{#1}{#2} \right)}
\renewcommand{\mod}[3]{\: #1 \equiv #2 \: (\mathrm{mod} \: #3) \:}
\newcommand{\nmod}[3]{\: #1 \centernot \equiv #2 \: (\mathrm{mod} \: #3) \:}
\newcommand{\ndiv}{\hspace{-4pt}\not \divides \hspace{2pt}}
\newcommand{\finfield}[1]{\mathbb{F}_{#1}}
\newcommand{\finunits}[1]{\mathbb{F}_{#1}^{\times}}
\newcommand{\ord}[1]{\mathrm{ord}\! \left(#1 \right)}
\newcommand{\quadfield}[1]{\Q \small(\sqrt{#1} \small)}
\newcommand{\vspan}[1]{\mathrm{span}\! \left\{#1 \right\}}
\newcommand{\galgroup}[1]{Gal \small(#1 \small)}
\newcommand{\ints}[1]{\mathcal{O}_{#1}}
\newcommand{\sm}{\! \setminus \!}
\newcommand{\norm}[3]{\mathrm{N}^{#1}_{#2}\left(#3\right)}
\newcommand{\qnorm}[2]{\mathrm{N}^{#1}_{\Q}\left(#2\right)}
\newcommand{\quadint}[3]{#1 + #2 \sqrt{#3}}
\newcommand{\pideal}{\mathfrak{p}}
\newcommand{\inorm}[1]{\mathrm{N}(#1)}
\newcommand{\tr}[1]{\mathrm{Tr} \! \left(#1\right)}
\newcommand{\delt}{\frac{1 + \sqrt{d}}{2}}
\renewcommand{\Im}[1]{\mathrm{Im}(#1)}
\newcommand{\modring}[1]{\Z / #1 \Z}
\newcommand{\modunits}[1]{(\modring{#1})^\times}
\renewcommand{\empty}{\varnothing}
\renewcommand{\d}[1]{\mathrm{d}#1}
\newcommand{\deriv}[2]{\frac{\d{#1}}{\d{#2}}}
\newcommand{\pderiv}[2]{\frac{\partial{#1}}{\partial{#2}}}
\newcommand{\parsq}[2]{\frac{\partial^2{#1}}{\partial{#2}^2}}

\newcommand{\atitle}[1]{\title{% 
	\large \textbf{Mathematics W4043 Algebraic Number Theory
	\\ Assignment \# #1} \vspace{-2ex}}
\author{Benjamin Church \\ \textit{Worked With Matthew Lerner-Brecher} }
\maketitle}

 
\newtheorem{theorem}{Theorem}[section]
\newtheorem{lemma}[theorem]{Lemma}
\newtheorem{proposition}[theorem]{Proposition}
\newtheorem{corollary}[theorem]{Corollary}


\begin{document}
\atitle{10}
 
\begin{enumerate}
\item[6.22]  Consider the equation $y^2 = x^3 - 2$. Reducing modulo $2$, $\mod{y}{x}{2}$ so either both $x$ and $y$ are even or they are both odd. If $x$ and $y$ are both even then $4 \divides x^3 - y^2 = 2$ which is a contradiction. Thus, $x$ and $y$ are even. We can rewrite this equation as,
\[x^3 = y^2 + 2 = (y + i \sqrt{2})(y - i \sqrt{2})\]
Therefore, the element $y^2 + 2 = (y + i \sqrt{2})(y - i \sqrt{2})$ is a square in $\Z[ i \sqrt{2}]$. Let $K = Q(\sqrt{-2})$ then $\ints{K} = \Z[\sqrt{-2}] = \Z[i \sqrt{2}]$ because $\nmod{-2}{1}{4}$. \bigskip \\ I claim that $a = y + i \sqrt{2}$ and $b = y - i \sqrt{2}$ are coprime. If $d \divides y + i \sqrt{2}$ and $d \divides y - i \sqrt{2}$ then $d \divides 2 i \sqrt{2}$. Therefore, $\qnorm{K}{d} \divides \qnorm{K}{2 i \sqrt{2}} = 8$. Therefore either $d = \pm 1$ or $\qnorm{K}{d}$ is even. However, $d \divides y + i \sqrt{2}$ so $\qnorm{K}{d} \divides \qnorm{K}{y + i \sqrt{2}} = y^2 + 2$ which is odd if $y$ is odd. Therefore $d = 1$. However, $\Z[\sqrt{-2}]$ is a PID and therefore a UFD. Thus, if $(y + i \sqrt{2})(y - i \sqrt{2})$ is a cube then $y + i \sqrt{2}$ is a cube as well. Now suppose that, 
\[y + i \sqrt{2} = (a + b i \sqrt{2})^3 = a^3 - 6ab^2 + (3a^2 - 2b^2) b i \sqrt{2} \]
with $a, b \in \Z$. Because $\sqrt{2}$ is irrational, the coeficients must themselves be equal. Therefore, $(3a^2 - 2 b^2)b = 1$ thus $b = \pm 1$ and $3a^2 - 2b^2 = \pm 1$ so $3a^2 = 3$ or $3a^2 = 1$. Since the latter is impossible, we have $a = \pm 1$ and $b = \pm 1$. Thus, $y = a^3 - 6ab^2 = a (a^2 - 6b^2) = a(1 - 6) = -5a$ which takes on the values $\pm 5$ because $a = \pm 1$. \bigskip \\ 
The corresponding values of $x$ are given by the norm of $a + b i \sqrt{2}$ because if $y + i \sqrt{2} = (a + b i \sqrt{2})^3$ then $\qnorm{K}{y + i \sqrt{2}} = y^2 + 2 = (\qnorm{K}{a + b i \sqrt{2}})^3 = x^3$. Thus, $x = \qnorm{K}{\pm 1 \pm i \sqrt{2}} = 1 + 2 = 3$. No other values of $x$ are possible because, if, using the known solution for $y$, $x^3 = y^2 + 2 = 25 + 2 = 27$ then $x = 3$. Therefore, the only solutions to $y^2 = x^3 - 2$ are $(x, y) = (3, \pm 5)$.  
\item[6.23]
\begin{enumerate}
\item Suppose that $A \vec{c} = \vec{c}$. Then for each $i$,
\[ \sum_{j = 1}^r a_{ij} c_j = 0 \quad \text{thus} \quad a_{ii} c_i + \sum_{j \neq i} a_{ij} c_j = 0\]
Therefore, 
\[|a_{ii}| |c_i| = \left| \sum_{j \neq i} a_{ij} c_j \right| \le \sum_{j \neq i} |a_{ij}| |c_j| \]
using the hypothesis and assuming that $|c_i| \neq 0$,
\[\left( \sum_{j \neq i} |a_{ij}| \right) |c_i| < |a_{ii}| |c_i|  \]
Let $c_m$ be the coefficients with the maximum absolute value. Then, $|c_j| \le |c_m|$ so, 
\[\left( \sum_{j \neq i} |a_{ij}| \right) |c_i| < |a_{ii}| |c_i| \le \sum_{j \neq i} |a_{ij}| |c_j| \le \left( \sum_{j \neq i} |a_{ij}| \right) |c_m|  \]
and thus, $|c_i| < |c_m|$ which cannot hold for every $i$ because then we can choose $i = m$ and $|c_m| < |c_m|$ is clearly false. Therefore, $c_m = 0$ which implies that $c_i = 0$ for each $i$ because the maximum absolute value of all these coefficients is zero. Thus, the null space of $A$ is trivial so $A$ is invertible because it is a square matrix.  

\item Introduce the embedding $\Phi : \ints{K} \to \R^{r_1} \times \C^{r_2}$ given by 
\[ \Phi(\alpha) = (\sigma_1(\alpha), \dots, \sigma_{r_1}(\alpha), \sigma_{r_1 + 1}(\alpha), \dots, \sigma_{r_1 + r_2}(\alpha))\]
where $\sigma_i$ runs over real embeddings and one of each conjugate pair of complex embeddings. Now for any positive real numbers $t_1, \cdots t_n \in \R$ with $n = r_1 + r_2$, consider the convex set,
\[B(t_1, \cdots, t_n) = \{x \in \R^{r_1} \times \C^{r_2} \mid |x_i| \le t_i \}\]
which has volume $V_B = 2^{r_1} \pi^{r_2} t_1 \cdots t_{r_1} \cdot t_{r_1 + 1}^2 \cdots t_{r_1 + r_2}^2$. By Minkowski's theorem, we are guaranteed that $B(t_1, \dots, t_n)$ containes a lattice point if $V_B \ge 2^n V_L$ where $V_L$ is the volume of a lattice cell. For a fixed $i$, we can ensure the volume satisfies this criterion by choosing $\tilde{t}_i > 2^{n - r_1} \pi^{-r_2} V_L$ then $B(1, \dots, \tilde{t}_i, \dots, 1)$ contains a non-zero lattice point. Therefore, $\exists \alpha_1 \in \ints{K}$ such that $\Phi(\alpha_1) \in B(1, \dots, t_i, \dots, 1)$. Now, we define a sequence of convex sets with $t_j = \frac{|\alpha_k|_j}{4}$ and $t_i = (4^k)^{r_1 + 2 r_2 - 1} \tilde{t}_i$ where $\alpha_k$ begins with $\alpha_1$ and then takes on the value of the last construced algebraic integer. Therefore, when $i \le r_1$,
\[V_{n} = 2^{r_1} \pi^{r_2} t_1 \cdots t_{r_1} \cdot t_{r_1 + 1}^2 \cdots t_{r_1 + r_2}^2 < 2^{r_1} \pi^{r_2} \frac{1}{(4^k)^{r_1 + 2 r_2 - 1}} (t_i)_k = 2^{r_1} \pi^{r_2} \tilde{t}_i > 2^n V_L\]  
since $t_j \le \frac{1}{4^k}$ because each $\alpha_k$ is choosen with $|\alpha_k|_j$ less than the previous $t_j$. Likewise when $i > r_1$ then
\[V_{n} = 2^{r_1} \pi^{r_2} t_1 \cdots t_{r_1} \cdot t_{r_1 + 1}^2 \cdots t_{r_1 + r_2}^2 \le 2^{r_1} \pi^{r_2} \frac{1}{(4^k)^{r_1 + 2 r_2 - 2}} (t_i)_k^2 = 2^{r_1} \pi^{r_2} (4^k)^{r_1 + 2 r_2} \tilde{t}_i > 2^n V_L\]  
therefore, $B_{k+1}$ contains a non-zero lattice point. Thus, take $\alpha_{k+1}$ such that $\Phi(\alpha_k) \in B_k$. Consider the ideals $(\alpha_k)$ which all have norm, 
\[|\qnorm{K}{\alpha_k}| = \prod\limits_{i = 1}^{r_1 + 2 r_2} |\sigma_i(\alpha_k)| < t_1 \cdots t_{r_1} \cdot t_{r_1 + 1}^2 \cdots t_{r_1 + r_2}^2 < 2^n V_{L} \] 
However, by Dedekind prime factorization, there are only finitly many ideals with norm less than some bound so we must have infintly many $\alpha_k$ which generate the same ideal. Take $(\alpha_k) = (\alpha_l)$  with $k > l$ then $\alpha_k = u_i \alpha_l$ for $u \in \ints{K}^\times$. However, we can bound the absolute values of this unit, when $j \neq i$,
$|u_i|_j = |\sigma_j(u_i)| = |\sigma_j(\tfrac{\alpha_k}{\alpha_l})| = \frac{|\sigma_j(\alpha_k)|}{|\sigma_j(\alpha_l)|} \le \frac{1}{4}$ because at each state the absolute value is reduced by at least $1/4$ and Minkowski's theorem ensures that none of the $\alpha_k$ are zero so none of the absolute values are zero. Also, because $|\qnorm{K}{u_i}| = |u_i|_1 \cdots |u_i|_{r_1} \cdot |u_i|_{r_1 + 1}^2 \cdots |u_i|_{r_1 + r_2}^2 = 1$ then $|u_i|_i > 1$ since each other absolute value is less than $1$. We have found a unit satisfying these criteria for each $i$. 
         
\item Consider the matrix of log-absolute values, i.e. $a_{ij} = \log{|u_i|_j}$ where $i$ and $j$ range from $1$ to $r = r_1 + r_2 - 1$. Now, because $|\qnorm{K}{u_i}| = |u_i|_1 \cdots |u_i|_{r_1}^2 \cdot |u_i|_{r_1 + 1} \cdots |u_i|_{r_1 + r_2}^2 = 1$ then $|u_i|_i > 1$ we have that,
\[ \sum_{j = 1}^{r + 1} \log{|u_i|_j} = \log{1} = 0 \] 
and thus,
\[ \sum_{j = 1}^{r} a_{ij} = \sum_{j = 1}^{r + 1} \log{|u_i|_j} - \log{|u_i|_{r+1}} > 0  \]
because $|u_i|_{r+1} \le \frac{1}{4}$ so $-\log{|u_i|_{r+1}} > 0$. For $j \neq i$, the same condition applies i.e. $\log{|u_i|_{r+1}} < 0$ because $|u_i|_{j} \le \frac{1}{4}$. Then,
\[|a_{ii}| = a_{ii} > - \sum_{j \neq i} a_{ij} = \sum_{j \neq i} |a_{ij}| \]  
because $|u_i|_i > 1$ so $a_{ii} = \log{|u_i|_i} > 0$ and the other terms are negative. Thus, by part (a), $a_{ij}$ is an invertable matrix so its rows are independent. Therefore, the set of units $\{u_1, \cdots, u_r \}$ is independent because if $u_1^{e_1} \cdots u_{r}^{e_r} = 1$ then for any $j$, 
\[\log{|u_1^{e_1} \cdots u_{r}^{e_r}|_j} = \sum_{i = 1}^{r} e_i \log{|u_i|_j} = \sum_{i = 1}^{r} e_i a_{ij} = \log{|1|_j} = 0 \implies e_i = 0\]
because the matrix has independent rows.  

\end{enumerate}
\item[2.]
\begin{enumerate}
\item Let $\bf{1}$ denote the function $\mathbf{1}(n) = 1$. Now consider the convolution,\[ (\mathbf{1} * \mathbf{1})(n) = \sum_{d \divides n} \mathbf{1}(d) \mathbf{1}(\tfrac{n}{d}) = \sum_{d \divides n} 1 = \left|\{ d \in \Zplus \mid d \divides n \} \right| = \tau(n)\]
\item Let $f$ and $g$ be multiplicative functions, now consider the convolution of $f$ and $g$ applied to coprime $a,b$,
\[ (f * g)(ab) = \sum_{d \divides ab} f(d) g(\tfrac{ab}{d}) \]
If $d \divides ab$ then let $d' = \frac{d}{(d, a)}$ which is coprime with $a' = \frac{a}{(d, a)}$. Futhermore, $d' \divides a'b$ but $(d', a') = 1$ so $d' \divides b$. Thus, $d = d' (d, a)$ which is a divisor of $a$ times a divisor of $b$. Futhermore, I claim that if $a$ and $b$ are coprime then these products are distinct. This holds because if $e,f \divides a$ and $g,h \divides b$ and $eg = fh$ therefore $e \divides fh$ and $f \divides eg$. However, $(e,h) = 1$ so $e \divides f$ but $(f, g) = 1$ so $f \divides e$. Thus, $e = f$ ad $g = h$. Therefore,     
\begin{align*}
(f * g)(ab) & = \sum_{d \divides ab} f(ab) g(\tfrac{ab}{d}) = \sum_{d_1 \divides a} \sum_{d_2 \divides b} f(d_1 d_2) g(\tfrac{ab}{d_1 d_2}) = \sum_{d_1 \divides a} \sum_{d_2 \divides b} f(d_1) f(d_2) g(\tfrac{a}{d_1}) g(\tfrac{b}{d_2}) \\ & = \left( \sum_{d_1 \divides a}  f(d_1) g(\tfrac{a}{d_1}) \right) \cdot \left( \sum_{d_2 \divides b} f(d_2) g(\tfrac{b}{d_2}) \right) = (f * g)(a) (f * g)(b) 
\end{align*} 

\item Define $f(n) = n$ and $\mu(1) = 1$ and $\mu(p) = -1$ and $\mu(n) = 0$ iff $n$ contains a square. First, we consider the function $f * \mu$ applied to prime powers. Because $f$ and $\mu$ are multiplicative, $f * \mu$ is as well so the result will extend to all integers. 
\[ (f * \mu)(p^k) = \sum_{d \divides p^k} f(d) \mu(\tfrac{p^k}{d})\]
The only divisors of $p^k$ which are not sent to zero by $\mu$ are $1$ and $p$ so we only need to sum over $d = p^k$ and $d = p^{k-1}$. Therefore, 
\[ (f * \mu)(p^k) = f(p^k)  \mu(\tfrac{p^k}{p^k}) + f(p^{k-1})  \mu(\tfrac{p^k}{p^{k-1}}) = p^k \mu(1) + p^{k-1} \mu(p) = p^k - p^{k-1} = \phi(p^k) \]
Because both $f * \mu$ and $\phi$ are multiplicative and agree for prime powers, by unique factorization they agree on all integers. Therefore, $f * \mu = \phi$. 

\item Define the function $\Lambda(p^k) = \log{p}$ and $\Lambda(n) = 0$ if $n$ is not a prime power. Now, define,
\[ D(s) = \sum_{n = 1}^\infty \frac{\Lambda(n)}{n^s}\]  
We will apply the theorem proved in class to conclude that if $f(n) = O(n^\beta)$ then,
\[ F(s) = \sum_{n = 1}^\infty \frac{f(n)}{n^s}\] coverges for $\mathrm{Re}(s) > 1 + \beta$. Take any $\beta > 0$ then because for any $n \in \Zplus$, $|\log(n)| < n$ we have the inequality, $\log(n^\beta) < n^\beta$ so $\log(n) < \frac{1}{\beta} n^\beta$. Therefore, 
\[ |\Lambda(n)| \le |\log(n)| < \frac{1}{\beta} n^\beta\]
so $\Lambda(n) = O(n^\beta)$ and thus $D(s)$ converges for $\mathrm{Re}(s) > 1 + \beta$ for every $\beta > 0$. Take any $s$ with $\mathrm{Re}(s) > 1$ then choose $\beta = \frac{\mathrm{Re}(s) - 1}{2} > 0$ then $\mathrm{Re}(s) > 1 + \beta = \frac{\mathrm{Re}(s) + 1}{2}$ so $D(s)$ converges on the right half plane $\mathrm{Re}(s) > 1$. \bigskip \\
Now in the right half plane $\mathrm{Re}(s) > 1$ on which $\zeta(s)$ is a holomorphic function, the function,
\[ f(s) = - \deriv{}{s} \log{\zeta(s)} = - \frac{\zeta'(s)}{\zeta(s)} \]   
exists and is holomorphic. Using the Euler product,
\begin{align*}
f(s) &= - \deriv{}{s} \log{ \prod_{p} \frac{1}{1 - p^{-s}}} = \deriv{}{s} \sum_{p} \log{\left(1 - p^{-s}\right)} = \sum_{p} \deriv{}{s} \log{\left(1 - p^{-s}\right)} = \sum_{p} \frac{p^{-s} \log{p}}{1 - p^{-s}} \\ & = \sum_{p} p^{-s} \log{p} \left(1 + \frac{1}{p^s} + \frac{1}{p^{2s}} + \cdots \right) = \sum_{p} \sum_{k = 1}^{\infty} \frac{\log{p}}{(p^k)^s} = \sum_{p} \sum_{k = 1}^{\infty} \frac{\Lambda(p)}{(p^k)^s} = \sum_{n = 1}^{\infty} \frac{\Lambda(p)}{n^s} = D(s)
\end{align*}  
where the sum can be extended from only prime powers to all positivie integers because $\Lambda(n) = 0$ for all $n$ which are not a prime power and the sums are all absolutly convergent so rearrangement poses no issue. 

\end{enumerate}

\end{enumerate}


\end{document}